Sections

SEARCH

\protect\hyperlink{site-content}{Skip to
content}\protect\hyperlink{site-index}{Skip to site index}

\href{https://myaccount.nytimes.com/auth/login?response_type=cookie\&client_id=vi}{}

\href{https://www.nytimes.com/section/todayspaper}{Today's Paper}

\href{https://www.nytimes.com/news-event/coronavirus}{The Coronavirus
Outbreak}

\begin{itemize}
\tightlist
\item
  live\href{https://www.nytimes.com/2020/08/01/world/coronavirus-covid-19.html}{Latest
  Updates}
\item
  \href{https://www.nytimes.com/interactive/2020/us/coronavirus-us-cases.html}{Maps
  and Cases}
\item
  \href{https://www.nytimes.com/interactive/2020/science/coronavirus-vaccine-tracker.html}{Vaccine
  Tracker}
\item
  \href{https://www.nytimes.com/interactive/2020/07/29/us/schools-reopening-coronavirus.html}{What
  School May Look Like}
\item
  \href{https://www.nytimes.com/live/2020/07/31/business/stock-market-today-coronavirus}{Economy}
\end{itemize}

Last Updated

July 24, 2020, 7:20 a.m. ET

July 24, 2020, 7:20 a.m. ET

\hypertarget{about-30-million-workers-are-collecting-jobless-benefits}{%
\section{About 30 Million Workers Are Collecting Jobless
Benefits}\label{about-30-million-workers-are-collecting-jobless-benefits}}

This briefing is no longer being updated. Follow live updates
\href{https://www.nytimes.com/live/2020/07/24/business/stock-market-updates-coronavirus}{here}.

\hypertarget{heres-what-you-need-to-know}{%
\subsubsection{Here's what you need to
know:}\label{heres-what-you-need-to-know}}

\begin{itemize}
\item
  \protect\hyperlink{roughly-one-in-five-workers-are-collecting-unemployment-benefits}{}

  Roughly one in five workers are collecting unemployment benefits.
\item
  \protect\hyperlink{coin-shortage-united-states-mint}{}

  The Mint has a plea for Americans: Change in your coins.
\item
  \protect\hyperlink{amc-postpones-the-reopening-of-its-theaters-to-mid-august}{}

  AMC postpones the reopening of its theaters to mid-August.
\item
  \protect\hyperlink{att-reports-customer-defections-as-americans-cut-back-in-the-pandemic}{}

  AT\&T reports customer defections as Americans cut back in the
  pandemic.
\item
  \protect\hyperlink{a-slide-in-big-technology-shares-drags-wall-street-lower}{}

  A slide in big technology shares drags Wall Street lower.
\item
  \protect\hyperlink{twitter-earnings.html}{}

  Twitter's user count surged but its revenue fell in the second
  quarter.
\end{itemize}

\hypertarget{roughly-one-in-five-workers-are-collecting-unemployment-benefits}{%
\subsection{\texorpdfstring{\protect\hyperlink{roughly-one-in-five-workers-are-collecting-unemployment-benefits}{Roughly
one in five workers are collecting unemployment
benefits.}}{Roughly one in five workers are collecting unemployment benefits.}}\label{roughly-one-in-five-workers-are-collecting-unemployment-benefits}}

Copied to clipboard.

\includegraphics{https://static01.nyt.com/images/2020/07/23/business/23markets-brf-report/merlin_174769635_abbef2aa-dc1d-4817-91cc-a67fa88a187f-articleLarge.jpg?quality=75\&auto=webp\&disable=upscale}

Getting a precise nationwide count of the number of
\href{https://www.nytimes.com/2020/04/09/business/economy/unemployment-claim-numbers-coronavirus.html}{people
collecting unemployment benefits} has been hampered since the start of
the coronavirus pandemic. Data from overwhelmed and understaffed state
offices has been inconsistent and strewn with errors. And there may be
some double-counting as the agencies struggle to clear out the flood of
new and backlogged claims.

On Thursday, the
\href{https://oui.doleta.gov/press/2020/072320.pdf}{Labor Department
reported} that the total number of people
\href{https://www.nytimes.com/2020/07/23/business/economy/unemployment-gig-workers-coronavirus.html}{claiming
unemployment insurance} for the week ending July 4 --- without any
seasonal adjustments --- equaled 31.8 million.

Ernie Tedeschi, a policy economist at Evercore ISI, estimates that the
actual figure is closer to 30 million, roughly one out of every five
workers --- a staggering number by any count.

The Labor Department categorizes the claims in separate batches. In one
are those filed through states' regular unemployment insurance systems,
a number that rose last week for the first time in months. A second
includes people who filed through the federal government's temporary
Pandemic Unemployment Assistance program, for workers not ordinarily
eligible for state benefits. A third and growing group includes
recipients who exhausted their regular benefits but are eligible for an
additional 13 weeks of
\href{https://www.nytimes.com/2020/07/21/business/economy/coronavirus-unemployment-benefits.html}{emergency
assistance} that Congress passed after the coronavirus outbreak.

Over the past couple of months, the labor market showed surprising gains
as businesses reopened and rehired workers. The overall jobless rate
dipped in June to 11.1 percent from a peak of 14.7 percent in April. But
troubling weaknesses may be reflected in the weekly tallies of laid-off
workers filing new applications, Mr. Tedeschi said.

``I'm surprised that the claims numbers haven't gotten better yet,'' he
said.

--- \href{https://www.nytimes.com/by/patricia-cohen}{Patricia Cohen}

\hypertarget{the-increase-in-new-state-jobless-filings-is-the-first-since-early-in-the-pandemic}{%
\subsection{\texorpdfstring{\protect\hyperlink{weekly-unemployment-numbers}{The
increase in new state jobless filings is the first since early in the
pandemic.}}{The increase in new state jobless filings is the first since early in the pandemic.}}\label{the-increase-in-new-state-jobless-filings-is-the-first-since-early-in-the-pandemic}}

Copied to clipboard.

Initial weekly unemployment claims,

both regular and those under the Pandemic Unemployment Assistance
program

6 million

5

4

3

2

1

0

'17

'18

'19

'20

Initial weekly unemployment claims, both regular and those under the
Pandemic Unemployment Assistance program

6 million

5

4

3

2

1

0

2017

2018

2019

2020

Pandemic Unemployment Assistance extends eligibility to some workers who
would not otherwise be able to apply for unemployment benefits, such as
part-time and self-employed workers. Regular claims are seasonally
adjusted but P.U.A. claims are not.

Source: Labor Department

By The New York Times

The government reported on Thursday that more than 1.4 million workers
\href{https://www.dol.gov/ui/data.pdf}{filed new claims for state
unemployment benefits} last week, the first time that the weekly tally
has risen in more than three months.

The upturn, from about 1.3 million in the two preceding weeks, comes
just days before an extra
\href{https://www.nytimes.com/2020/07/21/business/economy/coronavirus-unemployment-benefits.html}{\$600-a-week
jobless benefit is set to expire}.

An additional 975,000 claims were filed last week by freelancers,
part-time workers and others who do not qualify for regular state
jobless aid but are eligible for benefits under an emergency federal
program, the Labor Department said. Unlike the state figures, that
number is not seasonally adjusted.

``At this stage, you're seeing all the wrong elements for recovery,''
said Gregory Daco, the chief United States economist at Oxford
Economics. ``A deteriorating health situation, a weakening labor market
and a softening path for demand.''

The report on Thursday has particular resonance: It reflects the week
that will be used by the department to calculate the June jobs data and
unemployment rate.

The stubbornly high rate of new weekly claims more than four months into
the coronavirus pandemic ``suggests that the nature of the downturn has
changed from early on,'' said Ernie Tedeschi, a policy economist at
Evercore ISI. It may mean that
\href{https://www.nytimes.com/2020/07/15/business/economy/economic-recovery-coronavirus-resurgence.html}{businesses
are shutting down again} as cases surge in some places, or that funds
from emergency federal loans through the Paycheck Protection Program are
running out, he said --- or worse, something more fundamental.

``It might be that businesses are running through their first line of
credit,'' he said, ``and now they're facing the music of an economy that
has recovered a little bit but not nearly enough.''

--- \href{https://www.nytimes.com/by/patricia-cohen}{Patricia Cohen}

\hypertarget{advertisement}{%
\subsubsection{Advertisement}\label{advertisement}}

\protect\hyperlink{after-dfp-ad-mid1}{Continue reading the main story}

\hypertarget{disney-delays-release-of-mulan-and-pushes-back-star-wars-and-avatar-films}{%
\subsection{\texorpdfstring{\protect\hyperlink{disney-delays-release-of-mulan-and-pushes-back-star-wars-and-avatar-films}{Disney
delays release of `Mulan' and pushes back `Star Wars' and `Avatar'
films.}}{Disney delays release of `Mulan' and pushes back `Star Wars' and `Avatar' films.}}\label{disney-delays-release-of-mulan-and-pushes-back-star-wars-and-avatar-films}}

Copied to clipboard.

\includegraphics{https://static01.nyt.com/images/2020/07/23/business/23markets-brf-disney/merlin_170340048_b61a2f06-a6dd-47c5-b89b-0cd2ad47c74b-articleLarge.jpg?quality=75\&auto=webp\&disable=upscale}

\textbf{Disney} on Thursday gave a worrisome update on its movie
business --- the largest in Hollywood, by far, and still mostly shut
down because of the pandemic --- by delaying the theatrical release of
its live-action ``Mulan'' indefinitely and pushing back three upcoming
``Star Wars'' movies and four scheduled ``Avatar'' sequels by one year
each.

The next ``Star Wars'' movie will not arrive until 2023, making for a
far less promising 2022 for Disney's movie and consumer products
divisions. ``Mulan'' was supposed to arrive in theaters on Aug. 21 after
being pushed back several times already.

``It's become clear that nothing can be set in stone when it comes to
how we release films during this global health crisis,'' Disney said in
a statement.

In a similar move, \textbf{Warner Bros.} on
Monday\href{https://www.nytimes.com/live/2020/07/20/business/stock-market-today-coronavirus/warner-bros-backs-off-its-aug-12-release-date-for-tenet?partner=IFTTT}{indefinitely
delayed Christopher Nolan's big-budget ``Tenet,''} which had been
scheduled to arrive in theaters on Aug. 12.

Some upcoming Disney movies are being delayed because the coronavirus
has halted movie production across Hollywood. But the ``Mulan''
postponement reflects the economics of blockbuster-style films and the
inability of theaters in some crucial markets --- New York, Los Angeles,
the San Francisco Bay Area --- to reopen without government approval,
the timing of which is impossible to predict. To release ``Mulan'' on
Aug. 21, Disney would need to start its advertising barrage now. But no
company wants to spend a minimum of \$150 million to market a movie
worldwide if it can't be sure that the advertised product will be
available.

Similarly, because these movies cost roughly \$200 million just to
produce, the only way to make them financially viable is to make them
available everywhere all at once, thwarting piracy as much as possible.
New York, Los Angeles and the Bay Area are the country's three biggest
markets for ticket sales; it would be a financial calamity to release a
``tent pole'' movie with even one of those areas offline.

Disney's next megamovie will now not arrive until at least November,
when ``Black Widow'' is set to roll into theaters.

Without its movie division to generate revenue, Disney's theme parks
have become even more important to the conglomerate. Disney reopened
Walt Disney World in Florida earlier this month. It has been accused of
irresponsibility from people concerned about visitor and worker health,
but Disney has developed what it believes are safe operating procedures.
And it gets to tell investors on its Aug. 4 earnings call that at least
something came back online in the quarter.

--- \href{https://www.nytimes.com/by/brooks-barnes}{Brooks Barnes}

\hypertarget{the-mint-has-a-plea-for-americans-change-in-your-coins}{%
\subsection{\texorpdfstring{\protect\hyperlink{coin-shortage-united-states-mint}{The
Mint has a plea for Americans: Change in your
coins.}}{The Mint has a plea for Americans: Change in your coins.}}\label{the-mint-has-a-plea-for-americans-change-in-your-coins}}

Copied to clipboard.

\includegraphics{https://static01.nyt.com/images/2020/07/23/business/23markets-brf-coins/merlin_141270804_502fd621-8fbe-4a3a-80ff-f587adda3f72-articleLarge.jpg?quality=75\&auto=webp\&disable=upscale}

Pennies and dimes are hard to find in many parts of America after
pandemic lockdowns disrupted their flow and kept households from
exchanging their coin jars for dollar bills.

The United States Mint wants you to know that you can be part of the
solution.

``We ask that the American public start spending their coins,'' the
Mint, which is part of the Treasury, implored in a release on Thursday.
``The coin supply problem can be solved with each of us doing our
part.''

Other options include depositing coins or exchanging them for cash.

The coin shortage has forced regional Federal Reserve banks, which
distributes coins, to institute a
\href{https://www.nytimes.com/2020/06/25/business/economy/coin-shortage-coronavirus.html}{rationing
system}. On June 30, the Fed
\href{https://www.frbservices.org/news/communications/071020-us-coin-task-force-members-confirmed.html?utm_source=home-071020\&utm_medium=banner\&utm_campaign=fedcash\&utm_content=3-coin-task-force}{established}
a coin task force to deal with the unfolding crisis, complete with
``industry leaders in the coin supply chain.''

The shortage has become a problem for many small businesses across
America and has been the topic of
\href{https://www.wataugademocrat.com/covid19/widespread-coin-shortages-a-symptom-of-pandemic/article_806c11c9-9b52-5ace-890a-dc250e8316cc.html}{local
news} coverage and of discussion on a corner of Reddit devoted to
prepping strategies.

Even big retailers are feeling the penny pinch --- \textbf{Walmart},
\textbf{CVS}, \textbf{Kroger} and other chains have begun asking
customers to pay with plastic when possible or to use exact change.

While digital payments have become prevalent, coins have remained
crucial to some parts of the economy: parking meters, vending machines,
amusement parks and
\href{https://cluballiance.aaa.com/the-extra-mile/articles/prepare/travel/10-tips-for-your-next-car-camping-adventure?rdl=midatlantic.aaa.com\&et_cid=sfmc:email:weekly-tem-email:062420-prepare-travel-10-tips-for-your-next-car-camping-adventure:\&promo=\&utm_source=sfmc\&utm_medium=email\&utm_campaign=weekly-tem-email\&utm_content=062420-prepare-travel-10-tips-for-your-next-car-camping-adventure\&utm_term=\&et_cid=719079\&et_rid=15517979\&linkid=A1_10-tips-for-your-next-car-camping-adventure_BTN\&et_jid=ET_EMAIL}{even
campground showers}. For the millions of households without bank
accounts, cash is an essential part of daily life.

``For millions of Americans, cash is the only form of payment and cash
transactions rely on coins to make change,'' the Mint said.

``As important as it is to get more coins circulating, safety is
paramount,'' it added. ``Please be sure to follow all safety and health
guidelines.''

--- \href{https://www.nytimes.com/by/jeanna-smialek}{Jeanna Smialek}

\hypertarget{amc-postpones-the-reopening-of-its-theaters-to-mid-august}{%
\subsection{\texorpdfstring{\protect\hyperlink{amc-postpones-the-reopening-of-its-theaters-to-mid-august}{AMC
postpones the reopening of its theaters to
mid-August.}}{AMC postpones the reopening of its theaters to mid-August.}}\label{amc-postpones-the-reopening-of-its-theaters-to-mid-august}}

Copied to clipboard.

\includegraphics{https://static01.nyt.com/images/2020/07/23/business/23markets-brf-amc/merlin_174565317_7c51299f-c600-4cde-9ecd-a11b8875f322-articleLarge.jpg?quality=75\&auto=webp\&disable=upscale}

\textbf{AMC Theatres}, the nation's largest cinema chain, delayed the
opening of its more than 1,000 theaters in the United States until
mid-to-late August. The move was not a surprise, given that it arrived
on the heels of the Warner Bros. announcement earlier this week that
``\href{https://www.nytimes.com/live/2020/07/20/business/stock-market-today-coronavirus\#warner-bros-backs-off-its-aug-12-release-date-for-tenet}{Tenet},''
its big-budgeted thriller from Christopher Nolan, would not be released
on its rescheduled date of Aug. 12. No new date has yet been set.

With the coronavirus showing no signs of abatement, the studios and
their movie theater partners have been playing a game of chicken,
postponing the return to moviegoing until the virus numbers show a
decline.
\href{https://www.nytimes.com/2020/07/13/business/hong-kong-disneyland-closing.html}{Disney},
which had rescheduled its live-action adaptation of ``Mulan'' for Aug.
21, has yet to announce a date shift but is likely to do so.

\href{https://www.nytimes.com/live/2020/07/07/business/stock-market-today-coronavirus\#several-movie-theater-chains-sue-new-jersey-over-its-reopening-plan}{Theater
chains} large and small are struggling with the ramifications of the
pandemic. Though they received initial assistance from the federal
government in the form of the Paycheck Protection Program, they are
still grappling with soaring fixed costs and zero revenue.

Overseas, AMC says that approximately one-third of its cinemas in Europe
and the Middle East are already open and operating normally.

--- \href{https://www.nytimes.com/by/nicole-sperling}{Nicole Sperling}

\hypertarget{advertisement-1}{%
\subsubsection{Advertisement}\label{advertisement-1}}

\protect\hyperlink{after-dfp-ad-mid2}{Continue reading the main story}

\hypertarget{a-payroll-tax-cut-will-not-be-in-the-next-relief-bill-the-treasury-secretary-says}{%
\subsection{\texorpdfstring{\protect\hyperlink{a-payroll-tax-cut-will-not-be-in-the-next-relief-bill-the-treasury-secretary-says}{A
payroll tax cut will not be in the next relief bill, the Treasury
secretary
says.}}{A payroll tax cut will not be in the next relief bill, the Treasury secretary says.}}\label{a-payroll-tax-cut-will-not-be-in-the-next-relief-bill-the-treasury-secretary-says}}

Copied to clipboard.

\includegraphics{https://static01.nyt.com/images/2020/07/23/business/23markets-brf-mnuchin/merlin_174846600_d60bf665-5db3-43d6-b1a6-0d189a52dfc4-articleLarge.jpg?quality=75\&auto=webp\&disable=upscale}

The Trump administration is dropping its insistence on a payroll tax cut
as the centerpiece of the upcoming economic relief bill in favor of more
direct payments to Americans, Treasury Secretary Steven Mnuchin said on
Thursday.

The
\href{https://www.nytimes.com/2020/07/23/business/payroll-tax-cut-trump-recession.html}{payroll
tax cut}, which was a priority for President Trump, emerged as an
obstacle as Senate Republicans have tried to coalesce around a stimulus
plan this week. Mr. Mnuchin said that the idea, which he thinks would
help stimulate the economy over the longer term, will not be in the
``base bill'' but that it could still emerge in future legislation.

``We think the payroll tax cut is a very good pro-growth policy,'' Mr.
Mnuchin said on CNBC. ``The president's focus is, he wants to get money
into people's pockets now.''

Mr. Mnuchin said that Republicans had agreed to a plan to continue
expanded unemployment insurance, which will expire at the end of the
month. He said that the proposal would replace approximately 70 percent
of a worker's lost wages so that the policy did not create incentives
for people not to return to their jobs.

``We want to make sure that the people who are out there that can't find
jobs do get a reasonable wage replacement,'' Mr. Mnuchin said.

He added that there would be tax credits to encourage businesses to
rehire workers. The plan would also replenish the Paycheck Protection
Program so that small businesses with revenue down by 50 percent or more
can apply for second loans.

His comments come as Senate Republicans on Thursday are expected to
unveil a roughly \$1 trillion coronavirus relief measure that will
allocate more than \$100 billion to schools, new aid for states to
conduct testing across the country and liability protections for
schools, hospitals and businesses.

The Treasury secretary also said that the Republican plan would include
liability protections for businesses that are seeking to reopen to
shield them from ``frivolous lawsuits.''

--- \href{https://www.nytimes.com/by/alan-rappeport}{Alan Rappeport} and
\href{https://www.nytimes.com/by/emily-cochrane}{Emily Cochrane}

\hypertarget{att-reports-customer-defections-as-americans-cut-back-in-the-pandemic}{%
\subsection{\texorpdfstring{\protect\hyperlink{att-reports-customer-defections-as-americans-cut-back-in-the-pandemic}{AT\&T
reports customer defections as Americans cut back in the
pandemic.}}{AT\&T reports customer defections as Americans cut back in the pandemic.}}\label{att-reports-customer-defections-as-americans-cut-back-in-the-pandemic}}

Copied to clipboard.

\includegraphics{https://static01.nyt.com/images/2020/07/23/business/23markets-brf-att/merlin_173339052_e9aa9413-58aa-4a13-b324-260923e65c23-articleLarge.jpg?quality=75\&auto=webp\&disable=upscale}

\textbf{AT\&T} reported second-quarter results Thursday, and the numbers
revealed customer defections across it businesses, including wireless,
broadband and satellite TV. The only bright spot appeared to be HBO Max,
but even there the numbers looked fuzzy. Here are the highlights:

\begin{itemize}
\item
  The wireless division saw a loss of 154,000 customers in its
  traditional service where people pay a monthly bill. The company added
  165,000 new customers to its prepaid plans. Factoring in resellers and
  other connected devices like tablets, AT\&T claims 171.4 million total
  wireless accounts, an 8 percent uptick from last year. The phone giant
  still has plenty of cash coming in the door, allowing it to pay down
  its enormous debt and shell out dividends.
\item
  The company's TV service, including satellite (formerly known as
  DirecTV) and its online cable platform, saw a net loss of 954,000
  customers, a downturn the company blamed on lower household spending
  because of the pandemic.
\item
  So what about HBO Max? The company's
  \href{https://www.nytimes.com/2020/05/26/business/media/hbo-max-netflix-streaming.html}{new
  streaming service} was introduced in late May, and in its first month
  it attracted 4.1 million customers. For some context, rival Disney+
  \href{https://twitter.com/edmundlee/status/1194661580539125762}{signed
  up 10 million} after the first 24 hours. AT\&T said it hoped to have
  50 million HBO Max customers by 2025.
\item
  To get there, AT\&T will need to convert more of its regular HBO
  customers to HBO Max customers. More than 23.5 million people who pay
  for regular HBO could switch to HBO Max.
\item
  What's making those numbers a bit hard to understand is that AT\&T
  sells two versions of HBO. Both cost the same, but HBO Max has much
  more content. AT\&T wants people to buy HBO Max or convert their
  existing HBO account to HBO Max. Because most people get HBO through
  their cable or satellite provider, AT\&T has already cut deals with
  most of them to allow their customers to
  \href{https://www.nytimes.com/article/hbo-max-amazon-roku.html}{make
  the switch}.
\item
  But that doesn't mean everyone has. Or can. Amazon, for example, sells
  HBO through its Amazon Channels service, but it hasn't been able to
  strike a deal with AT\&T to convert those people to HBO Max customers.
  AT\&T's chief executive,
  \href{https://www.nytimes.com/2020/04/24/business/media/att-ceo-john-stankey-randall-stephenson.html}{John
  Stankey}, had some harsh words for Amazon on the earnings call
  following the report. ``We've tried repeatedly to make HBO Max
  available on Amazon,'' he said, ``Unfortunately, Amazon has taken an
  approach of treating HBO Max and its customers differently'' than
  other services.
\item
  Customers may also be confused about HBO versus HBO Max, or may not be
  aware they can even make the switch --- a marketing challenge for
  AT\&T.
\item
  Revenue at the HBO division was down 5.2 percent to \$1.6 billion, a
  result of more people cutting their cable and satellite accounts. But
  AT\&T is spending more --- costs were up about a third to \$1.5
  billion --- to invest in HBO Max. Content is expensive.
\item
  The company's cable network division that includes CNN, TNT and TBS
  (where it airs sports programming) took a big hit to its advertising
  business with a 37 percent drop to \$796 million, the biggest
  shortfall across the media group.
\end{itemize}

--- \href{https://www.nytimes.com/by/edmund-lee}{Edmund Lee}

\hypertarget{another-retail-casualty-ann-taylors-parent-company-files-for-bankruptcy}{%
\subsection{\texorpdfstring{\protect\hyperlink{another-retail-casualty-ann-taylors-parent-company-files-for-bankruptcy}{Another
retail casualty: Ann Taylor's parent company files for
bankruptcy.}}{Another retail casualty: Ann Taylor's parent company files for bankruptcy.}}\label{another-retail-casualty-ann-taylors-parent-company-files-for-bankruptcy}}

Copied to clipboard.

\includegraphics{https://static01.nyt.com/images/2020/07/09/business/00virus-ascena/merlin_95485880_6c016d0c-35cb-426c-b980-98f6b53551bf-articleLarge.jpg?quality=75\&auto=webp\&disable=upscale}

The pandemic has taken a
\href{https://www.nytimes.com/2020/07/05/business/coronavirus-malls-department-stores-bankruptcy.html}{heavy
toll on retailers}, especially apparel sellers and other mall-based
chains that might have otherwise stayed afloat, perhaps even for a short
period, without turning to bankruptcy court.

The latest company to file for bankruptcy was
\textbf{\href{https://www.nytimes.com/2020/07/23/business/ascena-bankruptcy-ann-taylor-lane-bryant.html}{Ascena
Retail Group}}, which on Thursday became at least the ninth prominent
retailer to file for bankruptcy since early May, following
\href{https://www.nytimes.com/2020/07/08/business/brooks-brothers-chapter-11-bankruptcy.html}{Brooks
Brothers},
\href{https://www.nytimes.com/2020/05/07/business/neiman-marcus-bankruptcy.html}{Neiman
Marcus Group} and
\href{https://www.nytimes.com/2020/05/15/business/jc-penney-bankruptcy-coronavirus.html}{J.C.
Penney}, among others.

Ascena, just a few years ago was one of the country's largest clothing
retailers for women and girls, will close ``a select number'' of its Ann
Taylor, Lane Bryant, LOFT and Lou \& Grey stores, as well as all of its
Catherines locations.

Ascena was known for decades as Dress Barn, the clothing chain founded
in 1962 by Roslyn S. Jaffe, who noticed that there were few options for
stylish and affordable women's work attire even as more women were
entering the work force. Dress Barn went public in 1983, around the time
that the ``power suit'' came into vogue, exemplifying women's desire to
take on the predominantly male corporate world.

--- Gillian Friedman and
\href{https://www.nytimes.com/by/sapna-maheshwari}{Sapna Maheshwari}

\hypertarget{advertisement-2}{%
\subsubsection{Advertisement}\label{advertisement-2}}

\protect\hyperlink{after-dfp-ad-mid3}{Continue reading the main story}

\hypertarget{cobbled-together-jobs-leave-workers-without-a-safety-net-in-hard-times}{%
\subsection{\texorpdfstring{\protect\hyperlink{cobbled-together-jobs-leave-workers-without-a-safety-net-in-hard-times}{Cobbled-together
jobs leave workers without a safety net in hard
times.}}{Cobbled-together jobs leave workers without a safety net in hard times.}}\label{cobbled-together-jobs-leave-workers-without-a-safety-net-in-hard-times}}

Copied to clipboard.

\includegraphics{https://static01.nyt.com/images/2020/07/23/business/23virus-briefing-cobble/23VIRUS-COBBLE-COMBO-articleLarge.jpg?quality=75\&auto=webp\&disable=upscale}

Having multiple jobs is business as usual for millions of Americans. But
many cobbled-together employment arrangements that enabled people to get
by when the jobless rate was skimming along at record lows collapsed
once the pandemic curbed or closed large swaths of the economy.

And when hard times hit, they are excluded from regular state
unemployment benefits.

``There's a misfit between the enormous volatility and part-time jobs
that make up the ways that people cobble together making money and the
system that's going to cut you a check,''
said\href{https://ssa.uchicago.edu/ssascholars/s-lambert}{Susan J.
Lambert}, a professor at the University of Chicago who studies
low-skilled hourly jobs.

The economic shock quickly exposed the mismatch between the reality of
making a living in 2020 and the systems built to protect workers. People
who rely on paychecks from different employers are already more likely
to have shifting schedules and unpredictable weekly paychecks, low
hourly wages and the absence of benefits like sick days and health
insurance.

They are also more likely to
be\href{https://www.rsfjournal.org/content/5/4/218}{Black, young and
without a college degree.}

``The rules of the game have changed,'' Ms. Lambert said, but
protections for workers, like jobless benefits, have not caught up.

--- \href{https://www.nytimes.com/by/patricia-cohen}{Patricia Cohen}

\hypertarget{months-into-joblessness-some-still-struggle-to-get-benefits}{%
\subsection{\texorpdfstring{\protect\hyperlink{months-into-joblessness-some-still-struggle-to-get-benefits}{Months
into joblessness, some still struggle to get
benefits.}}{Months into joblessness, some still struggle to get benefits.}}\label{months-into-joblessness-some-still-struggle-to-get-benefits}}

Copied to clipboard.

\includegraphics{https://static01.nyt.com/images/2020/07/23/business/23markets-brf-backlog2/merlin_174399429_11c88dd0-b16b-4025-88a6-421e7151a36b-articleLarge.jpg?quality=75\&auto=webp\&disable=upscale}

States have been whittling away at the backlog of unemployment claims,
but persistent delays in some places have continued.

Behnaz Mansouri, an attorney at the Unemployment Law Project in
Washington State, said her office was still averaging 200 phone calls a
week from people who had received no benefits after waiting months, or
who had inexplicably had them cut off.

Recently there has been some slow progress, she said. A number of people
who had appealed a decision in March, April and May were beginning to be
called in for a hearing. Those who have waited the longest, Ms. Mansouri
said, are often those who have disabilities or don't speak English well.

In Oklahoma, hundreds of frustrated workers
\href{https://www.washingtonpost.com/national/a-very-dark-feeling-hundreds-camp-out-in-oklahoma-unemployment-lines/2020/07/20/44d59cb6-c77a-11ea-a99f-3bbdffb1af38_story.html}{camped
out overnight} hoping to sort out delays with their unemployment claims
at one of the large-scale processing sessions that officials were
holding around the state.

The pain of job losses can be found in every corner of the country, but
Black men have had particular difficulties, said
\href{https://www.gse.harvard.edu/faculty/peter-blair}{Peter Q. Blair},
a co-director of the Project on Workforce at the Harvard Graduate School
of Education.

The government's
\href{https://www.bls.gov/news.release/empsit.t02.htm}{June jobs report}
showed that although unemployment for every other group declined from
May, the rate for African-American males over 20 rose to 15.8 percent
from 15.3 percent.

``It's important that we look at the way in which this crisis is having
a disparate effect on the African-American community, particularly Black
men,'' he said.

--- \href{https://www.nytimes.com/by/patricia-cohen}{Patricia Cohen}

\hypertarget{landlords-jump-the-gun-as-the-eviction-moratorium-wanes}{%
\subsection{\texorpdfstring{\protect\hyperlink{landlords-jump-the-gun-as-the-eviction-moratorium-wanes}{Landlords
jump the gun as the eviction moratorium
wanes.}}{Landlords jump the gun as the eviction moratorium wanes.}}\label{landlords-jump-the-gun-as-the-eviction-moratorium-wanes}}

Copied to clipboard.

\includegraphics{https://static01.nyt.com/images/2020/07/22/business/23virus-briefing-evictions/merlin_174815295_11edc9e6-4c9a-49c7-a279-7256ef3c6958-articleLarge.jpg?quality=75\&auto=webp\&disable=upscale}

The four-month pause that has protected millions of Americans from
eviction cases is set to expire at the end of this week. But that hasn't
stopped landlords across the country from trying to get a head start
forcing renters out,
\href{https://www.nytimes.com/2020/07/23/business/evictions-moratorium-cares-act.html}{reports
Matthew Goldstein}.

\begin{quote}
Landlords in Tucson, Ariz., filed dozens of eviction cases last month
despite the federal moratorium, which was put in place because of the
coronavirus crisis. Legal aid lawyers had to go to court to stop the
eviction of a San Antonio renter who had lost her job during a citywide
stay-at-home order. And in
\href{https://www.nhlp.org/wp-content/uploads/Douglas-County-Order-of-Dismissal.pdf}{Omaha},
a court found that a struggling renter's attempted eviction had violated
the emergency law.

As the number of Covid-19 cases has
\href{https://www.nytimes.com/interactive/2020/us/coronavirus-us-cases.html}{surged
across the country}, a disturbing trend has emerged: landlords
commencing eviction proceedings even though the CARES Act relief law
currently protects about 12 million tenants living in qualifying
properties.

State and local governments have also issued eviction moratoriums, but
the CARES Act is the furthest reaching, covering as many as 12.3 million
renters living in an apartment complex or single-family home financed
with a federally backed mortgage. But like other moratoriums, it's about
to expire: After Friday, landlords can begin filing eviction notices for
failure to pay rent. It will be at least 30 days after that before any
tenants are kicked out.
\end{quote}

\hypertarget{advertisement-3}{%
\subsubsection{Advertisement}\label{advertisement-3}}

\protect\hyperlink{after-dfp-ad-mid4}{Continue reading the main story}

\hypertarget{a-slide-in-big-technology-shares-drags-wall-street-lower}{%
\subsection{\texorpdfstring{\protect\hyperlink{a-slide-in-big-technology-shares-drags-wall-street-lower}{A
slide in big technology shares drags Wall Street
lower.}}{A slide in big technology shares drags Wall Street lower.}}\label{a-slide-in-big-technology-shares-drags-wall-street-lower}}

Copied to clipboard.

\includegraphics{https://static01.nyt.com/images/2020/07/23/world/23markets-brf-markets/merlin_174795525_2655ea3c-18e0-41d0-9885-db89454af95f-articleLarge.jpg?quality=75\&auto=webp\&disable=upscale}

Stocks on Wall Street tumbled on Thursday, with shares of large
technology companies leading the decline, as investors considered the
latest earnings reports and a rise in unemployment claims by workers in
the United States.

The S\&P 500 fell more than 1 percent in the afternoon, while the
tech-heavy Nasdaq composite was down by more than 2 percent, as shares
of \textbf{Apple}, \textbf{Microsoft}, and \textbf{Alphabet} all slid.
Large technology companies have a outsize influence on the stock market
indexes because of their sheer size.

Microsoft tumbled more than 4 percent even after reporting earnings late
on Wednesday that were better than Wall Street had expected. Apple was
one of the worst performing stocks on the S\&P 500 with a decline of
more than 4.5 percent.

Wall Street's negative tone was also affected by news that more than 1.4
million workers filed new claims for
\href{https://www.dol.gov/ui/data.pdf}{state unemployment benefits} last
week, the first time in more than three months that the weekly tally has
risen. The rise in claims comes as a surge in coronavirus cases around
the United States has prompted some states to reinstate restrictions on
public gatherings and close bars and restaurants again.

Investors have shaken off concerns about the impact the lockdowns might
have on the economy, in part because lawmakers in Washington are in the
middle of negotiations over a spending plan that will help offset some
of the damage. On Wednesday, Senate Republican leaders and White House
officials said that they
\href{https://www.nytimes.com/2020/07/22/us/politics/coronavirus-stimulus.html?action=click\&module=Top\%20Stories\&pgtype=Homepage}{had
reached an agreement in principle} on a proposal to give more than \$100
billion to schools, send additional checks directly to Americans and
provide \$16 billion for states to conduct testing and contact tracing.

But on Thursday, the negative tone spread to other markets as well.
Crude oil prices, for example, which can reflect expectations for the
economy, were also lower.

--- \href{https://www.nytimes.com/by/kevin-granville}{Kevin Granville}
and Mohammed Hadi

\hypertarget{catch-up-buzzfeed-lays-off-workers-dyson-and-dow-plan-job-cuts-and-more}{%
\subsection{\texorpdfstring{\protect\hyperlink{catch-up-buzzfeed-lays-off-workers-dyson-and-dow-plan-job-cuts-and-more}{Catch
up: Buzzfeed lays off workers, Dyson and Dow plan job cuts, and
more.}}{Catch up: Buzzfeed lays off workers, Dyson and Dow plan job cuts, and more.}}\label{catch-up-buzzfeed-lays-off-workers-dyson-and-dow-plan-job-cuts-and-more}}

Copied to clipboard.

\begin{itemize}
\item
  🤳🏻 \textbf{BuzzFeed} said it had laid off 50 employees, including 10
  people in its news division.The employees were among a group of 74 who
  were furloughed in the spring, the company said. Buzzfeed cut the
  salaries of its U.S. workers earning more than \$40,000, and top
  executives had their pay reduced by 25 percent. Jonah Peretti, the
  company's chief executive, said he would forgo his salary until the
  crisis passed. Employees will get between four and eight weeks of
  severance, and BuzzFeed will pay for insurance through September 30, a
  spokesman said.
\item
  🧹 The British technology and manufacturing company \textbf{Dyson}
  announced plans on Thursday to cut 900 jobs, about 6 percent of its
  global staff. Two-thirds of the job losses will be in Britain and
  apply to people in retail and customer service roles. The company said
  it hadn't put any staff on furlough during the pandemic or used
  government money in any of the countries it operates in.
\item
  🏦 Senator Mitt Romney, Republican of Utah, confirmed that he would
  oppose Judy Shelton's nomination to the Federal Reserve Board. It
  remains unclear whether Ms. Shelton, an unorthodox economist with
  close ties to the Trump administration, has enough votes to clear the
  full Senate. The Senate Banking Committee
  \href{https://www.nytimes.com/2020/07/21/business/economy/shelton-federal-reserve-trump-senate.html}{voted
  to advance} Ms. Shelton's nomination this week, as several Republicans
  who had once expressed doubts about her suitability decided to vote in
  her favor. Now, she and fellow Fed nominee Christopher Waller must win
  a simple majority vote in the full Senate to gain confirmation.
\item
  ✈️ \textbf{Southwest Airlines} and \textbf{American Airlines} on
  Thursday reported deep losses in the second quarter of the year, as a
  meek rebound in travel slowed in recent weeks with the spread of
  coronavirus infections and travel restrictions nationwide. For
  Southwest, revenue declined 83 percent from the same quarter last
  year, resulting in an overall loss of \$915 million. American saw
  revenue fall 86 percent, giving way to a \$2 billion loss. The losses
  represent a reversal of fortune for both airlines, which earned
  hundreds of millions of dollars in profit during the same three months
  in 2019.
\item
  🧪 \textbf{Dow}, the American chemicals company, announced plans to cut
  6 percent of its global work force, nearly 2,200 jobs, as it published
  its second quarter earnings on Thursday. The Michigan-based company
  said it aimed to save \$500 million this year, increasing the previous
  target from \$350 million. Its sales declined 24 percent in the second
  quarter, compared with last year.
\end{itemize}

\hypertarget{site-index}{%
\subsection{Site Index}\label{site-index}}

\hypertarget{site-information-navigation}{%
\subsection{Site Information
Navigation}\label{site-information-navigation}}

\begin{itemize}
\tightlist
\item
  \href{https://help.nytimes.com/hc/en-us/articles/115014792127-Copyright-notice}{©~2020~The
  New York Times Company}
\end{itemize}

\begin{itemize}
\tightlist
\item
  \href{https://www.nytco.com/}{NYTCo}
\item
  \href{https://help.nytimes.com/hc/en-us/articles/115015385887-Contact-Us}{Contact
  Us}
\item
  \href{https://www.nytco.com/careers/}{Work with us}
\item
  \href{https://nytmediakit.com/}{Advertise}
\item
  \href{http://www.tbrandstudio.com/}{T Brand Studio}
\item
  \href{https://www.nytimes.com/privacy/cookie-policy\#how-do-i-manage-trackers}{Your
  Ad Choices}
\item
  \href{https://www.nytimes.com/privacy}{Privacy}
\item
  \href{https://help.nytimes.com/hc/en-us/articles/115014893428-Terms-of-service}{Terms
  of Service}
\item
  \href{https://help.nytimes.com/hc/en-us/articles/115014893968-Terms-of-sale}{Terms
  of Sale}
\item
  \href{https://spiderbites.nytimes.com}{Site Map}
\item
  \href{https://help.nytimes.com/hc/en-us}{Help}
\item
  \href{https://www.nytimes.com/subscription?campaignId=37WXW}{Subscriptions}
\end{itemize}
