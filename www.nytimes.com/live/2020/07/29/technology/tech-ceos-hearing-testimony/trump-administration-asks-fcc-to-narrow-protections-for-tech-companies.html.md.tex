Sections

SEARCH

\protect\hyperlink{site-content}{Skip to
content}\protect\hyperlink{site-index}{Skip to site index}

\href{https://myaccount.nytimes.com/auth/login?response_type=cookie\&client_id=vi}{}

\href{https://www.nytimes.com/section/todayspaper}{Today's Paper}

Last Updated

July 29, 2020, 6:44 p.m. ET

July 29, 2020, 6:44 p.m. ET

\hypertarget{lawmakers-from-both-sides-take-aim-at-big-tech-executives}{%
\section{Lawmakers From Both Sides Take Aim at Big Tech
Executives}\label{lawmakers-from-both-sides-take-aim-at-big-tech-executives}}

\hypertarget{heres-what-you-need-to-know}{%
\subsubsection{Here's what you need to
know:}\label{heres-what-you-need-to-know}}

\begin{itemize}
\item
  \protect\hyperlink{democrats-dug-in-with-accusations-of-anticompetitive-behavior}{}

  Democrats dug in with accusations of anticompetitive behavior.
\item
  \protect\hyperlink{republicans-focused-on-bias-concerns-about-platforms}{}

  Republicans focused on bias concerns about platforms.
\item
  \protect\hyperlink{heres-which-tech-ceo-was-asked-the-most-questions-by-lawmakers}{}

  Here's which tech C.E.O. was asked the most questions by lawmakers.
\item
  \protect\hyperlink{questions-on-google-focused-on-its-search-engine-and-relationship-with-pentagon}{}

  Questions on Google focused on its search engine and relationship with
  Pentagon.
\item
  \protect\hyperlink{lawmakers-said-documents-show-facebook-tried-to-neutralize-a-competitive-threat}{}

  Lawmakers said documents show Facebook tried to neutralize a
  ``competitive threat.''
\item
  \protect\hyperlink{what-ceos-said}{}

  Here's a final tally of the C.E.O.s' catchphrases.
\item
  \protect\hyperlink{what-to-expect-from-the-hearing}{}

  What to expect from the hearing.
\end{itemize}

\hypertarget{democrats-dug-in-with-accusations-of-anticompetitive-behavior}{%
\subsection{\texorpdfstring{\protect\hyperlink{democrats-dug-in-with-accusations-of-anticompetitive-behavior}{Democrats
dug in with accusations of anticompetitive
behavior.}}{Democrats dug in with accusations of anticompetitive behavior.}}\label{democrats-dug-in-with-accusations-of-anticompetitive-behavior}}

Copied to clipboard.

Democrats on the committee quickly dug into the issue of competition,
citing documents obtained from inside the tech companies that they said
showed their anti-competitive conduct.

Representative David Cicilline, the chairman of the antitrust
subcommittee, grilled Sundar Pichai, Google's chief executive, about how
Google steers traffic to its own search pages and products.
Representative Jerrold Nadler of New York asked Mark Zuckerberg,
Facebook's top executive, about emails he wrote describing Instagram as
a potentially disruptive competitor before the company acquired the
firm. And Representative Hank Johnson of Georgia pushed Tim Cook of
Apple on whether his company exerts unfair dominance over app developers
in its app store.

In multiple cases, the chief executives evaded the questions, claiming
not to know specifics about the documents or interactions in question.

Mr. Cicilline, who has led the investigation into the tech giants for
more than a year, opened the hearing with a broadside against the
companies, saying their dominance harms the economy and leaves consumers
with no choice but to use their products.

``Any single action by one of these companies can affect hundreds of
millions of us in profound and lasting ways,'' Mr. Cicilline, a Rhode
Island Democrat, said in his opening statement. ``Simply put: They have
too much power.''

Mr. Cicilline is in control of many aspects of the hearing, including
how many rounds of questions lawmakers get. That may allow him to extend
lines of questioning in an attempt to dig deeper than an initial five
minutes allows.

Mr. Cicilline, who used to be the mayor of Providence,
\href{https://slack-redir.net/link?url=https\%3A\%2F\%2Fwww.nytimes.com\%2F2019\%2F12\%2F08\%2Ftechnology\%2FDavid-Cicilline-antitrust-tech.html}{has
become a prominent foe} of the technology platforms from his perch as
the top Democrat on the once-quiet subcommittee. For more than a year,
his staff has led the investigation, conducting hundreds of hours of
interviews and collecting 1.3 million documents. The team has grown to
include Lina Khan, a legal scholar who wrote a major law review note on
Amazon's power, and Phillip Berenbroick, previously the policy director
at the consumer group Public Knowledge.

Mr. Cicilline spent recent months negotiating to secure the appearance
by the chief executives. The process was not always friendly. When the
committee demanded that Mr. Bezos testify, Amazon responded with a
noncommittal letter. Mr. Cicilline threatened to subpoena Mr. Bezos
before the company agreed to make him available to answer the panel's
questions.

``Our founders would not bow before a king,'' Mr. Cicilline said
Wednesday. ``Nor should we bow before the emperors of the online
economy.''

--- \href{https://www.nytimes.com/by/david-mccabe}{David McCabe}

\hypertarget{republicans-focused-on-bias-concerns-about-platforms}{%
\subsection{\texorpdfstring{\protect\hyperlink{republicans-focused-on-bias-concerns-about-platforms}{Republicans
focused on bias concerns about
platforms.}}{Republicans focused on bias concerns about platforms.}}\label{republicans-focused-on-bias-concerns-about-platforms}}

Copied to clipboard.

Republicans spent much of their time steering away from antitrust and
competition, instead asking about the tech giants' efforts in China and
unproven claims that the companies suppress conservative views.

Representative Jim Jordan of Ohio, the top Republican on the Judiciary
Committee, spent his opening statements listing anecdotes where
Republican officials had been subject to enforcement actions by the
platforms' rules. (He did not mention that conservative publications and
figures routinely rank among the top performing pages on Facebook and
other platforms.)

``I'll just cut to the chase, Big Tech's out to get conservatives,''
said Mr. Jordan. He later asked Mr. Pichai whether or not Google would
take efforts to help Democratic presidential nominee Joseph R. Biden Jr.

The claims about conservative bias are a persistent, if largely
unproven, gripe among Republicans. President Trump, Attorney General
William P. Barr and lawmakers like Mr. Jordan and Senator Ted Cruz of
Texas have all raised concerns that Facebook, Twitter and YouTube
purposely downplay or remove conservative voices on their sites.

The suspicions rise from the accurate perception that Silicon Valley is
dominated by liberal-leaning workers. In November 2018, Facebook removed
an ad by an anti-abortion group endorsing Senator Marsha Blackburn,
Republican of Tennessee. Facebook said it did so because an image on the
ad that appeared to violate its community norms. That example and others
have fueled suspicion of conservative censorship.

Mr. Trump recently issued an executive order curtailing safe harbors for
internet companies in retaliation against his perceptions of bias. The
order was issued after
\href{https://slack-redir.net/link?url=https\%3A\%2F\%2Fwww.nytimes.com\%2F2020\%2F05\%2F26\%2Ftechnology\%2Ftwitter-trump-mail-in-ballots.html}{Twitter
labeled} a set of his tweets in late May for misinformation.

Representative Matt Gaetz, a Republican from Florida, asked the chief
executive of Google about the company's decision to drop a Defense
Department project after employees expressed concerns.

When they were asking about antitrust, some Republicans expressed
caution about tighter regulation.

``Big isn't inherently bad,'' said Representative James Sensenbrenner, a
Wisconsin Republican. Representative Ken Buck, a Republican of Colorado,
said: ``Our witnesses have taken ideas born out of a dorm room, a
garage. You have enjoyed the freedom to succeed. I do not believe big is
necessarily bad. In fact, big is often a force for good.''

--- \href{https://www.nytimes.com/by/david-mccabe}{David McCabe} and
\href{https://www.nytimes.com/by/cecilia-kang}{Cecilia Kang}

\hypertarget{advertisement}{%
\subsubsection{Advertisement}\label{advertisement}}

\protect\hyperlink{after-dfp-ad-mid1}{Continue reading the main story}

\hypertarget{heres-which-tech-ceo-was-asked-the-most-questions-by-lawmakers}{%
\subsection{\texorpdfstring{\protect\hyperlink{heres-which-tech-ceo-was-asked-the-most-questions-by-lawmakers}{Here's
which tech C.E.O. was asked the most questions by
lawmakers.}}{Here's which tech C.E.O. was asked the most questions by lawmakers.}}\label{heres-which-tech-ceo-was-asked-the-most-questions-by-lawmakers}}

Copied to clipboard.

We tracked which tech C.E.O. was under the most scrutiny by tallying the
number of questions they are getting asked. The results were striking.

Executive

Number of Questions

Mark Zuckerberg

62

\begin{verbatim}
                        background: #7f3636
                    ">
\end{verbatim}

62

Jeff Bezos

59

\begin{verbatim}
                        background: #7f3636
                    ">
\end{verbatim}

59

Tim Cook

35

\begin{verbatim}
                        background: #7f3636
                    ">
\end{verbatim}

35

Sundar Pichai

61

\begin{verbatim}
                        background: #7f3636
                    ">
\end{verbatim}

61

--- \href{https://www.nytimes.com/by/kellen-browning}{Kellen Browning}

\hypertarget{questions-on-google-focused-on-its-search-engine-and-relationship-with-pentagon}{%
\subsection{\texorpdfstring{\protect\hyperlink{questions-on-google-focused-on-its-search-engine-and-relationship-with-pentagon}{Questions
on Google focused on its search engine and relationship with
Pentagon.}}{Questions on Google focused on its search engine and relationship with Pentagon.}}\label{questions-on-google-focused-on-its-search-engine-and-relationship-with-pentagon}}

Copied to clipboard.

\includegraphics{https://static01.nyt.com/images/2020/07/29/business/29vid-tech-hearing-1/29vid-tech-hearing-1-videoSixteenByNine3000.jpg}

Sundar Pichai, chief executive of Google's parent company Alphabet, was
a consistent target of aggressive questions --- surprising, given that
he had testified a year ago and that he has the lowest profile of the
executives testifying Wednesday --- about its search engine and the
company's decision to withdraw from a Pentagon project after employee
protests about the work.

Representative David Cicilline, the chairman of the antitrust
subcommittee, accused Google of lifting content from other websites to
keep users within what he called the ``walled garden'' of its search
engine in order to make more money from advertising.

``The evidence seems very clear to me as Google became the gateway to
the internet, it began to abuse its power and use its surveillance over
the web traffic to identify competitive threats and crush them,'' Mr.
Cicilline said.

Mr. Pichai disagreed with that characterization and fell back on the
company's talking points that Google search has lots of competitors for
specific categories, such as Amazon in shopping. He also said that the
majority of Google's search results did not carry ads and that it was
acting in the best interest of users when it highlights answers to
queries. Google's global market share in search is 92 percent, according
to data from Statcounter, an online research tool.

Because Google is so dominant, other websites rely on the search engine
for traffic. In recent years, the Silicon Valley giant has started to
devote real estate at the top of search results to providing its own
answers for information about local businesses, flights and hotels. This
has angered other websites whose traffic has slid as Google surfaces
more information on its own search results.

Republicans zeroed in on
\href{https://slack-redir.net/link?url=https\%3A\%2F\%2Fwww.nytimes.com\%2F2018\%2F06\%2F01\%2Ftechnology\%2Fgoogle-pentagon-project-maven.html}{Google
pulling out of an effort} to help the Pentagon build technology systems
to analyze drone footage to identify particular objects like buildings,
vehicles and people. Google's employees protested the company's work on
the project.

Representatives Ken Buck from Colorado and Matt Gaetz from Florida, two
of the Republicans on the panel, questioned why Google pulled back from
the Pentagon, while continuing to operate an artificial intelligence lab
in China. Mr. Pichai denied one of their accusations that Google still
works with the Chinese military and noted that the company still worked
with the U.S. military, including a cybersecurity project with the
Defense Department.

--- \href{https://www.nytimes.com/by/daisuke-wakabayashi}{Daisuke
Wakabayashi}

\hypertarget{lawmakers-said-documents-show-facebook-tried-to-neutralize-a-competitive-threat}{%
\subsection{\texorpdfstring{\protect\hyperlink{lawmakers-said-documents-show-facebook-tried-to-neutralize-a-competitive-threat}{Lawmakers
said documents show Facebook tried to neutralize a ``competitive
threat.''}}{Lawmakers said documents show Facebook tried to neutralize a ``competitive threat.''}}\label{lawmakers-said-documents-show-facebook-tried-to-neutralize-a-competitive-threat}}

Copied to clipboard.

\includegraphics{https://static01.nyt.com/images/2020/07/29/business/29tech-hearing-hotdocs/merlin_175079946_aebc7771-9252-4f51-b55d-0d8de1c69480-articleLarge.jpg?quality=75\&auto=webp\&disable=upscale}

The House judiciary antitrust subcommittee has said it gathered 1.3
million documents about Facebook, Google, Amazon and Apple over the
course of its 13-month investigation into the power of the companies'
businesses. At the hearing, lawmakers began
\href{https://twitter.com/HouseJudiciary/status/1288540745637474306?s=20}{rolling
some of those documents out}.

Several of the documents were about Facebook and the desire of its chief
executive, Mark Zuckerberg, to buy the photo-sharing app Instagram as a
way of quashing a competitive threat. The social network
\href{https://dealbook.nytimes.com/2012/04/09/facebook-buys-instagram-for-1-billion/\#:~:text=Facebook\%20is\%20not\%20waiting\%20for,stock\%2C\%20the\%20company\%20said\%20Monday.}{bought
Instagram in 2012} for about \$1 billion in cash and stock.

In those documents, which were reviewed by The New York Times, Mr.
Zuckerberg pressed Kevin Systrom, a co-founder of Instagram, to submit
to Facebook's original acquisition offer of \$500 million. In other
correspondence, Facebook's chief financial officer at the time
specifically pointed at Instagram as a ``competitive threat'' that
needed to be dealt with.

The documents were evidence that Facebook viewed Instagram as a
``powerful threat that could siphon business away from Facebook,''
Representative Jim Sensenbrenner, Republican of Wisconsin, said in the
hearing. ``Rather than compete with it, Facebook bought it.''

In response, Mr. Zuckerberg said that while it seemed in hindsight that
Instagram's success was an inevitability, it was far from certain at the
time. Instagram had many competitors at that point, he said, including
now defunct start-ups such as Path.

``The acquisition has done wildly well not just because of the founders'
talent, but because we invested heavily in building up the
infrastructure and promoting it,'' Mr. Zuckerberg said. ``And I think
that this has been an American success story.''

--- \href{https://www.nytimes.com/by/mike-isaac}{Mike Isaac}

\hypertarget{advertisement-1}{%
\subsubsection{Advertisement}\label{advertisement-1}}

\protect\hyperlink{after-dfp-ad-mid2}{Continue reading the main story}

\hypertarget{tim-cook-pressed-on-apples-app-store}{%
\subsection{\texorpdfstring{\protect\hyperlink{tim-cook-pressed-on-apples-app-store}{Tim
Cook pressed on Apple's App
Store.}}{Tim Cook pressed on Apple's App Store.}}\label{tim-cook-pressed-on-apples-app-store}}

Copied to clipboard.

\includegraphics{https://static01.nyt.com/images/2020/07/29/business/29tech-hearing-apple/29tech-hearing-apple-articleLarge.jpg?quality=75\&auto=webp\&disable=upscale}

In the hearing's first several hours, Apple's Tim Cook was largely
ignored. In the last hour, he found himself on the defense.

First, Representatives Val Demings and Lucy Kay McBath, Democrats from
Florida and Georgia, needled him on why Apple removed parental-control
apps shortly after Apple introduced its own competing tool in 2018.
\href{https://www.nytimes.com/2019/04/27/technology/apple-screen-time-trackers.html}{The
Times reported about the removal of the apps last year}.

Mr. Cook said Apple pulled the apps because of privacy concerns, not
competition. Ms. McBath then pointed to an email that appeared to show a
top Apple executive, Phil Schiller, telling a concerned parent that they
could now use Apple's parental-control tool instead. Mr. Cook said he
could not see the email on his screen.

Mr. Cook then found himself defending Apple's recent demands to collect
a commission from Airbnb and ClassPass after the companies shifted to
selling virtual classes because of the pandemic,
\href{https://www.nytimes.com/2020/07/28/technology/apple-app-store-airbnb-classpass.html}{as
The Times reported this week}.

Representative Jerrold Nadler, Democrat of New York, asked: ``Isn't this
pandemic profiteering?''

Mr. Cook responded that Apple's rules require companies that sell
digital services to pay Apple's commission, but that Apple was working
with companies that had made business changes because of the pandemic.

Apple told The Times this week that it was still negotiating with Airbnb
and ClassPass on the fees. Earlier this month, ClassPass pulled its
virtual classes from its iPhone app because Apple told the company that
its deadline for complying with the rule had passed, according to a
person close to ClassPass who spoke on the condition of anonymity to
discuss private negotiations.

Apple faces accusations that it arbitrarily enforces its rules on app
developers, killing some of their businesses on a whim.

In Tim Cook's opening statement, he said that Apple's App Store rules
are ``applied equally to every developer.''

Earlier in the hearing, Democrats on the House antitrust subcommittee
tried to show that wasn't true.

Representative Hank Johnson, Democrat of Georgia, asked if Baidu, the
Chinese search giant, got special treatment. Mr. Cook responded that he
wasn't sure. The committee then released documents that appeared to show
Mr. Cook telling Baidu's chief executive in a 2014 email that Baidu
would be on an ``app review fast track'' and that two employees would
help manage the process.

\begin{quote}
Documents from the Hearing on ``Online Platforms and Market Power:
Examining the Dominance of Amazon, Apple, Facebook and Google"
\href{https://t.co/E8auYYSeMn}{pic.twitter.com/E8auYYSeMn}

--- House Judiciary Dems (@HouseJudiciary)
\href{https://twitter.com/HouseJudiciary/status/1288543144158597124?ref_src=twsrc\%5Etfw}{July
29, 2020}
\end{quote}

Mr. Johnson later pointed out that Apple now lets Amazon avoid Apple's
30 percent commission on its video-streaming service --- one of the main
complaints against Apple by developers --- in exchange for making Amazon
and Apple products work better together. Mr. Cook responded that any
other company could get the same deal.

Mr. Cook argued that Apple had to treat app developers fairly and had to
be competitive in the commission it charges. ``We have fierce
competition at the developer side and the customer side,'' he said.
``It's so competitive, I would describe it as a street fight for market
share in the smartphone business.''

In reality, the market for smartphone software is a clear duopoly. Apple
and Google make the software that underpins virtually every smartphone
in the world.

--- \href{https://www.nytimes.com/by/jack-nicas}{Jack Nicas}

\hypertarget{bezos-got-pushed-on-amazons-relationship-with-third-party-sellers}{%
\subsection{\texorpdfstring{\protect\hyperlink{bezos-got-pushed-on-amazons-relationship-with-third-party-sellers}{Bezos
got pushed on Amazon's relationship with third-party
sellers.}}{Bezos got pushed on Amazon's relationship with third-party sellers.}}\label{bezos-got-pushed-on-amazons-relationship-with-third-party-sellers}}

Copied to clipboard.

\includegraphics{https://static01.nyt.com/images/2020/07/29/business/29vid-tech-hearing-2/29vid-tech-hearing-2-videoSixteenByNineJumbo1600-v2.jpg}

Facing Congress for the first time, Jeff Bezos, Amazon's founder and
chief executive, was forced to defend one of Amazon's sources of great
pride: its relationship with the many third-party sellers whose products
fill its online store.

At the start of the hearing, Mr. Bezos introduced himself as a lucky and
humble example of the success of American democracy --- the son of a
plucky mother and a supportive, immigrant father who ``fostered my
curiosity and encouraged me to dream big.'' He said he brought that
ethos to Amazon, saying the company's growth has benefited Americans.

``Customer obsession has driven our success,'' he said.

But once Mr. Bezos faced questions, almost entirely from Democrats, he
had to respond to whether Amazon harms the sellers whose products make
up about 60 percent of its sales.

Representative Pramila Jayapal, a Democrat whose district includes
Amazon's Seattle headquarters, said former Amazon employees told the
committee that employees treat proprietary seller data like ``a candy
shop'' they can mine to develop products Amazon's own, competing house
brand. Representative Lucy McBath said that, when sellers talked to the
committee, ``they use the words like bullying, fear, and panic to
describe their relationship with Amazon.'' Representative David
Cicilline, the chairman of the antitrust subcommittee, said one seller
compared Amazon to a drug dealer.

Mr. Bezos said he disagreed with Mr. Cicilline's characterization, and
he told Ms. McBath that ``third party sellers in aggregate are doing
extremely well on Amazon.'' He said to Ms. Jayapal that Amazon has ``a
policy against using seller-specific data to aid our private label
business, but I can't guarantee you that that policy has never been
violated.''

He several times said that sellers have benefited from Amazon's growth
and investment. He said when Amazon decided two decades ago to invite
third-party sellers to offer products on its retail website, Amazon
thought that more selection would let both Amazon and the sellers
thrive.

When faced with data from Mr. Cicilline that Amazon controls 75 percent
of all online marketplace sales, Mr. Bezos responded, ``With great
respect, I do have a different opinion on that.'' He said that sellers
have ``a lot of options. ``I believe Amazon is a great one and we have
worked very hard,'' he said. ``I think we are the best one.''

Ms. Jayapal pushed Mr. Bezos on what happens to employees who violate
its internal policies. She said Amazon has ``access to data that far
exceeds the sellers on your platforms with whom you compete,'' such as
how many shoppers looked at an item but did not buy it.

Mr. Bezos began responding that he was ``very proud of what we have done
for third-party sellers on this platform,'' before Ms. Jayapal cut him
off saying she was running out of time.

--- \href{https://www.nytimes.com/by/karen-weise}{Karen Weise}

\hypertarget{heres-a-final-tally-of-the-ceos-catchphrases}{%
\subsection{\texorpdfstring{\protect\hyperlink{what-ceos-said}{Here's a
final tally of the C.E.O.s'
catchphrases.}}{Here's a final tally of the C.E.O.s' catchphrases.}}\label{heres-a-final-tally-of-the-ceos-catchphrases}}

Copied to clipboard.

How often do tech titans repeat themselves? How many times did the chief
executives fall back on buzzwords and catchphrases? And how frequently
did they bring up their rivals (TikTok! Walmart! Each other!) to
downplay their companies' power?

To answer these questions, we kept track of how often Jeff Bezos of
Amazon, Sundar Pichai of Google, Tim Cook of Apple and Mark Zuckerberg
of Facebook used certain arguments and phrases throughout the course of
the antitrust hearing.

\hypertarget{we-are-not-that-big}{%
\subsubsection{We Are Not That Big}\label{we-are-not-that-big}}

\hypertarget{each-time-a-ceo-argued-that-his-company-was-not-actually-that-powerful-because-its-market-share-was-small-or-its-influence-was-limited}{%
\paragraph{Each time a C.E.O. argued that his company was not actually
that powerful because its market share was small or its influence was
limited.}\label{each-time-a-ceo-argued-that-his-company-was-not-actually-that-powerful-because-its-market-share-was-small-or-its-influence-was-limited}}

\begin{longtable}[]{@{}ll@{}}
\toprule
\textbf{On Repeat} & \textbf{Count}\tabularnewline
\midrule
\endhead
Mark Zuckerberg & 2\tabularnewline
Jeff Bezos & 1\tabularnewline
Tim Cook & 3\tabularnewline
Sundar Pichai & 3\tabularnewline
\bottomrule
\end{longtable}

\hypertarget{we-are-good-for-america}{%
\subsubsection{We Are Good for America}\label{we-are-good-for-america}}

\hypertarget{each-time-a-ceo-boasted-about-how-his-company-has-added-jobs-fueled-economic-growth-accelerated-innovation-or-otherwise-helped-the-country}{%
\paragraph{Each time a C.E.O. boasted about how his company has added
jobs, fueled economic growth, accelerated innovation or otherwise helped
the
country.}\label{each-time-a-ceo-boasted-about-how-his-company-has-added-jobs-fueled-economic-growth-accelerated-innovation-or-otherwise-helped-the-country}}

\begin{longtable}[]{@{}ll@{}}
\toprule
\textbf{On Repeat} & \textbf{Count}\tabularnewline
\midrule
\endhead
Mark Zuckerberg & 4\tabularnewline
Jeff Bezos & 3\tabularnewline
Tim Cook & 6\tabularnewline
Sundar Pichai & 11\tabularnewline
\bottomrule
\end{longtable}

\hypertarget{we-will-get-back-to-you}{%
\subsubsection{We Will Get Back to You}\label{we-will-get-back-to-you}}

\hypertarget{each-time-a-ceo-didnt-directly-answer-a-question-saying-instead-that-he-would-respond-after-the-company-looked-into-the-matter}{%
\paragraph{Each time a C.E.O. didn't directly answer a question, saying
instead that he would respond after the company looked into the
matter.}\label{each-time-a-ceo-didnt-directly-answer-a-question-saying-instead-that-he-would-respond-after-the-company-looked-into-the-matter}}

\begin{longtable}[]{@{}ll@{}}
\toprule
\textbf{On Repeat} & \textbf{Count}\tabularnewline
\midrule
\endhead
Mark Zuckerberg & 7\tabularnewline
Jeff Bezos & 3\tabularnewline
Tim Cook & 3\tabularnewline
Sundar Pichai & 13\tabularnewline
\bottomrule
\end{longtable}

\hypertarget{we-are-not-the-ones-to-worry-about}{%
\subsubsection{We Are Not the Ones to Worry
About}\label{we-are-not-the-ones-to-worry-about}}

\hypertarget{each-time-a-ceo-tried-to-shift-attention-by-citing-a-competitor-or-the-specter-of-how-china-could-dominate-tech-if-their-own-companies-were-curtailed}{%
\paragraph{Each time a C.E.O. tried to shift attention by citing a
competitor or the specter of how China could dominate tech if their own
companies were
curtailed.}\label{each-time-a-ceo-tried-to-shift-attention-by-citing-a-competitor-or-the-specter-of-how-china-could-dominate-tech-if-their-own-companies-were-curtailed}}

\begin{longtable}[]{@{}ll@{}}
\toprule
\textbf{On Repeat} & \textbf{Count}\tabularnewline
\midrule
\endhead
Mark Zuckerberg & 6\tabularnewline
Jeff Bezos & 10\tabularnewline
Tim Cook & 13\tabularnewline
Sundar Pichai & 8\tabularnewline
\bottomrule
\end{longtable}

--- \href{https://www.nytimes.com/by/kellen-browning}{Kellen Browning}

\hypertarget{advertisement-2}{%
\subsubsection{Advertisement}\label{advertisement-2}}

\protect\hyperlink{after-dfp-ad-mid3}{Continue reading the main story}

\hypertarget{tech-executives-looked-like-they-work-in-well-tech-offices}{%
\subsection{\texorpdfstring{\protect\hyperlink{tech-executives-looked-like-they-work-in-well-tech-offices}{Tech
executives looked like they work in, well, tech
offices.}}{Tech executives looked like they work in, well, tech offices.}}\label{tech-executives-looked-like-they-work-in-well-tech-offices}}

Copied to clipboard.

In the pandemic era, videoconferencing has become a primary means of
conversation. Also, we've become very judgmental about the decor of our
co-workers. So how did Big Tech's executives fare in their congressional
hearing on Wednesday?

We were inspired by the
\href{https://slack-redir.net/link?url=https\%3A\%2F\%2Ftwitter.com\%2Fratemyskyperoom}{popular
Twitter account RoomRater}, which judges and rates the backgrounds of
people on video calls on a scale of one to 10. In that spirit, here are
our armchair ratings for the rooms in which the C.E.O.s of Google,
Apple, Amazon and Facebook appeared during their testimonies.

Image

Credit...Pool photo by Mandel Ngan

\textbf{Sundar Pichai:} Tasteful minimalism, muted and cold color
palette. Sitting in front of a midcentury modern bureau, with a small
stack of unidentified books and small pottery with a pleasant aloe
plant. Reading from printed documents. Textured wall hanging, likely of
stone.

\emph{7/10 for being relatable without going over the top.}

Image

\textbf{Tim Cook:} Extremely minimal background, even for Apple. Opted
for bare, taupe-colored walls with a slate planter flanking him. A
smattering of greenery livens up the background, but not too wild as if
it were a jungle. The hint of a computer screen --- a Macbook --- in
front of him for notes.

\emph{6/10 because we expected something more sleek from Apple.}

Image

\textbf{Jeff Bezos:} The warmest setting of the bunch. Mr. Bezos sat in
what looks like his Seattle office, backed by a full wall of built-in
bookcases. Gold-plated statue looks kind of like a funky atom or a
scientific award. Nice pottery and vases.

\emph{8/10 for the cool Pacific Northwest dad office vibes. Two points
subtracted for his connectivity issues.}

Image

\textbf{Mark Zuckerberg:} Almost completely devoid of character. Stark
white wooden plank background --- perhaps shiplap, the favorite of home
decorating star Joanna Gaines? The only distinguishing marks are two
little knobs in the wood that look like drawer handles. No plants, no
books, no warmth. Reading from a teleprompter. The setting is so
inoffensive it borders on offensive. But points for whoever staged his
lighting and camera. The focus is, unavoidably, Mr. Zuckerberg.

\emph{4/10 for its complete lack of personality. (Next time show us a
bookshelf.)}

--- \href{https://www.nytimes.com/by/mike-isaac}{Mike Isaac}

\hypertarget{the-ceos-dressed-to-project-humility-our-fashion-critic-writes}{%
\subsection{\texorpdfstring{\protect\hyperlink{the-ceos-dressed-to-project-humility-our-fashion-critic-writes}{The
C.E.O.s dressed to project humility, our fashion critic
writes.}}{The C.E.O.s dressed to project humility, our fashion critic writes.}}\label{the-ceos-dressed-to-project-humility-our-fashion-critic-writes}}

Copied to clipboard.

\includegraphics{https://static01.nyt.com/images/2020/07/30/fashion/30ZOOMSUITS-COMBO/29ZOOMSUITS-COMBO-articleLarge.jpg?quality=75\&auto=webp\&disable=upscale}

They didn't look like titans. They didn't look like masters of the
universe. They didn't look like ``emperors of the online economy,'' as
Representative David Cicilline, chairman of the House Judiciary
Committee and Democrat from Rhode Island, called them.

``They'' --- the four chief executives of Big Tech, Jeff Bezos of
Amazon; Mark Zuckerberg of Facebook; Tim Cook of Apple; and Sundar
Pichai of Alphabet, the parent company of Google --- didn't even look
all that big.

In fact, beamed in from their offices because of coronavirus concerns,
facing down the mask-clad congressmen who were socially distanced from
each other on the wood-paneled stage of the hearing room in the Rayburn
office building like an establishment army, they looked more like boys
dressed up in their graduation suits than the four horsemen of the
digital apocalypse whose planetary power was a threat to us all.

The costumes were donned with purpose.

Mr. Zuckerberg, for example, framed against a plain white background
that resembled barn siding, wore a blue suit and a blue and white
checked tie that had been pulled down and was slightly askew, as though
he had stuck one finger inside so he could take a deep breath.

Mr. Cook chose a light gray tie --- the same gray as his glasses frames
--- the knot listing just off to one side, and a dark gray suit, with a
whole Zen planter's worth of verdant greenery spilling out behind him.
He sipped from a mug of tea.

Mr. Pichai also appeared in a subtly patterned gray tie, though his
echoed the patterned artwork on the wall behind him and perfectly
matched his gray suit. Which matched his hair and beard, which matched
the gray pottery on the filing cabinet behind him, out of which bloomed
his own healthily lush green plant --- one part of an artistic and
minimal still life. He sat with his hands clasped on the desk in front
of him, radiating a sort of beneficent calm.

And Mr. Bezos, in his first-ever appearance in Congress, offset his dark
suit and tie with some homey light wood shelving, scattered with vases
and other decorative objects, and sustained himself with snacks kept
just offscreen.

Snacks! He's just like you and me.

Which was, of course, the point. If you are trying to convince a group
of lawmakers that the words they keep using to describe you ---
``dominant,'' ``power,'' ``billions,'' ``trillions'' --- are not nearly
the whole story, you don't want to limit your message to your humble
beginnings and crazy dreams. You want to channel Clark Kent, rather than
Superman.

--- \href{https://www.nytimes.com/by/vanessa-friedman}{Vanessa Friedman}

\hypertarget{there-are-many-investigations-into-the-tech-companies-heres-where-they-all-stand}{%
\subsection{\texorpdfstring{\protect\hyperlink{there-are-many-investigations-into-the-tech-companies-heres-where-they-all-stand}{There
are many investigations into the tech companies. Here's where they all
stand.}}{There are many investigations into the tech companies. Here's where they all stand.}}\label{there-are-many-investigations-into-the-tech-companies-heres-where-they-all-stand}}

Copied to clipboard.

\includegraphics{https://static01.nyt.com/images/2020/07/29/business/29tech-hearing-inquiries/merlin_163192332_bc0f35e4-7fc0-481a-bec0-f76d02126a92-articleLarge.jpg?quality=75\&auto=webp\&disable=upscale}

The tech giants are under investigation from numerous federal and state
antitrust officials, as well as by the lawmakers holding today's
hearing.

The Justice Department's investigation of Google appears to be the
furthest along.
\href{https://www.nytimes.com/2020/06/25/technology/barr-google-investigation.html}{The
agency is expected to soon announce a case against Google}, focusing on
alleged antitrust violations in online advertising.

The Federal Trade Commission
is\href{https://www.nytimes.com/2020/07/17/technology/ftc-facebook-investigation.html}{preparing
to depose}Mark Zuckerberg, the chief executive of Facebook, and other
top executives at the company for its investigation of the social
network. That inquiry appears to focus on whether Facebook illegally
maintained a monopoly in social networking by killing off competition
through its acquisitions of Instagram and WhatsApp. That investigation
may not wrap up before the end of the year.

Other investigations are moving forward, but not as swiftly as the
Google investigation. The Justice Department is also investigating
Apple's power over the app store, along with state attorneys general.
The agency has Facebook under review as well, looking at the company's
position in online advertising. But that investigation appears to be
moving slowly.

State investigators have been
\href{https://www.nytimes.com/2020/06/12/technology/state-inquiry-antitrust-amazon.html}{looking
into whether Amazon abuses its power} over sellers on the tech giant's
site. The F.T.C. is also investigating Amazon, but that appears to be
moving slowly.

--- \href{https://www.nytimes.com/by/cecilia-kang}{Cecilia Kang}

\hypertarget{trump-administration-asks-fcc-to-narrow-protections-for-tech-companies}{%
\subsection{\texorpdfstring{\protect\hyperlink{trump-administration-asks-fcc-to-narrow-protections-for-tech-companies}{Trump
administration asks F.C.C. to narrow protections for tech
companies.}}{Trump administration asks F.C.C. to narrow protections for tech companies.}}\label{trump-administration-asks-fcc-to-narrow-protections-for-tech-companies}}

Copied to clipboard.

The Trump administration asked the Federal Communications Commission
this week to narrow its interpretation of a law that shields internet
platforms like Facebook and YouTube from certain lawsuits over the
content they host.

The request, which stems from an executive order President Trump signed
in May, is part of a growing push by the president and his allies, who
say that tech companies are removing or suppressing conservative
content. Despite evidence that conservative sites and figures perform
well online, the president, along with much of his conservative base,
have repeatedly criticized the platforms over instances in which
conservative content was removed or otherwise moderated for violating a
platform's rules.

In a petition on Monday, the Department of Commerce asked the commission
to clarify that the law, known as Section 230, does not protect a
platform when it moderates or highlights user content based on a
``reasonably discernible viewpoint or message, without having been
prompted to, asked to, or searched for by the user.'' It would also
limit the circumstances under which platforms are protected from
liability over their users' content.

Kayleigh McEnany, the White House spokeswoman, said in a statement on
Wednesday morning that the president wants the F.C.C. ``to clarify that
Section 230 does not permit social media companies that alter or
editorialize users' speech to escape civil liability.''

Mr. Trump weighed in later on Twitter:

\begin{quote}
If Congress doesn't bring fairness to Big Tech, which they should have
done years ago, I will do it myself with Executive Orders. In
Washington, it has been ALL TALK and NO ACTION for years, and the people
of our Country are sick and tired of it!

--- Donald J. Trump (@realDonaldTrump)
\href{https://twitter.com/realDonaldTrump/status/1288506554585505793?ref_src=twsrc\%5Etfw}{July
29, 2020}
\end{quote}

The petition is now in the hands of the F.C.C., an independent agency
currently led by a Republican chairman, Ajit Pai, who was appointed to
the position by Mr. Trump. ``The F.C.C. will carefully review the
petition,'' said Brian Hart, a spokesman for the commission.

--- \href{https://www.nytimes.com/by/david-mccabe}{David McCabe}

\hypertarget{dont-only-blame-congress-if-this-hearing-goes-off-the-rails}{%
\subsection{\texorpdfstring{\protect\hyperlink{dont-only-blame-congress-if-this-hearing-goes-off-the-rails}{Don't
(only) blame Congress if this hearing goes off the
rails.}}{Don't (only) blame Congress if this hearing goes off the rails.}}\label{dont-only-blame-congress-if-this-hearing-goes-off-the-rails}}

Copied to clipboard.

Members of Congress have been
\href{https://www.thewrap.com/senator-orrin-hatch-facebook-biz-model-zuckerberg/}{mocked}
for asking ridiculous questions in technology hearings like these. That
might happen again today, but it won't be entirely their fault.

These big tech companies intentionally make themselves hard to
understand.

Few people outside these companies can truly examine how Amazon
influences prices of products we buy on its site
or\href{https://www.bloomberg.com/news/articles/2019-08-05/amazon-is-squeezing-sellers-that-offer-better-prices-on-walmart}{at
other retailers}; or assess fears that
Google\href{https://themarkup.org/google-the-giant/2020/07/28/google-search-results-prioritize-google-products-over-competitors}{funnels
people to its own websites},
Apple\href{https://www.nytimes.com/interactive/2019/09/09/technology/apple-app-store-competition.html}{steers
people to its own apps} or Facebook peers into what we do online to
\href{https://www.nytimes.com/2018/12/05/technology/facebook-emails-privacy-data.html}{squash
its rivals}. All of this is, by design, shrouded in secrecy and mystery.

Big Tech shouldn't want it to stay that way. Even companies like
Facebook and Google are asking for more government guidance and rules
around thorny topics like protecting elections and preventing hate
speech on their sites. That means that the public and the tech companies
have a vested interest in making these fact-finding sessions as
productive as possible.

\href{https://www.nytimes.com/2020/07/29/technology/congress-big-tech.html}{Read
more in On Tech}.

\emph{You can}
\href{https://www.nytimes.com/newsletters/signup/OT}{\emph{sign up
here}} \emph{for On Tech with Shira Ovide, a newsletter each weekday
about how technology is reshaping our lives and world.}

--- \href{https://www.nytimes.com/by/shira-ovide}{Shira Ovide}

\hypertarget{advertisement-3}{%
\subsubsection{Advertisement}\label{advertisement-3}}

\protect\hyperlink{after-dfp-ad-mid1}{Continue reading the main story}

\hypertarget{the-executives-are-testifying-remotely-using-ciscos-webex-videoconferencing}{%
\subsection{\texorpdfstring{\protect\hyperlink{the-executives-are-testifying-remotely-using-ciscos-webex-videoconferencing}{The
executives are testifying remotely, using Cisco's Webex
videoconferencing.}}{The executives are testifying remotely, using Cisco's Webex videoconferencing.}}\label{the-executives-are-testifying-remotely-using-ciscos-webex-videoconferencing}}

Copied to clipboard.

\includegraphics{https://static01.nyt.com/images/2020/07/29/business/29tech-hearing-webex/merlin_174831297_dbba0a25-3406-4981-8177-40356c08d534-articleLarge.jpg?quality=75\&auto=webp\&disable=upscale}

Congressional hearings usually involve witnesses appearing in dark
suits, with their entourages sitting behind them and lawmakers
questioning them from above as phalanxes of photographers snap pictures
and videographers stream the proceedings from a cavernous room at the
Capitol.

Not this time.

The C.E.O.s of Amazon, Apple, Facebook and Google are all appearing on
Wednesday before a House subcommittee virtually because of the
coronavirus pandemic. Remotely beaming into the hearing adds a wrinkle
of digital complexity, with any note-passing from aides and underlings
most likely happening off-camera.

And while many of the tech giants make
\href{https://www.nytimes.com/2020/04/24/technology/zoom-rivals-virus-facebook-google.html}{their
own video-calling software}, none will be using their own tools.
Instead, they will all be joining via Cisco's Webex videoconferencing
service.

Webex has been the go-to service for Congress since the pandemic began.
It has been certified by the House's administration committee for being
secure and meeting ``business and technical requirements,'' a House
administration spokesman, Peter Whippy, said.

In that time, Webex has been used for more than 100 congressional
hearings, said Jean Rosauer, Webex's head of government sector. Cisco
added that it had experienced more than triple its normal volume of
virtual meetings through Webex in recent months.

``Congressional hearings --- such as the upcoming House Judiciary
Committee hearing --- have traditions, policies and procedures, and we
had to ensure those could be conducted virtually and securely,'' Ms.
Rosauer said in a statement. She added that Cisco was ``incredibly
proud'' to play a role in keeping Congress connected.

--- \href{https://www.nytimes.com/by/kellen-browning}{Kellen Browning}

\hypertarget{watch-live-lawmakers-restart-grilling-of-executives}{%
\subsection{\texorpdfstring{\protect\hyperlink{watch-live-lawmakers-restart-grilling-of-executives}{Watch
live: Lawmakers restart grilling of
executives.}}{Watch live: Lawmakers restart grilling of executives.}}\label{watch-live-lawmakers-restart-grilling-of-executives}}

Copied to clipboard.

\includegraphics{https://static01.nyt.com/images/2020/07/29/us/29antitrust-vidcover/29hpantitrust-videoSixteenByNine3000-v2.jpg}

\hypertarget{big-techs-rivals-spoke-out-ahead-of-the-hearing}{%
\subsection{\texorpdfstring{\protect\hyperlink{big-techs-rivals-spoke-out-ahead-of-the-hearing}{Big
Tech's rivals spoke out ahead of the
hearing.}}{Big Tech's rivals spoke out ahead of the hearing.}}\label{big-techs-rivals-spoke-out-ahead-of-the-hearing}}

Copied to clipboard.

\includegraphics{https://static01.nyt.com/images/2020/07/29/business/29tech-hearing-rivals/merlin_164279103_480dfcec-5c97-492c-b524-1e7885551ae2-articleLarge.jpg?quality=75\&auto=webp\&disable=upscale}

Many competitors to Google, Facebook, Apple and Amazon have been busy
talking to House lawmakers for months about those companies' power. And
some deliberately spoke out this week to position themselves for how
they would be portrayed in the hearing and to influence the questioning.

TikTok, the Chinese-owned video app, issued a statement from
\href{https://www.nytimes.com/2020/05/18/business/media/tiktok-ceo-kevin-mayer.html}{its
chief executive}, Kevin Mayer, on Wednesday morning. In it, he addressed
how the app --- which Facebook is likely to cite in the hearing as an
example of how competition in social networking is thriving --- has been
dealing with scrutiny because of its Chinese ownership.

``We have received even more scrutiny due to the company's Chinese
origins,'' Mr. Mayer said
\href{https://newsroom.tiktok.com/en-us/fair-competition-and-transparency-benefits-us-all}{in
the statement}. ``We accept this and embrace the challenge of giving
peace of mind through greater transparency and accountability. We
believe it is essential to show users, advertisers, creators and
regulators that we are responsible and committed members of the American
community that follows U.S. laws.''

He also pointed to Facebook's willingness to launch ``copycat
products,'' like Reels, a TikTok look-alike. Facebook has had a history
of emulating competing products.

``Let's focus our energies on fair and open competition in service of
our consumers, rather than maligning attacks by our competitor ---
namely Facebook --- disguised as patriotism and designed to put an end
to our very presence in the U.S.,'' Mr. Mayer said.

Other tech companies also seized on the hearing to air their thoughts.
Tim Sweeney, chief executive of Epic Games, the Cary, N.C.-based maker
of
\href{https://www.nytimes.com/2018/07/25/arts/what-is-fortnite-battle-royale-nyt.html}{the
hit game Fortnite}, lashed out at Apple and Google for price gouging and
unfair policies in what he called their ``app store monopolies.''

``Both stores significantly obstruct competition,'' Mr. Sweeney said in
an interview on Tuesday. He particularly criticized Apple's 30 percent
fee on payments for digital goods, which he said made it difficult for
smaller players to offer artists a better deal.

Apple has said the
\href{https://www.nytimes.com/2020/07/28/technology/apple-app-store-airbnb-classpass.html}{30
percent commission it takes from many apps} in its App Store is a
standard fee. Mr. Sweeney called that argument ``silly nonsense.''
Epic's version of an app store charges its developers a 12 percent fee.

Mr. Sweeney, who began programming on an Apple II Plus computer in 1982
and founded Epic nine years later, said he felt a responsibility to
speak out.

``Every tech company that does business in this world is going to have
to live with the power we give these other companies,'' he said.

--- \href{https://www.nytimes.com/by/mike-isaac}{Mike Isaac} and
\href{https://www.nytimes.com/by/erin-griffith}{Erin Griffith}

\hypertarget{advertisement-4}{%
\subsubsection{Advertisement}\label{advertisement-4}}

\protect\hyperlink{after-dfp-ad-mid2}{Continue reading the main story}

\hypertarget{what-to-expect-from-the-hearing}{%
\subsection{\texorpdfstring{\protect\hyperlink{what-to-expect-from-the-hearing}{What
to expect from the
hearing.}}{What to expect from the hearing.}}\label{what-to-expect-from-the-hearing}}

Copied to clipboard.

\includegraphics{https://static01.nyt.com/images/2020/07/29/business/29tech-hearing-ledeall/29tech-hearing-ledeall-articleLarge.jpg?quality=75\&auto=webp\&disable=upscale}

After lawmakers collected hundreds of hours of interviews and obtained
more than 1.3 million documents about Amazon, Apple, Facebook and
Google, their chief executives will testify before Congress at 1 p.m. on
Wednesday to defend their powerful businesses.

The captains of the New Gilded Age ---
\href{https://www.nytimes.com/2020/07/27/business/jeff-bezos-amazon-congress.html}{Jeff
Bezos of Amazon}, Tim Cook of Apple, Mark Zuckerberg of Facebook and
Sundar Pichai of Google --- will appear together before Congress for the
first time to justify their business practices. Members of the House
judiciary's antitrust subcommittee
\href{https://www.nytimes.com/2019/06/11/technology/antitrust-hearing.html}{have
investigated the internet giants} for more than a year on accusations
that they have stifled rivals and harmed consumers. The exact contents
of the documents they've collected are unknown, although they are said
to include documents related to some of the companies' acquisitions and
internal communications among top executives.

It is set to be a bizarre spectacle, with four men who run companies
worth nearly \$5 trillion combined --- and who include two of the
world's richest individuals --- primed to argue that their businesses
are not really that powerful after all.

And it will be a first in another way: Mr. Zuckerberg, Mr. Pichai, Mr.
Bezos and Mr. Cook will all be testifying via videoconference, rather
than rising side-by-side for a swearing-in at a witness table in
Washington.

At the hearing, the 15 members of the antitrust subcommittee will have
five minutes for each question. Representative David Cicilline, Democrat
of Rhode Island and the chairman of the subcommittee, will control the
number of rounds of questioning, potentially stretching the hearing into
the evening.

The antitrust issues facing Apple, Facebook,
\href{https://www.nytimes.com/2019/06/02/business/google-antitrust-investigation.html}{Google
and Amazon} are complex and vastly different.

Amazon is accused of abusing its role as both a retailer and a platform
hosting third-party sellers on its marketplace. Apple has been accused
of unfairly using its clout over its App Store to block rivals and to
force apps to pay high commissions. Rivals have said Facebook has a
monopoly in social networking. Alphabet, the parent company of Google,
is dealing with multiple antitrust allegations because of Google's
dominance in online advertising, search and smartphone software.

Democrats may also veer off the topic of antitrust and bring up concerns
about misinformation on social media. Some Republicans are expected to
sidetrack discussion with their concerns of liberal bias at the Silicon
Valley companies and accusations that conservative voices are censored.

--- \href{https://www.nytimes.com/by/cecilia-kang}{Cecilia Kang},
\href{https://www.nytimes.com/by/jack-nicas}{Jack Nicas} and
\href{https://www.nytimes.com/by/david-mccabe}{David McCabe}

\hypertarget{todays-hearing-has-echoes-of-bill-gates-22-years-ago}{%
\subsection{\texorpdfstring{\protect\hyperlink{todays-hearing-has-echoes-of-bill-gates-22-years-ago}{Today's
hearing has echoes of Bill Gates, 22 years
ago.}}{Today's hearing has echoes of Bill Gates, 22 years ago.}}\label{todays-hearing-has-echoes-of-bill-gates-22-years-ago}}

Copied to clipboard.

\includegraphics{https://static01.nyt.com/images/2020/08/03/business/03tech-hearing-gates/03tech-hearing-gates-articleLarge.jpg?quality=75\&auto=webp\&disable=upscale}

The tech industry is an engine of innovation, job creation and American
economic prowess. Competition is flourishing, and just a click away.
Sure, we do well, but consumers are the big winners.

That was the gist of
\href{https://archive.nytimes.com/www.nytimes.com/library/tech/98/03/biztech/articles/04microsoft.html}{Bill
Gates's testimony} before a Senate panel more than two decades ago. And
it's a safe bet the same themes will feature prominently when the
leaders of Amazon, Apple, Facebook and Google testify on Wednesday.

There are differences, but this week's appearance by tech executives is
reminiscent of the congressional grilling Microsoft's chief faced 22
years ago.

In 1998, the spotlight was squarely on Mr. Gates, co-founder of
Microsoft, the tech behemoth of the personal computer era. This time,
the leaders of four big technology companies will be in the dock,
appearing remotely because of a pandemic.

Today, more issues are in play. In the late 1990s, the concern was that
Microsoft would use its dominance in the PC market to stifle internet
upstarts. The sheer market muscle of today's tech giants is a worry, but
so is the role they play broadly in commerce and communication,
influencing public opinion and politics.

When Mr. Gates testified, a formal investigation of Microsoft by federal
regulators and dozens of states was well underway. The same is true now
for Google and Facebook, while Amazon and Apple are also facing
antitrust scrutiny.

There can be gotcha moments. Under pointed questioning, Mr. Gates
rhetorically bobbed and weaved, refusing to use the M-word: monopoly.

But when Jim Barksdale, head of Netscape, the internet company most in
Microsoft's sights, testified that day, he asked the spectators to raise
their hands if they used a PC.

About three-quarters of the room did. Then, how many of them used
Microsoft's Windows operating system? Almost the same number of hands
flew up again.

``That,'' Mr. Barksdale said, ``is a monopoly.''

--- \href{https://www.nytimes.com/by/steve-lohr}{Steve Lohr}

\hypertarget{site-index}{%
\subsection{Site Index}\label{site-index}}

\hypertarget{site-information-navigation}{%
\subsection{Site Information
Navigation}\label{site-information-navigation}}

\begin{itemize}
\tightlist
\item
  \href{https://help.nytimes.com/hc/en-us/articles/115014792127-Copyright-notice}{©~2020~The
  New York Times Company}
\end{itemize}

\begin{itemize}
\tightlist
\item
  \href{https://www.nytco.com/}{NYTCo}
\item
  \href{https://help.nytimes.com/hc/en-us/articles/115015385887-Contact-Us}{Contact
  Us}
\item
  \href{https://www.nytco.com/careers/}{Work with us}
\item
  \href{https://nytmediakit.com/}{Advertise}
\item
  \href{http://www.tbrandstudio.com/}{T Brand Studio}
\item
  \href{https://www.nytimes.com/privacy/cookie-policy\#how-do-i-manage-trackers}{Your
  Ad Choices}
\item
  \href{https://www.nytimes.com/privacy}{Privacy}
\item
  \href{https://help.nytimes.com/hc/en-us/articles/115014893428-Terms-of-service}{Terms
  of Service}
\item
  \href{https://help.nytimes.com/hc/en-us/articles/115014893968-Terms-of-sale}{Terms
  of Sale}
\item
  \href{https://spiderbites.nytimes.com}{Site Map}
\item
  \href{https://help.nytimes.com/hc/en-us}{Help}
\item
  \href{https://www.nytimes.com/subscription?campaignId=37WXW}{Subscriptions}
\end{itemize}
