Sections

SEARCH

\protect\hyperlink{site-content}{Skip to
content}\protect\hyperlink{site-index}{Skip to site index}

\href{https://myaccount.nytimes.com/auth/login?response_type=cookie\&client_id=vi}{}

\href{https://www.nytimes.com/section/todayspaper}{Today's Paper}

\href{https://www.nytimes.com/news-event/coronavirus}{The Coronavirus
Outbreak}

\begin{itemize}
\tightlist
\item
  live\href{https://www.nytimes.com/2020/08/01/world/coronavirus-covid-19.html}{Latest
  Updates}
\item
  \href{https://www.nytimes.com/interactive/2020/us/coronavirus-us-cases.html}{Maps
  and Cases}
\item
  \href{https://www.nytimes.com/interactive/2020/science/coronavirus-vaccine-tracker.html}{Vaccine
  Tracker}
\item
  \href{https://www.nytimes.com/interactive/2020/07/29/us/schools-reopening-coronavirus.html}{What
  School May Look Like}
\item
  \href{https://www.nytimes.com/live/2020/07/31/business/stock-market-today-coronavirus}{Economy}
\end{itemize}

Last Updated

July 31, 2020, 7:05 a.m. ET

July 31, 2020, 7:05 a.m. ET

\hypertarget{big-tech-earnings-surge-as-economy-slumps}{%
\section{Big Tech Earnings Surge as Economy
Slumps}\label{big-tech-earnings-surge-as-economy-slumps}}

This briefing is no longer being updated. Follow live updates
\href{https://www.nytimes.com/live/2020/07/31/business/stock-market-today-coronavirus}{here}.

The percentage decrease in G.D.P.

is by far the biggest on record.

+

4

\%

+

2

0

--

2

--

4

--

6

--9.5\%

Percentage change from previous quarter

--

8

--

10

1950

1960

1970

1980

1990

2000

2010

2020

The percentage decrease in G.D.P. is by far the biggest on record.

+

4

\%

+

2

0

--

2

--

4

--

6

--9.5\%

Percentage change from previous quarter

--

8

--

10

1950

1960

1970

1980

1990

2000

2010

2020

The percentage decrease in G.D.P. is by far the biggest on record.

+

4

\%

+

2

0

--

2

--

4

--

6

Percentage change from previous quarter

--9.5\%

--

8

--

10

1950

1960

1970

1980

1990

2000

2010

2020

Source: Bureau of Economic Analysis

By Karl Russell

\hypertarget{heres-what-you-need-to-know}{%
\subsubsection{Here's what you need to
know:}\label{heres-what-you-need-to-know}}

\begin{itemize}
\item
  \protect\hyperlink{amazons-earnings-double-as-sales-surge}{}

  Amazon's earnings double as sales surge.
\item
  \protect\hyperlink{alphabets-revenue-drops-but-beats-wall-street-expectations}{}

  Alphabet's revenue drops, but beats Wall Street expectations.
\item
  \protect\hyperlink{ford-made-1-1-billion-profit-in-the-second-quarter-even-as-sales-tumbled}{}

  Ford made \$1.1 billion profit in the second quarter even as sales
  tumbled.
\item
  \protect\hyperlink{apple-blows-past-expectations-with-surging-sales-and-profits}{}

  Apple blows past expectations with surging sales and profits.
\item
  \protect\hyperlink{the-us-economys-contraction-in-the-second-quarter-was-the-worst-on-record}{}

  The U.S. economy's contraction in the second quarter was the worst on
  record.
\end{itemize}

\hypertarget{facebook-nearly-doubles-its-profit-but-warns-of-fallout-from-ad-boycotts}{%
\subsection{\texorpdfstring{\protect\hyperlink{facebook-nearly-doubles-its-profit-but-warns-of-fallout-from-ad-boycotts}{Facebook
nearly doubles its profit, but warns of fallout from ad
boycotts.}}{Facebook nearly doubles its profit, but warns of fallout from ad boycotts.}}\label{facebook-nearly-doubles-its-profit-but-warns-of-fallout-from-ad-boycotts}}

Copied to clipboard.

\includegraphics{https://static01.nyt.com/images/2020/07/30/business/30markets-brf-facebook/merlin_174655161_89b625c0-9697-484c-8543-ca8e17245182-articleLarge.jpg?quality=75\&auto=webp\&disable=upscale}

Mark Zuckerberg, the chief executive of \textbf{Facebook}, displayed the
social network's enormous financial clout on Thursday, even as the
company has dealt with regulatory scrutiny and advertiser boycotts.

Facebook's
\href{https://investor.fb.com/investor-news/press-release-details/2020/Facebook-Reports-Second-Quarter-2020-Results/default.aspx}{revenue
for the second quarter} rose 11 percent from a year earlier to \$18.7
billion, while profit jumped 98 percent to \$5.2 billion. The results
were well above analysts' estimates of \$17.3 billion in revenue with a
profit of \$3.9 billion, according to data provided by FactSet.

More than three billion people come to Facebook or one of its family of
apps on a regular basis, as the services have overtaken much of the
developed world. And some 2.47 billion people use one or more of
Facebook's apps every day.

The company said its number of monthly active users rose 12 percent from
a year ago and added that it was seeing record levels of engagement and
usage this year because of the coronavirus pandemic and the
shelter-in-place orders around the world.

``We're glad to be able to provide small businesses the tools they need
to grow and be successful online during these challenging times,'' said
Mr. Zuckerberg, who was
\href{https://www.nytimes.com/2020/07/29/technology/big-tech-hearing-apple-amazon-facebook-google.html}{grilled
by lawmakers on Wednesday} over Facebook's power. ``And we're proud that
people can rely on our services to stay connected when they can't always
be together in person.''

Facebook's earnings have long been a bright spot for the Silicon Valley
company. Despite increasing scrutiny from regulators, questions about
\href{https://www.nytimes.com/2018/02/17/technology/indictment-russian-tech-facebook.html}{its
role in subverting elections} and how people use the platform to spread
misinformation, users have continued coming back to its services.
Because of this, advertisers have consistently spent money on the
platform.

But that started changing in late June when a grass-roots campaign, Stop
Hate For Profit, rallied many of the top advertisers on Facebook to
\href{https://www.nytimes.com/2020/06/30/technology/facebook-advertising-boycott.html}{pull
back their spending} because of issues of hate speech on the site.
Facebook has tried to assuage the concerns, but has made it clear that
it will not change its policy about free speech on the site based on
outside threats to the business.

Facebook cautioned investors on Thursday that fallout from the ad
boycott was noticeable in July. The company warned that greater economic
turmoil from the pandemic could also eventually affect its bottom line.

--- \href{https://www.nytimes.com/by/mike-isaac}{Mike Isaac}

\hypertarget{amazons-earnings-double-as-sales-surge}{%
\subsection{\texorpdfstring{\protect\hyperlink{amazons-earnings-double-as-sales-surge}{Amazon's
earnings double as sales
surge.}}{Amazon's earnings double as sales surge.}}\label{amazons-earnings-double-as-sales-surge}}

Copied to clipboard.

\includegraphics{https://static01.nyt.com/images/2020/07/30/business/30markets-brf-amazon/30markets-brf-amazon-articleLarge.jpg?quality=75\&auto=webp\&disable=upscale}

Buoyed by a pandemic-induced surge in online shopping, \textbf{Amazon}
on Thursday reported record sales and profits in the latest quarter.

\href{https://www.nytimes.com/2020/05/22/technology/amazon-coronavirus-target-walmart.html}{Amazon
had \$88.9 billion in quarterly sales}, up 40 percent from a year
earlier. Profit doubled, to \$5.2 billion, even though the company
invested heavily to improve the safety in Amazon's warehouses.

Analysts expected the company to have \$81.4 billion in sales and \$665
million in profit, according estimates compiled by FactSet, a financial
data firm. Shares in the company jumped more than 6 percent in
after-hours trading.

``Simply put, Covid-19, in our view, has injected Amazon with a growth
hormone,'' Tom Forte, an analyst at the investment bank D.A. Davidson \&
Company, wrote in a recent note to investors.

The profit came even as Amazon invested \$9 billion in expanding
warehouses and other capital expenditures to increase its capacity.
``It's a good problem to have,'' Brian Olsavsky, the company's finance
chief, said on a call with reporters.

In April, Jeff Bezos, Amazon's chief executive,
\href{https://www.nytimes.com/2020/04/30/technology/amazon-stock-earnings-report.html}{told}
investors to expect no operating profit, and maybe even a loss, as the
company planned to spend about \$4 billion on coronavirus-related
expenses, including temporary pay increases, declines in warehouse
efficiency because of social distancing and \$300 million for testing
its work force for the virus.

``If you're a shareowner in Amazon, you may want to take a seat, because
we're not thinking small,'' he said at the time.

Amazon had been paying workers an extra \$2 an hour, but that benefit
expired in May. At the end of June, it announced
\href{https://blog.aboutamazon.com/operations/a-thank-you-bonus-for-amazon-front-line-employees-and-partners}{one-time
``thank you'' bonus} of \$500 for full-time associates in its
warehouses.

But even those costs did not compare to the immense surge in demand,
with online retail sales up 48 percent. As Americans have stayed home
during the virus, they have flocked to online shopping.

``E-commerce is off the charts right now,'' said Guru Hariharan, a
former Amazon employee whose company, CommerceIQ, helps major consumer
brands manager their Amazon business. The initial shock of panic buying
has subsided, but ``demand is starting to stabilize, at a much higher
level,'' he said. Mr. Olsavsky said customers returned to buying more
profitable products, like clothing, versus lower-margin groceries and
cleaning suppliers.

One of the biggest challenges Amazon had expected was keeping up with
demand, as the virus
\href{https://www.nytimes.com/2020/05/19/technology/amazon-coronavirus-workers.html}{flared
among its workers} and the communities where they live. Mr. Olsavsky
said the company was able to fulfill more orders than it had previously
expected. The number of products it sold grew 57 percent, but the number
of employees it had was up just 34 percent.

On the call with reporters, Amazon declined to say if it would be paying
its warehouse workers more in the current quarter. It said that
pandemic-related expenses would fall to \$2 billion in the quarter.

Sales at Amazon's lucrative cloud computing business, whose customers
range from major corporations to start-ups, grew 29 percent, to \$10.8
billion, falling short of analyst expectations, though it was more
profitable than they expected.

--- \href{https://www.nytimes.com/by/karen-weise}{Karen Weise}

\hypertarget{advertisement}{%
\subsubsection{Advertisement}\label{advertisement}}

\protect\hyperlink{after-dfp-ad-mid1}{Continue reading the main story}

\hypertarget{alphabets-revenue-drops-but-beats-wall-street-expectations}{%
\subsection{\texorpdfstring{\protect\hyperlink{alphabets-revenue-drops-but-beats-wall-street-expectations}{Alphabet's
revenue drops, but beats Wall Street
expectations.}}{Alphabet's revenue drops, but beats Wall Street expectations.}}\label{alphabets-revenue-drops-but-beats-wall-street-expectations}}

Copied to clipboard.

\includegraphics{https://static01.nyt.com/images/2020/07/30/business/30markets-brf-google/merlin_165422706_2d67c3e6-7d07-4987-b4a0-dde1ea884c81-articleLarge.jpg?quality=75\&auto=webp\&disable=upscale}

\textbf{Alphabet}, the parent company of \textbf{Google}, reported its
first-ever decline in quarterly revenue on Thursday, hurt by a slowdown
in spending by advertisers.

Alphabet said its revenue fell 2 percent to \$38.3 billion in the second
quarter compared with a year ago. The decline came largely from lower
sales of advertisements that run alongside its search results because of
the coronavirus pandemic, although the company posted an increase in
revenue from YouTube ads and its cloud-computing business.

The results were the first time that quarterly revenue had declined in
its 17 years as a publicly traded company, but Alphabet exceeded
analysts' expectations for revenue and profit. Net profit totaled \$6.96
billion, down 30 percent from a year ago.

--- \href{https://www.nytimes.com/by/daisuke-wakabayashi}{Daisuke
Wakabayashi}

\hypertarget{ford-made-11-billion-profit-in-the-second-quarter-even-as-sales-tumbled}{%
\subsection{\texorpdfstring{\protect\hyperlink{ford-made-1-1-billion-profit-in-the-second-quarter-even-as-sales-tumbled}{Ford
made \$1.1 billion profit in the second quarter even as sales
tumbled.}}{Ford made \$1.1 billion profit in the second quarter even as sales tumbled.}}\label{ford-made-11-billion-profit-in-the-second-quarter-even-as-sales-tumbled}}

Copied to clipboard.

\includegraphics{https://static01.nyt.com/images/2020/07/30/business/30market-brf-ford/merlin_168723951_901c005f-c9a2-41d8-a9bc-57af4151e16b-articleLarge.jpg?quality=75\&auto=webp\&disable=upscale}

\textbf{Ford Motor} said Thursday it earned \$1.1 billion in the second
quarter as a large one-time gain in the value of its investment in an
autonomous driving company more than offset losses in its main business.

Without the gain, from its stake in Argo AI, Ford lost \$1.9 billion
excluding interest and taxes. The result was better than Ford's earlier
forecast of a pretax loss of \$5 billion.

The coronavirus pandemic forced Ford and other automakers to close
factories for nearly two months starting in March. On Wednesday,
\textbf{General Motors} said it lost \$758 million in the second
quarter.

Ford said in a statement that it expected ``no further significant
coronavirus-related disruptions to production'' in the second half of
the year. But the company also said it was not expecting ``meaningful
change in the current economic conditions.''

Ford's chief executive, Jim Hackett, said in a conference call that
reductions in spending and an efficient restart of production enabled
the company to avoid the kind of dire results it forecast in April. In
after-hours trading, Ford shares rose more than 2 percent to about
\$6.90.

The automaker forecast that it would earn \$500 million to \$1.5 billion
on a pretax basis in the third quarter, amid weaker demand for new
vehicles, parts and services.

Ford's deliveries of new vehicles fell by half to 645,000 in the second
quarter largely because of the pandemic. Its auto operations lost money
in every region in the world, including a \$954 million setback in North
America.

The company used up \$5.3 billion in cash in the quarter, but said it
still has \$39 billion on hand at the end of June.

Ford recorded a gain of \$3.5 billion from a transaction related to an
alliance it formed with \textbf{Volkswagen}, which bought a stake in
Argo AI.

Ford has been trying for three years to streamline its operations and
return to robust profits, but has come under criticism for slow
progress. Mr. Hackett said Ford was pinning its hopes on new vehicles it
unveiled in the last few weeks: a redesigned version of its F-150 pickup
truck and a line of rugged sport-utility vehicles that will be marketed
under the Bronco name. The company said it had already taken
reservations for 150,000 Broncos.

--- \href{https://www.nytimes.com/by/neal-e-boudette}{Neal E. Boudette}

\hypertarget{apple-blows-past-expectations-with-surging-sales-and-profits}{%
\subsection{\texorpdfstring{\protect\hyperlink{apple-blows-past-expectations-with-surging-sales-and-profits}{Apple
blows past expectations with surging sales and
profits.}}{Apple blows past expectations with surging sales and profits.}}\label{apple-blows-past-expectations-with-surging-sales-and-profits}}

Copied to clipboard.

\includegraphics{https://static01.nyt.com/images/2020/07/30/business/30markets-brf-apple/merlin_170516238_7ed7cf4f-4da4-4c49-a5d0-90ee8cc868a5-articleLarge.jpg?quality=75\&auto=webp\&disable=upscale}

From April through June, millions of people lost their jobs, thousands
of businesses closed --- and Apple made a further \$11.25 billion in
profits.

A global
\href{https://www.nytimes.com/live/2020/07/30/business/stock-market-today-coronavirus/the-us-economys-contraction-in-the-second-quarter-was-the-worst-on-record}{economic
slowdown} in the second quarter did not faze one of the world's richest
and most resilient companies, as people kept buying Apple devices en
masse and paid the tech giant billions of dollars more for apps and
services on those gadgets.

Apple said its sales rose 11 percent to \$59.7 billion and its profits
increased 12 percent to \$11.25 billion. Both figures handily beat
analysts' expectations, with Wall Street having forecast declines in
both areas.

Revenue rose for all of Apple's product categories and in all of its
geographic areas, unusual success even by Apple's lofty standards.

Sales were particularly strong for iPads and Mac computers, as the
public was increasingly forced to work and socialize virtually because
of the pandemic. Revenue also surged in its internet-services business,
which includes Apple's cut of sales from the App Store, the subject of
antitrust investigations in
\href{https://www.nytimes.com/2020/07/28/technology/amazon-apple-facebook-google-antitrust-hearing.html}{the
United States} and
\href{https://www.nytimes.com/2020/06/16/business/apple-app-store-european-union-antitrust.html}{Europe}.
Even the iPhone, which remains the company's biggest seller, notched a
slight increase in sales for only the second time in the past seven
quarters.

Luca Maestri, Apple's finance chief, said in an interview that the shift
to working and learning from home had led more people to splurge on
Apple's devices. ``Our products and services are very relevant to our
customers' lives, and in some cases, even more during the pandemic than
ever before,'' he said.

But while the pandemic has
\href{https://www.nytimes.com/2020/03/23/technology/coronavirus-facebook-amazon-youtube.html}{further
entrenched the biggest tech companies' power}, Mr. Maestri disputed the
idea that it had been good for business, saying the quarter would have
been even stronger without it. ``We believe we've lost several billion
dollars because of the pandemic,'' he said.

Investors have flocked to Apple's shares as a safe haven from an
economic recession, pushing its stock price up about 30 percent this
year to a roughly \$1.67 trillion value.

--- \href{https://www.nytimes.com/by/jack-nicas}{Jack Nicas}

\hypertarget{advertisement-1}{%
\subsubsection{Advertisement}\label{advertisement-1}}

\protect\hyperlink{after-dfp-ad-mid2}{Continue reading the main story}

\hypertarget{the-us-economys-contraction-in-the-second-quarter-was-the-worst-on-record}{%
\subsection{\texorpdfstring{\protect\hyperlink{the-us-economys-contraction-in-the-second-quarter-was-the-worst-on-record}{The
U.S. economy's contraction in the second quarter was the worst on
record.}}{The U.S. economy's contraction in the second quarter was the worst on record.}}\label{the-us-economys-contraction-in-the-second-quarter-was-the-worst-on-record}}

Copied to clipboard.

Economic output fell at its fastest pace on record last spring as the
coronavirus pandemic forced
\href{https://www.nytimes.com/2020/07/17/business/economy/how-to-save-economy.html?action=click\&module=RelatedLinks\&pgtype=Article}{businesses
across the United States to close their doors} and kept millions of
Americans shut in their homes for weeks.

Gross domestic product --- the broadest measure of goods and services
produced --- fell 9.5 percent in the second quarter of the year, the
\href{https://www.bea.gov/sites/default/files/2020-07/gdp2q20_adv.pdf}{Commerce
Department said Thursday}. On an annualized basis, the
\href{https://www.nytimes.com/2020/07/29/business/economy/us-gdp-report.html}{standard
way of reporting quarterly economic data}, G.D.P. fell at a rate of 32.9
percent.

G.D.P. shrank \$1.8 trillion in the 2nd quarter.

\$20

trillion

18

\$17.2

trillion

16

14

12

Gross domestic product, adjusted for

inflation and seasonality, at annual rates

10

'04

'06

'08

'10

'12

'14

'16

'18

'20

G.D.P. shrank \$1.8 trillion in the second quarter.

\$20

trillion

18

\$17.2

trillion

16

14

12

Gross domestic product, adjusted for inflation and seasonality, at
annual rates

10

'04

'06

'08

'10

'12

'14

'16

'18

'20

Source: Bureau of Economic Analysis

By Karl Russell

The collapse was unprecedented in its speed and breathtaking in its
severity. The only possible comparisons in modern American history came
during the Great Depression and the demobilization after World War II,
both of which occurred before the advent of modern economic statistics.

Unlike past recessions, this one was a result of a conscious decision to
suspend economic activity to slow the spread of the virus. Congress
pumped trillions of dollars into the economy to sustain households and
businesses, limit long-term damage and allow for a rapid rebound.

The plan worked at first. In recent weeks, however,
\href{https://www.nytimes.com/2020/07/29/health/coronavirus-future-america.html}{cases
have surged in much of the country}. Data from public and private
sources indicate
\href{https://www.nytimes.com/2020/07/15/business/economy/economic-recovery-coronavirus-resurgence.html?action=click\&module=RelatedLinks\&pgtype=Article}{a
pullback in economic activity}, reflecting consumer unease and renewed
shutdowns.

``In another world, a sharp drop in activity would have been just a
good, necessary blip while we addressed the virus,'' said Heather
Boushey, president of the Washington Center for Equitable Growth, a
progressive think tank. ``From where we sit in July, we know that this
wasn't just a short-term blip.''

--- \href{https://www.nytimes.com/by/ben-casselman}{Ben Casselman}

\hypertarget{143-million-filed-new-state-unemployment-claims-last-week}{%
\subsection{\texorpdfstring{\protect\hyperlink{1-43-million-filed-new-state-unemployment-claims-last-week}{1.43
million filed new state unemployment claims last
week.}}{1.43 million filed new state unemployment claims last week.}}\label{143-million-filed-new-state-unemployment-claims-last-week}}

Copied to clipboard.

\includegraphics{https://static01.nyt.com/images/2020/07/30/business/30markets-brf-jobless-numbers1/merlin_174916698_cb3beb88-4cb3-4ec0-ad7e-44c9eb33375f-articleLarge.jpg?quality=75\&auto=webp\&disable=upscale}

The number of Americans filing new claims for state unemployment
benefits totaled 1.43 million last week, the Labor Department
\href{https://oui.doleta.gov/press/2020/073020.pdf}{reported Thursday}.

It was the 19th straight week that the tally exceeded one million, an
unheard-of figure before the coronavirus pandemic. And it was the second
weekly increase in a row after nearly four months of declines, a sign of
how the rebound in cases has undercut the economy's nascent recovery.
Claims for the previous week totaled 1.42 million.

New claims for Pandemic Unemployment Assistance, the government's
program aimed at covering freelancers, the self-employed and other
workers not covered by traditional unemployment benefits, totaled
830,000, down from 975,000 the week before. Those numbers, unlike the
figures for state claims, are not seasonally adjusted.

Initial weekly unemployment claims,

both regular and those under the Pandemic Unemployment Assistance
program

6 million

5

4

3

2

1

0

Feb.

March

April

May

June

July

Initial weekly unemployment claims, both regular and those under the
Pandemic Unemployment Assistance program

6 million

5

4

3

2

1

0

Feb.

March

April

May

June

July

Pandemic Unemployment Assistance extends eligibility to some workers who
would not otherwise be able to apply for unemployment benefits, such as
part-time and self-employed workers. Regular claims are seasonally
adjusted but P.U.A. claims are not.

Source: Labor Department

By Ella Koeze

``We're still in a desperate situation,'' said Diane Swonk, chief
economist at the accounting firm Grant Thornton in Chicago. Noting that
weekly claims were in the 200,000 range before the pandemic brought
widespread shutdowns in March, she added, ``This is unique in terms of
the speed and magnitude of the job losses.''

What's more, fears are growing that after rebounding strongly in May and
June, the economy has run out of steam, with many states reversing the
reopening of businesses.

``Everyone wants to keep putting on rose-colored glasses, but it's
blinding us to the reality of the situation and what we have to deal
with,'' Ms. Swonk said.

At the same time, the \$600 supplemental weekly unemployment payment
from the federal government
\href{https://www.nytimes.com/2020/07/29/business/economy/unemployment-benefits-coronavirus.html}{is
ending}, a potentially crippling financial blow to millions. Republicans
have proposed replacing the supplement with a \$200 weekly payment,
while Democrats want to extend it in full. ``We're nowhere close to a
deal,'' Mark Meadows, the White House chief of staff,
\href{https://www.nytimes.com/2020/07/29/business/economy/virus-aid-trump.html}{said
Wednesday}.

--- \href{https://www.nytimes.com/by/nelson-d-schwartz}{Nelson D.
Schwartz}

\hypertarget{consumer-spending-plummeted-in-the-second-quarter-but-not-across-the-board}{%
\subsection{\texorpdfstring{\protect\hyperlink{consumer-spending-plummeted-in-the-second-quarter-but-not-across-the-board}{Consumer
spending plummeted in the second quarter, but not across the
board.}}{Consumer spending plummeted in the second quarter, but not across the board.}}\label{consumer-spending-plummeted-in-the-second-quarter-but-not-across-the-board}}

Copied to clipboard.

\includegraphics{https://static01.nyt.com/images/2020/07/30/business/30markets-gdp-numbersfolo/merlin_174720708_743258e8-06c3-4d13-b711-a6a03eafea11-articleLarge.jpg?quality=75\&auto=webp\&disable=upscale}

Consumer spending, the bedrock of the U.S. economy, plunged 10.1 percent
in the second quarter, the Commerce Department
\href{https://www.bea.gov/sites/default/files/2020-07/gdp2q20_adv.pdf}{reported
Thursday}. It was by far the biggest drop on record. But the decline
wasn't across the board --- and the details help paint a picture of life
in a pandemic.

Spending on services fell 13.3 percent, led by a near-total collapse in
spending on restaurant meals and recreation, the department's report on
quarterly economic output noted. Health care spending fell sharply, too,
as patients canceled elective procedures and delayed routine care.

Spending on goods was a different story. Overall goods expenditures fell
a modest 3 percent, and some quarantine-friendly categories actually had
increases. Spending on recreational vehicles and related goods rose
nearly 9 percent as consumers sought ways to travel without getting on
airplanes.

Other parts of the economy showed large contractions. Business
investment, residential construction and trade --- both imports and
exports --- all fell by double-digit percentages. One exception:
Spending by the federal government rose 4.1 percent as Congress moved to
prevent deeper economic damage. (That figure reflects only a small
fraction of the government stimulus efforts, much of which are
considered ``transfer payments'' that aren't counted in gross domestic
product.)

--- \href{https://www.nytimes.com/by/ben-casselman}{Ben Casselman}

\hypertarget{advertisement-2}{%
\subsubsection{Advertisement}\label{advertisement-2}}

\protect\hyperlink{after-dfp-ad-mid3}{Continue reading the main story}

\hypertarget{stocks-drop-as-economic-numbers-highlight-the-pandemics-toll}{%
\subsection{\texorpdfstring{\protect\hyperlink{stocks-drop-as-economic-numbers-highlight-the-pandemics-toll}{Stocks
drop as economic numbers highlight the pandemic's
toll.}}{Stocks drop as economic numbers highlight the pandemic's toll.}}\label{stocks-drop-as-economic-numbers-highlight-the-pandemics-toll}}

Copied to clipboard.

\includegraphics{https://static01.nyt.com/images/2020/07/30/world/30markets-brf-markets02sub/merlin_174877557_226a3d6b-d69c-414f-b1f9-9166bce0493a-articleLarge.jpg?quality=75\&auto=webp\&disable=upscale}

Stocks slid on Thursday as economic reports from the United States and
Germany showed the toll of the coronavirus outbreak on growth, but a
rally in shares of big technology companies, ahead of their earnings
reports, helped minimize the blow to Wall Street.

The S\&P 500 fell about half a percent, while shares in Europe were down
by more than 2 percent. The Nasdaq composite climbed as \textbf{Apple,
Amazon, Alphabet} and \textbf{Facebook} all rose. The largest technology
companies often set the direction of the broad market because of their
sheer size.

Oil prices were also lower, as were shares of energy companies.
\textbf{ConocoPhillips} slid after the company said its earnings plunged
by more than analysts had expected.

Financial stocks, closely tied to the cyclical ups and downs of the
American economy, slumped too, as long-term interest rates --- set by
the yields on government bonds --- continued to plumb some of the lowest
levels in history.

The yield on the 10-year Treasury note fell to 0.55 percent on Thursday
morning. Such yields help set the price of the loans banks make and
significantly influence their profitability.

The U.S. economy shrank by 9.5 percent in the second quarter, while
Germany's economy shrank by 10.1 percent. On an annualized basis, the
standard way of reporting quarterly economic data, U.S. gross domestic
product fell at a rate of 32.9 percent, which is the sharpest drop on
record.

Data released at the same time showed that 1.43 million Americans filed
new state unemployment claims, the second week in which that number has
risen and a figure that highlights the persistence of the economic
downturn.

The grim data came a day after Jerome H. Powell, the Federal Reserve
chair,
\href{https://www.nytimes.com/2020/07/29/business/economy/federal-reserve-meeting-interest-rates.html}{told
reporters} that the ``pace of recovery looks like it has slowed,''
pointing to debit and credit card spending and hiring trends. He added,
``The path forward for the economy is extraordinarily uncertain and will
depend in large part on our success in keeping the virus in check.''

--- \href{https://www.nytimes.com/by/kevin-granville}{Kevin Granville}
and \href{https://www.nytimes.com/by/matt-phillips}{Matt Phillips}

\hypertarget{comcast-saw-10-million-sign-ups-for-its-streaming-service-peacock}{%
\subsection{\texorpdfstring{\protect\hyperlink{comcast-saw-10-million-sign-ups-for-its-streaming-service-peacock}{Comcast
saw 10 million sign-ups for its streaming service
Peacock.}}{Comcast saw 10 million sign-ups for its streaming service Peacock.}}\label{comcast-saw-10-million-sign-ups-for-its-streaming-service-peacock}}

Copied to clipboard.

\includegraphics{https://static01.nyt.com/images/2020/07/30/business/30markets-brf-comcast/merlin_174119592_7cafa828-c257-466c-be1a-117015e817b5-articleLarge.jpg?quality=75\&auto=webp\&disable=upscale}

\textbf{Comcast}, the largest cable operator in the U.S., reported on
Thursday \$23.7 billion in revenue and \$7.9 billion in adjusted profit
for the second quarter, beating expectations. Here are the highlights:

\begin{itemize}
\tightlist
\item
  \textbf{Peacock}, its new streaming product, \textbf{attracted 10
  million sign-ups} in its first three months. It differs from other
  platforms like HBO Max (which
  \href{https://www.nytimes.com/2020/07/23/business/media/att-hbo-max.html}{netted
  4.1 million} in one month) and Netflix in that it is free and relies
  on advertising for revenue. (There is a paid tier that features more
  content but still includes ads.) The strategy is reminiscent of the
  original broadcast system,
  \href{https://www.nytimes.com/2019/01/31/business/locast-streaming-free-network-tv.html}{which
  is also free}. Comcast hopes to have 35 million users by 2024.
\end{itemize}

\begin{itemize}
\tightlist
\item
  With most of the country under lockdown, Comcast \textbf{added 323,000
  more broadband customers, but it lost 477,000 pay TV subscribers}.
  People switched to cheaper streaming alternatives as wallets tightened
  under the pandemic. It's not a bad trade for Comcast, since a
  broadband subscriber tends to add more profit than a video one.
\end{itemize}

\begin{itemize}
\tightlist
\item
  At \textbf{NBCUniversal}, the lack of sports and the shutdown of movie
  theaters and theme parks hurt the division. Sales fell 25 percent to
  \$6.1 billion. \textbf{Theme parks took a \$399 million loss} for the
  quarter, and the Universal Studios division saw sales decline nearly a
  fifth to \$1.2 billion.
\end{itemize}

\begin{itemize}
\tightlist
\item
  But a
  \href{https://www.nytimes.com/2020/07/28/business/media/universal-amc-movies-at-home.html}{significant
  deal} was struck this week between \textbf{Universal and AMC
  Entertainment}, the nation's largest theater chain, that could recast
  the economics of the film industry. The studio can now sell movies on
  streaming 17 days after it runs in theaters, collapsing the usual
  90-day window. Movies tend to make most of its box office dollars in
  the first two weekends, so the new terms appear to benefit the studio.
  In other words, there will be more reasons to stay home.
\end{itemize}

--- \href{https://www.nytimes.com/by/edmund-lee}{Edmund Lee}

\hypertarget{what-else-is-happening-united-warns-of-more-furloughs-california-pizza-kitchen-files-for-bankruptcy}{%
\subsection{\texorpdfstring{\protect\hyperlink{what-else-is-happening-united-warns-of-more-furloughs-california-pizza-kitchen-files-for-bankruptcy}{What
else is happening: United warns of more furloughs, California Pizza
Kitchen files for
bankruptcy.}}{What else is happening: United warns of more furloughs, California Pizza Kitchen files for bankruptcy.}}\label{what-else-is-happening-united-warns-of-more-furloughs-california-pizza-kitchen-files-for-bankruptcy}}

Copied to clipboard.

\begin{itemize}
\item
  \textbf{United Airlines} warned its pilots on Thursday that it might
  need to expand planned furloughs if demand for flights remained deeply
  depressed and a vaccine was not mass produced by the end of next year.
  The airline previously said that it could furlough up to one third of
  its pilots, or 3,900 people, this year and next. ``That may not prove
  to be enough,'' an executive said in a memo to pilots.
\item
  \textbf{California Pizza Kitchen} filed for bankruptcy protection in
  Texas on Thursday. The company, which operates more than 200 locations
  in the United States and internationally, said it would use the
  restructuring process to close unprofitable locations and cut debt,
  and planned to emerge from bankruptcy in less than three months. The
  company is the latest dining chain to file for Chapter 11 protection
  during the pandemic, following Chuck E. Cheese's parent company, CEC
  Entertainment, and NPC International, the largest U.S. franchisee of
  Pizza Hut.
\item
  \textbf{Volkswagen} said on Thursday it fell into the red during the
  first six months of 2020 after sales plunged 23 percent compared with
  a year earlier. But the company, the world's largest carmaker, said
  vehicle sales, which were down by more than half in May, had begun to
  recover.
\item
  \textbf{Airbus} reported a big loss for the first half and vowed to
  conserve cash; \textbf{AstraZeneca} reported a 26 percent rise in
  earnings for its first half as sales of new drugs beat forecasts;
  \textbf{Credit Suisse} beat expectations, thanks to a surge in trading
  revenue; trading also aided \textbf{Shell}, which reported a
  smaller-than-expected loss, and \textbf{Total}, which disclosed a
  surprise profit; and \textbf{Nestlé} announced an 18 percent rise in
  first-half profit but warned of slowing growth for the rest of the
  year.
\end{itemize}

\hypertarget{site-index}{%
\subsection{Site Index}\label{site-index}}

\hypertarget{site-information-navigation}{%
\subsection{Site Information
Navigation}\label{site-information-navigation}}

\begin{itemize}
\tightlist
\item
  \href{https://help.nytimes.com/hc/en-us/articles/115014792127-Copyright-notice}{©~2020~The
  New York Times Company}
\end{itemize}

\begin{itemize}
\tightlist
\item
  \href{https://www.nytco.com/}{NYTCo}
\item
  \href{https://help.nytimes.com/hc/en-us/articles/115015385887-Contact-Us}{Contact
  Us}
\item
  \href{https://www.nytco.com/careers/}{Work with us}
\item
  \href{https://nytmediakit.com/}{Advertise}
\item
  \href{http://www.tbrandstudio.com/}{T Brand Studio}
\item
  \href{https://www.nytimes.com/privacy/cookie-policy\#how-do-i-manage-trackers}{Your
  Ad Choices}
\item
  \href{https://www.nytimes.com/privacy}{Privacy}
\item
  \href{https://help.nytimes.com/hc/en-us/articles/115014893428-Terms-of-service}{Terms
  of Service}
\item
  \href{https://help.nytimes.com/hc/en-us/articles/115014893968-Terms-of-sale}{Terms
  of Sale}
\item
  \href{https://spiderbites.nytimes.com}{Site Map}
\item
  \href{https://help.nytimes.com/hc/en-us}{Help}
\item
  \href{https://www.nytimes.com/subscription?campaignId=37WXW}{Subscriptions}
\end{itemize}
