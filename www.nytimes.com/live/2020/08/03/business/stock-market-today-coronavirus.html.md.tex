Sections

SEARCH

\protect\hyperlink{site-content}{Skip to
content}\protect\hyperlink{site-index}{Skip to site index}

\href{https://myaccount.nytimes.com/auth/login?response_type=cookie\&client_id=vi}{}

\href{https://www.nytimes.com/section/todayspaper}{Today's Paper}

\href{https://www.nytimes.com/news-event/coronavirus}{The Coronavirus
Outbreak}

\begin{itemize}
\tightlist
\item
  live\href{https://www.nytimes.com/2020/08/03/world/coronavirus-covid-19.html}{Latest
  Updates}
\item
  \href{https://www.nytimes.com/interactive/2020/us/coronavirus-us-cases.html}{Maps
  and Cases}
\item
  \href{https://www.nytimes.com/interactive/2020/science/coronavirus-vaccine-tracker.html}{Vaccine
  Tracker}
\item
  \href{https://www.nytimes.com/2020/08/02/us/covid-college-reopening.html}{College
  Reopening}
\item
  \href{https://www.nytimes.com/live/2020/08/03/business/stock-market-today-coronavirus}{Economy}
\end{itemize}

Last Updated

Aug. 3, 2020, 11:20 p.m. ET

Aug. 3, 2020, 11:20 p.m. ET

\hypertarget{boeing-737-max-moves-one-step-closer-to-flying-again}{%
\section{Boeing 737 Max Moves One Step Closer to Flying
Again}\label{boeing-737-max-moves-one-step-closer-to-flying-again}}

\hypertarget{heres-what-you-need-to-know}{%
\subsubsection{Here's what you need to
know:}\label{heres-what-you-need-to-know}}

\begin{itemize}
\item
  \protect\hyperlink{faa-says-boeing-has-effectively-mitigated-defects-in-the-737-max}{}

  F.A.A. says Boeing has `effectively mitigated' defects in the 737 Max.
\item
  \protect\hyperlink{small-businesses-got-emergency-loans-but-not-what-they-expected}{}

  Small businesses got emergency loans, but not what they expected.
\item
  \protect\hyperlink{covid-19-misinformation-websites-benefit-from-google-and-amazon-ad-networks-study-says}{}

  Covid-19 misinformation websites benefit from Google and Amazon ad
  networks, study says.
\item
  \protect\hyperlink{tech-rally-lifts-markets-and-the-nasdaq-reaches-a-record}{}

  Tech rally lifts markets, and the Nasdaq reaches a record.
\item
  \protect\hyperlink{lord-taylor-and-the-owner-of-mens-wearhouse-file-for-bankruptcy}{}

  Lord \& Taylor and the owner of Men's Wearhouse file for bankruptcy.
\end{itemize}

\hypertarget{faa-says-boeing-has-effectively-mitigated-defects-in-the-737-max}{%
\subsection{\texorpdfstring{\protect\hyperlink{faa-says-boeing-has-effectively-mitigated-defects-in-the-737-max}{F.A.A.
says Boeing has `effectively mitigated' defects in the 737
Max.}}{F.A.A. says Boeing has `effectively mitigated' defects in the 737 Max.}}\label{faa-says-boeing-has-effectively-mitigated-defects-in-the-737-max}}

Copied to clipboard.

\includegraphics{https://static01.nyt.com/images/2020/08/03/business/03Markets-Brf-Boeing/merlin_158452899_47c48d86-bf32-4c5f-9c78-c2d814405b4a-articleLarge.jpg?quality=75\&auto=webp\&disable=upscale}

The Federal Aviation Administration on Monday proposed changes that
\textbf{Boeing} must make to the 737 Max, potentially clearing the way
for
\href{https://www.nytimes.com/2020/07/15/business/boeing-737-max-return.html}{the
plane to start flying again} by the end of the year.

The changes include updating the plane's flight control software,
revising crew procedures and rerouting internal wiring. Once formally
published, the proposal will be open to public comment for 45 days,
after which the agency will issue a final ruling.

The agency concluded in a related report published on Monday that its
proposal was in line with Boeing's recommendations. The report said the
company's recommendations had sufficiently addressed the problems that
contributed to two fatal crashes, resulting in the worldwide grounding
of the jet.

``The F.A.A. has preliminarily determined that Boeing's proposed changes
to the 737 Max design, flight crew procedures and maintenance procedures
effectively mitigate the airplane-related safety issues that
contributed'' to crashes in Indonesia and Ethiopia that killed 346
people, the agency said.

The Max has been grounded since March 2019,
\href{https://www.nytimes.com/2020/01/29/business/boeing-737-max-costs.html}{costing
Boeing billions} of dollars.

Once the F.A.A.'s proposal is official, Boeing can begin to make the
changes and ready the planes for flight, a process that could take more
than a week per jet and involves system checks, deep cleaning and
software updates. The company will have to do that for hundreds of
planes that customers have already received and hundreds more that
Boeing has made but not delivered.

Several other obstacles remain before the F.A.A. lifts its grounding
order on the plane, including the development of pilot training
requirements and the review and filing of additional documentation.

--- \href{https://www.nytimes.com/by/niraj-chokshi}{Niraj Chokshi}

\hypertarget{the-chicago-fed-president-says-its-up-to-congress-to-save-the-economy}{%
\subsection{\texorpdfstring{\protect\hyperlink{the-chicago-fed-president-says-its-up-to-congress-to-save-the-economy}{The
Chicago Fed president says it's up to Congress to save the
economy.}}{The Chicago Fed president says it's up to Congress to save the economy.}}\label{the-chicago-fed-president-says-its-up-to-congress-to-save-the-economy}}

Copied to clipboard.

\includegraphics{https://static01.nyt.com/images/2020/08/03/business/03markets-brf-fed/merlin_174643971_c2bde9e9-1cb8-4956-987a-7f1f383f28fc-articleLarge.jpg?quality=75\&auto=webp\&disable=upscale}

The president of the \textbf{Federal Reserve Bank of Chicago} said the
central bank had limited room to do more and that Congress would need to
\href{https://www.nytimes.com/2020/08/03/business/fed-official-says-a-hard-lockdown-could-get-virus-under-control.html}{support
the economy} if the United States faced a full-fledged second wave of
coronavirus infections.

``The punchline ought to be that the ball is in Congress's court,''
Charles Evans, president of the Chicago Fed, said on a call with
reporters.

The Federal Reserve cut its primary interest rate, the federal funds
rate, to
\href{https://www.nytimes.com/2020/07/29/business/economy/federal-reserve-meeting-interest-rates.html}{near-zero}
in March. Central banks abroad have cut borrowing costs into negative
territory, but Fed officials have been consistently skeptical that such
policies will be effective.

``We can't lower the funds rate. Negative interest rates aren't going to
be the tool that we decide to use at this point, or probably at any
point,'' Mr. Evans said.

The Fed might be able to use a policy that would cap certain interest
rates --- an approach commonly called ``yield curve control'' --- but
Mr. Evans suggested that slightly higher medium-term rates are not the
real economic problem. The Fed has the ability to offer
\href{https://www.nytimes.com/2020/07/28/business/economy/coronavirus-federal-reserve-policy.html}{emergency
loans to backstop turbulent markets}, but businesses and governments
might need grants to make it through.

``At the moment, it's really fiscal policy that needs to be addressing
this,'' he said. Even now, more congressional support is needed to shore
up the economy as the pandemic wears on, Mr. Evans said.

The Chicago Fed chief was not alone in arguing that recently lapsed
expanded unemployment benefits should be in some way addressed, as
\href{https://www.nytimes.com/2020/08/02/us/politics/coronavirus-jobless-aid.html}{Congress
debates}the future of its pandemic response.

Thomas Barkin, president of the \textbf{Federal Reserve Bank of
Richmond},
\href{https://www.reuters.com/article/usa-fed-barkin/feds-barkin-says-economy-faces-sinkhole-without-more-fiscal-support-idUSW1N2B8003}{warned
that}the pandemic might be creating an economic ``sinkhole,'' rather
than the pothole policymakers initially believed they were facing.
Robert S. Kaplan, president of the \textbf{Federal Reserve Bank of
Dallas}, said in an interview on Bloomberg television on Monday that the
policies
\href{https://www.bloomberg.com/news/articles/2020-08-03/extension-of-jobless-benefits-to-buoy-growth-fed-s-kaplan-says?srnd=premium\&sref=oZtxD6sa}{had
helped} to bolster consumers, adding that ``it's important that we see
an extension of it.''

--- \href{https://www.nytimes.com/by/jeanna-smialek}{Jeanna Smialek}

\hypertarget{advertisement}{%
\subsubsection{Advertisement}\label{advertisement}}

\protect\hyperlink{after-dfp-ad-mid1}{Continue reading the main story}

\hypertarget{small-businesses-got-emergency-loans-but-not-what-they-expected}{%
\subsection{\texorpdfstring{\protect\hyperlink{small-businesses-got-emergency-loans-but-not-what-they-expected}{Small
businesses got emergency loans, but not what they
expected.}}{Small businesses got emergency loans, but not what they expected.}}\label{small-businesses-got-emergency-loans-but-not-what-they-expected}}

Copied to clipboard.

\includegraphics{https://static01.nyt.com/images/2020/08/04/business/00sba-disasterloan1/merlin_174739206_54c92148-8b18-476d-8b70-74438b7780d2-articleLarge.jpg?quality=75\&auto=webp\&disable=upscale}

For nearly 70 years, the Small Business Administration's disaster relief
program has helped companies recover from catastrophes including
wildfires, hurricanes and earthquakes. But it has never faced anything
like \href{https://www.nytimes.com/news-event/coronavirus}{the
coronavirus crisis}.

Besieged by more than eight million applicants --- and operating in the
shadow of the hastily assembled
\href{https://www.nytimes.com/2020/04/26/business/ppp-small-business-loans.html}{Paycheck
Protection Program} --- the disaster relief effort has given out more
money in the past few months than it had in its entire history.

But the demand has created a problem that is hobbling hundreds of
thousands of applicants: The agency, afraid of running out of cash,
capped its coronavirus loans at a fraction of what companies can
normally borrow --- even though the program has handed out less than
half the \$360 billion it can lend.

Caroline Keefer, a clothing designer in Los Angeles, had expected to
qualify for a loan of at least \$500,000 based on a complex formula
devised by the agency. But when her loan offer arrived in May, it was
for \$150,000 --- the ceiling the S.B.A. quietly put in place that
month. Qualified companies can usually take loans of
\href{https://www.sba.gov/about-sba/sba-newsroom/press-releases-media-advisories/sba-provide-disaster-assistance-loans-small-businesses-impacted-coronavirus-covid-19}{up
to \$2 million}.

The limit has crimped Ms. Keefer's efforts to salvage a business that
did \$2 million in sales last year. Her company,
\href{https://www.riverandskycalifornia.com/}{River + Sky}, sells
directly to merchants like boutiques, department stores and hotel spa
shops.

Nearly 400,000 businesses have run into the \$150,000 limit, according
to
\href{https://www.sba.gov/funding-programs/loans/coronavirus-relief-options/economic-injury-disaster-loans\#section-header-5}{the
agency's data}. S.B.A. representatives declined to comment on the cap or
why it was imposed.

--- \href{https://www.nytimes.com/by/stacy-cowley}{Stacy Cowley}

\hypertarget{covid-19-misinformation-websites-benefit-from-google-and-amazon-ad-networks-study-says}{%
\subsection{\texorpdfstring{\protect\hyperlink{covid-19-misinformation-websites-benefit-from-google-and-amazon-ad-networks-study-says}{Covid-19
misinformation websites benefit from Google and Amazon ad networks,
study
says.}}{Covid-19 misinformation websites benefit from Google and Amazon ad networks, study says.}}\label{covid-19-misinformation-websites-benefit-from-google-and-amazon-ad-networks-study-says}}

Copied to clipboard.

\includegraphics{https://static01.nyt.com/images/2020/08/03/business/03markets-disinfo-study/merlin_175010625_448d75b2-ac99-4191-8dac-8609e49f60c6-articleLarge.jpg?quality=75\&auto=webp\&disable=upscale}

Websites publishing coronavirus-related misinformation are being
supported financially by tapping into internet advertising networks
owned by \textbf{Google} and \textbf{Amazon},
\href{https://comprop.oii.ox.ac.uk/research/covid19-disinfo-seo/}{according
to a new Oxford University study}.

Google and Amazon operate two of the world's largest internet
advertising networks and serve as central repositories for millions of
ads that are distributed to websites around the world. But because the
systems are largely automated, they are vulnerable to abuse, said Philip
Howard, director of the Oxford Internet Institute, who co-wrote the
study.

Social media platforms like \textbf{Facebook}, \textbf{Twitter} and
\textbf{YouTube} have been widely criticized as being the primary tool
for sharing coronavirus-related misinformation. But the researchers said
the advertising networks provide the financial oxygen to websites that
are publishing much of the dubious and misleading information about the
pandemic.

Mr. Howard, who previously assisted the Senate investigation into
Russian disinformation efforts, said Google and Amazon should consider
developing a blacklist that blocks website with a history of sharing
false and misleading information about the pandemic from being able to
use their advertising networks.

He said the pandemic had spawned a niche world of websites that ``preys
on people's insecurities.'' Many of the sites sell products promising to
cure or prevent the disease. Even though the sites have a relatively low
audience, they muddy the information ecosystem and undermine the broader
public health response.

``They are hucksters, fraudsters, peddling misinformation,'' he said.
``It would be a public service to take them down.''

Amazon and Google did not respond to requests for comment.

--- \href{https://www.nytimes.com/by/adam-satariano}{Adam Satariano}

\hypertarget{tech-rally-lifts-markets-and-the-nasdaq-reaches-a-record}{%
\subsection{\texorpdfstring{\protect\hyperlink{tech-rally-lifts-markets-and-the-nasdaq-reaches-a-record}{Tech
rally lifts markets, and the Nasdaq reaches a
record.}}{Tech rally lifts markets, and the Nasdaq reaches a record.}}\label{tech-rally-lifts-markets-and-the-nasdaq-reaches-a-record}}

Copied to clipboard.

Stocks rallied on Monday, led higher by large technology companies, as
investors weighed a mixed bag of business news against continuing
concerns about the spread of the coronavirus.

The S\&P 500 rose nearly 1 percent, and the tech-heavy Nasdaq composite
climbed 1.5 percent and hit a record. Shares in Europe and Asia also
rose.

After the gain on Monday, the S\&P 500 is now less than 3 percent below
its high, which it reached in late February before the rapid spread of
the coronavirus and concern about the economic damage it would cause
sent markets into a tailspin. The index ended July with its fourth
consecutive monthly gain.

A big factor behind the recovery has been the success of big technology
companies like \textbf{Amazon} and \textbf{Apple}, which have thrived
during the pandemic as demand for their products and services rose with
consumers stuck at home. Last week, several of the largest technology
companies reported blockbuster earnings.

On Monday, it was \textbf{Microsoft} that led the big-tech rally. The
stock rose more than 5 percent after President Trump said that
\href{https://www.nytimes.com/2020/08/03/technology/trump-tiktok-microsoft.html}{Microsoft
could pursue an acquisition} of \textbf{TikTok} in the United States.
The president's statement came after a weekend of headlines about the
potential takeover of the video sharing platform, which Mr. Trump has
said is a national security threat because it is owned by a Chinese
company.

Speaking at the White House on Monday, Mr. Trump said that TikTok would
shut down on Sept. 15 unless Microsoft or another company purchased it.

Also lifting market sentiment on Monday, the Institute for Supply
Management said its measure of manufacturing activity in the United
States rose for a second consecutive month in July, and that ``sentiment
was generally optimistic'' among manufacturers as orders increased.
Similarly, the
\href{https://www.markiteconomics.com/Public/Home/PressRelease/c4e32989182e4296964138d78fcc1305}{IHT
Purchasing Managers' Index} for manufacturing in the euro area reflected
the first expansion in activity since early 2019.

Still,
\href{https://www.nytimes.com/2020/08/02/world/coronavirus-covid-19.html?action=click\&module=Top\%20Stories\&pgtype=Homepage}{concerns
about the spread of the virus} continued through the weekend, with
tightening restrictions in Manila and Melbourne, Australia. In the
United States, Dr. Deborah L. Birx, the Trump administration's
coronavirus coordinator, said that the country was in a ``new phase'' of
the pandemic, and that it was much more extensive than the spring
outbreaks in major cities like New York and Seattle.

--- \href{https://www.nytimes.com/by/kevin-granville}{Kevin Granville}
and Mohammed Hadi

\hypertarget{advertisement-1}{%
\subsubsection{Advertisement}\label{advertisement-1}}

\protect\hyperlink{after-dfp-ad-mid2}{Continue reading the main story}

\hypertarget{lord--taylor-and-the-owner-of-mens-wearhouse-file-for-bankruptcy}{%
\subsection{\texorpdfstring{\protect\hyperlink{lord-taylor-and-the-owner-of-mens-wearhouse-file-for-bankruptcy}{Lord
\& Taylor and the owner of Men's Wearhouse file for
bankruptcy.}}{Lord \& Taylor and the owner of Men's Wearhouse file for bankruptcy.}}\label{lord--taylor-and-the-owner-of-mens-wearhouse-file-for-bankruptcy}}

Copied to clipboard.

\includegraphics{https://static01.nyt.com/images/2020/08/02/business/02LordandTaylor-Bankruptcy1/02LordandTaylor-Bankruptcy1-articleLarge.jpg?quality=75\&auto=webp\&disable=upscale}

\textbf{Lord \& Taylor} and the company behind \textbf{Men's Wearhouse}
and \textbf{Jos. A. Bank} filed for bankruptcy protection on Sunday, the
latest American retailers to fall victim to the coronavirus outbreak.

The department store chain
\href{https://www.nytimes.com/2020/08/02/business/Lord-and-Taylor-Bankruptcy.html}{Lord
\& Taylor} traces its roots to 1826, and had been floundering for years.
\textbf{\href{https://www.nytimes.com/2020/08/03/business/tailored-brands-mens-wearhouse-bankruptcy.html}{Tailored
Brands}}\href{https://www.nytimes.com/2020/08/03/business/tailored-brands-mens-wearhouse-bankruptcy.html}{,}
which once dominated the market for men's suits through Men's Wearhouse
and Jos. A. Bank, saw demand plummet for its corporate clothing with the
pandemic keeping America's office workers at home.

They join a roster of bankruptcy filings since the beginning of May that
includes
\textbf{\href{https://www.nytimes.com/2020/05/07/business/neiman-marcus-bankruptcy.html}{Neiman
Marcus}}, ****
\textbf{\href{https://www.nytimes.com/2020/05/03/business/j-crew-bankruptcy-coronavirus.html}{J.
Crew}},
\textbf{\href{https://www.nytimes.com/2020/05/15/business/jc-penney-bankruptcy-coronavirus.html}{J.C.
Penney}},
\textbf{\href{https://www.nytimes.com/2020/07/08/business/brooks-brothers-chapter-11-bankruptcy.html}{Brooks
Brothers}} and the owner of
\textbf{\href{https://www.nytimes.com/2020/07/23/business/ascena-bankruptcy-ann-taylor-lane-bryant.html}{Ann
Taylor}}\href{https://www.nytimes.com/2020/07/23/business/ascena-bankruptcy-ann-taylor-lane-bryant.html}{}and
\textbf{Loft}.

Tailored Brands had approximately 1,400 stores and 18,000 employees. It
had already
\href{https://www.businesswire.com/news/home/20200721005319/en/Tailored-Brands-Announces-Plans-Reduce-Headcount-Close}{announced
plans} in July to eliminate 20 percent of its corporate jobs and close
up to 500 stores, and on Sunday, the company said that it planned to use
the restructuring process to cut its debt by at least \$630 million.

Lord \& Taylor was
\href{https://www.nytimes.com/2019/08/28/business/lord-taylor-sold-le-tote.html}{acquired
last year} by the clothing rental start-up \textbf{Le Tote} in an
unusual \$100 million deal. Now Le Tote and Lord \& Taylor are both
seeking Chapter 11 protection from their creditors. The companies said
in a filing on Sunday that they operated 38 locations, which had been
temporarily closed since March.

Abby Homer, a representative for Le Tote and Lord \& Taylor, declined to
comment on the filing. The filing said that the company had 651
employees, which appears to exclude Lord \& Taylor's thousands of store
workers, and brought in around \$250 million in revenue last year.

--- \href{https://www.nytimes.com/by/sapna-maheshwari}{Sapna Maheshwari}
and Gillian Friedman

\hypertarget{interviewing-for-a-job-online-here-are-some-tips}{%
\subsection{\texorpdfstring{\protect\hyperlink{interviewing-for-a-job-online-here-are-some-tips}{Interviewing
for a job online? Here are some
tips.}}{Interviewing for a job online? Here are some tips.}}\label{interviewing-for-a-job-online-here-are-some-tips}}

Copied to clipboard.

Just as \href{https://www.nytimes.com/news-event/coronavirus}{the
coronavirus pandemic} emptied offices, in many cases it also did away
with in-person job interviews. Acing a job interview conducted over Zoom
or Google Hangout isn't easy. The good news is that the interview basics
still apply --- such as researching the company and thinking ahead on
the questions you might be asked.

The Times's Julie Weed
\href{https://www.nytimes.com/2020/08/03/business/online-job-interview-tips.html}{shares}
how to adapt classic interview techniques to the new world of internet
interviews.

--- Julie Weed

\hypertarget{hsbc-is-caught-in-the-middle-as-china-and-the-west-do-battle}{%
\subsection{\texorpdfstring{\protect\hyperlink{hsbc-is-caught-in-the-middle-as-china-and-the-west-do-battle}{HSBC
is caught in the middle as China and the West do
battle.}}{HSBC is caught in the middle as China and the West do battle.}}\label{hsbc-is-caught-in-the-middle-as-china-and-the-west-do-battle}}

Copied to clipboard.

\includegraphics{https://static01.nyt.com/images/2020/07/31/business/00hongkong-biz-1/merlin_158775735_fef2c85e-434f-4236-9e23-41d2471c43a1-articleLarge.jpg?quality=75\&auto=webp\&disable=upscale}

Like Hong Kong, \textbf{HSBC} has long sat at the crossroads between
East and West, a big global bank based in Britain that has reveled in
and profited from its deep relationship with China. And like Hong Kong,
it is now caught in a new era of confrontation between Beijing and major
Western governments.

In China, HSBC has been accused of ``setting traps'' to ensnare the
Chinese tech giant, \textbf{Huawei}. In Britain, it has been admonished
for seeming to back Huawei's ambitions in the country.

Straddling neutral ground is no longer an option.
\href{https://www.nytimes.com/2020/05/31/business/hong-kong-china-business.html}{HSBC
got called out} in China for not publicly backing the new national
security law in Hong Kong. When the bank
\href{https://www.nytimes.com/2020/06/03/business/china-hong-kong-damage.html}{eventually
expressed support} on its Chinese social media account, members of the
British Parliament demanded an explanation and urged HSBC to rescind the
statement.

Global businesses are increasingly under pressure to pick sides as the
United States and its allies target the political and economic agenda of
China.

``There are multiple tailwinds pushing the global business world toward
this highly geopolitically sensitive environment where the landscape has
shifted fundamentally, and you can no longer be agnostic,'' said Jude
Blanchette, a China scholar at the Center for Strategic and
International Studies in Washington. ``It is the logical extension of
this new paradigm where economic security is now considered national
security.''

--- \href{https://www.nytimes.com/by/alexandra-stevenson}{Alexandra
Stevenson}

\hypertarget{advertisement-2}{%
\subsubsection{Advertisement}\label{advertisement-2}}

\protect\hyperlink{after-dfp-ad-mid3}{Continue reading the main story}

\hypertarget{overcrowded-housing-invites-covid-19-even-in-silicon-valley}{%
\subsection{\texorpdfstring{\protect\hyperlink{overcrowded-housing-invites-covid-19-even-in-silicon-valley}{Overcrowded
housing invites Covid-19, even in Silicon
Valley.}}{Overcrowded housing invites Covid-19, even in Silicon Valley.}}\label{overcrowded-housing-invites-covid-19-even-in-silicon-valley}}

Copied to clipboard.

\includegraphics{https://static01.nyt.com/images/2020/08/02/business/01virus-crowding4/merlin_174571227_6cc85450-b2d1-408f-8524-0bb45729a4e6-articleLarge.jpg?quality=75\&auto=webp\&disable=upscale}

Overcrowding, not density, has defined many coronavirus hot spots.
Service workers' quarters skirting Silicon Valley are no exception. The
Times's Conor Dougherty
\href{https://www.nytimes.com/2020/08/01/business/economy/housing-overcrowding-coronavirus.html}{reports}:

\begin{quote}
It was not surprising when three-quarters of the house tested positive.
There were 12 people in three bedrooms, with a bathroom whose door
frequently required a knock and a kitchen where dinnertime shifts
extended from 5 p.m. well into the evening.

Karla Lorenzo, a Guatemalan immigrant who cleaned houses in San
Francisco and Silicon Valley, lived in the big room along the driveway.
Big is a relative term when a room has five people in it. She and her
partner, Abel, slept in a queen-size bed along the wall. There was a
crib for the baby at the foot, with the older children's bunk bed next
to that. The other housemates had similar layouts.

Living among many people, as Ms. Lorenzo put it in Spanish, you cannot
really avoid your housemates. The sounds, the smells, the moods ---
everyone is pressed against all of it, and they understood that if one
of them got \href{https://www.nytimes.com/news-event/coronavirus}{the
coronavirus}, the rest probably would.

That happened in April, and now the house is returning to health. Abel,
referred to by his first name because his immigration status is
uncertain, is home after three weeks in the hospital, where Ms. Lorenzo
feared he would die alone gasping for air. And she is no longer
squirreled in the closet where she spent days to avoid giving the virus
to the children.

Now comes a second struggle: figuring out how to pay rent.
\end{quote}

--- \href{https://www.nytimes.com/by/conor-dougherty}{Conor Dougherty}

\hypertarget{what-else-is-happening-jc-penney-will-close-for-thanksgiving}{%
\subsection{\texorpdfstring{\protect\hyperlink{what-else-is-happening-jc-penney-will-close-for-thanksgiving}{What
else is happening: J.C. Penney will close for
Thanksgiving.}}{What else is happening: J.C. Penney will close for Thanksgiving.}}\label{what-else-is-happening-jc-penney-will-close-for-thanksgiving}}

Copied to clipboard.

\includegraphics{https://static01.nyt.com/images/2020/08/03/business/03markets-brf-jcpenney/03markets-brf-jcpenney-articleLarge.jpg?quality=75\&auto=webp\&disable=upscale}

\begin{itemize}
\item
  \textbf{J.C. Penney}, the cornerstone of American malls, was the
  latest retailer to announce it would not open Thanksgiving Day this
  year. Other major chains, including \textbf{Walmart} and
  \textbf{Target}, have said they would push the start of holiday
  shopping back to Black Friday, often casting the change as in honor of
  the work of their front-line employees. J.C. Penney, which filed for
  bankruptcy in May,
  \href{https://companyblog.jcpnewsroom.com/2020/08/03/jcpenney-to-close-stores-on-thanksgiving-day-2020/}{said
  in a statement} that the move was intended to allow both associates
  and customers ``to stay safe, relax, and enjoy the day.''
\item
  \textbf{HSBC}, Europe's largest bank, reported a 96 percent drop in
  profit in the second quarter to \$192 million, as the bank increased
  provisions for bad loans by \$3.8 billion. The bank, which does almost
  half of its business in Asia, also confirmed it would speed up a
  restructuring plan that would cut 35,000 jobs.
\item
  \textbf{Marathon Petroleum}, the largest U.S. independent refiner,
  announced Sunday that it had sold its
  \href{https://www.nytimes.com/2020/08/02/business/marathon-petroleum-speedway-7-11.html}{Speedway
  gas station chain} to the Japanese retail group that owns
  \textbf{7-Eleven} for \$21 billion in cash. The sale of Speedway, one
  of the country's largest convenience store chains with nearly 4,000
  outlets, is the biggest corporate deal in the oil sector since the
  coronavirus slashed demand for fuel early this year.
\end{itemize}

\hypertarget{site-index}{%
\subsection{Site Index}\label{site-index}}

\hypertarget{site-information-navigation}{%
\subsection{Site Information
Navigation}\label{site-information-navigation}}

\begin{itemize}
\tightlist
\item
  \href{https://help.nytimes.com/hc/en-us/articles/115014792127-Copyright-notice}{©~2020~The
  New York Times Company}
\end{itemize}

\begin{itemize}
\tightlist
\item
  \href{https://www.nytco.com/}{NYTCo}
\item
  \href{https://help.nytimes.com/hc/en-us/articles/115015385887-Contact-Us}{Contact
  Us}
\item
  \href{https://www.nytco.com/careers/}{Work with us}
\item
  \href{https://nytmediakit.com/}{Advertise}
\item
  \href{http://www.tbrandstudio.com/}{T Brand Studio}
\item
  \href{https://www.nytimes.com/privacy/cookie-policy\#how-do-i-manage-trackers}{Your
  Ad Choices}
\item
  \href{https://www.nytimes.com/privacy}{Privacy}
\item
  \href{https://help.nytimes.com/hc/en-us/articles/115014893428-Terms-of-service}{Terms
  of Service}
\item
  \href{https://help.nytimes.com/hc/en-us/articles/115014893968-Terms-of-sale}{Terms
  of Sale}
\item
  \href{https://spiderbites.nytimes.com}{Site Map}
\item
  \href{https://help.nytimes.com/hc/en-us}{Help}
\item
  \href{https://www.nytimes.com/subscription?campaignId=37WXW}{Subscriptions}
\end{itemize}
