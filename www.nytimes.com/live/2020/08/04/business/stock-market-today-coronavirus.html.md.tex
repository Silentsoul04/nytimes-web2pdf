Sections

SEARCH

\protect\hyperlink{site-content}{Skip to
content}\protect\hyperlink{site-index}{Skip to site index}

\href{https://myaccount.nytimes.com/auth/login?response_type=cookie\&client_id=vi}{}

\href{https://www.nytimes.com/section/todayspaper}{Today's Paper}

\href{https://www.nytimes.com/news-event/coronavirus}{The Coronavirus
Outbreak}

\begin{itemize}
\tightlist
\item
  live\href{https://www.nytimes.com/2020/08/04/world/coronavirus-cases.html}{Latest
  Updates}
\item
  \href{https://www.nytimes.com/interactive/2020/us/coronavirus-us-cases.html}{Maps
  and Cases}
\item
  \href{https://www.nytimes.com/interactive/2020/science/coronavirus-vaccine-tracker.html}{Vaccine
  Tracker}
\item
  \href{https://www.nytimes.com/2020/08/02/us/covid-college-reopening.html}{College
  Reopening}
\item
  \href{https://www.nytimes.com/live/2020/08/04/business/stock-market-today-coronavirus}{Economy}
\end{itemize}

LiveUpdated~

Aug. 4, 2020, 9:40 p.m. ET

Aug. 4, 2020, 9:40 p.m. ET

\hypertarget{disney-hits-58-million-subscribers}{%
\section{Disney+ Hits 58 Million
Subscribers}\label{disney-hits-58-million-subscribers}}

This briefing is no longer being updated. Follow live updates
\href{https://www.nytimes.com/2020/08/04/world/coronavirus-cases.html}{here}.

RIGHT NOW

\protect\hyperlink{fox-corporations-plunging-profit-is-cushioned-by-fox-news}{Fox
Corporation's plunging profit is cushioned by Fox News.}

\hypertarget{heres-what-you-need-to-know}{%
\subsubsection{Here's what you need to
know:}\label{heres-what-you-need-to-know}}

\begin{itemize}
\item
  \protect\hyperlink{disney-lost-4-7-billion-last-quarter-but-its-newest-business-was-a-big-hit}{}

  Disney lost \$4.7 billion last quarter, but its newest business was a
  big hit.
\item
  \protect\hyperlink{trading-in-kodak-shares-comes-under-scrutiny}{}

  Trading in Kodak shares comes under scrutiny.
\item
  \protect\hyperlink{nbcuniversal-to-cut-about-10-percent-of-its-work-force}{}

  NBCUniversal to cut about 10 percent of its work force.
\item
  \protect\hyperlink{the-ad-giant-publicis-has-parted-ways-with-an-executive-over-his-virus-tweets}{}

  The ad giant Publicis has `parted ways' with an executive over his
  virus tweets.
\item
  \protect\hyperlink{catch-up-uber-extends-its-work-from-home-policy-to-june-2021}{}

  Catch up: Uber extends its `work from home' policy to June 2021.
\end{itemize}

\hypertarget{disney-lost-47-billion-last-quarter-but-its-newest-business-was-a-big-hit}{%
\subsection{\texorpdfstring{\protect\hyperlink{disney-lost-4-7-billion-last-quarter-but-its-newest-business-was-a-big-hit}{Disney
lost \$4.7 billion last quarter, but its newest business was a big
hit.}}{Disney lost \$4.7 billion last quarter, but its newest business was a big hit.}}\label{disney-lost-47-billion-last-quarter-but-its-newest-business-was-a-big-hit}}

Copied to clipboard.

\includegraphics{https://static01.nyt.com/images/2020/08/04/business/04virus-disney3/merlin_174646827_1c3ac65b-d259-404b-8f35-6bcc15319c1e-articleLarge.jpg?quality=75\&auto=webp\&disable=upscale}

\textbf{The Walt Disney Company} reported doomsday financial results on
Tuesday, with padlocked theme parks, idled cruise ships, postponed film
releases, darkened Broadway marquees, closed mall stores and the absence
of live sports on ESPN as a result of the coronavirus pandemic all
contributing to \$4.72 billion in quarterly losses.

But Disney's newest and, as far as many investors are concerned, most
important business ---
\href{https://www.nytimes.com/2020/04/08/business/disney-plus-50-million-subscribers.html}{streaming}
--- experienced growth as people quarantined at home. Disney said it had
more than 100 million subscribers worldwide across its Disney+, Hulu and
ESPN+ streaming services. Disney+ has about 58 million by itself, an
astounding number for a platform that is less than nine months old.

Crossing the 100 million threshold was ``a significant milestone and a
reaffirmation of our direct-to-consumer strategy, which we view as key
to the future growth of our company,'' Bob Chapek, Disney's chief
executive officer, said in a statement.

In total, revenue in the quarter that ended on June 27 added up to
\$11.78 billion, falling from \$20.26 billion and underscoring the
\href{https://www.nytimes.com/2020/05/04/business/media/coronavirus-disney.html}{extreme
difficulties} that the world's largest entertainment company has faced
during the coronavirus pandemic. During the quarter, Disney furloughed
an estimated 100,000 employees, slashed executive pay by up to 50
percent and took out a \$5 billion line of credit to bolster its
liquidity, on top of \$8.25 billion secured in March.

Per-share losses in the quarter, the third in Disney's fiscal year,
totaled \$2.61 --- a stark departure from the spectacular growth the
company delivered from 2006, when it bought Pixar, to last year, when it
swallowed the majority of Rupert Murdoch's entertainment empire. In the
same period last year, Disney had a profit of 79 cents per share.

Excluding one-time items, Disney squeaked out per share profit for the
most recent quarter of eight cents, better than analysts were expecting.

Disney's share price has been remarkably buoyant, however, because
investors have focused on comeback efforts --- the return of some
sporting events and a retrofitted Walt Disney World that
\href{https://www.nytimes.com/2020/07/11/business/florida-coronavirus-disney-world-reopening.html}{reopened
to a limited number of visitors} in mid-July --- and the success of
Disney's streaming division. Hulu has been on a roll because of new
programming, including the FX-supplied series ``Mrs. America,'' which
received eight Emmy nominations. Disney+ created a cultural thunderclap
in early July, when it released a live capture of the original
``Hamilton'' stage production.

--- \href{https://www.nytimes.com/by/brooks-barnes}{Brooks Barnes}

\hypertarget{fox-corporations-plunging-profit-is-cushioned-by-fox-news}{%
\subsection{\texorpdfstring{\protect\hyperlink{fox-corporations-plunging-profit-is-cushioned-by-fox-news}{Fox
Corporation's plunging profit is cushioned by Fox
News.}}{Fox Corporation's plunging profit is cushioned by Fox News.}}\label{fox-corporations-plunging-profit-is-cushioned-by-fox-news}}

Copied to clipboard.

\includegraphics{https://static01.nyt.com/images/2020/08/04/business/04Markets-brf-fox/merlin_159022344_cb64f69e-c38c-42e0-9289-a8a3e261f4fb-articleLarge.jpg?quality=75\&auto=webp\&disable=upscale}

\textbf{Fox News} remains the profit king at Rupert Murdoch's diminished
media empire. The cable news behemoth was the lone bright spot at its
parent company \textbf{Fox Corporation}, which reported a drop in
quarterly profit and revenue because of impacts from the coronavirus
pandemic.

Profit plummeted 69 percent to \$145 million in three months ending in
June because of a decrease in advertising and an increase in costs,
including a significant impairment charge related to exiting a rights
agreement with U.S. Golf Association.

Since Mr. Murdoch sold the majority of his business to \textbf{The Walt
Disney Company}, his Fox empire has slimmed down to focus on sports and
news. Fox News has benefited from people in lockdown, but the absence of
major sports programming has severely cut into the business.

Revenue was down 4 percent to \$2.5 billion, slightly better than Wall
Street's expectations of \$2.3 billion. Fox News helped dampen the
downturn. The cable news channel saw an uptick in advertising despite a
recent
\href{https://www.nytimes.com/2020/06/18/business/media/tucker-carlson-advertisers-ratings.html}{marketing
boycott} of the network's biggest prime time star, Tucker Carlson.

Lachlan Murdoch, the chief executive, characterized the performance as a
strong result in spite of the pandemic. ``We entered the Covid-19 crisis
on sound operational and financial footing and we expect to emerge from
this pandemic more competitive, more focused and even more strongly
positioned to deliver value for our viewers, partners and shareholders
in the years ahead,'' he said in a statement accompanying the results.

The division that houses Fox News saw a 12 percent jump in profit to
\$674 million, accounting for 90 percent of the company's total
operating profit. The company also benefited from contractual rate
increases that cable and satellite operators pay to carry Fox's
networks. In addition to Fox News, the company also owns Fox
broadcasting and the FS1 cable sports channel.

As at other media networks, Fox saw an enormous defection in total ad
revenue, which dropped 22 percent to \$712 million. Carriage fees rose
7.7 percent to \$1.5 billion.

On the earnings call after the report, Mr. Murdoch underscored how
sports had started to return, including Major League Baseball, despite
flare-ups in coronavirus cases among some teams. He sounded an
optimistic note about the return of football by September, saying ``we
fully expect college football and N.F.L. to come back.''

--- \href{https://www.nytimes.com/by/edmund-lee}{Edmund Lee}

\hypertarget{advertisement}{%
\subsubsection{Advertisement}\label{advertisement}}

\protect\hyperlink{after-dfp-ad-mid1}{Continue reading the main story}

\hypertarget{trading-in-kodak-shares-comes-under-scrutiny}{%
\subsection{\texorpdfstring{\protect\hyperlink{trading-in-kodak-shares-comes-under-scrutiny}{Trading
in Kodak shares comes under
scrutiny.}}{Trading in Kodak shares comes under scrutiny.}}\label{trading-in-kodak-shares-comes-under-scrutiny}}

Copied to clipboard.

\includegraphics{https://static01.nyt.com/images/2020/08/04/business/04markets-brf-kodak/merlin_92635288_b4435354-3c40-4e46-a1e0-c4356dc455b6-articleLarge.jpg?quality=75\&auto=webp\&disable=upscale}

A surge in shares of \textbf{Eastman Kodak} before a deal was announced
with the Trump administration to produce critical components for the
pharmaceutical industry has come under scrutiny after a senator called
for a federal investigation and news reports suggested that one was
already underway.

On Monday, Senator Elizabeth Warren, Democrat of Massachusetts, called
on
\href{https://www.warren.senate.gov/imo/media/doc/2020.08.03\%20Letter\%20to\%20SEC\%20re\%20Kodak\%20stock\%20trades.pdf}{U.S.
securities regulators}to investigate trading in shares of Eastman Kodak
before the company disclosed that it would receive
\href{https://www.nytimes.com/live/2020/07/28/business/stock-market-today-coronavirus/the-united-states-will-lend-kodak-765-million-to-make-drug-components}{a
\$765 million federal loan} to produce ingredients to make critical
drugs in the United States.

Then on Tuesday,
\href{https://www.wsj.com/articles/kodak-loan-disclosure-and-stock-surge-under-sec-investigation-11596559126}{The
Wall Street Journal reported} that the Securities and Exchange
Commission had begun a preliminary inquiry. Ms. Warren applauded the
news in a
\href{https://twitter.com/SenWarren/status/1290697801882632193}{post on
Twitter} that linked to The Journal article.

Kodak, based in Rochester, N.Y., said in a statement it ``intends to
fully cooperate with any potential inquiries.'' Arielle Patrick, a
spokeswoman for the company, said Kodak never intended for news about
the loan to run in a local publication before the
\href{https://www.nytimes.com/live/2020/07/28/business/stock-market-today-coronavirus\#the-united-states-will-lend-kodak-765-million-to-make-drug-components}{Trump
administration} announced on July 28 that the company had been tapped to
work on a potential treatment for Covid-19 and other ailments.

An official with the S.E.C. declined to comment. Kodak had not received
any notification from the commission as of Tuesday, said a person
familiar with the matter who spoke on condition of anonymity because the
matter was not public.

The timing of the loan to Kodak, best known for its camera and film
processing business, raised controversy because the official
announcement
\href{https://www.nytimes.com/2020/07/31/business/kodak-ceo-stock-options.html}{came
a day after the company}had awarded Jim Continenza, the company's
chairman and chief executive, 1.75 million in stock options.

Kodak awarded those stock options to Mr. Continenza at the same time it
was alerting local media in Rochester about the impending loan deal. At
least one news outlet in Rochester jumped the gun on that news --- one
potential reason behind the 25 percent surge in Kodak shares on July 27.
The stock closed at \$2.62 that day and rose more than 1,000 percent
over the next two days on news of the Trump deal.

Within 48 hours of the options being granted to Mr. Continenza, they
were worth about \$50 million. For now, though, any gains in the value
of those options are just theoretical as Mr. Continenza has yet to
exercise them to buy shares.

In her letter, Ms. Warren said there were ``questions about how Kodak
handled what appears to be `non-intentional disclosure of material
nonpublic information.''' She said the company might have violated a
securities rule intended to handle such inadvertent disclosures.

--- \href{https://www.nytimes.com/by/matthew-goldstein}{Matthew
Goldstein}

\hypertarget{nbcuniversal-to-cut-about-10-percent-of-its-work-force}{%
\subsection{\texorpdfstring{\protect\hyperlink{nbcuniversal-to-cut-about-10-percent-of-its-work-force}{NBCUniversal
to cut about 10 percent of its work
force.}}{NBCUniversal to cut about 10 percent of its work force.}}\label{nbcuniversal-to-cut-about-10-percent-of-its-work-force}}

Copied to clipboard.

\includegraphics{https://static01.nyt.com/images/2020/08/04/business/04-markets-brf-nbc/merlin_171541782_e88614f8-e41a-4ba6-b436-5ba2b74f6824-articleLarge.jpg?quality=75\&auto=webp\&disable=upscale}

\textbf{NBCUniversal}, the media giant that includes Universal Pictures,
the NBC broadcast network and several cable channels, started layoffs
this week because of the affects of the coronavirus pandemic, according
to two people familiar with the matter.

The company plans to eliminate about 10 percent of its full time work
force of 35,000.

The pandemic has cut into sports broadcasts, closed movie theaters and
shut down theme parks. A large portion of the staff cuts will happen at
NBCUniversal's theme parks group, which took a \$399 million loss in the
second quarter, the only unit to lose money in the period.

The company recently reopened its locations in Florida and Japan after
closing for several months. Its California location remains closed.

Total sales for NBCUniversal, owned by \textbf{Comcast}, fell 25 percent
to \$6.1 billion in the second quarter, as the virus continued to wipe
out spending. Sales at the Universal Studios division declined nearly a
fifth to \$1.2 billion as the country waited for theaters to more fully
open.

--- \href{https://www.nytimes.com/by/edmund-lee}{Edmund Lee}

\hypertarget{the-ad-giant-publicis-has-parted-ways-with-an-executive-over-his-virus-tweets}{%
\subsection{\texorpdfstring{\protect\hyperlink{the-ad-giant-publicis-has-parted-ways-with-an-executive-over-his-virus-tweets}{The
ad giant Publicis has `parted ways' with an executive over his virus
tweets.}}{The ad giant Publicis has `parted ways' with an executive over his virus tweets.}}\label{the-ad-giant-publicis-has-parted-ways-with-an-executive-over-his-virus-tweets}}

Copied to clipboard.

The ad giant \textbf{Publicis Groupe} cut ties on Tuesday with an
executive over his Twitter posts about the coronavirus pandemic.

Tom Goodwin, who became the company's head of futures and insight in
January,
\href{https://twitter.com/tomfgoodwin/status/1289981216335073280?s=20}{posted}
over the weekend that he found ``the total obsession with Covid deaths
over all other deaths entirely gruesome. 7,500 Americans die every day
but only the ones with this precise new Virus matter.''

The post, like
\href{https://twitter.com/tomfgoodwin/status/1251848439035502592?s=20}{several
others} about
\href{https://twitter.com/tomfgoodwin/status/1236483014277894144?s=20}{the
crisis} in recent months, incited outrage from many in the advertising
industry. On Sunday, Tom Morton, the U.S. chief strategy officer of
R/GA, urged Mr. Goodwin
\href{https://twitter.com/tommorton/status/1289992807038332928?s=20}{to
stop posting}: ``Please no more clickbait contrarianism. You're better
than this.'' Mr. Goodwin hit back with a
\href{https://twitter.com/tomfgoodwin/status/1289995369313497090?s=20}{expletive-laced
response}, telling Mr. Morton to
``\href{https://twitter.com/tomfgoodwin/status/1289994688275988481?s=20}{get
off your lofty perch}'' and mocking his ``sourdough baking
home-schooling.''

Publicis ``parted ways'' with Mr. Goodwin because his actions on social
media ``do not meet the standard of conduct we expect of our company's
employees and were not aligned with our values,'' the advertising firm
said in a statement first reported by the trade publication
\href{https://www.adweek.com/agencies/after-coronavirus-tweets-tom-goodwin-is-out-at-publicis-groupe/}{AdWeek}.

Mr. Goodwin did not immediately respond to a request for comment, but
seemed to address the uproar by
\href{https://twitter.com/tomfgoodwin/status/1290340999672332291}{posting
on Monday} that ``in the free thinking world of 2020 we are only
`allowed' to have perfectly aligned ``genuine concerns,'' while adding
that it was ``time to repeat that I'm
\href{https://twitter.com/tomfgoodwin/status/1290325449529274370?s=20}{not
a voice of Publicis}.''

--- \href{https://www.nytimes.com/by/tiffany-hsu}{Tiffany Hsu}

\hypertarget{advertisement-1}{%
\subsubsection{Advertisement}\label{advertisement-1}}

\protect\hyperlink{after-dfp-ad-mid2}{Continue reading the main story}

\hypertarget{loans-are-harder-to-get-even-as-interest-rates-are-low}{%
\subsection{\texorpdfstring{\protect\hyperlink{loans-are-harder-to-get-even-as-interest-rates-are-low}{Loans
are harder to get, even as interest rates are
low.}}{Loans are harder to get, even as interest rates are low.}}\label{loans-are-harder-to-get-even-as-interest-rates-are-low}}

Copied to clipboard.

\includegraphics{https://static01.nyt.com/images/2020/08/04/business/04borrow1/04borrow1-articleLarge-v2.jpg?quality=75\&auto=webp\&disable=upscale}

The economic crisis caused by the pandemic has driven
\href{https://www.nytimes.com/2020/07/16/business/mortgage-rates-below-3-percent.html}{interest
rates to rock-bottom levels}, meaning there has hardly been a better
time to borrow. But with tens of million of people out of work and
coronavirus infections surging in many parts of the country,
\href{https://www.nytimes.com/2020/08/04/your-money/mortgage-loans-credit-cards-coronavirus.html}{qualifying
for a loan} --- from mortgages to auto loans --- has become more trying,
even for well-positioned borrowers.

Lenders that have
\href{https://www.nytimes.com/2020/07/14/business/big-banks-quarterly-results.html}{set
aside billion of dollars} for future defaults have also tightened their
standards, often requiring higher credit scores, heftier down payments
and more documentation. Some, such as \textbf{Wells Fargo} and
\textbf{Chase}, have temporarily eliminated home equity lines of credit,
while Wells Fargo also stopped cash-out refinancing.

It's not unusual for lenders to tighten the credit reins during a
downturn, but the current situation has made it especially challenging
for them to get an accurate read on consumers' financial health.
Borrowers have been able to pause mortgages, halt student loan payments
and delay paying their tax bills, while millions of households have
received an extra \$600 weekly in unemployment benefits. Those forms of
government support could be masking an underlying condition.

``It makes it hard for a lender to understand what the consumer's true
state of credit quality is and their ability to pay back a loan,'' said
Peter Maynard, senior vice president of global data and analytics at the
\textbf{Equifax} credit bureau.

--- \href{https://www.nytimes.com/by/tara-siegel-bernard}{Tara Siegel
Bernard}

\hypertarget{black-owned-businesses-face-a-double-blow-as-the-pandemic-strikes-minority-communities}{%
\subsection{\texorpdfstring{\protect\hyperlink{black-owned-businesses-face-a-double-blow-as-the-pandemic-strikes-minority-communities}{Black-owned
businesses face a double blow as the pandemic strikes minority
communities.}}{Black-owned businesses face a double blow as the pandemic strikes minority communities.}}\label{black-owned-businesses-face-a-double-blow-as-the-pandemic-strikes-minority-communities}}

Copied to clipboard.

\includegraphics{https://static01.nyt.com/images/2020/08/04/business/04-markets-brf-fedsurvey/04-markets-brf-fedsurvey-articleLarge.jpg?quality=75\&auto=webp\&disable=upscale}

Black-owned businesses have been hit particularly hard by the pandemic
and the resulting economic crisis, a new
\href{https://www.newyorkfed.org/smallbusiness/small-business-credit-survey-2020}{study
by the Federal Reserve} found, underlining how minority communities have
borne a disproportionate cost of the virus.

The analysis, from the Federal Reserve Bank of New York, shows that
\href{https://www.nytimes.com/interactive/2020/06/18/us/coronavirus-black-owned-small-business.html}{Black-owned
businesses} are heavily concentrated in areas that were hard-hit by the
outbreak. And those businesses were already vulnerable before the crisis
began, according to the briefing, written by Claire Kramer Mills and
Jessica Battisto.

``Counties with the highest concentration of COVID-19 are also the areas
with the highest concentration of Black businesses and networks,'' the
authors wrote, noting that ``weaker cash positions, weaker bank
relationships, and pre-existing funding gaps left Black firms with
little cushion entering the crisis.''

Black-owned businesses now appear to be closing at a faster rate than
those owned by other minority groups.

While the overall number of active business owners fell 22 percent
between February and April, the number of active Black business owners
dropped by 41 percent. Other minority groups also saw a major cost: The
number of Latino business owners fell by 32 percent, and Asian business
owners dropped by 26 percent. Those figures are based on a University of
California Santa Cruz analysis of census data.

Even as minority communities shouldered heavier burdens amid the
\href{https://www.nytimes.com/2020/04/07/us/coronavirus-race.html}{coronavirus
crisis}, they have sometimes failed to access federal help amid what the
authors call ``stark'' coverage gaps.

In counties with the densest Black-owned business activity, Paycheck
Protection Program loan coverage rates were typically lower than 20
percent, the report found. That was ``not too different'' from the
national coverage rate of 17.7 percent, according to the authors, but
there was big variation across counties.

For instance, only 7 percent of businesses in the Bronx, 11.3 percent in
Queens, and 11.6 percent in the Michigan county that is home to Detroit
received the forgivable loans.

Probably at play: Black communities often lack access to banks in the
best of times. Black-owned businesses are far less likely than
white-owned businesses to have stable banking relationships.

Lack of credit access in communities of color raises ``questions that
have heightened significance when banks are relied on to administer
federal, taxpayer-supported relief programs, as is the case with PPP,''
the brief said.

--- \href{https://www.nytimes.com/by/jeanna-smialek}{Jeanna Smialek}

\hypertarget{stocks-climb-as-us-lawmakers-continue-talks-over-coronavirus-relief}{%
\subsection{\texorpdfstring{\protect\hyperlink{stocks-climb-as-us-lawmakers-continue-talks-over-coronavirus-relief}{Stocks
climb as U.S. lawmakers continue talks over coronavirus
relief.}}{Stocks climb as U.S. lawmakers continue talks over coronavirus relief.}}\label{stocks-climb-as-us-lawmakers-continue-talks-over-coronavirus-relief}}

Copied to clipboard.

Stocks rose on Tuesday, extending a rally that has lifted technology
shares to new highs as lawmakers in Washington continued to try to pin
down a coronavirus relief package.

Investors have one eye on corporate earnings reports, and the other on
lawmakers who are discussing the latest aid bill to help people and
businesses hit by the economic crisis. Negotiations reconvened on
Tuesday to try to reach an agreement on how to extend aid to tens of
millions of Americans who lost crucial
\href{https://www.nytimes.com/2020/07/30/business/unemployment-payments-change.html}{unemployment
benefits} at the end of July. Economists have warned that permanent
damage could be wrought on the economy without action.

Trading on Tuesday was unsteady, with the S\&P 500 falling back into
negative territory at several points throughout the day. But sentiment
was also lifted by a report showing an uptick in factory orders in June,
another indication of rebounding economic activity. By the end of the
day, the S\&P 500 climbed about 0.4 percent, and the Nasdaq composite
rose to another record.

On the earnings front, the
\href{https://www.nytimes.com/live/2020/08/04/business/stock-market-today-coronavirus/bp-to-step-up-renewable-investment-as-it-reports-a-huge-loss}{London-based
oil
giant}\textbf{\href{https://www.nytimes.com/live/2020/08/04/business/stock-market-today-coronavirus/bp-to-step-up-renewable-investment-as-it-reports-a-huge-loss}{BP}}
reported a \$16.8 billion quarterly loss, and cut its dividend in half
for the first time in a decade. The company also said it would increase
its investments in low-carbon energy, like solar and wind power, by
tenfold in a decade, while cutting its oil and gas production by 40
percent. Its shares rose despite the huge loss.

The gain on Tuesday adds to a steady climb for stocks that has lifted
the S\&P 500 to within 3 percent of its record. That has been fueled by
government spending, the efforts of the Federal Reserve to backstop the
economy and a surge in shares of technology stocks like \textbf{Apple},
\textbf{Amazon} and \textbf{Microsoft} --- which have reported higher
profits as more people work and shop from home.

--- Mohammed Hadi

\hypertarget{advertisement-2}{%
\subsubsection{Advertisement}\label{advertisement-2}}

\protect\hyperlink{after-dfp-ad-mid3}{Continue reading the main story}

\hypertarget{catch-up-uber-extends-its-work-from-home-policy-to-june-2021}{%
\subsection{\texorpdfstring{\protect\hyperlink{catch-up-uber-extends-its-work-from-home-policy-to-june-2021}{Catch
up: Uber extends its `work from home' policy to June
2021.}}{Catch up: Uber extends its `work from home' policy to June 2021.}}\label{catch-up-uber-extends-its-work-from-home-policy-to-june-2021}}

Copied to clipboard.

\begin{itemize}
\item
  \textbf{Uber} said it extended its timetable for employees to work
  from home, saying that they will not be asked to go back to the office
  until June 2021. Previously, Uber told employees that they could
  expect to return by the end of September. But as coronavirus cases
  continued to rise in the United States, the goal of returning safely
  in the fall became less realistic. In Europe, some of Uber's offices
  have already reopened. If the pandemic subsides enough for U.S.
  offices to reopen before June, employees would have the option to
  return early, a spokesman said.
\item
  \textbf{Wynn Resorts} said on Tuesday that
  \href{https://wynnresortslimited.gcs-web.com/news-releases/news-release-details/wynn-resorts-limited-reports-second-quarter-2020-results?field_nir_news_date_value\%5Bmin\%5D=}{its
  operating revenue plunged 94.8 percent} to \$85.7 million for the
  second quarter, compared with \$1.66 billion in the same period last
  year. The company, based in Las Vegas, reported a net loss of \$637.6
  million for the April-to-June period. Like other casinos, Wynn
  Resorts' operations closed in the spring and reopened in June but with
  limited capacity because of health restrictions, including a limited
  number of seats per table and slots machines spaced farther apart to
  accommodate social distancing.
\item
  \textbf{\href{http://booking.com/}{Booking.com}} plans to reduce its
  global work force of more than 17,500 employees by up to 25 percent as
  the coronavirus pandemic continues to take a devastating toll on the
  travel industry. \textbf{Booking Holdings}, the parent company of
  \href{http://booking.com/}{Booking.com}, will make announcements to
  employees beginning in September on a country by country basis,
  \href{https://www.sec.gov/ix?doc=/Archives/edgar/data/1075531/000107553120000042/bkng-20200804.htm}{according
  to a securities filing}. Booking Holdings, which owns other travel
  websites including \textbf{Kayak} and
  \textbf{\href{http://priceline.com/}{Priceline.com},} reported a 51
  percent drop in gross travel bookings in the first quarter of 2020
  compared to the same period last year.
\end{itemize}

\hypertarget{site-index}{%
\subsection{Site Index}\label{site-index}}

\hypertarget{site-information-navigation}{%
\subsection{Site Information
Navigation}\label{site-information-navigation}}

\begin{itemize}
\tightlist
\item
  \href{https://help.nytimes.com/hc/en-us/articles/115014792127-Copyright-notice}{©~2020~The
  New York Times Company}
\end{itemize}

\begin{itemize}
\tightlist
\item
  \href{https://www.nytco.com/}{NYTCo}
\item
  \href{https://help.nytimes.com/hc/en-us/articles/115015385887-Contact-Us}{Contact
  Us}
\item
  \href{https://www.nytco.com/careers/}{Work with us}
\item
  \href{https://nytmediakit.com/}{Advertise}
\item
  \href{http://www.tbrandstudio.com/}{T Brand Studio}
\item
  \href{https://www.nytimes.com/privacy/cookie-policy\#how-do-i-manage-trackers}{Your
  Ad Choices}
\item
  \href{https://www.nytimes.com/privacy}{Privacy}
\item
  \href{https://help.nytimes.com/hc/en-us/articles/115014893428-Terms-of-service}{Terms
  of Service}
\item
  \href{https://help.nytimes.com/hc/en-us/articles/115014893968-Terms-of-sale}{Terms
  of Sale}
\item
  \href{https://spiderbites.nytimes.com}{Site Map}
\item
  \href{https://help.nytimes.com/hc/en-us}{Help}
\item
  \href{https://www.nytimes.com/subscription?campaignId=37WXW}{Subscriptions}
\end{itemize}
