The Times and Your Data

\hypertarget{privacy-policy}{%
\section{Privacy Policy}\label{privacy-policy}}

The Times and Your Data

\hypertarget{the-trust-of-our-readers-is-essential}{%
\section{The trust of our readers is
essential.}\label{the-trust-of-our-readers-is-essential}}

Our readers should know how we handle their data.

We created this page to answer our readers' questions about how we use
their personal data. This topic is important to many readers, and we
value their right to understand how it works.

This F.A.Q. applies to nytimes.com readers. If you want to dive deeper
into the details, please see our \href{/privacy/privacy-policy}{Privacy
Policy}.

\hypertarget{frequently-asked-questions}{%
\paragraph{Frequently asked
questions}\label{frequently-asked-questions}}

1.

Why does The Times collect (and use) my data?

At The Times, we aim to create the best possible reader experience
across every medium. This involves knowing certain things about our
readership. For example, knowing which articles you read helps us
understand your interests. That information lets us select the types of
articles we show you in certain parts of the app or site. (This article
selection process is still guided by our journalistic judgment, and
doesn't impact large portions of the app or site.)

It is important to note that The Times is primarily funded by
subscriptions and advertising. Both functions require the use of
readers' data. For example, we use readers' data to identify who may be
interested in a subscription in order to show you Times ads on other
websites. Our advertisers ask us to use reader data so their ads can be
targeted at the right audiences.

2.

What type of data is collected about me when I'm accessing The Times?

Different types of data are collected based on the different Times
services you use. There are effectively two types of data collection:
direct and indirect.

\textbf{Direct data collection} means you are submitting data to us. For
example, when you create an account with The Times, you submit your
email address to set up and personalize it.

\textbf{Indirect data collection} takes place passively as you interact
with our site or apps. Our tracking technologies collect data about your
reading behavior, like which articles you read or how often you visit
The Times. Third-party advertisers collect behavioral data associated
with their ads shown on our site or apps. Additionally, we collect data
about readers from sources like privately owned databases and social
media platforms.

3.

What does The Times do with data it collects on me?

We use it to cater our journalism and other offerings to you, like by
recommending stories that may be of interest. The specific data we use
depends on the offering, and how you are accessing it. We also use your
data to tailor your experience, which includes marketing and
advertising.

4.

How are you keeping my data safe?

We have implemented organizational, technological and physical
safeguards designed to keep our readers' data secure internally. The
Times is dedicated to building, maintaining and upgrading the measures
we take to protect your data.

5.

What are some ways I can protect my data?

We encourage our readers to take steps to safeguard their own data. We
recently published a guide on this topic:
\href{https://www.nytimes.com/guides/privacy-project/how-to-protect-your-digital-privacy}{How
to Protect Your Digital Privacy}.

6.

Why am I given the option to use Facebook and Google for log-in?

Our readers have voiced a desire for our log-in process to be even
simpler.

If you sign up via Facebook or Google, they share limited data with us
so we can create an account for you. We do not share any data about your
reading behavior with Facebook or Google when you sign up. To learn
more, see our
\href{https://help.nytimes.com/hc/en-us/articles/115014887628-Social-login}{social
login page}.

7.

How can I learn more about how companies use personal data online?

The Times regularly publishes pieces on the topic, including in
\href{https://www.nytimes.com/2019/11/04/business/secret-consumer-score-access.html}{Business},
\href{https://www.nytimes.com/2019/11/19/technology/artificial-intelligence-dawn-song.html}{Technology}
and
\href{https://www.nytimes.com/2019/11/24/smarter-living/privacy-online-how-to-stop-advertiser-tracking-opt-out.html}{Smarter
Living}.

Our Opinion series,
\href{https://www.nytimes.com/series/new-york-times-privacy-project}{The
Privacy Project}, explains how online privacy affects your daily life
--- and shows you how to take control of your data.

\hypertarget{explore-more}{%
\subsubsection{EXPLORE MORE}\label{explore-more}}

\href{/privacy/privacy-policy}{}

Our Privacy Policy

\href{/privacy/cookie-policy}{}

Our Cookie Policy

\hypertarget{any-other-questions}{%
\subsubsection{Any other questions?}\label{any-other-questions}}

Our team is here to help with any questions or concerns you may have
about our privacy policy. Please send an email to
\href{mailto:privacy@nytimes.com}{\nolinkurl{privacy@nytimes.com}}.

©2020 The New York Times Company

\href{/privacy}{Privacy F.A.Q.}\href{/privacy/privacy-policy}{Privacy
Policy}\href{/privacy/cookie-policy}{Cookie
Policy}\href{/privacy/california-notice}{California
Notice}\href{https://help.nytimes.com/hc/en-us/articles/115014893428-Terms-of-service}{Terms
of Service}

The Times and your Data

\hypertarget{main-menu}{%
\subsection{Main Menu}\label{main-menu}}

\href{/privacy}{Privacy F.A.Q.}\href{/privacy/privacy-policy}{Privacy
Policy}\href{/privacy/cookie-policy}{Cookie Policy}
