The Times and your Data

\hypertarget{main-menu}{%
\subsection{Main Menu}\label{main-menu}}

\href{/privacy}{Privacy F.A.Q.}\href{/privacy/privacy-policy}{Privacy
Policy}\href{/privacy/cookie-policy}{Cookie Policy}

The Times and Your Data

\hypertarget{privacy-policy}{%
\section{Privacy Policy}\label{privacy-policy}}

The Times and Your Data

\hypertarget{the-trust-of-our-readers-is-essential}{%
\section{The trust of our readers is
essential.}\label{the-trust-of-our-readers-is-essential}}

\hypertarget{the-new-york-times-company-privacy-policy}{%
\section{The New York Times Company Privacy
Policy}\label{the-new-york-times-company-privacy-policy}}

Last Updated on July 1, 2020

At The Times, our mission is to seek the truth and help people
understand the world. We want you to understand how we handle your data.
We also want you to know your rights and choices.

\begin{enumerate}
\def\labelenumi{\arabic{enumi}.}
\item
  1

  \href{what-information-do-we-gather-about-you}{What Information Do We
  Gather About You?}
\item
  2

  \href{what-do-we-do-with-the-information-we-gather}{What Do We Do With
  The Information We Collect About You?}
\item
  3

  \href{with-whom-do-we-share-the-information-we-gather}{With Whom Do We
  Share The Information We Gather?}
\item
  4

  \href{what-are-your-rights}{What Are Your Rights?}
\item
  5

  \href{what-about-sensitive-personal-information}{What About Sensitive
  Personal Information?}
\item
  6

  \href{how-long-do-you-retain-data}{How Long Do You Retain Data?}
\item
  7

  \href{how-do-you-protect-my-information}{How Do You Protect My
  Information?}
\item
  8

  \href{are-there-guidelines-for-children}{Are There Guidelines for
  Children?}
\item
  9

  \href{how-is-my-information-transferred-internationally}{How Is
  Information Transferred Internationally?}
\item
  10

  \href{what-is-our-legal-basis}{What Is Our Legal Basis?}
\item
  11

  \href{links-to-third-party-services}{What About Links to Third-Party
  Services?}
\item
  12

  \href{how-are-changes-to-this-privacy-policy-communicated}{How Are
  Changes to This Privacy Policy Communicated?}
\item
  13

  \href{how-can-you-contact-us-who-is-the-controller-of-your-personal-information}{How
  Can You Contact Us? Who Is the Controller of Your Personal
  Information?}
\end{enumerate}

This policy describes how we handle your data when you use ``Times
Services,'' listed below:

\begin{itemize}
\tightlist
\item
  The New York Times newspaper, plus our International Edition
\item
  Our websites, like nytimes.com
\item
  Our apps, like the New York Times app and the New York Times Crossword
  app
\item
  Our email newsletters, like Cooking and Morning Briefing
\item
  Our pages or ads on social media networks, like our Facebook and
  Instagram pages
\item
  Anywhere we gather information from you and refer you to this Privacy
  Policy
\end{itemize}

How we handle information about you depends on which Times Services you
use --- and how you use them. We use different information about print
subscribers than website visitors.

Be aware that certain Times Services work differently. Some have
additional terms that supplement this policy (e.g.,
\href{https://help.nytimes.com/hc/en-us/articles/360004901454-Reader-submission-terms}{Reader
Submissions}). Others refer to a different privacy policy altogether, so
this one does not apply.

1.

What Information Do We Gather About You?

The information we gather about you depends on the context. By and
large, it's information about you that can personally identify you ---
either on its own or when combined with other information.

The following describes the information we collect and how we obtain it.

\textbf{A) Information Collected Through Times Services.}

\begin{enumerate}
\def\labelenumi{\arabic{enumi}.}
\tightlist
\item
  \textbf{Information That You Voluntarily Give Us}

  \begin{itemize}
  \item
    \textbf{For Registration:}

    When you sign up for a Times Service (e.g., a subscription), we
    collect your contact information and account credentials. Once
    you're registered, we assign you a unique ID number. This ID number
    helps us recognize you when you're signed in.

    For some Times Services, you can instead sign up by linking your
    Apple, Facebook or Google account. See
    ``\protect\hyperlink{anchor-question1-sectionB}{From Other
    Sources}'' below.

    If you register for an event or conference, we might ask for
    additional information (e.g., your company name, your job title or
    your dietary restrictions).
  \item
    \textbf{For Billing:}

    To process payments or donations, we collect and use your payment
    information.

    This can include your name, your address, your telephone number,
    your email address, your credit or debit card information and any
    other relevant information.
  \item
    \textbf{For User-Generated Content:}

    We offer you the ability to post content that other users can read
    (e.g., comments or recipe reviews). Anyone can read, collect and use
    any personal information that accompanies your posts. See the
    \href{https://help.nytimes.com/hc/en-us/articles/115014792387-Comments}{Comments
    F.A.Q.}, or read
    ``\href{https://help.nytimes.com/hc/en-us/articles/115014893428-Terms-of-service\#3}{User-Generated
    Content}'' in our Terms of Service for more information.

    We do not have to publish any of your content. If the law requires
    us to take down, remove or edit your personal information, we will
    comply to the required extent.
  \item
    \textbf{For Contests, Sweepstakes and Special Offers:}

    When you sign up for these, you give us your name, email and any
    other required information.
  \item
    \textbf{For Reader Surveys, Research, Panels and Experience
    Programs:}

    We gather information through questionnaires, surveys and feedback
    programs. We also conduct similar research for advertisers. We ask
    you for your consent to use this information when you participate in
    these programs and events.
  \item
    \textbf{During Contact With Our Call Centers:}

    We collect information from you when you place an order over the
    phone or contact customer service through one of our toll-free
    numbers.
  \item
    \textbf{Personal Contacts Data:}

    We never scan your device for your contacts or upload this data.

    With your consent, we do comply with your requests to collect data
    about your friends, family or acquaintances (e.g., Refer a Friend
    campaigns). This functionality is only meant for U.S. residents. By
    using it, you acknowledge and agree that both you and your contacts
    are based in the United States --- and that you have everyone's
    consent for us to use their contact information.
  \end{itemize}
\item
  \textbf{Information Collected Automatically}

  \begin{itemize}
  \item
    \textbf{With Tracking Technologies in Your Browser and Mobile Apps:}

    These technologies include cookies, web beacons, tags and scripts,
    software development kits (or SDKs) and beyond.

    We track and store data about how you visit and use Times Services,
    particularly through our websites and apps. The items we log
    include:

    \begin{itemize}
    \tightlist
    \item
      Your IP address
    \item
      Your location
    \item
      Your operating system
    \item
      Your browser
    \item
      Your browser language
    \item
      The URLs of any pages you visit on our sites and apps
    \item
      Device identifiers
    \item
      Advertising identifiers
    \item
      Other usage information.
    \end{itemize}

    We combine this data with other information we collect about you.
    For more information about tracking methods on Times Services, and
    how to manage them, read our
    \href{https://www.nytimes.com/subscription/dg-cookie-policy/cookie-policy.html}{Cookie
    Policy}.

    If your browser doesn't accept our cookies, you can't access certain
    parts of our websites (e.g., your account on nytimes.com). Because
    the ``Do Not Track'' browser-based standard signal has yet to gain
    widespread acceptance, we don't currently respond to those signals.
  \item
    \textbf{With GPS Technologies:}

    Some of our apps can provide content based on your GPS location, if
    you enable this feature (e.g., the New York Times Real Estate app).
    Your GPS location is your exact location.

    You choose whether to enable GPS features when you first install the
    app. You can edit that setting on your device at any time. If you
    enable these features, your GPS location can be found by satellite,
    cell phone tower or Wi-Fi and used by the app. If you save a
    location-based search in your history, that data moves to our
    service provider's servers --- see
    \protect\hyperlink{anchor-question2-sectionE}{below} for the
    definition of service provider.

    If you do not enable GPS location-based services, or if a specific
    app does not have location-based features (e.g., the New York Times
    app), we don't collect your precise GPS location. We do collect your
    IP address, which can establish your approximate location. Ads on
    our sites and apps may be targeted based on this approximate
    location, but are never targeted based on your GPS location.
  \end{itemize}
\end{enumerate}

\textbf{B) Information Collected From Other Sources.}

\begin{enumerate}
\def\labelenumi{\arabic{enumi}.}
\item
  \textbf{Privately Owned Databases:}

  Marketing, data analytic and social media-owned databases give us
  access to a range of information --- like public data, survey data and
  beyond. This data sometimes includes your mailing address, your
  gender, your age, your household income and other demographic data.
\item
  \textbf{Social Media Platforms and Other Third-Party Services:}

  (Social media platforms include Facebook. Third-party services include
  Google, Kindle and Nook.)

  You can link your social media or other third-party account to a Times
  Service. By linking the services, you authorize us to collect, store
  and use any information they may give us (e.g., your email address).
  You can disconnect your nytimes.com registration from third-party
  accounts at any time.

  We also receive information from you when you interact with our pages,
  groups, accounts or posts on social media platforms. This includes
  aggregate data on our followers (e.g., age, gender and location),
  engagement data (e.g., ``likes,'' comments, shares, reposts and
  clicks), awareness data (e.g., number of impressions and reach) and
  individual users' public profiles.

  For more information, refer to our
  \href{https://help.nytimes.com/hc/en-us/articles/115014887628-Social-login}{social
  login},
  \href{https://help.nytimes.com/hc/en-us/articles/115014889068-Kindle-subscribers}{Kindle}
  and
  \href{https://help.nytimes.com/hc/en-us/articles/115014917867-NOOK-subscribers}{Nook}
  F.A.Q.
\item
  \textbf{Workplace and Schools:}

  When your employer or school buys an organizationwide subscription to
  nytimes.com, they sometimes provide us with your name and organization
  email address to grant you access as a user.
\end{enumerate}

\textbf{A note about future updates:}

We are always improving our products and services, and we create new
features regularly. These updates sometimes require us to collect new
information, or use what we already have differently. If there is a
significant or material change in the way we handle your personal
information, we will notify you as detailed below.

\href{app}{Back to top}

2.

What Do We Do With the Information We Gather?

\textbf{A) We provide the Times Services.}

We use your information to help you use and navigate Times Services,
such as:

\begin{itemize}
\tightlist
\item
  Making a Times Service available to you
\item
  Arranging access to your account
\item
  Providing customer service
\item
  Responding to your inquiries, requests, suggestions or complaints
\item
  Completing your payments and transactions
\item
  Sending service-related messages (e.g., a change in our terms and
  conditions)
\item
  Saving your reading list, recipes or property searches
\item
  Displaying your Crossword stats
\item
  Letting you take part in paid services, polls, promotions, surveys,
  panels, research and comments.
\end{itemize}

\textbf{B) We Personalize Your Experience.}

We track your interests and reading habits (e.g., the articles you read)
to personalize your reading experience using technology like algorithmic
recommendations and machine learning. This is how we highlight articles
you might be interested in and de-emphasize articles you've already
read. For more information about content personalization on Times
Services, you can read the
\href{https://help.nytimes.com/hc/en-us/articles/360003965994-Personalization}{Personalization
F.A.Q.} We also show you prices, promotions, products or services we
believe you'll find interesting, based on demographic and usage data.

\textbf{C) We Allow You to Share User-Generated Content.}

Any information you disclose in your content becomes public --- along
with your chosen screen name and uploaded photo.

\textbf{D) We Develop Products and Services, and Do Analysis.}

We analyze data on our users' subscription, purchase and usage
behaviors. This helps us make business and marketing decisions.

For example, our analysis lets us predict preferences and price points
for our products and services. It helps us determine whether our
marketing is successful. It also shows us characteristics about our
readers, which we sometimes share in aggregate with advertisers.

Google Analytics is one of the analytics providers we use. You can find
out \href{https://policies.google.com/technologies/partner-sites}{how
Google Analytics uses data} and
\href{https://tools.google.com/dlpage/gaoptout}{how to opt out of Google
Analytics}.

\textbf{E) We Carry Out Administrative Tasks.}

\begin{itemize}
\tightlist
\item
  For auditing: We verify that our internal processes work as intended
  and comply with legal, regulatory and contractual requirements.
\item
  For fraud and security monitoring: We detect and prevent cyberattacks
  or unauthorized robot activities.
\item
  For customer satisfaction: We assess users' satisfaction with Times
  Services and our customer care team.
\end{itemize}

The above activities can involve outside companies, agents or
contractors (``service providers'') with whom we share your personal
information for these purposes (discussed further below).

\textbf{F) We Offer Sweepstakes, Contests and Other Promotions.}

You can take part in our sweepstakes, contests and other promotions.
Some might have additional rules about how we use and disclose your
personal information.

\textbf{G) We Allow for Personalized Advertising on Times Services and
Create Audiences for Third Party Advertisers.}

We gather data and work with \href{/privacy/third-party}{third parties}
to show you personalized ads. This data comes from ad tracking
technologies set by us or the third party (e.g., cookies), the
information you provide (e.g., your email address), your use of Times
Services (e.g., your reading history), information from advertisers or
advertising vendors (e.g., demographic data) and anything inferred from
any of this information. We only use or share this information in a
manner that does not reveal your identity.

For example, we use Google to serve ads on Times Services. Google uses
cookies or unique device identifiers, in combination with their own
data, to show you ads based on you visiting nytimes.com and other sites.
You can opt out of the use of the Google cookie by visiting the
\href{https://policies.google.com/technologies/ads?hl=en}{related Google
privacy policy}.

We also create specific audiences that allow us to serve you
personalized advertising on our sites and apps, on behalf of
advertisers. To do this, we combine information we collect through
surveys (from subscribers, registered users and non-registered visitors)
with information we collect automatically using tracking technologies as
you use the Times Services. This combined information is used to build
models for measuring users against demographic and interest-based
attributes. With the help of service providers, these measurements are
then transformed into pools of users grouped by common attributes. Each
group is associated with a random ID that is then passed to our ad
server for us in targeting advertising campaigns on Times Services.

Another example is our affiliate link vendors, which we use in our
guides and product recommendations. Times Services include links that
will send you to affiliate vendor URLs and other services not operated
or controlled by us. These affiliate vendors use cookies and other
technologies to collect information about your navigation from the Times
Services to the merchant you are visiting. If you buy a product after
following a link to an affiliate link vendor's URL, we may earn a
commission.

\textbf{Additional notes:}

\begin{itemize}
\item
  For more about targeted advertising, and how to opt out with your
  specific browser and device, go to the
  \href{http://optout.aboutads.info/?c=2\&lang=EN}{DAA Webchoices
  Browser Check} and
  \href{http://optout.networkadvertising.org/?c=1}{NAI Opt Out of
  Interest-Based Advertising}. You can
  \href{https://youradchoices.com/appchoices}{download the AppChoices
  app} to opt out in mobile apps. You can also follow the instructions
  in the \protect\hyperlink{anchor-question4}{``What Are Your Rights?''}
  section below.
\item
  We try to limit how our third-party advertising technology vendors use
  the information they gather from you. Many of these providers require
  us to enter into contracts that allow them to optimize their own
  services and products, or that help them create their own.

  Essentially, these providers combine any information they gather about
  you through Times Services with information they receive from their
  other clients. This helps them target ads to you on behalf of their
  other clients, not just us.
\item
  These third parties sometimes use other services in order to serve
  ads; check \href{/privacy/third-party}{their privacy policies for more
  details}. For further information on tracking technologies and your
  rights and choices regarding them, see the applicable
  \href{https://www.nytimes.com/subscription/dg-cookie-policy/cookie-policy.html}{Cookie
  Policy}.
\end{itemize}

\textbf{H) We Advertise Times Services to You.}

We market our properties to you. Sometimes we use marketing vendors to
do this.

We serve ads through websites, locations, platforms and services
operated and owned by third parties. Often these ads are targeted at
people who have visited or registered for a Times Service but have not
subscribed or purchased anything. The ads are also targeted at people
with similar traits or behaviors to our subscribers or customers.

We target our advertising to these users by uploading an encrypted
customer list to a third party, or by incorporating a tracking
technology from a third party onto our Times Service. The third party
then matches individuals who appear in both our data and their data.
Because of how this matching process works, the third party can't read
our encrypted customer list if they don't already have it.

To opt out of receiving these matched ads, contact the applicable
\href{/privacy/third-party}{third parties}. For example, when we use
``Custom Audience'' to serve you our ad through Facebook, you should be
able to hover over the box in the right corner of that Facebook ad and
opt out. We are not responsible for any third party's failure to comply
with opt-out requests.

We periodically send you targeted email newsletters or promotional
emails. For information on opting out of these emails, see
\protect\hyperlink{anchor-question4}{``What Are Your Rights?''}

\textbf{I) We Aggregate (or De-identify) Personal Information Into
Larger Findings.}

Sometimes we aggregate or de-identify information so that it can no
longer identify you, under applicable laws. This helps us better
understand and represent our users, such as when we measure ad
performance, create advertising interest-based segments or compile
survey results. We can use and disclose this aggregated or de-identified
information for any purpose, unless an applicable law says otherwise.

\href{app}{Back to top}

3.

With Whom Do We Share the Information We Gather?

\textbf{A) Within The New York Times Company:}

We share your information with our affiliates for the purposes listed
here. See
\href{https://help.nytimes.com/hc/en-us/articles/360004990014}{a list of
our affiliates}.

\textbf{B) With Service Providers:}

We work with service providers, as defined
\protect\hyperlink{anchor-question2-sectionE}{above}, to carry out
certain tasks:

\begin{itemize}
\tightlist
\item
  Processing your payments
\item
  Fulfilling your orders
\item
  Maintaining technology and related infrastructure
\item
  Offering you customer service
\item
  Serving and targeting ads
\item
  Measuring ad performance
\item
  Presenting surveys
\item
  Shipping you products and mailings
\item
  Distributing emails
\item
  List processing and analytics
\item
  Managing and analyzing research
\item
  Managing promotions
\end{itemize}

When performing these tasks, service providers often have access to your
personal information.

We sometimes allow them to use aggregated or de-identified information
for other purposes, in accordance with applicable laws

\textbf{C) With Other Third Parties:}

There are situations when we share your information with third parties
beyond our service providers. We never share your email address with
these third parties without your consent, except in encrypted form to
engage in the matched ads process described
\protect\hyperlink{anchor-question2-sectionH}{above}.

\begin{enumerate}
\def\labelenumi{\arabic{enumi}.}
\item
  If you're a U.S. print subscriber, we may share your name and mailing
  address (among other information) with other reputable companies that
  want to market to you by mail.
\item
  We share information about our live event and conference attendees
  (e.g., your name, your company or your job title) with the event
  sponsors. In those cases we notify you when you provide us the
  information.
\item
  We share information about participants in our sweepstakes, contests
  and similar promotions with the promotions' sponsors. In those cases
  we notify you when you provide us the information.
\item
  We process payments you make through Times Services with external
  services.

  There are two ways this can happen:

  \begin{itemize}
  \tightlist
  \item
    We collect your information and share it with the third-party
    service for processing.
  \item
    The third-party service collects your information for processing.
  \end{itemize}
\item
  In the event of a reorganization, merger, sale, joint venture,
  assignment, transfer or other disposition of all or any portion of our
  business, assets or stock (including in connection with any bankruptcy
  or similar proceedings), we would have a legitimate interest in
  disclosing or transferring your information to a third party --- such
  as an acquiring entity and its advisers.
\item
  We can preserve or share personal information if the law requires us
  to do so. We can also preserve or share personal information if we
  believe it would be necessary to:

  \begin{itemize}
  \tightlist
  \item
    Comply with the law or with legal process
  \item
    Protect and defend our rights and property
  \item
    Protect against misuse or unauthorized use of the Times Services
  \item
    Protect the safety or property of our users or the general public
    (e.g., if you provide false information or attempt to pose as
    someone else, we could share your information to help investigations
    into your actions)
  \item
    Cooperate with government authorities, which could be outside your
    country of residence.
  \end{itemize}
\item
  We disclose public activities in our RSS feeds, APIs and other
  distribution formats. Your public activities could thus appear on
  other websites, blogs or feeds.
\end{enumerate}

\href{app}{Back to top}

4.

What Are Your Rights?

\textbf{A) How Do I Opt Out of Email, Phone, Mail and Push
Notifications?}

The opt-out methods described below are limited to the email address,
phone or device used. They won't affect subsequent subscriptions.

\begin{enumerate}
\def\labelenumi{\arabic{enumi}.}
\item
  \textbf{Email:}

  We offer a variety of commercial emails and email newsletters. You can
  unsubscribe from them by following the ``unsubscribe'' instructions
  near the bottom of the email. You can also email us at
  \href{mailto:privacy@nytimes.com}{\nolinkurl{privacy@nytimes.com}}.

  You can \href{https://myaccount.nytimes.com/seg/settings}{manage your
  nytimes.com newsletter preferences}.
\item
  \textbf{Mail or Telephone Promotions:}

  You can ask us to unsubscribe from our mail or telephone
  solicitations. You can also ask us to not share your information with
  third parties for marketing purposes. To do so, email us at
  \href{mailto:privacy@nytimes.com}{\nolinkurl{privacy@nytimes.com}}
  with ``Opt Out'' in the subject line, and your account number and
  phone number in the body of the email.

  You can write to us at: Customer Care, P.O. Box 8041, Davenport, IA
  52808-8041 --- or, for International Edition customers, The New York
  Times International Edition, Immeuble Le Lavoisier, 4, Place des
  Vosges, CS 10001, 92052 Paris La Défense Cedex, France. Please include
  your account number and phone number in the body of the letter.
\item
  \textbf{Push Notifications:}

  You can opt out any time by adjusting your device settings, or
  uninstalling our app.
\item
  \textbf{Text Messages:}

  You can opt out of text alerts any time by replying ``STOP,'' or any
  alternative keyword we've shared with you.
\end{enumerate}

We complete any opt-out request as quickly as we can. This opt-out
request won't prohibit us from sending you important nonmarketing
notices.

\textbf{B) How Do You Access, Change, Delete, Update or Exercise Your
Other Rights in Relation to Your Personal Information?}\\
In some parts of the world, you have the right to:

\begin{itemize}
\tightlist
\item
  Access, modify, or delete the personal information we have about you
\item
  Receive an electronic copy of the personal information we have about
  you, for data portability
\item
  Restrict, or object to, how we process personal information about you
\item
  Not receive discriminatory treatment by us for the exercise of your
  privacy rights.
\end{itemize}

You have the right to object to the processing of your personal
information based on our legitimate interest or that of a third party
--- unless we demonstrate compelling legitimate grounds for the
processing of, or the keeping of, your personal information for the
establishment, exercise or defense of legal claims.

If you'd like to exercise any of the above rights, contact us via
\href{https://www.nytimes.com/data-subject-request}{this form} or by
calling us at our toll-free number, 1-800-NYTIMES. In your request,
please be specific. State the information you want changed, whether
you'd like your information suppressed from our database or whether
there are limitations you'd like us to put on how we use your personal
information. Please use the email address linked to that personal
information --- we only complete requests on the information linked to
your email address. To verify your identity, we will email the email
address you provide us, and which matches our records, and wait for your
response. In some instances we may also ask for additional information.
This is how we verify your identity before complying.

You can designate an authorized agent to make a request on your behalf.
In order to do that, please provide the agent with written permission,
signed by you, authorizing the agent to submit the request on your
behalf. The agent must submit that written permission along with the
request. We will contact you to verify your identity --- and the
authorized agent's permission --- before a response to the request is
sent.

We'll respond to your request in a manner consistent with applicable
law.

We might need to keep certain information for recordkeeping purposes, or
to complete a transaction you began prior to requesting a change or
deletion (e.g., if you make a purchase or enter a promotion, you might
not be able to change or delete the personal information provided until
after the completion of the purchase or promotion).

In some cases, your request doesn't ensure complete removal of the
content or information (e.g., if another user has reposted your
content).

If you'd like, you can lodge a complaint with a data protection
authority. A
\href{https://ec.europa.eu/newsroom/article29/item-detail.cfm?item_id=612080}{list
of E.U. data protection authorities} is available.

\textbf{C) How Do You Manage Your Digital and Home Delivery Accounts?}

You can update your account information and see
\href{https://myaccount.nytimes.com}{your transaction history} (for
\href{https://customercare.inyt.com}{International Edition print
subscribers}). If you need assistance, call our toll-free number,
1-800-NYTIMES.
\href{https://subscribe.inyt.com/footer?requestAction=displayContactIht}{Other
local numbers} are available.

It works differently if you subscribed via Apple's App Store or Google
Play. Register with us to access the Account area, and contact Apple or
Google for your transaction history.

\href{app}{Back to top}

5.

What About Sensitive Personal Information?

We generally don't want to gather any sensitive information about you.
This includes:

\begin{itemize}
\tightlist
\item
  Your social security number
\item
  Your racial or ethnic origin
\item
  Your political opinions
\item
  Your religion or other beliefs
\item
  Your health, biometric or genetic characteristics
\item
  Any trade union membership
\item
  Any criminal background
\end{itemize}

There are rare situations when we request this information (e.g., a
reader survey asks about your political leanings), but you can decline
to answer. Outside those situations we would prefer you never share that
information with us.

\href{app}{Back to top}

6.

How Long Do You Retain Data?

It depends. We store your personal information for as long as needed, or
permitted, based on the reason why we obtained it (consistent with
applicable law). This means we might retain your personal information
even after you close your account with us.

When deciding how long to keep your information, we consider:

\begin{itemize}
\tightlist
\item
  How long we've had a relationship with you or provided a Times Service
  to you
\item
  Whether we are subject to any legal obligations (e.g., any laws that
  require us to keep transaction records for a certain period of time
  before we can delete them)
\item
  Whether we have taken any legal positions (e.g., in connection with
  any statutes of limitation).
\end{itemize}

Rather than delete your data, we might de-identify it by removing
identifying details.

\href{app}{Back to top}

7.

How Do You Protect My Information?

We protect your personal information with a series of organizational,
technological and physical safeguards --- but we cannot guarantee its
absolute security. We recommend that you use complex and unique
passwords for your Times accounts and for third-party accounts linked to
them. Do not share your password with anyone.

If you have reason to believe your interaction with us is no longer
secure, notify us immediately.

\href{app}{Back to top}

8.

Are There Guidelines for Children?

Times Services are intended for a general audience, and are not directed
at children under (13) years of age.

We do not knowingly gather personal information (as defined by the U.S.
Children's Privacy Protection Act, or COPPA) in a manner not permitted
by COPPA. If you are a parent or guardian and you believe we have
collected information from your child in a manner not permitted by law,
contact us at
\href{mailto:privacy@nytimes.com}{\nolinkurl{privacy@nytimes.com}}. We
will remove the data to the extent required by applicable laws.

\href{app}{Back to top}

9.

How Is Information Transferred Internationally?

The New York Times Company is headquartered in the United States. If you
are located outside the United States, your information is collected in
your country and then transferred to the United States --- or to another
country in which we (or our affiliates or service providers) operate.

If we transfer your data out of the European Economic Area (E.E.A.), we
implement at least one of the three following safeguards:

\begin{itemize}
\tightlist
\item
  We transfer your information to countries that have been recognized by
  the European Commission as providing an adequate level of data
  protection according to E.E.A. standards (see the
  \href{https://ec.europa.eu/info/law/law-topic/data-protection/international-dimension-data-protection/adequacy-decisions_en}{full
  list of these countries}).
\item
  We use a service provider in the United States that is
  \href{https://www.privacyshield.gov/welcome}{Privacy Shield}
  certified.
\item
  We take steps to ensure that the recipient is bound by E.U. Standard
  Contractual Clauses to protect your personal data. You can see a
  \href{https://ec.europa.eu/info/law/law-topic/data-protection/international-dimension-data-protection/standard-contractual-clauses-scc_en}{copy
  of these clauses}.
\end{itemize}

In certain situations, the courts, law enforcement agencies, regulatory
agencies or security authorities in those countries might be entitled to
access your personal information.

\href{app}{Back to top}

10.

What Is Our Legal Basis?

In some jurisdictions, like the European Union and the European Economic
Area, we only collect, use or share information about you when we have a
valid reason. This is called ``lawful basis.'' Specifically, this is one
of the following:

\begin{itemize}
\item
  The consent you provide to us at the point of collection of your
  information
\item
  The performance of the contract we have with you
\item
  The compliance of a legal obligation to which we are subject or
\item
  The legitimate interests of The Times, a third party or yourself.
  ``Legitimate interest'' is a technical term under international laws,
  including the European Union General Data Protection Regulation. It
  means that there are good reasons for the processing of your personal
  information, and that we take measures to minimize the impact on your
  privacy rights and interests. ``Legitimate interest'' also refers to
  our use of your data in ways you would reasonably expect and that have
  a minimal privacy impact.

  We have a legitimate interest in gathering and processing personal
  information, for example: (1) to ensure that our networks and
  information are secure; (2) to administer and generally conduct
  business within The New York Times Company; (3) to prevent fraud; and
  (4) to conduct our marketing activities.
\end{itemize}

\href{app}{Back to top}

11.

Links to Third-Party Services?

Some Times Services contain links to third-party websites, resources,
vendors and advertisers. These third parties are not Times Services. We
do not control (and are not responsible for) third party content or
privacy practices. Any personal data you provide to them is not covered
by this Privacy Policy.

\href{app}{Back to top}

12.

How Are Changes to This Privacy Policy Communicated?

We periodically update this Privacy Policy. We will post any changes on
this page by updating this policy.

If we make a significant or material change in the way we collect, use
or share your personal information, we will notify you at least 30 days
prior to the changes taking effect. We will do this via email or
prominent notice within Times Services. If you object to any change, you
can stop using the Times Services.

After we post any changes on this page, your continued use of Times
Services is subject to the updated Privacy Policy.

\href{app}{Back to top}

13.

How Can You Contact Us? Who Is the Controller of Your Personal
Information?

If you have any questions, email us at
\href{mailto:privacy@nytimes.com}{\nolinkurl{privacy@nytimes.com}} or
write us at:

The New York Times Company\\
620 Eighth Avenue\\
New York, N.Y. 10018\\
Attn.: Privacy Counsel

We can also be reached by phone at 1-800-NYTIMES (see
\href{https://subscribe.inyt.com/footer?requestAction=displayContactIht}{a
list of our local telephone numbers outside the United States}).

The New York Times Company is referred to in this Privacy Policy as
``The Times,'' ``we'' or ``our.''

Certain Times Services operate as independent controllers of your
personal information. Wirecutter operates as an independent controller
of personal information collected through the Wirecutter site available
at nytimes.com/wirecutter, pages or ads on social media networks, email
messages sent by Wirecutter, your offline contacts and any other service
offered by Wirecutter (collectively, the ``Wirecutter Services''). If
you have any questions regarding Wirecutter, email us at
\href{mailto:privacy@thewirecutter.com}{\nolinkurl{privacy@thewirecutter.com}}
or write us at:

Wirecutter, Inc.\\
c/o The New York Times Company\\
620 Eighth Avenue\\
New York, N.Y. 10018\\
Attn.: Privacy Counsel

Wirecutter operates the Wirecutter Services in accordance with the
practices disclosed in this Privacy Policy. With respect to the
Wirecutter Services, Wirecutter, Inc. is referred to in this Privacy
Policy as included in ``The Times,'' ``we'' or ``our.'' In this Privacy
Policy, Wirecutter Services are included under ``Times Services.''

\href{app}{Back to top}

©2020 The New York Times Company

\href{/privacy}{Privacy F.A.Q.}\href{/privacy/privacy-policy}{Privacy
Policy}\href{/privacy/cookie-policy}{Cookie
Policy}\href{/privacy/california-notice}{California
Notice}\href{https://help.nytimes.com/hc/en-us/articles/115014893428-Terms-of-service}{Terms
of Service}
