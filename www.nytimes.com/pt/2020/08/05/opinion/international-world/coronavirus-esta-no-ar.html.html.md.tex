Sections

SEARCH

\protect\hyperlink{site-content}{Skip to
content}\protect\hyperlink{site-index}{Skip to site index}

\href{https://myaccount.nytimes.com/auth/login?response_type=cookie\&client_id=vi}{}

\href{https://www.nytimes.com/section/todayspaper}{Today's Paper}

\href{/section/opinion}{Opinion}\textbar{}Sim, o coronavírus está no ar

\href{https://nyti.ms/3fq0Qko}{https://nyti.ms/3fq0Qko}

\begin{itemize}
\item
\item
\item
\item
\item
\end{itemize}

Advertisement

\protect\hyperlink{after-top}{Continue reading the main story}

\href{/section/opinion}{Opinion}

Supported by

\protect\hyperlink{after-sponsor}{Continue reading the main story}

\hypertarget{sim-o-coronavuxedrus-estuxe1-no-ar}{%
\section{Sim, o coronavírus está no
ar}\label{sim-o-coronavuxedrus-estuxe1-no-ar}}

A transmissão do vírus pelos aerossóis importa e, talvez, muito mais do
que ainda sabemos.

Por Linsey C. Marr

Dra. Marr é professora de engenharia.

\begin{itemize}
\item
  Aug. 5, 2020
\item
  \begin{itemize}
  \item
  \item
  \item
  \item
  \item
  \end{itemize}
\end{itemize}

\includegraphics{https://static01.nyt.com/images/2020/07/30/opinion/05marr-PORT/30Marr-articleLarge.jpg?quality=75\&auto=webp\&disable=upscale}

\href{https://www.nytimes.com/2020/07/30/opinion/coronavirus-aerosols.html}{Read
in
English}\href{https://www.nytimes.com/es/2020/08/01/espanol/opinion/coronavirus-aire.html}{Leer
en español}

Finalmente, a Organização Mundial da Saúde (OMS) reconheceu formalmente
que o SARS-CoV-2, o vírus que causa a COVID-19,
\href{https://www.nytimes.com/2020/07/09/health/virus-aerosols-who.html}{é
transmitido pelo ar} e pode ser transportado em minúsculas
\href{https://www.nature.com/articles/d41586-020-02058-1}{partículas de
aerossol}.

Quando tossimos ou espirramos, falamos ou apenas respiramos,
naturalmente expelimos gotículas (pequenas partículas de fluido) e
aerossóis (partículas ainda menores de fluido). No entanto, até o início
deste mês, a OMS --- bem como os Centros de Controle e Prevenção de
Doenças dos EUA ou a agência de Saúde Pública da Inglaterra ---
\href{https://www.who.int/news-room/commentaries/detail/modes-of-transmission-of-virus-causing-covid-19-implications-for-ipc-precaution-recommendations}{havia
alertado} principalmente sobre a transmissão do novo coronavírus por
meio de contato direto e gotículas liberadas a curta distância.

A organização havia alertado sobre aerossóis somente em circunstâncias
extraordinárias, como após a intubação e outros
\href{https://www.who.int/publications/i/item/WHO-2019-nCoV-IPC-2020.4}{procedimentos
médicos} relacionados a pacientes infectados em hospitais.

Após vários meses de
\href{https://www.nature.com/articles/d41586-020-00974-w\#ref-CR5}{insistência
de cientistas}, em 9 de julho, a OMS mudou de posição, passando da
negação à uma
\href{https://www.who.int/news-room/commentaries/detail/transmission-of-sars-cov-2-implications-for-infection-prevention-precautions}{aceitação
parcial e relutante}: ``Requerem-se mais estudos para determinar se é
possível detectar o SARS-CoV- 2 viável ​​em amostras de ar retiradas de
ambientes onde não sejam realizados procedimentos que geram aerossóis, e
o papel dos aerossóis na transmissão''.

Sou engenheira civil e ambiental e estudo como os vírus e as bactérias
se espalham pelo ar. Também estou entre
\href{https://www.nytimes.com/es/2020/07/06/espanol/ciencia-y-tecnologia/coronavirus-transmision-aire.html}{os
239 cientistas} que assinaram
\href{https://academic.oup.com/cid/article/doi/10.1093/cid/ciaa939/5867798}{uma
carta aberta} no final de junho exortando a OMS a considerar mais
seriamente o risco de transmissão aérea.

Um mês depois, acredito que a transmissão do SARS-CoV-2 por aerossóis é
muito mais importante do que tem sido reconhecido oficialmente até o
momento.

Em um \href{https://www.nature.com/articles/s41598-020-69286-3}{estudo
revisado} por outros pesquisadores publicado na revista Nature em 29 de
julho, pesquisadores do Centro Médico da Universidade de Nebraska
descobriram que as gotículas de aerossol coletadas em quartos de
hospitais de pacientes com COVID-19 continham o coronavírus.

Isso confirma os resultados de um
\href{https://www.medrxiv.org/content/10.1101/2020.05.31.20115154v1}{estudo}
(não revisado) do final de maio, no qual descobriu-se que pacientes com
COVID-19 liberavam SARS-CoV-2 simplesmente exalando, sem tossir ou até
mesmo sem falar. Os autores desse estudo disseram que esse achado
implicaria que a transmissão aérea ``influencia significativamente'' a
propagação do vírus.

Aceitar essas conclusões não mudaria muito as recomendações atuais sobre
as melhores práticas. A melhor proteção contra o SARS-CoV-2, esteja ele
presente em gotículas ou em aerossóis, é essencialmente a mesma:
mantenha a distância e use a máscara.

As descobertas recentes são sobretudo um lembrete importante de que
também devemos estar vigilantes sobre a abertura de janelas e a melhoria
da circulação de ar em ambientes fechados. Além disso, elas contribuem
para a evidência de que a qualidade das máscaras faciais e seu ajuste
correto na face também são fatores importantes.

De acordo com a definição da OMS, uma ``gotícula'' é
\href{https://www.who.int/news-room/commentaries/detail/modes-of-transmission-of-virus-causing-covid-19-implications-for-ipc-precaution-recommendations}{uma
partícula de mais de cinco micrômetros} que não percorre uma distância
maior que um metro.

Na realidade, não há limites claros ou significativos entre as gotículas
e os aerossóis --- de cinco micrômetros ou de qualquer tamanho. São
todas pequenas gotículas de líquido cujo tamanho varia de muito pequeno
a microscópico.

(Estou consultando historiadores da medicina para identificar os
fundamentos científicos da definição da OMS e, até agora, não
encontramos explicação razoável.)

É verdade que as gotículas tendem a voar pelo ar como balas de canhão em
miniatura e caem no chão rapidamente, enquanto os aerossóis podem
permanecer suspensos por muitas horas.

Porém, os princípios básicos da física também sugerem que uma gota de
cinco micrômetros que cai da boca de um adulto de altura média leve
cerca de meia hora para chegar ao chão, e nesse período a gotícula pode
viajar muitos metros numa corrente de ar. As gotículas expelidas com a
tosse ou no espirro também
\href{https://academic.oup.com/jid/advance-article/doi/10.1093/infdis/jiaa189/5820886}{percorrem
distâncias muito maiores} que um metro.

Outro equívoco comum: a importância dos aerossóis só havia sido
reconhecida de uma forma limitada até agora, como algo suspenso no ar e
que o vento dissipa: uma ameaça distante.

No entanto, antes que eles possam ir longe, os aerossóis devem viajar
pelo ar próximo --- isso significa que eles também são um perigo a curta
distância. E são ainda mais perigosos porque, assim como a fumaça do
cigarro, os aerossóis se concentram mais perto da pessoa infectada (ou
fumante) e são diluídos no ar à medida que se afastam.

Um estudo realizado por cientistas da Universidade de Hong Kong e da
Universidade de Zhejiang, em Hangzhou, China, e revisado por outros
pesquisadores, foi publicado em junho na revista Building and
Environment com a conclusão de que ``quanto menores as gotículas
exaladas, mais importante é a rota aérea de curta distância''.

Então, o que significa tudo isso exatamente, na prática?

Você pode entrar numa sala vazia e contrair o vírus se uma pessoa
infectada, que já foi embora, estava lá antes de você? Talvez, mas
provavelmente apenas se a sala for pequena e mal ventilada.

O vírus pode flutuar para cima e para baixo dos edifícios através de
aberturas de ventilação ou tubos\protect\hyperlink{_msocom_1}{{[}MV1{]}}
? Talvez, embora isso não tenha sido confirmado.

As pesquisas sugerem que é mais provável que os aerossóis sejam
relevantes em contextos mais simples.

Vejamos
\href{https://www.nytimes.com/2020/04/20/health/airflow-coronavirus-restaurants.html}{o
caso de um restaurante em} Cantão, no sul da China, no início deste ano,
em que um cliente infectado com o SARS-CoV-2 contaminou um total de nove
pessoas sentadas a sua mesa e em outras duas mesas.

Yuguo Li, professor de engenharia da Universidade de Hong Kong, e seus
colegas
\href{https://www.medrxiv.org/content/10.1101/2020.04.16.20067728v1}{analisaram
as gravações das câmeras de vigilância} do restaurante e numa
pré-publicação (não revisada) divulgada em abril afirmaram que não
encontraram evidências de contato próximo entre os clientes.

Neste caso, a transmissão não pode ser atribuída às gotículas, pelo
menos não a que ocorreu entre as pessoas que não estavam sentadas a mesa
da pessoa infectada: as gotículas teriam caído no chão antes de chegar
às outras mesas.

Contudo, as três mesas estavam em uma seção mal ventilada do restaurante
e um aparelho de ar condicionado circulava o ar entre elas. Também vale
ressaltar que nenhum empregado do restaurante e nenhum outro cliente se
contagiou, incluindo os que estavam sentados em duas mesas fora do
alcance do fluxo do ar condicionado.

Num caso semelhante, acredita-se que uma pessoa infectou 52 das outras
60 pessoas presentes em
\href{https://www.nytimes.com/2020/05/12/health/coronavirus-choir.html}{um
ensaio de cor}al no condado de Skagit, Washington, em março.

Analisei esse evento, com a ajuda de vários colegas de diferentes
universidades, e em
\href{https://www.medrxiv.org/content/10.1101/2020.06.15.20132027v2}{uma
pré-publicação (não revisada) lançada em junho}, concluímos que os
aerossóis provavelmente foram a principal via de transmissão.

Os presentes tinham usado desinfetante para as mãos e evitaram abraços e
apertos de mãos, o que limitava o potencial de contágio por contato
direto ou pelas gotículas. Por outro lado, a sala era mal ventilada, o
ensaio durou muito tempo (2,5 horas) e é sabido que cantar produz
microgotas de aerossol e
\href{https://www.atsjournals.org/doi/abs/10.1164/arrd.1968.98.2.297}{facilita
a propagação de doenças como a tuberculose}.

E o surto no navio Diamond Princess que saiu do Japão no início deste
ano? Por volta de 712 dos 3.711 passageiros foram infectados.

O professor Li e outros também
\href{https://www.medrxiv.org/content/10.1101/2020.04.09.20059113v1}{investigaram
esse caso} e, em uma pré-publicação (não revisada) de abril, concluíram
que não havia transmissão entre os quartos depois que as pessoas foram
colocadas em quarentena: o sistema de ar condicionado do navio não
espalhou o vírus a distâncias maiores.

Segundo o estudo, a causa mais provável de transmissão foi o contato
próximo com pessoas infectadas ou objetos contaminados antes dos
passageiros e tripulantes serem isolados. (Os pesquisadores não
definiram exatamente o que queriam dizer com ``contato'', nem
esclareceram se isso incluía gotículas ou aerossóis a curta distância.)

No entanto, outra
\href{https://www.medrxiv.org/content/10.1101/2020.07.13.20153049v1}{pré-publicação
recente} (não revisada) referente ao caso Diamond Princess concluiu que
``a inalação de aerossóis provavelmente foi o fator predominante na
transmissão da COVID-19'' entre os passageiros da embarcação.

Pode parecer lógico, ou intuitivo, deduzir que as gotículas maiores
contêm mais vírus que as gotículas menores de aerossol, mas esse não é o
caso.

\href{https://www.thelancet.com/journals/lanres/article/PIIS2213-2600(20)30323-4/fulltext}{Um
artigo publicado esta semana} em The Lancet Respiratory Medicine que
analisou os aerossóis produzidos pela tosse e a expiração de pacientes
com várias infecções respiratórias revelou ``uma predominância de
patógenos de pequenas partículas'' (menos de cinco micrômetros). ``Não
há evidências'', concluiu o estudo, ``de que alguns patógenos se
transportem somente em gotículas grandes''.

\href{https://www.medrxiv.org/content/10.1101/2020.07.13.20041632v2}{Uma
pré-publicação recente} (não revisada) de pesquisadores do Centro Médico
da Universidade de Nebraska revelou que as amostras de vírus retiradas
de aerossóis emitidos por pacientes com COVID-19 eram contagiosas.

Alguns cientistas
\href{https://jamanetwork.com/journals/jama/fullarticle/2768396}{argumentaram}
que o mero fato de que os aerossóis podem conter SARS-CoV-2 não prova
por si só que eles possam causar uma infecção e que se o SARS-CoV-2
fosse disseminado principalmente por aerossóis, haveria mais evidência
de transmissões de longa distância.

Concordo com a ideia que a transmissão a longa distância por aerossol
provavelmente não é significativa, mas considero que, no conjunto,
muitas das evidências até o momento sugerem que a transmissão \emph{a
curta distância} por aerossol é significativa, possivelmente muito
significativa, e certamente mais significativa que o spray direto de
gotículas.

As implicações práticas são simples:

\begin{itemize}
\item
  \textbf{O distanciamento social é muito importante}. Isso nos mantém
  afastados das áreas com maior concentração de expulsões respiratórias.
  Portanto, é importante manter-se a pelo menos dois metros de distância
  de outras pessoas, embora quanto mais longe você estiver, mais seguro
  estará.
\item
  \textbf{Use máscara}. Elas ajudam a bloquear os aerossóis liberados
  pelo portador.
  \href{https://ucsf.app.box.com/s/blvolkp5z0mydzd82rjks4wyleagt036}{As
  evidências científicas} também sugerem que as máscaras
  \href{https://www.nytimes.com/2020/07/27/health/coronavirus-mask-protection.html?campaign_id=154\&emc=edit_cb_20200727\&instance_id=20696\&nl=coronavirus-briefing\&regi_id=65413713\&segment_id=34503\&te=1\&user_id=bd32fbf008e5183a7928ed61c60669f7}{impedem
  a inalação de aerossóis} ao redor.
\end{itemize}

Quando se trata de máscaras, o tamanho importa.

A melhor máscara é o respirador N95 ou KN95, que, encaixado corretamente
ao rosto, filtra a exalação e impede o portador de inalar pelo menos
95\% das gotículas de aerossol.

A eficácia das máscaras cirúrgicas contra aerossóis varia muito.

\href{https://pubmed.ncbi.nlm.nih.gov/23498357/}{Um estudo de 2013}
revelou que as máscaras cirúrgicas reduziram a exposição aos vírus de
influenza entre 10 a 98\% (dependendo do modelo da máscara).

Um artigo recente descobriu que as máscaras cirúrgicas podem impedir
completamente a expulsão dos coronavírus sazonais no ar.

Pelo que sei, ainda não foi realizado estudo semelhante para o
SARS-CoV-2, mas essas descobertas também podem se aplicar a esse vírus,
pois é semelhante aos coronavírus sazonais, em termos de tamanho e
estrutura.

Meu laboratório vem testando máscaras de pano num manequim que aspira ar
pela boca de maneira realista. Descobrimos que mesmo um lenço
frouxamente amarrado sobre a boca e o nariz bloqueou a passagem de pelo
menos a metade das gotas de aerossol maiores que dois micrômetros.

Também descobrimos que especialmente para aerossóis muito pequenos
(menores que um micrômetro) é mais eficaz usar um tecido macio (mais
fácil de ajustar à face) do que um pano mais rígido (que, apesar de
oferecer um filtro melhor, geralmente não se encaixa bem ao rosto e
deixa lacunas).

\begin{itemize}
\item
  \textbf{Evite multidões}. Quanto mais pessoas ao seu redor, maior a
  probabilidade de alguém estar infectado. Evite multidões,
  especialmente em ambientes fechados, onde os aerossóis podem se
  acumular.
\item
  \textbf{Ventilação importa}. Mantenha janelas e portas abertas. Ajuste
  o regulador do sistema de ar condicionado e aquecimento. Faça
  manutenção regular dos filtros nesses sistemas. Adicione purificadores
  de ar portáteis ou instale a tecnologia de lâmpada germicida UV para
  remover ou matar partículas de vírus no ar.
\end{itemize}

Não está claro o quanto esse coronavírus é transmitido através de
aerossóis em comparação com gotículas ou contato com superfícies
contaminadas. Mesmo
\href{https://journals.plos.org/plospathogens/article?id=10.1371/journal.ppat.1008704}{no
caso da gripe}, estudada há décadas, também não sabemos ainda a resposta
dessa pergunta.

No entanto, o que sabemos até agora é: os aerossóis são relevantes na
transmissão da COVID-19, e provavelmente ainda mais do que pudemos
provar até agora.

Linsey C. Marr é Charles P. Lunsford Professor de Engenharia Civil e
Ambiental no Instituto Politécnico da Virgínia e na Universidade
Estadual. \href{https://twitter.com/linseymarr}{@linseymarr}

Advertisement

\protect\hyperlink{after-bottom}{Continue reading the main story}

\hypertarget{site-index}{%
\subsection{Site Index}\label{site-index}}

\hypertarget{site-information-navigation}{%
\subsection{Site Information
Navigation}\label{site-information-navigation}}

\begin{itemize}
\tightlist
\item
  \href{https://help.nytimes.com/hc/en-us/articles/115014792127-Copyright-notice}{©~2020~The
  New York Times Company}
\end{itemize}

\begin{itemize}
\tightlist
\item
  \href{https://www.nytco.com/}{NYTCo}
\item
  \href{https://help.nytimes.com/hc/en-us/articles/115015385887-Contact-Us}{Contact
  Us}
\item
  \href{https://www.nytco.com/careers/}{Work with us}
\item
  \href{https://nytmediakit.com/}{Advertise}
\item
  \href{http://www.tbrandstudio.com/}{T Brand Studio}
\item
  \href{https://www.nytimes.com/privacy/cookie-policy\#how-do-i-manage-trackers}{Your
  Ad Choices}
\item
  \href{https://www.nytimes.com/privacy}{Privacy}
\item
  \href{https://help.nytimes.com/hc/en-us/articles/115014893428-Terms-of-service}{Terms
  of Service}
\item
  \href{https://help.nytimes.com/hc/en-us/articles/115014893968-Terms-of-sale}{Terms
  of Sale}
\item
  \href{https://spiderbites.nytimes.com}{Site Map}
\item
  \href{https://help.nytimes.com/hc/en-us}{Help}
\item
  \href{https://www.nytimes.com/subscription?campaignId=37WXW}{Subscriptions}
\end{itemize}
