Sections

SEARCH

\protect\hyperlink{site-content}{Skip to
content}\protect\hyperlink{site-index}{Skip to site index}

\href{https://www.nytimes.com/section/politics}{Politics}

\href{https://myaccount.nytimes.com/auth/login?response_type=cookie\&client_id=vi}{}

\href{https://www.nytimes.com/section/todayspaper}{Today's Paper}

\href{/section/politics}{Politics}\textbar{}Republican v. Democratic
U.S. Coronavirus Relief Proposals

\url{https://nyti.ms/304A9gz}

\begin{itemize}
\item
\item
\item
\item
\item
\end{itemize}

Advertisement

\protect\hyperlink{after-top}{Continue reading the main story}

Supported by

\protect\hyperlink{after-sponsor}{Continue reading the main story}

\hypertarget{republican-v-democratic-us-coronavirus-relief-proposals}{%
\section{Republican v. Democratic U.S. Coronavirus Relief
Proposals}\label{republican-v-democratic-us-coronavirus-relief-proposals}}

By Reuters

\begin{itemize}
\item
  July 27, 2020
\item
  \begin{itemize}
  \item
  \item
  \item
  \item
  \item
  \end{itemize}
\end{itemize}

WASHINGTON --- The Republican leadership of the U.S. Senate has
introduced its proposal for the next coronavirus relief package, a \$1
trillion plan called the Heals Act.

Below is a look at how it compares with the Heroes Act that passed the
Democratic-controlled House of Representatives in May:

UNEMPLOYMENT BENEFITS

Republican plan: Would reduce the expanded unemployment benefit from the
current \$600 per week, which expires on Friday, to \$200 a week, in
addition to state unemployment benefits, and extend the program for two
more months. After that, states are to pay employees about 70\% of the
income they had before they lost their jobs.

Democratic plan: Extends weekly enhanced unemployment payments of \$600
through January 2021.

DIRECT PAYMENTS

Senate Republican plan: Includes \$1,200 per individual; Senate Majority
Leader Mitch McConnell promised "even more support for families who care
for vulnerable adult dependents."

Democratic plan: \$1,200 per family member, up to \$6,000 per household.

LIABILITY PROTECTION:

Republican plan: Includes as one of its core proposals measures to
protect businesses and institutions from coronavirus-related lawsuits if
they are following government guidelines.

Democratic plan: did not include anything on this and Democratic leaders
have pushed back against the idea.

SCHOOLS

Republican plan: Includes \$70 billion for helping schools to reopen and
\$30 billion for colleges and universities.

Democratic plan: \$100 billion to support the educational needs of
states, school districts and institutions of higher education in
response to coronavirus.

TESTING, CONTACT TRACING AND TREATMENT

Republican plan: \$16 billion for coronavirus testing; \$25 billion for
hospitals.

Democratic plan: \$75 billion for testing, tracing and isolation
measures, and to support hospitals and healthcare providers and ensure
free access to treatment for individuals.

PAYCHECK PROTECTION PROGRAM

Democratic plan: Gives small businesses more flexibility with how they
use loans from this program (previously they were required to use 75\%
for payroll expenses, or be forced to pay it back as a loan)

Republican plan: Would allow the hardest-hit smallest employers, whose
revenue has declined by 50\% or more, to get a second forgivable loan
under the program. To qualify, businesses must have 300 or fewer
employees.

STATE AND LOCAL GOVERNMENTS

Democratic plan: Nearly \$1 trillion in aid to state, local, territorial
and tribal governments to help pay first responders, healthcare workers
and teachers.

Republican plan: Does not include new money, but Republicans said it
would give state and local leaders more flexibility in spending the
\$150 billion passed into law in March.

(Reporting by David Morgan and Susan Cornwell; Editing by Tom Brown)

Advertisement

\protect\hyperlink{after-bottom}{Continue reading the main story}

\hypertarget{site-index}{%
\subsection{Site Index}\label{site-index}}

\hypertarget{site-information-navigation}{%
\subsection{Site Information
Navigation}\label{site-information-navigation}}

\begin{itemize}
\tightlist
\item
  \href{https://help.nytimes.com/hc/en-us/articles/115014792127-Copyright-notice}{©~2020~The
  New York Times Company}
\end{itemize}

\begin{itemize}
\tightlist
\item
  \href{https://www.nytco.com/}{NYTCo}
\item
  \href{https://help.nytimes.com/hc/en-us/articles/115015385887-Contact-Us}{Contact
  Us}
\item
  \href{https://www.nytco.com/careers/}{Work with us}
\item
  \href{https://nytmediakit.com/}{Advertise}
\item
  \href{http://www.tbrandstudio.com/}{T Brand Studio}
\item
  \href{https://www.nytimes.com/privacy/cookie-policy\#how-do-i-manage-trackers}{Your
  Ad Choices}
\item
  \href{https://www.nytimes.com/privacy}{Privacy}
\item
  \href{https://help.nytimes.com/hc/en-us/articles/115014893428-Terms-of-service}{Terms
  of Service}
\item
  \href{https://help.nytimes.com/hc/en-us/articles/115014893968-Terms-of-sale}{Terms
  of Sale}
\item
  \href{https://spiderbites.nytimes.com}{Site Map}
\item
  \href{https://help.nytimes.com/hc/en-us}{Help}
\item
  \href{https://www.nytimes.com/subscription?campaignId=37WXW}{Subscriptions}
\end{itemize}
