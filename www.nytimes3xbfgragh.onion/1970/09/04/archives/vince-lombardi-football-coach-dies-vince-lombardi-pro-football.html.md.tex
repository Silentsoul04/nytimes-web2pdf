Sections

SEARCH

\protect\hyperlink{site-content}{Skip to
content}\protect\hyperlink{site-index}{Skip to site index}

\href{https://myaccount.nytimes3xbfgragh.onion/auth/login?response_type=cookie\&client_id=vi}{}

\href{https://www.nytimes3xbfgragh.onion/section/todayspaper}{Today's
Paper}

Archives\textbar{}Vince Lombardi, Football Coach, Dies

\url{https://nyti.ms/1GZThtc}

\begin{itemize}
\item
\item
\item
\item
\item
\end{itemize}

Advertisement

\protect\hyperlink{after-top}{Continue reading the main story}

Supported by

\protect\hyperlink{after-sponsor}{Continue reading the main story}

\hypertarget{vince-lombardi-football-coach-dies}{%
\section{Vince Lombardi, Football Coach,
Dies}\label{vince-lombardi-football-coach-dies}}

By William N. Wallace

\begin{itemize}
\item
  Sept. 4, 1970
\item
  \begin{itemize}
  \item
  \item
  \item
  \item
  \item
  \end{itemize}
\end{itemize}

\includegraphics{https://s1.graylady3jvrrxbe.onion/timesmachine/pages/1/1970/09/04/355790322_360W.png?quality=75\&auto=webp\&disable=upscale}

See the article in its original context from\\
September 4, 1970, Page
1\href{https://store.nytimes3xbfgragh.onion/collections/new-york-times-page-reprints?utm_source=nytimes\&utm_medium=article-page\&utm_campaign=reprints}{Buy
Reprints}

\href{http://timesmachine.nytimes3xbfgragh.onion/timesmachine/1970/09/04/355790322.html}{View
on timesmachine}

TimesMachine is an exclusive benefit for home delivery and digital
subscribers.

About the Archive

This is a digitized version of an article from The Times's print
archive, before the start of online publication in 1996. To preserve
these articles as they originally appeared, The Times does not alter,
edit or update them.

Occasionally the digitization process introduces transcription errors or
other problems; we are continuing to work to improve these archived
versions.

Vince Lombardi, the profes sional football coach who symbolized
toughness and dedi cation in sports, died of intesti nal cancer
yesterday in George town Hosiptal in Washington. His age was 57.

His wife of 30 years, Marie, and his son, Vincent, were at the bedside.

Lombardi, who guided the Green Bay Packers to the pre mier position in
the National Football League in the nineteen sixties was seeking to do
the same with the Washington Red skins in the nineteen‐seventies.

The Redskins had long been losers. But Vincent Thomas Lombardi had never
associated with losers in his 31 years as a football coach.

Last year, his first in Wash ington, the Redskins had their first
winning record in 14 sea sons.

``Winning isn't everything,'' Lombardi once insisted. ``It is the only
thing.''

Under his direction the Green Bay Packers won six division titles and
five national Football League championships in nine seasons between 1959
and 1967. This was professional football's best winning record and Lom
bardi was acclaimed as the sport's best coach.

He retired from coaching after the 1967 season, when he was 53 years
old. But his wife and his close friends wondered how long he could stay
away from the sidelines. The answer: one year. Most pro football games
are played on Sunday afternoons and dur ing the season that Lombardi
confined himself to the duties of the Packers' general mana ger he said,
``I miss the fire on Sunday.'' Edward Bennett Wil lianms, president of
the Red skins, early in 1969 offered Lombardi a position as coach,
general manager and owner of 5 per cent of the team's stock, and the
offer was quickly ac cepted.

``Everyone wants to own something sometime. Isn't that right?'' asked
Lombardi in ex plaining why he resigned the Packer post with five years
re maining on his contract.

Lombardi was a symbol of authority.

``When he says `Sit down,' don't even bother to look for chair,'' one of
the Packer play ers explained.

``He's fair. He treats us all the same---like dogs,'' said Hen ry
Jordan, another Packer.

``He coaches through fear,'' said Bill Curry, a sensitive player
Lombardi let go.

Most of his athletes accepted his demanding ways and biting criticisms.
His primary target was a player named Marvin Fleming, who said in
reflection, ``I didn't mind. When I came to him I didn't have anything.
He taught me how to be winner.''

Another Packer, Jerry Kra mer, said, ``His whippings, his cussings and
his driving all fade; his good qualities en dure.''

\textbf{`It's for Them'}

Lombardi admitted that his scoldings sometimes were merely for effect.
During his last season at Green Bay, when he was goading an aging team
to another championship, he said, ``I have to go on that field every day
and whip peo ple. It's for them, not just me. I'm getting to be an
animal.''

Lombardi was always a hard man when it came to football. In college, at
Fordham where he graduated with honors in 1937, he played guard on a
famous line called the Seven Blocks of Granite. He was the smallest of
the group at 5 feet 8 inches and 175 pounds. ``But he hit like 250,'' a
teammate said.

The son of an immigrant Ital ian butcher, Lombardi was born June 11,
1913, and grew up in the Sheepshead Bay sec tion of Brooklyn. He went to
Cathedral High School and St. Francis Preparatory School be fore
Fordham. He had ambi tions to study for the Roman Catholic priesthood
for a while, but after graduation he went to law school for a year.

He supported himself by play ing for a minor league football team, the
Brooklyn Eagles, and serving as an insurance investi gator. But a
coaching career was calling and in 1939 he joined the faculty at St. Ce
celia High School in Englewood, N. J. For an annual salary or \$1,700 he
was an assistant foot ball coach and a teacher of physics, chemistry,
algebra and Latin.

\textbf{36 Victories in Row}

Lombardi stayed at St. Ceci lia for seven years. He soon was head coach
of the foot ball, basketball and baseball squads. His football teams won
six state championships and had a string of 36 victories in a row.

He returned to Fordham to coach the freshmen in 1947 and served as an
assistant in 1948. When Ed Danowski was reappointed head coach for the
1949 season, Lombardi left and joined Col. Earl Blaik's staff at the
United States Military Academy.

Life at West Point suited Lombardi and he was strongly influenced by
Colonel Blaik, who had his own hero, Gen. Douglas MacArthur. Lombardi,
too, became a disciple of Gen eral MacArthur and in ensuing years he
attempted to inspire his teams by quoting one or the other of the
military men with sayings such as, ``If you can walk, you can run.''

Pro football beckoned in 1954 when the New York Giants put together a
new coaching staff under Jim Lee Howell, who del egated the offense to
Lom bardi.

``Vince didn't understand our game,'' said Frank Gifford, one of his
stars. ``At first we play ers were showing him. But by the end of the
season he was showing us.''

\textbf{Green Bay's Offer}

Lombardi's opportunity to be a head coach did not come until 1959, when
he was 46, which is considered old in that line of work. The Green Bay
Pack ers, a community‐owned team in a city of only 70,000, were losers
and troubled financially.

The directors offered Lom bardi the job as coach and general manager. He
insisted upon full authority and they gave it to him. The prior coach,
Ray McLean, had a team that won only one game in 12. With a nucleus of
the same play ers, Lombardi's first Packer team won seven of 12 games
and tied for third place in the western division of the N.F.L.

The next season they were first but lost the league cham pionship to the
Phila delphia Eagles. Then the parade began, with league titles in 1961,
1962, 1965, 1966 and 1967, plus Super Bowl victories over the American
League champion in 1967 and 1968.

During his span of nine sea sons as head coach, Lombardi saw his teams
win 141 games, lose 39 and tie four. He in sisted that the Packers never
lost. Time merely ran out on them.

\textbf{Avenue Named for Him}

Green Bay, the smallest city in the league, became national ly known,
and the citizens adulated Lombardi. They named the street outside the
stadium Lombardi Avenue.

One year when the Los An geles Rams were striving to woo him away, the
directors gave Lombardi 320 acres of apple orchards in nearby Door Coun
ty.

Under the rules of their in corporation, the Packers could not pay their
1,700 stockhold ers any dividends. The money piled up and Lombardi spent
it in enlarging the stadium and building a magnificent field house.
After the winning of the first championship, he bought the players wives
mink stoles.

To keep a touted rookie from Texas, Donny Anderson, from signing with
the rival American League, Lombardi agreed to pay the young man the high
est bonus in pro football's his tory, \$600,000.

Lombardi, who had a keen appreciation for money, relat ed winning to
business success in pro football. ``The teams that win the most make the
most money,'' he said.

Although the Packers had an nual profits as high as \$800‐ 000, Lombardi
insisted upon keeping players' salaries ``in line.'' Jim Ringo, a center
who once played a game for Lombardi with 14 painful boils, held out for
more money one season. He was traded the next.

The Packer fullback star, Jim, Taylor, exercised the option clause in
his contract and be came a free agent so he could sign for more money
with an other team. The other running star, Paul Hornung, retired the
same year. ``We'll miss Hornung,'' Lombardi said. ``The other fellow
we'll re place.''

Hornung, like Gifford, was a favorite of the coach. Lom bardi was deeply
hurt when Hornung was suspended for the 1963 season for gambling in
violation of his contract, but Lombardi quickly forgave him.

The Catholic and military in fluences upon Lombardi were strong. After
the assassination of Senator Robert F. Kennedy, whom he knew, Lombardi
said, ``What's the matter with the world? There has been complete
breakdown of mental discipline.''

In speaking before an au dience of businessmen, Lom bardi said, ``There
is an abuse of freedom in our ciety---freedom without respon sibility.''

He deplored the long hair, the sideburns and mustaches' of youth. He
told a Redskin rookie, Trenton Jackson, ``You could run faster if you
didn't have that thing on your lip.'' Jackson shaved off the mus tache
at lunchtime.

\textbf{No Mystery, He Said}

Lombardi maintained there was no mystery to the Packer success.
``Coaches who can out line plays on a blackboard are a dime a dozen,''
he once said. ``The ones who win get inside their players and moti vate
them.''

Perhaps there was no mys tery. But the Packers had a los ing record the
first season aft er he retired as coach.

Lombardi believed in attack ing strength. ``Hit them at their strongest
point,'' he said. Be fore their first regular season game in his first
year at Green Bay, Lombardi told the Pack ers in the locker room, ``Go
through that door and bring back victory.''

Bill Forester, a tackle, said, ``I jumped up and hit my arm on my
locker. It was the worst injury I had all year.''

Lombardi loved to laugh, and his friends delighted in his company. But
he put off strangers, and the public re garded him with both awe and
fear.

Rene Carpenter, the form er wife of the astronaut Scott Carpenter,
described a recep tion held for Lombardi when he first came to
Washington. ``All of a sudden my skirt was too short and my back too
bare,'' she said. ``We were re duced to feeling like children.''

\textbf{Underwent Surgery Twice}

Although the seriousness of Lombardi's condition had been known to close
friends, no pub lic announcement on his condi tion was made until Wednes
day when Mrs. Lombardi au thorized a statement describ ing him as
suffering from ``an extraordinarily virulent form of cancer.''

He underwent two opera tions. After the first, on June 27, he appeared
to be recover ing and was released on July 14. However, his condition de
teriorated and he had to be operated on a second time, on July 27.

In addition to his wife, the former Marie Planitz, and son, the coach
leaves a daughter, Mrs. Susan Bickham of Green Bay, Wis.; his parents,
Mr. and Mrs. Harry Lombardi of Brook lyn; two brothers, Joseph of
Englewood, N. J., and Harold, of San Rafael, Calif, a sister, Mrs.
Claire Brandshagen of Hazlett, N. J., and six grand children.

The body will lie at Joseph Gawler's Sons Funeral Home, Wisconsin Avenue
and Harri son Street, N.W. in Washing ton, on Friday after 2 P.M. and on
Saturday and Sunday, from 10 A.M. to 10 P.M. at The Abbey, 888 Lexington
Avenue, at 66th Street in New York.

A mass will be celebrated at St. Patrick's Cathedral on Monday at 11
A.M. by Terence Cardinal Cooke, Archbishop of the Diocese of New York,
de scribed by a Redskins spokes man as ``a great admirer and old
friend.''

Burial will be that afternoon at Mount Olivet Cemetery Mid dletown
Township near Red Bank, N. J.

Arthur Daley pays tribute to Vince Lombardi in Sports of The Times on
Page 32.

Advertisement

\protect\hyperlink{after-bottom}{Continue reading the main story}

\hypertarget{site-index}{%
\subsection{Site Index}\label{site-index}}

\hypertarget{site-information-navigation}{%
\subsection{Site Information
Navigation}\label{site-information-navigation}}

\begin{itemize}
\tightlist
\item
  \href{https://help.nytimes3xbfgragh.onion/hc/en-us/articles/115014792127-Copyright-notice}{©~2020~The
  New York Times Company}
\end{itemize}

\begin{itemize}
\tightlist
\item
  \href{https://www.nytco.com/}{NYTCo}
\item
  \href{https://help.nytimes3xbfgragh.onion/hc/en-us/articles/115015385887-Contact-Us}{Contact
  Us}
\item
  \href{https://www.nytco.com/careers/}{Work with us}
\item
  \href{https://nytmediakit.com/}{Advertise}
\item
  \href{http://www.tbrandstudio.com/}{T Brand Studio}
\item
  \href{https://www.nytimes3xbfgragh.onion/privacy/cookie-policy\#how-do-i-manage-trackers}{Your
  Ad Choices}
\item
  \href{https://www.nytimes3xbfgragh.onion/privacy}{Privacy}
\item
  \href{https://help.nytimes3xbfgragh.onion/hc/en-us/articles/115014893428-Terms-of-service}{Terms
  of Service}
\item
  \href{https://help.nytimes3xbfgragh.onion/hc/en-us/articles/115014893968-Terms-of-sale}{Terms
  of Sale}
\item
  \href{https://spiderbites.nytimes3xbfgragh.onion}{Site Map}
\item
  \href{https://help.nytimes3xbfgragh.onion/hc/en-us}{Help}
\item
  \href{https://www.nytimes3xbfgragh.onion/subscription?campaignId=37WXW}{Subscriptions}
\end{itemize}
