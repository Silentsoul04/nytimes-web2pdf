Sections

SEARCH

\protect\hyperlink{site-content}{Skip to
content}\protect\hyperlink{site-index}{Skip to site index}

\href{https://www.nytimes3xbfgragh.onion/section/us}{U.S.}

\href{https://myaccount.nytimes3xbfgragh.onion/auth/login?response_type=cookie\&client_id=vi}{}

\href{https://www.nytimes3xbfgragh.onion/section/todayspaper}{Today's
Paper}

\href{/section/us}{U.S.}\textbar{}FIRST MAJOR CUTS IN SOCIAL SECURITY
PROPOSED IN DETAILED REAGAN PLAN

\url{https://nyti.ms/2a3kl1T}

\begin{itemize}
\item
\item
\item
\item
\item
\end{itemize}

Advertisement

\protect\hyperlink{after-top}{Continue reading the main story}

Supported by

\protect\hyperlink{after-sponsor}{Continue reading the main story}

\hypertarget{first-major-cuts-in-social-security-proposed-in-detailed-reagan-plan}{%
\section{FIRST MAJOR CUTS IN SOCIAL SECURITY PROPOSED IN DETAILED REAGAN
PLAN}\label{first-major-cuts-in-social-security-proposed-in-detailed-reagan-plan}}

By \href{https://www.nytimes3xbfgragh.onion/by/david-e-rosenbaum}{David
E. Rosenbaum, Special To the New York Times}

\begin{itemize}
\item
  May 13, 1981
\item
  \begin{itemize}
  \item
  \item
  \item
  \item
  \item
  \end{itemize}
\end{itemize}

\includegraphics{https://s1.graylady3jvrrxbe.onion/timesmachine/pages/1/1981/05/13/118220_360W.png?quality=75\&auto=webp\&disable=upscale}

See the article in its original context from\\
May 13, 1981, Section A, Page
1\href{https://store.nytimes3xbfgragh.onion/collections/new-york-times-page-reprints?utm_source=nytimes\&utm_medium=article-page\&utm_campaign=reprints}{Buy
Reprints}

\href{http://timesmachine.nytimes3xbfgragh.onion/timesmachine/1981/05/13/118220.html}{View
on timesmachine}

TimesMachine is an exclusive benefit for home delivery and digital
subscribers.

About the Archive

This is a digitized version of an article from The Times's print
archive, before the start of online publication in 1996. To preserve
these articles as they originally appeared, The Times does not alter,
edit or update them.

Occasionally the digitization process introduces transcription errors or
other problems; we are continuing to work to improve these archived
versions.

The Reagan Administration today announced details of a proposal for the
first significant reduction in Social Security benefits since the
retirement system was adopted 46 years ago, and the plan was immediately
assailed by some members of Congress and groups representing the
elderly.

Legislative authorities said that the prospects for the proposals in
Congress depended on whether lobbying by the Text of Schweiker
statement, page A29. Reagan Administration could overcome the intense
pressure that is likely to be placed on senators and representatives by
their elderly constituents.

A New York State conference on aging, convened to prepare for a White
House session later this year, put aside its agenda today to condemn
President Reagan's proposal. (Page B1.)

The most controversial elements of the plan would sharply reduce the
pensions of workers who retire before the age of 65, reduce somewhat the
payments to those who retire at the age of 65 and older and abolish the
limit on how much the elderly can earn without losing retirement
benefits.

For the most part, the benefit reductions would apply only to workers
who retire after the first of next year. The only proposed change
affecting the more than 36 million Americans who are already retired and
drawing benefits is one that would postpone from July until October,
beginning next year, the annual cost-of-living increase in payments.

An outline of the Reagan plan was divulged yesterday by White House
officials, and Richard S. Schweiker, the Secretary of Health and Human
Services, described it in detail at a news conference this morning.

Mr. Schweiker said that the reductions in benefits were necessary to
prevent the retirement fund from running out of money as early as next
year.

''The crisis is inescapable,'' the Secretary said. ''It is here. It is
now. It is serious. And it must be faced.'' In addition to the
short-term financing problem, the Social Security system faces a deficit
early in the next century when Americans born right after World War II
begin to retire. By the year 2030, according to actuaries, there could
be only three workers for every retired person, compared with five
workers for each retired person now. Mr. Schweiker said that the
Administration's plan would meet that long-term problem, as well as the
short-run difficulties.

Senator Robert J. Dole, Republican of Kansas, and Senator William L.
Armstrong, Republican of Colorado, praised the proposals and promised to
begin hearings on the legislation soon. Mr. Dole is chairman of the
Finance Committee and Mr. Armstrong is chairman of the Social Security
Subcommittee.

The chairman of the House Social Security subcommittee, Representative
J.J. Pickle, Democrat of Texas, said that his panel would ''consider
these proposals so that we can come to some bipartisan approach.''

Representative Claude Pepper, Democrat of Florida, the chairman of the
Select Committee on Aging, termed the plan ''insidious'' and ''cruel,''
and Senator Howard M. Metzenbaum accused the Administration of a
''breach of its promise'' to the elderly.

Criticism was also voiced by organized labor, but perhaps the most
politically significant statements came from representatives of the
elderly. A Plan to Pressure Congress

Laurie Fiori, legislative representative of the American Association of
Retired Persons, which has 12.5 million members, said, ''We've got a
grass-roots network, and we're going to put them to task on this issue
to put pressure on their members of Congress.''

The House and Senate have both taken preliminary steps on the Social
Security issue this year but have delayed conclusive action while
waiting for the Administration's proposal.

Administration officials calculated that the Reagan proposals would save
about \$9 billion in the fiscal year 1982, which begins Oct. 1, and
\$46.4 billion over the next five fiscal years.

Here is how some of the proposals would work: Early Retirement

Workers can now retire at the age of 62 and receive for the rest of
their lives 80 percent of the benefits they would be entitled to if they
had retired at the age of 65. The Administration would reduce that
proportion to 55 percent, meaning that workers who retire at the age of
62 next year would receive on the average \$126 a month less than they
would get under present law. About 70 percent of workers now retire and
begin drawing benefits before the age of 65.

The proposal would not affect workers who took early retirement before
next Jan. 1, and some career experts in the Social Security
Administration said they expected a rush to retirement late this year if
the plan is enacted.

In addition, benefits would be abolished to children of workers who take
early retirement. Now, children under 18, or if they are in school under
the age of 22, are eligible for benefits on the basis of their retired
parents' wage record. Initial Benefits

The Administration would alter the formula under which initial benefits
are calculated for those who retire at the age of 65 and afterward. John
A. Svahn, the Social Security Commissioner, said that the current
formula overcompensated for the rate of inflation. The change would mean
that the average worker retiring at the age of 65 in 1987 would receive
about 4 percent less each month than he would get if the law was not
changed and that some workers would get nearly 9 percent less per month.
Cost of Living

Payment of the annual cost-of-living increase in Social Security
benefits, which is based on the increase in the Consumer Price Index,
would be delayed from July until October. The saving in the next fiscal
year would be \$3.3 billion, based on a projected inflation rate of
somewhat more than 8 percent.

The change would take effect next year and would reduce the annual
benefits to all Social Security recipients every year thereafter.
Beneficiaries will continue to get an increase of more than 11 percent
this July. Federal and State Employees

Federal and some state government employees are not covered under Social
Security but have their own retirement systems. Under the current law,
such a person can retire from his government job, begin drawing a
pension, work for a few years in a job covered by Social Security and
then draw disproportionately high Social Security benefits in addition
to his other pension. The Administration would take the other pension
into account when calculating Social Security benefits, thus lowering
those benefits. Retirement Earnings

Currently, those who continue to work after the age of 65 have their
benefits reduced by \$1 for each \$2 they earn in excess of \$5,500 a
year until they are 72 years old. The Administration would raise the
earnings ceiling to \$10,000 in 1983 and \$15,000 in 1983 and lift it
entirely after that. The proposal would cost an estimated \$6.5 billion
over the next five fiscal years. Disability

Several changes would be made in the Social Security disability system.
Workers would be declared disabled only on the basis of medical factors
and not age, education and work experience as is now the case. The
workers claiming disability would also have to show that the prognosis
was that they could not work for 24 months, instead of 12 months, which
is now the test. Such workers would also have to have worked for 30 of
the last 40 calendar quarters, instead of 20 of 40, and would have to
wait six months, intead of five months, before being eligible for
benefits. Sick Pay

Sick pay, which is not now subject to Social Security taxes, would be
taxed for the first six months a worker was ill and off the job. Tax
Rates

Social Security taxes are now 6.65 percent of the first \$29,700 of
earned income and the rate is scheduled to increase gradually until it
reaches 7.65 percent in 1990. Under today's proposal, the rate would be
lowered when the money in the Social Security trust funds reached 50
percent of the amount that would have to be paid out in the following
year. Administration officials said that, under current economic
projections, the tax rate might be lowered slightly in the next few
years until it was 6.45 percent in 1990.

Advertisement

\protect\hyperlink{after-bottom}{Continue reading the main story}

\hypertarget{site-index}{%
\subsection{Site Index}\label{site-index}}

\hypertarget{site-information-navigation}{%
\subsection{Site Information
Navigation}\label{site-information-navigation}}

\begin{itemize}
\tightlist
\item
  \href{https://help.nytimes3xbfgragh.onion/hc/en-us/articles/115014792127-Copyright-notice}{©~2020~The
  New York Times Company}
\end{itemize}

\begin{itemize}
\tightlist
\item
  \href{https://www.nytco.com/}{NYTCo}
\item
  \href{https://help.nytimes3xbfgragh.onion/hc/en-us/articles/115015385887-Contact-Us}{Contact
  Us}
\item
  \href{https://www.nytco.com/careers/}{Work with us}
\item
  \href{https://nytmediakit.com/}{Advertise}
\item
  \href{http://www.tbrandstudio.com/}{T Brand Studio}
\item
  \href{https://www.nytimes3xbfgragh.onion/privacy/cookie-policy\#how-do-i-manage-trackers}{Your
  Ad Choices}
\item
  \href{https://www.nytimes3xbfgragh.onion/privacy}{Privacy}
\item
  \href{https://help.nytimes3xbfgragh.onion/hc/en-us/articles/115014893428-Terms-of-service}{Terms
  of Service}
\item
  \href{https://help.nytimes3xbfgragh.onion/hc/en-us/articles/115014893968-Terms-of-sale}{Terms
  of Sale}
\item
  \href{https://spiderbites.nytimes3xbfgragh.onion}{Site Map}
\item
  \href{https://help.nytimes3xbfgragh.onion/hc/en-us}{Help}
\item
  \href{https://www.nytimes3xbfgragh.onion/subscription?campaignId=37WXW}{Subscriptions}
\end{itemize}
