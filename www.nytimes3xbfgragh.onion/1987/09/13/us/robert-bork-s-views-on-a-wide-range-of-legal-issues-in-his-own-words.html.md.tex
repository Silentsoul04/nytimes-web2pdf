Sections

SEARCH

\protect\hyperlink{site-content}{Skip to
content}\protect\hyperlink{site-index}{Skip to site index}

\href{https://www.nytimes3xbfgragh.onion/section/us}{U.S.}

\href{https://myaccount.nytimes3xbfgragh.onion/auth/login?response_type=cookie\&client_id=vi}{}

\href{https://www.nytimes3xbfgragh.onion/section/todayspaper}{Today's
Paper}

\href{/section/us}{U.S.}\textbar{}Robert Bork's Views on a Wide Range of
Legal Issues, in His Own Words

\url{https://nyti.ms/29naTKX}

\begin{itemize}
\item
\item
\item
\item
\item
\end{itemize}

Advertisement

\protect\hyperlink{after-top}{Continue reading the main story}

Supported by

\protect\hyperlink{after-sponsor}{Continue reading the main story}

\hypertarget{robert-borks-views-on-a-wide-range-of-legal-issues-in-his-own-words}{%
\section{Robert Bork's Views on a Wide Range of Legal Issues, in His Own
Words}\label{robert-borks-views-on-a-wide-range-of-legal-issues-in-his-own-words}}

Special to the New York Times

\begin{itemize}
\item
  Sept. 13, 1987
\item
  \begin{itemize}
  \item
  \item
  \item
  \item
  \item
  \end{itemize}
\end{itemize}

\includegraphics{https://s1.graylady3jvrrxbe.onion/timesmachine/pages/1/1987/09/13/080387_360W.png?quality=75\&auto=webp\&disable=upscale}

See the article in its original context from\\
September 13, 1987, Section 1, Page
36\href{https://store.nytimes3xbfgragh.onion/collections/new-york-times-page-reprints?utm_source=nytimes\&utm_medium=article-page\&utm_campaign=reprints}{Buy
Reprints}

\href{http://timesmachine.nytimes3xbfgragh.onion/timesmachine/1987/09/13/080387.html}{View
on timesmachine}

TimesMachine is an exclusive benefit for home delivery and digital
subscribers.

About the Archive

This is a digitized version of an article from The Times's print
archive, before the start of online publication in 1996. To preserve
these articles as they originally appeared, The Times does not alter,
edit or update them.

Occasionally the digitization process introduces transcription errors or
other problems; we are continuing to work to improve these archived
versions.

Following are excerpts from writings, speeches and court opinions by
Judge Robert H. Bork on a range of issues: ON THE SUPREME COURT

We have a Court which is creating individual rights which are not to be
found in the Constitution by any standard method of interpretation. The
Court itself, from time to time, admits that, and more significantly,
the defenders of the Court's performance admit it.

...What the courts are doing. . .is in fact to create new constitutional
values which are nothing more than the imposition of upper-middle-class
values on the society.

''The liberty of free men, among other things, is the liberty to make
laws, which is increasingly being denied.''

- 1982 speech to the Yale University Federalist Society. ON ABORTION

Roe v. Wade is an unconstitutional decision, a serious and wholly
unjustifiable judicial usurpation of state legislative authority. I also
think that Roe v. Wade is by no means the only example of such
unconstitutional behavior by the Supreme Court. . .Without any warrant
in the Constitution, the courts have required so many basic and
unsettling changes in American life and government that a political
response was inevitable.

- 1981 Senate testimony on pending legislation. ON AFFIRMATIVE ACTION

Justice Powell's middle position - universities may not use raw racial
quotas but may consider race, among other factors, in the interest of
diversity among the student body - has been praised as a statesmanlike
solution to an agonizing problem. It may be. Unfortunately, in
constitutional terms, his argument is not ultimately persuasive.

...Mr. Justice Brennan's opinion for the four-member bloc that approved
racial quotas fares no better.

...The argument offends both ideas of common justice and the Fourteenth
Amendment's guarantee of equal protection to persons, not classes.

- 1978 Op-Ed article in The Wall Street Journal. ON WAR POWERS

I think there is no reason to doubt that President Nixon had ample
constitutional authority to order the attack upon the sanctuaries in
Cambodia seized by North Vietnamese and Vietcong forces. That authority
arises both from the inherent powers of the Presidency and from
Congressional authorization. The real question in this situation is
whether Congress has the constitutional authority to limit the
President's discretion with respect to this attack. Any detailed
intervention by Congress in the conduct of the Vietnamese conflict
constitutes a trespass upon powers the Constitution reposes exclusively
in the President.

...This conclusion in no way detracts from Congress's war powers, for
that body retains control of the issue of war or peace.

- Remarks at a 1971 symposium.

As expiation for Vietnam, we have the War Powers Resolution, an attempt
by Congress to share in detailed decisions about the deployment of U.S.
armed forces in the world. It is probably unconstitutional and certainly
unworkable.

- 1978 Op-Ed article in The Wall Street Journal. ON ANTITRUST AND
CONGRESS

Congress as a whole is institutionally incapable of the sustained,
rigorous and consistent thought that the fashioning of a rational
antitrust policy requires. No group of that size could accomplish the
task. Large bodies simply do not reason coherently together. Congress
could, for example, adopt a useful general goal in the Sherman Act. It
proved unable to supply sensible specifics in the Clayton and
Robinson-Patman Acts.

- ''The Antitrust Paradox,''1978. ON LIBEL AND ORIGINAL INTENT

The law of the First Amendment must not try to make public dispute safe
and comfortable for all the participants. That would only stifle the
debate.

...The American press is extraordinarily free and vigorous, as it should
be. . .Yet, the area in which legal doctrine is currently least adequate
to preserve press freedom is the area of defamation law.

...Judge Scalia's dissent implies that the idea of evolving
constitutional doctrine should be anathema to judges who adhere to a
philosophy of judicial restraint. But. . .when there is a known
principle to be explicated, the evolution of doctrine is inevitable.

...In a case like this, it is the task of the judge in this generation
to discern how the framers' values, defined in the context of the world
they knew, apply to the world we know. The world changes, in which
unchanging values find their application.

...If, over time, the libel action becomes a threat to the central
meaning of the First Amendment, why should not judges adapt their
doctrines?

...A judge who refuses to see new threats to an established
constitutional value, and hence provides a crabbed interpretation that
robs a provision of its full, fair and reasonable meaning, fails in his
judicial duty.

- Concurring opinion in Ollman v. Evans, 1984.

Advertisement

\protect\hyperlink{after-bottom}{Continue reading the main story}

\hypertarget{site-index}{%
\subsection{Site Index}\label{site-index}}

\hypertarget{site-information-navigation}{%
\subsection{Site Information
Navigation}\label{site-information-navigation}}

\begin{itemize}
\tightlist
\item
  \href{https://help.nytimes3xbfgragh.onion/hc/en-us/articles/115014792127-Copyright-notice}{©~2020~The
  New York Times Company}
\end{itemize}

\begin{itemize}
\tightlist
\item
  \href{https://www.nytco.com/}{NYTCo}
\item
  \href{https://help.nytimes3xbfgragh.onion/hc/en-us/articles/115015385887-Contact-Us}{Contact
  Us}
\item
  \href{https://www.nytco.com/careers/}{Work with us}
\item
  \href{https://nytmediakit.com/}{Advertise}
\item
  \href{http://www.tbrandstudio.com/}{T Brand Studio}
\item
  \href{https://www.nytimes3xbfgragh.onion/privacy/cookie-policy\#how-do-i-manage-trackers}{Your
  Ad Choices}
\item
  \href{https://www.nytimes3xbfgragh.onion/privacy}{Privacy}
\item
  \href{https://help.nytimes3xbfgragh.onion/hc/en-us/articles/115014893428-Terms-of-service}{Terms
  of Service}
\item
  \href{https://help.nytimes3xbfgragh.onion/hc/en-us/articles/115014893968-Terms-of-sale}{Terms
  of Sale}
\item
  \href{https://spiderbites.nytimes3xbfgragh.onion}{Site Map}
\item
  \href{https://help.nytimes3xbfgragh.onion/hc/en-us}{Help}
\item
  \href{https://www.nytimes3xbfgragh.onion/subscription?campaignId=37WXW}{Subscriptions}
\end{itemize}
