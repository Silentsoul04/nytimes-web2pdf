Sections

SEARCH

\protect\hyperlink{site-content}{Skip to
content}\protect\hyperlink{site-index}{Skip to site index}

\href{https://www.nytimes3xbfgragh.onion/section/us}{U.S.}

\href{https://myaccount.nytimes3xbfgragh.onion/auth/login?response_type=cookie\&client_id=vi}{}

\href{https://www.nytimes3xbfgragh.onion/section/todayspaper}{Today's
Paper}

\href{/section/us}{U.S.}\textbar{}Biden Admits Plagiarism in School But
Says It Was Not 'Malevolent'

\url{https://nyti.ms/29nboor}

\begin{itemize}
\item
\item
\item
\item
\item
\end{itemize}

Advertisement

\protect\hyperlink{after-top}{Continue reading the main story}

Supported by

\protect\hyperlink{after-sponsor}{Continue reading the main story}

\hypertarget{biden-admits-plagiarism-in-school-but-says-it-was-not-malevolent}{%
\section{Biden Admits Plagiarism in School But Says It Was Not
'Malevolent'}\label{biden-admits-plagiarism-in-school-but-says-it-was-not-malevolent}}

By E. J. Dionne Jr., Special To the New York Times

\begin{itemize}
\item
  Sept. 18, 1987
\item
  \begin{itemize}
  \item
  \item
  \item
  \item
  \item
  \end{itemize}
\end{itemize}

\includegraphics{https://s1.graylady3jvrrxbe.onion/timesmachine/pages/1/1987/09/18/624487_360W.png?quality=75\&auto=webp\&disable=upscale}

See the article in its original context from\\
September 18, 1987, Section A, Page
1\href{https://store.nytimes3xbfgragh.onion/collections/new-york-times-page-reprints?utm_source=nytimes\&utm_medium=article-page\&utm_campaign=reprints}{Buy
Reprints}

\href{http://timesmachine.nytimes3xbfgragh.onion/timesmachine/1987/09/18/624487.html}{View
on timesmachine}

TimesMachine is an exclusive benefit for home delivery and digital
subscribers.

About the Archive

This is a digitized version of an article from The Times's print
archive, before the start of online publication in 1996. To preserve
these articles as they originally appeared, The Times does not alter,
edit or update them.

Occasionally the digitization process introduces transcription errors or
other problems; we are continuing to work to improve these archived
versions.

Senator Joseph R. Biden Jr., fighting to salvage his Presidential
campaign, today acknowledged ''a mistake'' in his youth, when he
plagiarized a law review article for a paper he wrote in his first year
at law school.

Mr. Biden insisted, however, that he had done nothing ''malevolent,''
that he had simply misunderstood the need to cite sources carefully. And
he asserted that another controversy, concerning recent reports of his
using material from others' speeches without attribution, was ''much ado
about nothing.''

Mr. Biden, the 44-year-old Delaware Democrat who heads the Senate
Judiciary Committee, addressed these issues at the Capitol in a morning
news conference he had called expressly for that purpose. The news
conference was held just before he presided over the third day of
hearings on the nomination of Judge Robert H. Bork to the Supreme Court.

To buttress his assertions of sincerity and openness, Mr. Biden released
a 65-page file, obtained by the Senator from the Syracuse University
College of Law, that he said contained all the records of his years
there. It disclosed relatively poor grades in college and law school,
mixed evaluations from teachers and details of the plagiarism.

Both the current dean of the law school and Mr. Biden's professor today
played down the incident of plagiarism. {[} Page A23. {]} Brushing aside
any suggestion that he might be forced to withdraw from the Presidential
race, Mr. Biden declared at the news conference, ''I'm in the race to
stay, I'm in the race to win, and here I come.'' Blames Rivals

Mr. Biden also suggested that the recent damaging information about him
had originated with other campaigns, which he did not identify, and that
it had emerged now because he was enjoying a chance in the limelight
with the Bork hearings.

''Look, I'm a big boy,'' he said. ''I've been in politics for 15 years.
This is not my style. If they want to do it this way, so be it.''

The file distributed by the Senator included a law school faculty
report, dated Dec. 1, 1965, that concluded that Mr. Biden had ''used
five pages from a published law review article without quotation or
attribution'' and that he ought to be failed in the legal methods course
for which he had submitted the 15-page paper.

The plagiarized article, ''Tortious Acts as a Basis for Jurisdiction in
Products Liability Cases,'' was published in the Fordham Law Review of
May 1965. Mr. Biden drew large chunks of heavy legal prose directly from
it, including such sentences as: ''The trend of judicial opinion in
various jurisdictions has been that the breach of an implied warranty of
fitness is actionable without privity, because it is a tortious wrong
upon which suit may be brought by a non-contracting party.'' Just One
Footnote

In his paper, Mr. Biden included a single footnote to the Fordham Law
Review article.

In a letter defending himself, dated Nov. 30, 1965, Mr. Biden pleaded
with the faculty not to dismiss him from the school.

''My intent was not to deceive anyone,'' Mr. Biden wrote. ''For if it
were, I would not have been so blatant.''

At another point, the young Mr. Biden said that ''if I had intended to
cheat, would I have been so stupid?''

''I value my word above all else,'' the impassioned letter said. ''This
is a fact which is known to all those who are or have been acquainted
with my character.'' Misunderstanding, He Says

Mr. Biden said today, as he did 22 years ago, that he had misunderstood
the rules of citation and footnoting.

''I was wrong, but I was not malevolent in any way,'' Mr. Biden said.
''I did not intentionally move to mislead anybody. And I didn't. To this
day I didn't.''

The faculty ruled that Mr. Biden would get an F in the course but would
have the grade stricken when he retook it the next year. Mr. Biden
eventually received a grade of 80 in the course, which, he joked today,
prevented him from falling even further in his class rank. Mr. Biden,
who graduated from the law school in 1968, was 76th in a class of 85.

The file also included Mr. Biden's transcript from his days as an
undergraduate at the University of Delaware. In his first three
semesters, his grades were C's or D's, with three exceptions: two A's in
physical education courses, a B in a course on ''Great English Writers''
and an F in R.O.T.C. The grades improved somewhat later but were never
exceptional. Biden's Defense on Speeches

As for the issue of borrowing speeches, Mr. Biden was insistent that he
had done nothing wrong. He said it was ''ludicrous'' to expect a
politician to attribute all the quotations of others, and he cited two
examples to support his argument.

One was from one of his adversaries for the Democratic nomination, the
Rev. Jesse Jackson, whom Mr. Biden described as ''a friend.'' Mr.
Jackson, Mr. Biden said, has used the same part of a speech by Hubert H.
Humphrey that Mr. Biden has been accused of improperly appropriating,
and Mr. Jackson has called him to say so.

Robert F. Kennedy, another of those whose speeches have been echoed by
Mr. Biden, also used passages without attribution, the Senator said.

Mr. Biden appealed to voters to accept him as ''a middle-class guy'' who
makes mistakes but tells the truth. Of his campaign chances, he said:
''It'll all be dependent on the American people looking at me. They're
going to look at me and say, 'Is Joe Biden being honest with me, or is
Joe Biden not being honest with me?'

''I'm being honest,'' Mr. Biden said firmly. Support in Senate

Mr. Biden won strong support from a number of Senate colleagues today.

At the Bork hearings, Senator Alan K. Simpson, Republican of Wyoming,
praised Mr. Biden with words similar to those once invoked by Theodore
Roosevelt in salute to politicians.

''I don't know where all this stuff will go with regard to your present
situation,'' said Mr. Simpson, one of the most popular members of the
Senate. ''Hang on tight. You have at least had the guts to throw
yourself in the public arena, to run for the Presidency. And that's
better than a lot of faint-hearted detractors will ever do in this
world, and they will be the ones who will try to sully you and pull you
down.

''And so more power to you as you grapple with that one.''

The Senate majority leader, Robert C. Byrd of West Virginia, said:
''Senator Biden is a man of good intentions who means well for America.
His credibility is good with me.'' Healey Wasn't Involved

In the course of his news conference, Mr. Biden also acknowleged that he
was mistaken when he implied on several occasions that it was Denis
Healey, a prominent British Labor Party official, who had given him a
videotape of another speech whose words the Senator later used. In
London, Mr. Healey's office denied giving Mr. Biden the tape, and today
the Senator said that in fact it had not come from Mr. Healey.

In addition, Mr. Biden said that in his talks invoking that speech, by
Neil Kinnock, the Labor Party leader, he had miscast some of his own
forebears, painting them as having rather more humble origins than they
in fact did. For example, borrowing Mr. Kinnock's sentiments, Mr. Biden
had said he was ''the first in his family ever to go to university.'' In
fact, Mr. Biden said today, ''there are Finnegans, my mother's family,
that went to college.''

Mr. Biden also appeared to signal a shift in the way he is casting
himself politically, toward an image as a leader of the ordinary middle
class rather than as a civil rights and antiwar firebrand.

''During the 60's, I was, in fact, very concerned about the civil rights
movement,'' he said. But at another point he said, ''I was not an
activist,'' adding:

''I worked at an all-black swimming pool in the east side of Wilmington,
Del. I was involved in what they were thinking, what they were feeling.
But I was not out marching. I was not down in not out marching. I was
not down in Selma. I was not anywhere else. I was a suburbanite kid who
got a dose of exposure to what was happening to black Americans.''

In an address to the New Jersey Democratic State Convention on Sept. 13,
1983, Mr. Biden appeared to suggest that he had been deeply involved in
civil rights battles.

''When I was 17, I participated in sit-ins to desegregate restaurants
and movie houses,'' he declared then. ''And my stomach turned upon
hearing the voices of Faubus and Wallace. My soul raged on seeing Bull
Connor and his dogs.''

Asked about the apparent inconsistency, Larry Rasky, the Senator's press
secretary, said that as a youth in Wilmington, Mr. Biden ''did
participate in action to desegregate one restaurant and one movie
theater.''

Near the end of his news conference, Mr. Biden issued a dramatic defense
of the man he considers himself to be. He offered a kind of rebuttal to
reporters who have insistently asked how, having once cast himself as
the candidate of a ''new generation'' who spoke often of the civil
rights and antiwar movements, he coulld have done so with little record
of participation in either movement as a young man. He called the
queries ''bizarre.''

''When I was at Syracuse,'' he said, ''I was married, I was in law
school, I wore sports coats. You're looking at a middle-class guy. I am
who I am. I'm not big on flak jackets and tie-dyed shirts. You know,
that's not me.''

Advertisement

\protect\hyperlink{after-bottom}{Continue reading the main story}

\hypertarget{site-index}{%
\subsection{Site Index}\label{site-index}}

\hypertarget{site-information-navigation}{%
\subsection{Site Information
Navigation}\label{site-information-navigation}}

\begin{itemize}
\tightlist
\item
  \href{https://help.nytimes3xbfgragh.onion/hc/en-us/articles/115014792127-Copyright-notice}{©~2020~The
  New York Times Company}
\end{itemize}

\begin{itemize}
\tightlist
\item
  \href{https://www.nytco.com/}{NYTCo}
\item
  \href{https://help.nytimes3xbfgragh.onion/hc/en-us/articles/115015385887-Contact-Us}{Contact
  Us}
\item
  \href{https://www.nytco.com/careers/}{Work with us}
\item
  \href{https://nytmediakit.com/}{Advertise}
\item
  \href{http://www.tbrandstudio.com/}{T Brand Studio}
\item
  \href{https://www.nytimes3xbfgragh.onion/privacy/cookie-policy\#how-do-i-manage-trackers}{Your
  Ad Choices}
\item
  \href{https://www.nytimes3xbfgragh.onion/privacy}{Privacy}
\item
  \href{https://help.nytimes3xbfgragh.onion/hc/en-us/articles/115014893428-Terms-of-service}{Terms
  of Service}
\item
  \href{https://help.nytimes3xbfgragh.onion/hc/en-us/articles/115014893968-Terms-of-sale}{Terms
  of Sale}
\item
  \href{https://spiderbites.nytimes3xbfgragh.onion}{Site Map}
\item
  \href{https://help.nytimes3xbfgragh.onion/hc/en-us}{Help}
\item
  \href{https://www.nytimes3xbfgragh.onion/subscription?campaignId=37WXW}{Subscriptions}
\end{itemize}
