Sections

SEARCH

\protect\hyperlink{site-content}{Skip to
content}\protect\hyperlink{site-index}{Skip to site index}

\href{https://www.nytimes3xbfgragh.onion/section/obituaries}{Obituaries}

\href{https://myaccount.nytimes3xbfgragh.onion/auth/login?response_type=cookie\&client_id=vi}{}

\href{https://www.nytimes3xbfgragh.onion/section/todayspaper}{Today's
Paper}

\href{/section/obituaries}{Obituaries}\textbar{}Alexis Lichine, 76, an
Author and Expert on Wine

\url{https://nyti.ms/29mdF36}

\begin{itemize}
\item
\item
\item
\item
\item
\end{itemize}

Advertisement

\protect\hyperlink{after-top}{Continue reading the main story}

Supported by

\protect\hyperlink{after-sponsor}{Continue reading the main story}

\hypertarget{alexis-lichine-76-an-author-and-expert-on-wine}{%
\section{Alexis Lichine, 76, an Author and Expert on
Wine}\label{alexis-lichine-76-an-author-and-expert-on-wine}}

By \href{https://www.nytimes3xbfgragh.onion/by/frank-j-prial}{Frank J.
Prial}

\begin{itemize}
\item
  June 2, 1989
\item
  \begin{itemize}
  \item
  \item
  \item
  \item
  \item
  \end{itemize}
\end{itemize}

\includegraphics{https://s1.graylady3jvrrxbe.onion/timesmachine/pages/1/1989/06/02/957489_360W.png?quality=75\&auto=webp\&disable=upscale}

See the article in its original context from\\
June 2, 1989, Section D, Page
15\href{https://store.nytimes3xbfgragh.onion/collections/new-york-times-page-reprints?utm_source=nytimes\&utm_medium=article-page\&utm_campaign=reprints}{Buy
Reprints}

\href{http://timesmachine.nytimes3xbfgragh.onion/timesmachine/1989/06/02/957489.html}{View
on timesmachine}

TimesMachine is an exclusive benefit for home delivery and digital
subscribers.

About the Archive

This is a digitized version of an article from The Times's print
archive, before the start of online publication in 1996. To preserve
these articles as they originally appeared, The Times does not alter,
edit or update them.

Occasionally the digitization process introduces transcription errors or
other problems; we are continuing to work to improve these archived
versions.

Alexis Lichine, an internationally known wine expert and author of
landmark reference books in the field, died of cancer yesterday at
Chateau Prieure-Lichine, his home in Bordeaux, France. He was 76 years
old and had been ill for about six months.

Mr. Lichine was widely credited with introducing many in several
generations of Americans to wine through his books and lectures and the
wines that bore his name.

He was born in 1913 in Moscow, where his father was a businessman. On
the eve of the Bolshevik Revolution, the family fled eastward through
Vladivostok to the United States. They lived in New York City for
several years before moving to Paris to join the sizable White-Russian
community there.

Mr. Lichine attended a lycee in Paris. His father ran tour buses for a
time and his son worked for him as a multilingual guide. Mr. Lichine
then took a job selling advertising space for the Paris edition of The
New York Herald Tribune, which took him back and forth across France
just as Prohibition was ending in the United States. A Job in a Wine
Shop

He saw a vast market opening up in the United States for French wines
and in 1935 returned to New York to take a job in a retail wine shop.

Two years later, he teamed up with Frank Schoonmaker, a New Yorker
writer turned wine importer. Mr. Lichine aways credited Raymond Badouin,
Mr. Schoonmaker's supplier in France, with teaching him the fundamentals
of tasting and buying wine.

During World War II, Mr. Lichine served as a major in Army intelligence
in southern France and Corsica. He later admitted that his superiors
were more interested in his credentials as a connoisseur and often sent
him out to ''requisition'' fine food and wines for high-level dinners
and conferences.

After the war, he returned to importing wine and divided his time
between New York and Bordeaux, where in 1955 he founded Alexis Lichine
\& Company, which he sold 11 years later to a British brewing
conglomerate.

In 1951, he brought out his first book, ''The Wines of France,'' which
went through five editions until it was revised in 1979 and became
''Alexis Lichine's Guide to the Wines and Vineyards of France'' (Alfred
A. Knopf). That book is in its third edition. His ''Encyclopedia of
Wines and Spirits'' (Knopf) first appeared in 1967 and is in its fifth
edition. A Former Priory

In 1951, he bought a run-down wine chateau, Chateau Prieure-Cantenac, in
the village of Cantenac, about 15 miles north of the city of Bordeaux.
The Prieure, which had been a Benedictine priory in the Middle Ages, had
fallen into virtual ruin, even though it had been rated as a
third-growth chateau in the 1855 classfication of Bordeaux chateaux. The
Committee of Classified Growths permitted Mr. Lichine to change the name
to Prieure-Lichine.

In 1953, he purchased parts of two famous vineyards in Burgundy and in
1955 put together a syndicate that included David Rockefeller, Paul
Manheim and Gilbert Kahn to purchase Chateau Lascombes, a prominent wine
property not far from his own Prieure. Lascombes, a second-growth
chauteau, was sold in 1971. Both chateaux are in the prestigious Margaux
region and use that name or appellation on their bottles.

During the 1960's and 70's, Mr. Lichine made three or four six-week
trips a year though the United States, selling wines, conducting
tastings and giving lectures. In Bordeaux, he startled and often
irritated the conservative wine establishment with his often irreverent
remarks about their wines in his two books and with his American-style
promotions for his wine and for himself. He erected large billboards in
the vineyards inviting tourists to his two chateaux. A Close
Relationship

He formed a close, if not always warm, relationship with Philippe de
Rothschild, the owner of the famed Chateau Mouton-Rothschild and a
skilled wine salesman and promoter. For many years, one of Mr. Lichine's
closest confidants in France was Seymour Weller, who ran Chateau
Haut-Brion for the Dillon family in the United States.

Mr. Lichine maintained a vast apartment on Fifth Avenue, at 81st Street,
in Manhattan but in later years spent most of his time at his beloved
Prieure, making his wines, planning new cellars and entertaining.

He was married three times, most recently to the film actress Arlene
Dahl, who has since remarried.

He is survived by a sister, Irene Belesiotis of St. Petersburg, Fla.; a
son, Sacha, of Bordeaux and Manhattan, and a daughter, Alexandra, of
Manhattan.

A Russian Orthodox funeral service is scheduled tomorrow in the Cantenac
church that was originally part of the Prieure, with burial in the
vineyards at Prieure-Lichine.

Advertisement

\protect\hyperlink{after-bottom}{Continue reading the main story}

\hypertarget{site-index}{%
\subsection{Site Index}\label{site-index}}

\hypertarget{site-information-navigation}{%
\subsection{Site Information
Navigation}\label{site-information-navigation}}

\begin{itemize}
\tightlist
\item
  \href{https://help.nytimes3xbfgragh.onion/hc/en-us/articles/115014792127-Copyright-notice}{©~2020~The
  New York Times Company}
\end{itemize}

\begin{itemize}
\tightlist
\item
  \href{https://www.nytco.com/}{NYTCo}
\item
  \href{https://help.nytimes3xbfgragh.onion/hc/en-us/articles/115015385887-Contact-Us}{Contact
  Us}
\item
  \href{https://www.nytco.com/careers/}{Work with us}
\item
  \href{https://nytmediakit.com/}{Advertise}
\item
  \href{http://www.tbrandstudio.com/}{T Brand Studio}
\item
  \href{https://www.nytimes3xbfgragh.onion/privacy/cookie-policy\#how-do-i-manage-trackers}{Your
  Ad Choices}
\item
  \href{https://www.nytimes3xbfgragh.onion/privacy}{Privacy}
\item
  \href{https://help.nytimes3xbfgragh.onion/hc/en-us/articles/115014893428-Terms-of-service}{Terms
  of Service}
\item
  \href{https://help.nytimes3xbfgragh.onion/hc/en-us/articles/115014893968-Terms-of-sale}{Terms
  of Sale}
\item
  \href{https://spiderbites.nytimes3xbfgragh.onion}{Site Map}
\item
  \href{https://help.nytimes3xbfgragh.onion/hc/en-us}{Help}
\item
  \href{https://www.nytimes3xbfgragh.onion/subscription?campaignId=37WXW}{Subscriptions}
\end{itemize}
