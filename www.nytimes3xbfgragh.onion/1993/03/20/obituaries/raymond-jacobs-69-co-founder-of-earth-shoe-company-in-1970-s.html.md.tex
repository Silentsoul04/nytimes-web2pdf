Sections

SEARCH

\protect\hyperlink{site-content}{Skip to
content}\protect\hyperlink{site-index}{Skip to site index}

\href{https://www.nytimes3xbfgragh.onion/section/obituaries}{Obituaries}

\href{https://myaccount.nytimes3xbfgragh.onion/auth/login?response_type=cookie\&client_id=vi}{}

\href{https://www.nytimes3xbfgragh.onion/section/todayspaper}{Today's
Paper}

\href{/section/obituaries}{Obituaries}\textbar{}Raymond Jacobs, 69,
Co-Founder Of Earth Shoe Company in 1970's

\url{https://nyti.ms/29jjRpm}

\begin{itemize}
\item
\item
\item
\item
\item
\end{itemize}

Advertisement

\protect\hyperlink{after-top}{Continue reading the main story}

Supported by

\protect\hyperlink{after-sponsor}{Continue reading the main story}

\hypertarget{raymond-jacobs-69-co-founder-of-earth-shoe-company-in-1970s}{%
\section{Raymond Jacobs, 69, Co-Founder Of Earth Shoe Company in
1970's}\label{raymond-jacobs-69-co-founder-of-earth-shoe-company-in-1970s}}

By \href{https://www.nytimes3xbfgragh.onion/by/bruce-lambert}{Bruce
Lambert}

\begin{itemize}
\item
  March 20, 1993
\item
  \begin{itemize}
  \item
  \item
  \item
  \item
  \item
  \end{itemize}
\end{itemize}

\includegraphics{https://s1.graylady3jvrrxbe.onion/timesmachine/pages/1/1993/03/20/070093_360W.png?quality=75\&auto=webp\&disable=upscale}

See the article in its original context from\\
March 20, 1993, Section 1, Page
10\href{https://store.nytimes3xbfgragh.onion/collections/new-york-times-page-reprints?utm_source=nytimes\&utm_medium=article-page\&utm_campaign=reprints}{Buy
Reprints}

\href{http://timesmachine.nytimes3xbfgragh.onion/timesmachine/1993/03/20/070093.html}{View
on timesmachine}

TimesMachine is an exclusive benefit for home delivery and digital
subscribers.

About the Archive

This is a digitized version of an article from The Times's print
archive, before the start of online publication in 1996. To preserve
these articles as they originally appeared, The Times does not alter,
edit or update them.

Occasionally the digitization process introduces transcription errors or
other problems; we are continuing to work to improve these archived
versions.

Raymond Jacobs, who, with his wife, brought a Danish sandal to America,
called it the Earth Shoe and created an indelible symbol of the 1970's,
died on Wednesday. He was 69.

Mr. Jacobs's heart failed at a nursing home in Torrington, Conn., after
a long illness, his family said.

He and his wife, Eleanor, founded the Earth Shoe Company in 1970 in
Manhattan. The sandal became instantly popular, and the company
eventually established 123 outlets to sell the sandal and other footwear
using the sunken-heel design, across the United States and in Europe.

The Earth Shoe is a wooden sandal with a sunken heel intended to
simulate the feeling of a bare foot sinking into soft ground. The shoe
soon became a counterculture symbol, along with tie-dyed fashions, long
hair and beards, the peace emblem, health food and back-to-nature
communes.

The Earth Shoe phenomenon began quite by chance. In 1969, on a trip to
Europe, the Jacobses made a side trip to Copenhagen, where they spotted
a shop that sold the sandals. They bought two pairs and found that they
relieved their tired feet and backaches. Impressed, they asked about
selling the shoes in the United States.

Anna Kalso, the owner, had rejected offers from conventional shoe
manufacturers, but she took a liking to the American couple after
checking their astrological signs; Mr. Jacobs was a Taurus and Mrs.
Jacobs was a Leo. Ms. Kalso found them to be astrologically correct.
Auspicious Opening Day

The Jacobses came home and opened a store on East 17th Street in 1970.
Opening day, by chance, turned out to be the first Earth Day, and they
dubbed their sandal the Earth Shoe. A rally in nearby Union Square
produced a stream of curious customers. Many were unable to pay, so the
Jacobses gave the shoes away and said, "Pay later." "They all did and
brought back friends to buy more," Mrs. Jacobs said.

Soon the Earth Shoe was featured in the Whole Earth Catalogue and then
in mainstream media. It was advertised in Time and on Johnny Carson's
"Tonight" show. A United States factory opened to meet demand.

The company dissolved in 1977, and the original Earth Shoe became a
nostalgia collectible. A pair is in the Metropolitan Museum of Art's
permanent collection.

The Earth Shoe came after Mr. Jacobs's first career in photography. He
won more than 50 art directors' awards for his advertising work. His art
photography is in the Museum of Modern Art's permanent collection and
the Family of Man Show. His portraits included Louis Armstrong, Gloria
Swanson, Douglas Fairbanks Jr., Salvador Dali and Eartha Kitt.

Besides his wife, his survivors include two daughters, Susan Jacobs of
Manhattan and Laura Jacobs-Pavlick of Litchfield, Conn., and a sister,
Sally Eckerle of Delray Beach, Fla.

Advertisement

\protect\hyperlink{after-bottom}{Continue reading the main story}

\hypertarget{site-index}{%
\subsection{Site Index}\label{site-index}}

\hypertarget{site-information-navigation}{%
\subsection{Site Information
Navigation}\label{site-information-navigation}}

\begin{itemize}
\tightlist
\item
  \href{https://help.nytimes3xbfgragh.onion/hc/en-us/articles/115014792127-Copyright-notice}{©~2020~The
  New York Times Company}
\end{itemize}

\begin{itemize}
\tightlist
\item
  \href{https://www.nytco.com/}{NYTCo}
\item
  \href{https://help.nytimes3xbfgragh.onion/hc/en-us/articles/115015385887-Contact-Us}{Contact
  Us}
\item
  \href{https://www.nytco.com/careers/}{Work with us}
\item
  \href{https://nytmediakit.com/}{Advertise}
\item
  \href{http://www.tbrandstudio.com/}{T Brand Studio}
\item
  \href{https://www.nytimes3xbfgragh.onion/privacy/cookie-policy\#how-do-i-manage-trackers}{Your
  Ad Choices}
\item
  \href{https://www.nytimes3xbfgragh.onion/privacy}{Privacy}
\item
  \href{https://help.nytimes3xbfgragh.onion/hc/en-us/articles/115014893428-Terms-of-service}{Terms
  of Service}
\item
  \href{https://help.nytimes3xbfgragh.onion/hc/en-us/articles/115014893968-Terms-of-sale}{Terms
  of Sale}
\item
  \href{https://spiderbites.nytimes3xbfgragh.onion}{Site Map}
\item
  \href{https://help.nytimes3xbfgragh.onion/hc/en-us}{Help}
\item
  \href{https://www.nytimes3xbfgragh.onion/subscription?campaignId=37WXW}{Subscriptions}
\end{itemize}
