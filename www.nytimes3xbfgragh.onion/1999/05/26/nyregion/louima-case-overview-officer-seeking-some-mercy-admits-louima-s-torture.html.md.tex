Sections

SEARCH

\protect\hyperlink{site-content}{Skip to
content}\protect\hyperlink{site-index}{Skip to site index}

\href{https://www.nytimes3xbfgragh.onion/section/nyregion}{New York}

\href{https://myaccount.nytimes3xbfgragh.onion/auth/login?response_type=cookie\&client_id=vi}{}

\href{https://www.nytimes3xbfgragh.onion/section/todayspaper}{Today's
Paper}

\href{/section/nyregion}{New York}\textbar{}THE LOUIMA CASE: THE
OVERVIEW; Officer, Seeking Some Mercy, Admits to Louima's Torture

\begin{itemize}
\item
\item
\item
\item
\item
\end{itemize}

Advertisement

\protect\hyperlink{after-top}{Continue reading the main story}

Supported by

\protect\hyperlink{after-sponsor}{Continue reading the main story}

THE LOUIMA CASE: THE OVERVIEW

\hypertarget{the-louima-case-the-overview-officer-seeking-some-mercy-admits-to-louimas-torture}{%
\section{THE LOUIMA CASE: THE OVERVIEW; Officer, Seeking Some Mercy,
Admits to Louima's
Torture}\label{the-louima-case-the-overview-officer-seeking-some-mercy-admits-to-louimas-torture}}

By \href{https://www.nytimes3xbfgragh.onion/by/david-barstow}{David
Barstow}

\begin{itemize}
\item
  May 26, 1999
\item
  \begin{itemize}
  \item
  \item
  \item
  \item
  \item
  \end{itemize}
\end{itemize}

See the article in its original context from\\
May 26, 1999, Section A, Page
1\href{https://store.nytimes3xbfgragh.onion/collections/new-york-times-page-reprints?utm_source=nytimes\&utm_medium=article-page\&utm_campaign=reprints}{Buy
Reprints}

\href{http://timesmachine.nytimes3xbfgragh.onion/timesmachine/1999/05/26/868671.html}{View
on timesmachine}

TimesMachine is an exclusive benefit for home delivery and digital
subscribers.

In a faltering, hesitant voice, Justin A. Volpe admitted yesterday that
he rammed a stick into Abner Louima's rectum and then thrust it in his
face, an act, the police officer acknowledged, intended to humiliate and
intimidate the handcuffed Haitian immigrant.

''If you tell anybody about this, I'll find you and kill you,'' Mr.
Volpe said he told Mr. Louima moments after the Aug. 9, 1997, assault in
the restroom of the 70th Precinct station house in Brooklyn.

By admitting guilt, Mr. Volpe hoped to be spared a life sentence for an
assault that cast a shadow on the entire New York Police Department. But
while he offered grim details about acts he has long denied, Mr. Volpe
did not implicate any other officers by name, even though four others
are still on trial in the case. {[}Page B1.{]}

As he confessed to six Federal crimes, Mr. Volpe at times struggled for
words when pressed to explain the forces that compelled him to torture
Mr. Louima.

''When you put the stick up towards his face, having shoved it into his
rectum, was a part of your effort to humiliate him?'' Judge Eugene H.
Nickerson of Federal District Court asked, his tone quietly insistent.

Mr. Volpe paused, unsure of himself. ''I was in shock at the time, Your
Honor,'' he said.

The judge repeated the question.

''I couldn't believe what happened,'' Mr. Volpe said, again seeming to
fumble. And then, ''I was mad.''

Still unsatisfied, Judge Nickerson tried once more. ''You intended to
humiliate him?''

''Yes,'' Mr. Volpe finally said, averting his eyes.

Mr. Volpe, 27, wept only once, at the end of the 45-minute hearing, when
he said to Judge Nickerson, ''Your Honor, if I could just let the record
reflect I'm sorry for hurting my family.''

The judge cut him off. ''You'll get a chance to do that when you come to
sentence,'' he said.

Mr. Volpe wiped the tears from his eyes, and then turned and looked into
the audience for his most outspoken defender, his father, Robert Volpe,
a retired police detective once renowned for solving art thefts. The son
managed a weak smile as guards escorted him from the courtroom.

''There are all different kinds of hell,'' Robert Volpe said later.
''It's not easy seeing your son taken away.''

Mr. Volpe did not apologize to Mr. Louima, but his lawyer, Marvyn M.
Kornberg, said he was clearly remorseful. ''When you plead guilty, I
think that's a sufficient apology,'' Mr. Kornberg said. ''The man's
facing life.''

Even by throwing himself on the mercy of the court, Mr. Volpe faces 30
years in a Federal prison at a minimum, his lawyers said, citing their
analysis of the sentencing guidelines that will be used by Judge
Nickerson. He could also be fined up to \$1.5 million for violating Mr.
Louima's civil rights.

No sentencing date was set yesterday. Mr. Kornberg said that the defense
would try to sway the judge toward leniency by presenting a report from
a psychologist that could trace such things as Mr. Volpe's state of mind
at the time of the attack.

''We intend to present a psychological profile of Justin Volpe,'' he
said. It is not uncommon for defense lawyers to offer such profiles at
sentencing.

The plea was seized on by Mayor Rudolph W. Giuliani as evidence that the
Police Department had turned a corner in its battle with corruption and
brutality.

''It destroys the myth of the blue wall of silence,'' the Mayor said,
alluding to the testimony of Mr. Volpe's fellow officers, who described
how Mr. Volpe had boasted of the attack on Mr. Louima, who he mistakenly
believed had punched him during a melee..

With his guilty plea, Mr. Volpe was automatically dismissed from the
police force. Since the attack, he had been on modified duty, meaning he
carried neither badge nor gun, but continued to be paid.

Mr. Louima was not in court to hear Mr. Volpe's guilty plea, but his
cousin, Samuel Nicolas, spoke for the family yesterday, telling
reporters outside the courtroom in Brooklyn, ''We'd just like to thank
God for keeping Abner alive.''

Mr. Nicolas also thanked Federal prosecutors and said the Louima family
''looks forward for the rest of justice to be done.''

Today, three other officers and a sergeant also charged in the case will
return to court as their trial continues. The jury was not present for
the plea, and the remaining defendants' lawyers conferred with Judge
Nickerson yesterday to decide how best to explain to the jurors Mr.
Volpe's sudden exit without tainting their impression of the evidence
against the remaining defendants.

Prosecutors declined to discuss the plea in detail. ''We're obviously
very pleased with today's development,'' said the United States
Attorney, Zachary W. Carter.

The courtroom was so crowded for Mr. Volpe's plea that when Robert Volpe
entered, one of the few remaining seats was a few feet from Mr. Carter.
The elder Mr. Volpe squeezed in; Mr. Carter moved across the aisle to
another spot.

Twenty minutes before the appointed hour, Justin Volpe slipped into the
courtroom through a side door. Accompanied by his lawyers and several
police officers, he appeared relaxed and polished, his hair immaculately
slicked back.

He had spent the night at his parents' home in Staten Island, eating
with his mother and father and visiting with his girlfriend as a
half-dozen watchful law enforcement officials camped outside with a
couple of hecklers. In court, he chatted with his lawyers as if he were
oblivious to his surroundings.

But minutes before Judge Nickerson walked in, Mr. Volpe searched the
court for his father, who, except for an uncle, was the only Volpe
relative to attend the hearing. When their eyes met, Robert Volpe gave
him a hang-tough sort of smile, and then a clerk called out, ''United
States of America versus Justin Volpe.'' Mr. Volpe's poise began to
fade.

Flanked by his lawyers, he faced Judge Nickerson, who peered back over
half-glasses and showed none of his usual humor. Judge Nickerson asked
him if he understood what he was about to do, and Mr. Volpe said he did.
His voice was low and throaty, full of street, like a young Marlon
Brando doing a ''Godfather'' routine.

When the moment of confession arrived, Mr. Volpe reached into his jacket
and pulled out a one-page statement. He took several deep breaths, and
rocked back and forth, and read in a monotone until he got to the
phrase, ''I sodomized . . .'' Then he paused. Then he took another
breath. Then he sniffled and finished the sentence: ''. . . Mr. Louima
by placing a stick in his rectum.''

In the audience, Robert Volpe's eyes were red, and he kept pulling at
his mustache and stroking his chin and rubbing the bridge of his nose.

''That's it, that's your statement?'' Judge Nickerson asked when Mr.
Volpe stopped reading.

''Yes, Your Honor, if you have any questions,'' he replied.

Judge Nickerson had many questions. Did he ever threaten to kill Mr.
Louima's family? No. Were Mr. Louima's hands in shackles when he was
assaulted in a police car? Yes. Was Mr. Louima handcuffed when Mr. Volpe
shoved the stick up his rectum? Yes.

''Did you put it to his mouth, close to his mouth?'' the judge asked.

''I put it in front of his mouth, yes,'' Mr. Volpe replied.

The Rev. Al Sharpton, sitting with Mr. Louima's mother near the back of
the courtroom, had a one-word description for Mr. Volpe's account:
''Chilling.''

Later, speaking to reporters outside the courthouse, Mr. Sharpton
lambasted Mr. Volpe's lawyer, Mr. Kornberg, for suggesting during his
opening statement that Mr. Louima's severe internal injuries could have
been caused by consensual sex.

Mr. Kornberg had told the jurors, ''You will hear from a forensic
pathologist and you will hear from other medical doctors that the
injuries sustained by Mr. Louima are not, I repeat not, consistent with
a nonconsensual insertion of an object into his rectum,'' and added that
feces from Mr. Louima found in the station restroom ''contains the DNA
of another male.''

Mr. Kornberg never asked Mr. Louima about that during his testimony for
the prosecution. And since Mr. Volpe pleaded guilty before his lawyers
began their defense, there was no testimony to support the contention.

Given Mr. Volpe's confession, such a defense theory was tantamount to
''a second rape,'' Mr. Sharpton said at a news conference outside court.
''This vindicates Abner's character. It vindicates those of us who stood
for Abner.''

A reporter called out one last question to Mr. Sharpton: ''Is it
Giuliani time?'' he asked, making a reference to Mr. Louima's famous
falsehood about what the police officers told him during the assault.

''It's justice time,'' Mr. Sharpton shouted, a rallying cry that was
embraced by a raucous crowd of protesters who hooted and booed when Mr.
Kornberg stepped to the microphones, where he has waged a daily defense
for his client.

Reporters repeatedly asked Mr. Kornberg if he owed Mr. Louima an apology
for insinuating that his injuries came from gay sex. Not at all, Mr.
Kornberg replied, pointing out that he never directly said Mr. Louima
was gay.

Still, Mr. Kornberg, who styles himself New York's premier lawyer for
police officers in trouble, was clearly on the defensive about his
defense of Mr. Volpe.

He acknowledged that he had been caught off guard by some of the police
officers who testified against Mr. Volpe and that ''there came a point
in time when the evidence became overwhelming.'' And he distanced
himself from Mr. Volpe's about-face when he said, ''I don't make
decisions for clients to plead guilty.''

But as he said these things, he was nearly drowned out by the din of the
demonstrators. And later, a few of the protesters pounded on the car
carrying Mr. Kornberg until it pulled away from the courthouse, where
Justin Volpe waited to be taken to jail.

Advertisement

\protect\hyperlink{after-bottom}{Continue reading the main story}

\hypertarget{site-index}{%
\subsection{Site Index}\label{site-index}}

\hypertarget{site-information-navigation}{%
\subsection{Site Information
Navigation}\label{site-information-navigation}}

\begin{itemize}
\tightlist
\item
  \href{https://help.nytimes3xbfgragh.onion/hc/en-us/articles/115014792127-Copyright-notice}{©~2020~The
  New York Times Company}
\end{itemize}

\begin{itemize}
\tightlist
\item
  \href{https://www.nytco.com/}{NYTCo}
\item
  \href{https://help.nytimes3xbfgragh.onion/hc/en-us/articles/115015385887-Contact-Us}{Contact
  Us}
\item
  \href{https://www.nytco.com/careers/}{Work with us}
\item
  \href{https://nytmediakit.com/}{Advertise}
\item
  \href{http://www.tbrandstudio.com/}{T Brand Studio}
\item
  \href{https://www.nytimes3xbfgragh.onion/privacy/cookie-policy\#how-do-i-manage-trackers}{Your
  Ad Choices}
\item
  \href{https://www.nytimes3xbfgragh.onion/privacy}{Privacy}
\item
  \href{https://help.nytimes3xbfgragh.onion/hc/en-us/articles/115014893428-Terms-of-service}{Terms
  of Service}
\item
  \href{https://help.nytimes3xbfgragh.onion/hc/en-us/articles/115014893968-Terms-of-sale}{Terms
  of Sale}
\item
  \href{https://spiderbites.nytimes3xbfgragh.onion}{Site Map}
\item
  \href{https://help.nytimes3xbfgragh.onion/hc/en-us}{Help}
\item
  \href{https://www.nytimes3xbfgragh.onion/subscription?campaignId=37WXW}{Subscriptions}
\end{itemize}
