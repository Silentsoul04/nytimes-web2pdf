Sections

SEARCH

\protect\hyperlink{site-content}{Skip to
content}\protect\hyperlink{site-index}{Skip to site index}

\href{https://www.nytimes3xbfgragh.onion/section/books}{Books}

\href{https://myaccount.nytimes3xbfgragh.onion/auth/login?response_type=cookie\&client_id=vi}{}

\href{https://www.nytimes3xbfgragh.onion/section/todayspaper}{Today's
Paper}

\href{/section/books}{Books}\textbar{}BOOKS OF THE TIMES; Substance Over
Sex In Kennedy Biography

\begin{itemize}
\item
\item
\item
\item
\item
\end{itemize}

Advertisement

\protect\hyperlink{after-top}{Continue reading the main story}

Supported by

\protect\hyperlink{after-sponsor}{Continue reading the main story}

BOOKS OF THE TIMES

\hypertarget{books-of-the-times-substance-over-sex-in-kennedy-biography}{%
\section{BOOKS OF THE TIMES; Substance Over Sex In Kennedy
Biography}\label{books-of-the-times-substance-over-sex-in-kennedy-biography}}

By David J. Garrow

\begin{itemize}
\item
  May 28, 2003
\item
  \begin{itemize}
  \item
  \item
  \item
  \item
  \item
  \end{itemize}
\end{itemize}

AN UNFINISHED LIFE

John F. Kennedy, 1917-1963

By Robert Dallek

Illustrated. 838 pages. Little, Brown \& Company. \$30.

The New York Times reported last November that Robert Dallek's coming
biography of President John F. Kennedy would disclose, based on newly
opened medical files, that the president had suffered far more dire
health problems than American voters and previous biographers had ever
known.

Mr. Dallek's success in winning access to those long-sealed documents
was not singular; he similarly persuaded a former Times reporter and
onetime Kennedy White House press aide, Barbara Gamarekian, to release
closed portions of a 1964 oral history in which she described Kennedy's
18-month sexual affair with Mimi, a young White House intern recently
identified as Marion Beardsley Fahnestock, now a 60-year-old Manhattan
church administrator.

But ''An Unfinished Life'' is no salacious exposé. Mr. Dallek, a
professor of history at Boston University and the author of several
well-respected scholarly books, including a two-volume biography of
Lyndon B. Johnson, instead offers an impressively judicious and balanced
account of Kennedy's life and presidency.

Mr. Dallek views Kennedy's medical records as more a testament to the
young president's stoic fortitude than evidence of political deception
and historical cover-up. ''Kennedy courageously surmounted his physical
suffering,'' and medical problems ''did not significantly undermine his
performance as president,'' Mr. Dallek writes. And Mimi, like Marilyn
Monroe and other rumored Kennedy paramours, receives little more than a
passing mention. Mr. Dallek confronts Kennedy's ''reckless womanizing''
directly, yet succinctly, and concludes that Kennedy's myriad dalliances
''were no impediment to his being an effective president.''

''An Unfinished Life'' may be a third-generation synthesis in Kennedy
biographies, following the initial romanticized celebrations of
''Camelot'' and then a shelf full of hostile tomes in which one
presidential girlfriend, Judith Campbell Exner, and the Chicago gangster
Sam Giancana played larger roles than Cabinet members and Congressional
leaders did. Mr. Dallek addresses all the negatives, but his consistent
assertion that Kennedy's policy record as president decisively trumps
his personal peccadilloes is cumulatively quite persuasive.

Mr. Dallek's first three books all concerned foreign policy, and ''An
Unfinished Life'' accords Kennedy's overseas challenges a fuller and
more engaging treatment than most domestic policy issues receive.

Three subjects dominate this account of the Kennedy presidency: Cuba,
Vietnam and Kennedy's tension-filled dealings with the Soviet prime
minister, Nikita S. Khrushchev. Mr. Dallek readily acknowledges how both
the 1961 Bay of Pigs debacle and the administration's continuing
obsession with ousting Fidel Castro showed ''Kennedy at his worst.'' He
also admits that ''Kennedy was as much in the grip of conventional cold
war thinking as most other Americans,'' yet his portrait of Kennedy's
behavior concerning Vietnam repeatedly emphasizes how Kennedy ''had no
intention of being drawn into an expansion of American ground forces in
Vietnam and the possibility of an open-ended war.''

Mr. Dallek concedes Kennedy's hesitancy and indecisiveness about Vietnam
but implies that the uncertainty was largely the result of exceptionally
divergent expert advice. ''The two of you did visit the same country,
didn't you?''' he quotes Kennedy asking two advisers just back from a
trip to Saigon. Yet Mr. Dallek insists that by 1963 Kennedy ''was more
skeptical than ever about putting in ground forces'' and would never
have expanded the war as President Johnson did in 1965.

Kennedy's handling of nuclear tensions with the Soviet Union was ''the
greatest overall achievement of his presidency,'' Mr. Dallek argues.
Kennedy's caution and restraint during the October 1962 Cuban missile
crisis, particularly in the face of aggressively warmongering advice
from most members of the Joint Chiefs of Staff, was ''a model of wise
statesmanship in a dire situation,'' he writes.

But Mr. Dallek's emphasis on Kennedy's foreign relations efforts runs
the risk of giving domestic policy issues short shrift. He cites
Kennedy's ''limited interest in domestic affairs'' and notes how a 1962
Gallup Poll showed that 63 percent of respondents thought issues of war
and peace were ''the most important problem facing the country,''
compared with just 6 percent who named civil rights.

Yet he heaps praise on a June 10, 1963, Kennedy speech on world peace,
which ''received barely a mention in the press,'' while decidedly
playing down an arguably far more important Kennedy address on civil
rights to a national television audience the next evening. Mr. Dallek
complains that Kennedy ''did not fully understand . . . the importance
of taking a moral stand on civil rights,'' but Kennedy's June 11
telecast voiced an explicit declaration that racial equality indeed was
''a moral issue'' that ''is as old as the Scriptures and is as clear as
the American Constitution.''

''An Unfinished Life'' is nonetheless an excellent biography, one that
will convince any fair-minded reader that the Kennedy presidency should
be remembered not for medical deceit and sexual high jinks but for
resolute caution during the cold war's most dangerous hours.

Advertisement

\protect\hyperlink{after-bottom}{Continue reading the main story}

\hypertarget{site-index}{%
\subsection{Site Index}\label{site-index}}

\hypertarget{site-information-navigation}{%
\subsection{Site Information
Navigation}\label{site-information-navigation}}

\begin{itemize}
\tightlist
\item
  \href{https://help.nytimes3xbfgragh.onion/hc/en-us/articles/115014792127-Copyright-notice}{©~2020~The
  New York Times Company}
\end{itemize}

\begin{itemize}
\tightlist
\item
  \href{https://www.nytco.com/}{NYTCo}
\item
  \href{https://help.nytimes3xbfgragh.onion/hc/en-us/articles/115015385887-Contact-Us}{Contact
  Us}
\item
  \href{https://www.nytco.com/careers/}{Work with us}
\item
  \href{https://nytmediakit.com/}{Advertise}
\item
  \href{http://www.tbrandstudio.com/}{T Brand Studio}
\item
  \href{https://www.nytimes3xbfgragh.onion/privacy/cookie-policy\#how-do-i-manage-trackers}{Your
  Ad Choices}
\item
  \href{https://www.nytimes3xbfgragh.onion/privacy}{Privacy}
\item
  \href{https://help.nytimes3xbfgragh.onion/hc/en-us/articles/115014893428-Terms-of-service}{Terms
  of Service}
\item
  \href{https://help.nytimes3xbfgragh.onion/hc/en-us/articles/115014893968-Terms-of-sale}{Terms
  of Sale}
\item
  \href{https://spiderbites.nytimes3xbfgragh.onion}{Site Map}
\item
  \href{https://help.nytimes3xbfgragh.onion/hc/en-us}{Help}
\item
  \href{https://www.nytimes3xbfgragh.onion/subscription?campaignId=37WXW}{Subscriptions}
\end{itemize}
