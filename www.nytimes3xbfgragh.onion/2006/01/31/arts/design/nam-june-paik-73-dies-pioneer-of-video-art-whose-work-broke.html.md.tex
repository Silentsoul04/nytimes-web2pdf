Sections

SEARCH

\protect\hyperlink{site-content}{Skip to
content}\protect\hyperlink{site-index}{Skip to site index}

\href{https://www.nytimes3xbfgragh.onion/section/obituaries}{Obituaries}

\href{https://myaccount.nytimes3xbfgragh.onion/auth/login?response_type=cookie\&client_id=vi}{}

\href{https://www.nytimes3xbfgragh.onion/section/todayspaper}{Today's
Paper}

\href{/section/obituaries}{Obituaries}\textbar{}Nam June Paik, 73, Dies;
Pioneer of Video Art Whose Work Broke Cultural Barriers

\begin{itemize}
\item
\item
\item
\item
\item
\end{itemize}

Advertisement

\protect\hyperlink{after-top}{Continue reading the main story}

Supported by

\protect\hyperlink{after-sponsor}{Continue reading the main story}

\hypertarget{nam-june-paik-73-dies-pioneer-of-video-art-whose-work-broke-cultural-barriers}{%
\section{Nam June Paik, 73, Dies; Pioneer of Video Art Whose Work Broke
Cultural
Barriers}\label{nam-june-paik-73-dies-pioneer-of-video-art-whose-work-broke-cultural-barriers}}

By \href{https://www.nytimes3xbfgragh.onion/by/roberta-smith}{Roberta
Smith}

\begin{itemize}
\item
  Jan. 31, 2006
\item
  \begin{itemize}
  \item
  \item
  \item
  \item
  \item
  \end{itemize}
\end{itemize}

\textbf{Correction Appended}

Nam June Paik, an avant-garde composer, performer and artist widely
considered the inventor of video art, died Sunday at his winter home in
Miami Beach. He was 73 and also lived in Manhattan.

Mr. Paik suffered a stroke in 1996 and had been in declining health for
some time, said his nephew, Ken Paik Hakuta, who manages his uncle's
studio in New York.

Mr. Paik's career spanned half a century, three continents and several
art mediums, ranging through music, theater and found-object art. He
once built his own robot. But his chief means of expression was
television, which he approached with a winning combination of visionary
wildness, technological savvy and high entertainment values. His work
could be kitschy, visually dazzling and profound, sometimes all at once,
and was often irresistibly funny and high-spirited.

At his best, Mr. Paik exaggerated and subverted accepted notions about
both the culture and the technology of television while immersing
viewers in its visual beauty and exposing something deeply irrational at
its center. He presciently coined the term "electronic superhighway" in
1974, grasping the essence of global communications and seeing the
possibilities of technologies that were barely born. He usually did this
while managing to be both palatable and subversive. In recent years, Mr.
Paik's enormous American flags, made from dozens of sleek monitors whose
synchronized patterns mixed everything from pinups to apple pie at high,
almost subliminal velocity, could be found in museums and corporate
lobbies.

Mr. Paik was affiliated in the 1960's with the anti-art movement Fluxus,
and also deserves to be seen as an aesthetic innovator on a par with the
choreographer Merce Cunningham and the composer John Cage. Yet in many
ways he was simply the most Pop of the Pop artists. His work borrowed
directly from the culture at large, reworked its most pervasive medium
and gave back something that was both familiar and otherworldly.

He was a shy yet fearless man who combined manic productivity and
incessant tinkering with Zen-like equanimity. A lifelong Buddhist, Mr.
Paik never smoked or drank and also never drove a car. He always seemed
amused by himself and his surroundings, which could be overwhelming: a
writer once compared his New York studio to a television repair shop
three months behind schedule.

Mr. Paik is survived by his wife, the video artist Shigeko Kubota.

Mr. Paik got to television by way of avant-garde music. He was born in
1932 in Seoul, Korea, into a wealthy manufacturing family. Growing up,
he studied classical piano and musical composition and was drawn to
20th-century music; he once said it took him three years to find an
Arnold Schoenberg record in Korea. In 1949, with the Korean War
threatening, the family fled to Hong Kong, and then settled in Tokyo.
Mr. Paik attended the University of Tokyo, earning a degree in
aesthetics and the history of music in 1956 with a thesis on
Schoenberg's work.

He then studied music at the University of Munich and the Academy of
Music in Freiburg and threw himself into the avant-garde music scene
swirling around Cologne. He also met John Cage, whose emphasis on chance
and randomness dovetailed with Mr. Paik's sensibility.

Over the next few years, Mr. Paik arrived at an early version of
performance art, combining cryptic musical elements -\/- usually spliced
audiotapes of music, screams, radio news and sound effects -\/- with
startling events. In an unusually Oedipal act during a 1960 performance
in Cologne, Mr. Paik jumped from the stage and cut off Cage's necktie,
an event that prompted George Maciunas, a founder of Fluxus, to invite
Mr. Paik to join the movement. At the 1962 Fluxus International Festival
for Very New Music in Wiesbaden, Germany, Mr. Paik performed "Zen for
Head," which involved dipping his head, hair and hands in a mixture of
ink and tomato juice and dragging them over a scroll-like sheet of paper
to create a dark, jagged streak.

In 1963, seeking a visual equivalent for electronic music and inspired
by Cage's performances on prepared pianos, Mr. Paik bought 13 used
television sets in Cologne and reworked them until their screens jumped
with strong optical patterns. In 1963, he exhibited the first art known
to involve television sets at the Galerie Parnass in Wuppertal, Germany.

In 1965 he made his New York debut at the New School for Social
Research: Charlotte Moorman, a cellist who became his longtime
collaborator, played his "Cello Sonata No. 1 for Adults Only,"
performing bared to the waist. A similar work performed in 1967 at the
Filmmakers Cinematheque in Manhattan resulted in the brief arrest of Ms.
Moorman and Mr. Paik. Mr. Paik retaliated with his iconic "TV Bra for
Living Sculpture," two tiny television screens that covered Ms.
Moorman's breasts.

Mr. Paik bought one of the first portable video cameras on the market,
in 1965, and the same year he exhibited the first installation involving
a video recorder, at the Galeria Bonino in New York. Although he
continued to perform, his interests shifted increasingly to the
sculptural, technological and environmental possibilities of video.

In 1969, Mr. Paik started showing pieces using multiple monitors. He
created bulky wood robotlike figures using old monitors and retrofitted
consoles, and constructed archways, spirals and towers, including one
60-feet tall that used 1,003 monitors. By the 1980's he was working with
lasers, mixing colors and forms in space, without the silvery
cathode-ray screen.

For his 2000 retrospective at the Guggenheim Museum, Mr. Paik arranged
monitors faceup on the rotunda's floor, creating a pondlike effect of
light and images. Overhead, one of the artist's most opulent laser
pieces cascaded from the dome in lightninglike zigzags -\/- an apt
metaphor for a career that never stopped surging forward.

Correction: February 1, 2006, Wednesday An obituary yesterday about Nam
June Paik, widely considered the inventor of video art, omitted a
survivor. He is his brother, Ken Paik, of Kamakura, Japan.

Advertisement

\protect\hyperlink{after-bottom}{Continue reading the main story}

\hypertarget{site-index}{%
\subsection{Site Index}\label{site-index}}

\hypertarget{site-information-navigation}{%
\subsection{Site Information
Navigation}\label{site-information-navigation}}

\begin{itemize}
\tightlist
\item
  \href{https://help.nytimes3xbfgragh.onion/hc/en-us/articles/115014792127-Copyright-notice}{©~2020~The
  New York Times Company}
\end{itemize}

\begin{itemize}
\tightlist
\item
  \href{https://www.nytco.com/}{NYTCo}
\item
  \href{https://help.nytimes3xbfgragh.onion/hc/en-us/articles/115015385887-Contact-Us}{Contact
  Us}
\item
  \href{https://www.nytco.com/careers/}{Work with us}
\item
  \href{https://nytmediakit.com/}{Advertise}
\item
  \href{http://www.tbrandstudio.com/}{T Brand Studio}
\item
  \href{https://www.nytimes3xbfgragh.onion/privacy/cookie-policy\#how-do-i-manage-trackers}{Your
  Ad Choices}
\item
  \href{https://www.nytimes3xbfgragh.onion/privacy}{Privacy}
\item
  \href{https://help.nytimes3xbfgragh.onion/hc/en-us/articles/115014893428-Terms-of-service}{Terms
  of Service}
\item
  \href{https://help.nytimes3xbfgragh.onion/hc/en-us/articles/115014893968-Terms-of-sale}{Terms
  of Sale}
\item
  \href{https://spiderbites.nytimes3xbfgragh.onion}{Site Map}
\item
  \href{https://help.nytimes3xbfgragh.onion/hc/en-us}{Help}
\item
  \href{https://www.nytimes3xbfgragh.onion/subscription?campaignId=37WXW}{Subscriptions}
\end{itemize}
