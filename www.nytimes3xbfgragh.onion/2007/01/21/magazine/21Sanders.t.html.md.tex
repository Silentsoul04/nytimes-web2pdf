Sections

SEARCH

\protect\hyperlink{site-content}{Skip to
content}\protect\hyperlink{site-index}{Skip to site index}

\href{https://myaccount.nytimes3xbfgragh.onion/auth/login?response_type=cookie\&client_id=vi}{}

\href{https://www.nytimes3xbfgragh.onion/section/todayspaper}{Today's
Paper}

The Socialist Senator

\begin{itemize}
\item
\item
\item
\item
\item
\end{itemize}

Advertisement

\protect\hyperlink{after-top}{Continue reading the main story}

Supported by

\protect\hyperlink{after-sponsor}{Continue reading the main story}

\hypertarget{the-socialist-senator}{%
\section{The Socialist Senator}\label{the-socialist-senator}}

By \href{https://www.nytimes3xbfgragh.onion/by/mark-leibovich}{Mark
Leibovich}

\begin{itemize}
\item
  Jan. 21, 2007
\item
  \begin{itemize}
  \item
  \item
  \item
  \item
  \item
  \end{itemize}
\end{itemize}

When Bernie Sanders visits a high-school class, as he does regularly,
students don't hear a speech, a focus-grouped polemic, a campaign pitch
or, heaven forbid, practiced one-liners. Nor, in all likelihood, do they
hear Sanders tell stories about his family, childhood or some hardship
he has endured. He makes no great effort to ``connect'' emotionally in
the manner that politicians strive for these days, and he probably
doesn't ``feel your pain'' either, or at least make a point of saying
so. It's not that Sanders is against connecting, or feeling your pain,
but the process seems needlessly passive and unproductive, and he
prefers a more dynamic level of engagement.

``I urge you all to argue with your teachers, argue with your parents,''
Sanders told a group of about 60 students at South Burlington High
School --- generally liberal, affluent and collegebound --- one
afternoon in mid-December.

The newly elected senator whipped his head forward with a force that
shifted his free-for-all frizz of white hair over his forehead.
(Journalistic convention in Vermont mandates that every Sanders story
remark on his unruly hair as early on as possible. It also stipulates
that every piece of his clothing be described as ``rumpled.'')

``C'mon, I'm not seeing enough hands in here,'' he said.

A senior named Marissa Meredyth raised hers, and Sanders flicked his
index finger at her as if he were shooting a rubber band. She bemoaned
recent cuts to college financial-aid programs.

Sanders bemoans these, too, but he'd rather provoke.

``How we going to pay for this financial aid?'' Sanders asked. ``Who in
here wants us to raise taxes on your parents to pay for this?''

Not many, based on the show of hands.

``O.K., so much for financial aid,'' Sanders said, shrugging.

Next topic: ``How many of you think it was a good idea to give the
president the authority to go to war in Iraq?''

No hands.

``C'mon, anyone?''

He paused, paced, hungry for dissent, a morsel before lunch. Sanders
says he thinks Iraq was a terrible idea, too, but he seemed to crave a
jolt to the anesthetizing hum of consensus in the room.

``Iraq is a huge and very complicated issue,'' Sanders said, finally.
(``Huge'' is Sanders favorite word, which he pronounces ``yooge,''
befitting a thick Brooklyn accent unsmoothed-over by 38 years in
Vermont.) He mentioned that Vermont has had more casualties in Iraq per
capita than any other state in the union, including one from South
Burlington High School.

``O.K., last call for an Iraq supporter,'' he said. Going once, going
twice.

By this point, Sanders's cheeks had turned a shade of dark pink with a
strange hint of orange. It's a notable Sanders trait; his face seems to
change color with the tenor of a conversation, like a mood ring. His
complexion goes orangey-pink when he's impatient (often when someone
else is speaking), purpley-pink when he's making a point or a softer
shade of pink when at rest, ``rest'' being a relative term.

Next question from Sanders: ``Should people in this country who want to
go to college be able to go, regardless of income?''

Wall-to-wall hands, with the exception of one belonging to Andy Gower, a
senior in a backward baseball cap who recently moved up from North
Carolina. Relatively conservative, Andy is a conspicuous outlier in the
class. Bernie knows how he feels, having spent eight terms as the lone
Socialist in Congress, and the first to serve in the House since the
1920s.

``Why do you think that?'' Sanders asked Andy.

He replied with a question of his own: ``Why should people who can
afford to go to college pay for people who can't?'' He was sheepish at
first but gained momentum. ``Why should people who are successful in
this society be burdened by people who aren't? It's just a fact of life.
Some people will succeed, and some people won't. And it's just the way
it's going to be and has always been.''

A few classmates smirked, shook their heads. But Sanders was suddenly
buoyant. He stomped forward, clapped twice --- provocation achieved.

Hands were shooting up everywhere, and Sanders contorted his mouth into
a goofy grin.

``At the end of the day, democracy is a tough process,'' Sanders said
finally, arms restored to their flailing default positions.

``The discussion we've had in here is at a higher level than what we
often have on the floor of the United States Congress,'' Sanders gushed,
for as much as he ever gushes, which is not much.

And given some of the things Sanders has said about the United States
Congress, maybe this wasn't such a gush after all.

Sanders has always been an easier fit in Vermont than in Washington.
Being a Socialist in the seat of two-party orthodoxy will do that. While
he has generally championed liberal Democratic positions over the years
--- and the Democratic Senatorial Campaign Committee endorsed his Senate
campaign --- Sanders has strenuously resisted calling himself a
Democrat. And he has clung to a mantle --- socialism --- that brings
considerable stigma, in large part for its association with
authoritarian communist regimes (which Sanders is quick to disavow).

But he does little to airbrush the red ``S'' from his political profile.
On the wall of his Congressional office hangs a portrait of Eugene V.
Debs, the Socialist Party presidential candidate of the early 20th
century. A poster in a conference room marks Burlington's sister-city
relationship with Puerto Cabeza, Nicaragua --- one of a few such
alliances he forged with cities in Marxist states during his 10-year
stint as mayor of Vermont's biggest city in the 1980s.

Socialism brings Sanders instant novelty in Washington and, in many
circles, instant dismissal as a freak. But Sanders's outcast status in
Washington probably owes as much to his jackhammer style as to any
stubborn ideology. It is a town filled with student body president types
--- and Sanders, for his part, finished a distant third when he ran to
be president of his class at James Madison High School in Brooklyn.

Few would describe Sanders's personality as ``winning'' in the classic
politician's sense. He appears to burn a disproportionate number of
calories smiling and making eye contact. ``Bernie is not going to win a
lot of `whom would you rather live on a desert island with' contests,''
says Garrison Nelson, a professor of political science at the University
of Vermont. No matter. Sanders's agitating style in Washington also
constitutes a basic facet of anticharm, antipolitician appeal at home.

``I'm not afraid of being called a troublemaker,'' Sanders says,
something he's been called many times, in many different ways, many of
them unprintable. ``But you have to be smart. And being smart means not
creating needless enemies for yourself.''

In this regard, Sanders has not always been smart, especially when he
was first elected to the House in 1990. He called Congress ``impotent''
and dismissed the two major parties as indistinguishable tools of the
wealthy. He said it wouldn't bother him if 80 percent of his colleagues
lost re-election --- not the best way to win friends in a new workplace.

``Bernie alienates his natural allies,'' Representative Barney Frank,
the Massachusetts Democrat, said at the time. ``His holier-than-thou
attitude --- saying in a very loud voice he is smarter than everyone
else and purer than everyone else --- really undercuts his
effectiveness.'' The late Joe Moakley, another Massachusetts Democrat,
waxed almost poetic in his derision for Sanders. ``He is out there
wailing on his own,'' Moakley said. ``He screams and hollers, but he is
all alone.''

Frank says he came to like and work well with Sanders, with whom he
served on the House Financial Services Committee. His early objections
were over Sanders's railing against both parties as if they were the
same. ``I think when he first got here, Bernie underestimated the degree
that Republicans had moved to the right,'' Frank told me. ``I get sick
of people saying `a curse on both your houses.' When you point out to
them that you agree with them on most things, they'll say, `Yeah, well,
I hold my friends up to a higher standard.' Well, O.K., but remember
that we're your friends.''

Among his House colleagues, ``Bernie's not a bad guy,'' is something I
heard a lot of. ``You appreciate Bernie the more you see him in
action,'' says Senator Chuck Schumer, the head of the Democratic
Senatorial Campaign Committee, who served with him for several years in
the House. A fellow Brooklynite who is nine years younger, Schumer
attended the same elementary school as Sanders (P.S. 197) and the same
high school (James Madison, which also graduated a third United States
senator, Norm Coleman, Republican of Minnesota). ``Bernie does tend to
grow on people, whether it's in the House or in Vermont,'' Schumer says.

But he has clearly grown bigger in Vermont, and more seamlessly. ``His
bumper stickers just say, `Bernie,' '' says Senator Patrick Leahy,
Vermont's senior Senator and a Democrat. ``You have to reach a certain
exulted status in politics to be referred to only by your first name.''

Sanders is particularly beloved in Burlington, which elected the
recovering fringe candidate as its mayor despite the Reagan landslide of
1980 --- thus christening the so-called ``People's Republic of
Burlington.'' Some supporters called themselves ``Sanderistas.''

His election to the Senate in November came at the expense of a
too-perfect Bernie foil --- Richard Tarrant, a well-barbered,
Bentley-driving Republican businessman who spent \$7 million of his own
money so he could lose by 33 percentage points.

``Congratulations, Bernie,'' a fan yells to Sanders outside his district
office in Burlington. Sanders was out for a quick bagel on a balmy
December morning, temperatures in the 60s --- another day of Al Gore
weather in the once-frozen north. He walked head down but kept getting
stopped. ``Now you gotta run for president, please,'' the congratulator
added, something Sanders gets a lot of too.

It is a reception that any natural, eager-to-please politician would
relish --- and accordingly, Sanders dispatches these glad-handing chores
with the visible joy of someone cleaning a litter box, coughing out his
obligatory thank yous and continuing on his way.

Sanders's popularity in Vermont brings up the obvious questions: to what
degree is he a quaint totem of the state, like the hermit thrush (the
state bird), and could a Socialist be elected to the Senate anywhere
else?

In recent years, Vermont has joined --- perhaps surpassed --- states
like Massachusetts and New York in the top tier of liberal outposts.
Several distinctions nurture the state's credentials: It was the first
place to legalize civil unions for same-sex partners; it is the home of
Phish, the countercultural rock-folk band and contemporary analog to the
Grateful Dead and of Ben and Jerry's ice cream (and its peacenik-themed
flavors); and it is host to cultural quirks and ordinances like not
allowing billboards, being the last state to get a Wal-Mart.

The state has also incubated several politicians who have achieved
national boogie-man status among Republicans. They include Leahy, the
Grateful Dead fan and chairman of the Senate Judiciary Committee; former
Senator James Jeffords, the liberal Republican who became an Independent
in 2001, giving Democrats a temporary majority; and Howard Dean, the
former governor whose presidential campaign boom (and perhaps fizzle)
was tied heavily to his association with Vermont's progressive politics.

Sanders fits snugly into this maverick's pantheon. But Leahy says his
fellow senator appeals to an antiestablishment strain in Vermont that is
not necessarily liberal. Leahy notes that he himself is the only
Democrat the state's voters have ever elected to the Senate. Before
1992, only one Democratic presidential candidate carried Vermont ---
Lyndon Johnson in 1964.

``A lot of the lower-income parts of our state are Republican,'' Leahy
says, adding that many of them are populated by rural libertarians who
are greatly suspicious of government intrusion into individual rights.
``I saw Bernie signs all over those parts of the state.''

Sanders opposes some federal gun-control laws, which has helped him in a
state where ``you grow up believing it is legal to shoot deer on the
statehouse lawn in Montpelier,'' says Luke Albee, a South Burlington
native who was Leahy's House chief of staff.

But again: Could Sanders be elected to the Senate anywhere else?

No, not as a Socialist, Schumer says. ``Even in New York State it would
be hard.''

Massachusetts? ``Maybe this year he could,'' Frank says, meaning 2006.
``But if he were running in any other state, he probably would have to
comb his hair.''

Leahy says that just any Socialist probably couldn't get elected in
Vermont, either. But Sanders has made himself known in a state small
enough --- physically and in terms of population --- for someone,
particularly a tireless someone, to insinuate himself into neighborly
dialogues and build a following that skirts ideological pigeonholes.
Indeed, there are no shortages of war veterans or struggling farmers in
Vermont who would seemingly have no use for a humorless aging hippie
peacenik Socialist from Brooklyn, except that Sanders has dealt with
many of them personally, and it's a good bet his office has helped them
procure some government benefit.

``People have gotten to know him as Bernie,'' Leahy says. ``Not as the
Socialist.''

Sanders calls himself as a ``democratic Socialist.'' When I asked him
what this meant, as a practical matter, in capitalist America circa
2007, he did what he often does: he donned his rhetorical Viking's
helmet and waxed lovingly about the Socialist governments of
Scandinavia. He mentioned that Scandinavian countries have nearly wiped
out poverty in children --- as opposed to the United States, where 18 to
20 percent of kids live in poverty. The Finnish government provides free
day care to all children; Norwegian workers get 42 weeks of maternity
leave at full pay.

But would Americans ever accept the kinds of taxes that finance the
Scandinavian welfare state? And would Sanders himself trade in the
United States government for the Finnish one? He is curiously,
frustratingly non-responsive to questions like this. ``I think there is
a great deal we can learn from Scandinavia,'' he said after a long
pause. And then he returns to railing about economic justice and the
rising gap between rich and poor, things he speaks of with a sense of
outrage that always seems freshly summoned.

Sanders crinkles his face whenever a conversation veers too long from
this kind of ``important stuff'' and into the ``silly stuff,'' like
clothes and style. ``I do not like personality profiles,'' Sanders told
me during our first conversation. He trumpets a familiar rant against
the media, its emphasis on gaffes, polls and trivial details.

``If I walked up on a stage and fell down, that would be the top
story,'' Sanders says. ``You wouldn't hear anything about the growing
gap between rich and poor.''

When I first met Sanders in person on Church Street, there were big
streaks of dried mud on his shoes and dried blood on his neck from what
looked to be a shaving mishap. His hair flew every which way in a gust
of wind. At six feet tall, he is wiry, but he walks with shoulders
hunched and elbows out, like a big, skulking bird. From a distance, he
looked as if he could be homeless.

Image

Bernie Sanders.Credit...Larry Fink for The New York Times

Closer in, the overwhelming impression made by Sanders is that of an
acute worrier. He evinces the wearied default manner of a longtime
insomniac, eyes weather-beaten with big lines and a perpetual slight
cringe. His brow appears close to collapse beneath the weight of an
invisible sandbag.

Richard Sugarman, a professor of religion at the University of Vermont
and a longtime friend, recalls that during Sanders's days as mayor,
constituents would sometimes call him at his listed home phone number in
the middle of the night. ``Someone would call at 3 a.m. and say, `Hey
Bernie, someone just threw a brick through my window, what should I do?'
He was as hands on as anyone. ... Does he have an off-mode? Not
really.''

Luke Albee, Leahy's former chief of staff, says: ``He has no hobbies. He
works. He doesn't take time off. Bernie doesn't even eat lunch. The idea
of building a fire and reading a book and going on vacation, that's not
something he does.''

As much as anything, this distills why Sanders has been an awkward fit
in the chummy realm of Capitol Hill. He is no pleaser or jokester by
anyone's prototype. I don't recall Sanders laughing more than two or
three times in the 48 hours I spent with him in Vermont. His one
memorably funny aside came when I asked if his Congressional office had
a dress code.

``Yes,'' he said. ``You can't come in if you're totally nude,'' he said.
He instituted the rule, he said, when his outreach director, Phil
Fiermonte, who is now sitting next to him, came to work naked.

``Totally nude,'' Sanders said. ``On three occasions.''

He was kidding, presumably.

Riding in the passenger seat of Fiermonte's car, Sanders was shouting
into a brick-size cellphone, the likes of which were all the rage in the
1990s. He was talking to a staff person who was about to meet with
someone from the office of Senator Edward Kennedy, chairman of Health,
Education, Labor and Pensions, one of five committees that Sanders will
sit on. Sanders voice filled the car.

``Dental care is yooge,'' Sanders boomed into the phone. This has been a
leitmotif of my visit --- Sanders's crusade to improve dental health
among Vermont's rural poor. He views this as an employment and economic
issue. ``How many employers are going to hire someone who doesn't have
teeth?'' he asks. ``You go around this state, and you will find a lot of
people with no teeth. It is their badge of poverty.''

Improving dental care for the poor is a classic Sanders issue: unsexy
and given to practical solutions and his obsessive attention. Sanders
sees bad dental care among the poor as a ``pothole issue'' in Vermont,
meaning it is pervasive and something that government should be active
in fixing (like potholes). Teeth are tangible, especially when they
hurt.

Sanders's car pulled into the parking lot of H.O. Wheeler Elementary
School in North Burlington, where he was visiting a drop-by dental
clinic. The notion of ``school-based dental care'' excites Sanders
immensely, and his gait speeds as he enters the school, past the main
office, a classroom and several school officials he has come to know
over multiple visits.

``If you're a kid, and you're having dental pain, you're not going to be
learning a lot,'' said Joseph Arioli, of Burlington's Community Health
Center and one of a half-dozen program administrators --- including a
dentist in scrubs --- convened around a dentist chair.

The clinic provides free access to dental care for kids at high risk of
neglecting their teeth. Students are typically seen during the school
day, which means they miss minimal class time and their parents don't
have to leave work to take them. Betsy Liley, a grant writer for the
city, says that many households in Vermont own just one toothbrush.

``Lemme guess, a lot of the dietary habits you see here are not great,''
Sanders said. Nods all around. He said he'd do his best to secure more
financing and vowed to return. And he told Liley that he might bring her
to Washington to testify before a Senate committee.

Walking out, Sanders didn't bother with goodbye --- just as he didn't
with hello --- only a thank you and a ``what you're doing here is
yooge'' over his shoulder.

``Great program,'' Sanders said in the car. He likes to check in
whenever possible. That's essentially what I did with Sanders in
Vermont: check in, with programs that he's been involved with or wants
to learn more about. He likes to hit lots of meetings, quick,
businesslike transactions.

Only once in six discussions I sat in on did Sanders indulge in a
personal anecdote. He was in his office talking to Sharon Moffat,
Vermont's acting commissioner of health, and the topic turned to dental
care.

``I have a personal story to tell you,'' Sanders said, and my ears
perked up as I fantasized of learning the ``Rosebud'' episode that might
explain Bernie's interest in teeth.

``I was in the House cloakroom about five years ago,'' Sanders said.
``And I was thirsty. I took a drink of grape juice. Blawww.''

He scrunched up his face.

``It was awful, awful. Then I looked at the label. The amount of junk
they put in there is unbelievable.''

Moffat nodded.

``Anyway, I no longer drink that stuff,'' Sanders said.

Sanders's parents were Jewish immigrants from Poland. His father, Eli, a
struggling paint salesman who saw his family wiped out in the Holocaust,
worried constantly about supporting his wife and two sons. His mother,
Dorothy, dreamed of living in a ``private home,'' but they never made it
beyond their three-and-a-half-room apartment on East 26th and Kings
Highway. She died at age 46, when Bernie was 19. ``Sensitivity to class
was imbedded in me then quite deeply,'' Sanders told me.

Sanders spent a year at Brooklyn College before transferring to the
University of Chicago, where he studied psychology and helped lead
protests against racially segregated housing on campus. He spent time on
a kibbutz in Israel after graduation and then moved to Vermont with his
first wife. ``I had always been captivated by rural life,'' he says. As
a child, Sanders attended Boy Scout camp upstate and used to cry on the
bus as it returned him to New York at the end of the summer.

In Vermont, Sanders worked many jobs for meager sums --- as a freelance
writer, filmmaker, carpenter and researcher, among other things.
(Sanders has one son, Levi, and three stepchildren from his marriage to
his second wife, Jane O'Meara Driscoll, the president of a small college
in Burlington whom he met at a party on the night of his first mayoral
victory.)

Politics came to dominate Sanders's life. He was an early member of
Vermont's Liberty Union party, an offshoot of the antiwar movement in
Vermont. He ran as the party's nominee for the Senate in a special
election in 1971 and finished with 2 percent of the vote. The following
year, he ran for governor and received 1 percent. He would run two more
times for statewide office that decade as a third-party candidate and
never come close.

That changed when he ran for mayor of Burlington in 1980, at Sugarman's
urging. Sugarman studied the race and believed Sanders could win, if few
others did. Sanders knocked on doors all over the city, campaigned day
and night and beat a six-term Democratic incumbent by 12 votes.

``People generally assumed this was a fluke and that he would be gone in
two years,'' said Peter Clavelle, a friend who succeeded Sanders as
mayor.

Sanders spoke out against poverty in the third world and made good-will
visits to the Soviet Union and Cuba, among other places that U.S. mayors
generally didn't travel to during that time. But a funny thing happened
on the way to what many had dismissed as a short-running circus. Sanders
undertook ambitious downtown revitalization projects and courted evil
capitalist entities known as ``businesses.'' He balanced budgets. His
administration sued the local cable franchise and won reduced rates for
customers. He drew a minor-league baseball team to town, the Vermont
Reds (named for the Cincinnatis, not the Commies).

Sanders's appeal in Vermont's biggest city blended the ``think
globally'' sensibility of a liberal college town with the ``act
locally'' practicality of a hands-on mayor. He offered sister-city
relations with the Sandinistas and efficient snowplowing for the
People's Republic of Burlington. Before Sanders's mayoral victory, Leahy
says, it was easy not to take him seriously. ``Then he got over that
barrier, and got elected. He fixed the streets, filled the potholes,
worked with the business community. He did what serious leaders do.'' He
was re-elected three times.

In a sense, Sanders's stint as mayor become a template for his
subsequent successes --- and limitations --- as a national officeholder.
In the House, he gained great publicity and favor as an audacious critic
with a geopolitical purview, but ultimately left his biggest mark with
small-bore diligence to the local realpolitik.

I was reminded of this when I asked Sanders in early January what his
immediate legislative goals would be in the Senate. He listed these
broad-brush priorities: 1) ending the Iraq war; 2) reversing the ``rapid
decline of the middle class'' (a corollary to ``addressing the gap
between rich and poor''); 3) reordering priorities in the federal
budget; and 4) enacting environmental laws to thwart global warming.
When I asked how he would translate any of his priorities into concrete
legislation, he nodded sheepishly and said, ``I'm in the process of
trying to figure that out now.'' It is an unsatisfying response somewhat
reminiscent of Sanders's all-purpose invocations of Scandinavia whenever
he's pressed on how his socialist philosophy can be applied to the
two-party system he exists in.

As a general rule, Sanders is much more convincing at proffering outrage
than solutions. He can do this in Vermont, in part, because he is an
entrenched political brand --- ``Bernie'' --- and voters will forgive a
little blowhardedness (if not demagoguery) from someone they basically
agree with and who has grown utterly familiar to their landscape, like
cows. Sanders can also pull this off because, as he did in the mayor's
office, he has buttressed his bomb-throwing with rock-solid attention to
the pothole matters of dental clinics, veterans' benefits, farm
subsidies, the kind of things an attentive politician operating in a
tiny state (with a population of just 620,000) can fashion a formidable
political base from.

After three terms as mayor, Sanders ran for Vermont's at-large House
seat in 1988 as an Independent and lost by a small margin to Peter
Smith, the Republican former lieutenant governor. He won a rematch in
1990.

``When I came into the House, no one knew what to do with me,'' Sanders
says. ``I was the only representative from Vermont, so I had no one to
help me. And I was the only Independent, so no one knew where to put me
in terms of committee.''

Sanders was known as something of a pragmatic gadfly in the House. His
grillings of former Federal Reserve Chairman Alan Greenspan became a
running burlesque, much awaited by many Hill and Federal Reserve
watchers whenever Greenspan appeared before the House Financial Services
Committee. (``Do you give one whit of concern for the middle class and
working families of this country?'' Sanders asked Greenspan in one
representative exchange.)

Sanders was not without his legislative triumphs. He was adept at
working with people with whom he otherwise disagreed sharply --- forging
alliances with conservatives like Representative Ron Paul, Republican of
Texas and a well-known libertarian, with whom he shared a common
hostility to the U.S.A. Patriot Act. In what might have been Sanders's
signature triumph of recent years, he was instrumental in striking a
provision from the Patriot Act that would have required librarians to
release data on what their patrons were reading.

But in keeping with his pragmatic gadfly's approach, Sanders was far
more accomplished at filing amendments to House bills than actually
writing and producing legislation of his own. He was also gifted at
drawing attention to his issues and (just as important) to himself. He
was the first congressman to lead a bus trip to Canada to help seniors
buy cheaper prescription drugs.

As he makes the transition to his new job, Sanders says his former House
colleagues have teased him about not becoming ``like the rest of them''
in the Senate. Sanders jokes about this, as much as he jokes about
anything. He says he will be required to enter a machine that zaps his
brain and transforms him ``into a member in good standing in the House
of Lords.''

``We're talking about a completely different animal here,'' Sanders
says. The House fosters a more hospitable habitat for the audacious and
eccentric; their ranks tend to be camouflaged by its larger numbers,
curtailed by strict time limits on floor speeches and reined in by the
outsize power of the House leadership. Senators can speak for as long as
they want and single-handedly buck the wishes of 99 other senators by
placing ``holds'' on bills and nominations. Tradition dictates that
senators exercise such privileges sparingly.

``There will be times when he causes the Democratic leadership some
agita,'' Schumer predicts. ``But knowing him, I think he's smart enough
not to make any gratuitous enemies. He might make enemies, but they
won't be gratuitous enemies.''

Sanders told me, ``You have to ask yourself, Did the people send me here
to give long speeches, or did they send me here to get things done?''

By ``you'' Sanders means himself, as his sleepless Socialist adventure
proceeds into the House of Lords.

On a quiet morning in mid-December, Sanders was sitting in his new
office in the basement of a Senate office building --- it is a temporary
office he will inhabit before he moves to another temporary office that
he will occupy until a permanent space opens up, probably around March.
It's all very exasperating, he said, this office-space situation. But he
asked that I keep the specifics of his exasperation out of the article.
He is trying to meet a stepped-up standard of tact and decorum in his
new home.

``Why can't we get these phone calls forwarded from the House office?''
Sanders asked a staff person who is working temporarily at a temporary
reception desk in the temporary-temporary office. Everything seems
temporary, but not as temporary as before. Sanders has a six-year term
now instead of a two-year one. Friends have advised him to pace himself,
curb his impatience. He would seem ill wired for this, but he is trying.
He even took a four-day vacation last month --- and to Palm Springs.

But now he has work to do, beginning with getting to know his
colleagues. ``Personal relationships are very important in the Senate,''
he told me. He likes the Senate majority leader, Harry Reid, a lot,
appreciates that he gave him the committee assignments that he wanted
--- Health, Education, Labor and Pensions; the Environment and Public
Works; Veterans' Affairs; Energy and Natural Resources; and the Budget.
And wouldn't you know, Reid has an interest in dental care, too. He grew
up dirt poor in Nevada, and his mother had no teeth. The first thing
Reid did when he got his first job --- at a gas station --- was buy her
a new set. So the Senate's leading Democrat gets the importance of
dental care, which could help save teeth in Vermont.

``Let's go somewhere else to talk,'' Sanders said, as we headed out the
door of his temporary-temporary office. ``We can get some coffee.''

We traversed a maze of hallways that lead into a Senate dining room.
``Can we sit down in here?'' he asked a busperson. Yes, but then Sanders
looked at a bunch of tables covered in white linen table clothes, not
what he had in mind.

We walked upstairs, in search of a quiet place in the new neighborhood,
on the Senate side. He kept navigating short hallways and turning back.
An elevator opened in front of Sanders. It said ``Senators Only.'' The
attendant invited him on, but he hesitated, turned away and began
looking for another route to wherever he was going.

Sanders zigzags the Capitol this way barely recognized, or acknowledged
(or congratulated, or urged to run for president). A few people stare at
the new senator as he walks by --- maybe because he looks lost, or
famous, or maybe just because he looks like a strange bird out of
Vermont.

Advertisement

\protect\hyperlink{after-bottom}{Continue reading the main story}

\hypertarget{site-index}{%
\subsection{Site Index}\label{site-index}}

\hypertarget{site-information-navigation}{%
\subsection{Site Information
Navigation}\label{site-information-navigation}}

\begin{itemize}
\tightlist
\item
  \href{https://help.nytimes3xbfgragh.onion/hc/en-us/articles/115014792127-Copyright-notice}{©~2020~The
  New York Times Company}
\end{itemize}

\begin{itemize}
\tightlist
\item
  \href{https://www.nytco.com/}{NYTCo}
\item
  \href{https://help.nytimes3xbfgragh.onion/hc/en-us/articles/115015385887-Contact-Us}{Contact
  Us}
\item
  \href{https://www.nytco.com/careers/}{Work with us}
\item
  \href{https://nytmediakit.com/}{Advertise}
\item
  \href{http://www.tbrandstudio.com/}{T Brand Studio}
\item
  \href{https://www.nytimes3xbfgragh.onion/privacy/cookie-policy\#how-do-i-manage-trackers}{Your
  Ad Choices}
\item
  \href{https://www.nytimes3xbfgragh.onion/privacy}{Privacy}
\item
  \href{https://help.nytimes3xbfgragh.onion/hc/en-us/articles/115014893428-Terms-of-service}{Terms
  of Service}
\item
  \href{https://help.nytimes3xbfgragh.onion/hc/en-us/articles/115014893968-Terms-of-sale}{Terms
  of Sale}
\item
  \href{https://spiderbites.nytimes3xbfgragh.onion}{Site Map}
\item
  \href{https://help.nytimes3xbfgragh.onion/hc/en-us}{Help}
\item
  \href{https://www.nytimes3xbfgragh.onion/subscription?campaignId=37WXW}{Subscriptions}
\end{itemize}
