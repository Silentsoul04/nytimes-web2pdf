Sections

SEARCH

\protect\hyperlink{site-content}{Skip to
content}\protect\hyperlink{site-index}{Skip to site index}

\href{https://www.nytimes3xbfgragh.onion/pages/business/media/index.html}{Media}

\href{https://myaccount.nytimes3xbfgragh.onion/auth/login?response_type=cookie\&client_id=vi}{}

\href{https://www.nytimes3xbfgragh.onion/section/todayspaper}{Today's
Paper}

\href{/pages/business/media/index.html}{Media}\textbar{}Uncle Ben, Board
Chairman

\begin{itemize}
\item
\item
\item
\item
\item
\end{itemize}

Advertisement

\protect\hyperlink{after-top}{Continue reading the main story}

Supported by

\protect\hyperlink{after-sponsor}{Continue reading the main story}

Advertising

\hypertarget{uncle-ben-board-chairman}{%
\section{Uncle Ben, Board Chairman}\label{uncle-ben-board-chairman}}

By \href{https://www.nytimes3xbfgragh.onion/by/stuart-elliott}{Stuart
Elliott}

\begin{itemize}
\item
  March 30, 2007
\item
  \begin{itemize}
  \item
  \item
  \item
  \item
  \item
  \end{itemize}
\end{itemize}

A racially charged advertising character, who for decades has been
relegated to a minor role in the marketing of the products that still
carry his name, is taking center stage in a campaign that gives him a
makeover --- Madison Avenue style --- by promoting him to chairman of
the company.

The character is Uncle Ben, the symbol for more than 60 years of the
Uncle Ben's line of rices and side dishes now sold by the food giant
Mars. The challenges confronting Mars in reviving a character as
racially fraught as Uncle Ben were evidenced in the reactions of experts
to a redesigned Web site (unclebens.com), which went live this week.

``This is an interesting idea, but for me it still has a very high
cringe factor,'' said Luke Visconti, partner at Diversity Inc. Media in
Newark, which publishes a magazine and Web site devoted to diversity in
the workplace.

``There's a lot of baggage associated with the image,'' Mr. Visconti
said, which the makeover ``is glossing over.''

Uncle Ben, who first appeared in ads in 1946, is being reborn as Ben, an
accomplished businessman with an opulent office, a busy schedule, an
extensive travel itinerary and a penchant for sharing what the company
calls his ``grains of wisdom'' about rice and life. A crucial aspect of
his biography remains the same, though: He has no last name.

Vincent Howell, president for the food division of the Masterfoods USA
unit of Mars, said that because consumers described Uncle Ben as having
``a timeless element to him, we didn't want to significantly change
him.''

``What's powerful to me is to show an African-American icon in a
position of prominence and authority,'' Mr. Howell said. ``As an
African-American, he makes me feel so proud.''

The previous reluctance to feature Uncle Ben prominently in ads stood in
stark contrast to the way other human characters like Orville
Redenbacher and Colonel Sanders personify their products. That reticence
can be traced to the contentious history of Uncle Ben as the black face
of a white company, wearing a bow tie evocative of servants and Pullman
porters and bearing a title reflecting how white Southerners once used
``uncle'' and ``aunt'' as honorifics for older blacks because they
refused to say ``Mr.'' and ``Mrs.''

Before the civil rights movement took hold, marketers of food and
household products often used racial and ethnic stereotypes in creating
brand characters and mascots.

In addition to Uncle Ben, there was Aunt Jemima, who sold pancake mix in
ads that sometimes had her exclaiming, ``Tempt yo' appetite;'' a
grinning black chef named Rastus, who represented Cream of Wheat hot
cereal; the Gold Dust Twins, a pair of black urchins who peddled a soap
powder for Lever Brothers; the Frito Bandito, who spoke in an
exaggerated Mexican accent; and characters selling powdered drink mixes
for Pillsbury under names like Injun Orange and Chinese Cherry --- the
latter baring buck teeth.

``The only time blacks were put into ads was when they were athletic,
subservient or entertainers,'' said Marilyn Kern Foxworth, the author of
``Aunt Jemima, Uncle Ben and Rastus: Blacks in Advertising Yesterday,
Today and Tomorrow.''

After the start of the civil rights movement, such characters became
``lightning rods'' in a period when consumers started to want ``images
our children could look up to and emulate,'' Ms. Kern Foxworth said.

Image

A Web site for Uncle Ben's, unclebens.com, offers a look at his
executive office. Credit...Newspaper ad and image of Uncle Ben in his
office, Masterfoods USA

As a result, most of those polarizing ad characters were banished when
marketers --- becoming more sensitive to the changing attitudes of
consumers --- realized they were no longer appropriate. A handful like
Uncle Ben, Aunt Jemima and the Cream of Wheat chef were redesigned and
kept on, but in the unusual status of silent spokescharacters, removed
from ads and reduced to staring mutely from packages.

Times, however, change, as evidenced by real-life figures as disparate
as Wally Amos, the founder of Famous Amos cookies; Oprah Winfrey; and
Senator Barack Obama, the Illinois Democrat who is running for
president. In advertising, there are now black authority figures serving
as spokesmen in multimillion-dollar campaigns, like Dennis Haysbert, for
Allstate, and James Earl Jones, for Verizon.

That helped executives at Masterfoods and its advertising agency,
TBWA/Chiat/Day, consider the risky step of reviving the character.

``There's no doubt we realized we had a very powerful asset we were not
using strongly enough,'' Mr. Howell said.

So about 18 months ago, the company and agency decided ``to reach out to
our consumers'' and gauge attitudes toward Uncle Ben, Mr. Howell said.
There were no negative responses or references to the stereotyped
aspects of the character, he said. Rather, the consumers ``focused on
positive images, quality, warmth, timelessness,'' he added, and ``the
legend of Uncle Ben.''

That encouraged the idea that ``we could bring him to life,'' Mr. Howell
said, sensitive to ``the sorts of concerns that are important to me as
an African-American.''

Joe Shands, a creative director at the Playa del Rey, Calif., office of
TBWA/Chiat/Day, said the freedom to use the character to sell the Uncle
Ben's brand was a welcome change from the years when ``all we've had to
work with is a portrait.'' ``We wanted to know if there was something
there we could utilize to talk about new products, existing products,
the values of the company,'' Mr. Shands said, adding that both black and
white consumers described the character as someone ``they know and
love.''

``Through the magic of marketing, we've made him the chairman,'' Mr.
Shands said. Uncle Ben's office, he said, is ``reflective of a man with
great wisdom who has done great things.''

Magazine ads in the campaign, which carries the theme ``Ben knows
best,'' present a painting of the character in a gold frame with the
chairman's title affixed on a plaque.

The painting is also on display on the home page of the redesigned Web
site, which offers a virtual tour of Ben's office. Visitors can browse
through his e-mail messages, examine his datebook and read his executive
memorandums.

``It's important consumers begin to hear from Uncle Ben,'' said Mr.
Howell of Masterfoods, who is based in Los Angeles.

Despite the character's impressive new credentials, some advertising
executives expressed skepticism that the campaign could avoid negative
overtones.

The ads are ``asking us to make the leap from Uncle Ben being someone
who looks like a butler to overnight being a chairman of the board,''
Ms. Kern Foxworth said. ``It does not work for me.''

Image

The longtime image of Aunt Jemima, in a photo taken between 1933 and
1951; a number of women, including Anna Robinson, shown here, portrayed
the character at events like state fairs.Credit...Bettmann/Corbis

``I applaud them for the effort and trying to move forward,'' she added,
but the decision to keep the same portrait of Uncle Ben, bow tie and
all, also dismayed her because ``they're trying so hard to hold onto
something I'm trying so hard to get rid of.''

Howard Buford, chief executive at Prime Access in New York, an agency
specializing in multicultural campaigns, said he gave the campaign's
creators some credit. ``It's potentially a very creative way to handle
the baggage of old racial stereotypes as advertising icons,'' he said,
but ``it's going to take a lot of work to get it right and make it ring
true.''

For instance, Mr. Buford said, noting all the ``Ben'' references in the
ads, ``Rarely do you have someone of that stature addressed by his first
name'' --- and minus any signs of a surname.

Mr. Buford, who is a real-life black leader of a company, likened the
promotion of Uncle Ben to the abrupt plot twists on TV series like
``Benson'' and ``Designing Women,'' when black characters in subservient
roles one season became professionals the next.

``It's nice that now, for the 21st century, they're saying this icon can
`own' a company,'' Mr. Buford said, ``but they're going to have to make
him a whole person.''

Mr. Visconti of Diversity Inc. Media struck a similar chord. He said he
would have turned Ben's office into ``a learning experience,''
furnishing it with, for example, books by Frederick Douglass and the
Rev. Dr. Martin Luther King Jr.

``I've never been in the office of African-Americans of this era who
didn't have something in their office showing what it took to get them
there,'' Mr. Visconti said.

The actual biography of Uncle Ben is at variance with his fanciful new
identity. According to Ms. Kern Foxworth's book and other reference
materials, there was a Ben --- no surname survives --- who was a Houston
rice farmer renowned for the quality of his crops. During World War II,
Gordon L. Harwell, a Texas food broker, supplied to the armed forces a
special kind of white rice, cooked to preserve the nutrients, under the
brand name Converted Rice.

In 1946, Mr. Harwell had dinner with a friend (or business partner) in
Chicago (or Houston) and decided that a portrait of the maitre d'hotel
of the restaurant, Frank Brown, could represent the brand, which was
renamed Uncle Ben's Converted Rice as it was being introduced to the
consumer market.

In coming months, visitors to the Uncle Ben's Web site will be able to
discover new elements of the character, Mr. Howell said, like full-body
digital versions of Uncle Ben and voice mail messages. The Web site was
designed by an agency, Tequila, that is a sibling of TBWA/Chiat/Day, and
the budget for the campaign, print and online, is estimated at \$20
million. TBWA/Chiat/Day is part of the TBWA Worldwide unit of the
Omnicom Group.

If the makeover for Uncle Ben is deemed successful, could there be
similar changes in store for other racially charged characters?

Last month, the Cream of Wheat chef got a new owner when B\&G Foods
completed a \$200 million deal to buy his brand, and its companion,
Cream of Rice, from Kraft Foods.

``We're doing consumer focus work right now to understand how important
the character is,'' said David L. Wenner, chief executive at B\&G in
Parsippany, N.J.

If any changes were to be made, ``you would need to be very careful,''
he added, ``and you would want to do it with dignity.''

Advertisement

\protect\hyperlink{after-bottom}{Continue reading the main story}

\hypertarget{site-index}{%
\subsection{Site Index}\label{site-index}}

\hypertarget{site-information-navigation}{%
\subsection{Site Information
Navigation}\label{site-information-navigation}}

\begin{itemize}
\tightlist
\item
  \href{https://help.nytimes3xbfgragh.onion/hc/en-us/articles/115014792127-Copyright-notice}{©~2020~The
  New York Times Company}
\end{itemize}

\begin{itemize}
\tightlist
\item
  \href{https://www.nytco.com/}{NYTCo}
\item
  \href{https://help.nytimes3xbfgragh.onion/hc/en-us/articles/115015385887-Contact-Us}{Contact
  Us}
\item
  \href{https://www.nytco.com/careers/}{Work with us}
\item
  \href{https://nytmediakit.com/}{Advertise}
\item
  \href{http://www.tbrandstudio.com/}{T Brand Studio}
\item
  \href{https://www.nytimes3xbfgragh.onion/privacy/cookie-policy\#how-do-i-manage-trackers}{Your
  Ad Choices}
\item
  \href{https://www.nytimes3xbfgragh.onion/privacy}{Privacy}
\item
  \href{https://help.nytimes3xbfgragh.onion/hc/en-us/articles/115014893428-Terms-of-service}{Terms
  of Service}
\item
  \href{https://help.nytimes3xbfgragh.onion/hc/en-us/articles/115014893968-Terms-of-sale}{Terms
  of Sale}
\item
  \href{https://spiderbites.nytimes3xbfgragh.onion}{Site Map}
\item
  \href{https://help.nytimes3xbfgragh.onion/hc/en-us}{Help}
\item
  \href{https://www.nytimes3xbfgragh.onion/subscription?campaignId=37WXW}{Subscriptions}
\end{itemize}
