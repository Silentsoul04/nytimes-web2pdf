Sections

SEARCH

\protect\hyperlink{site-content}{Skip to
content}\protect\hyperlink{site-index}{Skip to site index}

\href{https://www.nytimes3xbfgragh.onion/section/books/review}{Book
Review}

\href{https://myaccount.nytimes3xbfgragh.onion/auth/login?response_type=cookie\&client_id=vi}{}

\href{https://www.nytimes3xbfgragh.onion/section/todayspaper}{Today's
Paper}

\href{/section/books/review}{Book Review}\textbar{}Fiction Chronicle

\begin{itemize}
\item
\item
\item
\item
\item
\end{itemize}

Advertisement

\protect\hyperlink{after-top}{Continue reading the main story}

Supported by

\protect\hyperlink{after-sponsor}{Continue reading the main story}

\hypertarget{fiction-chronicle}{%
\section{Fiction Chronicle}\label{fiction-chronicle}}

By Alison McCulloch

\begin{itemize}
\item
  March 23, 2008
\item
  \begin{itemize}
  \item
  \item
  \item
  \item
  \item
  \end{itemize}
\end{itemize}

\textbf{THE LABRADOR PACT, by Matt Haig. (Viking, \$23.95.)} In ``Henry
IV, Part 1,'' Falstaff much prefers survival (and a good breakfast) to
honor and duty: ``What is that word honor? ... Who hath it? He that died
o' Wednesday. Doth he feel it? No.'' The demands of virtue are also a
problem for Prince, the canine narrator of Haig's curiously affecting
take on ``Henry,'' which tells the story of a loyal Labrador's efforts
to hold his human family together. If only he follows the pact of his
breed, Prince tells himself, he can save these people from themselves
--- from ``all their lies and tensions and betrayals and injustices.''
His friend Falstaff, a chubby mixed-breed he meets in the park, begs to
differ. ``Duty schmooty,'' he bays (or however fictional dogs utter
their dialogue) when faced with Prince's insistence on ``duty over
all.'' Dark, comic and quite brilliantly adult, Haig's thinking animals
never stray into the sickly sweet zone. As the duplicities mount (among
both species), Prince's Labrador dogma proves inadequate to the task:
``Whereas dogs can learn to suppress their instincts,'' he realizes,
``for humans there is no hope.''

\textbf{THE DIVING POOL: Three Novellas, by Yoko Ogawa. (Picador, paper,
\$13.)} Still waters run dark in these bright yet eerie novellas, whose
crisp, almost guileless prose hides unexpected menace. In the title
story, Aya, a high school student, is obsessed with the graceful body of
a young diving enthusiast and the suffering of a child on whom she
unleashes a strangely calm cruelty: ``I wanted to savor every one of
Rie's tears, to run my tongue over the damp, festering, vulnerable
places in her heart and open the wounds even wider.'' Another novella is
the perplexing account of two sisters, one apparently keeping a record
of the other's pregnancy. At first, the mother-to-be seems the more
troubled, with her bizarre cravings and visits to a psychiatrist. Then
again, perhaps it's the diarist we should be keeping an eye on. In the
final novella, a lonely woman helps settle her young cousin into her old
college residence, which has fallen into disrepair since the unexplained
disappearance of a student. This nameless narrator can't stay away from
the dorm or its spooky manager, who is missing a leg and both arms and
says he is dying. She begins visiting every day, only belatedly
wondering why her cousin never seems to be around. Stephen Snyder's
elegant translations from the Japanese whet the appetite for more.

Image

\textbf{PLAYING THE CHANGES}\\
\emph{Milt Hinton's Life in Stories and Photographs. By Milt Hinton,
David Berger and Holly Maxson. (Vanderbilt University, \$75.)}

The bassist Milt Hinton also snapped pictures from the 1920s until his
death, at 90, in 2000. Among his famous subjects: Billie Holiday, above,
in her last recording session, in New York, 1959.Credit...Milt
Hinton/The Milton J. Hinton Photographic Collection

\textbf{DERVISHES, by Beth Helms. (Picador, paper, \$14.)} Secrets and
lies swirl around the characters in this first novel like the heat and
dust of 1970s Ankara, which Helms evokes so well. Grace and her
12-year-old daughter, Canada, are part of the embassy set, adrift in a
country with which they never really connect. This sense of dislocation
seems to have its roots in the mysterious work of Grace's husband, who
is frequently called away on clandestine missions to unknown
destinations. The novel opens with just such a summons, but this is one
from which Canada's father will not return. Helms then rewinds to
recount the period leading up to that late-night phone call. For mother
and daughter, it's a time of unraveling: Grace makes bad judgment calls
and has an affair with Canada's riding instructor, while her daughter,
who is both a corrupting and a corruptible child, tries to raise
herself. As Helms deftly maps the damage these two have wrought on
themselves and those around them, Canada wonders how they will cope:
``How would we live outside the close and sheltering world we'd always
bucked against, but always known? We would be suddenly, in all ways,
unmoored, ill equipped, stark naked.''

\textbf{THE SHADOW YEAR, by Jeffrey Ford. (Morrow/HarperCollins,
\$25.95.)} Ford travels deep into the wild country that is childhood in
this novel --- part mystery, part fantasy --- about a struggling family
on Long Island in the 1960s. The story's ``strange changeling year''
begins in the last days of August as a prowler is reported outside a
neighbor's window. The unnamed narrator, a sixth grader, and his
brother, Jim, decide to investigate, enlisting the mysterious powers of
their quirky sister, Mary, who tells them what's going to happen by
moving the pieces on a model of the neighborhood they keep in the
cellar. Jim has built this miniature world, Botch Town, out of
``buttons, Dixie cups, ice cream sticks, bottles and assorted other
discarded items'' on a table their father set up for electric trains ---
a project nixed by ``the money troubles.'' Ford's attention is always on
the kids, keeping their alcoholic mother and workaholic father largely
at a distance. But the observations and adventures of these sharp,
wayward children provide more than enough depth to be satisfying. And,
of course, there's the whodunit: What happened to young Charlie Edison?
Who is the prowler? And do the answers to these questions involve the
man who cruises the streets in a large white car?

\textbf{THE EXECUTOR: A Comedy of Letters, by Michael Krüger. (Harcourt,
\$23.)} Rudolf, a famous, curmudgeonly German writer living in Italy,
has committed suicide, leaving his longtime friend, the narrator of this
taut little novel, to sort out Rudolf's literary legacy and unlock the
secret of his last, great masterwork, a novel ``meant to transform the
genre itself.'' What a perfect setup for an author with the authority of
Michael Krüger (poet, novelist, editor and longtime head of a major
German publishing house) to ponder writers and writing. This he does
through the dead man, who, as remembered by the narrator, had almost
nothing good to say about his art or its practitioners: ``On the subject
of contemporary literature he was excoriating, disparaging it as
`middle-class prose' ''; ``for him, imagination was a petit-bourgeois
fad''; ``nothing was more repugnant to him than a gathering of
authors.'' Along the way, Krüger scatters names like bread crumbs ---
Pavese, Valéry, Lucentini, Proust, Kierkegaard, Levi and many more ---
adding to the rather bombastic tone of the book, translated by John
Hargraves. The mystery of the masterwork is provocative, but Rudolf is
no Céline --- there's not enough genius to make his misanthropy (and
misogyny, for that matter) worth caring about.

Advertisement

\protect\hyperlink{after-bottom}{Continue reading the main story}

\hypertarget{site-index}{%
\subsection{Site Index}\label{site-index}}

\hypertarget{site-information-navigation}{%
\subsection{Site Information
Navigation}\label{site-information-navigation}}

\begin{itemize}
\tightlist
\item
  \href{https://help.nytimes3xbfgragh.onion/hc/en-us/articles/115014792127-Copyright-notice}{©~2020~The
  New York Times Company}
\end{itemize}

\begin{itemize}
\tightlist
\item
  \href{https://www.nytco.com/}{NYTCo}
\item
  \href{https://help.nytimes3xbfgragh.onion/hc/en-us/articles/115015385887-Contact-Us}{Contact
  Us}
\item
  \href{https://www.nytco.com/careers/}{Work with us}
\item
  \href{https://nytmediakit.com/}{Advertise}
\item
  \href{http://www.tbrandstudio.com/}{T Brand Studio}
\item
  \href{https://www.nytimes3xbfgragh.onion/privacy/cookie-policy\#how-do-i-manage-trackers}{Your
  Ad Choices}
\item
  \href{https://www.nytimes3xbfgragh.onion/privacy}{Privacy}
\item
  \href{https://help.nytimes3xbfgragh.onion/hc/en-us/articles/115014893428-Terms-of-service}{Terms
  of Service}
\item
  \href{https://help.nytimes3xbfgragh.onion/hc/en-us/articles/115014893968-Terms-of-sale}{Terms
  of Sale}
\item
  \href{https://spiderbites.nytimes3xbfgragh.onion}{Site Map}
\item
  \href{https://help.nytimes3xbfgragh.onion/hc/en-us}{Help}
\item
  \href{https://www.nytimes3xbfgragh.onion/subscription?campaignId=37WXW}{Subscriptions}
\end{itemize}
