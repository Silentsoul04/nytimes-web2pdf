Sections

SEARCH

\protect\hyperlink{site-content}{Skip to
content}\protect\hyperlink{site-index}{Skip to site index}

\href{https://www.nytimes3xbfgragh.onion/section/arts/design}{Art \&
Design}

\href{https://myaccount.nytimes3xbfgragh.onion/auth/login?response_type=cookie\&client_id=vi}{}

\href{https://www.nytimes3xbfgragh.onion/section/todayspaper}{Today's
Paper}

\href{/section/arts/design}{Art \& Design}\textbar{}Revisiting Max's,
Sanctuary for the Hip

\begin{itemize}
\item
\item
\item
\item
\item
\end{itemize}

Advertisement

\protect\hyperlink{after-top}{Continue reading the main story}

Supported by

\protect\hyperlink{after-sponsor}{Continue reading the main story}

\hypertarget{revisiting-maxs-sanctuary-for-the-hip}{%
\section{Revisiting Max's, Sanctuary for the
Hip}\label{revisiting-maxs-sanctuary-for-the-hip}}

\includegraphics{https://static01.graylady3jvrrxbe.onion/images/2010/09/05/arts/05maxs-span/05maxs-span-articleLarge.jpg?quality=75\&auto=webp\&disable=upscale}

By \href{https://www.nytimes3xbfgragh.onion/by/randy-kennedy}{Randy
Kennedy}

\begin{itemize}
\item
  Sept. 1, 2010
\item
  \begin{itemize}
  \item
  \item
  \item
  \item
  \item
  \end{itemize}
\end{itemize}

IF a fiction writer were to sit down and conjure up a Manhattan
nightspot where a John Chamberlain sculpture flanked the jukebox and
Debbie Harry waited tables, where the earth artist Robert Smithson held
court with Waylon Jennings, where struggling artists could cash their
checks and pick up their mail, where the New York Dolls and Charlie Rich
played (in the same year!) and where an unknown named Bob Marley once
opened for a slightly less unknown named
\href{http://www.youtube.com/watch?v=xOQfo3406XE}{Bruce Springsteen}, he
would probably be scoffed at for fabulist excess.

But when Mickey Ruskin, a shy, strange-looking impresario with a chipped
gold tooth, opened
\href{http://classic.maxskansascity.com/index2.php}{Max's Kansas City}
on a nowhere stretch of Park Avenue South in 1965, it became that kind
of fact-trumps-fiction place, ultimately one of the few New York clubs
that could be said to have lived up to its legend. And the legend was
not inconsiderable: It played an important role in nurturing at least
two art movements (Minimalism and Pop); it enshrined a new, subversive
generation of rock music; and it helped give birth to the counterculture
itself, or at least provided it with a dazzling ideal.

``Truly, the F.B.I. would have done well by itself to close the place
down,'' said the sculptor Forrest Myers, known as Frosty, who helped
design the bar and restaurant for Ruskin with the help of the painter
Neil Williams.

Back in the 1950s the Cedar Tavern was the most famous artists' bar in
the world, but got that way mostly because its drinks were cheap and
because it was near painters' studios in Greenwich Village. Max's, on
the other hand, was out of the way and a little too expensive for people
without regular paychecks, but it may have been the first New York bar
designed expressly for artists. It became ``a kind of Ellis Island'' for
a wave of them who came to the city in the 1960s and '70s, said Anton
Perich, a photographer and early video auteur who worked there after
arriving from Paris, failing as a busboy but allowed by Ruskin to stick
around and take pictures. ``It was the place where I felt safe,'' Mr.
Perich said.

The demise of the original Max's in 1974 (it would continue as more of a
straight-ahead music club under new ownership, one of the crucibles of
punk, until closing in 1981) was brought about not by the F.B.I. but by
Ruskin's tax problems and increasing drug use. But beginning Sept. 15 in
Chelsea many scattered pieces of its history --- including some never
made public before, discovered in old film files --- will be reassembled
in two exhibitions, one at the
\href{http://www.stevenkasher.com/html/home.asp}{Steven Kasher Gallery}
and another, focusing on Max's artist regulars, at the
\href{http://www.lorettahoward.com/}{Loretta Howard Gallery}. In
conjunction with the Kasher show Abrams Image is also publishing ``Max's
Kansas City: Art, Glamour, Rock and Roll,'' a raucous photo book with
reminiscences of the club from the guitarist Lenny Kaye, the artist
Lorraine O'Grady and others, along with reproductions of time-yellowed
artifacts like an Andy Warhol bar tab (\$774.73 for September 1969,
minus a \$200 credit for a work of art identified only as ``Marilyn
Monroe'').

While mountains of words have been devoted to Max's over the years in
the memoirs of musicians and artists, the new book is only the second
extensive treatment of the club's history, following ``High on
Rebellion: Inside the Underground at Max's Kansas City,'' now out of
print, a 1998 collection of photos and interviews edited by Yvonne
Sewall-Ruskin, Mickey Ruskin's longtime companion, who now runs the
\href{http://www.maxskansascity.org/}{Max's Kansas City Project}, a
nonprofit philanthropy that helps artists and promotes drug-abuse
prevention.

The paucity of publications has been a result, in part, of the strict
control that Mickey Ruskin exercised over the taking of pictures and
most other kinds of documentation of the doings inside the club, which
often involved casual nudity and more-than-casual drug use.
(Amphetamines were the controlled substance of choice at the beginning,
in the pre-cocaine days.) Newspaper photographers were rarely allowed
in, creating a much different atmosphere than the one that prevailed at
Studio 54 when it opened in 1977.

``It was an oasis, and nobody there wanted a record of what they were
doing,'' said Mr. Kasher, who edited the Abrams book and helped discover
previously unknown pictures of Max's from insiders like the music
executive and writer Danny Fields, Mr. Perich and others. ``This was a
long time before the era of blogs and YouTube.''

As suggested by the title of the Loretta Howard show --- ``Artists at
Max's Kansas City 1965-1974: Hetero-Holics and Some Women Too'' --- the
bar was mostly a boy's club, as most of the art world then still was.

But female artists like Dorothea Rockburne, Lynda Benglis and Eve Hesse
also hung out there. The artist and philosopher Adrian Piper did a
well-remembered performance piece inside the bar in 1970, walking around
with a blindfold and earplugs in a place that was all about looking and
listening. Besides the Chamberlain crushed-metal sculpture, a Frank
Stella on the wall and a red-fluorescent Dan Flavin work dominating the
back room, the bar featured a collage by Ms. Rockburne and photographs
by Brigid Berlin. Ruskin also usually deployed women --- like the Warhol
followers Abigail Rosen and Dorothy Dean, a Harvard-trained editor ---
to control access at the front door.

In her memoir
\href{http://www.nytimes3xbfgragh.onion/2010/01/31/books/review/Carson-t.html?_r=1\&ref=patti_smith}{``Just
Kids,''} published this year, the singer Patti Smith recalls being taken
there for the first time in 1969 by Robert Mapplethorpe; they shared a
salad and stared toward the back room, the holiest of holies, rendered
so by Warhol, who had held court there for many years. After lots of
hanging out the pair were finally admitted and seated at the round table
where the coolest kids --- the lead singers, the transvestites, the
successful artists and the Factory regulars --- still sat.

``Robert was at ease,'' Ms. Smith wrote, ``because, at last, he was
where he wanted to be. I can't say I felt comfortable at all. The girls
were pretty but brutal.''

When Ruskin first asked Mr. Myers and Mr. Williams in 1965 to take a
look at the site a bit north of Union Square where he wanted to open the
club, then occupied by a run-down Southern-food restaurant, the two
artists were confused.

``We thought, `How are we going to get people over here?'~'' Mr. Myers
recalled recently. ``After 5 o'clock that neighborhood in those days
would just die.''

It was the first restaurant interior Mr. Myers had ever designed, and he
said his idea was make it look sleek and clean, like a gallery space,
with red booths and white walls.

\href{https://www.nytimes3xbfgragh.onion/slideshow/2010/09/01/arts/design/20100905-MAXS.html}{}

\hypertarget{an-artists-oasis}{%
\subsection{An Artist's Oasis}\label{an-artists-oasis}}

11 Photos

View Slide Show ›

\includegraphics{https://static01.graylady3jvrrxbe.onion/images/2010/09/01/arts/20100905-MAXS-slide-SF4Z/20100905-MAXS-slide-SF4Z-jumbo.jpg?quality=75\&auto=webp\&disable=upscale}

Courtesy Anton Perich and Steven Kasher Gallery, NYC

``Mickey was into art,'' Mr. Myers said, ``and so we decided that this
was not going to be a working-class bar, or a poets' bar. It was going
to be an artists' bar.''

Maurice Tuchman, a longtime curator at the Los Angeles County Museum of
Art and the curator of the Loretta Howard exhibition, said he remembered
visiting Max's in the mid-1960s because it was a place where any good
contemporary-art curator had to check in.

``And I was so bowled over by the presence of so much new art that was
so prominently displayed all over the place,'' he said. ``Mickey didn't
talk about it. No one talked about it much, but there it was. It sent an
intense subliminal message that art was the subject at hand.''

Mr. Perich added: ``They were not in museums, back then, these pieces.
The only place you knew you could see them was at Max's.''

Flavin's red-light work, which memorialized victims of the Vietnam War,
and Mr. Chamberlain's sculpture next to the jukebox, evocative of James
Dean's crashed car, always struck him as ``pieces of the
American-dream-gone-wrong puzzle,'' he said, a theme that resonated with
a crowd desperately trying to discover a new kind of American dream.

Throughout Ruskin's tenure at Max's, that crowd was usually full of
artists, many of them unknown and never to be known. They came for a
free steam-table lunch of chicken wings and chili served every
afternoon. And they stayed into the night.

``Basically if you were an artist, he wouldn't keep you out,'' Mr. Myers
said. ``Which is unusual because artists at that time didn't really have
much social power.''

In an interview conducted in the early 1970s by Mr. Fields, the music
producer, Ruskin --- a middle-class New Jersey boy who left a job as a
lawyer
t\href{http://www.nytimes3xbfgragh.onion/2003/05/11/nyregion/neighborhood-report-union-square-archetypal-host.html}{o
seek a more exciting life} --- described how little he knew at first
about art and music even as his club came to revolve around them.

His first bar, the Tenth Street Coffeehouse in the East Village, became
a poets' hangout through almost no effort of his own, except his
welcoming spirit. When he opened the Ninth Circle in Greenwich Village
in 1962, he hoped only that it would become a ``beatnik'' hangout, as he
told Mr. Fields. But Mr. Williams began drinking there and introduced
Ruskin to fellow artists like Mr. Myers, Mr. Chamberlain, Larry Poons
and Carl Andre.

It was a propitious moment not only for the New York art scene but for
Ruskin as a club owner. ``Poets really aren't drinkers, and artists
are,'' Ruskin explained, one of his sociological aperçus that is often
repeated.

The story goes, however, that a poet, Joel Oppenheimer, was responsible
for the club's odd name. He suggested the Kansas City part because of
the general feeling that it would sound more authentic for a place
featuring steaks; Max's was either borrowed from the poet Max Finstein
or, more probably, added simply because it sounded like a reliable
restaurant proprietor's name.

Yet no one sought out the spot for its food. Even the dried chickpeas
that were always on the table and strangely featured on the sign out
front (``Steak, lobster, chickpeas,'' though the sign made it look like
``Steak, chick, lobster, peas'') were sometimes too hard to eat and
better used for throwing.

Most of the art that once hung or sat in the bar has long been dispersed
to the winds, much of it sold by Ruskin, who died from a drug overdose
in 1983 at the age of 50. At the Kasher gallery exhibition the Flavin
light sculpture will be recreated. (It couldn't be obtained for the
show; a version of the work sold for \$662,000 at Christie's in 2009.)
The old Max's space itself is now an upscale Korean deli, where only the
steam tables evoke its past life.

In the interview with Mr. Fields, printed in the Abrams book, Ruskin
often sounded elegiac, even at what probably should have been a high
point for him, with the fame of his establishment and of his role in
creating it assured.

``I wonder, if there is no Max's, does that mean that there is no Mickey
Ruskin?'' he pondered. ``Is Max's all I ever want to do?''

It wasn't quite. Before his death he created other nightspots, including
one in TriBeCa, long (too long, as it turned out) before TriBeCa became
what it is today; it closed for lack of business. But it was only on
Park Avenue South that Ruskin proved to be a kind of prophet, one who
seemed to understand that a place as successful at defining its times as
Max's would not be allowed to outlast them.

Advertisement

\protect\hyperlink{after-bottom}{Continue reading the main story}

\hypertarget{site-index}{%
\subsection{Site Index}\label{site-index}}

\hypertarget{site-information-navigation}{%
\subsection{Site Information
Navigation}\label{site-information-navigation}}

\begin{itemize}
\tightlist
\item
  \href{https://help.nytimes3xbfgragh.onion/hc/en-us/articles/115014792127-Copyright-notice}{©~2020~The
  New York Times Company}
\end{itemize}

\begin{itemize}
\tightlist
\item
  \href{https://www.nytco.com/}{NYTCo}
\item
  \href{https://help.nytimes3xbfgragh.onion/hc/en-us/articles/115015385887-Contact-Us}{Contact
  Us}
\item
  \href{https://www.nytco.com/careers/}{Work with us}
\item
  \href{https://nytmediakit.com/}{Advertise}
\item
  \href{http://www.tbrandstudio.com/}{T Brand Studio}
\item
  \href{https://www.nytimes3xbfgragh.onion/privacy/cookie-policy\#how-do-i-manage-trackers}{Your
  Ad Choices}
\item
  \href{https://www.nytimes3xbfgragh.onion/privacy}{Privacy}
\item
  \href{https://help.nytimes3xbfgragh.onion/hc/en-us/articles/115014893428-Terms-of-service}{Terms
  of Service}
\item
  \href{https://help.nytimes3xbfgragh.onion/hc/en-us/articles/115014893968-Terms-of-sale}{Terms
  of Sale}
\item
  \href{https://spiderbites.nytimes3xbfgragh.onion}{Site Map}
\item
  \href{https://help.nytimes3xbfgragh.onion/hc/en-us}{Help}
\item
  \href{https://www.nytimes3xbfgragh.onion/subscription?campaignId=37WXW}{Subscriptions}
\end{itemize}
