Sections

SEARCH

\protect\hyperlink{site-content}{Skip to
content}\protect\hyperlink{site-index}{Skip to site index}

\href{https://www.nytimes3xbfgragh.onion/section/world/africa}{Africa}

\href{https://myaccount.nytimes3xbfgragh.onion/auth/login?response_type=cookie\&client_id=vi}{}

\href{https://www.nytimes3xbfgragh.onion/section/todayspaper}{Today's
Paper}

\href{/section/world/africa}{Africa}\textbar{}Violent End to an Era as
Qaddafi Dies in Libya

\begin{itemize}
\item
\item
\item
\item
\item
\item
\end{itemize}

Advertisement

\protect\hyperlink{after-top}{Continue reading the main story}

Supported by

\protect\hyperlink{after-sponsor}{Continue reading the main story}

\hypertarget{violent-end-to-an-era-as-qaddafi-dies-in-libya}{%
\section{Violent End to an Era as Qaddafi Dies in
Libya}\label{violent-end-to-an-era-as-qaddafi-dies-in-libya}}

\includegraphics{https://static01.graylady3jvrrxbe.onion/images/2011/10/21/world/africa/21libyaspan/21libyaspan-articleLarge-v2.jpg?quality=75\&auto=webp\&disable=upscale}

By \href{https://www.nytimes3xbfgragh.onion/by/kareem-fahim}{Kareem
Fahim},
\href{https://www.nytimes3xbfgragh.onion/by/anthony-shadid}{Anthony
Shadid} and
\href{https://www.nytimes3xbfgragh.onion/by/rick-gladstone}{Rick
Gladstone}

\begin{itemize}
\item
  Oct. 20, 2011
\item
  \begin{itemize}
  \item
  \item
  \item
  \item
  \item
  \item
  \end{itemize}
\end{itemize}

MISURATA, Libya --- Col. Muammar el-Qaddafi's last moments Thursday were
as violent as the uprising that overthrew him.

In a cellphone
\href{https://www.facebookcorewwwi.onion/photo.php?v=10150362178348094}{video}
that went viral on the Internet, the deposed Libyan leader is seen
splayed on the hood of a truck and then stumbling amid a frenzied crowd,
seemingly begging for mercy. He is next seen on the ground, with
fighters grabbing his hair. Blood pours down his head, drenching his
golden brown khakis, as the crowd shouts, ``God is great!''

Colonel Qaddafi's body was shown in later photographs, with bullet holes
apparently fired into his head at what forensic experts said was close
range, raising the possibility that he was executed by anti-Qaddafi
fighters.

The official version of events offered by Libya's new leaders --- that
Colonel Qaddafi was killed in a cross-fire --- did not appear to be
supported by the photographs and videos that streamed over the Internet
all day long, raising questions about the government's control of the
militias in a country that has been divided into competing regions and
factions.

The conflicting accounts about how he was killed seemed to reflect an
instability that could trouble Libya long after the euphoria fades about
the demise of Colonel Qaddafi, who ruled Libya for nearly 42 years and
is the first of the autocrats to be killed in the Arab Spring uprisings.

\includegraphics{https://static01.graylady3jvrrxbe.onion/images/2011/10/21/world/Libya-2-sub/Libya-2-sub-jumbo.jpg?quality=75\&auto=webp\&disable=upscale}

At the same time, the flood of good news for the former rebels prompted
a collective sigh of relief and quieted talk of rivalries, as strangers
congratulated one another in the streets.

For weeks, as the fight for Surt, Colonel Qaddafi's hometown and final
redoubt in the eight-month conflict, reached a bloody climax, NATO
forces and Libyan fighters had watched for an attempt by his armed
loyalists to flee and seek safety elsewhere. Soon after dawn, they did,
leaving urban bunkers in the Mediterranean town and heading west, said a
senior Western official in Europe knowledgeable about NATO's operations
in Libya.

Around 8:30 a.m. local time, a convoy slipped out of a fortified
compound in Surt, the scene of one of the civil war's bloodiest and
longest battles and a city that was on the verge of falling to Colonel
Qaddafi's opponents.

Before the convoy had traveled two miles, NATO officials said, it was
set upon by an American Predator drone and a French warplane. With the
attack the convoy ``was stopped from progressing as it sought to flee
Surt but was not destroyed,'' Defense Minister Gérard Longuet of France
said.

Only two vehicles in the convoy were hit, neither carrying Colonel
Qaddafi, a Western official said. But the rest of the convoy was forced
to detour and scatter. Anti-Qaddafi fighters rapidly descended on the
scene, telling Reuters they saw people fleeing through some nearby woods
and gave pursuit.

A field leader in Surt, who gave his name to Al Jazeera television as
Mohammed al-Laith, said that Colonel Qaddafi fled from a Jeep in the
convoy and dived into a large drainage pipe. After a gun battle backed
by his guards, he emerged. Mr. Laith told Al Jazeera that the former
Libyan leader had a Kalashnikov in one hand, a pistol in the other.

Image

Surt was Colonel Qaddafi's hometown and final redoubt in the eight-month
conflict.Credit...The New York Times

``What's happening?'' he quoted him as asking as he came out.

The video on Al Jazeera shows Colonel Qaddafi wounded, but clearly
alive. The network quoted a fighter saying that he had begged for help.
``Show me mercy!'' he was said to have cried. There was little of that,
in the video at least.

One fighter is seen pulling his hair, and others beat his limp body. Two
fighters interviewed by Al Jazeera said someone had struck his head with
a gun butt.

Omran Shaaban, 21, a Misurata fighter who claimed to have been the
first, along with a friend, to find Colonel Qaddafi, said he was already
wounded in the head and chest and bleeding in the drainage pipe and then
whisked away to an ambulance. Precisely how he died after that, Mr.
Shaaban said, was unclear.

By all accounts, he was then taken in an ambulance to Misurata, a
coastal town to the west that fought perhaps the most ferocious battle
against Colonel Qaddafi's government and whose fighters still celebrate
their reputation for martial prowess.

Holly Pickett, a freelance photojournalist working in Surt, reported in
a Twitter feed that she had seen Colonel Qaddafi's body in an ambulance
headed for Misurata, along with 10 fighters inside with him. It was
unclear from her posts whether he was dead. ``From the side door, I
could see a bare chest with bullet wound and a bloody hand. He was
wearing gold-colored pants,'' she
\href{http://twitter.com/\#!/hollypickett/status/127078287125131264}{said}
in one post.

Within an hour of the news of Colonel Qaddafi's death, Libyans were
celebrating. ``We have been waiting for this moment for a long time,''
Mahmoud Jibril, the prime minister of the Transitional National Council,
the interim government, said. ``Muammar Qaddafi is dead.'' He was
speaking at a news conference in Tripoli. Mahmoud Shammam, the council's
chief spokesman, called it ``the day of real liberation. We were serious
about giving him a fair trial. ~It seems God has some other wish.''

\includegraphics{https://static01.graylady3jvrrxbe.onion/images/2011/10/20/world/video-tc-111020-shadid/video-tc-111020-shadid-videoSmall.jpg}

At least one of Colonel Qaddafi's feared sons, Muatassim, was also
killed on Thursday, Libyan officials said, and there were unconfirmed
reports that another, Seif al-Islam, had been captured or wounded.

The Arab Twittersphere lighted up with gleeful comments, many of them
hinting at a similar fate awaiting other Arab dictators who have sought
to crush popular uprisings --- most notably President Ali Abdullah Saleh
of Yemen and President Bashar al-Assad of Syria. One of them, also
referring to former President Zine el-Abidine Ben Ali of Tunisia and
former President Hosni Mubarak of Egypt, read: ``Ben Ali escaped,
Mubarak is in jail, Qaddafi was killed. Which fate do you prefer, Ali
Abdullah Saleh? You can consult with Bashar.'' Another was more direct:
``Bashar al-Assad, how do you feel today?''

No videos or photos appeared to show Colonel Qaddafi alive after the
ambulance spirited him away from Surt, though there was a debate over
who exactly was responsible for his death. NATO never claimed the
airstrike killed him, and some officials of the Transitional National
Council made clear he died at their own hands.

A reporter accompanying Ali Tarhouni, the interim government's oil and
finance minister, who visited Misurata to view the body, saw Colonel
Qaddafi splayed out on a mattress in the reception room of a private
home, shirtless, with bullet wounds in the chest and temple and blood on
his arms and hair. Three medical officials arrived, presumably to
conduct more forensic tests. News agencies quoted a spokesman for the
council in Benghazi as saying a doctor had examined Colonel Qaddafi's
corpse in Misurata and found he had been shot in the head and abdomen.
The shot to the head was visible in photos that followed.

A remarkable feature of the Arab revolts is the degree to which almost
every incident is documented, usually by cellphone camera images. They
are almost instantly fed to the Internet and satellite channels, or
ferried by e-mail.

A flurry of images followed Colonel Qaddafi's death. In one, broadcast
by Al Jazeera, his body is half-naked, bleeding on the pavement. Even
more dramatic is a video posted on
\href{http://www.youtube.com/watch?v=KEPnIKI3Ivg\&feature=youtu.be\&t=41s\&skipcontrinter=1}{YouTube}.
Celebrating fighters surround his corpse, which appears to have been
washed. Clearly visible is a gunshot wound to his forehead.

\href{https://www.nytimes3xbfgragh.onion/slideshow/2011/10/20/world/africa/20111021-LIBYA.html}{}

\hypertarget{battle-for-libya--oct-20-2011}{%
\subsection{Battle for Libya \textbar{} Oct. 20,
2011}\label{battle-for-libya--oct-20-2011}}

14 Photos

View Slide Show ›

Thaier Al-Sudani/Reuters

A forensic pathologist in New York, Dr. Michael Baden, said in observing
the photos that there were as many as two bullet wounds and possibly
four in Colonel Qaddafi's head. From what he saw, he believed the shots
were fired at fairly close range.

``It looks more like an execution than something that happened during a
struggle,'' said Dr. Baden, a former New York City medical examiner.
``Two pretty identical-looking wounds like that would have been hard to
do from a distance.''

Late into the night, Libyans celebrated Colonel Qaddafi's death, as did
some elsewhere in the Arab world, seeing it as a lesson to autocrats in
Yemen and Syria. ``It is a historic moment,'' said Abdel Hafez Ghoga, a
spokesman for the Transitional National Council. ``It is the end of
tyranny and dictatorship. Qaddafi has met his fate.''

Western leaders who helped the anti-Qaddafi fighters throughout the
conflict also hailed Colonel Qaddafi's demise.

``We can definitely say that the Qaddafi regime has come to an end,''
President Obama said. ``The dark shadow of tyranny has been lifted, and
with this enormous promise the Libyan people now have a great
responsibility to build an inclusive and tolerant and democratic Libya
that stands as the ultimate rebuke to Qaddafi's dictatorship.''

But occasionally voiced in the Middle East was unease at the violence of
the moment, the fact that a bloody revolution ended with yet more
bloodshed. ``It's not acceptable to kill a person without trying him,''
said Louay Hussein, a Syrian opposition figure in Damascus. ``I prefer
to see the tyrant behind bars.''

Advertisement

\protect\hyperlink{after-bottom}{Continue reading the main story}

\hypertarget{site-index}{%
\subsection{Site Index}\label{site-index}}

\hypertarget{site-information-navigation}{%
\subsection{Site Information
Navigation}\label{site-information-navigation}}

\begin{itemize}
\tightlist
\item
  \href{https://help.nytimes3xbfgragh.onion/hc/en-us/articles/115014792127-Copyright-notice}{©~2020~The
  New York Times Company}
\end{itemize}

\begin{itemize}
\tightlist
\item
  \href{https://www.nytco.com/}{NYTCo}
\item
  \href{https://help.nytimes3xbfgragh.onion/hc/en-us/articles/115015385887-Contact-Us}{Contact
  Us}
\item
  \href{https://www.nytco.com/careers/}{Work with us}
\item
  \href{https://nytmediakit.com/}{Advertise}
\item
  \href{http://www.tbrandstudio.com/}{T Brand Studio}
\item
  \href{https://www.nytimes3xbfgragh.onion/privacy/cookie-policy\#how-do-i-manage-trackers}{Your
  Ad Choices}
\item
  \href{https://www.nytimes3xbfgragh.onion/privacy}{Privacy}
\item
  \href{https://help.nytimes3xbfgragh.onion/hc/en-us/articles/115014893428-Terms-of-service}{Terms
  of Service}
\item
  \href{https://help.nytimes3xbfgragh.onion/hc/en-us/articles/115014893968-Terms-of-sale}{Terms
  of Sale}
\item
  \href{https://spiderbites.nytimes3xbfgragh.onion}{Site Map}
\item
  \href{https://help.nytimes3xbfgragh.onion/hc/en-us}{Help}
\item
  \href{https://www.nytimes3xbfgragh.onion/subscription?campaignId=37WXW}{Subscriptions}
\end{itemize}
