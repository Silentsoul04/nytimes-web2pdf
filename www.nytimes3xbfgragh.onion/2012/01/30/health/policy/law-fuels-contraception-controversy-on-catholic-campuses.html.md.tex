Sections

SEARCH

\protect\hyperlink{site-content}{Skip to
content}\protect\hyperlink{site-index}{Skip to site index}

\href{https://www.nytimes3xbfgragh.onion/section/health/policy}{Money \&
Policy}

\href{https://myaccount.nytimes3xbfgragh.onion/auth/login?response_type=cookie\&client_id=vi}{}

\href{https://www.nytimes3xbfgragh.onion/section/todayspaper}{Today's
Paper}

\href{/section/health/policy}{Money \& Policy}\textbar{}Ruling on
Contraception Draws Battle Lines at Catholic Colleges

\begin{itemize}
\item
\item
\item
\item
\item
\item
\end{itemize}

Advertisement

\protect\hyperlink{after-top}{Continue reading the main story}

Supported by

\protect\hyperlink{after-sponsor}{Continue reading the main story}

Political Science \textbar{} Women and Faith

\hypertarget{ruling-on-contraception-draws-battle-lines-at-catholic-colleges}{%
\section{Ruling on Contraception Draws Battle Lines at Catholic
Colleges}\label{ruling-on-contraception-draws-battle-lines-at-catholic-colleges}}

By \href{https://www.nytimes3xbfgragh.onion/by/denise-grady}{Denise
Grady}

\begin{itemize}
\item
  Jan. 29, 2012
\item
  \begin{itemize}
  \item
  \item
  \item
  \item
  \item
  \item
  \end{itemize}
\end{itemize}

Bridgette Dunlap, a Fordham University law student, knew that the
school's health plan had to pay for birth control pills, in keeping with
New York state law. What she did not find out until she was in an
examining room, ``in the paper dress,'' was that the student health
service --- in keeping with Roman Catholic tenets --- would simply
refuse to prescribe them.

As a result, students have had to go to Planned Parenthood or private
doctors to get prescriptions. Some, unable to afford the doctor visits,
gave up birth control pills entirely. In November, Ms. Dunlap, 31, who
was raised a Catholic and was educated at parochial schools, organized a
one-day, off-campus clinic staffed by volunteer doctors who wrote
prescriptions for dozens of women.

Many Catholic colleges decline to prescribe or cover birth control,
citing religious reasons. Now they are under pressure to change. This
month the Obama administration, citing the medical case for birth
control, made a politically charged decision that the new health care
law requires insurance plans at Catholic institutions to cover birth
control without co-payments for employees, and that may be extended to
students. But Catholic organizations are resisting the rule, saying it
would force them to violate their beliefs and finance behavior that
betrays Catholic teachings.

``We can't just lie down and die and let religious freedom go,'' said
Sister Mary Ann Walsh, a spokeswoman for the United States Conference of
Catholic Bishops.

The administration's rule has now run headlong into a dispute over
values as Republican presidential contenders compete for the most
conservative voters. In an election season that features Newt Gingrich
and Rick Santorum, who have stressed their Catholic faith, scientific
thinking on the medical benefits of birth control has clashed with
deeply held religious and cultural beliefs.

The Obama administration relied on the recommendations of the Institute
of Medicine, an independent group of doctors and researchers that
concluded that birth control is not just a convenience but is medically
necessary ``to ensure women's health and well-being.''

About half of all pregnancies in the United States are unplanned, and
about 4 of 10 of those end in abortion, according to the Institute of
Medicine report, which was released in July. It noted that providing
birth control could lower both pregnancy and abortion rates. It also
cited studies showing that women with unintended pregnancies are more
likely to be depressed and to smoke, drink and delay or skip prenatal
care, potentially harming fetuses and putting babies at increased risk
of being born prematurely and having low birth weight.

But the Republican candidates have said that moral and religious values
weigh heavily in birth control issues. Andrea Saul, a spokeswoman for
Mitt Romney, said in an e-mail that he regarded the administration's
rule requiring religious employers to furnish birth control as wrong.
``This is a direct attack on religious liberty and will not stand in a
Romney presidency,'' she said. Mr. Romney has also pledged to end a
federal program, Title X, that provides family planning services to
millions of women.

Mr. Santorum has taken the position that health insurance plans should
not be required to cover birth control. He also favors allowing states
to decide whether to ban birth control. He and Mr. Gingrich both support
``personhood'' initiatives that would legally declare fertilized eggs to
be persons, effectively banning not just all abortions but also certain
contraceptives, including IUDs and some types of birth control pills.

Mr. Gingrich wants to withdraw government money from Planned Parenthood
because it performs abortions in addition to providing contraceptives,
though the federal money cannot be used for abortion.

The Obama administration has itself not been consistent in following
experts' advice on birth control. In December, it overruled scientists
at the Food and Drug Administration and blocked increased access to an
emergency contraceptive, citing potential risks to young girls who might
use them without parental help. The decision was widely seen as an
effort to avoid the ire of socially conservative voters and to defuse
anger about its pending rule requiring the provision of birth control in
insurance plans of Catholic institutions.

The Catholic Church considers it morally wrong to prevent conception by
any artificial means, including condoms, IUDs, birth control pills and
sterilization.

Some Catholic colleges are likely to ask for a yearlong delay in
implementing the rule on birth control coverage, said Michael
Galligan-Stierle, president of the Association of Catholic Colleges and
Universities. In the longer run, he predicted in a statement that either
Congress or the Supreme Court would invalidate the rule. Belmont Abbey
College, which is Catholic, and the interdenominational Colorado
Christian University have already sued the Department of Health and
Human Services, arguing that the birth control requirement violates the
right to freedom of religion.

Birth control is considered a ``preventive service'' under the new
health care law, but Mr. Galligan-Stierle said such services should be
limited to preventing disease, not pregnancy.

\includegraphics{https://static01.graylady3jvrrxbe.onion/images/2012/01/30/us/CONTRACEPTION/CONTRACEPTION-jumbo.jpg?quality=75\&auto=webp\&disable=upscale}

``We do not happen to think pregnancy is disease,'' he said. ``We think
it's a gift of love of two people and our creator.''

Despite Catholic teachings, surveys have found that 98 percent of
sexually active Catholic women, as in the general population, have used
contraceptives.

At Catholic universities, some students support the right of the schools
to uphold religious doctrine. But others, particularly professional and
graduate students, have found the restrictions on birth control coverage
onerous. Undergraduates are often covered by their parents' insurance,
but graduate students are usually on their own and are more likely to be
married or in relationships and in regular need of birth control.

At some schools, students say the rules are so stringent they have a
hard time getting coverage even if they need birth control pills for
strictly medical reasons.

One recent Georgetown law graduate, who asked not to be identified for
reasons of medical privacy, said she had polycystic ovary syndrome, a
condition for which her doctor prescribed birth control pills. She is
gay and had no other reason to take the pills. Georgetown does not cover
birth control for students, so she made sure her doctor noted the
diagnosis on her prescription. Even so, coverage was denied several
times. She finally gave up and paid out of pocket, more than \$100 a
month. After a few months she could no longer afford the pills. Within
months she developed a large ovarian cyst that had to be removed
surgically --- along with her ovary.

``If I want children, I'll need a fertility specialist because I have
only one working ovary,'' she said.

A spokeswoman for Georgetown, Stacy Kerr, said that problems like this
were rare and that doctors at the health service knew how to help
students get coverage for contraceptives needed for medical reasons.

Asked if Georgetown would begin covering birth control under the new
rule, she said, ``We will be reviewing and evaluating the new
regulations, ever mindful of our Catholic and Jesuit identity and
mission.''

Some Georgetown professors question the wisdom of the university's
current policy. ``I wish Catholic institutions would have more open
conversation about how bans on birth control can increase abortion rates
among students,'' said Robin L. West, a law school professor. ``Both are
contrary to Catholic teaching, but abortion as I understand it is the
graver of the sins, and certainly the greater injury to the fetus and
the woman.''

The university declined to comment on her remarks.

A 23-year-old who asked that her name not be used said she became
pregnant while studying at Fordham. In high school, she said, she had
taken birth control pills, but she gave them up at Fordham because she
could not afford the doctor visit needed for a prescription. She and her
boyfriend were using condoms when she became pregnant. Though Catholic,
she considered abortion, but chose to have the baby. She said she knew
six other Fordham students who had become pregnant and had abortions.

Senior Catholic officials said that students at Catholic universities
should know what to expect, and that those who disagree with the
policies can choose to go elsewhere. ``No one would go to a Jewish
barbecue and expect pork chops to be served,'' Mr. Galligan-Stierle
said.

At Fordham Law School on Tuesday night, Ms. Dunlap and five other law
students who had worked against the university's birth control policy
sat together at a lecture by Archbishop Timothy M. Dolan, who had
personally asked President Obama to exclude Catholic institutions from
the contraception requirement and called the decision against the church
``unconscionable.''

During his lecture, Archbishop Dolan criticized people who postponed
conception with ``chemicals and latex,'' calling them part of the
``culture of death.''

Ms. Dunlap and her colleagues were feeling proud: they had just won a
small victory, persuading Fordham to change its Web site to explain the
birth control policy more clearly. Now, they wrote down questions on
index cards, expecting them to be put to the archbishop after his
speech. One concerned contraception.

The moderator read through the questions and deemed some of them too
``pointed.''

``If I don't ask your question,'' he said, ``I either apologize or I
don't care.''

Ms. Dunlap's queries did not make the cut. Her frustration nearly
brought her to tears.

``I can't believe they didn't take our questions,'' she said, adding
that the moderator was trying to silence disagreement. ``It dishonors
the law school.''

Advertisement

\protect\hyperlink{after-bottom}{Continue reading the main story}

\hypertarget{site-index}{%
\subsection{Site Index}\label{site-index}}

\hypertarget{site-information-navigation}{%
\subsection{Site Information
Navigation}\label{site-information-navigation}}

\begin{itemize}
\tightlist
\item
  \href{https://help.nytimes3xbfgragh.onion/hc/en-us/articles/115014792127-Copyright-notice}{©~2020~The
  New York Times Company}
\end{itemize}

\begin{itemize}
\tightlist
\item
  \href{https://www.nytco.com/}{NYTCo}
\item
  \href{https://help.nytimes3xbfgragh.onion/hc/en-us/articles/115015385887-Contact-Us}{Contact
  Us}
\item
  \href{https://www.nytco.com/careers/}{Work with us}
\item
  \href{https://nytmediakit.com/}{Advertise}
\item
  \href{http://www.tbrandstudio.com/}{T Brand Studio}
\item
  \href{https://www.nytimes3xbfgragh.onion/privacy/cookie-policy\#how-do-i-manage-trackers}{Your
  Ad Choices}
\item
  \href{https://www.nytimes3xbfgragh.onion/privacy}{Privacy}
\item
  \href{https://help.nytimes3xbfgragh.onion/hc/en-us/articles/115014893428-Terms-of-service}{Terms
  of Service}
\item
  \href{https://help.nytimes3xbfgragh.onion/hc/en-us/articles/115014893968-Terms-of-sale}{Terms
  of Sale}
\item
  \href{https://spiderbites.nytimes3xbfgragh.onion}{Site Map}
\item
  \href{https://help.nytimes3xbfgragh.onion/hc/en-us}{Help}
\item
  \href{https://www.nytimes3xbfgragh.onion/subscription?campaignId=37WXW}{Subscriptions}
\end{itemize}
