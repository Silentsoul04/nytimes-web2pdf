Sections

SEARCH

\protect\hyperlink{site-content}{Skip to
content}\protect\hyperlink{site-index}{Skip to site index}

\href{https://www.nytimes3xbfgragh.onion/section/arts/design}{Art \&
Design}

\href{https://myaccount.nytimes3xbfgragh.onion/auth/login?response_type=cookie\&client_id=vi}{}

\href{https://www.nytimes3xbfgragh.onion/section/todayspaper}{Today's
Paper}

\href{/section/arts/design}{Art \& Design}\textbar{}Mike Kelley, an
Artist With Attitude, Dies at 57

\begin{itemize}
\item
\item
\item
\item
\item
\end{itemize}

Advertisement

\protect\hyperlink{after-top}{Continue reading the main story}

Supported by

\protect\hyperlink{after-sponsor}{Continue reading the main story}

\hypertarget{mike-kelley-an-artist-with-attitude-dies-at-57}{%
\section{Mike Kelley, an Artist With Attitude, Dies at
57}\label{mike-kelley-an-artist-with-attitude-dies-at-57}}

By \href{https://www.nytimes3xbfgragh.onion/by/holland-cotter}{Holland
Cotter}

\begin{itemize}
\item
  Feb. 1, 2012
\item
  \begin{itemize}
  \item
  \item
  \item
  \item
  \item
  \end{itemize}
\end{itemize}

Mike Kelley, one of the most influential American artists of the past
quarter century~and a pungent commentator on American class, popular
culture and youthful rebellion, was found dead on Wednesday at his home
in South Pasadena, Calif. He was 57.

Sgt. Robert Bartl of the South Pasadena police said it appeared that Mr.
Kelley had committed suicide. Speaking to The Associated Press, he said
a friend of Mr. Kelley's had told investigators that Mr. Kelley had been
depressed after breaking up with a girlfriend.

An autopsy was to be performed, Sergeant Bartl said.

Mr. Kelley was born in Wayne, Mich., a suburb of Detroit, to a working
class Roman Catholic family in October 1954. His father was in charge of
maintenance for a public school system; his mother was a cook in the
executive dining room at Ford Motor Company. He had early aspirations to
be a novelist, but doubted his talent and found writing was too
difficult, so he turned his energies to art, through painting,
object-making and through music.

Image

Mike Kelley in 2009.Credit...Hiroko Masuike/The New York Times

In high school he immersed himself in Detroit's heavy metal music
subculture, and that involvement continued through college at the
University of Michigan in Ann Arbor. There he performed in a proto-punk
noise band called
\href{http://en.wikipedia.org/wiki/Destroy_All_Monsters_\%28band\%29}{Destroy
All Monsters} with three other artists, Jim Shaw, Niagara and Carey
Loren, creating work that, with its combination of anti-establishment
politics and Dada theatrics, had close connections to performance art.

He brought this interest with him to graduate school in 1978 at the
California Institute of the Arts in Valencia, Calif. There he formed a
second art-band, ~``The Poetics,'' with fellow students John Miller and
Tony Oursler. He absorbed, with some resistance, the school's overriding
focus on Conceptual Art and theory, eased into by the embracing approach
of teachers like John Baldessari, Laurie Anderson and Douglas Huebler.

He began creating multimedia installations that synthesized large-scale
drawings and paintings, often incorporating his own writing, along with
sculptures, videos (one was based on the television show ``Captain
Kangaroo''), and performances, often scatological and sadomasochistic in
nature. Although he stopped performing in 1986 --- he later said that he
always had to get drunk to do it --- the other formal elements remained
constants in his art.

A certain tone or attitude remained constant, too. The shorthand term
for it is abjection, a deliberate immersion in the gross-out anarchy
associated with youth culture. But to see only that was to miss the deep
and covered-up strain of poetry in his work,~evident in a series of
sculptural pieces using children's stuffed animals sewn onto or covered
over with hand-knitted afghans.

\includegraphics{https://static01.graylady3jvrrxbe.onion/images/2012/02/02/arts/altKELLEY2-obit/altKELLEY2-obit-jumbo.jpg?quality=75\&auto=webp\&disable=upscale}

On one level, the pieces were sardonic send-ups of aesthetic trends like
Minimalism, which Mr. Kelley despised as elitist. On another, they took
aim at the strain of too-easy sentimentality he found repellent in
popular culture. At yet another level, these pieces, with their martyred
dolls and ruined promise of warmth, were innocence-and-experience
metaphors, suggesting the trauma of hurt and loss that underlay the
juvenile delinquent antics that surrounded them.

By the mid-1980s, he was already gaining attention nationally and
internationally. His career took off earlier in Europe than it did in
the United States; he found enthusiastic audiences in France and
Germany, at a time when Americans still didn't know quite what to do
with him, this artist who made drawings of garbage, parodied both
religious art and underground politics, and made pieces with titles like
``Plato's Cave, Rothko's Chapel, Lincoln's Profile.'' Mystifying as they
were at the time, they have given inspiration to countless young artists
since.

The band Sonic Youth used Mr. Kelley's work on the album cover for
``Dirty,'' released in 1992.

Mr. Kelley began having regular one-man exhibitions at Metro Pictures in
Manhattan in 1982, and at Rosamund Felsen Gallery in Los Angeles the
following year. In 2005, he had his first solo show at Gagosian gallery
in New York City, which was representing him at his death. ~A
retrospective, ~``Mike Kelley: Catholic Tastes,'' appeared at the
Whitney Museum of American Art in 1993 and traveled to Los Angeles and
Munich; a second retrospective appeared at the Museum of Contemporary
Art in Barcelona in 1997; and a third was at the Tate Liverpool in 2004.

Work by Mr. Kelley will be in the upcoming Whitney Biennial; it will be
his eighth appearance in that show.

Mr. Kelley is survived by a brother, George.

Advertisement

\protect\hyperlink{after-bottom}{Continue reading the main story}

\hypertarget{site-index}{%
\subsection{Site Index}\label{site-index}}

\hypertarget{site-information-navigation}{%
\subsection{Site Information
Navigation}\label{site-information-navigation}}

\begin{itemize}
\tightlist
\item
  \href{https://help.nytimes3xbfgragh.onion/hc/en-us/articles/115014792127-Copyright-notice}{©~2020~The
  New York Times Company}
\end{itemize}

\begin{itemize}
\tightlist
\item
  \href{https://www.nytco.com/}{NYTCo}
\item
  \href{https://help.nytimes3xbfgragh.onion/hc/en-us/articles/115015385887-Contact-Us}{Contact
  Us}
\item
  \href{https://www.nytco.com/careers/}{Work with us}
\item
  \href{https://nytmediakit.com/}{Advertise}
\item
  \href{http://www.tbrandstudio.com/}{T Brand Studio}
\item
  \href{https://www.nytimes3xbfgragh.onion/privacy/cookie-policy\#how-do-i-manage-trackers}{Your
  Ad Choices}
\item
  \href{https://www.nytimes3xbfgragh.onion/privacy}{Privacy}
\item
  \href{https://help.nytimes3xbfgragh.onion/hc/en-us/articles/115014893428-Terms-of-service}{Terms
  of Service}
\item
  \href{https://help.nytimes3xbfgragh.onion/hc/en-us/articles/115014893968-Terms-of-sale}{Terms
  of Sale}
\item
  \href{https://spiderbites.nytimes3xbfgragh.onion}{Site Map}
\item
  \href{https://help.nytimes3xbfgragh.onion/hc/en-us}{Help}
\item
  \href{https://www.nytimes3xbfgragh.onion/subscription?campaignId=37WXW}{Subscriptions}
\end{itemize}
