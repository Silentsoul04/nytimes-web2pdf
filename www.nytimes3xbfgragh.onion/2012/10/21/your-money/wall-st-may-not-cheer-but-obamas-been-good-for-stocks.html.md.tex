Sections

SEARCH

\protect\hyperlink{site-content}{Skip to
content}\protect\hyperlink{site-index}{Skip to site index}

\href{https://www.nytimes3xbfgragh.onion/section/your-money}{Your Money}

\href{https://myaccount.nytimes3xbfgragh.onion/auth/login?response_type=cookie\&client_id=vi}{}

\href{https://www.nytimes3xbfgragh.onion/section/todayspaper}{Today's
Paper}

\href{/section/your-money}{Your Money}\textbar{}Wall St. May Not Cheer,
but Obama's Been Good for Stocks

\url{https://nyti.ms/WxiasM}

\begin{itemize}
\item
\item
\item
\item
\item
\end{itemize}

Advertisement

\protect\hyperlink{after-top}{Continue reading the main story}

Supported by

\protect\hyperlink{after-sponsor}{Continue reading the main story}

\href{/column/business-strategies}{Strategies}

\hypertarget{wall-st-may-not-cheer-but-obamas-been-good-for-stocks}{%
\section{Wall St. May Not Cheer, but Obama's Been Good for
Stocks}\label{wall-st-may-not-cheer-but-obamas-been-good-for-stocks}}

By \href{https://www.nytimes3xbfgragh.onion/by/jeff-sommer}{Jeff Sommer}

\begin{itemize}
\item
  Oct. 20, 2012
\item
  \begin{itemize}
  \item
  \item
  \item
  \item
  \item
  \end{itemize}
\end{itemize}

``ARE you better off than you were four years ago?''

Ronald Reagan asked that memorable question at the end of
\href{http://www.youtube.com/watch?v=kXFEh4cdCog\&feature=related}{the
presidential debate of October 1980}, one week before Election Day.

Millions of voters answered no --- sweeping President Jimmy Carter from
office and installing the Republican Party in the White House for the
next 12 years.

In the presidential debate last Tuesday on Long Island, and through much
of this campaign season, Mitt Romney has been raising a similar
question. He said last week that middle-class incomes have declined
since President Obama took office, while gasoline prices have risen. And
much as Mr. Reagan did several decades ago, Mr. Romney suggested that
Americans are worse off today than they were four years earlier.

By some objective measures, we \emph{are} worse off --- although that is
at least partly because a severe recession was well under way and the
economy was shrinking when Mr. Obama first walked into the White House
as president. Based primarily on the grim fundamental economic data,
several forecasting models have been showing for months that Mr. Obama
faced a very difficult road to re-election.

Yet for the most part, he has been holding his own in the polls and the
prediction markets. At the moment, he is widely considered to have an
advantage in the all-important Electoral College.

But why? Elections, of course, are not decided on economics alone.
``Campaigns matter, at least a little,'' said
\href{http://www4.uwm.edu/letsci/polisci/faculty/holbrook.cfm}{Thomas M.
Holbrook, professor of political science} at the University of
Wisconsin, Milwaukee. ``The issues and the candidates matter, at least a
little.'' Yet to the extent that the economy determines the election's
outcome, it's possible that the stock market holds part of the
explanation for Mr. Obama's outsize electoral strength.

Through Friday, since Mr. Obama's inauguration --- his first 1,368 days
in office --- the Dow Jones industrial average has gained 67.9 percent.
That's an extremely strong performance --- the fifth best for an
equivalent period among all American presidents since 1900. The
\href{http://www.bespokeinvest.com/}{Bespoke Investment Group}
calculated those returns for The New York Times.

The best showing occurred in Franklin D. Roosevelt's first term, when
the market rose by a whopping 238.1 percent. Of course, that followed a
calamitous decline. When his term started, the Dow had fallen to
one-fourth of its former peak. In 2008, the year before Mr. Obama took
office, the Dow declined by roughly one-third.

After Franklin Roosevelt, the next-best market performances occurred
under Calvin Coolidge, Bill Clinton and Dwight Eisenhower. These
exceptionally strong markets helped all of them win the next election
--- and the stock market is undoubtedly helping Mr. Obama, too, even if
he isn't saying much about it.

``What I find most ironic about these numbers is that one of Obama's
weaknesses is said to be his economic record,'' said Paul T. Hickey,
Bespoke's co-founder. ``While the stock market isn't a complete
reflection of the economy, it is an important indicator, and the stock
market is one of the great things the president has working in his
favor. But it's a sensitive subject. With Wall Street so despised by the
average American right now, it's probably something he doesn't want to
be too quick to trumpet. But facts are facts.''

The market's relationship to the economy and to the political system is
complex and varying from era to era. But a simple truth is that the
market's rise and fall has an enormous effect on the wealth of ordinary
Americans --- and on whether they \emph{feel} themselves to be wealthy.

Some market effects on wealth are straightforward enough. With the
relative decline of traditional pensions, and the rise of
defined-contribution plans like 401(k)'s, the well-being of middle-class
households is intimately connected to the stock market's fortunes. That
was shockingly evident in 2008, when the Dow declined by more than 33
percent. That market fall, on top of a sharp drop in housing prices,
contributed to a 19 percent decline in the total net worth of American
households that year, according to figures compiled by the Federal
Reserve.

Relatively low housing prices continue to depress aggregate wealth
levels, the Fed's Flow of Funds report shows. But when stock market
values are factored in, the rising value of financial assets ---
including stocks --- has restored the total wealth of American
households. Using this measure, it is now higher than when Mr. Obama's
presidency began,
\href{http://www.federalreserve.gov/releases/z1/Current/z1.pdf}{Fed
figures} indicate.

Unfortunately for millions of Americans, that can't be said for median
household income, which was only \$51,023 last year, down from \$53,206
in 2009, according to Census Bureau figures adjusted for inflation.
While it appears to have risen in 2012, median household income almost
certainly has not returned to its former level. And
\href{http://elections.nytimes3xbfgragh.onion/2012/debates/presidential/2012-10-16\#fact-checks}{rising
income inequality} means that the hardship has been felt
disproportionately by poor people, who face a double-whammy: they don't
have financial assets that have grown in value.

Presidential election forecasting models using ``objective data'' like
personal income, gross domestic product and unemployment have captured
the gloomy realities of the economy, painting a relatively bleak picture
of Mr. Obama's electoral chances. But
\href{http://www.columbia.edu/\&}{Robert S. Erikson, a professor of
political science at Columbia University}, said, ``If you include
subjective data --- emphasizing expectations about the economy --- you
get a very different forecast.''
\href{http://papers.ssrn.com/sol3/papers.cfm?abstract_id=2110791}{In a
paper}, Professor Erikson and
\href{http://www.cla.temple.edu/politicalscience/faculty/wlezien.shtml}{Christopher
Wlezien of Temple University} described a forecasting model using
consumer sentiment measures and leading economic indicators. That model
predicts an Obama victory ``quite strongly,'' Professor Erikson said in
an interview.

The performance of the stock market, at least to some extent, is a
subjective rendering of market participants' expectations for the
corporate economy. It's a complicated subject, partly because market
prices tend to rise when interest rates are low, as they are now, thanks
largely to efforts of the Fed and other central banks.

Do these relatively high prices really reflect the future outlook for
the economy? It's not certain. But the stock market is an important
component of many measures of leading economic indicators, and it is
clearly painting a rosier picture than the retrospective headline
numbers about the economy --- like the 1.3 percent annual growth rate of
\href{http://www.bea.gov/newsreleases/glance.htm}{G.D.P. in the second
quarter} or the 7.8 percent unemployment rate in September.

IT'S often said that Wall Street prefers Mitt Romney to Mr. Obama, Mr.
Hickey observed, yet the stock market has flourished under the president
--- and under Democratic presidents generally. Since 1900, it has
returned 7.1 percent annually when Democrats have occupied the White
House, and only 3 percent under Republicans.

Mr. Carter was an exception. The market declined 0.3 percent in his
first 1,368 days in office --- a performance more than 68 percentage
points worse than that for Mr. Obama.

Are you better off than you were four years ago? For stock portfolios,
at least, the last four years have been bumpy but they haven't been bad
at all.

Advertisement

\protect\hyperlink{after-bottom}{Continue reading the main story}

\hypertarget{site-index}{%
\subsection{Site Index}\label{site-index}}

\hypertarget{site-information-navigation}{%
\subsection{Site Information
Navigation}\label{site-information-navigation}}

\begin{itemize}
\tightlist
\item
  \href{https://help.nytimes3xbfgragh.onion/hc/en-us/articles/115014792127-Copyright-notice}{©~2020~The
  New York Times Company}
\end{itemize}

\begin{itemize}
\tightlist
\item
  \href{https://www.nytco.com/}{NYTCo}
\item
  \href{https://help.nytimes3xbfgragh.onion/hc/en-us/articles/115015385887-Contact-Us}{Contact
  Us}
\item
  \href{https://www.nytco.com/careers/}{Work with us}
\item
  \href{https://nytmediakit.com/}{Advertise}
\item
  \href{http://www.tbrandstudio.com/}{T Brand Studio}
\item
  \href{https://www.nytimes3xbfgragh.onion/privacy/cookie-policy\#how-do-i-manage-trackers}{Your
  Ad Choices}
\item
  \href{https://www.nytimes3xbfgragh.onion/privacy}{Privacy}
\item
  \href{https://help.nytimes3xbfgragh.onion/hc/en-us/articles/115014893428-Terms-of-service}{Terms
  of Service}
\item
  \href{https://help.nytimes3xbfgragh.onion/hc/en-us/articles/115014893968-Terms-of-sale}{Terms
  of Sale}
\item
  \href{https://spiderbites.nytimes3xbfgragh.onion}{Site Map}
\item
  \href{https://help.nytimes3xbfgragh.onion/hc/en-us}{Help}
\item
  \href{https://www.nytimes3xbfgragh.onion/subscription?campaignId=37WXW}{Subscriptions}
\end{itemize}
