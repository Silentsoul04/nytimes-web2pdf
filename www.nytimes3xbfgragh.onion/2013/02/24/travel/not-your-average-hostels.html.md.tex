Sections

SEARCH

\protect\hyperlink{site-content}{Skip to
content}\protect\hyperlink{site-index}{Skip to site index}

\href{https://www.nytimes3xbfgragh.onion/section/travel}{Travel}

\href{https://myaccount.nytimes3xbfgragh.onion/auth/login?response_type=cookie\&client_id=vi}{}

\href{https://www.nytimes3xbfgragh.onion/section/todayspaper}{Today's
Paper}

\href{/section/travel}{Travel}\textbar{}This Is a Hostel?

\url{https://nyti.ms/W85ry0}

\begin{itemize}
\item
\item
\item
\item
\item
\end{itemize}

Advertisement

\protect\hyperlink{after-top}{Continue reading the main story}

Supported by

\protect\hyperlink{after-sponsor}{Continue reading the main story}

Update

\hypertarget{this-is-a-hostel}{%
\section{This Is a Hostel?}\label{this-is-a-hostel}}

By Christine Ajudua

\begin{itemize}
\item
  Feb. 22, 2013
\item
  \begin{itemize}
  \item
  \item
  \item
  \item
  \item
  \end{itemize}
\end{itemize}

Grupo Habita, the Mexico City-based team behind the upscale Hôtel
Americano in New York, is known for taking an unconventional approach to
hospitality. Still, its latest property --- Downtown Mexico, a hotel
within a 17th-century palace in the Centro Histórico district of Mexico
City --- houses a bit of a surprise: the company's first hostel,
Downtown Beds.

Downtown Beds occupies the palace's former service quarters. ``The space
had the bones for a youthful project,'' said Carlos Couturier, managing
partner at Grupo Habita, which created an upscale hotel at the opposite
end of the building. ``There was an intimate patio and a rooftop that
could be transformed into something cool.'' The local architecture firm
Cherem Serrano kept the original Catalan vaulted ceilings, painted the
wooden floors white and installed up to eight bespoke lattice-brick
bunks in each room, as well as en-suite bathrooms with rain showers. The
patio is now a ``chela'' garden (that's slang for beer) and the rooftop
has a swimming pool and bar that draws a steady stream of locals.
There's also a kitchen serving Mexican street snacks, a screening room,
table tennis, foosball and free use of bikes. ~

``People don't come to Downtown Beds because it's cheap; we have had
guests pay with Amex black cards,'' said Mr. Couturier, whose company
also plans to open a hostel in Mazunte, Oaxaca, in two years. ``They
come because it's fun and different.''

Clearly, Downtown Beds is not your traditional hostel, nor could its
guests be defined as typical backpackers. There are no chores required,
no lockout hours or curfews, and linens and toiletries are provided in
each of the 17 rooms, whether private or shared. It is one of the latest
examples of a global, industrywide trend focused on accommodating
design-conscious 20- and 30-somethings who are seeking out the scene
(via Facebook, Twitter, Instagram) for reasons beyond saving a buck.

Image

Downtown Beds in Mexico City.Credit...Undine Prohl

``We're seeing more and more travelers who can afford to stay at hotels,
yet choose to stay at hostels for the social experience,'' said Aaron
Chaffee, director of hostels at Hostelling International USA, who noted
that many modern hostels are offering the same amenities as hotels:
private rooms, concierge service, Wi-Fi, restaurants and bars. And, of
course, stylish interiors.

According to Mr. Chaffee, the trend has its roots in Asia, known for its
capsule hotels, and Europe, largely considered the vanguard of
hosteling. There is, for example, Matchbox, which opened near
Singapore's Chinatown in 2011. It calls itself a ``concept hostel.''
Think breakfast all day (Indian rojak or Malay cookies) and pod-style
bunks with panels that open and close, in case you'd like to chat with
your neighbor.

Meanwhile, outside Munich, the German Youth Hostel Association has
tapped the Laboratory for Visionary Architecture, also known as LAVA, an
eco-conscious local firm, to transform the circa-1930 Berchtesgaden
Youth Hostel. ``We were commissioned to rethink what a hostel could be
in the age of boutique hotels,'' said the LAVA director Tobias
Wallisser. The first section reopened just over a year ago with
natural-wood ``cocoon'' bunks, energy-efficient wood pellet heating and
cantilevered window nooks affording views of the Bavarian Alps, where a
resident outfitter arranges mountain biking and ski trips; the next
phase, with a bistro and lounge, is set to be completed by 2015, along
with LAVA's second hostel, in Bayreuth.

And in Reykjavik, a group of former soccer player and filmmaker friends
recently turned a disused biscuit factory --- originally scouted by the
Icelandic director Oskar Thor Axelsson for his movie ``Black's Game''
--- into Kex Hostel. With a retro barbershop, a gastropub and a music
venue-slash-art gallery that stages events from the likes of Sigur Ros
(or Russell Crowe and Patti Smith, who recently gave an impromptu
performance), it has a cult following among travelers and locals.

Increasingly, the hostel is being reinvented abroad --- in places like
South America. Inspired by his backpacking trips around the world,
28-year-old Guilherme Perez left a career in banking to create the
old-meets-new, minimalist-chic We Hostel Design back home in São Paulo,
Brazil. After finding the location --- a whitewashed, early-20th-century
mansion in the Vila Mariana neighborhood --- Mr. Perez asked the
architect Felipe Hess to design the seven dorm rooms, two private rooms
and multiple common spaces (including a ``glass room'' with wraparound
windows and a low slate roof, where guests are invited to leave their
mark in chalk) with a black-and-white color scheme. The hostel's
creative director, the 26-year-old user-experience designer Rodrigo
Marangoni, embedded the space with smartphone-compatible QR codes,
allowing guests to download everything from subway maps to, say, a song
by the São Paulo rapper Criolo.

\includegraphics{https://static01.graylady3jvrrxbe.onion/images/2013/02/24/travel/24UPDATE5/24UPDATE5-articleLarge.jpg?quality=75\&auto=webp\&disable=upscale}

We Hostel Design is a family-financed project. However, ``investors are
starting to realize that there is money to be made in this business,''
said David Chapman, director general of the Amsterdam-based World Youth
Student and Educational Travel Confederation. According to the latest
international hosteling survey --- which was produced by~STAY WYSE, the
accommodation sector of the World Youth Student and Educational (WYSE)
Travel Confederation, in collaboration with Hostelling International,
Hostelworld.com, HostelBookers.com and others --- the industry is now
valued at \$34 billion, with the global economic downturn acting as a
boon.

That said, the profitability of hostels has gone largely unnoticed ---
until now. ``There's been an influx of high-profile brands in the
market,'' said Mr. Chapman, ``with hostels that are challenging two- and
three-star hotels. The difference between these two options is basically
the letter `s.' It stands for `social.'~''

The United States has been rather resistant to hosteling, though a
number of hoteliers have started experimenting. Last year, in South Lake
Tahoe, Calif., the former Joie de Vivre executive Christian Strobel
opened the adventure-friendly Basecamp with amenities (communal dinners,
fire pits, a rooftop hot tub) meant to encourage guest interaction. Its
50 rooms are all private, but some have bright orange bunk beds and
sleep up to eight. More recently, the ``straight-friendly urban
resort,'' Out NYC, opened eight hostel-style Sleep Share rooms featuring
roomy bunks with personal TVs and privacy curtains; access to the
hotel's nightclub, spa and 24-hour gym is included.

In Florida, the Sydell Group, which runs the Nomad Hotel in New York,
describes its latest property --- Freehand Miami --- as ``the first
upscale hostel in the U.S.'' It opened last December in an Art Deco
building a short stroll from the luxury hotels lining South Beach, with
239 beds divided among 63 rooms.

``We wanted a happy, summer camp vibe,'' said Robin Standefer of Roman
\& Williams, the New York-based design firm responsible for the hostel's
boho-chic interiors (not to mention Manhattan's Ace Hotel). The wooden
bunks have Mexican blankets, bolster pillows, linen shutters and
built-in nightstands with reading lights and power outlets so you can
recharge your smartphone while you sleep. Staff members wear T-shirts by
the emerging American fashion duo Timo Weiland, custom beach cruisers
are available for rent, and the bar, in an overgrown tropical garden
with mismatched vintage furnishings and a swimming pool, is a favorite
among Miami's hipster set. Bartenders mix cocktails (\$11) using herbs
grown on site; there are \$3 Miller High Lifes and bottles of Krug for
\$350 a pop.

Image

Matchbox in Singapore.Credit...Matchbox the Concept Hostel

``People are generally putting up hotels that have some sort of
lifestyle or design element,'' said the Sydell Group founder and chief
executive Andrew Zobler. ``It was only natural that would happen in the
hostel space, but it really hadn't reached the U.S.''

For many travelers, though, a hostel is just a cheap place to crash; in
this country, the word tends to connote an environment akin to its
pronunciation. Set on changing that perception, the Sydell Group plans
to open Freehand locations nationwide (next up: New York). Still,
designer uniforms and Champagne aside, there is something to be said for
the romance of roughing it --- the journey, not the lodging being the
point. If you ask Mr. Chaffee, ``a hostel is only well designed if it
supports what it was designed to do: provide a social space for
travelers to meet up, go out and explore the location and then return to
reflect on the experiences of the day.''

Farryn Weiner, a 27-year-old New Yorker, is one such traveler. Whether
for work (as the global director of digital and social media for Michael
Kors) or pleasure, she is on the road about 10 days a month, stopping
anywhere from a business hotel in Tokyo to a tented camp at Coachella.
Recently, Ms. Weiner booked a bed in one of Freehand's ``quad'' rooms.
``It was like a mix of a hostel and a boutique hotel,'' she said.
``Everyone was sharing tips on where they were going and what they were
doing; I walked away with all sorts of connections. For me, at this
moment in my life, that's worth more than a five-star spa.''

\textbf{THE LIST}

\textbf{Basecamp,} 4143 Cedar Avenue, South Lake Tahoe, Calf.;
\href{http://basecamphotels.com}{basecamphotels.com}; rooms from \$129,
including breakfast.

\textbf{Berchtesgaden Youth Hostel,} 6 Struberberg, Bischofswiesen,
Germany;
\href{http://www.jugendherberge.de/en/hostels/search/portrait/jh.jsp?IDJH=656}{jugendherberge.de};
beds from 18 euros, or about \$24 at \$1.31 to the euro.

Image

Basecamp in South Lake Tahoe, Calif.Credit...Eva Kolenko

\textbf{Downtown Beds,} 30 Isabel la Católica, Mexico City;
\href{http://www.downtownbeds.com}{www.downtownbeds.com}; beds from 200
pesos, or about \$16 at 12.5 pesos to the dollar, and private rooms from
550 pesos, including breakfast.

\textbf{Freehand Miami,} 2727 Indian Creek Drive, Miami Beach, Fla.;
\href{http://thefreehand.com}{thefreehand.com}; beds from \$30, private
rooms from \$150, including breakfast.

\textbf{Kex Hostel,} Skulagata 28, Reykjavik, Iceland;
\href{http://kexhostel.is}{kexhostel.is}; beds from 2,300 kronur, or
about \$18 at 125 kronur to the dollar, and private rooms from 9,400
kronur.

\textbf{Matchbox,} 39 Ann Siang Road, Singapore;
\href{http://matchbox.sg}{matchbox.sg}; beds from 45 Singapore dollars,
or about \$37 at 1.2 Singapore dollars to the U.S. dollar, including
breakfast.

\textbf{The Out NYC,} 510 West 42nd Street, New York;
\href{http://theoutnyc.com}{theoutnyc.com}; beds from \$99, private
rooms from \$259.

\textbf{We Hostel Design,} Rua Morgado de Mateus, 567, São Paulo,
Brazil; \href{http://wehostel.com.br}{wehostel.com.br}; beds from 42
Brazilian reais, or about \$22 at 1.9 reais to the dollar, and private
rooms from 150 reais, including breakfast.

Advertisement

\protect\hyperlink{after-bottom}{Continue reading the main story}

\hypertarget{site-index}{%
\subsection{Site Index}\label{site-index}}

\hypertarget{site-information-navigation}{%
\subsection{Site Information
Navigation}\label{site-information-navigation}}

\begin{itemize}
\tightlist
\item
  \href{https://help.nytimes3xbfgragh.onion/hc/en-us/articles/115014792127-Copyright-notice}{©~2020~The
  New York Times Company}
\end{itemize}

\begin{itemize}
\tightlist
\item
  \href{https://www.nytco.com/}{NYTCo}
\item
  \href{https://help.nytimes3xbfgragh.onion/hc/en-us/articles/115015385887-Contact-Us}{Contact
  Us}
\item
  \href{https://www.nytco.com/careers/}{Work with us}
\item
  \href{https://nytmediakit.com/}{Advertise}
\item
  \href{http://www.tbrandstudio.com/}{T Brand Studio}
\item
  \href{https://www.nytimes3xbfgragh.onion/privacy/cookie-policy\#how-do-i-manage-trackers}{Your
  Ad Choices}
\item
  \href{https://www.nytimes3xbfgragh.onion/privacy}{Privacy}
\item
  \href{https://help.nytimes3xbfgragh.onion/hc/en-us/articles/115014893428-Terms-of-service}{Terms
  of Service}
\item
  \href{https://help.nytimes3xbfgragh.onion/hc/en-us/articles/115014893968-Terms-of-sale}{Terms
  of Sale}
\item
  \href{https://spiderbites.nytimes3xbfgragh.onion}{Site Map}
\item
  \href{https://help.nytimes3xbfgragh.onion/hc/en-us}{Help}
\item
  \href{https://www.nytimes3xbfgragh.onion/subscription?campaignId=37WXW}{Subscriptions}
\end{itemize}
