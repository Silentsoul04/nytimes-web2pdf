Sections

SEARCH

\protect\hyperlink{site-content}{Skip to
content}\protect\hyperlink{site-index}{Skip to site index}

\href{https://www.nytimes3xbfgragh.onion/section/world/europe}{Europe}

\href{https://myaccount.nytimes3xbfgragh.onion/auth/login?response_type=cookie\&client_id=vi}{}

\href{https://www.nytimes3xbfgragh.onion/section/todayspaper}{Today's
Paper}

\href{/section/world/europe}{Europe}\textbar{}Italy's Historic
Multicultural Compromise

\url{https://nyti.ms/1dHePfs}

\begin{itemize}
\item
\item
\item
\item
\item
\end{itemize}

Advertisement

\protect\hyperlink{after-top}{Continue reading the main story}

Supported by

\protect\hyperlink{after-sponsor}{Continue reading the main story}

Letter From Europe

\hypertarget{italys-historic-multicultural-compromise}{%
\section{Italy's Historic Multicultural
Compromise}\label{italys-historic-multicultural-compromise}}

By Celestine Bohlen

\begin{itemize}
\item
  March 24, 2014
\item
  \begin{itemize}
  \item
  \item
  \item
  \item
  \item
  \end{itemize}
\end{itemize}

BRUNICO, Italy --- There's no mistaking the border between the province
of South Tyrol and the rest of Italy. The sign by the highway tells it
all, welcoming visitors in three languages, German first. Italian radio
stations begin to fade out, replaced by the sing-song accent of Tyrolean
German, broadcast locally or from nearby Austria.

These are the sights and sounds of a historic compromise, achieved over
painful decades of war, repression and terrorism. It has taken almost a
century for modern Italy to accommodate a predominantly German-speaking
population in the mountainous border region it annexed from Austria in
1919.

Still, South Tyrol, or Alto Adige as it is known in Italian, has now
settled down into a peaceful, prosperous, bilingual --- even trilingual,
including the local Ladin language --- harmony that some say could be a
model for other multicultural territories in dispute across Europe.

Crimea, wrested this month from Ukraine by Russia, is just another
example. Its history is very different: Its annexation was
overwhelmingly approved by the dominant Russian-speaking population but
denounced by the rest of the world. In contrast, Italy's annexation of
South Tyrol was decided by the victorious nations of World War I, and
reaffirmed at the end of World War II, over the heads and against the
wishes of a population that was 92.2 percent German-speaking in 1910.

Those numbers changed after Italians were deliberately resettled here in
the 1930s, and again in the 1950s. Today, according to the 2011 census,
German speakers make up 61.5 percent of the population, while Italian
speakers account for 23.1 percent, and Ladin speakers 4 percent.

Here in this small, picturesque town, some 25 miles from the Austrian
border, German is unquestionably the default language. Yet everyone also
speaks Italian, conversing fluently with a steady stream of Italian
tourists.

It's a complicated arrangement, but it works. The provincial government,
dominated by the historic German-speaking party, controls virtually all
aspects of public life --- hospitals, schools and roads, although not
the police --- under the terms of an exceptionally generous autonomy
granted by the Italian government in 1972.

Concessions made by all sides explain how the region was able to subdue
competing nationalist resentments. German speakers had to heal wounds
left by a brutal program of forced Italianization carried out under
Mussolini in the 1930s. Italian speakers had to shelve memories of a
wave of terrorism by German-speaking separatists, which led to 361
attacks and 21 deaths between 1956 and 1988.

``Everyone had to renounce their own egoism,'' said Karl Bernardi, the
owner of a gourmet food and wine store in the center of Brunico, known
as Bruneck in German. ``Above all, it required a will to compromise.''

Mr. Bernardi, whose mother tongue and first name are German, even though
his last name is Italian, acknowledged that the process of
reconciliation took time, help from a handful of far-thinking political
leaders and even the shock of violence.

``The terrorism was important,'' he said candidly. ``It was a signal
that these issues were deadly serious.''

Those issues involve complex questions of culture, language and
identity, and also broader notions of citizenship. Most South Tyroleans,
regardless of the language they speak, have come to terms with a system
that allows different groups to keep their culture, without sacrificing
a sense of local and national community.

South Tyrol's patchwork of privileges --- which include a fiscal regime
that allows 90 percent of tax revenues to stay in the province --- have
helped make it one of the most prosperous, well-administered regions of
Italy, and a source of envy for other parts of the country. Most
important, the expansion of home rule has succeeded in making both
communities feel at home here, which may be the key lesson for Crimea
and other disputed regions.

Giorgio Fusaro moved here from southern Italy eight years ago, attracted
first by the beauty of the mountains, and then by the economic
opportunity available for his children. ``Here you can feel like a
citizen,'' Mr. Fusaro said.

Advertisement

\protect\hyperlink{after-bottom}{Continue reading the main story}

\hypertarget{site-index}{%
\subsection{Site Index}\label{site-index}}

\hypertarget{site-information-navigation}{%
\subsection{Site Information
Navigation}\label{site-information-navigation}}

\begin{itemize}
\tightlist
\item
  \href{https://help.nytimes3xbfgragh.onion/hc/en-us/articles/115014792127-Copyright-notice}{©~2020~The
  New York Times Company}
\end{itemize}

\begin{itemize}
\tightlist
\item
  \href{https://www.nytco.com/}{NYTCo}
\item
  \href{https://help.nytimes3xbfgragh.onion/hc/en-us/articles/115015385887-Contact-Us}{Contact
  Us}
\item
  \href{https://www.nytco.com/careers/}{Work with us}
\item
  \href{https://nytmediakit.com/}{Advertise}
\item
  \href{http://www.tbrandstudio.com/}{T Brand Studio}
\item
  \href{https://www.nytimes3xbfgragh.onion/privacy/cookie-policy\#how-do-i-manage-trackers}{Your
  Ad Choices}
\item
  \href{https://www.nytimes3xbfgragh.onion/privacy}{Privacy}
\item
  \href{https://help.nytimes3xbfgragh.onion/hc/en-us/articles/115014893428-Terms-of-service}{Terms
  of Service}
\item
  \href{https://help.nytimes3xbfgragh.onion/hc/en-us/articles/115014893968-Terms-of-sale}{Terms
  of Sale}
\item
  \href{https://spiderbites.nytimes3xbfgragh.onion}{Site Map}
\item
  \href{https://help.nytimes3xbfgragh.onion/hc/en-us}{Help}
\item
  \href{https://www.nytimes3xbfgragh.onion/subscription?campaignId=37WXW}{Subscriptions}
\end{itemize}
