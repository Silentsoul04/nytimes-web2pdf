Sections

SEARCH

\protect\hyperlink{site-content}{Skip to
content}\protect\hyperlink{site-index}{Skip to site index}

\href{https://myaccount.nytimes3xbfgragh.onion/auth/login?response_type=cookie\&client_id=vi}{}

\href{https://www.nytimes3xbfgragh.onion/section/todayspaper}{Today's
Paper}

\href{/section/upshot}{The Upshot}\textbar{}The Motherhood Penalty vs.
the Fatherhood Bonus

\url{https://nyti.ms/1qDxZeE}

\begin{itemize}
\item
\item
\item
\item
\item
\item
\end{itemize}

Advertisement

\protect\hyperlink{after-top}{Continue reading the main story}

Supported by

\protect\hyperlink{after-sponsor}{Continue reading the main story}

Upshot

Gender Divide

\hypertarget{the-motherhood-penalty-vs-the-fatherhood-bonus}{%
\section{The Motherhood Penalty vs. the Fatherhood
Bonus}\label{the-motherhood-penalty-vs-the-fatherhood-bonus}}

By \href{http://www.nytimes3xbfgragh.onion/by/claire-cain-miller}{Claire
Cain Miller}

\begin{itemize}
\item
  Sept. 6, 2014
\item
  \begin{itemize}
  \item
  \item
  \item
  \item
  \item
  \item
  \end{itemize}
\end{itemize}

One of the worst career moves a woman can make is to have children.
Mothers are less likely to be hired for jobs, to be perceived as
competent at work or to be paid as much as their male colleagues with
the same qualifications.

For men, meanwhile, having a child is good for their careers. They are
more likely to be hired than childless men, and tend to be paid more
after they have children.

These differences persist even after controlling for factors like the
hours people work, the types of jobs they choose and the salaries of
their spouses. So the disparity is not because mothers actually become
less productive employees and fathers work harder when they become
parents --- but because employers expect them to.

The data about the motherhood penalty and the fatherhood bonus present a
clear-cut look at American culture's ambiguous feelings about gender and
work. Even in the age of ``\href{http://leanin.org/book/}{Lean In},''
when women with children run Fortune 500 companies and head the Federal
Reserve, traditional notions about fathers as breadwinners and mothers
as caregivers remain deeply ingrained. Employers, it seems, have not yet
caught up to the fact that women can be both mothers and valuable
employees.

This bias is most extreme for the parents who can least afford it,
according to new data from
\href{http://www.umass.edu/sociol/faculty_staff/bios/budig.html}{Michelle
Budig}, a sociology professor at the University of Massachusetts,
Amherst, who has studied the parenthood pay gap for 15 years.
High-income men get the biggest pay bump for having children, and
low-income women pay the biggest price, she said in a
\href{http://www.thirdway.org/publications/853}{paper} published this
month by \href{http://www.thirdway.org/}{Third Way}, a research group
that aims to advance moderate policy ideas. ``Families with lower
resources are bearing more of the economic costs of raising kids,'' she
said in an interview.

Cultural assumptions aside, here is the reality:
\href{http://www.bls.gov/cps/wlf-databook-2013.pdf}{71 percent} of
mothers with children at home work, according to the Bureau of Labor
Statistics, and women are the sole or primary breadwinner in
\href{http://www.pewsocialtrends.org/2013/05/29/breadwinner-moms/}{40
percent} of households with children, according to data from the Pew
Research Center.

Yet much of the pay gap seems to arise from old-fashioned notions about
parenthood. ``Employers read fathers as more stable and committed to
their work; they have a family to provide for, so they're less likely to
be flaky,'' Ms. Budig said. ``That is the opposite of how parenthood by
women is interpreted by employers. The conventional story is they work
less and they're more distractible when on the job.''

Ms. Budig found that on average, men's earnings increased more than 6
percent when they had children (if they lived with them), while women's
decreased 4 percent for each child they had. Her study was based on data
from the National Longitudinal Survey of Youth from 1979 to 2006, which
tracked people's labor market activities over time. Childless, unmarried
women earn 96 cents for every dollar a man earns, while married mothers
earn 76 cents, widening the gap.

The gap persisted even after Ms. Budig controlled for factors like
experience, education, hours worked and spousal incomes. It's true that
fathers sometimes work more after children, but that explains at most 16
percent of their bonus, she found. And some mothers cut back on hours or
accept lower-paying jobs that are more family-friendly, but that
explains only a quarter to a third of the motherhood penalty.

\includegraphics{https://static01.graylady3jvrrxbe.onion/images/2014/09/07/business/7-UP-WAGES/7-UP-WAGES-articleLarge.jpg?quality=75\&auto=webp\&disable=upscale}

The majority of it, research suggests, is because of discrimination. ``A
lot of these effects really are very much due to a cultural bias against
mothers,'' said
\href{https://sociology.stanford.edu/people/shelley-correll}{Shelley J.
Correll}, a sociology professor at Stanford University and director of
the school's Clayman Institute for Gender Research.

Ms. Correll co-wrote a
\href{http://gender.stanford.edu/sites/default/files/motherhoodpenalty.pdf}{study}
at Cornell in which the researchers sent fake résumés to hundreds of
employers. They were identical, except on some there was a line about
being a member of the parent-teacher association, suggesting that the
applicant was a parent. Mothers were half as likely to be called back,
while fathers were called back slightly more often than the men whose
résumés did not mention parenthood. In a similar study done in a
laboratory, Ms. Correll asked participants how much they would pay job
applicants if they were employers. Mothers were offered on average
\$11,000 less than childless women and \$13,000 less than fathers.

In her research, Ms. Correll found that employers rate fathers as the
most desirable employees, followed by childless women, childless men and
finally mothers. They also hold mothers to harsher performance standards
and are less lenient when they are late.

There was one exception in Ms. Budig's study: Women in the top 10
percent of earners lost no income when they had children, and those in
the top 5 percent received bonuses, similar to men. She speculated that
in these
\href{http://www.nytimes3xbfgragh.onion/2014/06/08/business/riches-come-to-women-as-ceos-but-few-get-there.html}{rarefied
jobs}, employers see high-performing women as more similar to men, and
that women might work more and negotiate for higher pay in order to
afford household and child care help.

At the other end of the earnings spectrum, low-income women lost 6
percent in wages per child, two percentage points more than the average.
For men, the largest bonuses went to white and Latino men who were
highly educated and in professional jobs. The smallest pay bumps went to
unmarried African-American men who had less education and had manual
labor jobs. ``The daddy bonus increases the earnings of men already
privileged in the labor market,'' Ms. Budig wrote.

That low-income workers benefit the least or suffer the most
economically from parenthood is perhaps not surprising. They are the
least likely to have
\href{http://www.nytimes3xbfgragh.onion/interactive/2014/08/13/us/starbucks-workers-scheduling-hours.html}{flexible
schedules} or benefits like paid parental leave. Low-wage women with
children under 6, when offspring need the most in-person care, paid a
wage penalty five times as great as that of higher-paid women with young
children, Ms. Budig found.

The data could be boiled down to hardheaded career advice: Men should
festoon their desks with baby photos and add PTA membership to their
résumés, and women should do the opposite. But ultimately, the solution
is a realization that in the 21st century, male and female employees are
not so different from one another.

``The best hope we have for getting rid of these effects,'' Ms. Correll
said, ``is policy that very much conveys that people have the right to
coordinate work and family.''

In Ms. Budig's previous work, she has found that two policies shrink the
motherhood penalty: publicly funded, high-quality child care for babies
and toddlers, and moderate-length paid parental leave. For instance, in
countries that promote more traditional gender roles, like Germany,
where new mothers are expected to take more than a year off work, the
motherhood
\href{http://www.nytimes3xbfgragh.onion/2014/08/10/upshot/can-family-leave-policies-be-too-generous-it-seems-so.html}{penalty
is very high}. Countries like Sweden with more progressive policies,
such as incentives for new fathers to also take leave, have a smaller
pay gap.

In the United States, most people eventually have children. That is a
reality that employers should understand --- as is the fact that now,
fathers, too, change diapers and pack lunches and mothers go to work.

Advertisement

\protect\hyperlink{after-bottom}{Continue reading the main story}

\hypertarget{site-index}{%
\subsection{Site Index}\label{site-index}}

\hypertarget{site-information-navigation}{%
\subsection{Site Information
Navigation}\label{site-information-navigation}}

\begin{itemize}
\tightlist
\item
  \href{https://help.nytimes3xbfgragh.onion/hc/en-us/articles/115014792127-Copyright-notice}{©~2020~The
  New York Times Company}
\end{itemize}

\begin{itemize}
\tightlist
\item
  \href{https://www.nytco.com/}{NYTCo}
\item
  \href{https://help.nytimes3xbfgragh.onion/hc/en-us/articles/115015385887-Contact-Us}{Contact
  Us}
\item
  \href{https://www.nytco.com/careers/}{Work with us}
\item
  \href{https://nytmediakit.com/}{Advertise}
\item
  \href{http://www.tbrandstudio.com/}{T Brand Studio}
\item
  \href{https://www.nytimes3xbfgragh.onion/privacy/cookie-policy\#how-do-i-manage-trackers}{Your
  Ad Choices}
\item
  \href{https://www.nytimes3xbfgragh.onion/privacy}{Privacy}
\item
  \href{https://help.nytimes3xbfgragh.onion/hc/en-us/articles/115014893428-Terms-of-service}{Terms
  of Service}
\item
  \href{https://help.nytimes3xbfgragh.onion/hc/en-us/articles/115014893968-Terms-of-sale}{Terms
  of Sale}
\item
  \href{https://spiderbites.nytimes3xbfgragh.onion}{Site Map}
\item
  \href{https://help.nytimes3xbfgragh.onion/hc/en-us}{Help}
\item
  \href{https://www.nytimes3xbfgragh.onion/subscription?campaignId=37WXW}{Subscriptions}
\end{itemize}
