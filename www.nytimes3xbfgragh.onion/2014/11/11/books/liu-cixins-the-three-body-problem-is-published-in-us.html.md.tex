Sections

SEARCH

\protect\hyperlink{site-content}{Skip to
content}\protect\hyperlink{site-index}{Skip to site index}

\href{https://www.nytimes3xbfgragh.onion/section/books}{Books}

\href{https://myaccount.nytimes3xbfgragh.onion/auth/login?response_type=cookie\&client_id=vi}{}

\href{https://www.nytimes3xbfgragh.onion/section/todayspaper}{Today's
Paper}

\href{/section/books}{Books}\textbar{}In a Topsy-Turvy World, China
Warms to Sci-Fi

\url{https://nyti.ms/10R9nGS}

\begin{itemize}
\item
\item
\item
\item
\item
\end{itemize}

Advertisement

\protect\hyperlink{after-top}{Continue reading the main story}

Supported by

\protect\hyperlink{after-sponsor}{Continue reading the main story}

\hypertarget{in-a-topsy-turvy-world-china-warms-to-sci-fi}{%
\section{In a Topsy-Turvy World, China Warms to
Sci-Fi}\label{in-a-topsy-turvy-world-china-warms-to-sci-fi}}

\includegraphics{https://static01.graylady3jvrrxbe.onion/images/2014/11/11/books/11scifi/11scifi-articleLarge.jpg?quality=75\&auto=webp\&disable=upscale}

By Amy Qin

\begin{itemize}
\item
  Nov. 10, 2014
\item
  \begin{itemize}
  \item
  \item
  \item
  \item
  \item
  \end{itemize}
\end{itemize}

BEIJING --- With a state-owned power plant in nearby Shanxi Province
temporarily shut down to reduce air pollution, one of its engineers, Liu
Cixin, is using the free time to work on his hobby: reigning as China's
best-selling science-fiction author.

Along with working on a new novel and advising on screenplay adaptations
of his earlier fiction, Mr. Liu, 51, has been promoting the English
translation of
``\href{http://www.tor.com/blogs/2014/10/repost-the-worst-of-all-possible-universes-and-the-best-of-all-possible-earths-three-body-and-chinese-science-fiction}{The
Three-Body Problem},'' the first book in his best-selling apocalyptic
space opera trilogy. Translated by Ken Liu, an award-winning
science-fiction writer in his own right who is based in the United
States (the men are not related), it is one of the few Chinese
science-fiction novels to be translated into English. It will be
released in the United States on Tuesday by Tor Books.

The success of the ``Three-Body'' series, as it is called in China, has
gained a following beyond the small but flourishing science-fiction
world here. Since the third book was published in 2010, each entry in
the series has sold about 500,000 copies in the original Chinese, making
Mr. Liu the best-selling Chinese science-fiction author in decades.

In addition to the usual high school and college-age fans of science
fiction, China's aerospace and Internet industries have embraced the
books. Many interpret the battle of civilizations depicted in the series
as an allegory for the ruthless competition in the nation's Internet
industry.

The series has also breathed new life into a genre that, here as
elsewhere, the literary establishment often marginalizes.

For decades, science fiction was subject to the whims of Communist Party
rule. The genre went from being a vehicle for popularizing science for
socialist purposes to drawing criticism in 1983 from party newspapers
for ``spreading pseudoscience and promoting decadent capitalist
elements.'' When the prestigious People's Literature literary magazine
published four of Mr. Liu's short stories in 2012, it was a sign that
the genre was back in official good graces.

At its core, science fiction capitalizes on uncertainty about the future
to push the boundaries of the reader's imagination. In fast-changing
China, stories that lay out what coming years may hold in store have
therefore found deeper resonance among readers.

``China is on the path of rapid modernization and progress, kind of like
the U.S. during the golden age of science fiction in the '30s to the
'60s,'' Mr. Liu said. ``The future in the people's eyes is full of
attractions, temptations and hope. But at the same time, it is also full
of threats and challenges. That makes for very fertile soil.''

Chinese science fiction serves another purpose in the eyes of Xia Jia, a
science-fiction writer and professor at Xian Jiaotong University.
``Chinese science fiction, in a way, has borne the weight of the
`Chinese dream' since the genre first appeared in China in the late Qing
dynasty,'' she said, referring to the turn of the 20th century.

``The dream is about wanting to overtake the Western countries and
become a very powerful modern China while still preserving these old
elements,'' she added. ``This is what we who write science fiction in
China have to grapple with.''

The ``Three-Body'' tomes chronicle a march of the human race into the
universe set against the recent past, the tumultuous years of the
Cultural Revolution. It is a classic science-fiction story in the style
of the British master Arthur C. Clarke, whose work Mr. Liu says he grew
up reading. ``Everything that I write is a clumsy imitation of Arthur C.
Clarke,'' he said.

The first book in the series explores the world of the Trisolarans, an
alien civilization on the brink of destruction. When a secret military
project in China attempts to make contact with aliens, the Trisolarans
capture the signals and decide to invade Earth. Back in China, people
split into two camps: those who welcome the aliens and those who want to
fight them.

The series is likely to be a change of pace for science-fiction fans in
the United States, where many leading contemporary writers in the genre
are rejecting classic alien-invasion plots in favor of those that take
on real-world issues like climate change or shifting gender roles.

``I don't think the demand for this kind of classical golden age science
fiction has necessarily gone away,'' Liz Gorinsky, an editor at Tor
Books, said of the decision to introduce the series to American readers.
``The `Three-Body' series sort of scratches the same itch that harkens
back to the kinds of books people read when they were kids.''

Some experts link the popularity of the ``Three-Body'' series to a
growing confidence among Chinese about their country's growing role on
the world stage.

``There have always been science-fiction stories that contemplated China
as a leader in the world,'' said
\href{http://paper-republic.org/authors/wu-yan/}{Wu Yan}, a
science-fiction scholar and professor at Beijing Normal University.
``People may have liked them, but, in their hearts, they didn't truly
believe them, or they thought it was really far off in the future. Now,
with the `Three-Body' series, people think, `Wow, it really could be
possible that China might be given a say in the fate of humankind.'~''

In the book, scientists attempt to solve the traditional
\href{http://news.sciencemag.org/physics/2013/03/physicists-discover-whopping-13-new-solutions-three-body-problem}{three-body
problem} in physics, in which the otherwise stable gravitational
interaction between two objects in space becomes random and
unpredictable when a third object is introduced.

Mr. Liu's revered status in the genre was evident this month at a book
signing at the fifth annual Chinese Nebula Awards, one of the largest
gatherings of science-fiction writers and fans in China. More than 2,000
people attended the events, held in an empty museum space on the western
outskirts of Beijing.

Mr. Liu stepped out of the elevator and made his way through the throngs
to the table. Gasps and excited whispers were audible among the hundreds
of fans who had lined up around the building. Two hours later, more than
100 fans, mostly college-age, were still waiting for a chance to meet
him. One, Wu Liang, 27, trembled with excitement after her encounter
with Mr. Liu.

``You just bury your head in the `Three-Body' books, and all of a sudden
you feel so energized,'' said Ms. Wu, who works at an Internet company
in Beijing and attended the event in full Jedi attire. ``It's really a
milestone in Chinese science fiction.''

Advertisement

\protect\hyperlink{after-bottom}{Continue reading the main story}

\hypertarget{site-index}{%
\subsection{Site Index}\label{site-index}}

\hypertarget{site-information-navigation}{%
\subsection{Site Information
Navigation}\label{site-information-navigation}}

\begin{itemize}
\tightlist
\item
  \href{https://help.nytimes3xbfgragh.onion/hc/en-us/articles/115014792127-Copyright-notice}{©~2020~The
  New York Times Company}
\end{itemize}

\begin{itemize}
\tightlist
\item
  \href{https://www.nytco.com/}{NYTCo}
\item
  \href{https://help.nytimes3xbfgragh.onion/hc/en-us/articles/115015385887-Contact-Us}{Contact
  Us}
\item
  \href{https://www.nytco.com/careers/}{Work with us}
\item
  \href{https://nytmediakit.com/}{Advertise}
\item
  \href{http://www.tbrandstudio.com/}{T Brand Studio}
\item
  \href{https://www.nytimes3xbfgragh.onion/privacy/cookie-policy\#how-do-i-manage-trackers}{Your
  Ad Choices}
\item
  \href{https://www.nytimes3xbfgragh.onion/privacy}{Privacy}
\item
  \href{https://help.nytimes3xbfgragh.onion/hc/en-us/articles/115014893428-Terms-of-service}{Terms
  of Service}
\item
  \href{https://help.nytimes3xbfgragh.onion/hc/en-us/articles/115014893968-Terms-of-sale}{Terms
  of Sale}
\item
  \href{https://spiderbites.nytimes3xbfgragh.onion}{Site Map}
\item
  \href{https://help.nytimes3xbfgragh.onion/hc/en-us}{Help}
\item
  \href{https://www.nytimes3xbfgragh.onion/subscription?campaignId=37WXW}{Subscriptions}
\end{itemize}
