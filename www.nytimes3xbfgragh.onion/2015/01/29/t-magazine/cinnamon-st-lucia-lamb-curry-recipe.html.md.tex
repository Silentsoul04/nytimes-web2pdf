Sections

SEARCH

\protect\hyperlink{site-content}{Skip to
content}\protect\hyperlink{site-index}{Skip to site index}

\href{https://myaccount.nytimes3xbfgragh.onion/auth/login?response_type=cookie\&client_id=vi}{}

\href{https://www.nytimes3xbfgragh.onion/section/todayspaper}{Today's
Paper}

A Warm Cinnamon Curry From a Farm in St. Lucia

\begin{itemize}
\item
\item
\item
\item
\item
\end{itemize}

Advertisement

\protect\hyperlink{after-top}{Continue reading the main story}

Supported by

\protect\hyperlink{after-sponsor}{Continue reading the main story}

\href{/column/food-matters}{Food Matters}

\hypertarget{a-warm-cinnamon-curry-from-a-farm-in-st-lucia}{%
\section{A Warm Cinnamon Curry From a Farm in St.
Lucia}\label{a-warm-cinnamon-curry-from-a-farm-in-st-lucia}}

By Greg Kessler

\begin{itemize}
\item
  Jan. 29, 2015
\item
  \begin{itemize}
  \item
  \item
  \item
  \item
  \item
  \end{itemize}
\end{itemize}

\emph{When he is not making photos, the fashion photographer and T
contributor Greg Kessler is also a farmer at Long Island's Quail Hill
farm. Occasionally, for T he explores more remote agricultural areas.}

\includegraphics{https://static01.graylady3jvrrxbe.onion/images/2015/01/29/t-magazine/29cinnamon-kessler-1/29cinnamon-kessler-1-articleLarge.jpg?quality=75\&auto=webp\&disable=upscale}

Last month I escaped Manhattan, where kale continues to dominate the
culinary scene, to a place where kale does not even grow: a 30-acre
organic mountainside farm in St. Lucia. The volcanic soil at Emerald
Estate, owned by the family behind the Anse Chastanet and Jade Mountain
Resort, instead boasts diverse varieties of pineapples, mangoes, cocoa
beans --- and cinnamon. Before this trip, I couldn't have told you what
cinnamon is, or where it comes from. But I now know that the dried,
curly sticks sold in grocery stores are cultivated from a family of
exotic trees. I watched as a third-generation farmer at Emerald Estate,
Martin Joseph, harvested cinnamon the way his grandmother taught him
when he was a boy.

After cutting the grayish outer bark off an area of the tree (only
three-quarters of the tree gets stripped because removing all the bark
completely around one area would kill the tree), Martin cuts into the
tree with a large knife and makes three long incisions. Immediately, he
then lightly taps the exposed wood with a hammer. Within a minute, the
inner bark lifts itself away from the tree, which Joseph gently peels
off with his blade --- one large, wide piece at a time, which he then
slices into a variety of sizes. The result is slightly curved sticks of
raw cinnamon, light brown and damp. It is only after the fragrant bark
dries that it begins to curl into its signature scroll shape.

After my lesson in gathering cinnamon, the executive chef of Jade
Mountain, Jeffrey Forest, indulged me by creating a cinnamon curry with
spiced lamb. The hearty dish, outlined below, is as much a complement to
the breeze off the Caribbean Sea as it is a balm against the East
Coast's freezing temperatures in the wake of Juno.

Image

Yellow rice pairs with cinnamon lamb curry for a hearty
dish.Credit...Greg Kessler

\textbf{Cinnamon Lamb Curry}

\emph{Yield: 6-8 servings}\\
\emph{Time: 30 minutes to prep, plus 3 hours to cook}

\textbf{For the lamb:}\\
2 tablespoons ground cinnamon\\
2 tablespoons Kosher salt\\
2 teaspoons black pepper\\
4 pounds lamb shoulder meat (2-inch cubes )\\
Extra-virgin olive oil

1. Mix the cinnamon, salt and pepper together in a small bowl. Season
the lamb generously with the mixture.

2. Heat a large heavy-bottomed pan over medium high heat. Pour enough
oil in the pan to cover the bottom by 1/8 inch.

3. Carefully add the seasoned lamb to the pan and brown on all sides.
You may need to do this in batches.

*Chef's tip: Season the lamb in small batches (or the lamb will get wet
and will not brown well). It is not necessary to cook the lamb through.
Set aside.

Image

A bay leaf can be substituted for cinnamon leaves while making the curry
sauce.Credit...Greg Kessler

\textbf{For the curry:}

SPICE MIXTURE:\\
3 tablespoons ground cinnamon\\
1 tablespoon cumin seed\\
1 teaspoon ground allspice\\
1 teaspoon ground coriander\\
½ teaspoon ground nutmeg\\
½ teaspoon ground fennel seed\\
1 teaspoon ground cardamom\\
1 teaspoon ground ginger\\
½ teaspoon ground star anise\\
1 teaspoon cayenne pepper\\
1 pinch cloves

1. Combine all ingredients and set aside.

SAUCE:\\
2 medium onions, sliced\\
6 cloves garlic, chopped\\
2 tablespoons fresh ginger, chopped\\
4 14-ounce cans coconut milk\\
2 cans diced tomatoes\\
2 cinnamon leaves (may substitute bay leaves)\\
2 cinnamon sticks\\
Salt and pepper to taste\\
4 cups prepared basmati rice\\
Extra-virgin olive oil

1. In a large heavy-bottomed saucepot set over low heat, cook the
onions, garlic and ginger in a few tablespoons of olive oil for 20
minutes (add more olive oil if it gets dry). Add the spice mixture for
the curry, above, and cook for 5 minutes longer. *Chef's note: If the
pan seems dry, add more oil to pan. It should look like a thick paste.

2. Add the coconut milk and tomatoes to mixture and bring up to a
simmer.

3. Add the browned lamb and return to a simmer. Let the mixture cook
gently for 2 to 2-and-a-half hours or until the lamb is very tender.

4. Be sure to taste the sauce after about an hour and adjust the salt
and pepper to your liking. You may also want to adjust the thickness
with a little water or white wine.

5. Serve over basmati rice.

*Chef's Tip: For a vegetarian option, substitute roasted winter
vegetables for the lamb (squashes, sweet potatoes, and turnips work
well). Be sure to cook the sauce for at least 45 minutes to develop the
flavors before adding in the vegetables. Finish off with some crisp
blanched haricots verts for freshness.

Advertisement

\protect\hyperlink{after-bottom}{Continue reading the main story}

\hypertarget{site-index}{%
\subsection{Site Index}\label{site-index}}

\hypertarget{site-information-navigation}{%
\subsection{Site Information
Navigation}\label{site-information-navigation}}

\begin{itemize}
\tightlist
\item
  \href{https://help.nytimes3xbfgragh.onion/hc/en-us/articles/115014792127-Copyright-notice}{©~2020~The
  New York Times Company}
\end{itemize}

\begin{itemize}
\tightlist
\item
  \href{https://www.nytco.com/}{NYTCo}
\item
  \href{https://help.nytimes3xbfgragh.onion/hc/en-us/articles/115015385887-Contact-Us}{Contact
  Us}
\item
  \href{https://www.nytco.com/careers/}{Work with us}
\item
  \href{https://nytmediakit.com/}{Advertise}
\item
  \href{http://www.tbrandstudio.com/}{T Brand Studio}
\item
  \href{https://www.nytimes3xbfgragh.onion/privacy/cookie-policy\#how-do-i-manage-trackers}{Your
  Ad Choices}
\item
  \href{https://www.nytimes3xbfgragh.onion/privacy}{Privacy}
\item
  \href{https://help.nytimes3xbfgragh.onion/hc/en-us/articles/115014893428-Terms-of-service}{Terms
  of Service}
\item
  \href{https://help.nytimes3xbfgragh.onion/hc/en-us/articles/115014893968-Terms-of-sale}{Terms
  of Sale}
\item
  \href{https://spiderbites.nytimes3xbfgragh.onion}{Site Map}
\item
  \href{https://help.nytimes3xbfgragh.onion/hc/en-us}{Help}
\item
  \href{https://www.nytimes3xbfgragh.onion/subscription?campaignId=37WXW}{Subscriptions}
\end{itemize}
