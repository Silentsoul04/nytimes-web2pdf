Sections

SEARCH

\protect\hyperlink{site-content}{Skip to
content}\protect\hyperlink{site-index}{Skip to site index}

\href{https://www.nytimes3xbfgragh.onion/section/world/asia}{Asia
Pacific}

\href{https://myaccount.nytimes3xbfgragh.onion/auth/login?response_type=cookie\&client_id=vi}{}

\href{https://www.nytimes3xbfgragh.onion/section/todayspaper}{Today's
Paper}

\href{/section/world/asia}{Asia Pacific}\textbar{}North Korea Exports
Forced Laborers for Profit, Rights Groups Say

\url{https://nyti.ms/1Dul9rQ}

\begin{itemize}
\item
\item
\item
\item
\item
\end{itemize}

Advertisement

\protect\hyperlink{after-top}{Continue reading the main story}

Supported by

\protect\hyperlink{after-sponsor}{Continue reading the main story}

\hypertarget{north-korea-exports-forced-laborers-for-profit-rights-groups-say}{%
\section{North Korea Exports Forced Laborers for Profit, Rights Groups
Say}\label{north-korea-exports-forced-laborers-for-profit-rights-groups-say}}

\includegraphics{https://static01.graylady3jvrrxbe.onion/images/2015/02/20/world/NKOREA/NKOREA-articleLarge.jpg?quality=75\&auto=webp\&disable=upscale}

By \href{http://www.nytimes3xbfgragh.onion/by/choe-sang-hun}{Choe
Sang-Hun}

\begin{itemize}
\item
  Feb. 19, 2015
\item
  \begin{itemize}
  \item
  \item
  \item
  \item
  \item
  \end{itemize}
\end{itemize}

SEOUL, South Korea --- When the North Korean carpenter was offered a job
in Kuwait in 1996, he leapt at the chance.

He was promised \$120 a month, an unimaginable wage for most workers in
his famine-stricken country, where most people are not allowed to travel
abroad.

But for Rim Il, the deal soured from the start: Under a moonlit night,
the bus carrying him and a score of other fresh arrivals pulled into a
desert camp cordoned off with barbed-wire fences.

There, 1,800 workers, sent by
\href{http://topics.nytimes3xbfgragh.onion/top/news/international/countriesandterritories/northkorea/index.html?inline=nyt-geo}{North
Korea} to earn badly needed foreign currency, were living together under
the watchful eyes of North Korean government supervisors, Mr. Rim said.
They worked from 7 a.m. to 7 p.m. or, often, midnight, seven days a
week, doing menial jobs at construction sites.

``We only took a Friday afternoon off twice a month but had to spend the
time studying books or watching videos about the greatness of our leader
back home,'' Mr. Rim said at a recent news conference in Seoul, the
South Korean capital. ``We were never paid our wages, and when we asked
our superiors about them, they said we should think of starving people
back home and thank the leader for giving us this opportunity of eating
three meals a day.''

Tens of thousands of North Koreans work long hours for little or no pay,
toiling in Chinese factories or Russian logging camps, digging military
tunnels in Myanmar, building monuments for African dictators, sweating
at construction sites in the Middle East or aboard fishing boats off
Fiji, according to former workers and human rights researchers.

For decades, North Korea has been accused of sending workers abroad and
of confiscating most of their wages. But in the years since
\href{http://topics.nytimes3xbfgragh.onion/top/reference/timestopics/people/k/kim_jongun/index.html?inline=nyt-per}{Kim
Jong-un} took over as leader, human rights researchers say, the program
has expanded rapidly as international sanctions have squeezed the
country's other sources of foreign currency, like illicit trading in
missile parts.

A 2012
\href{http://en.nksc.co.kr/our-work/research/reports/nksc-report-on-the-human-rights-conditions-of-overseas-north-korean-laborers/}{study}
by the \href{http://en.nksc.co.kr/}{North Korea Strategy Center}, a
group in Seoul that works with North Korean defectors, and the private
Korea Policy Research Center estimated that 60,000 to 65,000 North
Koreans were working in more than 40 countries, providing the state with
\$150 million to \$230 million a year. That number has since grown to
100,000, human rights researchers said.

``North Korea is exploiting their labor and salaries to fatten the
private coffers of Kim Jong-un,'' said Ahn Myeong-chul, head of NK
Watch, a human rights group in Seoul. ``We suspect that Kim is using
some of the money to buy luxury goods for his elite followers and
finance the recent building boom in Pyongyang that he has launched to
show off his leadership.''

In a
\href{http://en.asaninst.org/contents/asan-report-beyond-the-coi-dprk-human-rights-report/}{report}
published late last year, the Seoul-based Asan Institute for Policy
Studies said that the revenue from overseas workers helped the North
Korean government bypass international sanctions, which have been
tightened in recent years.

``Earnings are not sent back as remittances, but appropriated by the
state and transferred back to the country in the form of bulk cash,'' it
said, noting that sanctions ban the transfer of bulk cash to the
Pyongyang government. ``Returning workers also act as mules to carry
hard currency earnings back to North Korea.''

NK Watch has collected the testimony of 13 former North Korean workers
now living in South Korea, in support of a petition to the United
Nations asking for an investigation into what it calls ``state-sponsored
slavery.'' The petition, to be filed next month to the United Nations'
special rapporteur on contemporary slavery, said the migrants worked a
minimum of 12 hours a day, were given a few days off a year, and
commonly received only 10 percent of their promised pay, or none at all.

NK Watch said that there had never been an official investigation into
the practice and that it was appealing to the United Nations in hopes of
building on the work of
\href{http://www.ohchr.org/EN/HRBodies/HRC/CoIDPRK/Pages/CommissionInquiryonHRinDPRK.aspx}{a
report} last year that documented widespread human rights abuses inside
North Korea. That report led to
\href{http://www.nytimes3xbfgragh.onion/2014/03/29/world/asia/un-north-korea.html}{a
recommendation} that the Security Council refer North Korea to the
International Criminal Court.

North Korea has
\href{http://www.nytimes3xbfgragh.onion/2014/09/14/world/asia/north-korea-says-reports-of-abuse-are-produced-by-political-racket.html}{dismissed
the report} as false and part of an American-sponsored effort to
overthrow its government.

The workers interviewed by NK Watch said they were victims of a chain of
exploitation and deception.

They described a system where government minders monitored their
movements and communications and required them to spy on one another.
The minders often confiscated the workers' passports.

``These workers face threats of government reprisals against them or
their relatives in North Korea if they attempt to escape or complain to
outside parties,'' the State Department said in
\href{http://www.state.gov/j/tip/rls/tiprpt/2014/?utm_source=NEW+RESOURCE:+Trafficking+in+Persons+R}{a
report} published last year. ``Workers' salaries are deposited into
accounts controlled by the North Korean government, which keeps most of
the money, claiming various `voluntary' contributions to government
endeavors.''

The Workers' Party, the ruling party in North Korea, instructed a group
in Kuwait to send home \$500,000 a month, more than its members' regular
salaries combined, a North Korean supervisor who worked there from 2011
to last year told NK Watch.

Former workers in Kuwait and elsewhere said they were forced to work
even longer hours and seek odd jobs in the local community, splitting
the earnings with government minders who demanded bribes in return for
allowing them such opportunities.

One worker told NK Watch that he received only \$160 in the three years
he worked in a Siberian logging camp in the 1990s, toiling up to 21
hours a day in temperatures often colder than minus 20 degrees
Fahrenheit.

He was told the rest of his wages were sent home to his family. But
families were given only coupons for state-owned stores, where there was
often nothing to buy, former workers said.

Still, in North Korea, the opportunity to work overseas was considered
such a privilege that the jobs had to be bought with bribes. Former
workers said their biggest fear was when supervisors threatened to send
them home when they failed to meet exorbitant production targets or
offer bribes. And compared with many of their compatriots at home, they
were well fed.

``Once, we were eating our bowls of rice, and one guy broke into tears
thinking of his starving children back home, and we all wept together,''
said a North Korean defector who worked in a Russian logging camp from
2000 to 2001. He gave only his last name, Kim, for fear of reprisal
against relatives who are still in the North.

Mr. Kim said he earned \$5.30 a day during the winter logging season. He
later learned that Chinese and Russian workers were earning \$30 a day
for doing much less.

Kim Yoon-tae, a researcher on North Korean human rights, said that the
international community could pressure countries that use North Korean
labor to honor basic international standards for labor protection,
including an end to the practice of giving workers' salaries to the
government.

Mr. Rim said he was paid in cash only once during the five months he
worked in Kuwait before he escaped into the South Korean Embassy there
in 1997. To celebrate the birthday of Kim Jong-il, Kim Jong-un's father
and predecessor, supervisors gave each worker about \$65 to buy
cigarettes.

``Our life was nothing but slavery,'' Mr. Rim said.

Advertisement

\protect\hyperlink{after-bottom}{Continue reading the main story}

\hypertarget{site-index}{%
\subsection{Site Index}\label{site-index}}

\hypertarget{site-information-navigation}{%
\subsection{Site Information
Navigation}\label{site-information-navigation}}

\begin{itemize}
\tightlist
\item
  \href{https://help.nytimes3xbfgragh.onion/hc/en-us/articles/115014792127-Copyright-notice}{©~2020~The
  New York Times Company}
\end{itemize}

\begin{itemize}
\tightlist
\item
  \href{https://www.nytco.com/}{NYTCo}
\item
  \href{https://help.nytimes3xbfgragh.onion/hc/en-us/articles/115015385887-Contact-Us}{Contact
  Us}
\item
  \href{https://www.nytco.com/careers/}{Work with us}
\item
  \href{https://nytmediakit.com/}{Advertise}
\item
  \href{http://www.tbrandstudio.com/}{T Brand Studio}
\item
  \href{https://www.nytimes3xbfgragh.onion/privacy/cookie-policy\#how-do-i-manage-trackers}{Your
  Ad Choices}
\item
  \href{https://www.nytimes3xbfgragh.onion/privacy}{Privacy}
\item
  \href{https://help.nytimes3xbfgragh.onion/hc/en-us/articles/115014893428-Terms-of-service}{Terms
  of Service}
\item
  \href{https://help.nytimes3xbfgragh.onion/hc/en-us/articles/115014893968-Terms-of-sale}{Terms
  of Sale}
\item
  \href{https://spiderbites.nytimes3xbfgragh.onion}{Site Map}
\item
  \href{https://help.nytimes3xbfgragh.onion/hc/en-us}{Help}
\item
  \href{https://www.nytimes3xbfgragh.onion/subscription?campaignId=37WXW}{Subscriptions}
\end{itemize}
