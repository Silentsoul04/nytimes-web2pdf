Sections

SEARCH

\protect\hyperlink{site-content}{Skip to
content}\protect\hyperlink{site-index}{Skip to site index}

\href{https://www.nytimes3xbfgragh.onion/section/travel}{Travel}

\href{https://myaccount.nytimes3xbfgragh.onion/auth/login?response_type=cookie\&client_id=vi}{}

\href{https://www.nytimes3xbfgragh.onion/section/todayspaper}{Today's
Paper}

\href{/section/travel}{Travel}\textbar{}Literary Louisiana

\url{https://nyti.ms/1ybHJnu}

\begin{itemize}
\item
\item
\item
\item
\item
\item
\end{itemize}

Advertisement

\protect\hyperlink{after-top}{Continue reading the main story}

Supported by

\protect\hyperlink{after-sponsor}{Continue reading the main story}

\hypertarget{literary-louisiana}{%
\section{Literary Louisiana}\label{literary-louisiana}}

\includegraphics{https://static01.graylady3jvrrxbe.onion/images/2015/04/12/travel/12LOUISIANA1/12LOUISIANA1-articleLarge-v4.jpg?quality=75\&auto=webp\&disable=upscale}

By Jennifer Moses

\begin{itemize}
\item
  April 10, 2015
\item
  \begin{itemize}
  \item
  \item
  \item
  \item
  \item
  \item
  \end{itemize}
\end{itemize}

A few months ago, I pulled off I-10 and stopped at a gas station in
LaPlace, La. As I gazed at the vodka and meat pies for sale inside, I
realized that I was home. Or at least some crucial nugget of what I call
``me'' was home. Although I haven't lived in south Louisiana for going
on seven years, the time I put in there, soaking up the land and
seascape and sky, and, most of all, the spiritual yearning that exists
cheek-by-jowl with drive-through daiquiri shops, poverty and Mardi Gras,
changed my perceptions profoundly.
\href{http://www.nytimes3xbfgragh.onion/travel/guides/north-america/united-states/louisiana/overview.html}{Louisiana}
--- south Louisiana, in particular --- may as well be its own separate
country, a place where the residents just happen to speak American
English, except when they're speaking Cajun French or Yat or some other
regional dialect.

All of which adds up to a place that produces writers the way France
produces cheese --- prodigiously, and with world-class excellence --- a
place that calls on its writers' talent and inspiration and, in turn, is
reflected back into the world through their words. And though the list
of Louisiana writers --- both homegrown and those drawn to the place as
if by seismic forces --- seems endless, I wanted to smell and hear and
see the places that had left a mark on at least a handful of them.

Image

Tennessee Williams, who called New Orleans one of "the last frontiers of
Bohemia."

I figured I'd start at literary ground zero, in this case, the oldest
part of the Big Easy, and from there work my way out and around and down
the bayou and along the river and over the swamps --- the
non-New-Orleans part of the state that the comic novelist John Kennedy
Toole called ``the heart of darkness, the true wasteland.'' Which, of
course, is just one point of view. Another would be from the mystery
writer James Lee Burke, who wrote: ``In the alluvial sweep of the land,
I thought I could see the past and the present and the future all at
once, as though time were not sequential in nature but took place
without a beginning or an end \ldots{} ''

In any case, the mainly 18th-century French Quarter, or Vieux Carré, has
been so suffused with literary associations that you can practically
hear the echoes of clattering typewriters. When the quarter was hovering
just on the genteel side of being a slum --- its heyday, actually, in
terms of number of writers per square foot --- Tennessee Williams, who
spent his early childhood in Mississippi before moving to St. Louis and
eventually New Orleans, said that it was one of the ``the last frontiers
of Bohemia.''

Today it's all pricey beignets and knickknacks shops, but back then,
just before the Second World War and into the midcentury, the Quarter
was filled with immigrants, streetwalkers, sailors, bars, brawling and
rooming houses, like the one at 722 Toulouse Street, where Tennessee
Williams's attic digs became the inspiration, if perhaps not the exact
setting, for Stella and Stanley's dreary flat in ``A Streetcar Named
Desire.'' Today 722 Toulouse is spruced up and operated by the Historic
New Orleans Collection, which uses it for its offices.

\includegraphics{https://static01.graylady3jvrrxbe.onion/images/2015/04/12/travel/12LOUISIANAJP2/12LOUISIANAJP2-articleLarge-v3.jpg?quality=75\&auto=webp\&disable=upscale}

But bricks and mortar aside, so potent was the vision that created
``Streetcar'' that every March, below the gallery of what's known as the
Upper Pontalba building on Jackson Square, a crowd gathers for the
Tennessee Williams/New Orleans Literary Festival to yell ``Stella!'' and
``Stanley!'' in the hopes of winning the Stanley and Stella shouting
contest. The four-story red-brick Pontalba buildings (there are two of
them, flanking Jackson Square) have long since gone from their original
aristocratic glory to seediness and back again. This time around they
possess the kind of coveted hipness that characterizes the dwellings of
tastemakers in other quarters of urban America, though as with
everything in New Orleans, the Pontalbas come with their own distinctive
flourish, including elaborate iron scrollwork. Truman Capote, who was
born in New Orleans, described the buildings in his short story ``Hidden
Gardens'' as ``the oldest, in some ways most somberly elegant, apartment
houses in America,'' even though they weren't originally apartment
houses in the modern sense at all, but rowhouses.

If it's grit --- or at least non-flash --- that you're after, veer off
the tourist track toward North Rampart Street, to the house at 1014
Dumaine Street that Williams owned from 1962 until his death in 1983.
There's not much to see: a 19th-century yellow house, with a path behind
a locked residents' gate leading back to separate units. The place
doesn't give anything up, but the walk itself is worth it for its quiet
non-glamour, and for the glimpses of peeling paint and cracked sidewalks
and things that at least hint at the seediness that characterized much
of the quarter during its literary heyday.

You can't even find \emph{that} New Orleans at 632 St. Peter Street, an
1842 house where Williams actually wrote ``A Streetcar Named Desire'' in
the 1940s, and where he heard what he called ``that rattletrap streetcar
that bangs through the Quarter.'' The building now sports a Fleurty Girl
store on its first floor; so irksome was this vision that I went off in
a tizzy of disappointment to the Carousel Bar \& Lounge at the
Beaux-Arts Hotel Monteleone, thinking seriously about comforting myself,
as many of the local literary greats did, with a Sazerac cocktail.
Indeed legions of literati loved the famous, circular watering hole,
including the Mississippian William Faulkner (who wrote ``Soldier's
Pay'' at 624 Pirate's Alley, now the home of Faulkner House Books, a
blink-and-you miss it cubbyhole, beloved by book-lovers). In the end I
skipped the booze and ordered a hamburger.

Image

The revolving Carousel Bar \& Lounge at the Hotel Monteleone, where
Hemingway, Faulkner and Tennessee Williams all enjoyed
drinks.Credit...William Widmer for The New York Times

New Orleans is much more than its oldest parts, though, and later
writers tended to go upriver, to the Johnny-come-lately neighborhoods of
the Garden District, the Irish Channel and Carrollton, where even a
casual drive-through is guaranteed to elicit an acute attack of real
estate envy, especially if you happen to swoon for, say, large, raised
center-hall Creole cottages with columns, gabled roofs and gardens
spilling over with flowering vines.

Anne Rice, who was born here, called New Orleans a ``strange, decadent
city full of antebellum houses,'' and whether or not you're into
vampires (I'm not), you might want to do a drive-by to see the author's
most famous residence, an 1857 Greek Revival Italianate mash-up in the
Garden District at 1239 First Street, or to 2524 St. Charles Avenue to
see the Marigny, a center-hall Greek Revival home also built in 1857,
where the author briefly lived when she was a teenager and which served
as the setting for ``Violin,'' about a ghostly violinist and musical
passion.

Kate Chopin, though born in St. Louis, also lived in the hood (and in
Cloutierville near Natchitoches), first at 443 Magazine Street,
described in the 1899 novel ``The Awakening'' as `` \ldots{} a large,
double cottage, with a broad front veranda, whose round, fluted columns
supported the sloping roof.'' I couldn't find that particular double
cottage, but saw others like it in the vicinity. You can also take in
the Chopins' residence --- eventually, Kate Chopin and her husband,
Oscar, had six children --- at 1413 Louisiana Street, a graceful house,
shaded by enormous live oaks. On the afternoon I was there, I gazed up,
imagining the author writing on the side balcony.

Image

Faulkner House Books in the French Quarter.Credit...William Widmer for
The New York Times

But at the rate I'm going, I'm never going to get as far as the north
shore of Lake Ponchartrain, so let's head over to the Irish Channel,
so-called because this sub-sliver of the Garden District, hugging the
river south of Magazine Street, was originally settled by Irish
immigrants in the 19th century, soon joined by Germans, Italians and
people of African ancestry, many of whom worked as stevedores. To this
day, the Irish Channel, with its cottages and shotgun homes, is where
you're most likely to hear a Yat accent (as in ``Where y'at?''), a New
Orleans dialect incorporating Southern American English with Irish,
German, Italian and other European speech patterns.

Yat is captured over and over again by the New Orleans native John
Kennedy Toole's uncanny ear for mimicry and rhythm in his masterpiece
and winner of a posthumous Pulitzer Prize for fiction, ``A Confederacy
of Dunces,'' which revolves around the adventures of the corpulent and
sometimes slightly delusional Ignatius J. Reilly: ``Santa says he likes
the communiss because he's lonely \ldots{} If he was to ax me to marry
him \ldots{} I wouldn't haveta think twice about it.'' The author
himself lived in a simple, one-story house at 7632 Hampson Street, in
Uptown-Carrollton, originally built in the late 19th century and
undergoing restoration on the October afternoon I strolled by.
(Hard-core fans might want to visit the statue of Ignatius J. Reilly in
front of the Hyatt on Canal Street, chow down on Lucky Dogs, or take in
a movie at the Prytania Theater.)

Just a few blocks from Hampson Street, on a quiet block of unassuming
houses, is 1820 Milan Street, where Walker Percy --- who was
instrumental in getting ``A Confederacy of Dunces'' published after its
author's suicide --- began writing the 1961 National Book Award-winning
``The Moviegoer.'' The novel follows its narrator's spiritual journey in
the days leading up to his 30th birthday, as he daydreams, meanders and
goes to the movies, all in a quest to simply be comfortable as a member
of the human race*.*

Image

Walker Percy in his Covington, La., yard in 1977.Credit...Jack
Thornell/Associated Press

Percy --- who was born in Alabama but lived either in or near New
Orleans most of his adult life --- felt the city as a kind of fever
dream, an atmosphere so redolent, so potent, so dripping in charm and
dazzle that it made it difficult for the artist to see past its
seductive appearance and get to the messiness of life, which is, after
all, the fodder of great literature. In an Esquire essay called ``Why I
Live Where I Live,'' Percy said: ``The occupational hazard of the writer
in New Orleans is a variety of the French flu, which might also be
called the Vieux Carré syndrome. One is apt to \ldots{} write
feuilletons or vignettes or catty romans à clef \ldots{}'' Binx Bolling,
the complex, searching, often lonely protagonist of ``The Moviegoer,''
says that he ``can't stand the old world atmosphere of the French
Quarter or the genteel charm of the Garden District'' and hence moves to
Gentilly, ``a middle-class suburb of New Orleans. Except for the banana
plants on the patios and the curlicues of iron on the Walgreen
drugstore, one would never guess it was a part of New Orleans.''

Perhaps to escape the lure of the literary equivalent of kitsch, Percy
also left New Orleans, in his case for Covington, some 40 miles away on
the North Shore of Lake Ponchartrain, another ``non-place,'' in the
author's estimation. Today Covington is a destination charm-spot filled
with coffee emporiums and boutiques. But in Percy's time it was quiet
and not at all chic, a place where he lunched at his favorite waffle
house and walked his Welsh corgi, Sweet Thing. Here, the author, a
Catholic convert, was free to pursue his, and his characters', search
for God in the everyday, whether in Louisiana, as in ``Lancelot,'' or
elsewhere --- North Carolina, for example, in ``The Second Coming.'' In
Covington's leafy historic district, I found myself on Lee Lane, where
19th-century, tin-roofed cottages are now being used mainly as antiques
shops.

I tried, but failed, to get a sense of the author at the French Mix, an
upscale-furnishings emporium, where, when it was his daughter's
bookshop, the Kumquat, he had an upstairs office. Nor could I find him
at St. Joseph Abbey on River Road, where he was an oblate, occasionally
attended Mass, and maintained friendships with at least three of the
monks --- and where he is buried. I did however find excellent, if
pricey, iced coffee at Coffee Rani, and realized, once again, that
there's no better place to find Walker Percy than inside the pages of
his many dazzling novels, which I'd fallen in love with years ago.

Image

The Bayou Teche winds through New Iberia.Credit...William Widmer for The
New York Times

His weren't the only novels I'd fallen for, however. And so I continued
northwest in pursuit of Bobby Pickens, his half-brother F. X., Burma,
Toinette and other characters who populate James Wilcox's 1983 novel
``Modern Baptists,'' and run afoul of human nature in fictional Tula
Springs:

\emph{Burma came to a full stop at the railroad tracks that ran parallel
to Tula Springs's main thoroughfare. The last time a train had come down
these tracks was in 1908, when Tula Springs was an important logging
town. But after most of the pine and cypress were chopped down, the
lumber company pulled out, and the railroad became nothing but a
dividing line between the side of town where Mr. Pickens lived and the
side of town where Mrs. Wedge went every Tuesday to pick up her maid.}

But where is Tula Springs, anyhow, other than in the imagination of the
author, who was born on Thomas Street in nearby Hammond and educated at
Yale? He went on to begin his writing career in New York City, and, a
couple of decades later, returned home, to head up the creative writing
department at Louisiana State University in Baton Rouge, which is where
I got to know him. Later I learned that Tula Springs --- where Wilcox's
novels ``North Gladiola,'' ``Miss Undine's Living Room,'' ``Sort of
Rich,'' ``Heavenly Days'' and ``Hunk City'' are also set --- is based in
part on Independence, La., a 40-minute drive from Covington, where in
fact the railroad tracks run parallel to the main thoroughfare, the
business district is about a block long, and there's a red-brick Baptist
church, a water tower and signs advertising mudbugs (crawfish), barbecue
and wings.

Like so much in Louisiana, if you're looking at Independence with eyes
hoping to see grandeur, glamour or charm --- European capitals, New
England villages --- you're not going to \emph{get it.} And that's
because, past the seductive beauty of the Big Easy, Louisiana is all
about nuance, possibility within the boundaries imposed by climate,
landscape and, in many places, poverty. Over all, it is flat, hot and
wet. But it is there --- in the interstices between the limited
\emph{real} and the spiritual \emph{perhaps ---} that the masterpieces
of Louisiana literature unfold. I love these places: small towns, weedy
railroad tracks, bridges over bayous leading to wetlands that give way
to the oil rig-dotted Gulf of Mexico. There are no signposts, no big
photo opportunities. Just a unique and, for me, magical way of being in
the world.

Image

Laura Plantation in Vacherie, west of New Orleans.Credit...Sara Essex
Bradley for The New York Times

Which is not to say that there aren't go-to sites, too, such as Laura
Plantation on the River Road in Vacherie, where, in the 1870s, Alcée
Fortier, a folklorist who lived up the road, began to write down the
stories that he heard former slaves telling their children in Creole
French patois. These tales, of Compair Lapin and Compair Bouki,
eventually entered the American canon as Br'er Rabbit, under the
authorship of Joel Chandler Harris. Whether or not you're a fan of that
original ``wascally wabbit,'' the Creole plantation itself, with its
slave cabins, family artifacts and raised Creole family home, surrounded
to this day with cane fields, is marvelous.

Versus Oscar, La., home of Ernest J. Gaines and the setting for most of
his novels that deal with themes springing from the legacy of slavery
and ongoing racism. On the surface, there isn't much going on even
though the local radio station, 88.3 (``The Voice of Pointe Coupee
Parish''), rocks. Still, across from the Gaines residence (marked by two
large ``G''s on the front gates) on False River Road (Highway 1) is the
sparkling oxbow False River Lake, today edged with vacation homes.
Nearby is the Pointe Coupee Parish Museum, a 1760 Creole-style log cabin
with one of its two rooms recreated as it would have looked back in the
back-then.

In many ways, Oscar seems a lot like the place depicted in ``A Lesson
Before Dying,'' in which Gaines wrote: ``All there was to see were old
white weather-houses, with smoke rising out of the chimneys and drifting
across the corrugated tin roofs overlooking the yard toward the field,
where some of the cane had been cut. \ldots{} A little farther over,
where another patch of cane was standing, tall and blue-green, you could
see the leaves swaying softly from a breeze.''

Before moving to California to live with his mother and stepfather,
Gaines was raised on River Lake plantation, where his parents were
sharecroppers, and as an adult built his home, on what had once been its
land, moving the tin-roofed plantation school to the back of his
property. You can still see the River Lake plantation house, which is
privately owned, from the road, and, if you're like me, drive down the
dirt road abutting it to see the cane being brought in. You can also
visit the pretty town of New Roads, where St. Augustine Catholic Church,
still serving primarily African-American worshipers, once ran a school
that Gaines attended, and where I got a bite to eat at Ma Mama's
Kitchen. The waitress called me ``honey'' and the menu featured
crawfish, crab cakes, étouffée and chops.

Now, over the swamps, to the small town of New Iberia, home-base of
James Lee Burke's Cajun detective Dave Robicheaux: crime in and around
da bayou. (The author himself, born in Texas and raised on the
Texas-Louisiana Gulf coast, now spends most of his time in Montana.) I
have to confess that I've long had a special affection for small, pretty
New Iberia, with its plantation homes and first-rate Bayou Teche Museum
next to the Art Deco Evangeline Theater. On Main Street you can follow
Detective Robicheaux and his buddies to Victor's Cafeteria, and get
fried frogs' legs, gumbo, soft rolls and all kinds of vegetables, or
meander over the Teche on Bridge Street, where Dave's fictional bait
shop was located. Not just the town, but all of Iberia Parish is dotted
with places from the author's many novels, but remember to bring soft
eyes so you don't miss the misty blue-greens, the march of live oaks and
the wooden houses, some with a quality of wistfulness from having
withstood decades, and even centuries, of hurricanes and humidity.

My final stop was also my starting place: Baton Rouge where Huey Long
built the nation's tallest state capitol building, a 34-story
Deco-Moderne skyscraper. Three years later when he was a United States
senator, he'd be assassinated outside his offices. Or, as Robert Penn
Warren put it in ``All the King's Men'': ``We came into the great lobby,
under the dome, where there was a blaze of light over the statues which
stood in statesmanlike dignity on pedestals to mark the quarters of the
place. \ldots{} I saw the two little spurts of pale-orange flame from
the muzzle of the weapon.''

On the Capitol's observation deck, I looked west over the Mississippi
toward the swamps and bayous, north toward great smoking petrochemical
plants, east toward the Big Easy and its dreamy charms, and finally,
between where I stood on the 27th floor and the Louisiana State
University Tiger football stadium, toward the old wooden house where my
husband and I raised our children, with its two looming oaks out back
and decades of stories.

Advertisement

\protect\hyperlink{after-bottom}{Continue reading the main story}

\hypertarget{site-index}{%
\subsection{Site Index}\label{site-index}}

\hypertarget{site-information-navigation}{%
\subsection{Site Information
Navigation}\label{site-information-navigation}}

\begin{itemize}
\tightlist
\item
  \href{https://help.nytimes3xbfgragh.onion/hc/en-us/articles/115014792127-Copyright-notice}{©~2020~The
  New York Times Company}
\end{itemize}

\begin{itemize}
\tightlist
\item
  \href{https://www.nytco.com/}{NYTCo}
\item
  \href{https://help.nytimes3xbfgragh.onion/hc/en-us/articles/115015385887-Contact-Us}{Contact
  Us}
\item
  \href{https://www.nytco.com/careers/}{Work with us}
\item
  \href{https://nytmediakit.com/}{Advertise}
\item
  \href{http://www.tbrandstudio.com/}{T Brand Studio}
\item
  \href{https://www.nytimes3xbfgragh.onion/privacy/cookie-policy\#how-do-i-manage-trackers}{Your
  Ad Choices}
\item
  \href{https://www.nytimes3xbfgragh.onion/privacy}{Privacy}
\item
  \href{https://help.nytimes3xbfgragh.onion/hc/en-us/articles/115014893428-Terms-of-service}{Terms
  of Service}
\item
  \href{https://help.nytimes3xbfgragh.onion/hc/en-us/articles/115014893968-Terms-of-sale}{Terms
  of Sale}
\item
  \href{https://spiderbites.nytimes3xbfgragh.onion}{Site Map}
\item
  \href{https://help.nytimes3xbfgragh.onion/hc/en-us}{Help}
\item
  \href{https://www.nytimes3xbfgragh.onion/subscription?campaignId=37WXW}{Subscriptions}
\end{itemize}
