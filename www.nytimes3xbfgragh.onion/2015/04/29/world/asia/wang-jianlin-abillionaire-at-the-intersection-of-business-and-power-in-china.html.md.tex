Sections

SEARCH

\protect\hyperlink{site-content}{Skip to
content}\protect\hyperlink{site-index}{Skip to site index}

\href{https://www.nytimes3xbfgragh.onion/section/world/asia}{Asia
Pacific}

\href{https://myaccount.nytimes3xbfgragh.onion/auth/login?response_type=cookie\&client_id=vi}{}

\href{https://www.nytimes3xbfgragh.onion/section/todayspaper}{Today's
Paper}

\href{/section/world/asia}{Asia Pacific}\textbar{}Wang Jianlin, a
Billionaire at the Intersection of Business and Power in China

\url{https://nyti.ms/1ziGdkm}

\begin{itemize}
\item
\item
\item
\item
\item
\end{itemize}

Advertisement

\protect\hyperlink{after-top}{Continue reading the main story}

Supported by

\protect\hyperlink{after-sponsor}{Continue reading the main story}

\hypertarget{wang-jianlin-a-billionaire-at-the-intersection-of-business-and-power-in-china}{%
\section{Wang Jianlin, a Billionaire at the Intersection of Business and
Power in
China}\label{wang-jianlin-a-billionaire-at-the-intersection-of-business-and-power-in-china}}

\includegraphics{https://static01.graylady3jvrrxbe.onion/images/2015/04/28/world/JP-WANG1/JP-WANG1-articleLarge.jpg?quality=75\&auto=webp\&disable=upscale}

By \href{http://www.nytimes3xbfgragh.onion/by/michael-forsythe}{Michael
Forsythe}

\begin{itemize}
\item
  April 28, 2015
\item
  \begin{itemize}
  \item
  \item
  \item
  \item
  \item
  \end{itemize}
\end{itemize}

HONG KONG --- He controls thousands of movie screens around the world,
serving more filmgoers than any other cinema chain. He has invested
billions of dollars in real estate projects across four continents. He
is building skyscrapers that will redraw the skylines of London and
Chicago. He is shopping for a Hollywood studio.

There are as many as 430 billionaires in
\href{http://topics.nytimes3xbfgragh.onion/top/news/international/countriesandterritories/china/index.html?inline=nyt-geo}{China},
more than in any country besides the United States. But Wang Jianlin
stands out, and not just because he is the richest person in Asia, with
a fortune estimated at more than \$35 billion.

As his real estate and entertainment empire expands overseas, Mr. Wang,
60, has emerged as the rare private-sector tycoon in a position to
advance Beijing's interests abroad, with clout in industries and
communities around the world.

Prime ministers send him thank-you notes, and
\href{http://www.nytimes3xbfgragh.onion/2013/09/23/business/global/chinese-titan-takes-aim-at-hollywood.html}{Hollywood's
biggest stars fly to China} when he summons them. In March, at an event
to woo foreign investors, he was one of only a dozen businessmen to meet
President Obama.

How the son of a foot soldier in Mao Zedong's Communist Revolution
catapulted into the top tier of the global elite is an archetypal story
of China's transition to capitalism and the outsize opportunities it
presents those with talent or connections --- or, in Mr. Wang's case,
both. His story, though, is also singular: He built one of the world's
most valuable real estate portfolios in a nation where the state retains
ownership of all land.

A yearlong examination of his success by The New York Times casts a
light on the murky intersection of business and power at the heights of
the Chinese economy, where market competition is often warped by the
whims of Communist Party leaders.

Entrepreneurs have powered rapid growth in China for more than three
decades. But even the most successful businessmen here must still reach
some accommodation with the party, which only a generation ago operated
a socialist planned economy.

Mr. Wang says he has prospered by delivering what ambitious party
officials crave: choice real estate developments that propel economic
growth and bolster their careers. In return, he says, the officials sell
him the rights to develop choice parcels of land at prices far below
what his competitors pay.

His conglomerate, Wanda Group, is best known in China for its signature
Wanda Plazas, massive shopping complexes with cinemas, office towers,
hotels and apartments. Since building the first one in the northeastern
city of Changchun in 2002, he has opened more than 100 of them in at
least 70 other Chinese cities, generating the revenue that now finances
his ambitions abroad.

But there is an aspect of his relationship with the authorities that Mr.
Wang never raises in interviews and that has gone unreported in the many
accounts of his success published in China and abroad: Relatives of some
of the nation's most powerful politicians and their business associates
own significant stakes in his company.

An extensive review of corporate records filed with the government
identified several such investments made from 2007 to 2011, when Wanda
was privately held and rarely sold shares to outsiders.

\href{http://www.bloomberg.com/news/articles/2012-06-29/xi-jinping-millionaire-relations-reveal-fortunes-of-elite}{Among
those given an early chance to buy a stake in his company was Qi
Qiaoqiao}, an active investor who is the elder sister of China's current
president, Xi Jinping. (She sold or transferred her shares in the
company in October 2013 to a longtime business associate.)

Other early investors included a business partner of the daughter of
former Prime Minister Wen Jiabao, and relatives of two other members of
the ruling Politburo at the time, Jia Qinglin and Wang Zhaoguo,
according to the records and interviews with family members and business
associates.

Together, their stakes in Wang Jianlin's real estate division, Dalian
Wanda Commercial Properties, were valued at \$1.1 billion
\href{http://dealbook.nytimes3xbfgragh.onion/2014/12/16/dalian-wanda-commercial-a-chinese-developer-raises-3-7-billion-in-i-p-o/}{when
it held an initial public offering in Hong Kong} in December. Their
shares in Mr. Wang's cinema subsidiary were valued at \$17.2 million
when it listed separately in January. Their holdings in both companies
are worth more than \$1.5 billion now.

There is no indication that any of the politicians whose relatives and
business associates owned shares in Wanda intervened on the company's
behalf in any of its dealings with the government. Nor is there evidence
that any of the politicians personally benefited from the windfall that
these investors reaped. The investors and officials did not respond to
written questions or could not be reached for comment.

Mr. Wang declined an interview request and did not respond to written
questions submitted to Wanda. But in public remarks, he often uses the
same phrase to describe how he manages his relationship with the
authorities: ``Stay close to the government and distant from politics.''

``It's a fact that China's economy is government-led, and the real
estate industry depends on approvals, so if you say you can ignore the
government in this business, I'd say that's impossible,'' Mr. Wang told
state television in a February interview. ``I'd say it's hypocritical
and fake to say that. ... But at the same time, for example, we don't
pay bribes.''

\textbf{A Remarkable Winning Streak}

It was September 2012, and students at the John F. Kennedy School of
Government at Harvard were spellbound by an unusual lecturer, a short,
plain-talking man with close-set eyes and a prominent widow's peak.

``In China, it's not easy for a company, especially a private company,
to be successful and grow,''
\href{http://www.wanda.cn/2013/chairman_0724/28.html}{he said through an
interpreter}. ``The hardships they face are many times greater than in
the United States.''

It would not have been a particularly incisive observation from a member
of the faculty. But the speaker that afternoon was Mr. Wang, and despite
the hardships that he described, he was on a remarkable winning streak.

Months earlier, Mr. Wang had purchased AMC Entertainment Holdings, the
second-largest theater chain in the United States. By year's end, his
empire in China would include 66 Wanda Plazas, 38 five-star hotels, 980
cinema screens and 57 department stores, not to mention 63 karaoke
saloons. Within a year, he would break ground on an \$8 billion movie
studio and theme park in the coastal city of Qingdao, flying in stars
such as Leonardo DiCaprio, Nicole Kidman and John Travolta to celebrate.

With success came the familiar trappings of the megarich. Mr. Wang
bought a Picasso painting at auction for \$28.2 million. His wife
socialized with Prince Albert of Monaco. His son hired a top Korean pop
group to perform at his 27th birthday party.

Not content with just owning a yacht, Mr. Wang bought the British
company that makes the luxury boats seen in James Bond films. In
January, his company announced it was buying a stake in a Spanish soccer
club.

``He is a force of nature,'' said Jeffrey Katzenberg, the chief
executive of DreamWorks Animation, who has known Mr. Wang for three
years. ``He is a very, very strong personality, and he is extremely
confident about what he is doing.''

Mr. Wang's father was a veteran of the Communist Party's Long March, the
arduous and deadly trek across China in the 1930s by the Communists that
hardened a generation of revolutionaries, and Mr. Wang himself, the
eldest of five brothers, followed him into the army as a teenager. In a
2013 appearance on state television, he recalled marching hundreds of
miles through knee-deep snow in training exercises during Mao's
fanatical Cultural Revolution.

``Many people couldn't make it,'' Mr. Wang said, but he did. He then
spent the next 16 years rising through the officer ranks, an experience
that colleagues say informs his management style. Even his most senior
Wanda executives are required to punch in and out, for example, and
tardiness is not tolerated, former employees say.

After leaving the military, Mr. Wang took a government job in the
northeastern port city of Dalian. His big break there came in 1988, when
he was transferred to a failing state-owned builder of residential
apartment blocks. By his own account, he secured a loan with the help of
an old army buddy and returned the company to profitability.

In 1992, the firm was restructured as one of China's first shareholding
companies, and over the next decade, Mr. Wang oversaw its privatization,
emerging as its majority owner.

The mayor of Dalian for much of that time was Bo Xilai, a politician who
was the son of an influential party elder and who would rise to the
Politburo before falling from power in a stunning corruption and murder
scandal in 2012.

Mr. Wang was an extraordinary benefactor to Dalian during Mr. Bo's
tenure, buying the city's soccer team, which helped make it a national
champion, and donating tens of millions of dollars to build schools.

Since Mr. Bo's fall, Mr. Wang has sought to distance himself from him.
In a recent interview, he said he struggled during those years because
he refused to pay bribes and did not get along with Mr. Bo. That limited
his access to land, he said, which Mr. Bo's administration would not
sell him.

``We still made money, but it wasn't a pleasant time,''
\href{http://www.bloomberg.com/news/articles/2015-02-09/wang-boy-soldier-turned-billionaire-pursues-hollywood-studios}{Mr.
Wang told Bloomberg Markets magazine}.

Because the state retains formal ownership of all land in China, those
who want to build on it must have the state's blessing. Local
governments have depended for years on the sale of long-term rights to
develop land to finance their operations. But who gets which parcels,
and at what price, is as much a political decision as an economic one.

Mr. Wang says his company is so good at delivering benefits to the
cities where it builds that local governments now compete for his
business and he turns down more than two out of every three proposals.

``We thus can take the initiative, and we have the bargaining power,''
he told the students at Harvard. That means he acquires land at less
than half the cost to his competitors, he added.

Even as property prices set records in China, the price that Dalian
Wanda paid for access to land fell by more than 40 percent from 2011 to
2014, according to its I.P.O. prospectus.

Wang Yongping, vice chairman of the China Commercial Real Estate
Association, said local officials were eager to work with Wanda because
of the tax revenue its projects bring them and because it has a
reputation for getting things done fast. The company can finish a Wanda
Plaza at what it calls Wanda Speed, or within 18 months.

``Chinese government officials love to have achievements when they are
in office,'' Wang Yongping said. ``Eighteen months might determine
whether they can become a district chief or a provincial party
secretary.''

But some in power have other reasons to root for Wang Jianlin.

\textbf{A Paper Trail to the Elite}

In July 2007, Mr. Wang had built only a handful of Wanda Plazas. He had
yet to make his first billion dollars, and few outside China had heard
of him or his company. The Hurun Rich List, which tracks the net worth
of the wealthiest people in China, ranked him 148th.

But late that month, a newly formed firm in Beijing, Minghao Holdings,
acquired a 2.5 percent stake in Mr. Wang's flagship company, Wanda
Group, becoming its largest external shareholder, according to corporate
records. There was only one other outside shareholder at the time, a
local real estate company in Dalian run by a friend.

Then, two months later, another newly established firm in Beijing,
Wugufeng Investment Consulting, took a 1.53 percent stake in the main
company that Mr. Wang used to hold his shares in Wanda Group, becoming
its fifth shareholder, the records show.

\includegraphics{https://static01.graylady3jvrrxbe.onion/images/2015/04/28/world/JP-WANG2/JP-WANG2-articleLarge.jpg?quality=75\&auto=webp\&disable=upscale}

The documents do not indicate why the two Beijing firms were invited to
become early investors in Mr. Wang's business. But the paper trail from
the companies leads to relatives of two men sitting at the time on the
Communist Party's ruling Politburo: Jia Qinglin, a longtime party boss
of the Beijing municipality who ranked fourth in the party leadership,
and Wang Zhaoguo, a senior legislator who was the leading force behind a
landmark law providing legal protections for private property.

Wang Zhaoguo's son, Wang Xinyu, was the controlling shareholder of the
first firm, which later transferred its stake in Wanda to a young woman
named Yang Xin, the records show. Ms. Yang is the Politburo member's
niece, according to Wang Zhao'an, a cousin who serves as party chief in
the family's home village in Hebei Province.

The owner of the second firm is listed in the records as Pan Yongbin,
63, who shares a business address, staff and phone numbers with a
Beijing investment firm run by Mr. Jia's son-in-law, Li Botan. Mr. Pan
is also listed as a board member of several firms owned by Mr. Li,
including his main investment company.

As Wanda prospered in the years that followed, the value of these early
stakes skyrocketed.

The firm controlled by Wang Zhaoguo's son paid less than \$500,000 for
its stake, according to Wanda's filings with the government, though the
firm's own records do not show the transaction. The documents are more
complete for the second firm, showing a payment of less than \$200,000
for its stake. Those two stakes are now worth more than \$640 million
and \$250 million.

Neither Politburo member was in a position to directly set the price or
approve the sale of land-use rights to Wanda. But party institutions
under the two men have showered Wang Jianlin with public recognition.

In March 2008, Mr. Wang was one of only three mainland billionaires
named to the standing committee of the Chinese People's Political
Consultative Congress, a national advisory body, led by Mr. Jia, that is
made up of people whom the party leadership deems influential. The
appointment amounted to a seal of approval by top party leaders, said
one Chinese businessman who serves on a similar body and spoke on the
condition of anonymity to protect his position.

In June 2007, Mr. Wang was also named an outstanding private
entrepreneur by a party-run industry association, the first of five
awards bestowed upon him over the next four years from groups led by or
associated with Wang Zhaoguo.

Honors like these can signal to local officials and potential business
partners that their recipients are well connected. ``At least within
China, people will be much more willing to do business with you, and
much less likely to offend you,'' said Victor Shih, a scholar at the
University of California, San Diego.

After the global financial crisis in late 2008, China's property market
took a nose dive. Real estate stocks in Shanghai ended the year down 65
percent.

But Wanda had a banner year, breaking ground on seven new Wanda Plaza
complexes --- setting a record --- and propelling Wang Jianlin to 20th
on the Hurun list of China's richest people, with a fortune estimated at
\$2.3 billion.

The next year, Wanda distributed shares to outsiders again, privately
selling an 8.5 percent stake in the company.

Among the eight new investors was a Beijing firm owned through a network
of holding companies by Qi Qiaoqiao, the sister of Xi Jinping, and her
husband, Deng Jiagui.

Ms. Qi and Mr. Xi are members of a privileged elite consisting of the
offspring of senior party officials. Their now deceased father, Xi
Zhongxun, was a military comrade of Mao and later became a vice premier
and a pioneer of economic reform.

Ms. Qi worked in a variety of political and military posts during her
career before becoming a businesswoman. During that time, her younger
brother, Mr. Xi, was working his way up through the ranks in provincial
government posts.

It is not known why Ms. Qi and her husband were offered the chance to
invest in Wanda. By 2009, she was already a wealthy investor with
business relationships across the country. Meanwhile, her brother had
become China's vice president and was the consensus candidate to become
the party's next leader.

Ms. Qi and Mr. Deng did not respond to a request for comment. But
records show the couple
\href{http://www.nytimes3xbfgragh.onion/2014/06/18/world/asia/chinas-president-xi-jinping-investments.html}{began
selling off or transferring hundreds of millions of dollars in
investments in 2012} as President Xi embarked on a high-profile
crackdown on official corruption. The motivation behind this sell-off is
unclear but it had the effect of reducing her brother's political
vulnerability as his campaign targeted thousands of officials.

In many cases, the couple sold their investments to individuals with no
clear connection to them.

But the shares in Wanda --- valued at \$240 million now, up from the
\$28.6 million the couple paid for them in 2009 --- were transferred to
a longtime business associate on Oct. 8, 2013. The records give no
indication of the price paid by the business associate, who has served
the couple in various corporate posts for more than a decade.

Another new shareholder in Wanda was an investment fund owned in part
and managed by a subsidiary of Tsinghua Holdings, the investment arm of
Beijing's prestigious Tsinghua University. At the time, the top official
at Tsinghua Holdings was Hu Haifeng, the son of Hu Jintao, then China's
president. There is no indication Hu Haifeng personally benefited from
or held any shares in Wanda himself.

Some of the individuals given an early chance to invest in Wanda are
difficult to identify. One of them, Jin Yi, acquired a stake in 2009
that is now worth about \$250 million. But the address on her identity
card is incomplete, and her name is not included on the official
registry of residents in the neighborhood listed.

Corporate records indicate, however, that Ms. Jin is a business
associate of
\href{http://www.nytimes3xbfgragh.onion/2012/10/26/business/global/family-of-wen-jiabao-holds-a-hidden-fortune-in-china.html}{Wen
Ruchun, the daughter of Wen Jiabao}, the prime minister from 2003 to
2013. Ms. Jin is listed as one of three partners in a Beijing real
estate firm alongside Lily Chang, an alias used by Ms. Wen.

Image

When Mr. Wang broke ground on an \$8 billion movie studio and theme park
in the coastal city of Qingdao in 2013, the stars he flew in to
celebrate included Leonardo DiCaprio.Credit...Jason Lee/Reuters

\textbf{Promoting Chinese Culture}

Seated between Lloyd Blankfein, the chief executive of Goldman Sachs,
and Nick Clegg, the deputy prime minister of Britain, Wang Jianlin
\href{http://dealbook.nytimes3xbfgragh.onion/2014/01/23/bristling-interrupts-amity-in-panel-on-u-s-china-europe-ties/}{had
little to say through much of a panel discussion last year at the World
Economic Forum in Davos}, Switzerland, the annual gathering of the
wealthy and powerful. But when a speaker suggested that China's focus on
territorial disputes diminished its influence in Asia, the billionaire
bristled.

``This is a discussion on economics today and shouldn't delve into
politics,'' Mr. Wang snapped. ``You are publicly saying bad things about
China --- at the very least, I don't think this is very polite.''

As his fortune has climbed and his investments have stretched overseas,
Mr. Wang has emerged as an outspoken advocate for his homeland. In
interviews and speeches, he tends to present himself as the pragmatic
face of big business in China, delivering a soothing message of
opportunity to foreign audiences anxious about the country's rise.

``Wang Jianlin is a perfect instrument for that from the party's point
of view,'' said Joseph Nye, the Harvard professor who coined the term
``soft power'' and was the panelist scolded by Mr. Wang at Davos.

Mr. Wang is effective in part because he is no longer simply a Chinese
real estate developer. As Beijing sought to cool its property sector in
recent years, he diversified by shifting investments abroad and into the
culture and entertainment sector, including his network of movie
theaters, which became the world's largest in 2012 with the purchase of
AMC's 4,000-plus screens in the United States.

The strategy coincided with a policy push by the Chinese leadership to
expand the nation's cultural influence both overseas and at home, where
younger generations have increasingly turned to Western music,
television and films.

A communiqué issued by the party's Central Committee in October 2011
cited an ``urgency for China to strengthen its cultural soft power and
global cultural influence.''

``After this document, Wanda started to put a lot of effort into
developing the cultural industry,'' said Zhang Lihua, a scholar at the
Tsinghua-Carnegie Center for Global Policy in Beijing.

Investors connected to senior party leaders were able to get in early on
Wanda's move. In December 2010, the investment firm of Mr. Jia's
son-in-law acquired a \$9 million stake in Wanda's cinema subsidiary,
records show. As of Monday, the stake was worth \$131 million.

The records also show another early investment that same month by an
investment fund run by New Horizon, a
\href{http://topics.nytimes3xbfgragh.onion/top/reference/timestopics/subjects/p/private_equity/index.html?inline=nyt-classifier}{private
equity} firm co-founded by Mr. Wen's son, Winston Wen. On Monday, that
stake was valued at \$526 million.

The Central Committee's decision resulted in new policies that benefited
real estate companies such as Wanda.

Local governments, for example, were told to prioritize the sale of
land-use rights to companies that built projects that promote Chinese
culture. And state-owned banks were told to offer loans for cultural
undertakings at home and abroad. The state export-import bank agreed to
help finance Wanda's overseas investments, backing its acquisition of
AMC.

In addition to investing in the movie industry, Wanda has opened a
series of amusement parks that promote Chinese culture, including one
that features a
\href{http://news.xinhuanet.com/house/wuxi/2014-02-28/c_119549270.htm}{building
in the shape of a Chinese teapot}. Mr. Wang says it will compete with a
Disneyland under construction in Shanghai.

``No matter how good Disney is, it is still American culture,'' he said
after the groundbreaking ceremony last year. ``We hope to use Chinese
culture.''

Xu Han, a scholar at the University of Pennsylvania who has studied the
company, said Wanda's overseas expansion was driven in part by Beijing's
desire to see the nation's premier companies build a presence abroad.
But he said the firm was more interested in profit than in directly
influencing public opinion about China.

``The central government's objective is very clear,'' he said. ``To have
a very successful Chinese company is good for China.''

When Mr. Wang bought AMC, he retained the American management team and
emphasized that his company would not dictate what films were shown in
American theaters.

But with China forecast to surpass North America in box office revenue
by 2018, Hollywood is already focused on serving Chinese filmgoers and
satisfying the censors who determine what foreign films can be shown in
Chinese theaters.

Mr. Wang often notes that the Chinese market will be twice the size of
the North American market by 2023.

Foreigners need to heed this new reality, he warned when he hosted Mr.
DiCaprio and Ms. Kidman for the 2013 groundbreaking of his Qingdao
studio.

``Those in the world film industry who realize this first and are among
the first to cooperate with China,'' he said, ``will be the first to
reap the benefits.''

Advertisement

\protect\hyperlink{after-bottom}{Continue reading the main story}

\hypertarget{site-index}{%
\subsection{Site Index}\label{site-index}}

\hypertarget{site-information-navigation}{%
\subsection{Site Information
Navigation}\label{site-information-navigation}}

\begin{itemize}
\tightlist
\item
  \href{https://help.nytimes3xbfgragh.onion/hc/en-us/articles/115014792127-Copyright-notice}{©~2020~The
  New York Times Company}
\end{itemize}

\begin{itemize}
\tightlist
\item
  \href{https://www.nytco.com/}{NYTCo}
\item
  \href{https://help.nytimes3xbfgragh.onion/hc/en-us/articles/115015385887-Contact-Us}{Contact
  Us}
\item
  \href{https://www.nytco.com/careers/}{Work with us}
\item
  \href{https://nytmediakit.com/}{Advertise}
\item
  \href{http://www.tbrandstudio.com/}{T Brand Studio}
\item
  \href{https://www.nytimes3xbfgragh.onion/privacy/cookie-policy\#how-do-i-manage-trackers}{Your
  Ad Choices}
\item
  \href{https://www.nytimes3xbfgragh.onion/privacy}{Privacy}
\item
  \href{https://help.nytimes3xbfgragh.onion/hc/en-us/articles/115014893428-Terms-of-service}{Terms
  of Service}
\item
  \href{https://help.nytimes3xbfgragh.onion/hc/en-us/articles/115014893968-Terms-of-sale}{Terms
  of Sale}
\item
  \href{https://spiderbites.nytimes3xbfgragh.onion}{Site Map}
\item
  \href{https://help.nytimes3xbfgragh.onion/hc/en-us}{Help}
\item
  \href{https://www.nytimes3xbfgragh.onion/subscription?campaignId=37WXW}{Subscriptions}
\end{itemize}
