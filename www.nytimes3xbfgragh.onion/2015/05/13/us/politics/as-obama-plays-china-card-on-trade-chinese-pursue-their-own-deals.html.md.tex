Sections

SEARCH

\protect\hyperlink{site-content}{Skip to
content}\protect\hyperlink{site-index}{Skip to site index}

\href{https://www.nytimes3xbfgragh.onion/section/politics}{Politics}

\href{https://myaccount.nytimes3xbfgragh.onion/auth/login?response_type=cookie\&client_id=vi}{}

\href{https://www.nytimes3xbfgragh.onion/section/todayspaper}{Today's
Paper}

\href{/section/politics}{Politics}\textbar{}As Obama Plays China Card on
Trade, Chinese Pursue Their Own Deals

\url{https://nyti.ms/1G4DzA6}

\begin{itemize}
\item
\item
\item
\item
\item
\end{itemize}

Advertisement

\protect\hyperlink{after-top}{Continue reading the main story}

Supported by

\protect\hyperlink{after-sponsor}{Continue reading the main story}

\hypertarget{as-obama-plays-china-card-on-trade-chinese-pursue-their-own-deals}{%
\section{As Obama Plays China Card on Trade, Chinese Pursue Their Own
Deals}\label{as-obama-plays-china-card-on-trade-chinese-pursue-their-own-deals}}

By \href{http://www.nytimes3xbfgragh.onion/by/david-e-sanger}{David E.
Sanger} and
\href{http://www.nytimes3xbfgragh.onion/by/edward-wong}{Edward Wong}

\begin{itemize}
\item
  May 12, 2015
\item
  \begin{itemize}
  \item
  \item
  \item
  \item
  \item
  \end{itemize}
\end{itemize}

WASHINGTON --- President Obama has toured Nike's headquarters and busy
American ports in recent weeks to try to convince Democrats that
rejecting the Trans-Pacific Partnership, a 12-nation trade deal that he
envisions as a crucial part of a legacy of national economic revival,
will undercut American power.

``If we don't write the rules for trade around the world, guess what?''
Mr. Obama warned at the headquarters of the sporting goods giant, whose
wares he uses when he works out. ``China will.''

Xi Jinping, the Chinese president and head of the Communist Party, is
making a parallel pitch --- but rooted in a very different strategy for
gaining global influence. Mr. Xi has essentially shrugged off the
question of whether his nation, the world's second-largest economy, will
join the pact. Instead, he has picked off American allies like Britain,
Germany and South Korea to join, against the administration's wishes,
the Asian Infrastructure Investment Bank, a project started by China in
part to keep its own state-owned firms busy building roads, dams and
power plants around Asia.

China is at the same time setting up other trade pacts around the region
so it can use its cash and enormous market leverage to strike deals more
advantageous to its interests. Mr. Xi was in Kazakhstan this month
plugging his ``One Belt, One Road'' initiative of construction from
Europe to Central Asia to the seas around China.

It is a subtle form of competition, but one that many in the Obama
administration see as the most important geopolitical power struggle in
the world today.

``It's not a black-and-white contest between us and China, even though
the president has presented it that way to sell it to Democrats,'' said
Michael J. Green, a Georgetown University professor who charts the
progress of the contest. ``The Pacific Partnership puts pressure on the
Chinese to up their game. We are somewhere between direct competition
over who will make the rules and a competitive liberalization that will
eventually create some common rules around the world.''

America's strategy since the 1990s, when Bill Clinton wooed Republican
votes to get China into the World Trade Organization, has been a
straightforward one: Entice China into institutions that operate
according to Western standards of trade rules, labor practices and the
protection of intellectual property, gradually changing the way a rising
power rises. Mr. Clinton made that case in visits to Beijing, arguing
that if China opened its doors to trade, new ideas and the Internet
would inevitably pressure its leaders toward democracy and freer
expression.

It was a view that, in retrospect, overestimated American influence and
underestimated the degree to which the Chinese believed they could amend
the global order to suit their own economic interests. So while Mr.
Obama plays the China card to sell the accord in the United States, the
Chinese are pursuing their own course.

China has been excluded from the negotiations on the trade deal because
it has been unwilling to sign on, so far, to the wide-ranging reforms of
its economy required of all members. It could join later on --- and
Chinese officials have left open that possibility, as have nations like
South Korea. But for now, China seems in no rush. Just as it created an
infrastructure bank to suit its own ambitions, it is assembling trade
agreements whose rules it can write by virtue of the huge size of its
market.

``The Chinese government's response is to build the free-trade
agreements that it can influence,'' said Li Daokui, a professor at the
School of Economics and Management at Tsinghua University in Beijing.
``I would say it was a mistake for the U.S. not to include China. If
China had been allowed to join at the beginning, the landscape would be
entirely different.''

Mr. Li said that if China had been part of the negotiations at the
start, it might not have pushed so hard across the region for its own
trade agreements, and ``everything would be much simpler.''

Perhaps for Mr. Xi. For Mr. Obama, it might have been even more complex.
The domestic politics of this deal are difficult enough without China as
a potential signatory, as shown by his failure to prevail in an
important vote in the Senate on Tuesday to get the authority he needs to
negotiate the trade agreement. (Administration officials insisted it was
a temporary and tactical setback.)

Kevin Rudd, the former Australian prime minister, has spent the last
year at Harvard and at the Asia Society studying the long-term future
between China and the West. He concluded in a recently published report
that in both China and the United States there was a rush ``to believe
that the two countries are now locked into some sort of irreversible and
increasingly fractious zero-sum game.'' Instead, he found the
relationship was still ``functioning reasonably effectively,'' but noted
that Mr. Xi had changed strategy.

``He has ended former paramount leader Deng Xiaoping's foreign policy
orthodoxy over the past 35 years of `hide your strength, bide your time,
never take the lead,' in favor of a more vigorous, activist and
assertive international policy to advance Chinese interests both in the
region and beyond,'' he wrote in Foreign Policy.

Shi Yinhong, a professor of international relations at Renmin
University, said Chinese leaders were well aware of how important
approval of the trade agreement, also known as the TPP, is for Mr. Obama
as he tries what he has often called a ``rebalancing'' toward Asia.

``The Chinese government knows the TPP is a major attempt by the U.S. to
win back economic leadership in the region,'' Mr. Shi said. ``China also
knows the Asia Pacific region is such a wide region, so you can have two
stages. One is led by the U.S., which is pushing the TPP. The other is
dominated by China.''

Mr. Shi said China was not worried about the TPP because Asia was a vast
enough region to allow for multiple trade agreements. ``This is far from
a zero-sum game,'' he said. ``In the future, both countries will find
places of cooperation as well as competition.''

American officials say the Chinese view has evolved. ``They have come
around to seeing that they are going to have to live in a TPP world, a
world with higher standards and increased competition for investment,''
Michael B. Froman, the United States trade representative, who is
negotiating the deal, said in an interview. ``We are already seeing that
investors are deciding to move to TPP countries where they will have a
stable labor system,'' protections for intellectual property and the
freedom to move data into and out of the country without government
restraints, which are part of the deal being negotiated. ``That's going
to force China to raise their game, too.''

Listen to the debate on Capitol Hill and it sounds a lot like the 1990s
--- discussion of manufacturing jobs being sent to China, fears that
American sovereignty will be ignored in trade disputes, revived
arguments about North American Free Trade Agreement, which was passed
more than two decades ago.

In Beijing, things sound very different. Last month, a Foreign Ministry
spokesman, Hong Lei, was asked about the TPP at a regularly scheduled
news conference. He responded that the TPP was one component that could
help build a
\href{http://www.chinadaily.com.cn/world/2014apec/2014-11/07/content_18885318.htm}{larger
free-trade area across Asia}, a plan that China laid out last year. That
plan, he said, ``will generate far more economic gains than all existing
regional free-trade arrangements once it is completed.''

Advertisement

\protect\hyperlink{after-bottom}{Continue reading the main story}

\hypertarget{site-index}{%
\subsection{Site Index}\label{site-index}}

\hypertarget{site-information-navigation}{%
\subsection{Site Information
Navigation}\label{site-information-navigation}}

\begin{itemize}
\tightlist
\item
  \href{https://help.nytimes3xbfgragh.onion/hc/en-us/articles/115014792127-Copyright-notice}{©~2020~The
  New York Times Company}
\end{itemize}

\begin{itemize}
\tightlist
\item
  \href{https://www.nytco.com/}{NYTCo}
\item
  \href{https://help.nytimes3xbfgragh.onion/hc/en-us/articles/115015385887-Contact-Us}{Contact
  Us}
\item
  \href{https://www.nytco.com/careers/}{Work with us}
\item
  \href{https://nytmediakit.com/}{Advertise}
\item
  \href{http://www.tbrandstudio.com/}{T Brand Studio}
\item
  \href{https://www.nytimes3xbfgragh.onion/privacy/cookie-policy\#how-do-i-manage-trackers}{Your
  Ad Choices}
\item
  \href{https://www.nytimes3xbfgragh.onion/privacy}{Privacy}
\item
  \href{https://help.nytimes3xbfgragh.onion/hc/en-us/articles/115014893428-Terms-of-service}{Terms
  of Service}
\item
  \href{https://help.nytimes3xbfgragh.onion/hc/en-us/articles/115014893968-Terms-of-sale}{Terms
  of Sale}
\item
  \href{https://spiderbites.nytimes3xbfgragh.onion}{Site Map}
\item
  \href{https://help.nytimes3xbfgragh.onion/hc/en-us}{Help}
\item
  \href{https://www.nytimes3xbfgragh.onion/subscription?campaignId=37WXW}{Subscriptions}
\end{itemize}
