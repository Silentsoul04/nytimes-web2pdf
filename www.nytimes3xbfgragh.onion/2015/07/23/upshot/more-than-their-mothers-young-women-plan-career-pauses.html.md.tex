Sections

SEARCH

\protect\hyperlink{site-content}{Skip to
content}\protect\hyperlink{site-index}{Skip to site index}

\href{https://myaccount.nytimes3xbfgragh.onion/auth/login?response_type=cookie\&client_id=vi}{}

\href{https://www.nytimes3xbfgragh.onion/section/todayspaper}{Today's
Paper}

\href{/section/upshot}{The Upshot}\textbar{}More Than Their Mothers,
Young Women Plan Career Pauses

\url{https://nyti.ms/1DurJKR}

\begin{itemize}
\item
\item
\item
\item
\item
\item
\end{itemize}

Advertisement

\protect\hyperlink{after-top}{Continue reading the main story}

Supported by

\protect\hyperlink{after-sponsor}{Continue reading the main story}

Upshot

Surveys Say

\hypertarget{more-than-their-mothers-young-women-plan-career-pauses}{%
\section{More Than Their Mothers, Young Women Plan Career
Pauses}\label{more-than-their-mothers-young-women-plan-career-pauses}}

By \href{http://www.nytimes3xbfgragh.onion/by/claire-cain-miller}{Claire
Cain Miller}

\begin{itemize}
\item
  July 22, 2015
\item
  \begin{itemize}
  \item
  \item
  \item
  \item
  \item
  \item
  \end{itemize}
\end{itemize}

Beginning in college, years before she planned to have children, Yi Gu
began strategizing about how to have a career that was flexible enough
to fit in family responsibilities.

She knew that arrangement wasn't realistic in her first two jobs:
banking, in which she worked very long hours, or consulting, in which
she traveled often. Instead, she saw those as preparation for the more
flexible job she took last year at age 31, in strategy at a major
pharmacy company. She became pregnant soon after.

``The definition of work-life balance keeps on changing,'' she said.
``Out of business school, not being married and not having kids,
anything less than 80 hours a week to me was balanced. Then in
consulting, it was if I traveled or had time during the week to hang out
with friends. Now with a kid, the definition has changed again.''

The youngest generation of women in the work force --- the millennials,
age 18 to early 30s --- is defining career success differently and less
linearly than previous generations of women. A variety of survey data
shows that educated, working young women are more likely than those
before them to expect their career and family priorities to shift over
time.

The surveys highlighted that two generations after women entered the
business world in large numbers, it can still be hard for women to work.
Even those with the highest career ambitions are more likely than their
predecessors to plan to scale back at work at certain times or to seek
out flexible jobs.

You might call them the planning generation: Their approach is less all
or nothing --- climb the career ladder or stay home with children ---
and more give and take.

In surveys of millennials who are college-educated professionals by the
\href{http://www.talentinnovation.org/}{Center for Talent Innovation}, a
research group, the young people said they saw their parents struggle
while working full time or leave the work force altogether, and wanted a
different option. ``They felt as if they were learning from generations
before them, and saw all of the downsides in both choices,'' said Laura
Sherbin, the center's director of research. ``Millennials are looking
for more of a balance.''

\href{http://www.hbs.edu/women50/docs/L_and_L_Survey_2Findings_12final.pdf}{A
survey of Harvard Business School alumni}, released as part of the
school's new gender initiative, found that 37 percent of millennial
women and 42 percent of those already married planned to interrupt their
career for family. That compared with 28 percent of Generation X women
and 17 percent of baby boomers.

The surveys also revealed that some younger women believe today's
economy has made it harder to be a working parent. In the Harvard
survey, fewer young women than older women said they expected to
successfully combine work and family or have a career equal to that of
their husband.

Half of women 30 and under said they thought their gender was a
disadvantage at work --- equal to the share of baby boomers who said the
same. Women were less likely than men to say they were satisfied with
their careers.

\includegraphics{https://static01.graylady3jvrrxbe.onion/images/2015/07/20/upshot/sub-21UP-Millennial/sub-21UP-Millennial-articleLarge.jpg?quality=75\&auto=webp\&disable=upscale}

\href{http://knowledge.wharton.upenn.edu/article/stew-friedman-new-work-family-choices-men-women/}{A
study of female students} graduating from the Wharton School at the
University of Pennsylvania found that while 78 percent of the business
school graduates in 1992 said they planned to have children, that share
had dropped to 42 percent by 2012. In some cases they did not want to,
and in others they did not think they could make it work because of a
lack of organizational support, according to Stewart Friedman, director
of Wharton's work/life integration project.

That belief extends beyond the narrow world of business school alumni. A
broader
\href{http://www.pewresearch.org/fact-tank/2015/03/10/women-still-bear-heavier-load-than-men-balancing-work-family/}{Pew
Research Center study} found that 58 percent of working millennial
mothers said being a working mother made it harder for them to get ahead
in their careers, compared with 38 percent of older women.

``With the boomers, there was a real ascendance in this idea of having
very egalitarian partnerships and the ability to have high-powered
careers, and that has diminished with Generation X and even more so with
this millennial generation,'' said Colleen Ammerman, assistant director
of the Harvard gender initiative.

Harvard Business School alumni are an elite group on an ambitious career
path. And they are also likely to earn enough that both partners would
not need to work. Young women do not seem to be lowering their ambitions
--- or ``leaving before you leave,'' as Sheryl Sandberg described it in
``Lean In.'' Their career goals, and their accomplishments in the years
immediately after business school, were indistinguishable from those of
men. Rather, they say, they are thinking ahead to some potentially tough
decisions.

``Just as we look at strategies of companies, a lot of H.B.S. people are
putting together strategies for their life,'' said Cheryl Han, 33, an
alumna and co-founder and chief executive of Keaton Row, a fashion
start-up. ``If I build that into my strategy, then I won't feel like I
failed, and maybe it makes you feel more certain about your future.''

Their approach is different from that of the women who paved the way for
theirgeneration to enter the upper tiers of business. Baby boomer women
were the first to work in professions in large numbers, and they were
less likely to say they planned to interrupt their careers and more
likely to say they expected to successfully combine their work and
family lives.

Younger women say they learned from the experiences of older
generations, and are determined to avoid their pitfalls. They are much
more likely than women in older generations to say that women in senior
leadership are critical to their success, both in navigating their
careers and in figuring out how to incorporate family responsibilities.

``They're anticipating that in some way they're going to have to dial
down or integrate their career and their life,'' said Caroline Ghosn,
chief executive of \href{http://www.levo.com/}{Levo}, an online
professional network focused on millennial women. ``This reality is
something that people are a lot more transparent and open about.''

By age 30, nearly half of the women in the Harvard study who were
married said they had chosen a job with more flexibility, 26 percent had
slowed down the pace of their career and 9 percent had declined a
promotion because of family responsibilities. Many of those interviewed
cited Ms. Sandberg, a fellow alumna.

Kwany Lui, 31, co-founder of Bundle Organics, a nutrition start-up,
said: ``The Sheryl Sandberg book says don't lean out in advance. How I
think about the message is currently I'm sitting here without children
and I don't have intentions to say, `O.K., next job I find, let me make
sure it's 40 hours a week.' I think when I have kids I might change my
mind, but until I get there I honestly don't know.''

\href{http://www.nytimes3xbfgragh.onion/2014/11/30/upshot/even-among-harvard-graduates-women-fall-short-of-their-work-expectations.html}{Women's
expectations} have declined: 66 percent of millennial women said they
expected their careers to be equal to those of their spouses, compared
with 79 percent of baby boomers. Three-quarters of millennial women said
they expected to succeed in combining their careers and family life, but
that is a significant drop from the 86 percent of baby boomer women who
said the same.

Men's attitudes are also beginning to change. Eventually, that could
lead to more shared responsibilities, though it is happening slowly. For
example, 13 percent of millennial men said they expected to interrupt
their careers for children. That is more than the 4 percent of
Generation X men and 3 percent of baby boomer men who said the same ---
but significantly less than the 37 percent of women who said so.

Advertisement

\protect\hyperlink{after-bottom}{Continue reading the main story}

\hypertarget{site-index}{%
\subsection{Site Index}\label{site-index}}

\hypertarget{site-information-navigation}{%
\subsection{Site Information
Navigation}\label{site-information-navigation}}

\begin{itemize}
\tightlist
\item
  \href{https://help.nytimes3xbfgragh.onion/hc/en-us/articles/115014792127-Copyright-notice}{©~2020~The
  New York Times Company}
\end{itemize}

\begin{itemize}
\tightlist
\item
  \href{https://www.nytco.com/}{NYTCo}
\item
  \href{https://help.nytimes3xbfgragh.onion/hc/en-us/articles/115015385887-Contact-Us}{Contact
  Us}
\item
  \href{https://www.nytco.com/careers/}{Work with us}
\item
  \href{https://nytmediakit.com/}{Advertise}
\item
  \href{http://www.tbrandstudio.com/}{T Brand Studio}
\item
  \href{https://www.nytimes3xbfgragh.onion/privacy/cookie-policy\#how-do-i-manage-trackers}{Your
  Ad Choices}
\item
  \href{https://www.nytimes3xbfgragh.onion/privacy}{Privacy}
\item
  \href{https://help.nytimes3xbfgragh.onion/hc/en-us/articles/115014893428-Terms-of-service}{Terms
  of Service}
\item
  \href{https://help.nytimes3xbfgragh.onion/hc/en-us/articles/115014893968-Terms-of-sale}{Terms
  of Sale}
\item
  \href{https://spiderbites.nytimes3xbfgragh.onion}{Site Map}
\item
  \href{https://help.nytimes3xbfgragh.onion/hc/en-us}{Help}
\item
  \href{https://www.nytimes3xbfgragh.onion/subscription?campaignId=37WXW}{Subscriptions}
\end{itemize}
