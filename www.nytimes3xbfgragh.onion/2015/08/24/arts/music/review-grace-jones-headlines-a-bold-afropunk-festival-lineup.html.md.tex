Sections

SEARCH

\protect\hyperlink{site-content}{Skip to
content}\protect\hyperlink{site-index}{Skip to site index}

\href{https://www.nytimes3xbfgragh.onion/section/arts/music}{Music}

\href{https://myaccount.nytimes3xbfgragh.onion/auth/login?response_type=cookie\&client_id=vi}{}

\href{https://www.nytimes3xbfgragh.onion/section/todayspaper}{Today's
Paper}

\href{/section/arts/music}{Music}\textbar{}Review: Grace Jones Headlines
a Bold Afropunk Festival Lineup

\url{https://nyti.ms/1V2Wkco}

\begin{itemize}
\item
\item
\item
\item
\item
\end{itemize}

Advertisement

\protect\hyperlink{after-top}{Continue reading the main story}

Supported by

\protect\hyperlink{after-sponsor}{Continue reading the main story}

\hypertarget{review-grace-jones-headlines-a-bold-afropunk-festival-lineup}{%
\section{Review: Grace Jones Headlines a Bold Afropunk Festival
Lineup}\label{review-grace-jones-headlines-a-bold-afropunk-festival-lineup}}

\includegraphics{https://static01.graylady3jvrrxbe.onion/images/2015/08/24/arts/24AFROPUNK1/24AFROPUNK1-articleLarge.jpg?quality=75\&auto=webp\&disable=upscale}

By \href{http://www.nytimes3xbfgragh.onion/by/ben-ratliff}{Ben Ratliff}

\begin{itemize}
\item
  Aug. 23, 2015
\item
  \begin{itemize}
  \item
  \item
  \item
  \item
  \item
  \end{itemize}
\end{itemize}

``Maybe I should take off my shoes and go totally tribal,'' Grace Jones
said.

At the end of the first night of this year's
\href{http://afropunkfest.com/}{Afropunk} Festival, on Saturday, she was
wearing very little: an under-bust corset, sheer tights and white body
paint in stripes and patterns. Which is to say that she was topless,
though by that point she'd been through a succession of wraps, gowns,
science-fiction headdresses and a skull mask. She sat down on a stage
riser, removed one of her high heels and pretended to throw it at the
audience, but held on. ``That was childish,'' she admitted, and cackled.

\includegraphics{https://static01.graylady3jvrrxbe.onion/images/2015/08/24/arts/24AFROPUNK2/24AFROPUNK2-articleLarge.jpg?quality=75\&auto=webp\&disable=upscale}

Then she picked up a hula hoop, stepped into it, and put it to use. For
all of her final song, ``Slave to the Rhythm,'' she kept the hoop going:
through its symphonic swells and its negative spaces, through her deep,
severe long-tones and incidental muttering, through introductions of
band members and a reference to her forthcoming autobiography.

At 67, Ms. Jones's confidence is extreme but relaxed; her whole
enterprise is bold, honest, provocative and peaceful, like the event she
was headlining. It was the 11th year of Afropunk, now held in Commodore
Barry Park in Fort Greene, Brooklyn, a two-day festival of black arts
and culture that started from underground energies in punk, hip-hop and
much else. It hasn't let go of the margins, but has inevitably moved
toward the center.

Image

Lauryn HillCredit...Nicole Fara Silver for The New York Times

You could ask what center, or notice that old centers can change their
meanings from year to year. Kelis, who performed Saturday, was a
commercial pop star at the beginning of the aughts, and has grown toward
meta-R\&B; SZA, also in Saturday's lineup, convincingly folded new
avant-garde aesthetics into soft-edged 1980s funk and soul.

And Lenny Kravitz, a mainstream rock-radio staple, was scheduled to
close out Sunday's lineup, which could be the clearest indication that
the festival has progressed into a different phase --- as well as the
fact that starting this year the festival is no longer free. Admission
was \$45 for a single day and \$75 for the weekend, though you could
volunteer your services for a few hours to earn a ticket. None of which
kept people away: The festival organizers estimated the crowd at 30,000
each day. On Saturday the park was packed, hot, and intense, with three
stages, vendors, and long lines for just about everything.

Image

SZACredit...Nicole Fara Silver for The New York Times

Striking, commanding, and chiding as she was, Ms. Jones acted as a
cooling service. She never seemed too far from laughing or making funny
interjections, mostly in reference to her own songs and story, with the
assumption that you knew it all. She has the gifts of her gothic voice,
and her slow, imposing movements, like a series of unfoldings. (At one
point she draped a long leg over the railing of an elevated perch at the
back of the stage, leaned backward and didn't need to do much but hold
the pose.) And she has her band, which articulated the cool, mysterious
feeling of her songs, in which reggae, rock, cabaret and electronic-pop
swirl together: The space-dub in ``Private Life,'' the robotic new wave
in her version of Roxy Music's ``Love Is the Drug,'' the frothy
Antillean party groove in a new, unreleased song, ``Shenanigans.''

Lauryn Hill preceded her with a concise but immediate performance of
songs that flowed into one another, including ``Peace of Mind,'' ``Mr.
Intentional,'' and ``Ex-Factor.'' Seated most of the time with an
acoustic guitar --- she started 40 minutes late and had her lights and
sound turned off early as a result --- she sang and rapped and cued the
band through hard, surging, urgent vamps: sort of soul, sort of gospel,
sort of Afrobeat, rolling on as if it could have continued much longer.

Death Grips, on a smaller stage right after Ms. Hill's set, played with
a similarly uninterrupted, flow-through feeling, though way more
explosive and less legible, and with zero sentimentality. Through five
years of confounding moves against record companies and its own
listeners (the band announced last year that it would break up, but who
knows?), its sound has held steady: A punk-noise-rap hybrid with Zach
Hill's assaultive live drums, MC Ride's bellowing warnings and a general
air of police-state dread and negativity. But the good feeling of the
festival accommodated them too: hundreds of fans near the stage jumped
in coordination to the jagged beats; behind that front flank, there was
more general pleasure in the audience than I've seen at their shows in
the past.

Advertisement

\protect\hyperlink{after-bottom}{Continue reading the main story}

\hypertarget{site-index}{%
\subsection{Site Index}\label{site-index}}

\hypertarget{site-information-navigation}{%
\subsection{Site Information
Navigation}\label{site-information-navigation}}

\begin{itemize}
\tightlist
\item
  \href{https://help.nytimes3xbfgragh.onion/hc/en-us/articles/115014792127-Copyright-notice}{©~2020~The
  New York Times Company}
\end{itemize}

\begin{itemize}
\tightlist
\item
  \href{https://www.nytco.com/}{NYTCo}
\item
  \href{https://help.nytimes3xbfgragh.onion/hc/en-us/articles/115015385887-Contact-Us}{Contact
  Us}
\item
  \href{https://www.nytco.com/careers/}{Work with us}
\item
  \href{https://nytmediakit.com/}{Advertise}
\item
  \href{http://www.tbrandstudio.com/}{T Brand Studio}
\item
  \href{https://www.nytimes3xbfgragh.onion/privacy/cookie-policy\#how-do-i-manage-trackers}{Your
  Ad Choices}
\item
  \href{https://www.nytimes3xbfgragh.onion/privacy}{Privacy}
\item
  \href{https://help.nytimes3xbfgragh.onion/hc/en-us/articles/115014893428-Terms-of-service}{Terms
  of Service}
\item
  \href{https://help.nytimes3xbfgragh.onion/hc/en-us/articles/115014893968-Terms-of-sale}{Terms
  of Sale}
\item
  \href{https://spiderbites.nytimes3xbfgragh.onion}{Site Map}
\item
  \href{https://help.nytimes3xbfgragh.onion/hc/en-us}{Help}
\item
  \href{https://www.nytimes3xbfgragh.onion/subscription?campaignId=37WXW}{Subscriptions}
\end{itemize}
