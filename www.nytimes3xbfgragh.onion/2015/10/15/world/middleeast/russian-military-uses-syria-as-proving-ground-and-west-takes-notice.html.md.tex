Sections

SEARCH

\protect\hyperlink{site-content}{Skip to
content}\protect\hyperlink{site-index}{Skip to site index}

\href{https://www.nytimes3xbfgragh.onion/section/world/middleeast}{Middle
East}

\href{https://myaccount.nytimes3xbfgragh.onion/auth/login?response_type=cookie\&client_id=vi}{}

\href{https://www.nytimes3xbfgragh.onion/section/todayspaper}{Today's
Paper}

\href{/section/world/middleeast}{Middle East}\textbar{}Russian Military
Uses Syria as Proving Ground, and West Takes Notice

\url{https://nyti.ms/1OD9wCU}

\begin{itemize}
\item
\item
\item
\item
\item
\item
\end{itemize}

Advertisement

\protect\hyperlink{after-top}{Continue reading the main story}

Supported by

\protect\hyperlink{after-sponsor}{Continue reading the main story}

\hypertarget{russian-military-uses-syria-as-proving-ground-and-west-takes-notice}{%
\section{Russian Military Uses Syria as Proving Ground, and West Takes
Notice}\label{russian-military-uses-syria-as-proving-ground-and-west-takes-notice}}

\includegraphics{https://static01.graylady3jvrrxbe.onion/images/2015/10/15/world/15military-web/15military-web-articleLarge.jpg?quality=75\&auto=webp\&disable=upscale}

By \href{http://www.nytimes3xbfgragh.onion/by/steven-lee-myers}{Steven
Lee Myers} and
\href{http://www.nytimes3xbfgragh.onion/by/eric-schmitt}{Eric Schmitt}

\begin{itemize}
\item
  Oct. 14, 2015
\item
  \begin{itemize}
  \item
  \item
  \item
  \item
  \item
  \item
  \end{itemize}
\end{itemize}

WASHINGTON --- Two weeks of air and missile strikes in
\href{http://topics.nytimes3xbfgragh.onion/top/news/international/countriesandterritories/syria/index.html?inline=nyt-geo}{Syria}
have given Western intelligence and military officials a deeper
appreciation of the transformation that Russia's military has undergone
under President
\href{http://topics.nytimes3xbfgragh.onion/top/reference/timestopics/people/p/vladimir_v_putin/index.html?inline=nyt-per}{Vladimir
V. Putin}, showcasing its ability to conduct operations beyond its
borders and providing a public demonstration of new weaponry, tactics
and strategy.

The strikes have involved aircraft never before tested in combat,
including the Sukhoi Su-34 strike fighter, which NATO calls the
Fullback, and a ship-based
\href{http://www.nytimes3xbfgragh.onion/2015/10/09/world/middleeast/russias-kalibr-cruise-missiles-a-new-weapon-in-syria-conflict.html}{cruise
missile} fired more than 900 miles from the Caspian Sea, which,
according to some analysts, surpasses the American equivalent in
technological capability.

Russia's jets have struck in support of Syrian ground troops advancing
from areas under the control of the Syrian government, and might soon
back an Iranian-led offensive that appeared to be forming in the
northern province of Aleppo on Wednesday. That coordination reflects
what American officials described as months of meticulous planning
behind Russia's first military campaign outside former Soviet borders
since the dissolution of the Soviet Union.

Taken together, the operations reflect what officials and analysts
described as a little-noticed --- and still incomplete --- modernization
that has been underway in Russia for several years, despite strains on
the country's budget. And that, as with Russia's intervention in
neighboring Ukraine, has raised alarms in the West.

In a
\href{http://www.ecfr.eu/publications/summary/russias_quiet_military_revolution_and_what_it_means_for_europe}{report}
this month for the European Council on Foreign Relations, Gustav Gressel
argued that Mr. Putin had overseen the most rapid transformation of the
country's armed forces since the 1930s. ``Russia is now a military power
that could overwhelm any of its neighbors, if they were isolated from
Western support,'' wrote Mr. Gressel, a former officer of the Austrian
military.

\includegraphics{https://static01.graylady3jvrrxbe.onion/images/2015/10/01/multimedia/russia-airstrikes/russia-airstrikes-videoSixteenByNine1050.jpg}

Russia's fighter jets are, for now at least, conducting nearly as many
strikes in a typical day against rebel troops opposing the government of
President Bashar al-Assad as the American-led coalition targeting the
Islamic State has been carrying out each month this year.

The operation in Syria --- still relatively limited --- has become, in
effect, a testing ground for an increasingly confrontational and defiant
Russia under Mr. Putin. In fact, as Mr. Putin himself suggested on
Sunday, the operation could be intended to send a message to the United
States and the West about the restoration of the country's military
prowess and global reach after decades of post-Soviet decay.

``It is one thing for the experts to be aware that Russia supposedly has
these weapons, and another thing for them to see for the first time that
they do really exist, that our defense industry is making them, that
they are of high quality and that we have well-trained people who can
put them to effective use,'' Mr. Putin said in an interview broadcast on
state television. ``They have seen, too, now that Russia is ready to use
them if this is in the interests of our country and our people.''

Russia's swift and largely bloodless takeover of Crimea in 2014 was
effectively a stealth operation, while its involvement in eastern
Ukraine, though substantial, was conducted in secrecy and obfuscated by
official denials of direct Russian involvement. The bombings in Syria,
by contrast, are being conducted openly and are being documented with
great fanfare by the Ministry of Defense in Moscow, which distributes
targeting video in the way the Pentagon did during the Persian Gulf war
in 1991.

That has also given officials and analysts far greater insight into a
military that for nearly a quarter-century after the collapse of the
Soviet Union was seen as a decaying, insignificant force, one so hobbled
by aging systems and so consumed by corruption that it posed little real
threat beyond its borders.

``We're learning more than we have in the last 10 years,'' said Micah
Zenko, a senior fellow at the Council on Foreign Relations, noting the
use of the new strike fighters and the new cruise missile, known as the
Kalibr. ``As it was described to me, we are going to school on what the
Russian military is capable of today.''

\href{https://www.nytimes3xbfgragh.onion/interactive/2015/10/14/world/europe/russia-frozen-zones-syria.html}{}

\includegraphics{https://static01.graylady3jvrrxbe.onion/images/2015/10/15/world/frozen5/frozen5-videoLarge.jpg}

\hypertarget{frozen-zones-how-russia-maintains-influence-in-the-post-cold-war-era}{%
\subsection{Frozen Zones: How Russia Maintains Influence in the
Post-Cold War
Era}\label{frozen-zones-how-russia-maintains-influence-in-the-post-cold-war-era}}

Modern Russia has inflamed conflict in former Soviet republics to create
``frozen zones,'' allowing it to influence events and confound its
opponents.

The capabilities on display in Syria --- and before that in Ukraine ---
are the fruits of Russia's short, victorious
\href{http://www.nytimes3xbfgragh.onion/2008/08/09/world/europe/09georgia.html?pagewanted=all}{war
in Georgia in 2008}. Although Russia crushed the American-trained forces
of Georgia's government, driving them from areas surrounding the
breakaway region of South Ossetia, Russia's ground and air forces
performed poorly.

The Russians lost three fighter jets and a bomber on the first day of
the war that August, and seven over all, according to an
\href{http://www.cast.ru/eng/?id=328}{analysis} conducted after the
conflict. Russian ground forces suffered from poor coordination and
communication, as well as episodes of so-called friendly fire.

In the war's aftermath, Mr. Putin, then serving as prime minister, began
a military modernization program that focused not only on high-profile
procurement of new weapons --- new aircraft, warships and missiles ---
but also on a less-noticed overhaul of training and organization that
included a reduction in the bloated officer corps and the development of
a professional corps of noncommissioned officers.

Russian military spending bottomed out in the mid-1990s but has risen
steadily under Mr. Putin and, despite the falling price of oil and
international sanctions imposed after the annexation of Crimea, it has
surged to its highest level in a quarter-century, reaching \$81 billion,
or 4.2 percent of the country's gross domestic product, a common measure
of military expenditure.

The Russian advancements go beyond new weaponry, reflecting an increase
in professionalism and readiness. Russia set up its main operations at
an air base near Latakia in northwestern Syria in a matter of three
weeks, dispatching more than four dozen combat planes and helicopters,
scores of tanks and armored vehicles, rocket and artillery systems, air
defenses and portable housing for as many as 2,000 troops. It was
Moscow's largest deployment to the Middle East since the Soviet Union
deployed in Egypt in the 1970s.

``What continues to impress me is their ability to move a lot of stuff
real far, real fast,'' Lt. Gen. Ben Hodges, the commander of United
States Army forces in Europe, said in an interview.

\href{https://www.nytimes3xbfgragh.onion/interactive/2015/09/30/world/middleeast/syria-control-map-isis-rebels-airstrikes.html}{}

\includegraphics{https://static01.graylady3jvrrxbe.onion/images/2015/09/30/world/middleeast/syria-control-map-isis-rebels-airstrikes-1443670697166/syria-control-map-isis-rebels-airstrikes-1443670697166-videoLarge-v12.jpg}

\hypertarget{syria-and-rebels-battle-for-aleppo-as-cease-fire-collapses}{%
\subsection{Syria and Rebels Battle for Aleppo as Cease-Fire
Collapses}\label{syria-and-rebels-battle-for-aleppo-as-cease-fire-collapses}}

A drastic escalation of fighting in Aleppo has shattered a partial
truce.

Since its
\href{http://www.nytimes3xbfgragh.onion/2015/10/01/world/europe/russia-airstrikes-syria.html?_r=0}{air
campaign started} on Sept. 30, Russia has quickly ramped up its
airstrikes from a handful each day to nearly 90 on some days, using more
than a half-dozen types of guided and unguided munitions, including
fragmentary bombs and bunker busters for hardened targets, American
analysts said.

Russia is not only bringing some of its most advanced hardware to the
fight, it has also deployed large field kitchens and even dancers and
singers to entertain the troops --- all signs that Moscow is settling in
for the long haul, American analysts said.

``They brought the whole package,'' said Jeffrey White, a former Middle
East analyst with the Defense Intelligence Agency now at the Washington
Institute for Near East Policy. ``It showed me they could deploy a
decent-sized expeditionary force.''

For now, Russia's focus in Syria is mainly an air campaign with some 600
marines on the ground to protect the air base in Latakia. Mr. Putin has
excluded the idea of sending in a larger ground force to assist the
Syrians.

Michael Kofman, an analyst with the CNA Corporation, a nonprofit
research institute, and a fellow at the Kennan Institute in Washington
who studies the Russian military, said that the operations over Syria
showed that Russia has caught up to the capabilities the United States
has used in combat since the 1990s. That nonetheless represented
significant progress given how far behind the Russians had fallen.

``Conducting night strikes, with damage assessments by drones, is a
tangible leap for Russia into a mix of 1990s and even current Western
combat ability,'' he said.

\includegraphics{https://static01.graylady3jvrrxbe.onion/images/2015/10/08/world/putin-video/putin-video-videoSixteenByNine1050.jpg}

The Russian Air Force suffered a series of training accidents over the
spring and summer --- losing at least five aircraft in a matter of
months --- which Mr. Kofman described as ``teething pains'' as pilots
increased operating tempo under Mr. Putin's orders. Even so, Russia's
aviation is ``often painted in the West as some sort of Potemkin
village, which is not the case.''

He and others said that the biggest surprise so far has been the missile
technology on display. The cruise missiles fired from Russian frigates
and destroyers in the Caspian Sea were first tested only in 2012. With a
range said to reach 900 miles, they had not been used in combat before,
and despite the loss of four cruise missiles that crashed in Iran in one
salvo, they represent a technological leap that could prove worrisome
for military commanders in NATO. He noted that the advances in missile
technologies improved the precision and firepower even of aging
Soviet-era ships or aircraft.

``This is an amazingly capable new weapon,'' he added.

Russia's state television network boasted on Monday that from the
Caspian, they could reach the Persian Gulf, the Arabian Peninsula and
the ``entire Mediterranean Sea.'' It went on to note that trials of the
missiles were underway aboard two ships in the Black Sea, which is
bordered by three NATO allies: Turkey, Bulgaria and Romania.

The Moskva, a guided-missile cruiser that is the flagship of Russia's
Black Sea Fleet, based in the newly annexed Crimea, has also deployed
with other ships off the coast of Syria, providing air defenses for the
aircraft and troops Russia has deployed. Those missiles effectively
protect the skies over Syrian territory under control of the government
from aerial incursions, and all but block the establishment of a no-fly
zone in Syria, as many have called for.

American officials say Russia has closely coordinated with its allies to
plan its current fight. Maj. Gen. Qassim Suleimani, the head of Iran's
paramilitary Quds Force, went to Moscow in late July in an apparent
effort to coordinate on the Russian offensive in Syria, and he is also
spearheading the Iranian effort to assist Iraqi militias. ``The broad
outlines were decided months ago,'' said Lt. Gen. Richard P. Zahner,
formerly the Army's top intelligence officer in Europe and in Iraq.

American officials, while impressed with how quickly Russia dispatched
its combat planes and helicopters to Syria, said air power had been used
to only a fraction of its potential, with indiscriminate fire common and
precision-guided munitions used sparingly. It is clear the Russians are
already harvesting lessons from the campaign to apply to their other
military operations, said David A. Deptula, a retired three-star Air
Force general who planned the American air campaigns in 2001 in
Afghanistan and in the gulf war.

``Essentially,'' he said, ``Russia is using their incursion into Syria
as an operational proving ground.''

Advertisement

\protect\hyperlink{after-bottom}{Continue reading the main story}

\hypertarget{site-index}{%
\subsection{Site Index}\label{site-index}}

\hypertarget{site-information-navigation}{%
\subsection{Site Information
Navigation}\label{site-information-navigation}}

\begin{itemize}
\tightlist
\item
  \href{https://help.nytimes3xbfgragh.onion/hc/en-us/articles/115014792127-Copyright-notice}{©~2020~The
  New York Times Company}
\end{itemize}

\begin{itemize}
\tightlist
\item
  \href{https://www.nytco.com/}{NYTCo}
\item
  \href{https://help.nytimes3xbfgragh.onion/hc/en-us/articles/115015385887-Contact-Us}{Contact
  Us}
\item
  \href{https://www.nytco.com/careers/}{Work with us}
\item
  \href{https://nytmediakit.com/}{Advertise}
\item
  \href{http://www.tbrandstudio.com/}{T Brand Studio}
\item
  \href{https://www.nytimes3xbfgragh.onion/privacy/cookie-policy\#how-do-i-manage-trackers}{Your
  Ad Choices}
\item
  \href{https://www.nytimes3xbfgragh.onion/privacy}{Privacy}
\item
  \href{https://help.nytimes3xbfgragh.onion/hc/en-us/articles/115014893428-Terms-of-service}{Terms
  of Service}
\item
  \href{https://help.nytimes3xbfgragh.onion/hc/en-us/articles/115014893968-Terms-of-sale}{Terms
  of Sale}
\item
  \href{https://spiderbites.nytimes3xbfgragh.onion}{Site Map}
\item
  \href{https://help.nytimes3xbfgragh.onion/hc/en-us}{Help}
\item
  \href{https://www.nytimes3xbfgragh.onion/subscription?campaignId=37WXW}{Subscriptions}
\end{itemize}
