Sections

SEARCH

\protect\hyperlink{site-content}{Skip to
content}\protect\hyperlink{site-index}{Skip to site index}

\href{https://www.nytimes3xbfgragh.onion/section/politics}{Politics}

\href{https://myaccount.nytimes3xbfgragh.onion/auth/login?response_type=cookie\&client_id=vi}{}

\href{https://www.nytimes3xbfgragh.onion/section/todayspaper}{Today's
Paper}

\href{/section/politics}{Politics}\textbar{}As Mayor, Bernie Sanders Was
More Pragmatist Than Socialist

\url{https://nyti.ms/1lKcLhp}

\begin{itemize}
\item
\item
\item
\item
\item
\end{itemize}

Advertisement

\protect\hyperlink{after-top}{Continue reading the main story}

Supported by

\protect\hyperlink{after-sponsor}{Continue reading the main story}

\hypertarget{as-mayor-bernie-sanders-was-more-pragmatist-than-socialist}{%
\section{As Mayor, Bernie Sanders Was More Pragmatist Than
Socialist}\label{as-mayor-bernie-sanders-was-more-pragmatist-than-socialist}}

\includegraphics{https://static01.graylady3jvrrxbe.onion/images/2015/11/25/us/26SANDERSweb1/26SANDERSweb1-articleLarge.jpg?quality=75\&auto=webp\&disable=upscale}

By
\href{http://www.nytimes3xbfgragh.onion/by/katharine-q-seelye}{Katharine
Q. Seelye}

\begin{itemize}
\item
  Nov. 25, 2015
\item
  \begin{itemize}
  \item
  \item
  \item
  \item
  \item
  \end{itemize}
\end{itemize}

BURLINGTON, Vt. --- When
\href{http://www.nytimes3xbfgragh.onion/interactive/2015/05/01/us/elections/bernie-sanders.html?inline=nyt-per}{Bernie
Sanders}, a self-declared socialist, served as mayor here in the 1980s,
he often complained that the United States had its priorities wrong,
that it should be diverting money from the military to domestic needs
like housing and health care.

So when dozens of antiwar activists blocked the entrance to the local
General Electric plant because it was manufacturing Gatling guns to
fight the socialists in Central America, the protesters expected the
mayor's full support.

Instead, he lined up with union officials and watched as the police made
arrests, saying later that in blocking the plant, the activists were
keeping workers from their jobs.

It was a classic example of how Mr. Sanders governed --- as a
pragmatist. He tended to talk globally but act locally, in this case
choosing the real and immediate socialist principle of protecting
workers over blocking the making of weapons to fight leftists abroad.
Although he often shouted about foreign affairs, Mr. Sanders was
consumed with running the city.

Now 74 and the junior senator from Vermont, Mr. Sanders sometimes cites
his eight years as mayor as he seeks the Democratic nomination for
president. His mayoralty was his only experience as a chief executive,
and it showed him to be a leader guided more by practicality than
ideology.

The mayor who was quick to condemn millionaires also imposed fiscal
discipline here in this laid-back blue-collar university town of 38,000
residents. He used a budget surplus not to experiment with a socialist
concept like redistributing wealth but to fix the city's deteriorating
streets. And he oversaw the revitalization of downtown, often working
with big business.

Back then, the Democrats were considered the old guard, his adversaries;
in many cases, Mr. Sanders aligned himself with Republicans to get
things done.

``Even though he talks revolution, he's an incrementalist,'' said
Richard Sugarman, a longtime friend and a professor of religion at the
University of Vermont. ``He knows that things will only be changed
little by little, one by one. That's why he's been effective.''

Critics on the right said their socialist mayor gave the city a bad
image, wasting time on foreign affairs, including trips to Nicaragua and
the Soviet Union. At the same time, critics on the left said he
compromised too much with business interests and did not go far enough
in pursuing socialist ideals. Over the span of his mayoralty, the number
of families living in poverty grew --- to 798 in 1990 from 563 in 1980,
an increase of 42 percent.

\includegraphics{https://static01.graylady3jvrrxbe.onion/images/2015/11/26/us/26-JPSANDERS1/26-JPSANDERS1-articleLarge.jpg?quality=75\&auto=webp\&disable=upscale}

Still, he was re-elected three times, each with an increasing share of
the vote. Under his watch, Burlington, Vermont's largest city, cropped
up on lists of the best places to live. U.S. News and World Report named
him one of the nation's 20 top mayors in 1987, crediting him with
preserving affordable housing, holding the line on property taxes and
making a serious push for home rule in a state where cities had little
autonomy.

``He learned how to use the levers of local government to improve
people's lives,'' said Peter Dreier, a professor of politics and public
policy at Occidental College who studied Burlington during Mr. Sanders's
mayoralty.

This was not necessarily what many had expected.

The arrival in the mayor's office in 1981 of a self-described socialist,
who hung a portrait of Eugene V. Debs on his wall, put Burlington on the
political map --- but as something of a joke. Garry Trudeau, creator of
``Doonesbury,'' called it ``The People's Republic of Burlington.'' Two
weeks after Mr. Sanders was elected, unseating the entrenched Democratic
mayor by just 10 votes, François Mitterrand, a socialist, was elected
president of France. This spawned the slogan ``As Burlington goes, so
goes France.''

But Mr. Sanders had more local concerns. Chief among them: the powerful
board of aldermen, now called the City Council.

Of the 13 aldermen, 11 opposed him and blocked everything he tried to
do. They were convinced that Mr. Sanders's whisker-thin victory had been
a fluke and were determined to stifle him. They fired his secretary.
They refused to let him appoint his own cabinet. The city clerk opened
his mail.

The voluble Mr. Sanders did not sit idly by. The Burlington Free Press
described that first year as ``one long shouting match.''

He finally gained his footing in March 1982 when he mounted a campaign
against some of the aldermen who faced re-election. He mobilized voters,
a tactic that would become a Sanders hallmark. And on Election Day, most
of the old guard Democrats were tossed out, bolstering his progressive
coalition of ``Sanderistas.''

``Bernie couldn't manage his way out of a paper bag,'' said Garrison
Nelson, a political scientist at the University of Vermont. ``But he
brought on board an extremely talented group of people.''

Mr. Sanders, frugal by nature, set the tone. And together, they
conducted the first audit of Burlington's pension system in a
quarter-century. They moved the city's money into higher-yielding
accounts. They raised fees for building permits and for utilities that
dug up the city's streets. And they ended the cronyism by which the
city's insurance contracts had been let, opening them to competitive
bidding.

Image

A supporter made a last-minute appeal for Mr. Sanders's re-election in
Burlington in 1983.Credit...Donna Light/Associated Press

Taken together, these moves saved the city hundreds of thousands of
dollars.

``Our slogan was we would `out-Republican the Republicans,''' said John
Franco Jr., who was assistant city attorney in the Sanders
administration. ``The Republicans on the board liked that, and so on
fiscal issues, they would side with us and we would have a governing
coalition.''

For all of his socialist oratory --- his first speech to the local
chamber of commerce denounced Washington's support for ``fascist
dictatorships'' in Latin America --- Mr. Sanders turned out to be good
for business. Even though he imposed new taxes, on hotels, restaurants
and bars, businesses did not flee.

Mr. Sanders also formed alliances of convenience, including one with
Antonio Pomerleau, a wealthy developer whom Mr. Sanders had criticized
during his first campaign for mayor. Mr. Pomerleau visited him after the
election. ``I said, `Congratulations, you're the mayor, but it's still
my town,''' Mr. Pomerleau, now 98, recalled in an interview. He told him
he was a Republican, but added, ``If you come up with good ideas for
Burlington, I'll back you up.'''

Mr. Sanders came up with several that Mr. Pomerleau found agreeable,
like raising the salaries of police officers. Mr. Sanders overlooked the
mogul's status as a 1 percenter, saying he was a self-made capitalist,
not a corporate capitalist, and relied on his advice. They remain
friends to this day.

If Mr. Sanders had a guiding political philosophy then, it might best be
described as an amalgam of economic pragmatism, political savvy and a
dash of his own brand of socialist theory. He defined that theory as
``opening up the doors of government, paying a special attention to the
needs of poor people and working people.''

This was the logic behind his support for the workers at the General
Electric plant making Gatling guns, which opened him to criticism from
activists on the left.

``It was a big disappointment that a fellow leftist did not support
us,'' said Jay Moore, a longtime Vermont political activist who was
among those who had blocked the General Electric plant.

But in other instances he was a hard bargainer, and he became a
practiced horse trader.

When he wanted to create a Community and Economic Development Office ---
in part to seize power from the Planning Department, an obstructionist
agency controlled by the old guard --- he won Republican backing by
promising to use it as an instrument of economic growth. But the office
also allowed Mr. Sanders to pursue his own agenda of creating more
affordable housing.

``The creation of CEDO was the beginning of creating a strong mayor,''
said Michael Monte, who was assistant director and later director of the
office. ``It became the administration's policy arm for a wide range of
proposals.''

Image

Waterfront Park in Burlington.Credit...Jacob Hannah for The New York
Times

Among them was Mr. Sanders's initiative to save the Northgate
Apartments, a huge, run-down complex of 336 townhouse-style units near
Lake Champlain. The Sanders administration created a nonprofit entity
that bought the complex from its private owners, stopping the proposed
conversion of Northgate into high-priced condos, which would have driven
out its low-income tenants.

``The key was to make sure the city didn't get gentrified,'' said Mr.
Nelson, the political scientist.

While many on the left applauded his efforts on housing, they were more
critical of Mr. Sanders's stance during the yearslong, convoluted battle
over development of the city's spectacular waterfront along Lake
Champlain.

Mr. Sanders wanted to open up the lakefront, long marred by a decrepit
rail yard, for public use. Eventually, that is what happened. But for a
time he backed a private proposal to build a complex of high-end condos,
hotel and commercial space that critics said would block views of the
lake and limit public access.

More deal maker than ideologue, Mr. Sanders later worked for a
compromise that scaled back the proposal and added public amenities like
green space. He said the compromise, supported by most of the aldermen,
was the best he could get and that the development would expand the
city's tax base, bringing millions of dollars into city coffers.

He then championed a \$6 million bond issue to pay for the
infrastructure and public amenities.

But environmentalists and others accused Mr. Sanders of selling out to
business interests. The dispute led to a highly contentious campaign
over the bond issue.

``We fought like hell,'' recalled Sandy Baird, then part of the Green
movement, now a professor at Burlington College. ``We wanted that land
open to the public.''

Mr. Sanders's side lost. In December 1985, the bond issue garnered 54
percent of the vote but not the two-thirds majority necessary to pass;
many working-class areas voted against it.

With the proposal dead, Mayor Sanders tacked again toward the pragmatic:
The city and state revived a lawsuit to claim the waterfront for public
use. After years of litigation, Vermont's highest court ruled in their
favor, clearing the way for the much-heralded public waterfront of today
--- free of large private high-end development along the shoreline.

Now, the waterfront is Burlington's most valuable asset. Residents and
tourists flock to its leafy open spaces, public docks, restaurants and
bike path.

In May, it provided a picture-perfect backdrop for Mr. Sanders to
\href{https://berniesanders.com/bernies-announcement/}{announce} that he
was running for president.

Advertisement

\protect\hyperlink{after-bottom}{Continue reading the main story}

\hypertarget{site-index}{%
\subsection{Site Index}\label{site-index}}

\hypertarget{site-information-navigation}{%
\subsection{Site Information
Navigation}\label{site-information-navigation}}

\begin{itemize}
\tightlist
\item
  \href{https://help.nytimes3xbfgragh.onion/hc/en-us/articles/115014792127-Copyright-notice}{©~2020~The
  New York Times Company}
\end{itemize}

\begin{itemize}
\tightlist
\item
  \href{https://www.nytco.com/}{NYTCo}
\item
  \href{https://help.nytimes3xbfgragh.onion/hc/en-us/articles/115015385887-Contact-Us}{Contact
  Us}
\item
  \href{https://www.nytco.com/careers/}{Work with us}
\item
  \href{https://nytmediakit.com/}{Advertise}
\item
  \href{http://www.tbrandstudio.com/}{T Brand Studio}
\item
  \href{https://www.nytimes3xbfgragh.onion/privacy/cookie-policy\#how-do-i-manage-trackers}{Your
  Ad Choices}
\item
  \href{https://www.nytimes3xbfgragh.onion/privacy}{Privacy}
\item
  \href{https://help.nytimes3xbfgragh.onion/hc/en-us/articles/115014893428-Terms-of-service}{Terms
  of Service}
\item
  \href{https://help.nytimes3xbfgragh.onion/hc/en-us/articles/115014893968-Terms-of-sale}{Terms
  of Sale}
\item
  \href{https://spiderbites.nytimes3xbfgragh.onion}{Site Map}
\item
  \href{https://help.nytimes3xbfgragh.onion/hc/en-us}{Help}
\item
  \href{https://www.nytimes3xbfgragh.onion/subscription?campaignId=37WXW}{Subscriptions}
\end{itemize}
