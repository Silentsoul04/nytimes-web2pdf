Sections

SEARCH

\protect\hyperlink{site-content}{Skip to
content}\protect\hyperlink{site-index}{Skip to site index}

\href{https://www.nytimes3xbfgragh.onion/section/sports}{Sports}

\href{https://myaccount.nytimes3xbfgragh.onion/auth/login?response_type=cookie\&client_id=vi}{}

\href{https://www.nytimes3xbfgragh.onion/section/todayspaper}{Today's
Paper}

\href{/section/sports}{Sports}\textbar{}Long-Hidden Details Reveal
Cruelty of 1972 Munich Attackers

\url{https://nyti.ms/1OsQ78A}

\begin{itemize}
\item
\item
\item
\item
\item
\end{itemize}

Advertisement

\protect\hyperlink{after-top}{Continue reading the main story}

Supported by

\protect\hyperlink{after-sponsor}{Continue reading the main story}

\hypertarget{long-hidden-details-reveal-cruelty-of-1972-munich-attackers}{%
\section{Long-Hidden Details Reveal Cruelty of 1972 Munich
Attackers}\label{long-hidden-details-reveal-cruelty-of-1972-munich-attackers}}

\includegraphics{https://static01.graylady3jvrrxbe.onion/images/2015/12/02/sports/02MUNICHweb1/02MUNICHweb1-articleLarge.jpg?quality=75\&auto=webp\&disable=upscale}

By \href{http://www.nytimes3xbfgragh.onion/by/sam-borden}{Sam Borden}

\begin{itemize}
\item
  Dec. 1, 2015
\item
  \begin{itemize}
  \item
  \item
  \item
  \item
  \item
  \end{itemize}
\end{itemize}

In September 1992, two Israeli widows went to the home of their lawyer.
When the women arrived, the lawyer told them that he had received some
photographs during his recent trip to Munich but that he did not think
they should view them. When they insisted, he urged them to let him call
a doctor who could be present when they did.

Ilana Romano and Ankie Spitzer, whose husbands were among the Israeli
athletes held hostage and
\href{http://timesmachine.nytimes3xbfgragh.onion/timesmachine/1972/09/06/issue.html}{killed
by Palestinian terrorists at the 1972 Olympics} in Munich, rejected that
request, too. They looked at the pictures that for decades they had been
told did not exist, and then agreed never to discuss them publicly.

The attack at the Olympic Village stands as one of sports' most
horrifying episodes. The eight terrorists, representing a branch of the
\href{http://topics.nytimes3xbfgragh.onion/top/reference/timestopics/organizations/p/palestine_liberation_organization/index.html?inline=nyt-org}{Palestine
Liberation Organization}, breached the apartments where the Israeli
athletes were staying before dawn on Sept. 5, 1972. That began an
international nightmare that lasted more than 20 hours and ended with a
disastrous failed rescue attempt.

The treatment of the hostages has long been a subject of speculation,
but a more vivid --- and disturbing --- account of the attack is
emerging. For the first time, Ms. Romano, Ms. Spitzer and other victims'
family members are choosing to speak openly about documentation
previously unknown to the public in an effort to get their loved ones
the recognition they believe is deserved.

Among the most jarring details are these: The Israeli Olympic team
members were beaten and, in at least one case, castrated.

``What they did is that they cut off his genitals through his underwear
and abused him,'' Ms. Romano said of her husband, Yossef. Her voice
rose.

\includegraphics{https://static01.graylady3jvrrxbe.onion/images/2015/12/02/sports/02MUNICHweb2/02MUNICHweb2-articleLarge.jpg?quality=75\&auto=webp\&disable=upscale}

``Can you imagine the nine others sitting around tied up?'' she
continued, speaking in Hebrew through a translator. ``They watched
this.''

Ms. Romano and Ms. Spitzer, whose husband, Andre, was a fencing coach at
the Munich Games and died in the attack, first described the extent of
the cruelty during an interview for the coming film ``Munich 1972 \&
Beyond,'' a documentary that chronicles the long fight by families of
the victims to gain public and official acknowledgment for their loved
ones. The film is expected to be released early next year.

In subsequent interviews with The New York Times, Ms. Spitzer explained
that she and the family members of the other victims only learned the
details of how the victims were treated 20 years after the tragedy, when
German authorities released hundreds of pages of reports they previously
denied existed.

Ms. Spitzer said that she and Ms. Romano, as representatives of the
group of family members, first saw the documents on that Saturday night
in 1992. One of Ms. Romano's daughters was to be married just three days
later, but Ms. Romano never considered delaying the viewing; she had
been waiting for so long.

The photographs were ``as bad I could have imagined,'' Ms. Romano said.
(The New York Times reviewed the photographs but has chosen not to
publish them because of their graphic nature.)

Mr. Romano, a champion weight lifter, was shot when he tried to
overpower the terrorists early in the attack. He was then left to die in
front of the other hostages and castrated. Other hostages were beaten
and sustained serious injuries, including broken bones, Ms. Spitzer
said. Mr. Romano and another hostage died in the Olympic Village; the
other nine were killed during a failed rescue attempt after they were
moved with their captors to a nearby airport.

Image

Ilana Romano, left, the widow of the weight lifter Yossef Romano, with
Spitzer in 2012.Credit...Jack Guez/Agence France-Presse --- Getty Images

It was not clear if the mutilation of Mr. Romano occurred before or
after he died, Ms. Spitzer said, though Ms. Romano said she believed it
happened afterward.

``The terrorists always claimed that they didn't come to murder anyone
--- they only wanted to free their friends from prison in Israel,'' Ms.
Spitzer said. ``They said it was only because of the botched-up rescue
operation at the airport that they killed the rest of the hostages, but
it's not true. They came to hurt people. They came to kill.''

For much of the past two decades, Ms. Spitzer, Ms. Romano and Pinchas
Zeltzer, the lawyer, mostly kept the grisly details to themselves,
though at least one prominent
\href{http://articles.latimes.com/2002/sep/05/sports/sp-munichmain05/4}{report
about the images surfaced}. When Ms. Romano returned home that first
night, she told her daughters the pictures were ``difficult'' but said
they should not ask her more about them. Her daughters agreed.

At various points over the next 20 years, Ms. Romano said, she did make
occasional references to the mutilation of her husband, but she always
kept the photographs of the episode hidden.

According to Ms. Spitzer, confusion about what had happened to the
victims existed from the beginning. The bodies of the victims were
identified by family or friends in Munich --- Ms. Romano said an uncle
of her husband identified his corpse but was shown only his face ---
and, as per Jewish law, burials were held almost immediately after the
bodies were flown back to Israel.

Since much of the attention from Israeli officials after the attacks
focused on security breaches and mistakes by German and Olympic
officials that had allowed the terrorists to strike, consideration of
the plight of the dead victims had been a priority only to their
families.

Image

Israeli and Olympics flags in~2002 near a plaque in honor of the members
of the Israeli Olympic team killed during the 1972 Munich
Olympics.Credit...Enric Marti/Associated Press

``We asked for more details, but we were told, over and over, there was
nothing,'' Ms. Spitzer said.

In 1992, after doing an interview with a German television station
regarding the 20th anniversary of the attack in which she expressed
frustration about not knowing exactly what happened to her husband and
his teammates, Ms. Spitzer was contacted by a man who said he worked for
a German government agency with access to reams of records about the
attack.

Initially, Ms. Spitzer said, the man, who remained anonymous, sent her
about 80 pages of police reports and other documents. With those
documents, Mr. Zeltzer, the lawyer, and Ms. Spitzer pressured the German
government into releasing the rest of the file, which included the
photographs.

After receiving the file, the victims' families sued the German
government, the Bavarian regional government and the city of Munich for
a ``deficient security concept'' and the ``serious mistakes'' that
doomed the rescue mission, according to the complaint. The suit was
ultimately dismissed because of statute-of-limitations regulations.

Nonetheless, the families have largely focused their efforts on ensuring
a place for remembrance of their loved ones in the fabric of the Olympic
movement. After decades of lobbying, the victims' families were
heartened when the International Olympic Committee, led by a new
president, Thomas Bach, agreed this year to help finance a permanent
memorial in Munich. There are also plans to remember the Munich victims
at the 2016 Summer Games in Rio de Janeiro.

At the moment, the victims will be included in a moment of remembrance
for all athletes who have died at the Olympics; Ms. Spitzer and Ms.
Romano continue to press for the Israeli athletes from Munich to be
remembered apart from athletes who died in competition, arguing that
their deaths were the result of unprecedented evil.

``The moment I saw the photos, it was very painful,'' Ms. Romano said.
``I remembered until that day Yossef as a young man with a big smile. I
remembered his dimples until that moment.''

She hesitated. ``At that moment, it erased the entire Yossi that I
knew,'' she said.

Advertisement

\protect\hyperlink{after-bottom}{Continue reading the main story}

\hypertarget{site-index}{%
\subsection{Site Index}\label{site-index}}

\hypertarget{site-information-navigation}{%
\subsection{Site Information
Navigation}\label{site-information-navigation}}

\begin{itemize}
\tightlist
\item
  \href{https://help.nytimes3xbfgragh.onion/hc/en-us/articles/115014792127-Copyright-notice}{©~2020~The
  New York Times Company}
\end{itemize}

\begin{itemize}
\tightlist
\item
  \href{https://www.nytco.com/}{NYTCo}
\item
  \href{https://help.nytimes3xbfgragh.onion/hc/en-us/articles/115015385887-Contact-Us}{Contact
  Us}
\item
  \href{https://www.nytco.com/careers/}{Work with us}
\item
  \href{https://nytmediakit.com/}{Advertise}
\item
  \href{http://www.tbrandstudio.com/}{T Brand Studio}
\item
  \href{https://www.nytimes3xbfgragh.onion/privacy/cookie-policy\#how-do-i-manage-trackers}{Your
  Ad Choices}
\item
  \href{https://www.nytimes3xbfgragh.onion/privacy}{Privacy}
\item
  \href{https://help.nytimes3xbfgragh.onion/hc/en-us/articles/115014893428-Terms-of-service}{Terms
  of Service}
\item
  \href{https://help.nytimes3xbfgragh.onion/hc/en-us/articles/115014893968-Terms-of-sale}{Terms
  of Sale}
\item
  \href{https://spiderbites.nytimes3xbfgragh.onion}{Site Map}
\item
  \href{https://help.nytimes3xbfgragh.onion/hc/en-us}{Help}
\item
  \href{https://www.nytimes3xbfgragh.onion/subscription?campaignId=37WXW}{Subscriptions}
\end{itemize}
