Sections

SEARCH

\protect\hyperlink{site-content}{Skip to
content}\protect\hyperlink{site-index}{Skip to site index}

\href{https://www.nytimes3xbfgragh.onion/section/movies}{Movies}

\href{https://myaccount.nytimes3xbfgragh.onion/auth/login?response_type=cookie\&client_id=vi}{}

\href{https://www.nytimes3xbfgragh.onion/section/todayspaper}{Today's
Paper}

\href{/section/movies}{Movies}\textbar{}Review: `Anomalisa' Pairs
Charlie Kaufman and Lonely Puppets

\url{https://nyti.ms/1JeqKqd}

\begin{itemize}
\item
\item
\item
\item
\item
\item
\end{itemize}

Advertisement

\protect\hyperlink{after-top}{Continue reading the main story}

Supported by

\protect\hyperlink{after-sponsor}{Continue reading the main story}

\hypertarget{review-anomalisa-pairs-charlie-kaufman-and-lonely-puppets}{%
\section{Review: `Anomalisa' Pairs Charlie Kaufman and Lonely
Puppets}\label{review-anomalisa-pairs-charlie-kaufman-and-lonely-puppets}}

\includegraphics{https://static01.graylady3jvrrxbe.onion/images/2015/12/30/arts/30ANOMALISAJP1/30ANOMALISAJP1-articleLarge-v2.jpg?quality=75\&auto=webp\&disable=upscale}

\begin{itemize}
\tightlist
\item
  Anomalisa\\
  **NYT Critic's Pick Directed by Duke Johnson, Charlie Kaufman
  Animation, Comedy, Drama, Romance R 1h 30m
\end{itemize}

By \href{https://www.nytimes3xbfgragh.onion/by/manohla-dargis}{Manohla
Dargis}

\begin{itemize}
\item
  Dec. 29, 2015
\item
  \begin{itemize}
  \item
  \item
  \item
  \item
  \item
  \item
  \end{itemize}
\end{itemize}

The sad and stingingly painful ``Anomalisa,'' a beautiful big-screen
whatsit, features a throng of whiners, malcontents and depressives along
with one bright soul who hasn't let disappointment break her. They're a
funny, odd group. Some register as generically prickly, full of vinegar
and spit (a few may just be tired after a day's work); others sag, as if
deflating one breath at a time under an unfathomable weight. And while
some carry their burden quietly and alone, others insist on sharing it,
like those people who take deep, accusatory sighs when you bump into
them on the subway.

\href{https://www.nytimes3xbfgragh.onion/interactive/2015/12/18/movies/anomalisa-behind-the-scenes.html}{}

\includegraphics{https://static01.graylady3jvrrxbe.onion/images/2015/12/20/arts/1220ANOMALISA2/1220ANOMALISA2-videoLarge-v2.jpg}

\hypertarget{showing-the-seams-in-anomalisa}{%
\subsection{Showing the Seams in
`Anomalisa'}\label{showing-the-seams-in-anomalisa}}

The directors Charlie Kaufman and Duke Johnson discuss the difficult
work of bringing puppets to life in their stop-motion film.

This is, in other words, the human comedy as brought to you by Charlie
Kaufman. He's best known for his dense, wily, rebuslike screenplays ---
including ``Being John Malkovich,'' ``Adaptation'' and ``Eternal
Sunshine of the Spotless Mind'' --- and least loved for ``Synecdoche,
New York'' (2008), the only other feature he directed before
``Anomalisa.'' A delirious, brutally undersung masterwork about a
tormented theater director who stars in his own self-devouring
production, ``Synecdoche'' closes with a voice providing the ultimate
stage direction: ``Die.'' It seemed like a portentous omen given that
Mr. Kaufman subsequently seemed to disappear for the next seven years.

He didn't; he was busy working, including on the
\href{http://articles.latimes.com/2005/sep/16/entertainment/et-ear16}{play}
that became ``Anomalisa.'' Like that production, the movie stars an
excellent David Thewlis as Michael, an author and motivational speaker
who has traveled to Cincinnati to deliver a speech. He meets a woman,
Lisa (Jennifer Jason Leigh), and they have an intense affair. At this
point it seems like a good idea to mention that all the characters in
the movie are stop-motion puppets. And that all the other roles are
performed by Tom Noonan, an invaluable vocalizer who creates a
supporting cast of thousands (well, dozens) through a voice that rises
and lowers, barks and purrs, and builds the ominous wall of sound that
opens and closes the movie, as if boxing it shut.

\includegraphics{https://static01.graylady3jvrrxbe.onion/images/2016/01/25/t-magazine/25tmag-kaufman/25tmag-kaufman-videoSixteenByNine1050.jpg}

Mr. Kaufman has a co-director this time out, Duke Johnson, who's a
partner in the production company that turned the play into an
animation. They make a seamless team. ``Anomalisa'' is a recognizably
Kaufmanesque creation in its anarchic and mordant humor, its singular
narrative beats and especially in its preoccupations (identity,
authenticity, loneliness, death, love, pleasure, the usual). And this
isn't the first time that Mr. Kaufman's work has involved puppets. The
lead character in ``Being John Malkovich,'' directed by Spike Jonze, is
an unhappy puppeteer who works with
\href{http://www.hubermarionettes.com/bjm/bjminterview.html}{marionettes}
that look like him and his wife. In a surreal turn, the puppeteer finds
a portal into the mind of the title character (played by Mr. Malkovich),
who becomes something of a puppet pulling strings of his own.

It's complicated, as are most of Mr. Kaufman's scripts; ``Anomalisa'' is
more narratively and philosophically streamlined. It also clocks in at a
well-timed 90 minutes, a relatively abbreviated length that fits this
hermetically sealed, precariously unoxygenated world, with its doll-size
scale, human avatars, fabricated environments and locked-down
protagonist. The filmmakers delay Michael's introduction, opening with a
babble that rises against a black screen: Enter, the great abyss! Next
up is a pale cloudy sky --- it's the most expansive image in the movie
as well as the only representation of the natural world --- a dreamscape
that's soon pierced by the plane taking Michael to his talk in
Cincinnati.

\includegraphics{https://static01.graylady3jvrrxbe.onion/images/2015/12/31/multimedia/movies-01012016-anoma/movies-01012016-anoma-videoSixteenByNine1050.jpg}

With his sallow complexion, drooping eyes and air of exhaustion (or
perhaps exasperation), Michael could be merely another business
traveler. Well, except that he's a puppet, one with
\href{http://www.nytimes3xbfgragh.onion/interactive/2015/12/18/movies/anomalisa-behind-the-scenes.html}{strange
black seams} that run along his hairline, down his chin and cut temple
to temple, dissecting his face into discrete quadrants. He's also a
puppet that is shortly popping a prescription pill, a moment that ---
with his melancholic resignation, the usual nightmare of plane travel,
the droningly familiar voices that swell around him --- rapidly makes
Michael feel somewhat real, more recognizable than not. That's because
while ``Anomalisa'' is filled with uncomfortably, at times sourly
humorous moments, it's also a horror movie about the agonizing, banal
surrealism of everyday life.

Mr. Kaufman's gift for quotidian horror remains startling; he's a whiz
at minor miseries. The story progresses through a series of squirmy
encounters with other characters who, despite variations in clothing and
hair, all have the same eerily blank faces. Once Michael breaks free of
the airport herd, most of these faces are attached to service workers of
one type or another who, with degrees of friendliness and hostility
masquerading as affability (or professionalism), roll out like dolls on
an assembly line. One after another, with voices that Mr. Noonan
distinguishes with modulations in pitch and an occasional curse, they
serve Michael: the asthmatic cabdriver; the obsequious hotel clerk and
bellhop; the room-service worker; the grumpy waitress.

\includegraphics{https://static01.graylady3jvrrxbe.onion/images/2015/12/30/arts/30ANOMALISAJP2/30ANOMALISAJP2-articleLarge.jpg?quality=75\&auto=webp\&disable=upscale}

Lisa turns out to be the exception to this manufactured nightmare (she's
the anomaly of the title), and Michael falls hard. ``Your voice!'' he
cries out in wonder, a moment of lyricism that the filmmakers tuck in
between unbuttoned clothes and an admirably uncomfortable, honest sex
scene. Lisa may seem like a mess --- she voluntarily enumerates her
supposed failings, like someone who's memorized other people's
criticisms of her --- yet she's glorious. And Ms. Leigh, who brings Lisa
to trembling life with soft mewls of feeling, perfectly timed pauses and
a poignant a cappella rendition of ``Girls Just Want to Have Fun,''
makes you see how much is at stake both for her and Michael. Whether he
can hear her is one heartskippingly moving question; whether he deserves
to is another.

\emph{``Anomalisa'' is rated R (Under 17 requires accompanying parent or
adult guardian). A puppet penis in action.}

Advertisement

\protect\hyperlink{after-bottom}{Continue reading the main story}

\hypertarget{site-index}{%
\subsection{Site Index}\label{site-index}}

\hypertarget{site-information-navigation}{%
\subsection{Site Information
Navigation}\label{site-information-navigation}}

\begin{itemize}
\tightlist
\item
  \href{https://help.nytimes3xbfgragh.onion/hc/en-us/articles/115014792127-Copyright-notice}{©~2020~The
  New York Times Company}
\end{itemize}

\begin{itemize}
\tightlist
\item
  \href{https://www.nytco.com/}{NYTCo}
\item
  \href{https://help.nytimes3xbfgragh.onion/hc/en-us/articles/115015385887-Contact-Us}{Contact
  Us}
\item
  \href{https://www.nytco.com/careers/}{Work with us}
\item
  \href{https://nytmediakit.com/}{Advertise}
\item
  \href{http://www.tbrandstudio.com/}{T Brand Studio}
\item
  \href{https://www.nytimes3xbfgragh.onion/privacy/cookie-policy\#how-do-i-manage-trackers}{Your
  Ad Choices}
\item
  \href{https://www.nytimes3xbfgragh.onion/privacy}{Privacy}
\item
  \href{https://help.nytimes3xbfgragh.onion/hc/en-us/articles/115014893428-Terms-of-service}{Terms
  of Service}
\item
  \href{https://help.nytimes3xbfgragh.onion/hc/en-us/articles/115014893968-Terms-of-sale}{Terms
  of Sale}
\item
  \href{https://spiderbites.nytimes3xbfgragh.onion}{Site Map}
\item
  \href{https://help.nytimes3xbfgragh.onion/hc/en-us}{Help}
\item
  \href{https://www.nytimes3xbfgragh.onion/subscription?campaignId=37WXW}{Subscriptions}
\end{itemize}
