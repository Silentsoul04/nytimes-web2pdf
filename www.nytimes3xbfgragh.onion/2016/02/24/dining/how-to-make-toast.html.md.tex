Sections

SEARCH

\protect\hyperlink{site-content}{Skip to
content}\protect\hyperlink{site-index}{Skip to site index}

\href{https://www.nytimes3xbfgragh.onion/section/food}{Food}

\href{https://myaccount.nytimes3xbfgragh.onion/auth/login?response_type=cookie\&client_id=vi}{}

\href{https://www.nytimes3xbfgragh.onion/section/todayspaper}{Today's
Paper}

\href{/section/food}{Food}\textbar{}The Secret to Good Toast? It's Your
Freezer

\url{https://nyti.ms/1XKXiuX}

\begin{itemize}
\item
\item
\item
\item
\item
\item
\end{itemize}

Advertisement

\protect\hyperlink{after-top}{Continue reading the main story}

Supported by

\protect\hyperlink{after-sponsor}{Continue reading the main story}

\hypertarget{the-secret-to-good-toast-its-your-freezer}{%
\section{The Secret to Good Toast? It's Your
Freezer}\label{the-secret-to-good-toast-its-your-freezer}}

\includegraphics{https://static01.graylady3jvrrxbe.onion/images/2016/02/24/dining/24TOAST2/24TOAST2-articleLarge.jpg?quality=75\&auto=webp\&disable=upscale}

By Emily Weinstein

\begin{itemize}
\item
  Feb. 23, 2016
\item
  \begin{itemize}
  \item
  \item
  \item
  \item
  \item
  \item
  \end{itemize}
\end{itemize}

Toast lovers, I have a modest proposal for you: Do not bother with bad
bread. Say goodbye to sweet, cottony, lightweight toast, the kind that
squishes under a butter knife or slumps under a blanket of jam.

Just get the good stuff instead, the best bread you're able to buy,
preferably handmade loaves with sturdy crusts and tender crumbs, imbued
with the flavors of fermentation. It's more expensive, and that's no
small thing. But unlike some other items for which you may pay more,
good bread is worth a little extra.

Image

Credit...Michael Kraus for The New York Times

Then always keep it on hand. I would argue that the best way to store
bread isn't to wrap it in foil, plastic or brown paper bags, sheath it
in a pillowcase or stash it in the breadbox. The best way to keep bread
is to put it into the freezer --- sliced.

The slicing is crucial here. (It's also a minor heresy, but hear me
out.) Home bread bakers know that a whole loaf freezes incredibly well.

But when you defrost it, you replicate the problem of a whole loaf fresh
out of the oven: Unless you have a full house, it's a race to finish it
before it goes stale. (Yes, you could make bread crumbs, but with
apologies to devotees of schnitzel and gratins, who actually needs that
many bread crumbs?) There are only two of us at my place, so a big loaf
of fresh bread is difficult to take down.

By contrast, slices of good bread in the freezer practically qualify as
convenience food: single serving and ever ready, the base of a
luxuriously simple breakfast, a satisfying lunch, a restorative snack, a
relaxed supper. And because you have stored your slices in the freezer,
they do not degrade in the quick and nasty style of sliced bread left to
languish at room temperature.

Here's what I do: Whenever I see an alluring loaf of bread, I buy it,
take it home, then start slicing, cutting about half the loaf into
thick, toaster-ready slices. I put the pieces in a plastic zipper bag
and pop them into the freezer. (Halved bagels work well, too.)

Then any time I want a piece of toast, I take a slice out of the freezer
and put it directly into the toaster. Professional bakers may blanch,
but I think the results are nearly as good as toasting a slice from a
day-old loaf on the counter.

You don't need any particular type of toaster. But you do need to think
of your freezer periodically. You don't want to just abandon bread to
time and freezer burn, though I can tell you from experience that
neglected slices will still work, even if the toasted texture won't be
nearly as good. Whatever you do, do not drag the microwave into this, no
matter how deeply frozen the bread. This is between you, the freezer and
the toaster.

Bear in mind that the fresher your bread is when it goes in the freezer,
the better your results will be. So gauge how much you want to eat
fresh, and just freeze the rest. You could even freeze it all at once, a
formidable supply of toast in case the craving strikes.

Advertisement

\protect\hyperlink{after-bottom}{Continue reading the main story}

\hypertarget{site-index}{%
\subsection{Site Index}\label{site-index}}

\hypertarget{site-information-navigation}{%
\subsection{Site Information
Navigation}\label{site-information-navigation}}

\begin{itemize}
\tightlist
\item
  \href{https://help.nytimes3xbfgragh.onion/hc/en-us/articles/115014792127-Copyright-notice}{©~2020~The
  New York Times Company}
\end{itemize}

\begin{itemize}
\tightlist
\item
  \href{https://www.nytco.com/}{NYTCo}
\item
  \href{https://help.nytimes3xbfgragh.onion/hc/en-us/articles/115015385887-Contact-Us}{Contact
  Us}
\item
  \href{https://www.nytco.com/careers/}{Work with us}
\item
  \href{https://nytmediakit.com/}{Advertise}
\item
  \href{http://www.tbrandstudio.com/}{T Brand Studio}
\item
  \href{https://www.nytimes3xbfgragh.onion/privacy/cookie-policy\#how-do-i-manage-trackers}{Your
  Ad Choices}
\item
  \href{https://www.nytimes3xbfgragh.onion/privacy}{Privacy}
\item
  \href{https://help.nytimes3xbfgragh.onion/hc/en-us/articles/115014893428-Terms-of-service}{Terms
  of Service}
\item
  \href{https://help.nytimes3xbfgragh.onion/hc/en-us/articles/115014893968-Terms-of-sale}{Terms
  of Sale}
\item
  \href{https://spiderbites.nytimes3xbfgragh.onion}{Site Map}
\item
  \href{https://help.nytimes3xbfgragh.onion/hc/en-us}{Help}
\item
  \href{https://www.nytimes3xbfgragh.onion/subscription?campaignId=37WXW}{Subscriptions}
\end{itemize}
