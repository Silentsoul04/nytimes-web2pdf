Sections

SEARCH

\protect\hyperlink{site-content}{Skip to
content}\protect\hyperlink{site-index}{Skip to site index}

\href{https://www.nytimes3xbfgragh.onion/section/us}{U.S.}

\href{https://myaccount.nytimes3xbfgragh.onion/auth/login?response_type=cookie\&client_id=vi}{}

\href{https://www.nytimes3xbfgragh.onion/section/todayspaper}{Today's
Paper}

\href{/section/us}{U.S.}\textbar{}After Missteps, U.S. Tightens Rules
for Espionage Cases

\url{https://nyti.ms/1T3m4BY}

\begin{itemize}
\item
\item
\item
\item
\item
\end{itemize}

Advertisement

\protect\hyperlink{after-top}{Continue reading the main story}

Supported by

\protect\hyperlink{after-sponsor}{Continue reading the main story}

\hypertarget{after-missteps-us-tightens-rules-for-espionage-cases}{%
\section{After Missteps, U.S. Tightens Rules for Espionage
Cases}\label{after-missteps-us-tightens-rules-for-espionage-cases}}

\includegraphics{https://static01.graylady3jvrrxbe.onion/images/2016/04/27/us/27espionage-web1/27espionage-web1-articleLarge.jpg?quality=75\&auto=webp\&disable=upscale}

By \href{http://www.nytimes3xbfgragh.onion/by/matt-apuzzo}{Matt Apuzzo}

\begin{itemize}
\item
  April 26, 2016
\item
  \begin{itemize}
  \item
  \item
  \item
  \item
  \item
  \end{itemize}
\end{itemize}

WASHINGTON --- The Justice Department has issued new rules that give
prosecutors in Washington greater oversight and control over national
security cases after the collapse of several high-profile prosecutions
led to allegations that Chinese-Americans were being singled out as
spies.

The new rules are intended to prevent such missteps, but without
undermining a counterespionage mission that is a top priority for the
Obama administration.

In December 2014, the Justice Department dropped charges against two
former Eli Lilly scientists, Guoqing Cao and Shuyu Li, who had been
accused of leaking proprietary information to a Chinese drugmaker. Three
months later, prosecutors
\href{http://www.nytimes3xbfgragh.onion/2015/05/10/business/accused-of-spying-for-china-until-she-wasnt.html?_r=0}{dropped
a case} against Sherry Chen, a government hydrologist in Ohio who had
been charged with secretly downloading information about dams.

Then in September, the Justice Department
\href{http://www.nytimes3xbfgragh.onion/2015/09/12/us/politics/us-drops-charges-that-professor-shared-technology-with-china.html}{dismissed
all charges} against a Temple University professor, Xiaoxing Xi, after
leading physicists testified that prosecutors had entirely misunderstood
the science underpinning their case.

``We cannot tolerate another case of Asian-Americans being wrongfully
suspected of espionage,'' Representative Judy Chu, Democrat of
California, said last fall. ``The profiling must end.''

While those cases raised the specter of Chinese espionage, none
explicitly charged the scientists as spies. The cases involved routine
criminal laws such as wire fraud, so national security prosecutors in
Washington did not oversee the cases.

In a letter last month to federal prosecutors nationwide, Deputy
Attorney General Sally Q. Yates said that would change. All cases
affecting national security, even tangentially, now require coordination
and oversight in Washington. That had always been the intention of the
rule, but Ms. Yates made it explicit.

``The term `national security issue' is meant to be a broad one,'' she
wrote.

Ms. Yates told federal prosecutors that consulting with experienced
national security prosecutors in Washington would help ``ensure prompt,
consistent and effective responses'' to national security cases.

The letter, which was not made public, was provided to The New York
Times by a government official.

John P. Carlin, the Justice Department's top national security
prosecutor,
\href{http://www.mainjustice.com/2014/10/21/dojs-national-security-division-reorganizes-for-cyber-and-corporate-espionage-threats/}{reorganized
his staff} in Washington in recent years to focus more aggressively on
preventing theft of America's trade secrets. The new rules mean that
espionage experts will review cases like Dr. Xi's. Such cases ``shall be
instituted and conducted under the supervision'' of Mr. Carlin or other
top officials, the
\href{https://www.justice.gov/usam/usam-9-90000-national-security}{rules}
say.

Peter R. Zeidenberg, a lawyer for the firm Arent Fox, who represented
Dr. Xi and Ms. Chen, called the new rules ``a very positive step.''

\includegraphics{https://static01.graylady3jvrrxbe.onion/images/2016/04/27/us/27espionage-web2/27espionage-web2-articleLarge.jpg?quality=75\&auto=webp\&disable=upscale}

``It's welcome, and it's overdue,'' he said. ``A bad reaction would be
`We're not going to do anything. Everything is fine.' ''

Several of the cases fell apart when defense lawyers confronted
prosecutors with new evidence or previewed the arguments they planned to
make in court. In traditional white-collar criminal investigations,
those conversations between prosecutors and defense lawyers often happen
before charges are filed. In cases involving even a whiff of espionage,
however, such conversations rarely happen. Authorities worry that
suspects, tipped off to the investigation, will run or destroy evidence.

The absence of those conversations makes it important, then, that such
cases receive an extra layer of review, defense lawyers said.

Ms. Yates did not mention the botched cases in her letter. But at the
Justice Department, they were regarded as unfortunate --- and perhaps
preventable --- black eyes that detracted from a string of successful
espionage prosecutions. The United States faces an onslaught of economic
espionage and other spying from China. Last year, Chinese hackers
\href{http://www.nytimes3xbfgragh.onion/2015/07/10/us/office-of-personnel-management-hackers-got-data-of-millions.html}{stole
a trove of government data} --- including Social Security numbers and
fingerprints --- on more than 21 million people.

Last month, Su Bin, a
\href{http://bigstory.ap.org/article/2eb94d3ffb764e81bddac91c948e1956/chinese-man-pleads-guilty-us-hacking-case}{Chinese
businessman}, pleaded guilty to trying to hack into American defense
contractors to steal information on the F-22 and F-35 fighter jets and
Boeing's C-17 military cargo plane. In January, a Chinese citizen
pleaded guilty to
\href{http://www.nytimes3xbfgragh.onion/2016/01/29/business/international/china-us-monsanto-dupont-corn.html}{trying
to steal corn seeds} from American companies and ship them to China to
replicate their genetic properties.

In the Obama administration's most direct confrontation with China,
\href{http://www.nytimes3xbfgragh.onion/2014/05/22/world/asia/hacking-charges-threaten-further-damage-to-chinese-american-relations.html}{the
Justice Department in 2014 charged five members}of the Chinese People's
Liberation Army with hacking into prominent American companies.

Mr. Zeidenberg and others have argued that rushed cases create suspicion
and unfairly tarnish reputations. In the case against the Eli Lilly
scientists, prosecutors were unsparing in their description.

``If the superseding indictment in this case could be wrapped up in one
word, that word would be `traitor,''' Cynthia Ridgeway, an assistant
United States attorney,
\href{http://www.ibj.com/articles/43949-lilly-scientists-stole-55-million-in-trade-secrets-indictment-alleges}{told
a federal court} in Indiana last year, according to the Indianapolis
Business Journal.

The Justice Department gave no explanation for later dropping the case,
saying only that it was done ``in the interests of justice.''

Prosecutors made a similar statement last year when dropping charges
against Dr. Xi. The dismissal suggested investigators did not understand
and did not do enough to learn the science before they brought charges.
Prosecutors had accused Dr. Xi, chairman of Temple's physics department,
with sharing schematics for a piece of American-made laboratory
equipment, a pocket heater, with China. After leading scientists ---
including the inventor of the pocket heater --- testified that the
schematics showed an entirely different type of heater, the Justice
Department dropped the case.

Though prosecutors dropped charges against Ms. Chen, the government has
said
\href{http://www.nytimes3xbfgragh.onion/2015/09/16/technology/chinese-american-cleared-of-spying-charges-now-faces-firing.html}{it
intends to fire her}. She is fighting that decision.

Advertisement

\protect\hyperlink{after-bottom}{Continue reading the main story}

\hypertarget{site-index}{%
\subsection{Site Index}\label{site-index}}

\hypertarget{site-information-navigation}{%
\subsection{Site Information
Navigation}\label{site-information-navigation}}

\begin{itemize}
\tightlist
\item
  \href{https://help.nytimes3xbfgragh.onion/hc/en-us/articles/115014792127-Copyright-notice}{©~2020~The
  New York Times Company}
\end{itemize}

\begin{itemize}
\tightlist
\item
  \href{https://www.nytco.com/}{NYTCo}
\item
  \href{https://help.nytimes3xbfgragh.onion/hc/en-us/articles/115015385887-Contact-Us}{Contact
  Us}
\item
  \href{https://www.nytco.com/careers/}{Work with us}
\item
  \href{https://nytmediakit.com/}{Advertise}
\item
  \href{http://www.tbrandstudio.com/}{T Brand Studio}
\item
  \href{https://www.nytimes3xbfgragh.onion/privacy/cookie-policy\#how-do-i-manage-trackers}{Your
  Ad Choices}
\item
  \href{https://www.nytimes3xbfgragh.onion/privacy}{Privacy}
\item
  \href{https://help.nytimes3xbfgragh.onion/hc/en-us/articles/115014893428-Terms-of-service}{Terms
  of Service}
\item
  \href{https://help.nytimes3xbfgragh.onion/hc/en-us/articles/115014893968-Terms-of-sale}{Terms
  of Sale}
\item
  \href{https://spiderbites.nytimes3xbfgragh.onion}{Site Map}
\item
  \href{https://help.nytimes3xbfgragh.onion/hc/en-us}{Help}
\item
  \href{https://www.nytimes3xbfgragh.onion/subscription?campaignId=37WXW}{Subscriptions}
\end{itemize}
