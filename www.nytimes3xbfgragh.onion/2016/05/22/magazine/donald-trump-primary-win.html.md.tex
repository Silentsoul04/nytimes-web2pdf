Mr. Trump's Wild Ride

\url{https://nyti.ms/1rTit3v}

\begin{itemize}
\item
\item
\item
\item
\item
\item
\end{itemize}

\includegraphics{https://static01.graylady3jvrrxbe.onion/images/2016/05/22/magazine/22cover/22cover-articleLarge-v4.jpg?quality=75\&auto=webp\&disable=upscale}

Sections

\protect\hyperlink{site-content}{Skip to
content}\protect\hyperlink{site-index}{Skip to site index}

Feature

\hypertarget{mr-trumps-wild-ride}{%
\section{Mr. Trump's Wild Ride}\label{mr-trumps-wild-ride}}

Down the homestretch with the impossible nominee.

Credit...Damon Winter/The New York Times. Prop stylists: Randi Brookman
Harris and Gozde Eker.

Supported by

\protect\hyperlink{after-sponsor}{Continue reading the main story}

By \href{http://www.nytimes3xbfgragh.onion/by/robert-draper}{Robert
Draper}

\begin{itemize}
\item
  May 18, 2016
\item
  \begin{itemize}
  \item
  \item
  \item
  \item
  \item
  \item
  \end{itemize}
\end{itemize}

`Have you seen the latest polls? I'm beating Hillary.''

Donald Trump was on the phone with a man he had never met, a Republican
delegate in Pennsylvania. It was May 2, one day before the Indiana
primary election, and the private plane bearing his last name in
gigantic letters was taxiing along a runway at Indianapolis
International Airport. Trump proceeded to quote the numbers to the man
in Pennsylvania: ahead of Clinton by 2 points in that day's Rasmussen
poll, 3 points behind her in the previous week's George Washington
University poll. These were the only two national polls at the moment
that did not show him lagging behind the Democrat by a wide margin in
the general election, but Trump was a businessman who preferred to
negotiate using numbers that were in his favor.

``I'd love your support, Phil,'' the candidate said as he squinted at
his own handwriting, a scrawl in black marker on a piece of paper. ``You
know, you're the only delegate I've talked to. But I saw you on
television, and you appreciate what I do --- I won your county by a
massive amount, and you're respectful of that, and I just appreciate
what you've said: `Having a moral obligation to support the winner' ---
I hadn't heard a delegate say that before.''

Trump thanked the delegate and hung up just as the Boeing 757 took off,
en route to a final campaign stop in South Bend. He settled into his
plush leather seat, beside a large cardboard box containing various
documents relating to the Trump Organization's sundry enterprises.
``It's hard negotiating elevator rates while you're running for
president,'' he said. On the table before him were some notes for a
speech on law and order prepared by his senior policy adviser, Stephen
Miller, who sat behind the candidate around a table with a few other
aides, including Trump's campaign manager, Corey Lewandowski. On the
more conventional presidential campaigns I have covered --- George W.
Bush, John McCain, Mitt Romney --- the candidate's mobile inner sanctum
was a hive of activity, the advisers hovering constantly over their
boss, rattling off the latest polling data or words of unsolicited
advice from a big donor. On Trump's plane, the aides spoke when spoken
to and otherwise kept to their labors on their laptops.

Trump's attention was on the large flat-screen TV on which various Fox
News pundits were forecasting his probable victory in Indiana's
Republican primary the following day and the bleak implications for his
opponent Ted Cruz. The Republican contest, they all seemed to agree, was
pretty much over. The 69-year-old billionaire now appeared destined to
be Clinton's opponent in the general election. The Fox commentators,
even the ones who favored Trump, seemed to struggle for the words to
convey this eventuality.

The candidate took in the good news with an oddly inert expression.
``Maybe I'll get beat tomorrow,'' he said, for at least the third time
that day. Not a single poll had given him cause for worry. But for all
his swagger, Trump had an awareness of unseen, deal-breaking
contingencies that held his triumphalism in check. He was compulsively
superstitious; twice on other plane trips I had seen him toss a few
granules of salt over his left shoulder after eating. And here he was,
on the day before he would effectively clinch his nomination, calling a
single obscure delegate in a state he had already won in a landslide ---
an implicit nod to the forces aligned against him before resuming the
affect of indomitability.

\includegraphics{https://static01.graylady3jvrrxbe.onion/images/2016/05/22/magazine/22trump4/22trump4-articleLarge.jpg?quality=75\&auto=webp\&disable=upscale}

On the TV, Fox had moved on from the election to footage of the smoky
aftermath of a bombing in Baghdad. Trump rose from his seat and walked
over to the screen for a closer look. ``Boy, this ISIS,'' he murmured.

I asked Trump if he had ever been to Iraq. ``Never!'' he said, sounding
horrified by the thought.

``What's the most dangerous place in the world you've been to?''

He contemplated this for a second. ``Brooklyn,'' he said, laughing.
``No,'' he went on, ``there are places in America that are among the
most dangerous in the world. You go to places like Oakland. Or Ferguson.
The crime numbers are worse. Seriously.''

It was a stark reminder of what set Trump apart from every other
politician in recent memory who had occupied his current position: how
little of the world he had seen beyond the archipelago of boardrooms,
golf courses and high-rise hotels he inhabited, how utterances that by
now would have torpedoed a more normal campaign continued to roll off
his tongue with impunity.

That Trump would emerge as the last candidate standing from a field that
once included 17 seemed at times unimaginable over the five spasmodic
weeks I had spent intermittently in the company of the Trump campaign.
More than during any other stretch over the past year, everyone --- at
times even Trump and his loyal advisers --- seemed hellbent on denying
him victory. Now it was clear that there would be no technicalities, as
some had long suspected, to keep the victory from him; no
self-administered fatal error, as so many had assumed. No, this was it:
the final stage of a process by which Americans accepted that this man,
wholly unlike any politician they had ever seen, was going to
definitely, not maybe, become the standard-bearer of one of the two
political parties of the most powerful nation on earth.

Image

Campaign headquarters on May 4, as John Kasich withdrew from the
presidential race and made Trump the de facto nominee.Credit...Damon
Winter/The New York Times

On the TV, the Fox News pundits were speaking consolingly of the
soon-to-be-vanquished Cruz's political future. Standing in front of the
oversize screen, Trump scoffed: ``I don't think he has much of a
future.'' He returned to his seat and proceeded to scratch out a few
notes for what would be his final speech as a Republican competing for
the nomination.

\textbf{``This is probably} the most successful club anywhere in the
world,'' Trump informed me. ``I have the best building and the best
location.'' It was early on the evening of March 23 at the wood-paneled
bar of Mar-a-Lago, Trump's private resort in Palm Beach, Fla.: an estate
that was envisioned after the death of its original owner, the cereal
magnate Marjorie Merriweather Post, as a winter presidential retreat and
that could conceivably be, by next January, a Trump-trademarked Camp
David. Trump strolled in wearing a navy blazer and white dress shirt ---
no tie --- and appearing slightly tanner than usual. We were supposed to
have met late that morning, to begin my several weeks of following the
campaign. But his communications director, Hope Hicks, emailed shortly
before the scheduled get-together: ``Something has come up, and the boss
is going to be occupied for a few hours.'' I deduced --- correctly, as
it turned out --- that Trump had ditched me for a golf game. It was the
first sunny day all week, and the previous evening the candidate had
crushed Cruz in Arizona, which occasioned some celebration. Now Trump
apologized for having kept me waiting. ``Are you going to have dinner
with us tonight?'' he asked.

Trump sat down across the table from me and next to Hicks and
Lewandowski, who were poring over their smartphones. Opposite them
loomed a painting of a much younger Trump in tennis whites. A waiter
materialized and poured him a Coke. (Trump says that he has never
touched alcohol.)

The month of March had been Trump's best thus far as a presidential
candidate. Although he had, early on, privately rated his chances of
winning the Republican nomination as one in 10, he now seemed poised to
do just that. On March 1, he clobbered Cruz across the South, winning
five of the seven primaries in the region that day --- victories that
wiped out hope, among the many Republicans who viewed Trump as an
apocalyptic threat to their party, that Cruz's support among
evangelicals would form a bulwark against the interloper. Two weeks
later, Trump decisively won Illinois and North Carolina, and seemed to
have squeaked by in Missouri, though the narrow margin there meant that
the result wasn't yet official.

More astounding, he won Florida, beating its native son, Senator Marco
Rubio, by nearly 19 points and forcing him out of the race. Less than
two months earlier, the first-term senator was the Republican Party's
favorite son: precocious and upbeat but exquisitely calibrated, never in
danger of wandering off-message --- in short, the antithesis of Donald
Trump. By early March, Trump had baited him into the tar pit, where he
was reduced to questioning the penis size of the man who called him
``Liddle Marco.''

Image

Hope Hicks, communications director.Credit...Damon Winter/The New York
Times

``He was branded beautifully,'' Trump said, slouching contentedly in his
chair. He turned to Lewandowski. ``Did they ever announce the results of
Missouri?''

``Sir, they're still certifying the counts of the delegates,''
Lewandowski said.

``Am I leading? Have they taken anything away from me?''

``So far you've lost a net of three votes.''

``So when will we know?''

``They're trying to certify this by Friday. They've allocated 25
delegates to you, 15 to Cruz --- there's still 12 out there.''

Trump's brow wrinkled. ``So are they saying I won Missouri by doing
that?'' he asked.

``Not yet,'' Lewandowski patiently explained. ``You've won a series of
congressional districts. You won five of them, which is 25 delegates.
Cruz won three --- so 15 for him.''

Distaste clouded Trump's face. Like most Americans, he had until
recently been almost completely ignorant of the obscure mechanics by
which a candidate became the party nominee. To win the nomination, he
needed the support of 1,237 delegates. Achieving this was not as
straightforward as simply winning the most votes in primaries. In each
state, lifelong party officials largely controlled the
delegate-selection process. This was the Republican establishment's last
front in its war against Trump --- and Trump feared, not without cause,
that his rivals would resort to whatever connivances were necessary to
deny him a 1,237 majority and throw the Republican convention into a
melee of multiple balloting and back-room deal-making.

Image

Donald Trump Jr.Credit...Damon Winter/The New York Times

``What I don't like,'' Trump said, ``is Cruz has a guy working for him''
--- his campaign manager, Jeff Roe --- ``that's one of the most powerful
guys in Missouri. So when I hear there's a revote'' --- there wasn't,
actually --- ``I know too much about politics, so I get it. And I don't
like it.''

Cruz was, perhaps, the only candidate as little-liked among Republican
Party hands as Trump was, but Trump plainly saw the Cruz campaign's
machinations as a reflection of the party establishment's ferocious
determination to stop him. It was no secret that many Republicans viewed
Trump as an explosive device poised to obliterate in a single blast the
party's economic orthodoxy and its ability to project an image of
tolerance. Trump himself had vowed to blow up the party's ``rigged
system.'' And yet he remained somewhat puzzled as to why the party was
so opposed to him. In his view, he had arrived on the scene as something
of a gift to the G.O.P. He had attracted to the polls hordes of
Americans who had previously given up on the party, or on politics as a
whole. Viewers were tuning in to the once-boring Republican debates in
ratings-smashing numbers --- and this, he argued, was ``100 percent
Donald Trump.'' The party had become too obsessed with ideology. ``One
thing I've seen over the years,'' he observed, ``is that the Democrats
stick together, and the Republicans eat their young. That's why they
lose so many elections. You know, a normal, very nice, very likable
Republican would be hard pressed to win.''

Trump did not accept the concern that his more incendiary statements had
alienated women and minorities and thereby made him unelectable. ``I'm
going to be better to women on women's issues than Hillary Clinton and
everybody else combined,'' he would later tell me. Now, sipping his
Coke, he cited his moderate-for-a-Republican view that Planned
Parenthood was a valuable women's health care organization, albeit one
that should not receive federal funding as long as it performed
abortions. ``Frankly, for the general election I think that's a very
good issue for me,'' he said. ``Structurally, it's very hard, almost
impossible, for a heavily conservative Republican to win, because of the
Electoral College. Whereas I bring in Michigan. Look at what I did in
Michigan --- I won it in a landslide, it wasn't even close. So I bring
in Michigan. I maybe bring in New York. Republicans don't even go for
the general election to campaign in New York, because there's no
chance.''

``Illinois!'' Hicks chimed in.

``I win Illinois,'' Trump said of a state in which, by the latest
polling from early March, he was trailing Clinton by 25 points and which
a Republican had not won since 1988.

``The reason they did an autopsy of the party,'' Hicks said, referring
to the Republican National Committee's internal analysis following the
defeat of 2012, ``was because the party was dead! People are accusing
Mr. Trump of killing the party --- well, that's already been done. He's
bringing the party back to life!''

Image

The stuffed animal that staffers call Lion Ted on the couch in the
campaign headquarters.Credit...Damon Winter/The New York Times

Trump said: ``By the way, I'm going to do great with the
African-American vote. One poll came out saying Donald Trump's going to
get 25 percent of the African-American vote.'' Trump was referring to
\href{http://www.surveyusa.com/client/PollReport.aspx?g=d950cadf-05ce-4148-a125-35c0cdab26c6}{last
September's SurveyUSA poll}, which has a margin of error of plus or
minus 10 percentage points. (In 1960 against Kennedy, Nixon received 32
percent of the black vote. Since then, the highest share of the black
vote any Republican nominee has received was Reagan's 14 percent in
1980.) ``And I said, Huh --- why not more? I'm going to do great with
the African-Americans. I'm going to bring back jobs. And I've had good
relations with them.'' And, he said, ``I'm going to do far better with
Hispanics than anyone thought. I have thousands working for me. When
this is over, one of my first pictures is going to be me at the Doral''
--- his golf resort near Miami --- ``with a thousand of my people
working there, most of whom are Hispanic and all who love Trump.''

As we moved to the patio for dinner, Trump signaled for Lewandowski and
Hicks to join us, which seemed to surprise them. We were seated at a
table that afforded a view of the beach while also placing the resort's
owner in the center of everyone else's attention. Trump accepted the
greetings, congratulations and selfie requests with rote magnanimity ---
posing for the camera phones, his forced wince of a smile looked as if
someone were grinding a shoe into his toe --- before dispatching each
well-wisher with an ``Enjoy your evening.'' He regarded the parade of
men in salmon- or lime-colored blazers with a flicker of amusement.
``Right out of central casting,'' he said.

Melania Trump joined us on the patio; Trump doted on her throughout the
meal, often touching her shoulder or leg and calling her ``baby.'' His
eldest son, Donald Jr., sat with his wife at a nearby table, as did
Trump's grandchildren and his youngest son, 10-year-old Barron.
Melania's soft-spokenness and Lewandowski and Hicks's deferentiality ---
both referred to Trump as ``sir'' and ``Mr. Trump'' --- lent the whole
tableau an Old World texture, like a Habsburg patriarch in repose.
``This is fun, right?'' Trump exclaimed. ``Really! We're having a good
time!''

Sometime after 10, he and his wife rose from the table and said good
night. Back in his bedroom just before midnight, he checked his Twitter
feed, as he often did when, he told me, he felt the passing urge to
``knock the crap out of'' somebody. Tonight, one of his eight million
Twitter followers had tweeted a pair of photographs: a flattering image
of Melania alongside one of Cruz's wife, Heidi, with a sort of
prune-faced expression, with the caption ``A picture is worth a thousand
words'' and the hashtag \#NEVERCRUZ. Trump retweeted it from his own
account --- his last public statement of the day.

The next morning, a Thursday, Lewandowski drove Hicks and me from
Mar-a-Lago to Trump's nearby golf resort in one of the candidate's many
cars. ``I'm Corey,'' Lewandowski, in shorts and loafers, explained to
the security guard at the entrance. Then, more emphatically: ``With Mr.
Trump's campaign.'' The guard eyed him skeptically as we drove past.

Image

Ivanka and Eric Trump.Credit...Damon Winter/The New York Times

Though he was Trump's top aide, Lewandowski was viewed by some political
observers in Washington as a glorified body man --- he seldom left the
candidate's side, and he lacked the blue-chip credentials usually
characteristic of front-running campaign strategists. Lewandowski
handled the details, not the vision. He was not a guru. Had he been,
Trump, who is his own guru, would not have hired him. In his briefcase,
Lewandowski carried a bulky black binder. It contained virtually
everything of significance in Trump's political universe: the daily,
weekly and monthly master schedules; the full staff list with everyone's
contact information; a similar list of the campaign's various
contractors; daily talking points for staff and surrogates; a running
tally of the delegate count; a list of Trump endorsers; a metrics chart
of field activities in each state, including the daily number of calls
made and doors knocked; position papers on each major issue; various
documents requiring the candidate's signature; and drafts of coming
speeches. When he was not taking orders from the candidate, he was on
the phone executing them, pacing around with his hand cupped over the
receiver like an offensive coordinator furtively calling in plays.

What Lewandowski did have in common with David Axelrod, Karl Rove and
other marquee strategists was a romanticized view of his candidate ---
one that even Trump, for all his self-regard, didn't seem to share.
Lewandowski saw him as a Braveheart-like hell-raiser tilting against a
party elite that had not seen fit to embrace either of them. Though
Lewandowski had kicked around in the political circles of New Hampshire
for much of the past two decades, he had never seen thousands of people
turn out to greet a candidate there the way they did his new boss. Nor
had he expected the campaigns of more experienced candidates run by
better-known consultants to collapse so quickly and spectacularly in the
face of Trump's challenge. Today, 15 months into the job, Lewandowski
plainly admitted that he was not this campaign's ``architect.'' Instead,
he described himself to me as ``a jockey on American Pharoah. You hold
on and give him a little bit of guidance. But you've got to let him
run.''

Over coffee in the club's sunny dining room overlooking the links,
Lewandowski and Hicks joked about the ``toxic infighting'' that some
media outlets had claimed was bedeviling the campaign. Its four
principals --- Lewandowski, Hicks, the deputy campaign manager, Michael
Glassner and the social-media director, Dan Scavino --- were, Hicks
insisted, extremely close. They had also been made aware of two things
by Trump: There was only one star of the campaign, and there was also
only one communications director. Unlike most who held her job title,
Hicks did not tend to the campaign's messaging strategy. Nor did Hicks,
who is 27, see it as her job to spend evenings sharing off-the-record
insights over drinks with the traveling press corps. The rest of the
Trump team felt similarly. This, combined with the campaign's unusually
long blacklist of media outlets it deemed unfair or unfriendly, had left
reporters with few of the usual means of interpreting the campaign's
inner doings, requiring them to rely instead on more far-flung sources.

Among those was Trump's longtime adviser Roger Stone, an inveterate
mischief-maker in the dark seams of American politics who lived by the
credo that, as he put it, ``the only thing worse than being talked about
is not being talked about.'' Depending on whom you believed, Stone had
either been dismissed by Trump last August or had quit. Trump had also
parted company with Stone's former protégé, Sam Nunberg, who worked for
Trump from 2011 until last August, when it was disclosed that he had
previously posted racist messages about Obama and the Rev. Al Sharpton
on his Facebook page. Nunberg no longer spoke to the candidate; Stone
remained on good terms with Trump but communicated with him
infrequently, usually when Trump called to compliment him on a TV
appearance. Both harbored an intense dislike for Lewandowski, who they
believed had tried to wall off their access to the candidate --- Stone,
whose formative years were spent working for the re-election campaign of
President Richard Nixon, described Lewandowski to me as having ``all of
Bob Haldeman's negative traits and none of his good ones'' --- and
merrily disseminated tales of his imminent professional demise.

Outside Trump World, these whispers dovetailed with a sense in the media
and the political class that a campaign that began as an odd novelty was
evolving into something darker. Trump's rhetoric had been inflammatory
since
\href{http://www.nbcnews.com/news/latino/donald-trump-announces-presidential-bid-trashing-mexico-mexicans-n376521}{his
announcement speech in June,} in which he castigated Mexico for sending
``rapists'' to the United States; in December, after a husband-and-wife
team of Islamic State sympathizers shot 35 people in San Bernardino,
Calif., he issued a statement calling for ``a total and complete
shutdown of Muslims entering the United States until our country's
representatives can figure out what is going on.'' Now reports and
videos were surfacing of Trump supporters flinging racial slurs and,
sometimes, attacking protesters at his rallies. ``If you see somebody
getting ready to throw a tomato, knock the crap out of them, would
you?'' Trump told a crowd in Iowa on Feb. 1. ``Seriously. O.K.? Just
knock the hell --- I promise you, I will pay for the legal fees.''

Image

Michael Glassner, deputy campaign manager, and a Secret Service
bomb-sniffing dog.Credit...Damon Winter/The New York Times

Then on March 8,
\href{http://www.nytimes3xbfgragh.onion/2016/03/30/us/politics/trump-campaign-manager-corey-lewandowski.html}{Lewandowski
grabbed the arm of Michelle Fields,} then a reporter for Breitbart News,
when she approached Trump at a campaign event at the golf club where we
were now sitting, leaving bruises. Fields filed a complaint, and the
stories now circulating portrayed a Trump campaign in a state of
``serious existential threat,''
\href{http://www.politico.com/story/2016/03/donald-trump-corey-lewandowski-220742}{as
one Politico article put it.} Stone had been quoted in that article, and
Nunberg, who would later announce his support for Cruz, had reached out
to Fields through an acquaintance and suggested lawyers to her.

Inside Trump World, these matters were regarded as drastically
overblown. Trump had no intention of punishing Lewandowski for the
Fields incident the way
\href{http://www.cnn.com/2016/02/22/politics/rick-tyler-marco-rubio-video-apology/}{Cruz
had thrown his national spokesman Rick Tyler overboard} the month before
for ill-advised Facebook and Twitter posts. Nevertheless, Trump quietly
issued the order that his rally venues for the time being be smaller,
and thus more easily controlled, even as he stood by his campaign
manager and defended his revolution as a nonviolent one.

At the golf resort, I brought up the more strategic criticism that had
been leveled at the campaign, that Trump needed to turn his guerrilla
squad into something resembling a more conventional operation, and asked
Lewandowski and Hicks how that might happen. ``Ever since we won Nevada,
all these guys have been calling us and saying we had to build out the
team,'' Hicks said. The campaign's core staffers had received this
advice with eye-rolls, recognizing it as a worldview at odds with their
own --- and from time to time would draw up imitation organizational
charts imagining what an expanded Trump World would look like:

Image

But a small cloud was gathering in the otherwise unblemished sky over
Palm Beach. That evening, a Wall Street Journal
\href{http://www.wsj.com/articles/ted-cruz-gains-in-louisiana-after-loss-there-to-donald-trump-1458861959}{article
by Reid J. Epstein was published} online under the headline ``Ted Cruz
Gains in Louisiana After Loss There to Donald Trump.'' Epstein wrote
that although Trump had won that state's primary, Cruz's team was
exploiting the state party's arcane rules to help draw many of the
delegates their way.

The man Trump called ``Lyin' Ted'' was running a campaign operation
that, in the view of Trump World, wasn't half as brilliant as the media
had given it credit for. After all, who had won the evangelical vote in
South Carolina? Who had swept nearly all of the South? Who had snatched
victory in Missouri from the jaws of Cruz's supposed wizard Roe? Still,
Cruz's campaign had found a different way to win.

Trump read the story at Mar-a-Lago the next day. Unnerved, he called
Roger Stone. ``Can they really steal this thing from me?'' Stone later
recalled Trump asking him.

Image

Paul Manafort, who oversees the campaign's delegate
operation.Credit...Damon Winter/The New York Times

Stone told him that yes, such a feat was entirely possible.

\textbf{The last time} anyone in the Republican Party had felt the need
to prepare for a brokered convention was 1976, when former Gov. Ronald
Reagan of California mounted an insurrectionary challenge to President
Gerald Ford. Among the operatives managing Ford's short but intense
convention-floor fight was Paul Manafort, a 27-year-old protégé of
Ford's campaign manager, the future secretary of state James Baker.

Manafort went on to advise several subsequent Republican presidential
campaigns, but since the mid-'80s, much of his counsel had been devoted
to helping foreign leaders including Ferdinand Marcos and Vladimir
Putin's ally in Ukraine, Viktor Yanukovych. Still, with his pinstripe
suits and white-shoe deftness, he represented a steady and low-profile
contrast to Trump's whippetlike campaign manager. He was also more than
25 years Lewandowski's senior --- a true peer to Trump, who often
referred to his traveling entourage as ``the kids.'' As it happened, he
lived on the 43rd floor of Trump Tower, and was Stone's former business
partner.

At Trump's request, Manafort had dinner the evening of March 24 with the
candidate at Mar-a-Lago. Manafort offered his services pro bono --- he
was already plenty wealthy, and presumably preferred an optimized blend
of influence and independence. Four days later, on the morning of
Monday, March 28, members of the campaign staff assembled at the
Washington office of Donald McGahn, Trump's campaign lawyer, for a
secret meeting. The team conferred for three and a half hours. The
manager of Trump's shoestring delegate operation, Ed Brookover, and his
deputies Brian Jack and Alan Cobb, began with a review of the campaign's
current delegate-hunting status state by state. But midway through the
presentation, the discussion spilled over into a deeper examination of
the state of the campaign --- of how the candidate's message should be
shaped and how his operation should be broadened.

As the newcomer in the room, Manafort was deferential but also pointed
in his observations. He told Lewandowski he was taking on a new role
now, according to two people present at the meeting. He was bigger than
just a campaign manager, he said. Senators would want to meet with him
directly, and he should leverage that when he was in Washington. Such
leveraging was, of course, exactly the skill of an establishment hand
like Manafort, not an outsider like Lewandowski. (A spokesman for
Manafort said he did not recall this being said.)

The next morning, March 29, Lewandowski turned himself in to police in
Jupiter, Fla., and was charged with simple battery for the incident with
Michelle Fields. Ultimately the state attorney for Palm Beach County
would decline to prosecute him. What lingered in significance, however,
was the complete senselessness of his denial that he had ever touched
Fields. (The episode was captured on video.) Instead, Lewandowski had
followed the example of his pugnacious boss, which he and Hicks
characterized to me during our meeting at Trump's golf resort in Palm
Beach: Don't back down. Double down.

Image

Memorabilia and fan art line the walls of the campaign
headquarters.Credit...Damon Winter/The New York Times

Trump, meanwhile, had other problems. He was now campaigning in
Wisconsin, where anti-Trump forces were mounting a fierce and skillfully
coordinated effort to deny him the nomination at the convention. ``I've
never said this before, but if I don't win it on the first ballot, the
dishonest establishment will never allow me to win,'' Trump told me
aboard his 757 on the morning of April 5. We were departing Milwaukee,
where voters were going to the polls, and the Fox News pundits on his TV
were dissecting what had been the worst two-week stretch of his young
political career --- one that had begun with his campaign manager's
arrest.

When one commentator made reference to Trump's recent ``unforced
errors,'' Trump said, ``O.K., you can turn the sound down now.'' Scavino
obliged.

Referring to the results of the Wisconsin primary that would arrive that
evening, Trump asked me, ``What do you think is going to happen?''

``You're probably going to lose,'' I said.

He shrugged. ``I have the whole machine against me.''

Surveying his recent setbacks, however, he allowed that he had perhaps
made some mistakes. He had come to regret his decision to retweet the
Heidi Cruz photo that night at Mar-a-Lago, which had dogged him for
weeks now. ``I could've done without it,'' he gruffly acknowledged.
``Some people were offended.'' I asked him if it was strategically wise
to have spent the past week in Wisconsin repeatedly attacking the
state's governor, the former presidential candidate Scott Walker ---
who, granted, was a Cruz supporter but who also enjoyed an 80 percent
favorability rating among the state's Republicans. ``Maybe not,'' Trump
mumbled. ``We'll see.''

Then there was his interview the previous week
\href{http://www.msnbc.com/hardball/watch/trump-s-hazy-stance-on-abortion-punishment-655457859717}{with
the MSNBC host Chris Matthews,} who asked him whether his pro-life views
meant that he also supported criminal penalties for a woman who had an
abortion. Trump had replied that yes, there should be ``some form of
punishment.'' Now he argued to me, rather unconvincingly, that he had
been misinterpreted: ``I didn't mean punishment for women like prison.
I'm saying women punish themselves. I didn't want people to think in
terms of `prison' punishment. And because of that I walked it back.'' A
more believable explanation, furnished by a senior adviser for the Trump
campaign, is this: Trump, a serial non-apologizer, initially saw nothing
wrong with his remark and refused to walk it back. Only when every
network chief executive and over 100 media outlets besieged the Trump
campaign with requests for additional comment on how women should be
punished for abortions did the Trump campaign turn to an ally: Chris
Christie, whose tenure as the Republican governor of the blue state of
New Jersey had given him experience placating both social conservatives
and the moderate voters Trump hoped to attract in the general election.
A member of Christie's political team helped draft a statement that
essentially repudiated Trump's earlier one.

Image

The subcellar at Trump Tower.Credit...Damon Winter/The New York Times

In any other presidential campaign, this string of failures would have
cost someone his or her job. But no heads had rolled in Trump World ---
a tacit acknowledgment by the candidate, perhaps, that responsibility
for the campaign resided in the man with the office on the 26th floor of
Trump Tower. The campaign's inner circle remained intact; Hicks now sat
directly behind Trump on the plane, pecking away at her laptop alongside
Lewandowski, whose eyes were haunted with fatigue and who had lost so
much weight recently (15 pounds, he would later tell me) that his blue
blazer drooped like a cloak around his shoulders. I asked Trump if his
campaign manager's job description had been affected by recent
developments. ``Zero,'' he insisted.

That evening Trump lost Wisconsin by 13 points to Cruz. Further setbacks
followed in Colorado and Wyoming, where Cruz's team outmaneuvered
Trump's in the delegate-apportioning process, as even some of Trump's
staff members would concede to me. Lewandowski thought highly of the
1993 Bill Clinton campaign documentary ``The War Room,'' and admiringly
regarded Clinton's team as a roomful of ``killers.'' The able but
mild-mannered Trump delegate crew, which included Jack, Brookover and
Barry Bennett --- all alumni of Dr. Ben Carson's recently shuttered
campaign --- did not seem to have the appetite for the jugular that
Cruz's team did.

At 8 in the morning on Saturday, April 16, Trump's top staff members
convened on the fifth floor of Trump Tower. Ten months into the race,
the candidate's headquarters looked more like the dingy redoubt of a
soon-to-be-disbanded mayoral campaign than the hypercaffeinated
situation room of a presidential front-runner. On ordinary days, no more
than eight or 10 staffers inhabited the warehouselike floor, which in
the manner of many campaigns was decorated like a politics-obsessed dorm
suite: a model White House topped with pink flamingos, life-size posters
of John Wayne and Ronald Reagan, an oversize plush lion the team had
named Lion Ted.
\href{http://nymag.com/daily/intelligencer/2016/04/inside-the-donald-trump-presidential-campaign.html}{A
recent description in New York Magazine} of its spartan condition
offended the building owner, who protested to me, ``It's this beautiful
raw space!'' He conceded that Hillary Clinton's campaign offices in
Brooklyn might be better appointed --- ``though she never had my
location.''

Manafort and Lewandowski had gathered the team to discuss the campaign's
new structure --- which would now have Manafort overseeing the entire
delegate operation and Lewandowski the campaign apparatus --- and to
introduce its new members, including Rick Wiley, the national political
director, previously Scott Walker's campaign manager. The candidate
strolled into the conference room. ``Wow, this looks like a professional
group of people,'' he said, smiling, according to two sources who were
present. ``All right, guys. I need you to go win. And we're going to
make sure you have what you need to win.''

After speaking for less than two minutes, Trump walked out. For the rest
of the meeting, much was said by everyone in the room, but nothing was
decided, because Manafort and Lewandowski had thoroughly opposing
visions of how the campaign should be run. The Washington-based
strategist believed it was time for Trump to close out the primaries by
taking a more scripted, mollifying approach. The campaign manager held
to the view that people attended a Trump rally fully expecting the same
type of raucous, unpredictable drama they saw at a sporting event. Trump
apparently was listening to both men now. But it was not obvious that
morning whose view would prevail --- or even which of the two had the
authority to give orders. One attendee told me that he came out with no
more clarity than he had before the meeting.

Image

The candidate in his campaign headquarters.Credit...Damon Winter/The New
York Times

When
\href{http://www.politico.com/story/2016/04/donald-trump-campaign-staff-222110}{Politico
broke the news of the secret meeting two days later,} on the evening of
April 18, Trump was en route to a campaign rally in Buffalo aboard his
smaller Citation X airplane, with Hicks, Lewandowski and Ivanka Trump's
husband, Jared Kushner, who was informally advising the campaign. It was
the day before the New York primary, and Trump sat in the front of the
eight-seater plane. ``I have to think about my speech now,'' he told me,
and began composing one on the spot. He leafed through various talking
points and issue memos, from which he culled a few ideas that he then
scribbled on another piece of paper. Once he was done with the other
documents, he tore them in half lengthwise and let the scraps flutter to
the floor.

The plane touched down at the airport, and the waiting fleet of black
sedans whisked the candidate and his entourage to the city's hockey
arena, where the rally would take place. Trump was posing for
photographs with campaign volunteers when Hicks's phone buzzed. It was
Paul Manafort, calling to try to head off another public-relations
controversy.

A woman whom Trump had briefly considered hiring in 2015 to help with
communications strategy,
\href{http://www.politico.com/blogs/2016-gop-primary-live-updates-and-results/2016/04/cheri-jacobus-trump-lewandowski-defamation-lawsuit-222103}{Cheri
Jacobus, was suing the candidate,} his campaign and Lewandowski for
libel after Trump tweeted that she had ``begged my people for a job,''
in addition to a few other disparaging remarks. Trump wanted to punch
back --- it was what he did; and in Lewandowski's view, the candidate's
brawling, politically incorrect impulses were what had made him the
front-runner to begin with.

At the candidate's direction, Hicks had prepared a statement chiding
Jacobus's threat. Manafort was now on the phone urging Trump not to
release the statement. Attacking Jacobus yet again struck him as
unnecessary, not to mention a distraction from the task at hand: winning
big in tomorrow's primary. It also flew in the face of Manafort's
publicly stated vow that his new client would now be evincing a more
``presidential'' affect.

Trump grew more red-faced as he heard Manafort out. Then he said,
``Don't tell me how to {[}expletive{]} do P.R.'' He stepped into a
private room to fix his hair, then posed for a few more photos with the
man who was about to introduce him to the crowd, Rex Ryan, head coach of
the Buffalo Bills. That evening, addressing a hockey arena filled with
perhaps 17,000 delirious Trumpophiles, he bellowed: ``I don't want to
really act more `presidential' until we win!''

\textbf{The following evening} at Trump Tower, the man who stepped out
before the press --- heralded by Sinatra's ``New York, New York'' --- to
celebrate his 35-point victory in his home state's primary appeared
uncharacteristically subdued. He referred to his vanquished opponent not
as ``Lyin' Ted'' but as ``Senator Cruz.'' He held his usual grievances
in check. After eight minutes, he departed the lectern without taking
any questions.

Manafort had managed to impose a veneer of Beltway respectability on the
campaign. More field organizers were now materializing in states like
Pennsylvania, where local volunteers had hitherto been left largely to
fend for themselves. Supporters who previously received no direction
from the campaign before going on TV to expound on the candidate's
policies --- ``I just make {[}expletive{]} up,'' Representative Duncan
Hunter of California confessed to a Trump senior adviser --- were now
receiving daily talking points.

But the moment-to-moment decision-making --- where to go, whom to see,
what to say and how to say it --- still rested almost exclusively upon
the whims of Trump and, secondarily, with the person in his immediate
proximity, who was almost always Lewandowski. That became apparent to me
on the morning of April 25, the day before the string of Northeastern
primaries that would restore Trump's indomitability. The candidate was
seated in the front of his Citation as it departed the airstrip of
Warwick, R.I. --- a stop that, Manafort and Lewandowski agreed, had been
a complete waste of the candidate's time, given that he was ahead of
Cruz there by 40 points. But when Trump told Lewandowski, ``I can't just
not go there,'' there was little point arguing. Lewandowski began making
calls to his advance team on Sunday morning. Some 24 hours later, Trump
walked into a sweaty and delirious tented gathering adjacent to a
Warwick hotel --- exulting, with customary hyperbole: ``We set this up
12 hours ago! There's thousands outside --- we need a bigger tent!''

Later that day, en route to West Chester, Pa., Trump's thoughts kept
wandering afield from politics. He sat with a large stack of newspaper
clippings --- some of them with handwritten notes from his daughter
Ivanka --- at his feet. To his right sat his 32-year-old son Eric, whom
I heard Trump refer to as ``honey.'' He perused some documents relating
to a land deal he was considering, pausing to fret over the fate of his
friend Tom Brady, the New England Patriots quarterback whose four-game
suspension for his role in the Deflategate scandal was upheld that
morning by a federal appeals court: ``He should've sued the N.F.L. in
Boston at the very beginning.'' He asked Lewandowski whether his
campaign schedule would allow him to attend the June 25 grand opening of
his Turnberry golf resort on the coast of Scotland.

``If we get to 1,237, you're there,'' Lewandowski said. ``If we're at
1,100, you're going nowhere.'' Trump scowled a bit but did not protest.
I was reminded that Trump was still fundamentally a real estate
developer with exactly zero previous campaign experience, who had gotten
this far by spending only a fraction of what his opponents had and
against the wishes of his party --- who was as new to the idea of a
Trump candidacy as the rest of us were.

Although his political maturation over the past year had not been
altogether linear, it seemed clear that an understanding of what his
candidacy meant to his supporters was taking root. Trump seemed aware,
despite his insistence that voters of all stripes were drawn to him,
that his constituency came chiefly from white working-class Americans
who felt left out of the Obama recovery and cheated by what they saw as
a rigged economic system. Playing to this sentiment, he had begun to
include in his speeches a litany of dire economic statistics pertaining
to whichever state he happened to be visiting at the time. The data,
compiled by Sam Clovis and Stephen Miller, senior policy advisers,
invariably cited the collapse of that local manufacturing sector over
the past two decades. It had become axiomatic in Trump World that
wherever jobs had been lost was also where Trump's voters could be
found. ``They're great people,'' he murmured back on the plane after the
event in Buffalo. ``And they want help.'' His face crinkled in disgust.
``They don't want \emph{hope}. They want help.''

It was a sobering reminder of the expectations that a President Trump
might find on his shoulders come January. But the moment passed, and his
mood seemed to regain altitude, the desperate souls on the rope line
reaggregating into an adoring mass of victory-assuring,
superlative-defying yugeness. ``So you've covered other people ---
nobody comes close to this,'' he said. ``Two guys from Fox said they've
never seen anything like it.''

We rose upward through the skies in the vehicle Trump referred to as
``just about the fastest plane made,'' eventually passing over the Ferry
Point golf course that Trump said he had built faster than anyone else
could, and finally toward the great Manhattan skyline that Trump had
made even greater --- a taste of what he could do for America, if its
great people would only let him.

Advertisement

\protect\hyperlink{after-bottom}{Continue reading the main story}

\hypertarget{site-index}{%
\subsection{Site Index}\label{site-index}}

\hypertarget{site-information-navigation}{%
\subsection{Site Information
Navigation}\label{site-information-navigation}}

\begin{itemize}
\tightlist
\item
  \href{https://help.nytimes3xbfgragh.onion/hc/en-us/articles/115014792127-Copyright-notice}{©~2020~The
  New York Times Company}
\end{itemize}

\begin{itemize}
\tightlist
\item
  \href{https://www.nytco.com/}{NYTCo}
\item
  \href{https://help.nytimes3xbfgragh.onion/hc/en-us/articles/115015385887-Contact-Us}{Contact
  Us}
\item
  \href{https://www.nytco.com/careers/}{Work with us}
\item
  \href{https://nytmediakit.com/}{Advertise}
\item
  \href{http://www.tbrandstudio.com/}{T Brand Studio}
\item
  \href{https://www.nytimes3xbfgragh.onion/privacy/cookie-policy\#how-do-i-manage-trackers}{Your
  Ad Choices}
\item
  \href{https://www.nytimes3xbfgragh.onion/privacy}{Privacy}
\item
  \href{https://help.nytimes3xbfgragh.onion/hc/en-us/articles/115014893428-Terms-of-service}{Terms
  of Service}
\item
  \href{https://help.nytimes3xbfgragh.onion/hc/en-us/articles/115014893968-Terms-of-sale}{Terms
  of Sale}
\item
  \href{https://spiderbites.nytimes3xbfgragh.onion}{Site Map}
\item
  \href{https://help.nytimes3xbfgragh.onion/hc/en-us}{Help}
\item
  \href{https://www.nytimes3xbfgragh.onion/subscription?campaignId=37WXW}{Subscriptions}
\end{itemize}
