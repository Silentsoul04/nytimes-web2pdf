Sections

SEARCH

\protect\hyperlink{site-content}{Skip to
content}\protect\hyperlink{site-index}{Skip to site index}

\href{https://www.nytimes3xbfgragh.onion/section/world/asia}{Asia
Pacific}

\href{https://myaccount.nytimes3xbfgragh.onion/auth/login?response_type=cookie\&client_id=vi}{}

\href{https://www.nytimes3xbfgragh.onion/section/todayspaper}{Today's
Paper}

\href{/section/world/asia}{Asia Pacific}\textbar{}Lured by Jobs in
Afghanistan, Nepalis Face Risks and Death

\begin{itemize}
\item
\item
\item
\item
\item
\end{itemize}

Advertisement

\protect\hyperlink{after-top}{Continue reading the main story}

Supported by

\protect\hyperlink{after-sponsor}{Continue reading the main story}

\hypertarget{lured-by-jobs-in-afghanistan-nepalis-face-risks-and-death}{%
\section{Lured by Jobs in Afghanistan, Nepalis Face Risks and
Death}\label{lured-by-jobs-in-afghanistan-nepalis-face-risks-and-death}}

\includegraphics{https://static01.graylady3jvrrxbe.onion/images/2016/06/29/world/29nepal-web/29nepal-web-articleLarge.jpg?quality=75\&auto=webp\&disable=upscale}

By \href{https://www.nytimes3xbfgragh.onion/by/kareem-fahim}{Kareem
Fahim} and Bhadra Sharma

\begin{itemize}
\item
  June 28, 2016
\item
  \begin{itemize}
  \item
  \item
  \item
  \item
  \item
  \end{itemize}
\end{itemize}

KABUL, Afghanistan --- With their wraparound sunglasses and easy smiles,
the Nepalese contractors who guarded the Canadian Embassy had become
respected fixtures in Kabul, part of a larger community of South Asian
security guards who stood sentry at foreign missions all around the
capital.

Driven to work in Afghanistan by
\href{https://www.washingtonpost.com/world/asia_pacific/nepal-once-known-for-farming-now-exports-people-migrants-earn-big-but-face-risks/2014/03/23/5858ca52-8441-11e3-bbe5-6a2a3141e3a9_story.html}{collapsed
economic prospects} back home, the contractors were able to send
desperately needed money back to their families. But nothing here is
without risk.

Last week, a Taliban suicide bomber
\href{https://www.nytimes3xbfgragh.onion/2016/06/21/world/asia/afghanistan-kabul-suicide-attack.html}{killed
15 men}, 13 Nepalese and two Indian contractors who helped secure the
embassy, striking the guards' commuter bus just after it had picked them
up at their residence compound. It was one of the deadliest attacks on
foreign workers in the capital --- and another example of how the South
Asian contractors who have become mainstays in places like Afghanistan
and Iraq are vulnerable in ways that many of their Western counterparts
are not.

Many of the Nepalese guards had worked for months just to recover the
\href{http://america.aljazeera.com/watch/shows/fault-lines/Episodes/2014/3/7/america-s-war-workers.html}{thousands
of dollars in broker fees} they had paid to secure jobs in Afghanistan.
One of the guards, Prem Bahadur Tamang, 38, said that they enjoyed fewer
privileges in their barracks than their ``white brothers.'' He added
that among other restrictions, for instance, they were prevented from
leaving their compound to go to a store.

And they were shuttled around Kabul in ordinary minibuses, not in the
armored cars that protect many Western contractors.

\includegraphics{https://static01.graylady3jvrrxbe.onion/images/2016/06/29/world/29nepal-web2/29nepal-web2-articleLarge.jpg?quality=75\&auto=webp\&disable=upscale}

The bodies of the latest victims arrived in Nepal last Wednesday. The
prime minister, K.P. Sharma Oli, laid garlands of marigold on the white
and red coffins, one by one. Around him, relatives of the security
guards sobbed or collapsed.

At least 24 security guards who had worked at the Canadian Embassy in
Kabul accompanied the bodies home, shaken by the deaths and angered that
their entreaties for better protection on the roads had been ignored.

``I lost longtime friends,'' said one of the guards, Mani Ram Khanal, a
17-year veteran of the Nepalese Army who returned to Nepal and called
the bombing ``the most shocking incident of my life.''

He said fewer of his colleagues would have died ``if safer vehicles were
used to transport security guards from one place to another,'' adding
that the guards had complained about the buses.

``They did not take it seriously,'' he said.

In response to questions about the Canadian Embassy's use of the
minibuses, a spokeswoman for Global Affairs Canada said in an email
message that she could not comment on specific security measures at the
mission. She said the agency took the safety of its personnel seriously
and did continuous, rigorous reviews of risk.

Sabre International, a private security firm contracted by the Canadian
Embassy to provide security at the mission, did not return messages and
phone calls seeking comment. Other foreign missions, including the
British Embassy, also used minibuses rather than armored vehicles to
transport Nepalese contractors, several guards said. A spokesman for the
British Embassy declined to comment, citing security concerns.

\href{https://www.nytimes3xbfgragh.onion/interactive/2015/09/29/world/asia/afghanistan-taliban-maps.html}{}

\includegraphics{https://static01.graylady3jvrrxbe.onion/images/2015/09/29/world/asia/afghanistan-taliban-maps-1443568419856/afghanistan-taliban-maps-1443568419856-videoLarge-v7.png}

\hypertarget{more-than-14-years-after-us-invasion-the-taliban-control-large-parts-of-afghanistan}{%
\subsection{More Than 14 Years After U.S. Invasion, the Taliban Control
Large Parts of
Afghanistan}\label{more-than-14-years-after-us-invasion-the-taliban-control-large-parts-of-afghanistan}}

At least one-fifth of the country is controlled or contested by the
Taliban.

About 150 Nepalese guards worked for Sabre in Afghanistan before the
bombing, Mr. Tamang said. Thousands of other contractors, including
veterans of British, Indian or Nepalese Gurkha units, are working around
the country for similar firms.

Mr. Tamang said he had worked for the company for five years but
abruptly returned home after the attack last week --- with barely any of
his clothes, saying he was still owed 20 days of wages. Other colleagues
in Kabul were also considering going back to Nepal, he added.

It was not an easy choice, he said, after desperation had driven many of
them to Afghanistan in the first place. ``We have no job opportunities
in Nepal,'' Mr. Tamang said, adding that he had paid a broker the
equivalent of about \$3,300 to get to Kabul, where he was paid \$950 a
month --- a fraction of the salaries paid to Western security
contractors.

Some of the Nepalese guards hoped to build a house. Others, like Mr.
Tamang, were trying to rebuild their lives after Nepal's devastating
earthquake last year. One of the security guards killed in the Kabul
attack had
\href{http://www.nepalitimes.com/blogs/thebrief/2016/06/23/lives-of-our-brothers/}{lost
a son and a daughter in the earthquake}, according to Nepalese media
reports.

The family of Lil Bahadur Gurung, another victim of the bombing, had
been forced to live in a makeshift shelter after their home was
destroyed, and was waiting for Mr. Gurung to send money home so they
could rebuild, according to his daughter, Anita Gurung. ``Our family
lost our breadwinner,'' she said.

Beyond need, there is also the allure of tradition for some who go
abroad.

Nepalese Gurkhas, renowned for their fearlessness, have served in the
British Army since the 19th century. In today's war zones, the
willingness of the Gurkhas to serve long hours in the most dangerous
posts is a source of pride for the soldiers, and it has kept them in
high demand.

But the reality of security jobs is often different from the dream.

In Kabul, the contractors had lived together in the fortified compound
behind a gas station on Jalalabad Street. And they were all but
sequestered when they were not at work in the embassy, spending their
time playing volleyball or Skyping their families, according to Mr.
Tamang.

The Nepalese guards had separate facilities from the 25 or so Western
contractors who also lived at the camp, as well as separate rules. The
Nepalese guards were not allowed to leave the compound, relying on local
Afghan guards to fetch essentials from a nearby store. And they were not
allowed to drink alcohol in their leisure time.

Three days after the attack, the Nepalese government announced that it
would restrict the travel of its citizens to Afghanistan and facilitate
trips for those who wanted to return. But the regulations are seen as
easy to circumvent, and conditions in Nepal are only becoming more
desperate, said Laxman Basnet, the Nepal-based general secretary of the
South Asian Regional Trade Union Council.

The attack might deter people for a few weeks, and give others second
thoughts, but ``there are no job opportunities in Nepal,'' he said.

With thousands of Nepalis working abroad, the latest deaths have added
to a grim procession of bodies returning to Nepal, several at a time,
every day, Mr. Basnet said. People have become inured to deaths, from
violence or after years of toiling abroad. ``It has sunk into our
psyche,'' he said.

It was only by luck that the toll from the bombing last week was not
higher, judging from the shrapnel that had torn the metal surrounding
shops nearby. The bomber struck early, as the Nepalese guards headed out
at 6 a.m. to relieve their colleagues on the night shift at the embassy.

Mr. Tamang said he had been on duty that morning, waiting for the men on
the bus to relieve him. Hours later, after telling the guards about the
attack, Canadian officials sent the men home in an armored vehicle.

Advertisement

\protect\hyperlink{after-bottom}{Continue reading the main story}

\hypertarget{site-index}{%
\subsection{Site Index}\label{site-index}}

\hypertarget{site-information-navigation}{%
\subsection{Site Information
Navigation}\label{site-information-navigation}}

\begin{itemize}
\tightlist
\item
  \href{https://help.nytimes3xbfgragh.onion/hc/en-us/articles/115014792127-Copyright-notice}{©~2020~The
  New York Times Company}
\end{itemize}

\begin{itemize}
\tightlist
\item
  \href{https://www.nytco.com/}{NYTCo}
\item
  \href{https://help.nytimes3xbfgragh.onion/hc/en-us/articles/115015385887-Contact-Us}{Contact
  Us}
\item
  \href{https://www.nytco.com/careers/}{Work with us}
\item
  \href{https://nytmediakit.com/}{Advertise}
\item
  \href{http://www.tbrandstudio.com/}{T Brand Studio}
\item
  \href{https://www.nytimes3xbfgragh.onion/privacy/cookie-policy\#how-do-i-manage-trackers}{Your
  Ad Choices}
\item
  \href{https://www.nytimes3xbfgragh.onion/privacy}{Privacy}
\item
  \href{https://help.nytimes3xbfgragh.onion/hc/en-us/articles/115014893428-Terms-of-service}{Terms
  of Service}
\item
  \href{https://help.nytimes3xbfgragh.onion/hc/en-us/articles/115014893968-Terms-of-sale}{Terms
  of Sale}
\item
  \href{https://spiderbites.nytimes3xbfgragh.onion}{Site Map}
\item
  \href{https://help.nytimes3xbfgragh.onion/hc/en-us}{Help}
\item
  \href{https://www.nytimes3xbfgragh.onion/subscription?campaignId=37WXW}{Subscriptions}
\end{itemize}
