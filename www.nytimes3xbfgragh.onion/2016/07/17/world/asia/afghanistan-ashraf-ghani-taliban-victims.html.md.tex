Sections

SEARCH

\protect\hyperlink{site-content}{Skip to
content}\protect\hyperlink{site-index}{Skip to site index}

\href{https://www.nytimes3xbfgragh.onion/section/world/asia}{Asia
Pacific}

\href{https://myaccount.nytimes3xbfgragh.onion/auth/login?response_type=cookie\&client_id=vi}{}

\href{https://www.nytimes3xbfgragh.onion/section/todayspaper}{Today's
Paper}

\href{/section/world/asia}{Asia Pacific}\textbar{}In Wartime, Ghani
Assumes Role of Comforting Afghans

\url{https://nyti.ms/2a6De84}

\begin{itemize}
\item
\item
\item
\item
\item
\end{itemize}

Advertisement

\protect\hyperlink{after-top}{Continue reading the main story}

Supported by

\protect\hyperlink{after-sponsor}{Continue reading the main story}

\hypertarget{in-wartime-ghani-assumes-role-of-comforting-afghans}{%
\section{In Wartime, Ghani Assumes Role of Comforting
Afghans}\label{in-wartime-ghani-assumes-role-of-comforting-afghans}}

\includegraphics{https://static01.graylady3jvrrxbe.onion/images/2016/07/12/world/xxghani1/xxghani1-articleLarge.jpg?quality=75\&auto=webp\&disable=upscale}

By \href{https://www.nytimes3xbfgragh.onion/by/kareem-fahim}{Kareem
Fahim}

\begin{itemize}
\item
  July 16, 2016
\item
  \begin{itemize}
  \item
  \item
  \item
  \item
  \item
  \end{itemize}
\end{itemize}

KABUL, Afghanistan --- The first condolence call was to the brother of a
dead police officer, struck down in a Taliban ambush last month.
President Ashraf Ghani spoke gently, offering a short prayer for the
dead and words of gratitude for the family's sacrifice. ``We will not
forget anyone,'' he vowed.

It would be impossible, though, for the president to remember everyone.

Calls to the families of the fallen are a dreaded burden of national
leaders everywhere, but
\href{http://topics.nytimes3xbfgragh.onion/top/reference/timestopics/people/g/ashraf_ghani/index.html}{Mr.
Ghani}, who has occasionally had to make dozens of calls in a day,
shoulders a heavier weight than most.

In 2015, his first full year in office, Afghan security forces sustained
their highest death toll in years, losing an estimated 6,000 soldiers,
police officers and other security personnel. And the casualty numbers
so far this year are higher than in the same period last year, according
to Afghan and American officials.

When Mr. Ghani began making condolence calls several months ago, it was
a rare display of respect for soldiers and officers long mistreated by
the government and now battered by the resurgent
\href{https://www.nytimes3xbfgragh.onion/topic/organization/taliban}{Taliban}.
The president's outreach also contrasts with the practices of his
immediate predecessor, Hamid Karzai, who showed solidarity with civilian
victims of the conflict but had a more ambivalent relationship with his
own security forces.

But each call is also a tacit acknowledgment of the government's
struggles on the battlefield and the steep challenge Mr. Ghani faces:
the unrelenting violence raging across Afghanistan's embattled
provinces, crowding out all other concerns, including the president's
pledges to transform
\href{https://www.nytimes3xbfgragh.onion/topic/destination/afghanistan}{Afghanistan}
into a prosperous, or at least viable, state.

``The job that I least wanted was to be the war president,'' Mr. Ghani
said in a recent interview, lamenting the four or more hours a day he
spends on security matters and not on economic development.

\includegraphics{https://static01.graylady3jvrrxbe.onion/images/2016/07/14/multimedia/ghani-afghanistan/ghani-afghanistan-videoSixteenByNine3000.jpg}

``We want to break out of the vicious cycle,'' he said. ``Our culture,
unfortunately, has made loss a routine.''

This month, President Obama
\href{http://www.nytimes3xbfgragh.onion/2016/07/07/world/asia/obama-afghanistan-troops.html}{delayed}
a planned drawdown of American troops from Afghanistan, citing the
continued threat from the Taliban and Afghanistan's ``precarious''
security. Pentagon officials have also recently relaxed the military's
rules of engagement to allow them to provide support, including
airstrikes, to Afghan security forces struggling to roll back the
Taliban's recent gains.

The Taliban
\href{http://www.nytimes3xbfgragh.onion/2015/10/14/world/asia/taliban-afghanistan-kunduz.html}{briefly
captured} the northern city of Kunduz last year, the first time they had
captured a major city in more than a dozen years. Afghan security forces
have faced their bloodiest challenge in southern Helmand Province, where
half of all the security deaths last year occurred.

Afghanistan's unity government, plagued by internal bickering, has been
unable to stem the loss of soldiers or persuade the Taliban to enter
into peace talks. And the climbing death toll has undercut Mr. Ghani's
assertions, nearly two years into his presidency, that Afghanistan's
catalog of woes are mainly inherited from his predecessor.

On the cruelest day, Mr. Ghani called relatives of 43 men, his office
said. Reporters from The New York Times were invited to watch Mr. Ghani
late last month as he made calls to the families of six police officers
and soldiers who were killed over two days.

The president is in some ways an unlikely consoler in chief. A longtime
academic who spent years in exile from Afghanistan, he is given more to
policy debates than to populism. He is also known for bursts of temper.

But sitting at his glass-topped desk recently, he was soothing as he
spoke to the relatives, praising the ``martyrs'' and promising financial
support. Between calls, he sat placidly, occasionally checking a name
off the list of the dead in front of him, before pushing a buzzer that
signaled to his aides to send the next call through.

\includegraphics{https://static01.graylady3jvrrxbe.onion/images/2016/07/12/world/xxghani2/xxghani2-articleLarge.jpg?quality=75\&auto=webp\&disable=upscale}

``He sacrificed himself to secure the country,'' the president told the
uncle of Rahmi Khoda, a police officer who had been killed by a roadside
bomb in Laghman Province, east of Kabul. ``I wish you patience,'' he
said to the uncle of a soldier named Habibullah, killed in the far
western province of Herat, near the border with Iran.

Mr. Ghani had a soft spot for the uncles, he said, having lost two of
his own on a single day decades ago, in an earlier era of war. ``There
is still a hole in my heart,'' he said.

The president spoke to the relatives privately, rather than on speaker
phone, so their reactions were hard to gauge. Mr. Ghani said the
complaints he heard were mostly related to money. The death benefit of
about \$2,300, or a year's salary, is small, he conceded. The
bureaucracy also makes it hard for families to receive the funds
quickly.

``If the president doesn't pay attention, unfortunately the system
doesn't pay attention,'' he said.

Contacted later, several of the relatives said they were grateful for a
show of high-level concern. Mahmud, the brother of the police officer
killed in the Taliban ambush, sobbed on the phone as recalled the
president's call, saying he felt he had been given the attention of the
``whole country and the whole government.''

But after the call, the family's headaches remained, waiting to be
solved only by a benevolent, high official, rather than an efficiently
operating state. ``We are a poor family. We do not have any assets or
land,'' said Mahmud, who uses a single name. So far, the family had
received only about \$440 in compensation, along with rice, flour and
cooking oil.

``If the government wants to, they will help us,'' he said. ``If they
don't want to, we cannot do anything.''

Haroun Mir, a political analyst in Kabul, said Mr. Ghani's calls were
among several measures aimed at improving the morale of the Afghan
security forces. At the very least, Mr. Mir said, they show greater
concern after scandalous and well-publicized episodes of official
neglect, including the government's
\href{http://www.nytimes3xbfgragh.onion/2016/04/18/world/asia/afghanistan-declares-soldiers-dead-then-alive-and-in-debt-for-funerals.html}{charging
of families for burials} after soldiers were declared dead in error.

Image

Mr. Ghani last month visited Afghan soldiers at Dawood hospital, where
several years ago Americans soldiers discovered patient
neglect.Credit...Sergey Ponomarev for The New York Times

After the Taliban
\href{http://www.nytimes3xbfgragh.onion/2016/07/01/world/asia/taliban-afghanistan-police-convoy-bombings.html}{attacked
a convoy} carrying police cadets last month, killing at least 33, Mr.
Ghani fired five senior officers for failing to prevent the attack.

``This is new in Afghanistan,'' Mr. Mir said. ``He is taking a lot of
responsibility himself.''

He added that Mr. Ghani's his frequent contact with army and police
commanders and vigorous denunciations of the Taliban also set him apart
from Mr. Karzai, who courted the insurgents, contributing to ``ambiguity
about who the enemy was.''

But there is no letup in the flood of casualties from the war. Hours
after the president completed his condolence calls, he visited Dawood
National Military Hospital in Kabul, where most of the patients he saw
had been wounded in the fighting in Helmand.

The hospital itself is a stark reminder of the government's past
negligence: In 2010,
\href{http://www.wsj.com/articles/SB10001424053111904480904576496703389391710}{American
officials discovered} injured soldiers dying in their beds from
starvation or medical neglect.

As he walked the wards, the president hailed the bravery of a soldier
who said his unit had cleared a highway and told Mohamed Dawood, who had
been wounded in the city of Kunduz, ``The country is breathing because
of you.''

Another soldier, Abdul-Jalil, needed more than a minute of the
president's time. ``I have not yet been promoted,'' he complained. ``I
don't know the reason.''

Mr. Ghani promised to do something about it and asked an aide to take a
note.

Advertisement

\protect\hyperlink{after-bottom}{Continue reading the main story}

\hypertarget{site-index}{%
\subsection{Site Index}\label{site-index}}

\hypertarget{site-information-navigation}{%
\subsection{Site Information
Navigation}\label{site-information-navigation}}

\begin{itemize}
\tightlist
\item
  \href{https://help.nytimes3xbfgragh.onion/hc/en-us/articles/115014792127-Copyright-notice}{©~2020~The
  New York Times Company}
\end{itemize}

\begin{itemize}
\tightlist
\item
  \href{https://www.nytco.com/}{NYTCo}
\item
  \href{https://help.nytimes3xbfgragh.onion/hc/en-us/articles/115015385887-Contact-Us}{Contact
  Us}
\item
  \href{https://www.nytco.com/careers/}{Work with us}
\item
  \href{https://nytmediakit.com/}{Advertise}
\item
  \href{http://www.tbrandstudio.com/}{T Brand Studio}
\item
  \href{https://www.nytimes3xbfgragh.onion/privacy/cookie-policy\#how-do-i-manage-trackers}{Your
  Ad Choices}
\item
  \href{https://www.nytimes3xbfgragh.onion/privacy}{Privacy}
\item
  \href{https://help.nytimes3xbfgragh.onion/hc/en-us/articles/115014893428-Terms-of-service}{Terms
  of Service}
\item
  \href{https://help.nytimes3xbfgragh.onion/hc/en-us/articles/115014893968-Terms-of-sale}{Terms
  of Sale}
\item
  \href{https://spiderbites.nytimes3xbfgragh.onion}{Site Map}
\item
  \href{https://help.nytimes3xbfgragh.onion/hc/en-us}{Help}
\item
  \href{https://www.nytimes3xbfgragh.onion/subscription?campaignId=37WXW}{Subscriptions}
\end{itemize}
