Sections

SEARCH

\protect\hyperlink{site-content}{Skip to
content}\protect\hyperlink{site-index}{Skip to site index}

\href{https://www.nytimes3xbfgragh.onion/section/world/asia}{Asia
Pacific}

\href{https://myaccount.nytimes3xbfgragh.onion/auth/login?response_type=cookie\&client_id=vi}{}

\href{https://www.nytimes3xbfgragh.onion/section/todayspaper}{Today's
Paper}

\href{/section/world/asia}{Asia Pacific}\textbar{}Afghans Go to Syria to
Fight for Its Government, and Anguish Results

\url{https://nyti.ms/2awhRvT}

\begin{itemize}
\item
\item
\item
\item
\item
\end{itemize}

Advertisement

\protect\hyperlink{after-top}{Continue reading the main story}

Supported by

\protect\hyperlink{after-sponsor}{Continue reading the main story}

\hypertarget{afghans-go-to-syria-to-fight-for-its-government-and-anguish-results}{%
\section{Afghans Go to Syria to Fight for Its Government, and Anguish
Results}\label{afghans-go-to-syria-to-fight-for-its-government-and-anguish-results}}

\includegraphics{https://static01.graylady3jvrrxbe.onion/images/2016/07/28/world/28afghansyria-web/28afghansyria-web-articleLarge.jpg?quality=75\&auto=webp\&disable=upscale}

By \href{https://www.nytimes3xbfgragh.onion/by/kareem-fahim}{Kareem
Fahim}

\begin{itemize}
\item
  July 28, 2016
\item
  \begin{itemize}
  \item
  \item
  \item
  \item
  \item
  \end{itemize}
\end{itemize}

HERAT, Afghanistan --- One woman here in the western Afghan city of
Herat said she had begged her son not to go fight in the Syrian war, but
he charged off anyway, leaving a wife and three children behind. A man
overhearing her story came over to say that his son had left two months
ago, and since then the family has been desperate for news about him.

Another woman, Khadija, whose son Hassan had joined Afghan brigades
fighting alongside the Syrian government, said he had been pulled into
the vicious conflict for the same reasons most of the young men in the
neighborhood had decided to go: ``He could not find work,'' she said.

A teenager standing on the edge of the group, listening to the parents,
said those were hardly isolated stories among the Afghan Shiites of
Herat. The neighborhood, he said, ``is full of them.''

Afghanistan has been hollowed out as its citizens have fled poverty and
war, many seeking work in Pakistan, Iran or Persian Gulf nations, or
risking the perilous trail to Europe. But this specific emigration
pattern --- of thousands of young men flowing into neighboring Iran and
then on to fight alongside the Syrian government and its allies --- has
provoked extraordinary anguish for families here and for Afghanistan's
government, particularly over the past year.

Leaving a country racked by decades of war, the young Afghans who choose
the path to Syria then fall into peril on the bloody front lines of
Aleppo, Homs or other battlegrounds. Iranian state news media and some
Afghan officials suggest that hundreds have been killed in battles over
the past year.

\includegraphics{https://static01.graylady3jvrrxbe.onion/images/2016/07/28/world/JP-AFGHANSYRIA/JP-AFGHANSYRIA-articleLarge.jpg?quality=75\&auto=webp\&disable=upscale}

Thousands of Afghans, almost all of them Shiite Muslims from the Hazara
ethnic minority, have fought in Syria in the past few years, serving in
brigades supporting the government of President Bashar al-Assad,
according to their relatives and commanders in Syria. Most of the Afghan
men are
\href{https://www.theguardian.com/world/2015/nov/05/iran-recruits-afghan-refugees-fight-save-syrias-bashar-al-assad}{recruited
or drawn from the Afghan diaspora within Iran}, a crucial ally of the
Assad government. But more and more of the men are
\href{https://www.theguardian.com/world/2016/jun/30/iran-covertly-recruits-afghan-soldiers-to-fight-in-syria}{coming
directly from Afghanistan} in the past year, in part because of the
increasingly dire state of the Afghan economy, relatives and officials
here say.

The promise of urgently needed salaries --- or at least compensation for
hardship or death, often paid by the Iranian government --- has done
little to comfort the families left behind, or to ease their regret at
the misery that forced their sons to flee in the first place.

In Khadija's case, her son Hassan's decision to go to Syria came after
her husband, who is disabled, lost his land. But she insisted that she
and her husband had urged Hassan not to go.

The exodus has also highlighted the government's failure to ease
\href{https://www.nytimes3xbfgragh.onion/2016/07/25/world/asia/hazara-afghanistan-victims.html}{the
suffering of the Hazaras}, long marginalized in Afghanistan and
discriminated against as Shiite Muslims. Faced with the embarrassing
spectacle of its citizens fighting for a foreign government, the
government has also had to contend with the loss of able-bodied men at a
time when it is desperate for recruits to fight the war at home against
Taliban insurgents.

Though the Afghan men who leave for Syria soon face the miseries of
another incessant war, they have one advantage over some other Afghan
migrants: They are less likely to be deported and forced to return to
Afghanistan. At the border crossing with Iran, a 90-minute drive from
Herat, at least 30 buses arrive several times a week, filled with
Afghans deported from Iran. Some carry families who have lived illegally
in Iran for years.

But most of the deported Afghans were young men --- some as young as 10,
according to aid workers with the International Organization for
Migration --- who stole across the border desperate to find work. Many
said they would return to Iran as soon as they could.

Image

Afghans who were deported from Iran unloaded their luggage from a bus
earlier this month in Herat Province, Afghanistan.Credit...Sergey
Ponomarev for The New York Times

Some of the Afghan fighters head to Syria for religious reasons, seeing
the battle as a war against Sunni extremists or choosing to defend
Shiite holy sites in Syria alongside other Shiite militiamen from
Lebanon or Iraq. Others were coerced or duped into fighting, say human
rights groups. But most were enticed by financial benefits, including
the promise of legal residence for the fighters and their families in
Iran, said Abdul Rahim Ghulami. He is a local official in Herat who said
his brother-in-law was a commander of an Afghan unit fighting in Aleppo.

Iran's government provides a few weeks of training and flies the men to
Syria, where they join one of the Afghan brigades. Those units are
sometimes viewed with suspicion by their own allies: In interviews in
Syria, some of the other fighters from pro-government militias
disparaged the Afghans as too young and poorly trained.

A shop owner in Damascus named Ahmed who works near the Sayyida Zainab
mosque, a revered site for Shiites, said the numbers of Afghan fighters
guarding the mosque had increased in the last six months. They were a
sorrowful lot who complained about their lives in Iran or Afghanistan
when he talked with them, he said, but said they faced little choice if
they wanted to support their families. At least if the men die, they die
as martyrs in a holy war, he added, giving only his first name because
he did not want to be punished.

Casualties among the Afghan fighters were high, said Mr. Ghulami, who
lived in Iran for 24 years. He said he visited the Iranian town of
Mashhad two months ago and saw that its Afghan quarter was blanketed
with black banners that signaled a house in mourning.

The size of the outflow from Afghanistan itself has been harder to
tally, because the government's disapproval has led families to stay
quiet. Mr. Ghulami, who serves as a local mayor in Jebrail, a Hazara
district of Herat with roughly 100,000 residents, estimated that 20
percent of the families there had someone serving in Syria. There was no
way to confirm that number: no funerals of Afghan fighters, and no black
banners to honor the dead.

But in Jebrail, along with another Hazara neighborhood of Herat, called
Khatim al-Anbiya, it is easy to find the relatives or friends the Afghan
fighters had left behind.

At the cigarette kiosk where he worked in Jebrail, a boy named Sayed Ali
remembered his neighbor and classmate, Habibullah, 20, who ran off to
Syria a few years ago, when he was still a teenager. This year, word
came back that Habibullah had been killed in the war. His family packed
up the house and moved to Iran, granted legal residence there because of
the boy's death, Sayed Ali said.

Another high school student, named Jawad, disappeared from his home in
Khatim al-Anbiya two winters ago, leaving his family to assume he had
gone to Iran to find work, according to his uncle, Mohamed Ibrahim.

When his parents last heard from Jawad, he said he was in Syria, and
told his father he was preparing to come home. Then, eight or nine
months ago, a man brought the news that Jawad had been shot in the head
and killed.

Mr. Ibrahim said he was not sure what took Jawad to Syria --- ``No one
can read anyone's heart,'' he said --- but said he thought the boy was
just looking for work. ``They go there because of poverty,'' he said.
Jawad's family left for Iran after he was killed, leaving behind their
one-story house and their cow, their most valuable possession, Mr.
Ibrahim said.

Yazdanbeg Yazdani, a 50-year-old resident of Jebrail with family in
Iran, said that a year ago, he received a call from Iran telling him his
younger brother, named Yunus, had joined the war as an officer and was
killed in a suicide bombing attack.

Mr. Yazdani was unsure why his brother, who was 48, felt compelled to
fight --- whether he supported the Syrian government, or had been forced
into battle, or simply needed the money. The brothers had been separated
decades ago, when Yunus moved to Syria --- their family fractured by
migration, like so many in Afghanistan. Mr. Yazdani could not attend his
brother's funeral, which was held in Iran. But his family there sent him
pictures of the service.

Advertisement

\protect\hyperlink{after-bottom}{Continue reading the main story}

\hypertarget{site-index}{%
\subsection{Site Index}\label{site-index}}

\hypertarget{site-information-navigation}{%
\subsection{Site Information
Navigation}\label{site-information-navigation}}

\begin{itemize}
\tightlist
\item
  \href{https://help.nytimes3xbfgragh.onion/hc/en-us/articles/115014792127-Copyright-notice}{©~2020~The
  New York Times Company}
\end{itemize}

\begin{itemize}
\tightlist
\item
  \href{https://www.nytco.com/}{NYTCo}
\item
  \href{https://help.nytimes3xbfgragh.onion/hc/en-us/articles/115015385887-Contact-Us}{Contact
  Us}
\item
  \href{https://www.nytco.com/careers/}{Work with us}
\item
  \href{https://nytmediakit.com/}{Advertise}
\item
  \href{http://www.tbrandstudio.com/}{T Brand Studio}
\item
  \href{https://www.nytimes3xbfgragh.onion/privacy/cookie-policy\#how-do-i-manage-trackers}{Your
  Ad Choices}
\item
  \href{https://www.nytimes3xbfgragh.onion/privacy}{Privacy}
\item
  \href{https://help.nytimes3xbfgragh.onion/hc/en-us/articles/115014893428-Terms-of-service}{Terms
  of Service}
\item
  \href{https://help.nytimes3xbfgragh.onion/hc/en-us/articles/115014893968-Terms-of-sale}{Terms
  of Sale}
\item
  \href{https://spiderbites.nytimes3xbfgragh.onion}{Site Map}
\item
  \href{https://help.nytimes3xbfgragh.onion/hc/en-us}{Help}
\item
  \href{https://www.nytimes3xbfgragh.onion/subscription?campaignId=37WXW}{Subscriptions}
\end{itemize}
