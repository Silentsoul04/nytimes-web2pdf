Sections

SEARCH

\protect\hyperlink{site-content}{Skip to
content}\protect\hyperlink{site-index}{Skip to site index}

\href{https://www.nytimes3xbfgragh.onion/section/politics}{Politics}

\href{https://myaccount.nytimes3xbfgragh.onion/auth/login?response_type=cookie\&client_id=vi}{}

\href{https://www.nytimes3xbfgragh.onion/section/todayspaper}{Today's
Paper}

\href{/section/politics}{Politics}\textbar{}Inside the Failing Mission
to Tame Donald Trump's Tongue

\url{https://nyti.ms/2b5WSNA}

\begin{itemize}
\item
\item
\item
\item
\item
\item
\end{itemize}

Advertisement

\protect\hyperlink{after-top}{Continue reading the main story}

Supported by

\protect\hyperlink{after-sponsor}{Continue reading the main story}

\hypertarget{inside-the-failing-mission-to-tame-donald-trumps-tongue}{%
\section{Inside the Failing Mission to Tame Donald Trump's
Tongue}\label{inside-the-failing-mission-to-tame-donald-trumps-tongue}}

\includegraphics{https://static01.graylady3jvrrxbe.onion/images/2016/08/14/us/14TRUMP1/14TRUMP1-articleInline-v2.jpg?quality=75\&auto=webp\&disable=upscale}

By \href{http://www.nytimes3xbfgragh.onion/by/alexander-burns}{Alexander
Burns} and
\href{http://www.nytimes3xbfgragh.onion/by/maggie-haberman}{Maggie
Haberman}

\begin{itemize}
\item
  Aug. 13, 2016
\item
  \begin{itemize}
  \item
  \item
  \item
  \item
  \item
  \item
  \end{itemize}
\end{itemize}

Donald J. Trump was in a state of shock: He had just
\href{http://www.nytimes3xbfgragh.onion/2016/06/21/us/politics/corey\%2Dlewandowski\%2Ddonald\%2Dtrump.html?_r=0}{fired
his campaign manager} and was watching the man
\href{http://www.cnn.com/videos/politics/2016/06/20/corey-lewandowski-full-intv-trump-campaign-manager-fired-bash-wolf.cnn}{discuss
his dismissal} at length on CNN. The rattled candidate's advisers and
family seized the moment for an intervention.

Joined by his daughter Ivanka and her husband, Jared Kushner, a cluster
of Mr. Trump's confidants pleaded with him to make that day --- June 20
--- a turning point.

He would have to stick to a teleprompter and end his freestyle
digressions and insults, like his repeated
\href{http://www.nytimes3xbfgragh.onion/2016/06/08/us/politics/paul\%2Dryan\%2Ddonald\%2Dtrump\%2Dgonzalo\%2Dcuriel.html}{attacks
on a Hispanic federal judge}. Paul Manafort, Mr. Trump's campaign
chairman, and Gov. Chris Christie of New Jersey argued that Mr. Trump
had an effective message, if only he would deliver it. For now, the
campaign's polling showed, too many voters described him in two words:
``unqualified'' and ``racist.''

Mr. Trump bowed to his team's entreaties, according to four people with
detailed knowledge of the meeting, who described it on the condition of
anonymity. It was time, he agreed, to get on track.

Nearly two months later, the effort to save Mr. Trump from himself has
plainly failed. He has repeatedly signaled to his advisers and allies
his willingness to change and adapt, but has grown only more volatile
and prone to provocation since then,
\href{http://www.nytimes3xbfgragh.onion/2016/08/01/us/politics/khizr\%2Dkhan\%2Dghazala\%2Ddonald\%2Dtrump\%2Dmuslim\%2Dsoldier.html}{clashing
with a Gold Star family},
\href{http://www.nytimes3xbfgragh.onion/2016/08/10/us/politics/donald-trump-hillary-clinton.html?action=click\&contentCollection=Politics\&module=RelatedCoverage\&region=Marginalia\&pgtype=article}{making
comments} that have been seen as inciting violence and linking his
\href{http://www.nytimes3xbfgragh.onion/2016/08/11/us/politics/trump-rally.html}{political
opponents to terrorism}.

Advisers who once hoped a Pygmalion-like transformation would refashion
a crudely effective political showman into a plausible American
president now increasingly concede that Mr. Trump may be beyond
coaching. He has ignored their pleas and counsel as his
\href{http://www.nytimes3xbfgragh.onion/2016/08/12/us/politics/donald-trump-obama-isis.html}{poll
numbers have dropped}, boasting to friends about the size of his crowds
and maintaining that he can read surveys better than the professionals.

\href{https://www.nytimes3xbfgragh.onion/interactive/2016/upshot/presidential-polls-forecast.html}{}

\includegraphics{https://static01.graylady3jvrrxbe.onion/images/2016/07/19/upshot/presidential-election-forecast-1468902507509/presidential-election-forecast-1468902507509-articleLarge.png}

\hypertarget{2016-election-forecast-who-will-be-president}{%
\subsection{2016 Election Forecast: Who Will Be
President?}\label{2016-election-forecast-who-will-be-president}}

The Upshot's presidential forecast, updated daily.

In private, Mr. Trump's mood is often sullen and erratic, his associates
say. He veers from barking at members of his staff to grumbling about
how he was better off following his own instincts during the primaries
and suggesting he should not have heeded their calls for change.

He broods about his souring relationship with the news media, calling
Mr. Manafort several times a day to talk about specific stories.
Occasionally, Mr. Trump blows off steam in bursts of boyish exuberance:
At the end of a fund-raiser on Long Island last week, he playfully
buzzed the crowd twice with his helicopter.

But in interviews with more than 20 Republicans who are close to Mr.
Trump or in communication with his campaign, many of whom insisted on
anonymity to avoid clashing with him, they described their nominee as
exhausted, frustrated and still bewildered by fine points of the
political process and why his incendiary approach seems to be
sputtering.

He is routinely preoccupied with perceived slights, for example raging
to aides after Senator Marco Rubio of Florida, in his
\href{http://www.nytimes3xbfgragh.onion/2016/06/23/us/politics/marco\%2Drubio\%2Dflorida\%2Dsenate\%2Drace.html}{re-election
announcement}, said he would stand up to the next president regardless
of party. In a visit to Capitol Hill in early July, Mr. Trump
\href{http://www.nytimes3xbfgragh.onion/2016/07/08/us/politics/donald-trump-republican-party.html}{bickered
with two Republican senators} who had not endorsed him; he needled
Representative Peter T. King of New York for having taken donations from
him over the years only to criticize him on television now.

And Mr. Trump has begun to acknowledge to associates and even in public
that he might lose. In an
\href{http://video.cnbc.com/gallery/?video=3000542306}{interview on
CNBC} on Thursday, he said he was prepared to face defeat.

``I'll just keep doing the same thing I'm doing right now,'' he said.
``And at the end, it's either going to work, or I'm going to, you know,
I'm going to have a very, very nice, long vacation.''

\includegraphics{https://static01.graylady3jvrrxbe.onion/images/2016/08/14/us/14trump-jp2/14trump-jp2-articleInline.jpg?quality=75\&auto=webp\&disable=upscale}

Jason Miller, a spokesman for Mr. Trump, said the Republican nominee was
still determined to win, and dismissed accounts that he was downcast.
Mr. Miller pointed to the crowds Mr. Trump attracts as a sign of
strength.

``Behind the scenes we have a very motivated and very focused candidate
in Donald Trump, who knows what he needs to do to win this race,'' Mr.
Miller said.

People around Mr. Trump and his operation say they are not ready to
abandon hope of a turnaround. But he is in a dire predicament,
Republicans say, because he is profoundly uncomfortable in the role of a
typical general election candidate, disoriented by the crosscurrents he
must now navigate and still relying impulsively on a pugilistic formula
that guided him to the nomination.

His advisers are still convinced of the basic potency of a sales pitch
about economic growth and a shake-up in Washington, and they aspire to
compete in as many as 21 states, despite Mr. Trump's perilous standing
in the four states --- Florida, Ohio, Pennsylvania and North Carolina
--- likely to decide the election.

Charles R. Black Jr., an influential Republican lobbyist supporting Mr.
Trump, said the campaign was in a continuing struggle to tame him.

``He has three or four good days and then makes another gaffe,'' Mr.
Black said. ``Hopefully, he can have some more good days.'' Of Mr.
Trump's advisers, Mr. Black said, ``They think he is making progress in
terms of being able to make set speeches and not take the bait on every
attack somebody makes on him.''

\href{https://www.nytimes3xbfgragh.onion/interactive/2016/08/09/us/elections/Bush-Rubio-and-Kasich-Donors-give-to-Clinton.html}{}

\includegraphics{https://static01.graylady3jvrrxbe.onion/images/2016/08/08/us/elections/Bush-Rubio-and-Kasich-Donors-give-to-Clinton-1470678226803/Bush-Rubio-and-Kasich-Donors-give-to-Clinton-1470678226803-largeHorizontalJumbo.png}

\hypertarget{donors-for-bush-kasich-and-christie-are-turning-to-clinton-more-than-to-trump}{%
\subsection{Donors for Bush, Kasich and Christie Are Turning to Clinton
More Than to
Trump}\label{donors-for-bush-kasich-and-christie-are-turning-to-clinton-more-than-to-trump}}

People who donated to establishment Republican candidates in the primary
season are more likely to give money to Hillary Clinton than to Donald
J. Trump.

Mr. Trump's advisers now hope to steady him by pairing him on the trail
with familiar, more seasoned figures --- people he views as peers and
enjoys spending time with, like former Mayor Rudolph W. Giuliani of New
York and former Gov. Mike Huckabee of Arkansas.

Mr. Giuliani, who campaigned with Mr. Trump early in the week, said he
did not see the candidate as unmoored or unhappy. If anyone was
disconcerted, Mr. Giuliani suggested, it was the people steering his
campaign.

``He doesn't seem to be as unnerved by these things that go wrong as the
people around him,'' Mr. Giuliani said. Still, he allowed, ``I think it
is true that maybe it took him a little while to realize that we're
moving from a primary campaign to a presidential campaign.''

Mr. Trump, he said, had become ``a little bit more realizing there are
certain days left and you've got to get messages out on those days.''

Even before Mr. Trump's most recent spate of incendiary comments,
Republicans who dealt with him after the primaries came away alarmed by
his obvious unease as the de facto party leader. After a meeting in late
May between Mr. Trump and Karl Rove, the architect of George W. Bush's
presidential victories, Mr. Rove told associates he was stunned by Mr.
Trump's poor grasp of campaign basics, including how to map out a
schedule and use data to reach voters.

\href{http://www.nytimes3xbfgragh.onion/2016/06/03/us/politics/karl-rove-donald-trump.html}{Sitting
with Mr. Rove} in the Manhattan apartment of a mutual friend, the casino
magnate Steve Wynn, Mr. Trump said he would compete in states like
Oregon, which has not voted Republican
\href{http://www.270towin.com/states/Oregon}{since Ronald Reagan's 1984
landslide}. Mr. Rove later told people he believed Mr. Trump was
confused and scared in anticipation of the general election, according
to people who have heard Mr. Rove's account.

Image

Karl Rove, a top adviser to President George W. Bush, at Central Park
last month, met with Mr. Trump in May, and told associates he was
stunned by Mr. Trump's poor grasp of campaign basics.Credit...Brad
Barket/Getty Images

A few weeks later, when Governor Christie brokered a meeting at Trump
Tower between Mr. Trump and governors from around the country, Mr. Trump
offered a desultory performance, bragging about his poll numbers,
listening passively as the governors talked about their states and then
sending them on their way.

Mr. Trump never asked them for their support, three people briefed on
the meeting said.

With donors, Mr. Trump has been an indifferent ambassador for his
campaign. He has resisted making fund-raising calls and, during at least
two major events in July, in New York and Chicago, burned valuable hours
with potential contributors by asking them to go around the room, one by
one, giving him their thoughts on whom he should pick as his running
mate.

That left little time for the donors to query Mr. Trump about policy or
strategy, or for him to reassure them about his campaign. Jay Bergman,
an Illinois oil executive who attended the event in Chicago, said he
wondered if Mr. Trump had taken that approach ``to avoid answering
questions.''

On matters of policy, too, Mr. Trump has engaged only fleetingly, and
idiosyncratically. Before delivering a
\href{http://www.nytimes3xbfgragh.onion/2016/08/09/us/politics/donald-trump-economy-speech.html}{policy
speech in Detroit} on Monday, he delegated the formation of an economic
plan to a few conservative economists outside his campaign, who
consulted him from time to time and ultimately haggled over the details
in his office as he followed their conversation.

Stephen Moore, a \href{http://www.heritage.org/}{Heritage Foundation}
fellow, said he and Arthur Laffer, the supply-side economist, had
tangled over the top tax bracket while Mr. Trump observed from behind
his desk, eventually siding with Mr. Moore. Mr. Trump, he said, also
expressed strong views about the taxation of interest on business loans,
citing his experience as a developer.

``He's a typical businessman, right?'' Mr. Moore said. ``He lets people
argue it out, and Arthur made his case and others made their case.''

``He's not the world's expert on the tax code,'' Mr. Moore added, ``but
he has very good intuition about how these things will affect real
people.''

At the last minute, Mr. Trump interjected to direct his advisers to
incorporate a tax deduction for the cost of child care in his economic
plan. The issue, which Mr. Trump had not discussed on the campaign
trail, is a favorite of his daughter Ivanka.

Mr. Trump's reliance on his family has only grown more pronounced. Mr.
Kushner, Mr. Trump's son-in-law, who has no background in politics, has
expanded his role: He now has broad oversight over areas including the
campaign's budget, messaging and strategy, with the power to approve
spending. Mr. Trump has also continued to seek advice from Corey
Lewandowski, the campaign manager whom Mr. Trump ousted in June at his
children's urging.

Efforts to bring in high-profile, experienced hands have been fruitless.
Mr. Kushner had suggested enlisting Steve Schmidt, Senator John McCain's
2008 presidential campaign manager, but despite having met once with Mr.
Trump during the primaries and speaking with him a few times, Mr.
Schmidt never signed on.

Mr. Trump's advisers believe he is nearly out of time to right his
campaign. On Tuesday, hours before his
\href{http://www.nytimes3xbfgragh.onion/2016/08/10/us/politics/donald-trump-hillary-clinton.html?action=click\&contentCollection=Politics\&module=RelatedCoverage\&region=Marginalia\&pgtype=article}{explosive
comment} about ``Second Amendment people'' taking action if Mrs. Clinton
is elected, his brain trust reassembled again at Trump Tower in a
reprise of their stern meeting in June.

They again urged Mr. Trump to adjust his tone and comportment. The top
pollster, Tony Fabrizio, gave an unvarnished assessment, warning that
Mr. Trump's numbers would only move in one direction, absent a major
change.

Mr. Trump, people briefed on the meeting said, digested the advice and
responded receptively.

It was time, he agreed, to get on track.

Advertisement

\protect\hyperlink{after-bottom}{Continue reading the main story}

\hypertarget{site-index}{%
\subsection{Site Index}\label{site-index}}

\hypertarget{site-information-navigation}{%
\subsection{Site Information
Navigation}\label{site-information-navigation}}

\begin{itemize}
\tightlist
\item
  \href{https://help.nytimes3xbfgragh.onion/hc/en-us/articles/115014792127-Copyright-notice}{©~2020~The
  New York Times Company}
\end{itemize}

\begin{itemize}
\tightlist
\item
  \href{https://www.nytco.com/}{NYTCo}
\item
  \href{https://help.nytimes3xbfgragh.onion/hc/en-us/articles/115015385887-Contact-Us}{Contact
  Us}
\item
  \href{https://www.nytco.com/careers/}{Work with us}
\item
  \href{https://nytmediakit.com/}{Advertise}
\item
  \href{http://www.tbrandstudio.com/}{T Brand Studio}
\item
  \href{https://www.nytimes3xbfgragh.onion/privacy/cookie-policy\#how-do-i-manage-trackers}{Your
  Ad Choices}
\item
  \href{https://www.nytimes3xbfgragh.onion/privacy}{Privacy}
\item
  \href{https://help.nytimes3xbfgragh.onion/hc/en-us/articles/115014893428-Terms-of-service}{Terms
  of Service}
\item
  \href{https://help.nytimes3xbfgragh.onion/hc/en-us/articles/115014893968-Terms-of-sale}{Terms
  of Sale}
\item
  \href{https://spiderbites.nytimes3xbfgragh.onion}{Site Map}
\item
  \href{https://help.nytimes3xbfgragh.onion/hc/en-us}{Help}
\item
  \href{https://www.nytimes3xbfgragh.onion/subscription?campaignId=37WXW}{Subscriptions}
\end{itemize}
