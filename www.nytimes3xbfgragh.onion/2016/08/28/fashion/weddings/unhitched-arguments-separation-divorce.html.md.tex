Sections

SEARCH

\protect\hyperlink{site-content}{Skip to
content}\protect\hyperlink{site-index}{Skip to site index}

\href{https://www.nytimes3xbfgragh.onion/section/fashion/weddings}{Love}

\href{https://myaccount.nytimes3xbfgragh.onion/auth/login?response_type=cookie\&client_id=vi}{}

\href{https://www.nytimes3xbfgragh.onion/section/todayspaper}{Today's
Paper}

\href{/section/fashion/weddings}{Love}\textbar{}Unhitched: The Same
Arguments, Then Finally a New Direction

\url{https://nyti.ms/2bIZTqD}

\begin{itemize}
\item
\item
\item
\item
\item
\end{itemize}

Advertisement

\protect\hyperlink{after-top}{Continue reading the main story}

Supported by

\protect\hyperlink{after-sponsor}{Continue reading the main story}

\href{/column/unhitched}{Unhitched}

\hypertarget{unhitched-the-same-arguments-then-finally-a-new-direction}{%
\section{Unhitched: The Same Arguments, Then Finally a New
Direction}\label{unhitched-the-same-arguments-then-finally-a-new-direction}}

\includegraphics{https://static01.graylady3jvrrxbe.onion/images/2016/08/28/fashion/28UNHITCHED/28UNHITCHED-articleInline.jpg?quality=75\&auto=webp\&disable=upscale}

By Louise Rafkin

\begin{itemize}
\item
  Aug. 25, 2016
\item
  \begin{itemize}
  \item
  \item
  \item
  \item
  \item
  \end{itemize}
\end{itemize}

In Unhitched, longtime couples tell the stories of their relationships,
from romance to vows to divorce to life afterward.

After 30 years, an empty nest leads to flight. With their children
grown, Richard Baum and Barbara Baum found themselves at a crossroad.
Exhausted by years of wrangling the same issues, change led to more
changes.

\textbf{Where did they grow up?}

In the same Chicago suburb. They were friendly in high school. Her
parents, who married for life, argued constantly. His parents stayed
together, and his family was very close, to the point of being wary of
outsiders.

\textbf{How did they become a couple?}

After college, they ran into each other in an elevator and discovered
they worked in the same building in Chicago. A friendship grew into
dating. They married two years later.

\textbf{What was it about the other?}

They shared friends, had common backgrounds and admired each other's
creativity. ``In retrospect, we may have been in love with the idea of
each other as opposed to who we really were,'' Barbara said.

\textbf{Why did they marry?}

All their friends were getting married.

``I remember thinking it wasn't quite right but could not articulate
why,'' she said. ``He really liked me, and I lacked self-esteem and
thought I might not find anybody else.''

``Being with Barbara felt like being a better version of myself,'' he
said. ``She would be a good mother, and she was good at a lot things I
wasn't good at.''

His parents were onboard ``as much as they could be for someone,'' he
said.

\textbf{Where did they live after marrying?}

At first in a dilapidated house near Lake Michigan, then in Chicago in a
high rise. After having two children, they moved to a suburb near where
they grew up and where her parents still lived, in order to have help
with child care.

\textbf{How were the early years?}

Busy. She had her own business; he practiced architecture and worked as
a futures options broker at the stock exchange. They had small children
and their aging parents needed help. They bought a fixer-upper, and
construction took a lot out of them.

``Having two careers was a blessing and a curse, but definitely impacted
our relationship,'' Richard said. ``At no point did I feel like I made a
mistake. I appreciated everything we had, but I was always tired.''

\textbf{Were they happy?}

Happy enough, she said. ``Mostly we didn't stop moving,'' she said. ``We
had to make life work for the kids and get the house finished.''

``I thought things could have been better,'' he said. ``I wondered if
Barbara really wanted to be married, and I felt like our intimacy was a
biological attraction rather than a deep connection.''

\textbf{First signs of trouble?}

When he was laid off from his architecture job and she began working a
full-time job, they drifted apart. The house construction went on and
on. She resented him always being tired, but working at the stock
exchange required him to start work early in the morning. They never
created a financial cushion.

``I typically overlook trouble, and maybe that was the true trouble,''
he said. ``Inside, I wondered why we couldn't fix things. But I was
perpetually resentful.''

\textbf{Did they try to work on things? Go to therapy?}

After seven years, they started couples therapy and went for many years
to various counselors, while also seeing individual therapists. She felt
like they had had the same arguments for 23 years.

``Nothing changed, and I'm not sure why not,'' she said. ``The
counselors didn't make it clear enough what the problems really were. I
expected him to change, and I'm not sure he wanted to. I think he
expected me to be more intimate and accepting and not so angry.''

``I wasn't opposed to changing,'' he said. ``I just didn't know how to.
I thought there were things about me that Barbara never appreciated or
gave me credit for. Unconditional love was missing.''

\textbf{First idea of splitting?}

She proposed divorce in the late 1990s and again in 2003. Both times she
backed off because of the children, and because she didn't want it to be
ugly, she said.

\textbf{The final breakup?}

In 2012 after the children were grown, they sold their house but had no
plan for what to do next. At that point they were living platonically.
Barbara rented a downtown apartment, but that didn't appeal to Richard.
He put his things in storage, and went to Atlanta, where his son was in
medical school. Though he thought they were ``on a break,'' after
attending a self-help seminar recommended by his daughter, he realized
the marriage was over.

``I'm not a planner, and she's a hyper-planner,'' he said. ``It's why we
got together and why we fell apart.''

``Kids are a wonderful distraction, but at some point that project is
done, and we couldn't agree on the next one,'' she said.

\textbf{How did they fare financially?}

Using a mediator, they split 50/50; he thought he made more concessions.
``Maintaining civility and family unity was more important than money,''
he said.

\textbf{Did they feel stigmatized?}

No. Some people were more judgmental than others, but those who cared
about them were supportive and understanding. Richard was in a new city,
so no one was aware of his past.

\textbf{How did their children react?}

Their son took it harder than their daughter, but she had moments of
sadness, too.

\textbf{Should they have divorced sooner?}

He: No. There's no point in looking in the rearview mirror.

She: Maybe there were too many second chances.

\textbf{What did they do to start over after the divorce?}

A year later, Barbara started dating. She didn't find anyone compatible,
but found many men wanted to ``sleep with her while they were sleeping
with about 10 other women,'' she said. She is happy with her
independence.

Richard made a commitment to doing things differently. He is again
practicing as an architect and is a part-owner in a restaurant. He has
began dating and is now in a new relationship. He said he and Barbara
discuss their dating lives.

\textbf{Is their new life better?}

For her, yes. It's less stressful, she said. ``I know more when to say
yes and when to say no, and its O.K. if someone isn't happy with my
answers,'' she said. ``Before, I tried to make others happy.''

For him, in most regards, yes, though not financially. ``My life is
wonderfully different,'' he said.

\textbf{Would they have done anything differently?}

She: ``Having grown up in a household where my parents fought, I didn't
want to fight with Richard. But I should have pushed harder for what I
thought was right for me and my family, financially and otherwise. I was
angry at myself for having married a romantic idea, but I never had
advice from my parents or anyone about marriage.''

He: ``I would have been much more receptive and accepting of Barbara's
priorities and planning. In short, I would have been a better partner.''

\textbf{Looking back, what advice would they offer?}

She: ``Don't cave on your beliefs to keep the peace, and don't overlook
trouble. Don't fall in love with an idea of how you want your spouse to
be. Fall in love with and accept the actual person as they really, truly
are. Don't imagine things will be different by getting mad or shutting
down or rearranging the furniture.''

He: ``Be aware of your unfulfilled expectations, your own thwarted
intentions and be responsible for your own emotions.''

\textbf{What is life like now?}

She still worries about him. ``He is family and the father of my
children,'' she said. ``I would do whatever I could for him without
sacrificing my own integrity.''

They talk regularly by phone.

\textbf{Has either person changed?}

She: ``I know myself better now and have more tolerance and patience.
Rich is a better communicator and has taken a hard look at himself.''

He: ``I'm more expressive, confident and aware of the impact I have on
others. My kids experienced me as unhappy and grumpy, and I didn't even
know it. If something isn't going well or the way I intended, I ask
myself what new actions I can take to get a different result.''

\textbf{Advice for others divorcing at their age?}

She: ``Don't play the blame game. Don't engage in drama. There is zero
upside to hurting each other. Be honest, and make it civil.''

He: ``Divorce is devastating to kids. For a long time that was a good
reason not to get divorced. But plenty of people survive it.

``I think we both resented each other for our differences,'' he added.
``I had a lot of good intentions but never put them into action. But
that's how I was conditioned to be.''

Advertisement

\protect\hyperlink{after-bottom}{Continue reading the main story}

\hypertarget{site-index}{%
\subsection{Site Index}\label{site-index}}

\hypertarget{site-information-navigation}{%
\subsection{Site Information
Navigation}\label{site-information-navigation}}

\begin{itemize}
\tightlist
\item
  \href{https://help.nytimes3xbfgragh.onion/hc/en-us/articles/115014792127-Copyright-notice}{©~2020~The
  New York Times Company}
\end{itemize}

\begin{itemize}
\tightlist
\item
  \href{https://www.nytco.com/}{NYTCo}
\item
  \href{https://help.nytimes3xbfgragh.onion/hc/en-us/articles/115015385887-Contact-Us}{Contact
  Us}
\item
  \href{https://www.nytco.com/careers/}{Work with us}
\item
  \href{https://nytmediakit.com/}{Advertise}
\item
  \href{http://www.tbrandstudio.com/}{T Brand Studio}
\item
  \href{https://www.nytimes3xbfgragh.onion/privacy/cookie-policy\#how-do-i-manage-trackers}{Your
  Ad Choices}
\item
  \href{https://www.nytimes3xbfgragh.onion/privacy}{Privacy}
\item
  \href{https://help.nytimes3xbfgragh.onion/hc/en-us/articles/115014893428-Terms-of-service}{Terms
  of Service}
\item
  \href{https://help.nytimes3xbfgragh.onion/hc/en-us/articles/115014893968-Terms-of-sale}{Terms
  of Sale}
\item
  \href{https://spiderbites.nytimes3xbfgragh.onion}{Site Map}
\item
  \href{https://help.nytimes3xbfgragh.onion/hc/en-us}{Help}
\item
  \href{https://www.nytimes3xbfgragh.onion/subscription?campaignId=37WXW}{Subscriptions}
\end{itemize}
