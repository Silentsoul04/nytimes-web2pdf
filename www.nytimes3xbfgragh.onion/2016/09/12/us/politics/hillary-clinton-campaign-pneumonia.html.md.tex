Sections

SEARCH

\protect\hyperlink{site-content}{Skip to
content}\protect\hyperlink{site-index}{Skip to site index}

\href{https://www.nytimes3xbfgragh.onion/section/politics}{Politics}

\href{https://myaccount.nytimes3xbfgragh.onion/auth/login?response_type=cookie\&client_id=vi}{}

\href{https://www.nytimes3xbfgragh.onion/section/todayspaper}{Today's
Paper}

\href{/section/politics}{Politics}\textbar{}Hillary Clinton's Doctor
Says Pneumonia Led to Abrupt Exit From 9/11 Event

\url{https://nyti.ms/2cFiCkr}

\begin{itemize}
\item
\item
\item
\item
\item
\item
\end{itemize}

Advertisement

\protect\hyperlink{after-top}{Continue reading the main story}

Supported by

\protect\hyperlink{after-sponsor}{Continue reading the main story}

\hypertarget{hillary-clintons-doctor-says-pneumonia-led-to-abrupt-exit-from-911-event}{%
\section{Hillary Clinton's Doctor Says Pneumonia Led to Abrupt Exit From
9/11
Event}\label{hillary-clintons-doctor-says-pneumonia-led-to-abrupt-exit-from-911-event}}

\includegraphics{https://static01.graylady3jvrrxbe.onion/images/2016/09/12/us/12campaign/12campaign-videoSixteenByNineJumbo1600-v2.jpg}

By \href{http://www.nytimes3xbfgragh.onion/by/jonathan-martin}{Jonathan
Martin} and \href{http://www.nytimes3xbfgragh.onion/by/amy-chozick}{Amy
Chozick}

\begin{itemize}
\item
  Sept. 11, 2016
\item
  \begin{itemize}
  \item
  \item
  \item
  \item
  \item
  \item
  \end{itemize}
\end{itemize}

\href{http://www.nytimes3xbfgragh.onion/interactive/2016/us/elections/hillary-clinton-on-the-issues.html?inline=nyt-per}{Hillary
Clinton} is being treated for pneumonia and dehydration, her doctor said
on Sunday, hours after she abruptly left a ceremony in New York honoring
the 15th anniversary of the Sept. 11 attacks and had to be helped into a
van by Secret Service agents.

The incident, which occurred after months of questions about her health
from her Republican opponent, Donald J. Trump, and his campaign, is
likely to increase pressure on Mrs. Clinton to address the issue and
release detailed medical records, which she has so far declined to do.

Mrs. Clinton was taken from the morning event at ground zero to the
Manhattan apartment of her daughter, Chelsea. About 90 minutes after
arriving there, Mrs. Clinton emerged from the apartment in New York's
Flatiron district. She waved to onlookers and posed for pictures with a
little girl on the sidewalk.

``I'm feeling great,'' Mrs. Clinton said. ``It's a beautiful day in New
York.''

Mrs. Clinton left in her motorcade without the group of reporters that
is designated to travel with her in public. A campaign spokesman, Nick
Merrill, indicated that she had returned to her Chappaqua, N.Y.,
residence sometime after 1 p.m., and Mrs. Clinton was not seen publicly
the rest of the day.

Mr. Merrill initially described Mrs. Clinton, the Democratic
presidential nominee, as feeling ``overheated'' at the commemoration
ceremony.

But just after 5 p.m., a campaign official said Mrs. Clinton's
physician, Dr. Lisa R. Bardack, had examined the candidate at her home
in Chappaqua, and Dr. Bardack said in a statement that Mrs. Clinton was
``rehydrated and recovering nicely.''

``Secretary Clinton has been experiencing a cough related to
allergies,'' Dr. Bardack's statement said, adding that on Friday
morning, after a prolonged cough, Mrs. Clinton was given a diagnosis of
pneumonia.

``She was put on antibiotics, and advised to rest and modify her
schedule,'' Dr. Bardack added. ``At this morning's event, she became
overheated and dehydrated.''

Dr. Bardack did not indicate what sort of pneumonia Mrs. Clinton had or
elaborate on the nature of the examination last week, whether Mrs.
Clinton had a fever today, or a host of other issues that could offer
more precise insights about her condition.

Late Sunday night, Mrs. Clinton's campaign said she was canceling her
plans to travel to California on Monday for what had been a planned
two-day trip there.

\includegraphics{https://static01.graylady3jvrrxbe.onion/images/2016/09/12/us/12campaign2/12campaign2-articleLarge.jpg?quality=75\&auto=webp\&disable=upscale}

\href{https://twitter.com/zgazda66/status/774993814025011200}{A video of
Mrs. Clinton} taken by an attendee at the ceremony captured what
appeared to be her legs buckling as she struggled to steady herself and
walk to her van. She required assistance from two Secret Service agents,
who held her on either side, to move off a curb and into the van.

Close-up images revealed that her feet were dragging as she was hoisted
into the vehicle.

The episode thrust questions about Mrs. Clinton's health and the
transparency of her campaign squarely into the last two months of the
race, which many polls show has grown tighter. For months Republicans
have, with scarce evidence, questioned the stamina of Mrs. Clinton, 68,
and claimed she is ill, often pointing to her repeated coughing bouts.

She has brushed off such claims. Mrs. Clinton and Mr. Trump, 70, have
shared substantially less information about their health than some
previous presidential candidates.

And Mrs. Clinton revealed that she had pneumonia and had been prescribed
medication only after the startling video emerged of her being unable to
walk under her own volition after the ceremony.

Her campaign initially did not offer any information about why she had
left early or her whereabouts. Twice during the day, she abandoned the
group of reporters assigned to cover her public movements. Campaign
officials did not respond to multiple inquiries about whether Mrs.
Clinton had been treated by a doctor or had taken any medications.

Even some members of Mrs. Clinton's campaign staff had been unaware of
her recent diagnosis. Huma Abedin, Mrs. Clinton's long-time aide, sent
an email to the full campaign staff on Sunday that included the doctor's
note, to share with them ``the full picture.''

``Onward as H.R.C. would say,'' Ms. Abedin wrote in the message, the
contents of which were disclosed by a Clinton aide who requested
anonymity to share an internal campaign email.

Temperatures were in the high 70s on Sunday morning in New York, and
humidity was high. Mr. Trump also attended the ceremony, as did many
other dignitaries.

Other attendees at the event said afterward that Mrs. Clinton had not
appeared ill when she first arrived at the former site of the World
Trade Center.

``She seemed fine,'' said Representative Peter T. King of New York, a
Republican, who recalled speaking briefly with Mrs. Clinton around 8:30
a.m.

But about an hour later there was a minor commotion, Mr. King said. A
number of New York's current and former elected officials had been
standing in silence as the names of the victims of the attacks were
read. Suddenly, Mrs. Clinton, a former New York senator, left her
position.

Mrs. Clinton emerged last week from an August mostly focused on private
fund-raisers with campaign events in Ohio, Illinois, Florida, North
Carolina and Missouri. On Wednesday, she participated in an
\href{https://www.nytimes3xbfgragh.onion/2016/09/08/us/politics/hillary-clinton-donald-trump-national-security.html}{NBC
forum on national security}, and on Friday she attended a Manhattan
fund-raiser where she characterized half of Mr. Trump's supporters as a
``\href{http://www.nytimes3xbfgragh.onion/2016/09/11/us/politics/hillary-clinton-basket-of-deplorables.html}{basket
of deplorables}.''

The candidates had taken their advertisements off the air to honor the
anniversary of the Sept. 11 attacks, and Mr. Trump said nothing when he
was asked on Sunday about Mrs. Clinton's condition.

But he and his supporters have aggressively
\href{http://www.nytimes3xbfgragh.onion/2016/09/08/upshot/release-your-medical-records-first-you-must-collect-them.html}{sought
to raise questions} about Mrs. Clinton's health in recent months. Mr.
Trump has highlighted her recurring cough and
\href{https://twitter.com/realdonaldtrump/status/770039317142069248}{wrote
on Twitter} last month that ``both candidates'' should ``release
detailed medical records.'' (Mr. Trump has issued only a limited summary
of his health.)

Questions about the health of presidential hopefuls are hardly new to
this campaign, but long before the exacting scrutiny of the modern media
environment, campaigns were often able to suppress information about the
ailments of candidates. Rumors about Franklin D. Roosevelt's health, for
example, pervaded his final presidential campaign in 1944, but he
campaigned vigorously and his aides kept the extent of the heart disease
that would kill him the following year out of the news.

More recently, Ronald Reagan, Bob Dole and Senator John McCain, each of
whom was the Republican presidential nominee while in his 70s, faced
questions about their physical condition.

``The physical demands of running for president, even with private
planes and Secret Service protection, are more difficult than the mental
demands,'' said Scott Reed, who managed Mr. Dole's 1996 campaign.

Mr. Trump has also been criticized for sharing few details about his
health, providing only a
\href{https://www.nytimes3xbfgragh.onion/politics/first-draft/2015/12/14/donald-trump-releases-medical-report-calling-his-health-extraordinary/}{brief
statement from his personal physician} in December.

Short on particulars like heart rate, cholesterol level or family
history, it described Mr. Trump in laudatory terms, saying that he would
be ``the healthiest individual ever elected to the presidency.''

The physician, Dr. Harold Bornstein, later
\href{http://www.nbcnews.com/news/us-news/trump-doctor-wrote-health-letter-just-5-minutes-limo-waited-n638526}{told
NBC News} that he had written the statement in five minutes while a
black car waited outside his office to collect it.

Mrs. Clinton's campaign has tried to bat away
\href{http://www.nytimes3xbfgragh.onion/2016/08/23/us/politics/clinton-trump-health.html}{rumors
about her health}, including releasing a letter from Mrs. Clinton's
doctor saying she was in ``excellent health.'' But aides have dismissed
such questions as a way to distract from the issue of Mr. Trump's not
releasing his tax returns.

In July 2015, Mrs. Clinton issued a detailed
\href{http://www.nytimes3xbfgragh.onion/politics/first-draft/2015/07/31/doctor-says-hillary-clinton-is-fit-to-serve/}{two-page
letter from her physician} that included a concussion she sustained in
2012, while she was secretary of state; it left her with a blood clot in
her head and double vision. Dr. Bardack, Mrs. Clinton's physician, said
those symptoms had been resolved within two months.

The candidate's husband, Bill Clinton, however, has said that she
``required six months of very serious work to get over'' the concussion
--- a statement that helped feed conspiracy theories among Republicans
that the injury was worse than initially disclosed, though there is no
medical evidence to support those theories.

Asked whether she was concerned that such questions about her health
would affect the election, as the polls have tightened, Mrs. Clinton
told reporters on her campaign plane last week: ``I'm not concerned
about the conspiracy theories. There are so many of them, I've lost
track of them.''

Advertisement

\protect\hyperlink{after-bottom}{Continue reading the main story}

\hypertarget{site-index}{%
\subsection{Site Index}\label{site-index}}

\hypertarget{site-information-navigation}{%
\subsection{Site Information
Navigation}\label{site-information-navigation}}

\begin{itemize}
\tightlist
\item
  \href{https://help.nytimes3xbfgragh.onion/hc/en-us/articles/115014792127-Copyright-notice}{©~2020~The
  New York Times Company}
\end{itemize}

\begin{itemize}
\tightlist
\item
  \href{https://www.nytco.com/}{NYTCo}
\item
  \href{https://help.nytimes3xbfgragh.onion/hc/en-us/articles/115015385887-Contact-Us}{Contact
  Us}
\item
  \href{https://www.nytco.com/careers/}{Work with us}
\item
  \href{https://nytmediakit.com/}{Advertise}
\item
  \href{http://www.tbrandstudio.com/}{T Brand Studio}
\item
  \href{https://www.nytimes3xbfgragh.onion/privacy/cookie-policy\#how-do-i-manage-trackers}{Your
  Ad Choices}
\item
  \href{https://www.nytimes3xbfgragh.onion/privacy}{Privacy}
\item
  \href{https://help.nytimes3xbfgragh.onion/hc/en-us/articles/115014893428-Terms-of-service}{Terms
  of Service}
\item
  \href{https://help.nytimes3xbfgragh.onion/hc/en-us/articles/115014893968-Terms-of-sale}{Terms
  of Sale}
\item
  \href{https://spiderbites.nytimes3xbfgragh.onion}{Site Map}
\item
  \href{https://help.nytimes3xbfgragh.onion/hc/en-us}{Help}
\item
  \href{https://www.nytimes3xbfgragh.onion/subscription?campaignId=37WXW}{Subscriptions}
\end{itemize}
