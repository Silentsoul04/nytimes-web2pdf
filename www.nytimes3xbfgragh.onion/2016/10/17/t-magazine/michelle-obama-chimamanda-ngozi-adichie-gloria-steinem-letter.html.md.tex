To the First Lady, With Love

\begin{itemize}
\item
\item
\item
\item
\item
\item
\end{itemize}

\includegraphics{https://static01.graylady3jvrrxbe.onion/images/2016/10/17/t-magazine/michelle-obama-slide-3G05/michelle-obama-slide-3G05-articleLarge.jpg?quality=75\&auto=webp\&disable=upscale}

Sections

\protect\hyperlink{site-content}{Skip to
content}\protect\hyperlink{site-index}{Skip to site index}

\hypertarget{to-the-first-lady-with-love}{%
\section{To the First Lady, With
Love}\label{to-the-first-lady-with-love}}

Four thank-you notes to Michelle Obama, who has spent the past eight
years quietly and confidently changing the course of American history.

.Credit...Collier Schorr

Supported by

\protect\hyperlink{after-sponsor}{Continue reading the main story}

By Chimamanda Ngozi Adichie, Gloria Steinem, Jon Meacham and Rashida
Jones

\begin{itemize}
\item
  Oct. 17, 2016
\item
  \begin{itemize}
  \item
  \item
  \item
  \item
  \item
  \item
  \end{itemize}
\end{itemize}

\hypertarget{by-chimamanda-ngozi-adichie}{%
\subsection{By Chimamanda Ngozi
Adichie:}\label{by-chimamanda-ngozi-adichie}}

She had rhythm, a flow and swerve, hands slicing air, body weight moving
from foot to foot, a beautiful rhythm. In anything else but a black
American body, it would have been contrived. The three-quarter sleeves
of her teal dress announced its appropriateness, as did her matching
brooch. But the cut of the dress scorned any ``future first lady''
stuffiness; it hung easy on her, as effortless as her animation. And a
brooch, Old World style accessory, yes, but hers was big and ebulliently
shaped and perched center on her chest. Michelle Obama was speaking. It
was the 2008 Democratic National Convention. My anxiety rose and
swirled, watching and willing her to be as close to perfection as
possible, not for me, because I was already a believer, but for the
swaths of America that would rather she stumbled.

She first appeared in the public consciousness, all common sense and
mordant humor, at ease in her skin. She had the air of a woman who could
balance a checkbook, and who knew a good deal when she saw it, and who
would tell off whomever needed telling off. She was tall and sure and
stylish. She was reluctant to be first lady, and did not hide her
reluctance beneath platitudes. She seemed not so much unique as true.
She sharpened her husband's then-hazy form, made him solid, more than
just a dream.

Image

Chimamanda Ngozi Adichie

But she had to flatten herself to better fit the mold of first lady. At
the law firm where they met before love felled them, she had been her
husband's mentor; they seemed to be truly friends, partners, equals in a
modern marriage in a new American century. Yet voters and observers,
wide strips of America, wanted her to conform and defer, to cleanse her
tongue of wit and barb. When she spoke of his bad morning-breath, a
quirky and humanizing detail, she was accused of emasculating him.

Because she said what she thought, and because she smiled only when she
felt like smiling, and not constantly and vacuously, America's cheapest
caricature was cast on her: the Angry Black Woman. Women, in general,
are not permitted anger --- but from black American women, there is an
added expectation of interminable gratitude, the closer to groveling the
better, as though their citizenship is a phenomenon that they cannot
take for granted.

``I love this country,'' she said to applause. She needed to say it ---
her salve to the hostility of people who claimed she was unpatriotic
because she had dared to suggest that, as an adult, she had not always
been proud of her country.

\includegraphics{https://static01.graylady3jvrrxbe.onion/images/2016/10/17/t-magazine/michelle-obama-slide-PRGV/michelle-obama-slide-PRGV-articleInline.jpg?quality=75\&auto=webp\&disable=upscale}

Of course she loved her country. The story of her life as she told it
was wholesomely American, drenched in nostalgia: a father who worked
shifts and a mother who stayed home, an almost mythic account of
self-reliance, of moderation, of working-class contentment. But she is
also a descendant of slaves, those full human beings considered human
fractions by the American state. And ambivalence should be her
birthright. For me, a foreign-raised person who likes America, one of
its greatest curiosities is this: that those who have the most reason
for dissent are those least allowed dissent.

Michelle Obama was speaking. I felt protective of her because she was
speaking to an America often too quick to read a black woman's
confidence as arrogance, her straightforwardness as entitlement.

She was informal, colloquial, her sentences bookended by the word
``see,'' a conversational fillip that also strangely felt like a mark of
authenticity. She seemed genuine. She was genuine. All over America,
black women were still, their eyes watching a form of God, because she
represented their image writ large in the world.

Her speech was vibrant, a success. But there was, in her eyes and
beneath her delivery and in her few small stumbles, a glimpse of
something somber. A tight, dark ball of apprehension. As though she
feared eight years of holding her breath, of living her life with a
stone in her gut.

Eight years later, her blue dress was simpler but not as eager to be
appropriate; its sheen, and her edgy hoop earrings, made clear that she
was no longer auditioning.

Her daughters were grown. She had shielded them and celebrated them, and
they appeared in public always picture perfect, as though their careful
grooming was a kind of reproach. She had called herself mom-in-chief,
and cloaked in that nonthreatening title, had done what she cared about.

She embraced veterans and military families, and became their listening
advocate. She threw open the White House doors to people on the margins
of America. She was working class, and she was Princeton, and so she
could speak of opportunity as a tangible thing. Her program Reach Higher
pushed high schoolers to go further, to want more. She jumped rope with
children on the White House grounds as part of her initiative to combat
childhood obesity. She grew a vegetable garden and campaigned for
healthier food in schools. She reached across borders and cast her light
on the education of girls all over the world. She danced on television
shows. She hugged more people than any first lady ever has, and she made
``first lady'' mean a person warmly accessible, a person both normal and
inspirational and a person many degrees of cool.

She had become an American style icon. Her dresses and workouts. Her
carriage and curves. Toned arms and long slender fingers. Even her
favored kitten heels, for women who cannot fathom wearing shoes in the
halfway house between flats and high heels, have earned a certain
respect because of her. No public figure better embodies that mantra of
full female selfhood: Wear what you like.

It was the 2016 Democratic Convention. Michelle Obama was speaking. She
said ``black boy'' and ``slaves,'' words she would not have said eight
years ago because eight years ago any concrete gesturing to blackness
would have had real consequences.

Image

\textbf{Céline} earrings, (212) 535-3703. \textbf{Chloé} top,
\href{http://www.barneys.com/}{barneys.com}. \textbf{Calvin Klein
Collection} top (worn underneath),
\href{http://www.calvinklein.us/}{calvinklein.com}.Credit...Collier
Schorr

She was relaxed, emotional, sentimental. Her uncertainties laid to rest.
Her rhythm was subtler, because she no longer needed it as her armor,
because she had conquered.

The insults, those barefaced and those adorned as jokes, the acidic
scrutiny, the manufactured scandals, the base questioning of legitimacy,
the tone of disrespect, so ubiquitous, so casual. She had faced them and
sometimes she hurt and sometimes she blinked but throughout she remained
herself.

Michelle Obama was speaking. I realized then that she hadn't been
waiting to exhale these past eight years. She had been letting that
breath out, in small movements, careful because she had to be, but
exhaling still.

\emph{Chimamanda Ngozi Adichie is the author of the novel ``Americanah''
and the book-length essay ``We Should All Be Feminists.'' A recipient of
a MacArthur Foundation ``genius'' grant, her words have been sampled by
Beyoncé and, most recently, on clothing from Dior's spring 2017
collection.}

\hypertarget{by-gloria-steinem}{%
\subsection{By Gloria Steinem:}\label{by-gloria-steinem}}

Michelle Obama came into my life in stages. I knew that, like her
husband, she was a Harvard-educated lawyer, but that unlike him, she had
grown up on the South Side of Chicago, with parents who had not gone to
college. When Barack Obama was a summer associate at her Chicago law
firm, they met because she was his mentor. After his successful campaign
for the U.S. Senate, I noticed that she chose not to go to Washington.
Instead, he commuted to their home and two daughters in Chicago where
Michelle had a big job as head of community affairs for a hospital.

Image

Gloria SteinemCredit...Carly Romeo

But she really entered my imagination once she became first lady, a
tall, strong, elegant and seriously smart woman who happened to live in
the White House. She managed to convey dignity and humor at the same
time, to be a mother of two daughters and insist on regular family
dinners, and to take on health issues and a national food industry
addicted to unhealthy profits. She did this despite an undertow of bias
in this country that subtly questioned everything she did. Was she too
strong, physically and intellectually, to be a proper first lady?

After a decade under a public microscope, she has managed what no other
first lady --- and few people in any public position --- have succeeded
in doing: She has lived a public life without sacrificing her privacy
and authenticity. She made her husband both more human and effective as
a president by being his interpreter and defender, but also someone we
knew was capable of being his critic. Eventually, she spoke up about the
pain of the racist assumptions directed at her, but she waited until her
husband could no longer be politically punished for her honesty. And she
has always been the best kind of mother, which means insisting that
fathers be equal parents. All of this she has done with honesty, humor
and, most important, kindness.

Recently, over the course of the Trump-Clinton presidential campaign,
Michelle has become one of the most effective public speakers of our
time. That's serious. To be less serious, she has always been a woman
who knows the difference between fashion (what outside forces tell you
to wear) and style (the way you express a unique self). At one lunch in
the White House for women who had been spokespeople and supporters in
President Obama's second campaign, she invited local public school
children to sing and perform. Those students, mostly African-American
kids, were spirited, talented and at ease in a White House that belongs
to them as much as to anyone in this country, yet they wouldn't have
been there without Michelle.

What will she choose to do next? That's up to her. She could do
anything, from becoming a U.S. senator from Illinois to campaigning for
the safety and education of girls globally. She could also choose to
lead a private life. Whatever she decides, I trust her judgment.

Though I'm old enough to remember Eleanor and Franklin D. Roosevelt in
the White House --- and all the couples and families since --- I have
never seen such balance and equal parenting, such love, respect,
mutuality and pleasure in each other's company. We will never have a
democracy until we have democratic families and a society without the
invented categories of both race and gender. Michelle Obama may have
changed history in the most powerful way --- by example.

\emph{Gloria Steinem, a feminist activist and writer, has been touring
America, campaigning for Hillary Clinton and promoting the paperback
edition of her travelogue ``My Life on the Road.''}

\hypertarget{by-jon-meacham}{%
\subsection{By Jon Meacham:}\label{by-jon-meacham}}

On a lovely early autumn day in her final October in the White House,
Michelle Obama stepped out onto a sunny South Lawn and, in a way, bid
farewell. The setting was her celebrated organic kitchen garden, but the
subtext seemed to go far beyond any single initiative. ``I have to tell
you that being here with all of you, overlooking this beautiful garden
--- and it is beautiful --- it's kind of an emotional moment,'' Mrs.
Obama said at a ceremony to unveil a bigger, fortified version of the
garden. ``We're having a lot of these emotional moments because
everything is the last. But this is particularly my baby, because this
garden is where it all started. So we're really coming full circle back
to the very beginning.'' She recalled conversations in 2008 about the
role she might play in an Obama presidency --- and noted, tellingly,
that the garden emerged after ``Barack actually won,'' to which she
added: ``He won twice.'' The gathered guests happily applauded.

Image

Jon MeachamCredit...Gasper Tringale

There, in a way, was the essential Michelle Obama, or at least the
essential observable version of herself: speaking of broad public good
(the garden, which was part of her campaign against childhood obesity)
while revealing an arch sense of competitiveness. My husband won; he won
twice. As their history-making time in the White House comes to an end,
it's worth pondering the lessons of the Age of Obama. My own view is
that both the president and the first lady have conducted themselves
splendidly in the White House, managing the most difficult of tasks with
apparent ease: projecting a grace that masked the ambition and the drive
that took them, at early ages, to the pinnacle of American life.

In this they have kept faith with a tradition that, in our country, is
as old as George Washington, who embodied the classical ideal of
Cincinnatus, the reluctant leader summoned from his plow to lead the
nation. President Obama gets much of the public credit for handling his
eight years coolly, but the first lady has been a critical element of
his success. She has chosen her shots carefully --- not least in
choosing to make the case against Donald Trump on the campaign trail in
2016 --- and is leaving the country with a warm impression of an
excellent mother, a steady spouse and a sensible, devoted American.

Not everyone agrees, of course; not everyone ever does. The Obama
skeptics and the Obama haters have from time to time questioned her
patriotism, but this is the same country that managed, in some quarters,
to hold Eleanor Roosevelt in contempt. The important thing is that Mrs.
Obama, a clear-eyed lawyer, found a way to withstand the scrutiny of the
spotlight. In point of fact, she did more than withstand it. To borrow a
phrase from William Faulkner, she not only endured it; she prevailed
over it.

How? By finding, or appearing to find, that most elusive of things in
the modern world: balance. She was not Mrs. Roosevelt or Mrs. Carter or
Mrs. Reagan or Mrs. Clinton, playing roles in affairs of state. Instead
she did what the first African-American first lady arguably had to do to
play a successful public role. In Voltaire's terms, she cultivated her
own garden, never threatening and never intimidating her neighbors. Much
more doubtless unfolded beneath the surface or behind closed doors;
history will sort that out. For now, it is enough to say that she is
leaving the White House a strong and popular figure with a lifetime of
good will and great reservoirs of capital on which to draw as she and
her husband write their next chapters.

Back in 2008, musing on the life she was about to enter, Mrs. Obama
recalled doubts about her garden --- a bit of projection, one suspects,
for doubts about the entire presidential enterprise. ``What if we
planted this garden and nothing grew?'' Mrs. Obama asked. ``We didn't
know about the soil, or the sunlight. And it's like, oh, my God, what if
nothing grows? \ldots{} It was like afterwards I remember telling Sam
{[}Kass, the former White House senior advisor for nutrition{]}, `This
better work, buddy. This better work.''' And so it did.

\emph{Jon Meacham, a Pulitzer Prize-winning biographer, is the author,
most recently, of ``Destiny and Power: The American Odyssey of George
Herbert Walker Bush.''}

\hypertarget{by-rashida-jones}{%
\subsection{By Rashida Jones:}\label{by-rashida-jones}}

The first time I met Michelle Obama was at the White House as part of a
mentoring initiative, for which the first lady had brought together a
dynamic group of women to speak to urban teenage girls about their
career goals. Olympians, actresses, producers, writers, an astronaut and
an Air Force general gathered in the West Wing to greet Michelle before
we headed out to various local schools. She was warm, gracious and
charming. She thanked us for coming, hugged everybody and made us all
feel like her friends. As first lady, she has ticked all the boxes:
loving wife, protective mother, health and fitness advocate, garden
enthusiast and, yes, style icon. These accomplishments have left
traditionalists feeling satisfied.

Image

Rashida JonesCredit...Noam Galai/Getty Images

But, as is always the way, her reputation as the perfect hostess invited
criticism from progressives. Enter Michelle Obama, outspoken activist, a
woman who isn't afraid to remind us she is a proud African-American
woman, which is, in itself, revolutionary. A former lawyer who speaks
out on behalf of gay rights and gun control, she delivered an
unforgettable speech at the Democratic National Convention earlier this
year, shining a clear, bright light on our country's shameful history.
Suddenly, the progressives were pleased and the traditionalists were
confused. The media wants to pin her down --- they've been trying since
Barack Obama took office in 2009. But you simply can't.

Michelle Obama embodies the modern, American woman, and I don't mean
that in any platitudinous or vague way. Rarely can someone express their
many identities at the same time while seeming authentic. My female
friends and I often talk about feeling like we're ``too much.'' We're
complicated; we want to be so many things. I want to be a boss and also
be vulnerable. I want to be outspoken and respected, but also sexy and
beautiful.

All women struggle to reconcile the different people that we are at all
times, to merge our conflicting desires, to represent ourselves honestly
and feel good about the inherent contradictions. But Michelle manages to
do this with poise, regardless of the scrutiny. That, to me, is the best
thing for feminism. Her individual choices force us to accept that being
a woman isn't just one thing. Or two things. Or three things. The
position of first lady is, unfortunately, symbolic, and that makes it
fair game for media analysis ad nauseam. But no think piece can fully
encompass a real woman.

If feminism's goal is equal opportunity and choice, Michelle makes me
feel like every choice is available. You can go to Princeton and
Harvard, you can rap with Missy Elliott, you can be a mother and a
lawyer and a powerful orator. You can champion the Lilly Ledbetter Fair
Pay Act, while also caring about fashion. You can dance with Ellen and
also fearlessly remind people, on live television, of the reality of
your position: ``I wake up every morning in a house that was built by
slaves. And I watch my daughters, two beautiful, intelligent, black
young women, playing with their dogs on the White House lawn.'' You can
be your husband's partner and supporter, and also use your cultural and
political capital to campaign for Hillary Clinton, unflinchingly
standing up to her ``locker room talk''-ing bully of an opponent with
the battle cry ``enough is enough!'' --- eloquently putting into words
what a lot of people, myself included, had been feeling.

Michelle Obama will have her own legacy, separate from her husband's.
And it will be that she was the first first lady to show women that they
don't have to choose. That it's okay to be everything.

\emph{Rashida Jones is a writer, actress and producer who stars in the
TBS comedy series ``Angie Tribeca,'' and most recently co-wrote an
episode of ``Black Mirror,'' premiering later this month on Netflix.}

Advertisement

\protect\hyperlink{after-bottom}{Continue reading the main story}

\hypertarget{site-index}{%
\subsection{Site Index}\label{site-index}}

\hypertarget{site-information-navigation}{%
\subsection{Site Information
Navigation}\label{site-information-navigation}}

\begin{itemize}
\tightlist
\item
  \href{https://help.nytimes3xbfgragh.onion/hc/en-us/articles/115014792127-Copyright-notice}{©~2020~The
  New York Times Company}
\end{itemize}

\begin{itemize}
\tightlist
\item
  \href{https://www.nytco.com/}{NYTCo}
\item
  \href{https://help.nytimes3xbfgragh.onion/hc/en-us/articles/115015385887-Contact-Us}{Contact
  Us}
\item
  \href{https://www.nytco.com/careers/}{Work with us}
\item
  \href{https://nytmediakit.com/}{Advertise}
\item
  \href{http://www.tbrandstudio.com/}{T Brand Studio}
\item
  \href{https://www.nytimes3xbfgragh.onion/privacy/cookie-policy\#how-do-i-manage-trackers}{Your
  Ad Choices}
\item
  \href{https://www.nytimes3xbfgragh.onion/privacy}{Privacy}
\item
  \href{https://help.nytimes3xbfgragh.onion/hc/en-us/articles/115014893428-Terms-of-service}{Terms
  of Service}
\item
  \href{https://help.nytimes3xbfgragh.onion/hc/en-us/articles/115014893968-Terms-of-sale}{Terms
  of Sale}
\item
  \href{https://spiderbites.nytimes3xbfgragh.onion}{Site Map}
\item
  \href{https://help.nytimes3xbfgragh.onion/hc/en-us}{Help}
\item
  \href{https://www.nytimes3xbfgragh.onion/subscription?campaignId=37WXW}{Subscriptions}
\end{itemize}
