Sections

SEARCH

\protect\hyperlink{site-content}{Skip to
content}\protect\hyperlink{site-index}{Skip to site index}

\href{https://www.nytimes3xbfgragh.onion/section/politics}{Politics}

\href{https://myaccount.nytimes3xbfgragh.onion/auth/login?response_type=cookie\&client_id=vi}{}

\href{https://www.nytimes3xbfgragh.onion/section/todayspaper}{Today's
Paper}

\href{/section/politics}{Politics}\textbar{}Californians Legalize
Marijuana in Vote That Could Echo Nationally

\url{https://nyti.ms/2eCbZiE}

\begin{itemize}
\item
\item
\item
\item
\item
\item
\end{itemize}

Advertisement

\protect\hyperlink{after-top}{Continue reading the main story}

Supported by

\protect\hyperlink{after-sponsor}{Continue reading the main story}

\hypertarget{californians-legalize-marijuana-in-vote-that-could-echo-nationally}{%
\section{Californians Legalize Marijuana in Vote That Could Echo
Nationally}\label{californians-legalize-marijuana-in-vote-that-could-echo-nationally}}

\includegraphics{https://static01.graylady3jvrrxbe.onion/images/2016/11/09/us/09marijuana/09marijuana-articleLarge.jpg?quality=75\&auto=webp\&disable=upscale}

By \href{https://www.nytimes3xbfgragh.onion/by/thomas-fuller}{Thomas
Fuller}

\begin{itemize}
\item
  Nov. 9, 2016
\item
  \begin{itemize}
  \item
  \item
  \item
  \item
  \item
  \item
  \end{itemize}
\end{itemize}

SAN FRANCISCO --- California, Massachusetts and Nevada legalized
marijuana on Tuesday in what advocates said was a reflection of the
country's changing attitude toward the drug.

Leading up to the election, recreational marijuana use was legal in four
states: Alaska, Colorado, Oregon and Washington, along with Washington,
D.C.

With the addition of California, Massachusetts and Nevada, the
percentage of Americans living in states where marijuana use is legal
for adults rose above 20 percent, from 5 percent.

Representative Earl Blumenauer, Democrat of Oregon and a supporter of
legalization, said Tuesday's votes would add to the pressure on the
federal government to treat cannabis like alcohol, allowing each state
to decide on its own regulations.

``The new administration is not going to want to continue this toxic and
nonproductive war on drugs,'' Mr. Blumenauer said.

The federal government's ban on the drug precludes the interstate sale
of cannabis, even among the states that have approved its use. But
Tuesday's votes created a marijuana bloc stretching down the West Coast,
and Gavin Newsom, the lieutenant governor of California, said he saw an
opportunity for the states where recreational marijuana is now legal to
``coordinate and collaborate'' on the issue, including applying pressure
in Washington to relax the federal ban.

A
\href{http://www.gallup.com/poll/196550/support-legal-marijuana.aspx}{Gallup
poll} in October found nationwide support for legalization at 60
percent, the highest level in the 47 years the organization has tracked
the issue.

Support is rising even though some public health experts warn that there
have been insufficient studies of the drug's effects and that law
enforcement agencies lack reliable tests and protocols to determine
whether a driver is impaired by marijuana.

Supporters in California portrayed legalization as both a social justice
and a criminal justice issue, saying the measure would help redress the
disproportionate numbers of arrests and convictions among minorities for
drug crimes.

``I think of this victory in California as a major victory,'' said
Lauren Mendelsohn, the chairwoman of the board of directors of Students
for Sensible Drug Policy, a group that has campaigned against the
government's war on drugs. ``It shows the whole country that prohibition
is not the answer to the marijuana question.''

Ms. Mendelsohn spoke at a celebration in Oakland for the passage of
Proposition 64, as California's legalization measure was known.

Supporters of legalization in California
\href{http://www.nytimes3xbfgragh.onion/2016/10/25/us/marijuana-legalization-ballot-measures.html}{vastly
outspent} opponents.

As of Nov. 6, pro-legalization committees in the state had raised around
\$23 million, according to the California secretary of state's office.
Chief among the backers were marijuana companies and tech entrepreneurs,
including Sean Parker, a founder of the file-sharing service Napster and
a former president of Facebook, who was the single largest donor to the
campaign. The anti-legalization campaign had spent less than \$2 million
in California.

Kevin Sabet, the president of Smart Approaches to Marijuana, one of the
country's major funders against marijuana legalization initiatives,
attributed the imbalance in campaign spending to investments by
marijuana companies hoping to profit if the industry was legalized.

``There's a lot of money to be made if marijuana is legal, not a lot of
money to be made if it remains illegal,'' he said.

Opponents of legalization say the adoption of medical marijuana laws in
more than 25 states has led to a popular perception that cannabis is
good for you. They have called for more studies on the drug's long-term
effects, particularly on the developing brains of young people.

``There is likely medical promise in the marijuana plant, but that is
different than saying smoked marijuana is medicine,'' Mr. Sabet said.
``We wouldn't smoke the opium plant to get the beneficial effects of
morphine.''

A bill to legalize marijuana in Vermont, supported by Gov. Peter
Shumlin, a Democrat, failed earlier this year. But in Massachusetts,
public support for legalization rose during the fall, even with
bipartisan opposition from the state's top elected officials ---
including Gov. Charlie Baker, a Republican, and Attorney General Maura
Healey, a Democrat --- and an organized anti-legalization campaign.

Lawmakers in Rhode Island were watching Massachusetts closely, and they
are expected to take up a legalization measure of their own now that one
has passed there.

Two other states --- Arizona and Maine --- were voting on recreational
marijuana legalization Tuesday. Arizona voted against the measure. In
Maine, a state with a libertarian streak that began decriminalizing
marijuana decades ago, the referendum on legalization drew scant funded
opposition.

Still, proponents of legalization said California would represent the
biggest victory because of its huge economy and population and also its
fertile soil and amenable climate.

Tuesday's vote reinforced the state's position as the epicenter of
marijuana cultivation for the country, a role it has had illicitly for
decades. Marijuana companies have been positioning themselves for the
prospect of interstate commerce, buying large plots of land in areas
that now grow vegetables and other crops.

The California measure, which passed with 56 percent approval, allows
people over 21 to possess limited amounts of marijuana for personal use
and also permits the personal cultivation of up to six plants in private
residences, provided they are shielded from public view. The sale of
recreational marijuana will not be allowed until licenses are issued, a
process that will take at least two years, said Steve DeAngelo, the
founder of Harborside, a medical marijuana dispensary in Oakland.

California officials expect additional tax revenue of around \$1 billion
from marijuana sales. The revenue is earmarked for the study of medical
marijuana, for the California Highway Patrol to develop procedures to
determine driver impairment due to marijuana consumption, for youth
education on drugs, and for the prevention of environmental damage from
marijuana production, among other programs.

Support for legalization in California cut across all age groups except
voters over 65, according to
\href{http://field.com/fieldpollonline/subscribers/Rls2555.pdf}{a Field
poll} released on Friday. Among those older voters, 42 percent were in
favor, and 57 percent were against.

A large majority of Republicans in the poll, 65 percent, were against
the measure, compared with 72 percent support among Democrats.

Support has been rising steadily since the 1960s, when only around 10
percent of California adults favored legalization, according to a 1969
Field poll, and legalization was the culmination of decades of
campaigning by proponents. A measure to decriminalize marijuana in 1972
was soundly rejected in California, with 66.5 percent of voters opposed
to it. In 1996, California
\href{http://www.nytimes3xbfgragh.onion/1996/11/17/us/votes-on-marijuana-are-stirring-debate.html}{voted}
to allow medical marijuana. But a 2010 measure to permit recreational
use
\href{http://www.nytimes3xbfgragh.onion/2010/11/03/us/politics/03ballot.html}{failed}.

In addition to Tuesday's votes on recreational marijuana, Arkansas,
Florida, Montana and North Dakota had medical marijuana initiatives on
the ballot. All four passed the legislation.

Advertisement

\protect\hyperlink{after-bottom}{Continue reading the main story}

\hypertarget{site-index}{%
\subsection{Site Index}\label{site-index}}

\hypertarget{site-information-navigation}{%
\subsection{Site Information
Navigation}\label{site-information-navigation}}

\begin{itemize}
\tightlist
\item
  \href{https://help.nytimes3xbfgragh.onion/hc/en-us/articles/115014792127-Copyright-notice}{©~2020~The
  New York Times Company}
\end{itemize}

\begin{itemize}
\tightlist
\item
  \href{https://www.nytco.com/}{NYTCo}
\item
  \href{https://help.nytimes3xbfgragh.onion/hc/en-us/articles/115015385887-Contact-Us}{Contact
  Us}
\item
  \href{https://www.nytco.com/careers/}{Work with us}
\item
  \href{https://nytmediakit.com/}{Advertise}
\item
  \href{http://www.tbrandstudio.com/}{T Brand Studio}
\item
  \href{https://www.nytimes3xbfgragh.onion/privacy/cookie-policy\#how-do-i-manage-trackers}{Your
  Ad Choices}
\item
  \href{https://www.nytimes3xbfgragh.onion/privacy}{Privacy}
\item
  \href{https://help.nytimes3xbfgragh.onion/hc/en-us/articles/115014893428-Terms-of-service}{Terms
  of Service}
\item
  \href{https://help.nytimes3xbfgragh.onion/hc/en-us/articles/115014893968-Terms-of-sale}{Terms
  of Sale}
\item
  \href{https://spiderbites.nytimes3xbfgragh.onion}{Site Map}
\item
  \href{https://help.nytimes3xbfgragh.onion/hc/en-us}{Help}
\item
  \href{https://www.nytimes3xbfgragh.onion/subscription?campaignId=37WXW}{Subscriptions}
\end{itemize}
