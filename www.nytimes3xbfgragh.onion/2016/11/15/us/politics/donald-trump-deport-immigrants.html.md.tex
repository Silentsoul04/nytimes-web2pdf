Sections

SEARCH

\protect\hyperlink{site-content}{Skip to
content}\protect\hyperlink{site-index}{Skip to site index}

\href{https://www.nytimes3xbfgragh.onion/section/politics}{Politics}

\href{https://myaccount.nytimes3xbfgragh.onion/auth/login?response_type=cookie\&client_id=vi}{}

\href{https://www.nytimes3xbfgragh.onion/section/todayspaper}{Today's
Paper}

\href{/section/politics}{Politics}\textbar{}What Donald Trump's Vow to
Deport Up to 3 Million Immigrants Would Mean

\url{https://nyti.ms/2ezSkon}

\begin{itemize}
\item
\item
\item
\item
\item
\end{itemize}

Advertisement

\protect\hyperlink{after-top}{Continue reading the main story}

Supported by

\protect\hyperlink{after-sponsor}{Continue reading the main story}

\hypertarget{what-donald-trumps-vow-to-deport-up-to-3-million-immigrants-would-mean}{%
\section{What Donald Trump's Vow to Deport Up to 3 Million Immigrants
Would
Mean}\label{what-donald-trumps-vow-to-deport-up-to-3-million-immigrants-would-mean}}

\includegraphics{https://static01.graylady3jvrrxbe.onion/images/2016/11/15/us/15immigexplainer1/15immigexplainer1-articleLarge.jpg?quality=75\&auto=webp\&disable=upscale}

By
\href{https://www.nytimes3xbfgragh.onion/by/julie-hirschfeld-davis}{Julie
Hirschfeld Davis} and
\href{http://www.nytimes3xbfgragh.onion/by/julia-preston}{Julia Preston}

\begin{itemize}
\item
  Nov. 14, 2016
\item
  \begin{itemize}
  \item
  \item
  \item
  \item
  \item
  \end{itemize}
\end{itemize}

WASHINGTON --- President-elect Donald J. Trump's promise to deport two
million to three million immigrants who have committed crimes suggested
that he would dramatically step up removals of both people in the United
States illegally and those with legal status. If carried out, the plan
potentially would require raids by a vastly larger federal immigration
force to hunt down these immigrants and send them out of the country.

Addressing the issue in an interview broadcast Sunday on the CBS program
``60 Minutes,'' Mr. Trump adopted a softer tone on immigrants than he
did during his campaign, when he called many of them rapists and
criminals. He instead referred to them as ``terrific people,'' saying
they would be dealt with only after the border had been secured and
criminals deported.

But by placing the number of people he aims to turn out of the country
as high as three million, Mr. Trump raised questions about which
immigrants he planned to target for deportation and how he could achieve
removals at that scale.

``If he wants to deport two to three million people, he's got to rely on
tactics that will divide communities and create fear throughout the
country,'' said Kevin Appleby, the senior director of international
migration policy at the Center for Migration Studies of New York. ``He
would have to conduct a sweep, or raids or tactics such as those, to
reach the numbers he wants to reach. It would create a police state, in
which they would have to be aggressively looking for people.''

The details are crucial to understanding the approach of a
president-elect who centered his campaign on a promise to build a border
wall and deport lawbreakers. On Monday,
\href{http://www.nytimes3xbfgragh.onion/2016/11/14/us/politics/barack-obama-trump.html}{President
Obama said} he would urge Mr. Trump to consider leaving in place his
executive actions that have shielded from deportation immigrants brought
to the United States illegally as children.

\hypertarget{a-look-at-the-numbers}{%
\subsection{A Look at the Numbers}\label{a-look-at-the-numbers}}

Asked on ``60 Minutes'' whether he would seek to deport ``millions and
millions of undocumented immigrants,'' Mr. Trump said his priority would
be to remove ``people that are criminal and have criminal records.''

``What we are going to do is get the people that are criminal and have
criminal records --- gang members, drug dealers, we have a lot of these
people, probably two million, it could be even three million. We are
getting them out of our country or we are going to incarcerate,'' Mr.
Trump said. ``But we're getting them out of our country, they're here
illegally.''

The Obama administration has
\href{https://www.dhs.gov/sites/default/files/publications/dhs-congressional-budget-justification-fy2013.pdf}{estimated}
that 1.9 million ``removable criminal aliens'' are in the United States.
That number includes people who hold green cards for legal permanent
residency and those who have temporary visas. It also includes people
who have been convicted of nonviolent crimes such as theft, not just
those found guilty of felonies or gang-related violence.

``They certainly have that many to start,'' said Jessica M. Vaughan,
director of policy studies at the Center for Immigration Studies, a
group that supports reduced immigration.

But even if Mr. Trump's numbers are correct --- and many immigration
activists dispute them --- it is not clear Mr. Trump could carry out
those deportations quickly without violating due process.

In many cases, convicts would have to go through immigration courts
before they could be deported. Those courts are overwhelmed with huge
backlogs, so obtaining deportation orders from judges can take many
months --- if not many years. Thousands of immigrants are serving jail
sentences that under current law cannot be curtailed. According to
official figures, as of June only about 183,000 immigrants had been
convicted of crimes and also had deportation orders so they could be
detained and removed quickly.

\hypertarget{targeting-criminals}{%
\subsection{Targeting Criminals}\label{targeting-criminals}}

Mr. Trump's approach would in some ways be a continuation of policies
Mr. Obama has pursued to focus immigration enforcement on convicted
criminals.

\includegraphics{https://static01.graylady3jvrrxbe.onion/images/2016/11/15/us/15immigexplainer2/15immigexplainer2-articleLarge.jpg?quality=75\&auto=webp\&disable=upscale}

In 2014, his administration issued guidelines instructing agents to make
criminals the highest priorities for their operations. In 2015,
according to Immigration and Customs Enforcement figures, the majority
of the 235,413 people deported --- 59 percent --- were convicted
criminals, while 41 percent were removed for immigration violations.

``Under the Obama administration we have already managed to calibrate
our policy with heavy emphasis on criminal aliens,'' said Muzaffar
Chishti, the director of the New York University School of Law office of
the Migration Policy Institute, a nonpartisan research group.

Since 2009, Mr. Obama has presided over the deportation of about 2.5
million immigrants, prompting sharp criticism from advocacy groups. He
did so in part to build political support for a broad revision of
immigration laws that would have provided a path to citizenship for
immigrants in the country illegally.

Under a now-defunct program known as Secure Communities, the Obama
administration used digital fingerprints shared by local law enforcement
departments to find and deport immigrants who had committed crimes.
Immigration and Customs Enforcement also partnered with local
authorities to prioritize the arrest and detention of criminal aliens.

Both measures helped drive deportations to roughly 400,000 per year
during Mr. Obama's first term. Multiplying that number by many times
would almost certainly require reinstituting a program like Secure
Communities and employing vastly more immigration agents, as well as
using more aggressive tactics to find and remove immigrants who may have
broken the law, according to Mr. Appleby of the Center for Migration
Studies of New York.

\hypertarget{resistance-from-cities}{%
\subsection{Resistance From Cities}\label{resistance-from-cities}}

If Mr. Trump seeks to revive programs of close cooperation between local
police and federal immigration authorities, he is likely to encounter
legal challenges and resistance from dozens of cities and counties that
have curtailed or rejected cooperation.

Mr. Trump has said he would cut off federal funding for cities that
refuse to help federal agents detain unauthorized immigrants. During his
campaign, he highlighted terrible crimes by immigrants he said had
escaped detection because of protective policies.

At a news conference in Chicago on Monday, Mayor Rahm Emanuel, a
Democrat, sought to ease fears of deportation and harassment as he
reiterated Chicago's status as a sanctuary city for immigrants.

``It is important for families that are anxious, it is important for
children and adolescents that are unsure because of Tuesday, to
understand the city of Chicago is your home,'' Mr. Emanuel said. ``You
are always welcome in this city.''

Cook County, where Chicago is, has adopted an especially restrictive
policy on ties between police and federal agents. Mr. Emanuel encouraged
immigrants to call a hotline for legal advice, and said Chicago would
quickly set up a municipal identification program to allow undocumented
immigrants access to city services.

Mayor Betsy Hodges of Minneapolis was defiant. ``I will continue to
stand by and fight for immigrants regardless of President-elect Trump's
threats,'' she said. ``If police officers were to do the work of ICE, it
would harm our ability to keep people safe and solve crimes.'' Mayor Ras
Baraka of Newark, said the city's protections would not change.

In California, lawmakers in a Legislature dominated by Democrats
rejected Mr. Trump's numbers and plans. ``It is erroneous and profoundly
irresponsible to suggest that up to three million undocumented
immigrants living in America are dangerous criminals,'' said Kevin de
León, the president pro tempore of the Senate. He said Mr. Trump's
figures were ``a thinly veiled pretense for a catastrophic policy of
mass deportation,'' and he told immigrants, ``the State of California
stands squarely behind you.''

The Los Angeles police chief, Charlie Beck, said his force would not
change its policies. ``We are not going to work in conjunction with
Homeland Security on deportation efforts,'' he said,
\href{http://www.latimes.com/local/lanow/la-me-ln-los-angeles-police-immigration-20161114-story.html}{according
to The Los Angeles Times}. ``That is not our job, nor will I make it our
job.''

Advertisement

\protect\hyperlink{after-bottom}{Continue reading the main story}

\hypertarget{site-index}{%
\subsection{Site Index}\label{site-index}}

\hypertarget{site-information-navigation}{%
\subsection{Site Information
Navigation}\label{site-information-navigation}}

\begin{itemize}
\tightlist
\item
  \href{https://help.nytimes3xbfgragh.onion/hc/en-us/articles/115014792127-Copyright-notice}{©~2020~The
  New York Times Company}
\end{itemize}

\begin{itemize}
\tightlist
\item
  \href{https://www.nytco.com/}{NYTCo}
\item
  \href{https://help.nytimes3xbfgragh.onion/hc/en-us/articles/115015385887-Contact-Us}{Contact
  Us}
\item
  \href{https://www.nytco.com/careers/}{Work with us}
\item
  \href{https://nytmediakit.com/}{Advertise}
\item
  \href{http://www.tbrandstudio.com/}{T Brand Studio}
\item
  \href{https://www.nytimes3xbfgragh.onion/privacy/cookie-policy\#how-do-i-manage-trackers}{Your
  Ad Choices}
\item
  \href{https://www.nytimes3xbfgragh.onion/privacy}{Privacy}
\item
  \href{https://help.nytimes3xbfgragh.onion/hc/en-us/articles/115014893428-Terms-of-service}{Terms
  of Service}
\item
  \href{https://help.nytimes3xbfgragh.onion/hc/en-us/articles/115014893968-Terms-of-sale}{Terms
  of Sale}
\item
  \href{https://spiderbites.nytimes3xbfgragh.onion}{Site Map}
\item
  \href{https://help.nytimes3xbfgragh.onion/hc/en-us}{Help}
\item
  \href{https://www.nytimes3xbfgragh.onion/subscription?campaignId=37WXW}{Subscriptions}
\end{itemize}
