Sections

SEARCH

\protect\hyperlink{site-content}{Skip to
content}\protect\hyperlink{site-index}{Skip to site index}

\href{https://www.nytimes3xbfgragh.onion/section/politics}{Politics}

\href{https://myaccount.nytimes3xbfgragh.onion/auth/login?response_type=cookie\&client_id=vi}{}

\href{https://www.nytimes3xbfgragh.onion/section/todayspaper}{Today's
Paper}

\href{/section/politics}{Politics}\textbar{}Trump Claims, With No
Evidence, That `Millions of People' Voted Illegally

\url{https://nyti.ms/2g7cF3I}

\begin{itemize}
\item
\item
\item
\item
\item
\item
\end{itemize}

Advertisement

\protect\hyperlink{after-top}{Continue reading the main story}

Supported by

\protect\hyperlink{after-sponsor}{Continue reading the main story}

\hypertarget{trump-claims-with-no-evidence-that-millions-of-people-voted-illegally}{%
\section{Trump Claims, With No Evidence, That `Millions of People' Voted
Illegally}\label{trump-claims-with-no-evidence-that-millions-of-people-voted-illegally}}

\includegraphics{https://static01.graylady3jvrrxbe.onion/images/2016/11/28/us/28TRANSITION2-web/28TRANSITION2-web-articleInline.jpg?quality=75\&auto=webp\&disable=upscale}

By \href{http://www.nytimes3xbfgragh.onion/by/michael-d-shear}{Michael
D. Shear} and
\href{http://www.nytimes3xbfgragh.onion/by/maggie-haberman}{Maggie
Haberman}

\begin{itemize}
\item
  Nov. 27, 2016
\item
  \begin{itemize}
  \item
  \item
  \item
  \item
  \item
  \item
  \end{itemize}
\end{itemize}

WASHINGTON --- President-elect Donald J. Trump said on Sunday that he
had fallen short in the popular vote in the general election only
because millions of people had voted illegally, leveling the baseless
claim as part of a \href{https://twitter.com/realDonaldTrump}{daylong
storm of Twitter posts} voicing anger about a three-state recount push.

``In addition to winning the Electoral College in a landslide, I won the
popular vote if you deduct the millions of people who voted illegally,''
Mr. Trump wrote Sunday afternoon.

The series of posts came one day after Hillary Clinton's campaign said
it would participate in a recount effort being undertaken in Wisconsin,
and potentially in similar pushes in Michigan and Pennsylvania, by Jill
Stein, who was the Green Party candidate. Mr. Trump's statements revived
claims he made during the campaign, as polls suggested he was losing to
Mrs. Clinton, about a rigged and corrupt system.

The Twitter outburst also came as Mr. Trump is laboring to fill crucial
positions in his cabinet, with his advisers enmeshed in a rift over whom
he should select as secretary of state. On Sunday morning, Kellyanne
Conway, a top adviser, extended a public campaign to undermine one
contender, Mitt Romney --- a remarkable display by a member of a
president-elect's team. In television appearances, she accused Mr.
Romney of having gone ``out of his way to hurt'' Mr. Trump during the
Republican primary contests.

Claims of wide-scale voter fraud have been advanced for years by
Republicans, though virtually no evidence of such improprieties has been
discovered --- especially on the scale of ``millions'' that Mr. Trump
claimed.

Late on Sunday, again without providing evidence, he referred in a
\href{https://twitter.com/realDonaldTrump/status/803033642545115140}{Twitter
post} to ``serious voter fraud in Virginia, New Hampshire and
California.''

A day earlier, Mr. Trump's transition team ridiculed the idea that
recounts were needed. ``This is a scam by the Green Party for an
election that has already been conceded,'' it said in a statement, ``and
the results of this election should be respected instead of being
challenged and abused.''

That message runs counter to the one Mr. Trump sent on Sunday with his
fraud claims --- if millions of people voted illegally, presumably
officials across the country would want to pursue large-scale ballot
recounts and fraud investigations.

But the Twitter posts could energize some of his supporters, who have
claimed online that Mrs. Clinton's two million-vote lead in the popular
vote has been faked. Mr. Trump at times promoted other conspiracy
theories during the campaign, including claiming that Senator Ted Cruz's
father was somehow tied to the assassination of President John F.
Kennedy.

Many of Mrs. Clinton's supporters have been galvanized by the notion
that vote recounts in the three states --- where Mr. Trump leads by a
combined total of about 100,000 votes --- could somehow overturn Mr.
Trump's commanding Electoral College victory. By announcing, three weeks
after Mrs. Clinton conceded, that it would participate in the Wisconsin
recount, her team has helped reignite the contentious atmosphere of the
campaign, of which Mr. Trump's Twitter barrages were a fixture. (By all
accounts, Mr. Trump types out many, though not all, of his own Twitter
posts.)

After spending almost five days in Palm Beach, Fla., where he celebrated
Thanksgiving at his Mar-a-Lago resort, Mr. Trump made no public
statements on Sunday other than those via Twitter. He returned in the
afternoon to Trump Tower in Manhattan.

Through the day, Mr. Trump appeared fixated on the recount and his
electoral performance. In a series of midafternoon Twitter posts, not
long before he boarded his flight, Mr. Trump boasted that he could have
easily won the ``so-called popular vote'' if he had campaigned only in
``3 or 4'' states, presumably populous ones.

``I would have won even more easily and convincingly (but smaller states
are forgotten)!'' he wrote.

The afternoon messages followed a string of early-morning Twitter posts
in which Mr. Trump railed against the recount efforts. In an
\href{https://twitter.com/realDonaldTrump/status/802849330176659456}{initial
post} at 7:19, he wrote: ``Hillary Clinton conceded the election when
she called me just prior to the victory speech and after the results
were in. Nothing will change.''

\includegraphics{https://static01.graylady3jvrrxbe.onion/images/2016/11/28/us/28TRANSITION1-web/28TRANSITION1-web-articleLarge.jpg?quality=75\&auto=webp\&disable=upscale}

He went on to quote a comment by Mrs. Clinton during one of their
debates, in which she said she was horrified by Mr. Trump's refusal to
say that he would accept the outcome of the election. And he noted that
in her concession speech, she had urged people to respect the vote
results.

```We have to accept the results and look to the future, Donald Trump is
going to be our President,''' Mr. Trump wrote, quoting Mrs. Clinton.

One person who spoke with Mr. Trump over the holiday weekend said the
president-elect had appeared to be preoccupied by suggestions that a
recount might be started, even as his aides played down any concerns.
Another friend said Mr. Trump felt crossed by Mrs. Clinton, who he
believed had conceded the race and accepted the results.

In a post on Medium, Marc Elias, the Clinton team's general counsel,
said the campaign would participate in Ms. Stein's recount effort with
little expectation that it would change the result, partly out of a
sense of duty to the millions who voted for Mrs. Clinton.

``We do so fully aware that the number of votes separating Donald Trump
and Hillary Clinton in the closest of these states --- Michigan --- well
exceeds the largest margin ever overcome in a recount,'' Mr. Elias said,
noting that Clinton campaign officials had found no ``actionable
evidence'' of hacking or attempts to tamper with the vote.

Late Sunday night, Mr. Elias responded on Twitter to Mr. Trump's
allegations,
\href{https://twitter.com/marceelias/status/803071609330397184}{writing},
``We are getting attacked for participating in a recount that we didn't
ask for by the man who won election but thinks there was massive
fraud.''

In Wisconsin, Mr. Trump leads by 22,177 votes. In Michigan, he has a
lead of 10,704 votes, and in Pennsylvania, his advantage is 70,638
votes.

Mr. Trump's aides echoed his concerns about the recount effort in
appearances on Sunday morning television news programs. Ms. Conway, who
was his campaign manager, said on NBC's ``Meet the Press'' that Mrs.
Clinton and her campaign advisers would have to decide ``whether they're
going to be a bunch of crybabies.''

As for the debate over Mr. Romney, Ms. Conway, echoing comments she
posted last week on Twitter, made clear that she opposed choosing Mr.
Romney as secretary of state.

``There was the `Never Trump' movement, and then there was Gov. Mitt
Romney,'' she said on ABC. During the primaries, Mr. Romney called Mr.
Trump a ``fraud'' and a ``phony.''

Ms. Conway said it was important for Mr. Trump to seek to unify the
Republican Party by making gestures to those who opposed his candidacy.
But, she added, ``I don't think the cost of party unity has to be the
secretary of state position.''

Moments after appearing on the show, Ms. Conway, who is under
consideration to be Mr. Trump's press secretary, wrote
\href{https://twitter.com/KellyannePolls/status/802884826520907776}{on
Twitter} that she had told Mr. Trump her opinion privately, ``and I'll
respect his decision.''

On ``Meet the Press,'' she said people felt ``betrayed'' by the idea
that Mr. Romney could get a top cabinet job. ``I'm not campaigning
against anyone,'' she said. ``I'm just a concerned citizen.''

``We don't even know if he voted for Donald Trump,'' she added.

Advertisement

\protect\hyperlink{after-bottom}{Continue reading the main story}

\hypertarget{site-index}{%
\subsection{Site Index}\label{site-index}}

\hypertarget{site-information-navigation}{%
\subsection{Site Information
Navigation}\label{site-information-navigation}}

\begin{itemize}
\tightlist
\item
  \href{https://help.nytimes3xbfgragh.onion/hc/en-us/articles/115014792127-Copyright-notice}{©~2020~The
  New York Times Company}
\end{itemize}

\begin{itemize}
\tightlist
\item
  \href{https://www.nytco.com/}{NYTCo}
\item
  \href{https://help.nytimes3xbfgragh.onion/hc/en-us/articles/115015385887-Contact-Us}{Contact
  Us}
\item
  \href{https://www.nytco.com/careers/}{Work with us}
\item
  \href{https://nytmediakit.com/}{Advertise}
\item
  \href{http://www.tbrandstudio.com/}{T Brand Studio}
\item
  \href{https://www.nytimes3xbfgragh.onion/privacy/cookie-policy\#how-do-i-manage-trackers}{Your
  Ad Choices}
\item
  \href{https://www.nytimes3xbfgragh.onion/privacy}{Privacy}
\item
  \href{https://help.nytimes3xbfgragh.onion/hc/en-us/articles/115014893428-Terms-of-service}{Terms
  of Service}
\item
  \href{https://help.nytimes3xbfgragh.onion/hc/en-us/articles/115014893968-Terms-of-sale}{Terms
  of Sale}
\item
  \href{https://spiderbites.nytimes3xbfgragh.onion}{Site Map}
\item
  \href{https://help.nytimes3xbfgragh.onion/hc/en-us}{Help}
\item
  \href{https://www.nytimes3xbfgragh.onion/subscription?campaignId=37WXW}{Subscriptions}
\end{itemize}
