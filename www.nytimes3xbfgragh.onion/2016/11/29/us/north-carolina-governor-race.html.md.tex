Sections

SEARCH

\protect\hyperlink{site-content}{Skip to
content}\protect\hyperlink{site-index}{Skip to site index}

\href{https://www.nytimes3xbfgragh.onion/section/us}{U.S.}

\href{https://myaccount.nytimes3xbfgragh.onion/auth/login?response_type=cookie\&client_id=vi}{}

\href{https://www.nytimes3xbfgragh.onion/section/todayspaper}{Today's
Paper}

\href{/section/us}{U.S.}\textbar{}In North Carolina, No End in Sight to
Governor's Race

\begin{itemize}
\item
\item
\item
\item
\item
\end{itemize}

Advertisement

\protect\hyperlink{after-top}{Continue reading the main story}

Supported by

\protect\hyperlink{after-sponsor}{Continue reading the main story}

\hypertarget{in-north-carolina-no-end-in-sight-to-governors-race}{%
\section{In North Carolina, No End in Sight to Governor's
Race}\label{in-north-carolina-no-end-in-sight-to-governors-race}}

\includegraphics{https://static01.graylady3jvrrxbe.onion/images/2016/11/30/us/30carolina-diptych/30carolina-diptych-articleInline.jpg?quality=75\&auto=webp\&disable=upscale}

By \href{https://www.nytimes3xbfgragh.onion/by/richard-fausset}{Richard
Fausset}

\begin{itemize}
\item
  Nov. 29, 2016
\item
  \begin{itemize}
  \item
  \item
  \item
  \item
  \item
  \end{itemize}
\end{itemize}

RALEIGH, N.C. --- Three weeks after Election Day, the North Carolina
governor's race remains an unresolved, hotly contested muddle. The
Democratic challenger, Roy Cooper, has declared victory. But the
Republican incumbent, Pat McCrory, trailing by thousands of votes in the
as-yet-unfinished tally, has refused to concede, as he and his allies
charge that the election was marred by numerous irregularities.

Mr. McCrory's foes are furious. Last week, they parked a U-Haul truck in
front of the governor's mansion and waved moving boxes adorned with the
face of Ray Charles as they sang, ``Hit the Road, Pat.'' On Monday
evening, hundreds gathered at the state Capitol, shouting, ``No stealing
our election!''

The North Carolina imbroglio is so complicated, with so many moving
pieces, that a spokesman for the State Board of Elections could not say
Tuesday when it might be resolved. It comes amid four years of pitched
ideological battles in a state sharply split between Democrats and
Republicans.

But it also comes amid a broader wave of skepticism about the integrity
of the basic mechanics of the American electoral process, including the
recount of the presidential results in three states sought by Jill
Stein, the Green Party candidate. The move enraged President-elect
Donald J. Trump, who this week said on Twitter that ``millions'' of
people had illegally voted for Hillary Clinton, a widely derided claim
for which he offered no evidence.

For some voters here, it can all feel like a failing of a divided
nation, and a divided state. The Rev. Phillip Fackler, an Episcopal
chaplain, said that it was difficult these days for liberals and
conservatives to sit down and work out their differences.

``Maybe it's easier to say, `Something was amiss,''' Mr. Fackler, 36,
said on Monday as he and hundreds of anti-McCrory protesters waited for
the start of the rally at the Capitol.

According to a preliminary tally by the State Board of Elections, Mr.
Cooper was leading Mr. McCrory by 9,813 votes as of Tuesday night ---
perilously close to the 10,000-vote difference that would prohibit Mr.
McCrory from demanding a recount. Mr. Cooper's lead has been growing
since the Nov. 8 election, as thousands of provisional votes from around
the state have been verified.

But nine of the state's 100 counties had still not completed certifying
their vote totals as of Tuesday night, in part because of challenges
filed by Mr. McCrory's supporters in dozens of counties, said a
spokesman for the State Board of Elections, Patrick Gannon.

A number of those challenges charge that the names of some voters
belonged to people who had died or to felons ineligible to vote. With
that in mind, Mr. McCrory's supporters have lashed out at Democrats who
have called on the governor to concede.

``Why is Roy Cooper so insistent on circumventing the electoral process
and counting the votes of dead people and felons?'' a McCrory campaign
spokesman, Ricky Diaz, said in a statement this month.

But Mr. Cooper's supporters argue that any alleged discrepancies are too
few in number to affect the outcome of the election --- and that the
Republicans' language is often misleading. Allison Riggs, the senior
attorney for the \href{https://www.southerncoalition.org/}{Southern
Coalition for Social Justice} in Durham, N.C., said that all of the dead
voters listed in the challenges were alive when they cast ballots during
the early-voting period but died before Election Day. (Such votes are
ineligible in North Carolina.)

The group also found that many of the people identified as ineligible to
vote because of a criminal record were, in fact, eligible.

Bob Hall, the executive director of the advocacy group
\href{http://nc-democracy.org/}{Democracy North Carolina}, said that 43
voters were listed in the county-by-county challenges as being
ineligible felons. The challenges listed 24 dead voters, he said.

On Nov. 20, Mr. Cooper, the state's attorney general, released a brief
video addressing his supporters.

``Because of your hard work, we have won this race for governor,'' he
said. He also said that Mr. McCrory was sowing ``confusion and fear''
with his postelection speeches.

The pro-McCrory forces say they simply want to ensure that the election
tally is correct. ``Any candidate, even candidates that the Democrats
hate, are entitled under the law to see what the count is, and to decide
whether they should call for a recount,'' Dallas Woodhouse, the
executive director of the state Republican Party, said on Tuesday.

Under North Carolina law, the governor appoints the five members of the
state elections board, and those members, in turn, appoint the members
of the 100 county boards. As a result, Republicans enjoy majorities on
all of these boards. Nonetheless, the governor's team has not fared well
with them thus far. Some county boards have already thrown out their
challenges.

On Monday, the McCrory team suffered a further blow when the state board
ruled that the challenged votes could be thrown out by county election
officials only if there were enough votes to determine the outcome of a
local election.

Ford Porter, a spokesman for Mr. Cooper's campaign, said Tuesday that a
number of the remaining challenges appeared unlikely to meet that
threshold.

``This is a devastating blow to the McCrory campaign and further
evidence that there is no path to victory for Governor McCrory,'' Mr.
Cooper's campaign manager, Trey Nix, said in a statement.

Mr. McCrory has announced his intention to call for a statewide recount.
But it seems he would prefer to make a last stand in liberal-dominated
Durham County. On election night, just before midnight, a tranche of
roughly 90,000 ballots were reported there, tipping the vote count in
Mr. Cooper's favor.

To many Republicans watching the results on TV, the sudden reversal of
fortune for their nominee came as a shock, especially because their
party was handily winning the presidential race in the state, and
Senator Richard Burr, a Republican, easily won re-election.

Since then, the Durham County board has rejected a protest filed by
Thomas Stark, a lawyer and McCrory supporter, who claimed that county
officials had engaged in ``malfeasance'' in counting those ballots
because they relied on inaccurate ballot machines.

But the state board has agreed to hear an appeal of that ruling on
Wednesday afternoon. If the board allows a Durham County recount, and
the results are the same as the original count, the McCrory team ``will
be prepared to withdraw its statewide recount request in the governor's
race,'' according to a statement released Saturday.

The unsettled governor's race is only one of many voting-related dramas
that have engulfed North Carolina recently. In August, a federal court
ruled that the state legislative map was unconstitutional because 28 of
the districts drawn by Republicans constituted racial gerrymanders.

On Tuesday, the court established a March 15, 2017, deadline for
redrawing the map --- and ruled that legislators in districts affected
by the redrawing stand for re-election next fall. Republican legislators
called the ruling a ``gross overreach'' and vowed to appeal.

More delays to a resolution of the governor's race could be in store
because of a federal lawsuit filed by Francis X. De Luca, president of
the \href{https://www.nccivitas.org/}{Civitas Institute}, a conservative
think tank based in Raleigh.

The suit, filed in United States District Court for the Eastern District
of North Carolina, concerns a state provision that allows residents to
register and vote on the same day during an early-voting period.

Mr. De Luca argues that such registrations cannot be verified until
after the state board certifies the election results, so he has asked
the court to order the state board not to include those votes in its
final tally until the registrations can be verified.

A hearing is scheduled for Dec. 8.

Advertisement

\protect\hyperlink{after-bottom}{Continue reading the main story}

\hypertarget{site-index}{%
\subsection{Site Index}\label{site-index}}

\hypertarget{site-information-navigation}{%
\subsection{Site Information
Navigation}\label{site-information-navigation}}

\begin{itemize}
\tightlist
\item
  \href{https://help.nytimes3xbfgragh.onion/hc/en-us/articles/115014792127-Copyright-notice}{©~2020~The
  New York Times Company}
\end{itemize}

\begin{itemize}
\tightlist
\item
  \href{https://www.nytco.com/}{NYTCo}
\item
  \href{https://help.nytimes3xbfgragh.onion/hc/en-us/articles/115015385887-Contact-Us}{Contact
  Us}
\item
  \href{https://www.nytco.com/careers/}{Work with us}
\item
  \href{https://nytmediakit.com/}{Advertise}
\item
  \href{http://www.tbrandstudio.com/}{T Brand Studio}
\item
  \href{https://www.nytimes3xbfgragh.onion/privacy/cookie-policy\#how-do-i-manage-trackers}{Your
  Ad Choices}
\item
  \href{https://www.nytimes3xbfgragh.onion/privacy}{Privacy}
\item
  \href{https://help.nytimes3xbfgragh.onion/hc/en-us/articles/115014893428-Terms-of-service}{Terms
  of Service}
\item
  \href{https://help.nytimes3xbfgragh.onion/hc/en-us/articles/115014893968-Terms-of-sale}{Terms
  of Sale}
\item
  \href{https://spiderbites.nytimes3xbfgragh.onion}{Site Map}
\item
  \href{https://help.nytimes3xbfgragh.onion/hc/en-us}{Help}
\item
  \href{https://www.nytimes3xbfgragh.onion/subscription?campaignId=37WXW}{Subscriptions}
\end{itemize}
