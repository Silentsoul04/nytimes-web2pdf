The Year's Most Captivating Film Performances

\url{https://nyti.ms/2hmwtxz}

\begin{itemize}
\item
\item
\item
\item
\item
\item
\end{itemize}

\includegraphics{https://static01.graylady3jvrrxbe.onion/images/2016/12/11/magazine/11cover2/11cover2-articleLarge.jpg?quality=75\&auto=webp\&disable=upscale}

Sections

\protect\hyperlink{site-content}{Skip to
content}\protect\hyperlink{site-index}{Skip to site index}

Great Performers

\hypertarget{the-years-most-captivating-film-performances}{%
\section{The Year's Most Captivating Film
Performances}\label{the-years-most-captivating-film-performances}}

The Times critics Wesley Morris and A.O. Scott discuss how the year's
best acting showed us two kinds of power: One found in professional
experience, the other in the experience of being alive.

Ashton Sanders of ``Moonlight''Credit...Jack Davison for The New York
Times

Supported by

\protect\hyperlink{after-sponsor}{Continue reading the main story}

By \href{https://www.nytimes3xbfgragh.onion/by/wesley-morris}{Wesley
Morris} and \href{https://www.nytimes3xbfgragh.onion/by/a-o--scott}{A.O.
Scott}

\begin{itemize}
\item
  Dec. 8, 2016
\item
  \begin{itemize}
  \item
  \item
  \item
  \item
  \item
  \item
  \end{itemize}
\end{itemize}

\textbf{WESLEY MORRIS:} What is it about watching someone become someone
else that has been so mesmerizing for so many people for so long? What
is it about a person's becoming his or her ultimate self that we can't
resist? It's all-powerful, all-consuming stuff. In life, that
consumption can be dangerous. But in art, I \emph{want} my mind
controlled by that power. I want an actor's skill, intelligence, energy,
body and face to overtake me. I love the movie magic of no longer seeing
an actor --- or especially a star --- but a person.

\textbf{A.O. SCOTT:} Me, too. And the best place to surrender to the
power you describe is still in a darkened theater, in spite of all the
competing performances we can sample on television and elsewhere. The
movies still feel larger than life and blissfully distant from it, even
as life grows crazier and more improbable than the movies. But the big
screen hardly holds a monopoly. Let's be honest: If we weren't limited
to movie stars here, we might have come up with a list of great
performers that included Beyoncé, Donald Trump, LeBron James and whoever
we suppose Elena Ferrante to be.

\includegraphics{https://static01.graylady3jvrrxbe.onion/images/2016/12/11/magazine/11atthemovies1-krisha/11atthemovies1-articleInline.jpg?quality=75\&auto=webp\&disable=upscale}

But what's most fascinating to me about movies now is how many different
kinds of acting coexist within the art form. In these pages, we have
some people who excel through the discipline of old-fashioned theatrical
technique and others who seem less like actors than like unwitting
documentary subjects. I marvel at the deep craft that Denzel Washington
and Viola Davis wield in bringing to life characters invented by August
Wilson (in ``Fences''). I also marvel at the apparent artlessness of
Krisha Fairchild (in ``Krisha'') and Sasha Lane (in ``American Honey''),
who don't seem to be pretending at all. Washington and Davis, playing a
long-married couple living in Pittsburgh in the 1950s, feel like
recognizable American archetypes; the characters, at least at first, are
as familiar as the people playing them. Krisha (the character) is a
middle-aged woman with a history of substance abuse and awful behavior
who pays a Thanksgiving visit to her semi-estranged family. Fairchild
(the performer) seems so much like that person that you wonder if she
might show up at \emph{your} next holiday meal. Similarly, you can't be
entirely sure that Sasha Lane won't ring your doorbell selling dubious
magazine subscriptions. Each performer pursues, and achieves, a distinct
kind of credibility.

\textbf{MORRIS:} Yes, this group of men and women presents a gamut of
what movie acting can be and do: knock you out, break your heart, scare
you, delight you, amuse and haunt you, whether through vertiginously
high style (hey there, Denzel and Viola) or plain-old naturalism.
Sometimes the feat is intangible, like the way that Ruth Negga's eyes
and frown, as she plays
\href{https://www.nytimes3xbfgragh.onion/2016/11/13/movies/loving-jeff-nichols-interview.html}{Mildred
Loving in ``Loving''} --- Jeff Nichols's film about the landmark
interracial-marriage case --- offer deep reserves of sadness. If
Modigliani ever painted the Delta blues, it would look something Negga's
expression in this movie. She doesn't have much to do in ``Loving,'' and
part of me wishes she were able to get up to more. But here's a woman
who can calibrate a long face so that it has a gravitational pull. And
that's really something.

Image

Ruth Negga in ``Loving'' (with Joel Edgerton).Credit...Focus Features

Negga works in the naturalist style you identified in Sasha Lane and
Krisha Fairchild. But it's funny: I hadn't thought of Fairchild as
practicing naturalism until just now. Fairchild has an arresting beauty
--- it's crystal-shop Sharon Stone --- and a flair for tragedy that
reminds me of Cassavetes-era Gena Rowlands. But part of what makes her
so good is the simple surprise of her talent: We had never seen this
woman before. In the opening scene, the minute she gets out of that car,
takes a few steps and then trudges back to get the suitcase she forgot,
I knew I was in for something other than --- or in addition to ---
realism. I knew I was watching a woman wielding a control over her
rawness. Her commitment becomes the drama. We're so absorbed in this
woman's state of mind that the simple dressing of a turkey becomes
Hitchcock-suspenseful.

\textbf{SCOTT:} And one of the most hackneyed conceits in all of movies
--- the family holiday from hell --- becomes the freshest, scariest,
most electrifying domestic nightmare anyone has ever shot.

Image

Trevante Rhodes in ``Moonlight.''Credit...A24

\textbf{MORRIS:} How can someone this good have gone unnoticed for this
long? There are lots of fresh faces in our group, and some of what
makes, say,
\href{https://www.nytimes3xbfgragh.onion/2016/01/30/movies/sundance-actors-ages-10-and-16-join-the-breakthrough-ranks.html}{Royalty
Hightower}, who is 11, so different from a legend like Isabelle Huppert
is that walking into ``The Fits,'' I really didn't know where that
little girl with the intoxicating studiousness was going to take me. I
just knew I'd follow her wherever that turned out to be. We talked at
some point about what unites the actors gathered for this issue, and I
really do think, whether it's Hightower or someone we see all the time,
like Washington or Huppert, it's surprise, don't you?

\textbf{SCOTT:} Yes, we're used to seeing the same thing repeated
everywhere --- not only in the entertainment we consume but also in our
interactions with the real world. Most of us, most of the time, are
content to rest on our assumptions about what other people are like.
Online, we traffic in algorithms, aggregations and demographic data
sets, the pseudoscientific dressing for old habits of generalization and
stereotyping. But a strong performance can smash through that
complacency with a force akin to love or friendship, reminding us that
the odd, the idiosyncratic, the irreducibly individual still have a
place in our standardized and quantified world. We quickly discover that
we've never seen anyone like this before.

What astonishes me is how many different forms this uniqueness can take,
how many distinct techniques there are for arriving at it. Troy Maxson,
Denzel Washington's character in ``Fences,'' sails onto the screen on
gusts of verbiage. He probably utters more words in the first two
minutes of that movie than all three of the actors who play Chiron do in
the entirety of ``Moonlight.'' But the character, born in August
Wilson's words, takes life in Washington's body. Troy's history --- a
brutal childhood, a spell in prison, a baseball career and a lot of sex,
liquor and manual labor in the midst of it all --- is all written in
Washington's posture. Shoulders back, belly forward, all the weight
centered in the hips, a kinetic vision of masculine grace and power worn
down and gone slightly to seed.

What happens in
``\href{https://www.nytimes3xbfgragh.onion/2016/10/21/movies/moonlight-review.html}{Moonlight}''
is something else altogether. It tells what might have been a standard
coming-of-age story, about a young man growing up poor, black and queer
in the Liberty City section of Miami. But instead of the usual plot
points, the story is fashioned out of moods and emotions, by the
flickerings of Chiron's consciousness and the stirrings of his desire.
It's a three-dimensional portrait in color and sound, and Alex Hibbert,
Ashton Sanders and Trevante Rhodes --- who play Chiron as a schoolboy, a
teenager and a grown man --- execute a \emph{coup de cinéma} that feels
to me utterly without precedent. A single soul occupies three distinct
bodies.

Image

Sasha Lane in ``American Honey'' (with Shia LaBeouf).Credit...A24

\textbf{MORRIS:} In American movies, children have often been asked to
be hams, to weaponize cuteness and spunk. But you get tired of kiddie
kleptomaniacs. You want to see some risks. Now the old European style of
acting has happily come into American independent filmmaking: lots of
stillness and quiet, these moments of privacy, solitude and thinking,
from actors who, to us, are strangers --- and often uncannily seasoned
young people. Sometimes all that quiet and stillness comes off as
bashful, as amateurism. But often, as in ``Moonlight,'' it's simply
confidence re-engineered. Each of the three chapters in Chiron's life is
handled by a different actor. Plenty of films use flashbacks or
prologues featuring a younger version of the protagonist to telescope an
adult hero's journey, and it can be a hokey device for any number of
reasons, including misbegotten casting. The coup you identify in
``Moonlight'' has precisely to do with the brilliance of its casting,
and for us, that starts with Hibbert, who plays the youngest Chiron with
top-secret assurance. Some actors really are hams. This kid was born
with the discipline of a vegetarian.

Hibbert's a peanut who can sit at a dining-room table or stand in a long
shot of an open field and draw you to him. He's got a serene,
self-protectively hardened face that conveys simultaneous hurt and
wonder and an awareness that the wonder can hurt. All of that complexity
is in the baton that's passed to Sanders, a very different performer
with a very different face, but he amplifies what Hibbert gives him and
adds physicality, volatility and grace. Sanders has two scenes that
involve walking, and the way he uses each gait to evoke a distinctly
different mood of dismay killed me. The director, Barry Jenkins, uses
two performances to establish these feelings and experiences, then hands
them off to Rhodes, whose final incarnation of Chiron strategically
buries all that has preceded him in rock-hard thuggery. I think he has
the hardest job. First, \emph{you} try following Hibbert and Sanders!
Second, he has to perform a hardness that we know is an act --- he's got
to exude \emph{some} of Hibbert and Sanders's innocence. Third, he has
to nail a crucial phone call for the movie's psychological junction to
make its thunderous ``click.'' When he takes that call, the experience
of watching what's happening on Rhodes's face, in his being, is not
unlike finding out that all Charles Foster Kane ever wanted was that
sled. That's so much of the movie right there: three performances
snapping into place with one drug dealer's dropped jaw.

Image

Royalty Hightower in ``The Fits.''Credit...Oscilloscope

\textbf{SCOTT:} And it's not just that --- as in ``Citizen Kane'' ---
the audience is finally understanding something about the character that
he already knew. Chiron at that moment is finally understanding himself,
and we are witnesses to the unlocking of his inner secret.

\textbf{MORRIS:} A few of these performances rely on a moment or two to
unlock what best illustrates a character's character. I'm thinking about
Casey Affleck, Viola Davis and Taraji P. Henson. The beauty of what
they've done with their roles has everything to do with patience ---
remembering that you're playing much more than a single, crucial moment
--- but also to do with an understanding of how real people work,
emotionally.

Image

Casey Affleck in ``Manchester by the Sea'' (with Michelle
Williams).Credit...Roadside Attractions

\textbf{SCOTT:} Real people are often emotionally opaque to others and
unavailable even to themselves. Perhaps the greatest challenge screen
actors face is capturing that opacity. Nonprofessional or untrained
actors sometimes have an advantage, because they haven't been taught the
tricks of controlled and sublimated expression, staples of the
curriculum in those acting schools that emphasize the psychological
basis of the craft. There is an enigmatic quality to the characters in
Italian neorealism, early Bresson and a lot of Kiarostami that comes
from the use of untrained actors.

I think Sasha Lane has some of that quality. In
``\href{https://www.nytimes3xbfgragh.onion/2016/10/02/movies/american-honey-open-highways-free-spirits.html}{American
Honey},'' her first film, she plays a girl named Star (of course she's
named Star), who escapes a grim family situation and runs off with a
feral crew of teenagers who roam the heartland selling magazine
subscriptions nobody wants. They pretend to be raising money for
scholarships or sports teams, but really they're the pawns in a grim
little scam, the main point of which seems to be their own exploitation.
There's a sequence in which Star breaks away from her mentor and
sometime-kind-of-boyfriend (played by Shia LaBeouf) to take a ride with
some middle-aged guys in cowboy hats who are out for a good time. So is
Star, but it might not be the same kind of good time, and the scene
unfolds with terrifying unpredictability. We don't know what's going to
happen to Star --- maybe something unspeakable, maybe something
uncomfortable, maybe something crazy and fun --- and we also don't know
what she wants to happen, how much control she's in or even thinks she's
in. But I don't think Lane is playing uncertainty or ambivalence in any
conventional sense. She's not communicating her character's
recklessness; she's matching it.

Image

Isabelle Huppert in ``Elle'' (with Laurent Lafitte).Credit...Sony
Pictures Classics/Everett

\textbf{MORRIS:} I love that way of thinking about a certain kind of
rawness. It's a risk, right? Andrea Arnold, the director of ``American
Honey,'' is a raw actor's director. She knows what to do with roughness
and edginess and daring. She's searching for that in her casting. When
it works, it's spellbinding for the audience. Harvey Keitel in any of
his primes had it. So have Jennifer Jason Leigh and Angela Bassett in
theirs. Joaquin Phoenix is my current favorite in this style of
post-Marlon Brando performance. Part of it is: How far in, out, up, down
will this actor go? Will he or she get to a place where all the training
and technique are gone and you're just watching an arsonist play with
matches? Guys like Nicolas Cage or Ed Harris or Tom Hardy can go too far
with the fire and sometimes seem to act just to burn everything down.

\textbf{SCOTT:} In the anti-Method acting manifesto ``True and False,''
David Mamet argues that what we respond to in a great performance is not
the dubious ``emotional truth'' of the character but rather the
existential courage of the actor. This is something more clearly seen
onstage, where there is no chance to retake or fix it in postproduction
and where the audience is always at least somewhat conscious of watching
``a real body in real time.'' But the camera also forces risks on those
who step in front of it. When I reflect on the moments in recent movies
that have stayed with me --- Taraji P. Henson arriving in her NASA
office, soaking wet, to claim the respect that is her due; Casey Affleck
walking away from Michelle Williams on a frigid New England street;
Sasha Lane plunging into a Texas swimming pool; Don Cheadle as Miles
Davis suffering the indignity of a visit to a Columbia University
dormitory in the midst of an absurd caper --- I think that the right
name for what I've witnessed is bravery.

\textbf{MORRIS:} Ah, that word! \emph{Bravery}. It's so easily overused
and misapplied. Bravery and fearlessness should be in the actor's job
description: to risk exposure, vulnerability, failure in an attempt to
achieve what our Method-trained friends would call ``truth'' and David
Mamet might call ``what I command.'' But obviously, bravery in screen
acting exists. Any woman who takes an assignment with Lars von Trier,
for example --- Emily Watson, Björk, Nicole Kidman, Bryce Dallas Howard,
Kirsten Dunst and especially Charlotte Gainsbourg --- embodies the
concept to a degree that abuts holiness and would alarm H.R.

Right now, given how few screen actors are expected --- or hired --- to
achieve anything close, it's a trait that feels increasingly endangered.
But what is it? A result of a seemingly hazardous workplace? The choice
to forgo likability in favor of being fully human? A willingness to push
yourself, regardless of the outcome, to some new expressive frontier?
Sasha Lane exceeds these definitions --- while working with Shia
LaBeouf, too, a terribly underrated performer whose self-regard is
crying out for a director to toss him a book of Nicolas Cage's matches.
I actually think the scene in which Taraji P. Henson explodes at her
co-workers in
``\href{https://www.nytimes3xbfgragh.onion/2016/05/22/movies/taraji-p-henson-octavia-spencer-hidden-figures-rocket-science-and-race.html}{Hidden
Figures}'' is more brave of the \emph{character} --- Katherine Johnson,
a black female mathematician working among scores of white men in the
early 1960s --- than of the woman playing her. That's a moment of
cathartic acting, of roaring release. She delivers that speech the way a
battleship delivers a torpedo. And accordingly, it does blow a hole in
the rest of a smartly done, crowd-pleasing movie, because you don't
forget that in addition to the space race, there's this \emph{other}
race thing going on. But what I found brave about the rest of her
performance is how demure she has allowed herself to be. Katherine seems
as surprised by her outburst as we and those men are. Until then, she's
playing everything \emph{but} anger.

Image

Kristen Stewart in ``Cafe Society'' (with Jesse
Eisenberg).Credit...Sabrina Lantos/Gravier Productions Inc.

\textbf{SCOTT:} There is an element of self-awareness, just this side of
winking, in the way Henson holds herself in check. At the moment, she is
known to all the sentient television-watching world as Cookie Lyon, a
volcano of rage, greed, maternal possessiveness and sexual need. One
fantasy that ``Hidden Figures'' invites is that the nerdy, demure,
slightly disheveled Katherine will respond to the slights and insults
thrown her way by Kirsten Dunst's character and the rest of the white
power structure by unleashing her inner Cookie. Of course the
stereotypical view of black womanhood embedded in that fantasy is part
of what ``Hidden Figures'' is attacking, and Henson is central to that
attack. Her self-control mirrors Katherine's self-control, but Henson is
also letting us know that she has always been in control. I would call
that assertion a form of bravery.

\textbf{MORRIS:} Truly brave, of course, is all some actors ever do.
That's all Isabelle Huppert ever does. And you'd think it would get old
watching her go out on yet another limb. But Huppert is why I'll never
stop going to the movies: She's all limbs. In
``\href{https://www.nytimes3xbfgragh.onion/2016/11/11/movies/elle-review-isabelle-huppert.html}{Elle},''
she plays a clammy video-game executive who is haunted by a sexual
assault. The appalling audacity of the movie, which Paul Verhoeven
directed, is its refusal to draw a line between rape and fetish. But the
movie is as much about Verhoeven's toying with propriety as it is about
an actress toying with herself. Her film persona is bound up in the
psychodynamics of sex and power. This might be her breeziest-yet
approach. And yet there are so many layers to what she's doing here:
repression, lust, shame, disgust, fun. I've never seen an actor so good
at and so seemingly obsessed with performing instinct, subtext and
offensiveness. That might get aggravating with someone else, because,
ultimately, it's meta-humanity she's after: Even when the women she's
playing just live in Paris, they never seem entirely from Earth. But you
always want to see how too-far she'll take things.

Image

Denzel Washington and Viola Davis in ``Fences.''Credit...David
Lee/Paramount Pictures

\textbf{SCOTT:} Even when she stays close to home, as in the finely
observed domestic drama of Mia Hansen-Love's ``Things to Come,'' she's
playing almost a doppelgänger of the ``Elle'' character --- a philosophy
professor whose marriage breaks up, who becomes a grandmother and whose
sexual interests point in what others might regard as an inappropriate
direction. She's also kind of mean, or at least shockingly direct. But
there's no violence in Hansen-Love's world, no sensationalism, nothing
lurid. Everything in tasteful Gallic moderation, except for Huppert, who
is a walking avatar of extremity. She collapses the distinction between
terror and rapture.

\textbf{MORRIS:} Huppert is among the last of a dying breed of
psychological star. That kind of acting has tended to be closely
associated with the Europeans and the Method people, but is it nuts to
watch Kristen Stewart work and think: She could be Huppert's daughter?

Image

Taraji P. Henson in ``Hidden Figures.''Credit...20th Century Fox

\textbf{SCOTT:} No more nuts than my own hunch, which is that
\href{https://www.nytimes3xbfgragh.onion/2016/08/17/t-magazine/entertainment/kristen-stewart-the-good-bad-girl.html}{Kristen
Stewar}t is the new Robert De Niro. Back in the '70s and '80s, De Niro's
reputation as the best actor in American movies rested on his ability to
vanish completely into each role, to effect a physical and psychological
transformation so total that you could barely recognize him from one
movie to the next. Some of what he did was a matter of what you might
call technical extremism: learning Sicilian dialect for ``The Godfather
Part II,'' pushing his body from sinewy fighting trim to has-been
bloatedness in ``Raging Bull.'' Stewart hasn't quite done that yet, but
she burrows as deeply as De Niro ever has into the interiors of her
characters, arranging her expressions, her carriage, her vocal
inflections --- even, it can seem, her height and bone structure ---
accordingly.

You could say that, having been made, perhaps reluctantly, into a movie
star by the ``Twilight'' movies, she has lately reinvented herself as
the character actor she might have always preferred to be. Apart from
her lead performance in Olivier Assayas's ``Personal Shopper,'' she has
been an ensemble player in 2016, with roles in Woody Allen's ``Cafe
Society,''
\href{https://www.nytimes3xbfgragh.onion/2016/10/16/magazine/the-quiet-menace-of-kelly-reichardts-feminist-westerns.html}{Kelly
Reichardt's ``Certain Women''} and Ang Lee's ``Billy Lynn's Long
Halftime Walk.'' But the fact that she's the most interesting person in
all of those movies suggests that her movie-star charisma is still
intact. She's just using it in subtle and occasionally subversive ways.

Image

Emma Stone in ``La La Land'' (with Ryan Gosling).Credit...Dale
Robinette/Lionsgate

\textbf{MORRIS:} I think Kristen Stewart is just about the best American
movie actress we have. Her bad romance with movie stardom has served her
well, because early exposure to its toxins might have fortified her
resistance to mere fame. Unlike with, say, Ben Affleck, there's no
tension or ambivalence between her being an actor and her being a star.
She appears to have rejected the latter to insist upon the value of the
former. Lots of people can have it both ways, but it's a balance that
takes a while to achieve. Look at how long it took for Meryl Streep and
Leonardo DiCaprio, whose allergy to overnight godliness foreshadowed
Stewart's. In the meantime, it's fascinating to watch her flirt with
stardom in the films she takes and the women she plays.

\textbf{SCOTT:} In
``\href{https://www.nytimes3xbfgragh.onion/2016/07/15/movies/cafe-society-review-woody-allen.html}{Cafe
Society},'' the Woody Allen movie, she takes what is, as written, an
almost entirely functional character --- the dream girl swooned over by
both a middle-aged Hollywood mogul and his ambitious nephew; a catalyst
of male desire and a mirror of masculine ego --- and makes her into the
only person in the film whose choices and desires really matter. In
``Certain Women,'' a much better movie, she slouches onto the screen
with self-effacing diffidence. You may wonder if Elizabeth Travis, a
young lawyer trying to earn some extra money teaching adult-ed classes
to disgruntled teachers in a middle-of-nowhere Western town, is in
possession of a backbone. Her posture is terrible. Her fashion sense is
worse. She seems entirely capable of standing in front of a room full of
people and vanishing from sight.

Except to a young ranch hand (Lily Gladstone), in whose eyes Elizabeth
is a dazzling, almost magical creature, the most intoxicating and
glamorous person she has ever encountered --- a dangerous and alluring
Edward Cullen to her own humble Bella Swan. But there is no winking from
Stewart herself, and none of the kind of ostentatious deglamorization
that stars sometimes traffic in when they are shopping for Academy
hardware. If this is realism, it's the kind that forces you to
acknowledge the gaps and blurry spaces in your previous conception of
reality.

Not that realism is everything. Even the drabbest, most earthbound
cinematic exploration of ordinary unhappiness is on some level a
fantasy.
``\href{https://www.nytimes3xbfgragh.onion/2016/11/06/movies/la-la-land-stars-ryan-gosling-emma-stone-and-los-angeles.html}{La
La Land},'' Damien Chazelle's brazen attempt to reinvigorate the
old-fashioned movie musical, is a fantasy on a lot of different levels.
One is Emma Stone's performance as a fresh-faced, quick-witted Los
Angeles girl next door who happens to be an aspiring movie star.

\textbf{MORRIS:} You and I are both guilty of pleading with the people
who make our movies to please, pretty please, put Taraji P. Henson and
Emma Stone in more of everything. And here they both are. It took only a
few moments in a couple of movies to realize that the women were stars.
They've got gale-force charisma that risks neglect and misuse. With
Henson, I worried that the movies would throw up their hands in
pre-emptive exasperation: ``We just don't the have the roles for her.''
And since her first Oscar nomination as Brad Pitt's mammy seven years
ago, they really haven't. With Stone, I worried that the only times we
would see her best stuff would be at awards shows and when she played
Peter Parker's damsel in a dreary Spider-Man reboot.

But ``La La Land'' is an ideal home for her sunlight and the clouds she
can gather to obscure it. Everything she's doing for Chazelle is on the
surface, which is just right for a movie that needs her to start singing
and dancing. You have to seem ready to join the joy --- or, as is often
the case here, the solemnity --- of having a song in your heart. One
reason to pick Stone over a lot of other actors for that kind of
readiness is that her intelligence doesn't get in her way. She has done
all her thinking before we see her. How many good actors have a hard
time giving themselves over to anything too small or normal, lest
someone miss them acting? Or: How many actors have to work hard at
appearing natural? Stone woke up like this. May the movies stay awake to
her!

\textbf{SCOTT:} My favorite Emma Stone moment in ``La La Land'' --- and
this is saying a lot --- is a very small morsel of multidimensional
bravura. Her character, Mia, is at an audition in the middle of a
stressed-out, ordinary, young-person-in-Los-Angeles-looking-for-a-break
kind of day. She's supposed to react as someone breaks up with her over
the phone, and Mia --- Emma --- gives it everything. Her eyes swell with
unshed tears; her mouth crumples and freezes into a pained half-smile.
And then someone interrupts, and the audition is over, and she makes her
exasperated way down a corridor lined with pretty, petite redheads just
like her. But we know she's not just another interchangeable aspirant.
We know, based on that audition, that Mia is a terrific actress. We
already knew that about Emma Stone, but now somehow we know it even
more.

\textbf{MORRIS:} ``La La Land'' arrives at a moment in which the movies
are diverging from Stardom 101: ``This is who I am, and it won't ever
change too much.'' Both she and her co-star Ryan Gosling --- who's also
excellent, and appears to have recovered from that give-me-the-matches
syndrome --- are working on established personas, in a classical style
that can result in serious work. (Ask Denzel Washington and Isabelle
Huppert.) But ever since De Niro, I think seriousness in American acting
has come to mean suffering and stress. His Jake LaMotta is a landmark
piece of acting from which people continue to extract the wrong lesson.
The achievement of that performance isn't its physicality --- it's the
psychological torture that defines and dictates it. (I have to say, the
toll of playing that character seemed to wreck his acting for a decade.)
That kind of madness might need the Method. But it has landed a lot of
performers in the pits of mannerism too. Natalie Portman has been one of
those actors: someone who needs the suffering to achieve the sublime.
Her playing
\href{https://www.nytimes3xbfgragh.onion/2016/11/30/fashion/jackie-kennedy-first-lady-natalie-portman.html}{Jacqueline
Kennedy} in Pablo Larraín's ``Jackie'' could have been more stunt work
--- the accent, the brittleness, the froideur. But I really like what an
exercise in state of mind she makes it. She's acting symbolism and
tragedy in a way only partly to do with her husband's assassination and
everything to do with performing docility, sympathy, wifeliness,
decorum, strength and bereavement for an expectant nation. Portman knows
what to do with a symbol: make it mean something.

She and Don Cheadle are the only actors in our group playing outsize
cultural icons. That kind of acting is a gamble, obviously, because it
can get too close to impersonation. We tend to value it nonetheless
because we have a historical record by which to measure it. So my bias
is always toward something completely new that has come from the inside
out, something that invents more than reinvents. But I'm into certain
reinventions too. Plain old stardom --- the way Stone pursues it in ``La
La Land'' --- isn't something we value as much in movies anymore. What
we value now is visible work. One of the many things I love about all
the fantastic film acting being done now is that some of these great
performers understand how invisible works, too.

Advertisement

\protect\hyperlink{after-bottom}{Continue reading the main story}

\hypertarget{site-index}{%
\subsection{Site Index}\label{site-index}}

\hypertarget{site-information-navigation}{%
\subsection{Site Information
Navigation}\label{site-information-navigation}}

\begin{itemize}
\tightlist
\item
  \href{https://help.nytimes3xbfgragh.onion/hc/en-us/articles/115014792127-Copyright-notice}{©~2020~The
  New York Times Company}
\end{itemize}

\begin{itemize}
\tightlist
\item
  \href{https://www.nytco.com/}{NYTCo}
\item
  \href{https://help.nytimes3xbfgragh.onion/hc/en-us/articles/115015385887-Contact-Us}{Contact
  Us}
\item
  \href{https://www.nytco.com/careers/}{Work with us}
\item
  \href{https://nytmediakit.com/}{Advertise}
\item
  \href{http://www.tbrandstudio.com/}{T Brand Studio}
\item
  \href{https://www.nytimes3xbfgragh.onion/privacy/cookie-policy\#how-do-i-manage-trackers}{Your
  Ad Choices}
\item
  \href{https://www.nytimes3xbfgragh.onion/privacy}{Privacy}
\item
  \href{https://help.nytimes3xbfgragh.onion/hc/en-us/articles/115014893428-Terms-of-service}{Terms
  of Service}
\item
  \href{https://help.nytimes3xbfgragh.onion/hc/en-us/articles/115014893968-Terms-of-sale}{Terms
  of Sale}
\item
  \href{https://spiderbites.nytimes3xbfgragh.onion}{Site Map}
\item
  \href{https://help.nytimes3xbfgragh.onion/hc/en-us}{Help}
\item
  \href{https://www.nytimes3xbfgragh.onion/subscription?campaignId=37WXW}{Subscriptions}
\end{itemize}
