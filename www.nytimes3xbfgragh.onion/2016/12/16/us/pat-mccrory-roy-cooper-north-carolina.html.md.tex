Sections

SEARCH

\protect\hyperlink{site-content}{Skip to
content}\protect\hyperlink{site-index}{Skip to site index}

\href{https://www.nytimes3xbfgragh.onion/section/us}{U.S.}

\href{https://myaccount.nytimes3xbfgragh.onion/auth/login?response_type=cookie\&client_id=vi}{}

\href{https://www.nytimes3xbfgragh.onion/section/todayspaper}{Today's
Paper}

\href{/section/us}{U.S.}\textbar{}North Carolina Governor Signs Law
Limiting Successor's Power

\url{https://nyti.ms/2hPYhP5}

\begin{itemize}
\item
\item
\item
\item
\item
\end{itemize}

Advertisement

\protect\hyperlink{after-top}{Continue reading the main story}

Supported by

\protect\hyperlink{after-sponsor}{Continue reading the main story}

\hypertarget{north-carolina-governor-signs-law-limiting-successors-power}{%
\section{North Carolina Governor Signs Law Limiting Successor's
Power}\label{north-carolina-governor-signs-law-limiting-successors-power}}

\includegraphics{https://static01.graylady3jvrrxbe.onion/images/2016/12/17/us/17carolina1/17carolina1-articleLarge.jpg?quality=75\&auto=webp\&disable=upscale}

By \href{https://www.nytimes3xbfgragh.onion/by/richard-fausset}{Richard
Fausset}

\begin{itemize}
\item
  Dec. 16, 2016
\item
  \begin{itemize}
  \item
  \item
  \item
  \item
  \item
  \end{itemize}
\end{itemize}

RALEIGH, N.C. --- Amid a tense and dramatic backdrop of outrage and
frustration, North Carolina's Republican-controlled legislature on
Friday approved a sweeping package of restrictions on the power of the
governor's office in advance of the swearing in of the Democratic
governor-elect, Roy Cooper.

Protesters spent a second day chanting and disrupting debate, as some
were arrested and led away from the state legislative building in
plastic wrist restraints.

Democratic lawmakers repeatedly referred to the move as a ``power grab''
carried out by a Republican Party upset that their candidate, Gov. Pat
McCrory, had lost the governor's race. Republicans countered by
emphasizing that they had suffered similar indignities for many decades
when Democrats controlled the legislature here.

State Senator Chad Barefoot, a Republican, said that the changes return
``power that was grabbed during Democratic administrations in the 1990s,
and some in the '70s.''

But some here said that Republicans' effort to hobble the incoming
governor had few parallels in recent North Carolina history.

``Sure, the Democrats don't have clean hands, but this is beyond
anything I've seen them do,'' said Bob Phillips, executive director of
the nonpartisan group Common Cause North Carolina. ``I think we're in
unprecedented, uncharted territory with this.''

Two major bills were approved by the legislature Friday. One of them,
which was quickly signed by the departing Gov. Pat McCrory, a
Republican, strips future governors of their power to appoint a majority
to the State Board of Elections. The number of board members was
expanded from five to eight, with the eight members to be evenly divided
between the two major parties.

It also changes the state court system, making it more difficult for the
losers of some superior court cases to appeal directly to the
Democratic-controlled Supreme Court.

A second bill, which had not been signed by the governor as of Friday
afternoon, strips the governor of his ability to name members of the
boards of state universities, and it reduces the number of state
employees the governor can appoint from 1,500 to 425.

Republicans, who once expanded the number of employees who serve at the
governor's pleasure in an effort to help Mr. McCrory, originally
proposed shrinking the number of such workers to 300 in advance of Mr.
Cooper's inauguration. The number was increased in an amendment filed by
Mr. Barefoot.

In another change, and one that could have the greatest impact in the
near term, the bill makes the governor's cabinet appointees subject to
approval by the State Senate. Republicans currently enjoy veto-proof
majorities in the House and the Senate, and the North Carolina
governorship is historically a relatively weak office. Cabinet
appointments are one of the major ways Mr. Cooper, a moderate Democrat,
might be able to influence the direction of the state.

The moves have mobilized North Carolina's sizable Democratic contingent,
who have been galvanized, over the last four years, by the ``Moral
Mondays'' protest movement led by the Rev. William Barber II, the
charismatic president of the state N.A.A.C.P.

On Friday afternoon, Mr. Barber entered the state legislative building,
triggering whoops and cheers from the roughly 200 protesters.

With the aid of a cane he made his way into the space outside the
legislative chambers, and encouraged the protesters to risk arrest by
knocking on the locked doors of the State House viewing gallery. Police
had forbidden protesters to knock.

``You have to decide if you want to in fact knock on that door,'' said
Mr. Barber.

Some knocked. A number of them were arrested. The building throbbed as
the protesters chanted, ``Let Us In.''

The raucous protests Friday, and the votes along strict party lines,
virtually guarantee that hyperpartisan political turmoil will continue
to be the norm in this deeply divided state. Democratic protests began
to swell here in 2013, after Mr. McCrory took office, and Republicans,
enjoying control of both the executive and legislative branches, began
rolling out an aggressive conservative agenda that limited ballot access
and, with the passage of the legislation known as House Bill 2,
curtailed gay and transgender rights.

That law, which set off boycotts and nationwide protests, is seen as a
main reason Mr. McCrory lost his bid for a second term, despite the fact
that he presided over an improving economy. Mr. McCrory further angered
Democrats by
\href{https://www.nytimes3xbfgragh.onion/2016/11/29/us/north-carolina-governor-race.html}{refusing
to concede the election} for nearly a month as his allies filed
challenges to the election results.

In a news conference Thursday, Mr. Cooper, the state's veteran attorney
general, threatened to sue the legislature over the changes. ``Once
more, the courts will have to clean up the mess the legislature made,
but it won't stop us from moving North Carolina forward,'' he said in a
statement on Friday.

Richard L. Hasen, a voting-rights expert and professor of law and
political science at the University of California, Irvine, has said that
the changes to the elections board could be challenged in state and
federal court. In a \href{https://electionlawblog.org/?p=89974}{blog
post}, Mr. Hasen wrote that a federal case might allege violations of
the Voting Rights Act, ``in part because the legislature would
potentially be diluting minority voting power and making minority voters
worse off, just at the time that their candidate of choice (Gov. Cooper)
is poised to assume power.''

Friday's debates at times found Republicans arguing that they could make
the changes, and Democrats questioning whether they should. Mr.
Barefoot, on the Senate floor, argued forcefully that a number of the
changes were well within the Republicans' legal rights, citing specific
passages from the state Constitution.

State Senator Joyce Waddell, a Democrat from Mecklenburg County, echoed
many other Democrats when she complained that the matters were being
decided in a hastily called special session that allowed less
opportunity for public input than normal.

``Why are we rushing?'' she said. ``There's no time to hear from
voters.''

Republicans defended some of the moves as good-government reform
efforts. Representative David R. Lewis, argued that the provision
changing the elections boards will instill greater confidence in the
electoral process. In addition to changing the state board, the law will
also make the state's 100 county elections boards four-member bodies
evenly split between the two parties. Before, they were set up so that
the governor's party had a 2-1 majority in each.

``We have been told that one of the most important things is for our
citizenry to have confidence and faith that the elections process is
fair,'' Mr. Lewis said, ``and that it is overseen in a way that does not
reflect the partisan bent, if you will, of those administering the
elections.''

Advertisement

\protect\hyperlink{after-bottom}{Continue reading the main story}

\hypertarget{site-index}{%
\subsection{Site Index}\label{site-index}}

\hypertarget{site-information-navigation}{%
\subsection{Site Information
Navigation}\label{site-information-navigation}}

\begin{itemize}
\tightlist
\item
  \href{https://help.nytimes3xbfgragh.onion/hc/en-us/articles/115014792127-Copyright-notice}{©~2020~The
  New York Times Company}
\end{itemize}

\begin{itemize}
\tightlist
\item
  \href{https://www.nytco.com/}{NYTCo}
\item
  \href{https://help.nytimes3xbfgragh.onion/hc/en-us/articles/115015385887-Contact-Us}{Contact
  Us}
\item
  \href{https://www.nytco.com/careers/}{Work with us}
\item
  \href{https://nytmediakit.com/}{Advertise}
\item
  \href{http://www.tbrandstudio.com/}{T Brand Studio}
\item
  \href{https://www.nytimes3xbfgragh.onion/privacy/cookie-policy\#how-do-i-manage-trackers}{Your
  Ad Choices}
\item
  \href{https://www.nytimes3xbfgragh.onion/privacy}{Privacy}
\item
  \href{https://help.nytimes3xbfgragh.onion/hc/en-us/articles/115014893428-Terms-of-service}{Terms
  of Service}
\item
  \href{https://help.nytimes3xbfgragh.onion/hc/en-us/articles/115014893968-Terms-of-sale}{Terms
  of Sale}
\item
  \href{https://spiderbites.nytimes3xbfgragh.onion}{Site Map}
\item
  \href{https://help.nytimes3xbfgragh.onion/hc/en-us}{Help}
\item
  \href{https://www.nytimes3xbfgragh.onion/subscription?campaignId=37WXW}{Subscriptions}
\end{itemize}
