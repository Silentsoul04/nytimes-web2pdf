Sections

SEARCH

\protect\hyperlink{site-content}{Skip to
content}\protect\hyperlink{site-index}{Skip to site index}

\href{https://www.nytimes3xbfgragh.onion/section/world/asia}{Asia
Pacific}

\href{https://myaccount.nytimes3xbfgragh.onion/auth/login?response_type=cookie\&client_id=vi}{}

\href{https://www.nytimes3xbfgragh.onion/section/todayspaper}{Today's
Paper}

\href{/section/world/asia}{Asia Pacific}\textbar{}As China Seeks Bigger
Role on World Stage, Xi Jinping Will Go to Davos World Economic Forum

\url{https://nyti.ms/2ih5eaT}

\begin{itemize}
\item
\item
\item
\item
\item
\end{itemize}

Advertisement

\protect\hyperlink{after-top}{Continue reading the main story}

Supported by

\protect\hyperlink{after-sponsor}{Continue reading the main story}

\hypertarget{as-china-seeks-bigger-role-on-world-stage-xi-jinping-will-go-to-davos-world-economic-forum}{%
\section{As China Seeks Bigger Role on World Stage, Xi Jinping Will Go
to Davos World Economic
Forum}\label{as-china-seeks-bigger-role-on-world-stage-xi-jinping-will-go-to-davos-world-economic-forum}}

\includegraphics{https://static01.graylady3jvrrxbe.onion/images/2017/01/11/world/11davos1/11davos1-articleLarge.jpg?quality=75\&auto=webp\&disable=upscale}

By \href{http://www.nytimes3xbfgragh.onion/by/edward-wong}{Edward Wong}

\begin{itemize}
\item
  Jan. 10, 2017
\item
  \begin{itemize}
  \item
  \item
  \item
  \item
  \item
  \end{itemize}
\end{itemize}

\href{http://cn.nytimes3xbfgragh.onion/china/20170111/davos-china-xi-jinping-trump/}{阅读简体中文版}

President
\href{https://www.nytimes3xbfgragh.onion/topic/person/xi-jinping?8qa}{Xi
Jinping} of
\href{https://www.nytimes3xbfgragh.onion/topic/destination/china?inline=nyt-geo}{China}
plans to stride into the snowy high-altitude conclave of the world's
financial elite next week, attending the
\href{https://www.weforum.org/agenda/2017/01/how-to-follow-davos-2017}{World
Economic Forum} at Davos, Switzerland, the first time a top Chinese
leader will put himself into the mix of political leaders and business
executives who view themselves as the masters of the global economy.

\href{http://www.fmprc.gov.cn/mfa_eng/xwfw_665399/s2510_665401/2511_665403/t1429431.shtml}{His
participation was announced} by the Chinese Foreign Ministry on Tuesday.
It is the latest, and in some ways the boldest, Chinese attempt to
compete with the United States' dominant position in world economic and
strategic institutions, a decades-long campaign that has been carried
out everywhere from the conference rooms of Asian central banks to the
waters of the South China Sea to the halls of the United Nations in New
York.

But it is unclear whether Mr. Xi, who rarely ventures beyond platitudes
in discussing the strategies of the world's second largest economy, can
take advantage of this time of transition when the rest of the world is
gauging whether the United States is pulling back from global
leadership.

Mr. Xi may sense an opening during a historic inflection point. He plans
to deliver a speech at Davos at a moment when the incoming United States
president, Donald J. Trump, has suggested that the United States should
withdraw from the traditional superpower role it has played since World
War II, including its leadership of a global free trade agenda.

In recent years, Davos had come to embody that American-led agenda --- a
gathering at an Alpine ski resort of some of the world's most powerful
figures in the realms of politics, media and technology. Discussions
each year have been set by Western leaders, not Asian ones.

But events of the past year brought into sharp focus a rise in populist
denunciations of globalization, free trade and inequality in some
Western nations, including the United States, with Davos frequently
mentioned by critics as a symbol of the root causes of their countries'
ailments.

China has benefited greatly from access to international trade markets
since its entry in 2001 into the World Trade Organization, and it could
now become the most vocal proponent of that system. Mr. Trump campaigned
on opposition to the existing global trade system and has denounced
China for competing unfairly against the United States.

At Davos, Mr. Xi plans to lead a delegation of senior officials, China's
wealthiest entrepreneurs and top executives of state-owned enterprises,
including the \href{http://www.poly.com.cn/english/1627.html}{China Poly
Group Corporation}, which has ties to the Chinese military.

People's Daily, the official Communist Party newspaper, published an
article Wednesday that said China could become the ``torchbearer of the
open trade system'' and boasted that Mr. Xi's visit ``will boost the
world's confidence in global governance.''

His appearance is a logical step in his country's evolution into a
globe-spanning superpower, a rapid transformation that has been marked
by bold symbolic gestures and events in the past decade, including the
2008 Summer Olympics in Beijing under President Hu Jintao, Mr. Xi's
predecessor.

Mr. Xi has carried on that theme with much more aggressive actions,
including overseeing construction of military infrastructure in the
\href{https://www.nytimes3xbfgragh.onion/2016/12/27/world/asia/south-china-sea-trump.html}{South
China Sea's contested waters} and establishing a
\href{https://www.nytimes3xbfgragh.onion/2015/12/05/business/international/china-creates-an-asian-bank-as-the-us-stands-aloof.html}{regional
lending bank} opposed by the United States.

Premiers of China, including the current one, Li Keqiang, have attended
Davos before, but the nation's president --- and head of the Communist
Party --- has never been to the gathering.

``Clearly it signals that Xi Jinping is now interested in writing both
China and himself in a grander way on the global diplomatic horizon,''
said Orville Schell, director of the Center on U.S.-China Relations at
the Asia Society. ``He feels it's time to really come out. Behind that
probably is an assumption and wishful thinking that the U.S. is in
disarray, Europe is feckless, and so on.''

``He'll be received almost as the number one citizen at Davos,'' Mr.
Schell added.

Mr. Xi plans to attend Davos on Jan. 17, during a state visit to
Switzerland from Jan. 15 to 18, said Lu Kang, a Foreign Ministry
spokesman, at a regularly scheduled news conference in Beijing on
Tuesday. Mr. Xi is expected to speak at the opening session of the
forum, which runs from Jan. 17 to 20.

Strong anti-globalization sentiments erupted last year in the movement
in Britain that culminated in the popular vote by British citizens to
leave the European Union. But it was Mr. Trump's election in November
that was the apotheosis of the move in the West toward isolationism, and
meetings and conversations at Davos --- whose theme this year is
``responsive and responsible leadership'' --- will take place in the
shadow of Mr. Trump's campaign promises and rhetoric.

``It's going to be very tempting for China to imagine for itself that
it's gained much more status after this election in a whole array of
global endeavors, including trade and
\href{https://www.nytimes3xbfgragh.onion/2017/01/10/world/asia/china-wants-to-be-a-climate-change-watchdog-but-cant-yet-lead-by-example.html}{on
climate change} and possibly other issues,'' Mr. Schell said. ``If the
U.S. is going to absent itself more --- and we don't know if that's the
case yet --- nature does abhor a vacuum. When a father grows old, the
son is sometimes able to fill the space.''

Victor Shih, a scholar of China's political economy at the University of
California, San Diego, said there are actually some global agenda
matters on which China's influence ``might have peaked in the medium
term.''

He pointed to China's push in recent years to have the renminbi counted
as an international currency. That has been undermined in the past year
by efforts of the Chinese central bank, the People's Bank of China, to
withdraw large quantities of offshore renminbi from circulation. The
bank has been doing that in order to try to prop up the value of the
renminbi and limit capital flight from China, including the transfer of
money to Hong Kong by wealthy Chinese.

``The People's Bank of China continues to claim that renminbi
internationalization is important, and of course, at Davos, President Xi
may continue to pay verbal homage to that agenda because it would be an
important sign of China's ascendance on the world stage,'' Mr. Shih
said. ``Yet, in the past year, we have seen renminbi deposits outside of
mainland China decline by hundreds of billions of renminbi.''

Among the Chinese tycoons expected to attend Davos are
\href{https://www.nytimes3xbfgragh.onion/2015/04/29/world/asia/wang-jianlin-abillionaire-at-the-intersection-of-business-and-power-in-china.html}{Wang
Jianlin}, founder of the Wanda Group, a property and cinema company, and
Jack Ma, founder of the Alibaba Group, China's biggest e-commerce
company.

They are commonly referred to as the two wealthiest men in China. Both
are investing in the United States, and Mr. Ma
\href{http://www.reuters.com/article/us-usa-trump-alibaba-idUSKBN14T1ZA}{met
with Mr. Trump on Monday} at Trump Tower in New York to discuss business
opportunities between companies in the two countries. (Within Mr.
Trump's family, his son-in-law, Jared Kushner, has been involved in
\href{https://www.nytimes3xbfgragh.onion/2017/01/07/us/politics/jared-kushner-trump-business.html}{personal
business negotiations} with top Chinese executives.)

Some members of the Congress, mostly Republicans, have called for
\href{https://www.nytimes3xbfgragh.onion/2016/10/06/world/asia/china-congress-media-fdi.html}{greater
scrutiny of Chinese investments} in the United States, including those
by Wanda, which recently bought the AMC movie theater chain and
Legendary Entertainment, a film production and financing company.

Advertisement

\protect\hyperlink{after-bottom}{Continue reading the main story}

\hypertarget{site-index}{%
\subsection{Site Index}\label{site-index}}

\hypertarget{site-information-navigation}{%
\subsection{Site Information
Navigation}\label{site-information-navigation}}

\begin{itemize}
\tightlist
\item
  \href{https://help.nytimes3xbfgragh.onion/hc/en-us/articles/115014792127-Copyright-notice}{©~2020~The
  New York Times Company}
\end{itemize}

\begin{itemize}
\tightlist
\item
  \href{https://www.nytco.com/}{NYTCo}
\item
  \href{https://help.nytimes3xbfgragh.onion/hc/en-us/articles/115015385887-Contact-Us}{Contact
  Us}
\item
  \href{https://www.nytco.com/careers/}{Work with us}
\item
  \href{https://nytmediakit.com/}{Advertise}
\item
  \href{http://www.tbrandstudio.com/}{T Brand Studio}
\item
  \href{https://www.nytimes3xbfgragh.onion/privacy/cookie-policy\#how-do-i-manage-trackers}{Your
  Ad Choices}
\item
  \href{https://www.nytimes3xbfgragh.onion/privacy}{Privacy}
\item
  \href{https://help.nytimes3xbfgragh.onion/hc/en-us/articles/115014893428-Terms-of-service}{Terms
  of Service}
\item
  \href{https://help.nytimes3xbfgragh.onion/hc/en-us/articles/115014893968-Terms-of-sale}{Terms
  of Sale}
\item
  \href{https://spiderbites.nytimes3xbfgragh.onion}{Site Map}
\item
  \href{https://help.nytimes3xbfgragh.onion/hc/en-us}{Help}
\item
  \href{https://www.nytimes3xbfgragh.onion/subscription?campaignId=37WXW}{Subscriptions}
\end{itemize}
