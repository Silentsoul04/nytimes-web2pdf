Sections

SEARCH

\protect\hyperlink{site-content}{Skip to
content}\protect\hyperlink{site-index}{Skip to site index}

\href{https://www.nytimes3xbfgragh.onion/section/books}{Books}

\href{https://myaccount.nytimes3xbfgragh.onion/auth/login?response_type=cookie\&client_id=vi}{}

\href{https://www.nytimes3xbfgragh.onion/section/todayspaper}{Today's
Paper}

\href{/section/books}{Books}\textbar{}Obama's Secret to Surviving the
White House Years: Books

\url{https://nyti.ms/2iClKCR}

\begin{itemize}
\item
\item
\item
\item
\item
\item
\end{itemize}

Advertisement

\protect\hyperlink{after-top}{Continue reading the main story}

Supported by

\protect\hyperlink{after-sponsor}{Continue reading the main story}

Critic's Notebook

\hypertarget{obamas-secret-to-surviving-the-white-house-years-books}{%
\section{Obama's Secret to Surviving the White House Years:
Books}\label{obamas-secret-to-surviving-the-white-house-years-books}}

\includegraphics{https://static01.graylady3jvrrxbe.onion/images/2017/01/16/us/16OBAMA-BOOKS-01/16OBAMA-BOOKS-01-articleInline-v2.jpg?quality=75\&auto=webp\&disable=upscale}

By \href{http://www.nytimes3xbfgragh.onion/by/michiko-kakutani}{Michiko
Kakutani}

\begin{itemize}
\item
  Jan. 16, 2017
\item
  \begin{itemize}
  \item
  \item
  \item
  \item
  \item
  \item
  \end{itemize}
\end{itemize}

\href{http://cn.nytstyle.com/international/20170117/obamas-secret-to-surviving-the-white-house-years-books/}{阅读简体中文版}\href{https://www.nytimes3xbfgragh.onion/es/2017/01/17/los-libros-que-ayudaron-a-barack-obama-en-la-casa-blanca}{Leer
en español}

Not since Lincoln has there been a president as fundamentally shaped ---
in his life, convictions and outlook on the world --- by reading and
writing as Barack Obama.

Last Friday, seven days before his departure from the White House, Mr.
Obama sat down in the Oval Office and talked about the indispensable
role that books have played during his presidency and throughout his
life --- from his peripatetic and sometimes lonely boyhood, when ``these
worlds that were portable'' provided companionship, to his youth when
they helped him to figure out who he was, what he thought and what was
important.

During his eight years in the White House --- in a noisy era of
information overload, extreme partisanship and knee-jerk reactions ---
books were a sustaining source of ideas and inspiration, and gave him a
renewed appreciation for the complexities and ambiguities of the human
condition.

``At a time when events move so quickly and so much information is
transmitted,'' he said, reading gave him the ability to occasionally
``slow down and get perspective'' and ``the ability to get in somebody
else's shoes.'' These two things, he added, ``have been invaluable to
me. Whether they've made me a better president I can't say. But what I
can say is that they have allowed me to sort of maintain my balance
during the course of eight years, because this is a place that comes at
you hard and fast and doesn't let up.''

{[}
\href{https://www.nytimes3xbfgragh.onion/2017/01/16/books/transcript-president-obama-on-what-books-mean-to-him.html}{Read
a transcript of the interview with Barack Obama} {]}

The writings of Lincoln, the Rev. Martin Luther King Jr., Gandhi and
Nelson Mandela, Mr. Obama found, were ``particularly helpful'' when
``what you wanted was a sense of solidarity,'' adding ``during very
difficult moments, this job can be very isolating.'' ``So sometimes you
have to sort of hop across history to find folks who have been similarly
feeling isolated, and that's been useful.'' There is a handwritten copy
of the Gettysburg Address in the Lincoln Bedroom, and sometimes, in the
evening, Mr. Obama says, he would wander over from his home office to
read it.

\includegraphics{https://static01.graylady3jvrrxbe.onion/images/2017/01/16/arts/16OBAMA-BOOKS3/16OBAMA-BOOKS3-articleLarge.jpg?quality=75\&auto=webp\&disable=upscale}

Like Lincoln, Mr. Obama taught himself how to write, and for him, too,
words became a way to define himself, and to communicate his ideas and
ideals to the world. In fact, there is a clear, shining line connecting
Lincoln and King, and President Obama. In speeches like the ones
delivered in
\href{https://www.nytimes3xbfgragh.onion/2015/07/04/arts/obamas-eulogy-which-found-its-place-in-history.html?_r=0}{Charleston}
and
\href{https://www.nytimes3xbfgragh.onion/2015/03/08/us/obama-in-selma-for-edmund-pettus-bridge-attack-anniversary.html}{Selma},
he has followed in their footsteps, putting his mastery of language in
the service of a sweeping historical vision, which, like theirs,
situates our current struggles with race and injustice in a historical
continuum that traces how far we've come and how far we have yet to go.
It's a vision of America as an unfinished project --- a continuing,
more-than-two-century journey to make the promises of the Declaration of
Independence real for everyone --- rooted both in Scripture and the
possibility of redemption, and a more existential belief that we can
continually remake ourselves. And it's a vision shared by the civil
rights movement, which overcame obstacle after obstacle, and persevered
in the face of daunting odds.

Mr. Obama's long view of history and the optimism (combined with a
stirring reminder of the hard work required by democracy) that he
articulated in his farewell speech last week are part of a hard-won
faith, grounded in his reading, in his knowledge of history (and its
unexpected zigs and zags), and his embrace of artists like Shakespeare
who saw the human situation entire: its follies, cruelties and mad
blunders, but also its resilience, decencies and acts of grace. The
playwright's tragedies, he says, have been ``foundational for me in
understanding how certain patterns repeat themselves and play themselves
out between human beings.''

\hypertarget{context-in-presidential-biographies}{%
\subsection{Context in Presidential
Biographies}\label{context-in-presidential-biographies}}

Presidential biographies also provided context, countering the tendency
to think ``that whatever's going on right now is uniquely disastrous or
amazing or difficult,'' he said. ``It just serves you well to think
about Roosevelt trying to navigate through World War II.''

Even books initially picked up as escape reading like the Hugo
Award-winning apocalyptic sci-fi epic
``\href{https://www.nytimes3xbfgragh.onion/2014/11/11/books/liu-cixins-the-three-body-problem-is-published-in-us.html}{The
Three-Body Problem}'' by the Chinese writer Liu Cixin, he said, could
unexpectedly put things in perspective: ``The scope of it was immense.
So that was fun to read, partly because my day-to-day problems with
Congress seem fairly petty --- not something to worry about. Aliens are
about to invade!''

In his searching 1995 book ``Dreams From My Father,'' Mr. Obama recalls
how reading was a crucial tool in sorting out what he believed, dating
back to his teenage years, when he immersed himself in works by Baldwin,
Ellison, Hughes, Wright, DuBois and Malcolm X in an effort ``to raise
myself to be a black man in America.'' Later, during his last two years
in college, he spent a focused period of deep self-reflection and study,
methodically reading philosophers from St. Augustine to Nietzsche,
Emerson to Sartre to Niebuhr, to strip down and test his own beliefs.

To this day, reading has remained an essential part of his daily life.
He recently gave his daughter Malia a Kindle filled with books he wanted
to share with her (including ``One Hundred Years of Solitude,'' ``The
Golden Notebook'' and ``The Woman Warrior''). And most every night in
the White House, he would read for an hour or so late at night ---
reading that was deep and ecumenical, ranging from contemporary literary
fiction (the last novel he read was Colson Whitehead's
``\href{https://www.nytimes3xbfgragh.onion/2016/08/03/books/review-the-underground-railroad-colson-whitehead.html?_r=0}{The
Underground Railroad}'') to classic novels to groundbreaking works of
nonfiction like Daniel Kahneman's ``Thinking, Fast and Slow'' and
Elizabeth Kolbert's
``\href{https://www.nytimes3xbfgragh.onion/2014/02/03/books/the-sixth-extinction-on-endangered-and-departed-species.html}{The
Sixth Extinction}.''

Image

``The Underground Railroad'' by Colson Whitehead.Credit...Patricia
Wall/The New York Times

Such books were a way for the president to shift mental gears from the
briefs and policy papers he studied during the day, a way ``to get out
of my own head,'' a way to escape the White House bubble. Some novels
helped him to better ``imagine what's going on in the lives of people''
across the country --- for instance, he found that Marilynne Robinson's
novels connected him emotionally to the people he was meeting in Iowa
during the 2008 campaign, and to his own grandparents, who were from the
Midwest, and the small town values of hard work and honesty and
humility.

Other novels served as a kind of foil --- something to argue with. V. S.
Naipaul's novel ``A Bend in the River,'' Mr. Obama recalls, ``starts
with the line `The world is what it is; men who are nothing, who allow
themselves to become nothing, have no place in it.' And I always think
about that line and I think about his novels when I'm thinking about the
hardness of the world sometimes, particularly in foreign policy, and I
resist and fight against sometimes that very cynical, more realistic
view of the world. And yet, there are times where it feels as if that
may be true.''

Writing was key to his thinking process, too: a tool for sorting through
``a lot of crosscurrents in my own life --- race, class, family. And I
genuinely believe that it was part of the way in which I was able to
integrate all these pieces of myself into something relatively whole.''

\hypertarget{a-writer-of-short-stories}{%
\subsection{A Writer of Short Stories}\label{a-writer-of-short-stories}}

Mr. Obama taught himself to write as a young man by keeping a journal
and writing short stories when he was a community organizer in Chicago
--- working on them after he came home from work and drawing upon the
stories of the people he met. Many of the tales were about older people,
and were informed by a sense of disappointment and loss: ``There is not
a lot of Jack Kerouac open-road, young kid on the make discovering
stuff,'' he says. ``It's more melancholy and reflective.''

That experience underscored the power of empathy. An outsider himself
--- with a father from Kenya, who left when he was 2, and a mother from
Kansas, who took him to live for a time in Indonesia --- he could relate
to many of the people he met in the churches and streets of Chicago, who
felt dislocated by change and isolation, and he took to heart his boss's
observation that ``the thing that brings people together to share the
courage to take action on behalf of their lives is not just that they
care about the same issues, it's that they have shared stories.''

This lesson would become a cornerstone of the president's vision of an
America where shared concerns --- simple dreams of a decent job, a
secure future for one's children --- might bridge differences and
divisions. After all, many people saw their own stories in his --- an
American story, as he said in his keynote address at the 2004 Democratic
National Convention possible ``in no other country on Earth.''

Image

President Obama reading ``Where the Wild Things Are'' to children at the
White House in 2014.Credit...Doug Mills/The New York Times

In today's polarized environment, where the internet has let people
increasingly retreat to their own silos (talking only to like-minded
folks, who amplify their certainties and biases), the president sees
novels and other art (like the musical ``Hamilton'') as providing a kind
of bridge that might span usual divides and ``a reminder of the truths
under the surface of what we argue about every day.''

He points out, for instance, that the fiction of Junot Díaz and Jhumpa
Lahiri speaks ``to a very particular contemporary immigration
experience,'' but at the same time tell stories about ``longing for this
better place but also feeling displaced'' --- a theme central to much of
American literature, and not unlike books by Philip Roth and Saul Bellow
that are ``steeped with this sense of being an outsider, longing to get
in, not sure what you're giving up.''

Mr. Obama entered office as a writer, and he will soon return to a
private life as a writer, planning to work on his memoirs, which will
draw on journals he's kept in the White House (``but not with the sort
of discipline that I would have hoped for''). He has a writer's
sensibility --- an ability to be in the moment while standing apart as
an observer, a novelist's eye and ear for detail, and a precise but
elastic voice capable of moving easily between the lyrical and the
vernacular and the profound.

He had lunch last week with five novelists he admires --- Dave Eggers,
Mr. Whitehead, Zadie Smith, Mr. Díaz and Barbara Kingsolver. He not only
talked with them about the political and media landscape, but also
talked shop, asking how their book tours were going and remarking that
he likes to write first drafts, long hand, on yellow legal pads.

Mr. Obama says he is hoping to eventually use his presidential center
website ``to widen the audience for good books'' --- something he's
already done with regular lists of book recommendations --- and then
encourage a public ``conversation about books.''

``At a time,'' he says, ``when so much of our politics is trying to
manage this clash of cultures brought about by globalization and
technology and migration, the role of stories to unify --- as opposed to
divide, to engage rather than to marginalize --- is more important than
ever.''

Advertisement

\protect\hyperlink{after-bottom}{Continue reading the main story}

\hypertarget{site-index}{%
\subsection{Site Index}\label{site-index}}

\hypertarget{site-information-navigation}{%
\subsection{Site Information
Navigation}\label{site-information-navigation}}

\begin{itemize}
\tightlist
\item
  \href{https://help.nytimes3xbfgragh.onion/hc/en-us/articles/115014792127-Copyright-notice}{©~2020~The
  New York Times Company}
\end{itemize}

\begin{itemize}
\tightlist
\item
  \href{https://www.nytco.com/}{NYTCo}
\item
  \href{https://help.nytimes3xbfgragh.onion/hc/en-us/articles/115015385887-Contact-Us}{Contact
  Us}
\item
  \href{https://www.nytco.com/careers/}{Work with us}
\item
  \href{https://nytmediakit.com/}{Advertise}
\item
  \href{http://www.tbrandstudio.com/}{T Brand Studio}
\item
  \href{https://www.nytimes3xbfgragh.onion/privacy/cookie-policy\#how-do-i-manage-trackers}{Your
  Ad Choices}
\item
  \href{https://www.nytimes3xbfgragh.onion/privacy}{Privacy}
\item
  \href{https://help.nytimes3xbfgragh.onion/hc/en-us/articles/115014893428-Terms-of-service}{Terms
  of Service}
\item
  \href{https://help.nytimes3xbfgragh.onion/hc/en-us/articles/115014893968-Terms-of-sale}{Terms
  of Sale}
\item
  \href{https://spiderbites.nytimes3xbfgragh.onion}{Site Map}
\item
  \href{https://help.nytimes3xbfgragh.onion/hc/en-us}{Help}
\item
  \href{https://www.nytimes3xbfgragh.onion/subscription?campaignId=37WXW}{Subscriptions}
\end{itemize}
