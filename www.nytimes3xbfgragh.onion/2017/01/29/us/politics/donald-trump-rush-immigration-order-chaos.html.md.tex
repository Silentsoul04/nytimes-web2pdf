Sections

SEARCH

\protect\hyperlink{site-content}{Skip to
content}\protect\hyperlink{site-index}{Skip to site index}

\href{https://www.nytimes3xbfgragh.onion/section/politics}{Politics}

\href{https://myaccount.nytimes3xbfgragh.onion/auth/login?response_type=cookie\&client_id=vi}{}

\href{https://www.nytimes3xbfgragh.onion/section/todayspaper}{Today's
Paper}

\href{/section/politics}{Politics}\textbar{}How Trump's Rush to Enact an
Immigration Ban Unleashed Global Chaos

\url{https://nyti.ms/2jLMK0Q}

\begin{itemize}
\item
\item
\item
\item
\item
\end{itemize}

Advertisement

\protect\hyperlink{after-top}{Continue reading the main story}

Supported by

\protect\hyperlink{after-sponsor}{Continue reading the main story}

\hypertarget{how-trumps-rush-to-enact-an-immigration-ban-unleashed-global-chaos}{%
\section{How Trump's Rush to Enact an Immigration Ban Unleashed Global
Chaos}\label{how-trumps-rush-to-enact-an-immigration-ban-unleashed-global-chaos}}

\includegraphics{https://static01.graylady3jvrrxbe.onion/images/2017/01/30/us/30RECONSTRUCT-01p/30RECONSTRUCT-01p-articleInline.jpg?quality=75\&auto=webp\&disable=upscale}

By \href{http://www.nytimes3xbfgragh.onion/by/michael-d-shear}{Michael
D. Shear} and \href{http://www.nytimes3xbfgragh.onion/by/ron-nixon}{Ron
Nixon}

\begin{itemize}
\item
  Jan. 29, 2017
\item
  \begin{itemize}
  \item
  \item
  \item
  \item
  \item
  \end{itemize}
\end{itemize}

WASHINGTON --- As President Trump signed a sweeping executive order on
Friday, shutting the borders to refugees and others from seven largely
Muslim countries, the secretary of homeland security was on a White
House conference call getting his first full briefing on the global
shift in policy.

Gen. John F. Kelly, the secretary of homeland security, had dialed in
from a Coast Guard plane as he headed back to Washington from Miami.
Along with other top officials, he needed guidance from the White House,
which had not asked his department for a legal review of the order.

Halfway into the briefing, someone on the call looked up at a television
in his office. ``The president is signing the executive order that we're
discussing,'' the official said, stunned.

The global confusion that has since erupted is the story of a White
House that rushed to enact, with little regard for basic governing, a
core campaign promise that Mr. Trump made to his most fervent
supporters. In his first week in office, Mr. Trump signed other
executive actions with little or no legal review, but his order barring
refugees has had the most explosive implications.

Passengers were barred from flights to the United States, customs and
border control officials got instructions at 3 a.m. Saturday and some
arrived at their posts later that morning still not knowing how to carry
out the president's orders.

``The details of it were not thought through,'' said Stephen Heifetz,
who served in the Justice and Homeland Security Departments, as well as
the C.I.A., under the previous three presidents. ``It is not surprising
there was mass confusion, and I expect the confusion and chaos will
continue for some time.''

Stephen K. Bannon, the chief White House strategist, oversaw the writing
of the order, which was done by a small White House team, including
Stephen Miller, Mr. Trump's policy chief. But it was first imagined more
than a year ago, when Mr. Trump, then a candidate for the Republican
nomination, reacted to terrorist attacks in San Bernardino, Calif., by
calling for a ``total and complete shutdown of Muslims entering the
United States.''

In the months that followed, Mr. Trump's campaign tried to back away
from the proposal, which was seen by Democrats as over-the-top campaign
rhetoric that would never be reality. Mr. Trump offered few details as
the campaign progressed, and as president-elect he promised to protect
the country from terrorists with only vague promises of ``extreme
vetting.''

But Mr. Bannon, who believes in highly restrictive immigration policies
and saw barring refugees as vital to shoring up Mr. Trump's political
base, was determined to make it happen. He and a small group made up of
the president's closest advisers began working on the order during the
transition so that Mr. Trump could sign it soon after taking office.

A senior administration official said that the order was drafted in
cooperation with some immigration experts on Capitol Hill and members of
the ``beachhead teams'' --- small groups of political appointees sent by
the new White House to be liaisons and begin work at the agencies.

James Jay Carafano, a vice president of the conservative Heritage
Foundation and a member of Mr. Trump's transition team, said that little
of that work was shared with career officials at the Homeland Security
Department, the State Department or other agencies.

There was ``a firewall between the old administration and the incoming
one,'' Mr. Carafano said.

One reason, he said, is that when the Trump transition team asked
pointed questions suggesting new policies to the career officials, those
questions were swiftly leaked to the news media, generating negative
stories. So the Trump team began to limit the information they discussed
with officials from the previous administration.

``Why share it with them?'' Mr. Carafano said.

R. Gil Kerlikowske, who served as commissioner of Customs and Border
Protection under former President Barack Obama, said that his staff had
little communication with Mr. Trump's transition team, who made no
mention of a bar on entry for people from certain countries.

White House officials in the meantime insisted to reporters at a
briefing that Mr. Trump's advisers had been in contact with officials at
the State and Homeland Security Departments for ``many weeks.''

One official added, ``Everyone who needed to know was informed.''

But that apparently did not include members of the president's own
cabinet.

Jim Mattis, the new secretary of defense, did not see a final version of
the order until Friday morning, only hours before Mr. Trump arrived to
sign it at the Pentagon.

Mr. Mattis, according to administration officials familiar with the
deliberations, was not consulted by the White House during the
preparation of the order and was not given an opportunity to provide
input while the order was being drafted. Last summer, Mr. Mattis sharply
criticized Mr. Trump's proposed ban on Muslim immigration as a move that
was ``causing us great damage right now, and it's sending shock waves
through the international system.''

Customs and Border Protection officers were also caught unaware.

They contacted several airlines late Friday that were likely to be
carrying passengers from the seven countries and ``instructed the
airlines to offload any passport holders from those countries,'' said a
state government official who has been briefed on the agency's actions.

It was not until 3 a.m. on Saturday that customs and border officials
received limited written instructions about what to do at airports and
border crossings. They also struggled with how to exercise the waiver
authority that was included in the executive order, which allowed the
homeland security secretary to let some individuals under the ban enter
the country case by case.

One customs officer, who declined to be quoted by name, said he was
given a limited briefing about what to do as he went to his post on
Saturday morning, but even managers seemed unclear. People at the agency
were blindsided, he said, and are still trying to figure things out,
even as people are being stopped from coming into the United States.

``If the secretary doesn't know anything, how could we possibly know
anything at this level?'' the officer said, referring to Mr. Kelly.

At the Citizenship and Immigration Service, staff members were told that
the agency should stop work on any application filed by a person from
any of the countries listed in the ban. Employees were told that
applicants should be interviewed, but that their cases for citizenship,
green cards or other immigration documents should be put on pause,
pending further guidance.

The timing of the executive order and the lack of advance warning had
homeland security officials ``flying by the seat of their pants,'' to
try to put policies in place, one official said.

By Saturday, as the order stranded travelers around the world and its
full impact became clear, Reince Priebus, the chief of staff, became
increasingly upset about how the program had been rolled out and
communicated to the public.

By Sunday morning, Mr. Priebus had to defend the immigration ban on
NBC's ``Meet the Press,'' where he insisted that the executive order was
rolled out smoothly. He also backpedaled on the policy and said that the
executive order's restrictions on entry to the United States would not
apply to legal permanent residents ``going forward.''

As White House officials also insisted on Sunday that the order had gone
through the usual process of scrutiny and approval by the Office of
Legal Counsel, the continuing confusion forced Mr. Kelly to clarify the
waiver situation. He issued a statement making clear that lawful
permanent residents --- those who hold valid green cards --- would be
granted a waiver to enter the United States unless information suggested
that they were a security threat.

But senior White House officials insisted on Sunday night that the
executive order would remain in force despite the change, and that they
were proud of taking actions that they said would help protect Americans
against threats from potential terrorists.

That assertion is likely to do little to calm the public furor, which
showed no signs of waning at the beginning of Mr. Trump's second full
week in the Oval Office.

Mr. Carafano said he believed that the substance of Mr. Trump's
executive order was neither radical nor unreasonable. But he said that
Mr. Trump's team could have delayed signing the order until they had
better prepared the bureaucracy to carry it out.

He also said the president and his team had not done a good job of
communicating to the public the purpose of the executive order.

``If there is a criticism of the administration, and I think there is, I
think they have done a rotten job of telling their story,'' he said.
``It is not like they did not know they were going to do this. To not
have a cadre of people out there defending the administration --- I
mean, really guys, they should have done this.''

Advertisement

\protect\hyperlink{after-bottom}{Continue reading the main story}

\hypertarget{site-index}{%
\subsection{Site Index}\label{site-index}}

\hypertarget{site-information-navigation}{%
\subsection{Site Information
Navigation}\label{site-information-navigation}}

\begin{itemize}
\tightlist
\item
  \href{https://help.nytimes3xbfgragh.onion/hc/en-us/articles/115014792127-Copyright-notice}{©~2020~The
  New York Times Company}
\end{itemize}

\begin{itemize}
\tightlist
\item
  \href{https://www.nytco.com/}{NYTCo}
\item
  \href{https://help.nytimes3xbfgragh.onion/hc/en-us/articles/115015385887-Contact-Us}{Contact
  Us}
\item
  \href{https://www.nytco.com/careers/}{Work with us}
\item
  \href{https://nytmediakit.com/}{Advertise}
\item
  \href{http://www.tbrandstudio.com/}{T Brand Studio}
\item
  \href{https://www.nytimes3xbfgragh.onion/privacy/cookie-policy\#how-do-i-manage-trackers}{Your
  Ad Choices}
\item
  \href{https://www.nytimes3xbfgragh.onion/privacy}{Privacy}
\item
  \href{https://help.nytimes3xbfgragh.onion/hc/en-us/articles/115014893428-Terms-of-service}{Terms
  of Service}
\item
  \href{https://help.nytimes3xbfgragh.onion/hc/en-us/articles/115014893968-Terms-of-sale}{Terms
  of Sale}
\item
  \href{https://spiderbites.nytimes3xbfgragh.onion}{Site Map}
\item
  \href{https://help.nytimes3xbfgragh.onion/hc/en-us}{Help}
\item
  \href{https://www.nytimes3xbfgragh.onion/subscription?campaignId=37WXW}{Subscriptions}
\end{itemize}
