Sections

SEARCH

\protect\hyperlink{site-content}{Skip to
content}\protect\hyperlink{site-index}{Skip to site index}

\href{https://www.nytimes3xbfgragh.onion/section/politics}{Politics}

\href{https://myaccount.nytimes3xbfgragh.onion/auth/login?response_type=cookie\&client_id=vi}{}

\href{https://www.nytimes3xbfgragh.onion/section/todayspaper}{Today's
Paper}

\href{/section/politics}{Politics}\textbar{}Rulings on Trump's
Immigration Order Are First Step on a Long Legal Path

\url{https://nyti.ms/2jKSxDu}

\begin{itemize}
\item
\item
\item
\item
\item
\end{itemize}

Advertisement

\protect\hyperlink{after-top}{Continue reading the main story}

Supported by

\protect\hyperlink{after-sponsor}{Continue reading the main story}

\hypertarget{rulings-on-trumps-immigration-order-are-first-step-on-a-long-legal-path}{%
\section{Rulings on Trump's Immigration Order Are First Step on a Long
Legal
Path}\label{rulings-on-trumps-immigration-order-are-first-step-on-a-long-legal-path}}

\includegraphics{https://static01.graylady3jvrrxbe.onion/images/2017/01/30/us/30LEGAL/30LEGAL-articleLarge.jpg?quality=75\&auto=webp\&disable=upscale}

By \href{http://www.nytimes3xbfgragh.onion/by/adam-liptak}{Adam Liptak}

\begin{itemize}
\item
  Jan. 29, 2017
\item
  \begin{itemize}
  \item
  \item
  \item
  \item
  \item
  \end{itemize}
\end{itemize}

WASHINGTON --- With what by legal standards was lightning speed, the
judicial branch responded to President Trump's immigration order on
Saturday night, telling the president that he had moved too fast in
barring people from seven predominantly Muslim nations from entering the
United States.

But the court orders, from judges in at least four cities, were just the
initial steps in litigation that may last for years.

The orders were provisional, aimed at maintaining the status quo. They
were limited in scope, applying only to people on their way to the
United States or already here. They did not rule on the larger question
of whether Mr. Trump's executive order was lawful.

They were a signal that the federal judiciary stands ready to assess the
limits of presidential power over immigration policy. But they gave only
the most preliminary hints about whether the courts will strike down
part or all of Mr. Trump's executive order.

Still, leaders of civil liberties groups were savoring their victories
in these early skirmishes.

``The courts can serve as a bulwark against these excesses,'' said
Anthony D. Romero, the executive director of the American Civil
Liberties Union, which represents the plaintiffs in one of the cases.
``Litigation is going to be a key tool for either undoing these policies
or slowing them down.''

In a statement issued early Sunday, the Department of Homeland Security
said it would comply with the court orders. But the department added
that little had changed.

``The president's executive orders remain in place --- prohibited travel
will remain prohibited, and the U.S. government retains its right to
revoke visas at any time if required for national security or public
safety,'' the statement said. ``No foreign national in a foreign land,
without ties to the United States, has any unfettered right to demand
entry into the United States or to demand immigration benefits in the
United States.''

The four cases --- in Boston, Brooklyn, Seattle and Alexandria, Va. ---
will move forward in the coming weeks, with briefs and hearings over
whether to make the bans on removing the travelers more permanent. At
the same time, the Trump administration may appeal the orders to federal
appellate courts.

David Cole, the A.C.L.U.'s legal director, said the order from Judge Ann
M. Donnelly of the Federal District Court in Brooklyn, which issued a
nationwide injunction, was an important first step.

``Making it stick will require both making sure that it's being followed
at borders around the country and upholding it on appeal,'' he said.

While the government has released some plaintiffs in the four cases,
others are being held while the cases move forward.

On Sunday, there were widespread reports that government officials at
airports were not complying with aspects of the orders. That may have
been a product of confusion and miscommunication, but some feared it was
an early sign of a potential constitutional showdown.

``The scariest scenario,'' said Peter J. Spiro, a law professor at
Temple University, ``is one in which the Department of Homeland Security
simply ignores the court orders. That would have been close to
unimaginable in prior administrations. In this one, unfortunately, it
would almost be unsurprising.''

The New York attorney general, Eric T. Schneiderman, sent
\href{https://ag.ny.gov/sites/default/files/dhs_cpb_letter_01_29_2016_final_rev_1.pdf}{a
letter} to federal officials Sunday demanding information about those
held at John F. Kennedy International Airport after receiving what he
called ``alarming reports'' that the federal government was planning to
violate the court order.

In an interview, Mr. Schneiderman said that the executive order was
unconstitutional and that he and other attorneys general were exploring
the possibility of legal action. ``There may be grounds for a claim that
this does damage to a state, if they're damaging our state
institutions'' like universities and hospitals, he said.

The president has broad power to control immigration.
\href{https://www.law.cornell.edu/uscode/text/8/1182}{A federal law}
allows him to ``suspend the entry of all aliens or any class of aliens
as immigrants or nonimmigrants'' if he determines that their entry
``would be detrimental to the interests of the United States.''

But \href{https://www.law.cornell.edu/uscode/text/8/1152}{another law}
appears to limit that power, saying that ``no person shall receive any
preference or priority or be discriminated against in the issuance of an
immigrant visa because of the person's race, sex, nationality, place of
birth or place of residence.''

Saturday's court orders, issued in haste, contained little reasoning.
Judge Donnelly, in Brooklyn,
\href{https://www.nytimes3xbfgragh.onion/interactive/2017/01/28/us/politics/trump-darweesh-decision-stay-refugee-ban.html}{wrote
that} there was a strong likelihood that the two plaintiffs in the case,
who seek to represent a class of refugees, visa holders and others,
could establish that their removal would violate the Constitution's due
process and equal protection guarantees.

Based on that conclusion and the ``irreparable injury'' the plaintiffs
would suffer, Judge Donnelly issued a temporary nationwide injunction
barring their removal. She did not order their release.

\href{https://aclum.org/wp-content/uploads/2017/01/6-TRO-Jan-29-2017.pdf}{The
order} in Boston, from Judge Allison D. Burroughs of the Federal
District Court there, was in one sense broader and in another narrower.
It limited border screening practices to those in place before Mr.
Trump's executive order, and it barred not only removal but also
detention. But it seemed to be limited to people arriving or held at
Logan International Airport there.

\href{https://www.justice4all.org/wp-content/uploads/2017/01/TRO-order-signed.pdf}{The
order} from Judge Leonie M. Brinkema of the Federal District Court in
Alexandria applied only to legal permanent residents being held at
Dulles International Airport.
\href{https://crosscut.com/wp-content/uploads/2017/01/Judge-Zilly_Order.pdf}{The
order} in Seattle, from Judge Thomas S. Zilly of the Federal District
Court there, applied to two plaintiffs.

It is possible that one of the four cases will definitively resolve an
important aspect of the legality of Mr. Trump's order. But the cases
were filed very quickly, and civil rights litigators generally like to
locate ideal plaintiffs and hone their legal theories before filing test
cases that could turn into legal landmarks.

The Brooklyn case, for instance, did not make a claim based on the First
Amendment's prohibition of government establishment of religion, though
some lawyers said that was the executive order's central flaw.

``The smoking gun they put in the executive order is the idea that they
would grant exceptions for minority religions,'' Mr. Romero said. Mr.
Trump has said that was meant to favor Christians over Muslims.

But opening such a challenge requires preparation, Mr. Cole said.

``The one thing you can't do under the establishment clause is
denomination favoritism,'' he said. ``That's a very promising claim, but
it requires the right plaintiff.''

Advertisement

\protect\hyperlink{after-bottom}{Continue reading the main story}

\hypertarget{site-index}{%
\subsection{Site Index}\label{site-index}}

\hypertarget{site-information-navigation}{%
\subsection{Site Information
Navigation}\label{site-information-navigation}}

\begin{itemize}
\tightlist
\item
  \href{https://help.nytimes3xbfgragh.onion/hc/en-us/articles/115014792127-Copyright-notice}{©~2020~The
  New York Times Company}
\end{itemize}

\begin{itemize}
\tightlist
\item
  \href{https://www.nytco.com/}{NYTCo}
\item
  \href{https://help.nytimes3xbfgragh.onion/hc/en-us/articles/115015385887-Contact-Us}{Contact
  Us}
\item
  \href{https://www.nytco.com/careers/}{Work with us}
\item
  \href{https://nytmediakit.com/}{Advertise}
\item
  \href{http://www.tbrandstudio.com/}{T Brand Studio}
\item
  \href{https://www.nytimes3xbfgragh.onion/privacy/cookie-policy\#how-do-i-manage-trackers}{Your
  Ad Choices}
\item
  \href{https://www.nytimes3xbfgragh.onion/privacy}{Privacy}
\item
  \href{https://help.nytimes3xbfgragh.onion/hc/en-us/articles/115014893428-Terms-of-service}{Terms
  of Service}
\item
  \href{https://help.nytimes3xbfgragh.onion/hc/en-us/articles/115014893968-Terms-of-sale}{Terms
  of Sale}
\item
  \href{https://spiderbites.nytimes3xbfgragh.onion}{Site Map}
\item
  \href{https://help.nytimes3xbfgragh.onion/hc/en-us}{Help}
\item
  \href{https://www.nytimes3xbfgragh.onion/subscription?campaignId=37WXW}{Subscriptions}
\end{itemize}
