Sections

SEARCH

\protect\hyperlink{site-content}{Skip to
content}\protect\hyperlink{site-index}{Skip to site index}

\href{https://www.nytimes3xbfgragh.onion/section/politics}{Politics}

\href{https://myaccount.nytimes3xbfgragh.onion/auth/login?response_type=cookie\&client_id=vi}{}

\href{https://www.nytimes3xbfgragh.onion/section/todayspaper}{Today's
Paper}

\href{/section/politics}{Politics}\textbar{}Travelers Stranded and
Protests Swell Over Trump Order

\url{https://nyti.ms/2jFy45B}

\begin{itemize}
\item
\item
\item
\item
\item
\item
\end{itemize}

Advertisement

\protect\hyperlink{after-top}{Continue reading the main story}

Supported by

\protect\hyperlink{after-sponsor}{Continue reading the main story}

\hypertarget{travelers-stranded-and-protests-swell-over-trump-order}{%
\section{Travelers Stranded and Protests Swell Over Trump
Order}\label{travelers-stranded-and-protests-swell-over-trump-order}}

\includegraphics{https://static01.graylady3jvrrxbe.onion/images/2017/01/30/us/30trump-web2/30trump-web2-videoSixteenByNine3000.jpg}

By \href{http://www.nytimes3xbfgragh.onion/by/peter-baker}{Peter Baker}

\begin{itemize}
\item
  Jan. 29, 2017
\item
  \begin{itemize}
  \item
  \item
  \item
  \item
  \item
  \item
  \end{itemize}
\end{itemize}

WASHINGTON --- Travelers were stranded around the world, protests
escalated in the United States and anxiety rose within President Trump's
party on Sunday as his order closing the nation to refugees and people
from certain predominantly Muslim countries provoked a crisis just days
into his administration.

The White House pulled back on part of
\href{https://www.nytimes3xbfgragh.onion/2017/01/28/us/refugees-detained-at-us-airports-prompting-legal-challenges-to-trumps-immigration-order.html?_r=0}{Mr.
Trump's temporary ban} on visitors from seven countries by saying that
it would not apply to those with green cards granting them permanent
residence in the United States. By the end of the day, the Department of
Homeland Security formally issued an order declaring such legal
residents exempt from the order.

But the recalibration did little to reassure critics at home or abroad
who saw the president's order as a retreat from traditional American
values. European leaders denounced the order, and some Republican
lawmakers called on Mr. Trump to back down. As of Sunday evening,
officials said no one was being held at American airports, although
lawyers said they believed that dozens were still being detained.

More than any of the myriad moves Mr. Trump has made in his frenetic
opening days in office, the immigration order has quickly come to define
his emerging presidency as one driven by a desire for decisive action
even at the expense of deliberate process or coalition building. It has
thrust the nine-day-old administration into its first constitutional
conflict, as multiple courts have intervened to block aspects of the
order, and into its broadest diplomatic incident, with overseas allies
objecting.

The White House was left to defend what seemed to many government
veterans like a slapdash process. Aides to Mr. Trump insisted they had
consulted for weeks with relevant officials, but the head of the customs
and border service in the Obama administration, who resigned on
inauguration day, said the incoming president's team never talked with
him about it.

White House officials blamed what they portrayed as a hyperventilating
news media for the confusion and said the order had been successfully
carried out. Only about 109 travelers were detained in the first 24
hours, out of the 325,000 who typically enter the United States in a
day, they said. As of Sunday evening, the Department of Homeland
Security said 392 green card holders had been granted waivers to enter.
That did not count many visitors who remained overseas now unable to
travel.

Reince Priebus, the White House chief of staff, said Mr. Trump simply
did what he had promised on the campaign trail and would not gamble with
American lives. ``We're not willing to be wrong on this subject,'' he
said on ``Face the Nation'' on CBS. ``President Trump is not willing to
take chances on this subject.''

The order bars entry to refugees from anywhere in the world for 120 days
and from Syria indefinitely. It blocks any visitors for 90 days from
seven designated countries: Iran, Iraq, Libya, Somalia, Sudan, Syria and
Yemen. The Department of Homeland Security initially said the order
would bar green card holders from those seven countries from returning
to the United States.

With thousands of protesters chanting outside his White House windows
and thronging the streets of Washington and other cities, Mr. Trump late
on Sunday defended his order. ``To be clear, this is not a Muslim ban,
as the media is falsely reporting,'' he said in a written statement.
``This is not about religion --- this is about terror and keeping our
country safe.''

He noted that the seven countries were identified by former President
Barack Obama's administration as sources of terrorism and that his order
did not affect citizens from dozens of other predominantly Muslim
countries. ``We will again be issuing visas to all countries once we are
sure we have reviewed and implemented the most secure policies over the
next 90 days,'' he said.

Mr. Trump expressed sympathy for victims of the long-running civil war
in Syria. ``I have tremendous feeling for the people involved in this
horrific humanitarian crisis in Syria,'' he said. ``My first priority
will always be to protect and serve our country, but as president, I
will find ways to help all those who are suffering.''

\includegraphics{https://static01.graylady3jvrrxbe.onion/images/2017/01/30/us/30cong/30cong-videoSixteenByNine3000.jpg}

While Mr. Trump denied that his action focused on religion, the first
iteration of his plan during his presidential campaign was framed as a
temporary ban on all Muslim visitors.

As late as Sunday morning, he made clear that his concern was for
Christian refugees, and part of his order gives preferential treatment
to Christians who try to enter the United States from majority-Muslim
nations.

In a Twitter post on Sunday morning, Mr. Trump deplored the killing of
Christians in the Middle East without noting the killings of Muslims,
who have been killed in vastly greater numbers in Iraq, Syria and
elsewhere.

``Christians in the Middle East have been executed in large numbers,''
he wrote. ``We cannot allow this horror to continue!''

His order, however, resulted in a second day of uncertainty at American
airports. The American Civil Liberties Union said it was investigating
reports that officials were not complying with court orders in New York,
Boston, Seattle, Los Angeles and Chicago.

New York's attorney general sent a letter to federal authorities
demanding a list of all individuals detained at Kennedy International
Airport. The Department of Homeland Security said on Sunday evening that
it was ``in compliance with judicial orders.''

Still, at Dulles International Airport outside Washington, even the
arrival of four Democratic members of Congress did not prompt customs
officers to acknowledge whether they were holding anyone or provide
lawyers access to anyone detained.

The lawmakers arrived after 3 p.m. and were rebuffed by police officers
when they tried to enter the Customs and Border Protection offices at
the airport. Representative Gerry Connolly, Democrat of Virginia, said
he was told to call the main office of the agency in Washington.

His staff got a legislative liaison from the customs service on the
phone, and ``they said we'll put you in touch with the deputy
commissioner,'' Mr. Connolly said.

``I said that's not acceptable,'' he continued. ``We want to talk to the
person in charge of operations at Dulles Airport. That's where the
problem is, and that's where the federal judicial ruling is
applicable.''

The clash over the order provoked emotional responses. At a news
conference, Senator Chuck Schumer, the Democratic minority leader from
New York, choked up as he vowed to ``claw, scrap and fight with every
fiber of my being until these orders are overturned.''

The mayors of New York, Chicago and Boston spoke out, as well. In
Dallas, Mayor Mike Rawlings personally offered regrets to four released
detainees at Dallas-Fort Worth Airport. ``We have wished them welcome,
and we have apologized from the depths of our heart,'' he said. Chelsea
Clinton joined a protest in New York.

The order roiled relations with America's traditional allies in Europe
and the Middle East. The spokesman for Chancellor Angela Merkel of
Germany said she ``is convinced that the resolute fight against
terrorism does not justify blanket suspicion on grounds of origin or
belief.''

Prime Minister Theresa May of Britain, who met with Mr. Trump in
Washington on Friday and has sought to forge a friendship with him,
initially declined to comment on the policy on Saturday when pressed by
reporters during a stop in Turkey.

But under pressure from opposition politicians, her spokesman later said
the British government did ``not agree with this kind of approach.''

The matter was especially sensitive in Muslim countries, and Mr. Trump
spoke by telephone on Sunday with King Salman of Saudi Arabia and Sheikh
Mohammed bin Zayed Al Nahyan, the crown prince of Abu Dhabi. White House
statements on the calls said they discussed the fight against terrorism
but did not say whether they discussed the immigration order, which did
not include their countries.

In Washington, protesters gathered by the thousands outside Mr. Trump's
front lawn to denounce his order and show solidarity with Muslim
Americans.

``Shame,'' they chanted, hoisting homemade signs toward the executive
mansion, where Mr. Trump was scheduled to host a private screening of
the movie ``Finding Dory.''

``No hate, no fear,'' they added later. ``Refugees are welcome here.''

Security fencing and reviewing stands still in place from the
inauguration prevented the crowd from getting more than a couple hundred
yards away from the building, but did not stop crowds from swelling
through the afternoon, when protesters departed to march to Capitol
Hill.

Some Republicans grew increasingly alarmed by the backlash to the order.
``This executive order sends a signal, intended or not, that America
does not want Muslims coming into our country,'' Senators John McCain of
Arizona and Lindsey Graham of South Carolina said in a statement. ``That
is why we fear this executive order may do more to help terrorist
recruitment than improve our security.''

Some conservative donors also criticized the decision. Officials with
the political network overseen by Charles G. and David H. Koch, the
billionaire conservative activists, released a statement on Sunday
criticizing Mr. Trump's handling of the issue.

``We believe it is possible to keep Americans safe without excluding
people who wish to come here to contribute and pursue a better life for
their families,'' said Brian Hooks, a chairman of the Kochs' donor
network. ``The travel ban is the wrong approach and will likely be
counterproductive.''

Senator Bob Corker, the chairman of the Foreign Relations Committee,
said the order was ``poorly implemented'' and urged the president to
``make appropriate revisions.'' Other Republicans were more circumspect.
Senator Mitch McConnell, the Republican majority leader, said the issue
would be decided by the courts.

Mr. Trump fired back at Mr. McCain and Mr. Graham on Twitter. ``They are
sadly weak on immigration,'' he wrote. ``Senators should focus their
energies on ISIS, illegal immigration and border security instead of
always looking to start World War III.''

Advertisement

\protect\hyperlink{after-bottom}{Continue reading the main story}

\hypertarget{site-index}{%
\subsection{Site Index}\label{site-index}}

\hypertarget{site-information-navigation}{%
\subsection{Site Information
Navigation}\label{site-information-navigation}}

\begin{itemize}
\tightlist
\item
  \href{https://help.nytimes3xbfgragh.onion/hc/en-us/articles/115014792127-Copyright-notice}{©~2020~The
  New York Times Company}
\end{itemize}

\begin{itemize}
\tightlist
\item
  \href{https://www.nytco.com/}{NYTCo}
\item
  \href{https://help.nytimes3xbfgragh.onion/hc/en-us/articles/115015385887-Contact-Us}{Contact
  Us}
\item
  \href{https://www.nytco.com/careers/}{Work with us}
\item
  \href{https://nytmediakit.com/}{Advertise}
\item
  \href{http://www.tbrandstudio.com/}{T Brand Studio}
\item
  \href{https://www.nytimes3xbfgragh.onion/privacy/cookie-policy\#how-do-i-manage-trackers}{Your
  Ad Choices}
\item
  \href{https://www.nytimes3xbfgragh.onion/privacy}{Privacy}
\item
  \href{https://help.nytimes3xbfgragh.onion/hc/en-us/articles/115014893428-Terms-of-service}{Terms
  of Service}
\item
  \href{https://help.nytimes3xbfgragh.onion/hc/en-us/articles/115014893968-Terms-of-sale}{Terms
  of Sale}
\item
  \href{https://spiderbites.nytimes3xbfgragh.onion}{Site Map}
\item
  \href{https://help.nytimes3xbfgragh.onion/hc/en-us}{Help}
\item
  \href{https://www.nytimes3xbfgragh.onion/subscription?campaignId=37WXW}{Subscriptions}
\end{itemize}
