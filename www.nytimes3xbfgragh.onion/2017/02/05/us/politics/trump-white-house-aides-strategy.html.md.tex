Sections

SEARCH

\protect\hyperlink{site-content}{Skip to
content}\protect\hyperlink{site-index}{Skip to site index}

\href{https://www.nytimes3xbfgragh.onion/section/politics}{Politics}

\href{https://myaccount.nytimes3xbfgragh.onion/auth/login?response_type=cookie\&client_id=vi}{}

\href{https://www.nytimes3xbfgragh.onion/section/todayspaper}{Today's
Paper}

\href{/section/politics}{Politics}\textbar{}Trump and Staff Rethink
Tactics After Stumbles

\url{https://nyti.ms/2kbB4UY}

\begin{itemize}
\item
\item
\item
\item
\item
\item
\end{itemize}

Advertisement

\protect\hyperlink{after-top}{Continue reading the main story}

Supported by

\protect\hyperlink{after-sponsor}{Continue reading the main story}

\hypertarget{trump-and-staff-rethink-tactics-after-stumbles}{%
\section{Trump and Staff Rethink Tactics After
Stumbles}\label{trump-and-staff-rethink-tactics-after-stumbles}}

\includegraphics{https://static01.graylady3jvrrxbe.onion/images/2017/02/06/us/06conflict-web02/06conflict-web02-articleInline-v2.jpg?quality=75\&auto=webp\&disable=upscale}

By \href{https://www.nytimes3xbfgragh.onion/by/glenn-thrush}{Glenn
Thrush} and
\href{http://www.nytimes3xbfgragh.onion/by/maggie-haberman}{Maggie
Haberman}

\begin{itemize}
\item
  Feb. 5, 2017
\item
  \begin{itemize}
  \item
  \item
  \item
  \item
  \item
  \item
  \end{itemize}
\end{itemize}

WASHINGTON --- President Trump loves to set the day's narrative at dawn,
but the deeper story of his White House is best told at night.

Aides confer in the dark because they cannot figure out how to operate
the light switches in the cabinet room. Visitors conclude their meetings
and then wander around, testing doorknobs until finding one that leads
to an exit. In a darkened, mostly empty West Wing, Mr. Trump's
provocative chief strategist, Stephen K. Bannon, finishes another
16-hour day planning new lines of attack.

Usually around 6:30 p.m., or sometimes later, Mr. Trump retires upstairs
to the residence to recharge, vent and intermittently use Twitter. With
his wife, Melania, and young son, Barron, staying in New York, he is
almost always by himself, sometimes in the protective presence of his
imposing longtime aide and former security chief, Keith Schiller. When
Mr. Trump is not watching television in his bathrobe or on his phone
reaching out to old campaign hands and advisers, he will sometimes set
off to explore the unfamiliar surroundings of his new home.

During his first two dizzying weeks in office, Mr. Trump, an outsider
president working with a surprisingly small crew of no more than a
half-dozen empowered aides with virtually no familiarity with the
workings of the White House or federal government, sent shock waves at
home and overseas with a
\href{https://www.nytimes3xbfgragh.onion/interactive/2017/us/politics/trump-agenda-tracker.html?_r=0}{succession
of executive orders} designed to fulfill campaign promises and taunt
foreign leaders.

``We are moving big and we are moving fast,'' Mr. Bannon said, when
asked about the upheaval of the first two weeks. ``We didn't come here
to do small things.''

But one thing has become apparent to both his allies and his opponents:
When it comes to governing, speed does not always guarantee success.

The bungled rollout of his
\href{https://www.nytimes3xbfgragh.onion/2017/01/29/us/trump-refugee-ban-muslim-executive-order.html}{executive
order barring immigrants} from seven predominantly Muslim countries, a
flurry of other miscues and embarrassments, and an approval rating lower
than that of any comparable first-term president in the history of
polling have Mr. Trump and his top staff rethinking an improvisational
approach to governing that mirrors his chaotic presidential campaign,
administration officials and Trump insiders said.

This account of the early days of the Trump White House is based on
interviews with dozens of government officials, congressional aides,
former staff members and other observers of the new administration, many
of whom requested anonymity. At the center of the story, according to
these sources, is a president determined to go big but increasingly
frustrated by the efforts of his small team to contain the backlash.

``What are we going to do about this?'' Mr. Trump pointedly asked an
aide last week, a period of turmoil briefly interrupted by the
successful rollout of his
\href{https://www.nytimes3xbfgragh.onion/2017/01/31/us/politics/supreme-court-nominee-trump.html}{Supreme
Court selection}, Judge Neil M. Gorsuch.

Chris Ruddy, the chief executive of Newsmax Media and an old friend of
the president's, said: ``I think, in his mind, the success of this is
going to be the poll numbers. If they continue to be weak or go lower,
then somebody's going to have to bear some responsibility for that.''

``I personally think that they're missing the big picture here,'' Mr.
Ruddy said of Mr. Trump's staff. ``Now he's so caught up, the
administration is so caught up in turmoil, perceived chaos, that the
Democrats smell blood, the protesters, the media smell blood.''

One former staff member likened the aggressive approach of the first two
weeks to D-Day, but said the president's team had stormed the beaches
without any plan for a longer war.

Clashes among staff are common in the opening days of every
administration, but they have seldom been so public and so pronounced
this early. ``This is a president who came to Washington vowing to shake
up the establishment, and this is what it looks like. It's going to be a
little sloppy, there are going to be conflicts,'' said Ari Fleischer,
President George W. Bush's first press secretary.

\includegraphics{https://static01.graylady3jvrrxbe.onion/images/2017/02/06/us/06conflict-web01/06conflict-web01-articleInline.jpg?quality=75\&auto=webp\&disable=upscale}

All this is happening as Mr. Trump, a man of flexible ideology but fixed
habits, adjusts to a new job, life and city.

Cloistered in the White House, he now has little access to his fans and
supporters --- an important source of feedback and validation --- and
feels increasingly pinched by the pressures of the job and the constant
presence of protests, one of the reasons he was forced to scrap a
planned trip to Milwaukee last week. For a sense of what is happening
outside, he watches cable, both at night and during the day --- too much
in the eyes of some aides --- often offering a bitter play-by-play of
critics like CNN's Don Lemon.

Until the past few days, Mr. Trump was telling his friends and advisers
that he believed the opening stages of his presidency were going well.
``Did you hear that, this guy thinks it's been terrible!'' Mr. Trump
said mockingly to other aides when one dissenting view was voiced last
week during a West Wing meeting.

But his opinion has begun to change with a relentless parade of bad
headlines.

Mr. Trump got away from the White House this weekend for the first time
since his inauguration, spending it in Palm Beach, Fla., at his private
club, Mar-a-Lago, posting Twitter messages angrily --- and in personal
terms --- about the
\href{https://twitter.com/realDonaldTrump/status/827867311054974976}{federal
judge} who put a nationwide halt on the travel ban. Mr. Bannon and
Reince Priebus, the two clashing power centers, traveled with him.

\includegraphics{https://static01.graylady3jvrrxbe.onion/images/2017/02/06/us/06travel-web/06travel-web-videoSixteenByNine3000.jpg}

By then, the president, for whom chains of command and policy minutiae
rarely meant much, was demanding that Mr. Priebus begin to put in effect
a much more conventional White House protocol that had been taken for
granted in previous administrations: From now on, Mr. Trump would be
looped in on the drafting of executive orders much earlier in the
process.

Another change will be a new set of checks on the previously unfettered
power enjoyed by Mr. Bannon and the White House policy director, Stephen
Miller, who oversees the implementation of the orders and who received
the brunt of the internal and public criticism for the rollout of the
travel ban.

Mr. Priebus has told Mr. Trump and Mr. Bannon that the administration
needs to rethink its policy and communications operation in the wake of
embarrassing revelations that key details of the orders were withheld
from agencies, White House staff and Republican congressional leaders
like Speaker Paul D. Ryan.

Mr. Priebus has also created a 10-point checklist for the release of any
new initiatives that includes signoff from the communications department
and the White House staff secretary, Robert Porter, according to several
aides familiar with the process.

Mr. Priebus bristles at the perception that he occupies a diminished
perch in the West Wing pecking order compared with previous chiefs. But
for the moment, Mr. Bannon remains the president's dominant adviser,
despite Mr. Trump's anger that he was not fully briefed on details of
the executive order he signed giving his chief strategist a seat on the
National Security Council, a greater source of frustration to the
president than the fallout from the travel ban.

It is partly because he is seen as having a clear vision on policy. But
it is also because others who had been expected to fill major roles have
been less confident in asserting their power.

Jared Kushner, Mr. Trump's son-in-law, occupies a central role in the
administration and has been present at most major decisions and photo
ops, but he is a father of young children who has taken to life in
Washington, and, along with his wife, Ivanka Trump, has already been
spotted at events around town.

Mr. Bannon has rushed into the vacuum, telling allies that he and Mr.
Miller have a brief window in which to push through their vision of Mr.
Trump's economic nationalism.

Mr. Bannon, whose website, Breitbart, was a magnet for white
nationalists and xenophobic speech, has also tried to reassure official
Washington. He has been careful to build bridges with the Republican
establishment, especially Mr. Ryan --- whom he once described as ``the
enemy'' and vowed to force out. He now talks regularly with Mr. Ryan to
coordinate strategy or plot their planned overhaul of the tax code.

Before he was ousted in November as transition chief, Gov. Chris
Christie of New Jersey, the Trump adviser with the most government
experience, helped prepare a detailed staffing and implementation plan
in line with the kickoff strategies of previous Republican presidents.

Image

Mr. Bannon, the chief strategist, and Mr. Priebus, the chief of staff,
are the two clashing power centers of Mr. Trump's White
House.Credit...Al Drago/The New York Times

It was discarded --- a senior Trump aide made a show of tossing it into
a garbage can --- for a strategy that prioritized the daily release of
dramatic executive orders to put opponents on the defensive.

Mr. Christie, who agrees in principle with the broad strokes of Mr.
Trump's immigration policy, says the president has been let down by his
staff.

``The president deserves better than the rollout he got on the
immigration executive order,'' Mr. Christie said. ``The fact is that
he's put forward a policy that, in my opinion, is significantly more
effective than what he had proposed during the campaign, yet because of
the botched implementation, they allowed his opponents to attack him by
calling it a Muslim ban.''

In the past few days, Mr. Trump's team has stressed its cohesion and the
challenges of jump-starting an administration that few outside its group
ever thought would exist.

``This team spent months in the foxhole together during the campaign,''
said Sean Spicer, the White House press secretary. ``We moved into the
White House as a unified team committed to enacting the president's
agenda.''

As part of Mr. Trump's Oval Office renovation, he ordered that four
hardback chairs be placed in a semicircle around his Resolute Desk now
heaped, in Trump Tower fashion, with memos and newspapers. They are an
emblem of Mr. Trump's in-your-face management style, but also a reminder
that in the White House, the seats always outlast the people seated in
them.

But finding enough skilled players to fill key slots has not been easy:
Mr. Spicer is serving double duty as communications director, a key
planning position, in addition to engaging in day-to-day combat with the
news media. Mr. Trump, several aides said, is used to quarterbacking his
own media strategy, and did not see the value of hiring an outsider.

An early plan was to give the communications job to Kellyanne Conway,
his former campaign manager and top TV surrogate, but the demands of the
job would have conflicted with Ms. Conway's other duties as a free-range
adviser to Mr. Trump with Oval Office walk-in privileges, according to
one aide.

Mr. Trump remains intensely focused on his brand, but the demands of the
job mean he spends less time monitoring the news media --- although he
recently upgraded the flat-screen TV in his private dining room so he
can watch the news while eating lunch.

He often has to wait until the end of the workday before grinding
through news clips with Mr. Spicer, marking the ones he does not like
with a big arrow in black Sharpie --- though he almost always makes time
to monitor Mr. Spicer's performance at the daily briefings, summoning
him to offer praise or criticism, a West Wing aide said.

Visitors to the Oval Office say Mr. Trump is obsessed with the décor ---
it is both a totem of a victory that validates him as a serious person
and an image-burnishing backdrop --- so he has told his staff to
schedule as many televised events in the room as possible.

To pass the time between meetings, Mr. Trump gives quick tours to
visitors, highlighting little tweaks he has made after initially
expecting he would have to pay for them himself.

Flanking his desk are portraits of Presidents Thomas Jefferson and
Andrew Jackson. He will linger on the opulence of the newly hung golden
drapes, which he told a recent visitor were once used by Franklin D.
Roosevelt but in fact were patterned for Bill Clinton. For a man who
sometimes has trouble concentrating on policy memos, Mr. Trump was
delighted to page through a book that offered him 17 window covering
options.

Ultimately, this is very much the White House that Mr. Trump wanted to
build. But while the world reckons with the effect he is having on the
presidency, he is adjusting to the effect of the presidency on him. He
is now a public employee. And the only boss Mr. Trump ever had in his
life was his father, a hard-driving developer the president still treats
with deep reverence.

With most of his belongings in New York, the only family picture on the
shelf behind Mr. Trump's desk is a small black-and-white photograph of
that boss,
\href{https://www.nytimes3xbfgragh.onion/2016/08/13/us/politics/fred-donald-trump-father.html}{Frederick
Christ Trump}.

Advertisement

\protect\hyperlink{after-bottom}{Continue reading the main story}

\hypertarget{site-index}{%
\subsection{Site Index}\label{site-index}}

\hypertarget{site-information-navigation}{%
\subsection{Site Information
Navigation}\label{site-information-navigation}}

\begin{itemize}
\tightlist
\item
  \href{https://help.nytimes3xbfgragh.onion/hc/en-us/articles/115014792127-Copyright-notice}{©~2020~The
  New York Times Company}
\end{itemize}

\begin{itemize}
\tightlist
\item
  \href{https://www.nytco.com/}{NYTCo}
\item
  \href{https://help.nytimes3xbfgragh.onion/hc/en-us/articles/115015385887-Contact-Us}{Contact
  Us}
\item
  \href{https://www.nytco.com/careers/}{Work with us}
\item
  \href{https://nytmediakit.com/}{Advertise}
\item
  \href{http://www.tbrandstudio.com/}{T Brand Studio}
\item
  \href{https://www.nytimes3xbfgragh.onion/privacy/cookie-policy\#how-do-i-manage-trackers}{Your
  Ad Choices}
\item
  \href{https://www.nytimes3xbfgragh.onion/privacy}{Privacy}
\item
  \href{https://help.nytimes3xbfgragh.onion/hc/en-us/articles/115014893428-Terms-of-service}{Terms
  of Service}
\item
  \href{https://help.nytimes3xbfgragh.onion/hc/en-us/articles/115014893968-Terms-of-sale}{Terms
  of Sale}
\item
  \href{https://spiderbites.nytimes3xbfgragh.onion}{Site Map}
\item
  \href{https://help.nytimes3xbfgragh.onion/hc/en-us}{Help}
\item
  \href{https://www.nytimes3xbfgragh.onion/subscription?campaignId=37WXW}{Subscriptions}
\end{itemize}
