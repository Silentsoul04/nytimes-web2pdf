\href{/section/fashion}{Fashion}\textbar{}A Designer, and a Collection,
Inspired by an Immigrant Father

\url{https://nyti.ms/2kPbmJv}

\begin{itemize}
\item
\item
\item
\item
\item
\end{itemize}

\includegraphics{https://static01.graylady3jvrrxbe.onion/images/2017/02/12/fashion/09KERBY1-WEB/09KERBY1-WEB-articleLarge.jpg?quality=75\&auto=webp\&disable=upscale}

Sections

\protect\hyperlink{site-content}{Skip to
content}\protect\hyperlink{site-index}{Skip to site index}

\hypertarget{a-designer-and-a-collection-inspired-by-an-immigrant-father}{%
\section{A Designer, and a Collection, Inspired by an Immigrant
Father}\label{a-designer-and-a-collection-inspired-by-an-immigrant-father}}

Kerby Jean-Raymond, the founder of Pyer Moss, is known for his activism.
This season, he turned his attention close to home.

This season, Kerby Jean-Raymond, designer, drew inspiration from his
father, who arrived in New York from Haiti in 1980, and who raised Kerby
alone after his mother died.Credit...Sasha Arutyunova for The New York
Times

Supported by

\protect\hyperlink{after-sponsor}{Continue reading the main story}

By Valeriya Safronova

\begin{itemize}
\item
  Feb. 8, 2017
\item
  \begin{itemize}
  \item
  \item
  \item
  \item
  \item
  \end{itemize}
\end{itemize}

For those who want their fashion designers to be both creative and
political, Kerby Jean-Raymond is a case study.

The designer, who turned 30 last year, and founded his label in 2013,
first made waves with his spring 2016 show, which featured a short film
about race relations in the United States. Since then, he has
\href{https://www.nytimes3xbfgragh.onion/2015/01/22/fashion/after-a-tragedy-like-the-charlie-hebdo-shooting-come-the-products.html}{designed
T-shirts} that listed the names of victims of police brutality;
\href{https://www.nytimes3xbfgragh.onion/2016/02/18/fashion/erykah-badu-debut-new-york-fashion-week.html}{traveled
to North Dakota} with supplies for the Dakota Access Pipeline
protesters; and created a collection that
\href{https://www.nytimes3xbfgragh.onion/2016/09/13/fashion/new-york-fashion-week-hood-by-air-altuzarra.html}{explored
the theme of economic inequality}.

This season, Mr. Jean-Raymond drew inspiration from his father,
Jean-Claude Jean-Raymond, 59, who arrived in New York from Haiti in 1980
and who raised Kerby alone after his wife, Vania Moss-Pierre (Pyer Moss
is her namesake), died when his son was seven. The collection is titled
``My Father as I Remember, 1980-1999,'' and is the first part of a
multiseason series called ``Stories About My Parents.''

It is also a timely reminder of the importance of the melting pot in the
current conversation around immigration.

\includegraphics{https://static01.graylady3jvrrxbe.onion/images/2017/02/09/fashion/09KERBY2-WEB/09KERBY2-WEB-articleInline.jpg?quality=75\&auto=webp\&disable=upscale}

``My dad was always so strict that I was scared to speak to him,'' Mr.
Jean-Raymond said recently, driving through East Flatbush, Brooklyn,
where he grew up. ``Haitian parents are very, `This is adults' business;
this is kids' business.'''

At the apartment where his father has lived since 1982, Mr. Jean-Raymond
scrolled through his phone as Jean-Claude Jean-Raymond looked on. ``He
randomly sent me these pictures on WhatsApp a few months ago and I
started designing my collection around them,'' Mr. Jean-Raymond said,
scrolling through the messaging app. ``It's an ode to the cars that he
drove, the excessive amount of jewelry he wore.''

Image

Jean-Claude Jean-Raymond, left, and Kerby Jean-Raymond.Credit...Sasha
Arutyunova for The New York Times

``I grew up thinking my father was tacky,'' Mr. Jean-Raymond said.
``There was no color coordination. It was whatever was cool. `These
sweatpants are cool. I'll wear them with these shoes that are cool.'''
Only recently did Kerby begin to truly admire his father's style. ``He
had less inhibitions,'' he said. ``I wasn't respectful of his swag
then.''

When Jean-Claude Jean-Raymond would take Kerby and his stepbrother to a
store called Dr. Jay's to buy new clothes for school, Kerby always ran
to the back. ``My stepbrother would head for Phat Farm, FUBU, those
sections, and I would look for anything that was made in Italy,'' he
said. ``There wasn't much to pick from. That kind of stuff they kept
behind the counter.''

Image

Only recently did Mr. Jean-Raymond begin to truly admire his father's
style. ``He had less inhibitions,'' he said. ``I wasn't respectful of
his swag then.''Credit...Sasha Arutyunova for The New York Times

``I grew up in a different Flatbush,'' Mr. Jean-Raymond added. ``When we
went outside for recess there would be drug dealers in the yard. We used
to take milk crates and hang them on the fence to play basketball.''

His father said, ``There was an area nearby called Vietnam,'' referring
to a stretch of several blocks near Flatbush Avenue. ``Day or night, you
couldn't walk in this area. It was too dangerous. But the same area now,
there's a Starbucks.''

``I never sold drugs,'' Kerby Jean-Raymond said. ``There were times when
gangs would approach me, but my father was way stronger than them. They
would come make threats and stuff, and I was like: `You don't know the
opposition I've got upstairs. I'm not scared of you.'''

Image

Mr. Jean-Raymond says he is still focused on unraveling the complex
relationship he has with his father.Credit...Sasha Arutyunova for The
New York Times

He did, however, have a lifelong obsession with sneakers, financed
starting when he was 13 by an after-school job at a sneaker store on
Flatbush Avenue called Ragga Muffin (he told the hiring manager that he
was 15). ``The way I was raised, you get a new pair of sneakers when the
old one gets messed up,'' Mr. Jean-Raymond said. ``But when I got to
high school, I started dating girls and trying to fit in, and I realized
everybody was collecting Jordans. When I would get my paychecks, I
wouldn't even take money. I would just trade them for sneakers.''

``It's completely different,'' he said of today's Ragga Muffin. Still,
he credited the place with inspiring his career path. Seeing the
less-than-high-quality pieces that sold while he worked there, he
thought to himself: ``I can make something better.''

Image

When he was 13, Mr. Jean-Raymond got an after-school job at a sneaker
store on Flatbush Avenue called Ragga Muffin. He credited the place with
inspiring his career path.Credit...Sasha Arutyunova for The New York
Times

Mr. Jean-Raymond had always been an impatient, antsy child, and to
channel his energy, his homeroom teacher at the High School of Fashion
Industries in Manhattan had him work as an intern with the designer Kay
Unger, who became a mentor.

Though he attended Hofstra University and earned a degree in business
law and entrepreneurship, he continued to freelance for Marc Jacobs,
Theory, Kenneth Cole and others, helping with showroom preparation,
draping and pattern making. Then came Pyer Moss, now based in
Manhattan's garment district.

Image

Part of the collection: a hand-drawn patch featuring an image of Mr.
Jean-Raymond's father in his youth, with his nickname printed
below.Credit...Sasha Arutyunova for The New York Times

Back at the office, Mr. Jean-Raymond went over several
pieces-in-progress with his team. ``I'm taking classic cuts, but I'm
doing them in textures that remind me of him,'' he said, referring to
his father. A bright silver, furry jacket lay on a table. Nearby was a
sleek tracksuit in black and purple, and a large leather jacket
purposely left unfinished. Also part of the collection: a hand-drawn
patch featuring an image of his father in his youth with his nickname,
``Didi,'' printed below.

He picked up a dark burgundy and black coat, inspired by a Maison
Margiela style his father wore when Mr. Jean-Raymond was a child, and
put it on. Maison Magiela and Yohji Yamamoto are the only two labels Mr.
Jean-Raymond now wears besides his own.

Image

Mr. Jean-Raymond in a coat inspired by a Maison Margiela style his
father wore when he was a child.Credit...Sasha Arutyunova for The New
York Times

``It was the first time I saw the word Margiela,'' he said of the
original piece. ``I could tell you exactly what he would wear it with. A
cream suit and a Van Heusen shirt and brown corduroy pants.'' He did not
love the look, back in the day. ``Now, I think that's fly.''

After wrapping up at his studio, Mr. Jean-Raymond headed to Le Soleil
Restaurant on 10th Avenue for lunch.

Image

After wrapping up at his studio, Mr. Jean-Raymond headed to Le Soleil
Restaurant on 10th Avenue for lunch.Credit...Sasha Arutyunova for The
New York Times

The Haitian restaurant is owned by his uncle, Alphonse Duval, and his
aunt, Rolande Bissereth, who is also Mr. Jean-Raymond's godmother. What
his father did not provide in emotional connection she provided in
spades, he said. He spent weekends with her at her home in Queens and at
the restaurant, working behind the counter, cleaning up and serving
food. ``If she was here now, I wouldn't be sitting,'' he said.

Image

The Haitian restaurant is owned by his uncle, Alphonse Duval, and his
wife Rolande Bissereth, who is also Mr. Jean-Raymond's
godmother.Credit...Sasha Arutyunova for The New York Times

Mr. Jean-Raymond ordered a selection of Haitian dishes: plantains, fried
goat with onions, conch, rice and beans, deep-fried accra (a root
vegetable) and a side of spicy chili sauce. One of his future
collections in the ``Stories About My Parents'' series will be about his
godmother and his mother.

Image

One of Mr. Jean-Raymond's future collections in the ``Stories About My
Parents'' series will be about his godmother and his
mother.Credit...Sasha Arutyunova for The New York Times

For now, Mr. Jean-Raymond is still focused on unraveling the complex
relationship he has with his father. ``We've never had an open line of
communication,'' he said. ``I hope this shows him that I've cared the
whole time. I just didn't say anything.''

Advertisement

\protect\hyperlink{after-bottom}{Continue reading the main story}

\hypertarget{site-index}{%
\subsection{Site Index}\label{site-index}}

\hypertarget{site-information-navigation}{%
\subsection{Site Information
Navigation}\label{site-information-navigation}}

\begin{itemize}
\tightlist
\item
  \href{https://help.nytimes3xbfgragh.onion/hc/en-us/articles/115014792127-Copyright-notice}{©~2020~The
  New York Times Company}
\end{itemize}

\begin{itemize}
\tightlist
\item
  \href{https://www.nytco.com/}{NYTCo}
\item
  \href{https://help.nytimes3xbfgragh.onion/hc/en-us/articles/115015385887-Contact-Us}{Contact
  Us}
\item
  \href{https://www.nytco.com/careers/}{Work with us}
\item
  \href{https://nytmediakit.com/}{Advertise}
\item
  \href{http://www.tbrandstudio.com/}{T Brand Studio}
\item
  \href{https://www.nytimes3xbfgragh.onion/privacy/cookie-policy\#how-do-i-manage-trackers}{Your
  Ad Choices}
\item
  \href{https://www.nytimes3xbfgragh.onion/privacy}{Privacy}
\item
  \href{https://help.nytimes3xbfgragh.onion/hc/en-us/articles/115014893428-Terms-of-service}{Terms
  of Service}
\item
  \href{https://help.nytimes3xbfgragh.onion/hc/en-us/articles/115014893968-Terms-of-sale}{Terms
  of Sale}
\item
  \href{https://spiderbites.nytimes3xbfgragh.onion}{Site Map}
\item
  \href{https://help.nytimes3xbfgragh.onion/hc/en-us}{Help}
\item
  \href{https://www.nytimes3xbfgragh.onion/subscription?campaignId=37WXW}{Subscriptions}
\end{itemize}
