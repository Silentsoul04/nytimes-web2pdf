Sections

SEARCH

\protect\hyperlink{site-content}{Skip to
content}\protect\hyperlink{site-index}{Skip to site index}

\href{https://www.nytimes3xbfgragh.onion/section/books/review}{Book
Review}

\href{https://myaccount.nytimes3xbfgragh.onion/auth/login?response_type=cookie\&client_id=vi}{}

\href{https://www.nytimes3xbfgragh.onion/section/todayspaper}{Today's
Paper}

\href{/section/books/review}{Book Review}\textbar{}Which Canonical Work
is Frequently and Frustratingly Misread?

\url{https://nyti.ms/2lrmY5F}

\begin{itemize}
\item
\item
\item
\item
\item
\end{itemize}

Advertisement

\protect\hyperlink{after-top}{Continue reading the main story}

Supported by

\protect\hyperlink{after-sponsor}{Continue reading the main story}

\href{/column/bookends}{Bookends}

\hypertarget{which-canonical-work-is-frequently-and-frustratingly-misread}{%
\section{Which Canonical Work is Frequently and Frustratingly
Misread?}\label{which-canonical-work-is-frequently-and-frustratingly-misread}}

By Rivka Galchen and Benjamin Moser

\begin{itemize}
\item
  Feb. 17, 2017
\item
  \begin{itemize}
  \item
  \item
  \item
  \item
  \item
  \end{itemize}
\end{itemize}

\emph{In Bookends, two writers take on questions about the world of
books. This week, Rivka Galchen and Benjamin Moser discuss which classic
books are commonly misunderstood.}

\textbf{By Rivka Galchen}

\emph{I didn't pay much attention to the way that Don Quixote's
delusions often made others suffer.}

Image

Rivka GalchenCredit...Illustration by R. Kikuo Johnson

Two weeks before he died unexpectedly at age 53, my father mailed a copy
of Miguel de Cervantes's ``Don Quixote'' to my freshman dorm room. It
wasn't only because it was the last that I heard from him that the gift
felt coded and meaningful. I had often seen my father eat ice cream with
a fork. The alligator on his belt was always upside down --- not
sometimes, but always, and when I asked him why, he said this was
because, he thought, even though he tried to put it right side up, he
was left-handed. My father was intelligent, but I had nevertheless seen
him curse at wheeled luggage that he was dragging along on the unwheeled
side. His hearing and vision both were weak, which seemed to me as a
child a natural manifestation of his preference for what he used to call
(with an emotional inflection I couldn't ken) ``the life of the mind.''
For these and other reasons, it was unimaginable to me that this man had
driven to the Sooner Fashion Mall, gone in and located the B. Dalton
booksellers, bought the book, found his way to the post office, stood in
line, known my address.

I read the book.

Upon that first reading, I found the Don Quixote that is the Don Quixote
many people have found and loved: a hero precisely because he sees a
noble horse where others see a skinny old nag, because he puts a wash
basin on his head and sets out for battle as a knight errant, and even
after being left beaten at the side of the road, he sets out again.

I didn't, on that first reading, pay much attention to the way Don
Quixote's delusions often made others suffer. Thirsty mules can't drink
from their trough because Don Quixote insists it's a baptismal font;
Sancho Panza is roughed up after Quixote doesn't pay his hotel bill; and
on and on.

As any follower of stories might guess, my reading of the book kept
changing, just as my ``reading'' of the gift of the book changed, too.

A gentle, pragmatic friend of mine suggested that the magic of ``Don
Quixote'' was that even though our hero is an obvious madman, it is
still sad at the end of the book, when he repents of his life and goes
sane. The reader, my friend said, has finally been converted to
Quixote's vision, just when death makes him refute it.

Nabokov, whose lectures on ``Don Quixote'' are mostly taken as an
expression of distaste --- ``a cruel and crude old book,'' he once
called it --- suggested that Cervantes had made a character who exceeded
in quality the book that had birthed him.

At a later moment, the book seemed to me to be about what a power move
it is to be ``eccentric'' and how that eccentricity coerces others into
serving your fantasy. I found Quixotism in the world at times
irritating, and at times cruel, and instead I saw the heroism of Don
Quixote's friends and neighbors, the ``normal'' people. I thought with
irritation about the impossibility of a female Quixote. I admired the
scene Cervantes makes of the girl from the traditional ``pastoral''
story showing up to disagree with how her tale gets told. I even came
back around to seeing Don Quixote, often enough, as the gentle hero I
had originally read him to be. The most common misreading of ``Don
Quixote'' is to see it as offering a stable and summarizable truth.

``Don Quixote'' had seemed random when it arrived. But I now remember
that it wasn't so random. In my first few weeks at school, I had told my
parents that my intention was to major in ``Spanish literature.'' (I
didn't know that there was no such major.) My parents both pointed out
that I knew almost no Spanish, and also that ``Spanish literature''
didn't lead to gainful employment. My father --- not quite as fully
``the space cadet'' as we had labeled him, as I continue to label him
--- said that if I wanted to be a writer (had I said that?) that I
should, like Primo Levi, for example, become a chemist.

I can say now, at age 40, that I completely agree with my father's
outlook. I was converted too late to that vision. I finally see the real
ambivalence of the gift: Be someone who makes it to the post office.
Sometimes.

\emph{\textbf{Rivka Galchen}} \emph{is a recipient of a William J.
Saroyan International Prize for Fiction, a Rona Jaffe Foundation
Writers' Award and a Berlin Prize, among other distinctions. Her fiction
and nonfiction have appeared in numerous publications, including
Harper's and The New Yorker, which selected her for their list of ``20
Under 40'' American fiction writers in 2010. Her debut novel, the
critically acclaimed ``Atmospheric Disturbances,'' was published in
2008, and her second book, a story collection titled ``American
Innovations,'' in 2014. Her most recent book is ``Little Labors.''}

◆ ◆ ◆

\textbf{By Benjamin Moser}

\emph{The Bible doesn't offer a consistent view of much of anything.}

Image

Benjamin MoserCredit...Illustration by R. Kikuo Johnson

The Bible is about sex, to judge by America's public discourse. In my
lifetime, it has rarely been invoked except in discussions about gay
marriage, contraception or abortion. ``What did the Bible have to say
about issue X or issue Y?'' The questions are anachronistic, and the
answers known in advance, since the people asking them have almost
always been those so obsessed with other people's sex lives ---
especially when those other people were gay or female --- that this
seemed to be the whole point of their religion.

This emphasis makes the Bible seem like a book written by right-wing
scolds --- as, indeed, parts of it were. But since it isn't a single
book but many, it doesn't offer a consistent view of much of anything.
Still, though written by different authors, in different languages, in
different countries, over the course of different centuries, one
political theme nevertheless runs through the whole, at least the Hebrew
parts: the many ways a nation can be lost.

For instance: The prophets tell how a group of refugees were brought, by
divine providence, into a rich new land, and made a mighty nation; and
then, forgetting the principles of their foundation, grew bloated and
impious: ``They are waxen fat, they shine: yea, they overpass the deeds
of the wicked: they judge not the cause, the cause of the fatherless,
yet they prosper; and the right of the needy do they not judge''
(Jeremiah 5:28).

Their leaders were liars --- ``For the rich men thereof are full of
violence, and the inhabitants thereof have spoken lies, and their tongue
is deceitful in their mouth'' (Micah 6:12) --- and cruel: ``But thine
eyes and thine heart are not but for thy covetousness, and for to shed
innocent blood, and for oppression, and for violence, to do it''
(Jeremiah 22:17). And they will be paid in their own coin: ``I will not
turn away the punishment thereof; because they sold the righteous for
silver, and the poor for a pair of shoes'' (Amos 2:6).

The aggression a nation inflicts on others will be visited upon it in
turn: ``Because you have plundered many nations, all the remnant of the
peoples shall plunder you'' (Habakkuk 2:8). And through Obadiah a nation
is warned: ``I have made thee small among the heathen: thou art greatly
despised''; and when the unrighteous are ruined they will not even have
the consolation of pity, we read in Jeremiah: ``For who shall have pity
upon thee, O Jerusalem? Or who shall bemoan thee?''

In a nation's ruin, Zephaniah mocked their former arrogance: ``This is
the rejoicing city that dwelt without care, that said in her heart, `I
am, and there is none besides me.' '' No wall can protect them from the
lecherous rulers set over them, punishment for their sins: ``Art thou
better than populous No, that was situate among the rivers, that had the
waters round about it, whose rampart was the sea, and her wall was from
the sea?'' (Nahum 3:8.)

Barely a word, if any, about condoms, or abortion, or gay marriage. And
if the prophets are not all without hope --- Esther, for example, brings
low a vicious racist; and others hope that repentance, prayer and
charity can turn back the fateful decree --- few are optimistic about a
people fallen so far from the principles that had made them great. That
fall, Jeremiah marveled, was entirely unnecessary: ``Her sun is gone
down while it was yet day.''

Empires fall, and usually deserve to: This is not a message with much
purchase among American politicians, especially those who most
ostentatiously flaunt their faith. If they read the Bible, they would
find the story, on page after page after page, of the predictable fate
of nations that abandon their covenants: ``Destruction upon destruction
is cried; for the whole land is spoiled; suddenly are my tents spoiled,
and my curtains in a moment.''

\emph{\textbf{Benjamin Moser}} \emph{is the author of ``Why This World:
A Biography of Clarice Lispector,'' a finalist for the National Book
Critics' Circle Award, and the general editor of the new translations of
Clarice Lispector at New Directions. A former New Books columnist at
Harper's Magazine, he is currently writing the authorized biography of
Susan Sontag. He lives in the Netherlands.}

Advertisement

\protect\hyperlink{after-bottom}{Continue reading the main story}

\hypertarget{site-index}{%
\subsection{Site Index}\label{site-index}}

\hypertarget{site-information-navigation}{%
\subsection{Site Information
Navigation}\label{site-information-navigation}}

\begin{itemize}
\tightlist
\item
  \href{https://help.nytimes3xbfgragh.onion/hc/en-us/articles/115014792127-Copyright-notice}{©~2020~The
  New York Times Company}
\end{itemize}

\begin{itemize}
\tightlist
\item
  \href{https://www.nytco.com/}{NYTCo}
\item
  \href{https://help.nytimes3xbfgragh.onion/hc/en-us/articles/115015385887-Contact-Us}{Contact
  Us}
\item
  \href{https://www.nytco.com/careers/}{Work with us}
\item
  \href{https://nytmediakit.com/}{Advertise}
\item
  \href{http://www.tbrandstudio.com/}{T Brand Studio}
\item
  \href{https://www.nytimes3xbfgragh.onion/privacy/cookie-policy\#how-do-i-manage-trackers}{Your
  Ad Choices}
\item
  \href{https://www.nytimes3xbfgragh.onion/privacy}{Privacy}
\item
  \href{https://help.nytimes3xbfgragh.onion/hc/en-us/articles/115014893428-Terms-of-service}{Terms
  of Service}
\item
  \href{https://help.nytimes3xbfgragh.onion/hc/en-us/articles/115014893968-Terms-of-sale}{Terms
  of Sale}
\item
  \href{https://spiderbites.nytimes3xbfgragh.onion}{Site Map}
\item
  \href{https://help.nytimes3xbfgragh.onion/hc/en-us}{Help}
\item
  \href{https://www.nytimes3xbfgragh.onion/subscription?campaignId=37WXW}{Subscriptions}
\end{itemize}
