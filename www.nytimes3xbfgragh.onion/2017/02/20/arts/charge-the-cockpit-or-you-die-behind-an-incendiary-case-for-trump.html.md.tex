Sections

SEARCH

\protect\hyperlink{site-content}{Skip to
content}\protect\hyperlink{site-index}{Skip to site index}

\href{https://www.nytimes3xbfgragh.onion/section/arts}{Arts}

\href{https://myaccount.nytimes3xbfgragh.onion/auth/login?response_type=cookie\&client_id=vi}{}

\href{https://www.nytimes3xbfgragh.onion/section/todayspaper}{Today's
Paper}

\href{/section/arts}{Arts}\textbar{}`Charge the Cockpit or You Die':
Behind an Incendiary Case for Trump

\url{https://nyti.ms/2loqxrc}

\begin{itemize}
\item
\item
\item
\item
\item
\end{itemize}

Advertisement

\protect\hyperlink{after-top}{Continue reading the main story}

Supported by

\protect\hyperlink{after-sponsor}{Continue reading the main story}

\hypertarget{charge-the-cockpit-or-you-die-behind-an-incendiary-case-for-trump}{%
\section{`Charge the Cockpit or You Die': Behind an Incendiary Case for
Trump}\label{charge-the-cockpit-or-you-die-behind-an-incendiary-case-for-trump}}

\includegraphics{https://static01.graylady3jvrrxbe.onion/images/2017/02/21/arts/21CLAREMONT2/21CLAREMONT2-articleLarge.jpg?quality=75\&auto=webp\&disable=upscale}

By
\href{https://www.nytimes3xbfgragh.onion/by/jennifer-schuessler}{Jennifer
Schuessler}

\begin{itemize}
\item
  Feb. 20, 2017
\item
  \begin{itemize}
  \item
  \item
  \item
  \item
  \item
  \end{itemize}
\end{itemize}

NEWPORT BEACH, Calif. --- At most dinner galas, the presence of a
Supreme Court justice would be reason enough to crow. But some of the
500 supporters of the \href{http://www.claremont.org/}{Claremont
Institute} who gathered here on Feb. 11 to see Justice Samuel A. Alito
Jr. accept a statesmanship award also had the group's links to an even
more powerful eminence on their minds.

``Many Claremonsters have the ear of this administration and may help
Trump take what he feels in his gut and migrate it to his head,'' Thomas
D. Klingenstein, the institute's board chairman, declared from the
stage, repurposing an old insult as a badge of honor.

``This is Claremont's moment,'' he continued. ``This is a time to judge
Claremont by its press. The more bad press, the better.''

Mr. Klingenstein was referring to the continuing furor around
\href{http://www.claremont.org/crb/basicpage/the-flight-93-election/}{``The
Flight 93 Election,''} an incendiary pro-Trump polemic that appeared in
September on the website of The Claremont Review of Books, the
institute's flagship publication. Published under the pseudonym Publius
Decius Mus, the essay compared the American republic to a hijacked
airliner, with a vote for Donald J. Trump as the risky, but
existentially necessary, course.

Decius' apocalyptic vision --- ``Charge the cockpit or you die'' ---
stirred intense rebuttals from the overwhelmingly anti-Trump
conservative intellectual establishment. Then The Weekly Standard
\href{http://www.weeklystandard.com/the-anonymous-pro-trump-decius-now-works-inside-the-white-house/article/2006623}{revealed}
that Decius was Michael Anton, a senior staff member at the National
Security Council, and a news media stampede was on.

The Intercept called his writings the ``intellectual source code of
Trumpism.'' Salon put him alongside Stephen K. Bannon and Stephen Miller
in the administration's ``white nationalist `genius bar,''' while the
conservative writer (and staunch Never-Trumper) William Kristol, writing
on Twitter, compared him to the Nazi theorist Carl Schmitt.

It certainly added up to a publicity coup for a small West Coast
institute known for summer seminars at which young conservatives immerse
themselves in the Federalist Papers and other classics of American
political thought. Suddenly, The Claremont Review, an erudite journal
with a mere 13,000 subscribers, was being hailed as the bible of
highbrow Trumpism --- ``crucially important,'' as the journalist Damon
Linker
\href{http://theweek.com/articles/678310/why-many-conservative-intellectuals-became-trumpists}{wrote},
``for anyone seeking to understand the evolution of the Republican and
conservative movement.''

\includegraphics{https://static01.graylady3jvrrxbe.onion/images/2017/02/21/arts/21CLAREMEONT1/21CLAREMEONT1-articleLarge.jpg?quality=75\&auto=webp\&disable=upscale}

Not that the journal is pro-Trump, mind you. Charles R. Kesler, the
editor, said that he had sought out ``robust debate,'' publishing some
Never-Trumpers alongside pro-Trumpers and those who call themselves
\href{http://www.claremont.org/crb/basicpage/the-reason-im-anti-anti-trump/}{``anti-anti-Trump.''}

The institute's president and chief executive, Michael Pack, also noted
that Claremont's affiliates --- who include the former Bush
administration lawyer John Yoo (a
\href{https://www.nytimes3xbfgragh.onion/2017/02/06/opinion/executive-power-run-amok.html}{strong
Trump critic}) and Senator Tom Cotton, a Republican from Arkansas (``one
of Trumpism's leading voices,'' as a headline in The Washington Post
recently
\href{https://www.washingtonpost.com/news/the-fix/wp/2017/02/07/how-tom-cotton-emerged-as-one-of-trumpisms-leading-voices/?utm_term=.df69e741f87a}{put
it}) --- were of many minds about the new president.

``The Claremont Institute stands for deep, serious thinking about
American founding principles,'' Mr. Pack said. ``We are not simply in
the partisan fight.''

But some Claremont-watchers take a darker view, saying the institute's
intellectual principles have been, to continue the aviation metaphor,
left on the runway.

``They have completely abandoned every principle on which they stood and
have endorsed a man who seems to have no interest in or knowledge of the
Constitution,'' Steven B. Smith, a political science professor at Yale,
said. ``Even though Trump may be completely unhinged, the idea is they
can nevertheless kind of slipstream behind him and smuggle their agenda
in.''

\hypertarget{from-leo-strauss-to-the-beach-boys}{%
\subsection{From Leo Strauss to the Beach
Boys}\label{from-leo-strauss-to-the-beach-boys}}

The Claremont mission, according to the institute's website, is to
``defeat progressivism'' and ``restore the principles of the American
Founding to their rightful, pre-eminent place in American life.''

But Mr. Kesler's office on the manicured campus of Claremont McKenna
College in Claremont, Calif., where he is a professor of government,
could not seem more distant from the corridors of power, let alone the
chaos of actually existing Trumpism. (The institute, founded in 1979, is
not affiliated with the college.)

The décor is standard-issue professor, down to the desk piled with books
relating to American political thought, including his own edition of the
\href{http://www.h-net.org/reviews/showrev.php?id=4621}{Federalist
Papers} --- the top-selling edition in the country, he noted --- and an
anthology of Woodrow Wilson's political writings.

Wilson, as it happens, is a favorite Claremont bête noire: the founder
of the modern ``administrative state,'' the unelected, unaccountable
rule-by-bureaucracy that has, the story goes, usurped the founders'
vision of rule by the people (and which reached its apogee with Barack
Obama, Mr. Kesler argued in his 2012 book,
\href{http://www.nytimes3xbfgragh.onion/2012/09/30/books/review/the-great-disconnect.html}{``I
Am the Change''}).

But Mr. Kesler, an unfailingly genial West Virginian, doesn't seem to
mind Wilson's intellectual company. ``I actually like Wilson,'' he said.
``Well, not politically.''

He takes a similarly cheerful view of the ruckus over the essay. He
brought up Mr. Kristol's ``intemperate'' tweet, but noted that his old
friend had recently come to the college for a conference about an essay
by the political philosopher Leo Strauss.

``A secret Straussian conclave!'' Mr. Kesler said with a laugh, then
paused. ``Don't call it that.''

Strauss's intensely close readings of Plato, Maimonides and Machiavelli
can seem remote from contemporary politics. But during the Bush
administration, much ink was spilled over the
\href{http://www.nytimes3xbfgragh.onion/2003/05/04/weekinreview/the-nation-leo-cons-a-classicist-s-legacy-new-empire-builders.html}{subterranean
influence} of Straussians like Mr. Kristol, who championed the war in
Iraq. And now some see the Claremont crowd's rising profile as revenge
of the so-called West Coast Straussians, as acolytes of
\href{https://www.nytimes3xbfgragh.onion/2015/01/12/us/politics/harry-v-jaffa-conservative-scholar-and-goldwater-muse-dies-at-96.html?_r=0}{Harry
V. Jaffa} --- the Claremont McKenna professor and Claremont Institute
patriarch who died in 2015 --- are known.

The Straussian lineages, and their
\href{http://www.politico.com/magazine/story/2016/11/conservative-intellectuals-trump-2016-politics-policy-214424}{fierce
schisms}, are notoriously complex. But Mr. Kesler, who studied under the
Straussian Harvey C. Mansfield at Harvard but came to Claremont in 1983,
gamely summed up the West Coast view as hinging on a more optimistic
take on America.

``The East Coast view was that America was a Lockean nation, purely
modern, based on radically individual and almost selfish rights: your
life, your liberty, your property,'' he said.

But Mr. Jaffa, a deep student of
\href{http://press.uchicago.edu/ucp/books/book/chicago/C/bo6950926.html}{Abraham
Lincoln}, ``thought that America was a heroic country,'' Mr. Kesler
continued. ``Not always, maybe only when it had to be. But it could
be.''

The tension between Athens (reason) and Jerusalem (faith) was one of
Strauss's great themes. But reconciling Athens and Mar-a-Lago may be a
challenge of an entirely different order.

Mr. Yoo, now a law professor at the University of California at
Berkeley, recalled with amusement a dinner last summer with a dozen
Claremonters at a Philadelphia restaurant famous for singing waiters,
where he was startled to realize that many of his Aristotle-quoting
tablemates were pro-Trump.

``Between arias, I had to listen to these very distinguished scholars go
on about the great intellectual virtues of Donald Trump,'' he said. ``It
was hard to keep down my wonderful Sicilian meal.''

Polemics like ``The Flight 93 Election'' represent something of a
departure for The Claremont Review, founded in 2000 as a conservative,
if eclectic, answer to The New York Review of Books. Previous
contributions by Mr. Anton, a former Jaffa student, include meditations
on French cooking, several articles about Tom Wolfe and perhaps the only
\href{http://www.claremont.org/crb/article/paradise-lost-and-regained/}{essay}
on the Beach Boys to cite both Brian Wilson and the social scientist
James Q. Wilson.

When Mr. Anton submitted an earlier essay written as Decius, Mr. Kesler
rejected it. ``It wasn't as well argued as I would have liked,'' he
said. (It was published in March, under the title ``Toward a Sensible,
Coherent Trumpism,'' by The Journal of American Greatness, a website
that has since closed and been reborn as
\href{https://amgreatness.com/}{American Greatness}.)

Decius' main target in the ``Flight 93'' essay was the fat-and-happy
conservative establishment --- Conservatism Inc. --- which had failed to
stand up to ``the ceaseless importation of third world foreigners,''
among other ills. Mr. Kesler said that he found the tone of apocalyptic
urgency exaggerated but compared the essay to the clarion call of Thomas
Paine's ``Common Sense.''

``Anton was trying to wake people up, and he did,'' Mr. Kesler said. ``I
was very happy to publish it. But I was also happy to publish others who
added some
\href{http://www.claremont.org/crb/basicpage/trump-and-prudence-a-reply-to-decius/}{caveats}.''

\hypertarget{anti-anti-trump-or-pro-trump}{%
\subsection{Anti-Anti-Trump or
Pro-Trump?}\label{anti-anti-trump-or-pro-trump}}

``The Flight 93 Election'' included some jabs at Mr. Kesler's more
cautious anti-anti-Trump position. Since then, he has moved in a more
clearly pro- direction.

In an essay in the latest issue of The Claremont Review, Mr. Kesler
calls Mr. Trump a ``common-sense conservative'' whose views on trade,
immigration and foreign policy represent a return to the pre-New Deal
Republican Party --- more Calvin Coolidge 2.0 than the tyrannical Caesar
``The Federalist Papers'' warned against.

Mr. Trump lacks a tyrant's ``dark political soul,'' Mr. Kesler said.
``He's more of a bull in a china shop.''

And what about charges that the bull is trampling the Constitution? Mr.
Kesler offered a perspective that was half wait-and-see, half
told-you-so.

``It's always amusing but heartening to hear liberals talk about the
separation of powers, the independent judiciary and other inhibitions on
executive action,'' he said. ``When you're the out-party, you discover
the utility of these constitutional protections.''

``Of course,'' he added, ``I'd be happier if they discovered not just
their utility but their nobility as well.''

Advertisement

\protect\hyperlink{after-bottom}{Continue reading the main story}

\hypertarget{site-index}{%
\subsection{Site Index}\label{site-index}}

\hypertarget{site-information-navigation}{%
\subsection{Site Information
Navigation}\label{site-information-navigation}}

\begin{itemize}
\tightlist
\item
  \href{https://help.nytimes3xbfgragh.onion/hc/en-us/articles/115014792127-Copyright-notice}{©~2020~The
  New York Times Company}
\end{itemize}

\begin{itemize}
\tightlist
\item
  \href{https://www.nytco.com/}{NYTCo}
\item
  \href{https://help.nytimes3xbfgragh.onion/hc/en-us/articles/115015385887-Contact-Us}{Contact
  Us}
\item
  \href{https://www.nytco.com/careers/}{Work with us}
\item
  \href{https://nytmediakit.com/}{Advertise}
\item
  \href{http://www.tbrandstudio.com/}{T Brand Studio}
\item
  \href{https://www.nytimes3xbfgragh.onion/privacy/cookie-policy\#how-do-i-manage-trackers}{Your
  Ad Choices}
\item
  \href{https://www.nytimes3xbfgragh.onion/privacy}{Privacy}
\item
  \href{https://help.nytimes3xbfgragh.onion/hc/en-us/articles/115014893428-Terms-of-service}{Terms
  of Service}
\item
  \href{https://help.nytimes3xbfgragh.onion/hc/en-us/articles/115014893968-Terms-of-sale}{Terms
  of Sale}
\item
  \href{https://spiderbites.nytimes3xbfgragh.onion}{Site Map}
\item
  \href{https://help.nytimes3xbfgragh.onion/hc/en-us}{Help}
\item
  \href{https://www.nytimes3xbfgragh.onion/subscription?campaignId=37WXW}{Subscriptions}
\end{itemize}
