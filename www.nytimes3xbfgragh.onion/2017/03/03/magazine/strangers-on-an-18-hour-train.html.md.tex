Sections

SEARCH

\protect\hyperlink{site-content}{Skip to
content}\protect\hyperlink{site-index}{Skip to site index}

\href{https://myaccount.nytimes3xbfgragh.onion/auth/login?response_type=cookie\&client_id=vi}{}

\href{https://www.nytimes3xbfgragh.onion/section/todayspaper}{Today's
Paper}

Strangers on an 18-Hour Train

\url{https://nyti.ms/2llfuCs}

\begin{itemize}
\item
\item
\item
\item
\item
\item
\end{itemize}

Advertisement

\protect\hyperlink{after-top}{Continue reading the main story}

Supported by

\protect\hyperlink{after-sponsor}{Continue reading the main story}

\href{/column/lives}{Lives}

\hypertarget{strangers-on-an-18-hour-train}{%
\section{Strangers on an 18-Hour
Train}\label{strangers-on-an-18-hour-train}}

\includegraphics{https://static01.graylady3jvrrxbe.onion/images/2017/03/05/magazine/05lives/05mag-05lives-t_CA0-articleLarge.jpg?quality=75\&auto=webp\&disable=upscale}

By Rafiq Ebrahim

\begin{itemize}
\item
  March 3, 2017
\item
  \begin{itemize}
  \item
  \item
  \item
  \item
  \item
  \item
  \end{itemize}
\end{itemize}

Many years ago, before my family immigrated to the United States from
Pakistan, we used to travel frequently by train. During a more recent
trip home, instead of flying back to Karachi from Lahore, I decided to
go by train again. I was interested to see what it was like traveling in
economy class. I got a seat in Car No. 3, and amid thunder and rain, the
train hissed out of the station. Its pace was slow. The compartment was
packed.

Those who had reserved seats occupied them, while others were perched on
the floor, next to the seats or even by the toilets, with the result
that it became difficult to move around or to use the toilet yourself.
Families with little children spread quilts and pillows on the seats and
on the floor.

A conductor entered the car and started checking tickets. The passenger
in Seat 54, a tall, middle-aged man with sharp gray eyes, had only an
unreserved ticket, so he was asked to vacate the seat. But he took the
conductor aside and returned in a few minutes to the same place. The
conductor, overlooking the mess everywhere in the car, smiled and went
about his duty.

If it were not for the rains in Punjab, the heat and dust would have
been unbearable. I noticed that the man sitting in Seat 54 kept watching
a young woman in a window seat with a little child on her lap. The
woman's eye fell on the man's face, and she immediately looked down and
adjusted her dupatta, her scarf.

The night wore on, and people began to close their eyes, but the seats
were so uncomfortable that only a very heavy sleeper could manage to get
any rest. The train continued its slow pace, stopping every so often at
another station. Because of the heat and suffocating air in the
compartment, many windows were kept open. The woman with the child on
her lap looked over at the man in Seat 54. He was still staring at her.
I was beginning to get angry with him. Even under such filthy and
uncomfortable circumstances, he couldn't resist indulging his desire to
gaze at an attractive woman. She began to look back at him with fire in
her eyes.

Turning her face away, she played with the child again for a while. The
train was approaching a station. I could see the familiar lights of
Khanewal, and as we stopped, a memory flashed through my mind. Two
decades earlier, whenever we traveled this route and stopped at this
station, my little daughter would urge me to take her out and buy some
ceramic toys from one of the stalls. We would buy the toys, and I would
enjoy a cup of tea in a clay cup at a tea stall. It was now 2 a.m., and
I got down from the train to recapture this pleasant memory. I was
drinking my tea when two burly men came near me and stood on either
side. I could feel something probing at my right. ``Take out your wallet
and give it to me,'' ordered one of them. I took it out and handed it to
him. The other man relieved me of my wristwatch. ``Have a safe
journey,'' one said before they both disappeared.

I was shaken, and not just because I lost my wallet and watch. The watch
was cheap, bought for four dollars from Kmart in Chicago, and there were
only a hundred rupees in my wallet; the rest of my money, my credit
cards and my IDs were safely tucked inside my shoes. But it struck me
that you should never try to recapture the memorable scenes of the past,
because you are likely to lose those memories forever. I finished the
cup of tea and returned to my seat on the train.

The train started again. The child was still awake on his mother's lap,
but the woman found it difficult to keep her eyes open. She was soon
lost in a short wink of sleep. Her head fell forward. A moment later,
the child began to climb the open window --- one leg went over it. The
man in Seat 54 leapt up and grabbed the child before he fell out.

The commotion woke up the woman. She seemed to be in a panic, and then
reality dawned. ``Here you are,'' the man said as he gave the child back
to her. ``Your child has been looking for an opportunity to crawl out of
the window,'' he said. ``That's why I have been watching the whole
time.'' He stretched his back and moved away. The woman was dumbfounded,
and so was I.

The woman had a few sips of water, then got up to thank the man, but he
was nowhere to be seen. The train moved on. Early in the morning, at
Drigh Road Station, the woman got up to get off the train. She searched
for the man again but couldn't find him.

``Bhai,'' she addressed me, calling me brother. ``If you see that man,
will you kindly thank him on my behalf?''

I nodded. Before I got down at Karachi Cantt, I searched the whole
compartment for him --- but he was gone.

Advertisement

\protect\hyperlink{after-bottom}{Continue reading the main story}

\hypertarget{site-index}{%
\subsection{Site Index}\label{site-index}}

\hypertarget{site-information-navigation}{%
\subsection{Site Information
Navigation}\label{site-information-navigation}}

\begin{itemize}
\tightlist
\item
  \href{https://help.nytimes3xbfgragh.onion/hc/en-us/articles/115014792127-Copyright-notice}{©~2020~The
  New York Times Company}
\end{itemize}

\begin{itemize}
\tightlist
\item
  \href{https://www.nytco.com/}{NYTCo}
\item
  \href{https://help.nytimes3xbfgragh.onion/hc/en-us/articles/115015385887-Contact-Us}{Contact
  Us}
\item
  \href{https://www.nytco.com/careers/}{Work with us}
\item
  \href{https://nytmediakit.com/}{Advertise}
\item
  \href{http://www.tbrandstudio.com/}{T Brand Studio}
\item
  \href{https://www.nytimes3xbfgragh.onion/privacy/cookie-policy\#how-do-i-manage-trackers}{Your
  Ad Choices}
\item
  \href{https://www.nytimes3xbfgragh.onion/privacy}{Privacy}
\item
  \href{https://help.nytimes3xbfgragh.onion/hc/en-us/articles/115014893428-Terms-of-service}{Terms
  of Service}
\item
  \href{https://help.nytimes3xbfgragh.onion/hc/en-us/articles/115014893968-Terms-of-sale}{Terms
  of Sale}
\item
  \href{https://spiderbites.nytimes3xbfgragh.onion}{Site Map}
\item
  \href{https://help.nytimes3xbfgragh.onion/hc/en-us}{Help}
\item
  \href{https://www.nytimes3xbfgragh.onion/subscription?campaignId=37WXW}{Subscriptions}
\end{itemize}
