Sections

SEARCH

\protect\hyperlink{site-content}{Skip to
content}\protect\hyperlink{site-index}{Skip to site index}

\href{https://www.nytimes3xbfgragh.onion/section/business}{Business}

\href{https://myaccount.nytimes3xbfgragh.onion/auth/login?response_type=cookie\&client_id=vi}{}

\href{https://www.nytimes3xbfgragh.onion/section/todayspaper}{Today's
Paper}

\href{/section/business}{Business}\textbar{}David Rockefeller,
Philanthropist and Head of Chase Manhattan, Dies at 101

\url{https://nyti.ms/2mHOusA}

\begin{itemize}
\item
\item
\item
\item
\item
\end{itemize}

Advertisement

\protect\hyperlink{after-top}{Continue reading the main story}

Supported by

\protect\hyperlink{after-sponsor}{Continue reading the main story}

\hypertarget{david-rockefeller-philanthropist-and-head-of-chase-manhattan-dies-at-101}{%
\section{David Rockefeller, Philanthropist and Head of Chase Manhattan,
Dies at
101}\label{david-rockefeller-philanthropist-and-head-of-chase-manhattan-dies-at-101}}

\href{https://www.nytimes3xbfgragh.onion/slideshow/2017/03/20/business/in-his-own-words-an-interesting-life.html}{}

\hypertarget{in-his-own-words-an-interesting-life}{%
\subsection{In His Own Words, `an Interesting
Life'}\label{in-his-own-words-an-interesting-life}}

15 Photos

View Slide Show ›

\includegraphics{https://static01.graylady3jvrrxbe.onion/images/2013/08/10/nyregion/Rockefeller-adv-obit-slide-16Q8/Rockefeller-adv-obit-slide-16Q8-articleLarge.jpg?quality=75\&auto=webp\&disable=upscale}

Michael Evans/The New York Times

By Jonathan Kandell

\begin{itemize}
\item
  March 20, 2017
\item
  \begin{itemize}
  \item
  \item
  \item
  \item
  \item
  \end{itemize}
\end{itemize}

David Rockefeller, the banker and philanthropist with the fabled family
name who controlled Chase Manhattan bank for more than a decade and
wielded vast influence around the world for even longer as he spread the
gospel of American capitalism, died on Monday morning at his home in
Pocantico Hills, N.Y. He was 101.

His son David Jr. confirmed the death.

Chase Manhattan had long been known as the Rockefeller bank, although
the family never owned more than 5 percent of its shares. But Mr.
Rockefeller was more than a steward. As chairman and chief executive
throughout the 1970s, he made it ``David's bank,'' as many called it,
expanding its operations internationally.

His stature was greater than any corporate title might convey, however.
His influence was felt in Washington and foreign capitals, in the
corridors of New York City government, in art museums, in great
universities and in public schools.

Mr. Rockefeller could well be the last of a less and less visible family
to have cut so imposing a figure on the world stage. As a peripatetic
advocate of the economic interests of the United States and of his own
bank, he was a force in global financial affairs and in his country's
foreign policy. He was received in foreign capitals with the honors
accorded a chief of state.

He was the last surviving grandson of John D. Rockefeller, the tycoon
who founded the Standard Oil Company in the 19th century and built a
fortune that made him America's first billionaire and his family one of
the richest and most powerful in the nation's history.

As an heir to that legacy, David Rockefeller lived all his life in
baronial splendor and privilege, whether in Manhattan (when he was a
boy, he and his brothers would roller skate along Fifth Avenue trailed
by a limousine in case they grew tired) or at his magnificent country
estates.

Imbued with the understated manners of the East Coast elite, he loomed
large in the upper reaches of a New York social world of glittering
black-tie galas. His philanthropy was monumental, and so was his art
collection, a museumlike repository of some 15,000 pieces, many of them
masterpieces, some lining the walls of his offices 56 floors above the
streets at Rockefeller Center, to which he repaired, robust and active,
well into his 90s.

In silent testimony to his power and reach was his Rolodex, a catalog of
some 150,000 names of people he had met as a banker-statesman. It
required a room of its own beside his office.

Spread out below that corporate aerie was a city he loved and influenced
mightily. He was instrumental in rallying the private sector to help
resolve New York City's fiscal crisis in the mid-1970s. As chairman of
the Museum of Modern Art for many years --- his mother had helped found
it in 1929 --- he led an effort to encourage corporations to buy and
display art in their office buildings and to subsidize local museums.
And as chairman of the New York City Partnership, a coalition of
business executives, he fostered innovation in public schools and the
development of thousands of apartments for lower-income and middle-class
families.

He was always aware of the mystique surrounding the Rockefeller name.

``I have never found it a hindrance,'' he once said with typical
reserve. ``Obviously, there are times when I'm aware that I'm treated
differently. There's no question that having financial resources, which,
thanks to my parents, I learned to use with some restraint and
discretion, is a big advantage.''

\hypertarget{an-ambassador-for-business}{%
\subsection{An Ambassador for
Business}\label{an-ambassador-for-business}}

With his powerful name and his zeal for foreign travel --- he was still
going to Europe into his late 90s --- Mr. Rockefeller was a formidable
marketing force. In the 1970s, his meetings with Anwar el-Sadat of
Egypt, Leonid Brezhnev of the Soviet Union and Zhou Enlai of China
helped Chase Manhattan become the first American bank with operations in
those countries.

``Few people in this country have met as many leaders as I have,'' he
said.

Some faulted him for spending so much time abroad. He was accused of
neglecting his responsibilities at Chase and failing to promote
aggressive, visionary managers. Under his leadership, Chase fell far
behind its rival Citibank, then the nation's largest bank, in assets and
earnings. There were years when Chase had the most troubled loan
portfolio among major American banks.

\includegraphics{https://static01.graylady3jvrrxbe.onion/images/2017/03/20/obituaries/21ROCKEFELLER1/21ROCKEFELLER1-articleInline.jpg?quality=75\&auto=webp\&disable=upscale}

``In my judgment, he will not go down in history as a great banker,''
John J. McCloy, a Rockefeller friend and himself a former Chase
chairman, told The Associated Press in 1981. ``He will go down as a real
personality, as a distinguished and loyal member of the community.''

Mr. Rockefeller's forays into international politics also drew
criticism, notably in 1979, when he and former Secretary of State Henry
A. Kissinger persuaded President Jimmy Carter to admit the recently
deposed shah of Iran into the United States for cancer treatment. The
shah's arrival in New York enraged revolutionary followers of Ayatollah
Ruhollah Khomeini, provoking them to seize the United States Embassy in
Iran and hold American diplomats hostage for more than a year. Mr.
Rockefeller was also assailed for befriending autocratic foreign leaders
in an effort to establish and expand his bank's presence in their
countries.

``He spent his life in the club of the ruling class and was loyal to
members of the club, no matter what they did,'' the New York Times
columnist David Brooks wrote in 2002, citing the profitable deals Mr.
Rockefeller had cut with ``oil-rich dictators,'' ``Soviet party bosses''
and ``Chinese perpetrators of the Cultural Revolution.''

Still, presidents as ideologically different as Mr. Carter and Richard
M. Nixon offered him the post of Treasury secretary. He turned them both
down.

After the death in 1979 of his older brother Nelson A. Rockefeller, the
former vice president and four-term governor of New York, David
Rockefeller stood almost alone as a member of the family with an outsize
national profile. Only Jay Rockefeller, a great-grandson of John D.
Rockefeller, had earned prominence, as a governor and United States
senator from West Virginia. No one from the family's younger generations
has attained or perhaps aspired to David Rockefeller's stature.

``No one can step into his shoes,'' Warren T. Lindquist, a longtime
friend, told The Times in 1995, ``not because they aren't good, smart,
talented people, but because it's just a different world.''

\hypertarget{a-privileged-life}{%
\subsection{A Privileged Life}\label{a-privileged-life}}

The youngest of six siblings, David Rockefeller was born in Manhattan on
June 12, 1915. His father, John D. Rockefeller Jr., the only son of the
oil titan, devoted his life to philanthropy. His mother, Abby Aldrich
Rockefeller, was the daughter of Nelson Aldrich, a wealthy senator from
Rhode Island.

Besides Nelson, born in 1908, the other children were Abby, who was born
in 1903 and died in 1976 after leading a private life; John D.
Rockefeller III, who was born in 1906 and immersed himself in
philanthropy until his death in an automobile accident in 1978;
Laurance, born in 1910, who was an environmentalist and died in 2004;
and Winthrop, born in 1912, who was governor of Arkansas and died in
1973.

David grew up in a mansion at 10 West 54th Street, the largest private
residence in the city at the time. It bustled with valets, parlor maids,
nurses and chambermaids. For dinner every night, his father dressed in
black tie and his mother in a formal gown.

Summers were spent at the 107-room Rockefeller ``cottage'' in Seal
Harbor, Me., and weekends at Kykuit, the family's country compound north
of New York City in Tarrytown, N.Y. The estate was likened to a feudal
fief. As Mr. Rockefeller wrote in his autobiography, ``Memoirs'' (2002),
``Eventually the family accumulated about 3,400 acres that surrounded
and included almost all of the little village of Pocantico Hills, where
most of the residents worked for the family and lived in houses owned by
Grandfather.''

In that bucolic setting, he developed a fascination for insects that
would lead to his building one of the largest beetle collections in the
world.

David was 21 when John D. Rockefeller died. ``He told amusing stories
and sang little ditties,'' Mr. Rockefeller recalled in 2002. ``He gave
us dimes.''

Mr. Rockefeller's sense of noblesse oblige was heightened by his early
education at the experimental Lincoln School in Manhattan, founded by
the American philosopher John Dewey and financed by the Rockefeller
Foundation to bring together children from varied social backgrounds. He
went on to study at Harvard, receiving his bachelor's degree in 1936,
and then spent a year at the London School of Economics, a hotbed of
socialist intellectuals. Mr. Rockefeller was awarded a Ph.D. in
economics from the University of Chicago in 1940.

Image

Mr. Rockefeller, second from left, with Donald H. Elliott, of the City
Planning Commission; Mayor John V. Lindsay; Richard Weinstein, of the
Office of Lower Manhattan Development; and Edmund F. Wagner, of the
Downtown-Lower Manhattan Association, at City Hall in 1972.Credit...Neal
Boenzi/The New York Times

Moved by the Great Depression at home and abroad, he stated in his
doctoral thesis that he was ``inclined to agree with the New Deal that
deficit financing during depressions, other things being equal, is a
help to recovery.'' The notion that a Rockefeller would take such a
liberal economic view was major news; the family, rock-ribbed
Republican, was known for its fierce opposition to President Franklin D.
Roosevelt, the New Deal's author.

After receiving his doctorate, Mr. Rockefeller became a secretary to
Fiorello H. La Guardia, New York's pugnacious, liberal Republican mayor.
In 1940, he married Margaret McGrath, known as Peggy, whom he had met at
a dance seven years earlier, when he was a Harvard freshman and she was
a student at the Chapin School in New York. His wife, a dedicated
conservationist, died at 80 in 1996.

Besides his son David, he is survived by his daughters, Abby
Rockefeller, Neva Goodwin, Peggy Dulany and Eileen Growald; 10
grandchildren and 10 great-grandchildren. Another son, Richard, died in
2014 at 65
\href{https://www.nytimes3xbfgragh.onion/2014/06/14/nyregion/richard-rockefeller-killed-in-new-york-plane-crash.html}{when
the small plane he was piloting crashed} shortly after takeoff from
Westchester County Airport.

Mr. Rockefeller enlisted in the Army in 1942, attended officer training
school and served in North Africa and France during World War II. He was
discharged a captain in 1945.

He began his banking career in 1946 as an assistant manager with the
Chase National Bank, which merged in 1955 with the Bank of Manhattan
Company to become Chase Manhattan.

Banking in the early postwar era was a gentleman's profession. Top
executives could attend to outside interests, using social contacts to
cultivate clients while leaving day-to-day management to junior
officers.

Mr. Rockefeller found plenty of time for such activities. In the late
1940s, he replaced his mother on the Museum of Modern Art's board and
eventually became its chairman. He courted art collectors. In 1968, he
put together a syndicate, including his brother Nelson and the CBS
chairman, William S. Paley, to buy Gertrude Stein's collection of modern
art. David and Peggy Rockefeller's own prized paintings --- by Cézanne,
Gauguin, Matisse, Picasso --- were lent to the museum permanently.

\hypertarget{putting-a-bank-in-order}{%
\subsection{Putting a Bank in Order}\label{putting-a-bank-in-order}}

Mr. Rockefeller's rise in banking was swift. By 1961, he was president
of Chase Manhattan and its co-chief executive with George Champion, the
chairman. Promoting expansion overseas, Mr. Rockefeller clashed with Mr.
Champion, who thought that the bank's domestic business was more
important. After Mr. Rockefeller replaced Mr. Champion as chairman and
sole chief executive in 1969, he was able to enlarge the bank's presence
on almost every continent. He said his brand of personal diplomacy,
meeting with heads of state, was crucial in furthering Chase's
interests.

``There were many who claimed these activities were inappropriate and
interfered with my bank responsibilities,'' Mr. Rockefeller wrote in his
autobiography. ``I couldn't disagree more.'' His ``so-called outside
activities,'' he insisted, ``were of considerable benefit to the bank
both financially and in terms of its prestige around the world.''

By 1976, Chase Manhattan's international arm was contributing 80 percent
of the bank's \$105 million in operating profit. But instead of
vindicating Mr. Rockefeller's avidity for banking abroad, those figures
underlined Chase's lagging performance at home. From 1974 to 1976, its
earnings fell 36 percent while those of its biggest rivals --- Bank of
America, Citibank, Manufacturers Hanover and J.P. Morgan --- rose 12 to
31 percent.

The 1974 recession hammered Chase, which had an unusually large
portfolio of loans in the depressed real estate industry. It also owned
more New York-related securities than any other bank in the mid-1970s,
when the city was edging toward bankruptcy. And among major banks, Chase
had the largest portfolio of nonperforming loans.

Chase also got caught up in a scandal in 1974. An internal audit
discovered that its bond trading account was overvalued by \$34 million
and that losses had been understated. A resulting \$15 million drain in
net income tarnished the bank's image. In 1975, the Federal Reserve and
the comptroller of the currency branded Chase a ``problem'' bank.

Even as he struggled to reverse Chase Manhattan's decline, Mr.
Rockefeller found time to address New York City's financial problems.
His involvement in municipal affairs dated to the early 1960s, when, as
founder and chairman of the Downtown-Lower Manhattan Association, he
recommended that a World Trade Center be built.

Image

Mr. Rockefeller with a Mark Rothko painting that he sold at auction in
2007.Credit...Todd Heisler/The New York Times

In 1961, largely at his instigation, Chase opened its 64-story
headquarters in the Wall Street area, a huge investment that helped
revitalize the financial district and encouraged the World Trade Center
project to proceed.

In the mid-1970s, with New York City facing a default on its debts
because of sluggish economic growth and uncontrolled municipal spending,
Mr. Rockefeller helped bring together federal, state and city officials
with New York business leaders to work out an economic plan that
eventually pulled the city out of its crisis.

At the same time, he put his bank's affairs in order. By 1981, he and
his protégé
\href{http://dealbook.nytimes3xbfgragh.onion/2012/08/27/willard-c-butcher-former-chief-of-chase-manhattan-dies-at-85/}{Willard
C. Butcher} had restored Chase Manhattan to full health. He yielded his
chairmanship to Mr. Butcher that year.

From 1976 to 1980, the bank's earnings more than doubled, and it
outperformed its archrival, Citibank, in returns on assets, a critical
indicator of a bank's profitability. Even after retiring from active
management in 1981, Mr. Rockefeller continued to serve Chase as chairman
of its international advisory council and to act as the bank's foreign
diplomat. He did not hesitate to criticize United States officials for
policies he considered mistaken.

He was notably harsh about President Carter. In 1980, he told The
Washington Post that Mr. Carter had not done ``what most other countries
do themselves, and expect us to do --- namely, to make U.S. national
interests our prime international objective.''

But Mr. Rockefeller also played the gadfly to Mr. Carter's far more
conservative successor, President Ronald Reagan. While the Reagan
administration was supporting anti-Marxist guerrillas in Africa, Mr.
Rockefeller took a 10-nation tour of the continent in 1982 and declared
that African Marxism was not a threat to the United States or to
American business interests.

Late in life, Mr. Rockefeller was involved in controversies over
Rockefeller Center, the Art Deco office building complex his father
built in the 1930s. In 1985, the Rockefeller family mortgaged the
property for \$1.3 billion, pocketing an estimated \$300 million. In
1989, the family sold 51 percent of the Rockefeller Group, which owned
Rockefeller Center and other buildings, to the Mitsubishi Estate Company
of Japan. Mitsubishi later increased its share to 80 percent.

The purchase represented the high tide of a buying spree of American
properties by Japanese corporations, and it opened the family to
criticism that it had surrendered an important national symbol to them.
When Japan's economic bubble burst in the early 1990s and Mitsubishi was
forced to declare Rockefeller Center in bankruptcy in 1995, Mr.
Rockefeller was criticized again, this time for allowing the site to
slip into financial ruin.

Before the year ended, Mr. Rockefeller put together a syndicate that
bought control of Rockefeller Center. Then, in 2000, it was sold in a
\$1.85 billion deal that severed the center's last ties with the
Rockefeller family.

As an octogenarian, Mr. Rockefeller, whose fortune was
\href{http://www.forbes.com/profile/david-rockefeller/}{estimated in
2012 at \$2.7 billion,} increasingly devoted himself to philanthropy,
donating tens of millions of dollars in particular to Harvard, the
Museum of Modern Art and
\href{http://www.rockefeller.edu/about/history}{the Rockefeller
University}, which John D. Rockefeller Sr. founded in 1901.

Even in his 90s, David Rockefeller continued to work at a pace that
would tire a much younger person. He spent more than half the year
traveling on behalf of Chase or groups like the Council on Foreign
Relations and the Trilateral Commission. In 2005, when he was
interviewed in his offices at Rockefeller Center, he remained physically
active, working with a trainer at the center's sports club.

He continued to collect art, including hundreds of paintings as well as
furniture and works in colored glass, porcelain and petrified wood.

That same year, he pledged a \$100 million bequest to the Museum of
Modern Art. Such giving became grist for the society pages. One
celebrity-filled fund-raising gala at the museum in 2005 drew 850 people
paying as much as \$90,000 for a table. The occasion was Mr.
Rockefeller's 90th birthday, and at the end of the evening, he was
presented with a birthday cake modeled after his house in Maine. Then it
was off to a week in southern France to continue the celebration with 21
members of his family.

With the book ``Memoirs'' in 2002, he became, at age 87, the first in
three generations of Rockefellers to publish an autobiography. Asked why
he wrote it, he replied in his characteristic reserved tone, ``Well, it
just occurred to me that I had led a rather interesting life.''

Advertisement

\protect\hyperlink{after-bottom}{Continue reading the main story}

\hypertarget{site-index}{%
\subsection{Site Index}\label{site-index}}

\hypertarget{site-information-navigation}{%
\subsection{Site Information
Navigation}\label{site-information-navigation}}

\begin{itemize}
\tightlist
\item
  \href{https://help.nytimes3xbfgragh.onion/hc/en-us/articles/115014792127-Copyright-notice}{©~2020~The
  New York Times Company}
\end{itemize}

\begin{itemize}
\tightlist
\item
  \href{https://www.nytco.com/}{NYTCo}
\item
  \href{https://help.nytimes3xbfgragh.onion/hc/en-us/articles/115015385887-Contact-Us}{Contact
  Us}
\item
  \href{https://www.nytco.com/careers/}{Work with us}
\item
  \href{https://nytmediakit.com/}{Advertise}
\item
  \href{http://www.tbrandstudio.com/}{T Brand Studio}
\item
  \href{https://www.nytimes3xbfgragh.onion/privacy/cookie-policy\#how-do-i-manage-trackers}{Your
  Ad Choices}
\item
  \href{https://www.nytimes3xbfgragh.onion/privacy}{Privacy}
\item
  \href{https://help.nytimes3xbfgragh.onion/hc/en-us/articles/115014893428-Terms-of-service}{Terms
  of Service}
\item
  \href{https://help.nytimes3xbfgragh.onion/hc/en-us/articles/115014893968-Terms-of-sale}{Terms
  of Sale}
\item
  \href{https://spiderbites.nytimes3xbfgragh.onion}{Site Map}
\item
  \href{https://help.nytimes3xbfgragh.onion/hc/en-us}{Help}
\item
  \href{https://www.nytimes3xbfgragh.onion/subscription?campaignId=37WXW}{Subscriptions}
\end{itemize}
