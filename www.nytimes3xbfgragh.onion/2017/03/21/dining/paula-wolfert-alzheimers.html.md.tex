Sections

SEARCH

\protect\hyperlink{site-content}{Skip to
content}\protect\hyperlink{site-index}{Skip to site index}

\href{https://www.nytimes3xbfgragh.onion/section/food}{Food}

\href{https://myaccount.nytimes3xbfgragh.onion/auth/login?response_type=cookie\&client_id=vi}{}

\href{https://www.nytimes3xbfgragh.onion/section/todayspaper}{Today's
Paper}

\href{/section/food}{Food}\textbar{}Her Memory Fading, Paula Wolfert
Fights Back With Food

\url{https://nyti.ms/2nxZpJN}

\begin{itemize}
\item
\item
\item
\item
\item
\item
\end{itemize}

Advertisement

\protect\hyperlink{after-top}{Continue reading the main story}

Supported by

\protect\hyperlink{after-sponsor}{Continue reading the main story}

\hypertarget{her-memory-fading-paula-wolfert-fights-back-with-food}{%
\section{Her Memory Fading, Paula Wolfert Fights Back With
Food}\label{her-memory-fading-paula-wolfert-fights-back-with-food}}

\includegraphics{https://static01.graylady3jvrrxbe.onion/images/2017/03/22/dining/22WOLFERT1/22WOLFERT1-articleLarge.jpg?quality=75\&auto=webp\&disable=upscale}

By \href{http://www.nytimes3xbfgragh.onion/by/kim-severson}{Kim
Severson}

\begin{itemize}
\item
  March 21, 2017
\item
  \begin{itemize}
  \item
  \item
  \item
  \item
  \item
  \item
  \end{itemize}
\end{itemize}

SONOMA, Calif. --- The first thing Paula Wolfert wants to make a guest
is coffee blended with butter from grass-fed cows and something called
brain octane oil. She waves a greasy plastic bottle of the oil around
her jumble of a kitchen like a preacher who has taken up a serpent.

Never mind that this is the woman who introduced tagines, Aleppo pepper
and cassoulet to American kitchens, wrote nine cookbooks and once
possessed a palate the food writer Ruth Reichl declared the best she'd
ever encountered.

Ms. Wolfert, 78, has dementia. She can't cook much, even if she wanted
to. Which, by the way, she doesn't.

She learned she probably had Alzheimer's disease in 2013, but she
suspected something wasn't right long before. Words on a page sometimes
made no sense. Complex questions started to baffle her. Since she has
always been an audacious and kinetic conversationalist with a touch of
hypochondria, friends didn't notice anything was wrong. Doctors spoke of
``senior moments.''

But she knew. One day, Ms. Wolfert went to make an omelet for her
husband, the crime novelist \href{http://www.williambayer.com/}{William
Bayer}. She had to ask him how.

The woman who once marched up to the French chef
\href{http://www.nytimes3xbfgragh.onion/2001/11/26/us/jean-louis-palladin-55-a-french-chef-with-verve-dies.html}{Jean-Louis
Palladin} and told him a dish didn't have enough salt can no longer
taste the difference between a walnut and a pecan, or smell whether the
mushrooms are burning. The list of eight languages she once understood
has been reduced to English. Maybe 40 percent of the words she knew have
evaporated.

\includegraphics{https://static01.graylady3jvrrxbe.onion/images/2017/03/22/dining/22WOLFERT2/22WOLFERT2-articleLarge.jpg?quality=75\&auto=webp\&disable=upscale}

``What am I going to do, cry about it?'' Ms. Wolfert said in an
interview at her home this month, the slap of her Brooklyn accent still
sharp. After all, she points out, her first husband left her in Morocco
with two small children and \$2,000: ``I cried for 20 minutes and I
thought, `This isn't going to do any good.'''

Still, her insatiable drive --- which took her to live with the Beat
Generation's most notable characters in Tangier in 1959 and then
propelled her like a pushy anthropologist into countless kitchens around
the world --- seems to be working just fine. Ms. Wolfert has been
collaborating with a writer on a biography to be published in April.
Instead of seeking out recipes, she is eating to save her mind.

Thus, the so-called bulletproof coffee she makes every morning and the
squares of dark chocolate she eats after lunch, in the belief they will
bolster her brainpower. In between, she eats a carbohydrate-free diet
built on salmon, berries and greens, along with extracts of turmeric,
cinnamon and eggplant.

The diet draws on an amalgam of theories she has culled from deep
internet research, her doctors, the other dementia patients she meets
with every week and long conversations with friends and experts on
FaceTime, her favorite place to chat.

``You can talk for an hour and a half, and it doesn't cost you a dime!''
she said. (The Southern food writer James Villas, her good friend,
lovingly calls her La Bouche --- the Mouth.)

She has happily lost 20 pounds. Friends say she looks remarkably good,
younger even. ``Turning back the clock, turning back the clock,'' she
chants cheerfully.

Image

Before food TV and celebrity chefs, cookbook authors like Ms. Wolfert
were the nation's gastronomic guides, traveling the cooking-school
circuit like celebrities.Credit...Jim Wilson/The New York Times

Ms. Wolfert hasn't even eaten bread, a true love, in over a year. ``I
don't remember it, but I don't care,'' she said. ``I don't want to be a
zombie.''

It would be hard to overstate the importance of Ms. Wolfert's work,
which introduced couscous and other classic Mediterranean dishes to
generations of cooks. The New York Times food writer Craig Claiborne
called her ``one of the leading lights in contemporary gastronomy.'' She
made Alice Waters fall in love with chicken cooked with preserved lemons
and olives in a tagine, and primed America for the Middle Eastern
flavors of
\href{https://www.nytimes3xbfgragh.onion/2017/02/07/dining/yotam-ottolenghi-baking-granita-kataifi.html?_r=0}{Yotam
Ottolenghi}, who remains a fan. The British chef Fergus Henderson chose
her cassoulet as his favorite recipe of all time.

A whole murderers' row of great American chefs --- Thomas Keller, David
Kinch, Judy Rodgers --- has said how much her work mattered. ``I have
always treasured and loved the vigor of her passionate and intellectual
approach to authenticity,'' Mario Batali said.

Ms. Wolfert started cooking as a young bride, taking classes from the
French instructor Dione Lucas, who was famous for her omelets. She
became Ms. Lucas's assistant, then picked up some cooking jobs arranged
for her by James Beard.

Discovering she was a complete failure as a line cook, she agreed to
move to
\href{http://www.nytimes3xbfgragh.onion/2011/10/05/dining/two-directions-for-moroccan-cuisine-modern-or-classic.html}{Morocco}
with her first husband. There, surrounded by expat writers and musicians
stuck in their web of drug-taking and drama, she found refuge in the
souks of Tangier and planted the seeds for what would eventually become
\href{https://www.harpercollins.com/9780060913960/couscous-and-other-good-food-from-morocco}{``Couscous
and Other Good Food From Morocco,''} which she published in 1973.

She branched out to southwestern France, Spain and other parts of the
Mediterranean, writing books at a time when America was waking up to the
culinary treasures beyond its borders. The concept of
\href{https://www.nytimes3xbfgragh.onion/2016/12/22/nyregion/chopped-cheese-ginia-bellafante.html}{culinary
Columbusing} had yet to surface, and the quest for authenticity in food
hadn't become sport.

Before food television and celebrity chefs, cookbook authors like her
were the nation's gastronomic guides, traveling the cooking-school
circuit like celebrities.

``I have come to call the people of that era `the Julia Child' of
whatever cuisine,'' said Celia Sack, who owns
\href{http://omnivorebooks.com/}{Omnivore Books} in San Francisco. Ms.
Sack buys the cookbook collections of the great cooking teachers of the
1970s and '80s, and sells them to younger cooks.

She recently put up for sale some cookbooks from Ms. Wolfert's personal
collection, which was deep and specific. A book on the polentas of
Venice stamped with Ms. Wolfert's name is selling for \$75.

Next month, a book about Ms. Wolfert will debut with an origin story as
unconventional as she is. ``Unforgettable: The Bold Flavors of Paula
Wolfert's Renegade Life'' is a biography interwoven with about 50
recipes. The author is Emily Kaiser Thelin, Ms. Wolfert's former editor
at Food and Wine, who has become as much a daughter as a biographer.

In 2006, Ms. Thelin inherited the magazine's Master Cook column, which
included contributions from Jacques Pépin and Jean-Georges Vongerichten.
``I always dealt with their assistants,'' Ms. Thelin said.

But Ms. Wolfert called her and said, ```O.K., you're my editor and you
need to know I can't write my way out of a paper bag.'''

Image

Emily Kaiser Thelin slicing salmon under the watchful eye of Ms.
Wolfert. Ms. Thelin is the author of a new book, ``Unforgettable: The
Bold Flavors of Paula Wolfert's Renegade Life.''Credit...Jim Wilson/The
New York Times

In 2008, Ms. Thelin traveled to Morocco to write about Ms. Wolfert for
the magazine. Young and intimidated, Ms. Thelin watched her in action.
She likens the adventure to ``a trip to Kitty Hawk with the Wright
Brothers.''

Ms. Thelin left the magazine in 2010 and moved from New York to Northern
California. The two women's friendship deepened, laced with long
conversations about food, reality TV and politics. Ms. Thelin was toying
with the idea of a biography. Then came the diagnosis. The biography
seemed more important than ever.

The proposal was praised but rejected by nearly a dozen editors,
including Dan Halpern, who as a young man slept free on Ms. Wolfert's
couch and later published her book
\href{https://www.harpercollins.com/9780061957550/the-food-of-morocco}{``The
Food of Morocco''}in 2009.

Ms. Wolfert, it seemed, was yesterday's news.

Eric Wolfinger, who is essentially the Annie Leibovitz of food
photography, suggested a
\href{https://www.kickstarter.com/projects/265444901/unforgettable-bold-flavors-from-a-renegade-life}{Kickstarter
campaign} and offered to shoot the pictures. It quickly raised over
\$91,000, including \$100 from Mr. Halpern. Andrea Nguyen, the noted
Vietnamese cookbook author, signed on to edit. Toni Tajima agreed to
design it. On April 4, it goes on
\href{https://www.amazon.com/Unforgettable-Flavors-Paula-Wolferts-Renegade/dp/1681882221/ref=la_B06XQXCSQW_1_1?s=books\&ie=UTF8\&qid=1490110907\&sr=1-1}{sale
for \$35 on Amazon} and through a website,
\href{http://www.unforgettablepaula.com/}{Unforgettable Paula}.

The book begins in a Jewish neighborhood in the Flatbush section of
Brooklyn, where Ms. Wolfert grew up with vision problems and a dieting
mother who fed her cottage cheese, melon and lettuce, and didn't like
her very much. It ends with tips for
\href{http://www.pbs.org/newshour/bb/health-july-dec13-wolfert_11-26/}{using
food to connect with someone suffering from dementia}, like cooking
recipes together that have a deeper, personal meaning, or understanding
that the hands of many older cooks may remember what to do when their
minds cannot.

The loving profile sometimes glosses over comments from critics (which
Ms. Wolfert still has quite a sharp memory for). More than a few editors
and cooks have found her demand for specific ingredients impossible, the
way she delivers extensive knowledge of certain cuisines insufferable
and her recipes so complex as to be unworkable.

Image

Lunch is served: oven-steamed salmon.Credit...Jim Wilson/The New York
Times

But Ms. Thelin, like many, is a true believer. ``I feel like every Paula
recipe seems to pull the rug out from under you,'' she said. ``You think
it's not going to work, but if you keep calm and follow the recipe it
does.''

Even though many of Ms. Wolfert's books never sold well, Ms. Thelin
said, they were almost always prescient. ``Alice Waters said if
`\href{http://www.slate.com/articles/news_and_politics/in_the_soup/1998/08/the_diva.html}{Grains
and Greens}' came out today, it would be a runaway best seller,'' she
said.

Ms. Wolfert still has lessons to teach her acolyte. On a recent
Saturday, Ms. Thelin spent the morning carefully blanching vegetables
that would be seasoned with pancetta in a recipe Ms. Wolfert adapted
from Michel Bras, a French chef whom Ms. Wolfert wrote about in 1987.

Then they moved onto salmon, using Ms. Wolfert's master recipe, which
calls for steaming the fish over a pan of hot water set in a roughly
250-degree oven. The fish cooks on a very thin pan until it's tender but
juicy and still bright.

Ms. Thelin pulled the fillet from the oven, considering how to cut the
soft fish into portions. Ms. Wolfert said she should have done so before
it was cooked, then took a pair of shears to the fillet. Ms. Thelin was
surprised by how tidy the technique was. She never would have thought to
use scissors.

``You're still teaching me things,'' she said.

Lunch stretched into the afternoon. Ms. Wolfert seemed energized by the
company and an opportunity to deliver stories with her favorite polished
punch lines. And because it's what food writers do, she promoted a new
book she had discovered:
\href{http://www.penguinrandomhouse.com/books/533132/the-spice-companion-by-lior-lev-sercarz/}{``The
Spice Companion,''} by
\href{https://www.nytimes3xbfgragh.onion/2016/11/02/dining/padma-lakshmi-lior-lev-sercarz-spice-guides.html}{Lior
Lev Sercarz}. Spices have given her a new culinary world to explore, at
least on paper.

She was so enamored of the book that she called Mr. Sercarz's New York
spice store, La Boite, to order a few of his blends to sprinkle on the
salmon at lunch.

Most of them she couldn't taste, but one, a blend called cancale,
stopped her. Salty and with a strong whiff of fennel and orange, it
somehow broke through. She could taste it.

``You know what it is?'' she said. ``It reminds me of Morocco.''

Recipes:
\href{https://cooking.nytimes3xbfgragh.onion/recipes/1018664-oven-steamed-salmon}{\textbf{Oven-Steamed
Salmon}} \textbar{}
\href{https://cooking.nytimes3xbfgragh.onion/recipes/1018665-cracked-green-olive-walnut-and-pomegranate-relish}{\textbf{Cracked
Green Olive, Walnut and Pomegranate Relish}} \textbf{\textbar{}}
\href{https://cooking.nytimes3xbfgragh.onion/68861692-nyt-cooking/4933834-paula-wolfert-recipes}{\textbf{More
Dishes From Paula Wolfert}}

Advertisement

\protect\hyperlink{after-bottom}{Continue reading the main story}

\hypertarget{site-index}{%
\subsection{Site Index}\label{site-index}}

\hypertarget{site-information-navigation}{%
\subsection{Site Information
Navigation}\label{site-information-navigation}}

\begin{itemize}
\tightlist
\item
  \href{https://help.nytimes3xbfgragh.onion/hc/en-us/articles/115014792127-Copyright-notice}{©~2020~The
  New York Times Company}
\end{itemize}

\begin{itemize}
\tightlist
\item
  \href{https://www.nytco.com/}{NYTCo}
\item
  \href{https://help.nytimes3xbfgragh.onion/hc/en-us/articles/115015385887-Contact-Us}{Contact
  Us}
\item
  \href{https://www.nytco.com/careers/}{Work with us}
\item
  \href{https://nytmediakit.com/}{Advertise}
\item
  \href{http://www.tbrandstudio.com/}{T Brand Studio}
\item
  \href{https://www.nytimes3xbfgragh.onion/privacy/cookie-policy\#how-do-i-manage-trackers}{Your
  Ad Choices}
\item
  \href{https://www.nytimes3xbfgragh.onion/privacy}{Privacy}
\item
  \href{https://help.nytimes3xbfgragh.onion/hc/en-us/articles/115014893428-Terms-of-service}{Terms
  of Service}
\item
  \href{https://help.nytimes3xbfgragh.onion/hc/en-us/articles/115014893968-Terms-of-sale}{Terms
  of Sale}
\item
  \href{https://spiderbites.nytimes3xbfgragh.onion}{Site Map}
\item
  \href{https://help.nytimes3xbfgragh.onion/hc/en-us}{Help}
\item
  \href{https://www.nytimes3xbfgragh.onion/subscription?campaignId=37WXW}{Subscriptions}
\end{itemize}
