Sections

SEARCH

\protect\hyperlink{site-content}{Skip to
content}\protect\hyperlink{site-index}{Skip to site index}

The Hawaii Cure

\url{https://nyti.ms/2nvgl3G}

\begin{itemize}
\item
\item
\item
\item
\item
\item
\end{itemize}

\includegraphics{https://static01.graylady3jvrrxbe.onion/images/2017/03/26/magazine/26hawaii1/26hawaii1-articleLarge-v2.jpg?quality=75\&auto=webp\&disable=upscale}

The Voyages Issue

\hypertarget{the-hawaii-cure}{%
\section{The Hawaii Cure}\label{the-hawaii-cure}}

A first trip to the island, in a desperate bid to escape the news.

Visitors at Volcanoes National Park on the Big Island.Credit...Dina
Litovsky/Redux, for The New York Times

Supported by

\protect\hyperlink{after-sponsor}{Continue reading the main story}

By Wells Tower

\begin{itemize}
\item
  March 21, 2017
\item
  \begin{itemize}
  \item
  \item
  \item
  \item
  \item
  \item
  \end{itemize}
\end{itemize}

Do not eat until you are full; eat until you are tired,'' calls Chief
Sielu Avea, a Polynesian entertainer who, according to his bio, is
``internationally known as the Coconut Man.'' Making our way to the
plastic table, paper plates wilting in our hands, we are tired already.

Here at the Chief's Luau, ``Aloha'' means last to the buffet. The
feeders in the ``Royal'' service tier (\$159 per ticket) got first crack
at the chafing dishes. And then team ``Paradise'' (\$119) went at the
sheet cake and roast pig. And if we stragglers in the Aloha group are
not enraptured with our feast of sweetly lacquered chicken chunks and
puffy dinner rolls, the fault is ours for booking steerage at \$87 a
head.

But you do not come to the Chief's Luau for the food. You come because
you have traveled thousands of miles only to fetch up in Waikiki Beach,
a concentrated zone of souvenir dealers and luggage-dragging hordes that
feels like a cultural protectorate of the airport. Hankering after
something incontestably Hawaiian, you end up on a charter bus bound for
the Chief's Luau at Sea Life Park 15 miles east on the Kalanianaole
Highway. Never mind that what is most purely Hawaiian about the luau is
its proficiency at extracting tourists' dollars. The luau leaves no
doubt: You are in Hawaii now.

Beyond the buffet, there are traditional activities. Under the
instruction of shirtless men in sarongs, you can fling a plastic spear
at grass. There is the weaving station, where the spectacle includes a
pregnant woman shoving her young daughter for trying to horn in on her
work at a frond headband. And there is a fire-starting clinic where we
rub sticks on logs in the hope of making flame. This proves no more
possible than it was in the forests of our childhoods, but we go on
rubbing in the faith that we are in a magical land where the laws of
physics bend toward human satisfaction.

And for many of us, it is a magical evening. The magic has to do with
the moon, the thud and rustle of the surf. The magic is working on Jed,
my 1½-year-old son. He is off to the side of the action, trying to
seduce a girl of 7 or so. She is engrossed with her tablet. A cultist of
the night sky, Jed touches her wrist, points overhead and says,
``Stars.'' The girl's eyes do not flicker from her screen.

My wife is similarly resistant to the enchantment. ``This luau is making
me feel bad about myself, and it is making me feel bad about humanity,''
she says. We are now watching an entertainment where Hawaiian women in
grass skirts dance the hula, and Hawaiian men with painted faces do a
grunting spear-dance and stick their tongues out tikistyle. To my wife,
this smacks uncomfortably of minstrelsy, which, yes, it does. But at
least it is a two-way minstrelsy. The dancers pretend to be tiki
warriors, and when the chief, in parting, bids us officially welcome to
``the land of happy people,'' we pretend to believe that such a place
exists.

\textbf{Can it be} true? The aloha spirit is real? Paradise on earth? An
Eden of happy Americans moated from our national ravages of malevolence,
contempt, uncertainty and fear?

Not until 2017 has Hawaii held for me even a vague temptation. The 50th
state has always seemed to me a meretricious luxury product whose
visitors bring happiness with them in the form of money. I am not
constitutionally geared for paradise. I am not one for cocktails
containing patio equipment, for lazing on talcum-soft sand, eyes gone to
pinwheels, grinning madly at the sun.

\includegraphics{https://static01.graylady3jvrrxbe.onion/images/2017/03/26/magazine/26hawaii2/26mag-26hawaii-t_CA4-articleLarge.jpg?quality=75\&auto=webp\&disable=upscale}

Hawaii is notoriously nice, and unremitting niceness is what I do not
want out of a vacation. This is because I'm cheap. I want a maximum
memory harvest for my travel dollar, and a trip rarely sticks in my
long-term storage cache without the sharp edges of mishap and discomfort
to snag on. I do not, for example, remember nice meals I have eaten so
clearly as the wet duckling I disgorged on a street in the Philippines,
and the delight this brought the locals. I cannot recall the nice hotels
I've stayed in half so well as the New Zealand jungle cabin where I
inadvertently slept on the rotting carcass of a rat and woke up with a
heart murmur.

But in a political moment so well supplied with nastiness, I don't need
to bunk with carrion. Give me a slack-keyed, macadamia-dusted holiday
where things are pretty and people are smiling, if only because it's in
their job description. In a gesture of spiritual surrender, I have
booked a five-day stay in the Hawaiian Islands with no greater hope for
the voyage than that it may be merely nice.

Our itinerary is at risk of proving mindlessly splendid: Oahu for two
nights, before we board a prop jet for three nights on Hawaii Island to
the east. But a 19-month-old, as it turns out, is excellent insurance
against a frictionless travel experience. Our first morning on Oahu, Jed
does me the kindness of waking up at 4 a.m. He insists that we dress and
begin making the most of our day. I put on an algae-print shirt I have
lacked the courage to wear since I bought it years ago in Thailand.

We are staying in a room at the Waikiki Beach Hilton, which, with its
ocean views and high-pressure shower head, is dangerously close to nice.
But in the corridor I am pleased to meet a fat and saucy cockroach,
thoughtfully dispatched, perhaps, by a concierge who has gotten wind of
my preferences. In live-and-let-live aloha spirit, I do not molest the
animal. My wife, however, in consideration of the sleeping guests the
roach might visit, bruises the creature with a sack of dirty diapers
before it jogs off down the hall.

Image

Sea-life encounter at Carlsmith National Beach Park,
Hawaii.Credit...Dina Litovsky/Redux, for The New York Times

In the lobby, we lay down \$12 for two coffees and one banana and browse
the morning paper, which proves a clemency from anticipated horrors. The
front page of The Honolulu Star-Advertiser bears not a single
presidential headline. ``Legislature Considers Funding to Combat Rat
Lungworm Disease'' is the story of the day.

Dawn finds us waterfront on Oahu's North Shore, downrange of the Banzai
Pipeline. The sand has a forthright cornmeal consistency. The water is
the blue of telegraph insulators. The waves transmit a disaster-movie
feeling with every crash, even after you have watched a thousand of them
land. The young and barely clad are out in force, demonstrating
physiques that can come only from long and rigorous hours of ignoring
national politics. Just up the shore, two young women are seriously
engaged in the business of aiming a big professional camera at the
tanned, professional butt of a third young woman who, I'm guessing, is a
big deal in a modeling niche I didn't know existed. One thing is sure:
No way will I be bathing here.

My son gives not a damn. He uncloaks fully his cloudlike body and hits
the sand like an oyster in a breading dredge. The day is perfect room
temp with a breeze. In the distant shallows, surfers shoot the tube or
gleam the curl or whatever that amazing thing is called. My wife and I
breakfast on fresh coconut --- neither sweet nor flavorful but fun to
gnaw, for the feeling that you've acquired termite superpowers. Jed
squats and tumbles and packs his nethers with 20-grit. ``Whoa! Whoa!
Whoa!'' is his ecstatic report on the sensation. I am right there with
him. It would be overselling things to claim that I've achieved
rapturous mind erasure my first morning in Hawaii, but this is, well,
rather nice.

For lunch we motor clockwise down the coast to the Kahuku Superette. The
Superette is a homely liquor shop/convenience store that from the
outside is easily pictured in a newscast with police lights flashing on
it. Inside, they dish out poké of world renown. Poké is sashimi salad
doused in soy and sesame and other things. We get a tub of traditional
shoyu poké and a tub of limu poké with crunchy bits of seaweed. The
place to gobble the Superette's poké is in your hot rental car in the
muddy parking lot. Gemlike blocks of tuna nearing a full cubic inch are
bright and salty as the sea.

Slide 1 of 4

1/4

Ahanalui Pool ``hot springs,'' Big Island

Credit...Dina Litovsky/Redux, for The New York Times

\begin{itemize}
\item
  \includegraphics{https://static01.graylady3jvrrxbe.onion/images/2017/03/26/magazine/26hawaii-voyages-slide-4DK8/26hawaii-vayages-slide-4DK8-superJumbo.jpg}
\item
  \includegraphics{https://static01.graylady3jvrrxbe.onion/images/2017/03/26/magazine/26hawaii-voyages-slide-H0LS/26hawaii-vayages-slide-H0LS-superJumbo.jpg}
\item
  \includegraphics{https://static01.graylady3jvrrxbe.onion/images/2017/03/26/magazine/26hawaii-voyages-slide-11HF/26hawaii-voyages-slide-11HF-superJumbo.jpg}
\item
  \includegraphics{https://static01.graylady3jvrrxbe.onion/images/2017/03/26/magazine/26hawaii-voyages-slide-7D85/26hawaii-vayages-slide-7D85-superJumbo.jpg}
\end{itemize}

Back in Honolulu, the Pearl Harbor Visitors Center is out of tickets to
the U.S.S. Arizona Memorial site, so we resolve to take in our ration of
history with a trudge around the Makiki neighborhood, where Barack Obama
grew up. It is an area of cinder-block buildings and auto-parts shops
well off the luau trail. On the sidewalks, hard-luck people push baby
strollers full of cans and bottles because the redemption center forbids
the use of grocery carts. Parking is free on the street, one of Makiki's
practical concessions to the paradise theme. No plaque marks the Punahou
Circle Apartments, where Obama lived during his middle- and high-school
years, and where, just before the 2008 election, he returned to visit
his maternal grandmother as she was dying. It is as regular an apartment
building as you could find anywhere in America, a putty-colored tower
whose minute balconies hold garbage bags, golf clubs, a vacuum cleaner
and one (small-size) American flag.

Nearby on King Street, we nip into the Baskin- Robbins where I heard
Obama worked in high school. It is the sort of cramped little parlor
that, if you had a job there, would make you sink into despair or go on
to be president. I ask the young woman on scooping duty if it's true
that Obama used to dip cones at this very counter, and she says,
``Yeah.'' No plaque in there either, just a newspaper clipping taped to
the sneeze guard next to the smoothie machine.

\textbf{``ENTER AT YOUR} OWN RISK.'' ``FISH AND EELS MAY BECOME
AGGRESSIVE.'' ``DANGEROUS SHOREBREAK.'' ``DO NOT ENTER IF YOU HAVE OPEN
WOUNDS DUE TO RISK OF BACTERIAL INFECTION.'' So runs some but not all of
the cautionary signage at the Ahalanui Warm Pond. My kind of place!

We have fled the Banzai Pipeline and the crowds of Waikiki to spend four
aimless days poking around Hawaii Island, a.k.a. the Big Island, the
easternmost landmass in the archipelago. The Big Island, which is larger
than all the other Hawaiian Islands combined, is big because the
volcanoes here (the only active ones in the state) keep making more
island every day.

Image

Relaxing in Waikiki at the beach.Credit...Dina Litovsky/Redux, for The
New York Times

The volcanoes also supply natural hot tubs like the Ahalanui Pond. The
only trouble is that renegade bacteria like a nice warm soak as much as
we do. If you don't want to go home majorly colonized, the internet
advises that you hit the pond early in the day, when the night seas have
rinsed the pool and the day's throng of bathers have not yet added their
personal contributions to the stew.

We arrive at the pond just after 9 a.m. It is a partly man-made,
not-quite Olympic-size lagoon walled with volcanic rock over which the
Pacific spills. Three other folks are breast stroking the green
shallows, none of them microbially ``hot'' in appearance. In we go.

Through heat-jellied water, my diver's mask reveals an aquarium of
striped fish and fish with long Hitchcockian faces and tiny minnows
hungrily scrumming at a scratch on my boy's knee. Now my son is
shrieking. I surface. Not shrieking but crowing. Jed, a connoisseur of
bath water, is sampling the pond by the bulging cheekful and finding it
superb.

As usual, Jed's judgment is on point. This pond is excellent, maybe the
closest I have ever found to my mind's ideal of the great American
swimming hole. It is a wallopingly beautiful place where admission is
free. No ``Royal''-access luxury cabanas, roving pedicurists or sling
chairs for rent. It is not up a mountain or deep in a jungle but near
enough to a parking lot that the infirm can enjoy it, too.

Image

On Kalakaua Avenue.Credit...Dina Litovsky/Redux, for The New York Times

By 10 a.m., a little bit of everybody is shouldering in for a wash.
There are local families with babies and senior citizens with foam
flotation noodles and tourists with sun-scalded calves the color of
Spam. Through modern advances in waterproofing, four young women have
brought their telephones with them into the pool, fending off a
potentially cloying surplus of timeless splendor. The bacteria deserve
credit, too, for their silent encouragement against loitering. After an
hour's swim, still free of visible rashes, we make for dry land.

Out in the poolside park, Saturday things are happening. A mom wonders
when the Alcoholics Anonymous meeting will clear out from the picnic
shanty and make way for her 2-year-old's birthday party. A guy is
washing his dog in the foot bath, near a sign that says ``no animals
allowed.'' Over in the parking lot, we are glad to find a man dealing
coconuts from his beat-up S.U.V. Shirtless, veiny and tan beneath a
blown-out wicker hat, he puts the nuts down on his tailgate and machetes
them with great flair. This coconut man (the second in our mounting
tally) seems a little offended when we ask what his coconuts cost. ``I
prefer donations,'' he says. ``I don't think of myself as a business.
I'm just out here trying to feed the people.'' My wife worms it out of
him that, really, he wants \$5 per nut. I hand him a 20 for two.
Clutching my money, he goes into a thing about how the green of the
coconut is the same green as the dollar. Then he tells me how coconut
water is chemically identical to human plasma and how World War II field
hospitals would transfuse soldiers with coconuts when they ran out of
blood. I have heard this fable before and know it to be hogwash, but I
say, ``Oh, wow,'' and await my \$10 change that does not appear to be
forthcoming. After a weirdly long interval of communing with my bill,
Coconut Man No.2 looks up at me and says in a put-out sort of way, ``Oh,
did you want some change?'' I allow that I do, and he produces it.

I go away full of gratitude for this fellow, not only because his
coconuts are very fine, but for nipping a budding and inconvenient fancy
that I might like to live here on the Big Island. His brand of coconut
palaver is, I suspect, common in these parts. Encountering it on any
sort of regular basis, straight-world mainlander that I am, would drive
me out of my mind.

Flouncing on Oahu's beaches has given our boy a taste for sand, so we
pile into the car in search of some. Motoring out, I feel my fondness
for the Big Island deepening. Cataracts of blossoming vine pour from the
roadside jungle. Tire-flattened mongooses make regular appearances on
the double yellow line. Even the roadkill here astounds! Real estate in
these parts would probably cost you a thumb, yet the houses are unfussy
hip-roofed bungalows built in a kind of army-base vernacular. While some
citizens keep spectacular gardens, this is also a place where if you
want to leave some old mattresses or an engine hoist in your yard, you
just go ahead and do it.

Image

The Chief's Luau.Credit...Dina Litovsky/Redux, for The New York Times

The internet directs us to Carlsmith Park, just east of Hilo. We are
dubious. The approach runs through a district of petrochemical tanks,
tire dealerships and self-storage facilities. Then, in the middle of all
that, hard by the highway, is a waterfront park. It has a couple of nice
microbeaches for sand freaks like Jed and undulant, green-turf realms
for pro-league picnicking. Another dumbfounding spot. A coastal shire
plunked down in outer Cleveland. And again: It's free!

A stealthy concentration of snorkels prowls the limpid bay. Something
must be going on down there. I don my rig. Not only are there
eye-popping rainbow sherbet fish in this water, but a gang of Hawaiian
green sea turtles are nodding around in the tide. I mean, whoa. There
goes one about eight feet away. I badly want to crowd it, but
turtle-crowding is forbidden by signs on the bank. So I just sort of
mooch about nearby, in a playing-it-cool-around-a-celebrity sort of way.
In what appears to be the general mode of Hawaiian magnificence, the
turtle's grandeur is even grander for its casualness. Majestic is not
the word for this unseaworthy looking creature, which resembles an
antique truck hubcap that rolled in off the overpass.

The turtles don't make a big deal of themselves, and topside, no one is
making a big deal of them either. Two teenagers are fretting out whether
tall guys are preferable to ``built'' guys. Somebody is cooking hot
dogs. A couple of older kids are letting my son hog their football, an
appreciated decency after Jed's frosty luau crush. It bears mentioning
how unaffectedly nice people are on this island --- from incidental
children, to park rangers, to our Airbnb host who for two days lets me
brazenly refer to him as Steve, only to let slip, in parting, that his
actual name is Sean.

The niceness and the beauty have the standard effect. Though the odds of
our returning here are minimal, my wife and I keep getting into grown-up
fantasy conversations about the quality of the local schools and whether
maintaining a big greenhouse could get us a farm-tax subsidy.

Image

Tourists on Waikiki Beach.Credit...Dina Litovsky/Redux, for The New York
Times

One gripe about the Big Island: There is reliable 3G coverage, bearing
news of an America we're not eager to get back to. A bigot shot an
Indian man in Kansas. The White House is extending its powers into the
nation's bathrooms. Lawmakers in our home state, North Carolina, want to
overturn a constitutional ban on secession.

Put that phone away.

\textbf{Our last full} day in Hawaii we have reserved for Kilauea, the
youngest and most effusive volcano on the island. In Hawaiian,
``Kilauea'' translates to ``spewing'' or ``much spreading.'' Since 1983,
it has been continuously spewing to the delectation of tourists and
geologists. The spew is distantly glimpsable at the volcano's caldera in
Hawai'i Volcanoes National Park. But who wants to glimpse lava from afar
when you can get close enough to jab it with a stick?

Getting to the lava spot is slightly tricky. The terrain is too unstable
to support a parking lot, so you have to walk or rent a bike and pedal
in. My wife and I agree that toddlers pair badly with molten earth, so
she dumps me at the trailhead where we'll reconnoiter later. It is a
four-and-a-half-mile pedal through a field of fissured lava that looks
like a giant brownie.

The road ends at a cordon where lava pilgrims, scores of them, make
their way down toward the viewing place. The lava lies in knuckles,
folds and challah braids. It creaks hollowly underfoot like frozen snow.

Image

Lava flow pyrotechnics at Volcanoes National Park on the Big
Island.Credit...Dina Litovsky/Redux, for The New York Times

And there, not too far but not too close, is the lava pouring into the
sea. It is a flaming cliff, exploding every couple of seconds in a
grayscale fireworks of liquid rock meeting water. In a concession to our
unsubtle political age, the cliff is doing a good impression of Abraham
Lincoln in profile with a vicious orange fulmination exploding from his
head.

Night starts coming on. I should be getting back, but inland, up the
grade, there are the city lights of lava much-spreading from the
volcano's flank.

Now this is really something. This lava, you can walk right up to it,
get close enough to sweat. It lies in glowing, buxom lobes, ticking
glassily as it cools. Tomorrow, we head home. There will be CNN in the
airport. But right now, all I can think is, Man, would I like to poke
that lava with a stick. ``Boy, I wish I had a stick to poke that lava
with,'' a nearby tourist says at the very instant I am thinking this.
Here, perhaps, lies some hope for our divided nation. We would all like
to poke lava with a stick.

Rain starts falling, an irrelevance in the lava's heat-field. No one
moves. The glow is transfixing, a campfire mesmerism of geologic scale.
``We are very fortunate,'' a woman beside me murmurs. Everybody stands
beautifully quiet, watching newborn wads of America bulge and slip
toward the sea. Lava at this range is powerful stuff. You can't get near
it and not become a stoned teenager. \emph{Whoa --- consciousness,
language, satellites, ginger snaps --- all made possible by this sloppy,
tectonic incontinence}. I have an idiot hunch that this will somehow
turn into a thought worth having if only I can sit out here another hour
or two and see how fine the lava looks in the full black of night. I
call my wife to say that it is frankly too miraculous out here for me to
be leaving anytime soon.

She is parked at the trailhead. She is hungry. A Kilauea situation has
happened in Jed's Pampers, and they're trapped in the car by the rain.
She has many valid opinions about spending even one more minute like
that. The lava is getting nicer by the instant. I could probably
embezzle another moment or two of watching it ripen, but I put my back
to the miraculous and get on with life.

Advertisement

\protect\hyperlink{after-bottom}{Continue reading the main story}

\hypertarget{site-index}{%
\subsection{Site Index}\label{site-index}}

\hypertarget{site-information-navigation}{%
\subsection{Site Information
Navigation}\label{site-information-navigation}}

\begin{itemize}
\tightlist
\item
  \href{https://help.nytimes3xbfgragh.onion/hc/en-us/articles/115014792127-Copyright-notice}{©~2020~The
  New York Times Company}
\end{itemize}

\begin{itemize}
\tightlist
\item
  \href{https://www.nytco.com/}{NYTCo}
\item
  \href{https://help.nytimes3xbfgragh.onion/hc/en-us/articles/115015385887-Contact-Us}{Contact
  Us}
\item
  \href{https://www.nytco.com/careers/}{Work with us}
\item
  \href{https://nytmediakit.com/}{Advertise}
\item
  \href{http://www.tbrandstudio.com/}{T Brand Studio}
\item
  \href{https://www.nytimes3xbfgragh.onion/privacy/cookie-policy\#how-do-i-manage-trackers}{Your
  Ad Choices}
\item
  \href{https://www.nytimes3xbfgragh.onion/privacy}{Privacy}
\item
  \href{https://help.nytimes3xbfgragh.onion/hc/en-us/articles/115014893428-Terms-of-service}{Terms
  of Service}
\item
  \href{https://help.nytimes3xbfgragh.onion/hc/en-us/articles/115014893968-Terms-of-sale}{Terms
  of Sale}
\item
  \href{https://spiderbites.nytimes3xbfgragh.onion}{Site Map}
\item
  \href{https://help.nytimes3xbfgragh.onion/hc/en-us}{Help}
\item
  \href{https://www.nytimes3xbfgragh.onion/subscription?campaignId=37WXW}{Subscriptions}
\end{itemize}
