Sections

SEARCH

\protect\hyperlink{site-content}{Skip to
content}\protect\hyperlink{site-index}{Skip to site index}

\href{https://myaccount.nytimes3xbfgragh.onion/auth/login?response_type=cookie\&client_id=vi}{}

\href{https://www.nytimes3xbfgragh.onion/section/todayspaper}{Today's
Paper}

How `Un-American' Became the Political Insult of the Moment

\url{https://nyti.ms/2nvk0P1}

\begin{itemize}
\item
\item
\item
\item
\item
\item
\end{itemize}

Advertisement

\protect\hyperlink{after-top}{Continue reading the main story}

Supported by

\protect\hyperlink{after-sponsor}{Continue reading the main story}

\href{/column/first-words}{First Words}

\hypertarget{how-un-american-became-the-political-insult-of-the-moment}{%
\section{How `Un-American' Became the Political Insult of the
Moment}\label{how-un-american-became-the-political-insult-of-the-moment}}

\includegraphics{https://static01.graylady3jvrrxbe.onion/images/2017/03/26/magazine/26firstwords/26mag-26firstwords-t_CA0-articleLarge.jpg?quality=75\&auto=webp\&disable=upscale}

By Beverly Gage

\begin{itemize}
\item
  March 21, 2017
\item
  \begin{itemize}
  \item
  \item
  \item
  \item
  \item
  \item
  \end{itemize}
\end{itemize}

Conservatives have long complained that ``I don't recognize my country
anymore.'' In Trump's America, though, it's liberals who see something
insufficiently, well, American. Issues once assumed to be settled ---
the desirability of racial tolerance, a general preference for
democratic processes --- now appear, suddenly and ominously, to be up
for debate. For many, this has created a sense of rupture, a disconnect
between what they had assumed would be happening at this moment in
history and what is actually taking place --- as if the 2016 election
shifted the nation into some dystopian reality that was never meant to
be.

This, at least, is the feeling reflected in the growing popularity of a
word once used to browbeat liberals and leftists. Today's White House,
according to Trump's critics, is not only wrongheaded or meanspirited
but ``un-American.''

``America is a nation of immigrants,'' Bruce Springsteen
\href{http://www.rollingstone.com/music/news/bruce-springsteen-slams-trump-america-is-a-nation-of-immigrants-w463952}{declared}
from a stage in Australia in the wake of Trump's first travel ban. ``And
we find this antidemocratic and fundamentally un-American.'' When the
White House introduced a revised version of the order earlier this
month, the newly chosen Democratic National Committee chairman, Tom
Perez, warned that ``this ban on Muslims is just as dangerous and
un-American as the last one.'' Others have invoked the word to assail
the president's name-calling, attacks on judges and contempt for the
media. ``This White House does not seem to value an independent press,''
CNN's
\href{http://www.cnn.com/2017/02/24/politics/jake-tapper-white-house-trump-unamerican-cnntv/index.html}{Jake
Tapper declared} last month. ``There is a word for that line of
thinking. The word is un-American.''

This line of attack remains just as popular on the right, and Trump
staff members have been quick to use it. On the day after Trump's speech
to Congress, Sebastian Gorka, a deputy assistant to the president,
called for unwavering support of the president's initiatives on behalf
of ``immigration crime'' victims: ``If you object to that, you are in
favor of pain, in favor of tragedy and in favor of chaos,'' he told one
interviewer, ``and that is un-American.'' The president himself has
joined in, too. ``The real scandal here is that classified information
is illegally given out by `intelligence' like candy,'' he tweeted in
mid-February, responding to accusations about his campaign's ties to
Russia. ``Very un-American!''

The charge can come across as an empty one, ready to be attached to any
given outrage. But its re-emergence reflects something far more serious
about the state of national discourse. On the surface, ``un-American''
implies consensus: It carries a punch only when everybody agrees what
``American'' is. But the word has historically gained traction at
moments when national consensus seems the most wobbly and uncertain. It
signals an urgent desire to find something --- anything --- that
Americans can still agree on.

For Trump's critics, it is aspirational, pointing toward an American
dream of liberty and equality that has never quite been realized on the
ground. For the administration, the word is a way of designating
insiders and outsiders, the righteous and the nonrighteous, those who
deserve the privileges of citizenship and those who do not. Coming from
the White House, it also carries an implied threat: Those who are
``un-American'' --- leakers, critics, reporters --- may need to be dealt
with as ``enemies of the people,'' another worrisome Trump phrase.

To label something ``un-American'' is to imagine America as it should
have been or as it might yet be but not as it ever was. Far from
bringing back a meaningless insult, the revival of the charge suggests
that something very big is now at stake: not only the direction of
federal policy or the partisan balance of power but America's identity
as a nation.

\textbf{In his farewell} speech, Barack Obama paid homage to ``the
beating heart of our American idea,'' describing how the promise of
``freedom to chase our individual dreams'' has long inspired the
nation's citizens. This notion --- that America is not simply another
nation-state but an ``idea'' --- is \emph{itself} the big American idea,
the ``beating heart'' of what's known as American exceptionalism. It's
what allows Americans to label certain things un-American even when
those things have existed for centuries within the territorial
boundaries of the United States. ``America is capable of being
un-American,'' lamented Richard Cahan, one of the authors of a recent
book on Japanese internment --- a seemingly absurd claim but, within the
exceptionalist tradition, an entirely plausible one.

Americans have been denouncing each other in this manner almost since
the dawn of the republic. By the early 20th century, the accusation
encompassed a vast and incoherent range of political convictions.
``Americans are very fond of classing as un-American anything they don't
like,'' The St. Louis Post-Dispatch complained in 1909, ticking off
tariffs and trusts, prohibition and social drinking, boycotts and ``the
intimidation of workingmen'' as just a few of the positions declared
``un-American'' by their opponents.

The First World War only raised the stakes of this rhetorical clash. In
an attempt to mobilize a divided country, President Woodrow Wilson
called for ``100 percent Americanism,'' effectively marginalizing anyone
not 100 percent on board with the war effort --- including thousands of
German immigrants (whose noncitizenship made them candidates for
internment) and antiwar dissenters like the Socialist leader Eugene Debs
(who was sentenced to 10 years in prison under Wilson's new Espionage
Act). After the war, the impulse to enforce ``Americanism'' helped to
justify deportation raids and the rapid growth of the Ku Klux Klan. By
1924, the same impulse had yielded the nation's first comprehensive
immigration law, which largely banned immigrants from Asia, Africa and
the Middle East and restricted entry from the heavily Catholic and
Jewish countries of southern and eastern Europe.

Throughout these years, the burgeoning civil liberties movement clung to
``un-American'' as a useful critique, insisting that restrictive speech
and immigration laws violated essential American principles. By the
1930s, though, ``un-American'' had begun to harden as a language of
repression, not resistance. In 1938, the House of Representatives
created its infamous Committee on Un-American Activities (HUAC). After
brief forays into investigating domestic fascism, the committee focused
almost exclusively on American communists, deemed dangerous both for
their suspected ties to the Soviet Union and for their ``un-American''
ideas about labor rights and racial equality.

The committee did target a few suspected spies, but HUAC's tribunals
served primarily to decimate the American left, creating an atmosphere
in which communists and noncommunists alike feared expressing their
political opinions. Even at the height of this conflict, though, a few
dissenters held on to an alternate view. By 1948, one F.B.I. memo noted,
some Americans were starting to see ``the House Committee on Un-American
Activities as being un-American, itself.''

For today's Democratic leadership, this viewpoint appears to hold
considerable appeal. In her now-notorious 2013 speech at Goldman Sachs,
Hillary Clinton warned against demagogues who ``play on people's fears''
to promote their ``backward-looking'' agenda. ``They have to be
rejected,'' she declared, ``because they are fundamentally
un-American.'' Two years later, Obama made a speech championing the
children of undocumented immigrants: ``When I hear folks talking as if
somehow these kids are different than my kids or less worthy in the eyes
of God, that somehow they are less worthy of our respect and
consideration and care, I think that's un-American,'' he said. ``I think
we should do better, because that's how America was made.''

The problem, of course, is that this is \emph{not} how America was made.
As a point of strategy, it may behoove Democrats to embrace patriotic or
nationalistic language --- to insist that there is more than one way to
make America great. As a matter of history, however, this tends to
obscure the bitter and enduring conflicts of the past. During election
season, many Democrats apparently believed their own story, assuming
that Americans were too dedicated to the expansion of liberty to elect
Donald Trump. His victory is a reminder that, despite the country's
fondness for aspirational rhetoric, our illiberal traditions have
serious staying power, too.

Today's Republicans seem to recognize this perfectly well. In the past
three years, Attorney General Jeff Sessions has praised the 1924
immigration act as a matter of national self-preservation. Newt Gingrich
has
\href{http://www.cnn.com/2016/06/14/politics/newt-gingrich-house-un-american-activities-committee/}{called
for a revival} of HUAC to contend with ISIS sympathizers. And the Ku
Klux Klan and American fascist movements have experienced levels of
attention and influence not seen since before World War II. Trump
complains that he is being subjected to ``witch hunts'' and
``McCarthyism'' but has advocated policies --- like a ``registry'' for
Muslims --- that hark back to the list-making and name-naming of the
HUAC era.

For the president's opponents, it may be comforting to believe that such
policies are un-American. But as the editors of The St. Louis
Post-Dispatch noted more than a century ago, ``the things we hurry to
denounce as un-American'' are often ``peculiarly and distinctly
American.''

``A better and broader view,'' they concluded, ``seems to be that these
problems are of our own creation, and are to be solved only by
ourselves.''

Advertisement

\protect\hyperlink{after-bottom}{Continue reading the main story}

\hypertarget{site-index}{%
\subsection{Site Index}\label{site-index}}

\hypertarget{site-information-navigation}{%
\subsection{Site Information
Navigation}\label{site-information-navigation}}

\begin{itemize}
\tightlist
\item
  \href{https://help.nytimes3xbfgragh.onion/hc/en-us/articles/115014792127-Copyright-notice}{©~2020~The
  New York Times Company}
\end{itemize}

\begin{itemize}
\tightlist
\item
  \href{https://www.nytco.com/}{NYTCo}
\item
  \href{https://help.nytimes3xbfgragh.onion/hc/en-us/articles/115015385887-Contact-Us}{Contact
  Us}
\item
  \href{https://www.nytco.com/careers/}{Work with us}
\item
  \href{https://nytmediakit.com/}{Advertise}
\item
  \href{http://www.tbrandstudio.com/}{T Brand Studio}
\item
  \href{https://www.nytimes3xbfgragh.onion/privacy/cookie-policy\#how-do-i-manage-trackers}{Your
  Ad Choices}
\item
  \href{https://www.nytimes3xbfgragh.onion/privacy}{Privacy}
\item
  \href{https://help.nytimes3xbfgragh.onion/hc/en-us/articles/115014893428-Terms-of-service}{Terms
  of Service}
\item
  \href{https://help.nytimes3xbfgragh.onion/hc/en-us/articles/115014893968-Terms-of-sale}{Terms
  of Sale}
\item
  \href{https://spiderbites.nytimes3xbfgragh.onion}{Site Map}
\item
  \href{https://help.nytimes3xbfgragh.onion/hc/en-us}{Help}
\item
  \href{https://www.nytimes3xbfgragh.onion/subscription?campaignId=37WXW}{Subscriptions}
\end{itemize}
