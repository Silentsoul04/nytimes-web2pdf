Sections

SEARCH

\protect\hyperlink{site-content}{Skip to
content}\protect\hyperlink{site-index}{Skip to site index}

\href{https://www.nytimes3xbfgragh.onion/section/politics}{Politics}

\href{https://myaccount.nytimes3xbfgragh.onion/auth/login?response_type=cookie\&client_id=vi}{}

\href{https://www.nytimes3xbfgragh.onion/section/todayspaper}{Today's
Paper}

\href{/section/politics}{Politics}\textbar{}In Major Defeat for Trump,
Push to Repeal Health Law Fails

\url{https://nyti.ms/2mYBXRC}

\begin{itemize}
\item
\item
\item
\item
\item
\item
\end{itemize}

Advertisement

\protect\hyperlink{after-top}{Continue reading the main story}

Supported by

\protect\hyperlink{after-sponsor}{Continue reading the main story}

\hypertarget{in-major-defeat-for-trump-push-to-repeal-health-law-fails}{%
\section{In Major Defeat for Trump, Push to Repeal Health Law
Fails}\label{in-major-defeat-for-trump-push-to-repeal-health-law-fails}}

\includegraphics{https://static01.graylady3jvrrxbe.onion/images/2017/03/25/us/25trump-republicans-3/25trump-republicans-3-videoSixteenByNine3000.jpg}

By \href{https://www.nytimes3xbfgragh.onion/by/robert-pear}{Robert
Pear}, \href{http://www.nytimes3xbfgragh.onion/by/thomas-kaplan}{Thomas
Kaplan} and
\href{http://www.nytimes3xbfgragh.onion/by/maggie-haberman}{Maggie
Haberman}

\begin{itemize}
\item
  March 24, 2017
\item
  \begin{itemize}
  \item
  \item
  \item
  \item
  \item
  \item
  \end{itemize}
\end{itemize}

WASHINGTON --- House Republican leaders, facing a revolt among
conservatives and moderates in their ranks, pulled legislation to repeal
the Affordable Care Act from consideration on the House floor Friday in
a major defeat for President Trump on the first legislative showdown of
his presidency.

``We're going to be living with Obamacare for the foreseeable future,''
the House speaker, Paul D. Ryan, conceded.

The failure of the Republicans' three-month blitz to repeal President
Barack Obama's signature domestic achievement exposed deep divisions in
the Republican Party that the election of a Republican president could
not mask. It cast a long shadow over the ambitious agenda that Mr. Trump
and Republican leaders had promised to enact once their party assumed
power at both ends of Pennsylvania Avenue.

And it was the biggest defeat of Mr. Trump's young presidency, which has
suffered many. His travel ban has been blocked by the courts.
Allegations of questionable ties to the Russian government forced out
his national security adviser, Michael T. Flynn. Tensions with key
allies such as Germany, Britain and Australia are high, and Mr. Trump's
approval ratings are at historic lows.

Republican leaders were willing to tolerate Mr. Trump's foibles with the
promise that he would sign into law their conservative agenda. The
collective defeat of the health care effort could strain that tolerance.

Mr. Trump, in a telephone interview moments after the bill was pulled,
tried to put the most flattering light on it. ``The best thing that
could happen is exactly what happened --- watch,'' he said.

\includegraphics{https://static01.graylady3jvrrxbe.onion/images/2017/03/24/us/24healthcare-trump/24healthcare-trump-videoSixteenByNine3000-v9.jpg}

``Obamacare unfortunately will explode,'' Mr. Trump said later. ``It's
going to have a very bad year.'' At some point, he said, after another
round of big premium increases, ``Democrats will come to us and say,
`Look, let's get together and get a great health care bill or plan
that's really great for the people of our country.'''

Mr. Trump expressed weariness with the effort, though its failure took a
fraction of the time that Democrats devoted to enacting the Affordable
Care Act in 2009 and 2010. ``It's enough already,'' the president said.

A major reason for the bill's demise was the opposition of members of
the conservative House Freedom Caucus, which wanted more aggressive
steps to lower insurance costs and to dismantle federal regulation of
insurance products.

In a day of high drama, Mr. Ryan rushed to the White House shortly after
noon on Friday to tell Mr. Trump he did not have the votes for a repeal
bill that had been promised for seven years --- since Mr. Obama signed
the landmark health care law. During a 3 p.m. phone call, the two men
decided to withdraw the bill rather than watch its defeat on the House
floor.

Mr. Trump later told journalists in the Oval Office that Republicans
were 10 to 15 votes short of what they needed to pass the repeal bill.

The effort to win passage had been relentless, and hardly hidden. Vice
President Mike Pence and Tom Price, the health secretary, visited
Capitol Hill on Friday for a late appeal to House conservatives, but
their pleas fell on deaf ears.

\includegraphics{https://static01.graylady3jvrrxbe.onion/images/2017/03/25/us/25ryan-presser-01/25ryan-presser-01-videoSixteenByNineJumbo1600-v3.jpg}

``You can't pretend and say this is a win for us,'' said Representative
Mark Walker of North Carolina, the chairman of the conservative
Republican Study Committee, who conceded it was a ``good moment'' for
Democrats.

``Probably that champagne that wasn't popped back in November may be
utilized this evening,'' Mr. Walker said.

At 3:30 p.m. on Friday, Mr. Ryan called Republicans into a closed-door
meeting to deliver the news that the bill would be withdrawn, with no
plans to try again. The meeting lasted five minutes. One of the
architects of the House bill, Representative Greg Walden, Republican of
Oregon and the chairman of the Energy and Commerce Committee, put it
bluntly: ``This bill's done.''

``We are going to focus on other issues at this point,'' he said.

The Republican bill would have repealed tax penalties for people without
health insurance, rolled back federal insurance standards, reduced
subsidies for the purchase of private insurance and set new limits on
spending for Medicaid, the federal-state program that covers more than
70 million low-income people. The bill would have repealed hundreds of
billions of dollars in taxes imposed by the Affordable Care Act and
would also have cut off federal funds to Planned Parenthood for one
year.

Mr. Ryan had said the bill included ``huge conservative wins.'' But it
never won over conservatives who wanted a more thorough eradication of
the Affordable Care Act. Nor did it have the backing of more moderate
Republicans who were anxiously aware of the Congressional Budget
Office's assessment that the bill would leave 24 million more Americans
without insurance in 2024, compared with the number who would be
uninsured under the current law.

The budget office also warned that in the short run, the Republicans'
legislation would drive insurance premiums higher. For older Americans
approaching retirement, the cost of insurance could have risen sharply.

With the House's most hard-line conservatives holding fast against the
bill, support for the legislation collapsed Friday after more and more
Republicans came out in opposition. They included Representatives Rodney
Frelinghuysen of New Jersey, the soft-spoken chairman of the House
Appropriations Committee, and Barbara Comstock of Virginia, whose
suburban Washington district went for the Democratic presidential
nominee, Hillary Clinton, in November.

``Seven years after enactment of Obamacare, I wanted to support
legislation that made positive changes to rescue health care in
America,'' Mr. Frelinghuysen said. ``Unfortunately, the legislation
before the House today is currently unacceptable as it would place
significant new costs and barriers to care on my constituents in New
Jersey.''

The bill died after Republican leaders, in a bid for conservative
support, agreed to eliminate federal standards for the minimum benefits
that must be provided by certain health insurance policies.

``It's so cartoonishly malicious that I can picture someone twirling
their mustache as they drafted it in their secret Capitol lair last
night,'' said Representative Jim McGovern, Democrat of Massachusetts.
``Republicans are killing the requirements that insurance plans cover
essential health benefits'' such as emergency services, maternity care,
mental health care, substance abuse treatment and prescription drugs.

Mr. Trump blamed Democrats for the bill's defeat, and they proudly
accepted responsibility.

``Let's just, for a moment, breathe a sigh of relief for the American
people that the Affordable Care Act was not repealed,'' said
Representative Nancy Pelosi of California, the House Democratic leader.

Defeat of the bill could be a catalyst if it forces Republicans and
Democrats to work together to improve the Affordable Care Act, which
members of both parties say needs repair. Democrats have been saying for
weeks that they want to work with Republicans on such changes, but
first, they said, Republicans must abandon their drive to repeal the
law.

\includegraphics{https://static01.graylady3jvrrxbe.onion/images/2017/03/25/us/25HEALTH-01-ryan/25HEALTH-01-ryan-articleLarge.jpg?quality=75\&auto=webp\&disable=upscale}

``Obamacare is the law of the land,'' Mr. Ryan said. ``It's going to
remain the law of the land until it's replaced.''

Whatever success Mr. Trump had in making business deals, he utterly
failed in his first effort at cutting a deal at the pinnacle of power in
Washington, Democrats said.

``This is not the art of the deal,'' said Representative Lloyd Doggett,
Democrat of Texas, alluding to Mr. Trump's best-selling book. ``It is
the art of the steal, of taking away insurance coverage from families
that really need it to provide tax breaks for those at the very top.''

Rejection of the repeal bill may prompt Republicans to reconsider the
political strategy they were planning to use for the next few years.

``We have to do some soul-searching internally to determine whether or
not we are even capable of functioning as a governing body,'' said
Representative Kevin Cramer, Republican of North Dakota. ``If `no' is
your goal, it's the easiest goal in the world to reach.''

Representative Robert Pittenger, Republican of North Carolina, offered
this advice to hard-line conservatives who helped sink the bill:
``Follow the example of Ronald Reagan. He was a master; he built
consensus. He would say, `I'll take 80 percent and come back for the
other 20 percent later.'''

Failure of the House effort leaves the Affordable Care Act in place,
with all the features Republicans detest.

``We tried our hardest,'' said Representative Michael C. Burgess of
Texas, chairman of the Energy and Commerce subcommittee on health.
``There were people who were not interested in solving the problem. They
win today.''

``The Freedom Caucus wins,'' he added. ``They get Obamacare forever.''

Advertisement

\protect\hyperlink{after-bottom}{Continue reading the main story}

\hypertarget{site-index}{%
\subsection{Site Index}\label{site-index}}

\hypertarget{site-information-navigation}{%
\subsection{Site Information
Navigation}\label{site-information-navigation}}

\begin{itemize}
\tightlist
\item
  \href{https://help.nytimes3xbfgragh.onion/hc/en-us/articles/115014792127-Copyright-notice}{©~2020~The
  New York Times Company}
\end{itemize}

\begin{itemize}
\tightlist
\item
  \href{https://www.nytco.com/}{NYTCo}
\item
  \href{https://help.nytimes3xbfgragh.onion/hc/en-us/articles/115015385887-Contact-Us}{Contact
  Us}
\item
  \href{https://www.nytco.com/careers/}{Work with us}
\item
  \href{https://nytmediakit.com/}{Advertise}
\item
  \href{http://www.tbrandstudio.com/}{T Brand Studio}
\item
  \href{https://www.nytimes3xbfgragh.onion/privacy/cookie-policy\#how-do-i-manage-trackers}{Your
  Ad Choices}
\item
  \href{https://www.nytimes3xbfgragh.onion/privacy}{Privacy}
\item
  \href{https://help.nytimes3xbfgragh.onion/hc/en-us/articles/115014893428-Terms-of-service}{Terms
  of Service}
\item
  \href{https://help.nytimes3xbfgragh.onion/hc/en-us/articles/115014893968-Terms-of-sale}{Terms
  of Sale}
\item
  \href{https://spiderbites.nytimes3xbfgragh.onion}{Site Map}
\item
  \href{https://help.nytimes3xbfgragh.onion/hc/en-us}{Help}
\item
  \href{https://www.nytimes3xbfgragh.onion/subscription?campaignId=37WXW}{Subscriptions}
\end{itemize}
