Trump vs. Congress: Now What?

\url{https://nyti.ms/2mDGP36}

\begin{itemize}
\item
\item
\item
\item
\item
\item
\end{itemize}

\includegraphics{https://static01.graylady3jvrrxbe.onion/images/2017/04/02/magazine/02trump6/02trump6-articleLarge-v2.jpg?quality=75\&auto=webp\&disable=upscale}

Sections

\protect\hyperlink{site-content}{Skip to
content}\protect\hyperlink{site-index}{Skip to site index}

Feature

\hypertarget{trump-vs-congress-now-what}{%
\section{Trump vs. Congress: Now
What?}\label{trump-vs-congress-now-what}}

After the president suffered his first defeat on Capitol Hill, can the
White House still make good on its legislative promises?

Donald Trump.Credit...Christopher Anderson/Magnum, for The New York
Times

Supported by

\protect\hyperlink{after-sponsor}{Continue reading the main story}

By \href{http://www.nytimes3xbfgragh.onion/by/robert-draper}{Robert
Draper}

\begin{itemize}
\item
  March 26, 2017
\item
  \begin{itemize}
  \item
  \item
  \item
  \item
  \item
  \item
  \end{itemize}
\end{itemize}

On Monday, Jan. 9, less than two weeks before President Trump's
inauguration, the House speaker, Paul Ryan, hosted a dinner at his
office in the Capitol with members of Trump's inner circle. The guests
included the president-elect's chief White House strategist, Stephen K.
Bannon; his son-in-law and family consigliere, Jared Kushner; his chief
of staff, Reince Priebus; his economic adviser, Gary Cohn; his nominee
for Treasury secretary, Steven Mnuchin; his incoming deputy chief of
staff, Rick Dearborn; and his legislative-affairs director, Marc Short.
The ostensible purpose of the dinner was to discuss the details of
Trump's legislative agenda --- in particular, the prospects for a
sweeping tax-reform measure that Republicans, and especially Ryan, have
been coveting for the past decade.

It was hoped that the dinner could also establish some sort of common
ground between Ryan and Bannon, the two figures who would arguably wield
the greatest influence over how Trump's campaign promises became law ---
or didn't. Ryan was a fixture among establishment Republicans even
before joining Mitt Romney's presidential ticket in 2012, his previous
labors on the House Budget Committee cementing his reputation as the
charts-and-graphs wizard of fiscal conservatism. Bannon, by contrast,
was a renegade autodidact who read Plato and had seemingly materialized
from nowhere to become the intellectual architect of Trump's campaign
and, later, administration.

Up to this point, Ryan had epitomized to Bannon everything that was
wrong with the Republican Party. Discussing the two parties'
shortcomings, Bannon later told me, ``What's that Dostoyevsky line:
Happy families are all the same, but unhappy families are unhappy in
their own unique ways?'' (He meant Tolstoy.) ``I think the Democrats are
fundamentally afflicted with the inability to discuss and have an adult
conversation about economics and jobs, because they're too consumed by
identity politics. And then the Republicans, it's all this theoretical
Cato Institute, Austrian economics, limited government --- which just
doesn't have any depth to it. They're not living in the real world.''

Breitbart News, the far-right media outlet Bannon ran before becoming
the chief executive of the Trump campaign in August, had described Ryan,
referring to his position on immigration, as ``arguably the most
pro-amnesty G.O.P. lawmaker in Congress'' --- an apostasy of nearly
impeachable proportions from Bannon's perspective. Worst of all, Ryan
all but abandoned Trump during the 2016 campaign. After the leak in
October of the damaging ``Access Hollywood'' tape, Ryan told fellow
Republican House members on a conference call, ``I am not going to
defend Donald Trump --- not now, not in the future.'' A Republican
lawmaker on the call told Trump what Ryan had said, yet another reason
for Bannon to regard himself as Ryan's worst enemy.

But as the dinner progressed, it became clear that Bannon and Ryan
actually had some ideas in common. Over memorably bad chicken Parmesan,
Ryan described his vision for a ``border-adjustment tax,'' which would
levy taxes on imports while offering exemptions for exports. His tax
package would include ``immediate expensing,'' he explained, in which
capital expenditures would be written off against profits in the first
year rather than over time. It also would abolish the alternative
minimum tax and the estate tax.

These were ideas Ryan had been pushing since 2008. Now they had Bannon's
attention. Taken together with a drastic reduction in corporate taxes,
Bannon believed, Ryan's scheme would spur a renaissance of a
manufacturing-based export economy, producing high-income labor in
keeping with Trump's populism. ``I would actually say,'' Bannon
remembers observing admiringly, ``that this tax reform comes as close to
a first step of economic nationalism as there is.''

``I would call it `responsible nationalism,' '' Ryan said, according to
Bannon.

Bannon laughed. ``You're going to have a lot of folks in the Senate say
this is breathtakingly radical.''

He meant it as a compliment. To Bannon, the entire world order --- from
the two political parties to the Wall Street reliance on leveraging to
multiculturalism --- was undergoing an extraordinary realignment, one
made manifest in the 2016 election. According to Bannon's vision,
economic nationalism would reorient priorities to the working class's
benefit. Trade deals, jobs programs, tax incentives, immigration
restrictions, environmental deregulation and even foreign policy would
ultimately serve to restore the primacy of those Trump called ``the
forgotten Americans.''

In March, when I spoke to Trump by phone, I asked him what the term
``economic nationalism'' meant to him. Compared with Bannon's
revolutionary fervor, his reply was surprisingly cautious. ``Well,
`nationalism' --- I define it as people who love the country and want it
to do good,'' he said. ``I don't see `nationalism' as a bad word. I see
it as a very positive word. It doesn't mean we won't trade with other
countries.''

Trump's tone was genial but also a touch defensive. His postelection
honeymoon had been short, if it existed at all. There were the
\href{https://www.nytimes3xbfgragh.onion/2017/02/05/us/politics/trump-white-house-aides-strategy.html}{administrative
intrigues} and
\href{http://fortune.com/2017/01/11/real-donald-trump-twitter-delete-account/}{self-inflicted
Twitter drama}, along with the questions about his
\href{https://www.nytimes3xbfgragh.onion/2017/02/14/us/politics/russia-intelligence-communications-trump.html}{campaign's
contacts with Russia}, which had already forced the
\href{https://www.nytimes3xbfgragh.onion/2017/02/13/us/politics/donald-trump-national-security-adviser-michael-flynn.html}{resignation}
of his national security adviser, Michael Flynn. Still, Trump's
legislative liaisons and their counterparts on Capitol Hill were
doggedly negotiating a rollout of the Trump Era, one that would fulfill
his most significant campaign promises --- those that could not be done
with just a stroke of Trump's own pen but required acts of Congress.

First, Obamacare would be repealed and replaced. Next, an austere budget
would be passed, with emergency funds allotted for the construction of a
wall along the Southern border. Then would come a tax-reform plan,
presumably of the type Ryan and Bannon discussed. And finally, a
bipartisan coalition would deliver a trillion-dollar infrastructure plan
to Trump's desk. If all this came to pass by the end of 2017, it would
lend some credence to Trump's pledge that this would be ``the busiest
Congress we've had in decades.'' But by March, this timetable was
looking like a formidable ``if.''

Trump himself seemed prone to distraction as he spoke to me from the
Oval Office. Though I was asking about his policy aims, his musings
swerved off to other vexations. More than once he denounced as ``fake
news'' reports about his administration's supposed disharmony. He
brought up his
\href{https://www.nytimes3xbfgragh.onion/2017/02/28/us/politics/trump-address-congress.html}{speech}
before the joint session of Congress in February, ``which I hope you
liked, but I certainly have gotten great reviews --- even the people who
hate me gave me the highest review.'' During the call, I could hear
Priebus nearby, occasionally murmuring encouragement.

Trump sounded more clipped and less jaunty on the call than he did
during the discursive chats I had with him last year on the
\href{https://www.nytimes3xbfgragh.onion/2016/05/22/magazine/donald-trump-primary-win.html}{campaign
trail}. The business of governing had little to do with any trade he had
previously practiced. In Congress, he was grappling with an arcane and
famously inefficient ecosystem over which he had little if any control
--- and people he incessantly derided on the campaign trail as being
``all talk and no action.'' I asked him if he still felt that way.
``It's like any other industry,'' he replied, somewhat morosely. ``I've
met some great politicians and some, to be honest, who aren't so hot.''

Trump wanted to make sure that he was given adequate credit for his
achievements, even in his administration's infancy. ``We've only been
here for a tiny speck of time,'' he said, ``and what I've done with
regulations, moving jobs back into the country, what I've done with
airplane pricing and buying is amazing. We've done a lot. I think we've
done more than anybody for this short period of time.'' Abraham Lincoln,
Franklin Roosevelt and Lyndon Johnson would take exception to this
claim. And Trump's significant actions to date have consisted entirely
of executive orders. What he has not yet demonstrated is his ability to
actually shepherd a bill into law.

The only major legislation that congressional committees have even seen
thus far is a bill to repeal and replace Obamacare, which met with a
stunning rebuke from Trump's own party, forcing Ryan to
\href{https://www.nytimes3xbfgragh.onion/2017/03/24/us/politics/health-care-affordable-care-act.html}{withdraw
the measure} on the afternoon of March 24. At this stage of his
presidency, Barack Obama had already signed into law his \$787 billion
economic-stimulus package and had moved on to holding White House
meetings on health care. It's conceivable that Trump could hit Day 100
with only minor symbolic legislative achievements to his name. For him
to avoid this ignominy, the 45th president will have to develop a
rapport with Washington's 535 federal deal makers, including the ones
who ``aren't so hot.''

\textbf{Whether Trump's agenda} succeeds will also depend in no small
measure on the ability of Bannon to expand his game beyond 1600
Pennsylvania Ave. At 63, and with a fortune reported to be in the tens
of millions of dollars --- partly through his investment in the company
that owns the syndication rights to ``Seinfeld'' --- Bannon is regarded
by Trump as a peer in the way that, say, the 45-year-old lifelong
politico Priebus is not. He is also approvingly seen as a fellow
workaholic by the president (whose only known hobbies are golf and
hate-watching CNN). And he is a deft operator who has learned from the
successes and failures of other Trump advisers. He has carefully not
claimed credit that the president would wish for himself and avoids
giving expansive interviews on his own controversial views that might
detract from his boss's celebrity. Like the former campaign manager
Corey Lewandowski, Bannon understands that power in Trump World derives
mainly from close and sustained physical proximity to the boss. Unlike
Lewandowski, Bannon immediately grasped the importance of maintaining
close relations with Jared Kushner, who factored heavily in
Lewandowski's dismissal from the Trump campaign last summer.

But like Kushner, Bannon has never worked in government or at a
policy-making institute and has no meaningful experience when it comes
to getting legislation passed. On the Hill, he has a few random
associations --- Senator Rand Paul of Kentucky and Representative John
Culberson of Texas among them. Otherwise, he remains a looming but
indistinct presence to the lawmakers who will be needed to pass most of
Trump's agenda.

Bannon's interest in this agenda predated his association with Trump.
One evening in January 2013, two guests showed up for dinner at the
Capitol Hill townhouse that Bannon liked to call the Breitbart Embassy.
One was the man Bannon would later describe to me as his ``mentor'':
Senator Jeff Sessions of Alabama. The other was Sessions's top aide and
protégé, a jittery 27-year-old named Stephen Miller.

\includegraphics{https://static01.graylady3jvrrxbe.onion/images/2017/04/02/magazine/02trump1/02trump1-articleInline.jpg?quality=75\&auto=webp\&disable=upscale}

Two months earlier, Obama decisively defeated Mitt Romney in the
presidential election, prompting Priebus, then the chairman of the
Republican National Committee, to commission an analysis of the state of
the party and its future, known colloquially in Washington as the
``autopsy,'' which would be delivered that spring. The only certainty
was that the report would urge Republicans to court the growing Latino
electorate --- which had voted for Obama by a 44-point margin that
November --- by championing comprehensive immigration reform. The three
men at the dinner table that night were among the few Republicans in
town who thoroughly rejected that conclusion.

Bannon wanted to talk to Sessions and Miller about a different report:
an article written by Sean Trende, the senior elections analyst for the
website RealClearPolitics, titled
``\href{http://www.realclearpolitics.com/articles/2012/11/08/the_case_of_the_missing_white_voters_116106.html}{The
Case of the Missing White Voters}.'' Trende observed that Obama's
victory was less a function of increased minority turnout than of the
fact that 6.6 million white voters who participated in the 2008 election
stayed home in 2012. The reason for this drop, Trende argued, was that
white working-class voters who did not approve of Obama but were
alienated by Romney's perceived elitism had not voted.

These votes were gettable, Bannon believed. As he would later tell me:
``The working class, and in particular the lower middle class,
understands something that's so obvious --- which is that they've
basically underwritten the rise of China. Their jobs, their raises,
their retirement accounts have all fueled the private equity and venture
capital that built China. Because China's really built on investments
and exports, right? People are smart enough to know that they're getting
played by both political parties. The two may be different on social
issues, but when it comes to fundamental economics, they're both the
same. That's why the American working class is interested in trade. It's
linked to their lives.''

Sessions shared Bannon's belief that the Republican Party needed to
emphasize immigration reduction, border security and the preservation of
working-class jobs through trade policy rather than courting Latino
voters with a bill he regarded as ``amnesty.'' As Sessions would write
in a memorandum to his Republican colleagues six months later, ``This
humble and honest populism --- in contrast to the administration's cheap
demagoguery --- would open the ears of millions who have turned away
from our party.''

At some point during the five-hour dinner, Bannon recalls blurting out
to Sessions, ``We have to run you for president.'' Just two years
earlier, in 2011, he made a similar pitch to Sarah Palin, after
completing a documentary about her called ``The Undefeated.'' Palin
demurred. She was enjoying her life of celebrity and wealth, she had
done little to immerse herself in policy minutiae and she was no doubt
unsettled by Bannon's warning that she stood little chance of defeating
Obama.

Now he delivered a similar message to Sessions. ``Look, you're not going
to win,'' he recalls saying. ``But you can get the Republican
nomination. And once you control the apparatus, you can make fundamental
changes. Trade is No.100 on the party's list. You can make it No.1.
Immigration is No.10. We can make it No.2.'' Acknowledging that the
drawling Alabama senator lacked Palin's charisma, Bannon said, ``You'll
be the anti-candidate.'' But Sessions told Bannon he did not see himself
running for president. ``It was pretty obvious by the end of the
night,'' Bannon recalled, ``that another candidate would have to do
it.''

Two months later, on March 15, 2013, Bannon happened to be attending the
Conservative Political Action Conference in Washington when Trump took
the stage. Trump had been a marginal figure at most in politics up to
that point, entertaining a Reform Party run in the 2000 election ---
when he speculated that he would probably take more votes from the
Democratic candidate than the Republican one --- and leading a
conspiratorial crusade in 2011 to force Obama to release his birth
certificate. The possibility that he might be a suitable host body for
Bannon's worldview had not occurred to Bannon before Trump spoke.

But Trump's grousing references to China's economic superiority, to 11
million ``illegals'' and to the erosion of America's manufacturing
sector were right out of Bannon's playbook. From his desk in the Russell
Senate Office Building, Stephen Miller, too, watched Trump's speech. By
2014, Miller was sending emails to friends expressing the hope that
Trump would run for president. By the time Trump announced his
candidacy, in June 2015, Sessions was officially uncommitted but
privately of the view that Trump was best suited to tap into the
movement that he, Miller and Bannon discussed over dinner more than two
years earlier.

Bannon's early support for Trump was manifest in Breitbart's breathless
coverage of his candidacy. In an email he sent on Aug. 30, 2015, to his
former filmmaking partner Julia Jones, Bannon explained that while
Republican candidates like Ted Cruz, Bobby Jindal, Ben Carson and Carly
Fiorina were ``all great,'' Trump represented a superior choice, because
he ``is a nationalist who embraces Senator Sessions's plan'' on
immigration. Still, recalls Sam Nunberg, Trump's first campaign
strategist, ``Steve kept all of his cards.'' He added: ``He was
respectful to some of the other ones who were running, like Walker and
Cruz and Carson. He didn't want to be seen as Trump-bart.'' When Trump
\href{https://www.nytimes3xbfgragh.onion/2015/07/19/us/politics/trump-belittles-mccains-war-record.html}{publicly
disparaged} John McCain's war-hero credentials, Bannon --- himself a
Navy veteran --- called Nunberg and demanded that Trump issue an
apology. (Trump did not.)

Image

President Trump delivering his first address to a joint session of
Congress on Feb. 28.Credit...Jim Lo Scalzo/EPA/Anadolu Agency/Getty
Images

Bannon was well positioned as a supportive but not sycophantic observer
by Aug. 13, 2016, when the Trump donor Rebekah Mercer read with alarm a
\href{https://www.nytimes3xbfgragh.onion/2016/08/14/us/politics/donald-trump-campaign-gop.html}{New
York Times account} of the Trump campaign's inability to handle its
mercurial candidate. At Mercer's behest, Bannon (whose website Mercer's
family helped underwrite) and Kellyanne Conway (who at that point was
receiving money from both a Mercer family political action committee and
the Trump campaign) flew out that day to East Hampton, N.Y., where Trump
was attending a dinner fund-raiser at the home of the New York Jets'
owner, Woody Johnson. After the dinner, Bannon and Conway huddled with
the candidate. Bannon remembers telling Trump, who at the time was
trailing Hillary Clinton by double digits in the polls, ``As long as you
stick to the message'' --- by which he meant economic nationalism ---
``you have a 100 percent probability of winning.''

A week after the election, in an
\href{http://www.hollywoodreporter.com/news/steve-bannon-trump-tower-interview-trumps-strategist-plots-new-political-movement-948747}{interview}
with the journalist Michael Wolff, Bannon offered a bold, sweeping
sketch of what the vision might mean in policy terms: ``Like
{[}Andrew{]} Jackson's populism, we're going to build an entirely new
political movement. It's everything related to jobs. The conservatives
are going to go crazy.'' Of course, some of the conservatives Bannon
intended to drive crazy possessed the congressional votes Bannon and
Trump would need to advance this agenda. Representative Jim Jordan of
Ohio, a leading conservative in the House, told me in March, ``I would
argue that populism, as long as it's rooted in conservative principle,
is a darn good thing.'' Jordan was smiling as he said it, but the note
of warning was hard to mistake.

\textbf{The last time} the Republican Party controlled all branches of
government in Washington was from 2003 to 2007. During that period, the
United States military toppled Saddam Hussein, Congress delivered tax
cuts for the wealthy and President George W. Bush appointed the reliably
conservative jurist Samuel A. Alito Jr. to the Supreme Court.

But in the collective view of conservatives, these years of the Bush
presidency were mostly characterized by betrayal and disappointment.
Goaded by Bush, congressional Republicans passed into law a new federal
entitlement (prescription drugs for senior citizens, also known as
Medicare Part D), ran up the deficit, promoted democratic ideals
overseas in the feckless manner of Woodrow Wilson, considered a pathway
to citizenship for undocumented immigrants and confirmed a Supreme Court
chief justice, John G. Roberts Jr., whose swing vote would later
\href{http://www.nytimes3xbfgragh.onion/2012/06/29/us/supreme-court-lets-health-law-largely-stand.html}{save
Obamacare} from judicial evisceration. ``My go-to line when I first ran
in 2008 was, `Republicans had the House, the Senate and the White House
--- and they blew it,' '' Representative Jason Chaffetz of Utah, the
chairman of the House Oversight and Government Reform Committee, told
me. ``Now we've got all three again, and I'm the guy who's in Congress,
not running for it. I don't want to be in a position where we're going
to blow it one more time.''

Chaffetz and other House conservatives freely acknowledge that Trump is
not cut from their cloth, but they say they could not care less as long
as he gives them what they want. Selecting Judge Neil Gorsuch to fill
the Supreme Court seat once held by Justice Antonin Scalia was ``the
best thing the president did in his first 50 days,'' Chaffetz told me.
He and his conservative colleagues have been cheered by Trump's
recruitment of former House colleagues and conservative stalwarts like
Vice President Mike Pence; Tom Price, the health and human services
secretary; and Mick Mulvaney, the Office of Management and Budget
director.

When Chaffetz and I spoke in March, he had met with the president twice
so far --- access he considered ``such a huge sea change'' from the
stony silence Republicans say they encountered from the Obama White
House. Most important, the Trump agenda's first three projected
legislative moves --- the Obamacare repeal and replacement, an austere
budget and tax reform --- were intended to keep conservatives happily in
Trump's camp. In turn, when the agenda moved on to less conservative
items like infrastructure and trade agreements, Trump and Bannon would
fully expect Republicans, including Ryan, to remember whose message
resonated most with working-class voters last year.

Representative Kevin McCarthy, the House majority leader, is Trump's
chief point of contact on the Hill. When McCarthy was a college student
and budding entrepreneur in Bakersfield, Calif., in the late 1980s, his
girlfriend at the time, now his wife, Judy, gave him an autographed copy
of Trump's ``The Art of the Deal.'' ``I thought it was great,'' he told
me. In McCarthy's view, Trump is a master of today's media, much as
Lincoln and Kennedy were in their own times. ``He's mastered
instantaneous Twitter,'' he said. ``It's like owning newspapers.''

Trump has found a kindred spirit in McCarthy, a coastal extrovert of
ambiguous ideological portfolio who (unlike Ryan) would far rather talk
about personalities than the tax code. And as the former minority leader
in the California Legislature during the governorship of Arnold
Schwarzenegger, McCarthy is experienced in the care and feeding of
celebrity egos. Since Trump's nomination, the two have spoken frequently
by phone --- to date, Trump has never been known to directly email or
text anyone --- about the cast of 535 characters with whom the president
must now deal.

But in the end, what Trump needs from the majority leader is not gossip
but votes --- 216 of them, to be exact, in the House. And McCarthy's
recent track record in obtaining majorities has not been the greatest.
In his previous capacity as House whip, he was thwarted by members of
his own party when it came to subjects as diverse as reauthorizing a
Patriot Act they deemed too intrusive, a farm bill they considered too
expensive and a border-security bill they regarded as too lenient. His
most reliable obstacles have been the three dozen or so House
conservatives known as the Freedom Caucus, a two-year-old group of
fiscal hard-liners. Early this year, McCarthy predicted to me that the
new president would quickly subjugate the Freedom Caucus. ``Trump is
strong in their districts,'' McCarthy told me. ``There's not a place for
them to survive in this world.''

Image

Stephen K. Bannon in the White House on March 13.Credit...Nicholas
Kamm/AFP/Getty Images

When we spoke on the morning of March 7, Trump assured me that he would
not bully the Obamacare-replacement bill's loudest Republican critics,
like the Freedom Caucus chairman, Representative Mark Meadows, on
Twitter: ``No, I don't think I'll have to,'' he said. ``Mark Meadows is
a great guy and a friend of mine. I don't think he'd ever disappoint me,
or the party. I think he's great. No, I would never call him out on
Twitter. Some of the others, too. I don't think we'll need to. Now,
they're fighting for their turf, but I don't think they're going to be
obstructionists. I spoke to Mark. He's got some ideas. I think they're
very positive.''

But on March 21, in a meeting with the Freedom Caucus about the bill,
Trump called out Meadows by name, saying, ``I'm going to come after you,
but I know I won't have to, because I know you'll vote `yes.' '' Meadows
remained a ``no'' on the bill, and among conservatives, he was far from
alone. One of the Freedom Caucus's most outspoken members,
Representative Raúl Labrador of Idaho, believes that the Trump White
House was led astray by Ryan's confidence that he knew what
conservatives wanted when drafting the bill. ``The legislation has to go
through the body, not the top,'' Labrador told me. ``And if our
leadership thinks now that we're a unified body, that they can do things
while ignoring us, that's not going to happen.''

Labrador is an affable but decidedly stubborn 49-year-old Mormon and
former immigration lawyer who moved as a child with his single mother
from Puerto Rico to Las Vegas. He was interviewed by the president-elect
for the post of interior secretary at Trump Tower last December ---
though Trump selected Labrador's House colleague Ryan Zinke for the post
a few days later.

For now, Labrador and other Freedom Caucus members have been willing to
blame House leaders like Ryan and McCarthy for drafting a health care
bill that was not to conservatives' liking. They aspire to remain
philosophical whenever Trump's daughter Ivanka persuades her father to
propose initiatives like paid family leave, as he did during his
joint-session speech. ``I didn't stand up when he said that,'' Labrador
said. ``That's the only part of the speech where I thought, That's not
even close to what my party stands for.''

To House conservatives like Labrador, the Republican Party stands for
limited government. To Trump and Bannon, big-ticket items like a border
wall and infrastructure take priority over shrinking America's debt. As
Chaffetz admitted to me, ``On the spending front, things could slip away
really quickly.''

\textbf{Trump's}
\href{https://www.nytimes3xbfgragh.onion/interactive/2017/03/16/us/politics/document-Trump-2018-Budget.html}{\textbf{budget
blueprint}} **** is regarded by deficit hawks as fundamentally
unserious, because it does not touch entitlements. Instead, it ravages
perennial (and already pint-size) conservative piñatas like foreign aid,
public broadcasting and the National Endowment for the Arts, in addition
to downsizing the Environmental Protection Agency and the Interior
Department --- cuts that focus on the 27 percent of the federal budget
that is not mandatory spending or devoted to defense. And for all the
Republicans' chesty rhetoric on cuts like these over the years, as a top
House Republican staff member told me, ``even the cabinet secretaries at
the E.P.A. and Interior are saying these cuts aren't going to happen.
They're going to protect their grant programs, their payments to states,
their Superfunds. So how do you cut 31 percent of the E.P.A. out of the
5 percent that isn't protected? And a bill that cuts all money for the
N.E.A. will not pass. For Republicans in the West'' --- states whose
vast rural areas benefit disproportionately from N.E.A. grants ---
``that's a re-election killer. The campaign commercials write
themselves.''

Labrador says he would defend Trump's cuts but doubts that many of his
colleagues would. ``What he's going to learn is that members of Congress
are unwilling to take the tough votes,'' he told me. ``When he learns
that, what's going to be the next step?'' In Labrador's view, Trump's
only sane recourse will be to accept the need for entitlement reform.
``At some point, the reality of the budget is going to have to hit
him,'' he said. ``You can have this economic nationalism --- Bannon is
very smart, he clearly helped him with his messaging, it was so
successful --- but at some point, that theory is going to hit reality.''

When I spoke with Trump, I ventured that, based on available evidence,
it seemed as though conservatives probably shouldn't hold their breath
for the next four years expecting entitlement reform. Trump's reply was
immediate. ``I think you're right,'' he said. In fact, Trump seemed much
less animated by the subject of budget cuts than the subject of spending
increases. ``We're also going to prime the pump,'' he said. ``You know
what I mean by `prime the pump'? In order to get this'' --- the economy
--- ``going, and going big league, and having the jobs coming in and the
taxes that will be cut very substantially and the regulations that'll be
going, we're going to have to prime the pump to some extent. In other
words: Spend money to make a lot more money in the future. And that'll
happen.'' A clearer elucidation of Keynesian liberalism could not have
been delivered by Obama.

The one clear point of agreement between the Trump economic nationalists
and the House conservatives is the one Ryan and Bannon identified over
dinner in January: tax reform. But in so doing, they will be picking a
fight that may prove perilous to Republicans. The border-adjustment-tax
proposal that Ryan floated to Bannon has never been able to get past K
Street lobbyists and wealthy Republican donors like the Koch brothers.

Image

A working lunch at The White House on March 1.Credit...Chip
Somodevilla/Getty Images

When I asked Trump if he was a fan of the border-adjustment tax, he
replied: ``I am. I'm the king of that.'' Almost no other country grafts
an import tax onto a corporate tax, and it's possible that enacting a
border-adjustment tax might well be in violation of the World Trade
Organization's agreements. Of course, Bannon has openly advocated
abandoning the W.T.O. anyway, because of China's membership in it.
Still, the specter of new taxes on American corporations, higher prices
for consumers and a jump in the dollar's value may compel an unusual
confederacy against the tax-reform plan.

Labrador predicts that the border-adjustment tax ``will have very little
political legs'' in the conservative House, while Senator Lindsey Graham
\href{http://thehill.com/homenews/senate/320306-graham-ryan-tax-plan-wont-get-10-votes-in-the-senate}{said
in February} that even in the Republican-controlled Senate, Ryan's tax
plan ``won't get 10 votes.'' Senator Heidi Heitkamp, a North Dakota
Democrat who has been outspoken in her willingness to work with Trump in
spite of the
\href{https://www.nytimes3xbfgragh.onion/2017/03/13/magazine/democratic-party-election-trump.html}{broader
stance of her party}, says, ``Let me tell you, I represent farmers, and
anyone who tells me that farm country benefits from a high dollar needs
to have a discussion with me.''

Perhaps the Republican faction most alarmed by Bannon's economic
nationalism is Washington's military hawks. John McCain is among those
not mollified by Trump's pledge of enacting ``one of the largest
increases in national-defense spending in American history.'' McCain
scoffed when I brought this up to him. ``Of course that's simply not
true,'' he said. ``When you look at 1981 and Reagan's commitment to
rebuilding the military, there's no comparison to this 3 percent
increase. It's a shell game, my friend.''

Despite his obvious differences with Trump, McCain was willing to work
with him --- but Bannon's presence seemed to confound such prospects.
``It's kind of interesting,'' McCain said, ``because I have decades of
experience with Kelly, with Mattis, with Dan Coats, McMaster,''
referring to Homeland Security Secretary John Kelly; Defense Secretary
James Mattis; Dan Coats, the director of national intelligence; and H.R.
McMaster, the national security adviser. ``We discuss issues all the
time. I think this is probably the finest national-security team that
I've ever observed. It's almost schizophrenic, in that I obviously don't
have conversations with Steve Bannon, but I do with Reince Priebus ---
he was my Republican chair in Wisconsin in my 2008 presidential
campaign. So it's almost a schizophrenic --- that's not the right word.
A very divided kind of relationship. Paradoxical.''

McCain acknowledged to me that economic nationalism was a global
movement and therefore not entirely ``the making of some members of the
Trump entourage.'' He then said: ``But it is an articulation that I
believe is strongly reminiscent of the 1930s. It certainly has unsettled
our allies and friends around the world, there's no doubt about that.''
Already, the senator asserted, the new administration's
\href{https://www.nytimes3xbfgragh.onion/2017/01/24/us/politics/wall-border-trump.html}{bellicosity
toward Mexico} has increased the likelihood that its citizens will elect
``a very left-wing, anti-American president.'' As for an import tax of
the sort favored by Bannon and Ryan, ``talk about harkening back to the
1930s,'' he said. ``It's unbelievable to me that they somehow think if
we start taxing goods coming across the border, that that's somehow not
going to be responded to by the Mexicans. Please. History shows this
sort of action gets you into a trade war.''

\textbf{Listening to McCain's} tirade, I found it evident that the
Bannon Effect might well cost the Trump White House at least one
Republican Senate vote on a number of central issues --- this at a time
when Republicans are clinging to a slender majority in the upper
chamber. In such cases, Trump could find himself asking for something
Obama was never able to count on: votes from the opposition.

Early in the afternoon of Feb. 9, several Democratic senators met with
Trump in the Roosevelt Room of the White House to discuss the Gorsuch
nomination and other matters. Among them were Heidi Heitkamp, Joe
Donnelly of Indiana, Joe Manchin of West Virginia and Jon Tester of
Montana. All four are moderates who are up for re-election in 2018 in
states Trump carried in 2016 by titanic margins --- the least of which,
in Donnelly's state, was nearly 20 points. If Democrats are to nurture
any hopes of retaking the Senate majority, they will need to hold these
four seats.

But if Donnelly, Heitkamp, Manchin and Tester need to be seen back home
as willing to work with Trump, the president needs them as well.
Republicans enjoy a precarious 52-to-48 advantage in the Senate. On
matters like the Supreme Court, Trump can count on all 52. On votes
requiring a simple majority, any two of those Republicans could fall
away, and Pence could preserve the win with a tiebreaking vote. But a
trio of fiscal hard-liners (like Ted Cruz, Rand Paul and Mike Lee),
military hawks (John McCain, Lindsey Graham and Marco Rubio) or social
moderates (Susan Collins, Lisa Murkowski and Shelley Moore Capito) could
deny Trump a majority, unless he could swing at least one Democrat to
his side.

That February afternoon in the Roosevelt Room, Donnelly thanked Trump
for
\href{https://www.nytimes3xbfgragh.onion/2016/12/01/business/economy/trump-carrier-pence-jobs.html}{negotiating
with Carrier}, the manufacturing company based in Indiana that had
threatened to move jobs to Mexico before Trump arm-twisted it into
keeping many of them in Indiana. But Donnelly urged him not to view that
episode as a ``one-off.'' He requested the president's support for his
End Outsourcing Act, which would give preferential treatment in awarding
federal contracts to businesses that kept jobs in America.

Image

Representatives Raúl Labrador, Mark Meadows and Jim Jordan at the
Capitol on March 24.Credit...Tom Williams/CQ Roll Call, via Getty Images

The words were scarcely out of Donnelly's mouth before Trump said, ``I'm
100 percent for that, and I'll do everything I can to help get it
passed.'' He then asked Pence, who was in the room, ``What do you think,
Mike?'' Trump was apparently unaware that Pence, as the governor of
Donnelly's state, had refused to back the senator's initiative, claiming
instead that burdensome federal regulations were to blame for
outsourcing. According to Donnelly, Pence gamely replied, ``If it's like
what Joe describes, I'll do everything I can to help.''

Donnelly, a thick-handed Irish Catholic with a barroom guffaw, had met
Trump once before. In January 2011, he was among the so-called Blue Dog
Coalition, composed of conservative House Democrats --- what remained of
them, anyway, after the previous November's disastrous midterms --- who
traveled to New York for their annual retreat. At a hotel conference
room in Midtown Manhattan, the 20 or so Blue Dogs received a procession
of guests, including Mayor Michael Bloomberg and former President Bill
Clinton. Only one of their scheduled appointments required that they go
to their guest --- and so they did, by bus, to Trump Tower.

Trump greeted them in his boardroom, with its commanding view of Central
Park. He was charming but also brash. ``Remember, at that point he
wasn't really talking about running for office,'' recalls one attendee,
former Representative Dan Boren of Oklahoma. ``But what strikes me was
how he talked about the same issues --- the wall, China --- that became
his stump speech years later.''

It was evident to the Blue Dogs that Trump was no Clinton or Bloomberg
when it came to the issues. Says former Representative Ben Chandler of
Kentucky, who was also in attendance: ``The difference in terms of
detailed knowledge of policy was stark. Trump just made bald assertions,
really.'' Particularly memorable to Chandler was Trump's insistence
``that one of the best things the country could do was slap a massive
tariff on the Chinese.'' Chandler continued: ``He seemed not to
understand that this would probably cause the entire world economy to
melt down by causing a huge trade war. What I remember more than
anything else was our general reaction afterward. And it was one of
disbelief.''

Today Donnelly remains offended by what he calls Trump's ``crazy
stuff,'' as well as the alternative to Obamacare that Trump supported.
But he does not begrudge Trump his showmanship. ``He came to the Carrier
plant,'' Donnelly said. ``I've been working on that issue since Day 1. I
was begging people in the Obama administration to come out and talk to
our workers. Donald Trump came out there. And Donald Trump talked to our
workers. You can tell people you care. But it matters if you show up.''

The Senate Democrat who, to outward appearances, seems closest to Trump
is Joe Manchin, who met face to face with the president-elect in Trump
Tower in December. Before the meeting, Bannon took the West Virginia
senator aside. ``The thing you need to know about Trump,'' Bannon said,
``is he doesn't care about the Republican Party and he doesn't care
about the Democratic Party. He just wants to put some wins on the board
for the country.'' In the meeting, Trump asked Manchin what could be
done for coal miners. Manchin replied that he should support his Miners
Protection Act, which would secure health benefits and pension funds for
retired miners. According to Manchin, Trump replied that he would
thoroughly support such a measure.

Later that month, Manchin went on ``Morning Joe'' --- the one show on
MSNBC that Trump has been known to watch --- to discuss, on the occasion
of the fourth anniversary of the Newtown school massacre, the need to
expand background checks on gun purchases. Within an hour after Manchin
was offscreen, his cellphone rang. It was Trump. Manchin was not
completely forthcoming about the conversation, but he did tell me that
he envisioned ``a complete opportunity'' for new gun-safety legislation.
Unlike with Obama, he said, ``no one thinks President Trump would do
anything that would take away your gun rights.''

In his conversations with Manchin and Donnelly, Trump was essentially
throwing his support behind a Democratic initiative without first
checking with the Republican Senate majority leader, Mitch McConnell, to
ask what he thought of those proposals. Had he done so, the answer in
each case would have been: not much. (Though on the coal miners'
legislation, Manchin said: ``We're seeing Mitch McConnell go from a `No,
no and hell no' to now dropping his own bill. Which is fine, so long as
we get it.'') Still, Trump may have little choice but to indulge
Democrats on some of their pet issues, given that he will need their
votes on two of the most critical pieces of his agenda: infrastructure
and trade deals.

Until now, Trump has divulged few details about this trillion-dollar
infrastructure venture. On the campaign trail, he frequently cited
America's crumbling roads and bridges. He bemoaned the potholes defiling
the runways at La Guardia Airport, where he parked his two planes.
During Donnelly's visit with Trump in the Roosevelt Room, the president
``talked about the Queens-Midtown Tunnel with the tiles falling off,
which he would see on his way to La Guardia,'' Donnelly recalled. (The
Metropolitan Transportation Authority denies that tiles are falling off
the tunnel.)

Image

Credit...Christopher Anderson/Magnum, for The New York Times

When I asked Trump for more specifics, he gingerly offered a few
morsels: ``This is something that's going to be a real infrastructure
bill, where real work is going to be done on bridges and roads and
airports and things that we're supposed to be doing. So it's not just a
political piece of paper. We're going to do infrastructure, and it's
going to be a very big thing.''

Trump's description struck me as uncharacteristically modest. Bannon had
evoked a more gleaming vision when he told me: ``Look, economic
nationalism is predicated on a state-of-the-art infrastructure for the
country, right? Broadband as good as Korea. Airports as good as China.
Roads as good as Germany. A rail system as good as France. If you're
going to be a world-class power, you've got to have a world-class
infrastructure.''

When I asked the president if his initiative might include such
features, he replied: ``Yes. It could, it could. You look at Japan and
China, where they have the fast trains, and we don't have any. You look
at other countries where we used to be the leader, and now we're the
laggard. It's not going to happen anymore.''

What also may not happen is House Republicans' supporting a
trillion-dollar bill that is at least somewhat reminiscent of the
stimulus bill they unanimously opposed eight years ago. It's also
possible that even moderate Democrats in swing states may face pressure
not to come to Trump's rescue. After all, the president remains
intensely unpopular among Democrats, who continue to nurture hopes that
Trump is one Russia connection away from impeachment. As a senior White
House official told me of Gorsuch's nomination to the Supreme Court:
``The comment we often get from Democrats is, `That's a great nominee.'
Oh, so you're voting for him? `I can't.' Why not? `My base would go
crazy, and I'd be primaried.' That environment has to change before we
can have any of these conversations.''

\textbf{On the morning} of Feb. 2, two Democratic leaders on trade
issues, Senator Ron Wyden of Oregon and Representative Richard Neal of
Massachusetts --- the ranking members of the Senate Finance and House
Ways and Means Committees ---
\href{https://www.nytimes3xbfgragh.onion/2017/02/02/us/politics/trump-tax-imports.html}{met
with Trump}, along with a few of his advisers and Republican lawmakers.
Trump had already greeted the day by threatening to yank federal funding
from the University of California at Berkeley after acts of violence had
forced the cancellation of the Breitbart editor Milo Yiannopoulos's
speech on campus, and by taunting Arnold Schwarzenegger's poor ratings
on ``The Apprentice'' during the National Prayer Breakfast. Disquiet
lingered from Trump's travel ban on refugees and his surly phone
conversation with the Australian prime minister the previous week. Amid
this chaos --- entirely to Bannon's liking and grating to nearly
everyone else in Washington --- actual legislative activity was slowly
unfolding.

Trump began the meeting by condemning the trade deals negotiated by his
predecessors. The press pool was then ushered out before the Democrats
could say anything in front of the cameras. When Neal was given a chance
to speak, he informed Trump, Pence, Bannon, Kushner and Commerce
Secretary Wilbur Ross that America had in fact prospered as a result of
past trade deals. Neal emphasized the crucial role that the Panama Canal
played in the economic vitality of the Eastern Seaboard. Other than
Ross, no one on Trump's team seemed aware of this. ``They were a bit
surprised,'' Neal later told me. He was also struck by the White House's
abhorrence of multilateral pacts, which seemed to him to be naïve. ``The
idea that you're going to negotiate 148 bilateral agreements with W.T.O.
members does not seem realistic,'' Neal said. ``The idea that we're all
of a sudden going to have a free-trade agreement with Great Britain,
that's going to take years to do.'' Later, Neal said, Ross privately
assured him that the Trump administration ``would not give up on
multilateral deals.''

Neal's lecture signified the start of what is likely to be a long and at
times contentious reckoning on the part of Trump and Bannon with the
limits of their nationalist rhetoric. Of all the legislative lifts, none
will be heavier than renegotiating trade agreements, which require a
simple majority approval by both the Senate and the House. Scrounging up
15 Democratic senators who are willing to vote along with 52 Republicans
would be a formidable enough task on any issue. But just as Democrats
like Neal in the Northeast would fight for a trade deal that benefits
their region, so will Republican lawmakers along the Southern border
rebel at an effort to repeal Nafta. As McCain told me, ``If you negated
Nafta, it would send my state into a severe recession.'' He assured me
that Trump's nationalist posture would not provoke only regional
opposition. He conjured up another Republican era --- not Reagan's, not
Bush's, but instead that of Herbert Hoover, when two Republican
lawmakers joined with a Republican president to design a protectionist
initiative that ultimately caused American exports to plummet during the
Great Depression. ``Somewhere,'' McCain said with a dark chuckle, ``Mr.
Smoot and Mr. Hawley are smiling.''

\textbf{On Thursday, March 23,} Trump hosted a morning meeting of
Freedom Caucus holdouts in the Cabinet Room. Jeff Duncan, a congressman
from South Carolina who was present, told me that Trump told them: ``I
need you guys. We need to put up a win. It's not just about needing to
repeal Obamacare --- though we do. It's also that a win here sets up a
win for tax reform and gives us momentum going into infrastructure. And
if the bill fails, it could derail all of that.''

With customary bravado, Trump told the conservative members that he
didn't want to squeak by with just a one-vote victory. ``I want all 237
of you,'' he said, according to Duncan, referring to the entire House
Republican conference. That included the more moderate members, who had
told Trump they felt that the White House wasn't paying sufficient
attention to their concerns. Later in the day, Trump hosted another
meeting with the moderates, where Representative Charlie Dent of
Pennsylvania informed Trump that he remained a ``no.'' According to an
attendee, Trump angrily informed Dent that he was ``destroying the
Republican Party'' and ``was going to take down tax reform --- and I'm
going to blame you.''

Until that day, Duncan had been an unyielding ``no'' on the bill. The
previous week, he delivered an impassioned speech to the vice president
and other Republicans, insisting that this vote constituted ``our
generation's rendezvous with destiny --- a real chance to roll back the
size and scope of the federal government, returning some liberty back to
the people through our actions to repeal Obamacare.'' In a text to me,
Duncan pointed to history: ``39 men in a hot room in 1787 had the
courage to break from the norm and empower a nation.''

But now the four-term congressman was, for the first time in his life,
sitting across the table from a president who was personally appealing
for his support. The White House was offering concessions and agreeing
to them in writing. Duncan left the meeting and spent a few hours
pondering, as he would later put it, ``the greater opportunity we as
Republicans have.'' By that evening, Trump had won Jeff Duncan's vote.

It wasn't enough. The next afternoon, Ryan
\href{https://www.nytimes3xbfgragh.onion/2017/03/24/us/politics/health-care-affordable-care-act.html}{pulled
the House health care bill}, conceding that neither he nor the White
House could muster enough votes.

``You get about nine months to do the big things,'' Kevin McCarthy, the
House majority leader, told me at the beginning of the year. Nine months
seemed like a long time then, the calendar spacious and the legislative
deal-making possibilities plentiful. But more than two of those months
are gone already --- and the path to future wins, as Trump foresaw in
his meeting with the Freedom Caucus, is now more complicated. When he
took office, Trump relished the prospect of becoming a new kind of deal
maker in the White House. By the time I spoke with him in early March,
however, he already seemed to be taking stock of the limits to his
powers. He still saw himself as the closer in chief --- but then that
was ``typical, I would think, of a president,'' he mused. ``Some more
than others.''

Advertisement

\protect\hyperlink{after-bottom}{Continue reading the main story}

\hypertarget{site-index}{%
\subsection{Site Index}\label{site-index}}

\hypertarget{site-information-navigation}{%
\subsection{Site Information
Navigation}\label{site-information-navigation}}

\begin{itemize}
\tightlist
\item
  \href{https://help.nytimes3xbfgragh.onion/hc/en-us/articles/115014792127-Copyright-notice}{©~2020~The
  New York Times Company}
\end{itemize}

\begin{itemize}
\tightlist
\item
  \href{https://www.nytco.com/}{NYTCo}
\item
  \href{https://help.nytimes3xbfgragh.onion/hc/en-us/articles/115015385887-Contact-Us}{Contact
  Us}
\item
  \href{https://www.nytco.com/careers/}{Work with us}
\item
  \href{https://nytmediakit.com/}{Advertise}
\item
  \href{http://www.tbrandstudio.com/}{T Brand Studio}
\item
  \href{https://www.nytimes3xbfgragh.onion/privacy/cookie-policy\#how-do-i-manage-trackers}{Your
  Ad Choices}
\item
  \href{https://www.nytimes3xbfgragh.onion/privacy}{Privacy}
\item
  \href{https://help.nytimes3xbfgragh.onion/hc/en-us/articles/115014893428-Terms-of-service}{Terms
  of Service}
\item
  \href{https://help.nytimes3xbfgragh.onion/hc/en-us/articles/115014893968-Terms-of-sale}{Terms
  of Sale}
\item
  \href{https://spiderbites.nytimes3xbfgragh.onion}{Site Map}
\item
  \href{https://help.nytimes3xbfgragh.onion/hc/en-us}{Help}
\item
  \href{https://www.nytimes3xbfgragh.onion/subscription?campaignId=37WXW}{Subscriptions}
\end{itemize}
