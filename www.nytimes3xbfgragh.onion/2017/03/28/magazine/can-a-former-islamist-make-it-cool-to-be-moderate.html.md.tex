Maajid Nawaz's Radical Ambition

\url{https://nyti.ms/2nHbY5l}

\begin{itemize}
\item
\item
\item
\item
\item
\item
\end{itemize}

\includegraphics{https://static01.graylady3jvrrxbe.onion/images/2017/04/02/magazine/02nawaz1/02mag-02nawaz-t_CA0-articleLarge.jpg?quality=75\&auto=webp\&disable=upscale}

Sections

\protect\hyperlink{site-content}{Skip to
content}\protect\hyperlink{site-index}{Skip to site index}

Feature

\hypertarget{maajid-nawazs-radical-ambition}{%
\section{Maajid Nawaz's Radical
Ambition}\label{maajid-nawazs-radical-ambition}}

The former Islamist has started a foundation to combat extremism among
Muslims --- and has made a lot of enemies in the process.

Maajid Nawaz near Russell Square in London.Credit...Sian Davey for The
New York Times

Supported by

\protect\hyperlink{after-sponsor}{Continue reading the main story}

By Thomas Chatterton Williams

\begin{itemize}
\item
  March 28, 2017
\item
  \begin{itemize}
  \item
  \item
  \item
  \item
  \item
  \item
  \end{itemize}
\end{itemize}

\textbf{I} met Maajid Nawaz on a drizzly afternoon in March, tucked in a
corner of the restaurant at the central London members' club he uses as
a satellite office. He was dabbing the chicken from his Caesar salad
into a mound of yellow English mustard, which he stopped doing for long
enough to load a video on his iPhone and slide it across the table. It
showed the Southern Poverty Law Center's Heidi Beirich, speaking at Duke
University about him. ``Let me just give you an example of Maajid Nawaz
--- our problem with him,'' she says. ``He believes that all mosques
should be surveilled. In other words, his opinion is that all Muslims
are potential terrorists.'' Nawaz, a Muslim himself, bristled with
frustration at the claim. In fact, he explained, he is on record making
the case against collective surveillance.

A former Islamist, for the past nine years Nawaz has made a name for
himself as an indefatigable anti-extremist activist. These days he
blends seamlessly into the sort of cosmopolitan circles that extremists
decry; at his club, dressed in an olive bomber jacket over fitted
workout sweats, he could have been a senior marketing exec or a
music-video director. At 39, Nawaz is handsome and vaguely famous
looking in person, prematurely silver-haired, with a widow's peak and
Mephistophelean soul patch that punctuates a politician's easy smile.
Whenever I saw him, he dapped me with one of those handclasp-half-hugs
that, to anyone of a certain age, serves as shorthand for an adolescence
steeped in the manners of hip-hop.

For Nawaz's detractors, of whom there are many, it's this very chameleon
quality, this at-homeness in disparate roles and spaces, that has earned
him a reputation as something of a charlatan, a preening opportunist
cashing in on his own sensational travails by means of society's
abundant anti-Muslim bias. This uncharitable narrative has shadowed him
from the outset, yet his point of view has only grown more relevant
after an exceptionally violent 2016 that saw coordinated suicide
bombings in Brussels and Istanbul; a mass shooting in a nightclub in
Orlando; the ambush and execution of a police officer and his partner
near Paris; a Bastille Day slaughter in Nice; ax and suicide bomb
attacks in Bavaria; the throat slitting of a Catholic priest in a church
in Normandy; pressure-cooker bombs in Manhattan and New Jersey; and a
massacre at a Christmas market in Berlin. And on March 22 this year in
London, a man mowed down pedestrians with his car near Parliament before
stabbing a police constable to death.

With each grisly new assault --- and the specter of Syria and the
Islamic State looming beyond it --- the voices of hatred and reaction in
the United States and throughout Britain and Europe found not only
sympathetic ears but also willing hands to pull levers in the voting
booths. Throughout the upheaval and backlash, Nawaz has remained a
constant presence in the media: on ``Real Time With Bill Maher,'' trying
to draw a distinction between religion and political dogma; in his book,
``Islam and the Future of Tolerance'' (co-written with the prominent
``new atheist'' Sam Harris), insisting that Islamism does have something
to do with Islam and that ISIS in fact possesses a plausible if terribly
ungenerous interpretation of the Quran. But whatever role Nawaz enjoys
as a public intellectual is inextricable from his personal celebrity as
a former fundamentalist. His work is his story, and his story is his
celebrity. In order to make his case against radicalism, he finds
himself in the not entirely enviable position of nonstop self-promotion.

On this front, he's as busy as ever. He is finishing a documentary based
on his book with Harris, but foremost on Nawaz's mind these days is the
2017 opening of the first new chapter of his anti-extremist
organization, the Quilliam Foundation, in the United States. ``Lots of
Muslims in America are basically liberals, but if you don't have a
visibly anti-extremist presence, then the Trumps of this world win''
through fear-mongering and misrepresentation, he says. ``Our presence is
needed in America to reassure the mainstream, whereas our presence is
needed in Europe to stop radicalization.''

Despite such deliberate affirmations and qualifications, there is
nonetheless confusion as to where Nawaz's sympathies actually lie.
According to Vice News, he has earned a ``terrorism'' designation on
Thomson Reuters World-Check, a risk-assessment database. (Thomson
Reuters would not confirm this.) But, last October, the Southern Poverty
Law Center took the incredible step of including him on a ``Field Guide
to Anti-Muslim Extremists,'' which they published with three other
research organizations. The guide listed 15 public figures, and Nawaz
was the only Muslim among them. (This is why Beirich brought him up at
Duke.) He was visibly furious whenever the topic came up and told me he
plans to crowdfund a legal response.

Though he and his allies, and even some of his opponents, have
complained to the S.P.L.C. --- there is a change.org petition to remove
him and the Somali-born atheist Ayaan Hirsi Ali, which has garnered
almost 12,000 signatures --- the group has not wavered on its position,
the costs of which have already been real for Quilliam. Nawaz claims
that the listing has compromised some funding for the organization. ``I
consider myself a liberal and wanted to work with liberals,'' he said,
shaking his head.

In reality, his views on Islamic extremism are more complex than these
labels allow, which is, arguably, one of the more compelling reasons to
listen to him on the subject.

\textbf{Early in Nawaz's} 2012 memoir, ``Radical: My Journey Out of
Islamist Extremism,'' there's an eyebrow-raising scene. The narrator, an
irreligious, N.W.A.-loving child, has resorted to strapping a knife
under his shirt for fear of the gangs of skinheads that stalk his Essex
suburb, Southend. He is 15, and on this afternoon, he is with his older
brother, Kaashif (identified by a pseudonym in the book), and a friend
who has converted to Islam. Neighborhood racists have chased the boys
with baseball bats and now have them cornered and outnumbered. The
skinheads' leader steps forward and asks to talk. Kaashif gestures to
the side of the road, where he and the skinhead fall into a tense and
private discussion. When the two return, the skinheads begin to retreat.
Incredulous, Maajid demands to know what his brother has told them.
Kaashif says he told the skinhead, ``We're Muslims, and we don't fear
death'' --- and, furthermore, that he was carrying a bomb in his
backpack.

The anecdote, which surfaces repeatedly in ``Radical'' and ultimately
swells to the dimensions of a creation myth, is quintessential Nawaz. On
one hand, it's a distillation of his larger rhetorical project,
capturing the confused and painful textures of contemporary Muslim
experience that can lead to the embrace of Islamism: an initial lack of
familiarity with religion; local grievance spun into a narrative of
global victimization; a tribal relation to other Muslims beyond racial
and ethnic categorization; the illusion of empowerment through threat of
violence. On the other hand, it has become emblematic of the
cantankerous, highly personal discourse that clings to the man himself:
For a number of reasons --- more on which later --- many of his critics
have come to claim that the anecdote is pure fabrication.

\includegraphics{https://static01.graylady3jvrrxbe.onion/images/2017/04/02/magazine/02nawaz2/02mag-02nawaz-t_CA2-articleLarge.jpg?quality=75\&auto=webp\&disable=upscale}

What's indisputable is that soon after that day in Southend, first
following Kaashif's example but then with a fervency that was entirely
his own, Nawaz threw himself into his new identity, falling under the
sway of Nasim Ghani, a charismatic young recruiter and future leader of
the British branch of Hizb ut-Tahrir, a multinational Islamic
revolutionary organization founded in 1953 in Jerusalem. H.T., as Nawaz
refers to it, advocates the imposition of Shariah law through
``bloodless'' coups in majority-Muslim countries first and ultimately in
the West as well. In other words, these were Islamists but not
jihadists, and the distinction isn't frivolous. Still, the line is a
porous one: Two H.T. leaders, Anjem Choudary and Omar Bakri Mohammad,
would go on to lead a splinter group of a far more deadly variety.

In September 2001, after stints of organizing and recruiting for H.T. in
London and Pakistan, Nawaz took his first wife and their infant son to
Alexandria, Egypt, where he posed as an Arabic-language student while
secretly proselytizing for the group. Though H.T. is legal in Britain,
it is banned in many majority-Muslim countries, including Egypt. In
2002, at 24, Nawaz was forcibly removed from his home, blindfolded and
thrown in the back of a van, one more Islamist caught up in the wide and
extralegal international crackdown on extremism in the wake of 9/11. He
spent his next four years in Egyptian prisons, where he claims to have
witnessed torture and where, in his solitude, he was able to memorize
half of the Quran.

A pivotal moment in Nawaz's moral education came when news of the 7/7
attacks in London reached the inmates at Tora, Egypt's prison notorious
for holding political dissidents. Four attackers bombed a bus and three
subway trains, claiming 52 lives. Nawaz writes that he suddenly ``felt
revulsion'' at the human cost of his ideas. A man Nawaz calls Omar, a
Dagestani bomb maker, had celebrated the slaughter. Nawaz is a hero in
his own telling of the ensuing exchange: He claims to have debated Omar
for the entirety of the day about the legitimacy of killing British
civilians, until the latter eventually conceded defeat. Nawaz writes,
``I felt that I had saved many future lives.''

In 2004, Amnesty International adopted Nawaz as a prisoner of conscience
and secured his return to London two years later. His was not an
overnight epiphany, but within two more years, he had graduated from the
School of Oriental and African Studies at the University of London,
renounced Islamism and H.T. and publicly reinvented himself as an
advocate for liberal democracy: a media-savvy expert on preventing
radicalization. His enemies, a long list made up of family members,
ex-friends and former H.T. associates, have publicly questioned his
conversion narrative. Ian Nisbet, a white convert to Islam who was
jailed with Nawaz in Egypt, told a reporter from Alternet that Nawaz
remained a fanatical Islamist after he was freed. Indeed, back in
England, Nawaz appeared on the BBC's ``Hardtalk'' program and declared
that his experience in Egypt left him convinced ``that there is a need
to establish this caliphate as soon as possible.'' In his defense, Nawaz
claims that making a clean break with a former life is both difficult
and genuinely confusing. He rather colorfully compares it to a breakup
with a lover. (Nawaz and his first wife split up around the time of his
departure from H.T.) He has also insisted that his public positions were
in part strategic: He didn't want to tip his hand to H.T. until he had
his exit plans in place.

Whatever the case, that same year, alongside a college friend named Ed
Husain, who had already made a name for himself with his own
reverse-conversion memoir, ``The Islamist,'' Nawaz co-founded the
Quilliam Foundation, which they named for William Quilliam, a British
convert who opened one of Britain's first mosques in the late 1880s.
Quilliam's first headquarters occupied the ground floor of a
brick-and-terra-cotta rowhouse overlooking the verdure of Russell
Square, practically the same view T.S. Eliot would have had when he
worked at Faber \& Faber, and just a block from two of the sites of the
7/7 attacks. As his critics constantly stress, Nawaz's timing was
convenient; the British government was then looking to finance
anti-extremist organizations and provided Quilliam with early funding.

Image

Nawaz at 12, in Southend, England.Credit...Photograph from Maajid Nawaz

Nawaz, then, is somewhat like British Petroleum when it is tasked with
cleaning up a catastrophic oil spill: His main qualification to do this
kind of decontamination work is precisely his experience as a
contaminator. As recently as the mid-1990s, Islamist ideology was
unpopular in British Muslim communities. ``We would have to convince
people of something that is strange to them,'' he told me of those days.
``We had to really hone our argumentative skills and our ability to
convince and influence people as that vanguard of the Islamist movement
in the West.'' He insists his background as an Islamist is what allows
him and others at Quilliam today not only to pinpoint Islamism's
weaknesses but also to employ the very same tenacious ability to
communicate ideas and influence people for the purpose now of advocating
liberal values. ``They're transferable skills'' is how he once put it to
me. What Nawaz seems to understand better than any of the other critics
of Islam he's so often lumped with is that Islamism is \emph{cool} ---
and it is cool in some of the same ways that punk rock and gangsta rap
and macho rebellion in general, whether symbolic or real, are
perennially seductive. As a result, countering it will have to mean
finding ways to, as he puts it, ``make it cool to be a liberal Muslim.''

And that may be harder than it seems. While the vast majority of British
Muslims today are certainly not flocking to join groups like H.T. ---
and many who have never been attracted to the ideology justifiably find
it irritating to be lectured by a man who was --- a sobering number
nonetheless have expressed views that would be very much at home in even
more extreme precincts. An online poll done in Britain following the 7/7
bombings, for example, showed that more than a fifth of British Muslims
felt some sympathy for the bombers' feelings and motives; more than half
said they could understand the bombers' behavior; and nearly a third
agreed that ``Western society is decadent and immoral and that Muslims
should seek to bring it to an end'' by nonviolent means. One incredible
Gallup report from 2009 found that 0 percent of British Muslims viewed
homosexual acts as morally acceptable. Though it is not at all clear
what pushes any given individual to cross the line into violence,
attitudes like these are what Nawaz and Quilliam have controversially
described as the ``mood music'' to terrorist acts.

It is this last contention that seems to be the crux of the S.P.L.C.
complaint against Nawaz, along with the disclosure that, in 2010,
Quilliam provided a list of nonviolent ``Islamist'' organizations to a
British counterterrorism official. But Nawaz justifies the move by
arguing that the distinction between violent and nonviolent Islamism is
far less rigid than many liberals would like to think. ``Now when these
guys are joining ISIS, the arguments have been made,'' he told me.
``What they're doing is just putting that last piece in the jigsaw: `I'm
going to go and fight for this cause.' But the ideology's already been
established. The surveys and the polls tell you that.''

\textbf{Before Quilliam moved} late last year to an undisclosed location
for security reasons, I visited Nawaz on several occasions. The
organization hummed with the energy and sense of mission of a tech
start-up. On one side were doors leading into a large and crowded room
where 20-odd analysts, academics and imams were doing the intellectual
grunt work that the foundation demands. On the other was the modest
office Nawaz used for himself. Though he is the face of the
organization, he is hardly the only employee with an exotic résumé. On
my first visit, my eyes fixed on a small prayer rug draped neatly over
the arm of a desk chair. ``Oh, that's for him!'' Nawaz quickly
clarified, referring to his officemate, Noman Benotman, the current
president of the organization and a former jihadist who fought the
Soviets in Afghanistan, tried to violently depose Muammar Gaddafi in the
1990s and later worked with Osama bin Laden and Ayman al-Zawahri in
Sudan.

Back across the hall there was also Dr. Usama Hasan, the head of Islamic
studies at the organization. The son of a conservative and influential
Salafi sheikh, Hasan used a break from his studies at Cambridge to
engage in jihad against the Soviets in Afghanistan (on a scale of
one-to-ISIS, he told me, ``our group maybe got up to about five''). And
yet, in 2011, after waging holy war in the east and after 25 years of
service at his father's London mosque, that kind of effort didn't count
for very much when he came under attack by hard-liners. He was forced to
stop delivering Friday prayers when 50 Muslim protesters stormed his
lecture and openly called for his execution. His offense had been to
venture that Islam could be compatible with modern theories of evolution
and that Muslim women should be allowed to uncover their hair in public.

Image

Nawaz speaking on stage at a fashion-industry event about global issues
in December in Oxfordshire, England.Credit...John Phillips/Getty Images

Aside from the life experience of some of its members and the issuance
of the occasional counter-fatwa, Quilliam is a standard left-of-center
think tank: a body of experts conducting research and providing advice
and ideas on specific political or social problems in support of liberal
democracy. The group works to shape public opinion from the top down,
making frequent media appearances, publishing reports that aim for the
highest levels of government (such as a critical 2009 investigation into
the ways British prisons incubate extremism) and periodically advising
government ministers and heads of state on matters of terrorism. But
they also engage ordinary Muslims and non-Muslims alike through outreach
work, organizing debate and training programs in Europe and the Middle
East.

All of this ought to make Quilliam a natural ally of progressives and of
institutions like the Southern Poverty Law Center, whose mission, after
all, is to advocate for the vulnerable. Yet that has not been the case.
Nawaz's layered arguments and concessions --- his insistence, for
example, that Islam does have something to do with Islamism --- provoke
a visceral suspicion among those who are concerned with fighting
Islamophobia above all. A term that you will hear with frequency from
Nawaz is ``the regressive left,'' as in purportedly progressive
institutions like the S.P.L.C. that, often starting from a legitimate
concern that Muslims en masse not be persecuted for the actions of a
few, nonetheless embody a perplexingly backward mind-set when it comes
to Islam. ``It's an Orientalist fetish,'' Nawaz says, ``a deeply
socially conservative Muslim who is medieval in their outlook is a
`real' Muslim, and anyone who's challenging that status quo is a
sellout.'' The left has, in Nawaz's view, forfeited what's best about
the liberal project, entirely conceding the right to speak in moral
absolutes and about universal values. ``The problem is you can't draw a
line with that reasoning: Why is what ISIS is doing bad, then?''

A core idea Quilliam espouses is that space must be claimed for secular
identities within Islam; the measure of Muslim authenticity would then
be a matter of individual imagination and will, not a test to pass or
fail. In other words, he would like to see many more Muslims thinking,
speaking and acting like him. Which is a big part of the reason it's
impossible to think of Quilliam independent of the outsize figure cut by
its co-founder, and why so much debate about the validity of the
organization's ideas comes down to a question of being for or against
Nawaz.

Attack pieces about Nawaz have practically become their own literary
genre. In the summer of 2015, The Guardian ran a deeply critical story
about him, which questioned the integrity of Nawaz's work with the
Cameron administration and took him to task for, among other misdeeds,
``sipping a skinny flat white'' coffee in front of the reporter. This
was followed, in January 2016, by a hard takedown at The New Republic,
whose writer, Nathan Lean, had earlier referred on Twitter to Nawaz as
Sam Harris's ``lap dog.'' Roughly a week after that piece came a longer,
even more personal attack at Alternet, which stood out in its attempt to
debunk, scene by scene, the events in ``Radical.'' The authors revisited
the subject of the bomb in the backpack and quoted Nawaz's older brother
as well as an anonymous cousin, who called the story ``imaginary.''
(Many of the sources in the Alternet article seem concerned that
Kaashif's ruse in the anecdote might be taken literally.) When I asked
him about it, Nawaz was dismissive. ``You go to a deeply wounded brother
that loved me all of his life, and I turn out to be not who he aspires
for me to be,'' he said. ``As a journalist, you can exploit that.'' He
shrugged soberly.

It is undeniable that one advantage, and shortcoming, of memoir as a
form lies in its ability to dominate the reader through an empirical
imbalance that can never be resolved in its entirety. Arguments, when
unsound, can be negated, but who can negate another person's lived
experience? It is a rhetorical tactic that is, in fact, most at home on
the left, where personal stories of grievance and oppression are
typically set in opposition to the status quo in the wider society.
Perhaps, then, this is why so much attention has been paid to Nawaz's
biography. If his life story can be shown to be contrived, the deeper
message, however compelling, can be pre-emptively dismissed: Not only is
the messenger's life not a genuine Muslim life, when seen from this
angle, it may even prove to be an anti-Muslim life.

\textbf{I saw Nawaz} in New York in September, while he was in town
fund-raising for Quilliam's American chapter. We had made plans to meet
at a Soho hotel for a drink, but he was running late. When I asked after
him, the concierge either didn't know his real name or pretended not to.
Nawaz and Benotman have been targeted by Al Qaeda and ISIS affiliates,
and he travels under an alias. When he finally arrived, we went down to
the bar, and he was in wonderful spirits. He's been criticized in the
British press for drinking and receiving a lap dance at a strip club,
but in situations like this, it's strange to think of Nawaz as having
been anything like a humorless extremist. Yet the bind he has made for
himself is a real one: He has to prove that liberal, moderate Islam can
be ``cool,'' while not coming off as too hip to convince the left of his
Muslim authenticity. He runs the very real risk of satisfying no one.

It reminded me of an observation that had been running through my head
since the previous winter, when Quilliam opened an art exhibit in London
called ``The Unbreakable Rope.'' Billed as ``an exploration of sexuality
in Islam,'' the show was co-curated by Nawaz's second wife, Rachel
Maggart (the couple had their first child in January), a lanky
32-year-old brunette from Tennessee by way of N.Y.U. In addition to the
regular Quilliam bodyguards, there were plainclothes counterterrorism
officers monitoring the site. Inside the venue, a shirtless, tattooed
Kuwaiti performance artist did preparatory stretches with his assistant
and a crystal ball. He would eventually be tied up in a corset and left
on the floor for guests to contemplate. The crowd sipped wine and soft
drinks and milled about the sparsely hung, mildly provocative artwork,
which was in fact beside the point. The point, of course, was that they
were even daring to do this in the first place.

I fell into conversation with Nawaz's mother and little sister and lost
track of time as the space filled up all around us. There were whites,
blacks, Persians, Arabs; people looked devout and nondevout, gay,
straight, young and old. Standing next to me was a man with the
voluminous beard of a cleric, turned out in an ankle-length djellaba,
ironed as crisply as a bedsheet at the Ritz, a pair of Nike Air Force 1s
and a flat-brimmed New Era cap printed with a four-letter expletive. He
looked like a cross between the leader of Hezbollah and the Bay Area
rapper Lil B. The room darkened and quieted, and Nawaz, brimming with
life, stepped into the middle of the crowd, whose diversity he lauded,
and thanked them all for coming. Like the B-boy he once aspired to be,
he thrills to the sound of his own voice flowing through the microphone.
``The first thing they do is try to silence us, and the first to suffer
are the creators!'' he told the room to enthusiastic applause. ``But
while you throw gays off the rooftops, we who are Muslims want to
respond like this!''

As I watched Nawaz bask in the applause of his most earnest admirers and
glanced back at the walls adorned with such unbearably unhip art, the
enormity of his task pressed itself upon me. After all, Islamism, like
good art, is innately subversive; it captures diffuse feelings of
alienation in a way that is difficult to fabricate. And therein lies the
biggest challenge confronting Quilliam in Europe and, as it seeks to
expand, in America: Though Nawaz himself is a star, there is something
both noble and perilously square about this kind of eat-your-peas forced
secularism.

Yet I'm convinced that Nawaz really does have his finger on the pulse of
one of the most urgent problems of the contemporary era, a problem that
is far too often mishandled or greeted with flat-out denial, through
ignorance, hatred and fear, certainly, but also as a result of the very
best of intentions. Without having planned to, I found myself at the
hotel bar in New York opening up to Nawaz about a recent train ride my
wife and I made in France. I watched an agitated young Arab man and his
wife, in full abaya, shut themselves inside the bathroom along with all
of their luggage. When they opened the door, the hair on my neck stood
up, and I braced myself for a fusillade that never came. Even as I
chastised myself for overreacting, I was convinced that the man
continued to behave strangely. My shame increased with each moment
nothing happened.

Nawaz listened intently to my story, but his eyes showed he'd long since
arrived at his answer. ``You're caught in a classically Catch-22
situation,'' he said. ``You've got two competing forces, which are
entirely legitimate. One is not wanting to racially profile, and the
other is not wanting to be the neighbor of the San Bernardino shooter
who didn't want to profile and, as a result, people lose their lives.
Or, more urgently, {[}you{]} just don't want to be the first person to
catch a bullet! On a human level, that is a perfectly natural reaction.
The fact that you're having these doubts is a good thing.''

Though he meant this defense of human prejudice to reassure me, it did
not. I almost wish he had accused me of Islamophobia --- at least then
the conversation might have achieved a certain black-and-white clarity.
But Nawaz, the consummate in-between thinker, then took care to layer on
several more shades of gray. ``I literally just tweeted, five minutes
before coming to see you, a picture of a blond ISIS child --- a child
with blond hair --- helping to execute people,'' he said, producing on
his phone a shocking image of a very young, Eastern European-looking boy
holding a gun in the desert. ``I said, `Trump, how you gonna profile
this?' ''

Advertisement

\protect\hyperlink{after-bottom}{Continue reading the main story}

\hypertarget{site-index}{%
\subsection{Site Index}\label{site-index}}

\hypertarget{site-information-navigation}{%
\subsection{Site Information
Navigation}\label{site-information-navigation}}

\begin{itemize}
\tightlist
\item
  \href{https://help.nytimes3xbfgragh.onion/hc/en-us/articles/115014792127-Copyright-notice}{©~2020~The
  New York Times Company}
\end{itemize}

\begin{itemize}
\tightlist
\item
  \href{https://www.nytco.com/}{NYTCo}
\item
  \href{https://help.nytimes3xbfgragh.onion/hc/en-us/articles/115015385887-Contact-Us}{Contact
  Us}
\item
  \href{https://www.nytco.com/careers/}{Work with us}
\item
  \href{https://nytmediakit.com/}{Advertise}
\item
  \href{http://www.tbrandstudio.com/}{T Brand Studio}
\item
  \href{https://www.nytimes3xbfgragh.onion/privacy/cookie-policy\#how-do-i-manage-trackers}{Your
  Ad Choices}
\item
  \href{https://www.nytimes3xbfgragh.onion/privacy}{Privacy}
\item
  \href{https://help.nytimes3xbfgragh.onion/hc/en-us/articles/115014893428-Terms-of-service}{Terms
  of Service}
\item
  \href{https://help.nytimes3xbfgragh.onion/hc/en-us/articles/115014893968-Terms-of-sale}{Terms
  of Sale}
\item
  \href{https://spiderbites.nytimes3xbfgragh.onion}{Site Map}
\item
  \href{https://help.nytimes3xbfgragh.onion/hc/en-us}{Help}
\item
  \href{https://www.nytimes3xbfgragh.onion/subscription?campaignId=37WXW}{Subscriptions}
\end{itemize}
