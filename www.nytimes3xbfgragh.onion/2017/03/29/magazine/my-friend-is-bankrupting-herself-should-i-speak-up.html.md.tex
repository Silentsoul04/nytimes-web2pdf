Sections

SEARCH

\protect\hyperlink{site-content}{Skip to
content}\protect\hyperlink{site-index}{Skip to site index}

\href{https://myaccount.nytimes3xbfgragh.onion/auth/login?response_type=cookie\&client_id=vi}{}

\href{https://www.nytimes3xbfgragh.onion/section/todayspaper}{Today's
Paper}

My Friend Is Bankrupting Herself. Should I Speak Up?

\url{https://nyti.ms/2ox5EdL}

\begin{itemize}
\item
\item
\item
\item
\item
\item
\end{itemize}

Advertisement

\protect\hyperlink{after-top}{Continue reading the main story}

Supported by

\protect\hyperlink{after-sponsor}{Continue reading the main story}

\href{/column/the-ethicist}{The Ethicist}

\hypertarget{my-friend-is-bankrupting-herself-should-i-speak-up}{%
\section{My Friend Is Bankrupting Herself. Should I Speak
Up?}\label{my-friend-is-bankrupting-herself-should-i-speak-up}}

By Kwame Anthony Appiah

\begin{itemize}
\item
  March 29, 2017
\item
  \begin{itemize}
  \item
  \item
  \item
  \item
  \item
  \item
  \end{itemize}
\end{itemize}

\includegraphics{https://static01.graylady3jvrrxbe.onion/images/2017/04/02/magazine/02ethicist/02ethicist-jumbo.jpg?quality=75\&auto=webp\&disable=upscale}

\emph{I've been friends with ``Cindy'' for 15 years. She's in her early
60s and never married. Because of job stress, she took early retirement
and lives on a reduced pension. Despite not having a lot of money, she
is generous, spoiling her family and friends with gifts.}

\emph{Two years ago she became romantically involved with a man,
``Sean,'' who was bankrupt and homeless. He started staying with her
right after they met. Cindy reveled in having a lover for the first time
in 30 years. Sean helped her lose weight and exercise. However, he is a
complete loser in my view: Not only does he rely on her for his cash,
transportation and cellphone needs, but he is also habitually drunk by 4
p.m.}

\emph{More than a year ago Cindy started using what credit she has to
start a business with Sean from scratch. She is spending thousands on
manufacturing, shipping and storage and long hours hawking and doing
administrative work. There's no indication that they are making any
profit; she admitted this and said she would give it one year, but she
is now saying two. Cindy's life is endless work and financial outlay ---
the conditions are worse than what made her want to take early
retirement.}

\emph{I spoke to a mutual friend about my fears for Cindy's financial
future. She agreed that Cindy might go bankrupt but pointed out that
Cindy does have a mortgage-free home. So I've remained silent. Recently
I found out Cindy was recovering from a minor stroke. Her fridge was
almost empty, and I could see the bloom had worn off the relationship
with Sean. While he chided her about her eating habits, she complained
of his constant drinking.}

\emph{When he moved in with her, she seemed cautious; she talked about
how in the province of our country there is a legal entitlement to
spousal support after cohabitation for three years. As this date
approaches, I am racked by the belief that I must do something to help
her break free. I thought of contacting her siblings, with whom I have
no relationship, to see if they would consider some sort of
intervention, but should I mind my own business?} Name Withheld

\textbf{This is a sad} story, and in its outlines, alas, not an
unfamiliar one. Given that Cindy is a mentally competent adult and a
friend of yours, though, the obvious person to talk to is not a sibling
of hers but Cindy herself. I don't mean that it would necessarily be
wrong to express your concern to a member of her family. But to do so
without first talking to her would be disrespectful. If she insists that
you not talk to her family, you should take her wishes seriously.

The modern ideal of autonomy means that we think each person should be
in charge of her own life. But this doesn't mean that friends and family
should avoid stepping in: Advice given out of love and concern for us is
no affront to our autonomy if it helps us think through our situation
more rationally.

All that granted, it's hard to intervene in cases like this. You're
asking someone to let you talk to her about a situation that is probably
a source of shame. If you really think that he's just taking advantage
of her, though, the time to speak up is now. It's what friends do.

\emph{Many years ago, a dear friend of mine told me she engaged in
behavior I consider abusive. She was a high school teacher. She got
drunk and took pills with two recent graduates (both 18) and then
engaged in a cutting ritual with one of them. They spent the night in
bed ``cuddling'' to comfort each other over the cuts they'd inflicted.
She found it a very arousing experience. I reacted strongly, told her
she could go to jail, that she had to quit her job, that she could no
longer be a teacher. She kept insisting it was O.K.: The girl was 18,
and it was consensual. I had a moral impulse to tell her principal, but
in my loyalty to her, I did not. Nothing happened. She kept her job and
never expressed much regret. Her lack of responsibility infuriated me,
and our friendship fizzled. I was left with the moral discomfort of
failing to report her.}

\emph{Recently, she wrote to me because she is suffering from a
degenerative disease and wants to ``make amends in case anything
happens.'' She also says she has dissociative identity disorder, which I
am not surprised by. She mentions a few things that happened between us
that she regrets, but nothing about the cutting incident. I forgive her
for all these other things; there are things I could ask her for
forgiveness for, too. The one thing I can't get past is her taking no
responsibility for abusing her student. My question is: Is it ethical
for me to bring it up? She is sick, and now officially mentally ill. She
may have suppressed the memory. Could bringing it up be a dangerous
trigger? Is it ethical to confront a mentally ill person about an abuse
she committed years ago when the abuse had nothing to do with you?} Name
Withheld

\textbf{I agree that} your friend's behavior sounds worrisome. But I
don't see that it was wrong merely because the other party had been a
student of hers. Once a student is an ex-student, and outside the
relationship of trust that exists between teachers and students, sexual
or semi-sexual acts like these are the business of the participants ---
provided there is full consent. When pills and alcohol are involved,
consent can be hard to establish. And of course, in all relationships,
there can be forms of power that it is abusive to exploit. If any of
that was the case here, what happened was wrong. Was it illegal? When it
comes to state intervention, the critical questions are: ``Was one of
the parties in a position of authority or trust over the other?'' and
``Was there genuine consent?''

If your reaction was based on considerations about consent or abuse of
power (and not just on the weirdness of what went on), you were right to
be critical. But your comments suggest that these considerations aren't
all that moved you. You wish you had done more at the time; you're angry
that she doesn't acknowledge the episode that so damaged your
relationship. That's why you want an opportunity to set her straight
about what happened.

I have no idea whether, in her current mental condition, she remembers
what happened or is capable of responding reasonably to your discussing
it. What matters, though, isn't whether she's mentally ill. It's whether
she can understand the issues you want to raise with her, as well as the
issues she herself has raised. If she can understand --- and isn't too
distressed by it --- you should feel free to tell her what's weighing so
heavily on your mind. But first, I think, you should try to get clear
about \emph{why} it weighs so much.

Advertisement

\protect\hyperlink{after-bottom}{Continue reading the main story}

\hypertarget{site-index}{%
\subsection{Site Index}\label{site-index}}

\hypertarget{site-information-navigation}{%
\subsection{Site Information
Navigation}\label{site-information-navigation}}

\begin{itemize}
\tightlist
\item
  \href{https://help.nytimes3xbfgragh.onion/hc/en-us/articles/115014792127-Copyright-notice}{©~2020~The
  New York Times Company}
\end{itemize}

\begin{itemize}
\tightlist
\item
  \href{https://www.nytco.com/}{NYTCo}
\item
  \href{https://help.nytimes3xbfgragh.onion/hc/en-us/articles/115015385887-Contact-Us}{Contact
  Us}
\item
  \href{https://www.nytco.com/careers/}{Work with us}
\item
  \href{https://nytmediakit.com/}{Advertise}
\item
  \href{http://www.tbrandstudio.com/}{T Brand Studio}
\item
  \href{https://www.nytimes3xbfgragh.onion/privacy/cookie-policy\#how-do-i-manage-trackers}{Your
  Ad Choices}
\item
  \href{https://www.nytimes3xbfgragh.onion/privacy}{Privacy}
\item
  \href{https://help.nytimes3xbfgragh.onion/hc/en-us/articles/115014893428-Terms-of-service}{Terms
  of Service}
\item
  \href{https://help.nytimes3xbfgragh.onion/hc/en-us/articles/115014893968-Terms-of-sale}{Terms
  of Sale}
\item
  \href{https://spiderbites.nytimes3xbfgragh.onion}{Site Map}
\item
  \href{https://help.nytimes3xbfgragh.onion/hc/en-us}{Help}
\item
  \href{https://www.nytimes3xbfgragh.onion/subscription?campaignId=37WXW}{Subscriptions}
\end{itemize}
