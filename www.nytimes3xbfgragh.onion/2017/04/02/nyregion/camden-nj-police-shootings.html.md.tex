Sections

SEARCH

\protect\hyperlink{site-content}{Skip to
content}\protect\hyperlink{site-index}{Skip to site index}

\href{/section/nyregion}{New York}\textbar{}Changes in Policing Take
Hold in One of the Nation's Most Dangerous Cities

\url{https://nyti.ms/2oq1zM3}

\begin{itemize}
\item
\item
\item
\item
\item
\end{itemize}

\includegraphics{https://static01.graylady3jvrrxbe.onion/images/2017/04/03/nyregion/03CAMDEN1/30CAMDEN1-articleLarge.jpg?quality=75\&auto=webp\&disable=upscale}

\hypertarget{changes-in-policing-take-hold-in-one-of-the-nations-most-dangerous-cities}{%
\section{Changes in Policing Take Hold in One of the Nation's Most
Dangerous
Cities}\label{changes-in-policing-take-hold-in-one-of-the-nations-most-dangerous-cities}}

It's a sort of Hippocratic ethos: Minimize harm, and try to save lives.
And in Camden, N.J., residents are noticing the results.

Officers Vidal Rivera, left, and Tyrrell Bagby in North Camden, a
neighborhood that has undergone seismic changes since the police started
walking the beat. Credit...Todd Heisler/The New York Times

Supported by

\protect\hyperlink{after-sponsor}{Continue reading the main story}

By \href{http://www.nytimes3xbfgragh.onion/by/joseph-goldstein}{Joseph
Goldstein}

\begin{itemize}
\item
  April 2, 2017
\item
  \begin{itemize}
  \item
  \item
  \item
  \item
  \item
  \end{itemize}
\end{itemize}

CAMDEN, N.J. --- Every few months, the police chief here asks which
officers wrote the most tickets.

Elsewhere, this might lead to praise, but in Camden --- where 40 percent
of residents live below the poverty line, the murder rate compares to
that of El Salvador and one of the most interesting experiments in
American policing is underway --- Chief J. Scott Thomson sees aggressive
ticket writing as a sign that his officers don't get the new program.

``Handing a \$250 ticket to someone who is making \$13,000 a year'' ---
\href{https://www.census.gov/quickfacts/table/PST045216/3410000/accessible}{around
the per capita income in the city} --- ``can be life altering,'' Chief
Thomson said in an interview last year, noting that it can make car
insurance unaffordable or result in the loss of a driver's license.
``Taxing a poor community is not going to make it stronger.''

Handling more vehicle stops with a warning, rather than a ticket, is one
element of Chief Thomson's new approach, which, for lack of another
name, might be called the Hippocratic ethos of policing: Minimize harm,
and try to save lives.

Officers are trained to hold their fire when possible, especially when
confronting people wielding knives and showing signs of mental illness,
and to engage them in conversation when commands of ``drop the knife''
don't work. This sometimes requires backing up to a safer distance. Or
relying on patience rather than anything on an officer's gun belt.

\includegraphics{https://static01.graylady3jvrrxbe.onion/images/2017/03/31/nyregion/xxcamden-police/xxcamden-police-videoSixteenByNine3000.jpg}

And Chief Thomson has told officers that when they respond to shootings
--- or after the police open fire --- they should carry the wounded into
their cruisers and rush to the hospital, rather than wait for an
ambulance.

Such changes were shaped partly by headlines and YouTube videos from far
beyond Camden, a city of some 80,000 that for decades has been
synonymous with blight and decline.

The
\href{https://www.nytimes3xbfgragh.onion/interactive/2014/08/13/us/ferguson-missouri-town-under-siege-after-police-shooting.html?_r=0}{unrest
in Ferguson}, Mo., after a police officer shot and killed an unarmed
black teenager, Michael Brown, in 2014, and the
\href{https://www.youtube.com/watch?v=LfXqYwyzQpM}{video} from Staten
Island of a dying
\href{https://www.nytimes3xbfgragh.onion/2015/06/14/nyregion/eric-garner-police-chokehold-staten-island.html}{Eric
Garner} gasping through a police chokehold, ignited a national dialogue
about policing and race. Police departments were pressured to reconsider
their policies for using force. Nationwide, many departments responded
by issuing \href{100000004995144/web/editing}{body-worn cameras};
turning to ``de-escalation'' training in an effort to shoot fewer
people; and paying more attention to how the police are perceived by
black residents.

Across the country, the political momentum for police reform has slowed
over the last year, even before the election of President Trump, whose
administration has taken the position that
\href{https://www.washingtonpost.com/news/the-fix/wp/2016/09/26/the-first-trump-clinton-presidential-debate-transcript-annotated/?utm_term=.7029ff794c26}{federal
efforts} to make the police more accountable
\href{https://www.nytimes3xbfgragh.onion/2017/02/28/us/politics/jeff-sessions-crime.html}{have
made them less effective}. Ambush attacks in
\href{https://www.nytimes3xbfgragh.onion/2016/07/09/us/dallas-attacks-what-we-know-baton-rouge-minnesota.html}{Dallas
and Baton Rouge}, La., last year left eight officers dead, shifting the
national discussion away from excessive force and toward the dangers
officers face.

\includegraphics{https://static01.graylady3jvrrxbe.onion/images/2017/04/03/nyregion/03CAMDEN3/30CAMDEN3-articleLarge.jpg?quality=75\&auto=webp\&disable=upscale}

But not in Camden, where changes have been openly received and are
taking hold within the department.

``The old police mantra was make it home safely,'' said
\href{http://webcache.googleusercontent.com/search?q=cache:O1CKTUJWfRgJ:camdencountypd.org/officer-of-the-week-tyrrell-bagby/+\&cd=1\&hl=en\&ct=clnk\&gl=us}{Tyrrell
Bagby}, 25, an affable second-generation Camden police officer. ``Now
we're being taught not only should we make it home safely, but so should
the victim and the suspect.'' Officer Bagby has saved 22 lives since
joining the force in 2014 by administering naloxone, a drug that
reverses opioid overdoses.

An early sign that Chief Thomson's message was taking hold among his
officers came on Nov. 9, 2015, when a 48-year-old man walked into a
Crown Fried Chicken, behaved menacingly toward customers and employees,
brandished a steak knife and left. Outside, officers ordered him to drop
the knife, according to video from police body cameras. But the man
began walking away, slashing the knife through the air as he went.

For several minutes, the officers formed a cordon around the man and
walked with him for a few blocks, trying to clear traffic ahead and
periodically instructing him to drop the knife.

\href{https://www.nytimes3xbfgragh.onion/interactive/2017/04/02/nyregion/camden-police-policy.html}{}

\includegraphics{https://static01.graylady3jvrrxbe.onion/images/2017/03/31/nyregion/camden-doc-promo/camden-doc-promo-articleLarge.png}

\hypertarget{camden-police-policy-on-escorting-victims}{%
\subsection{Camden Police Policy on Escorting
Victims}\label{camden-police-policy-on-escorting-victims}}

A directive from Chief John S. Thompson of the Camden Police Department
on how police officers should treat seriously injured people.

The crisis ended when the man did just that. Had the episode taken place
a year before, ``we would more than likely have deployed deadly force
and moved on,'' Chief Thomson said.

The chief said he had stressed to his officers that the department
``does not treat repositioning as retreating,'' and that backing up to
put a car between a suspect and an officer ``is not an act of
cowardice.''

Few
\href{https://www.youtube.com/watch?v=YtVUMT9P8iw\&feature=youtu.be}{videos}
like it have emerged in the annals of American policing.

Another lifesaving initiative in Camden, actually a mandate, is for
officers to drive gunshot victims to a hospital if waiting for an
ambulance would cause a delay. The policy, known as ``scoop and go,''
was modeled after
\href{https://news.upenn.edu/news/survival-rates-similar-gunshot-stabbing-victims-whether-brought-hospital-police-or-ems-penn-med}{a
longstanding Philadelphia policy}. But in much of the country, officers
view picking up victims as the ambulance crews' job.

Sgt. Angel Nieves, 45, a 17-year Camden officer, said the policy
``stunned'' him when it was put into effect in November 2015. He had
been taught to ``keep your distance --- you don't know what these guys
have,'' alluding to H.I.V.

Image

Chief J. Scott Thomson in his office, and an officer during
de-escalation training with a woman carrying a knife.Credit...Todd
Heisler/The New York Times

Then he thought of ``what happened in places like Ferguson,'' where
officers had
\href{https://www.nytimes3xbfgragh.onion/2014/08/24/us/michael-brown-a-bodys-timeline-4-hours-on-a-ferguson-street.html?_r=0}{left
Mr. Brown's body on the street, provoking outrage}. ``In light of what
happened there,'' he said, ``any department that doesn't go with a
`scoop and go' policy is just asking for it'' --- that is, asking for
trouble.

Chief Thomson, 45, who leads the department of 400 officers, is
president of a prominent police research group and has emerged as a
significant voice in American policing.

But he is an unlikely reformer. A Camden officer since 1994, he became
chief in 2008 mainly because he was next in a fast-moving line. The
department had gone through five chiefs in five years.

``They looked at me and said, `Well, he looks like he won't get indicted
in the next six months --- he'll do,'' Chief Thomson recalled.

Image

Tee Tee Nobles of North Camden said he has only recently become
comfortable letting his young daughters play outdoors.Credit...Todd
Heisler/The New York Times

The force was, he said, ``apathetic, lethargic and corrupt,'' and yet
still the ``most effective government agency in Camden.''

The city, across the Delaware River from Philadelphia, was once a
manufacturing powerhouse --- this is where Campbell's invented condensed
soup in a can and where RCA built many of the nation's first television
sets. But the city fell into a long decline.

Today there are glimmers of optimism. The Philadelphia 76ers opened a
training facility here, and a few major companies are moving to Camden.
But it is still a contender for the poorest and most dangerous city in
America.

Grandmothers warn children, ``Play in the streets, die in the streets.''
The streets are not meant as a metaphor. Just being outside is
considered dangerous.

A Roman Catholic nun in Camden, Sister Helen Cole of Guadalupe Family
Services, a social services agency, periodically hears from suburban
friends offering to donate bicycles. ``I don't take them, because our
kids in this community, they will not ride bikes outside,'' she said.

Image

Officer Dennis Smarth on a Camden corner where drug activity has been
common.Credit...Todd Heisler/The New York Times

The number of homicides in Camden has
\href{https://www.nytimes3xbfgragh.onion/2014/09/01/nyregion/camden-turns-around-with-new-police-force.html}{dropped
significantly since 2012}, when the city recorded 67, the most on
record; last year,
\href{http://www.nj.com/camden/index.ssf/2017/01/camden_4th_most_dangerous_us_city_real_estate_comp.html}{the
total was 44}. In 2013, the remnants of the Camden force ---
\href{http://www.nytimes3xbfgragh.onion/2011/03/07/nyregion/07camden.html}{half
had been laid off} --- were disbanded. A new department was formed,
again with Chief Thomson at its helm. It was a maneuver that lowered
salaries and pension obligations. It allowed the chief to bring on new
officers and a new culture.

The improvements in public safety since then are particularly strong in
North Camden, a neighborhood 10 blocks long and about that many wide,
full of single-family homes, many long abandoned. Addicts from the
suburbs often drove there to buy heroin from street dealers.

In 2013, police officers were sent to walk patrols in the neighborhood
for 12 hours. They were told to knock on doors and introduce themselves.
If they needed to use a bathroom, they had better make some friends. The
city razed abandoned homes. Drug dealers were arrested or pushed indoors
or out of the neighborhood. Initially, at least, residents were
discouraged from congregating outdoors.

In interviews, several residents who had been stopped by the police, or
even arrested, grudgingly conceded that things were better.

``Metro came out beasting --- they locked everybody up,'' recalled Tee
Tee Nobles, 28.

Since then, however, he has felt it safe enough to let his daughters,
ages 8 and 2, run around outdoors. Before, he said, ``you don't let them
outside.''

Advertisement

\protect\hyperlink{after-bottom}{Continue reading the main story}

\hypertarget{site-index}{%
\subsection{Site Index}\label{site-index}}

\hypertarget{site-information-navigation}{%
\subsection{Site Information
Navigation}\label{site-information-navigation}}

\begin{itemize}
\tightlist
\item
  \href{https://help.nytimes3xbfgragh.onion/hc/en-us/articles/115014792127-Copyright-notice}{©~2020~The
  New York Times Company}
\end{itemize}

\begin{itemize}
\tightlist
\item
  \href{https://www.nytco.com/}{NYTCo}
\item
  \href{https://help.nytimes3xbfgragh.onion/hc/en-us/articles/115015385887-Contact-Us}{Contact
  Us}
\item
  \href{https://www.nytco.com/careers/}{Work with us}
\item
  \href{https://nytmediakit.com/}{Advertise}
\item
  \href{http://www.tbrandstudio.com/}{T Brand Studio}
\item
  \href{https://www.nytimes3xbfgragh.onion/privacy/cookie-policy\#how-do-i-manage-trackers}{Your
  Ad Choices}
\item
  \href{https://www.nytimes3xbfgragh.onion/privacy}{Privacy}
\item
  \href{https://help.nytimes3xbfgragh.onion/hc/en-us/articles/115014893428-Terms-of-service}{Terms
  of Service}
\item
  \href{https://help.nytimes3xbfgragh.onion/hc/en-us/articles/115014893968-Terms-of-sale}{Terms
  of Sale}
\item
  \href{https://spiderbites.nytimes3xbfgragh.onion}{Site Map}
\item
  \href{https://help.nytimes3xbfgragh.onion/hc/en-us}{Help}
\item
  \href{https://www.nytimes3xbfgragh.onion/subscription?campaignId=37WXW}{Subscriptions}
\end{itemize}
