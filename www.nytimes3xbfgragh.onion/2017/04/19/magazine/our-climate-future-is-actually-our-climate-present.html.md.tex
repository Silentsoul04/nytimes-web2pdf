Our Climate Future Is Actually Our Climate Present

\url{https://nyti.ms/2pAL6lE}

\begin{itemize}
\item
\item
\item
\item
\item
\item
\end{itemize}

\includegraphics{https://static01.graylady3jvrrxbe.onion/images/2017/04/23/magazine/23cover/23cover-articleLarge.jpg?quality=75\&auto=webp\&disable=upscale}

Sections

\protect\hyperlink{site-content}{Skip to
content}\protect\hyperlink{site-index}{Skip to site index}

The Climate Issue

\hypertarget{our-climate-future-is-actually-our-climate-present}{%
\section{Our Climate Future Is Actually Our Climate
Present}\label{our-climate-future-is-actually-our-climate-present}}

How do we live with the fact that the world we knew is going and, in
some cases, already gone?

Credit...Illustration by Christoph Niemann

Supported by

\protect\hyperlink{after-sponsor}{Continue reading the main story}

By Jon Mooallem

\begin{itemize}
\item
  April 19, 2017
\item
  \begin{itemize}
  \item
  \item
  \item
  \item
  \item
  \item
  \end{itemize}
\end{itemize}

\textbf{A} few years ago, a locally famous blogger in San Francisco,
known as Burrito Justice, created an exquisitely disorienting map, with
help from a cartographer named Brian Stokle, and started selling copies
of it online. The map \href{https://burritojustice.com/200ft/}{imagined
the city in the year 2072}, after 60 years of rapid sea-level rise
totaling 200 feet. At present, San Francisco is a roughly square-shaped,
peninsular city. But on the map, it is severed clean from the mainland
and shaved into a long, fat smudge. The shape of the land resembles a
sea bird diving underwater for prey, with odd bays chewing into the
coastlines and, farther out, a sprawl of bulging and wispy islands that
used to be hills. If you lived in San Francisco, it was a map of where
you already were and, simultaneously, where you worried you might be
heading. ``The San Francisco Archipelago,'' Burrito Justice called it
--- a formerly coherent city in shards.

The map wasn't science; it didn't even pretend to be. I want to be very
clear about that, because I worry it's reckless to inject any more false
facts into a conversation about climate change. Projecting the effect of
sea-level rise on a specific location typically involves recondite
computer models and calculations; Burrito Justice was just a fascinated
hobbyist, futzing around on his laptop in his backyard. His entire
premise was unscientific; for now, it is unthinkable that seas will rise
so high so quickly. Even as most credible scientific estimates keep
increasing and the poles melt faster than imagined, those estimates
currently reach only between six and eight feet by the year 2100. That's
still potentially cataclysmic: Water would push into numerous cities,
like Shanghai, London and New York, and displace hundreds of millions of
people. And yes, there are some fringe, perfect-storm thought
experiments out there that can get you close to 200 feet by the end of
the century. But in truth, Burrito Justice settled on that number only
because that's how high he needed to jack up the world's oceans if he
wanted to wash out a particular road near his house. He has a friendly
rivalry with another blogger, who lives in an adjacent neighborhood
known for being a cloistered hamlet, and Burrito Justice thought it
would be funny to see it literally become an island. So again: The map
wasn't science. It didn't pretend to be. The point, initially, was just
to needle this other guy named Todd.

Still, the San Francisco Archipelago has always stuck with me, because,
almost in spite of itself, it managed to convey something peculiar and
destabilizing about our climatological future. Burrito Justice hadn't
just redrawn the geography of a place; he'd also carried a sense of that
place forward in time. And by transposing some of the grit and silly
shibboleths of contemporary city life onto that alternate landscape, the
map (and the little blog posts he wrote to accompany it) prodded you to
entertain the possibility that this ruined future might not feel like an
emergency to those living it, that life in that archipelago might have
all the richness, realness and inanity of ours.

There were, most obviously, the breezy, optimistic names given to every
new feature of the redrawn city, as though its ever-peppy real estate
agents had gone on rebranding neighborhoods as the landscape drowned.
Climate change, in this scenario, had more in common with gentrification
than with a natural disaster: a ceaseless upheaval of familiar spaces
that left old-timers shaking their heads, then kept accelerating.
Instead of Telegraph Hill rising north of Market Street downtown,
Telegraph Island now offered a tranquil view of Market Shoals. Dolores
Park was gone. But Cape Dolores jutted toward it, overlooking the
submerged Mission District --- now Mission Gulf. The former San
Francisco Zoo, out at Ocean Beach, was labeled San Francisco Aquarium.

Life went on, in other words --- albeit in some bleak and greatly
diminished capacity. Taco boats replaced taco trucks, the public-transit
agency's ``sea bus'' system exaggerated its on-time performance
statistics and the city government was offering to extend the notorious
tax break it offered Twitter in 2011 if the tech company relocated to
``disadvantaged Nob Island.'' The only people who remembered us, or
validated our earlier reality, came off as loopy, Nimby activists aiming
to obstruct development on one of the new coasts. ``Old San Francisco is
still alive in our hearts and minds,'' a statement from the Submerged
Historic San Francisco Preservation Association insists, ``even if only
the tops of the buildings can be seen!''

The map was a joke. But the longer I looked at it, the less funny and
more upsetting it got. I pictured the first apartment my wife and I
rented in San Francisco, how I'd parked the car out front while, just
home from the hospital, she carried our first baby up the stairs. Then I
pictured that all under water, and a man pushing off in his kayak for a
paddle far overhead.

\includegraphics{https://static01.graylady3jvrrxbe.onion/images/2017/04/23/magazine/23essay3/23essay3-articleLarge.jpg?quality=75\&auto=webp\&disable=upscale}

\textbf{The future we've been} warned about is beginning to saturate the
present. We tend to imagine climate change as a destroyer. But it also
traffics in disruption, disarray: increasingly frequent and more
powerful storms and droughts; heightened flooding; expanded ranges of
pests turning forests into fuel for wildfires; stretches of inhospitable
heat. So many facets of our existence --- agriculture, transportation,
cities and the architecture they spawned --- were designed to suit
specific environments. Now they are being slowly transplanted into
different, more volatile ones, without ever actually moving.

We're accustomed to hearing about the tragically straightforward cases
of island nations that will simply disappear: countries like Tuvalu and
Kiribati that face the possibility of having to broker the wholesale
resettlement of their people in other countries. Yet there must also be,
in any corner of the planet, and for each human living on it, a
threshold at which a familiar place becomes an unfamiliar one: an
altered atmosphere, inundated by differentness and weirdness, in which,
on some level, we'll live on, in exile. The Australian philosopher Glenn
Albrecht describes this feeling as ``solastalgia'': ``a form of
homesickness one gets when one is still at `home.' ''

Some communities will face new problems and varieties of weather; in
others, existing ones will intensify. Already-vulnerable societies ---
the poor, the poorly governed --- may be stressed to grim breaking
points. Consider the mass starvation in South Sudan, Nigeria, Yemen and
Somalia, where a total of nearly a million and a half children are
predicted to die this year --- and that climate change is projected to
worsen the kind of droughts that caused it. Consider, too, a 2015
Department of Defense report, which framed climate change as a
geopolitical ``threat multiplier'' that will ``threaten domestic
stability in a number of countries,'' and cited a study showing how a
five-year drought in Syria contributed to the outbreak of the current
conflict there. Nonetheless, denial is coming back in fashion among the
most powerful. We have a president who dismisses climate change as a
hoax, and a budget director who belittles government programs to study
and adapt to our new reality as a ``waste of your money.''

Still, we insulate ourselves from the disorientation and alarm in other,
more pernicious ways, too. We seem able to normalize catastrophes as we
absorb them, a phenomenon that points to what Peter Kahn, a professor of
psychology at the University of Washington, calls ``environmental
generational amnesia.'' Each generation, Kahn argues, can recognize only
the ecological changes its members witness during their lifetimes. When
we spoke recently, Kahn pointed to the living conditions in megacities
like Kolkata, or in the highly polluted, impoverished areas affected by
Houston's oil refineries, where he conducted his initial research in the
early '90s. In Houston, Kahn found that two-thirds of the children he
interviewed understood that air and water pollution were environmental
issues. But only one-third believed \emph{their} neighborhood was
polluted. ``People are born into this life,'' Kahn told me, ``and they
think it's normal.''

A University of British Columbia fisheries scientist, Daniel Pauly, hit
upon essentially the same idea around the same time, recognizing that as
populations of large fish collapsed, humanity had gone on obliviously
fishing slightly smaller species. One result, Pauly wrote, was a
``creeping disappearance'' of overall fish stocks behind ever-changing
and ``inappropriate reference points.'' He called this impaired vision
``shifting baseline syndrome.''

There are, however, many subtler shifts in our awareness that can't be
as precisely demarcated. Scenarios that might sound dystopian or
satirical as broad-strokes future projections unassumingly materialize
as reality. Last year, melting permafrost in Siberia released a strain
of anthrax, which had been sealed in a frozen reindeer carcass,
sickening 100 people and killing one child. In July 2015, during the
hottest month ever recorded on earth (until the following year), and the
hottest day ever recorded in England (until the following summer), the
Guardian newspaper had to shut down its live-blogging of the heat wave
when the servers overheated. And low-lying cities around the world are
experiencing increased ``clear-sky flooding,'' in which streets or
entire neighborhoods are washed out temporarily by high tides and storm
surges. Parts of Washington now experience flooding 30 days a year, a
figure that has roughly quadrupled since 1960. In Wilmington, N.C., the
number is 90 days. But scientists and city planners have conjured a term
of art that defuses that astonishing reality: ``nuisance flooding,''
they call it.

Image

Credit...Illustration by Christoph Niemann

Kahn calls our environmental generational amnesia ``one of the central
psychological problems of our lifetime,'' because it obscures the
magnitude of so many concrete problems. You can wind up not looking
away, exactly, but zoomed in too tightly to see things for what they
are. Still, the tide is always rising in the background, swallowing
something. And the longer you live, the more anxiously trapped you may
feel between the losses already sustained and the ones you see coming.

Such shifting baselines muddle the idea of adaptation to climate change,
too. Adaptation, Kahn notes, can mean anything from the human eye's
adjusting to a darker environment within a few milliseconds to wolves'
changing into dogs over thousands of years. It doesn't always mean
progress, he told me; ``it's possible to adapt and diminish the quality
of human life.'' Adapting to avoid or cope with the suffering wrought by
climate change might gradually create \emph{other} suffering. And
because of environmental generational amnesia, we might never fully
recognize its extent. Think of how Shel Silverstein's Giving Tree,
nimbly accommodating each of the boy's needs, eventually winds up a
stump.

On the most fundamental level, Kahn argues, we are already adapting to
climate change through a kind of tacit acquiescence, the way people in a
city like Beijing accept that simply breathing the air outside can make
them sick. ``People are aware --- they're coughing and wheezing,'' he
told me, ``but they're not staging political revolutions.'' Neither are
we. And, Kahn went on, we risk imprisoning ourselves, through gradual
adaptation, into a condition of ``unfulfilled flourishing.'' A wolf
becomes a dog, genetically; it \emph{wants} to fetch tennis balls and
sleep at the foot of your bed. But imagine a dog that isn't yet a dog,
that still wants to be a wolf.

Sure, I told him, but at some point it would all be too much.
Potentially, Kahn said. But assumptions about the future, no matter how
self-evident they may feel, don't automatically come true. ``The amazing
thing is that none of this seems to work the way we think it should.
When I was growing up in the Bay Area in the 1970s, the traffic was
really bad. And I said, If it just gets a little bit worse, you're going
to have a major upheaval in consciousness. And every five years it got
worse.'' He went silent for a second, then continued, ``I'm just
thinking about how many five-year periods I've lived through.''

\textbf{One more thing} about Burrito Justice and the origins of his
archipelago map: Shortly after moving to San Francisco in the early
2000s, he happened upon a map of the city from 1853. Like other cities
--- New York, Boston, Seattle --- San Francisco expanded its natural
coastline with thousands of acres of ``made land,'' filling in mud flats
and harbors with phenomenal amounts of debris and sand. But much of this
happened after 1853; on the map Burrito Justice was looking at, San
Francisco was smaller --- physically smaller. And he was struck by how
much its former shape might resemble its future one. It wouldn't take
much water for climate change to unmake the made land. The city would
revert to its previous version, as though leveled by some cosmic
control-Z.

As Burrito Justice described this to me on the phone one recent
afternoon, I thought of a woman in San Francisco named Pamela Buttery,
whom I'd heard about on National Public Radio in January. Buttery owned
a condo in the Millennium Tower, a waterfront skyscraper downtown. But
the tower had started sinking at an irregular angle, even before its
completion in 2010; by now, it has tilted six inches and sunk a foot
into the hodgepodge Victorian landfill on which it was constructed.
Buttery lived on the 57th floor. ``I've moved on into a depression about
it,'' she said. Though she used to unwind by putting golf balls, the
reporter noted that even this didn't ``give her the same joy it once
did. No matter which way she hits them, they all end up in the same
corner.'' And I realized that if someone in 1853 had tried to anticipate
the texture and oddities of future life in his artificially expanding
city, and imagined a woman who can't satisfactorily putt golf balls on
the 57th floor because her luxury condo is sinking into old garbage ---
well, I probably would have bought a copy of that guy's map, too.

The future is always somebody else's present --- it will very likely
feel as authentic, and only as horrific, as our moment does to us. But
the present is also somebody else's future: We are already standing on
someone else's ludicrous map. Except none of us are in on the joke, and
I'm guessing that it won't feel funny any time soon.

Advertisement

\protect\hyperlink{after-bottom}{Continue reading the main story}

\hypertarget{site-index}{%
\subsection{Site Index}\label{site-index}}

\hypertarget{site-information-navigation}{%
\subsection{Site Information
Navigation}\label{site-information-navigation}}

\begin{itemize}
\tightlist
\item
  \href{https://help.nytimes3xbfgragh.onion/hc/en-us/articles/115014792127-Copyright-notice}{©~2020~The
  New York Times Company}
\end{itemize}

\begin{itemize}
\tightlist
\item
  \href{https://www.nytco.com/}{NYTCo}
\item
  \href{https://help.nytimes3xbfgragh.onion/hc/en-us/articles/115015385887-Contact-Us}{Contact
  Us}
\item
  \href{https://www.nytco.com/careers/}{Work with us}
\item
  \href{https://nytmediakit.com/}{Advertise}
\item
  \href{http://www.tbrandstudio.com/}{T Brand Studio}
\item
  \href{https://www.nytimes3xbfgragh.onion/privacy/cookie-policy\#how-do-i-manage-trackers}{Your
  Ad Choices}
\item
  \href{https://www.nytimes3xbfgragh.onion/privacy}{Privacy}
\item
  \href{https://help.nytimes3xbfgragh.onion/hc/en-us/articles/115014893428-Terms-of-service}{Terms
  of Service}
\item
  \href{https://help.nytimes3xbfgragh.onion/hc/en-us/articles/115014893968-Terms-of-sale}{Terms
  of Sale}
\item
  \href{https://spiderbites.nytimes3xbfgragh.onion}{Site Map}
\item
  \href{https://help.nytimes3xbfgragh.onion/hc/en-us}{Help}
\item
  \href{https://www.nytimes3xbfgragh.onion/subscription?campaignId=37WXW}{Subscriptions}
\end{itemize}
