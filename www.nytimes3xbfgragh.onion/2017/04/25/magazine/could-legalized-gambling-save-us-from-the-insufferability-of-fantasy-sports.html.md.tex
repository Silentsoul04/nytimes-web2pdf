Sections

SEARCH

\protect\hyperlink{site-content}{Skip to
content}\protect\hyperlink{site-index}{Skip to site index}

\href{https://myaccount.nytimes3xbfgragh.onion/auth/login?response_type=cookie\&client_id=vi}{}

\href{https://www.nytimes3xbfgragh.onion/section/todayspaper}{Today's
Paper}

Could Legalized Gambling Save Us From the Insufferability of Fantasy
Sports?

\url{https://nyti.ms/2pgerUS}

\begin{itemize}
\item
\item
\item
\item
\item
\item
\end{itemize}

Advertisement

\protect\hyperlink{after-top}{Continue reading the main story}

Supported by

\protect\hyperlink{after-sponsor}{Continue reading the main story}

\href{/column/on-sports}{On Sports}

\hypertarget{could-legalized-gambling-save-us-from-the-insufferability-of-fantasy-sports}{%
\section{Could Legalized Gambling Save Us From the Insufferability of
Fantasy
Sports?}\label{could-legalized-gambling-save-us-from-the-insufferability-of-fantasy-sports}}

\includegraphics{https://static01.graylady3jvrrxbe.onion/images/2017/04/30/magazine/30onsports2/30mag-30onsports-t_CA0-articleLarge.jpg?quality=75\&auto=webp\&disable=upscale}

By Jay Caspian Kang

\begin{itemize}
\item
  April 25, 2017
\item
  \begin{itemize}
  \item
  \item
  \item
  \item
  \item
  \item
  \end{itemize}
\end{itemize}

In the early 1990s, Bill Bradley, the New Jersey senator and former New
York Knick, argued several times in front of Congress that legalizing
sports betting would dehumanize athletes and lead to the rampant
corruption of children. This was no cynical political crusade; Bradley
was a true believer. He liked to declare that athletes were not
``roulette chips'' and tell the story of a game he once played in
Madison Square Garden. His Knicks were up by 5 at the end when the other
team hit a meaningless basket to cut the final lead to 3. A confused
Bradley heard cheers in the crowd, and when someone told him why --- the
opponent had just lost by fewer points and beat the spread predicted by
oddsmakers --- his eyes were opened to the sordid callousness of the
gambling world. Bradley helped push through the Professional and Amateur
Sports Protection Act in 1992, all but banning sports betting outside
Nevada and driving a growing industry that employed hundreds
underground.

As it turned out, the country's millions of sports gamblers didn't share
Bradley's concern for the sanctity of athletes like Bill Bradley or the
children who might grow up to be just like him. In practical terms, the
Sports Protection Act has mostly failed, and gambling has effectively
grown to be an inescapable part of big-time sports. Nobody knows how
many Americans bet illegally on sports, but the American Gaming
Association estimates that roughly \$150 billion is wagered annually.
(By contrast, \$4.5 billion was bet in Nevada sports books last year.) A
gambler with a hunch about, say, a Golden State Warriors playoff game
this month can use a debit card on any one of dozens of offshore online
betting sites. If the gut feeling runs a bit more toward whether a
specific player can handle the big stage, he or she can pick him in a
daily-fantasy lineup and enter a contest to win thousands.

Sometime in the next four months, the Supreme Court is likely to decide
whether to hear an appeal by Gov. Chris Christie and the State of New
Jersey that would essentially let states determine whether to allow
sports betting within their borders. What the court will do is not
certain: The debate is largely being framed in the language of states'
rights versus intrusive and economically harmful federal overreach, and
the position the Justice Department will take in any opinion it files is
unclear. But most in the gambling industry believe that some court case
either now or in the near future --- seven other states have recently
expressed interest in the added tax, infrastructure and tourism revenue
that would come from legalized gambling --- will succeed in repealing or
gutting the Sports Protection Act.

Even the professional leagues, which for years mostly opposed
legalization on the grounds that expanded gambling would corrupt their
product, have started to prepare for changes. Adam Silver, the
commissioner of the N.B.A., has publicly come out in favor of regulated
sports betting. Last month, the N.F.L. announced that the Oakland
Raiders would be moving to Las Vegas, putting an N.F.L. team in the
heart of the gambling industry. ``There's an irreversible momentum for
sports gambling,'' Daniel Wallach, a gambling expert at the law firm
Becker \& Poliakoff, told me. The rise of daily-fantasy sports (whose
legality is accepted by many states) and the placement of N.H.L. and
N.F.L. franchises in Las Vegas have diluted the spirit of the law and
spawned challenges in court. ``We could be looking at legal sports
betting in this country in three years,'' Wallach said. ``We've reached
the point of no return.''

\textbf{Legal gambling on} sports doesn't necessarily mean we will soon
follow the example of European and Asian soccer teams that put the logos
of online betting sites on the front of their jerseys. There's still a
sizable public-relations gap to overcome --- nobody really likes sports
bettors. It's why the charming gamblers in movies tend to be cardsharps
or pool hustlers and rarely sports bettors. In cards and pool and games
of chance, the people who get hurt are the house or the sucker who
willingly took on the hustler. In sports betting, everyone's a victim:
the teammates, the fans, the innocent spirit of American youth. The
potential ruin doesn't stop with the gambler but with the corruption of
the sport itself.

This, of course, is a dubious fear --- those who gamble away their
savings at the craps table are no different morally speaking from those
who blow their money betting on the Cavaliers. And if it sometimes seems
as if betting scandals are routine in Europe and Asia --- allegations of
match-fixing in the lower circuits of professional tennis have
circulated for years, for example, as have claims of Asian gambling
moguls fixing prominent European soccer matches --- the legality of
sports betting in a particular country hardly matters when the sharp
behind an illicit operation lives in Malaysia and bets millions in a
casino in Macau. Regulated sports gambling in New Jersey is unlikely to
lead to a new generation of high-powered match fixers in the United
States.

\includegraphics{https://static01.graylady3jvrrxbe.onion/images/2017/04/30/magazine/30onsports1/30mag-30onsports-t_CA1-articleLarge.jpg?quality=75\&auto=webp\&disable=upscale}

The larger fantasy-sports industry, however, offers both a model and a
cautionary tale for how sports betting could grow once it moves into the
light of day. For both fantasy sports and gambling, a vast market has
emerged for any bit of information that might provide an edge, and its
reach has been evident for years, since N.F.L. broadcasts introduced
fantasy stock tickers and pregame segments full of references to obscure
metrics like ``yards after catch'' and ``defense-adjusted value over
average.'' Over the past 20 or so years, the world of fantasy sports has
broadened its originally stultifying conversations --- arguments over
how many yards the running back Tiki Barber would gain against the
Cowboys' defense --- into inquiries about how games are actually won on
the field, whether by examining the way a baseball catcher frames
pitches or by establishing the most efficient spots to shoot from on a
basketball court. Fantasy talk evolved into real insight.

If gambling wants to shake off its bad associations, fueled in no small
part by the scammers who offer their stone-cold lock of the week for a
one-time payment of just \$49.99, it will need to share all the data
that goes into making an informed sports bet. ``There's nobody in this
country who knows more about predictive analytics than the guys working
in sports betting,'' Brian Musburger, the founder of the new media
venture Vegas Stats \& Information Network, told me. For years, Brian's
uncle, Brent, the longtime CBS sportscaster, always tried to wink at the
gamblers watching his games. If a seemingly meaningless field goal
covered the spread, Brent found a way to work in a reference to the
people who had just sighed in great, full-bodied relief. Brent left ESPN
and joined his nephew's start-up this year.

The Vegas Stats network, which debuted in February, still does not have
many viewers (if you type VSIN into Google, search results for ``vain''
still pop up), but if it succeeds, there may be an unexpected benefit to
traditionalists like Bradley who want sports to be about wins and
losses, free from financial concerns or selfish individual
accomplishments. Fantasy might have inspired smarter ways to look at
actual game play, but it has also tended to make the experience of
watching sports, whether with friends or at a bar, downright unbearable.

We've long since normalized the horror of listening to someone telling
you about the tragedy that befell a fantasy team, but if that talk were
instead about whether the Cowboys were going to cover the spread,
wouldn't that conversation come closer to the actual, idealized point of
the game? There are hundreds of reasons that a team might win a football
game by, say, 10 or more points. Each one requires you to consider
offenses, defensive schemes or, at the very worst, minor details like
the rumors about the quarterback's recent trip to Cabo San Lucas or what
your equipment-manager cousin told you about the impact of last night's
rain on turf conditions --- and every such discussion beats talking
about your fantasy squad. If Bradley wanted fans to root based on the
play of their favorite team, he banned the wrong pastime.

Advertisement

\protect\hyperlink{after-bottom}{Continue reading the main story}

\hypertarget{site-index}{%
\subsection{Site Index}\label{site-index}}

\hypertarget{site-information-navigation}{%
\subsection{Site Information
Navigation}\label{site-information-navigation}}

\begin{itemize}
\tightlist
\item
  \href{https://help.nytimes3xbfgragh.onion/hc/en-us/articles/115014792127-Copyright-notice}{©~2020~The
  New York Times Company}
\end{itemize}

\begin{itemize}
\tightlist
\item
  \href{https://www.nytco.com/}{NYTCo}
\item
  \href{https://help.nytimes3xbfgragh.onion/hc/en-us/articles/115015385887-Contact-Us}{Contact
  Us}
\item
  \href{https://www.nytco.com/careers/}{Work with us}
\item
  \href{https://nytmediakit.com/}{Advertise}
\item
  \href{http://www.tbrandstudio.com/}{T Brand Studio}
\item
  \href{https://www.nytimes3xbfgragh.onion/privacy/cookie-policy\#how-do-i-manage-trackers}{Your
  Ad Choices}
\item
  \href{https://www.nytimes3xbfgragh.onion/privacy}{Privacy}
\item
  \href{https://help.nytimes3xbfgragh.onion/hc/en-us/articles/115014893428-Terms-of-service}{Terms
  of Service}
\item
  \href{https://help.nytimes3xbfgragh.onion/hc/en-us/articles/115014893968-Terms-of-sale}{Terms
  of Sale}
\item
  \href{https://spiderbites.nytimes3xbfgragh.onion}{Site Map}
\item
  \href{https://help.nytimes3xbfgragh.onion/hc/en-us}{Help}
\item
  \href{https://www.nytimes3xbfgragh.onion/subscription?campaignId=37WXW}{Subscriptions}
\end{itemize}
