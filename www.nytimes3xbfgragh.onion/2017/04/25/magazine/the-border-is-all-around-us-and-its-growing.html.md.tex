Sections

SEARCH

\protect\hyperlink{site-content}{Skip to
content}\protect\hyperlink{site-index}{Skip to site index}

\href{https://myaccount.nytimes3xbfgragh.onion/auth/login?response_type=cookie\&client_id=vi}{}

\href{https://www.nytimes3xbfgragh.onion/section/todayspaper}{Today's
Paper}

The Border Is All Around Us, and It's Growing

\url{https://nyti.ms/2pgeHU3}

\begin{itemize}
\item
\item
\item
\item
\item
\item
\end{itemize}

Advertisement

\protect\hyperlink{after-top}{Continue reading the main story}

Supported by

\protect\hyperlink{after-sponsor}{Continue reading the main story}

\href{/column/first-words}{First Words}

\hypertarget{the-border-is-all-around-us-and-its-growing}{%
\section{The Border Is All Around Us, and It's
Growing}\label{the-border-is-all-around-us-and-its-growing}}

\includegraphics{https://static01.graylady3jvrrxbe.onion/images/2017/04/30/magazine/30firstwords/30mag-30firstwords-t_CA0-articleLarge.jpg?quality=75\&auto=webp\&disable=upscale}

By Laila Lalami

\begin{itemize}
\item
  April 25, 2017
\item
  \begin{itemize}
  \item
  \item
  \item
  \item
  \item
  \item
  \end{itemize}
\end{itemize}

The Border Patrol agent watched our Prius approach, then signaled for us
to stop. Behind him stood several others in green uniforms, hands
resting on holsters, eyes hidden behind sunglasses. German shepherds
panted in the heat. ``Are you all U.S. citizens?'' the agent asked,
leaning against the driver's-side window and glancing around our car.
``Yes,'' said one of my companions, an artist from Iowa. ``Yes,'' echoed
the other, a poet from Connecticut. Then it was my turn. ``Yes,'' I
said. The agent's gaze lingered on me for a moment. Then he stood up and
waved us through the border.

Except this was not a border: This was the middle of Interstate 10
between El Paso and Marfa, Tex. No matter. At the Sierra Blanca
checkpoint, agents can make arrests for drugs or weapons, share
information with federal agencies and turn undocumented immigrants over
to Immigration and Customs Enforcement. There are many such checkpoints
scattered throughout the continental United States --- borders within
borders.

Borders mark the contours of nations, states, even cities, defining them
by separating them from all others. A border can be natural --- an
ocean, a river, a chain of mountains --- or it can be artificial,
splitting a homogeneous landscape into two. Often it is highly literal,
announcing itself in the shape of a concrete wall, a sand berm, a tall
fence topped with barbed wire. But whatever form it takes, a border
always conveys meaning. Hours before my encounter with the Border
Patrol, as the airplane I was on began its descent, I saw from my window
seat the wall that separates El Paso from Ciudad Juárez, Mexico. On one
side were gleaming towers, giant freeways and sprawling parks; on the
other, homes huddling together in the afternoon light, winding streets
and patches of dry grass. Here you will find safety and prosperity, the
wall seemed to say, but over there lie danger and poverty. It's a
message that ignores the cities' joint history, language and cultures.
But it is simple --- one might say simplistic --- and that is what gives
it power.

For much of the United States' history, national frontiers were fluid,
expanding through territorial conquest and purchases. But at the start
of the 20th century, as Arizona and New Mexico approached statehood and
the country's continental borders became stable, so did the desire to
secure them and police them --- first through congressional acts that
prohibited immigration from certain countries and later through the
building of fences and walls. During his campaign for the presidency,
Donald Trump often promised to extend a wall along the Southern border
and have Mexico pay for it. At his rallies, this promise was met with
cheers and chants of ``Build that wall!'' When Vicente Fox, a former
president of Mexico, declared that his nation had no intention of paying
for any such wall, Trump's response was, ``The wall just got 10 feet
higher.'' The more it was challenged, the higher it became, as if
literalizing the border could make all debate about it disappear.

Whether the administration can find the money to construct an immense
border wall remains to be seen. In the meantime, the legal apparatus
around it is already being built. This month, speaking to Customs and
Border Protection officers in Nogales, Ariz., Attorney General Jeff
Sessions promised them ``more tools in your fight against criminal
aliens'' --- including charging immigrants who repeatedly cross into the
United States illegally with felonies and, when possible, with document
fraud and aggravated identity theft, which can carry mandatory prison
time. His language was the language of war: Nogales, Sessions said, was
``ground zero'' in the fight to secure the border, a place where
``ranchers work each day to make an honest living'' while under threat
from ``criminal organizations that turn cities and suburbs into war
zones, that rape and kill innocent civilians.'' Under the new
administration, he said, his Justice Department was prepared for the
fight: ``It is here, on this sliver of land, on this border, where we
first take our stand.''

In this kind of rhetoric, the border separates not just nationals from
foreigners, rich from poor and north from south, but also order from
chaos, civilization from barbarians, decent people from criminals.
Location becomes character, with everything that designation entails. A
person is either American and an honest worker, or she is not American
and is a criminal alien. The two categories are seen as inherent and
inflexible.

In January, Trump signed an executive order temporarily barring
nationals of seven Muslim-majority countries from entering the United
States, even if they were green-card holders or refugees who had already
been cleared for resettlement. But as reports emerged of families
separated by the order, passengers stranded thousands of miles from
home, even an infant being denied scheduled surgery, the full effect of
this virtual wall revealed itself. By the time the ban was lifted by
federal courts, the experience had already brought to the national
consciousness a renewed awareness of what happens at the border. In this
in-between space, rights we take for granted disappear. At points of
entry to the United States, nobody --- not even American citizens or
permanent residents --- is fully protected by the Fourth Amendment,
which safeguards against unreasonable searches and seizures. Customs and
Border Protection officers can search luggage as well as phones, tablets
and laptops. Occasionally they ask for online browsing histories and
passwords to social-media accounts. A poorly phrased joke on Twitter, a
compromising picture on a private Instagram account, a Facebook argument
with a crazy uncle --- all these could be readable by C.B.P. officers,
entirely at their discretion. Again, the border sends a message: Watch
what you say.

The border's messages always carry with them hints of violence. In the
19th century, the American frontier was a place of conquest, a place
where laws did not apply and deadly clashes could happen at any moment.
That aura of risk and brutality still hangs over airports, the closest
thing we have to frontier outposts and the gateways to cross-border
travel. When travelers step into the secure area of an airport, they
leave behind bottles of water, take off their shoes and expose their
bodies to X-rays, all for the sake of protecting themselves from the
potential violence of terrorists. But this system can perpetrate
violence by itself. This month, a ticketed passenger who refused to give
up his seat on a United Airlines flight from O'Hare International
Airport in Chicago was brutally dragged away by the police. David Dao's
injuries, according to his lawyer, Thomas Demetrio, include a
concussion, a broken nose, two lost teeth and sinus damage that could
require surgery. Demetrio went on to ask, ``Are we going to just
continue to be treated like cattle?''

This dehumanization is a common feature of the border. Some years ago,
returning home from a holiday in Morocco, my husband and I passed
through immigration at Kennedy Airport. The border agent glanced at my
passport, which lists Morocco as my place of birth. Then she looked at
my husband's and, with a chuckle, asked him how many camels he had
traded for me. Even in my shock, I understood that what the agent was
trying to assert was her own authority, her superiority over me. If I
had dared to challenge her, I might have ended up subject to a secondary
search and further questioning. My silence was the price that the border
demanded.

Border walls are literal expressions of our worst fears. Terrorists,
rapists, drug dealers and various ``bad hombres'' are all said to come
from somewhere else; drawing lines, we are told, will keep us safe from
them. But the lines keep multiplying. What formally counts as the
border, according to the United States government, is not just the lines
separating the United States from Canada and Mexico, but any American
territory within 100 miles of the country's perimeter, whether along
land borders, ocean coasts or Great Lakes shores. That 100-mile strip of
land encompasses almost entirely the states of Connecticut, Delaware,
Florida, Hawaii, Maine, Massachusetts, Michigan, New Hampshire, New
Jersey, New York, Rhode Island and Vermont --- along with the most
populated parts of many others, including California and Illinois. In
total, the 100-mile-wide border zone is home to two-thirds of the
nation's population.

This is such a staggering fact that it bears repeating: The vast
majority of Americans, roughly 200 million, are effectively living in
the border zone. Any of these people could one day face checkpoints like
the one I went through in Sierra Blanca, Tex. They can be asked about
their citizenship and, if they fail to persuade the agent --- because of
how they look, act or sound --- they can be detained. The Justice
Department established these regulations in 1953 and, though they
periodically attract attention, they have never been changed. As we move
to erect and enforce more borders, this is another message worth
apprehending: Borders do not simply keep others out. They also wall us
in.

Advertisement

\protect\hyperlink{after-bottom}{Continue reading the main story}

\hypertarget{site-index}{%
\subsection{Site Index}\label{site-index}}

\hypertarget{site-information-navigation}{%
\subsection{Site Information
Navigation}\label{site-information-navigation}}

\begin{itemize}
\tightlist
\item
  \href{https://help.nytimes3xbfgragh.onion/hc/en-us/articles/115014792127-Copyright-notice}{©~2020~The
  New York Times Company}
\end{itemize}

\begin{itemize}
\tightlist
\item
  \href{https://www.nytco.com/}{NYTCo}
\item
  \href{https://help.nytimes3xbfgragh.onion/hc/en-us/articles/115015385887-Contact-Us}{Contact
  Us}
\item
  \href{https://www.nytco.com/careers/}{Work with us}
\item
  \href{https://nytmediakit.com/}{Advertise}
\item
  \href{http://www.tbrandstudio.com/}{T Brand Studio}
\item
  \href{https://www.nytimes3xbfgragh.onion/privacy/cookie-policy\#how-do-i-manage-trackers}{Your
  Ad Choices}
\item
  \href{https://www.nytimes3xbfgragh.onion/privacy}{Privacy}
\item
  \href{https://help.nytimes3xbfgragh.onion/hc/en-us/articles/115014893428-Terms-of-service}{Terms
  of Service}
\item
  \href{https://help.nytimes3xbfgragh.onion/hc/en-us/articles/115014893968-Terms-of-sale}{Terms
  of Sale}
\item
  \href{https://spiderbites.nytimes3xbfgragh.onion}{Site Map}
\item
  \href{https://help.nytimes3xbfgragh.onion/hc/en-us}{Help}
\item
  \href{https://www.nytimes3xbfgragh.onion/subscription?campaignId=37WXW}{Subscriptions}
\end{itemize}
