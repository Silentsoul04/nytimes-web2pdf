Rei Kawakubo, Interpreter of Dreams

\url{https://nyti.ms/2q4aMYp}

\begin{itemize}
\item
\item
\item
\item
\item
\item
\end{itemize}

\includegraphics{https://static01.graylady3jvrrxbe.onion/images/2017/04/30/magazine/30kawakubo2/30kawakubo2-articleLarge.jpg?quality=75\&auto=webp\&disable=upscale}

Sections

\protect\hyperlink{site-content}{Skip to
content}\protect\hyperlink{site-index}{Skip to site index}

Feature

\hypertarget{rei-kawakubo-interpreter-of-dreams}{%
\section{Rei Kawakubo, Interpreter of
Dreams}\label{rei-kawakubo-interpreter-of-dreams}}

The 74-year-old force behind the avant-garde label Comme des Garçons
makes otherworldly clothes that express hidden desires and fears.

Credit...Erik Madigan Heck for The New York Times. Headpiece by Julien
d'Ys. Model: Saskia de Brauw.

Supported by

\protect\hyperlink{after-sponsor}{Continue reading the main story}

Photographs by Erik Madigan Heck and Text by Leanne Shapton

\begin{itemize}
\item
  April 26, 2017
\item
  \begin{itemize}
  \item
  \item
  \item
  \item
  \item
  \item
  \end{itemize}
\end{itemize}

Rei Kawakubo is a 74-year-old Japanese designer who has been making
monochromatic, boiled, overstuffed, unraveling, surreal clothes since
she began her label, Comme des Garçons, in 1969. In the '80s, an era of
triangular fluorescent shoulders and moussed bangs, she sent widow's
weeds trailing down the runway. In the '90s, her clothes grew tumors
that resembled bustles, blurring the lines between illness and beauty,
what to look at and what to look away from. While her stores stock
simple cotton poplin shirts as well as more challenging lumpen jackets
and her runway shows present one-off sculptures, the through-line is a
thoughtful rejection of \emph{should} and a strong feeling of
\emph{perhaps}. The name of her label, which translates to ``Like the
Boys,'' is a quiet, defiant shrug.

Over the decades, Kawakubo has returned to certain proportions and
themes. A Peter Pan collar. An A-line skirt. A uniform. A cocoon. Her
variations on these natural and childlike shapes present a distilled and
dreamy version of dress. When I think of her clothes, my mind fishes up
an image I saw over 20 years ago in a magazine: Two women stood in a
dirt road. One wore a dress that struck me as utterly unflattering,
until it didn't. Then it struck me as beautiful. Innocent. The shape of
a doll's dress. The skirt hit midcalf, flat shoes, legs sturdy. In that
moment, I understood the difference between being looked at and looking
at yourself. Squarely and generously. The difference between sartorial
safety and statement.

Kawakubo's clothes don't move from day to evening. They don't flatter.
They don't slim. They don't fit perfectly or offer comfort or
reassurance. But then, given a beat, they do all of the above. They are
not simply clothes: They are ideas. They are feelings. Kawakubo once
said that her collections begin as ``nonverbal, abstract images inside
of me.''

In May, the Metropolitan Museum of Art's Costume Institute will present
an exhibition of Kawakubo's work --- the first show of a living fashion
designer since an Yves Saint Laurent exhibit in 1983. It will bring
together nearly 40 years of her baffling and beautiful designs for a
wonderfully awkward parade.

On a recent spring day, Erik Madigan Heck photographed six pieces from
Kawakubo's fall 2017 collection --- called ``Future of Silhouette'' ---
on the Dutch model Saskia de Brauw. The clothes are made from raw
material that Kawakubo calls ``nonfabrics'': what appeared to be
cauterized rubber, rug pads, cotton batting, stuffing and duct tape. The
hairstylist and artist Julien d'Ys's headpieces, which also appeared in
the runway show, were made of pot scrubbers from a Paris supermarket.

When I arrived at the studio where the shoot took place, the garments
were still sealed in enormous FedEx boxes, crated like Count Dracula for
their voyage from Tokyo. They emerged with help from a white-gloved team
and were hung on rolling racks.

In a YouTube video I watched of the Paris catwalk show, the pieces
looked like foil balls, lint bunnies, craft paper and wool felt. In
person, they still did. But close up, they were meticulously and
gorgeously constructed; their rough-hewed, blunted exteriors encased in
thoughtful and ergonomic scaffolding, buttresses of padded wiring,
canvas boning, industrial-strength crinolines.

Kawakubo reluctantly titles her collections after they are designed:
``Mud-Dyed'' (spring 1985); ``Ultrasimple'' (spring 1993); ``Abstract
Excellence'' (spring 2004); ``Bad Taste'' (fall 2008); ``2 Dimensions''
(fall 2012). She claims she just does this for journalists, preferring
observers to come to their own conclusions. I found myself nicknaming
the six pieces: \emph{Bubble Bath}, \emph{Calamari}, \emph{Linty},
\emph{Jiffy Pop}, \emph{Vase} and \emph{Red}.

Since her spring 2014 collection, ``Not Making Clothing,'' Kawakubo has
stopped showing conventionally wearable ensembles on the runway, and her
collections have become more extreme, unwieldy and conceptual. This
later work illustrates the version of clothes that are inside our heads
and hearts and bodies: the contradictions, the indecisions and the
architecture of the subconscious. Looking at Kawakubo's Rorschach-like
new creations, I'm led through a free association of my own memories and
emotions.

\textbf{Opening image:} A white suit, belted at the waist, appears
contained, but a babble of valves and blisters erupts at both ends in
herpetic chaos. We are organic and alive, reactive.

At least once every couple of days, I think about Isaac Newton and his
equal and opposite reactions. Or the psychoanalyst Adam Phillips's
suggestion that one's excesses are proportional to one's poverties.

In 2008, I saw Fiona Shaw as Winnie in the Samuel Beckett play ``Happy
Days.'' Her Winnie was a cheerful wife who in the first act was buried
up to her waist, then in the second act up to her neck, in a mound of
dirt. She prattled on to her husband and dug around in her black
matelassé purse. ``So little to say, so little to do, and the fear so
great,'' she chirps.

This garment is a land mine of artistic references. There is
\href{https://www.theguardian.com/lifeandstyle/2016/mar/14/louise-bourgeois-feminist-art-sculptor-bilbao-guggenheim-women\#img-3}{a
photograph} of the artist Louise Bourgeois, taken in 1975. She is
standing in front of her New York City stoop, wearing a latex cast of
her bubbling sculpture ``Avenza'' from the late '60s. Its clusters
presage Kawakubo's nippled creation, and the similarity is reassuring
and funny, like mother-and-daughter matching outfits. Speaking of these
organic forms, Bourgeois said, ``If you hold a naked child against your
naked breast, it is not the end of softness, it is the beginning of
softness, it is life itself.''

Like Beckett and Bourgeois, Kawakubo maps the empty spaces within us.
Pulling the self inside out and outside in. Palettes of mud, pillowcases
of doorknobs, bags of ice.

\includegraphics{https://static01.graylady3jvrrxbe.onion/images/2017/04/30/magazine/30kawakubo1/30mag-30kawakubo-t_CA0-articleLarge.jpg?quality=75\&auto=webp\&disable=upscale}

\textbf{Above:} A volcanic froth is barely, inadequately contained by a
tough rubber carapace. Softest polyester stuffing spills out from black
armor. It's a leather jacket thrown over a bubble bath. This could
describe a few people I know.

At a certain point in my late 30s, I sought out quilted things. Quilted
jackets, quilted boots, quilted trousers, quilted quilts. I found a long
pink coat from the '80s with duffel buttons, a Bill Blass parka that
resembled a sectioned bar of milk chocolate, a Norma Kamali bomber with
puffy linebacker shoulders. A huge comforter in muted taupes, stitched
in a grid of rainbow curves.

I'd had a baby the year before. My body was different, bigger in places,
smaller in others. I was happier with how I looked. I liked looking
older. But as my baby grew, my fears grew. I wanted to be buffered,
padded. I wanted more room between outside and inside. For her. For us.

Being a mother made me more punk, more angry, righteous, foul-mouthed
than I had ever been. It also made me more vulnerable. Visiting a lake
one summer, I drove to an auto-body shop and bought a rubber inner tube
for \$5. I inflated it, got inside and pushed off the dock. This made me
near-perfectly happy.

Image

Credit...Erik Madigan Heck for The New York Times. Model: Saskia de
Brauw.

\textbf{Above:} A reptile of lint stirs and rears its head. Inside, the
lining is black, there is no zipper, the body is swallowed whole. I
think of boa constrictors, of great whites, of barrels going over
Niagara Falls. And of ectoplasm.

In
\href{http://www.slate.com/blogs/the_vault/2014/04/09/spirit_photography_craig_and_george_falconer_images_of_ghosts.html}{spirit
photographs} from the turn of the 20th century, ectoplasm appears to be
made of wool, cotton or paper. In the course of a séance, a medium would
produce it from his or her mouth, nose or navel, and it would then,
ideally, form the figure of a lost loved one. The spirit photographs are
truly weird, and also laughable. Gobs of cheesecloth stream from the
face of someone in a trance state. Ragged balls of fabric form
potato-like clumps stuck to a neck.

I want to believe in ectoplasm, in the figuration of love from the
transmission of spirit. I want there to be a physical spontaneous
supernatural sneeze of form. Life after death.

I want a paranormal dress.

Image

Credit...Erik Madigan Heck for The New York Times. Headpiece by Julien
d'Ys. Model: Saskia de Brauw.

\textbf{Above:} An exploded metallic popcorn kernel. I approach and am
reflected, distorted in its crinkly, convex foil surfaces. It is all
outer space. The model's fingers barely poke out of the ``sleeves.''
Inside, the space is lined in smooth, undyed cotton.

This is a persona I bring to cocktail parties. The figure I see in my
head, caricatured from the inside out, trying to connect but only
reflecting and being reflected. Foiling. It feels like my inability to
retain what is said to me in the face of my self-absorption. Waking up
in the night cringing at small talk, blurted inanities, perceived
slights.

But it's like a croquembouche of exposure and erasure. As a child, I had
a nightmare whenever I had a fever. In it, I was miniature, enveloped in
a large mass. It was suffocating and bulbous, this mass. Sometimes I saw
it as if it were part of a cartoon panel: It filled the panel and
ballooned out of it. I was a speck, and then less than a speck. It was a
horrible dream. Sometimes I catch corners of it, when I am near sleep,
or when I am overheated. I wonder if it is about being in the womb or
the birth canal. Is it prenatal, or is it about love and death, outer
space and deep space, sea anemone and siren?

What does Kawakubo dream?

Image

Credit...Erik Madigan Heck for The New York Times. Model: Saskia de
Brauw.

\textbf{Above:} A man sweeps his hands in an hourglass shape to describe
a woman's figure. Inside the piece, there is little space; it is hot
when the model steps out of it. There is room for one elbow but not the
other.

A designer's female dummies are called Judy (males are called James).
These come in different sizes and proportions, their silhouettes
changing with the times --- with the trends of idealized bodies. Is the
dummy a canvas or a narrator? I wonder if, in this design, Kawakubo is
writing auto-fiction, externalizing the experience of herself as a
designer with her Judys.

I have two Comme des Garçons dresses that I bought secondhand. One is
stiff white cotton, like a nurse's uniform but wider, with translucent
strips down the sides and a Peter Pan collar. The other is ivory cotton
voile with a panel of green silk velvet wrapping halfway around the
belly and ending in a long, gaping seam. I wore this once to a dinner. I
had been concealing my pregnancy because a subchorionic hematoma made
miscarriage likely. My friend saw the dress and raised an eyebrow.

My daughter is 4 now. Kawakubo's designs remind me of my daughter's
peculiarities of dress. The sash has to be tied in the front. The trim
on the sleeves can't be blue. Seams are unbearable, and tags need to be
cut out. She likes to wear the same dress every day. She doesn't look in
the mirror to see how she appears, so I think about how her choices must
make her feel. Her body is not an image yet, not a projection or a pose
--- it's a boundary.

Image

Credit...Erik Madigan Heck for The New York Times. Headpiece by Julien
d'Ys. Model: Saskia de Brauw.

\textbf{Above:} A red couch upturned to fit through a doorway, or to
barricade it. It has no arms but is not armless. It stands at attention
and slouches into its pockets. Royal Guard and Charlie Chaplin.

Kawakubo's palettes are minimal and pure. A perfect strawberry red.
Bubble-gum pink. Years and years of black, and deep naval, navy blue.

I've always had a kind of synesthesia for clothes: Sweaters have to be
red. Collars have to be white. Trousers, blue. My school pictures show
missing teeth and variations of pigtails, but the collars are always
white, the sweaters are always red.

Sometimes I pull a sweater over my head and wait a moment before pushing
my arms through the sleeves. I'm looking for a German word for the
comfort, relief even, of not putting your arms through the armholes.
It's a form of self-soothing, generating something like the
anxiety-reducing squeeze box developed by the animal scientist Temple
Grandin, who has written about her autism. It calms my sensitivity.

When I get out of a pool after a swim, I pull a towel over my shoulders,
binding my arms to my sides at the elbow in a swaddle. Again there is
this moment of blissful containment.

Fashion media often insist on ``What We Want Now,'' a forecast of desire
and perhaps the economic or political climate. But Kawakubo's clothes
erase the reality of my body within clothes. Hers remind me of
innocence: not childishness, but the innocence that we have when we are
4 and we don't know what we look like but know what we want to wear and
why.

My daughter wakes yelling from her afternoon nap.

``Shhhh.'' I wrap her up and hold her. ``It's only a nightmare.''

``Don't look at me,'' my daughter says, in her dress.

Then she takes off all her clothes.

Advertisement

\protect\hyperlink{after-bottom}{Continue reading the main story}

\hypertarget{site-index}{%
\subsection{Site Index}\label{site-index}}

\hypertarget{site-information-navigation}{%
\subsection{Site Information
Navigation}\label{site-information-navigation}}

\begin{itemize}
\tightlist
\item
  \href{https://help.nytimes3xbfgragh.onion/hc/en-us/articles/115014792127-Copyright-notice}{©~2020~The
  New York Times Company}
\end{itemize}

\begin{itemize}
\tightlist
\item
  \href{https://www.nytco.com/}{NYTCo}
\item
  \href{https://help.nytimes3xbfgragh.onion/hc/en-us/articles/115015385887-Contact-Us}{Contact
  Us}
\item
  \href{https://www.nytco.com/careers/}{Work with us}
\item
  \href{https://nytmediakit.com/}{Advertise}
\item
  \href{http://www.tbrandstudio.com/}{T Brand Studio}
\item
  \href{https://www.nytimes3xbfgragh.onion/privacy/cookie-policy\#how-do-i-manage-trackers}{Your
  Ad Choices}
\item
  \href{https://www.nytimes3xbfgragh.onion/privacy}{Privacy}
\item
  \href{https://help.nytimes3xbfgragh.onion/hc/en-us/articles/115014893428-Terms-of-service}{Terms
  of Service}
\item
  \href{https://help.nytimes3xbfgragh.onion/hc/en-us/articles/115014893968-Terms-of-sale}{Terms
  of Sale}
\item
  \href{https://spiderbites.nytimes3xbfgragh.onion}{Site Map}
\item
  \href{https://help.nytimes3xbfgragh.onion/hc/en-us}{Help}
\item
  \href{https://www.nytimes3xbfgragh.onion/subscription?campaignId=37WXW}{Subscriptions}
\end{itemize}
