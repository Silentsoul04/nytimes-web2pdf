Sections

SEARCH

\protect\hyperlink{site-content}{Skip to
content}\protect\hyperlink{site-index}{Skip to site index}

\href{https://myaccount.nytimes3xbfgragh.onion/auth/login?response_type=cookie\&client_id=vi}{}

\href{https://www.nytimes3xbfgragh.onion/section/todayspaper}{Today's
Paper}

Letter of Recommendation: `Pistol Pete's Homework Basketball'

\url{https://nyti.ms/2poAMQo}

\begin{itemize}
\item
\item
\item
\item
\item
\item
\end{itemize}

Advertisement

\protect\hyperlink{after-top}{Continue reading the main story}

Supported by

\protect\hyperlink{after-sponsor}{Continue reading the main story}

\href{/column/letter-of-recommendation}{Letter of Recommendation}

\hypertarget{letter-of-recommendation-pistol-petes-homework-basketball}{%
\section{Letter of Recommendation: `Pistol Pete's Homework
Basketball'}\label{letter-of-recommendation-pistol-petes-homework-basketball}}

\includegraphics{https://static01.graylady3jvrrxbe.onion/images/2017/04/30/magazine/30lor/30mag-30lor-t_CA1-articleLarge.jpg?quality=75\&auto=webp\&disable=upscale}

By Barrett Swanson

\begin{itemize}
\item
  April 27, 2017
\item
  \begin{itemize}
  \item
  \item
  \item
  \item
  \item
  \item
  \end{itemize}
\end{itemize}

When I was 12, shortly after I joined a youth-basketball team, my father
summoned me to the living room. He was holding a set of VHS tapes called
``Pistol Pete's Homework Basketball,'' which he thought might help me
get ready for the coming season. Naturally, I was dubious. In what sense
could basketball be considered homework? And who was Pistol Pete
Maravich anyway? With the solemnity that other fathers adopt when
passing down heirloom rifles, Dad handed me the tapes and said, ``Just
watch.''

I popped the first one into the VCR, but Maravich's appearance did
little to curb my skepticism. Gawky and bird-thin, he wore a Chicago-cop
mustache and spoke in the gentle exhortations of a youth pastor.
``Practice, practice, practice,'' he said, shaking a finger at the
camera. ``Remember: Don't forget to do your homework.'' What I didn't
know at the time was that Maravich was a legend; he set the scoring
record in the N.C.A.A. and made five trips to the N.B.A. All-Star Game
throughout the 1970s. But he was most famous for his on-court wizardry,
for his nimble behind-the-back passes and persuasive head-fakes that
made even the most adroit defenders look ornamental and weak.

``Homework Basketball,'' which was released in 1987, started with
lessons on the fundamentals, like shooting and dribbling. But as the
videos went on, Maravich began to initiate his young viewers into a more
esoteric set of skills. ``I'm also going to teach you today how to do
the creative --- the more advanced --- type of pass, the artistic type
of pass.'' The video then furnished a bevy of drills whose names sounded
like dance moves that you would execute only at a wedding: ``Different
Strokes,'' ``Scrambled Eggs,'' ``The Laid-Back.'' Alone in the penumbral
glow of the gym, Maravich pirouetted across the hardwood, schooling
invisible foes with breakneck sleights of hand. Again and again, I
rewound the trick-shot sequence, watching as Maravich leapt into the air
and looped the ball between his legs before completing a reverse layup.

Throughout that basketball season, I rose before dawn to begin my
regimen. In the gloom of the driveway, I followed the Pistol's workout,
lassoing the ball between my legs (sometimes wearing a blindfold). Then
I rehearsed behind-the-back passes, first with my right arm, then with
my left, aiming for a bull's-eye I had painted on the wall of our
garage. My efforts were downright monastic --- asceticism in the service
of its opposite.

Yet whenever I applied Maravich's instructions to actual games, my
coaches grimaced at the theatrics. Occasionally it would work out: I
still recall faking a behind-the-back pass and juking my opponent, only
to retract the ball and flip it against the backboard for an uncontested
layup. But just as often my glitzy passes veered into the hands of
waiting defenders. So enamored was I with the aesthetic possibilities of
the sport, I overlooked the simple urgency of competition. Eventually,
my flashiness earned me a spot on the bench and, once I got to high
school, a berth on the J.V. squad.

Now in my early 30s, I rarely play basketball anymore, fearful as I am
of shredding an ankle or mangling a knee. These days, if Pistol Pete's
videos remain ``instructional'' for me, it's because they insist that
glimmers of artistry can live, however briefly, in activities we might
otherwise regard as brute and mechanistic. A friend of mine claims that
he cannot watch sports because their simplistic teleology --- putting a
ball in a basket, lobbing a puck into a goal --- distills ``the futility
of the whole human experience.'' Sports, for him, expose the extent to
which all of us are obsessively zeroed in on the next mark, the next
burst of glory, all of which will be swiftly and inexorably forgotten.
But it's not the primal exigencies of winning that compel my attention.
Rather, it's those moments when something unexpected, something artful,
flashes out of the roil of bodies in space.

On certain days, I'll jog past a public park where scrums of
20-somethings hotfoot across the concrete and will feel an odd twinge in
the gut. One recent afternoon, I actually summoned the courage to join
them. On a fast-break, something miraculous happened. I caught an outlet
pass and, blazing across center court, I was momentarily overcome by the
ghost of that boy --- the kid toiling in his driveway --- at which point
I dished a no-look pass through a defender's legs and hit my teammate
midstride, who finished things off with a balletic dunk. Heads turned.
Teammates said, \emph{O.K., O.K.!}

This is what will be remembered, I thought, when the ground takes us
back. The score will be forgotten, but what will endure are those rare
moments when the plodding human body escaped its mortal form and
entered, if only for a second, the realm of grace.

At 40, not long after he released the videos, Pistol Pete Maravich died
during a game of pickup basketball. Unknown to anyone, Maravich
included, he was born with a rare heart defect that left him without a
left coronary artery. Cardiologists who reviewed his autopsy were
flummoxed that he managed to play basketball for so long. In light of
this knowledge, I find it tempting to read retroactively into his career
a kind of gorgeous desperation, to wonder whether, on some level, he
felt the burden of his truncated fate, a questing urge to linger in the
spotlight while it was still his.

Advertisement

\protect\hyperlink{after-bottom}{Continue reading the main story}

\hypertarget{site-index}{%
\subsection{Site Index}\label{site-index}}

\hypertarget{site-information-navigation}{%
\subsection{Site Information
Navigation}\label{site-information-navigation}}

\begin{itemize}
\tightlist
\item
  \href{https://help.nytimes3xbfgragh.onion/hc/en-us/articles/115014792127-Copyright-notice}{©~2020~The
  New York Times Company}
\end{itemize}

\begin{itemize}
\tightlist
\item
  \href{https://www.nytco.com/}{NYTCo}
\item
  \href{https://help.nytimes3xbfgragh.onion/hc/en-us/articles/115015385887-Contact-Us}{Contact
  Us}
\item
  \href{https://www.nytco.com/careers/}{Work with us}
\item
  \href{https://nytmediakit.com/}{Advertise}
\item
  \href{http://www.tbrandstudio.com/}{T Brand Studio}
\item
  \href{https://www.nytimes3xbfgragh.onion/privacy/cookie-policy\#how-do-i-manage-trackers}{Your
  Ad Choices}
\item
  \href{https://www.nytimes3xbfgragh.onion/privacy}{Privacy}
\item
  \href{https://help.nytimes3xbfgragh.onion/hc/en-us/articles/115014893428-Terms-of-service}{Terms
  of Service}
\item
  \href{https://help.nytimes3xbfgragh.onion/hc/en-us/articles/115014893968-Terms-of-sale}{Terms
  of Sale}
\item
  \href{https://spiderbites.nytimes3xbfgragh.onion}{Site Map}
\item
  \href{https://help.nytimes3xbfgragh.onion/hc/en-us}{Help}
\item
  \href{https://www.nytimes3xbfgragh.onion/subscription?campaignId=37WXW}{Subscriptions}
\end{itemize}
