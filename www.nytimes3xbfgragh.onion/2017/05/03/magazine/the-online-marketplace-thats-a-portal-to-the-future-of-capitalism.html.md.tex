Sections

SEARCH

\protect\hyperlink{site-content}{Skip to
content}\protect\hyperlink{site-index}{Skip to site index}

\href{https://myaccount.nytimes3xbfgragh.onion/auth/login?response_type=cookie\&client_id=vi}{}

\href{https://www.nytimes3xbfgragh.onion/section/todayspaper}{Today's
Paper}

The Online Marketplace That's a Portal to the Future of Capitalism

\url{https://nyti.ms/2qxtEPF}

\begin{itemize}
\item
\item
\item
\item
\item
\item
\end{itemize}

Advertisement

\protect\hyperlink{after-top}{Continue reading the main story}

Supported by

\protect\hyperlink{after-sponsor}{Continue reading the main story}

The Money Issue

\hypertarget{the-online-marketplace-thats-a-portal-to-the-future-of-capitalism}{%
\section{The Online Marketplace That's a Portal to the Future of
Capitalism}\label{the-online-marketplace-thats-a-portal-to-the-future-of-capitalism}}

\includegraphics{https://static01.graylady3jvrrxbe.onion/images/2017/05/07/magazine/07alibaba/07mag-07alibaba-t_CA0-articleLarge.jpg?quality=75\&auto=webp\&disable=upscale}

By \href{https://www.nytimes3xbfgragh.onion/by/john-herrman}{John
Herrman}

\begin{itemize}
\item
  May 3, 2017
\item
  \begin{itemize}
  \item
  \item
  \item
  \item
  \item
  \item
  \end{itemize}
\end{itemize}

About three years ago, like millions of other Instagram users, I became
acquainted with an app called Wish. At first, my awareness was purely
subliminal. Wish ads appeared in my social-media feeds without
explanation or context, peddling a deliriously weird selection of
heavily discounted products: a smartwatch for under \$20; selfie sticks
for around \$5; a ``self-stirring mug.'' By the end of 2015, though,
these posts were appearing with striking frequency. That year, Wish's
parent company was reported to be one of the largest advertisers on
Facebook and Instagram during the holiday season.

Finally, I relented. A friend and I signed up and agreed on an amount
each of us would spend on the other. The packages, ordered over the
course of New Year's Eve, would be surprises. Ordering on Wish, I soon
found out, is dangerously easy. (When I signed up, the first purchase
was free; I chose yellow toe socks.) Shopping involves scrolling through
an intoxicating admixture of goods: Commodity necessities appear next to
fast fashion and knockoff apparel; extraordinarily cheap but on-trend
electronics mingle with what I can only describe as global manufacturing
overspill. Among the items I sent to my friend, on our modest budget: a
laser pointer; 100-count ``super strong'' small magnets; a functioning
violin; a spare part for the window mechanism on an Audi A6; a
deep-V-neck sweater; and of course, the self-stirring mug. Shipping was
often free, or only a dollar. The items were extraordinarily well
reviewed, often by thousands of customers. The deals seemed, if not
exactly too good to be true, at least economically unfeasible --- which,
close enough.

My experience as a recipient was more informative. A sampling of my
haul, clearly selected under the influence: ``Macho Man'' Randy
Savage-style sunglasses with built-in Bluetooth speakers; a pillow cover
emblazoned with unrepeatably offensive text; a T-shirt with an obscene
graphic print; a pale silkish robe; a bag of loose Sri Lankan coins; a
fondant mold shaped like a human fetus; and a 12-hole ceramic ocarina.
Packages arrived in waves over the next month, carrying with them a hint
at what makes operations like Wish possible: their shipping labels. Most
were classified as ``ePacket,'' courtesy of the United States Postal
Service and China Post.

These shipments were made in accordance with a bilateral trade agreement
between the United States and China that originated in 2010, meant to
address the rising tide of cross-border e-commerce. Items up to 4.4
pounds --- more than the weight of, for example, a violin and bow ---
can be shipped as ePackets, at extremely low rates with tracking numbers
and delivery confirmation. Tracking is crucial for foreign sellers that
are up against consumer skepticism and comparatively slow shipping
times. The deal seems to have worked: ePacket's usage has ballooned in
recent years.

This obscure trade deal has become the quiet conduit for an explosion in
a new and underexamined American consumer behavior: buying things
directly from their countries of manufacture. (Similar agreements also
exist with Hong Kong, Singapore and South Korea.) This, obviously,
presents a problem for the stores and retailers accustomed to serving as
importers themselves. Brick-and-mortar retailers are already
experiencing a grim 2017, shedding tens of thousands of jobs a month
under pressure from e-commerce. Cross-border purchases compound the
issue: Because of ePacket, and the decades-old international postal
agreements that serve as its foundation, lightweight product shipments
from China are heavily subsidized by the U.S.P.S. ``It's providing an
artificially low rate,'' says Jim Campbell, a consultant and lawyer
specializing in international postal law. ``It's redistributing wealth,
and the winners are essentially the big exporters.'' Accordingly, these
agreements have drawn intense criticism from American retailers large
and small. In 2015, an Amazon representative testified in front of
Congress about what he called a ``completely unnecessary and illogical''
system.

It's tempting --- and not entirely unfair --- to see this surge in
American cross-border shopping as nothing more than a strange side
effect of international postal policy. But that would be too tidy. Wish
certainly illuminates the peculiarities of international shipping, but
it casts a much brighter light on the state of globalized manufacturing
and commerce. In fact, it offers a somewhat convincing vision of what
they might become in the near future.

The Walmart shopping experience is largely opaque with respect to the
chain's dependence on imports. A product's country of origin is
relegated to labels or printed on mandatory stickers, subordinate to a
curated and thoroughly American shopping experience. Wish wastes no such
effort on concealing its international character. Its product selection
feels like a churning, infinite cascade; its lack of any sort of
organizing principle is part of the reason it's so hard to stop
scrolling. More recently, I ordered a smartwatch for the cost of a combo
meal. The seller has shipped more than 13,000 units through Wish, and
the watch has been reviewed more than 2,000 times. A single day produced
ratings written in English, Spanish, Danish, Cebuano, German, French and
Romanian.

Wish shoppers are hardly less alienated from the labor that makes their
purchases possible than Walmart customers or iPhone owners are. But
their experience at least hews closer to the global economic situation
as it truly exists. Services like this offer us a preview of a
maximalist capitalist future, in which the near-entirety of current-day
retail --- stores, humans and even storelike websites --- have been
identified as gatekeepers or sources of friction and accordingly
obliterated. Consumers, for their part, seem unfazed. ``Came before it
was supposed to,'' one reviewer wrote of my still-yet-to-be-delivered
smartwatch. ``Working good.'' Expectations, too, are recalibrating.
``Disappointed in programs and functionality,'' wrote a different
reviewer of --- this warrants repeating --- \emph{an operational \$7.60
smartwatch}. ``It's O.K.,'' another wrote, ``for the price you pay.''

Advertisement

\protect\hyperlink{after-bottom}{Continue reading the main story}

\hypertarget{site-index}{%
\subsection{Site Index}\label{site-index}}

\hypertarget{site-information-navigation}{%
\subsection{Site Information
Navigation}\label{site-information-navigation}}

\begin{itemize}
\tightlist
\item
  \href{https://help.nytimes3xbfgragh.onion/hc/en-us/articles/115014792127-Copyright-notice}{©~2020~The
  New York Times Company}
\end{itemize}

\begin{itemize}
\tightlist
\item
  \href{https://www.nytco.com/}{NYTCo}
\item
  \href{https://help.nytimes3xbfgragh.onion/hc/en-us/articles/115015385887-Contact-Us}{Contact
  Us}
\item
  \href{https://www.nytco.com/careers/}{Work with us}
\item
  \href{https://nytmediakit.com/}{Advertise}
\item
  \href{http://www.tbrandstudio.com/}{T Brand Studio}
\item
  \href{https://www.nytimes3xbfgragh.onion/privacy/cookie-policy\#how-do-i-manage-trackers}{Your
  Ad Choices}
\item
  \href{https://www.nytimes3xbfgragh.onion/privacy}{Privacy}
\item
  \href{https://help.nytimes3xbfgragh.onion/hc/en-us/articles/115014893428-Terms-of-service}{Terms
  of Service}
\item
  \href{https://help.nytimes3xbfgragh.onion/hc/en-us/articles/115014893968-Terms-of-sale}{Terms
  of Sale}
\item
  \href{https://spiderbites.nytimes3xbfgragh.onion}{Site Map}
\item
  \href{https://help.nytimes3xbfgragh.onion/hc/en-us}{Help}
\item
  \href{https://www.nytimes3xbfgragh.onion/subscription?campaignId=37WXW}{Subscriptions}
\end{itemize}
