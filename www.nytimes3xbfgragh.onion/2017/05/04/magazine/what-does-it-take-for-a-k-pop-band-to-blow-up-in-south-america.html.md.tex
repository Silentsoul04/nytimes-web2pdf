Sections

SEARCH

\protect\hyperlink{site-content}{Skip to
content}\protect\hyperlink{site-index}{Skip to site index}

\href{https://myaccount.nytimes3xbfgragh.onion/auth/login?response_type=cookie\&client_id=vi}{}

\href{https://www.nytimes3xbfgragh.onion/section/todayspaper}{Today's
Paper}

What Does It Take for a K-Pop Band to Blow Up in South America?

\url{https://nyti.ms/2pJZWsI}

\begin{itemize}
\item
\item
\item
\item
\item
\item
\end{itemize}

Advertisement

\protect\hyperlink{after-top}{Continue reading the main story}

Supported by

\protect\hyperlink{after-sponsor}{Continue reading the main story}

The Money Issue

\hypertarget{what-does-it-take-for-a-k-pop-band-to-blow-up-in-south-america}{%
\section{What Does It Take for a K-Pop Band to Blow Up in South
America?}\label{what-does-it-take-for-a-k-pop-band-to-blow-up-in-south-america}}

\includegraphics{https://static01.graylady3jvrrxbe.onion/images/2017/05/07/magazine/07nichepopstar/07mag-07nichepopstar-t_CA1-articleInline.jpg?quality=75\&auto=webp\&disable=upscale}

By Jeff Benjamin

\begin{itemize}
\item
  May 4, 2017
\item
  \begin{itemize}
  \item
  \item
  \item
  \item
  \item
  \item
  \end{itemize}
\end{itemize}

\href{https://www.nytimes3xbfgragh.onion/es/2017/05/08/la-inesperada-popularidad-del-pop-coreano-en-chile/}{Leer
en español}

If you were watching Chilean TV on a Friday morning last month, you
might have encountered a surprising scene: the interruption of a
newscast in the name of \emph{pop coreano}.

The Chilevisión reporter was inside Santiago International Airport,
trying her best to get a glimpse of BTS --- an abbreviation for the
Korean term \emph{Bangtan Sonyeondan,} or ``bulletproof boy scouts'' in
English --- one of K-pop's most famous boy bands, arriving for its
sold-out March 11 and 12 concert dates at the Movistar Arena. She
couldn't get a comment from the group, but her cameraperson did get a
good shot of them as they and their huge entourage passed hordes of
screaming and banner-toting fans. She also found a young woman whose
gleaming smile appeared to be on the verge of breaking into fan-girl
tears. ``Was it worth the wait?'' the reporter asked. ``It was,'' the
girl replied with a happy crack in her voice. ``The wait, the whole
night.''

K-pop has taken most of the world by storm, but Chile represents a
recent and somewhat unlikely conquest. None of the traditional radio
stations have shown any interest in playing it. Until now, the nation's
musical imports have tended to be Latin-inflected --- sounds like
reggaeton and hip-hop --- or American-bred pop. Then there are the
eclectic homegrown sensations like Mon Laferte (known for her
experimental mix of blues and electronic rock), Camila Moreno (an
alternative-pop singer-songwriter) and Gepe (a New Age Chilean folk
singer who blends, among other influences, '60s and '70s Andean music
with electronic pop).

So BTS and its peers have been forced to sneak in via the internet. One
important entry point has been Coca-Cola FM, the soda giant's online
radio platform, which is largely unknown to American audiences but
popular in Chile, with an estimated daily listenership of 40,000. Every
Friday, the network runs a K-pop program hosted by a Chilean D.J.,
Rodrigo Gallina. Social media has played a tremendous role, too: BTS has
more than five million Twitter followers, and to date BTS has spent 22
weeks at No. 1 on Billboard's Social 50 chart, which ranks the global
online activity of artists' social-media fan pages and weekly song
plays. A dedicated BTS Chile Twitter account, run by three fans,
regularly posts Spanish translations of news articles about the band and
of the band's own posts on social media. Some Chilean fans tune in
directly to Korea's popular V app, where artists hold live-streaming
broadcasts that fans anywhere in the world can join and ask questions
(BTS's channel has more than 4.7 million followers).

The band's online popularity had become so entrenched in Chile that tour
promoters didn't even bother with a traditional media push. Fans waited
outside the arena box office to purchase tickets, which ranged from \$38
to \$212, for up to a week in advance. All 12,500 tickets for what was
supposed to be just one show sold out in a record-breaking two hours.
``The speed of those sales made us immediately begin preparations for a
second show,'' says Gonzalo Garcia, the C.E.O. and founder of NoiX
Productions, which focuses strictly on bringing Asian artists to Chile
and other Latin American countries. Eventually, NoiX ran eight days of
print ads in the daily newspaper La Tercera, but simply as a thank-you
to fans who had bought tickets.

The Backstreet Boys, One Direction and the Jonas Brothers have all
played in Chile, but Korean boy bands are a recent phenomenon, having
begun to visit the country only since 2012. Before its performances last
month, BTS did play a show in Chile on its last tour in August 2015, but
promoters booked only about half of the Movistar Arena. In the months
since, though, the group's popularity has skyrocketed in the K-pop world
--- the ``Wings'' album was South Korea's best-selling album of 2016 ---
with fans worldwide seeming to connect with the band's increasingly
polished sound and relatable-yet-hopeful messages. ``We talk about our
own turmoil and mental breakdowns as honestly as possible in the music
and {[}how{]} it grows with us as we get older,'' the band member Rap
Monster says. ``We believe Chilean fans tend to connect to those values,
maybe a little deeper than fans in other countries,'' he added.

The band's delighted handlers struggle to find adequate metrics to
capture the extent of BTS mania in Chile. There is money, of course;
Garcia says ticket sales from this year's two-day concert series
``exceeded the \$2 million mark,'' and this figure doesn't even include
revenue from the enormous amount of BTS merchandise (like a \$45 glowing
light stick or a hand-held fan with the members' faces printed on it for
\$9) sold at the arena. For BTS's South Korean record label, Big Hit
Entertainment, the most remarkable measure is online engagement, which
it seems to monitor very closely. ``We crosschecked with
social-media-channel statistics to confirm the level of loyalty and fan
base in the country,'' says Yandi Park, a concert business manager for
Big Hit. ``We did expect to have good ticket sales because the promoters
were also confident ... but did not anticipate the sellout in minutes.''
Then there are the organic expressions of fan fervor: In parks and
public sites throughout Chile, thousands of fans have begun to gather
frequently to learn K-pop dances together.

Perhaps the most impressive metric, though, is also the most alarming
one. Owners of the Movistar Arena told Garcia that the audience screams
alone during the BTS concerts this year --- that is, at moments when the
band was not even performing --- reached an earsplitting 127 decibels,
well past the noise level at which permanent hearing loss becomes a
serious concern. The promoter proudly reports it as the loudest ever
recorded at the arena. ``Audience screams alone,'' Garcia repeats, a
note of awe creeping into his voice. ``It was madness.''

Advertisement

\protect\hyperlink{after-bottom}{Continue reading the main story}

\hypertarget{site-index}{%
\subsection{Site Index}\label{site-index}}

\hypertarget{site-information-navigation}{%
\subsection{Site Information
Navigation}\label{site-information-navigation}}

\begin{itemize}
\tightlist
\item
  \href{https://help.nytimes3xbfgragh.onion/hc/en-us/articles/115014792127-Copyright-notice}{©~2020~The
  New York Times Company}
\end{itemize}

\begin{itemize}
\tightlist
\item
  \href{https://www.nytco.com/}{NYTCo}
\item
  \href{https://help.nytimes3xbfgragh.onion/hc/en-us/articles/115015385887-Contact-Us}{Contact
  Us}
\item
  \href{https://www.nytco.com/careers/}{Work with us}
\item
  \href{https://nytmediakit.com/}{Advertise}
\item
  \href{http://www.tbrandstudio.com/}{T Brand Studio}
\item
  \href{https://www.nytimes3xbfgragh.onion/privacy/cookie-policy\#how-do-i-manage-trackers}{Your
  Ad Choices}
\item
  \href{https://www.nytimes3xbfgragh.onion/privacy}{Privacy}
\item
  \href{https://help.nytimes3xbfgragh.onion/hc/en-us/articles/115014893428-Terms-of-service}{Terms
  of Service}
\item
  \href{https://help.nytimes3xbfgragh.onion/hc/en-us/articles/115014893968-Terms-of-sale}{Terms
  of Sale}
\item
  \href{https://spiderbites.nytimes3xbfgragh.onion}{Site Map}
\item
  \href{https://help.nytimes3xbfgragh.onion/hc/en-us}{Help}
\item
  \href{https://www.nytimes3xbfgragh.onion/subscription?campaignId=37WXW}{Subscriptions}
\end{itemize}
