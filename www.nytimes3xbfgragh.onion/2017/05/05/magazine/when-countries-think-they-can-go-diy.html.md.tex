Sections

SEARCH

\protect\hyperlink{site-content}{Skip to
content}\protect\hyperlink{site-index}{Skip to site index}

\href{https://myaccount.nytimes3xbfgragh.onion/auth/login?response_type=cookie\&client_id=vi}{}

\href{https://www.nytimes3xbfgragh.onion/section/todayspaper}{Today's
Paper}

When Countries Think They Can Go D.I.Y.

\url{https://nyti.ms/2pcQ8n1}

\begin{itemize}
\item
\item
\item
\item
\item
\item
\end{itemize}

Advertisement

\protect\hyperlink{after-top}{Continue reading the main story}

Supported by

\protect\hyperlink{after-sponsor}{Continue reading the main story}

The Money Issue

\hypertarget{when-countries-think-they-can-go-diy}{%
\section{When Countries Think They Can Go
D.I.Y.}\label{when-countries-think-they-can-go-diy}}

\includegraphics{https://static01.graylady3jvrrxbe.onion/images/2017/05/07/magazine/07endpaper1/07endpaper1-articleInline.jpg?quality=75\&auto=webp\&disable=upscale}

By Jaime Lowe

\begin{itemize}
\item
  May 5, 2017
\item
  \begin{itemize}
  \item
  \item
  \item
  \item
  \item
  \item
  \end{itemize}
\end{itemize}

As a business plan, going it alone may please patriots, but consumers
can be a harder sell. ``I grew up in Brazil, and until the early 1990s
we had the technology Americans had in the '60s and '70s --- like dial
phones and TVs without remote controls, since our TVs and phones were
all manufactured in Brazil,'' says Thomas Fujiwara, an economist at
Princeton University. Here are a few more places that pursued the D.I.Y.
approach to illogical conclusions.

Image

\textbf{Proton}Credit...Goh Seng Chong/Bloomberg, via Getty Images

\hypertarget{malaysia}{%
\subsection{\texorpdfstring{\textbf{Malaysia}}{Malaysia}}\label{malaysia}}

The Malaysian national car project was announced in 1979, and the
resulting automaker, Proton, began selling vehicles in 1985. They were
intended, in the words of Mahathir Mohamad, the country's prime minister
at the time, to be a ``symbol of Malaysians as a dignified people.'' The
company quickly captured 64 percent of the domestic automobile market
--- but the government was losing \$15,000 on each car. Repeated
bailouts enabled Proton to reach peak production in 1993 --- 500,000
cars --- and its greatest market share in 2001, with 53 percent. But
Proton's portion of that market sank to 14 percent by last year, and the
government is now seeking an international partner.

Image

\textbf{Minitel}Credit...SSPL/Getty Images

\hypertarget{france}{%
\subsection{\texorpdfstring{\textbf{France}}{France}}\label{france}}

In the 1980s, the French began connecting themselves to the Minitel, a
console like a black-and-white TV connected to dial-up phone lines. In
1997, Jacques Chirac, then the president of France, said: ``Today a
baker in Aubervilliers knows perfectly how to check his bank account on
the Minitel. Can the same be said of the baker in New York?'' Users
could search a national phone directory, buy clothing and train tickets,
make restaurant reservations, read newspapers, exchange messages and
participate in online sex chats more than a decade before equivalent
services were accessible through computers. At its peak, the Minitel had
an estimated 25 million users --- but the internet rendered the boxy
beige terminal obsolete, and the free service from France Télécom was
turned off in 2012.

Image

\textbf{Cockta}Credit...Tommy Alven, via Shutterstock

\hypertarget{yugoslavia}{%
\subsection{\texorpdfstring{\textbf{Yugoslavia}}{Yugoslavia}}\label{yugoslavia}}

In 1952 the director of the state-run beverage company Slovenijavino, in
the former Yugoslavia, insisted that it produce something comparable to
Coca-Cola. The following year, a chemical engineer named Emerik Zelinka
developed what would become Cockta (from ``cocktail''), a mixture of
rose hip, lemon and medicinal herbs. The resulting drink was brown,
bittersweet and uncaffeinated; one slogan was ``You'll Always Remember
Your First.'' Today the Croatian firm Atlantic Grupa, which owns Cockta,
considers its limited market to be driven by nostalgia and has put it in
a portfolio that includes four of what were once Yugoslavia's most
popular brands: Cedevita vitamin drinks, Argeta fish and meat spreads
and the peanut-flavored corn snack Smoki.

Image

\textbf{BlackBerry}Credit...Sally Felton/iStock, via Getty Images

\hypertarget{argentina}{%
\subsection{\texorpdfstring{\textbf{Argentina}}{Argentina}}\label{argentina}}

In 2010 the government of Argentina's president, Cristina Fernández de
Kirchner, imposed tariffs of up to 40 percent on goods from abroad.
Cellphone imports were banned altogether, so to fill the need, a factory
that made BlackBerrys opened in 2011 in the southern region of Tierra
del Fuego. It took nearly two years to build the Argentine BlackBerry
from scratch. By the time the device reached consumers, it cost twice as
much as newer versions for sale in the United States, and a black market
emerged for foreign-made BlackBerrys (smugglers were bringing them into
the country in their socks). The factory closed after a few years of
operation.

Advertisement

\protect\hyperlink{after-bottom}{Continue reading the main story}

\hypertarget{site-index}{%
\subsection{Site Index}\label{site-index}}

\hypertarget{site-information-navigation}{%
\subsection{Site Information
Navigation}\label{site-information-navigation}}

\begin{itemize}
\tightlist
\item
  \href{https://help.nytimes3xbfgragh.onion/hc/en-us/articles/115014792127-Copyright-notice}{©~2020~The
  New York Times Company}
\end{itemize}

\begin{itemize}
\tightlist
\item
  \href{https://www.nytco.com/}{NYTCo}
\item
  \href{https://help.nytimes3xbfgragh.onion/hc/en-us/articles/115015385887-Contact-Us}{Contact
  Us}
\item
  \href{https://www.nytco.com/careers/}{Work with us}
\item
  \href{https://nytmediakit.com/}{Advertise}
\item
  \href{http://www.tbrandstudio.com/}{T Brand Studio}
\item
  \href{https://www.nytimes3xbfgragh.onion/privacy/cookie-policy\#how-do-i-manage-trackers}{Your
  Ad Choices}
\item
  \href{https://www.nytimes3xbfgragh.onion/privacy}{Privacy}
\item
  \href{https://help.nytimes3xbfgragh.onion/hc/en-us/articles/115014893428-Terms-of-service}{Terms
  of Service}
\item
  \href{https://help.nytimes3xbfgragh.onion/hc/en-us/articles/115014893968-Terms-of-sale}{Terms
  of Sale}
\item
  \href{https://spiderbites.nytimes3xbfgragh.onion}{Site Map}
\item
  \href{https://help.nytimes3xbfgragh.onion/hc/en-us}{Help}
\item
  \href{https://www.nytimes3xbfgragh.onion/subscription?campaignId=37WXW}{Subscriptions}
\end{itemize}
