Sections

SEARCH

\protect\hyperlink{site-content}{Skip to
content}\protect\hyperlink{site-index}{Skip to site index}

\href{https://www.nytimes3xbfgragh.onion/section/arts/television}{Television}

\href{https://myaccount.nytimes3xbfgragh.onion/auth/login?response_type=cookie\&client_id=vi}{}

\href{https://www.nytimes3xbfgragh.onion/section/todayspaper}{Today's
Paper}

\href{/section/arts/television}{Television}\textbar{}Review: `I Love
Dick' Sketches an Artistic Love Triangle

\url{https://nyti.ms/2pAbhI9}

\begin{itemize}
\item
\item
\item
\item
\item
\end{itemize}

Advertisement

\protect\hyperlink{after-top}{Continue reading the main story}

Supported by

\protect\hyperlink{after-sponsor}{Continue reading the main story}

\hypertarget{review-i-love-dick-sketches-an-artistic-love-triangle}{%
\section{Review: `I Love Dick' Sketches an Artistic Love
Triangle}\label{review-i-love-dick-sketches-an-artistic-love-triangle}}

\includegraphics{https://static01.graylady3jvrrxbe.onion/images/2017/05/12/arts/12ILOVEDICK1/12ILOVEDICK1-articleInline.jpg?quality=75\&auto=webp\&disable=upscale}

By \href{https://www.nytimes3xbfgragh.onion/by/james-poniewozik}{James
Poniewozik}

\begin{itemize}
\item
  May 11, 2017
\item
  \begin{itemize}
  \item
  \item
  \item
  \item
  \item
  \end{itemize}
\end{itemize}

Watching ``I Love Dick'' is like attending an exhibition for which the
artist has supplied her own curator's notes. It's an experience as much
as a story: arresting, disorienting and provocative. It's also very
conscious of explaining to you how and why it arrests, disorients and
provokes.

The form of ``I Love Dick,'' whose eight-episode first season arrives on
Friday on Amazon, fits its subject. Adapted by Jill Soloway
(``\href{https://www.nytimes3xbfgragh.onion/watching/recommendations/watching-tv-transparent?auto=true}{Transparent}'')
and the playwright Sarah Gubbins from a cult novel by Chris Kraus, it's
art TV about artists, a love triangle as Conceptual performance.

The first corner of that triangle is Dick Jarrett (Kevin Bacon), a
famous abstract sculptor and rancher in Marfa, Tex. (The novel's Dick
was based on the media theory scholar Dick Hebdige; this one is inspired
by the Marfa artist Donald Judd.)

As a sideline, Dick runs a residency fellowship program, where he
collects artists and intellectuals like prickly cactuses. One makes
video game art; another studies the aesthetics of pornography. Enter
Sylvère (Griffin Dunne), Dick's newest fellow and a Holocaust scholar,
who arrives from Brooklyn with his wife, Chris (Kathryn Hahn), an
experimental director whose latest work has just been dropped from the
Venice Film Festival.

Over dinner with Chris and Sylvère, Dick condescendingly suggests that
she failed because women are lousy filmmakers. ``They have to work from
behind their oppression,'' he says, ``which makes for some bummer
movies.''

\includegraphics{https://static01.graylady3jvrrxbe.onion/images/2017/05/12/arts/12dick/12dick-articleInline.jpg?quality=75\&auto=webp\&disable=upscale}

Chris is infuriated, but also turned on. (Chris and Sylvère have been in
a sexual dry spell, and remember, we're talking Cowboy Kevin Bacon
here.) She channels this rage-lust into a series of ``Dear Dick''
letters, prose-poem mash notes that punctuate the series, in all-caps,
white-on-red screen graphics, like
\href{http://www.nytimes3xbfgragh.onion/slideshow/2017/02/13/t-magazine/the-pictures-generations-greatest-hits/s/pictures-generation-slide-MYOG.html}{Barbara
Kruger aphorisms}: ``I WANT TO OWN EVERYTHING THAT HAPPENS TO ME NOW.''

Chris plans to keep the letters to herself, but reads one to Sylvère. It
has an aphrodisiac effect. Dick becomes an invisible third player in
their marriage and a source of creative arousal --- until the letters
become public, and the season spins into drama and farce.

A theme here --- and in case you miss it, the characters explain it
explicitly --- is the experience of women in an art world that has
historically seen them as nudes to be painted, forms to inspire. Chris
flips this by making the great-man artist into her muse. (``It's
humiliating,'' he admits.)

Ms. Hahn, a frequent comedy actress whose dramatic talent Ms. Soloway
showcased in ``Transparent'' and the movie ``Afternoon Delight,'' shows
fantastic range as Chris: raw-nerved, hyperverbal and caught up in her
own head.

Dick, whom Mr. Bacon plays as dry as jerky, is her opposite: coolly
dismissive and laconic. His work is rugged and phallic, austere and
withholding. One of his signature pieces is a brick on a table. He
refuses to title his sculptures and says he hasn't read a book in 10
years because he's ``post-idea.''

More accurately, he's all idea, a beefcake avatar of linear thinking and
pretension. (In a hilarious art-porn fantasy sequence, Chris imagines
him shirtless, shearing a lamb in the middle of the road.)

Thematically, making Dick a symbol as much as a person works. Ms.
Soloway has said she wants the series to exemplify
\href{https://www.nytimes3xbfgragh.onion/2017/05/05/arts/television/i-love-dick-amazon-chris-kraus-and-jill-soloway.html}{the
``female gaze''} in art. But, together with the familiarity of Chris and
Sylvère's frustrated-intellectual-couple dynamic, it makes it harder to
invest in the relationship triangle. The characters' tendency to
explicate the story's themes is also distancing, albeit plausible for a
series about theory-conscious aesthetes.

I've likened ``Transparent,'' Ms. Soloway's remarkable Amazon series
about a transgender woman and her family, to
\href{https://www.nytimes3xbfgragh.onion/2015/12/09/arts/television/season-2-of-transparent-expands-its-view-of-the-pfefferman-clan.html}{``The
Wire.''} Both are fueled by social mission, which can be the death of
nuance, yet they make their messages organic rather than preachy. To
push the analogy, ``I Love Dick'' might be her
``\href{https://www.nytimes3xbfgragh.onion/watching/recommendations/watching-tv-treme}{Treme}''
(David Simon's New Orleans follow-up to ``The Wire''), exploring similar
themes --- here, feminism, identity and power --- as expressed through
culture and art.

``I Love Dick,'' like ``Transparent,'' owns its seriousness and its
characters, but it has a sense of humor about it. Early on, Sylvère is
introduced to the president of the fellowship's board, who, he's told,
is ``a big fan of the Holocaust.''

But the series is best when it does what art does: to express what can't
be said literally, to be the painting --- or the brick --- and not the
plaque next to it.

Ms. Soloway directs the fifth and best episode, a transfixingly
visualized 20-minute collection of monologues by women in the Marfa
community about art and their sexual awakenings. Roberta Colindrez is
luminous and intense as Devon, a lesbian playwright and working-class
Marfa native. If the series has future seasons, it has a strong ensemble
to build on.

If the first season doesn't entirely hang together, it's bracingly
risk-taking. At its best, it captures the artistic process in a way that
TV rarely does, and it works as a kind of video art itself. But as with
some other recent experimental series --- ``The Young Pope,'' for
instance --- it's best to realize that going in.

Toward the end of the season, Chris mentions to Dick that she's seen a
new work of his --- a line of boulders outdoors, snaking toward the
horizon. ``Did you like it?'' he asks.

She hedges: ``I just need a little bit of time to, you know, just
process it.''

``I Love Dick,'' too, feels as if it might work better in memory than in
the moment --- once you've had some time to sit with it for a while, and
tune out the noise.

Advertisement

\protect\hyperlink{after-bottom}{Continue reading the main story}

\hypertarget{site-index}{%
\subsection{Site Index}\label{site-index}}

\hypertarget{site-information-navigation}{%
\subsection{Site Information
Navigation}\label{site-information-navigation}}

\begin{itemize}
\tightlist
\item
  \href{https://help.nytimes3xbfgragh.onion/hc/en-us/articles/115014792127-Copyright-notice}{©~2020~The
  New York Times Company}
\end{itemize}

\begin{itemize}
\tightlist
\item
  \href{https://www.nytco.com/}{NYTCo}
\item
  \href{https://help.nytimes3xbfgragh.onion/hc/en-us/articles/115015385887-Contact-Us}{Contact
  Us}
\item
  \href{https://www.nytco.com/careers/}{Work with us}
\item
  \href{https://nytmediakit.com/}{Advertise}
\item
  \href{http://www.tbrandstudio.com/}{T Brand Studio}
\item
  \href{https://www.nytimes3xbfgragh.onion/privacy/cookie-policy\#how-do-i-manage-trackers}{Your
  Ad Choices}
\item
  \href{https://www.nytimes3xbfgragh.onion/privacy}{Privacy}
\item
  \href{https://help.nytimes3xbfgragh.onion/hc/en-us/articles/115014893428-Terms-of-service}{Terms
  of Service}
\item
  \href{https://help.nytimes3xbfgragh.onion/hc/en-us/articles/115014893968-Terms-of-sale}{Terms
  of Sale}
\item
  \href{https://spiderbites.nytimes3xbfgragh.onion}{Site Map}
\item
  \href{https://help.nytimes3xbfgragh.onion/hc/en-us}{Help}
\item
  \href{https://www.nytimes3xbfgragh.onion/subscription?campaignId=37WXW}{Subscriptions}
\end{itemize}
