Sections

SEARCH

\protect\hyperlink{site-content}{Skip to
content}\protect\hyperlink{site-index}{Skip to site index}

\href{https://myaccount.nytimes3xbfgragh.onion/auth/login?response_type=cookie\&client_id=vi}{}

\href{https://www.nytimes3xbfgragh.onion/section/todayspaper}{Today's
Paper}

In the Footsteps of Marcel Proust

\url{https://nyti.ms/2qiBu1T}

\begin{itemize}
\item
\item
\item
\item
\item
\end{itemize}

Advertisement

\protect\hyperlink{after-top}{Continue reading the main story}

Supported by

\protect\hyperlink{after-sponsor}{Continue reading the main story}

\hypertarget{in-the-footsteps-of-marcel-proust}{%
\section{In the Footsteps of Marcel
Proust}\label{in-the-footsteps-of-marcel-proust}}

\includegraphics{https://static01.graylady3jvrrxbe.onion/images/2017/07/05/t-magazine/proust-slide-KMAD/proust-slide-KMAD-articleLarge.jpg?quality=75\&auto=webp\&disable=upscale}

By William Friedkin

\begin{itemize}
\item
  May 15, 2017
\item
  \begin{itemize}
  \item
  \item
  \item
  \item
  \item
  \end{itemize}
\end{itemize}

I MARRIED JEANNE Moreau in 1977 at a town hall in Paris. Moreau was one
of the most revered actresses of her generation, and we were attended by
a notable group: Jacques Chirac, the city's soon-to-be mayor, spoke, and
our witnesses were the film director Alain Resnais, who had introduced
me to Jeanne, and his wife, Florence Malraux, daughter of the writer
André Malraux.

After sips of Champagne and a brief ceremony, of which I did not
comprehend one word, Jeanne and I took a long walk in the Tuileries
Garden accompanied by a cluster of paparazzi. It was my first marriage,
her second. I've seen pictures of myself on our wedding day and I appear
shell-shocked and confused. That first year we spent the summer at her
chateau in La Garde-Freinet, a medieval village, on 150 acres of
farmland in the hills behind St.-Tropez. I had no prospect of work. My
most recent film,
``\href{http://www.imdb.com/title/tt0076740/}{Sorcerer},'' which I
thought to be my best, had been rejected by critics and audiences. I
drifted into the sedentary life of the French countryside, begun with
long morning walks into the village for a coffee and croissant. The cafe
owner and the patrons were dismissive and a large graffiti on the stone
wall leading to town read: ``Parisians Go Home.'' I could only imagine
how they felt about Americans.

In the evenings, after dinner, Jeanne would read Marcel Proust's
seven-volume novel,
``\href{http://www.penguinrandomhouse.com/books/220059/the-modern-library-in-search-of-lost-time-complete-and-unabridged-6-book-bundle-by-marcel-proust-translated-by-c-k-scott-moncrieff-terence-kilmartin-and-andreas-mayor-revised-by-d-j-enright/}{À
la Recherche du Temps Perdu}'' (``In Search of Lost Time''). She would
begin reading to me in French, then translate it into English.
Gradually, I became caught up in the novel's language, its complex
structure and the intertwined lives of the many characters. After two
years, Jeanne and I realized we were culturally displaced in each
other's worlds. Our marriage ended, but not my love for Proust. I
continued to read the novel, often with difficulty, until the revelation
of its final volume. Then I would make time every day to go over parts
of it again, sometimes only certain passages, like a favored piece of
recorded music.

This went on for 10 years, in which I devoured every biography and essay
about Proust I could find and became familiar with his life, which
seemed to closely parallel his work. It was a solitary pursuit. The only
other person I knew in Hollywood who appreciated the novel was the actor
Louis Jourdan, who lived with his wife of many years in Beverly Hills in
a single-level house surrounded by books, recordings and antiques. Louis
was always cast as an archetypal French lover but his passions were
literature and music. I got to know him well in his later years. I would
visit him two or three days a week. When he died in 2015, he left me his
annotated copy of Jean-Yves Tadie's definitive biography of Proust, with
Louis's handwritten notes and observations on every page.

In the late '80s, I returned to Paris with the sole purpose of walking
in Proust's footsteps, of seeing the places in which he lived and wrote
about. For the most part I'm intimidated by masterpieces, and not
inspired to attempt one. I could never compose music having listened to
Beethoven, or play an instrument after hearing Martha Argerich or Miles
Davis. I love many works of literature, but I'm not obsessed with seeing
Macondo, where Col. Aureliano Buendía faced the firing squad, or East
Egg, where Gatsby gazed toward the green light. Reading Proust, though,
and experiencing the way he seems to capture a life in full through the
novel's huge compilation of very small moments, made me want to see the
genesis of these moments for myself.

It's impossible to experience Dickens's London today, but some of
Proust's world still exists, much as it did in his time, especially in
Illiers-Combray, which the writer visited as a child and where he set a
great deal of his novel. Encountering this place in person would be, for
me, like visiting a monument. Seeing the Lincoln Memorial doesn't deepen
your understanding of Lincoln but it does make you think about what he
represents. And while I didn't believe Proust's novel was autobiography,
I hoped that viewing his inspirations firsthand would help me locate the
source of the great novel's transformational power.

\includegraphics{https://static01.graylady3jvrrxbe.onion/images/2017/05/15/t-magazine/proust-slide-OSHT/proust-slide-OSHT-articleLarge.jpg?quality=75\&auto=webp\&disable=upscale}

I BEGAN ALL THOSE years ago at the Ritz in Paris, where I stayed in the
Marcel Proust suite, formerly a private dining room on the hotel's
second floor, where Proust often hosted small dinner parties. Proust had
been friends with the maître d'hôtel at the Ritz, who served as a
partial model for the character Aimé in the novel. The suite had a
marble bathroom, and a high window looking down to the garden; in the
room, an elegant chandelier hung from a trompe l'oeil ceiling of blue
sky with puffy clouds. It was sparsely furnished with Louis XV
reproductions. For me it felt like a kind of sacred space, and it
retained its power even though it had undergone many changes since
Proust's time. (A large television set perched on a draped table seemed
out of place.) I was told by the hotel manager that the room was
reserved for Proust to entertain whenever he could venture out from his
cork-lined bedroom at 102 Boulevard Haussmann, where he often lay
bedridden from asthma. No doubt he absorbed inspiration from
conversations here, ones that made their way into his writing. His
curiosity about the inner lives of his characters was constant and his
senses were acute to stimuli that might have gone unnoticed by others.

I then rode the Paris subway to the Saint-Lazare station, a short walk
to the Lycée Condorcet, the secondary school that Proust attended from
1882, when he was 11, until 1889. Among its hundreds of notable
graduates were the painters Pierre Bonnard and Henri de
Toulouse-Lautrec, the writer Alexandre Dumas, fils, and the photographer
Henri Cartier-Bresson. Built in the 18th century around a central
courtyard, the school was formerly a Capuchin monastery. I entered the
original two-story Mansard-roofed building through a large blue portico
flanked by two Tuscan columns. I had no appointment. I was met there by
a well-dressed middle-aged woman who introduced herself as the school's
archivist. ``Can I help you?'' she asked, cautiously.

``Did you know Marcel Proust went to this school?'' I asked quietly.

She returned my shyness with scorn. ``You must be an American,'' she
said, to which I awkwardly confessed. She appeared skeptical.

``Why are you interested in Marcel Proust?''

I told her that his work inspired me and that I wanted to find out
everything I could about him.

``Why don't you read one of the biographies? They must have them in
translation.'' I told her I had read everything I could find.

``And you're not satisfied?''

``Only more curious.''

``Are you a writer?''

``No.''

She asked who I was and I told her I was a filmmaker, but I didn't want
to make a film about Proust.

She stared at me as though wondering if I was joking. She must have
decided I wasn't, because her attitude became sympathetic. ``Would you
like to see some of the work he did while he was here?''

I didn't expect that. I don't know what I expected, but she abruptly
left the room. For a long time I watched the students playing soccer in
the courtyard or talking in groups or reading alone.

She returned, proudly displaying a stack of Xeroxed papers and handed
them to me. They were about an inch thick.

``Here are some of his writings. Most of them are in the Bibliothèque
Nationale, but here are copies of the few papers we still have.''

There were some early short stories written when he was 13, some papers
in Latin and Greek, biology and chemistry. On his final year report
card, his philosophy teacher Alphonse Darlu had written an assessment
that was translated for me as: ``He works as hard as his affliction
allows.'' I found this to be faint praise, a stunning dismissal of the
young man who would become one of the world's great novelists, whose
work would transcend the test of time. I thanked her again and shook her
hand. ``You won't find `À la Recherche' in that stack,'' she said,
smiling as I left.

Image

A park on Boulevard Haussmann near the apartment where Proust lived for
13 years, where he wrote most of ``In Search of Lost
Time.''Credit...Patrick Tourneboeuf

PROUST LIVED in the five-story apartment building on Boulevard Haussmann
for 13 years, beginning in 1906. He described the neighborhood to his
friends as ``ugly'' and noisy, with bad air. But the apartment on the
second floor had sentimental value for him. His maternal great-uncle
Louis Weil, who died 10 years before Proust moved in, had owned the
building. Proust spent precious time there with his family, and it was
here that he began to structure his memories, transforming the lives of
his family and friends and organizing the notebooks begun in 1908 as a
series of essays against the literary criticism of Charles-Augustin
Sainte-Beuve. Sainte-Beuve had put forth the idea that you couldn't
appreciate the work of an author without also knowing about his personal
life. Proust angrily disagreed and these essays, along with fragments of
his unfinished novel ``Jean Santeuil,'' evolved into ``In Search of Lost
Time.''

When I visited the former apartment building on the tree-lined
Boulevard, it had long since been remade as a branch of a large
international banking firm. The office of the director of this division
was once the salon of Proust's apartment. Now it was efficient and well
appointed, but without charm. There was a plain mirror over an unusable
fireplace, ordinary wall sconces, off-white walls.

The present occupant was distinguished, well dressed, friendly and
bemused by my pilgrimage. He was well aware of the former famous tenant
and apologetic that the room no longer reflected his taste. There was
only one work of art in the office and it was on the wall behind his
desk, a reproduction of Jacques-Émile Blanche's familiar ``Portrait of
Marcel Proust,'' the original of which is in the Musée d'Orsay. In
Proust's time, the salon was next to the cork-lined bedroom where he
wrote most of his novel in his brass bed, using his knees as a desk.
That was gone.

But it was possible to see a recreation of the bedroom, with some of its
original furniture, at the Musée Carnavalet in the Marais. The original
building was constructed in the 16th century as a private residence and
has since been turned into a museum celebrating the history of Paris.
There were more than 100 exhibition areas with objects from the 16th
century to the present: historical paintings, photographs, street signs,
furniture, a large model of the Bastille carved from its actual stone.
There were recreated period rooms belonging to famous Parisians.
Wandering through the museum was like being in a large,
bric-a-brac-filled department store. Off in a cramped corner were the
reassembled pieces of furniture from Proust's bedroom, including a
five-paneled Chinese screen, a velvet armchair that belonged to his
father and a writing desk, used mostly for piling books. He kept his
notebooks and writing materials on an old rosewood end table beside the
bed. Two other tables are adrift in this cramped tableau, one of which
was used for his morning coffee tray, usually served with milk and
croissants.

The original Boulevard Haussmann apartment was spacious but crammed with
furniture, with double windows always covered by padded blue satin
drapes. The bedspread was blue satin as well and there was a chandelier,
which was never lit when Proust was working. The only light was from a
long-stemmed, green-shaded lamp on the bedside table. Imagining the
furniture in the museum placed around a much larger room, I had the
sense of Proust's isolation, of a recluse devoted to transcribing his
memories and imagination.

\href{https://www.nytimes3xbfgragh.onion/slideshow/2017/05/15/t-magazine/around-marcel-prousts-illiers-combray.html}{}

\hypertarget{around-marcel-prousts-illiers-combray}{%
\subsection{Around Marcel Proust's
Illiers-Combray}\label{around-marcel-prousts-illiers-combray}}

8 Photos

View Slide Show ›

Patrick Tourneboeuf

IN 1971, ON THE centenary of his birth, the town of Illiers formally
added the name Proust gave it in his novel --- Combray. Since then it's
been known as Illiers-Combray. I'm not aware of any other place in the
world that has acknowledged its fictional counterpart so thoroughly. I
made the drive from Paris to Illiers-Combray in about two hours. Seen
from a distance as you approach, standing like a sentinel, is the church
of Saint-Jacques, one of the inspirations for Saint-Hilaire in the
novel.

The permanent population of the town today is about 3,400; in Proust's
time, it was around the same. Marcel and his older brother Robert were
born in Auteuil, in western Paris, but that house has long been
demolished.

It was in Illiers-Combray that I came closest to Proust's world. He
visited this ordinary little town on at least three occasions as a
child, and it left a profound impression. Though he took liberties with
its characters and locations, using it as a kind of sketchbook, the
place seemed instantly familiar. Here, along the Rue de Saint-Esprit
(now called Rue du Docteur Proust) is the former home of his aunt and
uncle, Jules and Élisabeth Amiot. This inspired Aunt Léonie's house in
the novel, and I could walk into every room. Though the house is now a
museum, it seemed to me exactly as it is in the book. The locations in
the novel are indicated throughout the town, but it has not been turned
into a tourist mecca. If anything, it's understated.

On evenings during the family's holidays in Illiers, Marcel, Robert,
their mother, Jeanne, and their father, Dr. Adrien Proust, would take
long walks after dinner along one or the other of two paths that went
off in opposite directions before circling and coming together. From the
Amiot house, they could walk along the Méréglise way (Méséglise in the
novel), a windy plain, south toward Tansonville, where Proust set
Swann's house; or they could walk in the opposite direction, along the
river Loir, on the path now known as the Guermantes way. The two roads,
or ``ways,'' are Proust's metaphor for the possibilities and diversities
of life.

Out of this beautiful but not extraordinary landscape, Proust created an
idyllic, original world. Any one of the hundreds of little towns around
Illiers, or thousands of others in rural France, could have been the
model for Combray. And it is likely that his depiction of it was equally
influenced by memories of his childhood in Auteuil. The places where he
lived were like a giant well from which he drew inspiration. As Samuel
Beckett observed, ``The whole of Proust's world comes out of a teacup.''

In writing this, I spent hours trying to describe Combray as it existed
in Proust's time and today. The differences are minor but the purpose of
my discovery has nothing to do with real estate. I've taken pictures of
the landscapes and, in Proust's words, ``the little parlour, the
dining-room ... the hall through which I would journey to the first step
of that staircase ... and, at the summit, my bedroom, with the little
passage through whose glazed door Mamma would enter ...'' I've seen the
dark space outside that door where Proust's young narrator would
anxiously await his mother's goodnight kiss, and the hawthorn blossoms
in his uncle Jules's garden which he called the Pré Catalan, and which
resembled ``a series of chapels.''

But the alchemy of his work is not found in the parks, the roads, the
flowers in bloom nor in the town's church or in the house itself. It
exists in the genius of a person who understood there was a connection
between everything --- that the roads we take inevitably lead to the
same place, a place within ourselves.

What Proust inspires in us is to see and to appreciate every seemingly
insignificant place or object or person in our lives; to realize that
life itself is a gift and all the people we've come to know have
qualities worth considering and celebrating --- in time.

Advertisement

\protect\hyperlink{after-bottom}{Continue reading the main story}

\hypertarget{site-index}{%
\subsection{Site Index}\label{site-index}}

\hypertarget{site-information-navigation}{%
\subsection{Site Information
Navigation}\label{site-information-navigation}}

\begin{itemize}
\tightlist
\item
  \href{https://help.nytimes3xbfgragh.onion/hc/en-us/articles/115014792127-Copyright-notice}{©~2020~The
  New York Times Company}
\end{itemize}

\begin{itemize}
\tightlist
\item
  \href{https://www.nytco.com/}{NYTCo}
\item
  \href{https://help.nytimes3xbfgragh.onion/hc/en-us/articles/115015385887-Contact-Us}{Contact
  Us}
\item
  \href{https://www.nytco.com/careers/}{Work with us}
\item
  \href{https://nytmediakit.com/}{Advertise}
\item
  \href{http://www.tbrandstudio.com/}{T Brand Studio}
\item
  \href{https://www.nytimes3xbfgragh.onion/privacy/cookie-policy\#how-do-i-manage-trackers}{Your
  Ad Choices}
\item
  \href{https://www.nytimes3xbfgragh.onion/privacy}{Privacy}
\item
  \href{https://help.nytimes3xbfgragh.onion/hc/en-us/articles/115014893428-Terms-of-service}{Terms
  of Service}
\item
  \href{https://help.nytimes3xbfgragh.onion/hc/en-us/articles/115014893968-Terms-of-sale}{Terms
  of Sale}
\item
  \href{https://spiderbites.nytimes3xbfgragh.onion}{Site Map}
\item
  \href{https://help.nytimes3xbfgragh.onion/hc/en-us}{Help}
\item
  \href{https://www.nytimes3xbfgragh.onion/subscription?campaignId=37WXW}{Subscriptions}
\end{itemize}
