Sections

SEARCH

\protect\hyperlink{site-content}{Skip to
content}\protect\hyperlink{site-index}{Skip to site index}

What Animals Taught Me About Being Human

\url{https://nyti.ms/2rllJ7y}

\begin{itemize}
\item
\item
\item
\item
\item
\item
\end{itemize}

\includegraphics{https://static01.graylady3jvrrxbe.onion/images/2017/05/21/magazine/21openeressay2/21openeressay2-articleLarge-v3.jpg?quality=75\&auto=webp\&disable=upscale}

The Health Issue

\hypertarget{what-animals-taught-me-about-being-human}{%
\section{What Animals Taught Me About Being
Human}\label{what-animals-taught-me-about-being-human}}

Surrounding myself with animals to feel less alone was a mistake: The
greatest comfort is in knowing their lives are not about us at all.

A family photograph of the writer with house sparrows in
1979.Credit...Alisdair Macdonald

Supported by

\protect\hyperlink{after-sponsor}{Continue reading the main story}

By Helen Macdonald

\begin{itemize}
\item
  May 16, 2017
\item
  \begin{itemize}
  \item
  \item
  \item
  \item
  \item
  \item
  \end{itemize}
\end{itemize}

Long ago, when I was 9 or 10, I wrote a school essay on what I wanted to
be when I grew up. I'll be an artist, and I'll keep a pet otter, I
announced confidently, before adding, ``as long as the otter is happy.''
When I got back my exercise book, my teacher had commented, ``But how
can you tell if an otter is happy?'' I boiled with indignation. Surely,
I thought, otters would be happy if they had a soft place to sleep,
could play, go exploring, make a friend (that would be me) and swim
around in rivers catching fish. The fish were my only concession to the
notion that an otter's needs might not match my own. It never occurred
to me that I might not understand the things an otter wanted, or
understand much of what an otter might be. I thought animals were just
like me.

I was an odd, solitary child with an early and all-consuming compulsion
to seek out wild creatures. Perhaps this was unfinished business related
to losing my twin at birth: a small girl searching for her missing half,
not knowing what she was looking for. I upended rocks for centipedes and
ants, followed butterflies between flowers, lay facedown in meadows
breathing in the perfume of roots and decay, transfixed by the sight of
tiny insects the size of punctuation marks making their laborious way up
blades of grass. I pored over field guides trying to learn the names of
all these creatures --- it seemed polite, like knowing the names of my
classmates at school. Viewed close up, the profusion of life in a few
square feet of vegetation astonished me, radically shifting my sense of
scale and widening my world beyond the modest familiarities of classroom
and home.

The creatures I met in the fields and woods around my house came to feel
like a secret family, though I spent a lot of time chasing and catching
them and not thinking much about how that made them feel. I was a child
kneeling to extract a grasshopper from the closed cage of one hand,
solemn with the necessity of gentleness, frowning as I took in the
details of its netted wings, heraldically marked thorax, abdomen as
glossy and engineered as jewelry. I wasn't just finding out what animals
looked like, but testing my capacity to navigate that perilous space
between harm and care that was partly about understanding how much power
over things I might have and partly how much power I had over myself,
knowing that I could so easily hurt them. At home I kept insects and
amphibians in a growing collection of glass aquariums and vivariums
arranged on bedroom shelves and windowsills. Later they were joined by
an orphaned crow, an injured jackdaw, a badger cub and a nest of baby
bullfinches rendered homeless by a neighbor's hedge-pruning. Looking
after this menagerie taught me a lot about animal husbandry, but in
retrospect my motives were selfish. Rescuing animals made me feel good
about myself; surrounded by them, I felt less alone.

My parents were wonderfully accepting of these eccentricities, putting
up with seeds scattered on kitchen countertops and bird droppings in the
hall with great good grace. But things weren't so easy at school. To use
a term from developmental psychology, social cognition wasn't my forte.
One morning I wandered off the court in the middle of a netball match to
identify some nearby birdcalls and was bewildered by the rage this
induced in my team. Things like this kept happening. I wasn't good at
teams. Or rules. Or any of the in-jokes and complicated allegiances of
my peer group. Unsurprisingly, I was bullied. To salve this growing,
biting sense of difference from my peers, I began to use animals to make
myself disappear. By concentrating hard enough on insects, or by holding
binoculars up to my eyes to bring wild birds close, I found that I could
make myself go away. This method of finding refuge from difficulty was
an abiding feature of my childhood.

By the time I was in my 30s, I thought I'd grown out of this habit. I
had been a falconer for many years, which was a surprising education in
emotional intelligence. It taught me to think clearly about the
consequences of my actions, to understand the importance of positive
reinforcement and gentleness in negotiating trust. To know exactly when
the hawk had had enough, when it would rather be alone. And most of all,
to understand that the other party in a relationship might see a
situation differently or disagree with me for its own good reasons.
These were lessons about respect, agency and other minds that, I am
embarrassed to confess, I was rather late in applying to people. I
learned them first from birds.

But when I was 37, my father died, and all these lessons were suddenly
forgotten. I wanted to be something as fierce and inhuman as a goshawk.
So I lived with one. Watching her soar and hunt over hillsides near my
home, I identified so closely with the qualities I saw in her that I
forgot my grief. But I also forgot how to be a person, and fell into a
deep depression. A hawk turned out to be a terrible model for living a
human life. Once again, I had tried to escape emotional difficulty by
filling my mind with a living creature. It was a failure, a mistake that
revealed in retrospect the deepest lesson that animals have taught me:
how easily and unconsciously we see other lives as mirrors of our own.

\textbf{Animals don't exist} in order to teach us things, but that is
what they have always done, and most of what they teach us is what we
think we know about ourselves. The purpose of animals in medieval
bestiaries, for example, was to give us lessons in how to live. I don't
know anyone who now thinks of pelicans as models of Christian
self-sacrifice, or the imagined couplings of vipers and lampreys as an
allegorical exhortation for wives to put up with unpleasant husbands.
But our minds still work like bestiaries. We thrill at the notion that
we could be as wild as a hawk or a weasel, possessing the inner ferocity
to go after the things we want; we laugh at animal videos that make us
yearn to experience life as joyfully as a bounding lamb. A photograph of
the last passenger pigeon makes palpable the grief and fear of our own
unimaginable extinction. We use animals as ideas to amplify and enlarge
aspects of ourselves, turning them into simple, safe harbors for things
we feel and often cannot express.

None of us see animals clearly. They're too full of the stories we've
given them. Encountering them is an encounter with everything you've
ever learned about them from previous sightings, from books, images,
conversations. Even rigorous scientific studies have asked questions of
animals in ways that reflect our human concerns. In the late 1930s, for
example, when the Dutch and Austrian ethologists Niko Tinbergen and
Konrad Lorenz towed models resembling flying hawks above turkey chicks,
they were trying to prove that these birds hatched with a hard-wired
image resembling an airborne bird of prey already in their minds that
compelled them to freeze in terror. While later research has suggested
it is very likely that young turkeys actually learn what to fear from
other turkeys, the earlier experiment is still valuable, not least for
what it says about human fears. To me it seems shaped by the historical
anxieties of a Europe threatened for the first time by large-scale
aerial warfare, when pronouncements were made that ``the bomber will
always get through,'' no matter how tight the national defense.

Simply knowing that fragment of history, and knowing that domesticated
turkey chicks freeze when hawks fly overhead, make them more complicated
creatures than farmyard poultry or oven-ready carcasses. For the more
time spent researching, watching and interacting with animals, the more
the stories they're made of change, turning into richer stories that can
alter not only what you think of the animal but also who you are. It has
broadened my notion of home to think of what that concept might mean to
a nurse shark or a migratory barn swallow; altered my notion of family
after I learned of the breeding systems of acorn woodpeckers, in which
several males and females together raise a nest of young. No one I know
thinks that humans should spawn like wave-borne grunion or subsist
entirely on flies. But the various lives of creatures have led me to
feel there might not be only one right way to express care, to feel
allegiance, a love for place, a way of moving through the world.

You cannot know what it is like to be a bat by screwing your eyes tight,
imagining membranous wings, finding your way through darkness by talking
to it in tones that reply to you with the shape of the world. As the
philosopher Thomas Nagel explained, the only way to know what it is like
to be a bat is to be a bat. But the imagining? The attempt? That is a
good and important thing. It forces you to think about what you don't
know about the creature: what it eats, where it lives, how it
communicates with others. The effort generates questions not just about
how being a bat is different but about how different the world might be
for a bat. For what an animal needs or values in a place is not always
what we need, value or even notice. Muntjac deer have eaten the
undergrowth where nightingales once nested in the forests near my home,
and now those birds have gone. What to my human eye is a place of
natural beauty is, for a nightingale, something like a desert. Perhaps
this is why I am impatient with the argument that we should value
natural places for their therapeutic benefits. It's true that time
walking in a forest can be beneficial to our mental health. But valuing
a forest for that purpose traduces what forests are. They are not there
for us alone.

For some weeks, I've been worried about the health of family and
friends. Today I've stared at a computer screen for hours. My eyes hurt.
My heart does, too. Feeling the need for air, I sit on the step of my
open back door and see a rook, a sociable species of European crow,
flying low toward my house through gray evening air. Straightaway I use
the trick I learned as a child, and all my difficult emotions lessen as
I imagine how the press of cooling air might feel against its wings. But
my deepest relief doesn't come from imagining I can feel what the rook
feels, know what the rook knows --- instead, it's slow delight in
recognizing that I cannot. These days I take emotional solace from
understanding that animals are not like me, that their lives are not
about us at all. The house it's flying over has meaning for both of us.
To me, it is home. To a rook? A way point on a journey, a collection of
tiles and slopes, useful as a perch or a thing to drop walnuts on in
autumn to make them shatter and let it winkle out the flesh inside.

Then there is something else. As it passes overhead, the rook tilts its
head to regard me briefly before flying on. And with that glance I feel
a prickling in my skin that runs down my spine, and my sense of place
shifts. The rook and I have shared no purpose. For one brief moment we
noticed each other, is all. When I looked at the rook and the rook
looked at me, I became a feature of its landscape as much as it became a
feature of mine. Our separate lives, for that moment, coincided, and all
my anxiety vanished in that one fugitive moment, when a bird in the sky
on its way somewhere else pulled me back into the world by sending a
glance across the divide.

Advertisement

\protect\hyperlink{after-bottom}{Continue reading the main story}

\hypertarget{site-index}{%
\subsection{Site Index}\label{site-index}}

\hypertarget{site-information-navigation}{%
\subsection{Site Information
Navigation}\label{site-information-navigation}}

\begin{itemize}
\tightlist
\item
  \href{https://help.nytimes3xbfgragh.onion/hc/en-us/articles/115014792127-Copyright-notice}{©~2020~The
  New York Times Company}
\end{itemize}

\begin{itemize}
\tightlist
\item
  \href{https://www.nytco.com/}{NYTCo}
\item
  \href{https://help.nytimes3xbfgragh.onion/hc/en-us/articles/115015385887-Contact-Us}{Contact
  Us}
\item
  \href{https://www.nytco.com/careers/}{Work with us}
\item
  \href{https://nytmediakit.com/}{Advertise}
\item
  \href{http://www.tbrandstudio.com/}{T Brand Studio}
\item
  \href{https://www.nytimes3xbfgragh.onion/privacy/cookie-policy\#how-do-i-manage-trackers}{Your
  Ad Choices}
\item
  \href{https://www.nytimes3xbfgragh.onion/privacy}{Privacy}
\item
  \href{https://help.nytimes3xbfgragh.onion/hc/en-us/articles/115014893428-Terms-of-service}{Terms
  of Service}
\item
  \href{https://help.nytimes3xbfgragh.onion/hc/en-us/articles/115014893968-Terms-of-sale}{Terms
  of Sale}
\item
  \href{https://spiderbites.nytimes3xbfgragh.onion}{Site Map}
\item
  \href{https://help.nytimes3xbfgragh.onion/hc/en-us}{Help}
\item
  \href{https://www.nytimes3xbfgragh.onion/subscription?campaignId=37WXW}{Subscriptions}
\end{itemize}
