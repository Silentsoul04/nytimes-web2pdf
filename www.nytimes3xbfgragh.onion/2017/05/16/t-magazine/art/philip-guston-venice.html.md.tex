Sections

SEARCH

\protect\hyperlink{site-content}{Skip to
content}\protect\hyperlink{site-index}{Skip to site index}

\href{https://myaccount.nytimes3xbfgragh.onion/auth/login?response_type=cookie\&client_id=vi}{}

\href{https://www.nytimes3xbfgragh.onion/section/todayspaper}{Today's
Paper}

An Iconic American Painter, Re-examined Through Poetry

\url{https://nyti.ms/2rn6dbc}

\begin{itemize}
\item
\item
\item
\item
\item
\end{itemize}

Advertisement

\protect\hyperlink{after-top}{Continue reading the main story}

Supported by

\protect\hyperlink{after-sponsor}{Continue reading the main story}

\hypertarget{an-iconic-american-painter-re-examined-through-poetry}{%
\section{An Iconic American Painter, Re-examined Through
Poetry}\label{an-iconic-american-painter-re-examined-through-poetry}}

\href{https://www.nytimes3xbfgragh.onion/slideshow/2017/05/16/t-magazine/philip-guston-in-venice.html}{}

\hypertarget{philip-guston-in-venice}{%
\subsection{Philip Guston in Venice}\label{philip-guston-in-venice}}

9 Photos

View Slide Show ›

\includegraphics{https://static01.graylady3jvrrxbe.onion/images/2017/05/16/t-magazine/16tmag-guston-slide-1YV1/16tmag-guston-slide-1YV1-articleLarge.jpg?quality=75\&auto=webp\&disable=upscale}

Lorenzo Palmieri/© The Estate of Philip Guston/Courtesy of the Estate,
Gallerie dell'Accademia and Hauser \& Wirth

By Kat Herriman

\begin{itemize}
\item
  May 16, 2017
\item
  \begin{itemize}
  \item
  \item
  \item
  \item
  \item
  \end{itemize}
\end{itemize}

In the last years of Philip Guston's life, the painter found comfort in
the melancholia of T.S. Eliot, especially his ``Four Quartets.'' At
least that is how his daughter, Musa Mayer, remembers it. ``I think he
connected to the way Eliot was able to find meaning in the darkness,''
Mayer says. ``Of course, throughout his life, he always felt a certain
allegiance to poetry. I think it's because of the metaphysical aspect of
poetry, the way in which symbols and imagery are used to suggest
something larger, something deeper.''

In a new exhibition at the Gallerie dell'Accademia di Venezia in Italy,
the curator Kosme de Barañano connects the dots between poetry and the
artist's work with a show titled ``Philip Guston and the Poets.'' The
survey hones in on five literary giants: D.H. Lawrence, W.B. Yeats,
Wallace Stevens, Eugenio Montale and Eliot, and uses their words to
explore the psychology behind Guston's oeuvre.

As the guardian of Guston's estate, Mayer was struck by Barañano's
proposal to create a survey of her father's work that abandoned
traditional structures. ``When I read Barañano's essay, I thought it
provided a unique opportunity to look at {[}Guston's{]} work through new
eyes.''

``His large shows have tended to be organized either chronologically or
stylistically,'' Mayer says. ``This is neither. Abstract paintings
aren't segregated from the figurative ones, they are mixed together to
reveal something new.''

Throughout the show, drawings and paintings are contextualized with
stanzas --- some are more direct than others. A painting from 1979
called ``East Coker-Tse'' sits alongside lines from the poem that
inspired it, the third installment of Eliot's ``Four Quartets*.''* ``It
is a death mask of Eliot, but also a kind of self-portrait,'' Mayer
explains. ``He painted it after a heart attack that nearly killed him.''
As if in response, the last line ``East Coker'' reads: ``In my end is my
beginning.''

The setting adds yet another layer to the story. Mayer first traveled to
Venice with her parents after graduating from high school in 1960 when
Guston was participating in the biennale. ``The Gallerie dell'Accademia
was one of his favorite places to see art in the world,'' Mayer says.
``He would be so thrilled and honored to have his show here. He was
deeply touched by Italian painters and the language they used.'' More
than 50 years later, poetry, painting and Venice come together to form a
window into the mind of a man obsessed with the weight of signs and
symbols.

Advertisement

\protect\hyperlink{after-bottom}{Continue reading the main story}

\hypertarget{site-index}{%
\subsection{Site Index}\label{site-index}}

\hypertarget{site-information-navigation}{%
\subsection{Site Information
Navigation}\label{site-information-navigation}}

\begin{itemize}
\tightlist
\item
  \href{https://help.nytimes3xbfgragh.onion/hc/en-us/articles/115014792127-Copyright-notice}{©~2020~The
  New York Times Company}
\end{itemize}

\begin{itemize}
\tightlist
\item
  \href{https://www.nytco.com/}{NYTCo}
\item
  \href{https://help.nytimes3xbfgragh.onion/hc/en-us/articles/115015385887-Contact-Us}{Contact
  Us}
\item
  \href{https://www.nytco.com/careers/}{Work with us}
\item
  \href{https://nytmediakit.com/}{Advertise}
\item
  \href{http://www.tbrandstudio.com/}{T Brand Studio}
\item
  \href{https://www.nytimes3xbfgragh.onion/privacy/cookie-policy\#how-do-i-manage-trackers}{Your
  Ad Choices}
\item
  \href{https://www.nytimes3xbfgragh.onion/privacy}{Privacy}
\item
  \href{https://help.nytimes3xbfgragh.onion/hc/en-us/articles/115014893428-Terms-of-service}{Terms
  of Service}
\item
  \href{https://help.nytimes3xbfgragh.onion/hc/en-us/articles/115014893968-Terms-of-sale}{Terms
  of Sale}
\item
  \href{https://spiderbites.nytimes3xbfgragh.onion}{Site Map}
\item
  \href{https://help.nytimes3xbfgragh.onion/hc/en-us}{Help}
\item
  \href{https://www.nytimes3xbfgragh.onion/subscription?campaignId=37WXW}{Subscriptions}
\end{itemize}
