Sections

SEARCH

\protect\hyperlink{site-content}{Skip to
content}\protect\hyperlink{site-index}{Skip to site index}

\href{https://myaccount.nytimes3xbfgragh.onion/auth/login?response_type=cookie\&client_id=vi}{}

\href{https://www.nytimes3xbfgragh.onion/section/todayspaper}{Today's
Paper}

Aleppo After the Fall

\url{https://nyti.ms/2rSJ1SG}

\begin{itemize}
\item
\item
\item
\item
\item
\item
\end{itemize}

Advertisement

\protect\hyperlink{after-top}{Continue reading the main story}

Supported by

\protect\hyperlink{after-sponsor}{Continue reading the main story}

Feature

\hypertarget{aleppo-after-the-fall}{%
\section{Aleppo After the Fall}\label{aleppo-after-the-fall}}

\includegraphics{https://static01.graylady3jvrrxbe.onion/images/2017/05/28/magazine/28syria12/28syria12-articleLarge-v5.jpg?quality=75\&auto=webp\&disable=upscale}

By \href{http://www.nytimes3xbfgragh.onion/by/robert-f-worth}{Robert F.
Worth}

\begin{itemize}
\item
  May 24, 2017
\item
  \begin{itemize}
  \item
  \item
  \item
  \item
  \item
  \item
  \end{itemize}
\end{itemize}

\textbf{O}ne morning in mid-December, a group of soldiers banged on the
door of a house in eastern Aleppo. A male voice responded from inside:
``Who are you?'' A soldier answered: ``We're the Syrian Arab Army. It's
O.K., you can come out. They're all gone.''

The door opened. A middle-aged man appeared. He had a gaunt,
distinguished face, but his clothes were threadbare and his teeth looked
brown and rotted. At the soldiers' encouragement, he stepped hesitantly
forward into the street. He explained to them, a little apologetically,
that he had not crossed his threshold in four and a half years.

The man gazed around for a moment as if baffled, his eyes filling with
tears. The regime of President Bashar al-Assad had just recaptured the
city after years of bombing and urban warfare that had made Aleppo a
global byword for savagery. This frail-looking man had survived at the
war's geographic center entirely alone, an urban Robinson Crusoe, living
on stocks of dry food and whatever he could grow in his small inner
courtyard. Now, as he stumbled through an alley full of twisted metal
and rubble, he saw for the first time that the front lines, marked by a
wall of sandbags, were barely 20 yards from his house.

Three months later, in March, he sat with me under the tall, spindly
orange tree in his courtyard and described how he barricaded himself in
when the fighting started. He goes by the name Abu Sami, and he has the
mild, patient manners of a scholar; he taught at Aleppo University
before the war. In the early days of the rebel takeover, he said, his
nephews used to drop by with fresh bread and meat. But starting in 2013,
the shelling grew worse, and he would go six months or more without
seeing another human face. There was no water, no electric light; he
gathered rainwater in buckets and boiled it and used a small solar panel
to charge his phone. He made vinegar from grapes he grew in the
courtyard. He treated his illnesses with aloe and other herbs he grew in
pots. Once, when a rotten tooth became too painful, he yanked it out
with pliers. He cowered by his bed when bombs shook the house to its
foundation.

\includegraphics{https://static01.graylady3jvrrxbe.onion/images/2017/05/28/magazine/28syria2/28mag-28syria-t_CA5-articleInline.jpg?quality=75\&auto=webp\&disable=upscale}

Image

Inside Abu Sami's home.Credit...Sebastián Liste/Noor Images, for The New
York Times

Most of all, he sustained himself by reading. He carried out a stack of
books from his bedroom to show me: treatises by Sigmund Freud, novels by
Henry Miller, histories of science and psychology and religion and
mythology and cooking, a book on radical theater by the American drama
critic Robert Brustein. Some were in Russian, a language he learned as a
young man. ``I read these things so I wouldn't have to think about
politics or current events,'' he said. He read plays --- Shakespeare and
Molière were favorites --- and in his solitude, he found that he was
able to see the entire drama acted out in his mind, as if it were
onstage.

He led me upstairs to see his dusty study, where the walls and ceiling
were shredded with dozens of small shrapnel holes that let in fingers of
sunlight. He picked up a bomb fragment, rolled it in his palm and
laughed. He pointed to the house next door, where his neighbor, a quiet
man who kept pigeons on the roof, had lived until a group of rebels
arrived, shouting, and dragged him from the door. They brought back his
corpse half an hour later.

I asked Abu Sami why he never left. He gave the same answer that so many
others gave me: because it was his house. And because day after day,
year after year, he kept thinking, Surely this war is about to end.

\textbf{In the eastern} Aleppo streets beyond Abu Sami's house, little
has changed since the December morning when he rediscovered his ruined
city. A narrow alley leads to an open area where a ``hell cannon'' still
sits, the homemade howitzer used by rebels to fire on
government-controlled western Aleppo. Beyond it, there are buildings
with pancaked roofs, evidence of Russian and Syrian bombs. There are
piles of rubble so high that entire streets remain impassable.
Throughout the former rebel zone that once proudly called itself ``free
Aleppo,'' there are hospitals and schools and houses --- it goes on for
miles --- that have been reduced to uneven heaps of stone and broken
concrete, where the faint smell of buried corpses still lingers.

In the United States, the drawn-out siege of Aleppo --- where the Syrian
regime and its Russian allies repeatedly bombed hospitals and civilian
areas --- was widely deplored as a war crime comparable to the worst
massacres of the Bosnian war during the 1990s. The refusal to intervene,
some said, was a defining moral failure of the Obama administration. On
the other side, regime supporters saw only the rebels' atrocities and
their manipulation of civilians for propaganda. The ``fall'' of Aleppo,
they said, was really the ``liberation'' of a city from terrorist rule,
and a sign that Assad had all but won the civil war.

Both portraits are false and self-serving. The Syrian tragedy started in
a moment of deceptive simplicity, when the peaceful protesters of the
2011 Arab Spring seemed destined to inherit the future. Chants for
freedom turned quickly to insurrection, bullets and war. But it took
some time for outsiders to recognize how different Syria was, how its
internal schisms --- like tightly coiled springs --- would provoke the
fears and ambitions of all its neighbors. The Saudis and Turks wanted to
replace Assad with a reliable Sunni client, while Iran and Hezbollah
held fast to their one foothold in the Arab world. Russia, which
intervened decisively in 2015, had its own motives: flouting American
designs and protecting a reliable autocrat. The United States, having
expected Assad to fall on his own, dithered over support for the rebels.

Aleppo became the rebels' last major urban redoubt. Its fall
reconfigured the Syrian battleground: The Saudis and Turks resigned
themselves to Assad's rule, and their rivals exulted in a victory that
seemed to justify years of blood and treasure. The Assad regime happily
stepped out of the limelight as the world's attention turned back to
capturing Raqqa, the ISIS capital in Syria's northeast.

Image

Ruins near the citadel in Aleppo's Old City.Credit...Sebastián
Liste/Noor Images, for The New York Times

One small measure of the regime's confidence was a renewed willingness
to let Western journalists --- including me --- travel the country. I
was under the usual police-state surveillance, with a minder from the
Information Ministry accompanying me during my travels outside Damascus.
But scarcely any Americans had been to Aleppo since the regime's victory
in December. The city had become a symbol of sorts, a sprawling
commercial hub where every faction seemed to have left its mark. I had
been trying to get back there for years; it was a place I loved when I
covered the region from Beirut. I wanted to wind back the clock and make
sense of how a city that seemed so averse to politics --- of any kind
--- had been torn apart.

Even Syrians have trouble answering that question. In March, I met a
lawyer named Anas Joudeh, who took part in some of the 2011 protests.
Joudeh no longer considers himself a member of the opposition. I asked
him why. ``No one is 100 percent with the regime, but mostly these
people are unified by their resistance to the opposition,'' Joudeh told
me. ``They know what they don't want, not what they want.'' In December,
he said, ``Syrians abroad who believe in the revolution would call me
and say, `We lost Aleppo.' And I would say, `What do you mean?' It was
only a Turkish card guarded by jihadis.'' For these exiled Syrians, he
said, the specter of Assad's crimes looms so large that they cannot see
anything else. They refuse to acknowledge the realities of a rebellion
that is corrupt, brutal and compromised by foreign sponsors. This is
true. Eastern Aleppo may not have been Raqqa, where ISIS advertised its
rigid Islamist dystopia and its mass beheadings. But as a symbol of
Syria's future, it was almost as bad: a chaotic wasteland full of
feuding militias --- some of them radical Islamists --- who hoarded food
and weapons while the people starved.

As for the regime's victory there, it probably would not have taken
place if Turkey had not withdrawn some of its rebel proxies to focus on
fighting the Kurds. Aleppo may have helped the regime's morale, but the
war is likely to grind on for years, sustained and manipulated by
outside powers. Assad needs them: His army has been decimated by war and
desertions. That may help explain his use of chemical weapons in the
town of Khan Sheikhoun in early April, which prompted the Trump
administration's slap-on-the-wrist missile strike. With his manpower
running out, Assad cares more about reinforcing his rule --- at any cost
--- than rehabilitating his reputation in the West, which might have
provided loans to help rebuild his shattered country.

All the same, Aleppo was a turning point, and in some ways an emblem of
the wider war. Its fall appears to have persuaded many ordinary Syrians
that the regime, for all its appalling cruelty and corruption, is their
best shot at something close to normality. This is almost certainly true
for the Trump administration too. President Trump may call Assad an
``animal'' and hint at more airstrikes, but he cannot unseat him,
because he knows that the alternative is not the kinder, gentler place
once dreamed of by opposition activists. It is anarchy, where the
warlords rule not from the presidential palace but from every town and
every street.

\textbf{F}irst they stole everything, then they burned everything,''
Freddy Marrache told me as I stumbled along in the darkness behind him.
Above us were vaulted medieval stone roofs, interrupted here and there
by huge shell holes. ``This was the spice market --- it's totally gone.
The front line was just here.'' Underfoot was a slurry of ash and
garbage. Marrache, a 48-year-old businessman with a pale, shaved head
and an air of quiet alertness, made frequent visits as a child to
Aleppo's Old City. The \emph{souqs} were its crown jewel, a cloacal maze
of market stalls packed with spices, fabrics, silks, leather, soaps,
gold, meat, fruit, carpets, toilet seats --- almost anything. It went on
for more than eight miles, one of the largest covered markets in the
world. Aleppans used to say that a blind man could find his way through
them by following the smells of the merchandise. Then, in 2012, the
rebels came, and the \emph{souqs} became the perfect refuge for urban
guerrillas.

Marrache and his sister, Marie-Michelle, had offered to show me the
remains of the Khan al Nahassine, a grand old house attached to the
\emph{souqs} that was built in 1539 and has been owned by their family
since the 1800s. I first saw it a decade earlier, when their mother,
Jenny Poche Marrache, walked me through it. She was then in her 60s, and
she had a sour elegance that seemed to match the place. I remember her
complaining about Islamists as she chain-smoked and concluding acidly in
French, ``Soon I'll be dead, and it won't matter.'' She waited patiently
as I stared at the parade of antiquities on every wall: paintings,
sculptures, documents, old musical instruments, photographs. That house
was a shrine to the old Levantine world of which Aleppo --- with its
polyglot traders, its mix of Europe and Asia, Christianity and Islam ---
had been the center. Jenny died in 2015, before she could see the extent
of the damage to the house.

Everything was gone. Even the copper wiring had been stripped out. The
walls were still there, but on the far side some rooms had collapsed
into the courtyard after heavy shelling. ``You remember the painting
that was here?'' Freddy asked. I did. It was of a woman in Renaissance
dress, the wife of the first Venetian consul in Aleppo. (Aleppo still
has honorary consulships for many European countries.) The painting was
memorable because it was painted in the same room where it hung; you
could recognize the other objects and the shape of the wall. Freddy
explained that the painting had turned up in Turkey, like much of what
the rebels had stolen across Aleppo. An Istanbul antiques dealer told
them that he would sell it back to them for \$20,000. When they
protested that it was stolen, Freddy told me, the dealer said
dismissively, ``I get things from Syria every day.''

Image

A wall where pictures used to hang in the Khan al
Nahassine.Credit...Sebastián Liste/Noor Images, for The New York Times

Back in the \emph{souqs}, I kept trying to superimpose my memories of
the place. We passed near the silk merchants' area, now blackened and
silent. Before 2011, I used to stop there and visit a flamboyant young
trader with a round, cherubic face. He would give me tea and drape me
with scarves. His little stall was covered with pictures of gay icons
like Judy Garland, a reference that his Syrian partners seemed not to
get (or perhaps they just didn't care). I still have his business card,
with a picture of Oscar Wilde and the quote: ``I can resist everything
except temptation.'' Aleppo in those days was a magnet for footloose
journalists and adventure tourists. We would spend hours getting lost in
the \emph{souqs} and then stop for drinks in the dimly lit bar at the
Hotel Baron, gazing at its old unpaid bar tab left by T.E. Lawrence, our
heads swimming with nostalgia for an era we knew only from books.

Now parts of the city were literally unrecognizable. In al-Hatab Square,
once one of the prettier spots in the Old City, I found only a giant,
uneven mound of rubble and earth that rose 15 feet above the street,
with grass growing in it. I almost stepped on an unexploded Turkish gas
bomb surrounded by yellow spring flowers. On the square's edges, half
the buildings were destroyed. It was hard to believe this was once an
orderly urban setting, lined with restaurants and hotels. The last time
I was in Aleppo, in late 2010, I stayed at a beautiful old boutique
hotel near the square, the Beit Wakil. I remember the owner taking me
down into a dark, earthen-walled subbasement to show me a network of
tunnels built centuries earlier. You could travel all the way to the
citadel --- the great medieval palace that towers over the Old City ---
without going aboveground, he said. They were built during the 17th
century, when intermittent wars often made streets too treacherous to
walk. ``Perhaps we will need them again,'' he said.

**W**hat destroyed Aleppo? It was not the sectarianism that is often
held up as a key to the Syrian war. It was not just ``terrorism,'' the
word used by regime apologists to fend off any share of blame. Those
things played a role, but the core of the conflict in Aleppo, as in much
of Syria, was a divide between urban wealth and rural poverty. It is not
new. Travelers in the Ottoman era used to describe the shocking gulf
between Aleppo's opulence and the countryside surrounding it, where
peasants lived in almost Stone Age conditions. Later, this divide mapped
onto the city itself, as eastern Aleppo spread and filled with poor
migrants. Deeply religious and mostly illiterate, smoldering with class
resentment, they became the foot soldiers of a violent insurgency led by
the Muslim Brotherhood in the 1970s. That rebellion burned for years and
culminated in the Syrian regime's notorious massacre of 10,000 to 30,000
people in the city of Hama in 1982. Hundreds of people were killed in
Aleppo, too, and a siege atmosphere marked the entire city. The Syrian
novelist Khaled Khalifa, who grew up in Aleppo during those years and
wrote a novel about it, told me in 2008 that the city's cosmopolitan
traditions had helped protect it. But he added: ``All this has harmed
Syrian society so much. If what happened in the 1980s were to happen
again, I think the Islamists would win.''

One tragedy of Aleppo is that this rift between rich and poor was slowly
mending in the years just before the 2011 uprisings. An economic
renaissance was underway, fueled by thousands of small factories on the
city's outskirts. The workers were mostly from eastern Aleppo, and the
owners from the west. A trade deal with Turkey, whose border is just 30
miles to the north, brought new business and tourists and optimism. I
remember sitting at cafe table with two Turkish traders just outside the
citadel in late 2009. Tourists thronged all around us, and the two men
talked excitedly about how new joint ventures were melting the animosity
between their country and Syria. ``Erdogan and Assad, they are like real
friends,'' one of them said, referring to President Recep Tayyip Erdogan
of Turkey.

This kind of optimism was one reason the revolution took so long to
reach Aleppo. All through 2011, as the rest of Syria erupted in protest,
its largest city was quiet. But by 2012, in the villages just beyond the
city's edges, weaponry was flowing in from across the Turkish border and
battalions were being formed. ``The countryside was boiling,'' I was
told by Adnan Hadad, an opposition activist who was there at the time
and belonged to the Revolutionary Military Council in Aleppo, a group
led by Syrian military officers who defected. The council was eager for
more European and American recognition and sensitive to Western calls
for the preservation of most of Syria's state institutions. But local
rural people tended to side with a more Islamist and less patient group
called Liwa al-Tawheed. Tawheed's members ``considered themselves more
authentic'' and had begun getting their own funding from Persian Gulf
donors, Hadad told me. In the spring of 2012, Tawheed's members began
pushing for a military takeover of Aleppo, accusing the council of
excessive caution and even secret deals with the regime. The council
resisted, saying they should move only when it was clear that the city's
people wanted them to. In July, Tawheed took matters into its own hands.
Armed insurgents flooded eastern and southwestern parts of the city,
taking over civilian houses as well as police stations in the name of
the revolution. Hadad considered the move a ``fatal mistake,'' he told
me, and resigned from the military council.

By then, eastern Aleppo had become a rebel stronghold. In early 2013,
elections for provincial councils took place, giving the rebels a
civilian veneer. But the councils, initially funded by the Syrian branch
of the Muslim Brotherhood, were soon under pressure from the Nusra
Front, the Syrian Qaeda affiliate, and other hard-line groups. Later,
ISIS forces captured parts of the city and forced residents to live by
their rigid code. In theory, Aleppo was an embattled showplace for the
Syrian revolution's aspirations. In fact, most civilians were dependent
on a patchwork of armed rebel factions for food and protection. The
constant pressure of war left almost no room for a real economy, and
many of the city's factories had been repurposed by the rebels as
military bases.

Now Aleppo's great economic engine lies in ruins. One afternoon, a
45-year-old factory owner named Ghassan Nasi took me to the industrial
area just west of Aleppo called Layramoon. The sounds of the city
dissipated as we drove west, and when the car stopped, there was an
eerie silence. An entire district that once hummed with 1,000 small
factories was now abandoned, most of its buildings shattered and burned.
``It is a 100 percent loss here,'' Nasi said. We walked down a dusty
street to his factory, a textile and dyeing house that employed 130
people who worked 24 hours a day in three shifts. The door still had its
metal filigree gate and marble steps. ``This is where workers stamped in
and out,'' he said.

Image

The roof of the Aleppo Eye Hospital, which rebels used as a military
headquarters.Credit...Sebastián Liste/Noor Images, for The New York
Times

Image

Inside the hospital.Credit...Sebastián Liste/Noor Images, for The New
York Times

Inside, the huge factory floor was burned black and strewn with rubble.
The rebels had used it to make weapons, he said. His old office had been
used to house prisoners. Nasi told me quietly that he collapsed to his
knees upon seeing it again last summer. ``I lost \$10 million in
machinery, \$4 million in land,'' he said. ``Even if we rebuild, the
machinery is gone, and with the sanctions, we cannot buy new
machinery.'' On top of that, there is inflation: The American dollar was
worth 47 Syrian pounds before the crisis, and now it trades unofficially
at about 520. And Turkey --- where much of the Aleppo factories'
machinery was transported and sold, often with the collusion of Syrian
owners who wanted to avoid losing everything --- now sells similar
textiles for less. Reviving Syrian industry, and the social glue it
might once have provided, is next to impossible.

I asked Nasi what had become of his workers. He said about 70 percent of
them joined the rebels. He didn't seem bitter or surprised about this.
Some lived nearby, so when the area was divided, they had little choice.
As for the others, they were poor and ill educated and religious, and
the rebels promised them a lot. ``The average salary for workers was
about a hundred dollars a week,'' he said. ``The rebels paid more.''

**F**or many Aleppans, caught up in a conflict they had tried to avoid,
the only rule was survival. On a warm spring morning in 2013, a
22-year-old man named Yasser lay bleeding in the middle of a street in
eastern Aleppo. Moments earlier, he had carried his mother, mortally
wounded by a sniper, into his grandparents' car. As he watched the car
pull away, three bullets struck his legs and left arm. He collapsed into
the street and could not move. Shots rang out over his head: regime
soldiers trading fire with rebels on either side of him. The soldiers
heard Yasser calling for help and told him to come toward them. ``I
can't move,'' he shouted. Then a rebel spoke from a nearby building,
promising to help. When he answered, a regime soldier called out, ``Who
are you talking to?'' The rebels quickly warned him not to answer or
they would kill him.

``I was very scared of both sides,'' Yasser told me later. ``If I went
to one side, the other would kill me.'' He lay there, his limbs going
numb, too frightened to move or speak for more than four hours.

I met Yasser in March in Sha'ar, the most devastated neighborhood in
eastern Aleppo. He was short and solidly built, with a snub nose and a
gruff manner. He was selling tomatoes and cucumbers from a stand, on a
block where many buildings were in ruins. Across the street was a fruit
stand, and next to it, a loud generator, set up by the government to
supply electricity. Surprising numbers of people walked the streets.
This place had been almost completely empty a few weeks earlier, but now
that Russian mine-clearing teams had been through and the rubble was
mostly pushed aside, Sha'ar's residents were returning to their homes.
(More than 100,000 went back to eastern Aleppo between January and
March, according to the International Organization for Migration.)
Yasser said he was one of the first people to come back, right after
what he --- like everyone else I met --- called the liberation. It was a
gesture of defiance, aimed at the rebels. ``What we lost, we will get it
back,'' he said. He wore military fatigues, and he told me he
re-enlisted in the military after he got out of the hospital in 2013.
``My blood type is O-Assad,'' he said.

Later, Yasser showed me the place where he was wounded. It was the first
time he'd been back since it happened, and the block had changed, like
most of eastern Aleppo. ``There was a checkpoint here, there were
sandbags there,'' he said. He pointed out the first-floor window where
an old man had talked to him through curtains as he lay on the street.
He showed me the building where he thought the sniper had been hiding,
about 100 yards away. He explained how his ordeal had ended: An
airstrike hit the building, and the sniper vanished. A man on a
motorbike rescued Yasser, carrying him to a house, where someone cleaned
his wounds. Later, he was taken to a hospital, where a doctor told him
that his mother was dead. The doctor put a needle in his arm and told
him to count to three, and he blacked out.

I found Yasser's story credible, and his uncle later backed it up. But
as I stood on the street with him, I found myself wondering: Did he
really know who shot him? Bullets were coming from each side. As he lay
there bleeding, whom was he more frightened of --- the rebels or the
regime? Yasser clearly knew how his government is portrayed in the West
and seemed defensive about it. He told me a rebel group tried to blame
the regime for his mother's death. Later, he said, the same group
admitted its guilt and offered blood money, which the family refused to
take. This seemed less plausible. He walked me down the street to his
uncle's house, where he said we would hear another story about what the
rebels had done.

Yasser's uncle was a big, heavyset man with a jowly face and a look of
weary resignation in his eyes. He welcomed us into his tiny apartment,
where he offered me a stool and sat down on his old brass bed. He sighed
and apologized for being unable to offer us tea. Then he showed us his
scarred arm and told us the story of how his family was devastated in
January 2013. He was driving his pregnant daughter to the hospital when
machine-gun fire riddled the car, killing his wife instantly and
wounding everyone else. He told me rebels from the Free Syrian Army
pulled them from the car and rushed them to a nearby hospital. I asked
who fired on them. ``I don't know,'' he said.

Image

In the Old City.Credit...Sebastián Liste/Noor Images, for The New York
Times

There was a silence. Until that moment, I had not heard anyone miss an
opportunity to blame the rebels. With my government minder looking on,
Yasser began asking where the gunfire had come from. Wasn't it from a
tall building nearby? Weren't the rebels in that building? The uncle
shrugged. ``I don't know,'' he said again. I had the impression that he
was profoundly depressed and past caring about what he was supposed to
say. Yasser kept pressing. Eventually, the uncle caved in and said,
Sure, it was probably the rebels. ``My wife died a martyr,'' he said.
``For the F.S.A., I cannot say, because they helped me and my wife. I
cannot say how they are with other people.'' And then quietly, he began
to cry.

**T**here is a billboard hanging over one of Aleppo's main intersections
that shows a soldier wearing a helmet, with his entire face shrouded in
shadow. Beneath it are the words ``Aleppo in Our Eyes.'' The image stuck
with me. Which Aleppo? Whose eyes? In Arabic, the expression conveys
affection, but the words also seemed to hint at the city's fragmented
loyalties, its atmosphere of enduring suspicion. It wasn't just regime
allegiance that made people like Yasser tailor their stories. He refused
to be photographed, and when I asked why, he hinted that the rebels
might still be able to find and kill him, even after the regime's
triumph. Another man described receiving a visit at his home in
government-controlled Aleppo from two ISIS members, who calmly
blackmailed him and went on their way, unhurried. The city had changed
hands so many times that no one could be fully confident whose eyes
would be watching them.

On my second day in the city, I went to see the Aleppo Eye Hospital, a
sprawling compound that the rebels had used as a military headquarters.
As we walked through the burned and shattered building, my government
minder and the soldiers guarding the place kept picking up markers of
the rebels' Islamist leanings. They weren't hard to find. A
fire-blackened car out front still had the Qaeda logo on its hood.
Inside, the rebels had put up paper signs to show how they used the
rooms: a room where Shariah rulings were handed out by a religious
sheikh, a document about Islamic punishments. There was a prison too,
and I later met a woman who seems to have been kept there. She had been
captured in a rural village, and the rebels killed her husband and then
moved her from place to place, intending to trade her for their own
prisoners. There were female jailers who beat and cursed her and called
her an infidel. She told me she was given a bottle of water to wash
herself with once every 10 to 15 days. During the final battle for
Aleppo, she often heard the sounds of bombs and mortars exploding
nearby, and her jailers would taunt her, saying Assad's bombs will kill
you.

As I walked through those ruins, it was clear enough that the rebels who
ruled eastern Aleppo had done some awful things there. Yet the whole
hospital tour was designed, at least in part, to mitigate or obscure a
very uncomfortable fact. The Assad regime repeatedly and deliberately
bombed hospitals in the rebel zone, even when there was no reason to
suspect that fighters were based there. No one would discuss this with
me during my time in Aleppo, even when I did not have the minder with
me. Instead, I had to speak to people who fled eastern Aleppo under the
terms of the deal to evacuate the city in December, when the regime
recaptured it. They were living in Idlib province, to the southwest,
which is held by rebels, and I spoke to them by Skype. One was a young
man who worked as a nurse at the Omar bin Abdul Aziz Hospital throughout
2016. He told me that the hospital was rendered inoperable 15 times by
regime airstrikes. Each time, engineers and doctors would rehabilitate
it, only to see it damaged again. When the regime soldiers got too
close, they moved to another hospital, called Al Quds. It was so crowded
that they sometimes tended the wounded in the street outside.

Image

The Aleppo Eye Hospital.Credit...Sebastián Liste/Noor Images, for The
New York Times

Stories like this have been amply documented and held up as evidence
that the Assad regime is guilty of war crimes on a wide scale. A
nongovernmental organization in Europe has been working for years to
gather documents that would tie the Syrian leadership to these crimes in
a Nuremberg-style trial. That prospect is remote, but there are signs
that Assad, too, may be worried about whose eyes are watching him. This
month, the State Department released satellite photographs suggesting
that the regime is burning the bodies of executed prisoners in a
crematory at the Sednaya prison complex, north of Damascus, in an
alleged effort to hide evidence. The Syrian regime called these charges
a ``new Hollywood plot.''

**O**ne afternoon, I was sitting in an Aleppo cafe with a 26-year-old
man who had just got out of the army after six years. He served all over
the country and had shrapnel wounds on his legs and back. His cellphone
rang, and I watched his eyes widen as he absorbed what was obviously bad
news. When he got off the phone, he told me that six friends from his
old unit had just been killed in Jobar, a suburb of Damascus. The rebels
had launched a complex attack using tunnels and multiple suicide
bombers. All in all, about 30 regime soldiers had been killed. That
attack was the start of a rebel offensive that reached the edge of
Damascus's Old City, keeping residents awake much of the night with
deafening blasts. The rebels mounted simultaneous assaults to the north,
forcing the roads to close for days. I had my own minor brush with the
rebel campaign. As I was driving from Aleppo to the Syrian coast, rebels
opened fire on the road to our right as we passed near the city of Homs.
A soldier yelled at us to move fast, and our driver gunned the engine
--- we must have hit 100 miles an hour on a tiny two-way road --- and
told us to duck our heads. The rebels were just a hundred yards away. I
heard their shellfire thumping in the distance almost everywhere I went.

It is impossible to live in government-controlled Syria without noticing
that there are almost no young men on the street. They are in the army,
or they are dead. Veterans must carry their military papers with them or
risk on-the-spot re-enlistment. At one checkpoint, government soldiers
tried to grab the young Spanish photographer I was working with, who is
easily mistaken for a Syrian; they wanted to recruit him. In Latakia, a
beach town in the regime's northwestern heartland, I met a 53-year-old
businessman named Munzer Nasser, who commands a militia composed almost
entirely of older men; there are no young men left in his village. One
of its members, he told me, is a 65-year-old whose three sons have all
been killed in the war. Behind the Assad regime's atrocities lies a fear
of demographic exhaustion. Its rebel opponents have no such worries:
They can draw on a vast well of Islamist sympathizers across the Arab
world.

These facts translate into a genuine gratitude --- in regime-controlled
areas --- toward Russia, whose military intervention in late 2015 may
have forestalled a total collapse. Many Syrians say they feel reassured
by the sight of Russian soldiers, because they (unlike the army and its
allied militias) are not likely to loot or steal. Some of my contacts in
regime-controlled areas are even learning Russian. In Latakia, some
people told me that their city might have been destroyed if not for the
Russians. The city has long been one of Syria's safe zones, well
defended by the army and its militias; there are tent cities full of
people who have fled other parts of the country, including thousands
from Aleppo. But in the summer of 2015, the rebels were closing in on
the Latakia city limits, and mortars were falling downtown. If the
rebels had captured the area --- where Alawites are the majority --- a
result would almost certainly have been sectarian mass murder. Many
people in the region would have blamed the United States, which armed
some of the rebels operating in the area. In this sense, the Russian
intervention was a lucky thing for the Obama administration too. Andrew
Exum, who worked in the Pentagon at the time, told me that the military
drew up contingency plans for a rapid collapse of the regime. The
planning sessions were talked about as ``catastrophic success.''

Yet Assad's popularity is due not only to his role as the guarantor of a
secular order. He has also cannily positioned himself as a unique
guardian against his own regime. Just before I arrived in Aleppo in
March, a high-ranking Republican Guard commander in the city issued a
public order declaring a crackdown on ``acts of looting, robbery and
assaults on public property and on the freedoms of citizens and their
private property.'' The order was a belated recognition of what had been
going on for months: an orgy of looting by the various paramilitary
groups that work alongside the Syrian Army, and even by elements of the
army itself.

Image

Inside the Great Mosque of Aleppo.Credit...Sebastián Liste/Noor Images,
for The New York Times

I heard complaints about this everywhere I went. Looting has become so
common that it has generated a new word: \emph{ta'feesh}, to steal
furniture. One reporter for the regime-friendly TV channel Al Mayadeen
said in a November interview that ``this systematic looting has exceeded
all limits to include murder as well as stealing and looting.'' He went
on to describe a ``rigorously organized'' process in which the
paramilitary groups followed the Syrian Army and pillaged at will,
sometimes ``dragging homeowners from their houses and robbing the houses
right in front of their eyes.'' Another common tactic, he said, was to
pour gasoline on walls and set a fire ``until the tiles on the floors
and walls expand due to the heat. Then they put out the fire, remove the
tiles and resell them.''

The reporter, an Aleppan named Rida al-Basha, described the
neighborhoods where this had taken place and named the militias,
including the notorious Tiger Forces, whose leaders include well-known
thugs. At the end of the interview, Basha said that he had pleaded with
the city government and other journalists to expose these crimes, but
that everyone was too frightened. After the report, Basha repeatedly
said he was threatened with death, and he is said to have fled the
country.

Publicly, the Syrian state deplores these crimes, but privately it seems
to condone them as a form of compensation for the paramilitary groups,
whose support Assad needs to supplement his decimated army. (The rebels
do it, too, and sometimes offer an Islamic justification: \emph{ghana'im
al-harb}, the spoils of war.) Only when the looting starts to spin out
of control, as it did in Aleppo in January and February, is there a
crackdown. But such systematized thievery has become entrenched in an
economy that is more corrupt than ever. Regime-allied armed groups often
set up checkpoints and extort taxes from farmers and businessmen, making
it that much harder to earn a living. ``You pay through the nose to
transport anything anywhere,'' I was told by a man who manufactures
plastics and has seen most of his profit margin disappear. ``Bashar
can't do anything about this. He is in survival mode.'' Meanwhile, war
profiteers (\emph{tujjar al-harb} --- another phrase you hear a lot in
Syria nowadays) have become well-known figures. I was amazed to see new,
lavish-looking restaurants in Damascus; some of them belong to men who
are said to have grown rich from crime. Members of the old Damascus
business elite wince when they describe the clientele in these places.
One friend told me, ``You see a guy in a business suit in a fancy bar
talking to a thuggish-looking guy in fatigues, and you understand the
conversation without hearing anything.'' Some of these men are also
widely said to sell oil to rebel groups for huge profits.

Late last year, Iran abruptly suspended oil deliveries, which have
become a lifeline for Syria. Iran acted because it was angry about the
amount of its fuel that was being diverted and sold to rebels by
regime-connected middlemen, I was told by a Syrian who has close ties to
Hezbollah, Iran's ally. The suspension created a serious fuel crisis in
winter. Iran resumed its supplies in mid-February, but Tehran has little
choice: It needs Assad as much as he needs it. There are reports of
similar tensions with the Russians, who are more interested in brokering
an end to the fighting than Assad is. The Syrian businessman put it like
this: ``Bashar is like a man with two false legs --- one is Russia and
one is Iran. He keeps hopping from one leg to the other, because the
ground he is standing on is very hot.''

All this may sound awfully precarious for Assad. But in a sense, it is
just a more extreme form of the game Assad and his father have played
for decades. The Assad regime arose after an unstable period during the
1950s and '60s, when Syria was shaken by coups and countercoups. Hafez
al-Assad, Bashar's father, triumphed in part by managing a constellation
of rivals who hated one another but were all dependent on him. They knew
that without him at the center, chaos would return, and that would be
bad for business. This is truer than ever today. And it has a secondary
effect, not unimportant: Many ordinary people now see Assad as their
only hedge against a far more toxic kind of chaos.

My Syrian businessman friend told me that he twice gathered about a
dozen people for dinner and offered them a hypothetical in strict
confidence. It is up to you to name the next president of Syria, he
said. Whom would you choose? The guests were all Syrians, and none
supported the regime. To his surprise, almost all of them named Assad.
When he asked why, the same answer came back again and again: Assad is
the only one who can protect us against his own devils.

Image

Displaced Aleppans at a temporary shelter about six miles from the
city.Credit...Sebastián Liste/Noor Images, for The New York Times

**W**hile I was in Syria, I found myself thinking now and again about
the vast street demonstrations I saw in Iran in 2009. This was just
after the disputed re-election of Mahmoud Ahmadinejad as president, when
millions of people marched peacefully through the streets of Tehran. The
crowd drew from every social class, every generation. A peaceful popular
movement seemed to have brought the theocratic regime to its knees. Soon
after the largest march, on June 15, the police and the Basij militia
came out in force, spraying tear gas and beating people with truncheons.
Protests went on for months, but eventually they dwindled to a hard
core, and the regime crushed the movement with relative ease. I often
wondered about all those people I saw in the streets on June 15. At the
time, their absence felt a bit like cowardice to me. Now it feels more
like a kind of earned political wisdom. They stayed home not because
they preferred the regime but because they did not want to risk death.
And perhaps because they did not want to see their country torn apart.

Anas Joudeh, the Damascus lawyer, told me that the absence of this kind
of wisdom is precisely what doomed Syria. For years, he said, he and his
friends cast around looking for someone to blame for the failure of the
2011 revolution. ``We often asked, If only this, or if only that,'' he
said. ``But now I feel that what happened was destiny. Because there are
no political or social forces in Syria. The regime emptied them out. So
when the regime looked to make a deal in 2011, there was no one there.''
I took Joudeh to be saying that the regime might have been willing to
share power, in some limited way, if the opposition had been more
organized, more conciliatory. Perhaps it is naïve to suggest that the
regime could have offered genuine reforms of its own accord. Police
states are not known for voluntarily giving up power in the interest of
building a better future. Assad has spoken the language of reform ever
since he inherited his role from his father in 2000, and he has never
followed through on any of it. Still, there is one story that has
haunted me.

On March 30, 2011, Assad delivered a televised speech to Syria's
rubber-stamp Parliament that is widely viewed in retrospect as a crucial
step in the country's descent into war. He had kept silent during the
previous two weeks of protest and violence. Some of his advisers and
proxies had hinted, in the days beforehand, that he would make historic
proposals, offering a hand to the protesters and paving the way for
genuine national reconciliation. Much of the region tuned in as Assad
walked up a red carpet into the Parliament building past a cheering
crowd. But his speech quickly turned into a familiar, embarrassing
spectacle, with lawmakers chanting his name and interrupting his speech
with fawning accolades. Assad delivered a hard-line speech deriding the
protesters as dupes of a foreign-backed plot to destroy the country. He
closed on an ominous note, saying: ``There is no compromise or middle
way in this. What is at stake is the homeland, and there is a huge
conspiracy. ... We have never hesitated in defending our causes,
interests and principles, and if we are forced into a battle, so be
it.''

One former regime official told me that he recalls watching the speech
with a sense of shock and dismay. He and other high-ranking officials
had heard in advance the details of what the speech was supposed to say.
It had been drafted, they were told, by Vice President Farouk al-Shara,
and it emphasized reconciliation with the protesters. Shara had received
input from several other top officials with similar inclinations. This
version of the speech even had the support of Hezbollah's leaders, who
believed that genuine gestures of compromise could head off a war, the
former official said. Other people close to the regime have echoed this
account, though there are analysts who are skeptical; it's almost
impossible to be sure about what happens in Assad's secretive inner
circle.

What is certain is that Assad did not deliver the speech that was
expected. Instead, the former official said, he scrapped it at the last
minute in favor of a much more aggressive text. ``When I heard the
speech, my feeling was --- we are in for a long fight,'' the former
official told me. ``I was in my office. We looked around at each other
and did not say a word.'' He remains convinced that if Assad had given
the other speech, the past six years would have unrolled very
differently, and oceans of blood might have been spared.

Image

A cemetery near the Old City.Credit...Sebastián Liste/Noor Images, for
The New York Times

**F**ew people in Syria have any patience for this kind of wistful talk.
Former regime critics like Joudeh now confine themselves to pressing for
the smallest-bore reforms: better training for the police and judiciary,
more local control in towns and cities, a diminished role for the Baath
Party and its outmoded Arab Nationalist bromides. But even this will not
happen while the war continues, and it may be even less likely
afterward. Assad has a genius for corrupting everyone around him, in
ways large and small (some of his advisers are said to be receiving land
in bombed-out rebel areas). Even giving in to these mild measures of
hope can start to make you feel dirty, as if you had been played for a
fool. The alternative, of course, is an ecumenical cynicism toward
everyone and everything. This is the default mentality for most Syrians
I know.

In my time in Syria, I met one person who seemed to evade both of these
traps. I hesitate to use the word ``hero,'' because he would violently
reject it. But he held onto a dogged civic idealism that was divorced
from hope of any kind. In a sense, he was the inverse of Abu Sami, the
professor who shut himself off from the war inside his home. He was a
57-year-old engineer named Tarif Attora, who appeared to be working
himself to death running a group called the Aleppo People's Initiative.
He and his teams repair water pipes and electricity lines and supply
food and medical aid to people in need. They do this in both regime and
rebel areas, unlike the White Helmets, the rescue group that was
lionized in an Academy Award-winning documentary. He is the only person
I know who has the unreserved admiration of both rebel leaders and
die-hard regime loyalists.

I met Attora in the initiative's office, where he sat at a battered desk
with a vast map of the city --- east and west --- on the wall behind
him. The desk was covered with stacked files and old coffee mugs, and he
interrupted our talk several times to bark instructions to site
managers. (``Don't strain the lines. You're getting 50. It should be 25
to 30.'') He has steel-gray hair that is cut short and flat on the top
of his head, and his face --- stark, creased with vertical lines,
square-jawed --- looks a bit like Albert Camus's might have if he lived
a decade longer.

The initiative started in July 2012, when Attora gathered roughly 30
engineers and other professionals --- all Aleppo residents --- to talk
about how they could help protect the city. Some were with the
opposition, some were not, and there were arguments. They agreed on one
thing: the need to keep the lights on and the water running. So they
asked the authorities for permission and began reaching out to all the
rebel groups. Soon they had hundreds of volunteers working with them and
repair crews going everywhere, even the front lines. The state water
company supplied pipes and materials, but apart from that, the
initiative is entirely self-funding, Attora said. Six of its members
have been killed. Many others have been wounded, including Attora.

He told me several harrowing stories about his work in what he called
``hot zones.'' Twice he came close to being killed, and his back is now
broken in two places. Jihadi groups were in control of Aleppo's main
power plant for more than a year, so he ended up dealing with them a
lot. Once, he saw something that left him traumatized for months. He
asked me not to report the details, because it might anger the people
involved and limit his ability to work with them. Despite the trauma ---
which still haunts him --- he did not want to jeopardize his ability
``to continue working in all areas,'' he said.

I asked Attora why he does it, and he hesitated. He seemed uncomfortable
dealing with abstractions. ``Freedom doesn't come from destroying the
country,'' he said as he put out what must have been his 10th cigarette
since our conversation started and lit another. ``Look, people consider
me opposition,'' he said. ``But the way I see opposition --- it doesn't
mean I must destroy my country and put us back 100 years. That kind of
opposition is a betrayal of the country, a betrayal of the ideals I've
grown up with.''

He seemed unsatisfied with his words, and he glanced around the room, as
if he were looking for an excuse to stop talking and get back to his
engineers. It was getting dark outside. ``We all served the politics of
other countries in our own land, whether we knew it or not,'' he said.
``Everybody has to wake up. To be brave, to admit they've made mistakes,
to come back to the right way.''

I stood up to shake Attora's hand and say goodbye. His face cracked into
a smile, and the phone rang. He picked it up, and instantly he was at
home again, supervising repairs on a power line that would probably be
blown up again tomorrow.

Advertisement

\protect\hyperlink{after-bottom}{Continue reading the main story}

\hypertarget{site-index}{%
\subsection{Site Index}\label{site-index}}

\hypertarget{site-information-navigation}{%
\subsection{Site Information
Navigation}\label{site-information-navigation}}

\begin{itemize}
\tightlist
\item
  \href{https://help.nytimes3xbfgragh.onion/hc/en-us/articles/115014792127-Copyright-notice}{©~2020~The
  New York Times Company}
\end{itemize}

\begin{itemize}
\tightlist
\item
  \href{https://www.nytco.com/}{NYTCo}
\item
  \href{https://help.nytimes3xbfgragh.onion/hc/en-us/articles/115015385887-Contact-Us}{Contact
  Us}
\item
  \href{https://www.nytco.com/careers/}{Work with us}
\item
  \href{https://nytmediakit.com/}{Advertise}
\item
  \href{http://www.tbrandstudio.com/}{T Brand Studio}
\item
  \href{https://www.nytimes3xbfgragh.onion/privacy/cookie-policy\#how-do-i-manage-trackers}{Your
  Ad Choices}
\item
  \href{https://www.nytimes3xbfgragh.onion/privacy}{Privacy}
\item
  \href{https://help.nytimes3xbfgragh.onion/hc/en-us/articles/115014893428-Terms-of-service}{Terms
  of Service}
\item
  \href{https://help.nytimes3xbfgragh.onion/hc/en-us/articles/115014893968-Terms-of-sale}{Terms
  of Sale}
\item
  \href{https://spiderbites.nytimes3xbfgragh.onion}{Site Map}
\item
  \href{https://help.nytimes3xbfgragh.onion/hc/en-us}{Help}
\item
  \href{https://www.nytimes3xbfgragh.onion/subscription?campaignId=37WXW}{Subscriptions}
\end{itemize}
