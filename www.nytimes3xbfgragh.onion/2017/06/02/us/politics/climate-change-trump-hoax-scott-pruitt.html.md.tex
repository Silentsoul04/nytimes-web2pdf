Sections

SEARCH

\protect\hyperlink{site-content}{Skip to
content}\protect\hyperlink{site-index}{Skip to site index}

\href{https://www.nytimes3xbfgragh.onion/section/politics}{Politics}

\href{https://myaccount.nytimes3xbfgragh.onion/auth/login?response_type=cookie\&client_id=vi}{}

\href{https://www.nytimes3xbfgragh.onion/section/todayspaper}{Today's
Paper}

\href{/section/politics}{Politics}\textbar{}Does Donald Trump Still
Think Climate Change Is a Hoax? No One Can Say

\url{https://nyti.ms/2rAPzsh}

\begin{itemize}
\item
\item
\item
\item
\item
\end{itemize}

Advertisement

\protect\hyperlink{after-top}{Continue reading the main story}

Supported by

\protect\hyperlink{after-sponsor}{Continue reading the main story}

\hypertarget{does-donald-trump-still-think-climate-change-is-a-hoax-no-one-can-say}{%
\section{Does Donald Trump Still Think Climate Change Is a Hoax? No One
Can
Say}\label{does-donald-trump-still-think-climate-change-is-a-hoax-no-one-can-say}}

\includegraphics{https://static01.graylady3jvrrxbe.onion/images/2017/06/03/us/03trump/03trump-articleLarge.jpg?quality=75\&auto=webp\&disable=upscale}

By \href{http://www.nytimes3xbfgragh.onion/by/peter-baker}{Peter Baker}

\begin{itemize}
\item
  June 2, 2017
\item
  \begin{itemize}
  \item
  \item
  \item
  \item
  \item
  \end{itemize}
\end{itemize}

WASHINGTON --- As a businessman,
\href{https://www.nytimes3xbfgragh.onion/topic/person/donald-trump?inline=nyt-per}{President
Trump} was a frequent and scornful critic of the concept of
\href{https://www.nytimes3xbfgragh.onion/section/climate?inline=nyt-classifier}{climate
change}. In the years before running for president, he called it
``nonexistent,'' ``mythical'' and a ``a total con job.'' Whenever snow
fell in New York, it seemed, he would mock the idea of global warming.

``Global warming has been proven to be a canard repeatedly over and over
again,''
\href{https://twitter.com/realDonaldTrump/status/185074709111644160?ref_src=twsrc\%5Etfw\&ref_url=https\%3A\%2F\%2Fwww.vox.com\%2Fpolicy-and-politics\%2F2017\%2F6\%2F1\%2F15726472\%2Ftrump-tweets-global-warming-paris-climate-agreement}{he
wrote} on Twitter in 2012. In another post later that year,
\href{https://twitter.com/realdonaldtrump/status/265895292191248385?lang=en}{he
said}, ``The concept of global warming was created by and for the
Chinese in order to make U.S. manufacturing non-competitive.'' A year
later,
\href{https://twitter.com/realDonaldTrump/status/408977616926830592?ref_src=twsrc\%5Etfw\&ref_url=https\%3A\%2F\%2Fwww.vox.com\%2Fpolicy-and-politics\%2F2017\%2F6\%2F1\%2F15726472\%2Ftrump-tweets-global-warming-paris-climate-agreement}{he
wrote} that ``global warming is a total, and very expensive, hoax!''

But on Friday, a day after Mr. Trump
\href{https://www.nytimes3xbfgragh.onion/2017/06/01/climate/trump-paris-climate-agreement.html?ref=todayspaper}{withdrew
the United States} from the Paris climate change accord, the White House
refused to say whether the president still considers climate change a
hoax. As other leaders around the world vowed to confront climate change
without the United States, Mr. Trump's advisers fanned out to defend his
decision and, when pressed, said they did not know his view of the
science underlying the debate.

``I have not had an opportunity to have that discussion,'' said Sean
Spicer, the White House press secretary.

\href{https://www.nytimes3xbfgragh.onion/interactive/2017/06/02/climate/trump-paris-green-climate-fund.html}{}

\includegraphics{https://static01.graylady3jvrrxbe.onion/images/2017/06/02/climate/trump-paris-green-climate-fund-1496444974956/trump-paris-green-climate-fund-1496444974956-articleLarge.jpg}

\hypertarget{what-is-the-green-climate-fund-and-how-much-does-the-us-actually-pay}{%
\subsection{What Is the Green Climate Fund and How Much Does the U.S.
Actually
Pay?}\label{what-is-the-green-climate-fund-and-how-much-does-the-us-actually-pay}}

In announcing his decision to exit the Paris accord, President Trump
scorned the Green Climate Fund. Does the United States contribute an
outsize share?

``I do not speak for the president,''
\href{http://www.cnn.com/2017/06/02/politics/donald-trump-climate-change-belief/}{said
Ryan Zinke}, the interior secretary.

``You should ask him that,''
\href{http://abcnews.go.com/US/trump-counselor-kellyanne-conway-wont-president-believes-global/story?id=47787361}{said
Kellyanne Conway}, the White House counselor.

Mr. Trump offered no opportunity for anyone to ask him that on Friday.
But his current views, whatever they may be, presumably shaped his
thinking as he evaluated whether to remain in the Paris accord. Given
that he promised on Thursday to seek to re-enter the pact on better
terms or negotiate an entirely new deal that he said would be fairer to
the United States, his acceptance or denial of climate science seems
likely to determine his approach.

In his speech announcing his decision, he did not address the science of
climate change or repeat any of the skepticism he has expressed for
years. Instead, he cast it largely in economic terms, arguing that
President
\href{https://www.nytimes3xbfgragh.onion/topic/person/barack-obama}{Barack
Obama} agreed to a bad deal for Americans that would handcuff the
economy and put the United States at a disadvantage against its
international competitors. He did not say the goal itself was pointless,
only that it would be too much of a burden.

But administration officials clearly saw no benefit in clarifying. If
they affirmed that he still believed climate change to be fake, they
would expose him to even more criticism at home and abroad and
complicate the lives of those advisers who accept the broad scientific
consensus. If they asserted that he had changed his mind and now agreed
that climate change is real, then they would have to explain a flip-flop
while risking criticism from his own base.

Moreover, recent weeks have reminded White House aides about the dangers
of making declarative statements about the president's beliefs or
actions only to have him contradict them within days or even hours. When
Mr. Trump
\href{https://www.nytimes3xbfgragh.onion/2017/05/09/us/politics/james-comey-fired-fbi.html}{fired
James B. Comey}, the F.B.I. director, he sent out his vice president and
top aides to give an explanation of his decision that quickly unraveled
after he gave an interview with a conflicting version of events.

Climate science deniers, cheered by his decision to pull out of the
Paris agreement, seemed willing to live without a clearer statement
taking on what they call the bogus claims of environmental advocates.

``I think his withdrawing us from Paris was the greatest action by a
president in my lifetime,'' said Steve Milloy, who runs a website,
JunkScience.com, which aims to debunk climate change and who served on
Mr. Trump's environmental transition team. ``And he explained his action
brilliantly. Most substantive explanation I've ever heard from a
president --- including Reagan.''

``What he believes,'' Mr. Milloy added, ``you need to get from him.''

Supporters of the Paris accord said the White House refusal to outline
Mr. Trump's beliefs on climate change indicated that he had not bothered
to inform himself on the issue before making a decision with enormous
consequences. ``By not admitting what his views on this, the White House
is just hiding the fact that Trump is too incurious to actually look
seriously at the issue,'' said Andrew Light, a former Obama State
Department official who helped negotiate the Paris pact.

Carol Browner, a former Environmental Protection Agency administrator
under Bill Clinton and adviser to Mr. Obama, said Mr. Trump's action
seemed founded on misinformation. ``Seems he accepts junk science in his
decision making which makes you wonder if next he will repeal bans on
indoor smoking and put lead back in paint,'' she said.

Scott Pruitt, the administrator of the E.P.A. and a longtime critic of
what he calls ``climate exaggerators,'' said the question of what Mr.
Trump believed about the science never came up during the
administration's deliberations over the Paris agreement.

\href{https://www.nytimes3xbfgragh.onion/interactive/2017/03/21/climate/how-americans-think-about-climate-change-in-six-maps.html}{}

\includegraphics{https://static01.graylady3jvrrxbe.onion/images/2017/03/21/climate/how-americans-think-about-climate-change-in-six-maps-1490111758035/how-americans-think-about-climate-change-in-six-maps-1490111758035-articleLarge.jpg}

\hypertarget{how-americans-think-about-climate-change-in-six-maps}{%
\subsection{How Americans Think About Climate Change, in Six
Maps}\label{how-americans-think-about-climate-change-in-six-maps}}

Americans overwhelmingly believe that global warming is happening, and
that carbon emissions should be scaled back. But fewer are sure that it
will harm them personally.

``What's interesting about all the discussions that we had through the
last several weeks have been focused on one singular issue: Is Paris
good or not for this country?'' he told reporters at the White House.
``That's the discussions I've had with the president.''

Mr. Pruitt and other administration officials defended Mr. Trump's
decision as a courageous action to protect the United States. ``We have
nothing to be apologetic about as a country,'' he said, noting that the
country has reduced its carbon emissions in recent years, attributing
that to innovation and technology rather than government regulation.

``So, when we look at issues like this, we are leading with action and
not words,'' he said. ``I also want to say that exiting Paris does not
mean disengagement.''

Describing his own views, Mr. Pruitt derided those ``climate
exaggerators,'' who he said make assertions with great certainty. Mr.
Pruitt said he has concluded that ``global warming is occurring, that
human activity contributes to it in some manner'' but ``measuring with
precision, from my perspective, the degree of human contribution is very
challenging.''

Mr. Trump has not been so shy in the past about his opinions on the
subject. At one point in 2009, he signed an open letter to Mr. Obama
published as an ad in newspapers supporting ``meaningful and effective
measures to control climate change,'' although that may have just
reflected the influence of the three adult children who also signed.

\href{https://www.nytimes3xbfgragh.onion/interactive/2017/06/02/climate/trump-paris-mayors.html}{}

\includegraphics{https://static01.graylady3jvrrxbe.onion/images/2017/06/01/climate/trump-paris-agreement-reactions-1496371339213/trump-paris-agreement-reactions-1496371339213-articleLarge.jpg}

\hypertarget{how-cities-and-states-reacted-to-trumps-decision-to-exit-the-paris-climate-deal}{%
\subsection{How Cities and States Reacted to Trump's Decision to Exit
the Paris Climate
Deal}\label{how-cities-and-states-reacted-to-trumps-decision-to-exit-the-paris-climate-deal}}

President Trump's decision to withdraw from the Paris climate agreement
drew immediate reaction from big-city mayors, governors and Congress
members.

But he soon found climate change to be a favorite target on Twitter,
mentioning the topic scores of times over the years, particularly during
cold weather spells. ``Any and all weather events are used by the GLOBAL
WARMING HOAXSTERS to justify higher taxes to save our planet!''
\href{https://twitter.com/realDonaldTrump/status/427556692109574146?ref_src=twsrc\%5Etfw\&ref_url=http\%3A\%2F\%2Fwww.motherjones.com\%2Fenvironment\%2F2016\%2F11\%2Ftrump-climate-timeline}{he
wrote} in 2014. ``They don't believe it \$\$\$\$!''

As he opened his presidential campaign, he told Hugh Hewitt, the
conservative radio host, that the weather changed naturally over time
and that there was not a major problem. ``I'm not a believer in global
warming,'' he said. ``I'm not a believer in man-made global warming.''

After he won the election last November, he tempered his views in
\href{https://www.nytimes3xbfgragh.onion/2016/11/23/us/politics/trump-new-york-times-interview-transcript.html}{an
interview} with The New York Times, saying that he believed ``there is
some connectivity'' between human activity and climate change and
promising to look at the issue with fresh eyes.

``I have a very open mind,'' he said. ``And I'm going to study a lot of
the things that happened on it and we're going to look at it very
carefully. But I have an open mind.''

By this week, however, Mr. Trump was no longer speaking his mind on the
question of the science, and neither were his aides.

Mr. Spicer said twice this week that he had not had the chance to ask
the president. Asked if he would find time to take the question to Mr.
Trump, he said, ``If I can, I will.''

Advertisement

\protect\hyperlink{after-bottom}{Continue reading the main story}

\hypertarget{site-index}{%
\subsection{Site Index}\label{site-index}}

\hypertarget{site-information-navigation}{%
\subsection{Site Information
Navigation}\label{site-information-navigation}}

\begin{itemize}
\tightlist
\item
  \href{https://help.nytimes3xbfgragh.onion/hc/en-us/articles/115014792127-Copyright-notice}{©~2020~The
  New York Times Company}
\end{itemize}

\begin{itemize}
\tightlist
\item
  \href{https://www.nytco.com/}{NYTCo}
\item
  \href{https://help.nytimes3xbfgragh.onion/hc/en-us/articles/115015385887-Contact-Us}{Contact
  Us}
\item
  \href{https://www.nytco.com/careers/}{Work with us}
\item
  \href{https://nytmediakit.com/}{Advertise}
\item
  \href{http://www.tbrandstudio.com/}{T Brand Studio}
\item
  \href{https://www.nytimes3xbfgragh.onion/privacy/cookie-policy\#how-do-i-manage-trackers}{Your
  Ad Choices}
\item
  \href{https://www.nytimes3xbfgragh.onion/privacy}{Privacy}
\item
  \href{https://help.nytimes3xbfgragh.onion/hc/en-us/articles/115014893428-Terms-of-service}{Terms
  of Service}
\item
  \href{https://help.nytimes3xbfgragh.onion/hc/en-us/articles/115014893968-Terms-of-sale}{Terms
  of Sale}
\item
  \href{https://spiderbites.nytimes3xbfgragh.onion}{Site Map}
\item
  \href{https://help.nytimes3xbfgragh.onion/hc/en-us}{Help}
\item
  \href{https://www.nytimes3xbfgragh.onion/subscription?campaignId=37WXW}{Subscriptions}
\end{itemize}
