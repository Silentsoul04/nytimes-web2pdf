Sections

SEARCH

\protect\hyperlink{site-content}{Skip to
content}\protect\hyperlink{site-index}{Skip to site index}

\href{https://www.nytimes3xbfgragh.onion/section/food}{Food}

\href{https://myaccount.nytimes3xbfgragh.onion/auth/login?response_type=cookie\&client_id=vi}{}

\href{https://www.nytimes3xbfgragh.onion/section/todayspaper}{Today's
Paper}

\href{/section/food}{Food}\textbar{}A Chef's Quest in India: Win Respect
for Its Cooking

\url{https://nyti.ms/2rQUk0P}

\begin{itemize}
\item
\item
\item
\item
\item
\item
\end{itemize}

Advertisement

\protect\hyperlink{after-top}{Continue reading the main story}

Supported by

\protect\hyperlink{after-sponsor}{Continue reading the main story}

\hypertarget{a-chefs-quest-in-india-win-respect-for-its-cooking}{%
\section{A Chef's Quest in India: Win Respect for Its
Cooking}\label{a-chefs-quest-in-india-win-respect-for-its-cooking}}

\href{https://www.nytimes3xbfgragh.onion/slideshow/2017/06/06/dining/asha-gomez-in-india.html}{}

\hypertarget{asha-gomez-in-india}{%
\subsection{Asha Gomez in India}\label{asha-gomez-in-india}}

11 Photos

View Slide Show ›

\includegraphics{https://static01.graylady3jvrrxbe.onion/images/2017/06/07/dining/07INDIA-slide-FYU4/07INDIA-slide-FYU4-articleLarge.jpg?quality=75\&auto=webp\&disable=upscale}

Evan Sung for The New York Times

By \href{http://www.nytimes3xbfgragh.onion/by/kim-severson}{Kim
Severson}

\begin{itemize}
\item
  June 6, 2017
\item
  \begin{itemize}
  \item
  \item
  \item
  \item
  \item
  \item
  \end{itemize}
\end{itemize}

MUNNAR, India --- In the shade of a
\href{https://www.youtube.com/watch?v=GKrD9SFhi-Q}{cardamom patch} on a
South Indian mountainside, Asha Gomez leaned against a tree and began to
cry.

She asked a photographer to stop taking pictures and sent a videographer
farther down the dirt path. Ms. Gomez, a chef from Atlanta who had
traveled for 22 hours to get to the land where she was born, needed a
moment.

``I think I had disconnected myself from this place in some way by
saying for so long that the U.S. was home,'' said Ms. Gomez, 47, who had
moved from the Indian state of Kerala to the state of Michigan as a
teenager. ``There is still so much a part of me here. I think I had
forgotten that.''

She wiped her tears and made her way back toward the cameras, more
committed than ever to the work she had set out to do when she landed at
Cochin International Airport a few days earlier.

Ms. Gomez had come to this land of ports, tea estates and spice gardens
not only to reconnect with a part of herself, but also to find new ways
to use her camera-ready personality and kitchen chops to lasso Kerala's
beautiful food culture and drag it back to the United States.

``I have to remove people from the mentality that all Indian food should
be clumped up into nine dishes that are not really Indian dishes,'' she
said. ``Not all Indian food belongs on a buffet line at \$4.99. Indian
food is 5,000 years of tradition and history, and it belongs right up
there with French cuisine.''

Her frustration over American interpretations of the beloved
\href{https://www.bbcgoodfood.com/howto/guide/top-10-foods-try-kerala}{coconut-scented
fish curries, dosas and carefully layered beef biryanis} of her homeland
echoes the lament of countless cooks who have immigrated from countries
like China, Mexico or Vietnam only to find their food mangled to meet
the limitations of a new country's palate and relegated to its
cheap-eats guides.

``I wish I could say to every immigrant cook in America, `Why do you
think your food should be any less than any other cuisine that comes
from anywhere else in the world?''' Ms. Gomez said.

It's not hard to see why: For one thing, unless that food is served in
an upscale setting, with polished service, it doesn't command the prices
or the critical respect afforded European or American cuisines.

\includegraphics{https://static01.graylady3jvrrxbe.onion/images/2017/06/07/dining/07INDIA1/07INDIA1-articleLarge-v2.jpg?quality=75\&auto=webp\&disable=upscale}

And even when the restaurant is fancy, the problem persists. Ms. Gomez
experienced it at her first restaurant, a fine-dining place in Atlanta
she named Cardamom Hill, after the spice-growing region that she was
touring last month. Customers would complain that she charged \$32 for a
complex fish curry with smoked tamarind, even when a fish entree at a
well-regarded new Southern restaurant not far away cost the same.

``That makes me see red immediately,'' said David Chang, the prolific
chef and restaurateur, whose parents immigrated to the United States
from South Korea. ``It's the worst kind of racism, because it's so
readily accepted.''

Even though there are some notable Indian chefs cooking in America,
integrating the kind of food Ms. Gomez loves won't come easy, said Mr.
Chang, who first met Ms. Gomez last week over fried chicken in Atlanta.
``Considering the time we're living in, having someone with that color
skin from that part of the world makes it a hard sell,'' he said. ``It's
probably not going to happen in one lifetime, and it is going to take
relentless media exposure.''

That's exactly why Ms. Gomez had invited a producer working on a show
for PBS; two videographers, who help create her web-based subscription
cooking show, ``Curry and Cornbread''; and two newspaper journalists to
join her in Kerala.

The trip was a relentless blur of food and road miles. One day Ms. Gomez
was picking out silky pomfret and river mullet to smear with masala, in
a makeshift kitchen on the banks of Fort Kochi, and the next she was in
a van grinding up a narrow mountain road to Kerala's vast tea estates,
or buying iron knives from a street vendor. By the end, she was happy to
order a steak and get an ayurvedic treatment at a seaside hotel.

It had been eight years since Ms. Gomez last stepped onto Indian soil.
She had come to adopt her son, Ethan, then a 3-year-old living in an
orphanage.

So much has happened since then. For one thing, she embarked on a
cooking career.

Ms. Gomez originally wanted to work in the beauty business. When the
recession hit the United States in 2008, she had been running a luxe
ayurvedic spa in Atlanta, where she had moved with her husband, Bobby
Palayam. As her clients finished their massages and facials, she would
feed them vegetarian biryanis and coconut-infused stir-fries bright with
turmeric and chiles.

Then the spa succumbed to the downturn, and some of the city's best
chefs, as well as clients who understood a special thing when they
tasted it, encouraged her to keep spreading the curry-and-coconut gospel
of Kerala. She did, first with a supper club and then, in 2012, at
Cardamom Hill. The menu drew parallels between the American South and
the Indian South, highlighting both regions'
\href{http://www.creativeloafing.com/food-drink/100-dishes/article/13073218/kerala-fried-chicken-and-waffles-at-cardamom-hill}{mutual
love of fried chicken}, braised pork and vegetables like okra and field
peas.

``You know that Kerala is in your kitchen when you have coconut oil,
curry leaves and mustard seeds sizzling in a chati,'' she likes to tell
people. ``That's our trinity.''

Two years later, just as Cardamom Hill had gained critical acclaim,
\href{http://www.atlantamagazine.com/dining-news/asha-gomez-on-closing-down-cardamom-hill-opening-spice-to-table-in-studioplex-friday/}{she
shut the place down}. She wasn't making any money, and was working so
hard she had no time for her son. ``I had lost my joy for cooking,'' she
said.

Image

Ms. Gomez stepped in to grill some fish at a stall in Kochi, Kerala. Her
trip was a relentless blur of food and road miles.Credit...Evan Sung for
The New York Times

Her next restaurant was a quiet, stylish Indian patisserie where she
served puff-pastry samosas and carrot cakes infused with black pepper.
She closed it in February. The reasons weren't much different.

Now she cooks for private clients at
\href{https://www.google.com/search?q=Third+Space\&oq=Third+Space\&aqs=chrome..69i57j0l5.659j0j9\&sourceid=chrome\&ie=UTF-8\#q=Third+Space+asha}{the
Third Space}, her kitchen and dining room in Atlanta's Old Fourth Ward.
She teaches cooking, consults for food companies and has become
\href{http://www.care.org/chefstable/chef-asha-gomez}{a ``chef
ambassador'' for CARE}. She receives more requests for public
appearances than she can say yes to. Her 2016 cookbook,
``\href{https://www.nytimes3xbfgragh.onion/2016/09/28/dining/asha-gomez-my-two-souths-cookbook-review.html?_r=0}{My
Two Souths},'' was nominated for a James Beard Award.

Ms. Gomez learned to cook from her mother, Hazel, and her three aunts,
who all lived near one another in a three-household compound in
Thiruvananthapuram (she said she preferred the old name, Trivandrum),
Kerala's capital city on the Arabian Sea. It was a dreamy childhood, in
a religiously diverse and literate region of India where young people
prefer American rock to Bollywood soundtracks.

The Portuguese had begun settling there in the 15th century, bringing
with them a love of pork and for the chiles that would come to mark
Kerala's food. It's how her family got its name, and why she grew up a
meat-eating Roman Catholic in a state where more than half the
population is Hindu.

Her father was a civil engineer who helped build bridges for a German
company. Her mother never set a table that wasn't beautiful.

Ms. Gomez grew up pulling mangoes from the trees and buying sugar cane
from the vendors outside her parochial school. At night, she would head
to the street stalls called thattukadas, for chunks of chicken with
crunchy fried shallots, garlic and curry leaves crisped in coconut oil.
She loves to eat the dish with flaky wheat
\href{http://food.ndtv.com/opinions/taste-of-kerala-what-it-takes-to-perfect-a-malabar-parotta-744958}{parathas},
made using a method that originated in her home state.

Her father was intent on moving the children to America for college. To
prepare, she and her older brothers were required to speak only English
at home and eat using cutlery instead of the tidy, one-handed finger
style many in Kerala use for their curry-soaked red rice and breakfast
puttus.

``I think he had seen enough of the world that he didn't want us to come
to this country and be outsiders,'' she said.

When she was 16, her father died of a heart attack. She and her mother
moved to Michigan, where her older brothers were already in college.
They eventually landed in Queens, where cousins encouraged her mother to
cater food for the Kerala diaspora.

``I hated it,'' Ms. Gomez said. ``Our apartment was so small I would
literally disinfect the bathtub, and I would have to wash the dishes in
there.''

Image

A traditional Sadya meal for a large Indian wedding is prepared in
Kochi, Kerala.Credit...Evan Sung for The New York Times

But Ms. Gomez fit seamlessly into New York life, developing the kind of
cultural fluidity that has allowed her to adapt Kerala's food for the
uninitiated while satisfying the people from her home state, who
lovingly call each other Mallus. ``As much as I love tradition, I am not
a traditionalist,'' she said. ``I'm an innovator.''

Sometimes innovators need to reconnect with their roots, which is why
she got up at 4 a.m. at one point to visit a Kerala catering kitchen
that for 100 years has been creating a classic Hindu celebratory meal
called a sadhya, served at almost every wedding and holiday gathering in
Kerala.

The meal, eaten from a banana leaf and centered on rice, can involve two
dozen dishes that vary from salty to sour to bitter to sweet. Always
there are spicy pickles, sambar and vegetables simmered in coconut milk
or sautéed with ground coconut and curry leaves. It's a celebratory
feast that Ms. Gomez makes a few times a year.

In a wet, torrid kitchen, she watched the cooks, with bare chests and
feet, add buckets of spices and vegetables to pots as big as wading
pools. In a few hours, they would have lunch ready for 1,500.

She talked shop with the owner, Mahadevan Iyer. He wanted to take her to
his factory, where he is packaging food for the American market. ``We
are interested in exporting delicious Kerala,'' he told her.

Only hours earlier, the young woman hired to do her hair and makeup had
also asked Ms. Gomez for help. She wanted to follow the chef back to
America to become her personal assistant.

``I knew the direction this was going to go,'' Ms. Gomez said. She
declined the offers.

She had ideas of her own. In Thekkady, a town surrounded by spice
gardens full of black pepper and nutmeg, she imagined customizing Indian
spice boxes for American chefs. As she wandered
\href{http://www.conservationindia.org/gallery/wildlife-in-tea-plantations-western-ghats}{the
vast tea estates of the Western Ghats mountains}, she mulled over a plan
to provide restaurants with tea service. She even pondered how she might
do for Kerala what the Danish chef René Redzepi has done for Nordic
cuisine, opening a small, seasonal white-tablecloth restaurant with
Michelin-star ambitions.

``Can you imagine what I could do in a place like this?'' she asked.

But there in the Cardamom Hills, she understood the deeper meaning of
the trip. She had found a way to merge her two homes.

``I want to leave an America behind where my son can be proud of both
his heritages,'' she said. ``I want him to carry both these places with
equal pride, hand in hand.''

Recipe:
\href{https://cooking.nytimes3xbfgragh.onion/recipes/1018782-kerala-roadside-chicken}{\textbf{Kerala
Roadside Chicken}}

\textbf{And to Drink ...}

Spicy dishes like this chile-infused fried chicken require specific
sorts of wines. You want to avoid wines high in alcohol, which will
augment the chile heat, and those that are oaky, which will clash.
Instead, you want fruity wines, preferably white, that are low in
alcohol and discernibly sweet.
\href{https://www.nytimes3xbfgragh.onion/2017/06/01/dining/wine-school-riesling-spatlese.html}{German
spätlese rieslings} would be ideal. A distant second would be whites
with residual sugar, like demi-sec Vouvray or certain pinot gris from
Alsace. But they would be higher in alcohol. I very much like dry
sparkling wines, like a top cava or extra-brut Champagne, though some
people believe that bubbles will increase the heat. If you're wed to a
red, try a fruity, spicy unoaked bottle from the Loire Valley. ERIC
ASIMOV

\href{https://www.facebookcorewwwi.onion/nytfood/}{\emph{Follow NYT Food
on Facebook}}\emph{,}
\href{https://instagram.com/nytfood}{\emph{Instagram}}\emph{,}
\href{https://twitter.com/nytfood}{\emph{Twitter}} \emph{and}
\href{https://www.pinterest.com/nytfood/}{\emph{Pinterest}}\emph{.}
\href{https://www.nytimes3xbfgragh.onion/newsletters/cooking}{\emph{Get
regular updates from NYT Cooking, with recipe suggestions, cooking tips
and shopping advice}}\emph{.}

Advertisement

\protect\hyperlink{after-bottom}{Continue reading the main story}

\hypertarget{site-index}{%
\subsection{Site Index}\label{site-index}}

\hypertarget{site-information-navigation}{%
\subsection{Site Information
Navigation}\label{site-information-navigation}}

\begin{itemize}
\tightlist
\item
  \href{https://help.nytimes3xbfgragh.onion/hc/en-us/articles/115014792127-Copyright-notice}{©~2020~The
  New York Times Company}
\end{itemize}

\begin{itemize}
\tightlist
\item
  \href{https://www.nytco.com/}{NYTCo}
\item
  \href{https://help.nytimes3xbfgragh.onion/hc/en-us/articles/115015385887-Contact-Us}{Contact
  Us}
\item
  \href{https://www.nytco.com/careers/}{Work with us}
\item
  \href{https://nytmediakit.com/}{Advertise}
\item
  \href{http://www.tbrandstudio.com/}{T Brand Studio}
\item
  \href{https://www.nytimes3xbfgragh.onion/privacy/cookie-policy\#how-do-i-manage-trackers}{Your
  Ad Choices}
\item
  \href{https://www.nytimes3xbfgragh.onion/privacy}{Privacy}
\item
  \href{https://help.nytimes3xbfgragh.onion/hc/en-us/articles/115014893428-Terms-of-service}{Terms
  of Service}
\item
  \href{https://help.nytimes3xbfgragh.onion/hc/en-us/articles/115014893968-Terms-of-sale}{Terms
  of Sale}
\item
  \href{https://spiderbites.nytimes3xbfgragh.onion}{Site Map}
\item
  \href{https://help.nytimes3xbfgragh.onion/hc/en-us}{Help}
\item
  \href{https://www.nytimes3xbfgragh.onion/subscription?campaignId=37WXW}{Subscriptions}
\end{itemize}
