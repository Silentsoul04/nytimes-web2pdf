The Man Behind Trump's Voter-Fraud Obsession

\url{https://nyti.ms/2siFKzB}

\begin{itemize}
\item
\item
\item
\item
\item
\item
\end{itemize}

\includegraphics{https://static01.graylady3jvrrxbe.onion/images/2017/06/18/magazine/18kobach1/18mag-18kobach-t_CA0-articleLarge.jpg?quality=75\&auto=webp\&disable=upscale}

Sections

\protect\hyperlink{site-content}{Skip to
content}\protect\hyperlink{site-index}{Skip to site index}

Feature

\hypertarget{the-man-behind-trumps-voter-fraud-obsession}{%
\section{The Man Behind Trump's Voter-Fraud
Obsession}\label{the-man-behind-trumps-voter-fraud-obsession}}

How Kris Kobach, the Kansas secretary of state, plans to remake America
through restrictive voting and immigration laws.

Kris Kobach.Credit...Jonno Rattman for The New York Times

Supported by

\protect\hyperlink{after-sponsor}{Continue reading the main story}

By Ari Berman

\begin{itemize}
\item
  June 13, 2017
\item
  \begin{itemize}
  \item
  \item
  \item
  \item
  \item
  \item
  \end{itemize}
\end{itemize}

**K**ris Kobach likes to bill himself as ``the A.C.L.U.'s worst
nightmare.'' The Kansas secretary of state, who was a champion debater
in high school, speaks quickly for a rural Midwesterner, with the
confidence of a man who holds degrees from Harvard, Oxford and Yale Law
School, and until January he hosted his own local radio show, which used
that line about the A.C.L.U. to introduce each episode. On March 3 he
strode into the Robert J. Dole Federal Courthouse in Kansas City, Kan.,
to face the latest lawsuit filed against him by the civil-liberties
organization. In an unusual arrangement for a secretary of state,
Kobach, 51, personally argues all of his cases. He seems to see it as a
perk of the job --- and a mission.

The A.C.L.U. has filed four suits against Kobach since he was elected in
2010. All of them challenge some aspect of his signature piece of
legislation, the Secure and Fair Elections Act, or SAFE Act, a 2011
state law that requires people to show a birth certificate, passport or
naturalization papers to register to vote. Kobach has long argued that
such a law is necessary to prevent noncitizens from registering to vote,
a phenomenon that he has repeatedly claimed is both pervasive and a
threat to democracy. The A.C.L.U. has countered that the real purpose of
the law is not to prevent fraud but to stop the existing electorate from
expanding and shifting demographically. The same principle informed the
``grandfather clauses'' of the Jim Crow era, which exempted most white
voters from literacy tests and poll taxes designed to disenfranchise
black voters. Even a seemingly small impediment to registration, like a
new ID requirement, favors the status quo, and in Kansas, and indeed
nationally, the status quo favors the Republican Party.

The Voting Rights Act of 1965 outlawed tactics that prevented blacks,
Hispanics and other minority groups from voting. But for decades,
Republicans have fought to circumvent the law by describing their
proposed restrictions --- requiring specific forms of identification to
vote, preventing early voting, purging voting rolls --- as colorblind
security measures, even though there is little evidence of any
individual voter fraud in the United States. The A.C.L.U. has repeatedly
argued that the Kansas law discriminated against minorities, young
people and low-income people, all of whom are more likely to be
registering for the first time and less likely to have immediate access
to citizenship papers, because they can't afford them or were more
transient and don't have copies of their documents at hand. No state has
been as aggressive as Kansas in restricting ballot access, and no
elected official has been as dogged as Kobach.

Standing before Judge Julie Robinson in Kansas City, Orion Danjuma, one
of the A.C.L.U. lawyers, noted that Kansas's proof-of-citizenship law
applied only to people registering or updating their registrations after
2013. ``Tens of thousands of Kansans have already been prevented from
registering to vote because of this requirement,'' Danjuma said --- one
in seven new registrants. Close to half of those were under 30.

Today the A.C.L.U. was arguing that a new program called Birth Link ---
which crosschecked flagged names on the list of voter registrations with
Kansas state birth records, conveniently automating the
proof-of-citizenship process --- discriminated against Kansas residents
who were born outside the state. ``The Birth Link policy is, in our
view, a constitutional smoking gun,'' Danjuma said. ``There's nothing
wrong at all with the fact that the Kansas Department of Vital Records
records people who were born in the state. The problem is when the state
starts to distribute benefits --- like the right to vote --- based on
whether or not you're in that database.''

For Kobach, the question of citizenship, and who has a rightful claim to
it, is at the heart of his lawsuits and legislation. Years before Donald
Trump began talking about building a wall, the fate of America's white
majority was a matter of considerable interest to Kobach, who once
agreed with a caller to his radio show that a rise in Latino immigration
could lead to the ``ethnic cleansing'' of whites and has written scores
of laws across the country to crack down on undocumented immigration. He
told The Associated Press in May that he met Trump through his son
Donald Trump Jr., with whom he has a mutual friend. Kobach has since
become close to the White House inner circle, including the president
and his chief strategist, Steve Bannon. Two weeks after the election,
Kobach met with Trump at his golf club in Bedminster, N.J., where the
president-elect was auditioning potential members of his cabinet before
the press, and was photographed holding a white paper outlining a
``Kobach Strategic Plan for First 365 Days.'' Though partly obscured,
what could be read of the document was a bullet-pointed wish-list of
right-wing policies that included ``extreme vetting'' and tracking of
``all aliens from high-risk areas,'' reducing ``intake of Syrian
refugees to zero,'' deporting a ``record number of criminal aliens in
the first year'' and the ``rapid build'' of a wall along the U.S.-Mexico
border.

Kobach did not go to work in the Trump administration: He said in May
that he had turned down two offered positions, one in the White House
and the other at the Department of Homeland Security, although The Wall
Street Journal reported in January that John Kelly, the secretary of the
Department of Homeland Security, had balked at making Kobach his deputy.
But on May 11 Trump named him vice chairman of a new Presidential
Advisory Commission on Election Integrity to be led by Vice President
Mike Pence. The commission will examine ``improper voter registrations
and improper voting'' --- issues that Kobach, with his high-profile
efforts in Kansas, almost single-handedly put on the Trump
administration's radar.

Kobach's plans represent a radical reordering of American priorities.
They would help preserve Republican majorities. But they could also
reduce the size and influence of the country's nonwhite population. For
years, Republicans have used racially coded appeals to white voters as a
means to win elections. Kobach has inverted the priorities, using
elections, and advocating voting restrictions that make it easier for
Republicans to win them, as the vehicle for implementing policies that
protect the interests and aims of a shrinking white majority. This has
made him one of the leading intellectual architects of a new nativist
movement that is rapidly gaining influence not just in the United States
but across the globe.

On June 8, Kobach announced his candidacy in the 2018 Kansas
gubernatorial race, telling a room full of supporters in the Kansas City
suburb of Lenexa that he had ``the honor of personally advising
President Trump, both before the election and after the election, on how
to reduce illegal immigration. And is he doing a good job?'' The crowd
cheered. If Kobach wins, he could be positioned to run for president as
the legal mind who can deliver the promise of Trumpism without the
baggage of Trump himself.

At the A.C.L.U. hearing, Kobach argued that his restrictive measures
were justified by the high stakes. ``We are preventing noncitizens from
voting in elections,'' he said. ``And when a few noncitizens vote, those
can swing a close election.'' Afterward, sipping a Diet Coke at the
restaurant in the Hilton Garden Inn across the street from the
courthouse, Kobach told me he wants his work in Kansas to become a model
for the rest of the country. Other state and federal laws would follow,
if only he could create ``the absolute best legal framework,'' he said.
``That's what I set out to do.''

\includegraphics{https://static01.graylady3jvrrxbe.onion/images/2017/06/18/magazine/18kobach3/18mag-18kobach-t_CA1-articleLarge.jpg?quality=75\&auto=webp\&disable=upscale}

\textbf{No one better} represents the kind of America that Kobach is
promoting than Kobach himself. He is tall and broad-shouldered and looks
like John Wayne. He was born in Wisconsin and moved to Topeka, Kan.,
when he was 7. In high school, he mowed lawns and worked at his father's
Buick dealership. After becoming class president, he went on to Harvard.

It was at Harvard that Kobach became a protégé of Prof. Samuel
Huntington, then the director of Harvard's Center for International
Affairs. Huntington had worked in the National Security Council under
President Jimmy Carter, but he is now best known for his dire warnings
about an inevitable ``clash of civilizations'' between various regional
and religious groups, including Islam and the West. Under Huntington's
guidance, Kobach wrote his senior thesis on how the movement to divest
from South Africa was misguided because international businesses were
already leading the way against apartheid. (Huntington, who had advised
South Africa's government, argued that a transition away from white
minority rule might require a period of ``enlightened despotism.'')

Kobach says Huntington ``touched on a lot of themes I've worked on with
immigration law,'' but he distances himself from some of Huntington's
more radical ideas. Two of those ideas, however, have played an
important role in the direction of the larger reactionary movement that
Kobach leads.

The first was that broad-based participation in a democracy was not
always a good thing. ``Some of the problems of governance in the United
States today stem from an excess of democracy,'' Huntington wrote in a
1975 report called ``The Crisis of Democracy,'' and there are
``potentially desirable limits to the indefinite extension of political
democracy.''

Huntington warned of the dangers of expanding the franchise to
previously disenfranchised and marginalized groups of voters. ``In
itself, this marginality on the part of some groups is inherently
undemocratic, but it has also been one of the factors which has enabled
democracy to function effectively,'' Huntington wrote. ``Marginal social
groups, as in the case of the blacks, are now becoming full participants
in the political system. Yet the danger of overloading the political
system with demands which extend its functions and undermine its
authority still remains.''

The second idea was that the changing demographics of the United States
would lead to a culture war between Anglo-Protestants and newer
immigrant groups, particularly Latinos. ``While Muslims pose the
immediate problem to Europe,'' Huntington wrote in his 1996 book ``The
Clash of Civilizations,'' ``Mexicans pose the problem for the United
States.''

He expanded on this view in his 2004 book ``Who Are We? The Challenges
to America's National Identity,'' denouncing the ``Hispanization'' of
the United States and claiming that many Mexican-American immigrants
``do not appear to identify primarily with the United States'' and were
``often contemptuous of American culture.''

Huntington's central thesis was that the country's ``Anglo-Protestant
culture'' was under siege: He warned that ``the large and continuing
influx of Hispanics threatens the pre-eminence of white Anglo-Protestant
culture and the place of English as the only national language. White
nativist movements are a possible and plausible response to these
trends.'' Five years later, in an essay in Foreign Policy, he amplified
the point: ``Demographically, socially and culturally, the reconquista
(reconquest) of the Southwest United States by Mexican immigrants is
well underway.''

Kobach enrolled at Yale Law School in 1992. In his final year,
California voters approved Prop. 187, a sweeping law, also known as the
Save Our State initiative, that for the first time restricted public
benefits, including education and health care, for undocumented
immigrants. The federal courts ultimately blocked the law, on the
grounds that California was overstepping federal immigration authority,
and it is now largely remembered as a political debacle. California had
been predominately Republican for decades, but a backlash from the
state's growing Hispanic population pushed Gov. Pete Wilson out of
office and flipped the state from more-or-less red to permanently blue.

Kobach says that it was not Huntington so much as Prop. 187 that sparked
his interest in immigration law. ``It was not popular at Yale Law
School, but I defended it,'' he said. ``It just struck me as obvious
that a state has the right to restrict its welfare benefits only to
those people who are U.S. citizens or are visiting the state legally.''

Jed Shugerman, a legal historian at Fordham Law School, attended a
debate at Yale as an undergraduate in which Kobach defended Prop. 187.
``While the other pro-187 debaters were careful to distinguish between
the `legal' and `illegal' process, Kobach struck me even then as far
more xenophobic than other Yale conservatives,'' Shugerman wrote on his
personal blog in May. ``His image at that moment is seared into my
memory, because I remember thinking, This dude is really smart and
really scary. Remember his name, because he'll be back with a
vengeance.''

Image

Bills that Kobach helped write or helped get passed in the State
Legislature.Credit...Jonno Rattman for The New York Times

\textbf{In 2001, Kobach} took a leave of absence from his job as a law
professor at the University of Missouri-Kansas City to become a White
House fellow in the George W. Bush administration. He was assigned to
the Justice Department a week before Sept. 11. While much of the
national-security establishment regarded the attacks as an intelligence
failure, Kobach viewed them as a failure of border security. Mark
Johnson, a partner at Dentons law firm in Kansas City who has known
Kobach for 25 years, says, ``It radicalized him on the issue of
immigration.''

Kobach grew close to Attorney General John Ashcroft, and when the
fellowship ended a year later, he stayed on as his chief adviser on
immigration and border-security issues. One of his first tasks was to
implement the National Security Entry-Exit Registration System, or
Nseers, a program he designed that required all male visa holders over
the age of 16 from 24 predominantly Muslim countries (and North Korea)
to be fingerprinted, photographed and interviewed by immigration
authorities. The program was controversial inside and outside the
government. The A.C.L.U. said in a statement that it ``mandated ethnic
profiling on a scale not seen in the United States since
Japanese-American internment during World War II and the `Operation
Wetback' deportations to Mexico of 1954.'' The Obama administration
halted the program in 2011. Nseers did not result in a single known
conviction on terrorism charges, but it did result in deportation
proceedings for nearly 14,000 Muslim men, many for minor immigration
violations. Today Kobach recalls it as a ``great success.''

In 2003, Kobach returned to Kansas to challenge Dennis Moore, a
Democrat, for his seat in Congress. The following year, he also
represented students in a lawsuit sponsored by the Federation for
American Immigration Reform, or FAIR, a far-right advocacy group, in a
lawsuit challenging a provision that allowed public universities to
charge undocumented residents of Kansas in-state tuition rates. (The
suit was unsuccessful.)

FAIR was founded in 1978 by John Tanton, an ophthalmologist in rural
Michigan. Tanton was initially concerned about how human population
growth was harming the environment, but increasingly embraced nativist
arguments that demonized all kinds of immigration, illegal and legal. He
was especially struck by a brazenly racist 1973 novel called ``The Camp
of the Saints,'' by a French author, Jean Raspail, depicting ``the end
of the white world'' after a fleet of savage refugees, led by an Indian
called ``the turd eater,'' overwhelm Europe. Tanton republished the book
in English, and it attracted some influential American readers,
including Steve Bannon, who has cited the book frequently.

Tanton argued that white people needed to take action against the
country's demographic changes. Tanton outlined his concerns in a 1986
memo, now available from The Southern Poverty Law Center, which labeled
FAIR a ``hate group.'' ``Will Latin American migrants bring with them
the tradition of the mordida (bribe), the lack of involvement in public
affairs, etc.?'' he asked in the memo addressed to colleagues at a
retreat of anti-immigration activists in 1986. ``As Whites see their
power and control over their lives declining, will they simply go
quietly into the night? Or will there be an explosion?'' The Los Angeles
Times studied FAIR's tax returns and found that it had received at least
\$600,000 in grants from the Pioneer Fund, a nonprofit foundation that
subsidizes research that claims to prove blacks and other minorities are
genetically inferior to whites.

Kobach's connection to Tanton --- in addition to representing FAIR in
court, he received contributions totaling \$10,000 from a
political-action committee run by Tanton's wife --- became an issue in
his congressional run. ``People and groups tied to white supremacists
gave Kobach thousands,'' said a TV ad run by Moore. ``One even hired
Kobach.'' But Kobach refused to return the donations or disavow Tanton
or FAIR. Instead, he made opposition to undocumented immigration the
centerpiece of his campaign, criticizing Moore for supporting what
Kobach described as ``amnesty'' and calling for the National Guard to
patrol the Mexican border.

Kobach lost the race by 11 points but earned national headlines for his
outspoken nativism. ``I want to just applaud you for your courage,''
Bill O'Reilly told him that year, during Kobach's first of many
appearances on O'Reilly's show. ``You're the first former administration
official to come up and really tell the folks what's going on.'' Kobach
became counsel to the Immigration Reform Law Institute, the legal arm of
FAIR, and began drafting a series of ordinances for cities around the
country, preventing landlords from knowingly renting to undocumented
immigrants or employers from hiring them. Most of the laws were defeated
in court because the federal government had the exclusive power to
enforce immigration laws. But Kobach's co-counsel, Michael Hethmon,
recognized their real purpose. He told the filmmakers of the 2009
documentary ``9500 Liberty,'' that the effort to institute one of
Kobach's model ordinances in Prince William County, Virginia, might best
be understood as ``a field study.''

\textbf{In 2006, Kobach} received a call from the Maricopa County
Attorney's office in Phoenix. Andrew Thomas, the county attorney, wanted
Kobach to defend his interpretation of the state's ``coyote law,'' which
in his view should allow undocumented immigrants to be charged as
co-conspirators when they were caught illegally crossing the border.
Kobach agreed.

Even as he remained active in his own state's politics, serving as
chairman of the Kansas Republican Party from 2007 to 2009, Kobach began
spending more time in Arizona. He struck up a friendship with Joe
Arpaio, the Maricopa County sheriff, who dubbed himself ``America's
toughest sheriff.'' Arpaio, one of the first local sheriffs who took it
upon himself to enforce federal immigration law, was also a flamboyantly
authoritarian figure who drew national attention for requiring his
inmates to wear pink underwear, work on chain gangs and live outdoors in
tents where temperatures reached 130 degrees. At the Justice Department,
Kobach had promoted an effort to deputize local police departments with
immigration-enforcement authority from Immigration and Customs
Enforcement. In 2007, Arpaio received such a deputization, and his
office within two years had arrested 33,000 undocumented immigrants,
many of them in highly publicized ``crime suppression'' sweeps.

In 2009, after Barack Obama took office, the Department of Homeland
Security rescinded Arpaio's immigration-enforcement powers. That same
year, the Justice Department began an investigation into Arpaio's
``discriminatory police practices and unconstitutional searches and
seizures.'' Not long after that, Arpaio hired Kobach to train all of his
deputies on how to comply with federal immigration law. ``I really want
to applaud what Maricopa County is doing,'' Kobach said in a video for
the trainings, calling the county a model for the nation. Despite the
federal government's jurisdiction over immigration, Kobach told Arpaio's
deputies they had ``inherent authority'' to enforce immigration laws,
based on a 2002 memo Kobach had requested from the Justice Department.
He listed several of the dozens of federal crimes undocumented
immigrants could be arrested for, including ``failure to carry an alien
registration card'' and ``failure to notify the federal government of a
change of address.''

Kobach also helped State Senator Russell Pearce, the foremost opponent
of undocumented immigration in the State Legislature, draft SB 1070, a
2010 bill that required the Arizona police to ask for citizenship papers
from anyone they had ``reasonable suspicion'' of being in the state
illegally.

Kobach counseled Pearce on how to make the bill even more sweeping. In
an email to Pearce before the law's final passage, Kobach said that a
person's violation of ``any county or municipal ordinance'' could lead
to an immigration query: ``This will allow police to use violations of
property codes (i.e., cars on blocks in the yard) or rental codes (too
many occupants of a rental accommodation) to initiate queries as well.''
After filing a lawsuit against SB 1070, the A.C.L.U. referred to it on
its website as the Show Me Your Papers law.

Arizona became the first state to act on another of Kobach's theories:
attrition through enforcement. Make life miserable enough for
immigrants, and they will leave of their own volition. As Pearce told
The Arizona Republic newspaper, ``Disneyland taught us that if you shut
down the rides, people leave the amusement park.'' Mitt Romney was
widely mocked when he used the word ``self-deportation'' during the 2012
election, but that was exactly what Kobach was trying to achieve in
Arizona.

The Arizona experiment didn't end well for the state or its principal
actors. The interpretation of the ``coyote law'' that Kobach came to
Arizona to defend was blocked in 2013. The Supreme Court struck down
three of four sections of SB 1070 and narrowed enforcement of the ``show
me your papers'' provision. The Justice Department sued Arpaio in 2012,
and the following year a federal court ruled that his immigration stops
violated federal law, including the Civil Rights Act of 1964, for
discriminating against Latinos. Arpaio was ultimately charged with
criminal contempt of court; he failed to be re-elected in November 2016.
Andrew Thomas, the county attorney, was disbarred by the Arizona Supreme
Court in 2012 for what it called an ``unholy collaboration'' with
Arpaio. Russell Pearce was recalled from the Senate in 2011 and then
resigned as vice chairman from the Arizona Republican Party after saying
on his radio show that Medicaid recipients should be sterilized, which
led to a public outcry.

But Kobach continued to thrive. In 2010, the same year SB 1070 passed in
Arizona, he ran for secretary of state in Kansas. ``My hope is that
Kansas will be to stopping election fraud what Arizona is to stopping
illegal immigration,'' he told The Kansas City Star. The position of
secretary of state was not an especially glamorous one, but it offered
an enormous amount of power by virtue of its authority to enforce state
voting laws, particularly as American elections were being decided by
increasingly narrow margins. During the 2000 election in Florida and the
2004 election in Ohio, Republican secretaries of state were at the
center of hotly disputed elections.

Kobach had not been a particularly popular figure in Kansas. When he was
chairman of the Kansas Republican Party, he introduced what he described
as a ``loyalty rule'' to expel moderate Republican party leaders, an
episode The Kansas City Star likened to the ``Kansas G.O.P.'s version of
Stalin's purges.'' But Kobach also had a growing constituency. ``He
should be running for president,'' Arpaio said when he came to Kansas to
campaign for Kobach, ``but we'll take secretary of state.''

\textbf{Just a few} days before Election Day in 2010, Kobach held a news
conference and announced that nearly 2,000 dead voters in the state were
still registered to vote. ``Every one of those 1,966 identities is an
opportunity for voter fraud waiting to happen,'' he said. Kobach singled
out one name, Alfred K. Brewer, who was born in 1900 and died in 1996,
but was in fact listed as having voted just that year. ``An Alfred K.
Brewer voted in the 2010 primary election,'' Kobach said. ``Is it the
same one? We are still trying to achieve confirmation of this, but it
certainly seems like a very real possibility.''

A reporter from The Wichita Eagle found Alfred K. Brewer very much
alive; he was in his front yard doing chores. ``I don't think this is
heaven, not when I'm raking leaves,'' Brewer said. He was 78 and had
been listed as being born in 1900 because Kansas didn't record birth
dates on voter registration forms back when he first registered. The
date of death was his father's, who had the same name but was born in
1904. Kobach could have avoided the embarrassing mix-up if he had called
Brewer before singling him out.

But Kansas voters, in the age of Obama, demonstrated an appetite for
Kobach's nativist brew of anti-immigration sentiment and voting
restrictions. He won handily and quickly set about turning a once-sleepy
office into a kind of laboratory for limiting access to the ballot.

In 2005, Kansas joined with three other Midwestern states in a regional
compact called the Interstate Voter Registration Crosscheck Program. The
program compared state records to find people registered to vote in more
than one place. On taking office, Kobach, recognizing the program's
potential, championed it to election officials around the country,
rapidly expanding its reach. The program now includes more than 30
states.

Crosscheck appeared to offer an appealing scientific certainty to the
hunt for fraud. But it could also be used to suppress the vote. The
program searches for double registrations using only voters' first and
last names and date of birth, and it generates thousands of false
matches --- John Smith in Kansas can easily be confused with John Smith
in Iowa. These false matches have in several instances led to people
being wrongly removed from voter rolls. In 2013, after Virginia joined
Crosscheck, and in the midst of a hotly contested governor's race, the
state board of elections sent counties a list of more than 57,000 voters
to purge because they were supposedly registered in other states. The
data was littered with errors: Lawrence Haake, then the registrar in
Chesterfield County, told The Richmond Times-Dispatch, ``We do need an
interstate checking mechanism, but I'm not real impressed with this
one.''

Crosscheck has led to outrageous headlines that make double voting seem
far more common than it is. In 2014, after North Carolina joined
Crosscheck, the head of the state board of elections reported that in
the 2012 general election, there were 35,750 voters in the state whose
first and last names and dates of birth matched those of individuals who
voted in the same election in a different state. Republican leaders of
the North Carolina Legislature called it ``alarming evidence of voter
fraud,'' and the conservative political strategist Dick Morris told Sean
Hannity on Fox News, ``It's the most important data I've read in a
year,'' adding that it was ``the first concrete evidence we've ever had
of massive voter fraud.'' But when North Carolina investigated the
numbers using additional data like the last four digits of voters'
Social Security numbers, eight cases of potential double voting were
referred to prosecutors and two people were convicted.

Some states, including Florida and Oregon, have withdrawn from
Crosscheck over concerns about its accuracy. In a 2016 paper,
researchers at Stanford, Harvard, Yale and the University of
Pennsylvania analyzed the lists of potential duplicate voter
registrations that Crosscheck sent to the state of Iowa before the 2012
and 2014 elections and found that ``200 legitimate voters may be impeded
from voting for every double vote stopped.''

Kobach's other major project was making the SAFE Act into a sustainable
model of election legislation, one that would stand up to scrutiny in
the courts. When it was made into law in April 2011, Kobach compared it
to the 19th Amendment, which gave women the right to vote. But the
A.C.L.U. immediately began to file legal challenges claiming that rather
than expand access to the ballot, the law was making it harder to vote.

One of the most significant challenges to the SAFE Act came from a
lawsuit in a different state. In June 2013, the Supreme Court ruled that
Arizona could not require proof of citizenship for those who registered
to vote using a federal registration form, which had the effect of
nullifying part of the SAFE Act. Justice Antonin Scalia, however, in
writing the majority opinion, noted that states like Arizona and Kansas
that wanted to implement proof-of-citizenship laws could petition the
Election Assistance Commission, which is a little-known federal agency
created after the 2000 presidential-election recount. Its approval, he
said, would be sufficient to make the laws constitutional. Kobach, who
filed just such a petition in 2012, promptly sent another request two
months later, but the acting executive director of the E.A.C. denied it.

A finalist for the permanent position of executive director at the
E.A.C. happened to be one of Kobach's own election commissioners in
Kansas, Brian Newby of Johnson County. Kobach was informed in April 2015
by staff at the Johnson County Election Office that Newby was being
audited for misusing county funds, but instead of admonishing Newby,
Kobach recommended him for a top federal job. The E.A.C., which was made
up of three commissioners, two of whom were Republicans, took Kobach's
advice, and Newby got the job in November 2015. Three months after Newby
took office, he unexpectedly changed the E.A.C.'s rules in Kobach's
favor.

The League of Women Voters sued the E.A.C. two weeks later. ``If the
Newby decision stands, then every state in the nation will be able to
require documentary proof of citizenship,'' the group's advocacy
director, Lloyd Leonard, told The New York Times. ``Citizenship
documents,'' like birth certificates and passports, are not things most
Americans carry around with them. That makes it impossible for groups
like the League of Women Voters to register voters at farmers markets or
public marches and demonstrations. When the SAFE Act went into effect,
eight of nine chapters of the Kansas League of Women Voters suspended
voter-registration activities; the Wichita chapter went from registering
4,000 voters in 2012 to just 465 in 2014.

The Obama Justice Department took the extraordinary step of refusing to
defend Newby's directive in federal court, so Kobach defended it
himself. The 10th Circuit Court of Appeals ruled against him, finding
that Newby, as executive director, did not have the authority to make
the decision without the consent of his commissioners.

\textbf{In 2015, Kobach} persuaded the Kansas Legislature to make him
the only secretary of state in the country with the power to
independently prosecute voter-fraud cases. He told The Kansas City Star
that this was ``the final piece in the puzzle in terms of preventing
voter fraud.'' Betty and Steven Gaedtke were two of the first people
Kobach charged. After retiring, the Gaedtkes left Olathe, Kan., to build
their dream house in the woods of the Arkansas Ozarks. Betty is a member
of the Quapaw Tribe of Oklahoma, which was based in Arkansas before
being forcefully relocated to Oklahoma in the 1800s, and she felt as if
she were returning home. She was elected to the tribal council and
became an advocate against sexual assault. ``She's very, very
civic-minded,'' her lawyer, Trey Pettlon, said.

Betty became an Arkansas resident and voted there in 2010. Her husband
moved down after her and, before he left, filled out absentee ballots
for each of them in Kansas. Then he settled in Arkansas before the 2010
election and voted there too, believing he had lawfully established
residency. County attorneys in Kansas declined to prosecute the
Gaedtkes, seeing the double voting as an honest mistake. But in October
2015, a month before the five-year statute of limitations expired,
Kobach charged them each with three misdemeanor counts of ``voting
without being qualified.'' The evidence was ``very strong that the
individuals in question intentionally voted multiple times in the same
election,'' he said.

``If I was convicted of that, I would've had to step down from my
tribe,'' Betty recalls. ``The whole experience was such a nightmare.''
Five days before the trial was set to begin, Kobach's office dropped the
charges against Betty. Steven pleaded guilty to one of the misdemeanors
and received a \$500 fine. ``I didn't even get to tell him: `This is
what I look like. I'm a good person,' '' Betty says of Kobach. ``I feel
like I was just a pawn for him.''

Though Kobach received the authority to prosecute fraud cases after
warning that voting by ``aliens'' was rampant, the nine convictions he
has won since 2015 have primarily been citizens 60 and over who own
property in two states and were confused about voting requirements. Only
one noncitizen has been convicted. A state representative, John
Carmichael, a Democrat from Wichita, told me these cases were ``show
trials to try and justify his prosecutorial authority,'' and he has
introduced a bill to repeal Kobach's prosecutorial power.

While Kobach searched for fraud cases, his SAFE Act had blocked the
registrations of 35,000 Kansans by September 2015. Then Kobach started
removing anyone from the rolls who didn't provide citizenship documents
within 90 days. ``It's no big deal,'' he told Fox News. ``Nobody's being
disenfranchised.'' In February 2016, the A.C.L.U. sued Kobach on behalf
of more than 18,000 Kansas voters who had unsuccessfully tried to
register at the Department of Motor Vehicles. A federal court found that
the SAFE Act violated the 1993 National Voter Registration Act, which
allowed voters to register at many government agencies. In response,
Kobach had an administrative rule passed which said that any Kansan who
registered at the D.M.V. but didn't show proof of citizenship could vote
in federal but not state elections. In July 2016, while Kobach was at
the Republican National Convention helping to draft the G.O.P. platform,
the A.C.L.U. sued him again. ``It seemed bonkers that someone would be
able to vote for president but not school board or City Council or
secretary of state,'' said Dale Ho, director of the A.C.L.U.'s Voting
Rights Project.

Marvin Brown, a 91-year-old World War II veteran, became the lead
plaintiff. Brown was the first person Kobach ever met who had paid a
poll tax. He paid \$2 to register on his 21st birthday in Arkansas in
1946, after returning from flying bomber planes over Germany during
World War II. ``I learned in civics it was your reasonable and honorable
duty to vote,'' he told me. He added that Sevier County was deciding
whether to allow alcohol sales and ``the main reason I registered was
cause they were voting for the sale of beer.''

Brown moved to Kansas in 1948 and worked as an electrician, ran a marina
in Arkansas and then moved back to the Kansas City suburbs to be closer
to his family. In 2015, he went down to the county government building
with his wife to register to vote in Kansas. ``We did everything we did
before,'' he said. ``Then we got this precious letter that said you have
to prove your citizenship. I got a little upset.'' Brown's ancestors had
fought for the Union in the Civil War and settled in Kansas afterward.
He flew so many bombing missions in World War II that the Air Force lost
count. ``I grew up in this country,'' he said. ``I'm 91 years old, and
this son of a buck is telling me I might not be a citizen. I told
Kobach, `That hurts me inside real deep.' ''

In court, Kobach questioned Brown's citizenship and said he didn't have
standing to sue. ``At this point, we don't even know that these
individuals are citizens,'' he said. ``We know that they are asserting
that Mr. Brown fought in the war and, of course, even that doesn't prove
your citizenship.'' A state court struck down the two-tiered election
system 10 days after the case was filed. ``The number of noncitizen
registrations are minuscule,'' the judge wrote, ``compared to the number
of voters that potentially will be unable to vote.''

\textbf{``I won the} popular vote if you deduct the millions of people
who voted illegally,'' Donald Trump tweeted on Nov. 27. When asked in an
ABC interview where Trump got that information, the president-elect's
adviser Kellyanne Conway named Kobach as a source of the claim. Three
days later, after Kobach certified the results of the 2016 election in
Kansas at the Capitol in Topeka, he told reporters, ``I think the
president-elect is absolutely correct when he says the number of illegal
votes cast exceeds the popular-vote margin between him and Hillary
Clinton.''

As evidence, Kobach pointed to a 2014 study whose lead author was an Old
Dominion University political scientist, Jesse Richman. It estimated
that ``6.4 percent of noncitizens voted in 2008.'' That finding was
quickly picked up by Breitbart (``Study: Voting by Non-Citizens Tips
Balance for Democrats'') and National Review (``Jaw-Dropping Study
Claims Large Numbers of Non-Citizens Vote in U.S.'') and was also cited
directly by Trump on the campaign trail.

Yet Richman's study was soon contested by other political scientists.
Richman had found 489 noncitizens in a much larger 2010 Harvard survey
of 55,400 American adults called the Cooperative Congressional Election
Study. In 2012, three political scientists who coordinated the original
C.C.E.S. study went back and re-interviewed 19,000 of the respondents.
They found only 85 who said they were noncitizens in the survey --- and
none of them could be matched to a valid voting record. ``Thus the best
estimate of the percentage of noncitizens who vote is zero,'' they
wrote.

In January 2017, nearly 200 leading political scientists signed an open
letter criticizing Richman's study. Kobach nevertheless recently
retained Richman as an expert witness in his ongoing battle with the
A.C.L.U., and Richman produced another eye-popping claim: 18,000
noncitizens were registered to vote in Kansas. To reach that number,
Richman identified 37 noncitizens on a list of temporary driver's
licenses in Kansas and found six who, he wrote in an expert report that
Kobach filed in court, ``had either registered to vote or attempted to
register to vote.'' He then divided those six people, representing 16
percent of a total of 37 people, by Kansas's estimated noncitizen
population of 114,000 and concluded that ``a very substantial number and
portion of noncitizens in Kansas have registered to vote or attempted to
register to vote --- more than 18,000.''

Brian Schaffner, a professor of political science at the University of
Massachusetts-Amherst who helped conduct the original C.C.E.S. study,
said that going from six people who may have registered to 18,000
noncitizens actually registering or trying to register was a huge leap.
``We don't know that any of them actually registered,'' Schaffner told
me. ``None of them are matched to a valid vote record.'' When Kobach
told the Kansas Legislature in February that ``18,000 aliens may be on
the Kansas voting rolls,'' the gallery erupted in laughter. Kobach threw
up his hands, looked back directly at the chamber and said, ``You can
perhaps do your own statistical analysis and submit it to the court.''

Kobach's chilling narrative of deceitful foreigners subverting democracy
has served him well. Making people believe that voter fraud is rampant
builds public support for policies that restrict access to the ballot.
And claims of illegal voting by noncitizens help justify Kobach's
hard-line anti-immigration agenda. This has given Kobach a powerful
political constituency, not least of which is the president himself. The
story Kobach tells about voter fraud is what persuaded Trump to create a
presidential commission on ``election integrity'' and name Kobach its
vice chairman. ``He's stated his own view publicly, which is consistent
with what he's told me privately,'' Kobach says of Trump's views on
voter fraud. ``He believes that it's a significant problem.''

The Trump commission marks a major step forward in Kobach's efforts to
nationalize his restrictions on voting. He'll have a presidential bully
pulpit and access to government resources that weren't previously
available, such as a nationwide database that includes noncitizens that
could be run against state voter rolls to generate new allegations. But
that Systematic Alien Verification for Entitlements database does not
automatically reveal the status of immigrants who become U.S. citizens,
which means thousands of noncitizens who are subsequently naturalized
could mistakenly be tagged as illegal voters. The commission will also
make policy recommendations at the federal and state level, which could
include support for suppressive policies like strict voter-ID laws and
voter-rolls purges.

Kobach says the National Voter Registration Act and the Voting Rights
Act, the country's cornerstone voting-rights laws, are being
misinterpreted. ``The N.V.R.A. has been abused by organizations like the
A.C.L.U.,'' Kobach told me. ``They've twisted the words to try and say
it prevents proof-of-citizenship laws.'' The Voting Rights Act is also
``being abused by the A.C.L.U.,'' he says. ``Now they're trying to
attack photo-ID laws using the Voting Rights Act by claiming, using very
flimsy evidence, that photo-ID laws disproportionately affect minority
populations more than others.'' Kobach wants proof-of-citizenship laws
to be adopted in every state.

In 2006, when he was still a law professor, Kobach spoke at a
candlelight gathering to oppose federal immigration reform, billed as a
Vigil to Save the American Worker, in Kansas City. The event was
sparsely attended, but Kobach spoke pessimistically to those who had
come with passion. He cited a line often attributed to Winston
Churchill. ``He said that his definition of a fanatic is `someone who
can't change his mind and won't change the subject,' '' Kobach said,
standing in the candlelight. ``And friends, if that's what a fanatic is,
then I guess I'm a fanatic. Because, when it comes to restoring the rule
of law, I can't change my mind and I won't change the subject.''

Advertisement

\protect\hyperlink{after-bottom}{Continue reading the main story}

\hypertarget{site-index}{%
\subsection{Site Index}\label{site-index}}

\hypertarget{site-information-navigation}{%
\subsection{Site Information
Navigation}\label{site-information-navigation}}

\begin{itemize}
\tightlist
\item
  \href{https://help.nytimes3xbfgragh.onion/hc/en-us/articles/115014792127-Copyright-notice}{©~2020~The
  New York Times Company}
\end{itemize}

\begin{itemize}
\tightlist
\item
  \href{https://www.nytco.com/}{NYTCo}
\item
  \href{https://help.nytimes3xbfgragh.onion/hc/en-us/articles/115015385887-Contact-Us}{Contact
  Us}
\item
  \href{https://www.nytco.com/careers/}{Work with us}
\item
  \href{https://nytmediakit.com/}{Advertise}
\item
  \href{http://www.tbrandstudio.com/}{T Brand Studio}
\item
  \href{https://www.nytimes3xbfgragh.onion/privacy/cookie-policy\#how-do-i-manage-trackers}{Your
  Ad Choices}
\item
  \href{https://www.nytimes3xbfgragh.onion/privacy}{Privacy}
\item
  \href{https://help.nytimes3xbfgragh.onion/hc/en-us/articles/115014893428-Terms-of-service}{Terms
  of Service}
\item
  \href{https://help.nytimes3xbfgragh.onion/hc/en-us/articles/115014893968-Terms-of-sale}{Terms
  of Sale}
\item
  \href{https://spiderbites.nytimes3xbfgragh.onion}{Site Map}
\item
  \href{https://help.nytimes3xbfgragh.onion/hc/en-us}{Help}
\item
  \href{https://www.nytimes3xbfgragh.onion/subscription?campaignId=37WXW}{Subscriptions}
\end{itemize}
