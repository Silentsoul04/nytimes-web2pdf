Sections

SEARCH

\protect\hyperlink{site-content}{Skip to
content}\protect\hyperlink{site-index}{Skip to site index}

\href{https://www.nytimes3xbfgragh.onion/section/politics}{Politics}

\href{https://myaccount.nytimes3xbfgragh.onion/auth/login?response_type=cookie\&client_id=vi}{}

\href{https://www.nytimes3xbfgragh.onion/section/todayspaper}{Today's
Paper}

\href{/section/politics}{Politics}\textbar{}Trump Team Met With Lawyer
Linked to Kremlin During Campaign

\url{https://nyti.ms/2uWPOMw}

\begin{itemize}
\item
\item
\item
\item
\item
\item
\end{itemize}

Advertisement

\protect\hyperlink{after-top}{Continue reading the main story}

Supported by

\protect\hyperlink{after-sponsor}{Continue reading the main story}

\hypertarget{trump-team-met-with-lawyer-linked-to-kremlin-during-campaign}{%
\section{Trump Team Met With Lawyer Linked to Kremlin During
Campaign}\label{trump-team-met-with-lawyer-linked-to-kremlin-during-campaign}}

\includegraphics{https://static01.graylady3jvrrxbe.onion/images/2017/07/08/us/08TRUMP1/08TRUMP1-articleInline.jpg?quality=75\&auto=webp\&disable=upscale}

By \href{http://www.nytimes3xbfgragh.onion/by/jo-becker}{Jo Becker},
\href{http://www.nytimes3xbfgragh.onion/by/matt-apuzzo}{Matt Apuzzo} and
\href{https://www.nytimes3xbfgragh.onion/by/adam-goldman}{Adam Goldman}

\begin{itemize}
\item
  July 8, 2017
\item
  \begin{itemize}
  \item
  \item
  \item
  \item
  \item
  \item
  \end{itemize}
\end{itemize}

\emph{Update: The Times is now reporting that Donald Trump Jr. was
promised damaging information about Hillary Clinton before the meeting.}
\href{https://www.nytimes3xbfgragh.onion/2017/07/09/us/politics/trump-russia-kushner-manafort.html?hp\&action=click\&pgtype=Homepage\&clickSource=story-heading\&module=first-column-region\&region=top-news\&WT.nav=top-news}{\emph{Read
the story}}\emph{.}

Two weeks after Donald J. Trump clinched the Republican presidential
nomination last year, his eldest son arranged a meeting at Trump Tower
in Manhattan with a Russian lawyer who has connections to the Kremlin,
according to confidential government records described to The New York
Times.

The previously unreported meeting was also attended by Mr. Trump's
campaign chairman at the time, Paul J. Manafort, as well as the
president's son-in-law, Jared Kushner, according to interviews and the
documents, which were outlined by people familiar with them.

While President Trump has been dogged by revelations of undisclosed
meetings between his associates and Russians, this episode at Trump
Tower on June 9, 2016, is the first confirmed private meeting between a
Russian national and members of Mr. Trump's inner circle during the
campaign. It is also the first time that his son Donald Trump Jr. is
known to have been involved in such a meeting.

Representatives of Donald Trump Jr. and Mr. Kushner confirmed the
meeting after The Times approached them with information about it. In a
statement, Donald Jr. described the meeting as primarily about an
adoption program. The statement did not address whether the presidential
campaign was discussed.

\href{https://www.nytimes3xbfgragh.onion/2017/01/06/us/politics/russia-hack-report.html}{American
intelligence agencies have concluded} that Russian hackers and
propagandists worked to tip the election toward Mr. Trump, and a special
prosecutor and congressional committees are now investigating whether
his campaign associates colluded with Russians. Mr. Trump has disputed
that, but the investigation has cast a shadow over his administration
for months.

Image

Natalia Veselnitskaya

Mr. Trump has also equivocated on whether the Russians were solely
responsible for the hacking. But in Germany on Friday, meeting President
Vladimir V. Putin for the first time as president, Mr. Trump
\href{https://www.nytimes3xbfgragh.onion/2017/07/07/world/europe/trump-putin-g20.html}{questioned
him about the hacking}. The Russian leader denied meddling in the
election.

The Russian lawyer invited to the Trump Tower meeting, Natalia
Veselnitskaya, is best known for mounting a multipronged attack against
the Magnitsky Act, an American law that blacklists suspected Russian
human rights abusers. The law so enraged Mr. Putin that he retaliated by
halting American adoptions of Russian children.

The adoption impasse is a frequently used talking point for opponents of
the Magnitsky Act. Ms. Veselnitskaya's campaign against the law has also
included attempts to discredit its namesake, Sergei L. Magnitsky, a
lawyer and auditor who died in mysterious circumstances in a Russian
prison in 2009 after exposing one of the biggest corruption scandals
during Mr. Putin's rule.

Ms. Veselnitskaya was formerly married to a former deputy transportation
minister of the Moscow region, and her clients include state-owned
businesses and a senior government official's son, whose company was
under investigation in the United States at the time of the meeting. Her
activities and associations had previously drawn the attention of the
F.B.I., according to a former senior law enforcement official.

In his statement, Donald Trump Jr. said: ``It was a short introductory
meeting. I asked Jared and Paul to stop by. We primarily discussed a
program about the adoption of Russian children that was active and
popular with American families years ago and was since ended by the
Russian government, but it was not a campaign issue at the time and
there was no follow up.''

He added: ``I was asked to attend the meeting by an acquaintance, but
was not told the name of the person I would be meeting with
beforehand.''

Late Saturday, Mark Corallo, a spokesman for the president's lawyer,
issued a statement implying that the meeting was a setup. Ms.
Veselnitskaya and the translator who accompanied her to the meeting
``misrepresented who they were,'' it said.

\includegraphics{https://static01.graylady3jvrrxbe.onion/images/2017/07/08/us/08TRUMP3/08TRUMP3-articleInline.jpg?quality=75\&auto=webp\&disable=upscale}

In an interview, Mr. Corallo explained that Ms. Veselnitskaya, in her
anti-Magnitsky campaign, employs a private investigator whose firm,
Fusion GPS,
\href{https://www.nytimes3xbfgragh.onion/2017/01/11/us/politics/donald-trump-russia-intelligence.html}{produced
an intelligence dossier} that contained
\href{https://www.nytimes3xbfgragh.onion/2017/01/11/us/politics/trump-intelligence-report-explainer.html}{unproven
allegations} against the president. In a statement, the firm said,
``Fusion GPS learned about this meeting from news reports and had no
prior knowledge of it. Any claim that Fusion GPS arranged or facilitated
this meeting in any way is false.''

Donald Trump Jr. had denied participating in any campaign-related
meetings with Russian nationals when he was interviewed by The Times in
March. ``Did I meet with people that were Russian? I'm sure, I'm sure I
did,'' he said. ``But none that were set up. None that I can think of at
the moment. And certainly none that I was representing the campaign in
any way, shape or form.''

Asked at that time whether he had ever discussed government policies
related to Russia, the younger Mr. Trump replied, ``A hundred percent
no.''

The Trump Tower meeting was not disclosed to government officials until
recently, when Mr. Kushner, who is also a senior White House aide, filed
a revised version of a form required to obtain a security clearance. The
Times
\href{https://www.nytimes3xbfgragh.onion/2017/04/06/us/politics/jared-kushner-russians-security-clearance.html}{reported
in April} that he had failed to disclose any foreign contacts, including
meetings with the Russian ambassador to the United States and the head
of a Russian state bank. Failure to report such contacts can result in a
loss of access to classified information and even, if information is
knowingly falsified or concealed, in imprisonment.

Mr. Kushner's advisers said at the time that the omissions were an
error, and that he had immediately notified the F.B.I. that he would be
revising the filing. They also said he had met with the Russians in his
official transition capacity as a main point of contact for foreign
officials.

In a statement on Saturday, Mr. Kushner's lawyer, Jamie Gorelick, said:
``He has since submitted this information, including that during the
campaign and transition, he had over 100 calls or meetings with
representatives of more than 20 countries, most of which were during
transition. Mr. Kushner has submitted additional updates and included,
out of an abundance of caution, this meeting with a Russian person,
which he briefly attended at the request of his brother-in-law Donald
Trump Jr. As Mr. Kushner has consistently stated, he is eager to
cooperate and share what he knows.''

Mr. Kushner's lawyers addressed questions about his disclosure but
deferred to Donald Trump Jr. on questions about the meeting itself.

Mr. Manafort, the former campaign chairman, also recently disclosed the
meeting, and Donald Trump Jr.'s role in organizing it, to congressional
investigators who had questions about his foreign contacts, according to
people familiar with the events.

A spokesman for Mr. Manafort declined to comment. In response to
questions, Ms. Veselnitskaya said the meeting lasted about 30 minutes
and focused on the Magnitsky Act and the adoption issue.

``Nothing at all was discussed about the presidential campaign,'' she
said, adding, ``I have never acted on behalf of the Russian government
and have never discussed any of these matters with any representative of
the Russian government.''

Because Donald Trump Jr. does not serve in the administration and does
not have a security clearance, he was not required to disclose his
foreign contacts. Federal and congressional investigators have not
publicly asked for any records that would require his disclosure of
Russian contacts. It is not clear whether the Justice Department was
aware of the meeting before Mr. Kushner disclosed it recently. Neither
Mr. Kushner nor Mr. Manafort was required to disclose the content of the
meeting in their government filings.

During the campaign, Donald Trump Jr. served as a close adviser to his
father, frequently appearing at campaign events. Since the president
took office, the younger Mr. Trump and his brother, who have worked for
the Trump Organization for most of their adult lives, assumed day-to-day
control of their father's real estate empire.

A quick internet search reveals Ms. Veselnitskaya as a formidable
operator with a history of pushing the Kremlin's agenda. Most notable is
her campaign against the Magnitsky Act, which provoked a Cold War-style,
tit-for-tat row with the Kremlin when President Barack Obama signed it
into law in 2012.

Under the law, some 44 Russian citizens have been put on a list that
allows the United States to seize their American assets and deny them
visas. The United States asserts that many of them are connected to
fraud exposed by Mr. Magnitsky, who after being jailed for more than a
year was found dead in his cell. A Russian human rights panel found that
he had been assaulted. To critics of Mr. Putin, Mr. Magnitsky, in death,
became a symbol of corruption and brutality in the Russian state.

An infuriated Mr. Putin has called the law an ``outrageous act,'' and,
in addition to banning American adoptions, compiled what became known as
an ``anti-Magnitsky'' blacklist of United States citizens.

Among those blacklisted was Preet Bharara, then the United States
attorney in Manhattan, who led high-profile convictions of Russian arms
and drug dealers. Mr. Bharara was
\href{https://www.nytimes3xbfgragh.onion/2017/03/11/us/politics/preet-bharara-us-attorney.html}{abruptly
fired} in March, after previously being asked to stay on by Mr. Trump.

One of Ms. Veselnitskaya's clients is Denis Katsyv, the Russian owner of
a Cyprus-based investment company called Prevezon Holdings. He is the
son of Petr Katsyv, the vice president of the state-owned Russian
Railways and a former deputy governor of the Moscow region. In a civil
forfeiture case prosecuted by Mr. Bharara's office, the Justice
Department alleged that Prevezon had helped launder money tied to a
\$230 million corruption scheme exposed by Mr. Magnitsky by parking it
in New York real estate and bank accounts. As a result, the government
froze \$14 million of its assets. Prevezon recently settled the case for
\$6 million without admitting wrongdoing.

Ms. Veselnitskaya and her client hired a team of political and legal
operatives that has worked unsuccessfully in Washington to repeal the
Magnitsky Act. They also tried but failed to keep Mr. Magnitsky's name
off a new law that takes aim at human-rights abusers across the globe.

Besides the private investigator whose firm produced the Trump dossier,
the lobbying team included Rinat Akhmetshin, an émigré to the United
States who once served as a Soviet military officer and who has been
called a Russian political gun for hire.

Ms. Veselnitskaya was also deeply involved in the making of
\href{https://www.nytimes3xbfgragh.onion/2016/06/10/world/europe/sergei-magnitsky-russia-vladimir-putin.html}{an
anti-Magnitsky film that} premiered just weeks before the Trump Tower
meeting. Titled ``The Magnitsky Act --- Behind the Scenes,'' the film
echoes the Kremlin line that the widely accepted version of Mr.
Magnitsky's life and death is wrong. The film claims that he was not
assaulted and alleges that he never testified that government officials
conspired to steal \$230 million in fraudulent tax rebates.

In the film's telling, the true culprit of the fraud was William F.
Browder, an American-born financier who hired Mr. Magnitsky to
investigate the fraud after he had three of his investment funds
companies in Russia seized. On RussiaTV5, a station whose owners are
known to be close to Mr. Putin, Ms. Veselnitskaya was lauded as ``one of
those who gave the film crew the real proofs and records of testimony.''

Mr. Browder, who stopped the screening of the film in Europe by
threatening libel suits, called the film a state-sponsored smear
campaign.

``She's not just some private lawyer,'' Mr. Browder said of Ms.
Veselnitskaya. ``She is a tool of the Russian government.''

John O. Brennan, the former C.I.A. director, testified in May that he
had been concerned last year by Russian government efforts to contact
and manipulate members of Mr. Trump's campaign. ``Russian intelligence
agencies do not hesitate at all to use private companies and Russian
persons who are unaffiliated with the Russian government to support
their objectives,'' he said.

The
\href{https://www.nytimes3xbfgragh.onion/2017/03/20/us/politics/fbi-investigation-trump-russia-comey.html}{F.B.I.
began a counterintelligence investigation} last July into Russian
contacts with any Trump associates. Agents focused on Mr. Manafort and a
pair of advisers,
\href{https://www.nytimes3xbfgragh.onion/2017/04/19/us/politics/carter-page-russia-trump.html}{Carter
Page} and Roger J. Stone.

Among those now under investigation is Michael T. Flynn, who was forced
to resign as Mr. Trump's national security adviser after it became known
that he had falsely denied speaking to the Russian ambassador about
sanctions imposed by the Obama administration over the election hacking.

Congress later discovered that Mr. Flynn had been paid more than
\$65,000 by companies linked to Russia, and that he had failed to
disclose those payments when he renewed his security clearance and
underwent an additional background check to join the White House staff.

In May, the
\href{https://www.nytimes3xbfgragh.onion/2017/05/09/us/politics/james-comey-fired-fbi.html}{president
fired the F.B.I. director}, James B. Comey, who days later provided
information about a meeting with Mr. Trump at the White House. According
to Mr. Comey, the president asked him to end the bureau's investigation
into Mr. Flynn; Mr. Trump has repeatedly denied making such a request.
Robert S. Mueller III, a former F.B.I. director, was then appointed
\href{https://www.nytimes3xbfgragh.onion/2017/05/17/us/politics/robert-mueller-special-counsel-russia-investigation.html}{as
special counsel}.

The status of Mr. Mueller's investigation is not clear, but he has
assembled a veteran team of prosecutors and agents to dig into any
possible collusion.

Advertisement

\protect\hyperlink{after-bottom}{Continue reading the main story}

\hypertarget{site-index}{%
\subsection{Site Index}\label{site-index}}

\hypertarget{site-information-navigation}{%
\subsection{Site Information
Navigation}\label{site-information-navigation}}

\begin{itemize}
\tightlist
\item
  \href{https://help.nytimes3xbfgragh.onion/hc/en-us/articles/115014792127-Copyright-notice}{©~2020~The
  New York Times Company}
\end{itemize}

\begin{itemize}
\tightlist
\item
  \href{https://www.nytco.com/}{NYTCo}
\item
  \href{https://help.nytimes3xbfgragh.onion/hc/en-us/articles/115015385887-Contact-Us}{Contact
  Us}
\item
  \href{https://www.nytco.com/careers/}{Work with us}
\item
  \href{https://nytmediakit.com/}{Advertise}
\item
  \href{http://www.tbrandstudio.com/}{T Brand Studio}
\item
  \href{https://www.nytimes3xbfgragh.onion/privacy/cookie-policy\#how-do-i-manage-trackers}{Your
  Ad Choices}
\item
  \href{https://www.nytimes3xbfgragh.onion/privacy}{Privacy}
\item
  \href{https://help.nytimes3xbfgragh.onion/hc/en-us/articles/115014893428-Terms-of-service}{Terms
  of Service}
\item
  \href{https://help.nytimes3xbfgragh.onion/hc/en-us/articles/115014893968-Terms-of-sale}{Terms
  of Sale}
\item
  \href{https://spiderbites.nytimes3xbfgragh.onion}{Site Map}
\item
  \href{https://help.nytimes3xbfgragh.onion/hc/en-us}{Help}
\item
  \href{https://www.nytimes3xbfgragh.onion/subscription?campaignId=37WXW}{Subscriptions}
\end{itemize}
