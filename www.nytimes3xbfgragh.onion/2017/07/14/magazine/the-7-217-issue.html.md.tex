Sections

SEARCH

\protect\hyperlink{site-content}{Skip to
content}\protect\hyperlink{site-index}{Skip to site index}

\href{https://myaccount.nytimes3xbfgragh.onion/auth/login?response_type=cookie\&client_id=vi}{}

\href{https://www.nytimes3xbfgragh.onion/section/todayspaper}{Today's
Paper}

The 7.2.17 Issue

\url{https://nyti.ms/2vjQCLP}

\begin{itemize}
\item
\item
\item
\item
\item
\end{itemize}

Advertisement

\protect\hyperlink{after-top}{Continue reading the main story}

Supported by

\protect\hyperlink{after-sponsor}{Continue reading the main story}

The Thread

\hypertarget{the-7217-issue}{%
\section{The 7.2.17 Issue}\label{the-7217-issue}}

\includegraphics{https://static01.graylady3jvrrxbe.onion/images/2017/07/16/magazine/16thread1/16thread1-articleLarge-v2.jpg?quality=75\&auto=webp\&disable=upscale}

July 14, 2017

\begin{itemize}
\item
\item
\item
\item
\item
\end{itemize}

\textbf{RE:}
\href{https://www.nytimes3xbfgragh.onion/2017/06/28/magazine/greetings-et-please-dont-murder-us.html}{\textbf{METI}}

\emph{Steven Johnson wrote about the methods --- and ethics --- behind
humanity's latest attempts to beam messages into space.}

\textbf{Steven Johnson's wonderful} piece detailing all the issues
surrounding interstellar/interplanetary contact certainly touched all
the relevant issues. I firmly support the SETI efforts to listen in on
the possibility of extraterrestrial life. But I also fully understand
how the METI program could lead to problems, even though cosmic
distances tend to mitigate those concerns. The idea I found most
interesting was the line: ``How can you send a message to a life-form
. . . that you know nothing at all about?'' There are quite a number of
life-forms here on Earth that are, demonstrably, quite intelligent and
about whose intellect and cognition we know very little. All my life I
have been intrigued by the possibility of extraterrestrial life. Still
am. But we could learn much about that possible communication by
learning to communicate with the ``alien'' species that share our world.
\emph{Carlos D. Martinez, Rego Park, N.Y.}

\includegraphics{https://static01.graylady3jvrrxbe.onion/images/2017/07/16/magazine/16thread2/16thread2-articleInline.jpg?quality=75\&auto=webp\&disable=upscale}

\textbf{We, as a society,} are not sufficiently mature to even address
any issue that might address our species' extinction.

We are not convinced of global warming, despite the evidence that it is
true. Darwin's views on evolution are, in the main, fact, regardless of
the refinement of the details; and prejudice within our species is
nearly genocidal with its impact. We are not eligible as a species or a
civilization to make decisions about possible extinction events or
issues. We need to shut up and get our social, political and scientific
houses in order before trying to invite unknown aliens to ``Come on
down, y'all.'' We might end up on the menu or in slave quarters. And
religious concerns should not even be on the table for any form of
consideration because there is not any factual evidence to support any
tenet of any religion: They must be taken on faith, and you must
believe, despite what the facts may belie concerning those beliefs.

The whole METI thing needs to be decided based on facts, of which we
have damned few, and taking a ``Wouldn't it be nice if. . .'' is hardly
a tactic designed to ensure our survival as a species. \emph{Jay Brown,
Seminole, Fla.}

\textbf{RE:}
\href{https://www.nytimes3xbfgragh.onion/2017/06/27/magazine/can-a-tech-start-up-successfully-educate-children-in-the-developing-world.html}{\textbf{BRIDGE}}

\emph{Peg Tyre wrote about a for-profit company's mission to educate
children in developing countries through a chain of inexpensive
schools.}

Image

Credit...Illustration by Giacomo Gambineri

\textbf{Peg Tyre's} ``The Bridge Effect'' dredged up many familiar
frustrations associated with corporate America's impulse to appropriate
culture and promote a hollow version of compassion. Being a
public-school teacher in Milwaukee, I have been able to observe this
movement from its genesis. It is an adherence to appearance that drives
the decision-making.

The essential flaw in for-profit education is the almost willful failure
to realize that it is not the act of providing ``customer service'' that
is at the core of education. It is the act of sustaining community ---
of compassionately communing with children and families in the context
of a culture --- that makes this effort succeed. This is the axiomatic
difference that explains why for-profit educational companies so often
fail. \emph{Jeff Cartier, Milwaukee}

Image

Credit...Illustration by Giacomo Gambineri

\textbf{I was struck} by how similar the story of Bridge International
Academies is to the proliferation of private charter schools in the
United States. Like many American private charter schools, Bridge
schools are set in very poor communities and feature a market-based
education model; founders who are entrepreneurs, not educators;
conflicts with local public schools; parents who are confused about the
benefits for their children; and teachers who are uncertified or
underprepared for their work. I hope that we learn from Bridge
International Academies and rely on other kinds of answers as we work
within communities to lift the lives of students and families who live
in poverty in the United States. \emph{Pamela S. Carroll, president,
Council of Academic Deans From Research Education Institutions, Orlando}

Advertisement

\protect\hyperlink{after-bottom}{Continue reading the main story}

\hypertarget{site-index}{%
\subsection{Site Index}\label{site-index}}

\hypertarget{site-information-navigation}{%
\subsection{Site Information
Navigation}\label{site-information-navigation}}

\begin{itemize}
\tightlist
\item
  \href{https://help.nytimes3xbfgragh.onion/hc/en-us/articles/115014792127-Copyright-notice}{©~2020~The
  New York Times Company}
\end{itemize}

\begin{itemize}
\tightlist
\item
  \href{https://www.nytco.com/}{NYTCo}
\item
  \href{https://help.nytimes3xbfgragh.onion/hc/en-us/articles/115015385887-Contact-Us}{Contact
  Us}
\item
  \href{https://www.nytco.com/careers/}{Work with us}
\item
  \href{https://nytmediakit.com/}{Advertise}
\item
  \href{http://www.tbrandstudio.com/}{T Brand Studio}
\item
  \href{https://www.nytimes3xbfgragh.onion/privacy/cookie-policy\#how-do-i-manage-trackers}{Your
  Ad Choices}
\item
  \href{https://www.nytimes3xbfgragh.onion/privacy}{Privacy}
\item
  \href{https://help.nytimes3xbfgragh.onion/hc/en-us/articles/115014893428-Terms-of-service}{Terms
  of Service}
\item
  \href{https://help.nytimes3xbfgragh.onion/hc/en-us/articles/115014893968-Terms-of-sale}{Terms
  of Sale}
\item
  \href{https://spiderbites.nytimes3xbfgragh.onion}{Site Map}
\item
  \href{https://help.nytimes3xbfgragh.onion/hc/en-us}{Help}
\item
  \href{https://www.nytimes3xbfgragh.onion/subscription?campaignId=37WXW}{Subscriptions}
\end{itemize}
