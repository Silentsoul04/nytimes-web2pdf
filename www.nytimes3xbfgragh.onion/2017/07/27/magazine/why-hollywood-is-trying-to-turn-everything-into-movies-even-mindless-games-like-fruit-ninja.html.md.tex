How to Make a Movie Out of Anything --- Even a Mindless Phone Game

\url{https://nyti.ms/2tMddEn}

\begin{itemize}
\item
\item
\item
\item
\item
\item
\end{itemize}

\includegraphics{https://static01.graylady3jvrrxbe.onion/images/2017/07/30/magazine/30fruitninja1/30fruitninja1-articleLarge.jpg?quality=75\&auto=webp\&disable=upscale}

Sections

\protect\hyperlink{site-content}{Skip to
content}\protect\hyperlink{site-index}{Skip to site index}

Feature

\hypertarget{how-to-make-a-movie-out-of-anything--even-a-mindless-phone-game}{%
\section{How to Make a Movie Out of Anything --- Even a Mindless Phone
Game}\label{how-to-make-a-movie-out-of-anything--even-a-mindless-phone-game}}

Hollywood is aggressively adapting material that doesn't have a
narrative or even any characters. But not all intellectual property is
created equal.

Credit...Illustration by Brosmind

Supported by

\protect\hyperlink{after-sponsor}{Continue reading the main story}

By Alex French

\begin{itemize}
\item
  July 27, 2017
\item
  \begin{itemize}
  \item
  \item
  \item
  \item
  \item
  \item
  \end{itemize}
\end{itemize}

In 2013, a movie producer named Tripp Vinson was thumbing through
Variety when he stumbled upon a confounding item: Phil Lord and
Christopher Miller, a pair of writers and directors, were working on
something called ``The Lego Movie.'' Vinson was baffled. ``I had no idea
where they were going to go with Legos,'' he says. ``There's no
character; no narrative; no theme. Nothing.''

A sharply handsome man in his mid-40s, Vinson has worked in Hollywood
for 14 years, racking up 19 producing credits. He's a journeyman
producer who specializes in popcorn flicks; over all, his films have an
average Rotten Tomatoes score of 30 (out of 100). Vinson may not win
Oscars, but he knows how to get his projects into theaters. He has
survived and advanced in Hollywood by quickly adapting to trends ---
what's selling and what's falling out of fashion. His filmography reads
like a map of Hollywood's shifting sands.

Vinson has produced a movie starring Pierce Brosnan as an aging master
thief (``After the Sunset,'' 2004); a movie about Coast Guard swimmers
with Kevin Costner (``The Guardian,'' 2006); and a psychological
thriller with Jim Carrey (``The Number 23,'' 2007). He has made two
movies about exorcisms, one with Laura Linney (``The Exorcism of Emily
Rose,'' 2005), the other with Anthony Hopkins (``The Rite,'' 2011); a
thriller about a psychic who helps the F.B.I. hunt down a serial killer,
also with Hopkins (``Solace,'' 2015); and a romantic comedy with Anna
Faris and Chris Evans, the guy who plays Captain America (``What's Your
Number,'' 2011). He has even made a dance-­competition movie (``Battle
of the Year,'' 2013).

Since Vinson got into the business, something has changed in Hollywood.
More and more movies are developed from intellectual property: already
existing stories or universes or characters that have a built-in fan
base. Vinson thinks it started in 2007, when the Writers Guild went on
strike. ``Before the strike, the studios were each making 20-­something
movies a year,'' he says. ``Back then, you could get a thriller made.
After the strike, they cut back dramatically on the number of films they
made. It became all about I.P.'' --- intellectual property. With fewer
bets to place, the studios became more cautious. ``The way to cut
through the noise is hitching yourself onto something customers have
some exposure to already,'' he says. ``Something familiar. You're not
starting from scratch. If you're going to work in the studio system, you
better have a really big I.P. behind you.''

So over time, Vinson has moved toward making movies backed by
intellectual property. He was the executive producer of the
so-bad-it-was-good ``Hansel \& Gretel: Witch Hunters'' (2013), which
barely broke even domestically but went on to record a worldwide gross
of \$226 million. He also produced the ``Journey To'' franchise
(``Journey to the Center of the Earth,'' 2008; ``Journey 2 the
Mysterious Island,'' 2012) based on Jules Verne's stories, which has
been solidly profitable, with a worldwide gross of over \$500 million.
(A third installment is in development.) He is now working with Disney
on a film about Snow White's sister, Rose Red. And following the trend
of taking successful movie concepts to TV, Vinson has started on a
serialized version of the Martin Scorsese film ``The Departed.''

But at least those were stories. Vinson didn't see how Legos could be
the basis of a feature-­length film. He watched in disbelief as the
movie raked in \$69 million its opening weekend, grossed almost \$470
million worldwide and was almost universally lauded by critics. ``It was
magical and fresh and really profitable,'' he recalls. The movie was
clever, telling the story of a Lego construction worker caught in a
battle between good and evil, which is eventually revealed to be all in
the imagination of a boy playing with his controlling father's Lego set.

Vinson started looking for undervalued I.P. to guide his next movie. He
wanted something an audience would already be familiar with, something
that was culturally ubiquitous but could be made new again. He started
his search in the public domain. He had succeeded with his Jules Verne
and Brothers Grimm adaptations, and besides, old material like that had
the advantage of being free. Nothing caught his eye.

Next he started looking around for a big-name console video game to
acquire. Perhaps something in the mold of ``Lara Croft: Tomb Raider'' or
the ``Resident Evil'' series, which has made well over a billion dollars
at the box office. ``The video-­game companies can be really hard,''
Vinson says. ``Ubisoft and Activision have their own in-house
film-­development arms. A lot of the others are hard to get rights from.
They feel like Hollywood can't figure out how to make a good video-­game
title. Why give it to them to have them screw it up? That can hurt game
sales.'' Not only were the companies difficult to bargain with, only a
few titles even made sense for an adaptation. Vinson's analysis revealed
that megaproperties like Call of Duty and Grand Theft Auto sold tens of
millions of units per installment, but after those top titles, sales
dropped to levels that would make an adaptation risky.

So Vinson started looking at mobile games. A cursory investigation
revealed that the very best selling mobile games didn't move tens of
millions or even a hundred million units --- they could reach into the
billions. He happened upon Fruit Ninja, a wildly popular series of games
that, since its debut in 2010, has been downloaded well over a billion
times. A million people play Fruit Ninja per day. He contacted
Half­brick, the company that developed the game.

``The call came completely out of left field,'' says Sam White,
Half­brick's vice president for entertainment and licensing. The company
had already been working on a TV series based on the game, but, White
says, ``a Hollywood film was like the holy grail.'' Vinson found the
mobile-­game developers at Half­brick to be more approachable than their
console counterparts. They're usually smaller, younger companies. They
see Hollywood as a good opportunity to sell more games. And, most
important, they aren't protective of already existing characters and
plotlines --- generally because they don't have any to speak of.

Vinson worked out a ``shopping agreement'' with Half­brick, a contract
that gave him exclusive film rights to Fruit Ninja for a limited period
so that he could recruit writers and then take a proposal to the
studios. If the project sold, Half­brick would then negotiate a deal to
sell the film rights to the studio, a deal that, based on the ubiquity
of the game, could run up into the high six figures. Vinson then
realized that he was faced with a formidable predicament. There are no
protagonists or antagonists in Fruit Ninja. There's no mythology. No
moral. The game play involves staring at a wall as pineapples,
watermelons, kiwis, apples and oranges fly up into view. The only thing
you do is swipe at the fruit with your finger, cutting them in half.
Sometimes there are bombs, and you're not supposed to swipe at those.
``There's a fun game to play, but that's it,'' Vinson says. ``The
challenge was: What the {[}expletive{]} am I going to do with Fruit
Ninja?''

\textbf{This trend toward} I.P.-­based movies has been profound. In
1996, of the top 20 grossing films, nine were live-­action movies based
on wholly original screenplays. In 2016, just one of the top 20 grossing
movies, ``La La Land,'' fit that bill. Just about everything else was
part of the Marvel universe or the DC Comics universe or the ``Harry
Potter'' universe or the ``Star Wars'' universe or the ``Star Trek''
universe or the fifth Jason Bourne film or the third ``Kung Fu Panda''
or a super-­high-­tech remake of ``Jungle Book.'' Just outside the top
20, there was a remake of ``Ghostbusters'' and yet another version of
``Tarzan.''

This year there is more of the same --- the third installment of
``XXX,'' the Smurfs, ``Pirates of the Caribbean'' (a franchise based on
a theme-park ride), a King Kong movie, Thor, the sequel to ``Blade
Runner,'' a remake of ``Beauty and the Beast,'' ``CHIPS,'' ``Power
Rangers,'' another ``Star Wars'' movie, a ``Guardians of the Galaxy''
sequel, two Stephen King adaptations (``The Dark Tower'' and ``It''),
``Wonder Woman,'' ``The Mummy,'' ``The War for the Planet of the Apes,''
a retelling of Agatha Christie's ``Murder on the Orient Express.'' Every
stripe of intellectual property is represented: from comic books to best
sellers; from the public domain to unnervingly recent source material
like ``Baywatch.''

This environment has fostered, in some producers, a sense of
desperation. When I asked Vinson if the changes his business has
undergone over the past decade have inspired him to panic, he told me:
``Absolutely. It's forced me to look at everything as though it could be
I.P.'' Increasingly, that means non­narrative I.P.: stuff with big
followings but no stories, or even characters, already cooked in.

``The Angry Birds Movie,'' which was based on a mobile game, was
released in 2016 and took in over \$349 million worldwide. The game
itself consisted of flinging birds at pigs, but it at least provided its
writer, Jon Vitti, with protagonists (the birds) and antagonists (the
pigs). There was also Adam Sandler's 2015 movie ``Pixels,'' a disaster
story that united characters from classic 1980s arcade games. Allspark,
a subsidiary of Hasbro, has scored two big successes with a pair of
movies based on the Ouija board. The first installment, ``Ouija,'' cost
an estimated \$5 million to make but managed to earn more than \$103
million in the worldwide box office; the sequel, ``Ouija: Origin of
Evil,'' made \$81 million on a reported \$9 million budget.

Since 2008, Hasbro has been run by Brian Goldner. He founded Hasbro
Studios in 2009, creating Allspark soon after. Hasbro has released five
Transformers movies, with two more in the works; it has made two G.I.
Joe films and has a third in development. Those toys were already made
into cartoons in the 1980s, so refashioning them into live-­action
movies for adults wasn't much of a stretch. The same can't be said for
the I.P. that underlies other movies Hasbro has in development ---
Monopoly and Magic: The Gathering --- or for those it is reported to
have tried to develop: Candy Land, Hungry Hungry Hippos, Furby,
Play-Doh.

\includegraphics{https://static01.graylady3jvrrxbe.onion/images/2017/07/30/magazine/30fruitninja2/30fruitninja2-articleLarge.png?quality=75\&auto=webp\&disable=upscale}

Goldner says the key to making movies from board games and toys is to
``focus on understanding the universal truth about the brand.'' For
example, the Ouija board comes with rules relating to its paranormal
mythology: Always say ``goodbye'' at the end of a session; never take it
into a graveyard. ``If you remember the movies,'' he says, ``all those
rules are broken --- and what results is a very scary situation.''

If the Ouija films were successful, ``Battleship'' (2012) offers a
cautionary tale about I.P. leading to narrative absurdities. Set off the
coast of Hawaii, the film tells of an American Navy crew's fight against
extraterrestrial invaders. The aliens render the Navy's radar systems
useless through some sort of electromagnetic interference, forcing it to
employ a grid-­based targeting method --- just as in the board game.

``Battleship'' cost over \$200 million to make. Domestic box office
returns were weak, and the movie was saved only by the nearly \$250
million it made internationally. It was a major critical flop too. It
wasn't apparent until well into the third act how the film was actually
related to the game. Goldner dismissed all that when we spoke. ``I have
people all the time on airplanes tell me that `Battleship' is one of
their more favorite action movies,'' he said. ``Our goal wasn't to do
`Battleship' the way you would do `Jumanji,' with two people playing a
board game.'' (``Jumanji,'' it's worth noting, was based on a children's
book about a nonexistent board game; there will be a ``Jumanji'' sequel
released this fall.) He continued: ``It's not one person yelling out
`F-7!' and the other one saying, `You hit my battleship.' It was
intended to be about the strategy behind Battleship, about the not
knowing what your opponent is doing, about the cat-and-mouse game.''

This summer's most prominent example of non­narrative I.P. is
\href{https://www.nytimes3xbfgragh.onion/2017/07/27/movies/the-emoji-movie-review.html}{``The
Emoji Movie,''} a film that dramatizes the imaginary lives of emojis.
The film's director and co-­writer, Tony Leondis, told me that ``The
Emoji Movie'' actually began with a quest for some other form of I.P.
About two years ago, he was thinking about what his next project should
be, and he asked himself: ``What are the newest and hottest toys out
there in the marketplace?'' He looked down at his phone and realized
they were right there in front of him: emojis. Everyone uses them.

Unlike board games, emojis don't have rules to play with. Or mythology.
They don't even exist in the real world. So Leondis created a universe
for them: The emojis live inside your phone and are on call 24/7,
waiting to be sent to your screen when needed. Each has to make the same
expression every time they're summoned. He created a character, Gene, a
``Meh'' emoji who is born multi­expressional, violating the rules of the
emoji universe. ``The idea that each emoji has one expression only was
the key to figuring out the whole story,'' Leondis told me. ``Then we
asked ourselves about the world: What do the apps look like to emojis?
What happens when you delete an app? And what would happen if emojis
were wreaking havoc inside other apps than their own?'' Leondis told me
that production moved along at a breakneck pace --- it was two years
from pitch to release. A lot of studios, he told me, think ``The Emoji
Movie'' has the potential to be the beginning of a multifilm franchise.

\textbf{In his search} to find a writer for his Fruit Ninja project,
Tripp Vinson reached out to every major talent agency in Hollywood. In
March 2016, he was introduced by an agent to the writing duo of J. P.
Lavin and Chad Damiani. They have been in Hollywood for 15 years and
have worked as partners for that entire span. They have never once had a
script make it to the big screen.

Lavin and Damiani, both in their mid-40s, took very different paths to
Hollywood. Lavin started as a playwright and earned an M.F.A. from
Carnegie Mellon; Damiani was as an announcer for World Championship
Wrestling in the 1990s. When W.C.W. was faltering, Damiani moved to
Hollywood. Lavin was already there, trying to get his writing career off
the ground. The two started thinking about reality-­show ideas. `` `Joe
Millionaire' had just come out. Only the worst ideas were selling,''
Lavin says. ``All I did was think about terrible reality-­show ideas.''
The pair came up with a reality competition show called ``Green Card.''
The concept was simple: An ultra-­nerdy American guy is set up with
beautiful contestants flown in from all over the globe, who compete for
his affection. The winner receives a green card. (The State Department
wouldn't allow it.) There were other near misses for the duo in the
reality field --- a competition called ``Jocks vs. Nerds'' that a
producer told them MTV liked so much it had considered putting the show
on TV five days a week. (The show never aired.) They developed a hybrid
scripted-­reality series called ``Anchorwoman'' (tag line: ``Would you
trust a bikini model to deliver the news?'') that Fox canceled after its
first night.

They also started writing spec scripts together. The first was titled
``WASPloitation,'' a comedy inspired by Martha Stewart's prison
sentence. Then they wrote ``Terminally Phil,'' in which a fraternity
fools a pledge into thinking he is dying so they don't get kicked off
campus. A zombie-­coal-­mining movie called ``Dead Canary'' was followed
shortly afterward by ``Kamikaze Love,'' an action comedy about a
down-on-his-luck bartender who falls madly in love with a Japanese woman
who has been trafficked into the United States to marry a Yakuza boss.
Every year, a Hollywood executive named Franklin Leonard conducts a
survey of popular but unproduced screenplays called the Black List. In
2007, ``Kamikaze Love'' made the cut, receiving more mentions by studio
executives than many movies that went on to be produced, including
``Slumdog Millionaire,'' ``The Wrestler'' and ``The Wolf of Wall
Street.'' Sony Screen Gems bought ``Kamikaze Love,'' and in the years
since, it has been passed from one Sony subsidiary to the next. Lavin
and Damiani aren't totally sure who has it now.

On the strength of that script, Lavin and Damiani started getting
commissions to develop other people's projects, a lot of them involving
I.P. Brett Ratner enlisted the pair to write the adaptation to the
comic-­book series ``Youngblood.'' The deal fell apart. They wrote ``Max
Steel,'' based on the Mattel toy property for Paramount. The movie ended
up being made, but not based on their script. Warner Brothers enlisted
them to write a screenplay for another comic-­book movie called
``Capeshooters.'' They were attached to a script based on the video game
Duke Nukem and another based on the 1964 kids' book ``Flat Stanley,''
about a boy who survives being smushed pancake flat and uses his new
condition for all manner of mischief.

When they were approached by Vinson, the first thing they did was
download Fruit Ninja. Lavin called Damiani after playing for a while.
They agreed: There was nothing there. Just fruit. Their work on projects
like ``Flat Stanley,'' though, had shown them that having less to work
with provided a greater degree of creative freedom. Lavin and Damiani
spent hours discussing the essence of Fruit Ninja. ``For me, it is the
messiness, the immediate release of destroying fruit,'' Damiani told me.
For Lavin, the soul of the game is the feeling of ``frenzy.'' ``There's
like a 60-­second version of it where you can see how fast you can kill
fruit,'' he says, which ``puts your brain in this weird, bizarre focused
place.'' As he sees it: ``This would be the movie to go see stoned. I
can imagine going in and seeing it in 3-D --- just imagine a
20-foot-high pineapple monster. That shot of yellow and orange. I'd go
see this movie a dozen times.''

While they were developing the movie, Damiani and Lavin were also
attending career days at elementary schools in the San Fernando Valley
and Hollywood. Sometimes they went to four classes a day. These gave
them the opportunity to do some informal market research. Every time
they brought up the script they were working on, they found the same
reaction. The kids would ``put their hand in the air, raise a finger and
start swiping like crazy.'' Lavin told me, ``Whatever movie we wrote, it
had to be an extension of that energy, that desire to tear up everything
in your path and take charge.''

Early on, Lavin and Damiani struggled to find a narrative entry point.
They started with the premise that there was a magic book and an evil
fruit overlord. Vinson rejected that idea. Their next concept involved
scientific experiments on fruit gone wrong. Vinson didn't like that
either. Eventually, a working narrative emerged: Every couple of hundred
years, a comet flies by Earth, leaving in its wake a parasite that
descends on a farm and infects the fruit. The infected fruit then search
for a human host. The only thing keeping humanity from certain doom is a
secret society of ninjas who kill the fruit and rescue the hosts by
administering the ``anti-­fruit.'' The produce-­slaying saviors are
recruited from the population based on their skill with the Fruit Ninja
game. With civilization in imminent danger, a cadre of unlikely heroes
materializes --- a little boy, a college-­age girl, two average guys.
The action starts after each of the story's heroes returns home after a
horrible day and plays Fruit Ninja to relieve some stress. Damiani told
me this aligns with the Fruit Ninja brand: ``Anybody can play. Anybody
can be a master.''

\textbf{With the story} intact, Vinson, Lavin and Damiani started
ironing out a pitch. They're known around town for being good in the
room. Lavin has a background in theater; Damaini does improv comedy and
teaches clowning. Their presentation was 35 minutes --- fast moving,
full of laughs. ``It felt a lot like how you develop clown work,''
Damiani told me. ``You play and improvise to keep the energy up --- and
register what works. I'm always looking for the hot spot --- the person
giving us the best energy. That might not be the big boss. It might be a
junior. Keep them laughing, and it spreads.'' They estimate that they
gave 25 to 30 presentations, five of which were at different film
divisions within Sony. They met with four different Chinese companies.
To keep their act feeling fresh, they added seemingly improvised asides
and digressions. ``If it's too polished, the execs feel like they're at
a TED Talk, and then you see the eyes go to the window,'' Damiani says.

Everyone they pitched was enthusiastic, but no one pulled the trigger.
``I love this. Can you come back and pitch it to my boss?'' was a
popular refrain. They presented in a room where an executive laughed so
hard that she cried. Still, no one was biting. Three months into the
process, they presented at New Line Cinema, a subsidiary of Warner
Brothers. New Line always has a strong slate of comedies and horror
movies, but family fare hasn't traditionally been a priority. Vinson
didn't think anything would come of the meeting. This time, however, the
decision makers were in the room. And they bought ``Fruit Ninja.''

This doesn't mean the movie will definitely be made. There are a million
considerations that could affect its production --- scheduling, budget,
the studio head's reading the script while in a foul mood. In fact,
Lavin said that his project's future will depend, to a certain extent,
on the performance of ``The Emoji Movie.'' ``We really want to see `The
Emoji Movie' succeed, because it's like proof of concept with these
I.P.s,'' he says. ``But the success of that movie can result in two very
different things. Both are a form of enthusiasm. One is: `OMG this is
happening! We like this script. Let's get moving.' Or it can result in:
`Wow, they did a really great job. Let's slow down and take a real good
look at what we're doing with ``Fruit Ninja.'' ' ''

The paths that ``Fruit Ninja'' could take from here are practically
infinite, anything from being spiked to having the Lego franchise's wild
success, which is perhaps best embodied by Bricksburg, the
15,000-­square-­foot facility in Hollywood where the producer Dan Lin
was part of a team that created the entire Lego universe. It is a shrine
to everything Lego --- there's a cactus made of Legos, a Lego bat signal
--- and to the possibilities of non­narrative I.P. This year has brought
the release of ``The Lego Batman Movie,'' which will be followed by
``The Lego Ninjago Movie'' and, in 2019, by a sequel to the first film.

I visited Bricksburg earlier this year, and Lin, who is 40-­something
but looks 20-­something, joined me at the end of the tour. He invited me
into a secret room, tucked down a small passageway hidden by a trap door
next to a water cooler, where he and his colleagues go for device-­free
brainstorming. He told me about how he came up with the idea for ``The
Lego Movie'' in 2009, after watching his 5-year-old son playing with the
bricks --- the sound effects and dialogue, the way he made those pieces
zoom around the room. The company was interested, he discovered, in
expanding their reach with children as they enter their teenage years,
which they called, Lin said, ``the dark ages.'' Lin wanted to make
something his son could watch. What resulted felt lightly subversive ---
the villain is named President Business --- but also managed to uphold
the basic tenets of the Lego brand: imagination, free play, creativity.

Lin didn't know when he was making ``The Lego Movie'' that it would
inspire so many other movies based on toys, games and apps. When I last
spoke with him, over the phone, I got the sense that birthing an entire
generation of cynically made movies weighed on him. Companies call him
all the time, he told me, asking if he can do for their company what he
did for Lego. ``You know, 95 percent of brands are not Legoizable,'' he
said. When I told him about the Fruit Ninja script, he let out a little
gasp. ``Oh, my gosh,'' he said. ``Who's making that?''

Advertisement

\protect\hyperlink{after-bottom}{Continue reading the main story}

\hypertarget{site-index}{%
\subsection{Site Index}\label{site-index}}

\hypertarget{site-information-navigation}{%
\subsection{Site Information
Navigation}\label{site-information-navigation}}

\begin{itemize}
\tightlist
\item
  \href{https://help.nytimes3xbfgragh.onion/hc/en-us/articles/115014792127-Copyright-notice}{©~2020~The
  New York Times Company}
\end{itemize}

\begin{itemize}
\tightlist
\item
  \href{https://www.nytco.com/}{NYTCo}
\item
  \href{https://help.nytimes3xbfgragh.onion/hc/en-us/articles/115015385887-Contact-Us}{Contact
  Us}
\item
  \href{https://www.nytco.com/careers/}{Work with us}
\item
  \href{https://nytmediakit.com/}{Advertise}
\item
  \href{http://www.tbrandstudio.com/}{T Brand Studio}
\item
  \href{https://www.nytimes3xbfgragh.onion/privacy/cookie-policy\#how-do-i-manage-trackers}{Your
  Ad Choices}
\item
  \href{https://www.nytimes3xbfgragh.onion/privacy}{Privacy}
\item
  \href{https://help.nytimes3xbfgragh.onion/hc/en-us/articles/115014893428-Terms-of-service}{Terms
  of Service}
\item
  \href{https://help.nytimes3xbfgragh.onion/hc/en-us/articles/115014893968-Terms-of-sale}{Terms
  of Sale}
\item
  \href{https://spiderbites.nytimes3xbfgragh.onion}{Site Map}
\item
  \href{https://help.nytimes3xbfgragh.onion/hc/en-us}{Help}
\item
  \href{https://www.nytimes3xbfgragh.onion/subscription?campaignId=37WXW}{Subscriptions}
\end{itemize}
