Losing It in the Anti-Dieting Age

\begin{itemize}
\item
\item
\item
\item
\item
\item
\end{itemize}

\includegraphics{https://static01.graylady3jvrrxbe.onion/images/2017/08/01/magazine/weight-watchers-top/weight-watchers-top-articleLarge-v2.jpg?quality=75\&auto=webp\&disable=upscale}

Sections

\protect\hyperlink{site-content}{Skip to
content}\protect\hyperlink{site-index}{Skip to site index}

Feature

\hypertarget{losing-it-in-the-anti-dieting-age}{%
\section{Losing It in the Anti-Dieting
Age}\label{losing-it-in-the-anti-dieting-age}}

The agonies of being overweight --- or running a diet company --- in a
culture that likes to pretend it only cares about health, not size.

Credit...Photo Illustration by Todd McLellan

Supported by

\protect\hyperlink{after-sponsor}{Continue reading the main story}

By Taffy Brodesser-Akner

\begin{itemize}
\item
  Aug. 2, 2017
\item
  \begin{itemize}
  \item
  \item
  \item
  \item
  \item
  \item
  \end{itemize}
\end{itemize}

James Chambers was watching membership sign-ups on Jan. 4, 2015, like a
stock ticker --- it was that first Sunday of the year, the day we all
decide that this is it, we're not going to stay fat for one more day. At
the time, he was Weight Watchers' chief executive, and he sat watching,
waiting for the line on the graph to begin its skyward trajectory.
Chambers knew consumer sentiment had been changing --- the company was
in its fourth year of member-recruitment decline. But they also had a
new marketing campaign to help reverse the generally dismal trend. But
the weekend came and went, and the people never showed up. More than
two-thirds of Americans were what public-health officials called
overweight or obese, and this was the oldest and most trusted diet
company in the world. Where were the people? Weight Watchers was at a
loss.

Chambers called Deb Benovitz, the company's senior vice president and
global head of consumer insights. ``We're having one of the worst
Januaries that anyone could have imagined,'' she remembers him telling
her. In the dieting business, January will tell you everything you need
to know about the rest of the year. ``Nothing like we had anticipated.''
Chambers and Benovitz knew that people had developed a kind of diet
fatigue. Weight Watchers had recently tried the new marketing campaign,
called ``Help With the Hard Part,'' an attempt at radical honesty. No
one wanted radical honesty. Chambers told Benovitz that they needed to
figure out what was going on and how to fix it before the February board
meeting.

Benovitz got to work. She traveled the country, interviewing members,
former members and people they thought should be members about their
attitudes toward dieting. She heard that they no longer wanted to talk
about ``dieting'' and ``weight loss.'' They wanted to become ``healthy''
so they could be ``fit.'' They wanted to ``eat clean'' so they could be
``strong.''

If you had been watching closely, you could see that the change had come
slowly. ``Dieting'' was now considered tacky. It was anti-feminist. It
was arcane. In the new millennium, all bodies should be accepted, and
any inclination to change a body was proof of a lack of acceptance of
it. ``Weight loss'' was a pursuit that had, somehow, landed on the wrong
side of political correctness. People wanted nothing to do with it.
Except that many of them did: They wanted to be thinner. They wanted to
be not quite so fat. Not that there was anything wrong with being fat!
They just wanted to call dieting something else entirely.

\href{http://jamanetwork.com/journals/jama/article-abstract/2608211}{A
study out of Georgia Southern University's Jiann-Ping Hsu College of
Public Health}, published in The Journal of the American Medical
Association in March, monitored attitudes toward losing weight over
three periods between 1988 and 2014. In the first period, 1988-94, 56
percent of fat adults reported that they tried to lose weight. In the
last period, 2009-14, only 49 percent said so.

The change had been spurred not just by dieting fatigue but also by real
questions about dieting's long-term efficacy. In Weight Watchers' own
research, the average weight loss in any behavior-modification program
is about a 5 percent reduction of body weight after six months, with a
return of a third of the weight lost at two years. There were studies
that appeared to indicate that the cycle of weight loss and weight gain
could cause long-term damage to the metabolism. Those studies led to
more studies, which suggested that once your body reaches a certain
weight, it is nearly impossible to exist at a much lower weight for an
extended period of time. Even more studies began to question whether or
not it's so bad to be fat in the first place; one notably suggested that
fatter people lived longer than thin ones.

These questions began to filter into the mainstream. Women's magazines
started shifting the verbal displays on their covers, from the
aggressive hard-body stance of old to one with gentler language,
acknowledging that perhaps a women's magazine doesn't know for sure what
size your body should be, or what size it can be: \emph{Get fit! Be your
healthiest! GET STRONG!} replaced diet language like \emph{Get lean!}
\emph{Control your eating! Lose 10 pounds this month!} In late 2015,
Women's Health, a holdout, announced in its own pages that it was doing
away with the cover phrases ``drop two sizes'' and ``bikini body.'' The
word ``wellness'' came to prominence. People were now fasting and eating
clean and cleansing and making lifestyle changes, which, by all
available evidence, is exactly like dieting.

Diet companies suffered for being associated with dieting. Lean Cuisine
repositioned itself as a ``modern eating'' company, not a diet company.
In fact, Lean Cuisine went so far in their pivot that in 2016 they
introduced a Google Chrome extension that would filter mentions of the
word ``diet'' and ``dieting''; it apparently did this to show that just
because it was called Lean Cuisine, that didn't mean it was a diet
company. You can't be held responsible for what your parents named you!

Weight Watchers saw all this happening and concluded that people didn't
have faith in diets. The company decided that what it offered was not a
diet program but a lifestyle program. It was a behavior-modification
program. (For the sake of expediency here, I will call its program a
diet because it prescribes amounts of food.) When Deb Benovitz returned
from her travels with news of dieting's new language changes, the
company realized that something had to change more than its marketing
approach.

Weight Watchers' chief science officer is Gary Foster, a psychologist
--- the first in that position, which previously had been held by
dietitians. What he and his team realized from Benovitz's research was
that dieters wanted a holistic approach to eating, one that helped
really change their bodies, yes, but in a way that was sustainable and
positive. He got to work creating a new approach that would become known
as Beyond the Scale: He used all available mind-body research to try to
figure out a way for members to appreciate benefits of the program
besides weight loss. This would help them stay on the program during
setbacks and beyond their weight-loss period and allow the program to
infiltrate their lives beyond mealtime and beyond plain old eating
suggestions.

The company would move away from giving its members goal weights. It
expanded its cognitive-behavioral strategies, which taught members to
challenge unhelpful thinking and to respond to their emotions with
reason, as opposed to with food or despair. It developed workshops that
used meditation and qigong and didn't once mention food or weight. It
updated its apps and introduced a social-media program, Connect. It
became as holistic-minded as the people told Benovitz they wanted a
program to be.

But Weight Watchers was still a company called Weight Watchers, and it
had to figure out a way to communicate all of this change to the public.
People had too many associations with the brand. It needed someone other
than the usual celebrity spokesdieter, a fat famous person who could be
paid somewhere between \$250,000 and \$2 million to do the talk show
circuit and People covers for a year. It needed someone who could
fast-track the message that it was worth taking a new look at Weight
Watchers.

When the company called Oprah Winfrey in July 2015, she was standing on
the lawn of her home in Maui with a sprained ankle, an injury she
sustained while hiking in the mountains. In the month since her
convalescence began, she had gained 17 pounds. Her struggles with weight
were, at this point, a cultural meme. How could you explain the failure
of someone so goal-oriented and successful --- someone so successful
that her name was invoked as a symbol of success as often as it was ever
used to summon her? Weight Watchers had reached out to her in the past,
but she politely declined. This time she bought a 10 percent stake in
the company for \$43 million, and Weight Watchers stockholders rejoiced.

But the verbal changes around dieting had indicated something deeper
than just a marketing issue; they pointed straight back to the fatigue
that was hurting Weight Watchers in the first place. So, yes, many
people celebrated the new partnership. But others --- meaning, anyone
who for a majority of their lives had been watching Oprah cycle up and
down through different sizes --- felt a little confused by the move.
What was Oprah, a person whose very brand meant enlightenment and
progress, doing on another diet? It was hard not to suspect that she was
trapped, like so many of us are, in a culture that says one thing about
fatness and means something very different.

\textbf{Back in 1963,} when Jean Nidetch held the first
what-would-be-known-as-Weight-Watchers meetings above a movie theater in
Queens, things seemed clearer: It was bad to be fat, and it was good to
be thin, and fat people wanted to be thin, and thin people wanted to
help them get there. Her memoirs, ``The Story of Weight Watchers,''
reads about as current as a cigarette ad featuring smoking babies. ``If
strawberry shortcake made you break out in purple spots, you wouldn't
eat it,'' she wrote. ``You'd be allergic to it. But, do you think fat is
prettier than purple spots? It's uglier and harder to get rid of.''

Its frankness seems like an anachronism now, but you have to consider
that at the time, this kind of straight talk was a glass of cold water
in the desert for many fat people, who privately wondered why it was so
hard for them to reduce the size of their bodies when it seemed so
effortless for the people who walked around thin. Nidetch lost her
weight in her late 30s, after a lifetime of self-loathing and
embarrassment; the last indignity was the time someone asked when her
baby was due when she was definitively not pregnant. She went to a
city-run obesity clinic, and when she left the program, she kept the
diet it gave her. She mimeographed it and handed it out to people whom
she had gathered to spread the word about how weight loss could provide
freedom and hope. (The diet would evolve from an eating plan to a more
democratic system of balanced exchanges to an absolute laissez-faire
system of points, as the company realized that the more autonomy and the
less deprivation people experienced in their dieting --- limitless
choices, if not limitless amounts --- the more likely they'd be to stay
on the diet.) But Nidetch knew that it wasn't just the food that was the
problem; it was the problem that was the problem. What fat people needed
was one another. They needed a space in which they could talk openly
about the physical struggles and daily humiliations of walking around in
a fat body, and just how much that sucked.

These same ideas were articulated more starkly a few years later, but
with a different prescription. In 1967, a fat man named Lew Louderback
unleashed an essay in The Saturday Evening Post arguing that the wisdom
around thinness could be applied only to thin people --- that fat people
suffered physically and psychologically when trying to maintain
thin-person weights, and that this maintenance seemed to be temporary at
best and largely destructive emotionally.

He went on to write a book called ``Fat Power,'' which helped give birth
to what would become known as the fat-acceptance movement. That movement
has varying degrees of militancy, but generally asks the public to put
aside its bias and learn something new --- to not think of fat people as
lazy; to not deny them medical care; to not exclude them from their
basic rights. It suggests that we re-examine what we think we know about
fatness, that we consider trying to love and care for our bodies at
whatever size they are now.

\includegraphics{https://static01.graylady3jvrrxbe.onion/images/2017/08/06/magazine/06weightwatchers3/06weightwatchers3-articleLarge.jpg?quality=75\&auto=webp\&disable=upscale}

There were more books and more essays and more challenges to the status
quo in the decades to come. In 2008, Linda Bacon, a researcher who holds
graduate degrees in physiology, psychology and exercise science with a
specialty in nutrition, wrote a seminal fat-acceptance book, ``Health at
Every Size: The Surprising Truth About Your Weight,'' which used
peer-reviewed research to bolster these ideas. She gave seminars to
doctors on fat phobia and weight bias in an effort to help them
understand how their views on obesi­ty were hurting their patients and
not allowing them to examine fatness neutrally. For example, there is
evidence that stress and discrimination play a strong role in the
insulin resistance and diabetes and heart disease for which weight
typi­cally takes the blame.

With the rise of social media, the movement began to infiltrate the
culture in other ways, too. Fat-acceptance and body-positivity activists
began posting pictures of themselves on Instagram --- just regular
pictures, defiant for their lack of apology. There were intuitive-eating
workshops and body-positivity training camps. There were bloggers and
authors asking exactly how much of your life you were willing to put off
in pursuit of a diet, or until you got to a certain weight, even
temporarily. Normal, nonmilitant, nonactivist people began asking
themselves if it was that bad to be fat --- if it was that unhealthy, or
that ugly, to be fat. And yet the most telling thing about the way the
fat-acceptance movement is received in our society may be that its
Wikipedia entry contains two quotes from people criticizing it before it
mentions even one person who espouses it. In this world, we are witness
to a moment when the word ``optimal'' is used in conjunction with the
word ``body,'' when people are trying to mold themselves into
high-performance, precision machines. The idea of a fat machine makes no
sense when you are easily fueled and refueled on Whole Foods and
Soylent.

In other words, all this activism didn't make the world more comfortable
with fat people or dieting. Society doesn't normally change the words
for things unless we're fundamentally uncomfortable with the concepts
beneath them. Consider the verbal game of chicken we've played with the
people all this affects: Fat people went from being called fat (which is
mean) to being called overweight (a polite-seeming euphemism that either
accidentally or not accidentally implies that there is a standard
weight) to being called zaftig/chubby/pleasingly plump (just don't) to
curvy (which seems to imbue size with a sexuality and optimism where it
should just be sexually and emotionally neutral) and back to fat
(because it's only your judgment of fat people that made it a bad word
in the first place, and maybe being fat isn't as bad as we've been made
to believe). It bears mentioning that Weight Watchers doesn't have a
standardized word for its demographic, but Foster uses the term ``people
with overweight.''

As the ideas that sprang from the fat-acceptance movement began to
trickle into the mainstream, fat people began to wonder what it might be
like to put all this aside and just live their lives. Some asked
themselves if they thought they could figure out a way to not want to be
thin; some began to ask themselves if they actually liked the way they
looked. They began to wonder if there was even a proven and effective
way to become and stay thin anyway. They began to ask themselves if they
should be dieting at all.

\textbf{Last fall, I was with Foster,} Weight Watchers' chief science
officer, as he walked the halls of Obesity Week, the annual conference
of the Obesity Society. The conference includes study presentations,
each one a possible clue to the mystery of fatness. We attended a
presentation on a new study of a weight-loss medication. People on the
medication lost weight, but once they were off, the weight came back. If
only we could get people's weight down, the presenters said, they could
have a fresh start. Out in the hall, Foster shook his head. ``There's a
bias and a stigma: `We'll give these people medication for a short
period, but then they've got to fly straight and get will power.' It's
nonsense. This tough love --- \emph{I'm going to be hard on myself} ---
you know, in some perverse way, if it were true, we might try to
leverage it, but it's not. The harder you are on yourself, the worse you
do.'' In his career before Weight Watchers, Foster was the founder and
director of the Center for Obesity Research and Education at Temple
University. Neutralizing the morality talk and stigma that surround
obesity, he says, would make it a lot easier to figure out how to deal
with it.

By the time of the conference, the Oprah-Weight Watchers partnership
\href{https://www.nytimes3xbfgragh.onion/2015/10/20/business/dealbook/shares-of-weight-watchers-jump-as-oprah-winfrey-takes-a-stake.html}{had
proved a clear success.} Within a year, the company was up to 2.8
million members; by the first quarter of 2017 it would be 3.6 million.
Oprah had brought an audience to Beyond the Scale, the holistic model
Foster helped create. He says that initial weight loss on the program in
2016 was up 15 percent from what it was the year before. Of course
people should want to manage their weight, he said, the same way they'd
want to manage their diabetes. ``It would seem preposterous if we would
say to people with diabetes, `Don't manage your diabetes.' '' Or their
asthma. All three are chronic conditions; why, he asks, would we assume
we should give up on weight? When people lose weight, he points out,
they see improvements in risk factors. Data is data. Modifying your
eating is hard, he says, but it's worth it. No one can tell you that
lowering weight doesn't also lower other health factors like
hypertension and high cholesterol and joint pain.

Maybe that's true. Most mainstream sources agree on this, but there are
definitely some researchers who don't; there are some who think that
people who end up fat have different physiologies, and that fatness is
just one component of them. Consider the ``uptick,'' as Foster calls it,
that comes after two years on a diet when, say, the person who lost 5
percent of her weight has gained a third of it back. Think about those
numbers. If you weigh 300 pounds, you will lose 15 pounds in six months.
You'll keep it off for a year or two, maybe. Five pounds is likely to
return. Of course, these are people who don't stay on a diet-maintenance
plan; but the average dieter certainly doesn't, and it's worth it to ask
why a person wouldn't stay on a program that offered such rewards. Is it
because they couldn't? It's worth it to ask if the programs are right
and all these humans trying very hard at them are wrong. And also, where
are the 300-pound people who want to lose just 15 pounds in the first
place? I haven't met those people. But mainly, what it comes down to is
this: Weight Watchers is designed to be successful only if you can stay
on Weight Watchers forever.

And there were also questions about dieting's long-term effects on the
body. A study done by the National Institute of Diabetes and Digestive
and Kidney Diseases, which is part of the National Institutes of Health.
The study followed contestants who had appeared on the eighth season of
``The Biggest Loser,'' all of whom had normal resting metabolisms for
their size when the season began. As the contestants experienced
radical, sweeps-week weight loss, their metabolisms slowed, and stayed
slow afterward. To maintain his weight loss, one contestant's resting
metabolism now required 800 calories fewer per day than a man of his
size. It might be that when you have been fat, your body doesn't behave
the way a thin body does, even when you become thinner.

Foster shook his head at that one. He hears about the ``Biggest Loser''
study a lot, but he doesn't think it conveys accurate information. It
uses a very small sample under extreme conditions. He cited his own and
others' studies examining the metabolic rate, fat distribution and
psychological state of people before they lose weight, after they lose
weight and after they regain the weight. ``Nothing is changed,'' he
said. ``I'm not saying that's a good outcome or something we should
celebrate --- but this idea that the act of managing your weight and
losing weight has somehow set you up to be in a worse spot just isn't
borne out by the science.''

Here is the thing about this particular debate at this particular
moment: Everyone has much the same data, but there are plenty of people
who would interpret the data differently from the way Foster does. I've
spoken to countless (I literally stopped keeping count) obesity
researchers and dietitians and biologists and doctors. The answer
becomes one of point of view: Is fat inherently bad, or can it be
neutral?

We can't answer that yet. There is still too much debate. So in the
meantime, a fat person has to consider the data she has access to ---
meaning studies, yes, but also her own experience and the experience of
her fat peers --- and ask: Do you believe that you, a fat person, can
ever be meaningfully thinner for a meaningful amount of time? Is a diet
successful if it stops being successful once you're done with it? I've
interviewed Foster before. Back in 2011, when he was at Temple, he
published a study about the efficacy of different kinds of diets. They
all led to similar losses, and they all led to similar rates of
recidivism. When I spoke with him back then, I asked him why we should
continue dieting if the outcomes were so bad. He was concerned that I
would suggest to my readers that dieting wasn't worth it. He told me
that people didn't need that kind of discouragement. This attitude is
what makes him so credible to me --- his message was the same long
before he worked for a corporation --- but it's also what makes this so
depressing.

I do not recommend being a fat person at Obesity Week. Over the years,
the event has become a week long, and it contains a robust trade show.
After Foster left me to go to a meeting, I walked the trade-show floor
and saw all the products being shown to the obesity specialists in
attendance. I watched a video of a new kind of retractor that will more
easily hold open belly skin while part of a stomach is cut out and sewn
up, because you can't eat as much if your stomach is made smaller. I
watched a person showing a model of a balloon you'd insert into the
stomach of a patient that would take up volume so that she wouldn't be
so hungry, to be removed later once her behavior had been modified. I
drank a smoothie with a superfood ingredient I can't pronounce or
remember while someone told me that my readers would really be
interested in their something-metrics plan for hydration and portion
control. I narrowed my eyes thoughtfully because it felt rude to be
drinking this guy's smoothie without taking him seriously. ``There's no
such thing as magic, Taffy,'' the smoothie man was saying. I nodded in
solemn agreement.

Before he left me, I told Foster that Obesity Week made me sad. First,
it was the profusion of educated people in the room studying me and my
people as if we were problems to solve. But second, it was because if
you have this many hundreds of smart and educated people trying to
figure this out, and nobody has anything for me but superfood and
behavior modification and an insertable balloon and the removal of an
organ, it must be that there is no way to solve fatness.

Foster doesn't see it that way, he told me. ``I look around this room,''
he said, ``and I see hope.''

Image

Oprah Winfrey on her show in 1988, with a wagon of fat to represent the
67 pounds she lost while on Optifast.Credit...Charles Bennet/Associated
Press

\textbf{By the time} Oprah announced that she was signing on with Weight
Watchers, I was celebrating my 25th anniversary of my first diet, at age
15, which I found in an issue of Shape magazine. I was 5-foot-3 and
weighed 110 pounds. In the intervening years, I did cleanses and had
colonics and refilled the prescriptions on three rounds of those diet
pills that made my teeth sweat and ate two shakes for lunch and just
protein and just good carbs (carbs are divided into good and bad, like
witches in Oz) and just liquid and just fruit until dinnertime and just
food the size of my fist and two glasses of lukewarm lemon water. I had
stood up in a room and said, ``Hello, my name is Taffy, and I am a
compulsive overeater.'' I had stuck my finger down my throat, a shot in
the dark that I hoped would be more sustainable than it was. I had South
Beached, I had Atkinsed, I had Slim-Fasted. Put it this way: The Amazon
algorithm recently recommended to me, based on my previous searches, a
book-and-CD combination, ``Hypnotic Gastric Band: The New Surgery-Free
Weight-Loss System,'' which offered a hypnotic equivalent to bariatric
surgery. Put it this way: When I arrived at Weight Watchers, despite the
fact that I was there as a journalist, I \emph{registered for the diet}
under the ra­tion­ale that this was experiential journalism. When I gave
my name at the counter, the person registering me furrowed her brow and
said: ``That's strange. There are three other people named Taffy
Akner.'' I said, ``No, those are all me.''

``In Brooklyn?''

``Yes, when I was in high school.''

``In Los Angeles?''

``Yes, right before I was married.'' I stopped her before she could go
on. ``They're all me.''

By then I was all in, as if I ever hadn't been. When I arrived at the
Union, N.J., meeting at 8 a.m. on a Saturday, it was a few weeks before
Thanksgiving. Thanksgiving is marketed as a fun, festive holiday of
family gathering, but everyone at that meeting knew the truth:
Thanksgiving is an existential threat. Thanksgiving is a killer.

The year leading up to Thanksgiving hadn't been much better for this
group. There had been family deaths and illness. There had been
foreclosures and unemployments and high-school reunions, and someone's
daughter was always baking sticky buns; someone's husband wanted to know
where his steak was; someone's son wanted to know why the meatloaf
tasted different; someone's co-worker was always leaving doughnuts and
bagels on the communal table at the office. The people, mostly women, in
the folding chairs had one rule, though: No matter what happened during
the week, you showed up. ``This is my church,'' a woman named Donna told
me. A few months before, she buried her mother on a Friday; on Saturday
she came to the meeting.

Dayna, the group leader, stood at the head of the room. How could you
not love Dayna? She took such care with her appearance --- she wore tall
boots and wrap dresses and makeup, even on Saturday mornings when
everyone else wore sweatpants at best (or leggings; leggings weigh
less). She gave them star-shaped stickers off a large roll when they
lost weight or when they had acted in their best interests over the
week. She remembered their names, even the ones who hadn't shown up in
months; she gave them hugs.

Today, Donna had gained weight. She had been holding steady at six
pounds short of her goal. Since 2009 --- \emph{2009} --- she had shown
up every week and by now had lost 132 pounds, which is an entire other
Donna. But these last six pounds, my God, what would it take? She'd been
down last week by a pound, and now that pound was back. She'd been going
to the gym ``religiously'' for two weeks, but thought maybe the not
going to the gym three weeks ago had caught up with her. Sometimes being
six pounds away from her goal was harder than being 321 pounds.

``I'm so frickin' aggravated,'' she said. She asked me how I did. I
shrugged and told her I had lost three pounds. ``But I just started, so,
. . . '' I said. I didn't want her to feel bad. Another woman, Amy,
whispered to me, ``You never want to say `I only lost, . . . ' because
then everyone will go, `Oh, jeez.' ''

I asked the women there, most of whom were repeat joiners as well:
Shouldn't we be moving toward acceptance? Here we all were --- smart,
accomplished, successful women (and one man) --- and we couldn't
maintain what was proved to be the most effective diet you could ever
try. If we couldn't stay on this, could we stay on anything? What if the
flaw wasn't in us but in the system?

They furrowed their brows and shook their heads and gave me funny looks.
What was I talking about? How could a fat person not want to be thin?
Donna's sisters were all on diabetes medication, and she wasn't. Her
back had hurt until about 20 pounds ago, and now she could crawl on the
floor with her grandson as if it were nothing.

I couldn't counter very hard. Each time I came to a meeting, I was
seduced by the possibility, by the clean, Calvinist logic, that if you
ate less you would weigh less, that your body would feed on itself and
its fat reserves until you became smaller and smaller and more pleasing
to the world and its standards --- until you practically disappeared (we
are a culture that fetishizes something called Size 0). I looked forward
to these meetings, feeling as if these people were the only ones who
seemed to truly understand my predicament. But my optimism and
motivation didn't survive my walk out the door. By the time I got to my
car, I had no idea what to do. I knew that if this could be done, I
would have done it, and yet I didn't know why I couldn't do it. Just eat
less, right? It's so simple!

About two years ago, I decided to yield to what every statistic I knew
was telling me and stop trying to lose weight at all. I decided to stop
dieting, but when I did, I realized I couldn't. I didn't know what or
how to eat. I couldn't fathom planning my food without thinking first
about its ability to help or hinder a weight-loss effort. I went to a
nutritional therapist to help figure this out (dieting, I have found, is
its own chronic condition), and I paid her every week so I could tell
her that there still had to be a way for me to lose weight. When she
reminded me that I was there because I had realized on my own that there
was no way to achieve this goal, I reminded this wonderful, patient
person that she couldn't possibly understand my desperation because she
was skinny. I had arthritis in my knees, I said. Morality and society
aside, they hurt. I have a sister with arthritis in her knees, too, but
she's skinny and her knees \emph{don't} hurt.

I went to an intuitive-eating class --- intuitive eating is where you
learn to feed yourself based only on internal signals and not external
ones like mealtimes or diet plans. Meaning it's just eating what you
want when you're hungry and stopping when you're full. There were six of
us in there, educated, desperate fat women, doing mindful-eating
exercises and discussing their pitfalls and challenges. We were given
food. We would smell the food, put the food on our lips, think about the
food, taste the food, roll the food around in our mouths, swallow the
food. Are you still hungry? Are you sure? The first week it was a
raisin. It progressed to cheese and crackers, then to cake, then to
Easter candy. We sat there silently, as if we were aliens who had just
arrived on Earth and were learning what this thing called food was and
why and how you would eat it. Each time we did the eating exercise, I
would cry. ``What is going on for you?'' the leader would ask. But it
was the same answer every time: I am 41, I would say. I am 41 and
accomplished and a beloved wife and a good mother and a hard worker and
a contributor to society and I am learning how to eat a goddamned
raisin. How did this all go so wrong for me?

They tried to soothe me. They told me that hatred of fat was a societal
construct, but I never understood why that should comfort me. I live in
society. I hurt my ankle playing tennis, activating an old injury, and
an internist I was seeing for the first time, without taking any medical
history or vital signs --- my blood pressure is pristine, just so you
know --- told me he couldn't do anything for me until I lost weight and
gave me a rusty photocopy about food exchanges. (Another doctor
prescribed three months of physical therapy, and now my ankle is fine.)
I was in Iceland, for a story assignment, and the man who owned my hotel
took me fishing and said, ``I'm not going to insist you wear a life
jacket, since I think you'd float, if you know what I mean.'' I ignored
him, and then afterward, back on land, after I fished cod like a Viking,
he said, ``I call that survival of the fattest.'' A woman getting into a
seat next to me on a plane said, ``Looks like this will be a cozy
ride,'' and a Manhattan taxi driver told me he liked to watch my
``jelly'' shake, by which I can only presume he meant a part of my body.
I have been asked if it was my first time taking Pilates at a studio
where I'm on my fifth 10-pack. I have been told at a yoga class that I
have ``a really great spirit'' and it's important that I ``just keep
coming.'' (I've been taking yoga for 12 years.) I was told by a
seamstress that she had never seen a bride not lose weight for her
wedding until she met me. A crazy man tried to give me candy outside the
Met, and when I politely declined he screamed at me that of course I
didn't want it, I was fat enough, and my sister asked me why I was so
upset, clearly that guy was crazy, and I said, ``You don't understand
because you're skinny,'' and on and on forever. (By the way, I am
writing this despite the myriad degradations that I know will appear in
my inbox and in the comments section when it is published. I am someone
who once wrote a body-image essay for a women's magazine in which a
comment in the margins from an editor read, ``Why doesn't she stop
eating so much?'')

Back in Union, Dayna stood at the front of the room. The conversation
had shifted to Thanksgiving foods, how sons home from college depend on
the stuffin' muffins, how husbands will know if there's no butter in the
mashed potatoes. Donna makes an Easter pie with more kinds of pork than
there are pigs roaming the Earth. Really, the group members were worried
that despite their weight loss, they would forget that they were really
fat people on the inside. Thanksgiving is a killer.

``It's just one day,'' Dayna said. And all those around her heaved heavy
sighs.

\textbf{``Please hold} for Ms. Winfrey.''

When Oprah called me, she was on the same mountain in Hawaii where she
sprained her ankle two years ago. After a monthslong search, Weight
Watchers had hired a new C.E.O., Mindy Grossman, formerly of the Home
Shopping Network. In her office, Grossman had talked to me about
personalizing the company's mobile app and creating greater moments for
connection. She is tan and very blond, with pink lipstick; she looks
like the second coming of Jean Nidetch. Weight Watchers had found its
business leader. She was joining the company after its fourth
consecutive quarter of revenue growth because it had finally found its
spiritual one.

On the release day of the commercial in which Oprah told the world she
loved bread and was excited to be able to eat it every day and still
lose weight, the graph line shot up tall and straight at Weight
Watchers. But a lot of us wondered if maybe Oprah had finally fallen out
of touch. She said in one commercial, ``Inside every overweight woman is
a woman she knows she can be,'' saying she'd been buried in her weight
to the point where she couldn't recognize herself, and the internet did
not love this sentiment, asking exactly why Oprah thought that women
were worthless if they weren't thin. They asked if she was
``disempowering women.'' They said her investment in Weight Watchers was
``bad news for women everywhere.'' One blogger wrote that she was
``disappointed that she is choosing to participate in and endorse a
company whose sole purpose is to tell women that they are not enough.''
The journalist Melissa Harris-Perry gave a five-minute ``Letter of the
Week'' on MSNBC saying: ``But, O! You are already precisely the woman so
many are striving to be,'' and ``there is not one thing that you have
done that would have been more extraordinary if you'd done it with a
25-inch waist.'' Oprah's \$43 million investment was now worth \$110
million. Maybe that's what this was all about in the first place.

Oprah was used to criticism. Back in 1985, Joan Rivers brought Oprah
\href{https://www.youtube.com/watch?v=TAtjDjZa2eA}{on ``The Tonight
Show,''} and without so much as a warning in the pre-interview, told her
she shouldn't have ``let'' her weight gain happen to her. ``You're a
pretty girl, and you're single,'' Rivers said. Oprah explained that she
had done everything so far --- everything! By 1985! She had done the
banana-hot-dog-egg diet (in which you just eat a banana and a hot dog
and an egg). She had done the pickles-and-peanut-butter diet (in which
all you eat are pickles and peanut butter).

In 1988, she pulled a wagon full of fat onto the stage of her show to
show off her 67-pound weight loss. In 1991, she went on the cover of
People and declared she was never dieting again. In 1996 she wrote a
book with Bob Greene about having found the solution. In 2002, she wrote
a story in her magazine, O, called ``What I Know for Sure About Making
Peace With My Body,'' in which she announced that she had made peace
with her body. In 2005, the cover of O, which usually features just one
Oprah, featured two: a thin one with an exposed midriff leaning on the
shoulder of another thin one in a fancy dress. In 2009, she published
another two-Oprah cover. This one was the midriff-bearing one from the
2005 cover, leaning on a larger Oprah in a purple jogging suit. The
cover line said, \emph{How did I let this happen again?}

Oprah sounds like Oprah when you talk with her --- she sings your name,
``Taffy!'' and her voice registers in you in a way that is as familiar
to your body as your mother's voice. She told me she doesn't care if
she's never skinny again. She cares that she feels as if she has
control. For her whole life, she said, her only goal has been to find a
higher level of consciousness, to remain more in the moment than she has
ever been in any other moment. She had never felt stress, even during
all those years when she was doing three shows a day. She just ate
instead. She had bags of potato chips, and people would say, ``Don't you
get stressed?'' and she'd think, What's stress? She had seen the
cultural changes for years. She knew that you were no longer supposed to
say that you wanted to diet or be thinner. You had to want ``fitness''
and ``strength'' and just general health. But this thinking was a
prison. So was the one where you just accept yourself and move on.
``This whole P.C. about accepting yourself as you are --- you should,
100 percent,'' she said. But it was that thinking that made her say yes
to Weight Watchers. ``It's a mechanism to keep myself on track that
brings a level of consciousness and awareness to my eating. It actually
is, for me, mindful eating, because the points are so ingrained now.''
Meaning, Oprah wasn't interested in ceding to a movement. She was
wondering how to finally make this work.

``In the particular moment in time that I got the call,'' she told me,
``I was desperate: What's going to work? I've tried all of the green
juices and protein shakes, and let's do a cleanse, and all that stuff.
That doesn't work. It doesn't last. What is going to be consistent, keep
me conscious and mindful?''

But this thing about acceptance? Why couldn't accepting herself mean not
accepting her weight? Why wasn't it an act of love to use any available
means to avoid her genetic predisposition to diabetes? Sure, she could
have abandoned her efforts. She could have gone hard on acceptance. A
million people would have bought ``Oprah's Guide to Body Acceptance.''
But she couldn't get there. ``For your heart to pump, pump, pump, pump,
it needs the least amount of weight possible to do that,'' she said.
``So all of the people who are saying, `Oh, I need to accept myself as I
am' --- I can't accept myself if I'm over 200 pounds, because it's too
much work on my heart. It causes high blood pressure for me. It puts me
at risk for diabetes, because I have diabetes in my family.''

I nodded into the phone because I didn't want Oprah to hear me crying. I
wanted to quit dieting, but had come to realize that dieting was all I
had. I was completely perplexed by food --- food! Stupid food! That's
what this was about! I dieted because I wanted to maintain hope that I
could one day manage my food intake, because my bewilderment around the
stuff was untenable. When I didn't have that hope, I was left with too
much worry about pain, about how much my knees hurt now and how much
more they would in just a few years. I could be enlightened about my
body. I could have acceptance. But nobody would tell that to the people
who saw me as a target; nobody would tell that to my knees.

And yet, I told Oprah, in admitting this, I couldn't stop feeling as if
I were betraying everyone I knew who was out there trying to find peace
with herself. I couldn't stop thinking that nothing would change in the
world until there was a kind of uprising.

``Oh, my God, Taffy,'' Oprah said. ``I have to have a talk with you. I
used to say this to my producers all the time. We are never going to win
with this show looking back to see what other people are doing on their
shows. The only way you win is to keep looking forward for yourself.
What's best for you?''

The ``you'' threw me. I didn't know if she meant ``you'' as in my body
or ``you'' as in me, and it occurred to me that she could mean both,
that some people think of those two things as the same thing. I treated
my body with such contempt, but my body wasn't different from me. There
were no two of me to put on a magazine cover, just the one of me.

Weight isn't neutral. A woman's body isn't neutral. A woman's body is
everyone's business but her own. Even in our attempts to free one
another, we were still trying to tell one another what to want and what
to do. It is terrible to tell people to try to be thinner; it is also
terrible to tell them that wanting to lose weight is hopeless and wrong.

I don't know if diets can work in the short term or the long term. For
the first time, I began to think that this was something worth being
made crazy over. Our bodies deserve our thoughts and our kindness, our
acceptance and our striving. Our bodies are what carry our thoughts and
our kindness and our acceptance and striving.

\textbf{On Saturday, March 18,} Donna, of the meeting in Union, made her
goal weight. Six weeks later, having maintained the weight, she became a
lifetime member. If she stayed within a few pounds of her goal, she
could keep using the program free. There were other lifetime members in
our meeting. There were also former lifetime members who were starting
over.

Eileen, a lifetime member who sat next to Donna at every meeting, had
bought her a plastic tiara. Donna wore leggings this time, not
sweatpants as usual, with her traditional Uggs and a fleece, and someone
pointed out that you could finally see her shape. She passed around some
old pictures; she was unrecognizable in them, if you could find her
behind all the other people in the picture.

``I don't think I'll ever feel like a thin person,'' Donna said. Her
hope is that she'll continue to at least look like one.

Dayna, near sobbing, gave her a bunch of star stickers off her roll.
``My heart just feels so happy today,'' she said.

We all cheered for Donna, and when I left, I walked around outside. A
skinny woman was eating a cupcake and talking on her phone, tonguing the
icing as if she were on ecstasy. Another skinny woman drank a regular Dr
Pepper as if it were nothing, as if it were just a drink. I continued
walking and stopped in front of a diner and watched through the window
people eating cheeseburgers and French fries and talking gigantically.
All these people, I looked at them as if they were speaking Mandarin or
discussing string theory, with their ease around their food and their
ease around their bodies and their ability to live their lives without
the doubt and self-loathing that brings me to my arthritic knees still.
There's no such thing as magic, Taffy. I shook my head at the
impossibility of it all, and sitting here writing this, I still do.

Advertisement

\protect\hyperlink{after-bottom}{Continue reading the main story}

\hypertarget{site-index}{%
\subsection{Site Index}\label{site-index}}

\hypertarget{site-information-navigation}{%
\subsection{Site Information
Navigation}\label{site-information-navigation}}

\begin{itemize}
\tightlist
\item
  \href{https://help.nytimes3xbfgragh.onion/hc/en-us/articles/115014792127-Copyright-notice}{©~2020~The
  New York Times Company}
\end{itemize}

\begin{itemize}
\tightlist
\item
  \href{https://www.nytco.com/}{NYTCo}
\item
  \href{https://help.nytimes3xbfgragh.onion/hc/en-us/articles/115015385887-Contact-Us}{Contact
  Us}
\item
  \href{https://www.nytco.com/careers/}{Work with us}
\item
  \href{https://nytmediakit.com/}{Advertise}
\item
  \href{http://www.tbrandstudio.com/}{T Brand Studio}
\item
  \href{https://www.nytimes3xbfgragh.onion/privacy/cookie-policy\#how-do-i-manage-trackers}{Your
  Ad Choices}
\item
  \href{https://www.nytimes3xbfgragh.onion/privacy}{Privacy}
\item
  \href{https://help.nytimes3xbfgragh.onion/hc/en-us/articles/115014893428-Terms-of-service}{Terms
  of Service}
\item
  \href{https://help.nytimes3xbfgragh.onion/hc/en-us/articles/115014893968-Terms-of-sale}{Terms
  of Sale}
\item
  \href{https://spiderbites.nytimes3xbfgragh.onion}{Site Map}
\item
  \href{https://help.nytimes3xbfgragh.onion/hc/en-us}{Help}
\item
  \href{https://www.nytimes3xbfgragh.onion/subscription?campaignId=37WXW}{Subscriptions}
\end{itemize}
