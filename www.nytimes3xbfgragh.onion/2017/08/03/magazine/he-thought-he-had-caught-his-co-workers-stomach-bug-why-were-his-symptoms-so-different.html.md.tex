Sections

SEARCH

\protect\hyperlink{site-content}{Skip to
content}\protect\hyperlink{site-index}{Skip to site index}

\href{https://myaccount.nytimes3xbfgragh.onion/auth/login?response_type=cookie\&client_id=vi}{}

\href{https://www.nytimes3xbfgragh.onion/section/todayspaper}{Today's
Paper}

He Thought He Had Caught His Co-Worker's Stomach Bug. Why Were His
Symptoms So Different?

\url{https://nyti.ms/2hq9oPz}

\begin{itemize}
\item
\item
\item
\item
\item
\item
\end{itemize}

Advertisement

\protect\hyperlink{after-top}{Continue reading the main story}

Supported by

\protect\hyperlink{after-sponsor}{Continue reading the main story}

\href{/column/diagnosis}{Diagnosis}

\hypertarget{he-thought-he-had-caught-his-co-workers-stomach-bug-why-were-his-symptoms-so-different}{%
\section{He Thought He Had Caught His Co-Worker's Stomach Bug. Why Were
His Symptoms So
Different?}\label{he-thought-he-had-caught-his-co-workers-stomach-bug-why-were-his-symptoms-so-different}}

\includegraphics{https://static01.graylady3jvrrxbe.onion/images/2017/08/06/magazine/06diagnosis1/06diagnosis1-articleInline.jpg?quality=75\&auto=webp\&disable=upscale}

By Lisa Sanders, M.d.

\begin{itemize}
\item
  Aug. 3, 2017
\item
  \begin{itemize}
  \item
  \item
  \item
  \item
  \item
  \item
  \end{itemize}
\end{itemize}

When Dr. Jennifer Girard walked into the darkened room, her first
thought was that the 61-year-old man looked too healthy to be in the
hospital. It was well past midnight, and though his pale blue eyes had
the puffiness of fatigue, he was awake and alert.

She introduced herself to her newest patient. She was a resident
finishing up her second year of training and was caring for the patients
who came in that night.

\hypertarget{a-sudden-fever}{%
\subsection{\texorpdfstring{\textbf{A Sudden
Fever}}{A Sudden Fever}}\label{a-sudden-fever}}

He sat up, put on his glasses and explained why he had come to the
hospital. He'd just returned home to Boston from a weeklong trip to
Edinburgh. Once back, his co-worker, a young woman half his age,
developed a horrible stomach illness --- lots of pain and diarrhea ---
that kept her home for the next two days. But he'd felt fine.

Around the time she was starting to feel better, he woke up feeling
tired and achy. A museum curator, he was leading a group that day to a
nearby town to view a collection, and he thought he could manage. But
that afternoon, on the way home in the tour bus, he began to shiver. He
went home and got into bed. He dug up a thermometer and stuck it under
his tongue. It read 101.

The next day he was too sick to work. He was expecting that he had the
same stomach bug his co-worker had. But though his stomach felt tender,
the violent symptoms never arrived. He just felt feverish and sore. When
his temperature rose to 103 the following day, he called his doctor's
office and went in the next afternoon. He was still blazing with fever,
and the doctor sent him to Beth Israel Deaconess Medical Center in
Boston.

\hypertarget{the-danger-of-a-helpful-drug}{%
\subsection{\texorpdfstring{\textbf{The Danger of a Helpful
Drug}}{The Danger of a Helpful Drug}}\label{the-danger-of-a-helpful-drug}}

In the E.R., he was given high-dose acetaminophen to bring his fever
down. And he spent the rest of the day being questioned, prodded, poked
and scanned. None of the tests were revealing, but no one wanted to send
him home, either. He understood why. For the past two years, he'd been
taking a medication called etanercept for Sjogren's syndrome. The drug
works by tamping down the immune system, and that limited his ability to
fight off infections.

Sjogren's is an autoimmune disorder --- a somewhat mysterious type of
disease in which the body's own white blood cells attack normal tissue
as if it were an invading organism --- in this case wreaking havoc on
the glands that make tears and saliva, as well as on some of the joints.
A weekly injection of etanercept had been helpful, eliminating his joint
pain. But he knew that because of this drug, he was almost helpless in
the face of a serious infection.

Once the acetaminophen kicked in, the patient felt much better. The E.R.
doctors decided to admit him overnight. If, as the doctors suspected, he
had a virus, he'd be fine the next day and could go home.

\includegraphics{https://static01.graylady3jvrrxbe.onion/images/2017/08/06/magazine/06diagnosis2/06mag-06diagnosis.t_CA1-articleLarge.jpg?quality=75\&auto=webp\&disable=upscale}

\hypertarget{in-search-of-infections}{%
\subsection{\texorpdfstring{\textbf{In Search of
Infections}}{In Search of Infections}}\label{in-search-of-infections}}

When Girard first heard about her new patient, she read up on the
infections to which people who took etanercept were susceptible. It was
a long list. Viruses couldn't be treated before diagnosis, so she
focused on bacterial infections. One bacterium that attacks those with
immune deficiencies is food-borne and called listeria. It is often found
in unpasteurized milk and cheeses and tends to cause horrible nausea,
vomiting and diarrhea, which he didn't have. Legionella pneumonia is
another infection more common in those who take the medication, but he
didn't have a cough and wasn't short of breath. A much more common bug,
staph aureus, could cause joint infections --- though again, he didn't
have any complaints about his joints.

Still, she planned to start him on broad-spectrum antibiotics to cover
all these threats. But when she met him, he didn't look like someone who
had a terrible infection. The doctors in the E.R. hadn't given him
antibiotics. She could hold off on the antibiotics and just watch and
wait --- but if he had one of the infections she'd read about, delaying
treatment even a few hours could be deadly. To make matters more
complicated, he thought he was allergic to penicillin --- though he
couldn't remember what happened when he took it. This shifted the
risk-benefit ratio. The normally low-stakes decision of whether to start
an antibiotic changed when doing so could be so dangerous.

She asked a colleague, another resident, to assess the patient. Girard
didn't tell her co-worker that she planned on an antibiotic; she just
asked her to see the patient and give a recommendation. The co-worker
saw the patient and agreed. Even though he looked well, if he had an
infection, he was practically defenseless against it. She should give
him the antibiotics.

\hypertarget{a-new-fever}{%
\subsection{\texorpdfstring{\textbf{A New
Fever}}{A New Fever}}\label{a-new-fever}}

Because he had a history of a penicillin allergy, Girard would have to
test him to make sure he could tolerate the medication, which was not
penicillin but a close cousin. Girard ordered a test dose to be given.
If he was going to have a serious reaction to the medicine, he would
have a rash or his blood pressure would drop. Neither happened. So he
got the full dose.

A couple of hours later, Girard got a call. The patient had spiked a
temperature of 102. She ran down to see him. His face was shiny with a
thin layer of sweat, and his cheeks were rosy. And though he looked
uncomfortable, his heart and blood pressure were fine. Thank goodness
she'd given him antibiotics, she thought to herself. She ordered another
dose of Tylenol, and when she came back for one final look at him, he
was sound asleep. His temperature was gone. The sweaty gleam and rosy
cheeks had disappeared.

\hypertarget{an-answer-emerges}{%
\subsection{\texorpdfstring{\textbf{An Answer
Emerges}}{An Answer Emerges}}\label{an-answer-emerges}}

The next night, when Girard returned to the hospital for her overnight
shift, the first thing she did was check on this patient. She had
admitted five other patients, but he was the one she worried about from
home after signing him over to the day team. In the hours since she last
saw him, his blood had grown an organism in the lab. It was too early to
know precisely what it was. But it was clear that it was bacteria robust
enough to grow quickly in a petri dish and that it wasn't staph aureus
or Legionella. The list of other possibilities was long, and none really
seemed to fit.

It was another two days before they could finally identify the bug. It
was listeria. Girard --- and just about everyone caring for the man ---
was surprised. As a physician, you get these pictures in your head about
diseases. For Girard, listeriosis was a disease that you got from
unpasteurized milk and soft cheeses and that caused nausea, vomiting and
diarrhea. She did a little more reading on the unusual infection and
found that while people with normal immune systems may experience these
symptoms, they usually recover quickly, like the patient's co-worker. In
pregnant women, people over 65 or anyone with an impaired immune system,
however, the infection can break out of the digestive tract ---
sometimes without manifesting the typical symptoms --- and get into the
bloodstream. From there it can spread like wildfire through the entire
body.

This form of the infection, called invasive listeriosis, can be fatal up
to 30 percent of the time in these patients. Indeed, though less common
than botulism, listeriosis can be just as deadly. Early treatment with
antibiotics can be lifesaving.

If this was a food-borne illness, which food had carried the bacteria?
Though the man and his co-worker did not have unpasteurized milk or soft
cheese on their trip, an internet search showed an outbreak of listeria
this summer in Scotland linked to smoked beef. Neither of them had eaten
smoked beef, but they had shared some smoked salmon. Exactly where
they'd picked up the bug remained a mystery.

Once the medical team knew what he had, they could tailor the
antibiotics. He completed his three-week treatment in July. His doctors
instructed him to take it easy for the rest of the summer, but at this
point he feels mostly back to normal and has returned to his museum and
the work he loves.

Advertisement

\protect\hyperlink{after-bottom}{Continue reading the main story}

\hypertarget{site-index}{%
\subsection{Site Index}\label{site-index}}

\hypertarget{site-information-navigation}{%
\subsection{Site Information
Navigation}\label{site-information-navigation}}

\begin{itemize}
\tightlist
\item
  \href{https://help.nytimes3xbfgragh.onion/hc/en-us/articles/115014792127-Copyright-notice}{©~2020~The
  New York Times Company}
\end{itemize}

\begin{itemize}
\tightlist
\item
  \href{https://www.nytco.com/}{NYTCo}
\item
  \href{https://help.nytimes3xbfgragh.onion/hc/en-us/articles/115015385887-Contact-Us}{Contact
  Us}
\item
  \href{https://www.nytco.com/careers/}{Work with us}
\item
  \href{https://nytmediakit.com/}{Advertise}
\item
  \href{http://www.tbrandstudio.com/}{T Brand Studio}
\item
  \href{https://www.nytimes3xbfgragh.onion/privacy/cookie-policy\#how-do-i-manage-trackers}{Your
  Ad Choices}
\item
  \href{https://www.nytimes3xbfgragh.onion/privacy}{Privacy}
\item
  \href{https://help.nytimes3xbfgragh.onion/hc/en-us/articles/115014893428-Terms-of-service}{Terms
  of Service}
\item
  \href{https://help.nytimes3xbfgragh.onion/hc/en-us/articles/115014893968-Terms-of-sale}{Terms
  of Sale}
\item
  \href{https://spiderbites.nytimes3xbfgragh.onion}{Site Map}
\item
  \href{https://help.nytimes3xbfgragh.onion/hc/en-us}{Help}
\item
  \href{https://www.nytimes3xbfgragh.onion/subscription?campaignId=37WXW}{Subscriptions}
\end{itemize}
