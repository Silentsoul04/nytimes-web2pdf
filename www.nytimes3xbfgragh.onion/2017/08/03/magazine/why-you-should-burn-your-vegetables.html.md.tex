Sections

SEARCH

\protect\hyperlink{site-content}{Skip to
content}\protect\hyperlink{site-index}{Skip to site index}

\href{https://myaccount.nytimes3xbfgragh.onion/auth/login?response_type=cookie\&client_id=vi}{}

\href{https://www.nytimes3xbfgragh.onion/section/todayspaper}{Today's
Paper}

Why You Should Burn Your Vegetables

\url{https://nyti.ms/2umba5w}

\begin{itemize}
\item
\item
\item
\item
\item
\item
\end{itemize}

Advertisement

\protect\hyperlink{after-top}{Continue reading the main story}

Supported by

\protect\hyperlink{after-sponsor}{Continue reading the main story}

\href{/column/magazine-eat}{Eat}

\hypertarget{why-you-should-burn-your-vegetables}{%
\section{Why You Should Burn Your
Vegetables}\label{why-you-should-burn-your-vegetables}}

\includegraphics{https://static01.graylady3jvrrxbe.onion/images/2017/08/06/magazine/06eat2/06mag-06eat.t_CA0-articleLarge.jpg?quality=75\&auto=webp\&disable=upscale}

By \href{http://www.nytimes3xbfgragh.onion/by/sam-sifton}{Sam Sifton}

\begin{itemize}
\item
  Aug. 3, 2017
\item
  \begin{itemize}
  \item
  \item
  \item
  \item
  \item
  \item
  \end{itemize}
\end{itemize}

Less than a year ago, for reasons perhaps best explored in a clinical
setting, my friend Brendan McCarthy started to build \emph{parrillas},
huge and heavy wood-burning grills of the sort common in Argentina, to a
design of his own making. He welded together a metal cart and attached
steel walls to three sides of it. He fabricated a fire cage to hold
burning wood on one side of the cart's top and hung a grill beside it,
on cables that allow it to be cranked up and down over embers raked from
the fire. You can cook six or seven racks of ribs or legs of lamb or an
entire case of eggplants on the thing. The first parrilla took more than
a month to build and weighs around 400 pounds.

McCarthy is not a cook, and only lately a welder. He is a professional
fishing guide. As such, he is a student of behavior --- of people as
well as of fish. He figured those who like to stalk saltwater game fish
in shallow water, or to chase them through the tide rips that mount off
the islands of the Northeastern coast, might thrill as well to cooking
on a parrilla, with its evocation of grilling in the wilds of Argentina,
after a day hunting monster trout. He put the parrillas up for sale.

This was, I thought, a little crazy. Build grills? Why not just offer
trips to fish in Argentina? He only shrugged. He wanted to build things
out of steel, the way someone might wake up and want one day to write a
poem, or to plant roses. And it turned out he wasn't wrong. If you want
one of his grills right now, you'll need to wait in line. He can't keep
up production. (The fishing gets in the way.)

Because McCarthy builds his parrillas in a shop behind my house, I've
been able to cook on an early model all summer long, searing steaks,
hanging chickens and pineapples on hooks from the bar that runs over the
grill and fire cage, fire-smoking chops, ribs, whole fish, clams and
pounds and pounds of vegetables. The cooking is different from
traditional grilling, because you can moderate the heat more easily,
cranking the grates high above the coals to where they're taking on more
smoke than heat, or lowering them into an inferno to apply a burn.

\includegraphics{https://static01.graylady3jvrrxbe.onion/images/2017/08/06/magazine/06eat1/06eat1-articleInline.jpg?quality=75\&auto=webp\&disable=upscale}

I've learned a lot from that burn, lessons that apply not just to big
machines like the parrilla but also to little charcoal grills shaped
like football helmets, to cheap braziers, to state-owned grills in parks
or at beaches, to anywhere you cook outside over fire. Most important:
Burn your vegetables. Scrape the char from the exterior, and what's left
is delicate and smoky, softened and sweet. It can be a revelation, a
primitive joy.

Every summer has a song. This year's is shaping up to be Luis Fonsi and
Daddy Yankee's ``Despacito,'' featuring Justin Bieber, which seems to
play from every third car in the parking lot at the beach. I think every
summer has a recipe too, and this one is mine, ``Despacito'' for the
grill: whole cabbage cooked over fire until it is blistered and black.
After grilling, I remove the exterior leaves, cut out the core and slice
the smoky, pale interior into shreds. I dress it with Mexican
\emph{crema} to create a hot slaw of incredible sweet-smokiness, the
perfect accompaniment to almost anything you'd think to cook on the
grill. It is lovely in tacos. It is a great match for ribs. It is
brilliant with corn.

And it is simple to make. Just put a couple of whole cabbages over a hot
fire and leave them there, turning every few minutes when you get a
chance, until they look like something tragic and ruined. You don't need
to season them, or to oil them, or to remove the thick outer leaves the
way you'd do if you were cooking them lightly or shredding them raw. You
just need to burn them, slowly and deeply, so that they soften within
and take on the flavor of fire.

When it comes to making the slaw, note that the crema is as important as
the cabbage. Crema is thinner than sour cream, a little more tangy,
slightly salty. It cloaks the shredded cabbage lightly and counteracts
the smokiness while at the same time elevating it. (It's the Bieber part
of the song.) You can buy crema at the market, or at least in some
markets, but don't fret if you can't. Just combine sour cream with heavy
cream (or buttermilk if you have any), some lime juice and salt, and let
it set on the counter at room temperature until it has come together
into velvety thickness. Feel like stirring into that some hot sauce, or
a tablespoon of adobo from a can of chipotles, some chopped cilantro, an
extra spray of lime? Please do. Hot slaw welcomes improvisation. Nothing
can ruin what started out ruined.

\textbf{Recipe:}
\href{https://cooking.nytimes3xbfgragh.onion/recipes/1018884-hot-slaw-mexican-style}{Hot
Slaw, Mexican-Style} \textbar{}
\href{https://cooking.nytimes3xbfgragh.onion/recipes/1018885-crema}{Crema}

Advertisement

\protect\hyperlink{after-bottom}{Continue reading the main story}

\hypertarget{site-index}{%
\subsection{Site Index}\label{site-index}}

\hypertarget{site-information-navigation}{%
\subsection{Site Information
Navigation}\label{site-information-navigation}}

\begin{itemize}
\tightlist
\item
  \href{https://help.nytimes3xbfgragh.onion/hc/en-us/articles/115014792127-Copyright-notice}{©~2020~The
  New York Times Company}
\end{itemize}

\begin{itemize}
\tightlist
\item
  \href{https://www.nytco.com/}{NYTCo}
\item
  \href{https://help.nytimes3xbfgragh.onion/hc/en-us/articles/115015385887-Contact-Us}{Contact
  Us}
\item
  \href{https://www.nytco.com/careers/}{Work with us}
\item
  \href{https://nytmediakit.com/}{Advertise}
\item
  \href{http://www.tbrandstudio.com/}{T Brand Studio}
\item
  \href{https://www.nytimes3xbfgragh.onion/privacy/cookie-policy\#how-do-i-manage-trackers}{Your
  Ad Choices}
\item
  \href{https://www.nytimes3xbfgragh.onion/privacy}{Privacy}
\item
  \href{https://help.nytimes3xbfgragh.onion/hc/en-us/articles/115014893428-Terms-of-service}{Terms
  of Service}
\item
  \href{https://help.nytimes3xbfgragh.onion/hc/en-us/articles/115014893968-Terms-of-sale}{Terms
  of Sale}
\item
  \href{https://spiderbites.nytimes3xbfgragh.onion}{Site Map}
\item
  \href{https://help.nytimes3xbfgragh.onion/hc/en-us}{Help}
\item
  \href{https://www.nytimes3xbfgragh.onion/subscription?campaignId=37WXW}{Subscriptions}
\end{itemize}
