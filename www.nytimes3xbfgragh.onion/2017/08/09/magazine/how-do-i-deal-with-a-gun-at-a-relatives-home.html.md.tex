Sections

SEARCH

\protect\hyperlink{site-content}{Skip to
content}\protect\hyperlink{site-index}{Skip to site index}

\href{https://myaccount.nytimes3xbfgragh.onion/auth/login?response_type=cookie\&client_id=vi}{}

\href{https://www.nytimes3xbfgragh.onion/section/todayspaper}{Today's
Paper}

How Do I Deal With a Gun at a Relative's Home?

\url{https://nyti.ms/2uFcAId}

\begin{itemize}
\item
\item
\item
\item
\item
\item
\end{itemize}

Advertisement

\protect\hyperlink{after-top}{Continue reading the main story}

Supported by

\protect\hyperlink{after-sponsor}{Continue reading the main story}

\href{/column/the-ethicist}{The Ethicist}

\hypertarget{how-do-i-deal-with-a-gun-at-a-relatives-home}{%
\section{How Do I Deal With a Gun at a Relative's
Home?}\label{how-do-i-deal-with-a-gun-at-a-relatives-home}}

By Kwame Anthony Appiah

\begin{itemize}
\item
  Aug. 9, 2017
\item
  \begin{itemize}
  \item
  \item
  \item
  \item
  \item
  \item
  \end{itemize}
\end{itemize}

\includegraphics{https://static01.graylady3jvrrxbe.onion/images/2017/08/13/magazine/13ethicist/13mag-13ethicist.t_CA0-articleInline.jpg?quality=75\&auto=webp\&disable=upscale}

\emph{The patriarch of our large family came out of the closet as an
elderly man nearing the end of his life; he now has a husband who is
much younger, whom I will call Tim. The family embraced Tim, but the
adjustment has been rocky, especially among some of the men. Tim has
earned back this trust by being both husband and physical caretaker of
our ailing relative. One recent evening, while his husband was ill, Tim
and I sat alone in their home. The conversation turned to gun politics;
I'm a closeted gun rights sympathizer. Perhaps sensing some undue
camaraderie, Tim stole away to the foyer, then returned with an unloaded
black shotgun and ammunition. Tim told me not to mention this to our
family --- or to my relative, who doesn't know about the shotgun even
though they live together. We also have children in the family, who
visit Tim and his husband with frequency, and I'm well aware of the
statistics about households that keep guns. I plan to advise Tim at
least to move the gun elsewhere, out of the house. However, that does
not seem to be enough. Having the gun in the house suggests a lack of
judgment; it seems like a serious breach of trust and, God forbid,
potentially dangerous. Doesn't this directly contravene Tim's claim to
being a responsible caretaker, an ethical impetus that overrides
confidentiality? Another twist: I also know that if the gun (or
ammunition) were discovered, it could be just the excuse our extremely
anti-gun family needs to disavow Tim. This seems like a bitter pileup of
issues --- gay equality, gun safety and family loyalty.} Name Withheld

\textbf{Let's be clear:} Tim didn't show you an Altoids tin filled with
crystal meth. Provided he has the necessary permits, he is entitled to
keep a gun in his home. The largest danger posed by firearms in the
household is that they will be used for suicide, which accounts for
nearly two-thirds of gun deaths. I assume you don't think he or your
relative is at risk for that. (If they were, the solution would involve
more than getting rid of a weapon.) True, there's good evidence that
people living in homes with guns are more likely to be homicide victims
as well. And obviously, accidents with guns do occur, and you need guns
around to have accidents with them. But a reasonable person who knows
all this might decide to keep a gun. According to the Pew Research
Center, more than 44 percent of American households have at least one
gun. Tim's not depressive or alcoholic --- or you would have mentioned
it --- and it's his home.

Certainly, guns should be stored where children can't get at them; Tim
should keep his locked up. (Maybe he does.) But I don't agree that
having an unloaded gun, even with its ammunition nearby, is evidence
that you're not a responsible spouse and caretaker. Our country is full
of responsible spouses and caretakers who have guns stored safely in
their houses.

In our divided country, though, people who disagree about gun ownership
and regulation seem to be split into two great tribes. Each regards
those on the other side as not just mistaken about policy but also
wicked or corrupt. (For what it's worth, I think guns should be more
heavily regulated; I don't think gun ownership is wicked.) The members
of your family are on one side of the divide; Tim is on the other.
Because of this, telling your kinfolk he has a gun, which he showed you
in confidence, will give them an excuse to do what some of them are
inclined to do anyway, which is repudiate him.

You don't quite explain the basis of that inclination; family dynamics
are complicated, and this patriarch's new household falls into the Grace
Paley category of Enormous Changes at the Last Minute. But your mention
of gay equality suggests that you think some of it has to do with their
being disconcerted by their patriarch's pairing off with a man. That is,
you fear they would use the permissible prejudice against gun owners to
excuse the now impermissible prejudice against gay people. In these
circumstances, sharing the confidence seems not only wrong in itself but
also likely to lead to some people's behaving badly. The only reason to
tell them would be if you thought the children were at risk because of
it, and that need not be the case.

The problem with Tim's conduct isn't having the gun; it's not letting
his husband know he has it. That is a breach of trust, especially if Tim
knows your relative would disapprove. I said it's Tim's home. But it's
not just his home. You could wonder whether someone who keeps that sort
of secret fully respects his spouse. So provided Tim can assure you that
the gun is stored safely when the kids visit, I'd focus on persuading
him to tell his husband he has it. Unlike you, his husband is in a
position to ask that the gun go, if that's how he feels. If Tim won't
tell him, you may have to consider telling him yourself. The duty of
confidentiality can be overridden by sufficiently weighty
considerations. But getting in between two spouses, even if one of them
is a close relation, is a pretty serious step.

\emph{I am an American and recent college graduate teaching English to
children in China. When I arrived, I had no teaching experience
whatsoever, and I did not study anything in college related to English
or teaching. I recently found out that I make about twice as much money
as local Chinese teachers, who all studied English and teaching in
university, have advanced teaching certificates and usually have at
least a few years of teaching experience. The company I work for
explains this by saying that Chinese teachers get a fair, competitive
wage for the city we are in, and that native English speakers would not
be attracted at anything close to this wage. And yet it seems so immoral
that I should get paid far more. What can I do to lessen my guilt?} Name
Withheld

\textbf{We live in} a world where wages are determined, in part, by the
sorts of market forces your employers have mentioned. There are lots of
ways in which these forces are modified by other ones. Some, like legal
regulation, can be legitimate; others, like racial and gender prejudice,
are not. Your case doesn't seem to pose such issues. The company wants
native speakers of English; if it paid them what it paid its Chinese
teachers (who are getting a competitive wage in the local market), it
would have fewer or none. You are working a long way from home and,
presumably, for a limited time. I don't think you need to feel bad about
the premium you currently command.

Advertisement

\protect\hyperlink{after-bottom}{Continue reading the main story}

\hypertarget{site-index}{%
\subsection{Site Index}\label{site-index}}

\hypertarget{site-information-navigation}{%
\subsection{Site Information
Navigation}\label{site-information-navigation}}

\begin{itemize}
\tightlist
\item
  \href{https://help.nytimes3xbfgragh.onion/hc/en-us/articles/115014792127-Copyright-notice}{©~2020~The
  New York Times Company}
\end{itemize}

\begin{itemize}
\tightlist
\item
  \href{https://www.nytco.com/}{NYTCo}
\item
  \href{https://help.nytimes3xbfgragh.onion/hc/en-us/articles/115015385887-Contact-Us}{Contact
  Us}
\item
  \href{https://www.nytco.com/careers/}{Work with us}
\item
  \href{https://nytmediakit.com/}{Advertise}
\item
  \href{http://www.tbrandstudio.com/}{T Brand Studio}
\item
  \href{https://www.nytimes3xbfgragh.onion/privacy/cookie-policy\#how-do-i-manage-trackers}{Your
  Ad Choices}
\item
  \href{https://www.nytimes3xbfgragh.onion/privacy}{Privacy}
\item
  \href{https://help.nytimes3xbfgragh.onion/hc/en-us/articles/115014893428-Terms-of-service}{Terms
  of Service}
\item
  \href{https://help.nytimes3xbfgragh.onion/hc/en-us/articles/115014893968-Terms-of-sale}{Terms
  of Sale}
\item
  \href{https://spiderbites.nytimes3xbfgragh.onion}{Site Map}
\item
  \href{https://help.nytimes3xbfgragh.onion/hc/en-us}{Help}
\item
  \href{https://www.nytimes3xbfgragh.onion/subscription?campaignId=37WXW}{Subscriptions}
\end{itemize}
