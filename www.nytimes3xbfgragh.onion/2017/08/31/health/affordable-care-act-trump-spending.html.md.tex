Sections

SEARCH

\protect\hyperlink{site-content}{Skip to
content}\protect\hyperlink{site-index}{Skip to site index}

\href{https://www.nytimes3xbfgragh.onion/section/health}{Health}

\href{https://myaccount.nytimes3xbfgragh.onion/auth/login?response_type=cookie\&client_id=vi}{}

\href{https://www.nytimes3xbfgragh.onion/section/todayspaper}{Today's
Paper}

\href{/section/health}{Health}\textbar{}Trump Administration Sharply
Cuts Spending on Health Law Enrollment

\url{https://nyti.ms/2wVQl6A}

\begin{itemize}
\item
\item
\item
\item
\item
\item
\end{itemize}

Advertisement

\protect\hyperlink{after-top}{Continue reading the main story}

Supported by

\protect\hyperlink{after-sponsor}{Continue reading the main story}

\hypertarget{trump-administration-sharply-cuts-spending-on-health-law-enrollment}{%
\section{Trump Administration Sharply Cuts Spending on Health Law
Enrollment}\label{trump-administration-sharply-cuts-spending-on-health-law-enrollment}}

\includegraphics{https://static01.graylady3jvrrxbe.onion/images/2017/09/01/science/01HEALTH/01HEALTH-articleLarge.jpg?quality=75\&auto=webp\&disable=upscale}

By \href{http://www.nytimes3xbfgragh.onion/by/abby-goodnough}{Abby
Goodnough} and
\href{https://www.nytimes3xbfgragh.onion/by/robert-pear}{Robert Pear}

\begin{itemize}
\item
  Aug. 31, 2017
\item
  \begin{itemize}
  \item
  \item
  \item
  \item
  \item
  \item
  \end{itemize}
\end{itemize}

WASHINGTON --- The Trump administration is slashing spending on
advertising and promotion for enrollment under the Affordable Care Act,
a move some critics charged was a blatant attempt to sabotage the law.

Officials with the Department of Health and Human Services, who insisted
on not being identified during a conference call with reporters, said on
Thursday that the advertising budget for the open enrollment period that
starts in November would be cut to \$10 million, compared with \$100
million spent by the Obama administration last year, a drop of 90
percent. Additionally, grants to about 100 nonprofit groups, known as
navigators, that help people enroll in health plans offered by the
insurance marketplaces will be cut to a total of \$36 million, from
about \$63 million.

The officials said the administration believed that the cuts were
necessary because of ``diminishing returns'' from advertising. They said
the number of first-time enrollees in Affordable Care Act coverage fell
by 42 percent this year, compared with 2016. In addition, they said that
many navigator groups failed to meet their enrollment targets last year,
despite sometimes receiving hundreds of thousands of dollars in federal
funds.

The Senate Democratic leader, Chuck Schumer of New York, denounced the
cutbacks. ``The Trump administration is deliberately attempting to
sabotage our health care system,'' he said. ``When the number of people
with health insurance declines and costs skyrocket, the American people
will know who's to blame.''

Since Republicans in Congress failed to pass legislation to repeal and
replace the law, a longstanding goal of the party, President Trump has
stepped up attacks on the legislation and repeatedly insisted it was
failing and insurance markets were imploding.

Over all, 12.2 million people selected or automatically re-enrolled in
marketplace plans for 2017, although the Trump administration said in
June that the number of customers who paid their premiums had dropped to
10.1 million. Similar drops have occurred in previous years after people
lost coverage after they failed pay their premiums.

The administration officials who briefed reporters on Thursday said it
made no sense to continue spending as much on promoting the law's
coverage options because most Americans know about them already. Instead
of television ads, the administration will run radio and digital ads and
send emails and texts to existing enrollees, they said. They added that
any outreach would emphasize that the enrollment period will be much
shorter this year --- from Nov. 1 to Dec. 15. Last year's open
enrollment period ran from Nov. 1 through Jan. 31.

``People are aware the products are out there,'' one official said,
``and they are aware they can sign up.''

Although no navigator group will lose its funding completely, the
amounts they get will be based on how close they came to meeting their
enrollment goal for 2017. If a group reached only 30 percent of its
enrollment goal, for example, it would receive 30 percent of the grant
it got last year.

``We are moving forward by matching funding to performance,'' one
official said.

Another official added that last year, navigator groups enrolled about
80,000 people --- less than half of what the groups aspired to, and only
0.7 percent of overall enrollees for 2017 --- in marketplace plans. Some
groups, they said, enrolled only a handful of people.

But the figures they cited appeared not to include people who met with
navigators to sort through coverage options but enrolled later on their
own --- a not insignificant group, according to past surveys on
enrollment.

Navigators defended their work. Alisha Keezer, a health care navigator
at the Maine Lobstermen's Association in Kennebunk, said the cutbacks
were ``shocking to all of us.''

``We had no forewarning,'' Ms. Keezer said. ``This is heartbreaking.
Here in Maine, we have helped many fishermen enroll in coverage ---
self-employed people who have never had health insurance before.''

The navigator groups whose funding may be cut are only in the 38 states
that rely on the federal Affordable Care Act marketplace,
HealthCare.gov. Twelve other states run their own marketplaces and fund
their own enrollment programs.

``It's very disappointing that the administration is minimizing the
importance of in-person assistance to millions of people who have relied
on it to understand how to enroll and how to use their insurance,'' said
Shelli D. Quenga, the program director at the Palmetto Project in South
Carolina, a nonprofit group that received about \$1 million to help with
outreach and enrollment in the last 12 months. ``These are not easy
discussions for people who may have been uninsured all their lives.''

Ms. Quenga said it was not true that most eligible people now know about
the coverage available under the Affordable Care Act. ``It's ludicrous
to believe that everyone now knows that the Affordable Care Act is alive
and well and open for enrollment,'' Ms. Quenga said. ``Many people
believe that the law is dead, or will be dead, based on the
administration's claims.''

Advertisement

\protect\hyperlink{after-bottom}{Continue reading the main story}

\hypertarget{site-index}{%
\subsection{Site Index}\label{site-index}}

\hypertarget{site-information-navigation}{%
\subsection{Site Information
Navigation}\label{site-information-navigation}}

\begin{itemize}
\tightlist
\item
  \href{https://help.nytimes3xbfgragh.onion/hc/en-us/articles/115014792127-Copyright-notice}{©~2020~The
  New York Times Company}
\end{itemize}

\begin{itemize}
\tightlist
\item
  \href{https://www.nytco.com/}{NYTCo}
\item
  \href{https://help.nytimes3xbfgragh.onion/hc/en-us/articles/115015385887-Contact-Us}{Contact
  Us}
\item
  \href{https://www.nytco.com/careers/}{Work with us}
\item
  \href{https://nytmediakit.com/}{Advertise}
\item
  \href{http://www.tbrandstudio.com/}{T Brand Studio}
\item
  \href{https://www.nytimes3xbfgragh.onion/privacy/cookie-policy\#how-do-i-manage-trackers}{Your
  Ad Choices}
\item
  \href{https://www.nytimes3xbfgragh.onion/privacy}{Privacy}
\item
  \href{https://help.nytimes3xbfgragh.onion/hc/en-us/articles/115014893428-Terms-of-service}{Terms
  of Service}
\item
  \href{https://help.nytimes3xbfgragh.onion/hc/en-us/articles/115014893968-Terms-of-sale}{Terms
  of Sale}
\item
  \href{https://spiderbites.nytimes3xbfgragh.onion}{Site Map}
\item
  \href{https://help.nytimes3xbfgragh.onion/hc/en-us}{Help}
\item
  \href{https://www.nytimes3xbfgragh.onion/subscription?campaignId=37WXW}{Subscriptions}
\end{itemize}
