Sections

SEARCH

\protect\hyperlink{site-content}{Skip to
content}\protect\hyperlink{site-index}{Skip to site index}

The Incarcerated Women Who Fight California's Wildfires

\url{https://nyti.ms/2wUIW7B}

\begin{itemize}
\item
\item
\item
\item
\item
\item
\end{itemize}

\hypertarget{wildfires-in-the-west}{%
\subsubsection{\texorpdfstring{\href{https://www.nytimes3xbfgragh.onion/spotlight/california-wildfires?name=styln-california-wildfires\&region=TOP_BANNER\&block=storyline_menu_recirc\&action=click\&pgtype=Article\&impression_id=8da7f380-f529-11ea-a612-19c737f1de2c\&variant=undefined}{Wildfires
in the West}}{Wildfires in the West}}\label{wildfires-in-the-west}}

\begin{itemize}
\tightlist
\item
  live\href{https://www.nytimes3xbfgragh.onion/2020/09/12/us/wildfires-live-updates.html?name=styln-california-wildfires\&region=TOP_BANNER\&block=storyline_menu_recirc\&action=click\&pgtype=Article\&impression_id=8da7f381-f529-11ea-a612-19c737f1de2c\&variant=undefined}{Fires
  Updates}
\item
  \href{https://www.nytimes3xbfgragh.onion/interactive/2020/us/fires-map-tracker.html?name=styln-california-wildfires\&region=TOP_BANNER\&block=storyline_menu_recirc\&action=click\&pgtype=Article\&impression_id=8da81a90-f529-11ea-a612-19c737f1de2c\&variant=undefined}{Maps
  of the Fires}
\item
  \href{https://www.nytimes3xbfgragh.onion/article/wildfires-photos-california-oregon-washington-state.html?name=styln-california-wildfires\&region=TOP_BANNER\&block=storyline_menu_recirc\&action=click\&pgtype=Article\&impression_id=8da81a91-f529-11ea-a612-19c737f1de2c\&variant=undefined}{Photos}
\item
  \href{https://www.nytimes3xbfgragh.onion/2020/09/10/us/climate-change-california-wildfires.html?name=styln-california-wildfires\&region=TOP_BANNER\&block=storyline_menu_recirc\&action=click\&pgtype=Article\&impression_id=8da81a92-f529-11ea-a612-19c737f1de2c\&variant=undefined}{A
  Climate Reckoning}
\item
  \href{https://www.nytimes3xbfgragh.onion/article/wildfires-california-oregon-washington.html?name=styln-california-wildfires\&region=TOP_BANNER\&block=storyline_menu_recirc\&action=click\&pgtype=Article\&impression_id=8da81a93-f529-11ea-a612-19c737f1de2c\&variant=undefined}{Answers
  to Your Questions}
\item
  \href{https://www.nytimes3xbfgragh.onion/2020/09/09/us/california-wildfires.html?name=styln-california-wildfires\&region=TOP_BANNER\&block=storyline_menu_recirc\&action=click\&pgtype=Article\&impression_id=8da81a94-f529-11ea-a612-19c737f1de2c\&variant=undefined}{Newsletter}
\end{itemize}

\includegraphics{https://static01.graylady3jvrrxbe.onion/images/2017/09/03/magazine/03firefighters1/03firefighters1-articleLarge.jpg?quality=75\&auto=webp\&disable=upscale}

Feature

\hypertarget{the-incarcerated-women-who-fight-californias-wildfires}{%
\section{The Incarcerated Women Who Fight California's
Wildfires}\label{the-incarcerated-women-who-fight-californias-wildfires}}

By choice, for less than \$2 an hour, the female inmate firefighters of
California work their bodies to the breaking point. Sometimes they even
risk their lives.

A crew from Rainbow Camp cutting the line on a small fire near Hemet, in
June.Credit...Peter Bohler for The New York Times

Supported by

\protect\hyperlink{after-sponsor}{Continue reading the main story}

By Jaime Lowe

\begin{itemize}
\item
  Aug. 31, 2017
\item
  \begin{itemize}
  \item
  \item
  \item
  \item
  \item
  \item
  \end{itemize}
\end{itemize}

\includegraphics{https://static01.graylady3jvrrxbe.onion/images/2017/01/29/podcasts/the-daily-album-art/the-daily-album-art-articleInline-v2.jpg?quality=75\&auto=webp\&disable=upscale}

\hypertarget{listen-to-this-article}{%
\subsubsection{Listen to This Article.}\label{listen-to-this-article}}

Jaime Lowe explores California's all-but-invisible line of defense
against the wildfires --- female inmate firefighters undertaking
grueling physical work and sometimes risking their lives.

\emph{To hear more audio stories from publishers like The New York
Times,}
\href{https://www.audm.com/?utm_source=nytmag\&utm_medium=embed\&utm_campaign=incarcerated_women_california}{\emph{download
Audm for iPhone or Android.}}

**S**hawna Lynn Jones climbed from the back of a red truck with ``L.A.
County Fire'' printed on its side. Ten more women piled out after her,
at a spot on the border of Agoura Hills and Malibu, in Southern
California. They could see flames in the vicinity of Mulholland Highway,
from a fire that had been burning for about an hour. Jones and her crew
wore helmets and yellow Nomex fire-retardant suits; yellow handkerchiefs
covered their mouths and necks. Each woman carried 50 pounds of
equipment in her backpack: gloves, flares, food, full water bottles,
safety and medical gear and an emergency shelter, in case they were
surrounded by flames. As the ``second saw,'' Jones was one of two women
who carried a chain saw with her. She was also one of California's 250
or so female-inmate firefighters.

Jones worked side by side with Jessica Ornelas, the ``second bucker,''
who collected whatever wood Jones cut down. Together they were
responsible for ``setting the line,'' which meant clearing potential
fuel from a six-foot-wide stretch of ground between whatever was burning
and the land they were trying to protect. If they did their job right, a
fire might be contained. But any number of things could quickly go wrong
--- a slight wind shift, the fall of a burning tree --- and the fire
would jump the break.

``This is what I get for wishing for live flames,'' Jones said to
Ornelas on the truck ride.

Image

Shawna Lynn Jones, who died working on a fire with Malibu 13-3 in
February 2016.Credit...From Diana Baez

It was just after 3 a.m. on Feb. 25, 2016, when Malibu 13-3, the
12-woman crew Jones belonged to, arrived at the Mulholland fire, ahead
of any aerial support or local fire trucks. The inmates --- including
men, roughly 4,000 prisoners fight wildfires alongside civilian
firefighters throughout California --- immediately went to work. They
operated in hookline formation, moving in order of rank, which was
determined by task and ability. Fire captains divide the line into the
cutting section and the scraping section. The first saw, or hook, leads;
second saw is next. The Pulaskis, nicknamed for their tool, a type of
shovel, follow. The McLeods, also named for their hand tool, rake the
scorched remains. Mulholland was Jones's first fire as second saw; she
was promoted the previous week. It took only four months for captains to
notice her after she began training, and she quickly rose from the back
of the hookline, where all inmates start, to the front.

This part of Southern California, inland from the Pacific Coast Highway,
is full of ravines and dry brush. Season after season, its protected
lands are prone to landslides, flash floods and wildfires. The women
scrambled over a slope that was full of loose soil and rocks, which made
digging the containment line --- a trench of sorts --- even more
challenging. ``It was very steep,'' Tyquesha Brown, a member of the crew
who was there, told me. ``The fire was jumping.'' As the crew moved
toward the flames, tools in hand, the firefighters kept a distance of 10
feet between each other and called out conditions.

\includegraphics{https://static01.graylady3jvrrxbe.onion/images/2017/09/03/magazine/03firefighters8-index/03firefighters8-index-articleInline.jpg?quality=75\&auto=webp\&disable=upscale}

Ornelas could tell that Jones was struggling with the weight of her
chain saw as they hiked up the slope. ``I was pushing her, she was
sliding down,'' Ornelas says. ``It was just too heavy for her. She
wasn't used to the weight.'' With every step they took forward, it felt
as if they were slipping at least one step back. But by 7:30 a.m., a
little more than a third of the fire was considered contained. Crew 13-3
had done its job: the fire didn't jump the line; it didn't threaten
homes or ranches or coastal properties.

By 10 the next morning, Jones was dead. She was 22. Her three-year
sentence had less than two months to go.

\textbf{California's inmate} firefighters choose to take part in the
grinding and dangerous work they do. And they get paid for it, though
not much. They have to pass a fitness test before they can qualify for
fire camps. But once they are accepted into a camp, the training they
receive, which often lasts as little as three weeks, is significantly
less than the three-year apprenticeship that full-time civilian
firefighters get.

Inmate labor in California goes back to the mid-19th century and the
earliest official state prison, located on the Waban, a 268-ton ship. In
1852, its prisoners slept on deck at night and spent their days building
San Quentin, the state's first permanent prison. By 1923, California's
road crews, made up of inmates who worked on highway construction, were
receiving wages, albeit low wages, for their labor. During World War II,
California turned its prisons into factories for the military industry
and moved inmates into the temporary forestry camps of the Civilian
Conservation Corps, a public work-relief program created during the
Depression. They built roads, harvested crops and repaired
infrastructure. In 1946, as part of Gov. Earl Warren's Prisoner
Rehabilitation Act, the state opened Camp Rainbow which --- under the
joint supervision of the state's Division of Forestry and the California
Department of Corrections (later renamed the California Department of
Corrections and Rehabilitation by Gov. Arnold Schwarzenegger) --- housed
inmates to clear fire lines. This setup was so cost-effective that by
1959 Gov. Edmund G. Brown promised to double the size of the
Conservation Camp Program. It now partners with Cal Fire and the Los
Angeles County Fire Department. ``Any fire you go on statewide, whether
it be small or large, the inmate hand crews make up anywhere from 50 to
80 percent of the total fire personnel,'' says Lt. Keith Radey, the
commander who is in charge of a camp where women train.

Image

Dionne Davis of Rainbow Camp.Credit...Peter Bohler for The New York
Times

Image

Sarah Meenahan of Rainbow Camp.Credit...Peter Bohler for The New York
Times

When they work, California's inmates typically earn between 8 cents and
95 cents an hour. They make office furniture for state employees, state
license plates, prison uniforms, anything that any state institution
might use. But wages in the forestry program, while still wildly low by
outside standards, are significantly better than the rest. At Malibu 13,
one of three conservation camps that house women, the commander, John
Scott, showed me a printout: Inmate firefighters can make a maximum of
\$2.56 a day in camp and \$1 an hour when they're fighting fires.

Those higher wages recognize the real dangers that inmate firefighters
face. In May, one man was crushed by a falling tree in Humboldt County;
in July, another firefighter died within a week after accidentally
cutting his leg and femoral artery on a chain saw. But, after visiting
three camps over a year and a half, I could see why inmates would accept
the risks. Compared with life among the general prison population, the
conservation camps are bastions of civility. They are less violent and
offer more space. They smell of eucalyptus, the ocean, fresh blooms.
They provide barbecue areas for families who visit; one camp has a small
cabin where relatives can stay with an inmate for up to three days. They
have woodworking areas, softball fields and libraries full of donated
mysteries and romance novels. ``I always up-talk the program,'' an
inmate named Amber Sapp told me. She noted how the quality of time
served is so much better than that in most correctional facilities.
``You see it on the women's faces, on the staff's faces.''

Still, when they're at work, the inmates look like chain gangs without
the chains, especially when out working in Malibu, where the average
annual household income is \$238,000. ``The pay is ridiculous,''
La'Sonya Edwards, 35, told me during a break from clearing a fire road.
``There are some days we are worn down to the core,'' she said. ``And
this isn't that different from slave conditions. We need to get paid
more for what we do.'' Edwards makes about \$500 a year in camp, plus
whatever she earns while on the fire line, which might add up to a few
hundred dollars in a month; the pay for a full-time civilian firefighter
starts at about \$40,000. In 1999, in a study funded by the Open Society
Institute, five prominent economists argued for basic worker rights,
including minimum wages, for inmates. Those standards have not been
widely embraced, however. David Fathi, the director of the A.C.L.U.
National Prison Project, who opposes all forms of prison labor, told me,
``I think one important question to ask is, if these people are safe to
be out and about and carrying axes and chain saws, maybe they didn't
need to be in prison in the first place.''

C.D.C.R. says that the firefighter program is intended to serve as
rehabilitation for the inmates. Yet they're being trained to work in a
field they will probably have trouble finding a job in when they get
out: Los Angeles County Fire won't hire felons and C.D.C.R. doesn't
offer any formal help to inmates who want firefighting jobs when they're
released.

This institutional disinterest makes more sense when inmate
firefighters, who are on-call continuously, are considered as a state
resource. The Conservation Camp Program saves California taxpayers
approximately \$100 million a year, according to C.D.C.R. Several
states, including Arizona, Nevada, Wyoming and Georgia employ prisoners
to fight fires, but none of them rely as heavily on its inmate
population as California does. In the fall of 2014, as the state's
courts were taking up the issue of overcrowded prisons, the office of
California's attorney general argued against shrinking the number of
inmates. Doing so, it claimed, ``would severely impact fire camp
participation, a dangerous outcome while California is in the middle of
a difficult fire season and severe drought.'' In 2015, Gov. Jerry Brown
told a local CBS affiliate, ``It's very important when we can quantify
that manpower, utilize it.''

Image

Sandra Rojas of Malibu Camp.Credit...Peter Bohler for The New York Times

Image

Marquet Jones, a chain-saw operator with Rainbow Camp.Credit...Peter
Bohler for The New York Times

After five years, that drought is over, thanks to a much-needed rainy
season earlier this year that produced the rare ``super bloom'' ---
vast, thick patches of orange, magenta and purple blossoms among the
lime-green grasses. And yet experts still worry about this year in
particular: The last time a drought ended, in 2010, the following fire
season was even more extreme than the previous one. Rain caused more
grass to grow in places it ordinarily wouldn't, and when summer
temperatures regularly top 100 degrees, that grass dries out and becomes
kindling. In addition, an estimated 102 million trees in California have
been killed by the bark beetle since 2010; the insect, which is the size
of a rice grain, has been attacking pines, oaks and cedars, leaving
behind dry wood husks and a heightened risk of large, severe wildfires.
The 2010 fire season was bad; this season could be catastrophic. A total
of more than 5,000 fires have burned 460,000 acres already. Faced with
the prospect of a state in flames, California continues to depend on its
inmate firefighters as a tenuous and all-but-invisible line of defense.

\textbf{``I lost count,''} Marquet Jones, a firefighter arrested for
first-degree burglary, told me with a shake of her head when I asked her
how many fires she had been on over the previous year. ``I don't know
how many fires there were last season, but all through last season.''
The fire season typically runs from mid-May through November.

She recalled her first fire last year, going into Napa Valley as
residents were evacuating. The town was burned over; cars were
blackened. She wondered what she had got herself into. Despite her fear
and strained nerves, she cut the containment line for 10 hours, almost
until dawn. The heavy labor and the danger create a bond among the crew
members. ``I can say, coming from the streets, when you're with your
fire crew, that's your family,'' Edwards said. Of the 30 or so women I
met, most were serving prison terms because of drug- or alcohol-related
crimes, nonviolent convictions that the state classifies as low-level.
All had been drawn to the forestry camps by the relative freedom and the
chance to make more money than they could doing other prison jobs. But
many said the real education they were getting had to do with making and
maintaining relationships. ``It helps you to work as a sister crew,''
Marquet said. ``You learn how to work with them, you know --- 'cause,
really all you have is each other when you're on a fire.''

Some inmates say they would work the fireline for free --- for the
experience, the training, the gratification of doing something useful.
``It feels good,'' Marquet said, ``when you see kids with signs saying,
`Thank you for saving my house, thank you for saving my dog.' It feels
good that you saved somebody's home, you know? Some people, they look
down on us because we're inmates.''

Image

Inmates preparing to cross from the California Department of Corrections
and Rehabilitation side of Rainbow Camp to the Cal Fire
side.Credit...Peter Bohler for The New York Times

Marquet, who is 27, already had two strikes against her when she was
arrested. ``I was just under the influence on meth and just felt like
doing something. When you're under that drug, you really just go with
the flow. You feel like you're invincible. Can't no one stop you ---
you're just the king or the queen of the world. I got under the
influence and started walking down the street, saw a house with the
window open and decided to go in. Through the window.'' Now, her young
boys --- Bernard and Unique, both under 10 --- live with her older
sister. They haven't been able to visit, but Marquet goes to evening
prayer meetings in one of the common spaces at Rainbow. ``I go Sunday,
Monday, Tuesday and Thursday,'' she said. ``And I'm starting to go
Friday, too. But it's not really church. It's `Moms in Prayer.' We pray
for our kids.''

\href{https://www.nytimes3xbfgragh.onion/spotlight/california-wildfires}{Wildfires
in the West ›}

\hypertarget{live-updates}{%
\subsection{\texorpdfstring{\href{https://www.nytimes3xbfgragh.onion/2020/09/12/us/wildfires-live-updates.html}{Live
Updates}}{Live Updates}}\label{live-updates}}

Updated~

Sept. 12, 2020, 2:53 p.m. ET

\begin{itemize}
\tightlist
\item
  \href{https://www.nytimes3xbfgragh.onion/2020/09/12/us/wildfires-live-updates.html\#link-f3961ff}{President
  Trump will visit California on Monday after destructive fires.}
\item
  \href{https://www.nytimes3xbfgragh.onion/2020/09/12/us/wildfires-live-updates.html\#link-7e503ae9}{Shifting
  weather may improve firefighting conditions on the West Coast.}
\item
  \href{https://www.nytimes3xbfgragh.onion/2020/09/12/us/wildfires-live-updates.html\#link-5e4c548d}{Oregon's
  fire marshal is temporarily replaced as firefighters battle blazes.}
\end{itemize}

There are three all-female camps: the one at Rainbow, between San Diego
and Los Angeles, also known as Conservation Camp No. 2; the one at
Malibu, or Conservation Camp No. 13; and one at Puerta la Cruz, just
east of Temecula, called Conservation Camp No. 14. Their grounds could
pass for spiritual retreats. They are places of calm as much as training
grounds; one inmate incarcerated in Malibu, for example, leads yoga and
meditation sessions. Vegetable gardens are tended by inmates after work
hours; there are the remnants of a boxing camp that complement the
weight-lifting facility. Malibu is kissed with salt air and shade;
Rainbow and Port are hiking paradises. All the inmates eat civilian food
cooked by other inmates: rib-eye steak and lobster and sometimes
all-you-can-eat shrimp. But the benefits of greater freedom and superior
food also come with a physical cost. ``Your feet are hot and tired, and
they have a pulse of their own,'' Marquet said. ``You feel like you
can't breathe, but you're breathing. Your face feels like it's about to
melt off, but it's there. It's just --- you have to be aware of
everything.'' Otherwise, she added, ``you're not going to survive.''

Image

Part of an inmate-led yoga class at Malibu Camp.Credit...Peter Bohler
for The New York Times

\textbf{Shawna Lynn Jones} could take apart her chain saw and put it
back together effortlessly. She could fix the machine when it kicked
back, sharpen the chain when it dulled, clean the clutch cover. The
calluses on her hands came from working the saw --- it was an extension
of her body. You don't get to be second saw without knowing your machine
intimately and taking your job seriously. The night of the Mulholland
fire, Jones was frustrated, according to Jessica Ornelas. It was taking
a long time for the civilian crews to get the hoses up the ravine. So
she ran down the rocky hillside and brought them up herself.

Jones didn't grow up with dreams of being a firefighter. She wanted to
be a police officer. The first photo her mom, Diana Baez, showed me was
of a cocky young girl of around 5 or 6 dressed up for career day. Jones
is wearing navy blue head-to-toe and aviator shades. She has a death
grip on a plastic baton and holds a leash tethered to the neck of a
stuffed Goofy doll. ``She always wanted to be a K-9 handler, and here
she was dressed like one,'' Baez said. We were sitting in a dark,
wood-paneled bar --- the Trap, a dusty oasis on the fringes of
Lancaster, a town already on the fringes of Southern California in the
high desert of the Antelope Valley. Before Jones was incarcerated, this
was her home. Her mom managed the bar; much of her extended family was
in a hard-rock band called Seconds to Centuries (SIIC) that played the
back room.

Image

Raven Vasquez, left, a chain-saw operator from Malibu Camp, and Megan
Clark, her ``puller,'' the person who pulls brush out of the way, work
the Detwiler fire in Mariposa County in July.Credit...Peter Bohler for
The New York Times

Jones was smart, but as a teenager she couldn't sit still in class.
Eventually she dropped out of high school to work at a mortuary owned by
a boyfriend's family. The job ended when the relationship did. She had a
string of boyfriends, most of them bad, and in May 2014, she was caught
sitting in a car next to one of them and a large quantity of crystal
methamphetamine. He had a lengthy record and didn't want to be locked up
for life. He told Jones he would bail her out if she took responsibility
for the drugs. Jones was convicted of possession with attempt to
distribute methamphetamine and of marijuana possession. The boyfriend
kept his promise and paid the \$30,000 bail, and Jones was sentenced to
three years' probation.

She was trapped in Lancaster. ``No one can get out of here, it's like
we're all stuck,'' Rosa Garcia, Jones's friend, said. Jones helped her
mom run the Trap's karaoke nights (screaming expletives of denial
whenever someone sang ``Like a Virgin''), and she made some extra money
by drawing on patrons' flat-billed snapback baseball caps. She sold
merchandise at her friends' shows; hustled pool; bummed cigarettes;
wrote poetry; smoked weed; and skateboarded, sometimes all night. In
some pictures from SIIC shows, her leggings are ripped and her eyeliner
is winged to perfection, and she's standing victoriously over a riotous
crowd. ``I could always count on Shawna being right there, right in
front of me, center stage, every single time,'' Jae Paige Dion, the lead
singer of SIIC, said. ``She had no problem getting in the mosh pits and
knocking down all the guys.'' Jones was fearless. Her Facebook photos
show her sticking her tongue out aggressively, flashing a middle finger
at a friend's cellphone camera; there are shots of her belly red and raw
from being slapped.

Image

Rainbow Crew No. 4 maintaining the line on a small fire near
Hemet.Credit...Peter Bohler for The New York Times

Within a year of the methamphetamine arrest, Jones was back in trouble.
She had violated parole at least three times --- stealing puppy food,
stealing groceries, selling marijuana, missing court dates --- before a
warrant was issued for her arrest. Jones decided to turn herself in. On
June 2, 2015, she wrote on her Facebook page: ``I can only handle so
much bad stuff at one time. and I have reached my quota for the year so
it can stop now. I want some good stuff to happen soon.'' The Trap
hosted a party. Rosa Garcia got the dollar taco guy to bring his truck
to the parking lot. ``We basically ordered one million tacos so that she
would remember what real food tastes like,'' Garcia said. Dion made her
a personalized T-shirt with her nickname, ``Baby Hooker,'' scrawled on
it, which everyone signed, and by the next day she was ready. Jones
hugged her mom, who was crying, and skated off on her longboard toward
the Lancaster courthouse to turn herself in. Jones admitted to the court
that she failed to comply with her probation conditions, and she was
sentenced to three years. She heard about the forestry program during
one of the 238 days she spent in the county jail: The women all spoke of
it as a prison Shangri-La --- lobster, shrimp, ocean breezes. Six months
after leaving the county jail, Jones was transferred to Malibu.

By November 2015, Jones was calling her mom weekly to tell her about the
training, about the exhaustion after sandbagging a hillside to prevent
flooding and about the optional weekend hikes that she always went on
through the canyons of Malibu. She had found something in this sort of
work, something she liked. It reminded her of a not-too-distant past. In
high school, she camped out with friends on Shaver Lake in the Sierra
Nevada mountains, plunged into cold lakes from rocky cliffs and
boogie-boarded at the beach. She etched her initials with her
boyfriend's, ``C.C. + S.J.,'' on the side of a rusted beach picnic
table. Her enthusiasm was so great it convinced her mother that Jones's
luck was changing. Baez was already planning her daughter's welcome-home
party.

Image

An inmate firefighter walks through scorched earth near
Hemet.Credit...Peter Bohler for The New York Times

On the morning of the Mulholland fire, Feb. 25, an unknown number
flashed on Diana Baez's cellphone around 10 a.m. It flashed again and
again and again. Baez kept declining the call until it seemed like
something she shouldn't ignore. ``There's been an accident,'' a man told
her when she answered. Baez, immediately hysterical, asked, ``Where is
my daughter?'' He paused and said, ``I can't tell you because she's an
inmate.'' An hour later, when the Lancaster sheriff's office called with
numbers and instructions, Baez scrawled as much information as she could
on her bedroom mirror using eyeliner. The sheriff told her that Jones
was not admitted under her birth name, because of her incarcerated
status. He told Baez that when she got to the U.C.L.A. hospital, she
should ask for ``Hawaii X.'' She arrived to find her daughter lying
unconscious on a gurney.

``The first thing I did when I opened that curtain and I saw her --- I
grabbed her --- right there, I grabbed her, and I said, `You promised
me,' '' Baez told me. ``She just called me two days before, and she
said, `Momma, I'm coming home in six weeks,' so I freaking told her,
`You promised me.' '' Baez hardly recognized her daughter. Her face was
swollen; her eyes were taped shut so that they wouldn't dry out; her
head had been shaved because the doctors were trying to drain a blood
clot. Baez crawled onto the gurney next to her daughter, but she
remained unresponsive.

The two police officers standing guard at the door to Jones's room tried
to explain what happened. Captains and representatives from C.D.C.R. all
tried to explain. But Baez could only cry and hold her daughter's hand.
She never left Jones. (A nurse had to force her to eat a snack of orange
juice and graham crackers.) Later, she found out from the intake
administrator what had happened on the ravine in Malibu.

The earth above Jones began giving way. At first it was just pebbles.
Then, the first chain saw shouted, ``Rock.'' But Jones couldn't hear
over the noise of her machine. The large stone fell suddenly, 100 feet
and, in an instant, struck her head. She was knocked out on her feet. A
fire captain strapped her into a stretcher, and a helicopter, there to
drop fire retardant, descended to retrieve the limp body.

Image

Firefighters from Crew 13-4 of Camp Malibu on a lunch break at Nicholas
Canyon Beach after completing a training exercise on Sept. 30,
2016.Credit...Peter Bohler for The New York Times

\textbf{There are three} ways to get to Malibu 13 --- from the Pacific
Coast Highway, from the circuitous back roads northeast of Malibu or by
way of C.D.C.R. transport. When new trainees arrive in a white bus, they
see no fences. They see off-duty inmates wearing orange jumpsuits half
on, white T-shirts on top and fire-rated boots laced loosely. They see
open-dorm barracks where they will sleep with their crew, in a line, as
if they could roll out of bed and fight fire within minutes of an alarm,
which they will do, sometimes multiple nights in a row. The crews are
always at work, even when they're not. They see visitors because
C.D.C.R. is proud of the program. And when they look at the communal
board on the L.A. County Fire side of the camp, they see a dedicated
plaque and several articles about Jones's death. Some people wrote notes
to Jones, now faded behind plexiglass. The Malibu community raised
\$4,000 for the ``Shawna Lynn Jones Fund.'' On the C.D.C.R. side of camp
there is another memorial --- five tree stumps and a rain stick with a
carved message: ``Like the wind, felt but not seen, my sweet Shawna may
you R.I.P.''

At a graduation last year of inmate firefighters at the California
Institution for Women, near Chino, where all female inmate firefighters
are trained, the mood was celebratory, almost exultant. One speaker
brought up Jones and asked, to great applause, that her life and her
death not go in vain. He said, ``She gave her life for this program, and
L.A. County made sure she did not leave without full dress.''

When I visited Rainbow, I asked a Cal Fire captain named Danny Ramirez
why the state wouldn't increase the incentive to join the program by
paying even a little bit more. He didn't have a ready answer. Which
brought up another puzzling aspect of the program: Why doesn't the state
get more out of its investment in training these women by hiring them
when they're released? Or at the very least, by creating a pathway to
employment? Ramirez said the idea ``to keep tags on the girls'' had come
up before. ``Some of these girls leave very interested in what they got
exposed to and say, `Oh I never knew this exists, how do I keep on doing
this?' And it's hard when they get out there because they do have a lot
of the same walls that they were facing before. But a program to keep
them guided and keep them on that path and keep them focused on
something instead of getting back into their old ways or old friends
would be awesome.''

Jones's body was driven from the coroner's department to Eternal Valley
Memorial Park and Mortuary, located between Lancaster and Los Angeles. A
fire company crew was on every overpass, standing on their trucks,
saluting in full uniform as Jones's body was driven underneath. Outside
her funeral, rows of sheriffs and deputies stood at attention, right
hands at their brows. Two fire trucks were parked at the entrance with
their ladders raised, crossed in tribute to her. Shawna Lynn Jones lived
as an inmate and died an honored firefighter. Baez received a customary
American flag, folded into a tight triangle. Someone told her, she says,
that in Shawna's four months as a firefighter, she made about \$1,000.

Advertisement

\protect\hyperlink{after-bottom}{Continue reading the main story}

\hypertarget{site-index}{%
\subsection{Site Index}\label{site-index}}

\hypertarget{site-information-navigation}{%
\subsection{Site Information
Navigation}\label{site-information-navigation}}

\begin{itemize}
\tightlist
\item
  \href{https://help.nytimes3xbfgragh.onion/hc/en-us/articles/115014792127-Copyright-notice}{©~2020~The
  New York Times Company}
\end{itemize}

\begin{itemize}
\tightlist
\item
  \href{https://www.nytco.com/}{NYTCo}
\item
  \href{https://help.nytimes3xbfgragh.onion/hc/en-us/articles/115015385887-Contact-Us}{Contact
  Us}
\item
  \href{https://www.nytco.com/careers/}{Work with us}
\item
  \href{https://nytmediakit.com/}{Advertise}
\item
  \href{http://www.tbrandstudio.com/}{T Brand Studio}
\item
  \href{https://www.nytimes3xbfgragh.onion/privacy/cookie-policy\#how-do-i-manage-trackers}{Your
  Ad Choices}
\item
  \href{https://www.nytimes3xbfgragh.onion/privacy}{Privacy}
\item
  \href{https://help.nytimes3xbfgragh.onion/hc/en-us/articles/115014893428-Terms-of-service}{Terms
  of Service}
\item
  \href{https://help.nytimes3xbfgragh.onion/hc/en-us/articles/115014893968-Terms-of-sale}{Terms
  of Sale}
\item
  \href{https://spiderbites.nytimes3xbfgragh.onion}{Site Map}
\item
  \href{https://help.nytimes3xbfgragh.onion/hc/en-us}{Help}
\item
  \href{https://www.nytimes3xbfgragh.onion/subscription?campaignId=37WXW}{Subscriptions}
\end{itemize}
