Sections

SEARCH

\protect\hyperlink{site-content}{Skip to
content}\protect\hyperlink{site-index}{Skip to site index}

\href{https://www.nytimes3xbfgragh.onion/section/us}{U.S.}

\href{https://myaccount.nytimes3xbfgragh.onion/auth/login?response_type=cookie\&client_id=vi}{}

\href{https://www.nytimes3xbfgragh.onion/section/todayspaper}{Today's
Paper}

\href{/section/us}{U.S.}\textbar{}Construction Cranes in Miami Become
Multi-Ton Threats During Irma

\url{https://nyti.ms/2eOVBxR}

\begin{itemize}
\item
\item
\item
\item
\item
\end{itemize}

Advertisement

\protect\hyperlink{after-top}{Continue reading the main story}

Supported by

\protect\hyperlink{after-sponsor}{Continue reading the main story}

\hypertarget{construction-cranes-in-miami-become-multi-ton-threats-during-irma}{%
\section{Construction Cranes in Miami Become Multi-Ton Threats During
Irma}\label{construction-cranes-in-miami-become-multi-ton-threats-during-irma}}

\includegraphics{https://static01.graylady3jvrrxbe.onion/images/2017/09/10/us/11xp-cranes/11xp-cranes-articleLarge.jpg?quality=75\&auto=webp\&disable=upscale}

By \href{https://www.nytimes3xbfgragh.onion/by/maggie-astor}{Maggie
Astor}

\begin{itemize}
\item
  Sept. 10, 2017
\item
  \begin{itemize}
  \item
  \item
  \item
  \item
  \item
  \end{itemize}
\end{itemize}

Construction cranes are a ubiquitous sight in Miami, symbols of a
growing city towering hundreds of feet above its streets. They are built
to withstand high winds, but maybe not to withstand Hurricane Irma.

Since
\href{https://www.nytimes3xbfgragh.onion/2017/09/10/us/hurricane-irma-florida.html}{Irma
made landfall} in the Florida Keys early on Sunday, bringing
hurricane-force winds to much of South Florida, parts of two cranes
collapsed:
\href{https://twitter.com/NWSMiami/status/906889237856997377}{one at 300
Biscayne Boulevard}, in the heart of downtown Miami, and
\href{https://twitter.com/FrancisSuarez/status/906964322580123651}{one
at 600 NE 30th Terrace}, about two miles away. Both sites are close to
the Biscayne Bay waterfront, where developers have sought to build
luxury housing in recent years.

\begin{quote}
Here's a better pic. Heard a loud crack, looked up and saw the crane
snapped and falling.
\href{https://twitter.com/CityofMiami?ref_src=twsrc\%5Etfw}{@CityofMiami}
\href{https://twitter.com/downtownMIA?ref_src=twsrc\%5Etfw}{@downtownMIA}
\href{https://twitter.com/wsvn?ref_src=twsrc\%5Etfw}{@wsvn}
\href{https://twitter.com/CBSMiami?ref_src=twsrc\%5Etfw}{@CBSMiami}
\href{https://t.co/NUCHUICsz2}{pic.twitter.com/NUCHUICsz2}

--- Gideon J. Apé (@GideonApe)
\href{https://twitter.com/GideonApe/status/906885388698038273?ref_src=twsrc\%5Etfw}{September
10, 2017}
\end{quote}

\begin{quote}
Able to physically confirm second crane collapse at 600 NE 30th ter.
\href{https://twitter.com/CityofMiami?ref_src=twsrc\%5Etfw}{@CityofMiami}
\href{https://t.co/Gz1CgSZsNe}{pic.twitter.com/Gz1CgSZsNe}

--- Mayor Francis Suarez (@FrancisSuarez)
\href{https://twitter.com/FrancisSuarez/status/906964322580123651?ref_src=twsrc\%5Etfw}{September
10, 2017}
\end{quote}

The strong winds prevented emergency workers from responding at first,
and the cranes' arms remained where they had fallen, dangling menacingly
above buildings. By the evening, however, the Miami Fire-Rescue
Department was able to reach both sites and cordon them off to keep
pedestrians and vehicles out. Capt. Ignatius Carroll, a spokesman for
the fire department, said there were no injuries.

The federal Occupational Safety and Health Administration will
investigate, Miami officials said in a statement.

OSHA's communications office did not return a call on Sunday afternoon.
The Miami city manager, Daniel Alfonso, did not respond to an email on
Sunday, and no one answered the phone at his office.

The 20- to 25-tower cranes now in use in Miami were designed to
withstand 145-mile-an-hour winds, city officials said last week.

At that point, Irma was a Category 5 hurricane with wind speeds of up to
185 m.p.h. Maurice Pons, deputy director of the city's building
department,
\href{https://twitter.com/CityofMiami/status/905185163558223872}{said in
a statement} then that he ``would not advise'' riding out the storm near
one of the cranes. (The wind speeds ultimately recorded in Miami were
lower.)

But even as officials warned that the cranes could pose a serious
threat, they said it was impossible to move them before the hurricane
arrived.

Tower cranes are hundreds of feet tall, and their counterbalances can
weigh as much as 30,000 pounds. That is about as much as 10 Honda
Civics, and it means the cranes cannot be moved on a dime.

Miami officials said on Twitter last week that it would take two weeks
to move them --- and two weeks ago, Hurricane Irma was not even a
tropical storm.

But \href{http://www.craneconsultant.com/about.php}{Thomas Barth}, who
owns Barth Crane Inspections in South Carolina and investigates crane
accidents, said the cranes' arms and counterbalances could have been
taken down with the help of crews from out of state if necessary.

Without the arms and counterbalances, only the cranes' towers would have
been left standing, and those are extremely strong, Mr. Barth said on
Sunday.

``They should have done that,'' he said. ``They could have taken those
cranes down, and Miami could have said `take them down,' but nobody
wants to give that order.''

South Florida has undergone
\href{https://www.nytimes3xbfgragh.onion/2017/09/07/climate/florida-hurricane-irma-damage.html}{frenetic
growth and development}, particularly along the coast, where developers
have ample incentive to build. Local zoning rules require high-rise
developers in certain areas to fortify their buildings against high
winds, but a hurricane can turn construction materials and equipment
into flying debris.

The vulnerability of high-rise construction sites was apparent five
years ago, when Hurricane Sandy battered the Northeast. In the hours
before the full force of the storm, a crane atop a 57th Street
construction site in Manhattan snapped and dangled 1,000 feet above the
ground in 80-mile-an-hour winds as officials
\href{http://www.nytimes3xbfgragh.onion/2012/11/07/nyregion/drama-behind-securing-crippled-crane-in-manhattan.html}{rushed
to safely secure it}.

Advertisement

\protect\hyperlink{after-bottom}{Continue reading the main story}

\hypertarget{site-index}{%
\subsection{Site Index}\label{site-index}}

\hypertarget{site-information-navigation}{%
\subsection{Site Information
Navigation}\label{site-information-navigation}}

\begin{itemize}
\tightlist
\item
  \href{https://help.nytimes3xbfgragh.onion/hc/en-us/articles/115014792127-Copyright-notice}{©~2020~The
  New York Times Company}
\end{itemize}

\begin{itemize}
\tightlist
\item
  \href{https://www.nytco.com/}{NYTCo}
\item
  \href{https://help.nytimes3xbfgragh.onion/hc/en-us/articles/115015385887-Contact-Us}{Contact
  Us}
\item
  \href{https://www.nytco.com/careers/}{Work with us}
\item
  \href{https://nytmediakit.com/}{Advertise}
\item
  \href{http://www.tbrandstudio.com/}{T Brand Studio}
\item
  \href{https://www.nytimes3xbfgragh.onion/privacy/cookie-policy\#how-do-i-manage-trackers}{Your
  Ad Choices}
\item
  \href{https://www.nytimes3xbfgragh.onion/privacy}{Privacy}
\item
  \href{https://help.nytimes3xbfgragh.onion/hc/en-us/articles/115014893428-Terms-of-service}{Terms
  of Service}
\item
  \href{https://help.nytimes3xbfgragh.onion/hc/en-us/articles/115014893968-Terms-of-sale}{Terms
  of Sale}
\item
  \href{https://spiderbites.nytimes3xbfgragh.onion}{Site Map}
\item
  \href{https://help.nytimes3xbfgragh.onion/hc/en-us}{Help}
\item
  \href{https://www.nytimes3xbfgragh.onion/subscription?campaignId=37WXW}{Subscriptions}
\end{itemize}
