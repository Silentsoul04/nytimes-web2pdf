Sections

SEARCH

\protect\hyperlink{site-content}{Skip to
content}\protect\hyperlink{site-index}{Skip to site index}

\href{https://www.nytimes3xbfgragh.onion/section/business}{Business}

\href{https://myaccount.nytimes3xbfgragh.onion/auth/login?response_type=cookie\&client_id=vi}{}

\href{https://www.nytimes3xbfgragh.onion/section/todayspaper}{Today's
Paper}

\href{/section/business}{Business}\textbar{}Irma May Force Florida
Insurers to Turn to Deeper Pockets

\url{https://nyti.ms/2ePAETt}

\begin{itemize}
\item
\item
\item
\item
\item
\end{itemize}

Advertisement

\protect\hyperlink{after-top}{Continue reading the main story}

Supported by

\protect\hyperlink{after-sponsor}{Continue reading the main story}

\hypertarget{irma-may-force-florida-insurers-to-turn-to-deeper-pockets}{%
\section{Irma May Force Florida Insurers to Turn to Deeper
Pockets}\label{irma-may-force-florida-insurers-to-turn-to-deeper-pockets}}

\includegraphics{https://static01.graylady3jvrrxbe.onion/images/2017/09/12/business/12insure_web1/12insure_web1-articleInline.jpg?quality=75\&auto=webp\&disable=upscale}

By \href{http://www.nytimes3xbfgragh.onion/by/mary-williams-walsh}{Mary
Williams Walsh}

\begin{itemize}
\item
  Sept. 11, 2017
\item
  \begin{itemize}
  \item
  \item
  \item
  \item
  \item
  \end{itemize}
\end{itemize}

With Hurricane Irma's destructive force having pushed north, Floridians
are beginning to check on what has become of their homes. They may also
want to check on their insurers.

The big national carriers like State Farm and Allstate cut back on
writing homeowners' insurance policies in Florida years ago, citing
catastrophic risks and unhelpful state regulators. Those reductions left
a vacuum that was filled, initially, with a state-owned insurer,
Citizens Property Insurance. Eventually, the state offered incentives to
coax some brave new insurers into the market.

As a result, all that stands between many Florida homeowners and
potential ruin is one state-owned insurer and dozens of relatively
little-known companies that do most --- or all --- of their business in
the state. They all have the benefit of the Florida Hurricane
Catastrophe Fund, which, with no major storms in the past 12 years, has
\$17 billion at the ready --- a sum that may not be nearly enough.

``This can really be a hurricane that can bust the insurance industry,''
said Shahid Hamid, a finance professor at Florida International
University's International Hurricane Research Center. ``When I saw
Irma's track, I was tremendously concerned.''

Whether policyholders can be made whole will most likely depend on
reinsurance --- the custom-tailored insurance that the Florida insurers
themselves take out --- and other financial vehicles that they have to
fall back on. But the variables at play --- the storm's path, the
companies' balance sheets and the details of their reinsurance contracts
--- could produce differing fortunes for the various companies.

By missing a direct hit on Miami, and inflicting less damage than
expected on Florida's Atlantic Coast, Irma turned out to be bad, but not
as bad as experts had feared.

AIR Worldwide, a catastrophe modeling firm in Boston, predicted on
Monday that Irma's insured damage in the United States would cost the
industry \$20 billion to \$40 billion. If insured property in the
Caribbean is included, the total projected losses would range from \$25
billion to \$55 billion, according to Kevin Long, a spokesman.

Insurer strength could be more of an issue in Florida than it was in
eastern Texas, where most of Hurricane Harvey's damage was caused by
inland flooding. Homeowners' insurers do not generally provide flood
insurance. While Irma brought flooding to Florida, too, the greater
damage was expected to be from high winds, a peril covered by standard
homeowners' policies.

Investors dumped the stocks of Florida insurers last week as they
anticipated the resulting claims and losses. Not all the insurers are
publicly traded, but the second largest in the Florida market, Universal
Property \& Casualty Insurance, saw the shares of its corporate parent,
Universal Insurance Holdings, drop by 15 percent on Thursday. It
regained some of that ground Friday and was up more than 13 percent
Monday.

Mr. Hamid led the development of the Florida Public Hurricane Loss
Model, which is used by state regulators to assess insurers' financial
strength, among other things. But he said it was impossible to predict
which insurers might be vulnerable, because the companies' exposures
vary by region and the terms of their reinsurance contracts are not
known.

Long before Hurricane Irma started bearing down, Mr. Hamid stress-tested
the larger homeowners' insurers in Florida and found that they would
withstand a worst-case scenario, as he defined it at the time: a storm
like Hurricane Andrew in 1992.

Until now, Andrew was ranked as Florida's most destructive storm,
causing \$27 billion in damage --- \$47 billion in today's dollars.
Twenty-two insurers failed, leaving a million policyholders without
coverage.

Joseph L. Petrelli, president of Demotech, an Ohio-based company that
reviews and rates regional and specialty insurers --- including about 50
in Florida that represent 60 percent of the state's homeowners'
insurance market --- suggested that Irma could be more significant
because it would cause such a wide swath of damage.

``It's the first time in history that every county in Florida was on
hurricane watch,'' he said.

Using the same benchmark as Mr. Hamid, Mr. Petrelli said he believed his
Florida client insurers could withstand a storm like Hurricane Andrew,
mainly because they currently hold more reinsurance than ever. He said
he could not speak for the other insurers because he does not rate them.

Reinsurance allows companies to transfer some of their exposure to
other, well-capitalized companies, adding capacity but reducing their
profits. Mr. Petrelli said that was why the national insurers pulled
back from Florida --- their reinsurance needs in other states were far
less expensive and did not undermine profits.

``We make them buy more reinsurance every year,'' he said of his Florida
clients, ``because the value of homes goes up every year, the number of
homes goes up every year, and the cost of repairs goes up every year.''
For the last few years, reinsurance prices had fallen, even for
companies in high-risk Florida, because the state has not had a major
storm since 2005.

Specialized vehicles known as catastrophe bonds have also been more
affordable for insurers. Michael Peltier, a spokesman for Citizens
Property Insurance, said that when the company issued its first
catastrophe bond, in 2012, investors demanded a yield of 21 percent.
This year, he said, the rate was only 7 percent. Between reinsurance and
catastrophe bonds, he said Citizens was able to increase its capacity to
absorb losses this year by \$1.3 billion, at a cost of \$97 million.

Industry experts think the losses from this year's storms could put an
end to the buyer's market for reinsurance, however.

The insurers pass at least some of the cost on to their customers.
Homeowners in Florida spent about \$1 billion on insurance in 1991, the
year before Hurricane Andrew. Last year, they spent nearly \$9 billion.

``Insurance rates are pretty high,'' said Mr. Hamid. ``The average rate
for \$300,000 coverage, built pre-1992, is \$11,500, and it's higher if
you're near the coast.''

He said the state's building codes had improved since Hurricane Andrew,
and people who upgrade older houses to the new standards can get ``a
huge discount'' on their premiums.

The largest homeowners' insurance company in Florida remains Citizens
Property Insurance. It was formed in 1993, and for many years it
dominated the market in Florida, causing consternation because its size
meant the state was bearing significant risk.

By 2011, Citizens estimated that a 100-year storm would cause \$24.5
billion in losses, and it had the resources --- including the state
catastrophe fund and reinsurance --- to cover just \$13 billion.

That year, Citizens opened up its books and let the private insurers buy
its most profitable policies, in big batches. The goal was to reduce the
role of the state, so that while big risks would still be there,
for-profit companies would bear more of them.

The initiative succeeded, said Mr. Peltier, the company spokesman.
Today, Citizens is about one-third its former size. With fewer
policyholders, it estimated that it would get only \$6.6 billion in
claims after a 100-year storm. It has enough resources to cover that
without taxpayers getting involved, company projections show. Citizens
now screens prospective customers to make sure it takes only those
deemed too risky for the companies to insure, like homeowners near the
coasts.

Citizens builds reserves, buys reinsurance and issues catastrophe bonds
to sophisticated investors, just as the private insurers do, Mr. Peltier
said. But Citizens also has a risk-management tool that private
companies do not: the taxing power of the state. If Hurricane Irma turns
out to be devastating enough to use up other resources, Citizens has the
authority to assess virtually all policyholders in the state, no matter
what type of insurance they hold or whether they bought it from
Citizens.

The private companies have nothing like that. If one exhausts its
resources, it will end up insolvent. In such cases, the company is
typically closed, and a state-led ``guarantee'' mechanism takes over,
seeking to pay homeowners as much as possible from what is left and
assessments on other insurers. Homeowners' claims would be unlikely to
be paid in full.

Advertisement

\protect\hyperlink{after-bottom}{Continue reading the main story}

\hypertarget{site-index}{%
\subsection{Site Index}\label{site-index}}

\hypertarget{site-information-navigation}{%
\subsection{Site Information
Navigation}\label{site-information-navigation}}

\begin{itemize}
\tightlist
\item
  \href{https://help.nytimes3xbfgragh.onion/hc/en-us/articles/115014792127-Copyright-notice}{©~2020~The
  New York Times Company}
\end{itemize}

\begin{itemize}
\tightlist
\item
  \href{https://www.nytco.com/}{NYTCo}
\item
  \href{https://help.nytimes3xbfgragh.onion/hc/en-us/articles/115015385887-Contact-Us}{Contact
  Us}
\item
  \href{https://www.nytco.com/careers/}{Work with us}
\item
  \href{https://nytmediakit.com/}{Advertise}
\item
  \href{http://www.tbrandstudio.com/}{T Brand Studio}
\item
  \href{https://www.nytimes3xbfgragh.onion/privacy/cookie-policy\#how-do-i-manage-trackers}{Your
  Ad Choices}
\item
  \href{https://www.nytimes3xbfgragh.onion/privacy}{Privacy}
\item
  \href{https://help.nytimes3xbfgragh.onion/hc/en-us/articles/115014893428-Terms-of-service}{Terms
  of Service}
\item
  \href{https://help.nytimes3xbfgragh.onion/hc/en-us/articles/115014893968-Terms-of-sale}{Terms
  of Sale}
\item
  \href{https://spiderbites.nytimes3xbfgragh.onion}{Site Map}
\item
  \href{https://help.nytimes3xbfgragh.onion/hc/en-us}{Help}
\item
  \href{https://www.nytimes3xbfgragh.onion/subscription?campaignId=37WXW}{Subscriptions}
\end{itemize}
