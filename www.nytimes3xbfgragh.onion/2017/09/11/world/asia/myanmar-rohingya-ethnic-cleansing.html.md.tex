Sections

SEARCH

\protect\hyperlink{site-content}{Skip to
content}\protect\hyperlink{site-index}{Skip to site index}

\href{https://www.nytimes3xbfgragh.onion/section/world/asia}{Asia
Pacific}

\href{https://myaccount.nytimes3xbfgragh.onion/auth/login?response_type=cookie\&client_id=vi}{}

\href{https://www.nytimes3xbfgragh.onion/section/todayspaper}{Today's
Paper}

\href{/section/world/asia}{Asia Pacific}\textbar{}Rohingya Crisis in
Myanmar Is `Ethnic Cleansing,' U.N. Rights Chief Says

\url{https://nyti.ms/2xUUiF4}

\begin{itemize}
\item
\item
\item
\item
\item
\end{itemize}

Advertisement

\protect\hyperlink{after-top}{Continue reading the main story}

Supported by

\protect\hyperlink{after-sponsor}{Continue reading the main story}

\hypertarget{rohingya-crisis-in-myanmar-is-ethnic-cleansing-un-rights-chief-says}{%
\section{Rohingya Crisis in Myanmar Is `Ethnic Cleansing,' U.N. Rights
Chief
Says}\label{rohingya-crisis-in-myanmar-is-ethnic-cleansing-un-rights-chief-says}}

\includegraphics{https://static01.graylady3jvrrxbe.onion/images/2017/09/12/world/12Myanmar/12Myanmar-articleInline.jpg?quality=75\&auto=webp\&disable=upscale}

By Nick Cumming-Bruce

\begin{itemize}
\item
  Sept. 11, 2017
\item
  \begin{itemize}
  \item
  \item
  \item
  \item
  \item
  \end{itemize}
\end{itemize}

GENEVA --- The United Nations' top human rights official accused Myanmar
on Monday of carrying out ``a textbook example of ethnic cleansing''
against Rohingya Muslims, hundreds of thousands of whom have crossed
into Bangladesh since late August to escape a military crackdown.

Zeid Ra'ad al-Hussein, the United Nations high commissioner for human
rights, said the military's ``brutal'' security campaign was in clear
violation of international law, and cited what he called refugees'
consistent accounts
of\href{https://www.nytimes3xbfgragh.onion/2017/09/02/world/asia/rohingya-myanmar-bangladesh-refugees-massacre.html?action=click\&contentCollection=Asia\%20Pacific\&module=RelatedCoverage\&region=Marginalia\&pgtype=article\&_r=0}{widespread
extrajudicial killings, rape and other atrocities}.

Mr. al-Hussein said the crackdown ``resembles a cynical ploy to forcibly
transfer large numbers of people without possibility of return,'' noting
that Myanmar had progressively stripped its Rohingya minority of civil
and political rights for decades.

``The situation seems a textbook example of ethnic cleansing,'' he said
in a keynote address before the United Nations Human Rights Council in
Geneva.

More than 300,000 Rohingya have
\href{https://www.nytimes3xbfgragh.onion/2017/09/08/world/asia/myanmar-rohingya-refugees-270000.html}{fled
to Bangladesh since Aug. 25}, when armed
\href{https://www.nytimes3xbfgragh.onion/2017/08/25/world/asia/myanmar-rakhine-killed-kofi-annan.html}{Rohingya
militants attacked police posts} and a military base in the western
state of Rakhine, which borders Bangladesh. The Myanmar authorities said
15 members of the security forces and 370 militants had been killed in
the fighting.

\includegraphics{https://static01.graylady3jvrrxbe.onion/images/2017/09/03/world/03rohingya-cover/03rohingya-cover-videoSixteenByNineJumbo1600.jpg}

Since then, Rohingya refugees arriving in Bangladesh have told
journalists, rights groups and others that soldiers, along with some
local residents, had set fire to numerous villages and had butchered
Rohingya men, women and children.

Some officials in Myanmar have said that Rohingya had set fire to their
own homes and villages. On Monday, Mr. al-Hussein called such
accusations a ``complete denial of reality'' that was damaging the
international standing of a leadership that had benefited from
considerable good will as the country emerged from decades of military
rule.

Mr. al-Hussein's comments added to mounting international criticism of
the military's actions in Rakhine. Some of it has
\href{https://www.nytimes3xbfgragh.onion/interactive/2017/09/09/opinion/kristof-nobel-prize-aung-san-suu-kyi-shame.html}{singled
out Daw Aung San Suu Kyi}, the de facto leader of the elected civilian
government, who was awarded the Nobel Peace Prize in 1991 for her
resistance to the military dictatorship. Ms. Aung San Suu Kyi does not
control Myanmar's military, but she has yet to criticize its crackdown
on the Rohingya.

On Friday, the Dalai Lama became the latest Nobel Peace Prize laureate
to raise the issue of her silence, following
\href{https://www.facebookcorewwwi.onion/DesmondTutuOfficial/posts/1136360939841237?pnref=story}{statements
from Bishop Desmond Tutu of South Africa} and the rights advocate
\href{https://twitter.com/Malala/status/904449772844711938}{Malala
Yousafzai of Pakistan}, both of whom called on Ms. Aung San Suu Kyi to
take action.

The Dalai Lama told journalists in Dharamsala, India, that those who
were persecuting Rohingya ``should remember Buddha,'' a pointed reminder
to the Buddhists who make up a majority of Myanmar's population. Some
Buddhist nationalists in Myanmar have campaigned for Muslims to be
driven out of the country.

The Buddha ``would definitely give help to those poor Muslims,'' the
Dalai Lama said.

On Sunday, leaders who had gathered in Astana, Kazakhstan, for a meeting
of the Organization of Islamic Cooperation issued a statement condemning
the ``systematic brutal acts'' against the Rohingya and asked Myanmar to
allow a United Nations fact-finding mission into the country to
investigate.

That mission was established after an earlier crackdown in Rakhine, in
October, also in response to a coordinated attack on security forces by
Rohingya militants. Myanmar's government has refused to cooperate with
the mission and has said it will not allow members of the group into the
country. The mission is scheduled to report to the United Nations rights
council this month.

The Organization of Islamic Cooperation is currently led by President
Recep Tayyip Erdogan of Turkey. His wife, Ermine Erdogan, traveled to
Bangladesh with a consignment of humanitarian aid last week, urging the
government in Dhaka to keep its borders open for Rohingya refugees.

The militant group blamed for the August attacks, the Arakan Rohingya
Salvation Army, declared a unilateral, one-month ceasefire on Sunday,
citing the need to allow the delivery of humanitarian aid and urging
Myanmar's military to lay down its arms. The government refused, saying
it would not negotiate with terrorists.

In his address on Monday, Mr. al-Hussein said he was appalled by
\href{https://www.nytimes3xbfgragh.onion/2017/09/06/world/americas/bangladesh-rohingya-land-mines.html}{reports
that Myanmar's military has placed mines} along the border with
Bangladesh. Amnesty International said on Sunday that it had
\href{https://www.amnesty.org/en/latest/news/2017/09/myanmar-army-landmines-along-border-with-bangladesh-pose-deadly-threat-to-fleeing-rohingya/}{documented
``what seems to be targeted use of land mines''} by Myanmar's security
forces at crossing points used by refugees.

The rights group said that one civilian near the border had been killed
and that three people, including two children, had been seriously
injured by mines in the past week.

``This is another low in what is already a horrific situation in
Rakhine,'' said Tirana Hassan, Amnesty's crisis response director.

Advertisement

\protect\hyperlink{after-bottom}{Continue reading the main story}

\hypertarget{site-index}{%
\subsection{Site Index}\label{site-index}}

\hypertarget{site-information-navigation}{%
\subsection{Site Information
Navigation}\label{site-information-navigation}}

\begin{itemize}
\tightlist
\item
  \href{https://help.nytimes3xbfgragh.onion/hc/en-us/articles/115014792127-Copyright-notice}{©~2020~The
  New York Times Company}
\end{itemize}

\begin{itemize}
\tightlist
\item
  \href{https://www.nytco.com/}{NYTCo}
\item
  \href{https://help.nytimes3xbfgragh.onion/hc/en-us/articles/115015385887-Contact-Us}{Contact
  Us}
\item
  \href{https://www.nytco.com/careers/}{Work with us}
\item
  \href{https://nytmediakit.com/}{Advertise}
\item
  \href{http://www.tbrandstudio.com/}{T Brand Studio}
\item
  \href{https://www.nytimes3xbfgragh.onion/privacy/cookie-policy\#how-do-i-manage-trackers}{Your
  Ad Choices}
\item
  \href{https://www.nytimes3xbfgragh.onion/privacy}{Privacy}
\item
  \href{https://help.nytimes3xbfgragh.onion/hc/en-us/articles/115014893428-Terms-of-service}{Terms
  of Service}
\item
  \href{https://help.nytimes3xbfgragh.onion/hc/en-us/articles/115014893968-Terms-of-sale}{Terms
  of Sale}
\item
  \href{https://spiderbites.nytimes3xbfgragh.onion}{Site Map}
\item
  \href{https://help.nytimes3xbfgragh.onion/hc/en-us}{Help}
\item
  \href{https://www.nytimes3xbfgragh.onion/subscription?campaignId=37WXW}{Subscriptions}
\end{itemize}
