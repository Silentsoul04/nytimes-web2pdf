Sections

SEARCH

\protect\hyperlink{site-content}{Skip to
content}\protect\hyperlink{site-index}{Skip to site index}

\href{https://www.nytimes3xbfgragh.onion/section/us}{U.S.}

\href{https://myaccount.nytimes3xbfgragh.onion/auth/login?response_type=cookie\&client_id=vi}{}

\href{https://www.nytimes3xbfgragh.onion/section/todayspaper}{Today's
Paper}

\href{/section/us}{U.S.}\textbar{}Hurricane Maria Updates: In Puerto
Rico, the Storm `Destroyed Us'

\url{https://nyti.ms/2yfmgvg}

\begin{itemize}
\item
\item
\item
\item
\item
\item
\end{itemize}

Advertisement

\protect\hyperlink{after-top}{Continue reading the main story}

Supported by

\protect\hyperlink{after-sponsor}{Continue reading the main story}

\hypertarget{hurricane-maria-updates-in-puerto-rico-the-storm-destroyed-us}{%
\section{Hurricane Maria Updates: In Puerto Rico, the Storm `Destroyed
Us'}\label{hurricane-maria-updates-in-puerto-rico-the-storm-destroyed-us}}

\includegraphics{https://static01.graylady3jvrrxbe.onion/images/2017/09/22/us/22live-storm2/22live-storm2-articleInline.jpg?quality=75\&auto=webp\&disable=upscale}

By The New York Times

\begin{itemize}
\item
  Sept. 21, 2017
\item
  \begin{itemize}
  \item
  \item
  \item
  \item
  \item
  \item
  \end{itemize}
\end{itemize}

\emph{Read the latest on}
\href{https://www.nytimes3xbfgragh.onion/2017/09/22/us/hurricane-maria-puerto-rico.html}{\emph{Hurricane
Maria}} \emph{with Friday's live updates.}

Puerto Rico remained in the throes of chaos and devastation Thursday as
the remnants of Hurricane Maria continued to dump rain on the island ---
up to three feet in some areas.

Flash flood warnings persisted,
\href{http://www.nhc.noaa.gov/news/AL152017_key_messages.png?023}{according
to the National Hurricane Center}, with ``catastrophic'' flooding
``especially in areas of mountainous terrain.''

The strikingly powerful storm had rendered an estimated 3.4 million
people without power, and with the territory's energy grid all but
destroyed, Gov. Ricardo Rosselló predicted a long period of recovery.
Anxious relatives in the mainland United States and elsewhere took to
social media in an effort to find news of their loved ones.

Late Thursday, the mayor of Toa Baja, a town in northern Puerto Rico,
told The New York Times that eight people had drowned there after
flooding. That brought to at least 10 the number who have died in Puerto
Rico as a result of Hurricane Maria, a toll that is expected to climb.

Puerto Rico faces numerous obstacles as it begins to emerge from the
storm: the weight of an extended debt and bankruptcy crisis; a recovery
process begun after Irma, which killed at least three people and left
nearly 70 percent of households without power; the difficulty of getting
to an island far from the mainland; and the strain on relief efforts by
the Federal Emergency Management Agency and other groups already spread
thin in the wake of several recent storms.

``Irma gave us a break, but Maria destroyed us,'' said Edwin Serrano, a
construction worker in Old San Juan.

The storm churned off the northern coast of the Dominican Republic as a
Category 3 hurricane on Thursday, and the National Hurricane Center
repeated hurricane warnings for late Thursday and early Friday morning
for the southeastern Bahamas and the Turks and Caicos.

\href{https://www.nytimes3xbfgragh.onion/interactive/2017/09/18/world/americas/hurricane-maria-tracking-map.html}{}

\includegraphics{https://static01.graylady3jvrrxbe.onion/images/2017/09/18/world/americas/hurricane-maria-tracking-map-1505756148075/hurricane-maria-tracking-map-1505756148075-articleLarge-v10.png}

\hypertarget{maps-hurricane-marias-path-across-puerto-rico}{%
\subsection{Maps: Hurricane Maria's Path Across Puerto
Rico}\label{maps-hurricane-marias-path-across-puerto-rico}}

Real-time map showing the position and forecast for Hurricane Maria, and
the storm's impact in Puerto Rico.

\textbf{Here's the latest:}

• At least eight people died after drowning in different parts of Toa
Baja, the mayor of the town, Bernardo Márquez, said in an interview. The
mayor said that storm surge and river overflows had burst through open
floodgates and that alarms did not sound when the floodgates were opened
because of faulty maintenance. Three children and a police officer were
among the dead, he said.

The newly reported deaths came after Governor Rosselló told CNN late
Wednesday that officials knew of one fatality in the commonwealth
involving a man who was hit by a board. The United States Coast Guard
also reported the
\href{https://content.govdelivery.com/accounts/USDHSCG/bulletins/1b8858e}{death
of a man aboard a capsized vessel} near Vieques, P.R.

• The death toll from Hurricane Maria has risen to at least 15 on the
small Caribbean island of Dominica, according to Prime Minister
Roosevelt Skerrit. Two people were also killed on the French Caribbean
island of Guadeloupe, officials said.

• Forecasters say Puerto Rico will see up to an additional eight inches
of rain through Saturday, adding to the several feet of rain that has
fallen on parts of the island. Caguas, in the central mountains, has
received the most rain on the island during Maria, 37.9 inches,
according to the National Weather Service.

• There is significant concern about the expected ``life-threatening''
storm surge of nine to 12 feet in the Bahamas and the Turks and Caicos,
according to Michael Brennan of the National Hurricane Center.

• President Trump said Thursday that he would visit Puerto Rico, but he
gave no details on the timing of the trip.

• In Puerto Rico, Governor Rosselló had previously set a 6 p.m. to 6
a.m. curfew effective until Saturday. On Thursday, he requested that
Gov. Andrew M. Cuomo of New York help with recovery efforts. Governor
Cuomo will travel to Puerto Rico with key emergency response officials
as early as Friday morning, his office said in a statement.

• In the United States Virgin Islands, Gov. Kenneth E. Mapp announced
that a 24-hour curfew would be scaled back on three islands beginning on
Friday morning. And although the full curfew remained in place on St.
Croix, Governor Mapp said residents would be allowed to leave their
homes to get supplies during a four-hour window on Friday.

• In the British Virgin Islands, government officials issued an all
clear at 9 a.m. on Thursday morning, lifting a curfew that had been in
effect since 6 p.m. on Tuesday.

\includegraphics{https://static01.graylady3jvrrxbe.onion/images/2017/09/22/world/22live-strom-1/22live-strom-1-articleInline.jpg?quality=75\&auto=webp\&disable=upscale}

• Sign up for the
\href{https://www.nytimes3xbfgragh.onion/newsletters/morning-briefing?mcubz=1}{Morning
Briefing} for hurricane news and a daily look at what you need to know
to begin your day.
\href{https://www.nytimes3xbfgragh.onion/interactive/2017/09/18/world/americas/hurricane-maria-tracking-map.html}{Follow
the storm's path with our maps}.

\hypertarget{restoring-power-is-a-priority-in-puerto-rico}{%
\subsection{Restoring power is a priority in Puerto
Rico.}\label{restoring-power-is-a-priority-in-puerto-rico}}

Image

San Juan, P.R., was plunged into darkness on Wednesday after Hurricane
Maria made landfall.Credit...Alex Wroblewski/Getty Images

Jenniffer González-Colón, Puerto Rico's nonvoting member of the House of
Representatives, told CNN on Thursday that the island appeared to have
been ``devastated,'' with power lines lying on the ground and rivers
flowing over bridges.

Ms. González-Colón, who spent much of the hurricane in a closet, said
restoring power was crucial, but added that the governor had estimated
that it could take a month or more to get electricity back for the whole
island. She suggested that without electricity, many of the pumps that
supplied residents with running water would not be functioning.

Complicating matters, more than 95 percent of the island's wireless cell
sites are out of service, said Ajit Pai, the chairman of the Federal
Communications Commission.

Ricardo Ramos, the chief executive of the government-owned Puerto Rico
Electric Power Authority, told CNN on Thursday that the island's power
infrastructure had essentially been ``destroyed.''

\hypertarget{a-seaside-area-is-smashed-by-the-storm}{%
\subsection{A seaside area is smashed by the
storm.}\label{a-seaside-area-is-smashed-by-the-storm}}

Image

A fallen sculpture in San Juan, P.R., on Thursday.Credit...Erika P.
Rodriguez for The New York Times

Residents and business owners in the Condado area of San Juan began to
trickle into the streets on Thursday to assess the havoc. Joggers ran
past what resembled a beachside battlefield. Bikers pedaled slowly,
taking in the overwhelming damage.

Condado, the tourist district of the island, which has seen a
reawakening of sorts with the opening of new hotels and restaurant
chains over the last couple of years, was ravaged. Windows were blown
out in the apartment buildings and hotels that line the promenade. A
restaurant lost its roof. Parque del Indio, a popular seaside park for
skaters and joggers, was blanketed with sand and water.

``It's total destruction,'' said Angie Mok, a property manager. ``This
will be a renaissance.''

Ms. Mok's fourth-floor seaside apartment was destroyed. Her apartment
had no shutters, and the wind rattled her belongings, while ankle-high
water soaked the floors.

In Old San Juan, which like most of the island was without reliable cell
service, people were thirsty for information. At Plaza de Armas,
residents sat on benches and stoops to share what information they had.
Those with radios were tuning in to the only station broadcasting in the
entire island.

Cristina Cardalda, 55, had just gotten her first phone call since Maria
hit --- it was her cousin in Florida checking in. ``I haven't heard
anything from anyone,'' she said.

\hypertarget{family-members-on-the-mainland-are-desperately-seeking-information}{%
\subsection{Family members on the mainland are `desperately seeking
information.'}\label{family-members-on-the-mainland-are-desperately-seeking-information}}

For Puerto Ricans living on the United States mainland, the tragic news
coming from the island has been magnified by the fact that many of them
have been unable to get in touch with friends and relatives, given the
sharp blow that Hurricane Maria dealt to the island's communications
infrastructure.

``We're all anxious, we're all desperately seeking information and we're
all on call to help Puerto Rico and give it whatever it needs,'' said
David Galarza Santa, 48, a Brooklyn resident who said he had been unable
to reach his family in the municipality of Florida, west of San Juan,
since noon Wednesday.

But Mr. Galarza was optimistic that his relatives there, including his
father and two older sisters, were doing well, in part because they had
all hunkered down at his father's sturdy cement house. He also noted
that Puerto Ricans are old hands when it comes to surviving devastating
storms.

More than five million Puerto Ricans live on the mainland, more than the
population of the island itself, and the worry and stress were widely
shared Thursday among those watching from afar. It was a feeling of
``impotence,'' said Eliezer Vélez, 44, of Atlanta.

``You're here, but your mind and your heart are on the island,'' he
said. ``We are here, but we belong there. I cannot describe the
frustration that I'm not there.''

\hypertarget{puerto-rico-is-in-perilous-shape-trump-says}{%
\subsection{Puerto Rico is in `perilous shape,' Trump
says.}\label{puerto-rico-is-in-perilous-shape-trump-says}}

\includegraphics{https://static01.graylady3jvrrxbe.onion/images/2017/09/22/us/22storm-alpha/22storm-alpha-videoSixteenByNine3000-v3.jpg}

``Puerto Rico was absolutely obliterated,'' President Trump said during
a meeting with President Petro Poroshenko of Ukraine at the United
Nations in New York on Thursday.

``Their electrical grid is destroyed,'' Mr. Trump added. ``It wasn't in
good shape to start off with. But their electrical grid is totally
destroyed. And so many other things.''

Mr. Trump has declared Puerto Rico a disaster area, and he said he was
consulting with Governor Rosselló and federal officials about the
recovery effort. ``We are going to start it with great gusto,'' he said.
``But it's in very, very, very perilous shape. Very sad what happened to
Puerto Rico.''

Despite the challenges his island faces, Governor Rosselló
\href{https://twitter.com/ricardorossello/status/910673245900361729}{said
on Twitter} that ``we will come out of this stronger than ever.''

In an interview with WAPA radio on Thursday, Mr. Rosselló said that
reopening the island's ports was a priority, because it would allow for
shipments of aid, including generators, food, cots and first aid
materials.

The White House has also declared the United States Virgin Islands a
disaster area, to make federal funding available for residents of St.
Croix.

Governor Mapp said that the western end of St. Croix had taken what he
called a ``beating.'' At least one school there appeared to have been
destroyed, and portions of another had been seriously damaged. The power
system was down on the island, he added.

Juan F. Luis Hospital has also been serious damaged, he said, adding
that dozens of patients would be evacuated as soon as Friday ---
potentially as far as South Carolina.

Governor Mapp said officials were not aware of ``any significant
number'' of people who had been injured by Maria or of any deaths
connected to the storm.

``The worst is finally behind us,'' Mr. Mapp said. ``Now it's really
time to march forward.''

\hypertarget{dominica-grapples-with-widespread-destruction}{%
\subsection{Dominica grapples with widespread
destruction.}\label{dominica-grapples-with-widespread-destruction}}

Mr. Skerrit, the prime minister, called the damage from the storm
``unprecedented'' in an interview with a television station on the
nearby island of Antigua on Thursday. He gave a death toll of 15, but
warned that the number might rise, with at least 20 people still missing
and dozens of villages yet to be assessed.

``We have many deaths, but it is just a miracle that we do not have
hundreds of deaths in the country,'' Mr. Skerrit said. ``Because when
you look at the destruction, people were in those homes.''

The storm made landfall in Dominica on Monday with winds of up to 155
miles per hour. Aerial views of the island showed houses and businesses
torn apart by the storm.

Mr. Skerrit described ``almost complete'' devastation: Power and water
have been cut across the island, communications are nearly impossible,
schools have been destroyed and the main hospital is without
electricity. Indigenous villages on the country's east coast have yet to
be assessed, and he said it would be a ``miracle'' if there was no loss
of life there.

Mr. Skerrit said residents had taken the risk seriously, and many
evacuated to shelters before the storm hit, which mitigated the loss of
life. Many displaced residents are still in shelters, he said, and some
are staying with neighbors in the few homes that would have survived,
but many do not know where to spend the night.

Mr. Skerrit will travel to New York on Friday to meet with international
leaders at the United Nations General Assembly.

\hypertarget{st-croix-a-center-of-earlier-relief-efforts-now-needs-help}{%
\subsection{St. Croix, a center of earlier relief efforts, now needs
help.}\label{st-croix-a-center-of-earlier-relief-efforts-now-needs-help}}

Image

Supplies for those affected by Hurricane Irma in St. Thomas were loaded
onto a boat in Christiansted, St. Croix, on Sunday.Credit...Chip
Somodevilla/Getty Images,

In St. Croix, which was the hardest hit of the United States Virgin
Islands, response efforts were already underway on Thursday night.

Emily Weston, who owns a bar and restaurant in Frederiksted, said some
grocery stores had opened, and distribution centers had been set up.
Kathy Tiddark, a Frederiksted resident who owns Caribbean Breeze
Apartments and Vacation Homes with her husband, said she was worried
about security after watching a nearby business get destroyed by
looters.

Still, the Tiddarks fared better than many others: Their approximately
300-year-old guesthouse remained intact.

``My neighbor's house was in my yard upside down,'' she said. ``The
whole front side of the house, the full patio and the roof just ripped
off and flipped upside down and landed right in our courtyard.''

Advertisement

\protect\hyperlink{after-bottom}{Continue reading the main story}

\hypertarget{site-index}{%
\subsection{Site Index}\label{site-index}}

\hypertarget{site-information-navigation}{%
\subsection{Site Information
Navigation}\label{site-information-navigation}}

\begin{itemize}
\tightlist
\item
  \href{https://help.nytimes3xbfgragh.onion/hc/en-us/articles/115014792127-Copyright-notice}{©~2020~The
  New York Times Company}
\end{itemize}

\begin{itemize}
\tightlist
\item
  \href{https://www.nytco.com/}{NYTCo}
\item
  \href{https://help.nytimes3xbfgragh.onion/hc/en-us/articles/115015385887-Contact-Us}{Contact
  Us}
\item
  \href{https://www.nytco.com/careers/}{Work with us}
\item
  \href{https://nytmediakit.com/}{Advertise}
\item
  \href{http://www.tbrandstudio.com/}{T Brand Studio}
\item
  \href{https://www.nytimes3xbfgragh.onion/privacy/cookie-policy\#how-do-i-manage-trackers}{Your
  Ad Choices}
\item
  \href{https://www.nytimes3xbfgragh.onion/privacy}{Privacy}
\item
  \href{https://help.nytimes3xbfgragh.onion/hc/en-us/articles/115014893428-Terms-of-service}{Terms
  of Service}
\item
  \href{https://help.nytimes3xbfgragh.onion/hc/en-us/articles/115014893968-Terms-of-sale}{Terms
  of Sale}
\item
  \href{https://spiderbites.nytimes3xbfgragh.onion}{Site Map}
\item
  \href{https://help.nytimes3xbfgragh.onion/hc/en-us}{Help}
\item
  \href{https://www.nytimes3xbfgragh.onion/subscription?campaignId=37WXW}{Subscriptions}
\end{itemize}
