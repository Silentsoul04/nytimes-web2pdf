Sections

SEARCH

\protect\hyperlink{site-content}{Skip to
content}\protect\hyperlink{site-index}{Skip to site index}

\href{https://www.nytimes3xbfgragh.onion/section/world/asia}{Asia
Pacific}

\href{https://myaccount.nytimes3xbfgragh.onion/auth/login?response_type=cookie\&client_id=vi}{}

\href{https://www.nytimes3xbfgragh.onion/section/todayspaper}{Today's
Paper}

\href{/section/world/asia}{Asia Pacific}\textbar{}In Grim Camps,
Rohingya Suffer on `Scale That We Couldn't Imagine'

\url{https://nyti.ms/2yxurDk}

\begin{itemize}
\item
\item
\item
\item
\item
\item
\end{itemize}

Advertisement

\protect\hyperlink{after-top}{Continue reading the main story}

Supported by

\protect\hyperlink{after-sponsor}{Continue reading the main story}

\hypertarget{in-grim-camps-rohingya-suffer-on-scale-that-we-couldnt-imagine}{%
\section{In Grim Camps, Rohingya Suffer on `Scale That We Couldn't
Imagine'}\label{in-grim-camps-rohingya-suffer-on-scale-that-we-couldnt-imagine}}

By \href{https://www.nytimes3xbfgragh.onion/by/ben-c-solomon}{Ben C.
Solomon}

\begin{itemize}
\item
  Sept. 29, 2017
\item
  \begin{itemize}
  \item
  \item
  \item
  \item
  \item
  \item
  \end{itemize}
\end{itemize}

BALUKHALI, Bangladesh --- Up to their ankles in mud, hundreds of
Rohingya refugees fought to the front of the crowd outside of their
\href{https://www.nytimes3xbfgragh.onion/video/world/asia/100000005453328/inside-a-refugee-camp-born-of-massacre.html?mcubz=1}{makeshift
camp}. An open-bed truck full of Bangladeshi volunteers was passing by,
tossing out donated goods at random: small bags of rice, a faded
SpongeBob SquarePants T-shirt, a cluster of dirty forks.

Entire families sloshed through the rain hoping to grab whatever they
could. One boy, no older than 6, squeezed his way to an opening where a
pair of oversize men's jeans came hurtling off a truck. He had to fight
off an older boy before he could run off with the prize.

There were already more than 200,000 ethnic Rohingya migrants stuck in
camps like this one, Balukhali, in southern Bangladesh. But over the
past month, at least 500,000 more --- more than half of the Rohingya
population thought to have been living in Myanmar --- are reported to
have fled over the border to take refuge, surpassing even the worst
month of the Syrian war's refugee tide.

As international leaders squabble over whether to punish Myanmar for the
military's methodical killing and uprooting of Rohingya civilians, the
recent arrivals are living in abjectly desperate conditions.

This is not so much a defined camp as a dense collection of bamboo and
tarp stacks. When I visited, children were wandering in the mud looking
for food and clothes. There are worries about cholera and tuberculosis.
With no toilets, what's left of the forest has become a vast, improvised
bathroom.

While the flow of refugees has greatly slowed in the past week, aid
organizations are still overwhelmed.

``It's on a scale that we couldn't imagine,'' said Kate White, the
Doctors Without Borders medical emergency manager in Bangladesh. ``This
is a small piece of land, and everyone is condensed into it. We just
can't scale up fast enough.''

For decades, the Muslim Rohingya of Myanmar, a minority concentrated in
the western state of Rakhine, have faced systemic repression by the
country's Buddhist majority, and particularly by the military. But what
happened in August, when the military and allied mobs began
\href{https://www.nytimes3xbfgragh.onion/interactive/2017/09/18/world/asia/rohingya-villages.html?_r=0}{burning
whole Rohingya villages}, was so much worse that the
\href{https://www.nytimes3xbfgragh.onion/2017/09/11/world/asia/myanmar-rohingya-ethnic-cleansing.html}{United
Nations is calling it ethnic cleansing}.

Across the camps, the escaped all have accounts of fire and cruelty.

Anwar Begum, 73, sat on the ground, her arm limp below the elbow. She
was in constant pain and could hardly focus her eyes. The army, she told
me, set fire to her village in Myanmar and cleared the people out. As
the civilians fled, one soldier singled her out, saying, ``You're not
welcome in Myanmar,'' and smashing her elbow with the butt of his rifle.
As her family dragged her away, the soldier had one last thing to say:
``Bring that to Bangladesh.''

What was once a loose network of camps has become a sprawl. Acres upon
acres of forest have been razed to make way for small cities of huts,
made from cheap black tarps covered in mud. Across the camp, men are
building them as fast as they can.

Every medical treatment post here has a line that snakes nearly around
the camp. Local doctors and foreign aid organizations like Doctors
Without Borders are scrambling to set up more clinics but can hardly
keep pace.

With just one main road connecting most of the district, aid groups are
struggling to reach the most remote camps. In Taink Khali, a nearly
30-minute hike off the main road, an Australian aid agency called
Disaster Response Group was setting up a mobile clinic in a tiny shack.
Dozens of women and children were waiting quietly in the hot sun.

``We're the first aid these people have seen,'' said Brad Stewart,
operations manager for the aid group, a small medical assistance
organization that most often serves backpackers in Nepal. His team of
four ex-military Australians were taping a bottle of hand sanitizer to a
tree.

``The immediate attention is going to the more established camps,'' he
said. ``Out here, we're just a drop in a very large bucket.''

For the hundreds still coming, they face a dangerous boat ride across
the river border into Bangladesh. On Thursday, dozens of Rohingya, many
of them children, were killed as a trawler carrying them capsized in the
Bay of Bengal.

Their bodies washed up on the bay alongside some survivors.

``The women and children couldn't swim,'' said Nuru Salam, 22. He had
tried to cross with his entire family when the boat tipped in the sea.
His son had died and he was waiting to find his wife's body. ``There are
still many more bodies to come.''

The physical challenges here are steep enough. But the Rohingya crisis
has shaken the entire region, exacerbating already severe sectarian and
political strains and worsening the relationship between Bangladesh and
Myanmar, in particular.

This is not the first wave of Rohingya that Bangladesh has had to
absorb. In 1978, around 200,000 Rohingya fled into the country. Most
returned to Myanmar after the two governments hammered out a
repatriation deal. Another influx came in the early 1990s, as well as in
2012 and in October of last year.

Even before this latest exodus from Myanmar, the stresses being placed
on this already poor country were considerable. Yet some of the locals
have shown remarkable kindness.

``I see that the host community here has been incredibly positive,''
said Karim Elguindi, head of the World Food Program's office in Cox's
Bazar, Bangladesh. ``I'm still surprised by the humanitarian response of
the government and the community.''

Mr. Elguindi has worked in Darfur, the strife-torn region of western
\href{https://www.nytimes3xbfgragh.onion/topic/destination/sudan?inline=nyt-geo}{Sudan},
and he noted that while the sheer concentration of arriving Rohingya was
unprecedented and the terrain challenging for aid distribution, the
Rohingya could at least count on basic security once they made it to
Bangladesh.

``Compared with Darfur, here they speak the same language,'' he said.
``The refugees in Darfur were in I.D.P. camps,'' referring to internally
displaced persons. ``They were still among enemies. Here, they are
relatively safe.''

With existing camps beyond their capacity, the Bangladeshi government is
racing to convert an additional 2,000 acres of land into settlements for
the new arrivals. But a report from a network of United Nations agencies
warned that Rohingya refugees had already arrived at the site before
adequate infrastructure and services had been set up. The local
authorities have begun limiting Rohingya refugees to the camps, setting
up police checkpoints to prevent them from leaving.

On Sunday, a Bangladeshi cabinet minister said that the government
\href{http://www.dhakatribune.com/bangladesh/2017/09/24/amu-no-plan-give-refugee-status-rohingya/}{did
not plan to give refugee status} to the newly arrived Rohingya --- a
stance that is complicating efforts to get them more aid. The
Bangladeshi government has said it hopes that Myanmar will eventually
take back the Rohingya.

The Myanmar government, however, has said that it will only repatriate
those with the correct documentation to prove they are from Rakhine. It
is unlikely that most of the Rohingya who recently fled to Bangladesh
brought such papers with them, if they ever possessed them at all.

So far, the bulk of the aid effort has fallen to groups of Bangladeshi
volunteers. Touched by the stories they have seen on local television,
many across the nation have started donation campaigns and driven long
distances to give what they can to struggling refugees.

``We couldn't just sit at home,'' said Abul Hossain, a volunteer who
lives six hours north of the camps. ``Last week we asked everyone in our
village to donate. We drove all night to bring it here.''

Mr. Hossain and his neighbors had hoped to hand the goods over to
government workers or foreign aid organizations like the United Nations,
but they say they have not been able to find any.

``We've been driving around since the morning looking for anyone to take
this,'' he said. ``So now, we're just handing it out ourselves.''

This has proved a dangerous method. Last week, CNN reported that
\href{http://edition.cnn.com/2017/09/17/asia/stampede-bangladesh-rohingya/index.html)}{a
woman and two children were killed} in a stampede as a group of
Bangladeshis threw food from a truck.

Since then, assistance has improved significantly. Food distribution
points have been set up by the Bangladeshi government to try to lessen
the need for the volunteer truck visits. But it is still a challenge.

``There are no roads. We've been carrying our equipment in the heavy
rain,'' said Ms. White, from Doctors Without Borders.

Near his truck at the entrance to the camp, Mr. Hossain unloaded some of
his goods as the lines grew around him.

``We expect these people will return home one day. For now we will help
them --- they are our brothers and sisters.'' Then his index finger went
up. ``But maybe not forever.''

Advertisement

\protect\hyperlink{after-bottom}{Continue reading the main story}

\hypertarget{site-index}{%
\subsection{Site Index}\label{site-index}}

\hypertarget{site-information-navigation}{%
\subsection{Site Information
Navigation}\label{site-information-navigation}}

\begin{itemize}
\tightlist
\item
  \href{https://help.nytimes3xbfgragh.onion/hc/en-us/articles/115014792127-Copyright-notice}{©~2020~The
  New York Times Company}
\end{itemize}

\begin{itemize}
\tightlist
\item
  \href{https://www.nytco.com/}{NYTCo}
\item
  \href{https://help.nytimes3xbfgragh.onion/hc/en-us/articles/115015385887-Contact-Us}{Contact
  Us}
\item
  \href{https://www.nytco.com/careers/}{Work with us}
\item
  \href{https://nytmediakit.com/}{Advertise}
\item
  \href{http://www.tbrandstudio.com/}{T Brand Studio}
\item
  \href{https://www.nytimes3xbfgragh.onion/privacy/cookie-policy\#how-do-i-manage-trackers}{Your
  Ad Choices}
\item
  \href{https://www.nytimes3xbfgragh.onion/privacy}{Privacy}
\item
  \href{https://help.nytimes3xbfgragh.onion/hc/en-us/articles/115014893428-Terms-of-service}{Terms
  of Service}
\item
  \href{https://help.nytimes3xbfgragh.onion/hc/en-us/articles/115014893968-Terms-of-sale}{Terms
  of Sale}
\item
  \href{https://spiderbites.nytimes3xbfgragh.onion}{Site Map}
\item
  \href{https://help.nytimes3xbfgragh.onion/hc/en-us}{Help}
\item
  \href{https://www.nytimes3xbfgragh.onion/subscription?campaignId=37WXW}{Subscriptions}
\end{itemize}
