Sections

SEARCH

\protect\hyperlink{site-content}{Skip to
content}\protect\hyperlink{site-index}{Skip to site index}

\href{https://www.nytimes3xbfgragh.onion/section/theater}{Theater}

\href{https://myaccount.nytimes3xbfgragh.onion/auth/login?response_type=cookie\&client_id=vi}{}

\href{https://www.nytimes3xbfgragh.onion/section/todayspaper}{Today's
Paper}

\href{/section/theater}{Theater}\textbar{}Review: `Springsteen on
Broadway' Reveals the Artist, Real and Intense

\url{https://nyti.ms/2z3YOl2}

\begin{itemize}
\item
\item
\item
\item
\item
\item
\end{itemize}

Advertisement

\protect\hyperlink{after-top}{Continue reading the main story}

Supported by

\protect\hyperlink{after-sponsor}{Continue reading the main story}

\hypertarget{review-springsteen-on-broadway-reveals-the-artist-real-and-intense}{%
\section{Review: `Springsteen on Broadway' Reveals the Artist, Real and
Intense}\label{review-springsteen-on-broadway-reveals-the-artist-real-and-intense}}

\includegraphics{https://static01.graylady3jvrrxbe.onion/images/2017/10/13/arts/13Springsteen2/13Springsteen2-articleLarge.jpg?quality=75\&auto=webp\&disable=upscale}

\begin{itemize}
\tightlist
\item
  Springsteen on Broadway\\
  **NYT Critic's Pick Broadway, Blues, Folk, Musical, Rock, Special
  Event 2 hrs. Closing Date: Dec. 15, 2018 Walter Kerr Theater, 219 W.
  48th St. 877-250-2929
\end{itemize}

By \href{http://www.nytimes3xbfgragh.onion/by/jesse-green}{Jesse Green}

\begin{itemize}
\item
  Oct. 12, 2017
\item
  \begin{itemize}
  \item
  \item
  \item
  \item
  \item
  \item
  \end{itemize}
\end{itemize}

There came a moment the other night, near the end of Bruce Springsteen's
overwhelming and uncategorizable Broadway show, when it seemed possible
to see straight through his many masks to some core truth of his being.

This was when the audience, which had mostly restrained itself through
the first 13 songs of the 15-song set, could no longer sit on its hands
as if in church. The show had been, to that point, quite solemn --- and
would continue to be.

But now, entire swaths of the Walter Kerr Theater, apparently unmindful
of downbeat lyrics like ``I ain't nothing but tired,'' started clapping
along to ``\href{https://www.youtube.com/watch?v=129kuDCQtHs}{Dancing in
the Dark},'' Mr. Springsteen's biggest hit, from 1984.

He stopped cold. ``I'll handle it myself,'' he said, shutting them down
with a small, sharky glint of a smile.

Would he ever! Make no mistake,
``\href{https://brucespringsteen.net/broadway/}{Springsteen on
Broadway},'' which opened on Thursday evening, is a solo act by a solo
artist with an artist's steel. Even though Patti Scialfa, his wife,
shows up to harmonize on two numbers, this is not a singalong arena show
or a roadhouse rouser. Even less does it try to be a feel-good Broadway
book musical or a slick, whitewashed jukebox like
``\href{http://www.nytimes3xbfgragh.onion/2005/11/07/theater/reviews/from-bluecollar-boys-to-doowop-sensation-a-bands-rise-and-fall.html}{Jersey
Boys.}''

Rather, ``Springsteen on Broadway'' is a painful if thrilling summing-up
at 68: a major statement about a life's work, but also a major revision
of it. As music acts go, it thus has more in common with Lena Horne's
revelatory
``\href{http://www.nytimes3xbfgragh.onion/1981/05/13/arts/theater-lena-horne-the-lady-and-her-music.html}{The
Lady and Her Music}'' from 1981 than with a greatest-hits concert by the
likes of
\href{https://www.timeout.com/newyork/theater/manilow-on-broadway-st-james-theatre-february-19-2013}{Barry
Manilow}.

Call it a greatest anti-hits concert: Many of the songs Mr. Springsteen
has chosen to sing are less familiar and more meditative than his
chart-toppers, and those that were chart-toppers are almost
unrecognizable.

That's why the show's version of ``Dancing in the Dark'' admits no
clapping; sung at a slower-than-usual tempo, and accompanied only by Mr.
Springsteen on acoustic guitar, it is no longer the casual invitation to
sex it seemed to be in its first incarnation. It is instead a parable
about the nihilism underlying such invitations.

Nor is ``\href{https://www.youtube.com/watch?v=EPhWR4d3FJQ}{Born in the
U.S.A.},'' also from 1984, the jingoistic anthem it once sounded like on
MTV, when the thrust engine roar of the E Street Band sent it into
orbit. With its choruses now spat away quickly and its bleak verses
about damaged veterans dwelt on, it is, as Mr. Springsteen says he
always intended, a ``protest song.''

\includegraphics{https://static01.graylady3jvrrxbe.onion/images/2017/10/13/arts/13Springsteen1/13Springsteen1-articleLarge.jpg?quality=75\&auto=webp\&disable=upscale}

This will not be news to fans who have been paying attention to him
during the 30 years since his sleeveless T-shirt and bandanna heyday, or
to anyone who has read his hair-raising 2016 memoir,
``\href{https://www.nytimes3xbfgragh.onion/2016/09/21/books/bruce-springsteen-memoir-born-to-run.html}{Born
to Run}.'' There he outlined an ideal of rock music as a ``culture
shaper'' and an ideal of himself as someone who would ``collide with the
times'' in order to change them. ``Springsteen on Broadway'' distills
the same daunting dream; its spoken portions, which make up about half
of the two-hour show, are mostly taken from the book or build on its
ideas.

Not that the music stops much even then; when telling stories about his
chaotic upbringing in the ``crap town'' of Freehold, N.J., he is often
picking at one of his guitars or vamping on the piano, his eyes
half-shut. The same two-chord figure continues for what seems like
hundreds of bars, making apt, trancelike accompaniments to a
claustrophobic tale of ``church, school, homework and string beans.''
Songs like ``\href{https://www.youtube.com/watch?v=1xY3q45EBt8}{Growin'
Up},'' ``\href{https://www.youtube.com/watch?v=77gKSp8WoRg}{My
Hometown}'' and ``\href{https://www.youtube.com/watch?v=d5xZmgFOuRA}{My
Father's House}'' draw their expressivity from (or despite) a similar
sense of confinement.

Admittedly, this is a well honed story, a self-portrait of a mask. There
have been many such masks over the years --- mama's boy, loner, stadium
stud, Woody Guthrie --- each developed through songs that would seem to
cancel each other out.

``Springsteen on Broadway'' sets out to reconcile that contradiction by
reshaping his glibber hits and personas along the lines of his mature,
``\href{https://www.youtube.com/watch?v=NKKpmbcSe5E}{Ghost of Tom
Joad}'' vision. But first he acknowledges the contradiction, using words
like ``fraud'' to describe himself: a bard of the working class who has
``never held an honest job'' and a rebel who despite his millions still
lives 10 minutes from the house he grew up in. Even his five-decade
career has been a ``magic trick,'' turning the unpromising tools of his
cheese-grater voice and ``hideous'' appearance into a vehicle for the
primal rock message that ``fun is a birthright.''

That's another mask: rock Dionysus. In any case, ``fun'' is not a word
I'd use to describe ``Springsteen on Broadway.'' (How about
``relentlessly serious'' for the marquee?) On Heather Wolensky's
abandoned warehouse set, under Natasha Katz's deeply shadowed lighting,
Mr. Springsteen comes off as the kind of character he often writes
about: a pink-slipped worker in a shuttered factory in a dying industry.
Fair enough. There is little left of the music business that could once
breed and elevate a performer like him, whose ear is tuned to the whole
world's injustice.

At other times, in the startling intimacy of the 939-seat theater (Mr.
Springsteen often plays stadiums that are vastly larger) and with the
help of Brian Ronan's you-are-there sound design, ``Springsteen on
Broadway'' seems like a radio monodrama broadcast from the deepest
interior of a single troubled soul. His voice, still quite capable of
what he calls in the memoir his ``Jersey-Pavarotti-via-Roy Orbison
singing,'' more often sounds like the howl of a dog caught in a
barbed-wire fence. His guitar sounds like the barbed wire.

Are there any other artists of Mr. Springsteen's stature who would
choose to drive a show --- on Broadway yet --- so close and so often
toward what he calls the ``suicide watch''? (He is credited as both
writer and director.) The opening grimness, unrelieved but for some
self-lacerating jokes, does not let up until
``\href{https://www.youtube.com/watch?v=x5kXnq5IjdU}{Thunder Road},''
about a half-hour along. The height of the evening's high spirits is
``\href{https://www.youtube.com/watch?v=k8OC43MqZXk}{Tenth Avenue
Freeze-Out},'' a ``boardwalk soul'' dance tune whose first words are
nevertheless ``Teardrops on the city.''

More often we are in the strange, exalted precincts of
``\href{https://www.youtube.com/watch?v=iywFZqtPlhU}{Long Walk Home}''
and the Sept. 11 psalm
``\href{https://www.youtube.com/watch?v=6i-fiRgbpr4}{The Rising}.'' Even
the songs he sings with his wife ---
``\href{https://www.youtube.com/watch?v=idnJnjV_8rg}{Brilliant
Disguise}'' and
``\href{https://www.youtube.com/watch?v=_91hNV6vuBY}{Tougher Than the
Rest}'' --- question the possibility of fidelity, both to oneself and to
others.

But he clearly knows where he's headed and that he can get there; at any
rate he has to keep this up five times a week through Feb. 3. (The run
is sold out, except for some tickets available by
\href{https://www.nytimes3xbfgragh.onion/2017/09/28/theater/bruce-springsteen-broadway-tickets-lottery.html}{digital
lottery} and those going for more than \$1,000 each on the resale
market.) Perhaps, like a priest, he enjoys the ritual. He does call
music his ``service,'' his ``long and noisy prayer.''

That's another mask, of course, but the thing about his masks is that
they're all real, made with enough craft to see and be seen through.
Indeed, as portraits of artists go, there may never have been anything
as real --- and beautiful --- on Broadway.

Advertisement

\protect\hyperlink{after-bottom}{Continue reading the main story}

\hypertarget{site-index}{%
\subsection{Site Index}\label{site-index}}

\hypertarget{site-information-navigation}{%
\subsection{Site Information
Navigation}\label{site-information-navigation}}

\begin{itemize}
\tightlist
\item
  \href{https://help.nytimes3xbfgragh.onion/hc/en-us/articles/115014792127-Copyright-notice}{©~2020~The
  New York Times Company}
\end{itemize}

\begin{itemize}
\tightlist
\item
  \href{https://www.nytco.com/}{NYTCo}
\item
  \href{https://help.nytimes3xbfgragh.onion/hc/en-us/articles/115015385887-Contact-Us}{Contact
  Us}
\item
  \href{https://www.nytco.com/careers/}{Work with us}
\item
  \href{https://nytmediakit.com/}{Advertise}
\item
  \href{http://www.tbrandstudio.com/}{T Brand Studio}
\item
  \href{https://www.nytimes3xbfgragh.onion/privacy/cookie-policy\#how-do-i-manage-trackers}{Your
  Ad Choices}
\item
  \href{https://www.nytimes3xbfgragh.onion/privacy}{Privacy}
\item
  \href{https://help.nytimes3xbfgragh.onion/hc/en-us/articles/115014893428-Terms-of-service}{Terms
  of Service}
\item
  \href{https://help.nytimes3xbfgragh.onion/hc/en-us/articles/115014893968-Terms-of-sale}{Terms
  of Sale}
\item
  \href{https://spiderbites.nytimes3xbfgragh.onion}{Site Map}
\item
  \href{https://help.nytimes3xbfgragh.onion/hc/en-us}{Help}
\item
  \href{https://www.nytimes3xbfgragh.onion/subscription?campaignId=37WXW}{Subscriptions}
\end{itemize}
