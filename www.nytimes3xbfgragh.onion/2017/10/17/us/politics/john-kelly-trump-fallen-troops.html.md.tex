Sections

SEARCH

\protect\hyperlink{site-content}{Skip to
content}\protect\hyperlink{site-index}{Skip to site index}

\href{https://www.nytimes3xbfgragh.onion/section/politics}{Politics}

\href{https://myaccount.nytimes3xbfgragh.onion/auth/login?response_type=cookie\&client_id=vi}{}

\href{https://www.nytimes3xbfgragh.onion/section/todayspaper}{Today's
Paper}

\href{/section/politics}{Politics}\textbar{}Top General's Grief Becomes
Political Talking Point for Trump

\url{https://nyti.ms/2kX68g8}

\begin{itemize}
\item
\item
\item
\item
\item
\item
\end{itemize}

Advertisement

\protect\hyperlink{after-top}{Continue reading the main story}

Supported by

\protect\hyperlink{after-sponsor}{Continue reading the main story}

\hypertarget{top-generals-grief-becomes-political-talking-point-for-trump}{%
\section{Top General's Grief Becomes Political Talking Point for
Trump}\label{top-generals-grief-becomes-political-talking-point-for-trump}}

\includegraphics{https://static01.graylady3jvrrxbe.onion/images/2017/10/18/us/18dc-obama/18dc-obama-articleLarge.jpg?quality=75\&auto=webp\&disable=upscale}

By \href{http://www.nytimes3xbfgragh.onion/by/mark-landler}{Mark
Landler}

\begin{itemize}
\item
  Oct. 17, 2017
\item
  \begin{itemize}
  \item
  \item
  \item
  \item
  \item
  \item
  \end{itemize}
\end{itemize}

WASHINGTON --- Four days after his son died in Afghanistan,
\href{https://www.nytimes3xbfgragh.onion/2017/07/28/us/politics/john-kelly-chief-of-staff-donald-trump.html?mcubz=3}{John
F. Kelly}, then a Marine Corps general, eulogized two Marines killed by
a truck bomb in Iraq. He made only a fleeting, oblique reference to his
son Second Lt. Robert Kelly as he paid tribute to the two other fallen
service members who held their ground for six seconds as a truck bore
down on them.

Now Mr. Kelly is President Trump's chief of staff, and the commander in
chief is testing his aide's long-held reluctance to discuss his loss.
Mr. Trump, in defending his handling of the
\href{https://www.nytimes3xbfgragh.onion/2017/10/06/world/africa/green-berets-niger-soldiers-killed.html}{deaths
of four Green Berets in Niger}, falsely claimed on Monday that President
Barack Obama did not contact the families of fallen troops. And on
Tuesday, Mr. Trump brought to light that Mr. Obama never called Mr.
Kelly after the death of his son.

Such phone calls are not routine, especially when the rate of
combat-related fatalities is high, as was the case in 2010, when
Lieutenant Kelly was killed after stepping on a land mine while leading
a platoon in Afghanistan. But Mr. Kelly is the highest-ranking American
military officer to lose a child in combat in Iraq or Afghanistan.

Mr. Trump called the families of the soldiers killed in Niger, the White
House said Tuesday, and the president has said that he would try to call
as many families of American troops killed on his watch ``when it's
appropriate.''

``You could ask General Kelly, `Did he get a call from Obama?''' Mr.
Trump said
\href{https://radio.foxnews.com/2017/10/17/president-donald-trump-on-tax-reform-were-the-highest-tax-nation-in-the-world-we-need-the-tax-cuts/}{in
an interview on Tuesday on Fox News Radio}. ``I believe his policy was
somewhat different than my policy. I can tell you my policy is I called
every one of them.''

A spokesman for Mr. Obama declined to comment.

But Mr. Trump's remarks have drawn angry rebukes from allies of the
former president because his claims about Mr. Obama are false --- he
called or met with relatives of multiple fallen service members. Former
military commanders, for their part, said Mr. Trump was politicizing one
of the saddest and most sacred duties of the presidency.

Mr. Obama invited Mr. Kelly and his wife, Karen, to a breakfast for Gold
Star families --- those who have lost children in combat --- at the
White House in May 2011. The couple were seated at a table with the
first lady, Michelle Obama. At the time, Mr. Kelly was the senior
military assistant to Defense Secretary Leon E. Panetta.

Mr. Kelly has not addressed the dispute. But colleagues who worked with
him at the Pentagon during that period said they did not recall him
expressing unhappiness with the way Mr. Obama handled the death of his
son. In a statement issued shortly after Lieutenant Kelly's death, Mr.
Kelly and his wife noted that ``the nation he served has honored us with
promoting him posthumously to first lieutenant of Marines.''

Other military commanders noted that neither Mr. Obama nor President
George W. Bush routinely called the families of fallen soldiers because
that would have been logistically impossible. It could also raise
questions about why one family merited a call but another did not.

``It's been my observation in all the years I've been in the military
that all presidents, as commander in chief, feel an enormous sense of
loss and compassion and pain about those who are killed under their
command,'' said Jack Keane, a retired four-star Army general. ``How they
express that is entirely up to them, and I think we should respect
that.''

\includegraphics{https://static01.graylady3jvrrxbe.onion/images/2017/10/18/us/18DC-Obama2/18DC-Obama2-articleLarge.jpg?quality=75\&auto=webp\&disable=upscale}

Referring to Mr. Trump's predecessors, Gen. Martin E. Dempsey, a former
chairman of the Joint Chiefs of Staff,
\href{https://twitter.com/Martin_Dempsey/status/920105739489816576}{said
on Twitter}: ``POTUS 43 \& 44 and first ladies cared deeply, worked
tirelessly for the serving, the fallen, and their families. Not
politics. Sacred Trust.''

Other former Obama officials castigated Mr. Trump for what they said was
a cynical mischaracterization of Mr. Obama's record, which included
regular visits to Dover Air Force Base to greet the coffins of returning
forces, pilgrimages to Arlington National Cemetery and visits to wounded
troops at Walter Reed National Military Medical Center.

``Stop the damn lying --- you're the President,'' former Attorney
General Eric H. Holder Jr.
\href{https://twitter.com/EricHolder/status/920130744059662336}{wrote on
Twitter}. ``I went to Dover AFB with 44 and saw him comfort the families
of both the fallen military \& DEA.''

When Mr. Trump made the comparison to Mr. Obama and other former
presidents at a news conference on Monday, he dialed back his initial
assertion that Mr. Obama did not call the families of fallen service
members when a reporter asked him about it again.

``President Obama, I think, probably did sometimes and maybe sometimes
he didn't,'' Mr. Trump said, standing in the Rose Garden alongside
Senator Mitch McConnell, Republican of Kentucky and the majority leader.
``That's what I was told. All I can do is ask my generals.''

On Tuesday, it became clear that Mr. Trump's source was Mr. Kelly, a
former commander of the military's Southern Command, whom he recruited
in July from the Department of Homeland Security to serve as chief of
staff. Officials credit Mr. Kelly with
\href{https://www.nytimes3xbfgragh.onion/2017/08/03/us/politics/john-kelly-chief-of-staff-trump.html}{imposing
discipline} on an unruly West Wing staff and on how information is given
to the president.

As a wartime commander, Mr. Kelly led troops in intense combat in
western Iraq. In 2003, he became the first Marine colonel since 1951 to
be promoted to brigadier general while in active combat.

The death of Mr. Kelly's son may have played a role in his selection to
the cabinet. During the transition, people close to Mr. Trump said he
wanted people on his national security team who understood personally
the hazards of sending Americans into battle.

But Mr. Kelly has been reluctant to discuss his son. He asked the hosts
of the eulogy he gave, in St. Louis in 2010, not to mention Lieutenant
Kelly when they introduced him.

Asked about his son on Memorial Day in an interview on Fox \& Friends,
Mr. Kelly, his voice choking, replied, ``Finest guy.''

``Wonderful guy. Wonderful husband. Wonderful son. Wonderful brother.
Brave beyond all get out,'' he added. ``His men still correspond with
us. They still mourn him as we do.''

Advertisement

\protect\hyperlink{after-bottom}{Continue reading the main story}

\hypertarget{site-index}{%
\subsection{Site Index}\label{site-index}}

\hypertarget{site-information-navigation}{%
\subsection{Site Information
Navigation}\label{site-information-navigation}}

\begin{itemize}
\tightlist
\item
  \href{https://help.nytimes3xbfgragh.onion/hc/en-us/articles/115014792127-Copyright-notice}{©~2020~The
  New York Times Company}
\end{itemize}

\begin{itemize}
\tightlist
\item
  \href{https://www.nytco.com/}{NYTCo}
\item
  \href{https://help.nytimes3xbfgragh.onion/hc/en-us/articles/115015385887-Contact-Us}{Contact
  Us}
\item
  \href{https://www.nytco.com/careers/}{Work with us}
\item
  \href{https://nytmediakit.com/}{Advertise}
\item
  \href{http://www.tbrandstudio.com/}{T Brand Studio}
\item
  \href{https://www.nytimes3xbfgragh.onion/privacy/cookie-policy\#how-do-i-manage-trackers}{Your
  Ad Choices}
\item
  \href{https://www.nytimes3xbfgragh.onion/privacy}{Privacy}
\item
  \href{https://help.nytimes3xbfgragh.onion/hc/en-us/articles/115014893428-Terms-of-service}{Terms
  of Service}
\item
  \href{https://help.nytimes3xbfgragh.onion/hc/en-us/articles/115014893968-Terms-of-sale}{Terms
  of Sale}
\item
  \href{https://spiderbites.nytimes3xbfgragh.onion}{Site Map}
\item
  \href{https://help.nytimes3xbfgragh.onion/hc/en-us}{Help}
\item
  \href{https://www.nytimes3xbfgragh.onion/subscription?campaignId=37WXW}{Subscriptions}
\end{itemize}
