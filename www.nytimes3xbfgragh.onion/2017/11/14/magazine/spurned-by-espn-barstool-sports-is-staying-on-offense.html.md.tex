Sections

SEARCH

\protect\hyperlink{site-content}{Skip to
content}\protect\hyperlink{site-index}{Skip to site index}

Spurned by ESPN, Barstool Sports Is Staying on Offense

\url{https://nyti.ms/2jpVWN4}

\begin{itemize}
\item
\item
\item
\item
\item
\item
\end{itemize}

\includegraphics{https://static01.graylady3jvrrxbe.onion/images/2017/11/19/magazine/19barstool1/19barstool1-articleLarge.jpg?quality=75\&auto=webp\&disable=upscale}

Feature

\hypertarget{spurned-by-espn-barstool-sports-is-staying-on-offense}{%
\section{Spurned by ESPN, Barstool Sports Is Staying on
Offense}\label{spurned-by-espn-barstool-sports-is-staying-on-offense}}

The insurgent media company has built a devoted following of what it
sees as ``average'' sports fans: unruly, occasionally toxic and
aggressively male.

Barstool employees in their New York offices: Paul Gulczynski (left) and
Francis Ellis.Credit...Dina Litovsky/Redux, for The New York Times

Supported by

\protect\hyperlink{after-sponsor}{Continue reading the main story}

By Jay Caspian Kang

\begin{itemize}
\item
  Nov. 14, 2017
\item
  \begin{itemize}
  \item
  \item
  \item
  \item
  \item
  \item
  \end{itemize}
\end{itemize}

In the offices of Barstool Sports, on two floors of a narrow building in
the Flatiron district of Manhattan, there's a full bar, a semicircle of
sticky leather recliners and a wall of flat-screen televisions. Almost
every other bit of square footage on the editorial floor is occupied by
a dude at work --- dudes carrying props, dudes spitting tobacco juice
into plastic bottles, dudes typing up blog posts and tracking page
views. The center of the space, marked by a giant menorah, operates as a
full-time modular studio, pumping out all sorts of content: Barstool
produces a Sirius XM radio show (soon to be a whole channel), a growing
list of podcasts and up to 90 blog posts a day. The dudes who make this
stuff are uniformly young and white. In the three days I spent at
Barstool headquarters, the only women I saw were Erika Nardini, the
company's chief executive; Asa Akira, the world-famous porn star who
co-hosted a Barstool podcast; and the security guard.

On an overcast day in June, Dave Portnoy, Barstool's wiry, 40-year-old
founder --- known to followers as El Presidente --- stood shirtless in
front of a green screen rapping, trying to recreate an Instagram video
LeBron James had posted from the gym. At the other end of the office, a
gaggle of dudes in impressively obscure jerseys hyped up a one-on-one
basketball game between two bloggers called Smitty and Gay Pat. I sat
down next to Noah Ives, a pale, slightly hunched intern, who was 21 and
recently graduated from a communications program at Syracuse University.
In decades past, Ives might have spent his summers at ESPN headquarters
in Bristol, Conn., coiling wires and fetching coffee. ``ESPN was
definitely on my mind when I got to college,'' he told me. ``But
Barstool is just a cooler brand that people my age just respect more.''
When I asked him what he meant by ``respect,'' Ives smirked and said:
``The takes are just much more relatable. It's like Pres says ---
Barstool is by the common man, for the common man.'' He paused before
adding: ``ESPN is just spitting facts and political correctness.''

Ives spoke the language of the culture war raging in sports media --- a
mix of marketing jargon and Bannon-lite populism directed, at all times,
at the self-proclaimed Worldwide Leader in Sports, which stands accused
of losing touch with the young bros whose attention it once owned. This
is surely part of why, early this fall, ESPN announced ``Barstool Van
Talk,'' a half-hour talk show that would air at 1 a.m. on a secondary
channel and feature two of Barstool Sports's most palatable
personalities: Consciously or not, the network wanted to co-opt the
resistance.

But the day before the Oct. 17 premiere of ``Van Talk,'' Sam Ponder,
host of ESPN's ``Sunday N.F.L. Countdown,'' tweeted ``Welcome to the
ESPN family'' to one of the show's hosts --- along with two screen
captures of a Barstool blog post from 2014, in which Portnoy called her
a ``Bible-thumping freak'' and wrote that her job's ``\#1 requirement''
was to ``make men hard.'' Other outlets, like USA Today's For the Win
blog, surfaced audio from a Barstool show in which Portnoy calls on
Ponder to be more ``slutty.'' Just six days after the first episode
aired, John Skipper, ESPN's president, announced the cancellation of the
show in a P.R. statement. ``I erred,'' he wrote, ``in assuming we could
distance our efforts from the Barstool site and its content.''

\includegraphics{https://static01.graylady3jvrrxbe.onion/images/2017/11/19/magazine/19barstool2/barstool2-articleLarge.jpg?quality=75\&auto=webp\&disable=upscale}

The cancellation sparked joy in sports media's more progressive ranks,
which tend to view Barstool with the same disdain that their colleagues
in political news might view Breitbart or The Daily Caller. Skipper's
statement was, however, paying Portnoy a kind of compliment. Only two
years ago, any business that wanted to partner with Barstool knew it
would be partnering with the unfiltered chauvinism that made Portnoy a
minor celebrity in Boston, his hometown. Now the president of the
biggest sports network in the world was admitting that he had believed
there was a way to temper Barstool for a mainstream market. ``Once upon
a time, they'' --- ESPN --- ``were the coolest people in the room,''
said Richard Deitsch, a media columnist for Sports Illustrated, in a
podcast discussing the controversy. ``That's not the case in 2017. It's
the opposite. They are polarizing; people dislike them on all sides.''
By partnering with Barstool, he said, they were trying ``to associate
themselves with who they believe are the cool kids in the room.''

Portnoy says he doesn't really think ESPN's future depends on the
implied politics of its on-air personalities, but he knows a branding
opportunity when he sees one. And the entrenched narrative --- that the
once-irreverent network has fed its soul to the hounds of political
correctness and liberal fake news --- is certainly an opportunity. As
professional athletes have knelt for the national anthem, criticized the
president and railed against police violence, ESPN has been repeatedly
accused of being too sympathetic to them and too liberal for its own
good. A company once built on an aggressively apolitical foundation has
somehow become a locus of almost every imaginable type of political
fight. As it has blundered its way through widely publicized incidents
like Jemele Hill's tweets about the president and the N.F.L., the
sports-radio host Clay Travis has taken to calling the network
``MSESPN''; the reporter Britt McHenry has suggested that she was fired
from the network for professing her conservative beliefs. Whether these
critiques are accurate seems largely irrelevant. Neither does it
particularly matter that Portnoy and his cast of bloggers are largely
liberal-leaning dudes from the breeding grounds of the coastal elite.
There exists a swarm of angry sports fans who maintain that they do not
want to talk about Colin Kaepernick or the national anthem, and Barstool
has cleared a space for them to gather and talk, mostly, about just how
much they don't want to talk about politics. They claim to be an
overlooked majority --- the vast market inefficiency that will richly
reward anyone who will let them watch their games, memes and funny
videos without having to feel bad about themselves. Barstool is their
safe space.

So on a balmy day in October, Portnoy --- who looks like Mark Zuckerberg
after five years of hard drinking and even harder tanning --- called an
``emergency press conference'' in his offices to address the
cancellation of ``Van Talk.'' Portnoy's addresses to readers tend to
ramble and veer off on tangents, but they do so triumphantly. This one
was no different: He sidled up to a makeshift lectern made out of a
water jug, stared straight into a camera and delivered an unapologetic
seven-minute rant. ``We're not going to let Mickey Mouse push us
around,'' he said, referring to Disney, which owns ESPN. ``There is
actually nothing that ESPN could have done to illustrate'' --- he meant
\emph{better} illustrate --- ``why we are rising and they are falling.''

\textbf{El Presidente,} who grew up in Swampscott, Mass., an
upper-middle-class suburb, isn't the most obvious choice for a champion
of the common man, but he does describe his own life as thoroughly,
unceasingly average. He says he did O.K. in school and was an O.K.
baseball player. He considers himself an average-looking guy. He
attended the University of Michigan, where, he says, he had average fun
with his average friends. His rise to prominence came not from his
skills as a sports analyst but from his ability to sniff out untapped
markets. After graduating from Michigan in 1999, he worked in the sales
department of a consulting firm in Boston but quickly tired of corporate
life. He wanted to start his own business, preferably in the gambling
scene. In 2003, on a trip to Las Vegas, he met with people in the
online-gaming industry and found them desperate for places to advertise.
Portnoy's idea was to start a sports publication that might be
attractive to poker advertisers --- and because the only ad model that
seemed viable at the time was in print, he planned to pass it out as a
tabloid at T stops throughout Boston.

Early in the decade, the only sports blogs with any significant audience
were Sports by Brooks --- which mostly aggregated news --- and the
writer Bill Simmons's column on AOL's Digital City Boston. (I later
worked for Simmons for three years at Grantland, the ESPN website where
he was editor in chief.) Portnoy revered Simmons and agreed with his
assessment that Boston's sports coverage, which was still centered in
the column inches of The Globe and The Herald, had grown stale and out
of touch with the common man. Not so for Portnoy: In an early mock-up,
calling himself Devilfish Dave, he wrote that ``the people at Barstool
Sports are a bunch of average Joes, who like most guys love sports,
gambling, golfing and chasing short skirts.''

Barstool went through the usual spate of early hardships, and there's
every chance that had it been born out of some actual journalistic
ideal, it would have folded within the year. But Portnoy's talent was
for gathering feedback from readers and advertisers and tweaking his
product accordingly: ``We could pivot really easily,'' he told me, ``and
chase money.'' The first breakthrough came when a local photographer
told Portnoy he should start putting photos of area women on the
tabloid's covers --- and offered to take the pictures himself. (A
version of this idea still exists on the website, under the title
``Local Smokeshow of the Day.'') Around the same time, Portnoy noticed
that readers seemed to respond more to stories about drinking, women and
gambling than day-to-day sports news. He sold ads to bars and breweries
and catered more and more to a certain archetypal Boston bro --- the
type who puts on a collared shirt to get blackout drunk every weekend
while ruefully cheering on the Red Sox. His writing voice fell into a
distinct rhythm, half-cocked and prone to fits of anger. When he finally
mustered up a web version of Barstool, it looked like a relic from the
1990s and often crashed, but he called such inconveniences ``the
Barstool difference'' --- a sign, he maintained, of true authenticity.

By the time Barstool began publishing, Simmons had started a national
column for ESPN and moved to Los Angeles to write for television.
Portnoy had Boston to himself. ``When Bill was writing on Digital
Cities, he was reaching regular guys like me,'' Portnoy told me. ``I'm
reaching the new me.''

Whether he knew it or not, Portnoy was also building a modern
online-media business well before its time --- with low overhead, an
investment in brand loyalty and diversified revenue streams that could
withstand fluctuations in advertising. He started hawking T-shirts and
merchandise on the site, another venture that fell prey to the Barstool
difference; printing and shipping could take months. He built up a
network of bloggers in other cities, like Dan Katz in Chicago and Kevin
Clancy in New York --- Big Cat and KFC, per their Barstool nicknames.
When Portnoy realized that readers were more invested in these bloggers
as personalities than in their opinions on sports, he began turning the
site into a sort of online reality show: Every office argument and
personal-life development was written up and fed to a growing legion of
``Stoolies.''

Image

Trent Ryan (left, with headphones) and Adam Ferrone (red
hat).Credit...Dina Litovsky/Redux, for The New York Times

Part of what Barstool offered these readers was escapism, something that
retains a lot of power among sports fans who still see games as a
nightly release from their responsibilities. The site's enduring slogan,
``Saturdays Are for the Boys,'' promises a day free from girlfriends and
wives. Search for the phrase on social media, and you'll find videos of
Stoolies relaxing at beach houses, on boats or at tailgates, surrounded
by nothing but shirtless men; in some, they actually push women out of
the camera's frame. Sports could also be a reprieve from office work. In
his initial mock-up, in 2003, Portnoy wrote that ``we don't take
ourselves very seriously and view working at Barstool Sports as a way to
avoid becoming slaves to cubicle life.'' When Clancy, who calls himself
``the king of average,'' started writing Barstool New York, he was
working as an accountant at Deloitte; when he saw that his tales of
mind-numbing corporate boredom were getting traction with readers, he
began writing a column called ``Cubicle Chronicles,'' grumbling
rancorously about everything from bad coffee to the ``fat secretary
blasting Dominic the {[}expletive{]} Donkey'' around Christmas.

The only thing the Stoolies wouldn't do, it turned out, was
``politics.'' For the most part, Portnoy and his readers employed the
time-honored bro tactic of saying they had no problem with anyone ---
until, of course, ``anyone'' started complaining. A particularly
illustrative example of this can be found in an article from 2009, in
which a reader informed Portnoy about the ``Fagbug,'' an art
installation aimed at raising awareness about homophobic violence. ``I
could give a {[}expletive{]} less if somebody is gay or not,'' Portnoy
responded, insisting that, much as he enjoyed the female anatomy, if
another man preferred the male one, then ``more power to him.'' But
what, he asked, was the point of the installation? ``I thought gay dudes
hated being called fags? Or is this like when a black person uses the
N-word as a compliment?'' He closed by saying all this ``fag talk''
reminded him of last night's television: Did anyone else see Adam
Lambert on ``American Idol''? Every line was aimed directly at dudes
who, like Portnoy, would not identify as bigots, but who also scratched
their heads at the weird tendencies of anyone who wasn't exactly like
them, self-proclaimed common men.

By 2010, Barstool was doing well enough that Portnoy had an office in
Milton, Mass., local pages for New York, Chicago and Philadelphia and a
handful of employees, including the future YouTube megastar Jenna
Mourey, a.k.a. Jenna Marbles. That year, he hired a local white rapper
named Sammy Adams and set up a tour of New England colleges. ``When we
showed up on the campuses, they had our signs on their dorms, people
were rushing after our bus,'' he told me. ``That was the first time I
really thought this might be bigger than I anticipated.''

The following year, he started a nationwide party tour called ``Barstool
Blackout.'' The college students who attended danced under blacklights
and occasionally --- the obvious implication --- drank until they
blacked out. (One of the slogans: ``By the C- student, for the C-
student.'') It was Barstool's first real encounter with controversy.
Early in 2012, students in the Boston area demonstrated against the
Blackout Parties, claiming that they promoted rape culture and
circulating Portnoy's writings on the subject. ``Just to make friends
with the feminists,'' he'd written on the site, ``I'd like to reiterate
that we don't condone rape of any kind at our Blackout Parties in
mid-January. However, if a chick passes out, that's a gray area
though.'' And: ``Though I never condone rape, if you're a Size 6 and
you're wearing skinny jeans, you kind of deserve to be raped, right?''
The parties, which were held at private spaces near campuses, went ahead
as planned. Portnoy issued no retractions or apologies. ``I think the
controversy probably helped us,'' he says now. ``Our fans liked that we
didn't back down. They realized that I was on their side.''

\textbf{In January 2016,} Portnoy stood in Times Square, dressed in a
tuxedo and flanked by Clancy, Katz and a Barstool editor, Keith
Markovich. After a few bars of Frank Sinatra's ``New York, New York,''
he made what he called a ``shocking'' announcement: ``I am no longer the
majority owner of Barstool Sports. We have taken investment from an
investment company called Chernin Digital.'' He went on to describe
Peter Chernin --- head of the Chernin Group, the former president of
News Corporation and the Fox film executive who greenlit ``Titanic'' and
``Avatar'' --- as a ``big swinging {[}expletive{]} at the cracker
factory,'' and alluded to Barstool's business and technological
shortcomings, all of which would presumably be fixed soon. ``When you're
a young comedian in the '80s,'' he said, ``and you graduate, right, you
had to send your résumé to `S.N.L.' Five years, all these little kids,
all these beautiful people --- there's only going to be one place to
send their résumé: Barstool Sports.''

In a blog post about the sale --- which concluded with the coy signoff
``PS --- I'm kinda rich now'' --- Portnoy added something prescient:
``Chernin knows about the Size 6 skinny-jean joke. They know about
Babygate.'' (The ``Babygate'' controversy stemmed from Portnoy's
speculating about the size of Tom Brady's baby's penis.) ``They know
about Al Jazeera.'' (In this one, Clancy questioned the legitimacy of
any news outlet with an Arabic-sounding name.) ``They get it.''

The Barstool acquisition was engineered by the president of Chernin
Digital, Mike Kerns. ``When I got access to Barstool's Google analytics,
I knew this was something different,'' he told me. ``They had something
like 20 percent of their visitors coming back about 20 times a day. I've
been in this business for two decades, and all their numbers bucked the
usual trends.'' Portnoy kept full editorial control; Chernin's bet was
that it could serve cheap content to his loyal fan base, which would
then pay for things like T-shirts, events and premium content. The brand
would be scaled up into something that could be sold to advertisers, big
media partners and even sports leagues. Every Barstool executive I spoke
to mentioned the possibility of opening branded sports bars across the
country; all of them talked about partnerships with networks. Since its
acquisition, Barstool has released a raft of popular podcasts ---
including ``Pardon My Take,'' which, with downloads running up to one
million per episode, is one of the biggest sports podcasts in the
country. It has partnered with Facebook on a roving pregame
college-football show (since canceled) and produced a widely watched
baseball show that regularly features former major-league players.

This bullish transition has been helmed by Erika Nardini, a 42-year-old
former marketing executive who once served as the chief marketing
officer for AOL. Nardini, who grew up playing sports with her younger
brother, seems uniquely qualified to deal with both the business of
turning Barstool into a national brand and the inevitable public- and
human-relations disasters that will arise along the way. She is also a
woman, and despite its growth since the Chernin acquisition, Barstool
still has to work around how bad its worst moments can get --- from
Portnoy's rape jokes to posts like the one a blogger named Chris
Spagnuolo wrote this summer: ``Is Rihanna Going to Make Being Fat the
Hot New Trend?''

Image

Erika Nardini.Credit...Dina Litovsky/Redux, for The New York Times

The Rihanna incident highlighted how much has changed since the Chernin
acquisition, but also how much has stayed more or less the same. The
post was quietly taken down. But Portnoy also opined on the site that he
thought the post wasn't ``as bad as many are making it out to be,'' and
that he was angry mostly because Spagnuolo had given ``feminists''
fodder to say ``there goes Barstool being Barstool again.'' And yet
Portnoy himself cannot seem to stop personally offering up more and more
of that fodder. A controversy last month, involving the terms of a
contract offered to a sports personality named Elika Sadeghi, began on
relatively professional footing. Within days, though, Barstool had
released a seven-minute animated video in which a cartoon Portnoy says
Sadeghi's surname sounds like ``the monkey from `The Lion King.' '' It
also portrays her hanging upside down over a boiling caldron.

I spoke to several women in sports media who have had run-ins with
Barstool. All described the same pattern: They would tweet something
critical of Barstool's statements about women, which would prompt a
response from Portnoy or one of his bloggers. Then came the swarm of
Stoolies on social media, who would harass them with misogynist slurs
and threats, often for days. Even random sports fans have been targeted.
A few years ago, a Cubs fan named Missy suffered a brain injury after a
fall; during her recovery, she found that she had trouble reading
anything longer than a paragraph, so she moved her usual sports-media
consumption over to Twitter. When she saw an article detailing the way
Clancy and an army of Stoolies had responded to the Al Jazeera incident,
Missy tweeted her support for the author of the article and women she
felt had been abused online. Stoolies responded almost immediately, with
three days of the usual misogynist epithets and vague threats. A year
later, she says, after another comment critical of Barstool, a reader
found photos she had posted memorializing a cousin who died of cancer
--- and reposted them on Twitter, tagging Barstool writers to do God
knows what with them.

There's a uniform response from Barstool employees about the worst of
the Stoolies. ``I hate seeing it,'' Katz told me. ``But it's just a few
idiots who have nothing better to do, and it sucks that people use them
to smear an entire company.'' The average Stoolie, Portnoy, Clancy and
Markovich all argue, is not a misogynist abuser but has been painted
with a broad brush by other media outlets. ``I'm used to it by now,''
Nardini said of the constant negative press surrounding Barstool's
attitudes toward women. ``Every time anyone mentions us in the media,
they're always going to write that requisite paragraph.'' She used to be
part of a network of female executives, she told me, but ``after they
heard I was coming here, every single one of them dropped me like a bad
habit.''

The wrath of the Stoolies can occasionally extend to Barstool's own
employees. ``There wasn't a single day that would go by without me
seeing the N-word in the comments,'' Maurice Peebles, Barstool's first
black employee, told me. Peebles ran Barstool's Philadelphia page for
three years. His administrator access allowed him to see that the racial
slurs were coming mostly from a concentrated number of IP addresses,
which meant that only a few readers were posting the slurs, and over
time the site's filters became better at blocking them. But he doesn't
absolve Barstool of all responsibility. ``They could've done more about
it,'' he says. ``None of the guys who worked at Barstool ever said
anything racist to me, but I don't know if they all understand what it's
like to see that word every day.''

Barstool's reputation ``was certainly listed as a risk,'' Kerns says.
``But I think time is on our side. The younger folks within agencies and
brands get Barstool and recognize the world is increasingly taking
itself less seriously.'' Over the past year, that time seemed to have
already arrived. Dunkin' Donuts, the advertiser most associated with
Boston sports, had long been wary of dealing with Portnoy, but this
year, Barstool dedicated an entire month to promoting the chain's new
energy drink. Wendy's had also expressed hesitation to partner with
Barstool, but this summer the company sponsored ``Barstool 5th Year,'' a
Snapchat channel specifically targeted at college students.

The question of whether Barstool should be held responsible for the
worst behaviors of its fans reflects a fundamental question facing
online media --- the same one at the core of Facebook's issues with fake
news, Twitter's with neo-Nazis and Reddit's with various toxic
communities. Unlike those companies, Barstool can't hide behind the
notion of being an open, neutral platform for the free speech of others.
Its readers may come from all sorts of backgrounds, but the core
Stoolies are an organic online community that grew under the caring,
thoughtful hand of their very own El Presidente. Every new-media venture
seeks out an ``organic online community'' like this --- one that can, in
Nardini's words, ``convert content into commerce.'' That community could
mean, say, subscribers of The Daily Skimm, an email for millennial women
that recaps the news in a peppy, corporate voice. But it can also mean
tribes of angry, disaffected young men who gather online to find shelter
from the floodwaters of political correctness. This leaves companies
like ESPN with a discomforting dilemma. Should they try to create their
own communities --- an almost impossible enterprise, especially with
young audiences who have grown up on completely independent, unfiltered
personalities on YouTube and social media? Or should they co-opt,
sanitize and scale audiences like the Stoolies?

\textbf{There are two} distinct visions of how Barstool could work at
the scale Nardini and Chernin envision. The first would involve running
back into the understanding embrace of the Stoolies and building an
uncouth, unapologetic brand aimed exclusively at boorish young men. Last
August, Barstool purchased Old Row, a site that posts frat-boy fight
videos and photos of college girls in bikinis and sells T-shirts
celebrating the Trump presidency. This month, Barstool announced that it
had bought Rough N Rowdy Brawl, an amateur boxing company from West
Virginia that features untrained locals knocking one another out. In an
``emergency press conference'' announcing the acquisition, Portnoy
thanked Ponder, saying the ESPN controversy had led to ``the biggest
couple weeks we've ever had.'' ``It does not matter if you like us, hate
us, whatever,'' he said. ``We speak directly to our own consumers.''

The other road is to take the popular material that has been built since
the Chernin acquisition and take another crack at entering media's
mainstream. I have friends and relatives --- the majority of whom would
be considered progressive, many of whom are not white --- who read
Barstool regularly, like its videos on Instagram and listen to ``Pardon
My Take.'' Some are vaguely aware that Portnoy has said disturbing
things about women, but they shrug it off in the same way they shrug off
the cloud of bad news that continually engulfs the N.F.L. The vast
majority of the Barstool content they consume ticks between
standard-fare aggregation (funny videos, memes, weird stories from
Florida) and genuinely enjoyable content aimed directly at men who, like
me, grew up watching sports and went to colleges where we watched sports
with our sports-watching friends.

During the N.H.L. finals in June, I went to the Barstool offices to
watch a recording of ``Pardon My Take.'' Katz, who is not as fat as he
claims to be on air, sat in a La-Z-Boy, idly watching hockey and
scribbling notes. His co-host, who goes by the pseudonym PFT Commenter,
tried out jokes about handshake lines and the superiority of the N.H.L.
to the N.B.A. When the game ended, they settled on a list of segments
and piled into a small recording studio, completely bare except for
poorly stapled acoustic tiles and posters of Chris Berman and Lenny
Dykstra.

Image

Dan Katz.Credit...Dina Litovsky/Redux, for The New York Times

They rattled through the show without second takes or pauses, the jokes
and banter falling into a familiar, rapid rhythm. Katz is the straight
man; he is mostly playing himself, an affable dude who loves his Chicago
sports and could easily slide into the chair of any ESPN opinion show.
PFT Commenter, who has shoulder-length hair and wears Hawaiian shirts,
has created a type of character that has never really been seen in
sports media --- a gag version of a commenter on the well-trafficked
N.F.L. blog Pro Football Talk, his tweets and columns filled with the
spelling errors, prejudices and leaps of logic that plague all open
forums about sports. In 2015, covering the early part of the
presidential campaign, in character, for SB Nation, he would pound
airplane bottles of Fireball whiskey and walk straight into scrums of
reporters; outside a Republican debate in Cleveland, he held up a sign
behind Chris Matthews that read, ``Is Joe Flacco a ELITE Quaterback?''

Almost everything about ``Pardon My Take'' is a densely referential
sports-fan in-joke. Even the title plays off two ESPN talk shows,
``Pardon the Interruption'' and ``First Take.'' If you've never watched
Chris Berman run through a highlight reel or admired the sports-yelling
talents of Stephen A. Smith, Katz and Commenter might as well be
speaking a foreign language. But most sports fans have watched hundreds
of hours of ESPN programming, absorbing all the tics, clichés and motifs
that Katz and Commenter have quilted together into a pitch-perfect
satirical pidgin. One of its catchiest elements derives from the
N.F.L.-coach habit of explaining some bit of masculine bluster by saying
``I'm a football guy'' --- in my time at Barstool, at least 70 percent
of conversations seemed to include some deadpan variation on ``I'm a
huge {[}something{]} guy.'' This entertaining mix has attracted an
impressive list of high-profile athletes and media figures to the show.
Before the Chernin deal, Portnoy had not seen the value in producing
podcasts, which he admitted to me was a ``big mistake''; now ``Pardon My
Take'' is Barstool's flagship product. Katz and Commenter were, until
October, proof that Barstool could be scrubbed clean and scaled up.

During our conversations, Portnoy kept bringing up ``Saturday Night
Live,'' mentioning it as a model for Barstool. What he meant was that he
wanted to create a broad cast of characters, each capable of his or her
own independent success. Barstool has hired the ESPN sideline reporter
Julie Stewart-Binks; Michael Rapaport, the character actor who hosts a
popular podcast; the former major-league pitcher Dallas Braden; Pat
McAfee, an N.F.L. punter who retired midcareer to sign with Portnoy; and
Adam Ferrone, a battle-rap champion who told me that he pestered
Barstool for two years for a job. By choosing Barstool, each seems to be
signing on with the gospel of Portnoy: Say what's on your mind --- and
if anyone has a problem with it, fight back.

\textbf{The morning after} Skipper sent out the ESPN memo canceling
``Van Talk,'' I met Portnoy in his hoarder's den of an office, where Tom
Brady memorabilia and stacks of paper occupied every surface. He seemed
unusually subdued. His nearly manic manner had dissipated into a fog of
half-formed sound bites and what felt like sincere anxiety. He expressed
regret over hurting Katz and Commenter's television prospects, denounced
ESPN's cowardice and called out the hypocrisy of any female journalist
at ESPN who had ever tweeted an edgy joke in the past. ``I used to think
of Barstool like a comedy club,'' he said. ``Just me talking to my guys.
But things have definitely changed.'' Then, after a pause, he seemed to
have a change of heart: ``ESPN thought they were going to get Barstool
without Barstool. How does that even work?''

Over the days to come, Portnoy and a handful of bloggers --- alongside
hundreds of Stoolie volunteers --- scrubbed through the social-media
accounts of women at ESPN who had spoken out about them, resurfacing
every comment that was even slightly off-color. (``I hate hypocrisy,''
Portnoy told me --- and that image, of ugly honesty triumphing over
hypocrisy, probably explains Barstool's appeal to young men better than
any of its content.) A week later, Henry Lockwood, the producer of
``Pardon My Take,'' tweeted that Britt McHenry had ``cankles,'' leading
to another spat. It was as if Barstool was doubling down on being more
Barstool than ever, even though ESPN wasn't the only partner that had
been scared off: Portnoy told me another network had backed away from a
deal, and that some advertisers had expressed concern.

We talked about something that happened a few hours before ESPN's
announcement, while Portnoy was recording one of his daily pizza reviews
--- a Barstool programming staple in which Portnoy tries to review every
pizza joint in Manhattan. That day, his guest was Jake Paul, the
20-year-old YouTube heel who might be the only person on the internet
better than Portnoy at turning hate and controversy into merchandising
opportunities. ``People know I'm a Jake Paul guy,'' Portnoy said. ``I
respect people who take over the internet, and this guy has got maybe
more haters than I do, which I also love.'' He ventured that ``if you
put Team 10'' --- the name of Paul's company --- ``with the Stoolies, I
think we can bring down, like, the entire country.''

Image

Glen Medoro (left) and Maria Ciuffo.Credit...Dina Litovsky/Redux, for
The New York Times

He was joking, but if companies like ESPN want to corral the millions of
young people who have cut cable cords, turned off ``SportsCenter'' and
flocked to unfiltered and anarchic internet personalities, they will
have to reckon with Jake Pauls and Dave Portnoys. The truth about ESPN's
supposed bias will not really matter: A lot of people, like Ives the
intern, believe that the Worldwide Leader in Sports no longer speaks to
them. Their grievances, like those of the angry men who fume over the
female cast of ``Ghostbusters'' or ethics in video-game journalism, will
seem absurdly petty, whether they're complaining about the rare yet
somehow oppressive sight of a female sportscaster or the unbearable
burden placed upon their consciences by a two-minute conversation about
Colin Kaepernick. But they will voice these grievances online with
enough volume and vitriol to worry even the most reasonable media
executive. And if that executive doesn't bend to their will, they will
seek out someone, anyone, who feels more authentic to their experience,
whatever that may mean. For huge media conglomerates, this dynamic might
matter only in the margins; ESPN surely has more immediate business
concerns. But gains in media right now occur only in the margins. The
market inefficiencies will not be ignored.

In his office, I asked Portnoy why he thought ESPN had been interested
in partnering with Barstool in the first place, given its past. A
half-smile crept over his face. ``You know, it's like that Batman
quote,'' he said. ``In a time of desperation, you turn to a man you
don't fully understand.''

Advertisement

\protect\hyperlink{after-bottom}{Continue reading the main story}

\hypertarget{site-index}{%
\subsection{Site Index}\label{site-index}}

\hypertarget{site-information-navigation}{%
\subsection{Site Information
Navigation}\label{site-information-navigation}}

\begin{itemize}
\tightlist
\item
  \href{https://help.nytimes3xbfgragh.onion/hc/en-us/articles/115014792127-Copyright-notice}{©~2020~The
  New York Times Company}
\end{itemize}

\begin{itemize}
\tightlist
\item
  \href{https://www.nytco.com/}{NYTCo}
\item
  \href{https://help.nytimes3xbfgragh.onion/hc/en-us/articles/115015385887-Contact-Us}{Contact
  Us}
\item
  \href{https://www.nytco.com/careers/}{Work with us}
\item
  \href{https://nytmediakit.com/}{Advertise}
\item
  \href{http://www.tbrandstudio.com/}{T Brand Studio}
\item
  \href{https://www.nytimes3xbfgragh.onion/privacy/cookie-policy\#how-do-i-manage-trackers}{Your
  Ad Choices}
\item
  \href{https://www.nytimes3xbfgragh.onion/privacy}{Privacy}
\item
  \href{https://help.nytimes3xbfgragh.onion/hc/en-us/articles/115014893428-Terms-of-service}{Terms
  of Service}
\item
  \href{https://help.nytimes3xbfgragh.onion/hc/en-us/articles/115014893968-Terms-of-sale}{Terms
  of Sale}
\item
  \href{https://spiderbites.nytimes3xbfgragh.onion}{Site Map}
\item
  \href{https://help.nytimes3xbfgragh.onion/hc/en-us}{Help}
\item
  \href{https://www.nytimes3xbfgragh.onion/subscription?campaignId=37WXW}{Subscriptions}
\end{itemize}
