Sections

SEARCH

\protect\hyperlink{site-content}{Skip to
content}\protect\hyperlink{site-index}{Skip to site index}

\href{https://www.nytimes3xbfgragh.onion/section/obituaries}{Obituaries}

\href{https://myaccount.nytimes3xbfgragh.onion/auth/login?response_type=cookie\&client_id=vi}{}

\href{https://www.nytimes3xbfgragh.onion/section/todayspaper}{Today's
Paper}

\href{/section/obituaries}{Obituaries}\textbar{}Charles Manson Dies at
83; Wild-Eyed Leader of a Murderous Crew

\url{https://nyti.ms/2jHBzuV}

\begin{itemize}
\item
\item
\item
\item
\item
\item
\end{itemize}

Advertisement

\protect\hyperlink{after-top}{Continue reading the main story}

Supported by

\protect\hyperlink{after-sponsor}{Continue reading the main story}

\hypertarget{charles-manson-dies-at-83-wild-eyed-leader-of-a-murderous-crew}{%
\section{Charles Manson Dies at 83; Wild-Eyed Leader of a Murderous
Crew}\label{charles-manson-dies-at-83-wild-eyed-leader-of-a-murderous-crew}}

\includegraphics{https://static01.graylady3jvrrxbe.onion/images/2017/11/21/us/charles-manson-obituary-slide-S1LM/charles-manson-obituary-slide-S1LM-videoSixteenByNineJumbo1600-v4.jpg}

By \href{http://www.nytimes3xbfgragh.onion/by/margalit-fox}{Margalit
Fox}

\begin{itemize}
\item
  Nov. 20, 2017
\item
  \begin{itemize}
  \item
  \item
  \item
  \item
  \item
  \item
  \end{itemize}
\end{itemize}

Charles Manson, one of the most notorious murderers of the 20th century,
who was very likely the most culturally persistent and perhaps also the
most inscrutable, died on Sunday in a hospital in Kern County, Calif.,
north of Los Angeles. He was 83 and had been behind bars for most of his
life.

The California Department of Corrections and Rehabilitation announced
his death in a news release. In accordance with federal and state
privacy regulations, no cause was given; he had been hospitalized in
January for intestinal bleeding but was ruled too frail to undergo
surgery.

Mr. Manson was a semiliterate habitual criminal and failed musician
before he came to irrevocable attention in the late 1960s as the
wild-eyed leader of the Manson family, a murderous band of young
drifters in California. Convicted of nine murders in all, he was known
in particular for the seven brutal killings collectively called the
Tate-LaBianca murders, committed by his followers on two consecutive
August nights in 1969.

The most famous of the victims was Sharon Tate, an actress who was
married to the film director Roman Polanski. Eight and a half months
pregnant,
\href{http://www.nytimes3xbfgragh.onion/1970/08/22/archives/coroner-details-the-tate-killing-says-actress-was-stabbed-16-times.html}{she
was killed with four other people}at her home in the Benedict Canyon
area of Los Angeles, near Beverly Hills.

The Tate-LaBianca killings and the seven-month trial that followed were
the subjects of fevered news coverage. To a frightened, mesmerized
public, the murders, with their undercurrents of sex, drugs, rock 'n'
roll and Satanism, seemed the depraved logical extension of the
anti-establishment, do-your-own-thing ethos that helped define the '60s.

Since then, the Manson family has occupied
\href{https://www.nytimes3xbfgragh.onion/2017/11/20/arts/charles-manson-pop-culture.html}{a
dark, persistent place in American culture} --- and American commerce.
It has inspired, among other things, pop songs, an opera, films, a host
of internet fan sites, T-shirts, children's wear and half the stage name
of the rock musician Marilyn Manson.

It has also been the subject of many nonfiction books, most famously
``Helter Skelter'' (1974), by
\href{https://www.nytimes3xbfgragh.onion/2015/06/10/us/vincent-t-bugliosi-manson-prosecutor-and-true-crime-author-dies-at-80.html}{Vincent
Bugliosi} and Curt Gentry. Mr. Bugliosi was the lead prosecutor at the
Tate-LaBianca trial.

\includegraphics{https://static01.graylady3jvrrxbe.onion/images/2017/01/07/us/charles-manson-obituary-slide-2CTZ/charles-manson-obituary-slide-2CTZ-articleLarge.jpg?quality=75\&auto=webp\&disable=upscale}

The Manson family came to renewed attention in 2008, when officials in
California, responding to long speculation that there were victims still
unaccounted for, searched a stretch of desert in Death Valley. There, in
a derelict place known as the Barker Ranch, Mr. Manson and his followers
had lived for a time in the late '60s. The search turned up no human
remains.

It was a measure of Mr. Manson's hold over his followers,
\href{https://www.nytimes3xbfgragh.onion/2017/11/20/us/what-ever-happened-to-the-manson-family.html}{mostly
young women}who had fled middle-class homes, that he was not physically
present at the precise moment that any one of the Tate-LaBianca victims
was killed. Yet his family swiftly murdered them on his orders, which,
according to many later accounts, were meant to incite an apocalyptic
race war that Mr. Manson called Helter Skelter. He took the name from
the title of a Beatles song.

Throughout the decades since, Mr. Manson has remained an enigma. Was he
a paranoid schizophrenic, as some observers have suggested? Was he a
sociopath, devoid of human feeling? Was he a charismatic guru, as his
followers once believed and his fans seemingly still do?

Or was he simply flotsam, a man whose life,
\href{http://www.nytimes3xbfgragh.onion/1970/01/04/archives/charlie-manson-one-mans-family-charlie-manson-one-mans-family.html}{The
New York Times wrote in 1970}, ``stands as a monument to parental
neglect and the failure of the public correctional system''?

Image

The body of Ms. Tate being removed from the rented home where she was
murdered.Credit...Associated Press

No Name Maddox, as Mr. Manson was officially first known, was born on
Nov. 12, 1934, to a 16-year-old unwed mother in Cincinnati. (Many
accounts give the date erroneously as Nov. 11.) His mother, Kathleen
Maddox, was often described as having been a prostitute. What is
certain, according to Mr. Bugliosi's book and other accounts, is that
she was a heavy drinker who lived on the margins of society with a
series of men.

Mr. Manson apparently never knew his biological father. His mother
briefly married another man, William Manson, and gave her young son the
name Charles Milles Manson.

Kathleen often disappeared for long periods --- when Charles was 5, for
instance, she was sent to prison for robbing a gas station --- leaving
him to bounce among relatives in Ohio, West Virginia and Kentucky. She
was paroled when Charles was 8 and took him back, but kept him for only
a few years.

\hypertarget{burglary-robbery-rape}{%
\subsection{Burglary, Robbery, Rape}\label{burglary-robbery-rape}}

From the age of 12 on, Charles was placed in a string of reform schools.
At one institution, he held a razor to a boy's throat and raped him.

Image

The Spahn Movie Ranch.Credit...Ralph Crane/The LIFE Picture Collection,
via Getty Images

Escaping often, he committed burglaries, auto thefts and armed
robberies, landing in between in juvenile detention centers and
eventually federal reformatories. He was paroled from the last one at
19, in May 1954.

Starting in the mid-1950s, Mr. Manson, living mostly in Southern
California, was variously a busboy, parking-lot attendant, car thief,
check forger and pimp. During this period, he was in and out of prison.

He was married twice: in 1955 to Rosalie Jean Willis, a teenage
waitress, and a few years later to a young prostitute named Leona. Both
marriages ended in divorce.

Mr. Manson was believed to have fathered at least two children over the
years: at least one with one of his wives, and at least one more with
one of his followers. The precise number, names and whereabouts of his
children --- a subject around which rumor and urban legend have long
coalesced --- could not be confirmed.

Image

Mr. Manson in court with Susan Atkins, seated, in October
1970.Credit...Associated Press

By March 1967, when Mr. Manson, then 32, was paroled from his latest
prison stay, he had spent more than half his life in correctional
facilities. On his release, he moved to the Bay Area and eventually
settled in the Haight-Ashbury district of San Francisco, the nerve
center of hippiedom, just in time for the Summer of Love.

There, espousing a philosophy that was an idiosyncratic mix of
Scientology, hippie anti-authoritarianism, Beatles lyrics, the Book of
Revelation and the writings of Hitler, he began to draw into his orbit
the rootless young adherents who would become known as the Manson
family.

Mr. Manson had learned to play the guitar in prison and hoped to make it
as a singer-songwriter. His voice was once compared to that of the young
Frankie Laine, a crooner who first became popular in the 1940s.

Mr. Manson's lyrics, by contrast, were often about sex and death, but in
the '60s, that did not stand out very much. (Songs he wrote were later
recorded by Guns N' Roses and Marilyn Manson.) Once he was famous, Mr.
Manson himself released several albums, including ``LIE,'' issued in
1970, and ``Live at San Quentin,'' issued in 2006.

Image

Mr. Manson in 1980. He learned to play the guitar in prison and hoped to
make it as a singer-songwriter.Credit...Mirrorpix, via Alamy

With his followers --- a loose, shifting band of a dozen or more --- Mr.
Manson left San Francisco for Los Angeles. They stayed awhile in the
home of Dennis Wilson, the Beach Boys' drummer. Mr. Manson hoped the
association would help him land a recording contract, but none
materialized. (The Beach Boys did later record a song, ``Never Learn Not
to Love,'' that was based on one written by Mr. Manson, although Mr.
Wilson, who sang it, gave it new lyrics and a new title --- Mr. Manson
had called it ``Cease to Exist'' --- and took credit for writing it.)

The Manson family next moved to the Spahn Movie Ranch, a mock Old West
town north of Los Angeles that was once a film set but had since fallen
to ruins. The group later moved to Death Valley, eventually settling at
the Barker Ranch.

The desert location would protect the family, Mr. Manson apparently
thought, in the clash of the races that he believed was inevitable. He
openly professed his hatred of black people, and he believed that when
Helter Skelter came, blacks would annihilate whites. Then, unable to
govern themselves, the blacks would turn for leadership to the Manson
family, who would have ridden out the conflict in deep underground holes
in the desert.

\hypertarget{a-frenzy-of-bloodshed}{%
\subsection{A Frenzy of Bloodshed}\label{a-frenzy-of-bloodshed}}

At some point, Mr. Manson seems to have decided to help Helter Skelter
along. Late at night on Aug. 8, 1969, he dispatched four family members
--- Susan Atkins, Patricia Krenwinkel, Charles Watson and Linda Kasabian
--- to the Tate home in the Hollywood hills. Mr. Manson knew the house:
Terry Melcher, a well-known record producer with whom he had dealt
fruitlessly, had once lived there.

Image

On Jan. 25, 1971, the jury found Mr. Manson, Patricia Krenwinkel, left,
and Susan Atkins, center, guilty of seven counts of murder each. Leslie
Van Houten, right, was found guilty of two counts.Credit...Associated
Press

Shortly after midnight on Aug. 9, Ms. Atkins, Ms. Krenwinkel and Mr.
Watson entered the house while Ms. Kasabian waited outside. Through a
frenzied combination of shooting, stabbing, beating and hanging, they
murdered Ms. Tate and four others in the house and on the grounds: Jay
Sebring, a Hollywood hairdresser; Abigail Folger, an heiress to the
Folger coffee fortune; Voytek (also spelled Wojciech) Frykowski, Ms.
Folger's boyfriend; and Steven Parent, an 18-year-old visitor. Ms.
Tate's husband, Mr. Polanski, was in London at the time.

Before leaving, Ms. Atkins scrawled the word ``pig'' in blood on the
front door of the house; in Mr. Manson's peculiar logic, the killings
were supposed to look like the work of black militants.

The next night, Aug. 10, Mr. Manson and a half-dozen followers drove to
a Los Angeles house he appeared to have selected at random. Inside, Mr.
Manson tied up the residents --- a wealthy grocer named Leno LaBianca
and his wife, Rosemary --- before leaving. After he was gone, several
family members stabbed the couple to death. The phrases ``Death to
Pigs'' and ``Healter Skelter,'' misspelled, were scrawled in blood at
the scene.

The seven murders went unsolved for months. Then, in the autumn of 1969,
the police closed in on the Manson family after Ms. Atkins, in jail on
an unrelated murder charge, bragged to cellmates about the killings.

Image

Charles Manson being taken to jail months after the brutal killings of
seven people in Los Angeles in 1969.Credit...Bettmann

On June 15, 1970, Mr. Manson, Ms. Atkins, Ms. Krenwinkel and a fourth
family member, Leslie Van Houten, went on trial for murder. Ms.
Kasabian, who had been present on both nights but said she had not
participated in the killings, became the prosecution's star witness and
was given immunity. Mr. Watson, who had fled to Texas, was tried and
convicted separately.

During the trial, the bizarre became routine. On one occasion, Mr.
Manson lunged at the judge with a pencil. On another, he punched his
lawyer in open court. At one point, Mr. Manson appeared in court with an
``X'' carved into his forehead; his co-defendants quickly followed suit.
(Mr. Manson later carved the X into a swastika, which remained
flagrantly visible ever after.)

Outside the courthouse, a small flock of chanting family members kept
vigil. One of them, Lynette Fromme, nicknamed Squeaky, would make
headlines herself in 1975 when she tried to assassinate President Gerald
R. Ford.

On Jan. 25, 1971, after nine days' deliberation, the jury found Mr.
Manson, Ms. Atkins and Ms. Krenwinkel guilty of seven counts of murder
each. Ms. Van Houten, who had been present only at the LaBianca murders,
was found guilty of two counts. All four were also convicted of
conspiracy to commit murder.

Image

Mr. Manson at a parole hearing in 2008. He was turned down for parole a
dozen times, most recently in 2012.Credit...Associated Press

On March 29, the jury voted to give all four defendants the death
penalty. In 1972, after capital punishment was temporarily outlawed in
California, their sentences were reduced to life in prison.

Mr. Manson was convicted separately of two other murders: those of Gary
Hinman, a musician killed by Manson family members in late July 1969,
and Donald Shea, a Barker Ranch stuntman killed late that August.
Altogether, Mr. Manson and seven family members were eventually
convicted of one to nine murders apiece.

Incarcerated in a series of prisons over the years, Mr. Manson passed
the time by playing the guitar, doing menial chores and making scorpions
and spiders out of thread from his socks. His notoriety made him a
target: In 1984, he was treated for second- and third-degree burns after
being doused with paint thinner by a fellow inmate and set ablaze.

Mr. Manson was turned down for parole a dozen times, most recently in
2012. Most of the other convicted family members remain in prison.
\href{http://www.nytimes3xbfgragh.onion/2009/09/26/us/26atkins.html}{Ms.
Atkins died in prison} in 2009, at 61, of natural causes.

Image

Mr. Manson in a 2011 California Department of Corrections photo. To the
end of his life, he denied having ordered the Tate-LaBianca murders.
Nor, as he replied to a question he was often asked, did he feel
remorse.Credit...California Department of Corrections, via Agence
France-Presse --- Getty Images

The Manson family was an inspiration for the television series
``Aquarius,'' broadcast on NBC in 2015 and 2016. A period drama set in
the late '60s, it starred David Duchovny as a Los Angeles police
detective who comes up against Mr. Manson (played by the British actor
Gethin Anthony) in the course of investigating a teenage girl's
disappearance.

To the end of his life, Mr. Manson denied having ordered the
Tate-LaBianca murders. Nor, as he replied to a question he was often
asked, did he feel remorse, in any case.

He said as much in 1986 in a prison interview with the television
journalist Charlie Rose.

``So you didn't care?'' Mr. Rose asked, invoking Ms. Tate and her unborn
child.

``Care?'' Mr. Manson replied.

He added, ``What the hell does that mean, `care'?''

Advertisement

\protect\hyperlink{after-bottom}{Continue reading the main story}

\hypertarget{site-index}{%
\subsection{Site Index}\label{site-index}}

\hypertarget{site-information-navigation}{%
\subsection{Site Information
Navigation}\label{site-information-navigation}}

\begin{itemize}
\tightlist
\item
  \href{https://help.nytimes3xbfgragh.onion/hc/en-us/articles/115014792127-Copyright-notice}{©~2020~The
  New York Times Company}
\end{itemize}

\begin{itemize}
\tightlist
\item
  \href{https://www.nytco.com/}{NYTCo}
\item
  \href{https://help.nytimes3xbfgragh.onion/hc/en-us/articles/115015385887-Contact-Us}{Contact
  Us}
\item
  \href{https://www.nytco.com/careers/}{Work with us}
\item
  \href{https://nytmediakit.com/}{Advertise}
\item
  \href{http://www.tbrandstudio.com/}{T Brand Studio}
\item
  \href{https://www.nytimes3xbfgragh.onion/privacy/cookie-policy\#how-do-i-manage-trackers}{Your
  Ad Choices}
\item
  \href{https://www.nytimes3xbfgragh.onion/privacy}{Privacy}
\item
  \href{https://help.nytimes3xbfgragh.onion/hc/en-us/articles/115014893428-Terms-of-service}{Terms
  of Service}
\item
  \href{https://help.nytimes3xbfgragh.onion/hc/en-us/articles/115014893968-Terms-of-sale}{Terms
  of Sale}
\item
  \href{https://spiderbites.nytimes3xbfgragh.onion}{Site Map}
\item
  \href{https://help.nytimes3xbfgragh.onion/hc/en-us}{Help}
\item
  \href{https://www.nytimes3xbfgragh.onion/subscription?campaignId=37WXW}{Subscriptions}
\end{itemize}
