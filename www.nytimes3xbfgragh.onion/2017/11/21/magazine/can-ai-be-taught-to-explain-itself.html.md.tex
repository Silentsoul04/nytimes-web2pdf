Can A.I. Be Taught to Explain Itself?

\url{https://nyti.ms/2hR2weQ}

\begin{itemize}
\item
\item
\item
\item
\item
\item
\end{itemize}

\includegraphics{https://static01.graylady3jvrrxbe.onion/images/2017/11/26/magazine/26explicableai1/26mag-explicableai-image1-articleLarge.jpg?quality=75\&auto=webp\&disable=upscale}

Sections

\protect\hyperlink{site-content}{Skip to
content}\protect\hyperlink{site-index}{Skip to site index}

Feature

\hypertarget{can-ai-be-taught-to-explain-itself}{%
\section{Can A.I. Be Taught to Explain
Itself?}\label{can-ai-be-taught-to-explain-itself}}

As machine learning becomes more powerful, the field's researchers
increasingly find themselves unable to account for what their algorithms
know --- or how they know it.

Credit...Photo illustration by Derek Brahney. Source photo: J.R.
Eyerman/The Life Picture Collection/Getty Images.

Supported by

\protect\hyperlink{after-sponsor}{Continue reading the main story}

By Cliff Kuang

\begin{itemize}
\item
  Nov. 21, 2017
\item
  \begin{itemize}
  \item
  \item
  \item
  \item
  \item
  \item
  \end{itemize}
\end{itemize}

\textbf{In} \textbf{September,} Michal Kosinski published a study that
he feared might end his career. The Economist broke the news first,
giving it a self-consciously anodyne title: ``Advances in A.I. Are Used
to Spot Signs of Sexuality.'' But the headlines quickly grew more
alarmed. By the next day, the Human Rights Campaign and Glaad, formerly
known as the Gay and Lesbian Alliance Against Defamation, had labeled
Kosinski's work ``dangerous'' and ``junk science.'' (They claimed it had
not been peer reviewed, though it had.) In the next week, the tech-news
site The Verge had run an article that, while carefully reported, was
nonetheless topped with a scorching headline: ``The Invention of A.I.
`Gaydar' Could Be the Start of Something Much Worse.''

Kosinski has made a career of warning others about the uses and
potential abuses of data. Four years ago, he was pursuing a Ph.D. in
psychology, hoping to create better tests for signature personality
traits like introversion or openness to change. But he and a
collaborator soon realized that Facebook might render personality tests
superfluous: Instead of asking if someone liked poetry, you could just
see if they ``liked'' Poetry Magazine. In 2014, they published a study
showing that if given 200 of a user's likes, they could predict that
person's personality-test answers better than their own romantic partner
could.

After getting his Ph.D., Kosinski landed a teaching position at the
Stanford Graduate School of Business and soon started looking for new
data sets to investigate. One in particular stood out: faces. For
decades, psychologists have been leery about associating personality
traits with physical characteristics, because of the lasting taint of
phrenology and eugenics; studying faces this way was, in essence, a
taboo. But to understand what that taboo might reveal when questioned,
Kosinski knew he couldn't rely on a human judgment.

Kosinski first mined 200,000 publicly posted dating profiles, complete
with pictures and information ranging from personality to political
views. Then he poured that data into an open-source facial-recognition
algorithm --- a so-called deep neural network, built by researchers at
Oxford University --- and asked it to find correlations between people's
faces and the information in their profiles. The algorithm failed to
turn up much, until, on a lark, Kosinski turned its attention to sexual
orientation. The results almost defied belief. In previous research, the
best any human had done at guessing sexual orientation from a profile
picture was about 60 percent --- slightly better than a coin flip. Given
five pictures of a man, the deep neural net could predict his sexuality
with as much as 91 percent accuracy. For women, that figure was lower
but still remarkable: 83 percent.

Much like his earlier work, Kosinski's findings raised questions about
privacy and the potential for discrimination in the digital age,
suggesting scenarios in which better programs and data sets might be
able to deduce anything from political leanings to criminality. But
there was another question at the heart of Kosinski's paper, a genuine
mystery that went almost ignored amid all the media response: \emph{How}
was the computer doing what it did? What was it seeing that humans could
not?

It was Kosinski's own research, but when he tried to answer that
question, he was reduced to a painstaking hunt for clues. At first, he
tried covering up or exaggerating parts of faces, trying to see how
those changes would affect the machine's predictions. Results were
inconclusive. But Kosinski knew that women, in general, have bigger
foreheads, thinner jaws and longer noses than men. So he had the
computer spit out the 100 faces it deemed most likely to be gay or
straight and averaged the proportions of each. It turned out that the
faces of gay men exhibited slightly more ``feminine'' proportions, on
average, and that the converse was true for women. If this was accurate,
it could support the idea that testosterone levels --- already known to
mold facial features --- help mold sexuality as well.

But it was impossible to say for sure. Other evidence seemed to suggest
that the algorithms might also be picking up on culturally driven
traits, like straight men wearing baseball hats more often. Or ---
crucially --- they could have been picking up on elements of the photos
that humans don't even recognize. ``Humans might have trouble detecting
these tiny footprints that border on the infinitesimal,'' Kosinski says.
``Computers can do that very easily.''

It has become commonplace to hear that machines, armed with machine
learning, can outperform humans at decidedly human tasks, from playing
Go to playing ``Jeopardy!'' We assume that is because computers simply
have more data-crunching power than our soggy three-pound brains.
Kosinski's results suggested something stranger: that artificial
intelligences often excel by developing whole new ways of seeing, or
even thinking, that are inscrutable to us. It's a more profound version
of what's often called the ``black box'' problem --- the inability to
discern exactly what machines are doing when they're teaching themselves
novel skills --- and it has become a central concern in
artificial-intelligence research. In many arenas, A.I. methods have
advanced with startling speed; deep neural networks can now detect
certain kinds of cancer as accurately as a human. But human doctors
still have to make the decisions --- and they won't trust an A.I. unless
it can explain itself.

This isn't merely a theoretical concern. In 2018, the European Union
will begin enforcing a law requiring that any decision made by a machine
be readily explainable, on penalty of fines that could cost companies
like Google and Facebook billions of dollars. The law was written to be
powerful and broad and fails to define what constitutes a satisfying
explanation or how exactly those explanations are to be reached. It
represents a rare case in which a law has managed to leap into a future
that academics and tech companies are just beginning to devote
concentrated effort to understanding. As researchers at Oxford dryly
noted, the law ``could require a complete overhaul of standard and
widely used algorithmic techniques'' --- techniques already permeating
our everyday lives.

Those techniques can seem inescapably alien to our own ways of thinking.
Instead of certainty and cause, A.I. works off probability and
correlation. And yet A.I. must nonetheless conform to the society we've
built --- one in which decisions require explanations, whether in a
court of law, in the way a business is run or in the advice our doctors
give us. The disconnect between how we make decisions and how machines
make them, and the fact that machines are making more and more decisions
for us, has birthed a new push for transparency and a field of research
called explainable A.I., or X.A.I. Its goal is to make machines able to
account for the things they learn, in ways that we can understand. But
that goal, of course, raises the fundamental question of whether the
world a machine sees can be made to match our own.

\textbf{``Artificial intelligence''} is a misnomer, an airy and
evocative term that can be shaded with whatever notions we might have
about what ``intelligence'' is in the first place. Researchers today
prefer the term ``machine learning,'' which better describes what makes
such algorithms powerful. Let's say that a computer program is deciding
whether to give you a loan. It might start by comparing the loan amount
with your income; then it might look at your credit history, marital
status or age; then it might consider any number of other data points.
After exhausting this ``decision tree'' of possible variables, the
computer will spit out a decision. If the program were built with only a
few examples to reason from, it probably wouldn't be very accurate. But
given millions of cases to consider, along with their various outcomes,
a machine-learning algorithm could tweak itself --- figuring out when
to, say, give more weight to age and less to income --- until it is able
to handle a range of novel situations and reliably predict how likely
each loan is to default.

Machine learning isn't just one technique. It encompasses entire
families of them, from ``boosted decision trees,'' which allow an
algorithm to change the weighting it gives to each data point, to
``random forests,'' which average together many thousands of randomly
generated decision trees. The sheer proliferation of different
techniques, none of them obviously better than the others, can leave
researchers flummoxed over which one to choose. Many of the most
powerful are bafflingly opaque; others evade understanding because they
involve an avalanche of statistical probability. It can be almost
impossible to peek inside the box and see what, exactly, is happening.

Rich Caruana, an academic who works at Microsoft Research, has spent
almost his entire career in the shadow of this problem. When he was
earning his Ph.D at Carnegie Mellon University in the 1990s, his thesis
adviser asked him and a group of others to train a neural net --- a
forerunner of the deep neural net --- to help evaluate risks for
patients with pneumonia. Between 10 and 11 percent of cases would be
fatal; others would be less urgent, with some percentage of patients
recovering just fine without a great deal of medical attention. The
problem was figuring out which cases were which --- a high-stakes
question in, say, an emergency room, where doctors have to make quick
decisions about what kind of care to offer. Of all the machine-learning
techniques students applied to this question, Caruana's neural net was
the most effective. But when someone on the staff of the University of
Pittsburgh Medical Center asked him if they should start using his
algorithm, ``I said no,'' Caruana recalls. ``I said we don't understand
what it does inside. I said I was afraid.''

The problem was in the algorithm's design. Classical neural nets focus
only on whether the prediction they gave is right or wrong, tweaking and
weighing and recombining all available morsels of data into a tangled
web of inferences that seems to get the job done. But some of these
inferences could be terrifically wrong. Caruana was particularly
concerned by something another graduate student noticed about the data
they were handling: It seemed to show that asthmatics with pneumonia
fared better than the typical patient. This correlation was real, but
the data masked its true cause. Asthmatic patients who contract
pneumonia are immediately flagged as dangerous cases; if they tended to
fare better, it was because they got the best care the hospital could
offer. A dumb algorithm, looking at this data, would have simply assumed
asthma meant a patient was likely to get better --- and thus concluded
that they were in less need of urgent care.

``I knew I could probably fix the program for asthmatics,'' Caruana
says. ``But what else did the neural net learn that was equally wrong?
It couldn't warn me about the unknown unknowns. That tension has
bothered me since the 1990s.''

The story of asthmatics with pneumonia eventually became a legendary
allegory in the machine-learning community. Today, Caruana is one of
perhaps a few dozen researchers in the United States dedicated to
finding more transparent new approaches to machine learning. For the
last six years, he has been creating a new model that combines a number
of machine-learning techniques. The result is as accurate as his
original neural network, and it can spit out charts that show how each
individual variable --- from asthma to age --- is predictive of
mortality risk, making it easier to see which ones exhibit particularly
unusual behavior. Immediately, asthmatics are revealed as a far outlier.
Other strange truths surface, too: For example, risk for people age 100
goes down suddenly. ``If you made it to this round number of 100,''
Caruana says, ``it seemed as if the doctors were saying, `Let's try to
get you another year,' which might not happen if you're 93.''

Caruana may have brought clarity to his own project, but his solution
only underscored the fact the explainability is a kaleidoscopic problem.
The explanation a doctor needs from a machine isn't the same as the one
a fighter pilot might need or the one an N.S.A. analyst sniffing out a
financial fraud might need. Different details will matter, and different
technical means will be needed for finding them. You couldn't, for
example, simply use Caruana's techniques on facial data, because they
don't apply to image recognition. There may, in other words, eventually
have to be as many approaches to explainability as there are approaches
to machine learning itself.

\includegraphics{https://static01.graylady3jvrrxbe.onion/images/2017/11/26/magazine/26explicableai2/26explicableai2-articleLarge.jpg?quality=75\&auto=webp\&disable=upscale}

\textbf{Three years ago,} David Gunning, one of the most consequential
people in the emerging discipline of X.A.I., attended a brainstorming
session at a state university in North Carolina. The event had the title
``Human-Centered Big Data,'' and it was sponsored by a government-funded
think tank called the Laboratory for Analytic Sciences. The idea was to
connect leading A.I. researchers with experts in data visualization and
human-computer interaction to see what new tools they might invent to
find patterns in huge sets of data. There to judge the ideas, and act as
hypothetical users, were analysts for the C.I.A., the N.S.A. and sundry
other American intelligence agencies.

The researchers in Gunning's group stepped confidently up to the white
board, showing off new, more powerful ways to draw predictions from a
machine and then visualize them. But the intelligence analyst evaluating
their pitches, a woman who couldn't tell anyone in the room what she did
or what tools she was using, waved it all away. Gunning remembers her as
plainly dressed, middle-aged, typical of the countless government agents
he had known who toiled thanklessly in critical jobs. ``None of this
solves my problem,'' she said. ``I don't need to be able to visualize
another recommendation. If I'm going to sign off on a decision, I need
to be able to justify it.'' She was issuing what amounted to a
broadside. It wasn't just that a clever graph indicating the best choice
wasn't the same as explaining why that choice was correct. The analyst
was pointing to a legal and ethical motivation for explainability: Even
if a machine made perfect decisions, a human would still have to take
responsibility for them --- and if the machine's rationale was beyond
reckoning, that could never happen.

Gunning, a grandfatherly military man whose buzz cut has survived his
stints as a civilian, is a program manager at the Defense Advanced
Research Projects Agency. He works in Darpa's shiny new midrise tower in
downtown Alexandria, Va. --- an office indistinguishable from the others
nearby, except that the security guard out front will take away your
cellphone and warn you that turning on the Wi-Fi on your laptop will
make security personnel materialize within 30 seconds. Darpa managers
like Gunning don't have permanent jobs; the expectation is that they
serve four-year ``tours,'' dedicated to funding cutting-edge research
along a single line of inquiry. When he found himself at the
brainstorming session, Gunning had recently completed his second tour as
a sort of Johnny Appleseed for A.I.: Starting in the 1990s, he has
founded hundreds of projects, from the first application of
machine-learning techniques to the internet, which presaged the first
search engines, to the project that eventually spun off as Siri, Apple's
voice-controlled assistant. ``I'm proud to be a dinosaur,'' he says with
a smile.

As of now, most of the military's practical applications of such
technology involve performing enormous calculations beyond the reach of
human patience, like predicting how to route supplies. But there are
more ambitious applications on the horizon. One recent research program
tried to use machine learning to sift through millions of video clips
and internet messages in Yemen to detect cease-fire violations; if the
machine does find something, it has to be able to describe what's worth
paying attention to. Another pressing need is for drones flying on
self-directed missions to be able to explain their limitations so that
the humans commanding the drones know what the machines can --- and
cannot --- be asked to do. Explainability has thus become a hurdle for a
wealth of possible projects, and the Department of Defense has begun to
turn its eye to the problem.

After that brainstorming session, Gunning took the analyst's story back
to Darpa and soon signed up for his third tour. As he flew across the
country meeting with computer scientists to help design an overall
strategy for tackling the problem of X.A.I., what became clear was that
the field needed to collaborate more broadly and tackle grander
problems. Computer science, having leapt beyond the bounds of
considering purely technical problems, had to look further afield --- to
experts, like cognitive scientists, who study the ways humans and
machines interact.

This represents a full circle for Gunning, who began his career as a
cognitive psychologist working on how to design better automated systems
for fighter pilots. Later, he began working on what's now called
``old-fashioned A.I.'' --- so-called expert systems in which machines
were given voluminous lists of rules, then tasked with drawing
conclusions by recombining those rules. None of those efforts was
particularly successful, because it was impossible to give the computer
a set of rules long enough, or flexible enough, to approximate the power
of human reasoning. A.I.'s current blossoming came only when researchers
began inventing new techniques for letting machines find their own
patterns in the data.

Gunning's X.A.I. initiative, which kicked off this year, provides \$75
million in funding to 12 new research programs; by the power of the
purse strings, Gunning has refocused the energies of a significant part
of the American A.I. research community. His hope is that by making
these new A.I. methods accountable to the demands of human psychology,
they will become both more useful and more powerful. ``The real secret
is finding a way to put labels on the concepts inside a deep neural
net,'' he says. If the concepts inside can be labeled, then they can be
used for reasoning --- just like those expert systems were supposed to
do in A.I.'s first wave.

\textbf{Deep neural nets,} which evolved from the kinds of techniques
that Rich Caruana was experimenting with in the 1990s, are now the class
of machine learning that seems most opaque. Just like old-fashioned
neural nets, deep neural networks seek to draw a link between an input
on one end (say, a picture from the internet) and an output on the other
end (``This is a picture of a dog''). And just like those older neural
nets, they consume all the examples you might give them, forming their
own webs of inference that can then be applied to pictures they've never
seen before. Deep neural nets remain a hotbed of research because they
have produced some of the most breathtaking technological
accomplishments of the last decade, from learning how to translate words
with better-than-human accuracy to learning how to drive.

To create a neural net that can reveal its inner workings, the
researchers in Gunning's portfolio are pursuing a number of different
paths. Some of these are technically ingenious --- for example,
designing new kinds of deep neural networks made up of smaller, more
easily understood modules, which can fit together like Legos to
accomplish complex tasks. Others involve psychological insight: One team
at Rutgers is designing a deep neural network that, once it makes a
decision, can then sift through its data set to find the example that
best demonstrates why it made that decision. (The idea is partly
inspired by psychological studies of real-life experts like
firefighters, who don't clock in for a shift thinking, These are the 12
rules for fighting fires; when they see a fire before them, they compare
it with ones they've seen before and act accordingly.) Perhaps the most
ambitious of the dozen different projects are those that seek to bolt
new explanatory capabilities onto existing deep neural networks. Imagine
giving your pet dog the power of speech, so that it might finally
explain what's so interesting about squirrels. Or, as Trevor Darrell, a
lead investigator on one of those teams, sums it up, ``The solution to
explainable A.I. is more A.I.''

Five years ago, Darrell and some colleagues had a novel idea for letting
an A.I. teach itself how to describe the contents of a picture. First,
they created two deep neural networks: one dedicated to image
recognition and another to translating languages. Then they lashed these
two together and fed them thousands of images that had captions attached
to them. As the first network learned to recognize the objects in a
picture, the second simply watched what was happening in the first, then
learned to associate certain words with the activity it saw. Working
together, the two networks could identify the features of each picture,
then label them. Soon after, Darrell was presenting some different work
to a group of computer scientists when someone in the audience raised a
hand, complaining that the techniques he was describing would never be
explainable. Darrell, without a second thought, said, Sure --- but you
could make it explainable by once again lashing two deep neural networks
together, one to do the task and one to describe it.

Darrell's previous work had piggybacked on pictures that were already
captioned. What he was now proposing was creating a new data set and
using it in a novel way. Let's say you had thousands of videos of
baseball highlights. An image-recognition network could be trained to
spot the players, the ball and everything happening on the field, but it
wouldn't have the words to label what they were. But you might then
create a new data set, in which volunteers had written sentences
describing the contents of every video. Once combined, the two networks
should then be able to answer queries like ``Show me all the double
plays involving the Boston Red Sox'' --- and could potentially show you
what cues, like the logos on uniforms, it used to figure out who the
Boston Red Sox are.

Call it the Hamlet strategy: lending a deep neural network the power of
internal monologue, so that it can narrate what's going on inside. But
do the concepts that a network has taught itself align with the reality
that humans are describing, when, for example, narrating a baseball
highlight? Is the network recognizing the Boston Red Sox by their logo
or by some other obscure signal, like ``median facial-hair
distribution,'' that just happens to correlate with the Red Sox? Does it
actually have the concept of ``Boston Red Sox'' or just some other
strange thing that only the computer understands? It's an ontological
question: Is the deep neural network really seeing a world that
corresponds to our own?

We human beings seem to be obsessed with black boxes: The highest
compliment we give to technology is that it feels like magic. When the
workings of a new technology is too obvious, too easy to explain, it can
feel banal and uninteresting. But when I asked David Jensen --- a
professor at the University of Massachusetts at Amherst and one of the
researchers being funded by Gunning --- why X.A.I. had suddenly become a
compelling topic for research, he sounded almost soulful: ``We want
people to make informed decisions about whether to trust autonomous
systems,'' he said. ``If you don't, you're depriving people of the
ability to be fully independent human beings.''

\textbf{A decade} in the making, the European Union's General Data
Protection Regulation finally goes into effect in May 2018. It's a
sprawling, many-tentacled piece of legislation whose opening lines
declare that the protection of personal data is a universal human right.
Among its hundreds of provisions, two seem aimed squarely at where
machine learning has already been deployed and how it's likely to
evolve. Google and Facebook are most directly threatened by Article 21,
which affords anyone the right to opt out of personally tailored ads.
The next article then confronts machine learning head on, limning a
so-called right to explanation: E.U. citizens can contest ``legal or
similarly significant'' decisions made by algorithms and appeal for
human intervention. Taken together, Articles 21 and 22 introduce the
principle that people are owed agency and understanding when they're
faced by machine-made decisions.

For many, this law seems frustratingly vague. Some legal scholars argue
that it might be toothless in practice. Others claim that it will
require the basic workings of Facebook and Google to change, lest they
face penalties of 4 percent of their revenue. It remains to be seen
whether complying with the law will mean a heap of fine print and an
extra check box buried in a pop-up window, some new kind of
warning-label system marking every machine-made decision or much more
profound changes.

If Google is one of the companies most endangered by this new scrutiny
on A.I., it's also the company with the greatest wherewithal to lead the
whole industry in solving the problem. Even among the company's
astonishing roster of A.I. talent, one particular star is Chris Olah,
who holds the title of research scientist --- a title shared by Google's
many ex-professors and Ph.D.s --- without ever having completed more
than a year of college. Olah has been working for the last couple of
years on creating new ways to visualize the inner workings of a deep
neural network. You might recall when Google created a hallucinatory
tool called Deep Dream, which produced psychedelic distortions when you
fed it an image and which went viral when people used it to create
hallucinatory mash-ups like a doll covered in a pattern of doll eyes and
a portrait of Vincent Van Gogh made up in places of bird beaks. Olah was
one of many Google researchers on the team, led by Alex Mordvintsev,
that worked on Deep Dream. It may have seemed like a folly, but it was
actually a technical steppingstone.

Olah speaks faster and faster as he sinks into an idea, and the words
tumbled out of him almost too quickly to follow as he explained what he
found so exciting about the work he was doing. ``The truth is, it's
really beautiful. There's some sense in which we don't know what it
means to see. We don't understand how humans do it,'' he told me, hands
gesturing furiously. ``We want to understand something not just about
neural nets but something deeper about reality.'' Olah's hope is that
deep neural networks reflect something deeper about parsing data ---
that insights gleaned from them might in turn shed light on how our
brains work.

Olah showed me a sample of work he was preparing to publish with a set
of collaborators, including Mordvintsev; it was made
\href{https://distill.pub/2017/feature-visualization/}{public} this
month. The tool they had developed was basically an ingenious way of
testing a deep neural network. First, it fed the network a random image
of visual noise. Then it tweaked that image over and over again, working
to figure out what excited each layer in the network the most.
Eventually, that process would find the platonic ideal that each layer
of the network was searching for. Olah demonstrated with a network
trained to classify different breeds of dogs. You could pick out a
neuron from the topmost layer while it was analyzing a picture of a
golden retriever. You could see the ideal it was looking for --- in this
case, a hallucinatory mash-up of floppy ears and a forlorn expression.
The network was indeed homing in on higher-level traits that we could
understand.

Watching him use the tool, I realized that it was exactly what the
psychologist Michal Kosinski needed --- a key to unlock what his deep
neural network was seeing when it categorized profile pictures as gay or
straight. Kosinski's most optimistic view of his research was that it
represented a new kind of science in which machines could access truths
that lay beyond human intuition. The problem was reducing what a
computer knew into a single conclusion that a human could grasp and
consider. He had painstakingly tested his data set by hand and found
evidence that the computer might be discovering hormonal signals in
facial structure. That evidence was still fragmentary. But with the tool
that Olah showed me, or one like it, Kosinski might have been able to
pull back the curtain on how his mysterious A.I. was working. It would
be as obvious and intuitive as a picture the computer had drawn on its
own.

Advertisement

\protect\hyperlink{after-bottom}{Continue reading the main story}

\hypertarget{site-index}{%
\subsection{Site Index}\label{site-index}}

\hypertarget{site-information-navigation}{%
\subsection{Site Information
Navigation}\label{site-information-navigation}}

\begin{itemize}
\tightlist
\item
  \href{https://help.nytimes3xbfgragh.onion/hc/en-us/articles/115014792127-Copyright-notice}{©~2020~The
  New York Times Company}
\end{itemize}

\begin{itemize}
\tightlist
\item
  \href{https://www.nytco.com/}{NYTCo}
\item
  \href{https://help.nytimes3xbfgragh.onion/hc/en-us/articles/115015385887-Contact-Us}{Contact
  Us}
\item
  \href{https://www.nytco.com/careers/}{Work with us}
\item
  \href{https://nytmediakit.com/}{Advertise}
\item
  \href{http://www.tbrandstudio.com/}{T Brand Studio}
\item
  \href{https://www.nytimes3xbfgragh.onion/privacy/cookie-policy\#how-do-i-manage-trackers}{Your
  Ad Choices}
\item
  \href{https://www.nytimes3xbfgragh.onion/privacy}{Privacy}
\item
  \href{https://help.nytimes3xbfgragh.onion/hc/en-us/articles/115014893428-Terms-of-service}{Terms
  of Service}
\item
  \href{https://help.nytimes3xbfgragh.onion/hc/en-us/articles/115014893968-Terms-of-sale}{Terms
  of Sale}
\item
  \href{https://spiderbites.nytimes3xbfgragh.onion}{Site Map}
\item
  \href{https://help.nytimes3xbfgragh.onion/hc/en-us}{Help}
\item
  \href{https://www.nytimes3xbfgragh.onion/subscription?campaignId=37WXW}{Subscriptions}
\end{itemize}
