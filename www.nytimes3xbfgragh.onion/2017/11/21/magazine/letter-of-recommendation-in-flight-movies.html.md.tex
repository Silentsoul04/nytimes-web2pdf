Sections

SEARCH

\protect\hyperlink{site-content}{Skip to
content}\protect\hyperlink{site-index}{Skip to site index}

\href{https://myaccount.nytimes3xbfgragh.onion/auth/login?response_type=cookie\&client_id=vi}{}

\href{https://www.nytimes3xbfgragh.onion/section/todayspaper}{Today's
Paper}

Letter of Recommendation: In-Flight Movies

\url{https://nyti.ms/2jJuKsC}

\begin{itemize}
\item
\item
\item
\item
\item
\end{itemize}

Advertisement

\protect\hyperlink{after-top}{Continue reading the main story}

Supported by

\protect\hyperlink{after-sponsor}{Continue reading the main story}

\href{/column/letter-of-recommendation}{Letter of Recommendation}

\hypertarget{letter-of-recommendation-in-flight-movies}{%
\section{Letter of Recommendation: In-Flight
Movies}\label{letter-of-recommendation-in-flight-movies}}

\includegraphics{https://static01.graylady3jvrrxbe.onion/images/2017/11/26/magazine/26lor/26mag-26lor-t_CA0-articleLarge.jpg?quality=75\&auto=webp\&disable=upscale}

By Meher Ahmad

\begin{itemize}
\item
  Nov. 21, 2017
\item
  \begin{itemize}
  \item
  \item
  \item
  \item
  \item
  \end{itemize}
\end{itemize}

The flight attendant passing a tomato juice to my seatmate did a
double-take when she saw my tear-streaked face. She touched my shoulder:
``Ma'am, are you doing O.K.?'' I wiped the gobs of snot from my nose,
pulled my headphones off and gave her that awkward, forced smile you do
when you're crying but want the other person to leave you alone. She
obliged.

I was watching ``Lion,'' the Garth Davis film about a young Indian boy
adopted by nice, white Australians, but I would have reacted the same
way if I were watching a David Attenborough nature documentary. I wasn't
even two minutes into the film when the first tear rolled down my cheek
dramatically and my lower lip started quivering.

Crying on planes is so common that it has prompted cheeky ``weep
warnings'' on Virgin Atlantic flights and myriad articles trying to
understand why we do it. The most accepted explanation is a simple
confluence of altitude, loneliness and the heightened emotions that
accompany the humiliating experience that is modern air travel.

I've been dealing with clinical anxiety, the most millennial of
maladies, since I was a teenager. I experienced my first full-fledged
anxiety attack in a Metro car in Paris, and after that I became a
panicked traveler regardless of whether I was on the ground or in the
air. Over time, my phobia shifted to become about crying itself ---
which then made it inevitable that I would cry, in that prophetic way
anxiety manifests itself. Once, on a short flight from Marseille to
Paris, I caught the panic creeping up slowly enough that I flagged the
flight attendant, hoping she could somehow reverse the oncoming
hyperventilation with a chic Air France accouterment. ``Just pretend she
is a biiiiig bus,'' she crooned in Franglais, assuming my anxiety
derived from a fear of flying rather than from my actual fear: people
seeing me cry in public.

Though I come on my mother's side from a long line of Shiites, who
encourage public grieving through institutionalized mourning called
\emph{matam,} there's little I find more mortifying than crying in front
of strangers. \emph{Matam} is the act of grieving Imam Hussain, the
Prophet Muhammad's grandson, who was martyred along with his followers
in a bloody battle in the year 680. Our tears and self-flagellations are
meant to be in remembrance of Hussain's agony, but it's actually a
chance for us to grieve about our own quotidian sufferings. It's
difficult not to cry when you attend a \emph{majlis,} or the
congregation for \emph{matam.}

My mother, my aunts, my grandmother and her sisters have between them
survived multiple wars, a violent escape from their small town in what
is now India during the Partition and an economic exodus that took their
children and grandchildren to far-flung places like Minneapolis and
Indianapolis, the opposite end of the world from Pakistan. Still, they
don't really cry outside a \emph{majlis.} The ritual allows them to
break down without any of the judgment that is otherwise associated with
being a wailing woman in public. The few times I've been with them, I
sat on the floor in the ladies' section of the mosque and watched as
they dissolved into sobbing, shaking piles of dark fabric. My face
stayed dry.

I pride myself in rarely crying or displaying any emotion at all, so
having panic attacks meant my anxieties came out all at once. They
weren't expected, and their erratic appearances are precisely what made
them debilitating. At my worst, I convinced myself that my fear of
weeping on planes would stop me from flying altogether, but then I began
relishing the idea of flipping through the in-flight movie catalog on a
clumsy touch-screen, calmly selecting a straight-to-DVD film that would
reduce me to tears.

On a plane, the principles of film selection are suspended as long as
we're at cruising altitude: It's fine --- welcome, even --- to watch a
bad movie. I've found myself dabbing my eyes through ``Murder, She
Baked,'' a series of films created by Hallmark Movies \& Mysteries, and
have somehow choked up at ``Taken 2'' \emph{and} ``Taken 3'' ---
anything in which the music swells. I had little interest in watching
``Lion'' in theaters or at home, but I knew it would give me the
opportunity to freely weep at someone else's anxieties instead of my
own. Above all, it gave me an excuse to cry when there seemed to be a
reason to, a semblance of control.

By the time the plane begins its descent, I'm blotchy-eyed but
emotionally sound; I'm still riding the high of postcrying calmness
while standing in the airport taxi line. The rhythm to the ritual is
comforting enough that I don't even need to discreetly nibble off-brand
Xanax in my seat; it has replaced my new-age meditation completely. The
idea of letting my weepy alter-ego make a controlled appearance is so
appealing that I sometimes wonder whether I travel just to give my tear
ducts a test run.

A kind woman sitting cater-corner to my seat watched me dissolve into
the hiccuppy kind of crying by the last scene of ``Lion,'' handing me
her allotted single in-flight napkin after seeing that mine was soaked.
Airlines are stingy with everything, but it seems inhumane that they
aren't willing to hand out more napkins during a service, given that
everyone acknowledges that in-flight movies make them cry. At Shiite
mosques, they like to put boxes of pink, perfumed Rose Petal brand
tissues around the room before people arrive for prayer. At least
they're prepared.

Advertisement

\protect\hyperlink{after-bottom}{Continue reading the main story}

\hypertarget{site-index}{%
\subsection{Site Index}\label{site-index}}

\hypertarget{site-information-navigation}{%
\subsection{Site Information
Navigation}\label{site-information-navigation}}

\begin{itemize}
\tightlist
\item
  \href{https://help.nytimes3xbfgragh.onion/hc/en-us/articles/115014792127-Copyright-notice}{©~2020~The
  New York Times Company}
\end{itemize}

\begin{itemize}
\tightlist
\item
  \href{https://www.nytco.com/}{NYTCo}
\item
  \href{https://help.nytimes3xbfgragh.onion/hc/en-us/articles/115015385887-Contact-Us}{Contact
  Us}
\item
  \href{https://www.nytco.com/careers/}{Work with us}
\item
  \href{https://nytmediakit.com/}{Advertise}
\item
  \href{http://www.tbrandstudio.com/}{T Brand Studio}
\item
  \href{https://www.nytimes3xbfgragh.onion/privacy/cookie-policy\#how-do-i-manage-trackers}{Your
  Ad Choices}
\item
  \href{https://www.nytimes3xbfgragh.onion/privacy}{Privacy}
\item
  \href{https://help.nytimes3xbfgragh.onion/hc/en-us/articles/115014893428-Terms-of-service}{Terms
  of Service}
\item
  \href{https://help.nytimes3xbfgragh.onion/hc/en-us/articles/115014893968-Terms-of-sale}{Terms
  of Sale}
\item
  \href{https://spiderbites.nytimes3xbfgragh.onion}{Site Map}
\item
  \href{https://help.nytimes3xbfgragh.onion/hc/en-us}{Help}
\item
  \href{https://www.nytimes3xbfgragh.onion/subscription?campaignId=37WXW}{Subscriptions}
\end{itemize}
