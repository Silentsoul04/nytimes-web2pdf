Sections

SEARCH

\protect\hyperlink{site-content}{Skip to
content}\protect\hyperlink{site-index}{Skip to site index}

\href{https://www.nytimes3xbfgragh.onion/section/well/move}{Move}

\href{https://myaccount.nytimes3xbfgragh.onion/auth/login?response_type=cookie\&client_id=vi}{}

\href{https://www.nytimes3xbfgragh.onion/section/todayspaper}{Today's
Paper}

\href{/section/well/move}{Move}\textbar{}Activity Trackers Don't Always
Work the Way We Want Them To

\url{https://nyti.ms/2hOPVZE}

\begin{itemize}
\item
\item
\item
\item
\item
\item
\end{itemize}

Advertisement

\protect\hyperlink{after-top}{Continue reading the main story}

Supported by

\protect\hyperlink{after-sponsor}{Continue reading the main story}

Well

\hypertarget{activity-trackers-dont-always-work-the-way-we-want-them-to}{%
\section{Activity Trackers Don't Always Work the Way We Want Them
To}\label{activity-trackers-dont-always-work-the-way-we-want-them-to}}

\includegraphics{https://static01.graylady3jvrrxbe.onion/images/2017/11/26/magazine/26mag-well/26mag-well-articleLarge.jpg?quality=75\&auto=webp\&disable=upscale}

By
\href{https://www.nytimes3xbfgragh.onion/by/gretchen-reynolds}{Gretchen
Reynolds}

\begin{itemize}
\item
  Nov. 21, 2017
\item
  \begin{itemize}
  \item
  \item
  \item
  \item
  \item
  \item
  \end{itemize}
\end{itemize}

Comparatively speaking, young people in the United States and England do
not move much. Studies indicate that most children reach their activity
peak at about age 7 and become more sedentary throughout adolescence.
Many parents probably hope that shiny new technologies, such as Fitbits
and other physical-activity monitors, might inspire our children to
become more active.

But a
\href{http://www.tandfonline.com/doi/abs/10.1080/19325037.2017.1343161}{recent
study published in The American Journal of Health Education} finds that
the gadgets frequently have counterproductive impacts on young people's
attitudes about exercise and the capabilities of their own bodies.

The new study, conducted by psychologists from Brunel University London
and the University of Birmingham, involved 100 healthy boys and girls
ages 13 and 14 from two middle schools in England. The schools were far
apart geographically and socioeconomically, representing a broad cross
section of adolescent society.

The researchers began by interviewing the young people and asking them
to fill out psychological questionnaires about how they felt about
exercise and their fitness. Then the scientists gave everyone an
activity monitor, which came preprogrammed with a goal of 10,000 steps
each day. The users' activities could be recorded on a ``leader board''
shared with friends, which would show who had been the most and least
active.

The teenagers were asked to use the monitors for two months, and then
complete more questionnaires and participate in focus-group discussions.
During the focus groups, almost all the young people expressed initial
enthusiasm for the monitors and said they had at first become more
active.

But the allure soon faded. After about a month, most of the teenagers
had begun to find the monitors chiding and irksome, making them feel
lazy if they did not manage 10,000 steps each day. Many also said they
now considered themselves more physically inept than they had at the
study's start, often because they were rarely near the top of the
activity leader boards. Most telling, a large percentage of the
adolescents reported feeling less motivated to be active now than before
getting the monitor. (The researchers did not directly track changes in
the young people's activity levels, because the study focused on
psychology.)

The problem with the monitors seemed to be that they had left the
teenagers feeling pressure and with little control over their
activities, as well as self-conscious about their physical abilities,
said Charlotte Kerner, a lecturer in youth sport and physical education
at Brunel University London, who led the study. The result was
frustration, self-reproach --- and less, not more, movement.

``You can't just give a child a Fitbit for Christmas and expect them to
be active,'' Kerner said. ``They will need educating on how best to
negotiate the features.'' Nudge them to set realistic step counts and
other fitness goals, she says, and to consider whether they want to
share their results with friends. For many young people, fitness may be
better achieved in private.

Advertisement

\protect\hyperlink{after-bottom}{Continue reading the main story}

\hypertarget{site-index}{%
\subsection{Site Index}\label{site-index}}

\hypertarget{site-information-navigation}{%
\subsection{Site Information
Navigation}\label{site-information-navigation}}

\begin{itemize}
\tightlist
\item
  \href{https://help.nytimes3xbfgragh.onion/hc/en-us/articles/115014792127-Copyright-notice}{©~2020~The
  New York Times Company}
\end{itemize}

\begin{itemize}
\tightlist
\item
  \href{https://www.nytco.com/}{NYTCo}
\item
  \href{https://help.nytimes3xbfgragh.onion/hc/en-us/articles/115015385887-Contact-Us}{Contact
  Us}
\item
  \href{https://www.nytco.com/careers/}{Work with us}
\item
  \href{https://nytmediakit.com/}{Advertise}
\item
  \href{http://www.tbrandstudio.com/}{T Brand Studio}
\item
  \href{https://www.nytimes3xbfgragh.onion/privacy/cookie-policy\#how-do-i-manage-trackers}{Your
  Ad Choices}
\item
  \href{https://www.nytimes3xbfgragh.onion/privacy}{Privacy}
\item
  \href{https://help.nytimes3xbfgragh.onion/hc/en-us/articles/115014893428-Terms-of-service}{Terms
  of Service}
\item
  \href{https://help.nytimes3xbfgragh.onion/hc/en-us/articles/115014893968-Terms-of-sale}{Terms
  of Sale}
\item
  \href{https://spiderbites.nytimes3xbfgragh.onion}{Site Map}
\item
  \href{https://help.nytimes3xbfgragh.onion/hc/en-us}{Help}
\item
  \href{https://www.nytimes3xbfgragh.onion/subscription?campaignId=37WXW}{Subscriptions}
\end{itemize}
