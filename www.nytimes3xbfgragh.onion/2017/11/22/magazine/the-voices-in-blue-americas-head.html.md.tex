Sections

SEARCH

\protect\hyperlink{site-content}{Skip to
content}\protect\hyperlink{site-index}{Skip to site index}

The Voices in Blue America's Head

\url{https://nyti.ms/2jMtKE8}

\begin{itemize}
\item
\item
\item
\item
\item
\item
\end{itemize}

\includegraphics{https://static01.graylady3jvrrxbe.onion/images/2017/11/26/magazine/26mag-media-image1/26mag-26media-t_CA0-articleLarge.jpg?quality=75\&auto=webp\&disable=upscale}

Feature

\hypertarget{the-voices-in-blue-americas-head}{%
\section{The Voices in Blue America's
Head}\label{the-voices-in-blue-americas-head}}

For years, liberals have tried, and failed, to create their own version
of conservative talk radio. Has Crooked Media finally figured it out?

From left: Tommy Vietor, Jon Lovett and Jon Favreau backstage before a
``Pod Tours America'' event in November.Credit...Brian Finke for The New
York Times

Supported by

\protect\hyperlink{after-sponsor}{Continue reading the main story}

By Jason Zengerle

\begin{itemize}
\item
  Nov. 22, 2017
\item
  \begin{itemize}
  \item
  \item
  \item
  \item
  \item
  \item
  \end{itemize}
\end{itemize}

It was early November, the day before Virginia's elections, and the
Democratic cavalry --- in the form of four podcast hosts crammed into a
Lyft --- was coming to the aid of Lt. Gov. Ralph Northam. ``Do you want
to kick things off with something light and funny?'' Jon Favreau asked
Jon Lovett as their ride --- an S.U.V. outfitted with neon lights and a
disco ball that were a bit discombobulating before 9 o'clock in the
morning --- took them to a Richmond campaign office. They'd be rallying
volunteers canvassing for Northam, the Democratic candidate for
governor, who was at the time commanding a perilously narrow lead in the
polls. ``I want to go toward the end for some earnestness,'' Favreau
said.

``You should do something real message-y,'' Tommy Vietor proposed.

``I'm expecting the `race speech' for G.O.T.V.,'' Dan Pfeiffer chimed
in.

It was a joke from the podcasters' past lives. As Barack Obama's chief
speechwriter for eight years, Favreau had a hand in some of his most
memorable oratory --- none more so than the 2008 campaign speech about
race that followed questions about Obama's relationship with the Rev.
Jeremiah Wright. Whenever a knotty issue arose in Obama's White House,
Pfeiffer and Vietor, who worked in the communications department, and
Lovett, a fellow speechwriter, would taunt Favreau: ``We need a `race
speech' for Simpson-Bowles,'' or: ``Write a `race speech' for the BP oil
spill.''

The night before, at the National, an 800-seat theater in Richmond,
Favreau and his co-hosts performed a sold-out live taping of ``Pod Save
America,'' a liberal political podcast and the flagship offering from
Crooked Media, the media company that Favreau, Lovett and Vietor started
in January. ``Pod Save America'' scored its first million-listener
episode within its first several weeks, and it now averages 1.5 million
listeners per show --- about as many people as Anderson Cooper draws on
prime-time CNN. Their podcast has come to occupy a singular perch in
blue America; where an NPR tote bag once signified a certain political
persuasion and mind-set, in the age of Trump, it's a ``Friend of the
Pod'' T-shirt. `` `Pod Save America,' '' says the Democratic strategist
Jesse Ferguson, ``is the voice of the `resistance.' ''

Its hosts have not shied away from making use of their newfound
influence. In 2017, ``Pod Save America'' has pointed its audience toward
an array of grass-roots groups on the left, partnering with MoveOn to
send nearly 2,000 listeners to Republican town-hall meetings, with Swing
Left to raise more than \$1 million for challengers to House Republicans
next year and with Indivisible to deluge Republican senators with tens
of thousands of phone calls in favor of preserving Obamacare. And in
Richmond, the hosts were lending their activist cachet and charisma to
Northam, a candidate who, Democrats worried, could use a lot more of
both.

``With Donald Trump winning the presidency, we have decided --- we've
realized --- that democracy is not just a job for politicians,'' Favreau
told the crowd at Northam's campaign office, amid half-empty doughnut
boxes and carafes of coffee. ``It's a job for every single American, and
that job doesn't just end on Election Day --- that job is an
every-single-day job. It is a fight.''

As the podcasters spoke, Northam looked on with what appeared to be a
mixture of bewilderment and admiration. He was the candidate, and the
only person in the room wearing a suit and tie --- the podcast hosts,
like the canvassers, were dressed in jeans and hoodies --- but it was
clear he knew he wasn't the star of this particular show. When it was
his turn to speak, the man who in 36 hours would be elected Virginia's
73rd governor recalled a conversation he had the day before with his two
20-something children. ``They said, `We heard you're going to be on
``Pod Save America''! Is that true?' '' Northam recounted. The crowd
laughed. ``Oh, my God!'' he exclaimed. ``I have finally arrived!''

\textbf{During the 2016 campaign,} Favreau, Lovett, Vietor and Pfeiffer
--- mostly as a lark --- hosted a popular politics podcast for Bill
Simmons's sports-and-pop-culture website The Ringer called ``Keepin' It
1600.'' But with Hillary Clinton expected to be sitting in the Oval
Office in 2017, ``we didn't want to be the people who criticized the
White House just to be interesting, nor did we want to be to the Clinton
administration what Hannity now is to the Trump administration,''
Pfeiffer says. ``We all assumed the election was the end of the road for
us.''

Favreau, Lovett and Vietor were in their 20s when they went to work for
Obama in the White House, and they had been somewhat adrift since
leaving it around the end of Obama's first term. They relocated to
California in search of a second act, but nothing quite stuck. Lovett
helped create a sitcom called ``1600 Penn'' about a wacky First Family,
but poor ratings and reviews led NBC to cancel it after one season.
Favreau and Vietor founded a strategic communications firm to pay the
bills while they nursed their own TV ambitions, but their projects --- a
campaign drama-comedy called ``Early States'' and a public-affairs show
that they pitched, with Lovett, as ``a millennial `Meet the Press' ''
--- were rejected by the networks and streaming services. ``Lots of
people in suits told us that politics was a crowded space as they
greenlit `CSI {[}expletive{]} West Hollywood' or whatever,'' Vietor
recalls.

The day after Trump's victory, Lovett was driving Favreau and Vietor to
The Ringer's Hollywood studios when his car ran out of gas. It was while
the three of them were pushing the Jeep Grand Cherokee down Sunset
Boulevard that they first started discussing what would become ``Pod
Save America'' and Crooked Media. They wanted to get involved in
politics again, but none of them had any desire to go back to Washington
or to work for a candidate. A podcast and a liberal media company, they
thought, could be their contribution to the anti-Trump resistance.

\includegraphics{https://static01.graylady3jvrrxbe.onion/images/2017/11/26/magazine/26mag-media-image2/26mag-media-image2-articleLarge.jpg?quality=75\&auto=webp\&disable=upscale}

In the early days of ``Pod Save America,'' the hosts leaned heavily on
their Obama connections; Obama himself was the guest on one of their
first episodes. But as the podcast rapidly built an audience, Democratic
politicians outside the Obama orbit began accepting their invitations,
or sometimes even asking to appear on the show --- even if they didn't
always know what exactly ``Pod Save America'' was. In a May episode,
Senator Amy Klobuchar of Minnesota confessed that it had been her
daughter's idea for her to be interviewed: ``I, for some reason, thought
it was a video, so I spent a lot of time wearing a hip outfit today, and
then I found out it was a podcast.''

More than 1,600 political podcasts --- most of them anti-Trump --- have
appeared since the 2016 election, according to RawVoice, a podcast
hosting and analytics company. ``Pod Save America,'' with nearly 120
million downloads to date, is the undisputed king of the field. But the
show's numbers alone do not quite capture the nature of its
accomplishment. With a shoestring budget and no organizational backing,
its hosts seem to have created something that liberals have spent almost
two decades, and hundreds of millions of dollars, futilely searching
for: the left's answer to conservative talk radio.

Air America, the nationwide liberal talk-radio network, declared
bankruptcy and stopped broadcasting in 2010 after six years of middling
to abysmal ratings. The independent cable network Current TV, which Al
Gore started with Joel Hyatt in 2005, tried to make itself a platform
for unapologetically liberal commentary --- at one point hiring Keith
Olbermann as its chief news officer --- but it was sold and shut down in
2013. Part of the problem with these earlier ventures was their
arms-race mentality: They offered liberals a mirror image of what
conservatives had, rather than something liberals might actually want.
``Olbermann was lefty O'Reilly,'' says Tim Miller, a Republican media
consultant and Crooked Media's token conservative contributor. ``Air
America was lefty Limbaugh.''

``Pod Save America,'' by contrast, has no conservative antecedent. The
craft-beer-bar-bull-session vibe of podcasts suits the left better than
the shouty antagonism of talk radio. ``Rather than trying to replicate
what's worked on the right, these podcasts aren't taking the same tropes
you see on Fox or hear on conservative talk radio and applying them to
the left,'' Miller says.

On ``Pod Save America,'' Favreau sits in what radio pros call ``the
power chair,'' dictating the topics and pace of the show; Lovett
provides comic relief; and Pfeiffer and Vietor contribute an earnest
wonkiness. A typical hourlong episode might consist of a breakdown of
the latest Republican tax-reform proposal, some war stories from the
Obama White House, a dispute about which host was more disruptive at a
recent ``Game of Thrones'' viewing party and an interview with Ta-Nehisi
Coates. ``It's down to earth and relaxed,'' Seth Moulton, a
Massachusetts congressman who appeared on ``Pod Save America'' in March,
told me. ``I think it's important for people to realize I'm a regular
person, and sometimes you don't get that when you see me in a suit on
CNN.''

Like conservative talk radio or Fox News, ``Pod Save America'' is an
authentic partisan response to the perceived failings of the mainstream
media. While many conservatives hate the mainstream media for its
supposed liberal bias, many liberals have come to despise what they see
as its tendency toward false equivalence --- a grievance particularly
inflamed by the coverage of Hillary Clinton during the 2016 campaign.
Liberals don't want a hermetically sealed media ecosystem of their own,
so much as one that does away with the pretense of kneejerk balance: a
media that's willing to say one side is worse than the other. ``I
screamed at the TV a lot in the White House,'' Favreau says. He and his
co-hosts particularly loathe the bipartisan on-air panels of blabbering
pundits that cable networks deployed during the election. ``If there is
one way that I would sum up what the 2016 election was on cable news,''
Lovett says, ``it was world-class journalists interviewing morons.''

``Pod Save America,'' to its hosts and its listeners, is a twice-weekly
reality check. ``I think that when you have a president gaslighting an
entire nation,'' Vietor says, ``there's a cathartic effect when you have
a couple of people who worked in the White House who are like: `Hey,
this is crazy. You're right, he's wrong.' ''

What is absent from the podcast, significantly, is any of the usual
liberal squeamishness (or, depending on your point of view, principle)
about using media as a tool of partisan advantage. Liberal activists
point regretfully to Jon Stewart and Stephen Colbert, who in their
Comedy Central heyday were happy to savage Republicans but refused to
champion Democrats: In 2010, the pair drew some 215,000 people to the
National Mall a few days before the midterm elections, only to keep the
rally strictly nonpartisan. ``Pod Save America,'' by contrast, isn't
afraid to, as Ben Wikler of MoveOn puts it, ``actually touch
Excalibur.'' At the theater in Richmond this month, shortly before
bringing Northam and the rest of Virginia's Democratic ticket onstage,
Favreau asked the crowd: ``Is everyone registered to vote? Is everyone
going to be doing phone-banking and canvassing? Because if not, you have
to leave.''

\textbf{Crooked Media's headquarters} consists of a few bargain-priced
rooms on La Cienega in a seedy section of West Hollywood, cater-corner
to a lingerie shop and across the street from a strip club. On the
summer afternoon I visited, I was greeted at the entrance by a
goldendoodle. Favreau, materializing behind the animal, said: ``This is
Lovett's dog, Pundit --- the thing that we hate and the thing that we've
become.''

The office, like the company itself, was still very much a work in
progress. An entire wall was covered with ``A Beautiful Mind''-style
scribbles about ``webseries,'' ``daily micropods'' and ``chat convos''
--- the handiwork of Tanya Somanader, who was the director of digital
rapid response in the Obama White House and is now Crooked Media's chief
content officer. ``This,'' she said, pointing at the wall and summoning
as portentous a tone as she could muster, ``is how you build a media
empire.''

The self-mockery about Crooked Media's ambitions belies how outsize
those ambitions are. In addition to ``Pod Save America,'' the company
now has six other podcasts and plans to roll out at least two more soon.
It has hired two producers, one from MTV and the other from the Oprah
Winfrey Network. In October, it poached a New Republic writer to helm
its website. A nationwide ``Pod Save America'' tour, Crooked Media's
first serious stab at live events, has so far played to sold-out
theaters in seven cities. For the 2018 midterms and the 2020
presidential race, Crooked Media is hoping to host candidate forums and
debates.

Acutely aware of the perils of their new operation resembling the old
political-media boys' club, the decidedly bro-ish ``Pod Save America''
hosts have slanted Crooked Media's growing podcast slate toward
non-white-male hosts, and the company's top two executives are women.
``Ideally what we're trying to build is a media company that's not about
one show, `Pod Save America,' but a whole bunch of new shows that are
not living and dying by the latest tweet,'' Vietor told me.

Still, the one show is serving them awfully well. An executive at
another podcasting company told me that assuming standard industry
rates, Crooked Media is most likely bringing in at least \$50,000 in
advertising revenue for each episode of ``Pod Save America'' --- which
at two episodes a week is about \$5 million a year. That has allowed the
company to turn away the many investors who have approached it. Peter
Chernin, whose Chernin Group acquired a reported 51 percent stake in the
media company Barstool Sports last year, was one of them. ``I think it's
more unusual than standard to turn down investors,'' Chernin told me,
``but it's been very smart on their part.''

Chernin was the president of Rupert Murdoch's News Corporation when the
company enlisted Roger Ailes to get Fox News off the ground in the late
1990s, and he sees some parallels between the conservative cable channel
and Crooked Media. ``This was true of Roger: It's not just a business
for these Crooked Media guys, it's a calling,'' he told me. ``The real
execution challenge is about authenticity. Does it feel authentic to the
audience? They certainly have that going for them.''

Few things are as inauthenticity-prone, however, as the political-pundit
business. On a Saturday morning in July, ``Pod Save America'' traveled
to the Pasadena Convention Center for Politicon, a two-day event that
has been hailed as the ``Comic-Con of politics,'' in which several
thousand political junkies pay \$80 apiece for the opportunity to see
their favorite cable-news talking heads in the flesh. When the Politicon
organizers first approached them about appearing, the ``Pod Save
America'' hosts recoiled at the idea --- ``Some of these people are
despicable,'' Lovett complained to the organizers about the other
invitees --- but they eventually reconsidered. After all, they had a
brand to promote.

As they stood at the threshold of the Politicon greenroom, the ``Pod
Save America'' hosts looked like patients about to go into surgery.
``I'd rather stay out here as long as possible,'' Vietor whispered.
Inside, Ann Coulter --- flanked by a couple of cops who were providing
security --- marked her territory, while the Coulter wannabe Tomi Lahren
paraded around with a camera crew in tow. In one corner, the Republican
rogue Roger Stone held court. In another, Chris Cillizza of CNN
dispensed conventional wisdom. The Crooked Media guys mostly talked
among themselves.

Then Vietor and Lovett spotted Bill Kristol, the founding editor of The
Weekly Standard and a neoconservative boogeyman to liberals during the
George W. Bush presidency, who emerged as one of the most forthright
conservative critics of Trump in 2016. They introduced themselves and
fell into conversation about the 2008 election. ``You guys had a good
team,'' Kristol said. ``It seems like another era: Hillary and Obama
debating the intricacies of whether you could do health care reform
without a mandate or with a mandate.''

``The primary was about whether the I.R.G.C.'' --- Iran's Islamic
Revolutionary Guard Corps --- ``should be a terrorist group!'' Vietor
marveled.

Soon, their discussion turned to the Trump administration. ``It's so
unbelievable that these guys are trying to run anything,'' Kristol
lamented. ``This level of total insanity is terrible.''

``You know what?'' Lovett told Kristol. ``I like you better lately. It's
like we're together to fight the aliens.''

``It's like we're all defending the Earth!'' Kristol said.

``But at some point the aliens will leave,'' Lovett reminded him. ``And
then we'll just be sitting at the same table being like, `Oh, right, we
hate each other.' ''

Advertisement

\protect\hyperlink{after-bottom}{Continue reading the main story}

\hypertarget{site-index}{%
\subsection{Site Index}\label{site-index}}

\hypertarget{site-information-navigation}{%
\subsection{Site Information
Navigation}\label{site-information-navigation}}

\begin{itemize}
\tightlist
\item
  \href{https://help.nytimes3xbfgragh.onion/hc/en-us/articles/115014792127-Copyright-notice}{©~2020~The
  New York Times Company}
\end{itemize}

\begin{itemize}
\tightlist
\item
  \href{https://www.nytco.com/}{NYTCo}
\item
  \href{https://help.nytimes3xbfgragh.onion/hc/en-us/articles/115015385887-Contact-Us}{Contact
  Us}
\item
  \href{https://www.nytco.com/careers/}{Work with us}
\item
  \href{https://nytmediakit.com/}{Advertise}
\item
  \href{http://www.tbrandstudio.com/}{T Brand Studio}
\item
  \href{https://www.nytimes3xbfgragh.onion/privacy/cookie-policy\#how-do-i-manage-trackers}{Your
  Ad Choices}
\item
  \href{https://www.nytimes3xbfgragh.onion/privacy}{Privacy}
\item
  \href{https://help.nytimes3xbfgragh.onion/hc/en-us/articles/115014893428-Terms-of-service}{Terms
  of Service}
\item
  \href{https://help.nytimes3xbfgragh.onion/hc/en-us/articles/115014893968-Terms-of-sale}{Terms
  of Sale}
\item
  \href{https://spiderbites.nytimes3xbfgragh.onion}{Site Map}
\item
  \href{https://help.nytimes3xbfgragh.onion/hc/en-us}{Help}
\item
  \href{https://www.nytimes3xbfgragh.onion/subscription?campaignId=37WXW}{Subscriptions}
\end{itemize}
