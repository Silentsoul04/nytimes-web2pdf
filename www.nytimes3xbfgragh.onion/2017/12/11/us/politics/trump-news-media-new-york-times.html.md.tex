Sections

SEARCH

\protect\hyperlink{site-content}{Skip to
content}\protect\hyperlink{site-index}{Skip to site index}

\href{https://www.nytimes3xbfgragh.onion/section/politics}{Politics}

\href{https://myaccount.nytimes3xbfgragh.onion/auth/login?response_type=cookie\&client_id=vi}{}

\href{https://www.nytimes3xbfgragh.onion/section/todayspaper}{Today's
Paper}

\href{/section/politics}{Politics}\textbar{}Trump Escalates His
Criticism of the News Media, Fueling National Debate

\url{https://nyti.ms/2jNATBl}

\begin{itemize}
\item
\item
\item
\item
\item
\end{itemize}

Advertisement

\protect\hyperlink{after-top}{Continue reading the main story}

Supported by

\protect\hyperlink{after-sponsor}{Continue reading the main story}

\hypertarget{trump-escalates-his-criticism-of-the-news-media-fueling-national-debate}{%
\section{Trump Escalates His Criticism of the News Media, Fueling
National
Debate}\label{trump-escalates-his-criticism-of-the-news-media-fueling-national-debate}}

\includegraphics{https://static01.graylady3jvrrxbe.onion/images/2017/12/12/us/politics/12dc-trump-1/12dc-trump-1-articleLarge.jpg?quality=75\&auto=webp\&disable=upscale}

By \href{http://www.nytimes3xbfgragh.onion/by/peter-baker}{Peter Baker}
and \href{http://www.nytimes3xbfgragh.onion/by/sydney-ember}{Sydney
Ember}

\begin{itemize}
\item
  Dec. 11, 2017
\item
  \begin{itemize}
  \item
  \item
  \item
  \item
  \item
  \end{itemize}
\end{itemize}

WASHINGTON --- President Trump has escalated his fiery attacks on the
news media, seizing on a recent string of mistaken reports to bolster
his case that he is being persecuted by a left-leaning establishment out
to bring him down and fueling a national debate over truth,
accountability and a free press.

In a series of broadsides reflecting his own profound grievances while
also resonating with his populist conservative base, Mr. Trump
castigated ABC News for a
\href{https://twitter.com/realdonaldtrump/status/937145025359761408?lang=en}{``horrendously
inaccurate and dishonest report,''} declared that CNN's slogan should be
``\href{https://twitter.com/realDonaldTrump/status/939485131693322240}{THE
LEAST TRUSTED NAME IN NEWS}'' and insisted that a Washington Post
reporter
\href{https://twitter.com/realDonaldTrump/status/939634404267380736}{``should
be fired}.''

While every president has groused about his coverage, Mr. Trump has
proved to be the most vocal and visceral news media critic in the Oval
Office in at least a generation. In recent days, news outlets have
provided him ammunition with reporting errors. But the barrage has
deepened concern among media executives about what they see as a
concerted campaign to discredit independent journalism.

The tension erupted Monday in the White House briefing room when Sarah
Huckabee Sanders, the president's press secretary, engaged in a testy
exchange with reporters. She said bias in news reports has gotten ``out
of control'' and ``should be taken seriously,'' dismissing journalists
who attributed recent errors to honest mistakes that were corrected.

``You cannot say that it's an honest mistake when you're purposefully
putting out information that you know to be false,'' she told reporters,
``or when you're taking information that hasn't been validated, that
hasn't been offered any credibility, and that has been continually
denied by a number of people, including people with direct knowledge of
an instance.''

``Are you speaking about the president?'' one reporter asked in light of
Mr. Trump's own history of making claims that have not been validated
and have been continually denied.

While news organizations targeted by Mr. Trump have corrected factual
errors and in one recent case suspended a reporter, Mr. Trump has never
backed down, for instance, from unsubstantiated claims that President
\href{https://www.nytimes3xbfgragh.onion/topic/person/barack-obama}{Barack
Obama} wiretapped
\href{https://www.nytimes3xbfgragh.onion/2017/03/04/us/politics/trump-obama-tap-phones.html}{Trump
Tower} or that millions of illegal immigrants
\href{https://www.nytimes3xbfgragh.onion/2017/01/23/us/politics/donald-trump-congress-democrats.html}{cast
votes} last year, swinging the popular tally against him.

Just last month, Mr. Trump shared with nearly 44 million Twitter
followers
\href{https://www.nytimes3xbfgragh.onion/2017/11/29/us/politics/trump-anti-muslim-videos-jayda-fransen.html}{anti-Muslim
videos} without verifying them; one purported to show a Dutch boy being
beaten by a ``Muslim migrant'' who in fact was not a Muslim migrant. Mr.
Trump issued no correction, and Ms. Sanders at the time said it did not
matter if the video were real because ``the threat is real.''

Mr. Trump has had a long history of interacting with the news media from
his days as a New York developer, but it was largely transactional as he
sought favorable coverage. During his business days, he planted items in
gossip columns and even called reporters and used a false name,
pretending to be his spokesman.

By his own account, he has always craved media attention, but his
encounters with the Washington press corps have turned increasingly
bitter. At one point, Mr. Trump labeled some outlets
\href{https://www.nytimes3xbfgragh.onion/2017/02/17/business/trump-calls-the-news-media-the-enemy-of-the-people.html}{``the
enemy of the American people},'' and at another, he said, ``It's frankly
\href{http://www.cnn.com/videos/politics/2017/10/11/donald-trump-nbc-media-write-whatever-it-wants-sot.cnn}{disgusting}
the way the press is able to write whatever they want to write.''

For the most part, Mr. Trump's blasts at the news media have been
rhetorical. But he has warned that he might go beyond name calling, such
as when
\href{https://www.nytimes3xbfgragh.onion/2017/10/11/us/politics/trump-nbc-fcc-broadcast-license.html}{he
threatened} to try to revoke a broadcasting license for NBC after a
\href{https://www.nbcnews.com/news/all/trump-wanted-dramatic-increase-nuclear-arsenal-meeting-military-leaders-n809701}{news
report} he challenged. Networks like NBC do not hold federal licenses
themselves, but their individual television stations do.

Mr. Trump's administration has taken aim at CNN's corporate ownership.
The Justice Department has
\href{https://www.nytimes3xbfgragh.onion/2017/11/20/business/dealbook/att-time-warner-merger.html}{gone
to court} to block AT\&T's purchase of Time Warner, which owns Turner
Broadcasting, CNN's corporate parent. AT\&T officials have said the
department insisted that AT\&T divest either Turner Broadcasting or its
valuable DirecTV service in return for approval, which government
officials denied.

Analysts said Mr. Trump's criticisms represented an effort to undermine
faith in journalism. ``It is a common thing in the authoritarian
playbook to discredit the media so that they are the only source that
can be trusted,'' said Indira Lakshmanan, who holds the Newmark chair in
journalism ethics at the Poynter Institute. ``Making it so there is no
objective truth is the most dangerous thing of all of this.''

\includegraphics{https://static01.graylady3jvrrxbe.onion/images/2017/12/12/us/politics/12dc-trump2/merlin_131168174_6747d637-2e12-4b74-87ee-b6096ab2a0da-articleLarge.jpg?quality=75\&auto=webp\&disable=upscale}

The raft of recent reporting errors made that easier. During a rally on
Friday in Florida, Mr. Trump berated news outlets. ``Did you see all the
corrections the media has been making?'' he asked. ``They've been
apologizing left and right.''

He ramped up attacks over the weekend on Twitter, calling for the firing
of
\href{https://twitter.com/realDonaldTrump/status/939480342779580416}{``lonely
Brian Ross at ABC News''} for
\href{https://www.nytimes3xbfgragh.onion/2017/12/02/us/brian-ross-suspended-abc.html}{misreporting}
that Mr. Trump as a candidate directed his adviser Michael T. Flynn to
contact Russians. (He did so after the election as president-elect.) He
denounced CNN for
\href{https://www.nytimes3xbfgragh.onion/2017/12/08/business/media/cnn-correction-donald-trump-jr.html}{erroneously
reporting} that his campaign received a heads-up email from WikiLeaks
before it released hacked Democratic Party documents. (The email came
after the release.)

Mr. Trump later targeted Dave Weigel, a reporter for The Post, who
tweeted a
\href{https://www.nytimes3xbfgragh.onion/2017/12/10/us/politics/trump-dave-weigel.html}{misleading
photograph} about the crowd size at Mr. Trump's Florida rally. ``Demand
apology \& retraction from FAKE NEWS WaPo!''
\href{https://twitter.com/realdonaldtrump/status/939616077356642304}{Mr.
Trump wrote}.
\href{https://twitter.com/daveweigel/status/939616810684514304}{Mr.
Weigel did just that, deleting the picture and expressing regret},
saying he did not realize it was taken before the rally started.
Unsatisfied, Mr. Trump said Mr. Weigel should be fired.

Mr. Trump's supporters said that he has a point --- that such mistakes
stem from a predisposition by reporters to believe the worst about him.
Rather than complain when Mr. Trump points out flawed stories, they
said, the news media should be more searching about its responsibility
to provide balanced coverage of the president.

``Naturally, the elite media responded --- not by admonishing Weigel
over his inexcusably inaccurate trolling --- but with their favorite
claim that Trump is the one man in America who does not have the First
Amendment right to criticize the media,''
\href{http://www.breitbart.com/big-journalism/2017/12/10/trump-blasts-wapos-dave-weigel-reporter-heckles-fake-photo/}{wrote
John Nolte} of Breitbart News, the site run by Mr. Trump's former chief
strategist, Stephen K. Bannon.

Mr. Trump turned his attention on Monday to The New York Times,
disputing
\href{https://www.nytimes3xbfgragh.onion/2017/12/09/us/politics/donald-trump-president.html}{an
article} describing his television habits. ``Another false story, this
time in the Failing @nytimes, that I watch 4-8 hours of television a
day,''
\href{https://twitter.com/realDonaldTrump/status/940223974985871360}{he
wrote}. ``Wrong! Also, I seldom, if ever, watch CNN or MSNBC, both of
which I consider Fake News. I never watch Don Lemon, who I once called
the `dumbest man on television!' Bad Reporting.''

Mr. Trump posted the message about a half-hour after a segment on
MSNBC's ``Morning Joe'' focusing on a line near the bottom of the
article reporting that Mr. Trump sometimes watched Mr. Lemon on CNN to
get worked up.

``We stand by our reporting, sourced from interviews with 60 advisers,
associates, friends and members of Congress, including many who interact
with President Trump every day,'' said Danielle Rhoades Ha, a
spokeswoman for The Times.

CNN issued a statement likening Mr. Trump to a bully. ``In a world where
bullies torment kids on social media to devastating effect on a regular
basis with insults and name calling, it is sad to see our president
engaging in the very same behavior himself,'' the statement said.
``Leaders should lead by example.''

The effect of the news media criticism remains debated. A
\href{https://www.politico.com/story/2017/10/18/trump-media-fake-news-poll-243884}{Politico/Morning
Consult} poll found that 46 percent of Americans believe the news media
makes up stories about the president. A separate study by the Poynter
Institute, on the other hand, found that overall trust and confidence in
the news media, while low, has actually increased somewhat under Mr.
Trump. The partisan divide, however, has become pronounced. Among
Democrats, nearly 75 percent expressed confidence in the news media
compared with only 19 percent of Republicans.

While previous presidents criticized the news media, analysts said that
Mr. Trump's attacks strike at the fundamental notion of truth in a way
that can make the customary response of news organizations standing by
their articles feel insufficient. It is not just the facts he is calling
into question, but the very institution of journalism, which some
believe demands a more vigorous reply from the mainstream media.

``When he attacks one of us, he's actually attacking all of us,'' said
Dean Baquet, the executive editor of The Times. ``We're now confronting
a guy who attacks us in ways that we've not seen before,'' he added.
``And I think maybe we should be thinking about ways to push back not
just on behalf of our particular institutions but to push back on behalf
of journalism itself.''

But a common response seems unlikely. ``Journalistic organizations are
by nature competitive, and it's sort of hard for them to unite that
way,'' said David Lauter, the Washington bureau chief for The Los
Angeles Times. ``And I think there's also always concern about making it
seem like you're creating an institutional fight rather than just
reporting the news.''

Cameron Barr, a managing editor at The Post, said competition is ``in
the DNA'' of news organizations. ``I'm a little wary of suggestions that
somehow journalistic institutions should be banding together against the
chief executive,'' he said. ``I think that quickly ends up in an
uncomfortable place, to say the least.''

Advertisement

\protect\hyperlink{after-bottom}{Continue reading the main story}

\hypertarget{site-index}{%
\subsection{Site Index}\label{site-index}}

\hypertarget{site-information-navigation}{%
\subsection{Site Information
Navigation}\label{site-information-navigation}}

\begin{itemize}
\tightlist
\item
  \href{https://help.nytimes3xbfgragh.onion/hc/en-us/articles/115014792127-Copyright-notice}{©~2020~The
  New York Times Company}
\end{itemize}

\begin{itemize}
\tightlist
\item
  \href{https://www.nytco.com/}{NYTCo}
\item
  \href{https://help.nytimes3xbfgragh.onion/hc/en-us/articles/115015385887-Contact-Us}{Contact
  Us}
\item
  \href{https://www.nytco.com/careers/}{Work with us}
\item
  \href{https://nytmediakit.com/}{Advertise}
\item
  \href{http://www.tbrandstudio.com/}{T Brand Studio}
\item
  \href{https://www.nytimes3xbfgragh.onion/privacy/cookie-policy\#how-do-i-manage-trackers}{Your
  Ad Choices}
\item
  \href{https://www.nytimes3xbfgragh.onion/privacy}{Privacy}
\item
  \href{https://help.nytimes3xbfgragh.onion/hc/en-us/articles/115014893428-Terms-of-service}{Terms
  of Service}
\item
  \href{https://help.nytimes3xbfgragh.onion/hc/en-us/articles/115014893968-Terms-of-sale}{Terms
  of Sale}
\item
  \href{https://spiderbites.nytimes3xbfgragh.onion}{Site Map}
\item
  \href{https://help.nytimes3xbfgragh.onion/hc/en-us}{Help}
\item
  \href{https://www.nytimes3xbfgragh.onion/subscription?campaignId=37WXW}{Subscriptions}
\end{itemize}
