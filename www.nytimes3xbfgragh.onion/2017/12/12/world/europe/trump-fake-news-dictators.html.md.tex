Sections

SEARCH

\protect\hyperlink{site-content}{Skip to
content}\protect\hyperlink{site-index}{Skip to site index}

\href{https://www.nytimes3xbfgragh.onion/section/world/europe}{Europe}

\href{https://myaccount.nytimes3xbfgragh.onion/auth/login?response_type=cookie\&client_id=vi}{}

\href{https://www.nytimes3xbfgragh.onion/section/todayspaper}{Today's
Paper}

\href{/section/world/europe}{Europe}\textbar{}`Fake News,' Trump's
Obsession, Is Now a Cudgel for Strongmen

\url{https://nyti.ms/2C7CEkq}

\begin{itemize}
\item
\item
\item
\item
\item
\item
\end{itemize}

Advertisement

\protect\hyperlink{after-top}{Continue reading the main story}

Supported by

\protect\hyperlink{after-sponsor}{Continue reading the main story}

\hypertarget{fake-news-trumps-obsession-is-now-a-cudgel-for-strongmen}{%
\section{`Fake News,' Trump's Obsession, Is Now a Cudgel for
Strongmen}\label{fake-news-trumps-obsession-is-now-a-cudgel-for-strongmen}}

\includegraphics{https://static01.graylady3jvrrxbe.onion/images/2017/12/13/world/13fakenews1/14fakenews1-videoSixteenByNine3000.jpg}

By \href{http://www.nytimes3xbfgragh.onion/by/steven-erlanger}{Steven
Erlanger}

\begin{itemize}
\item
  Dec. 12, 2017
\item
  \begin{itemize}
  \item
  \item
  \item
  \item
  \item
  \item
  \end{itemize}
\end{itemize}

\href{http://cn.nytimes3xbfgragh.onion/world/20171213/trump-fake-news-dictators/}{阅读简体中文版}\href{http://cn.nytimes3xbfgragh.onion/world/20171213/trump-fake-news-dictators/zh-hant/}{閱讀繁體中文版}\href{https://www.nytimes3xbfgragh.onion/es/2017/12/14/los-politicos-autoritarios-adoptan-la-excusa-de-noticias-falsas-de-trump}{Leer
en español}

BRUSSELS --- President Trump routinely invokes the phrase ``fake news''
as a rhetorical tool to undermine opponents, rally his political base
and
\href{https://www.nytimes3xbfgragh.onion/2017/12/11/us/politics/trump-news-media-new-york-times.html}{try
to discredit} a mainstream American media that is aggressively
investigating his presidency.

But he isn't the only leader enamored with the phrase. Following Mr.
Trump's example, many of the world's autocrats and dictators are taking
a shine to it, too.

When Amnesty International released a report about
\href{https://www.amnesty.org/en/press-releases/2017/02/syria-investigation-uncovers-governments-secret-campaign-of-mass-hangings-and-extermination-at-saydnaya-prison/}{prison
deaths in Syria}, the Syrian president, Bashar al-Assad, retorted that
``we are living in a fake-news era.'' President Nicolás Maduro of
Venezuela, who is steadily
\href{https://www.nytimes3xbfgragh.onion/2017/07/29/world/americas/venezuela-nicolas-maduro-constituent-assembly.html}{rolling
back democracy} in his country, blamed the global media for ``lots of
false versions, lots of lies,'' saying ``this is what we call `fake
news' today.''

In Myanmar, where international observers accuse the military of
conducting a genocidal campaign against the Rohingya Muslims, a security
official
\href{https://www.nytimes3xbfgragh.onion/2017/12/02/world/asia/myanmar-rohingya-denial-history.html}{told
The New York Times} that ``there is no such thing as Rohingya,'' adding:
``It is fake news.'' In Russia, a Foreign Ministry spokeswoman, Maria
Zakharova,
\href{https://www.nytimes3xbfgragh.onion/2017/02/22/world/europe/russia-fake-news-media-foreign-ministry-.html?_r=0}{told
a CNN}reporter to ``stop spreading lies and fake news.'' Her ministry
now uses
\href{https://www.nytimes3xbfgragh.onion/2017/02/22/world/europe/russia-fake-news-media-foreign-ministry-.html?_r=0}{a
big red stamp, ``FAKE,''} on its website to label news stories it
dislikes.

Around the world, authoritarians, populists and other political leaders
have seized on the phrase ``fake news'' --- and the legitimacy conferred
upon it by an American president --- as a tool for attacking their
critics and, in some cases, deliberately undermining the institutions of
democracy. In countries where press freedom is restricted or under
considerable threat --- including Russia, China, Turkey, Libya, Poland,
Hungary, Thailand, Somalia and others --- political leaders have invoked
``fake news'' as justification for beating back media scrutiny.

Just this week, the official newspaper of the Chinese Communist Party,
\href{http://en.people.cn/mobile/new/content.html?cI=1002\&nI=9302932\&aT=m}{People's
Daily}, used Mr. Trump's words to undercut critical media coverage of an
increasingly authoritarian Beijing.

\includegraphics{https://static01.graylady3jvrrxbe.onion/images/2017/12/14/world/14fakenews2/14fakenews2-articleLarge.jpg?quality=75\&auto=webp\&disable=upscale}

``If the president of the United States claims that his nation's leading
media outlets are a stain on America,'' the paper wrote, ``then negative
news about China and other countries should be taken with a grain of
salt, since it is likely that bias and political agendas are distorting
the real picture.''

Not quite a year into his presidency, Mr. Trump has shaken the global
status quo, with his ``America First'' ethos, his disdain for global
trade and multilateral treaties, and his testy relationships with many
traditional allies (and seemingly warm embrace of many traditional
rivals). But the president's mantra of ``fake news'' stirs different
concerns among many foreign politicians and analysts, who fear it erodes
public confidence in democratic institutions at a time when
\href{https://www.nytimes3xbfgragh.onion/2016/11/29/world/americas/western-liberal-democracy.html}{populism
and authoritarianism are returning} in many regions.

``Trump doesn't only talk about fake news, but attacks the media as fake
news, and that's an attack on the free press,'' said
\href{http://www.europarl.europa.eu/meps/en/96945/Marietje_SCHAAKE_home.html}{Marietje
Schaake}, a Dutch member of the European Parliament who focuses on human
rights and the digital landscape. ``As the leader of a country that
traditionally defends human rights, that's very serious, and of course
it has a major impact worldwide.''

Richard Javad Heydarian, a political scientist at De La Salle University
in Manila and the author of a book on President Rodrigo Duterte of the
Philippines, said that American soft power, long rooted in advocacy of
democracy, was ``in a state of total collapse,'' allowing strongman
leaders like Mr. Duterte greater leeway to ignore democratic norms.

``With Trump in power, no one is talking about human rights, only fake
news, and that's great for Duterte,'' he said. ``They both see
themselves as populists facing a conspiracy of liberal elites. They
think they are victims of fake news.''

Though the term ``fake news'' has been around at least since the 1890s,
\href{https://www.merriam-webster.com/words-at-play/the-real-story-of-fake-news}{according
to Merriam-Webster}, Mr. Trump is most responsible for making it a big
part of the current global conversation. Now it is so common that
Collins Dictionary decided to make it this year's
\href{http://www.independent.co.uk/news/uk/home-news/fake-news-word-of-the-year-2017-collins-dictionary-donald-trump-kellyanne-conway-antifa-corbynmania-a8032751.html}{``word
of the year,''} finding in early November that the use of the term had
risen by 365 percent since 2016.

Image

President Trump with President Rodrigo Duterte of the Philippines at the
31st Asean summit meeting in Manila last month.Credit...Doug Mills/The
New York Times

Helen Newstead, Collins's head of language content, said that `` `Fake
news,' either as a statement of fact or as an accusation, has been
inescapable this year, contributing to the undermining of society's
trust in news reporting.''

The problem, of course, is that fake news is a real problem, especially
on social media. United States intelligence agencies have concluded that
Russia used fake news reports as part of an effort to interfere in the
2016 presidential election on behalf of Mr. Trump. The presence of fake
news in the globalized stream of media content helps blur the line with
traditional, fact-based news.

How much the fake-news epithet has damaged journalism, however, is
difficult to say, given the pre-existing difficulties of doing
untrammeled reporting in countries where the media is already under the
thumb of the state and where journalists have been murdered or
imprisoned, not simply insulted or mocked. But there is little question
that social media, with its huge reach and its vulnerability to bots and
manipulation, has helped to
\href{https://www.nytimes3xbfgragh.onion/2016/12/25/us/politics/fake-news-claims-conservatives-mainstream-media-.html}{amplify
criticism} from political leaders and undermine trust in traditional
journalism.

``Trump has succeeded in building an alternative reality separate from
the mainstream media's efforts at democratic, rational politics,'' said
\href{https://reutersinstitute.politics.ox.ac.uk/people/john-lloyd}{John
Lloyd}, a senior research fellow at the
\href{https://reutersinstitute.politics.ox.ac.uk/}{Reuters Institute for
the Study of Journalism}at the University of Oxford. ``Of course
journalists make mistakes, but those errors are amplified by the charge
of `fake news,' '' he said. ``The mainstream media is portrayed as the
tool of an arrogant, out-of-touch elite, who use that tool to keep down
the marginalized.''

The fake-news narrative also complicates the work of democracy advocates
in countries where democracy is already under assault. Kenneth Roth, the
executive director of Human Rights Watch, said that ``the sad irony is
that Trump's greatest harm to human rights may not be his infatuation
with abusive strongmen but his undermining of the fact-based discourse
that is essential for reining them in.''

He added: ``In countries where the judicial system is unable or
unwilling to enforce rights --- most countries --- the human rights
movement's main tool is to investigate and publicize official
misconduct. Autocrats go to great lengths to avoid that shaming, because
it tends to delegitimize them before their public and their peers.''

Image

Inside the newsroom of The Cambodia Daily in Phnom Penh in August. Prime
Minister Hun Sen shut down the newspaper and jailed journalists; his
government also banned the opposition party.Credit...Omar Havana for The
New York Times

Some analysts say Mr. Trump's success at creating an alternative reality
and disparaging an adversarial media both copies and augments the
tactics of Russia's president, Vladimir V. Putin, noting that Mr.
Putin's propagandists ``create a barrage of fake facts'' on politically
sensitive topics such as the conflict in Ukraine in order to sow public
cynicism and uncertainty. Russia and China also create ``positive'' fake
news on social media to inspire patriotism at home.

``People accept these versions or are confused by them, unclear as to
what is correct,'' said Mr. Lloyd, author of ``The Power and the Story:
The Global Battle for News and Information.'' ``Putin above all has
grasped this and uses it against his enemies. The concept of `fake news'
is used to tar any uncomfortable fact.''

Other governments have also embraced the phrase, especially to attack
media outlets that Mr. Trump constantly disparages. One glaring example
came in Libya, after
\href{https://www.nytimes3xbfgragh.onion/2017/11/19/world/africa/libya-migrants-slavery.html}{CNN
aired video} showing a migrant being auctioned as a slave. Libyan
leaders responded by using Mr. Trump's attacks against CNN to try to
cast doubt on the network's report.

Prime Minister Hun Sen of Cambodia, who was put in charge of the
occupied country by the Vietnamese Army more than 30 years ago, shut
down
\href{https://www.nytimes3xbfgragh.onion/2017/09/03/world/asia/cambodia-daily-newspaper.html}{The
Cambodia Daily} and jailed journalists and recently
\href{https://www.nytimes3xbfgragh.onion/2017/11/16/world/asia/cambodia-court-opposition.html}{banned
the opposition party}. Now he also has focused his attacks on Western
media for writing about issues from corruption and repression to sex
trafficking. ``I would like to send a message to the president that your
attack on CNN is right,'' Mr. Hun Sen said in August. ``American media
is very bad.''

Prime Minister Najib Razak of Malaysia, embroiled in a scandal in which
billions of dollars disappeared from the state investment fund,
repeatedly calls accusations against him ``fake news,'' including what
he called ``a well-known foreign newspaper,'' presumably a reference to
The Wall Street Journal, which has reported on
th\href{http://www.wsj.com/specialcoverage/malaysia-controversy}{e
disappearance of the funds}. Mr. Trump once called Mr. Najib his
``favorite prime minister.'' He also has hailed his ``great
relationship'' with Mr. Duterte, the Philippine president, who has
blamed ``fake news'' for coverage of his war on drug traffickers, which
has killed thousands of Filipinos, many without trial.

Many media organizations are now introducing features to verify facts
for readers. In France, Le Monde's
\href{http://www.lemonde.fr/verification/}{Décodex}was launched in
January as part of the fact-checking section of its website. In Britain,
the BBC is
\href{http://www.bbc.com/news/entertainment-arts-42242630}{starting a
project} to help secondary school pupils to identify real news and
filter out fake or false information.

But it is a different matter when the president of the United States is
the source, Ms. Schaake said. ``There is significant damage to the
credibility of the United States as the defender of human rights and
democratic principles, of which press freedom is one of the pillars,''
she said.

Advertisement

\protect\hyperlink{after-bottom}{Continue reading the main story}

\hypertarget{site-index}{%
\subsection{Site Index}\label{site-index}}

\hypertarget{site-information-navigation}{%
\subsection{Site Information
Navigation}\label{site-information-navigation}}

\begin{itemize}
\tightlist
\item
  \href{https://help.nytimes3xbfgragh.onion/hc/en-us/articles/115014792127-Copyright-notice}{©~2020~The
  New York Times Company}
\end{itemize}

\begin{itemize}
\tightlist
\item
  \href{https://www.nytco.com/}{NYTCo}
\item
  \href{https://help.nytimes3xbfgragh.onion/hc/en-us/articles/115015385887-Contact-Us}{Contact
  Us}
\item
  \href{https://www.nytco.com/careers/}{Work with us}
\item
  \href{https://nytmediakit.com/}{Advertise}
\item
  \href{http://www.tbrandstudio.com/}{T Brand Studio}
\item
  \href{https://www.nytimes3xbfgragh.onion/privacy/cookie-policy\#how-do-i-manage-trackers}{Your
  Ad Choices}
\item
  \href{https://www.nytimes3xbfgragh.onion/privacy}{Privacy}
\item
  \href{https://help.nytimes3xbfgragh.onion/hc/en-us/articles/115014893428-Terms-of-service}{Terms
  of Service}
\item
  \href{https://help.nytimes3xbfgragh.onion/hc/en-us/articles/115014893968-Terms-of-sale}{Terms
  of Sale}
\item
  \href{https://spiderbites.nytimes3xbfgragh.onion}{Site Map}
\item
  \href{https://help.nytimes3xbfgragh.onion/hc/en-us}{Help}
\item
  \href{https://www.nytimes3xbfgragh.onion/subscription?campaignId=37WXW}{Subscriptions}
\end{itemize}
