Sections

SEARCH

\protect\hyperlink{site-content}{Skip to
content}\protect\hyperlink{site-index}{Skip to site index}

\href{https://myaccount.nytimes3xbfgragh.onion/auth/login?response_type=cookie\&client_id=vi}{}

\href{https://www.nytimes3xbfgragh.onion/section/todayspaper}{Today's
Paper}

A Pastry Fit for a King --- or a Queen

\url{https://nyti.ms/2oXpNi0}

\begin{itemize}
\item
\item
\item
\item
\item
\end{itemize}

Advertisement

\protect\hyperlink{after-top}{Continue reading the main story}

Supported by

\protect\hyperlink{after-sponsor}{Continue reading the main story}

\href{/column/on-dessert}{On Dessert}

\hypertarget{a-pastry-fit-for-a-king--or-a-queen}{%
\section{A Pastry Fit for a King --- or a
Queen}\label{a-pastry-fit-for-a-king--or-a-queen}}

\includegraphics{https://static01.graylady3jvrrxbe.onion/images/2017/12/24/magazine/24mag-ondessert1/24mag-24ondessert-t_CA0-articleLarge.jpg?quality=75\&auto=webp\&disable=upscale}

By Dorie Greenspan

\begin{itemize}
\item
  Dec. 20, 2017
\item
  \begin{itemize}
  \item
  \item
  \item
  \item
  \item
  \end{itemize}
\end{itemize}

Winter in Paris always surprises me. From one January to the next, I
forget how very short the days are, how damp the air is, how often it
seems we're on the brink of snow but how seldom it arrives, how rare a
day of sunshine is and then how weak the light can be. But January
brings compensatory pleasures: hot chocolate, \emph{vin chaud} (a mix of
mulled wine, brandy and orange), roasted chestnuts bought on the street,
freshly griddled crepes rolled into paper cones and the \emph{galette
des rois,} a pastry sold only during this gray month.

``Galette'' is a word that can still trip me up in translation: It can
be a pancake, a buckwheat crepe, a chubby cookie or a double-crusted
pastry, which describes the \emph{galette des rois}. Created to
celebrate Epiphany --- the day the Three Kings (\emph{les rois}) brought
gifts to the infant Jesus --- the galette is beloved throughout northern
France (in the south, they make a brioche cake, a \emph{gâteau des rois}
that resembles a New Orleans king cake).

The galette has two components: a pair of puff-pastry circles and a
frangipani-type filling. Although pastry chefs have taken to creating
fillings with fruits (dried or roasted), chocolate, rose, coconut,
citrus and various nuts, the form remains true to tradition. The edges
of the galette are scalloped, the better to show the dramatic rise of
the pastry and to seal in the velvety almond cream that is the perfect
counterpoint to the flaky crusts. (If a galette doesn't shatter into
hundreds of buttery shards, then a measure of its character is lost.)
The top, baked to a burnished matte mahogany, is etched in a spare
pattern with the tip of a knife --- no icing, no frosting, no frippery
or frills. In the world of prettily decorated pastries, the galette is
as plain as toast.

In the years when I was a frequent tourist in Paris, I wasn't much
interested in galettes. I was even miffed that for weeks these large
pastries overwhelmed patisserie windows and pushed aside most other
specialties, the ones that visitors could eat on the run or savor in a
hotel room. It was only when I moved to Paris and had a home and friends
of my own that I came to appreciate it and to understand one reason
locals love it: It's as much a party game as a pastry. Like a birthday
cake, it's an invitation to gather and celebrate.

Every galette comes with a crown --- some from the best patisseries are
intricately designed, but most are made of gold cardboard --- and a
charm. In earlier times, the charm was a dried bean, a \emph{fève} ---
it's still the name for the trinket, even though, as with the crowns,
today's \emph{fèves}, made of porcelain, can be quite elaborate. From
Epiphanies past, I've saved \emph{fèves} shaped like macarons (one was
black --- so chic and so unusual), sheaves of wheat, hearts and stylized
beans. If the charm is hidden in your portion, you get the crown and the
title of king or queen for the day. How that portion becomes yours is
always a matter of luck: Many French families have the youngest person
in the room get under the table and call out who should get the next
slice of galette.

My friend Simon Maurel, now 40, is still the baby in his family, and so
each year he folds his six-foot frame as gracefully as he can and crawls
under his parents' dining-room table to proclaim the order in which all
will be served the first galette of the season. If you live in France,
it's almost guaranteed that a galette will be a weekly (if not more
frequent) indulgence --- you're bound to have one at school, in the
office, at home, at a neighbor's, a friend's or even the local wine
shop. (Champagne and sparkling cider are excellent accompaniments.)

I always have a galette get-together at my apartment in Paris, and when
I return to America, I do it again, this time having baked my own
galette. Because it uses puff pastry, which rises spectacularly almost
no matter what you do, the dessert is automatically striking. And
because it's more craft than art, talent or skill, it's a project of
parts: The almond filling is made ahead, the pastry circles cut ahead,
the galette assembled and then chilled again. The steps fit nicely
between bouts with a tough crossword puzzle or getting the place ready
for guests. A whole almond makes a good \emph{fève,} but because I have
the charms I've pocketed, I'll use one of them. Sometimes more --- I
consider it a baker's prerogative to stack the deck. I disregard
tradition for the delight of seeing my friends bent over the galette,
guessing which slice might hold the bean, betting on their picks,
moaning theatrically when their slice is bare or brandishing the coveted
\emph{fève} in triumph.

\textbf{Recipe:}
\href{https://cooking.nytimes3xbfgragh.onion/recipes/1019114-galette-des-rois}{Galette
des Rois}

Advertisement

\protect\hyperlink{after-bottom}{Continue reading the main story}

\hypertarget{site-index}{%
\subsection{Site Index}\label{site-index}}

\hypertarget{site-information-navigation}{%
\subsection{Site Information
Navigation}\label{site-information-navigation}}

\begin{itemize}
\tightlist
\item
  \href{https://help.nytimes3xbfgragh.onion/hc/en-us/articles/115014792127-Copyright-notice}{©~2020~The
  New York Times Company}
\end{itemize}

\begin{itemize}
\tightlist
\item
  \href{https://www.nytco.com/}{NYTCo}
\item
  \href{https://help.nytimes3xbfgragh.onion/hc/en-us/articles/115015385887-Contact-Us}{Contact
  Us}
\item
  \href{https://www.nytco.com/careers/}{Work with us}
\item
  \href{https://nytmediakit.com/}{Advertise}
\item
  \href{http://www.tbrandstudio.com/}{T Brand Studio}
\item
  \href{https://www.nytimes3xbfgragh.onion/privacy/cookie-policy\#how-do-i-manage-trackers}{Your
  Ad Choices}
\item
  \href{https://www.nytimes3xbfgragh.onion/privacy}{Privacy}
\item
  \href{https://help.nytimes3xbfgragh.onion/hc/en-us/articles/115014893428-Terms-of-service}{Terms
  of Service}
\item
  \href{https://help.nytimes3xbfgragh.onion/hc/en-us/articles/115014893968-Terms-of-sale}{Terms
  of Sale}
\item
  \href{https://spiderbites.nytimes3xbfgragh.onion}{Site Map}
\item
  \href{https://help.nytimes3xbfgragh.onion/hc/en-us}{Help}
\item
  \href{https://www.nytimes3xbfgragh.onion/subscription?campaignId=37WXW}{Subscriptions}
\end{itemize}
