Sections

SEARCH

\protect\hyperlink{site-content}{Skip to
content}\protect\hyperlink{site-index}{Skip to site index}

\href{/section/politics}{Politics}\textbar{}Stoking Fears, Trump Defied
Bureaucracy to Advance Immigration Agenda

\url{https://nyti.ms/2DCJqPP}

\begin{itemize}
\item
\item
\item
\item
\item
\item
\end{itemize}

\includegraphics{https://static01.graylady3jvrrxbe.onion/images/2017/12/20/us/24dc-immigration1/00dc-immigration1-articleLarge.jpg?quality=75\&auto=webp\&disable=upscale}

Trump's Way

\hypertarget{stoking-fears-trump-defied-bureaucracy-to-advance-immigration-agenda}{%
\section{Stoking Fears, Trump Defied Bureaucracy to Advance Immigration
Agenda}\label{stoking-fears-trump-defied-bureaucracy-to-advance-immigration-agenda}}

The changes have had far-reaching consequences, both for the immigrants
who have sought to make a new home in this country and for America's
image in the world.

President Trump at the White House in July. His visceral approach to
immigration defined his campaign and has shaped the first year of his
presidency.Credit...Gabriella Demczuk for The New York Times

Supported by

\protect\hyperlink{after-sponsor}{Continue reading the main story}

By \href{http://www.nytimes3xbfgragh.onion/by/michael-d-shear}{Michael
D. Shear} and
\href{https://www.nytimes3xbfgragh.onion/by/julie-hirschfeld-davis}{Julie
Hirschfeld Davis}

\begin{itemize}
\item
  Dec. 23, 2017
\item
  \begin{itemize}
  \item
  \item
  \item
  \item
  \item
  \item
  \end{itemize}
\end{itemize}

\href{https://www.nytimes3xbfgragh.onion/es/2017/12/27/como-trump-transformo-las-politicas-migratorias-de-estados-unidos}{Leer
en español}

WASHINGTON --- Late to his own meeting and waving a sheet of numbers,
President Trump stormed into the Oval Office one day in June, plainly
enraged.

Five months before, Mr. Trump had dispatched federal officers to the
nation's airports
\href{https://www.nytimes3xbfgragh.onion/2017/01/27/us/politics/trump-syrian-refugees.html}{to
stop travelers from several Muslim countries} from entering the United
States in
\href{https://www.nytimes3xbfgragh.onion/2017/01/28/us/refugees-detained-at-us-airports-prompting-legal-challenges-to-trumps-immigration-order.html}{a
dramatic demonstration} of how he would deliver on his campaign promise
to fortify the nation's borders.

But so many foreigners had flooded into the country since January, he
vented to his national security team, that it was making a mockery of
his pledge. Friends were calling to say he looked like a fool, Mr. Trump
said.

According to six officials who attended or were briefed about the
meeting, Mr. Trump then began reading aloud from the document, which his
domestic policy adviser, Stephen Miller, had given him just before the
meeting. The document listed how many immigrants had received visas to
enter the United States in 2017.

More than 2,500 were from Afghanistan, a terrorist haven, the president
complained.

Haiti had sent 15,000 people. They ``all have AIDS,'' he grumbled,
according to one person who attended the meeting and another person who
was briefed about it by a different person who was there.

Forty thousand had come from Nigeria, Mr. Trump added. Once they had
seen the United States, they would never ``go back to their huts'' in
Africa, recalled the two officials, who asked for anonymity to discuss a
sensitive conversation in the Oval Office.

As the meeting continued, John F. Kelly, then the secretary of homeland
security, and Rex W. Tillerson, the secretary of state, tried to
interject, explaining that many were short-term travelers making
one-time visits. But as the president continued, Mr. Kelly and Mr.
Miller turned their ire on Mr. Tillerson, blaming him for the influx of
foreigners and prompting the secretary of state to throw up his arms in
frustration. If he was so bad at his job, maybe he should stop issuing
visas altogether, Mr. Tillerson fired back.

Tempers flared and Mr. Kelly asked that the room be cleared of staff
members. But even after the door to the Oval Office was closed, aides
could still hear the president berating his most senior advisers.

Sarah Huckabee Sanders, the White House press secretary, denied on
Saturday morning that Mr. Trump had made derogatory statements about
immigrants during the meeting.

``General Kelly, General McMaster, Secretary Tillerson, Secretary
Nielsen and all other senior staff actually in the meeting deny these
outrageous claims,'' she said, referring to the current White House
chief of staff, the national security adviser and the secretaries of
state and homeland security. ``It's both sad and telling The New York
Times would print the lies of their anonymous `sources' anyway.''

While the White House did not deny the overall description of the
meeting, officials strenuously insisted that Mr. Trump never used the
words ``AIDS'' or ``huts'' to describe people from any country. Several
participants in the meeting told Times reporters that they did not
recall the president using those words and did not think he had, but the
two officials who described the comments found them so noteworthy that
they related them to others at the time.

The meeting in June reflects Mr. Trump's visceral approach to an issue
that defined his campaign and has indelibly shaped the first year of his
presidency.

Seizing on immigration as the cause of countless social and economic
problems, Mr. Trump entered office with an agenda of symbolic but
incompletely thought-out goals, the product not of rigorous policy
debate but of emotionally charged personal interactions and an instinct
for tapping into the nativist views of white working-class Americans.

Like many of his initiatives, his effort to change American immigration
policy has been executed through a disorderly and dysfunctional process
that sought from the start to defy the bureaucracy charged with
enforcing it, according to interviews with three dozen current and
former administration officials, lawmakers and others close to the
process, many of whom spoke on the condition of anonymity to detail
private interactions.

But while Mr. Trump has been repeatedly frustrated by the limits of his
power, his efforts to remake decades of immigration policy have gained
increasing momentum as the White House became more disciplined and adept
at either ignoring or undercutting the entrenched opposition of many
parts of the government. The resulting changes have had far-reaching
consequences, not only for the immigrants who have sought to make a new
home in this country, but also for the United States' image in the
world.

``We have taken a giant steamliner barreling full speed,'' Mr. Miller
said in a recent interview. ``Slowed it, stopped it, begun to turn it
around and started sailing in the other direction.''

It is an assessment shared ruefully by Mr. Trump's harshest critics, who
see a darker view of the past year. Frank Sharry, the executive director
of America's Voice, a pro-immigration group, argues that the president's
immigration agenda is motivated by racism.

``He's basically saying, `You people of color coming to America seeking
the American dream are a threat to the white people,''' said Mr. Sharry,
an outspoken critic of the president. ``He's come into office with an
aggressive strategy of trying to reverse the demographic changes
underway in America.''

\includegraphics{https://static01.graylady3jvrrxbe.onion/images/2017/12/24/us/24dc-immigration2/00dc-immigration2-articleLarge.jpg?quality=75\&auto=webp\&disable=upscale}

\hypertarget{a-pledge-with-appeal}{%
\subsection{A Pledge With Appeal}\label{a-pledge-with-appeal}}

Those who know Mr. Trump say that his attitude toward immigrants long
predates his entry into politics.

``He's always been fearful where other cultures are concerned and always
had anxiety about food and safety when he travels,'' said Michael
D'Antonio, who interviewed him for the biography ``The Truth About
Trump.'' ``His objectification and demonization of people who are
different has festered for decades.''

Friends say Mr. Trump, a developer turned reality TV star, grew to see
immigration as a zero-sum issue: What is good for immigrants is bad for
America. In 2014, well before becoming a candidate,
\href{https://twitter.com/realdonaldtrump/status/496640747379388416?ref_src=twcamp\%5Eshare\%7Ctwsrc\%5Eios\%7Ctwgr\%5Ecom.google.Gmail.ShareExtension}{he
tweeted}: ``Our government now imports illegal immigrants and deadly
diseases. Our leaders are inept.''

But he remained conflicted, viewing himself as benevolent and wanting to
be liked by the many immigrants he employed.

Over time, the anti-immigrant tendencies hardened, and two of his early
advisers, Roger J. Stone Jr. and Sam Nunberg, stoked that sentiment. But
it was Mr. Trump who added an anti-immigrant screed to
\href{https://www.nytimes3xbfgragh.onion/2015/06/17/us/politics/donald-trump-runs-for-president-this-time-for-real-he-says.html}{his
Trump Tower campaign announcement} in June 2015 in New York City without
telling his aides.

``When do we beat Mexico at the border? They're laughing at us, at our
stupidity,'' Mr. Trump ad-libbed. ``They're sending people that have
lots of problems, and they're bringing those problems,'' he continued.
``They're bringing drugs; they're bringing crime; they're rapists.''

During his campaign, he
\href{https://www.nytimes3xbfgragh.onion/2015/11/25/nyregion/a-definitive-debunking-of-donald-trumps-9-11-claims.html}{pushed
a false story about Muslims celebrating in Jersey City} as they watched
the towers fall after the Sept. 11, 2001, attacks in New York. He said
illegal immigrants were like ``vomit'' crossing the border. And he made
pledges that he clearly could not fulfill.

``We will begin moving them out, Day 1,'' he
\href{https://www.nytimes3xbfgragh.onion/2016/09/02/us/politics/transcript-trump-immigration-speech.html}{said
at a rally in August 2016}, adding, ``My first hour in office, those
people are gone.''

Democrats and some Republicans recoiled, calling Mr. Trump's messaging
damaging and divisive. But for the candidate, the idea of securing the
country against outsiders with a wall had intoxicating appeal, though
privately, he acknowledged that it was a rhetorical device to whip up
crowds when they became listless.

Senator Tom Cotton, Republican of Arkansas, whom Mr. Trump consults
regularly on the matter, said it was not a stretch to attribute Mr.
Trump's victory to issues where Mr. Trump broke with a Republican
establishment orthodoxy that had disappointed anti-immigrant
conservatives for decades.

``There's no issue on which he was more unorthodox than on
immigration,'' Mr. Cotton said.

Image

``We have taken a giant steamliner barreling full speed,'' the adviser
Stephen Miller said of Mr. Trump's immigration efforts. ``Slowed it,
stopped it, begun to turn it around and started sailing in the other
direction.''Credit...Stephen Crowley/The New York Times

\hypertarget{ban-restarts-enforcement}{%
\subsection{Ban Restarts Enforcement}\label{ban-restarts-enforcement}}

Mr. Trump came into office with a long list of campaign promises that
included not only
\href{https://www.nytimes3xbfgragh.onion/2017/01/25/world/americas/trump-mexico-border-wall.html}{building
the wall} (and making Mexico pay for it), but creating a
``\href{http://www.msnbc.com/msnbc/trump-vows-humane-deportation-force}{deportation
force},''
\href{https://www.nytimes3xbfgragh.onion/politics/first-draft/2015/12/07/donald-trump-calls-for-banning-muslims-from-entering-u-s/}{barring
Muslims from entering the country} and immediately
\href{https://www.nytimes3xbfgragh.onion/2016/11/15/us/politics/donald-trump-deport-immigrants.html}{deporting
millions of immigrants with criminal records}.

Mr. Miller and other aides had the task of turning those promises into a
policy agenda that would also include an assault against a
pro-immigration bureaucracy they viewed with suspicion and disdain.
Working in secret, they drafted a half-dozen executive orders. One would
crack down on
\href{https://www.nytimes3xbfgragh.onion/interactive/2016/09/02/us/sanctuary-cities.html}{so-called
sanctuary cities}. Another proposed changing the definition of a
criminal alien so that it included people arrested --- not just those
convicted.

But mindful of his campaign promise to quickly impose ``extreme
vetting,'' Mr. Trump decided his first symbolic action would be an
executive order to place a worldwide ban on travel from nations the
White House considered compromised by terrorism.

With no policy experts in place, and deeply suspicious of career civil
servants they regarded as spies for President Barack Obama, Mr. Miller
and a small group of aides started with an Obama-era law that identified
seven terror-prone ``countries of concern.'' And then they skipped
practically every step in the standard White House playbook for creating
and introducing a major policy.

The National Security Council never convened to consider the travel ban
proposal. Sean Spicer, the White House press secretary at the time, did
not see it ahead of time. Lawyers and policy experts at the White House,
the Justice Department and the Homeland Security Department were not
asked to weigh in. There were no talking points for friendly surrogates,
no detailed briefings for reporters or lawmakers, no answers to
frequently asked questions, such as whether green card holders would be
affected.

The announcement of the travel ban on a Friday night, seven days after
Mr. Trump's inauguration, created
\href{https://www.nytimes3xbfgragh.onion/2017/01/29/us/politics/white-house-official-in-reversal-says-green-card-holders-wont-be-barred.html}{chaotic
scenes at the nation's largest airports}, as hundreds of people were
stopped, and set off widespread confusion and loud protests. Lawyers for
the government raced to defend the president's actions against
\href{https://www.nytimes3xbfgragh.onion/2017/01/29/us/politics/trump-immigration-refugee-order.html}{court
challenges}, while aides struggled to explain the policy to
\href{https://www.nytimes3xbfgragh.onion/2017/01/29/us/politics/republicans-congress-trump-refugees.html}{perplexed
lawmakers} the next night at a black-tie dinner.

White House aides resorted to Google searches and frenzied scans of the
United States Code to figure out which countries were affected.

But for the president, the chaos was the first, sharp evidence that he
could exert power over the bureaucracy he criticized on the campaign
trail.

``It's working out very nicely,'' Mr. Trump told reporters in the Oval
Office the next day.

At a hastily called Saturday night meeting in the Situation Room, Mr.
Miller told senior government officials that they should tune out the
whining.

Sitting at the head of the table, across from Mr. Kelly, Mr. Miller
repeated what he told the president: This is what we wanted --- to turn
immigration enforcement back on.

Mr. Kelly, who shared Mr. Trump's views about threats from abroad, was
nonetheless livid that his employees at homeland security had been
called into action with no guidance or preparation. He told angry
lawmakers that responsibility for the rollout was ``all on me.''
Privately, he told the White House, ``That's not going to happen
again.''

Image

Demonstrating in January against the detention of travelers and refugees
by border officials at Kennedy International Airport. The announcement
of a travel ban set off widespread confusion and loud
protests.Credit...Christopher Lee for The New York Times

\hypertarget{forced-to-back-down}{%
\subsection{Forced to Back Down}\label{forced-to-back-down}}

Amid the turbulent first weeks, Mr. Trump's attempt to bend the
government's immigration apparatus to his will began to take shape.

The ban's message of ``keep out'' helped drive down illegal border
crossings as much as 70 percent, even without being formally put into
effect.

Immigration officers rounded up 41,318 undocumented immigrants during
the president's first 100 days, nearly a 40 percent increase. The
Justice Department began hiring more immigration judges to speed up
deportations. Officials threatened to hold back funds for sanctuary
cities. The flow of refugees into the United States slowed.

Mr. Trump ``has taken the handcuffs off,'' said Steven A. Camarota of
the Center for Immigration Studies, an advocacy group that favors more
limits on immigration.

Mr. Obama had been criticized by immigrant rights groups for excessive
deportations, especially in his first term. But Mr. Camarota said that
Mr. Trump's approach was ``a distinct change, to look at what is
immigration doing to us, rather than what is the benefit for the
immigrant.''

The president, however, remained frustrated that the shift was not
yielding results.

By early March, judges across the country had blocked his travel ban.
Immigrant rights activists were crowing that they had thwarted the new
president. Even Mr. Trump's own lawyers told him he had to give up on
defending the ban.

Attorney General Jeff Sessions and lawyers at the White House and
Justice Department had decided that waging an uphill legal battle to
defend the directive in the Supreme Court would fail. Instead, they
wanted to devise a narrower one that could pass legal muster.

The president, though, was furious about what he saw as backing down to
politically correct adversaries. He did not want a watered-down version
of the travel ban, he yelled at Donald F. McGahn II, the White House
counsel, as the issue came to a head on Friday, March 3, in the Oval
Office.

It was a familiar moment for Mr. Trump's advisers. The president did not
mind being told ``no'' in private, and would sometimes relent. But he
could not abide a public turnabout, a retreat. At those moments, he
often exploded at whoever was nearby.

As Marine One waited on the South Lawn for Mr. Trump to begin his
weekend trip to Palm Beach, Fla., Mr. McGahn insisted that
administration lawyers had already promised the court that Mr. Trump
would issue a new order. There was no alternative, he said.

``This is bullshit,'' the president responded.

With nothing resolved, Mr. Trump, furious, left the White House. A
senior aide emailed a blunt warning to a colleague waiting aboard Air
Force One at Joint Base Andrews in Maryland: ``He's coming in hot.''

Already mad at Mr. Sessions, who the day before
\href{https://www.nytimes3xbfgragh.onion/2017/03/02/us/politics/jeff-sessions-russia-trump-investigation-democrats.html}{had
recused himself} in the Russia investigation, Mr. Trump refused to take
his calls. Aides told Mr. Sessions he would have to fly down to
Mar-a-Lago to plead with the president in person to sign the new order.

Over dinner that night with Mr. Sessions and Mr. McGahn, Mr. Trump
relented. When he was back in Washington, he signed the new order. It
was an indication that he had begun to understand --- or at least,
begrudgingly accept --- the need to follow a process.

Still, one senior adviser later recalled never having seen a president
so angry signing anything.

\includegraphics{https://static01.graylady3jvrrxbe.onion/images/2017/12/27/us/politics/ice-raid-trump/ice-raid-trump-videoSixteenByNine3000.jpg}

\hypertarget{soft-spot-for-dreamers}{%
\subsection{Soft Spot for `Dreamers'}\label{soft-spot-for-dreamers}}

As a candidate, Mr. Trump had repeatedly contradicted himself about the
deportations he would pursue, and whether he was opposed to any kind of
path to citizenship for undocumented immigrants. But he also courted
conservative voters by describing an Obama-era policy as an illegal
amnesty for the immigrants who had been brought to the United States as
children.

During the transition, his aides drafted an executive order to end the
program, known as Deferred Action for Childhood Arrivals. But the
executive order was held back as the new president struggled with
conflicted feelings about the young immigrants, known as Dreamers.

``We're going to take care of those kids,'' Mr. Trump had pledged to
Senator Richard J. Durbin during a private exchange at his Inauguration
Day luncheon.

The comment was a fleeting glimpse of the president's tendency to seek
approval from whoever might be sitting across from him, and the power
that personal interactions have in shaping his views.

In 2013, Mr. Trump met with a small group of Dreamers at Trump Tower,
hoping to improve his standing with the Hispanic community. José Machado
told Mr. Trump about waking up at the age of 15 to find his mother had
vanished --- deported, he later learned, back to Nicaragua.

``Honestly,'' Mr. Machado said of Mr. Trump, ``he had no idea.''

Mr. Trump appeared to be touched by the personal stories, and insisted
that the Dreamers accompany him to his gift shop for watches, books and
neckties to take home as souvenirs. In the elevator on the way down, he
quietly nodded and said, ``You convinced me.''

Aware that the president
\href{https://www.nytimes3xbfgragh.onion/2017/02/26/us/politics/daca-dreamers-immigration-trump.html}{was
torn about the Dreamers}, Jared Kushner, his son-in-law, quietly reached
out in March to Mr. Durbin, who had championed legislation called the
Dream Act to legalize the immigrants, to test the waters for a possible
deal.

After weeks of private meetings on Capitol Hill and telephone
conversations with Mr. Durbin and Senator Lindsey Graham, a South
Carolina Republican supportive of legalizing the Dreamers, Mr. Kushner
invited them to dinner at the six-bedroom estate he shares with his
wife, Ivanka Trump.

But Mr. Durbin's hope of a deal faded when he arrived to the house and
saw who one of the guests would be.

``Stephen Miller's presence made it a much different experience than I
expected,'' Mr. Durbin said later.

Image

Jairo Reyes of Arkansas, center left, and Karen Cuadillo of Florida,
young undocumented immigrants shielded from deportation under an
Obama-era policy, on Capitol Hill in September.Credit...Tom Brenner/The
New York Times

\hypertarget{confronting-the-deep-state}{%
\subsection{Confronting the `Deep
State'}\label{confronting-the-deep-state}}

Even as the administration was engaged in a court battle over the travel
ban, it began to turn its attention to another way of tightening the
border --- by limiting the number of refugees admitted each year to the
United States. And if there was one ``deep state'' stronghold of Obama
holdovers that Mr. Trump and his allies suspected of undermining them on
immigration, it was the State Department, which administers the refugee
program.

At the department's Bureau of Population, Refugees and Migration, there
was a sense of foreboding about a president who had once warned that any
refugee might be a ``Trojan horse'' or part of a ``terrorist army.''

Mr. Trump had already used the travel ban to cut the number of allowable
refugees admitted to the United States in 2017 to 50,000, a fraction of
the 110,000 set by Mr. Obama. Now, Mr. Trump would have to decide the
level for 2018.

At an April meeting with top officials from the bureau in the West
Wing's Roosevelt Room, Mr. Miller cited statistics from the
restrictionist Center for Immigration Studies that indicated that
resettling refugees in the United States was far costlier than helping
them in their own region.

Mr. Miller was visibly displeased, according to people present, when
State Department officials pushed back, citing another study that found
refugees to be a net benefit to the economy. He called the contention
absurd and said it was exactly the wrong kind of thinking.

But the travel ban had been a lesson for Mr. Trump and his aides on the
dangers of dictating a major policy change without involving the people
who enforce it. This time, instead of shutting out those officials, they
worked to tightly control the process.

In previous years, State Department officials had recommended a refugee
level to the president. Now, Mr. Miller told officials the number would
be determined by the Department of Homeland Security under a new policy
that treated the issue as a security matter, not a diplomatic one.

When he got word that the Office of Refugee Resettlement had drafted a
55-page report showing that refugees were a net positive to the economy,
Mr. Miller swiftly intervened, requesting a meeting to discuss it. The
study never made it to the White House; it was shelved in favor of a
three-page list of all the federal assistance programs that refugees
used.

At the United Nations General Assembly in September, Mr. Trump cited the
Center for Immigration Studies report, arguing that it was more
cost-effective to keep refugees out than to bring them into the United
States.

``Uncontrolled migration,'' Mr. Trump declared, ``is deeply unfair to
both the sending and receiving countries.''

Image

Border wall prototypes in San Diego. The disorganization that marked Mr.
Trump's earliest actions on immigration have given way to a more
disciplined approach that has yielded concrete results.Credit...Jenna
Schoenefeld for The New York Times

\hypertarget{more-disciplined-approach}{%
\subsection{More Disciplined Approach}\label{more-disciplined-approach}}

Cecilia Muñoz, who served as Mr. Obama's chief domestic policy adviser,
said she was alarmed by the speed with which Mr. Trump and his team have
learned to put their immigration agenda into effect.

``The travel ban was a case of bureaucratic incompetence,'' she said.
``They made rookie mistakes. But they clearly learned from that
experience. For the moment, all of the momentum is in the direction of
very ugly, very extreme, very harmful policies.''

By year's end, the chaos and disorganization that marked Mr. Trump's
earliest actions on immigration had given way to a more disciplined
approach that yielded concrete results,
\href{https://www.nytimes3xbfgragh.onion/2017/07/28/us/politics/john-kelly-chief-of-staff-donald-trump.html}{steered
in large part by Mr. Kelly}, a retired four-star Marine general. As
secretary of homeland security, he had helped unleash immigration
officers who felt constrained under Mr. Obama. They arrested 143,000
people in 2017, a sharp uptick, and deported more than 225,000.

Later, as White House chief of staff, Mr. Kelly quietly persuaded the
president to drop his talk of Mexico paying for the wall. But he has
advocated on behalf of the president's restrictionist vision,
\href{https://www.nytimes3xbfgragh.onion/2017/10/25/us/politics/trump-kelly.html}{defying
his reputation} as a moderator of Mr. Trump's hard-line instincts.

In September,
\href{https://www.nytimes3xbfgragh.onion/2017/09/24/us/politics/new-order-bars-almost-all-travel-from-seven-countries.html}{a
third version of the president's travel ban} was issued with little
fanfare and new legal justifications. Then, Mr. Trump overruled
objections from diplomats, capping refugee admissions at 45,000 for
2018, the lowest since 1986. In November, the president
\href{https://www.nytimes3xbfgragh.onion/2017/11/20/us/haitians-temporary-status.html}{ended
a humanitarian program} that granted residency to 59,000 Haitians since
a 2010 earthquake ravaged their country.

As the new year approached, officials began considering
\href{https://www.nytimes3xbfgragh.onion/2017/12/21/us/trump-immigrant-families-separate.html}{a
plan to separate parents from their children} when families are caught
entering the country illegally, a move that immigrant groups called
draconian.

At times, though, Mr. Trump has shown an openness to a different
approach. In private discussions, he returns periodically to the idea of
a ``comprehensive immigration'' compromise, though aides have warned him
against using the phrase because it is seen by his core supporters as
code for amnesty. During a fall dinner with Democratic leaders, Mr.
Trump explored the possibility of a bargain to legalize Dreamers in
exchange for border security.

Mr. Trump even told Republicans recently that he wanted to think bigger,
envisioning a deal early next year that would include a wall, protection
for Dreamers, work permits for their parents, a shift to merit-based
immigration with tougher work site enforcement, and ultimately, legal
status for some undocumented immigrants.

The idea would prevent Dreamers from sponsoring the parents who brought
them illegally for citizenship, limiting what Mr. Trump refers to as
``chain migration.''

``He wants to make a deal,'' said Mr. Graham, who spoke with Mr. Trump
about the issue last week. ``He wants to fix the entire system.''

Yet publicly, Mr. Trump has only employed the absolutist language that
defined his campaign and has dominated his presidency.

After an Uzbek immigrant was arrested on suspicion of
\href{https://www.nytimes3xbfgragh.onion/2017/10/31/nyregion/police-shooting-lower-manhattan.html}{plowing
a truck into a bicycle path} in Lower Manhattan in October, killing
eight people, the president seized on the episode.

Privately, in the Oval Office, the president expressed disbelief about
the visa program that had admitted the suspect, confiding to a group of
visiting senators that it was yet another piece of evidence that the
United States' immigration policies were ``a joke.''

Even after a year of progress toward a country sealed off from foreign
threats, the president still viewed the immigration system as plagued by
complacency.

``We're so politically correct,'' he complained to reporters in the
cabinet room, ``that we're afraid to do anything.''

Advertisement

\protect\hyperlink{after-bottom}{Continue reading the main story}

\hypertarget{site-index}{%
\subsection{Site Index}\label{site-index}}

\hypertarget{site-information-navigation}{%
\subsection{Site Information
Navigation}\label{site-information-navigation}}

\begin{itemize}
\tightlist
\item
  \href{https://help.nytimes3xbfgragh.onion/hc/en-us/articles/115014792127-Copyright-notice}{©~2020~The
  New York Times Company}
\end{itemize}

\begin{itemize}
\tightlist
\item
  \href{https://www.nytco.com/}{NYTCo}
\item
  \href{https://help.nytimes3xbfgragh.onion/hc/en-us/articles/115015385887-Contact-Us}{Contact
  Us}
\item
  \href{https://www.nytco.com/careers/}{Work with us}
\item
  \href{https://nytmediakit.com/}{Advertise}
\item
  \href{http://www.tbrandstudio.com/}{T Brand Studio}
\item
  \href{https://www.nytimes3xbfgragh.onion/privacy/cookie-policy\#how-do-i-manage-trackers}{Your
  Ad Choices}
\item
  \href{https://www.nytimes3xbfgragh.onion/privacy}{Privacy}
\item
  \href{https://help.nytimes3xbfgragh.onion/hc/en-us/articles/115014893428-Terms-of-service}{Terms
  of Service}
\item
  \href{https://help.nytimes3xbfgragh.onion/hc/en-us/articles/115014893968-Terms-of-sale}{Terms
  of Sale}
\item
  \href{https://spiderbites.nytimes3xbfgragh.onion}{Site Map}
\item
  \href{https://help.nytimes3xbfgragh.onion/hc/en-us}{Help}
\item
  \href{https://www.nytimes3xbfgragh.onion/subscription?campaignId=37WXW}{Subscriptions}
\end{itemize}
