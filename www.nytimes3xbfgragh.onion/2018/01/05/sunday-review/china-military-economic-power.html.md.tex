Sections

SEARCH

\protect\hyperlink{site-content}{Skip to
content}\protect\hyperlink{site-index}{Skip to site index}

\href{/section/opinion/sunday}{Sunday Review}\textbar{}A Chinese Empire
Reborn

\url{https://nyti.ms/2EcHv43}

\begin{itemize}
\item
\item
\item
\item
\item
\end{itemize}

\includegraphics{https://static01.graylady3jvrrxbe.onion/images/2018/01/07/opinion/07wong-headerstill/07wong-headerstill-articleLarge.jpg?quality=75\&auto=webp\&disable=upscale}

News Analysis

\hypertarget{a-chinese-empire-reborn}{%
\section{A Chinese Empire Reborn}\label{a-chinese-empire-reborn}}

The Communist Party's emerging empire is more the result of force than a
gravitational pull of Chinese ideas.

A military parade in Beijing in 2015.Credit...Jonah M. Kessel/The New
York Times

Supported by

\protect\hyperlink{after-sponsor}{Continue reading the main story}

By \href{https://www.nytimes3xbfgragh.onion/by/edward-wong}{Edward Wong}

\begin{itemize}
\item
  Jan. 5, 2018
\item
  \begin{itemize}
  \item
  \item
  \item
  \item
  \item
  \end{itemize}
\end{itemize}

\href{https://cn.nytimes3xbfgragh.onion/opinion/20180108/china-military-economic-power/}{阅读简体中文版}\href{https://cn.nytimes3xbfgragh.onion/opinion/20180108/china-military-economic-power/zh-hant/}{閱讀繁體中文版}

I am the son of two empires, the United States and China. I was born in
and raised around Washington in the Nixon-to-Reagan era, but my parents
grew up in
\href{http://www.nytimes3xbfgragh.onion/2009/12/04/world/asia/04taishan.html}{villages}
in southern China. My father was
\href{https://www.nytimes3xbfgragh.onion/2017/06/18/world/asia/mongolian-warriors-and-communist-soldiers-a-frontier-town-in-china.html}{a
member of the People's Liberation Army} in the 1950s, the first decade
of Communist rule, before he soured on the revolution and left for Hong
Kong.

So it was with excitement that I landed in Beijing in April 2008 to
start an assignment with The New York Times that stretched to almost a
decade. I had just spent nearly four years reporting on
\href{http://www.nytimes3xbfgragh.onion/2007/06/03/weekinreview/03wong.html}{the
bloody failure of the American imperial project in Iraq}, and now I was
in the metropole that was building a new world order.

China had entered a honeymoon phase with other nations. For years,
anticipation had built for the 2008 Summer Olympics. Though China had
\href{http://www.nytimes3xbfgragh.onion/2008/03/24/world/asia/24tibet.html}{suppressed
a Tibetan uprising} that spring, it earned international good will after
a
\href{http://www.nytimes3xbfgragh.onion/2008/05/13/world/asia/13scene.html}{devastating
earthquake}.

People flocked to Beijing for China's ``coming out'' party. Foreign
leaders gawked at
\href{http://www.nytimes3xbfgragh.onion/2008/07/13/arts/design/13build.html}{gleaming
architecture} and
\href{http://www.nytimes3xbfgragh.onion/2008/08/09/sports/olympics/09china.html}{opening
ceremonies} that signaled the nation's ambitions. After the festivities
ended, the world arrived at another inflection point --- the implosion
of the American financial system and the global economic crisis. China's
growth buttressed both the world economy and a belief among its
officials that its economic and political systems could rival those of
the United States.

\includegraphics{https://static01.graylady3jvrrxbe.onion/images/2018/01/07/opinion/sunday/07Wong1/merlin_107001671_f0f2e50b-7c0c-4c73-b182-db369f12d6c7-articleLarge.jpg?quality=75\&auto=webp\&disable=upscale}

Though unabashedly authoritarian, China was a magnet. I was among many
who thought it might forge a confident and more open identity while
ushering in a vibrant era of new ideas, values and culture, one
befitting its superpower status. When I ended my China assignment last
year, I no longer had such expectations.

From
\href{https://www.nytimes3xbfgragh.onion/2017/11/09/business/donald-trump-china-trade-xi-jinping.html}{trade}
to the
\href{https://www.nytimes3xbfgragh.onion/2017/08/17/business/dealbook/alibaba-sales-revenue-first-quarter-profit.html}{internet},
from
\href{https://www.nytimes3xbfgragh.onion/2017/02/02/magazine/the-parachute-generation.html}{higher
education} to
\href{https://www.nytimes3xbfgragh.onion/2015/04/29/world/asia/wang-jianlin-abillionaire-at-the-intersection-of-business-and-power-in-china.html}{Hollywood},
China is shaping the world in ways that people have only begun to grasp.
Yet the emerging imperium is more a result of the Communist Party's
exercise of hard power, including
\href{http://www.nytimes3xbfgragh.onion/2009/12/21/world/asia/21china.html}{economic
coercion}, than the product of a gravitational pull of Chinese ideas or
\href{http://www.nytimes3xbfgragh.onion/interactive/world/asia/culture-and-control-in-china-series.html}{contemporary
culture}.

Of the global powers that dominated the 19th century, China alone is a
rejuvenated empire. The Communist Party commands a vast territory that
the ethnic-Manchu rulers of the Qing dynasty cobbled together through
war and diplomacy. And the dominion could grow: China is using its
military to test potential control of disputed borderlands from the
\href{https://www.nytimes3xbfgragh.onion/2014/06/17/world/asia/spratly-archipelago-china-trying-to-bolster-its-claims-plants-islands-in-disputed-waters.html}{South
China Sea} to the
\href{https://www.nytimes3xbfgragh.onion/2017/07/26/world/asia/dolam-plateau-china-india-bhutan.html}{Himalayas},
while firing up nationalism at home. Once again, states around the world
pay homage to the court, as in 2015 during a huge
\href{https://www.nytimes3xbfgragh.onion/2015/09/03/world/asia/beijing-turns-into-ghost-town-as-it-gears-up-for-military-parade.html?_r=0}{military
parade}.

For decades, the United States was a global beacon for those who
embraced certain values --- the rule of law, free speech, clean
government and human rights. Even if policy often fell short of those
stated ideals, American ``soft power'' remained as potent as its armed
forces. In the post-Soviet era, political figures and scholars regarded
that American way of
\href{https://www.jstor.org/stable/1148580?seq=1\#page_scan_tab_contents}{amassing
power through attraction} as a central element of forging a modern
empire.

China's rise is a blunt counterpoint. From 2009 onward, Chinese power in
domestic and international realms has become synonymous with brute
strength, bribery and browbeating --- and the Communist Party's empire
is getting stronger.

Image

A poster for a Chinese high-speed train at the construction site for a
bridge spanning the Mekong River near Luang Prabang, Laos.Credit...Adam
Dean for The New York Times

At home, the party has imprisoned
\href{https://www.nytimes3xbfgragh.onion/2015/07/23/world/asia/china-crackdown-human-rights-lawyers.html}{rights
lawyers},
\href{https://www.nytimes3xbfgragh.onion/2017/12/14/opinion/net-neutrality-china-internet.html}{strangled
the internet}, compelled companies and universities to
\href{https://www.nytimes3xbfgragh.onion/2017/02/12/opinion/china-the-party-corporate-complex.html}{install
party cells}, and planned for a potentially Orwellian
\href{https://www.nytimes3xbfgragh.onion/2018/01/04/business/china-alibaba-privacy.html?_r=0}{``social
credit'' system}. Abroad, it is building military installations on
disputed Pacific reefs and infiltrating cybernetworks. It pushes
\href{https://www.nytimes3xbfgragh.onion/2017/05/13/business/china-railway-one-belt-one-road-1-trillion-plan.html?_r=0}{the
``One Belt, One Road'' infrastructure initiative} across Eurasia, which
will have benefits for other nations but will also allow China to
pressure them to do business with Chinese state-owned enterprises,
\href{http://www.nytimes3xbfgragh.onion/2009/12/21/world/asia/21china.html}{as
it has done in recent years} throughout Asia and Africa.

So far, Chinese soft power plays a minor role. For one thing, the party
insists on
\href{http://www.nytimes3xbfgragh.onion/interactive/world/asia/culture-and-control-in-china-series.html}{tight
control of cultural production}, so Chinese popular culture has little
global appeal next to that of the United States or even South Korea.

No nation knows China's hard ways better than Norway. China
\href{https://www.nytimes3xbfgragh.onion/2016/12/19/world/europe/china-norway-nobel-liu-xiaobo.html}{punished
it by breaking diplomatic and economic ties} for six years after the
independent Nobel committee in 2010 gave the Peace Prize to Liu Xiaobo,
a pro-democracy writer imprisoned in China (he
\href{https://www.nytimes3xbfgragh.onion/2017/07/13/world/asia/liu-xiaobo-dead.html}{died
of cancer in July}).

President Xi Jinping is the avatar of the new imperium. The 19th Party
Congress in October was his
\href{https://www.nytimes3xbfgragh.onion/2017/10/24/world/asia/china-xi-jinping-communist-party.html}{victory
lap}. Party officials enshrined ``Xi Jinping Thought'' in the party
constitution, putting him on par with Mao Zedong. Mr. Xi said China had
entered a ``new era'' of strength and the party would be the arbiter of
public life. Mr. Xi holds appeal for foreign leaders aspiring to
strongman status --- President Trump openly admires him.

Many Chinese people told me they still believed the country's top
leaders looked out for ordinary people, even if the party was rotting.
This belief was rooted in abstract hope rather than empirical evidence.
It was like peering through the
\href{http://www.nytimes3xbfgragh.onion/2013/08/04/sunday-review/life-in-a-toxic-country.html}{toxic
air enveloping Chinese cities} in search of blue sky.

The culture of hard power goes from top to bottom. In the provinces,
party officials move quickly to suppress any challenges to their
authority. When they sense rising mass resistance, they buy off or
imprison the leaders.

I saw this in my first year in China, when officials separately broke
the will of parents furious over
\href{http://www.nytimes3xbfgragh.onion/2008/10/17/world/asia/17milk.html}{deadly
tainted milk} and ones grieving over thousands of children
\href{http://www.nytimes3xbfgragh.onion/2008/09/05/world/asia/05china.html}{who
had died in shoddily built schools} during the Sichuan earthquake. I
learned this was typical of the approach taken by Chinese officials.
Most Chinese do not run afoul of the party, but those who do pay a high
price.

The abuse of power is frequent, and many Chinese say corruption is their
top concern. All other issues, from environmental degradation to
\href{http://www.nytimes3xbfgragh.onion/2012/10/26/business/global/family-of-wen-jiabao-holds-a-hidden-fortune-in-china.html}{wealth
inequality}, are linked to it. Mr. Xi is canny enough to capitalize on
the discontent: He leads an anticorruption drive that allows him to oust
rivals and enforce party discipline.

None of that results in the rule of law. And China's domestic security
budget has exceeded that of its military in recent years, even as both
grow rapidly, highlighting the nation's investment in hard power.

\includegraphics{https://static01.graylady3jvrrxbe.onion/images/2018/01/12/world/05tibet-1/05tibet-1-videoSixteenByNineJumbo1600.jpg}

I learned in 2016 that Tashi Wangchuk, a young entrepreneur who had
spoken to me about his advocacy for broader Tibetan language education,
had been detained in his hometown, Yushu, by police officers. In
microblog posts, Mr. Tashi had asked local officials to promote true
bilingual education, and he had
\href{https://www.nytimes3xbfgragh.onion/2015/11/29/world/asia/china-tibet-language-education.html?_r=0}{appeared
in 2015 in Times articles and video}.

Mr. Tashi is the kind of citizen China should value --- someone working
within the law to recommend policies that would benefit ordinary people
and ease tensions. But two years later, Mr. Tashi remains imprisoned. A
court tried him on Thursday for ``inciting separatism'' despite
\href{https://www.nytimes3xbfgragh.onion/2017/01/18/world/asia/china-tibetan-education-advocate.html}{criticism
from Western diplomats and human rights groups}.

The party's style of rule threatens to turn sentiments against China
even as the empire grows in stature. History teaches us about an
inevitable dialectic: Power creates resistance. While the state can bend
people to its will, those people meet it with fear and suspicion. The
United States learns this lesson each time it
\href{https://www.nytimes3xbfgragh.onion/column/vietnam-67}{over-relies
on hard power}.

I traveled often to the
\href{http://www.nytimes3xbfgragh.onion/2010/07/25/world/asia/25tibet.html}{frontier
regions} because it was there that
\href{https://www.nytimes3xbfgragh.onion/2016/06/19/world/asia/china-climate-change-nu-river-greenhouse-gases.html}{the
dynamic of power and resistance} was most evident, and that I got the
\href{https://www.nytimes3xbfgragh.onion/interactive/2016/10/25/world/asia/china-climate-change-resettlement.html}{clearest
look} at how China treats
\href{http://www.nytimes3xbfgragh.onion/2009/02/26/world/asia/26tibet.html}{its
most vulnerable citizens}, those
\href{https://www.nytimes3xbfgragh.onion/interactive/2016/10/25/world/asia/china-climate-change-resettlement.html}{outside
mainstream ethnic Han culture}. No other areas better embody the idea of
imperial China. Conquered by the Manchus and reabsorbed by Mao,
\href{https://www.nytimes3xbfgragh.onion/interactive/2016/10/24/world/asia/living-in-chinas-expanding-deserts.html}{these
lands} make up at least one-quarter of Chinese territory. Party
officials fear they are like the Central Asian regions under Soviet rule
--- always on the verge of rebellion and eager to break free.

Image

Police officers on patrol in Kashgar, Xinjiang, in northwestern China.
Beijing fears unrest among the Muslim Uighurs of the
region.Credit...Gilles Sabrié for The New York Times

In October 2016, I quietly entered the sprawling Tibetan Buddhist
settlement of
\href{https://www.nytimes3xbfgragh.onion/2016/11/28/world/asia/china-takes-a-chain-saw-to-a-center-of-tibetan-buddhism.html}{Larung
Gar} and watched the government-ordered demolition of homes of monks and
nuns. In parts of Xinjiang populated by ethnic Uighurs, the tension is
even greater, fueled by
\href{http://www.nytimes3xbfgragh.onion/2009/07/12/weekinreview/12wong.html}{cycles
of violence and repression}. Uighurs speak in hushed tones of
\href{http://www.nytimes3xbfgragh.onion/2008/10/19/world/asia/19xinjiang.html}{restrictions
on Islam} and mass detentions. Signs across Xinjiang forbid long beards
and full veils, and surveillance cameras are everywhere. On my last
reporting trip in China, to
\href{https://www.nytimes3xbfgragh.onion/2017/07/19/world/asia/dodging-chinese-police-in-kashgar-a-silk-road-oasis-town.html}{the
Silk Road oasis of Kashgar}, I saw police patrols in riot gear rounding
up young men.

An important bellwether is Hong Kong, the former British colony from
which my parents emigrated to the United States. On this southern
frontier, as in the west, the party works
\href{https://www.nytimes3xbfgragh.onion/2017/08/17/world/asia/hong-kong-joshua-wong-jailed-umbrella-movement.html}{to
silence the voices of students, politicians and other residents critical
of its rule}. Agents have even abducted booksellers. But those moves
have actually led to more resistance and strengthened
\href{https://www.nytimes3xbfgragh.onion/2014/10/08/world/asia/hong-kong-people-looking-in-mirror-see-fading-chinese-identity.html}{Hong
Kong and Cantonese identity}. They have also stoked greater fears of
Beijing among citizens of Taiwan, the self-governing island that the
party longs to rule.

It is not a stretch to say the party's ways of governance perpetuate a
lack of trust by the Chinese in their institutions and fellow citizens.
And its international policies light the kindling of resistance
overseas, from
\href{https://www.nytimes3xbfgragh.onion/2017/12/19/world/australia/australia-china-backlash-influence.html}{Australia}
to
\href{http://www.nytimes3xbfgragh.onion/2013/06/07/world/africa/ghana-arrests-chinese-in-gold-mining-regions.html}{Ghana}.

Chinese citizens and the world would benefit if China turns out to be an
empire whose power is based as much on ideas, values and culture as on
military and economic might. It was more enlightened under its most
glorious dynasties. But for now, the Communist Party embraces hard power
and coercion, and this could well be what replaces the fading liberal
hegemony of the United States on the global stage.

It will not lead to a grand vision of world order. Instead, before us
looms a void.

Image

Nuns and monks in Larung Gar, a monastic settlement in Sichuan Province.

Credit...Gilles Sabrié for The New York Times

Advertisement

\protect\hyperlink{after-bottom}{Continue reading the main story}

\hypertarget{site-index}{%
\subsection{Site Index}\label{site-index}}

\hypertarget{site-information-navigation}{%
\subsection{Site Information
Navigation}\label{site-information-navigation}}

\begin{itemize}
\tightlist
\item
  \href{https://help.nytimes3xbfgragh.onion/hc/en-us/articles/115014792127-Copyright-notice}{©~2020~The
  New York Times Company}
\end{itemize}

\begin{itemize}
\tightlist
\item
  \href{https://www.nytco.com/}{NYTCo}
\item
  \href{https://help.nytimes3xbfgragh.onion/hc/en-us/articles/115015385887-Contact-Us}{Contact
  Us}
\item
  \href{https://www.nytco.com/careers/}{Work with us}
\item
  \href{https://nytmediakit.com/}{Advertise}
\item
  \href{http://www.tbrandstudio.com/}{T Brand Studio}
\item
  \href{https://www.nytimes3xbfgragh.onion/privacy/cookie-policy\#how-do-i-manage-trackers}{Your
  Ad Choices}
\item
  \href{https://www.nytimes3xbfgragh.onion/privacy}{Privacy}
\item
  \href{https://help.nytimes3xbfgragh.onion/hc/en-us/articles/115014893428-Terms-of-service}{Terms
  of Service}
\item
  \href{https://help.nytimes3xbfgragh.onion/hc/en-us/articles/115014893968-Terms-of-sale}{Terms
  of Sale}
\item
  \href{https://spiderbites.nytimes3xbfgragh.onion}{Site Map}
\item
  \href{https://help.nytimes3xbfgragh.onion/hc/en-us}{Help}
\item
  \href{https://www.nytimes3xbfgragh.onion/subscription?campaignId=37WXW}{Subscriptions}
\end{itemize}
