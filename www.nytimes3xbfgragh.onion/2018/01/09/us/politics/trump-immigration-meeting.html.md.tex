Sections

SEARCH

\protect\hyperlink{site-content}{Skip to
content}\protect\hyperlink{site-index}{Skip to site index}

\href{https://www.nytimes3xbfgragh.onion/section/politics}{Politics}

\href{https://myaccount.nytimes3xbfgragh.onion/auth/login?response_type=cookie\&client_id=vi}{}

\href{https://www.nytimes3xbfgragh.onion/section/todayspaper}{Today's
Paper}

\href{/section/politics}{Politics}\textbar{}Trump's Negotiation on
Immigration, Unfolding on Camera

\url{https://nyti.ms/2Etvmrh}

\begin{itemize}
\item
\item
\item
\item
\item
\item
\end{itemize}

Advertisement

\protect\hyperlink{after-top}{Continue reading the main story}

Supported by

\protect\hyperlink{after-sponsor}{Continue reading the main story}

News Analysis

\hypertarget{trumps-negotiation-on-immigration-unfolding-on-camera}{%
\section{Trump's Negotiation on Immigration, Unfolding on
Camera}\label{trumps-negotiation-on-immigration-unfolding-on-camera}}

\includegraphics{https://static01.graylady3jvrrxbe.onion/images/2018/01/10/us/politics/10dc-assess/merlin_132069492_c40882fb-1672-4ba7-a625-efa4dcef4418-articleLarge.jpg?quality=75\&auto=webp\&disable=upscale}

By \href{http://www.nytimes3xbfgragh.onion/by/peter-baker}{Peter Baker}

\begin{itemize}
\item
  Jan. 9, 2018
\item
  \begin{itemize}
  \item
  \item
  \item
  \item
  \item
  \item
  \end{itemize}
\end{itemize}

WASHINGTON --- For 55 minutes, with cameras rolling, President Trump
engaged in a vigorous discussion of immigration with congressional
leaders of both parties in a setting usually reserved for bland talking
points and meaningless photo opportunities.

In effect, the president and his visitors threw away the blah-blah
scripts and negotiated possible legislation in front of the nation. ``I
hope we've given you enough material,'' a pleased Mr. Trump joked with
reporters as he finally ushered them out of the Cabinet Room in the
White House.

That was the point. After days in which his very
\href{https://www.nytimes3xbfgragh.onion/2018/01/06/us/politics/trump-genius-mental-health.html}{fitness
for office} was debated, Mr. Trump appeared intent on demonstrating that
he could handle the presidency. He was in command of the meeting while
inviting input. He did not berate anyone. He did not call anyone
derogatory nicknames. He signaled that he was open to compromise.

The bar, of course, was historically low given that Democrats and even
some Republicans have been describing him as so unstable that he should
be removed from office. For his advisers, the meeting was a relief, a
chance to reset the narrative and make Mr. Trump look more like a
traditional president. And his critics, grading on a curve, called it a
welcome change, a moment of constructive engagement that they hoped
would lead to more.

Yet it was a measure of Mr. Trump's political weakness that anyone
seemed surprised.

He did not lapse into incoherence but neither did he demonstrate mastery
of policy details after a year in office. At one point, Senator Dianne
Feinstein of California seemed to lead him into agreeing to an
immigration deal on terms that she and fellow Democrats have sought,
only to have an alarmed Representative Kevin McCarthy of California, the
House Republican leader, jump in to steer him back toward his own
policy.

Indeed, Mr. Trump made clear once again that the details of governance
do not really matter to him as much as success, telling congressional
leaders that he would approve whatever they send him. ``I will be
signing it,'' he said. ``I'm not going to say, `Oh, gee, I want this or
I want that.' I'll be signing it.''

The meeting was extraordinary in the sense that it played out in front
of cameras and on the record. President Barack Obama on a few occasions
participated in similarly extended conversations with congressional
leaders in front of the news media, most notably a
\href{http://www.nytimes3xbfgragh.onion/2010/02/26/health/policy/26health.html}{tense
session} with Republicans over health care.

But the news media more typically is invited into the room for only the
first few minutes of such a meeting to record the traditional platitudes
before being kicked out. After Tuesday's meeting, lawmakers said that
they were surprised Mr. Trump allowed cameras to stay and some noted
wryly that they had become props in his latest reality television show.

Still, it was almost possible afterward to hear the collective exhale of
Republicans who themselves nursed private doubts about the president's
capacity long before Michael Wolff's
\href{https://www.nytimes3xbfgragh.onion/2018/01/04/business/media/michael-wolff-trump.html}{new
book}, ``Fire and Fury: Inside the Trump White House,'' prompted the
president to play up his
\href{https://twitter.com/realDonaldTrump/status/949618475877765120}{``mental
stability''} and insist that he is a
\href{https://twitter.com/realDonaldTrump/status/949619270631256064}{``very
stable genius.''}

Senator Lindsey Graham, Republican of South Carolina, who was a harsh
critic when he faced Mr. Trump in the Republican primaries in 2016 but
has lately become one of his chief allies, called Tuesday's session
``the most fascinating meeting I've been involved with in 20-plus years
in politics.''

Democrats used other adjectives. ``It's fairly bizarre that the
president has to show that he's actually balanced and sane,'' Senator
Mazie K. Hirono, Democrat of Hawaii, said in an interview afterward.

Did he succeed in doing that? ``More than some other kinds of other
behaviors that he's engaged in,'' she said cautiously.

``It was more constructive than I believed it was going to be,'' added
Senator Michael Bennet, Democrat of Colorado. ``I went into it thinking
that it was just an opportunity to have a photo op and maybe allow him
to demonstrate that he was not the person he was accused of being over
the weekend. But I think it was more constructive than that.''

Mr. Trump used the occasion to suggest restoring the
\href{http://www.nytimes3xbfgragh.onion/2010/11/17/us/politics/17memo.html}{banned
congressional practice} of allowing members to designate money for
projects in their districts, a process known as earmarks but
conventionally called pork. For Mr. Trump, it was a nod to Washington
institutionalists who say such spending makes it easier to forge
legislative compromises, but it seemed to conflict with his promise to
``drain the swamp.''

Mr. Trump's outreach to Democrats came on the same day the ``America
First'' president announced that he would
\href{https://www.nytimes3xbfgragh.onion/2018/01/09/us/politics/trump-davos-world-economic-forum.html}{travel
this month to Davos, Switzerland}, for the World Economic Forum, a
gathering that is ground zero for the wealthy elite and globalist
leaders that he has derided. By happenstance, it was the same day that
Stephen K. Bannon, the architect of the president's build-the-wall
nationalist appeal and enemy of all things Davos, was
\href{https://www.nytimes3xbfgragh.onion/2018/01/09/us/politics/steve-bannon-breitbart-trump.html?hp\&action=click\&pgtype=Homepage\&clickSource=story-heading\&module=first-column-region\&region=top-news\&WT.nav=top-news}{pushed
out} of Breitbart News after his critical comments in Mr. Wolff's book.

Added together, it gave the impression of a shift, but Mr. Trump has
been down this road before. He
\href{https://www.nytimes3xbfgragh.onion/2017/03/01/us/politics/trump-undocumented-immigrants.html}{broached
the idea} of comprehensive immigration legislation on the same day he
addressed Congress for the first time nearly a year ago, then set about
enacting some of the toughest anti-immigration policies in generations.
He cut a
\href{https://www.nytimes3xbfgragh.onion/2017/09/06/us/politics/house-vote-harvey-aid-debt-ceiling.html}{short-term
spending deal} with Democrats in the fall, predicting a new bipartisan
era, only to torpedo any follow-up with Twitter attacks and a move back
to the right.

Recalling that, Democrats said they were curbing their expectations. But
they held out hope that a coming spending deadline might force action.
``I see all of the potential stars aligned,'' said Senator Robert
Menendez of New Jersey. ``Doesn't mean that it will happen. But I see
the stars aligned for the possibility of something happening.''

He added that he could agree to concessions on bolstering border
security and ending the diversity lottery visa program, two of Mr.
Trump's priorities, if the president likewise made compromises. ``We're
willing to give,'' Mr. Menendez said. ``Some of these things for me are
still hard, but I'm willing to do that if we achieve a real goal here
and if Republicans are reasonable about what they're seeking.''

As in the past, however, Mr. Trump got immediate pushback on Tuesday
from conservative allies who quickly accused him of betraying the very
promises he won the presidency on. In exchange for border security, he
said he would sign legislation to replace the Deferred Action for
Childhood Arrivals, an Obama-era program that protected younger
immigrants who were brought into the country illegally as children but
that Mr. Trump scrapped, citing legal grounds.

``This DACA lovefest confirms a main thesis of Michael Wolff's book:
When Bannon left. liberal Dems Jared, Ivanka, Cohn \& Goldman Sachs took
over,'' Ann Coulter, the conservative commentator,
\href{https://twitter.com/AnnCoulter/status/950796797131227137}{wrote on
Twitter}, referring to Jared Kushner, Ivanka Trump and Gary D. Cohn, the
president's chief economic adviser. She
\href{https://twitter.com/AnnCoulter/status/950798708895027200}{added},
``Nothing Michael Wolff could say about @realDonaldTrump has hurt him as
much as the DACA lovefest right now.''

Mr. Trump said he was prepared for the pressure if the two sides moved
beyond a quick initial deal and tackled a broader comprehensive
immigration bill down the road.

``I'll take the heat, I don't care. I don't care,'' he told lawmakers.
``I'll take all the heat you want to give me, and I'll take the heat off
both the Democrats and the Republicans. My whole life has been heat. I
like heat, in a certain way.''

The next weeks will test that because the temperature is sure to keep
going up.

Advertisement

\protect\hyperlink{after-bottom}{Continue reading the main story}

\hypertarget{site-index}{%
\subsection{Site Index}\label{site-index}}

\hypertarget{site-information-navigation}{%
\subsection{Site Information
Navigation}\label{site-information-navigation}}

\begin{itemize}
\tightlist
\item
  \href{https://help.nytimes3xbfgragh.onion/hc/en-us/articles/115014792127-Copyright-notice}{©~2020~The
  New York Times Company}
\end{itemize}

\begin{itemize}
\tightlist
\item
  \href{https://www.nytco.com/}{NYTCo}
\item
  \href{https://help.nytimes3xbfgragh.onion/hc/en-us/articles/115015385887-Contact-Us}{Contact
  Us}
\item
  \href{https://www.nytco.com/careers/}{Work with us}
\item
  \href{https://nytmediakit.com/}{Advertise}
\item
  \href{http://www.tbrandstudio.com/}{T Brand Studio}
\item
  \href{https://www.nytimes3xbfgragh.onion/privacy/cookie-policy\#how-do-i-manage-trackers}{Your
  Ad Choices}
\item
  \href{https://www.nytimes3xbfgragh.onion/privacy}{Privacy}
\item
  \href{https://help.nytimes3xbfgragh.onion/hc/en-us/articles/115014893428-Terms-of-service}{Terms
  of Service}
\item
  \href{https://help.nytimes3xbfgragh.onion/hc/en-us/articles/115014893968-Terms-of-sale}{Terms
  of Sale}
\item
  \href{https://spiderbites.nytimes3xbfgragh.onion}{Site Map}
\item
  \href{https://help.nytimes3xbfgragh.onion/hc/en-us}{Help}
\item
  \href{https://www.nytimes3xbfgragh.onion/subscription?campaignId=37WXW}{Subscriptions}
\end{itemize}
