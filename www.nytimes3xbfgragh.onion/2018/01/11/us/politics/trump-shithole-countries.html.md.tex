Sections

SEARCH

\protect\hyperlink{site-content}{Skip to
content}\protect\hyperlink{site-index}{Skip to site index}

\href{https://www.nytimes3xbfgragh.onion/section/politics}{Politics}

\href{https://myaccount.nytimes3xbfgragh.onion/auth/login?response_type=cookie\&client_id=vi}{}

\href{https://www.nytimes3xbfgragh.onion/section/todayspaper}{Today's
Paper}

\href{/section/politics}{Politics}\textbar{}Trump Alarms Lawmakers With
Disparaging Words for Haiti and Africa

\url{https://nyti.ms/2EzkEQe}

\begin{itemize}
\item
\item
\item
\item
\item
\item
\end{itemize}

Advertisement

\protect\hyperlink{after-top}{Continue reading the main story}

Supported by

\protect\hyperlink{after-sponsor}{Continue reading the main story}

\hypertarget{trump-alarms-lawmakers-with-disparaging-words-for-haiti-and-africa}{%
\section{Trump Alarms Lawmakers With Disparaging Words for Haiti and
Africa}\label{trump-alarms-lawmakers-with-disparaging-words-for-haiti-and-africa}}

\includegraphics{https://static01.graylady3jvrrxbe.onion/images/2018/01/12/us/politics/12dc-cong1/12dc-cong1-articleLarge.jpg?quality=75\&auto=webp\&disable=upscale}

By
\href{https://www.nytimes3xbfgragh.onion/by/julie-hirschfeld-davis}{Julie
Hirschfeld Davis},
\href{http://www.nytimes3xbfgragh.onion/by/sheryl-gay-stolberg}{Sheryl
Gay Stolberg} and
\href{http://www.nytimes3xbfgragh.onion/by/thomas-kaplan}{Thomas Kaplan}

\begin{itemize}
\item
  Jan. 11, 2018
\item
  \begin{itemize}
  \item
  \item
  \item
  \item
  \item
  \item
  \end{itemize}
\end{itemize}

\href{https://www.nytimes3xbfgragh.onion/es/2018/01/12/donald-trump-migantes-haiti-noruega/}{Leer
en español}

WASHINGTON --- President Trump on Thursday balked at an immigration deal
that would include protections for people from Haiti and some nations in
Africa, demanding to know at a White House meeting why he should accept
immigrants from ``shithole countries'' rather than from places like
Norway, according to people with direct knowledge of the conversation.

Mr. Trump's remarks, the latest example of his penchant for racially
tinged remarks denigrating immigrants, left members of Congress from
both parties attending the meeting in the Oval Office alarmed and
mystified. He made them during a discussion of an emerging bipartisan
deal to give legal status to immigrants illegally brought to the United
States as children, those with knowledge of the conversation said,
speaking on the condition of anonymity to discuss the meeting.

When Mr. Trump heard that Haitians were among those who would benefit
from the proposed deal, he asked whether they could be left out of the
plan, asking, ``Why do we want people from Haiti here?''

The comments were reminiscent of
\href{https://www.nytimes3xbfgragh.onion/2017/12/23/us/politics/trump-immigration.html}{ones
the president made last year} in an Oval Office meeting with cabinet
officials and administration aides, during which he complained about
admitting Haitians to the country, saying that they all had AIDS, as
well as Nigerians, who he said would never go back to their ``huts,''
according to officials who heard the statements in person or were
briefed on the remarks by people who had. The White House vehemently
denied last month that Mr. Trump made those remarks.

In a written statement, Raj Shah, the White House deputy press
secretary, did not deny the account of the meeting on Thursday or
directly address Mr. Trump's comments.

``Certain Washington politicians choose to fight for foreign countries,
but President Trump will always fight for the American people,'' Mr.
Shah said. ``Like other nations that have merit-based immigration,
President Trump is fighting for permanent solutions that make our
country stronger by welcoming those who can contribute to our society,
grow our economy and assimilate into our great nation.''

But the president's vulgar language on a delicate issue left the fate of
the broader immigration debate in limbo and had the potential to torpedo
the chances of achieving the deal being sought to protect about 800,000
undocumented immigrants brought to the country as children. And they
drew a backlash from Republican and Democratic lawmakers, many of whom
called Mr. Trump's utterances unacceptable at best and plainly racist at
worst.

Representative Mia Love, a Republican of Utah who is of Haitian descent,
demanded an apology from the president, saying his comments were
``unkind, divisive, elitist, and fly in the face of our nation's
values.''

\href{https://www.nytimes3xbfgragh.onion/interactive/2018/01/11/us/politics/trump-approval.html}{}

\includegraphics{https://static01.graylady3jvrrxbe.onion/images/2018/01/08/us/politics/trump-approval-1515625719653/trump-approval-1515625719653-articleLarge-v3.jpg}

\hypertarget{a-year-later-trump-is-less-popular-across-voting-blocs-see-by-how-much}{%
\subsection{A Year Later, Trump Is Less Popular Across Voting Blocs. See
by How
Much.}\label{a-year-later-trump-is-less-popular-across-voting-blocs-see-by-how-much}}

President Trump's approval rating fell across many demographic groups
over his first year in office, including among those seen as important
to his base.

``This behavior is unacceptable from the leader of our nation,'' Ms.
Love went on in an emotional statement that noted her heritage and that
said her parents ``never took a thing'' from the government while
achieving the American dream. ``The president must apologize to both the
American people and the nations he so wantonly maligned.''

``As an American, I am ashamed of the president,'' said Representative
Luis V. Gutiérrez, Democrat of Illinois. ``His comments are
disappointing, unbelievable, but not surprising.'' He added, we can now
``say with 100 percent confidence that the president is a racist who
does not share the values enshrined in our Constitution or Declaration
of Independence.''

The reactions were extraordinary bipartisan rebukes to a sitting
president, but they only fanned what has been a long-simmering debate
over Mr. Trump's views and talk on race.

Mr. Trump sought to have the final word late Thursday,
\href{https://twitter.com/realDonaldTrump/status/951675713089888256}{posting
on Twitter shortly before midnight}: ``The Democrats seem intent on
having people and drugs pour into our country from the Southern Border,
risking thousands of lives in the process. It is my duty to protect the
lives and safety of all Americans. We must build a Great Wall, think
Merit and end Lottery \& Chain. USA!''

As a candidate, Mr. Trump, who rose to political prominence questioning
the validity of President Barack Obama's birth certificate, branded
Mexican immigrants rapists and criminals, called for a ban on Muslims
entering the United States and was slow to disavow the support of the
former Ku Klux Klan leader David Duke.

As the president, Mr. Trump has ordered a broad immigration crackdown
while privately railing against immigrants from predominantly black
countries and has repeatedly stoked racial divisions, denouncing ``both
sides'' for violence after a white supremacist rally in Charlottesville,
Va., and singling out black athletes for failing to stand for the
national anthem before their games.

The episode at the White House, first
\href{https://www.washingtonpost.com/politics/trump-attacks-protections-for-immigrants-from-shithole-countries-in-oval-office-meeting/2018/01/11/bfc0725c-f711-11e7-91af-31ac729add94_story.html?utm_term=.8de714470fbf}{reported
by The Washington Post}, unfolded as Mr. Trump was hosting a meeting
with Senator Lindsey Graham, Republican of South Carolina, and Richard
J. Durbin, Democrat of Illinois, who are working to codify the
protections in the Deferred Action for Childhood Arrivals program, or
DACA, the Obama-era initiative that provided temporary work permits and
reprieves from deportation to immigrants brought to the United States as
children by their parents.

Also present were Representative Kevin McCarthy, Republican of
California and the majority leader; Senator David Perdue, Republican of
Georgia; Senator Tom Cotton, Republican of Arkansas; and Representative
Robert W. Goodlatte, Republican of Virginia and the chairman of the
Judiciary Committee.

None of the lawmakers would comment on Mr. Trump's remarks.

The plan outlined by Mr. Graham and Mr. Durbin, according to people
familiar with it, would also include more than \$2.5 billion for border
security and a grant of protected status for the parents of the
immigrants, known as Dreamers, who would be barred from sponsoring their
parents for citizenship.

Mr. Trump grew angry as the group detailed another aspect of the deal
--- a move to end the diversity visa lottery program and use some of the
50,000 visas that are annually distributed as part of the program to
protect vulnerable populations who have been living in the United States
under what is known as Temporary Protected Status. That was when Mr.
Durbin mentioned Haiti, prompting the president's criticism.

When the discussion turned to African nations, those with knowledge of
the conversation added, Mr. Trump asked why he would want ``all these
people from shithole countries,'' adding that the United States should
admit more people from places like Norway.

About 83 percent of Norway's population is ethnic Norwegian, according
to a 2017
\href{https://www.cia.gov/library/publications/the-world-factbook/geos/no.html}{C.I.A.
fact book}, making the country overwhelmingly white.

Mr. Trump has long argued that the United States should base legal
immigration on merit and skills rather than family ties, seeking new
entrants who are highly educated, capable of assimilating and unlikely
to use government programs for the poor. Some people familiar with his
comments argued privately on Thursday night that the president had only
tried to press that point, using salty language.

But it was the language he used that shocked and appalled many lawmakers
and created a public outcry --- the vulgar phrase Mr. Trump uttered
quickly began trending on Twitter --- overshadowing the substance of the
DACA talks, and with it, the future of the immigrants at risk of
deportation should those discussions fail.

Representative Cedric L. Richmond, Democrat of Louisiana and the
chairman of the Congressional Black Caucus, called the president's
closed-door comments ``yet another confirmation of his racially
insensitive and ignorant views'' and said they reinforced ``the concerns
that we hear every day, that the president's slogan, `Make America Great
Again,' is really code for `Make America White Again.'''

Senator Richard Blumenthal, Democrat of Connecticut, described the
comments as ``the most odious and insidious racism masquerading poorly
as immigration policy,'' and argued that they would make it more
difficult for the two parties to reach consensus on an immigration deal.

``It inflames and encourages the worst instinct and the basest dark side
of immigration issues,'' Mr. Blumenthal said. He added that he had
spoken with several Senate colleagues who expressed ``a combination of
disbelief and a sense of repugnance'' at what the president had said.

Mr. Durbin spoke with reporters briefly after the White House gathering,
but did not share what the president had said. Looking dejected, he said
he and Mr. Graham had gone to meet with Mr. Trump hoping to get the
president's blessing for their bipartisan plan.

\includegraphics{https://static01.graylady3jvrrxbe.onion/images/2018/01/15/us/politics/tv-news-still/tv-news-still-videoSixteenByNineJumbo1600.jpg}

``The president is not prepared to do that at this moment,'' Mr. Durbin
said, adding, ``I don't know what happens next.''

The meeting had gotten off to a grim start after Mr. Durbin and Mr.
Graham, who had been summoned by Mr. Trump to discuss their compromise
proposal, arrived to find a gaggle of Republicans they had not expected,
including immigration hard-liners who have been skeptical of a DACA
deal, filing into the room to discuss the plan.

One of them, Senator Tom Cotton, Republican of Arkansas, later offered a
blistering appraisal of the Durbin-Graham deal, calling it ``a joke of a
proposal'' and arguing that Democrats had not offered Republicans
``anything legitimate in return'' for accepting legal status for the
Dreamers.

The White House session was the second time this week that Mr. Trump has
met with members of Congress to address the fate of the Dreamers, whose
protections under DACA are set to expire in March after Mr. Trump
\href{https://www.nytimes3xbfgragh.onion/2017/09/05/us/politics/trump-daca-dreamers-immigration.html}{moved
to rescind the policy} in the fall.

On Tuesday, Mr. Trump convened a televised bipartisan negotiating
session on the issue in which he said he wanted lawmakers to negotiate a
``bill of love'' for the DACA recipients and tasked the Democratic and
Republican leaders in both the House and the Senate to negotiate a final
deal.

Senator John Cornyn, Republican of Texas and the No. 2 Senate
Republican, said Thursday that any immigration compromise could not
simply be hammered out by a small group of senators, referring to Mr.
Durbin and Mr. Graham.

``I think what the president told them is it's fine for them to have
negotiated what they think is a reasonable proposal, but what they need
to do is share that with others so that it will have broad enough
support to actually get passed,'' Mr. Cornyn said, adding, ``We need to
have more than six votes for a proposal.''

When he first moved to rescind the DACA program, Mr. Trump gave
lawmakers six months to come up with a replacement. But Democrats have
pressed to include a solution in a broad spending package, which must be
completed by a deadline of next Friday, when a short-term bill funding
the government will expire. They are under extraordinary pressure from
their progressive base to withhold their votes to keep the government
open unless the immigration measure is included.

One of the Senate negotiators, Jeff Flake, Republican of Arizona, cast
doubt on passing a bill by that deadline. But he expressed hope that the
deal being put together by Mr. Durbin's group could move ahead soon.

``It's the only game in town,'' Mr. Flake said. ``There's no other
bill.''

Advertisement

\protect\hyperlink{after-bottom}{Continue reading the main story}

\hypertarget{site-index}{%
\subsection{Site Index}\label{site-index}}

\hypertarget{site-information-navigation}{%
\subsection{Site Information
Navigation}\label{site-information-navigation}}

\begin{itemize}
\tightlist
\item
  \href{https://help.nytimes3xbfgragh.onion/hc/en-us/articles/115014792127-Copyright-notice}{©~2020~The
  New York Times Company}
\end{itemize}

\begin{itemize}
\tightlist
\item
  \href{https://www.nytco.com/}{NYTCo}
\item
  \href{https://help.nytimes3xbfgragh.onion/hc/en-us/articles/115015385887-Contact-Us}{Contact
  Us}
\item
  \href{https://www.nytco.com/careers/}{Work with us}
\item
  \href{https://nytmediakit.com/}{Advertise}
\item
  \href{http://www.tbrandstudio.com/}{T Brand Studio}
\item
  \href{https://www.nytimes3xbfgragh.onion/privacy/cookie-policy\#how-do-i-manage-trackers}{Your
  Ad Choices}
\item
  \href{https://www.nytimes3xbfgragh.onion/privacy}{Privacy}
\item
  \href{https://help.nytimes3xbfgragh.onion/hc/en-us/articles/115014893428-Terms-of-service}{Terms
  of Service}
\item
  \href{https://help.nytimes3xbfgragh.onion/hc/en-us/articles/115014893968-Terms-of-sale}{Terms
  of Sale}
\item
  \href{https://spiderbites.nytimes3xbfgragh.onion}{Site Map}
\item
  \href{https://help.nytimes3xbfgragh.onion/hc/en-us}{Help}
\item
  \href{https://www.nytimes3xbfgragh.onion/subscription?campaignId=37WXW}{Subscriptions}
\end{itemize}
