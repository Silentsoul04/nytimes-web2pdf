Sections

SEARCH

\protect\hyperlink{site-content}{Skip to
content}\protect\hyperlink{site-index}{Skip to site index}

\href{https://myaccount.nytimes3xbfgragh.onion/auth/login?response_type=cookie\&client_id=vi}{}

\href{https://www.nytimes3xbfgragh.onion/section/todayspaper}{Today's
Paper}

\href{/section/upshot}{The Upshot}\textbar{}Hate Paperwork? Medicaid
Recipients Will Be Drowning in It

\url{https://nyti.ms/2Deofmp}

\begin{itemize}
\item
\item
\item
\item
\item
\item
\end{itemize}

Advertisement

\protect\hyperlink{after-top}{Continue reading the main story}

Upshot

Supported by

\protect\hyperlink{after-sponsor}{Continue reading the main story}

\hypertarget{hate-paperwork-medicaid-recipients-will-be-drowning-in-it}{%
\section{Hate Paperwork? Medicaid Recipients Will Be Drowning in
It}\label{hate-paperwork-medicaid-recipients-will-be-drowning-in-it}}

Kentucky's new Medicaid waiver will ask low-income people to jump over
hurdles to keep their coverage. Evidence suggests that many will fail.

By \href{http://www.nytimes3xbfgragh.onion/by/margot-sanger-katz}{Margot
Sanger-Katz}

\begin{itemize}
\item
  Jan. 18, 2018
\item
  \begin{itemize}
  \item
  \item
  \item
  \item
  \item
  \item
  \end{itemize}
\end{itemize}

\includegraphics{https://static01.graylady3jvrrxbe.onion/images/2017/12/22/upshot/00up-tower/00up-tower-articleLarge.jpg?quality=75\&auto=webp\&disable=upscale}

In 2003, Washington State was facing a budget crisis and wanted to
reduce spending on Medicaid. Instead of requiring people to establish
their eligibility annually, the legislature began requiring them to do
so twice a year, and added some paperwork. It worked: Enrollment in the
health insurance program fell by more than 40,000 children in a year.

In the early 2000s, Louisiana wanted to maximize the number of eligible
children who signed up for coverage, so officials simplified the sign-up
process. It also worked: Enrollment surged, and the number of
administrative cancellations fell by 20 percentage points.

In 2006, Congress asked states to verify the citizenship of
beneficiaries by seeking their birth certificates. Across the country,
children's Medicaid enrollment dropped, despite scant evidence that
ineligible children had been signed up. That policy was reversed as part
of the Affordable Care Act in 2010.

The Trump administration's decision to approve a first-of-its-kind work
requirement for Kentucky's Medicaid program last week has inspired
concern that the program will
\href{https://www.nytimes3xbfgragh.onion/2018/01/11/upshot/medicaid-work-requirements-trump.html?_r=0}{leave
behind Medicaid beneficiaries} who are unable to find or keep work. But
a large body of social science suggests that the mere requirement of
documenting work hours is likely to cause many eligible people to lose
coverage, too.

``Without being tremendously well organized, it can be easy to fail,''
said Donald Moynihan, a professor of public affairs at the University of
Wisconsin-Madison, who is writing a book on the effects of
administrative burdens. Researchers have studied the ways complexity can
reduce sign-ups for
\href{https://institutional.vanguard.com/iam/pdf/CRRATEP_AutoEnrollDefault.pdf?cbdForceDomain=false}{workplace
pension plans}, participation in food stamps and
\href{http://law.osu.edu/electionlaw/litigation/documents/Ohio559.pdf}{turnout
in elections}, he noted. ``These sorts of little barriers are ways in
which humans get tripped up all the time when they're trying to do
something that might benefit them.''

Anyone who has ever forgotten to pay a bill on time, or struggled to
assemble all the necessary forms of identification before heading to the
D.M.V., is likely to sympathize with how administrative hurdles can
stymie someone. But these may be especially daunting for the poor, who
tend to have less stable work schedules and less access to resources
that can simplify compliance: reliable transportation, a bank account,
internet access. There is also a lot of research about the Medicaid
program, specifically, that shows that sign-ups fall when states make
their program more complicated.

The Kentucky program won't just create a work requirement for some
beneficiaries; it will set up a broader obstacle course of
administrative rules. Many beneficiaries will be asked to pay monthly
premiums to the state to retain their coverage, as little as \$1 a month
for some very poor families, who are unlikely to have bank accounts.

They will be asked to notify Medicaid officials any time their income
changes. Their benefits could rise or fall depending on whether they get
an annual checkup, or take a financial literacy course. Beneficiaries
who fail to renew their coverage promptly at the end of a year will be
locked out for as long as six months. Beneficiaries who are ``medically
frail'' can get an exemption from the work requirement, but they will
need to submit a doctor's note.

Kentucky officials argue that the changes will give beneficiaries
\href{https://www.nytimes3xbfgragh.onion/2018/01/12/health/kentucky-medicaid-work.html?rref=collection\%2Fbyline\%2Fabby-goodnough\&action=click\&contentCollection=undefined\&region=stream\&module=stream_unit\&version=latest\&contentPlacement=1\&pgtype=collection}{more
dignity and promote personal responsibility}. But they also estimate
that
\href{https://www.medicaid.gov/Medicaid-CHIP-Program-Information/By-Topics/Waivers/1115/downloads/ky/ky-health-pa2.pdf}{around
100,000 fewer people} will be enrolled in the program by the end of five
years. There are about 1.3 million Medicaid recipients in Kentucky.

Medicaid premiums are not unprecedented, and they've been studied. The
results have typically been
\href{https://www.kff.org/medicaid/issue-brief/the-effects-of-premiums-and-cost-sharing-on-low-income-populations-updated-review-of-research-findings/}{reductions
in enrollment}. Laura Dague, a professor of health policy at Texas A\&M
who has studied the effects of small premiums, found that both frequency
and cost tend to
\href{https://ccf.georgetown.edu/wp-content/uploads/2012/03/Dague-Premiums.pdf}{cause
drops in enrollment}. ``Any time the state requires more contact from
the eligible population or enrolled population, you lose people at those
places of contact,'' she said.

Kentucky now will ask its beneficiaries to interact with the state
monthly, both to pay premiums and to document the time they've worked or
pursued work. A few missed premiums or work filings could cost them
their coverage, even if they continue to work the required number of
hours.

Renewals, historically, have also been associated with drops in
enrollment, even among eligible people. Washington State officials were
surprised by their experiment in checking families' eligibility, said
Mary Wood, an assistant director of the state's Medicaid program. When
the state asked the families why they were dropping coverage for their
children, they learned that many families remained eligible but had
simply lost track of the paperwork.

``If you were busy and you forgot to do it, you would lose coverage,''
she said. Two years after making the policy change, Washington State
reversed it, as part of a broader effort at simplifying the sign-up
process. Since then, the state has invested heavily in making renewal a
default for most families, by searching electronic databases to figure
out who is eligible without having to ask them.

Other states have seen similar effects. Jason Helgerson, New York's
Medicaid director, was the Medicaid director in Wisconsin in 2008 when
the state started using data from other public programs to sign up
people for Medicaid.

``We picked up almost 100,000 people on our first day under the new
rule,'' he said.

Wisconsin and Washington haven't been the only states to push for
simplification. While states have typically tended to turn dials of
complexity up or down depending on state budget constraints and
political control, Medicaid programs started trending in the same
direction in recent years.

Louisiana made enrolling every eligible child a priority, and, by
\href{http://ccf.georgetown.edu/wp-content/uploads/2008/03/Louisiana-Expansion-Simplification-Outreach.pdf}{making
renewals easier}, it was able to slash plan cancellations.

The Affordable Care Act helped
\href{https://www.healthaffairs.org/do/10.1377/hpb20110127.77698/full/}{codify
that simplification}, by
\href{https://kaiserfamilyfoundation.files.wordpress.com/2013/04/8391.pdf}{requiring
all states} to use the same definitions of income, to use electronic
databases whenever possible, and to eliminate
\href{http://ccf.georgetown.edu/wp-content/uploads/2008/03/Louisiana-Expansion-Simplification-Outreach.pdf}{other
eligibility rules} that made it harder and more time consuming for
people to seek coverage. The
\href{https://www.nytimes3xbfgragh.onion/2015/01/29/upshot/the-goal-was-simplicity-instead-theres-a-many-headed-medicaid.html}{goal
of simplification} has been to enroll as many eligible people as
possible, and drive down the uninsured rate.

The Kentucky waiver will be a shift in the opposite direction, toward
more rules, paperwork and ways for Medicaid patients to lose their
coverage if they don't keep up. Nine other states have asked to make
similar program changes. Champions for the policy say the new rules will
help teach low-income people to take more responsibility for their
health.

Advocates for the poor are disappointed. ``It's sad to me that we're
going to have to go through this proof again if these things really get
implemented,'' said Judith Solomon, a vice president for health policy
at the liberal Center on Budget and Policy Priorities. ``There is no
doubt in my mind that the experience will be the same.''

Advertisement

\protect\hyperlink{after-bottom}{Continue reading the main story}

\hypertarget{site-index}{%
\subsection{Site Index}\label{site-index}}

\hypertarget{site-information-navigation}{%
\subsection{Site Information
Navigation}\label{site-information-navigation}}

\begin{itemize}
\tightlist
\item
  \href{https://help.nytimes3xbfgragh.onion/hc/en-us/articles/115014792127-Copyright-notice}{©~2020~The
  New York Times Company}
\end{itemize}

\begin{itemize}
\tightlist
\item
  \href{https://www.nytco.com/}{NYTCo}
\item
  \href{https://help.nytimes3xbfgragh.onion/hc/en-us/articles/115015385887-Contact-Us}{Contact
  Us}
\item
  \href{https://www.nytco.com/careers/}{Work with us}
\item
  \href{https://nytmediakit.com/}{Advertise}
\item
  \href{http://www.tbrandstudio.com/}{T Brand Studio}
\item
  \href{https://www.nytimes3xbfgragh.onion/privacy/cookie-policy\#how-do-i-manage-trackers}{Your
  Ad Choices}
\item
  \href{https://www.nytimes3xbfgragh.onion/privacy}{Privacy}
\item
  \href{https://help.nytimes3xbfgragh.onion/hc/en-us/articles/115014893428-Terms-of-service}{Terms
  of Service}
\item
  \href{https://help.nytimes3xbfgragh.onion/hc/en-us/articles/115014893968-Terms-of-sale}{Terms
  of Sale}
\item
  \href{https://spiderbites.nytimes3xbfgragh.onion}{Site Map}
\item
  \href{https://help.nytimes3xbfgragh.onion/hc/en-us}{Help}
\item
  \href{https://www.nytimes3xbfgragh.onion/subscription?campaignId=37WXW}{Subscriptions}
\end{itemize}
