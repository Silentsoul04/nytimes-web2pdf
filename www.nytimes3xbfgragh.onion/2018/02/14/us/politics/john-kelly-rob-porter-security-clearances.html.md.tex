Sections

SEARCH

\protect\hyperlink{site-content}{Skip to
content}\protect\hyperlink{site-index}{Skip to site index}

\href{https://www.nytimes3xbfgragh.onion/section/politics}{Politics}

\href{https://myaccount.nytimes3xbfgragh.onion/auth/login?response_type=cookie\&client_id=vi}{}

\href{https://www.nytimes3xbfgragh.onion/section/todayspaper}{Today's
Paper}

\href{/section/politics}{Politics}\textbar{}Trump, a Week After Porter
Resigned, Says He's `Totally Opposed' to Domestic Violence

\url{https://nyti.ms/2BZ8ra6}

\begin{itemize}
\item
\item
\item
\item
\item
\item
\end{itemize}

Advertisement

\protect\hyperlink{after-top}{Continue reading the main story}

Supported by

\protect\hyperlink{after-sponsor}{Continue reading the main story}

\hypertarget{trump-a-week-after-porter-resigned-says-hes-totally-opposed-to-domestic-violence}{%
\section{Trump, a Week After Porter Resigned, Says He's `Totally
Opposed' to Domestic
Violence}\label{trump-a-week-after-porter-resigned-says-hes-totally-opposed-to-domestic-violence}}

\includegraphics{https://static01.graylady3jvrrxbe.onion/images/2018/02/15/us/politics/15dc-clearance/merlin_133463066_f2d67582-b67d-443d-8a49-39d0fbceeb1a-articleLarge.jpg?quality=75\&auto=webp\&disable=upscale}

By
\href{https://www.nytimes3xbfgragh.onion/by/julie-hirschfeld-davis}{Julie
Hirschfeld Davis},
\href{http://www.nytimes3xbfgragh.onion/by/maggie-haberman}{Maggie
Haberman} and
\href{http://www.nytimes3xbfgragh.onion/by/michael-d-shear}{Michael D.
Shear}

\begin{itemize}
\item
  Feb. 14, 2018
\item
  \begin{itemize}
  \item
  \item
  \item
  \item
  \item
  \item
  \end{itemize}
\end{itemize}

WASHINGTON --- One week after Rob Porter, his staff secretary, resigned
amid spousal abuse allegations, President Trump said on Wednesday that
he was ``totally opposed to domestic violence,'' his first condemnation
of the alleged conduct behind a scandal that has engulfed the White
House.

His statement, which members of both parties had said was long overdue,
came as John F. Kelly, Mr. Trump's chief of staff, faced new questions
about his handling of Mr. Porter's case, including how he could have
held a temporary high-level security clearance for more than a year in
light of the allegations, and as committees in both the House and the
Senate announced they would investigate the circumstances surrounding
the granting of Mr. Porter's clearance.

Mr. Kelly told senior aides last fall to put an immediate end to
granting new interim security clearances like the one given to Mr.
Porter and directed them to resolve any issues preventing employees who
held them at the time from receiving a full clearance, according to two
people familiar with the discussion.

At a meeting in the West Wing, Mr. Kelly said he was assigning Kirstjen
Nielsen, then his deputy, to enforce the new policy, the people said.
But it is not known whether Mr. Kelly, Ms. Nielsen or any other senior
officials sought to delve into why Mr. Porter was operating with only an
interim clearance.

\href{https://www.nytimes3xbfgragh.onion/2018/02/07/us/politics/rob-porter-resigns-abuse-white-house-staff-secretary.html}{Mr.
Porter resigned} last week after allegations that he had abused his two
former wives were
\href{http://www.dailymail.co.uk/news/article-5359731/Ex-wife-Rob-Porter-Trumps-secretary-tells-marriage.html}{reported
by The Daily Mail}. In the following days, Mr. Trump repeatedly declined
to comment publicly on the episode --- or on the issue of domestic
violence --- even as he went out of his way to praise Mr. Porter and
express sympathy with him, noting that he had denied the accusations.

But during a tax event on Wednesday at the White House, the president
finally weighed in, saying: ``I'm totally opposed to domestic violence
of any kind. Everyone knows that, and it almost wouldn't even have to be
said.''

The F.B.I. learned about the allegations by the two women soon after Mr.
Trump was inaugurated last year. But spokesmen and spokeswomen for the
White House have insisted that no senior White House officials knew of
them until last week. They have said that the career government
employees at the White House Personnel Security Office who processed the
clearances did not tell them about the allegations uncovered by the
F.B.I.

``I am interested in how someone with credible allegations of domestic
abuse, plural, can be hired,'' said Representative Trey Gowdy,
Republican of South Carolina and the chairman of the Oversight and
Government Reform Committee, which will look into Mr. Porter's clearance
and the granting of interim clearances generally. He called domestic
abuse ``a particularly insidious crime'' that bears serious
consideration in both the hiring and clearance process.

Senator Charles E. Grassley, Republican of Iowa and the chairman of the
Judiciary Committee, said that he had instructed his staff to also begin
looking into the clearance process and into whether the proper protocols
were followed at the White House, while six Democratic senators said
that they were concerned about whether there had been ``any mishandling
of classified information'' because of the interim clearances.

The six senators asked Christopher A. Wray, the F.B.I. director, for the
names of other employees at the White House who were working ``without
being able to obtain a permanent security clearance.''

Mr. Porter was allowed to continue serving in his post with an interim
clearance while awaiting approval from the Personnel Security Office for
a permanent clearance. The little-known office of about a dozen
employees, on the fifth floor of the New Executive Office Building down
the block from the White House, receives background information from the
F.B.I. and determines whether officials should be cleared to have access
to sensitive information.

Mr. Wray testified on Tuesday that the F.B.I. had sent the White House a
preliminary background report on Mr. Porter in March, a full
investigation in July and a more detailed accounting in November. The
agency closed its inquiry in January.

But White House officials say the security office, where only one out of
three officials who had to sign off on granting Mr. Porter's clearance
had weighed in, never made a final recommendation.

When senior White House aides have been flagged for concern during prior
administrations, that information was typically sent to the general
counsel at the Office of Administration, which oversees the security
office, who would quickly pass the information to the White House
counsel or the deputy chief of staff, according to people familiar with
the process.

Several former White House officials and people familiar with the
security clearance process argue that it is highly unusual and dangerous
for people at Mr. Porter's level to do their jobs indefinitely without
the proper clearance.

``The presumption is not that you get clearance; the presumption is
actually that you do not get clearance, and you have to prove yourself
worthy of having access to vital information,'' said Max Stier, the
president and chief executive of the Partnership for Public Service, a
nonprofit organization that specializes in federal government management
issues.

``Interim clearance is simply to allow people to operate for a short
time in their jobs while they await a full check, but if you get red
flags in that interim clearance process, then it means that you
shouldn't have that interim clearance,'' he added.

Even as Mr. Kelly's past efforts to deal with security clearance issues
at the White House were becoming clearer, his shifting public responses
to the revelations of Mr. Porter's past were coming under further
scrutiny.

Three people briefed on the situation said that Mr. Kelly learned that
the accusations would be published in The Mail last Tuesday, before
leaving for a visit to Capitol Hill. In a meeting with a group of aides,
including several from the press office, everyone agreed that Mr. Porter
would have to resign, the people briefed on the situation said, and a
statement from Mr. Kelly was drafted to provide to The Mail.

But Mr. Porter continued to deny the accusations from his former wives.
One aide in the discussions pushed back on the belief that Mr. Porter
should resign, saying that these were mere allegations, and that if Mr.
Porter were forced out over them, other people could be forced from
their posts any time an allegation was made. Other aides agreed, and
argued for waiting for the story to play out.

At that point, they reached out to Mr. Kelly, who had left for the visit
to the Capitol, by phone, the people said, and he said he agreed,
telling them to make his statement about Mr. Porter more supportive. Mr.
Kelly dictated specific language that he wanted in the statement to
Sarah Huckabee Sanders, the White House press secretary.

A short time later, The Mail published Mr. Kelly's statement calling Mr.
Porter ``a man of true integrity and honor,'' someone with whom he was
``proud to serve,'' and who had faced ``vile'' accusations from his
former wives.

But soon after the article appeared, Mr. Kelly, who by then had returned
to the White House, heard from someone with more detailed knowledge of
the allegations against Mr. Porter that more damning information was
about to come out, and that the chief of staff should not put himself in
the position of being Mr. Porter's main defender. The people briefed on
the discussions would not identify that person.

The conversation prompted Mr. Kelly to go back to Mr. Porter, this time
telling him that he ``knows what he has to do,'' according to those
briefed on the discussion.

Mr. Porter agreed to resign, and told his staff that he was stepping
down, a White House official said. But the next morning, Wednesday of
last week, he told White House aides he wanted to leave on his own terms
and help with the transition. Mr. Kelly told him that he had to leave
his job, but he agreed to let Mr. Porter attempt a more graceful exit
with an unclear departure date, those briefed said.

The scandal has
\href{https://www.nytimes3xbfgragh.onion/2018/02/09/us/politics/trump-porter-abuse.html}{placed
Mr. Kelly's job in jeopardy}, leading Mr. Trump to
\href{https://www.nytimes3xbfgragh.onion/2018/02/08/us/politics/kelly-trump.html}{complain
privately about him} and sound out confidants about potential
replacements, including Gary D. Cohn, the director of his National
Economic Council, and Representative Kevin McCarthy, Republican of
California and the majority leader.

Mr. Trump is said to seem more favorable toward Mr. McCarthy in some of
his discussions, seeing him as someone who would be a more willing
subordinate than Mr. Cohn might be, according to a person with direct
knowledge of the discussions. Yet in other conversations, Mr. Trump has
indicated that Mr. Cohn is his pick.

Several of Mr. Trump's advisers believe the president, who has a long
history of quizzing aides about one another behind their backs without
taking action, might just be venting. And his interactions with Mr.
Kelly have remained mostly positive, according to two West Wing advisers
who have witnessed them together.

Still, Mr. Kelly is facing a restive staff that includes some aides who
have grown increasingly upset with his handling of the Porter situation,
including a constantly shifting set of explanations and timelines that
have raised further questions about the White House's credibility.

As new information about the scandal emerged Wednesday, officials twice
postponed, then ultimately canceled, the daily briefing by the press
secretary. Press aides told reporters the question-and-answer session
had been scrapped in light of a
\href{https://www.nytimes3xbfgragh.onion/2018/02/14/us/parkland-school-shooting.html?hp\&action=click\&pgtype=Homepage\&clickSource=story-heading\&module=span-ab-top-region\&region=top-news\&WT.nav=top-news}{fatal
school shooting} in South Florida.

Advertisement

\protect\hyperlink{after-bottom}{Continue reading the main story}

\hypertarget{site-index}{%
\subsection{Site Index}\label{site-index}}

\hypertarget{site-information-navigation}{%
\subsection{Site Information
Navigation}\label{site-information-navigation}}

\begin{itemize}
\tightlist
\item
  \href{https://help.nytimes3xbfgragh.onion/hc/en-us/articles/115014792127-Copyright-notice}{©~2020~The
  New York Times Company}
\end{itemize}

\begin{itemize}
\tightlist
\item
  \href{https://www.nytco.com/}{NYTCo}
\item
  \href{https://help.nytimes3xbfgragh.onion/hc/en-us/articles/115015385887-Contact-Us}{Contact
  Us}
\item
  \href{https://www.nytco.com/careers/}{Work with us}
\item
  \href{https://nytmediakit.com/}{Advertise}
\item
  \href{http://www.tbrandstudio.com/}{T Brand Studio}
\item
  \href{https://www.nytimes3xbfgragh.onion/privacy/cookie-policy\#how-do-i-manage-trackers}{Your
  Ad Choices}
\item
  \href{https://www.nytimes3xbfgragh.onion/privacy}{Privacy}
\item
  \href{https://help.nytimes3xbfgragh.onion/hc/en-us/articles/115014893428-Terms-of-service}{Terms
  of Service}
\item
  \href{https://help.nytimes3xbfgragh.onion/hc/en-us/articles/115014893968-Terms-of-sale}{Terms
  of Sale}
\item
  \href{https://spiderbites.nytimes3xbfgragh.onion}{Site Map}
\item
  \href{https://help.nytimes3xbfgragh.onion/hc/en-us}{Help}
\item
  \href{https://www.nytimes3xbfgragh.onion/subscription?campaignId=37WXW}{Subscriptions}
\end{itemize}
