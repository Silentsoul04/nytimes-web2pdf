Sections

SEARCH

\protect\hyperlink{site-content}{Skip to
content}\protect\hyperlink{site-index}{Skip to site index}

\href{https://www.nytimes3xbfgragh.onion/pages/automobiles/wheels/index.html}{Wheels}

\href{https://myaccount.nytimes3xbfgragh.onion/auth/login?response_type=cookie\&client_id=vi}{}

\href{https://www.nytimes3xbfgragh.onion/section/todayspaper}{Today's
Paper}

\href{/pages/automobiles/wheels/index.html}{Wheels}\textbar{}More Luxury
Buyers Ditch the Imports and Pick Up a Truck

\url{https://nyti.ms/2C2r8di}

\begin{itemize}
\item
\item
\item
\item
\item
\end{itemize}

Advertisement

\protect\hyperlink{after-top}{Continue reading the main story}

Supported by

\protect\hyperlink{after-sponsor}{Continue reading the main story}

\href{/column/wheels}{Wheels}

\hypertarget{more-luxury-buyers-ditch-the-imports-and-pick-up-a-truck}{%
\section{More Luxury Buyers Ditch the Imports and Pick Up a
Truck}\label{more-luxury-buyers-ditch-the-imports-and-pick-up-a-truck}}

\includegraphics{https://static01.graylady3jvrrxbe.onion/images/2018/02/16/business/16WHEELS1/merlin_133871898_68f70028-e776-4d29-8755-b92b7a9361e5-articleLarge.jpg?quality=75\&auto=webp\&disable=upscale}

By \href{https://www.nytimes3xbfgragh.onion/by/neal-e-boudette}{Neal E.
Boudette}

\begin{itemize}
\item
  Feb. 15, 2018
\item
  \begin{itemize}
  \item
  \item
  \item
  \item
  \item
  \end{itemize}
\end{itemize}

When Lee Victorian was looking for an upscale car to complement his
wife's BMW last year, he was leaning toward an Audi A6 --- a sedan whose
acceleration, refinement and dazzling array of advanced technologies,
like automatic braking and radar-based cruise control, he found
alluring.

But what he drove off the lot was an entirely different kind of premium
vehicle, and one more luxury buyers are choosing: a pickup truck.

Mr. Victorian, a retired Michigan state trooper, bought a Raptor version
of the Ford F-150. The Raptor is a truck with the soul of a racecar: It
has a 450-horsepower engine, a 10-speed transmission, electronic ride
settings for seven different road surfaces, big chrome wheels, a power
tailgate, cameras at all four corners and an adaptive cruise control
system similar to the Audi's. With all those options, the sticker price
came to about \$80,000.

``Man, this truck is so slick,'' Mr. Victorian said. ``I stop at a light
and people give me the thumbs up and take pictures of it. The truck is
the celebrity.''

For the last few years, the auto industry has been roiled by a
significant shift in consumer tastes. In droves, Americans are turning
their backs on family sedans and small cars and flocking to bigger,
roomier models like sport utility vehicles and trucks. In January, two
of every three new vehicles sold were classified as trucks, including
S.U.V.s, pickups, minivans and the lighter cousins of S.U.V.s known as
crossovers.

Now a new dimension to this trend is emerging: Even upscale buyers who
long favored Lexus, Cadillac, Jaguar and the German luxury brands are
gravitating to trucks and S.U.V.s. What they are buying are often
special-edition, fully loaded models, like Mr. Victorian's Raptor, that
sell for as much as or more than BMW's flagship 7 Series sedan.

``We are seeing it,'' said Tom Libby, an auto industry analyst at the
research firm IHS Markit. ``There is movement from luxury cars to luxury
trucks.''

General Motors' GMC brand --- which sells only trucks and S.U.V.s ---
accounted for 11.3 percent of domestic sales of models with an average
price of \$60,000 or more in 2017, according to data from Edmunds.com.
Five years earlier, the brand made up a mere 0.1 percent of those sales.

Ford and Chevrolet saw similar but smaller jumps, driven by increasing
high-end truck and S.U.V. sales. At the same time, the portion of
over-\$60,000 sales for luxury brands including Porsche, Mercedes-Benz,
Lexus, Jaguar and Cadillac shrank.

That is providing a tailwind for the Detroit automakers when overall
new-vehicle sales in the United States are slowing. General Motors, Ford
and Fiat Chrysler, with its Jeep brand, dominate in trucks and S.U.V.s,
and now they're scrambling to roll out more high-end versions.

It's a competitive --- and crucial --- segment. With demand for cars
shriveling, the Detroit three and even some foreign manufacturers
acknowledge they are now losing money on many of the cars they sell. But
a \$60,000 truck can generate tens of thousands of dollars in operating
profit.

At a recent investor conference, G.M. outlined a plan to produce more of
the pricey Denali versions of GMC S.U.V.s and trucks. The company showed
data indicating that the Denali line had an average sale price of
\$56,000 --- more than the average price of a BMW, a Mercedes-Benz or an
Audi.

``This thing,'' G.M.'s president, Dan Ammann, said of the Denali line,
``is a money machine.''

The other Detroit carmakers are heading in the same direction.

In October, Ford began making new versions of its eight-passenger Ford
Expedition and Lincoln Navigator full-size S.U.V.s, and already has
decided to make 25 percent more this year than it originally planned. In
January, Navigators sold for an average of \$77,000, thanks to strong
sales of the top-of-the-line Black Label edition.

Fiat Chrysler is preparing to add more Jeep models, including a pickup
and a full-size Grand Wagoneer.

In 2017, S.U.V.s and crossovers made up 41 percent of the market in the
United States, up from 30 percent in 2013, according to Autodata. Luxury
cars have gone in the opposite direction: They made up 5.4 percent of
the market last year, down from 7.5 percent four years earlier.

And the priciest S.U.V.s and trucks are selling fastest. The high-end
Lariat, King Ranch and Raptor models make up more than half of all F-150
sales, up from one-third a few years ago. Denali editions account for 29
percent of GMC's sales, up from 21 percent.

Low gasoline prices are one reason that sales of high-end trucks are
rising. Years ago, pickups and big S.U.V.s often traveled only 11 or 12
miles on a gallon of gas. Today, their fuel economy is often double
that.

``The complaint that S.U.V.s are horrible on gas is not such a roadblock
anymore,'' said Mark Scarpelli, owner of two Chevrolet dealerships and a
Chrysler-Dodge-Jeep franchise in Illinois.

\includegraphics{https://static01.graylady3jvrrxbe.onion/images/2018/02/16/business/16WHEELS2/merlin_133871958_e99afd80-fff1-49d3-a367-14dc8f64ea80-articleLarge.jpg?quality=75\&auto=webp\&disable=upscale}

At the same time, automakers have appointed special-edition S.U.V.s and
trucks with the same kinds of advanced technologies and comfort features
that consumers once found only in luxury cars. Want an interior trimmed
in African mahogany? You can get it in the Black Label Navigator. Want
an S.U.V. that accelerates like a Porsche (and isn't a Porsche)? Try the
Jeep Grand Cherokee Trackhawk and its 707-horsepower V8.

Chuck Ducher, a retired school psychologist in Onsted, Mich., just
bought an F-150 Lariat with a bevy of options, including heated rear
seats. ``I can put my mother back there, and she's in heaven,'' he said.
``There's no doubt in my mind this is a luxury vehicle.''

Wes Lutz, owner of a Chrysler-Dodge-Jeep dealership in Jackson, Mich.,
said he was surprised at the way customers were snapping up the most
expensive models on his lot. This month, he had two Trackhawks this
month, each with a sticker price of \$93,000.

``They won't be here more than a few weeks,'' he said. ``It's
incredible. We never used to play in that price range.''

Increasing competition from upmarket S.U.V.s and trucks is adding to the
struggles of the luxury makers. Most have long relied on cars for the
bulk of their sales, and are suffering now that bigger vehicles are in
favor. In 2017, for example, BMW's sales to individual customers at
dealerships in the United States fell more than 5 percent, according
data shared among automakers. The decline in BMW's total sales was less
because of a big jump in sales to rental car fleets, a type of customer
that luxury brands tended to shun in the past.

Out in Tacoma, Wash., Gary Gilchrist sees the trend just about every
week at his GMC dealership.

``We've been taking in Lexuses on trade-ins, BMWs,'' he said. This
month, he said, a customer turned in a 2012 BMW 550i and bought a
\$71,000 GMC Sierra Denali pickup.

``People used to want German cars for the image factor,'' Mr. Gilchrist
said. ``Now, if you have a Denali, you get that. People turn their heads
to look.''

Advertisement

\protect\hyperlink{after-bottom}{Continue reading the main story}

\hypertarget{site-index}{%
\subsection{Site Index}\label{site-index}}

\hypertarget{site-information-navigation}{%
\subsection{Site Information
Navigation}\label{site-information-navigation}}

\begin{itemize}
\tightlist
\item
  \href{https://help.nytimes3xbfgragh.onion/hc/en-us/articles/115014792127-Copyright-notice}{©~2020~The
  New York Times Company}
\end{itemize}

\begin{itemize}
\tightlist
\item
  \href{https://www.nytco.com/}{NYTCo}
\item
  \href{https://help.nytimes3xbfgragh.onion/hc/en-us/articles/115015385887-Contact-Us}{Contact
  Us}
\item
  \href{https://www.nytco.com/careers/}{Work with us}
\item
  \href{https://nytmediakit.com/}{Advertise}
\item
  \href{http://www.tbrandstudio.com/}{T Brand Studio}
\item
  \href{https://www.nytimes3xbfgragh.onion/privacy/cookie-policy\#how-do-i-manage-trackers}{Your
  Ad Choices}
\item
  \href{https://www.nytimes3xbfgragh.onion/privacy}{Privacy}
\item
  \href{https://help.nytimes3xbfgragh.onion/hc/en-us/articles/115014893428-Terms-of-service}{Terms
  of Service}
\item
  \href{https://help.nytimes3xbfgragh.onion/hc/en-us/articles/115014893968-Terms-of-sale}{Terms
  of Sale}
\item
  \href{https://spiderbites.nytimes3xbfgragh.onion}{Site Map}
\item
  \href{https://help.nytimes3xbfgragh.onion/hc/en-us}{Help}
\item
  \href{https://www.nytimes3xbfgragh.onion/subscription?campaignId=37WXW}{Subscriptions}
\end{itemize}
