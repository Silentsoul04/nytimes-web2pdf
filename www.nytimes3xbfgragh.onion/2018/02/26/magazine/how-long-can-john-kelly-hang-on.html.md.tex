How Long Can John Kelly Hang On?

\url{https://nyti.ms/2sYWYUB}

\begin{itemize}
\item
\item
\item
\item
\item
\item
\end{itemize}

\includegraphics{https://static01.graylady3jvrrxbe.onion/images/2018/03/04/magazine/04mag-kelly1/04mag-kelly1-articleLarge-v3.jpg?quality=75\&auto=webp\&disable=upscale}

Sections

\protect\hyperlink{site-content}{Skip to
content}\protect\hyperlink{site-index}{Skip to site index}

Feature

\hypertarget{how-long-can-john-kelly-hang-on}{%
\section{How Long Can John Kelly Hang
On?}\label{how-long-can-john-kelly-hang-on}}

Last year, Democrats and Republicans alike agreed that if anyone could
bring order to the Trump administration, it was the retired four-star
Marine general. Were they wrong?

John Kelly, the White House chief of staff, observing a meeting between
President Trump and members of Congress.Credit...Tom Brenner for The New
York Times

Supported by

\protect\hyperlink{after-sponsor}{Continue reading the main story}

By \href{https://www.nytimes3xbfgragh.onion/by/matt-flegenheimer}{Matt
Flegenheimer}

\begin{itemize}
\item
  Feb. 26, 2018
\item
  \begin{itemize}
  \item
  \item
  \item
  \item
  \item
  \item
  \end{itemize}
\end{itemize}

On a Saturday afternoon last July, the day after John Kelly agreed to
become President Trump's chief of staff, an email arrived in his inbox.
``Congratulations!'' it began. ``(I think!)''

The sender was Philippe Reines, Hillary Clinton's longtime aide and
image buffer since her time in the Senate. Reines met Kelly while
working for Clinton at the State Department, when Kelly was a senior
military aide to Leon Panetta, then the defense secretary. The two
stayed in touch. After Kelly left military service in 2016, he joined
the advisory board at Reines's consulting firm, Beacon Global
Strategies, cashing a couple of checks before he was summoned to Trump
Tower.

``Can't say I'm rooting for your boss, but I'm absolutely rooting for
you,'' Reines wrote to Kelly. ``Especially if it takes the edges off
him.'' He offered some unsolicited advice: Stay off television. Tend to
the mystique. Let the Kellyanne Conways and Anthony Scaramuccis talk up
the president on cable. ``You don't want to be in that basket,'' Reines
said.

Kelly replied the next day. ``Thanks for taking the time,'' he wrote.
``I came to these same conclusions. I may be in this job for a day, or a
few years, but I will stay true to my values. We are in dangerous times,
Philippe, and the POTUS --- any POTUS --- needs all the help he can get.
What I do I do for the country. That's been my North Star for 46+ years
of service and it's worked thus far.''

Seven months later, over Presidents' Day weekend, Air Force One touched
down, as it often does, in Florida, for a spell of distinctly Trumpian
president-ing: Trump visited a hospital that treated victims of the
Parkland school massacre before decamping to Mar-a-Lago, where a Studio
54-style disco party was waiting. The next night, he shared a meal at
the club with his adult sons, Don Jr. and Eric, and Geraldo Rivera.
Near, but not so near, was Kelly, dining at another table. Some guests
approached Kelly to pay tribute, thanking him for keeping the president
on course. ``That's what the White House needs: discipline,'' Wayne
Allyn Root, a Trump-loving radio host who introduced himself to Kelly
that night, told me. ``I won't say he's a celebrity. That's probably a
bad word. But he's a person that's respected by everybody.''

For most of his 67 years, Kelly, a retired four-star Marine general who
projects the weary asceticism of a TV cop perpetually two weeks from
retirement, might have been out of place among Mar-a-Lago's bronzed
faces and gilded trimmings. But by mid-February, this was one of the few
rooms where he could still expect such a warm reception. Back in
Washington, the
\href{https://www.nytimes3xbfgragh.onion/2018/02/19/us/politics/trump-kelly-porter.html}{scandal}
surrounding his handling of abuse claims against a top aide, Rob Porter,
hummed through its second week --- an unusual longevity in a White House
where news of a Trump lawyer's pre-election hush payment to a porn star
had come and gone without great consequence. More surprising still was
how quickly, and unshakably, the crisis attached itself to Kelly, whose
sins --- praising the aide too forcefully before his departure,
purportedly sitting on the allegations for months without acting ---
felt airlifted from an era of more traditional Washington cover-ups.
Kelly held onto his job through the weekend, which did not initially
seem like a given. But Trump had already been musing privately about
possible replacements.

Even among Trump critics, Kelly once inspired uncommon sympathy. While
other high-level officials, like
\href{https://www.nytimes3xbfgragh.onion/2017/07/19/us/politics/trump-interview-sessions-russia.html}{Jeff
Sessions} and
\href{https://www.nytimes3xbfgragh.onion/2017/10/17/magazine/rex-tillerson-and-the-unraveling-of-the-state-department.html}{Rex
Tillerson}, had invited doubts about how long they could possibly
tolerate working in the administration, Kelly's responsibilities seemed
uniquely masochistic: He was the chief disciplinarian in a famously
undisciplined White House. ``You never run into somebody like Trump in
the military,'' Panetta told me. ``They'd usually get kicked out.'' The
job itself was premised on a paradox: If Trump weren't Trump, Kelly's
position would be bearable. And if Trump weren't Trump, you would not
need a John Kelly.

Now, Kelly's struggle has grown lonelier --- informed, even before the
Porter affair, by yawning cracks in his once-broad base of support. Some
of Trump's deepest skeptics had convinced themselves, despite a career's
worth of counterevidence, that a well-traveled military man might temper
the president's instincts, particularly on immigration. Then there were
Kelly's friends, who had already worried that the job was bending Kelly
to its will, and not the other way around.

\textbf{Shortly after} Election Day, Senator Tom Cotton, the Arkansas
Republican, went to Trump Tower to see the president-elect. Trump was
preparing to staff his cabinet, and he asked Cotton, an Iraq war
veteran, to name the best general of this generation. Cotton chose
Kelly. ``The president didn't know about General Kelly and asked me to
tell him more,'' Cotton told me. In December, Trump nominated Kelly to
run the Department of Homeland Security.

Unlike many prospective additions to the administration, Kelly had
stayed out of the 2016 campaign. In an
\href{http://foreignpolicy.com/2016/07/11/ex-general-to-top-brass-stay-out-of-the-cesspool-of-domestic-politics/}{interview
with Foreign Policy} that July, he said he would be willing to serve in
either a Trump or a Clinton administration but made plain his distaste
for the ``cesspool of domestic politics.'' ``To join in the political
fray, I don't think it convinces anyone,'' he said, chiding fellow
military leaders who had lined up behind candidates. ``It just becomes a
talking point on CNN.''

His confirmation process was mostly incident-free, specked with the kind
of John Wayne dialogue that had dazzled Trump from the start. Before
Kelly's Senate hearing, an aide who helped him prepare, Blain Rethmeier,
noticed the general had neglected to attach a flag pin to his lapel.
``Blain,'' Kelly told him, declining the pin, ``I am an American flag.''
He was confirmed on Inauguration Day, 88 votes to 11. ``A great
choice,'' Senator Chuck Schumer, the Democratic leader, said at the
time.

Kelly grew up in Boston's Brighton section, the Catholic-school-educated
son of a postal worker; fellow Marines later joked that his accent
required a Boston-to-English translator. In his own telling, he was
shaped by two forces in the neighborhood: men who had worn the military
uniform --- his father, his uncles, everyone on the block --- and the
ubiquitous drug use among his peers. ``He would claim that growing up in
South Boston, he lost most of everybody he grew up with to drugs,'' Adam
Isacson, the director of defense oversight at the Washington Office on
Latin America, told me, recalling meetings with Kelly when he led the
United States Southern Command. ``It was part of his persona.''

By the time the Iraq war began, Kelly, by then a three-decade veteran of
the corps, had become the first Marine promoted to brigadier general in
an active combat zone in over half a century, according to the Marines.
On Barack Obama's first trip to Iraq as the presumed Democratic nominee
in July 2008, Kelly rode with him in an armored truck across Anbar
Province. His career advanced rapidly during Obama's first term, and in
2012, he was named to head the Southern Command (``Southcom''), charged
with overseeing United States military operations in Central and South
America and the Caribbean.

Under Obama, Kelly --- a typically conservative Marine, friends say ---
was nominally tasked with steering the country toward policies he often
abhorred. After officials pushed the repeal of ``don't ask, don't
tell,'' Kelly seethed. ``Marines will die from this,'' he told
colleagues at the Pentagon. (The Trump White House disputes this.) He
was also a fierce defender of the status quo at Guantánamo Bay, which
fell under his command, publicly criticizing efforts to close the
facility and chafing at media accounts that humanized those being held
there. When some detainees began a hunger strike during Obama's second
term, Kelly feared that the issue was being framed too sympathetically.
His charges were instructed to call the act ``long-term nonreligious
fasting'' instead.

But perhaps Kelly's most instructive experiences at Southcom involved
the scourge of trafficking --- drugs, weapons, people. From command
headquarters, just outside Miami, he tracked narcotics production in
Colombia and Guatemala. At a forum in 2015, Kelly recalled meeting an
official from Customs and Border Protection who oversaw some 200 miles
of the Mexico-Texas line. ``I asked her how much cocaine she got last
year. She said, proud as she could be, `642 pounds,' '' Kelly said.
``That's pocket litter to me. I got 191 metric tons last year.'' He also
charted the rise of MS-13, an international criminal gang that had
become a security threat across Central America.

Kelly seemed to see immigration almost entirely through the prism of
security --- paralleling Trump's campaign, which nodded to the zero-sum
economic view of immigration prominent on the right for years, but
dwelled far more on
\href{https://www.nytimes3xbfgragh.onion/2016/09/02/us/politics/transcript-trump-immigration-speech.html}{blood-and-guts
anecdotes} of violent crime by immigrants. Although Southcom's area of
responsibility does not include Mexico, this did not discourage Kelly
from holding forth on the country, making claims that, according to
Obama-administration officials, sometimes contradicted the intelligence
of Northcom, the Pentagon command covering Mexico. At the height of the
Ebola scare of 2014, he
\href{https://www.youtube.com/watch?v=_MzR6Kl5VPI}{suggested publicly}
that the disease would spark a stampede across the border, incensing
White House staff members who thought he was stoking panic. ``If it
breaks out, it's literally, `Katie, bar the door,' '' Kelly said. ``And
there will be mass migration into the United States.''

\textbf{As he assumed} control at Homeland Security, Kelly implied that
his views on immigration were more nuanced than Trump's, infused with
compassion for a region whose leaders he had come to know personally. In
preconfirmation testimony, Kelly accused the United States of ``ignoring
what our drug demand does to the people of Central and South America,''
whose countries had devolved at times into ``nearly failed
narco-states.'' He positioned himself as a moderate voice in sessions
with lawmakers, casting doubt on the wisdom of a massive border wall.
Speaking to a group of Democrats last year, he suggested that
undocumented immigrants without a serious criminal footprint would not
be enforcement priorities. ``He said, `I'm the best thing to happen to
DACA recipients,' '' Representative Nanette Diaz Barragán, a California
Democrat, told me.

The benefit of the doubt lasted nine days. On Jan. 29, Trump signed an
\href{https://www.nytimes3xbfgragh.onion/2017/01/29/us/politics/donald-trump-rush-immigration-order-chaos.html}{executive
order} barring entry to the United States by residents of seven
predominantly Muslim countries, inciting large
\href{https://www.nytimes3xbfgragh.onion/2017/01/30/magazine/the-chaos-at-kennedy-airport-and-the-chaos-to-come.html}{protests
at airports} across the country --- and a backlash against Democrats who
had voted to confirm Kelly, whose department was left to carry out the
order. Kelly
\href{https://www.washingtonpost.com/world/national-security/in-their-courtrooms-theyre-protected-by-people-like-me-dhs-secretary-weighs-in-on-legal-dispute-over-trump-ban/2017/02/07/5e37fc4e-ed4e-11e6-9662-6eedf1627882_story.html}{told
angry lawmakers} that responsibility for the chaos was ``all on me.'' In
reality, according to an exhaustive
\href{https://www.oig.dhs.gov/sites/default/files/assets/2018-01/OIG-18-37-Jan18.pdf}{Inspector
General's report} released early this year, Kelly and his team were
caught almost entirely off-guard. He told investigators that ``he had
assumed that White House staff had proactively engaged Congress and
other stakeholders'' before the order was signed, according to the
report. Publicly, Kelly cheered the spirit of the measure,
\href{https://www.nytimes3xbfgragh.onion/2017/06/11/us/politics/as-trump-sounds-urgent-note-on-travel-ban-a-vetting-revamp-grinds-on.html}{arguing}
that a federal court order blocking the ban was preventing the nation
from doing ``all that we can to weed out potential wrongdoers from these
locations.'' Privately, he faulted the execution. ``That's not going to
happen again,'' he told the White House.

At the same time, Kelly was making good on a signature Trump campaign
promise. Immigration officers arrested more than 140,000 people in 2017,
a sharp uptick. ``We questioned the fact that many of these arrests were
taking place when parents were dropping off their children to go to
school,'' Representative Nydia Velázquez, a Democrat from New York who
met with Kelly at the time, told me. ``He didn't back down.''

By then, most Democrats had seen enough. ``He's disappointing to me,''
Schumer decided by February of last year, suggesting in an interview
that Kelly was probably ``regretting going in.''

Kelly betrayed no second thoughts in public. He also began taking on
duties that appeared to be outside his jurisdiction. When news broke
that Trump's son-in-law, Jared Kushner, worked to establish a secret
communications channel with Russian diplomats during the transition, it
was Kelly who
\href{https://www.nytimes3xbfgragh.onion/2017/05/28/us/politics/trump-returns-to-us-and-to-berating-newsmedia-on-twitter.html}{defended
the effort} on television as ``a good thing.'' In May, addressing a
private breakfast with former diplomats and foreign-policy experts,
Kelly said he had suggested to Tillerson, the ever-beleaguered secretary
of state, that he could ``take care of Central America'' while Tillerson
confronted first-order headaches like North Korea, according to a person
in the room. (A White House spokesman disputes this account.)

It was no secret, by then, that Trump had grown disenchanted with his
first chief of staff, Reince Priebus, who presided over a team consumed
by squabbling factions, endless leaks and overbroad walk-in privileges
for presidential face time. (During one Oval Office meeting Trump had
with New York Times reporters in April, no fewer than 20 people came and
went.) In an interview with The Times last December, Kelly said that he
told Trump around this time that he did not believe the president ``was
being well served by the staff'' in some respects. A month and a half
later, Kelly recalled, Trump called and said, ``I need you to be
chief.''

\textbf{Kelly set out} first to slay the meandering, oversize meetings
he loathed. ``I see these people,'' he used to tell staff members at
Homeland Security, after returning from the White House. ``I don't even
know who they are or what they do.'' Almost immediately, he sought to
institute new rules: Meetings were to be tight, targeted and
surprise-free. Once, Vice President Mike Pence showed up for one
unexpectedly. ``You guys have the meeting,'' Kelly grumbled, walking
off, according to a White House official who witnessed the exchange.

For decades, the White House chief of staff's mandate has been a kind of
tough love --- the capacity to close the door to the Oval Office and
tell the president what he does not want to hear. ``Above all, you are
the honest broker of information,'' Chris Whipple, the author of
``\href{https://www.nytimes3xbfgragh.onion/2017/03/24/books/review/the-head-honchos-head-honcho.html}{The
Gatekeepers},'' a history of White House chiefs, told me. True to this
template, Kelly clamped down on the free flow of information to Trump,
who once rifled through
\href{https://www.nytimes3xbfgragh.onion/2017/03/04/us/politics/trump-obama-tap-phones.html}{Breitbart
articles} and conspiracy-stuffed printouts with impunity. Some executive
riffraff was expelled altogether. ``He fired me like a gentleman,'' says
Anthony Scaramucci, who lasted 11 days as communications director and
scolds anyone who suggests it was 10. Those who dared attempt an
unsanctioned chat with the president could expect a Kelly follow-up:
``You want to be chief of staff?''

The early purges, which included the exits of the advisers
\href{https://www.nytimes3xbfgragh.onion/2017/08/18/us/politics/steve-bannon-trump-white-house.html}{Stephen
K. Bannon} and
\href{https://www.nytimes3xbfgragh.onion/2017/08/25/us/politics/sebastian-gorka-leaves-white-house.html}{Sebastian
Gorka}, restored a measure of good will toward Kelly among Democrats.
But again, Kelly's kinship with Trump on immigration was underestimated.
``Part of that is the Marine in him, part of that is the Irish guy in
Boston who believes that in the end, you really do have to abide by the
laws,'' Panetta, a friend of Kelly's, told me. ``I think that's what's
coming out now.''

In November, as Homeland Security was set to extend residency permits
for tens of thousands of Hondurans living in the United States, Kelly
made an 11th-hour plea to the department's acting secretary to
reconsider the move. When the administration debated lowering the annual
cap on refugees --- should it stay at 110,000? Fall to 50,000, the
minimum recommended by Defense and State Department officials? Land
somewhere in between? --- Kelly offered his take: If it were his call,
he said, the number would be between zero and one. The administration
settled on 45,000.

Even as Kelly has driven out the most flamboyant West Wing agitators ---
Scaramucci, Omarosa Manigault Newman, Bannon, Gorka --- it has not gone
unnoticed that
\href{https://www.nytimes3xbfgragh.onion/2017/10/09/us/politics/stephen-miller-trump-white-house.html}{Stephen
Miller}, the 32-year-old senior policy adviser and Trump's nativist id
on immigration policy since the campaign, has thrived on Kelly's watch.
``He turns out to have been more hawkish than I might have expected,''
Mark Krikorian, the executive director of the hard-line Center for
Immigration Studies, told me of Kelly. ``It's a pleasant surprise.''

In January, several Senate moderates believed they were close to
securing Trump's support for a compromise measure protecting the young
undocumented immigrants known as Dreamers, in exchange for
border-security funding and other policy adjustments. Trump invited some
of them to a
\href{https://www.nytimes3xbfgragh.onion/2018/01/09/us/politics/trump-immigration-meeting.html}{televised
summit} at the White House, where he told them he would approve any
legislation they brought him. ``I'm not going to say, `Oh, gee, I want
this, or I want that,' '' he said. ``I'll be signing it.'' He even
assured Senator Dianne Feinstein, Democrat of California, that he
welcomed a stand-alone bill protecting the Dreamers from deportation ---
anathema to the negotiating position of Republicans in Congress, some of
whom rushed to dissuade him at the table.

Kelly sat silently through the televised session. But after about an
hour, when the cameras were dismissed and the meeting began in earnest,
he unburdened himself in what four attendees described as a caustic
scolding. ``You've been fiddling around for years on immigration,'' he
told lawmakers. The time had come, he said before leaving in a huff, to
``do your job.'' Attendees were handed a document labeled ``MUST
HAVE'S,'' outlining the administration's demands: billions in
border-wall funding, an end to ``extended chain migration'' and a move
toward a ``merit-based system'' for legal immigration. These were
requirements long pushed by Kelly and Miller and likely to sink any deal
with Democrats.

``What's this?'' Trump asked, according to three people present,
eyeballing the list of what were ostensibly his own policy directives.
He suggested the papers were unhelpful and could be disregarded.
Lawmakers left the meeting unsure if the president knew his own
administration's position, or if he was even responsible for it.

Days later, a bipartisan group of senators led by Dick Durbin of
Illinois and Lindsey Graham of South Carolina thought again that they
could sell Trump on their plans, only to be
\href{https://www.nytimes3xbfgragh.onion/2018/01/19/us/politics/trump-durbin-immigration-daca.html}{thwarted
anew}. This time, they did not conceal their frustrations with Kelly in
particular. ``I don't think he was well served by his staff,'' Graham
\href{http://transcripts.cnn.com/TRANSCRIPTS/1801/16/sitroom.02.html}{said}
of the president to reporters at the Capitol. And Kelly, he said, was
``part of the staff.''

In another meeting with Hispanic congressional Democrats later in
January, Kelly made the case once more for a ``merit-based system'' for
legal immigration. Members reminded him what he was asking of them.
``He's saying this to 25 members of the Congressional Hispanic Caucus!''
Representative Luis Gutiérrez, an Illinois Democrat, told me. ``My mom
came with a fifth-grade education. Someone stood up and said to him, `So
you don't think we should even be here?' We're the children of those
parents. And we're members of Congress.''

Kelly's active role in immigration policy, Whipple told me, was highly
unusual for a chief of staff, setting Kelly apart from even otherwise
partisan warriors like Dick Cheney, who served as chief to Gerald Ford,
and Rahm Emanuel, the first to hold the position under Obama. ``This is
abnormal,'' Whipple said. ``He's been more partisan than almost any
chief of staff I can think of.''

\textbf{It took a} mass shooting and another round of Russia intrigue to
elbow speculation about Kelly's job status, post-Porter scandal,
temporarily out of the news. ``Trying to keep below the radar
particularly after the Porter issue and my involvement was so
inaccurately covered,'' Kelly told me in an email, declining an
interview. But his standing has not necessarily rebounded. This fate
seems to flow, in part, from the heightened initial expectations of him.
But it also speaks to his shortage of allies in Trump's inner circle.
Rivals, sensing a power vacuum, are wasting no time, spawning a
succession of leaks and counterleaks that evoke the Priebus-era West
Wing.

Today Kelly holds the support of two incongruous constituencies: those
who cheer him on immigration and those who assume, despite the strikes
against him, that he is preventing catastrophes no one can see. But this
base is shrinking. ``One Donald is bad enough,'' Reines, one of the last
Democratic holdouts,
\href{https://twitter.com/PhilippeReines/status/961585598871064576}{tweeted}
as Kelly cycled through conflicting explanations of the Porter timeline
in February. ``We don't need two.''

Older friends have greeted recent events with a deeper despair. Some war
buddies, eager to publicly support Kelly when he took the job, have
begged off entirely. ``I'd prefer not to talk anymore about him,'' Mark
Hertling, a retired three-star general who served with Kelly in Iraq,
told me, ``given what I've seen lately.''

For John Allen, a retired four-star general who has known Kelly since
the late 1970s and endorsed Clinton in 2016, the first distress signal
seemed to come in October. Trump had
\href{https://www.nytimes3xbfgragh.onion/2017/10/18/us/politics/trump-widow-johnson-call.html}{upset}
the widow of La David T. Johnson, an Army sergeant who was killed in an
ambush in Niger, by telling her that her husband ``knew what he signed
up for.'' Defending himself afterward, Trump falsely accused Obama of
not contacting the families of fallen troops at all, adding (truthfully)
that Obama
\href{https://www.nytimes3xbfgragh.onion/2017/10/17/us/politics/john-kelly-trump-fallen-troops.html}{did
not call Kelly} when his son First Lt. Robert Kelly was killed in
Afghanistan in 2010.

Kelly, who had long avoided discussing his loss in detail,
\href{https://www.nytimes3xbfgragh.onion/2017/10/19/us/politics/john-kelly-son-trump.html}{confirmed
as much} from the lectern of the White House briefing room. ``It must
have been enormously painful for him,'' Allen told me. ``John is very
private in his grief.'' But Kelly went on to accuse Frederica Wilson, a
Florida congresswoman who knew the widow and listened to Trump's phone
call with her, of making self-aggrandizing remarks years earlier. A
video from the time quickly
\href{https://www.nytimes3xbfgragh.onion/2017/10/20/us/politics/trump-kelly-congresswoman-wilson-niger.html}{proved
him wrong}, but Kelly
\href{https://www.cnn.com/2017/10/30/politics/john-kelly-frederica-wilson-apologize/index.html}{never
apologized}. To friends who had winced often at Trump's conduct, but
never Kelly's, it was agonizing to watch.

In the months since, Trump and Kelly have found new reasons to grow sick
of each other. Even before the Porter maelstrom began dominating Trump's
cable-news diet, the president had been smarting for weeks over Kelly's
suggestion to Fox News that the president had
``\href{http://video.foxnews.com/v/5714423988001/\#sp=show-clips}{evolved}''
on his wall demands. Both episodes stirred latent frustrations with
Kelly's imperious style, which had grated on Trump. ``He's a free
spirit,'' Roger Stone, Trump's longtime informal adviser, said of the
president. ``Nobody handles Donald Trump. Nobody manages him. He resents
those who try.'' Trump has been
\href{https://www.nytimes3xbfgragh.onion/2018/02/14/us/politics/john-kelly-rob-porter-security-clearances.html}{floating
Gary Cohn}, his economic adviser, as a possible replacement.

But on immigration, Kelly's legacy, such as it is, may already be
secure. His tumble has coincided with what was supposed to be a period
of congressional progress on the issue --- testing the priorities of a
president who would still like a wall but thrills at the prospect of any
signing ceremony. After appearing inclined at times toward an agreement
with Democrats before Kelly helped reel him back, Trump has by now
wholly convinced them that he is not to be trusted to cut a deal. Once
held up as the administration's most credible cross-aisle emissary,
Kelly has instead become the figure --- even more than Miller, from whom
Democrats expected nothing less --- most closely associated with White
House intransigence. ``He's not a martyr, and he's not a hostage,''
Juliette Kayyem, a former assistant secretary at Homeland Security under
Obama who has worked with Kelly and who initially cheered his addition
to the administration, told me. ``John Kelly is not saving us.''

There is a favorite book of Kelly's that he has said he rereads at key
moments of his career. It is ``The General,'' a 1936 novel by C.S.
Forester, set among British forces around World War I. Principally, it
is about a commander unable to meet the moment. ``He is a brave guy, a
dedicated guy, a noble guy,'' Kelly
\href{http://foreignpolicy.com/2017/07/31/homeland-security-secretary-john-kelly-discusses-c-s-foresters-the-general-2/}{said}
of the protagonist, Herbert Curzon, in a collection of book
recommendations by military leaders published last year, ``but a guy who
in the end has become a corps commander --- a three-star general --- and
when presented with an overwhelming German attack couldn't figure out
how to deal with it because he'd never developed himself
intellectually.''

Outmatched by his circumstances, Curzon resolves, at least, to fall on
his terms. ``He didn't know the great lessons of the great master, if
you will,'' Kelly said, ``and then he just decided one day to go down to
his horse, grab his sword, and attack --- with the intent of dying.''

Advertisement

\protect\hyperlink{after-bottom}{Continue reading the main story}

\hypertarget{site-index}{%
\subsection{Site Index}\label{site-index}}

\hypertarget{site-information-navigation}{%
\subsection{Site Information
Navigation}\label{site-information-navigation}}

\begin{itemize}
\tightlist
\item
  \href{https://help.nytimes3xbfgragh.onion/hc/en-us/articles/115014792127-Copyright-notice}{©~2020~The
  New York Times Company}
\end{itemize}

\begin{itemize}
\tightlist
\item
  \href{https://www.nytco.com/}{NYTCo}
\item
  \href{https://help.nytimes3xbfgragh.onion/hc/en-us/articles/115015385887-Contact-Us}{Contact
  Us}
\item
  \href{https://www.nytco.com/careers/}{Work with us}
\item
  \href{https://nytmediakit.com/}{Advertise}
\item
  \href{http://www.tbrandstudio.com/}{T Brand Studio}
\item
  \href{https://www.nytimes3xbfgragh.onion/privacy/cookie-policy\#how-do-i-manage-trackers}{Your
  Ad Choices}
\item
  \href{https://www.nytimes3xbfgragh.onion/privacy}{Privacy}
\item
  \href{https://help.nytimes3xbfgragh.onion/hc/en-us/articles/115014893428-Terms-of-service}{Terms
  of Service}
\item
  \href{https://help.nytimes3xbfgragh.onion/hc/en-us/articles/115014893968-Terms-of-sale}{Terms
  of Sale}
\item
  \href{https://spiderbites.nytimes3xbfgragh.onion}{Site Map}
\item
  \href{https://help.nytimes3xbfgragh.onion/hc/en-us}{Help}
\item
  \href{https://www.nytimes3xbfgragh.onion/subscription?campaignId=37WXW}{Subscriptions}
\end{itemize}
