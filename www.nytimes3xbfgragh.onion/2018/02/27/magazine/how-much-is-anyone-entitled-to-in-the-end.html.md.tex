Sections

SEARCH

\protect\hyperlink{site-content}{Skip to
content}\protect\hyperlink{site-index}{Skip to site index}

\href{https://myaccount.nytimes3xbfgragh.onion/auth/login?response_type=cookie\&client_id=vi}{}

\href{https://www.nytimes3xbfgragh.onion/section/todayspaper}{Today's
Paper}

How Much Is Anyone `Entitled' To, in the End?

\url{https://nyti.ms/2sYXwcX}

\begin{itemize}
\item
\item
\item
\item
\item
\item
\end{itemize}

Advertisement

\protect\hyperlink{after-top}{Continue reading the main story}

Supported by

\protect\hyperlink{after-sponsor}{Continue reading the main story}

\href{/column/first-words}{First Words}

\hypertarget{how-much-is-anyone-entitled-to-in-the-end}{%
\section{How Much Is Anyone `Entitled' To, in the
End?}\label{how-much-is-anyone-entitled-to-in-the-end}}

\includegraphics{https://static01.graylady3jvrrxbe.onion/images/2018/03/04/magazine/04mag-firstwords/04mag-04firstwords-t_CA0-articleLarge.jpg?quality=75\&auto=webp\&disable=upscale}

By Carina Chocano

\begin{itemize}
\item
  Feb. 27, 2018
\item
  \begin{itemize}
  \item
  \item
  \item
  \item
  \item
  \item
  \end{itemize}
\end{itemize}

When I was in my 20s, I worked at a video-game company in Silicon
Valley. We were expected to work 12-hour days and at least one weekend
day per week. The pay was terrible, as was the product. Even so, it was
part of the job to feel lucky to have it, and, being young and
replaceable, we were in no position to complain. To complain would have
made us ``entitled,'' which we were not entitled to be. The company, on
the other hand, was entitled to claim our every waking hour, and to
treat its rush to meet deadlines for clunky, slow-moving shooter games
as a collective labor of love. It felt difficult to square the
democratic world outside the office with the feudal arrangement inside
it; I spent a lot of time marveling at the cognitive dissonance of it
all.

It wasn't just that office, of course. There has always been a
disconnect between what we claim to deserve as human beings and what we
believe is deserved in specific circumstances --- by citizens or
noncitizens, workers or owners, ``makers'' or ``takers.'' Lately,
``entitled'' is the scalpel we use to divide one from the other,
affirming the things we believe are owed to us, then turning around and
shaming others for expecting anything beyond that. It lets us claim that
certain rights are fundamental, granted by a higher authority, and it
lets us accuse others of being grasping, arrogant and superior. It swaps
hats, darting frantically from post to post, defending and attacking as
the need arises.

Any survey of news and headlines will illustrate the tangled nature of
the word. For every negative usage, a positive one seems to follow; for
every allowance we're granted, another will soon be taken away.
``Millennials have an entitlement complex,'' you will read, but ``All
Americans are entitled to opinions.'' Then there's ``Millennials in the
workplace --- not entitled, just different.'' An article about the
travel arrangements of the head of the Environmental Protection Agency
fashions a quote from a senator into a headline that affirms as it
undermines: ``Pruitt `Entitled' to Fly First Class.'' Other benefits
might have been earned the hard way: After Georgina Chapman announced
that she was leaving Harvey Weinstein, numerous articles speculated that
she was legally ``entitled'' to at least \$300,000 for each year they
were married. One article declares that Melania Trump is ``entitled'' to
some privacy, while another notes that noncitizens brought to this
country as children are not ``entitled'' to court-appointed immigration
lawyers. On a recent episode of ``The View,'' the host Joy Behar
challenged Senator Kirsten Gillibrand over her calls for Al Franken to
resign his Senate seat, noting that the president had been accused of
worse sins and was not about to step down. Gillibrand agreed that she'd
like to see Trump resign too, then carefully divided the things that
were owed to Franken from the things that were not: ``He's entitled to a
hearing. He is. But he's not entitled to my silence, Joy.''

This double-sided use of entitlement tends not to tell us a great deal
about the subject it's applied to, whether that subject is privacy,
legal representation or airline upgrades. It does, however, tend to
reveal a great deal about us, the people having the conversation ---
about what we believe human beings should expect from life and about
what we owe to one another. And it hints at all the bad feeling and
zero-sum thinking that pervade our arguments over who deserves what: our
envy, our fear of missing out, our paranoia that the rights of others
are constantly threatening to impinge on our own.

\textbf{The way we} use ``entitled'' feels less strange when you
consider that its two meanings have slightly different origins. Once
upon a time, ``entitlement'' was, quite literally, something granted by
a higher authority: A ``noble'' person pleased the sovereign, perhaps by
acquiring land or fighting bravely, and in return the sovereign bestowed
upon him special rights and privileges, perks not afforded to a
commoner, including the right not to have their estate expropriated by
the crown. ``Entitlement'' was a recognition of service, a promise of
specific legal benefits. In some cases it came with an actual title,
like the feudal ``baron.''

Something similar has been true in the United States, where
``entitlements'' refers to government benefit programs not subject to
budgetary discretion --- assistance that the state was obligated, on
levels both legal and societal, to provide to all eligible citizens.
According to the linguist Geoff Nunberg, it was sometime after the
publication of ``The Culture of Narcissism,'' an influential 1979 book
by Christopher Lasch, that a certain negative connotation began to
spread. Legitimate ``entitlement'' and an illegitimate \emph{sense} of
entitlement merged. You might now be an entitled person: someone whose
privilege leads to arrogance, snobbery and rudeness, someone who expects
to be waited on, provided for, deferred to.

This was an unfortunate development for those who depended on
entitlement programs, who were increasingly cast as undeserving ---
people who ``chose'' to be poor, via laziness and lack of
responsibility, and yet felt ``entitled'' to governmental support. Even
before Lasch's book, Ronald Reagan, during his first run for president,
held up an actual career criminal --- a con woman, identity thief and
suspected kidnapper believed to have bilked the government of untold
thousands of dollars --- as a symbol of excess, conjuring around her a
world of fantastical ``welfare queens'': enemies of the bootstrapping
individualist spirit of America, coddled and demanding and overly
dependent on programs they were, in fact, legally entitled to use. The
apotheosis of this concept was laid out by Mitt Romney in 2012, when he
declared that 47 percent of voters were dependent on the government, saw
themselves as victims and believed ``that they are entitled to health
care, to food, to housing, to you name it. That that's an entitlement.''
It was not his job, he argued, to care about such people, whom he'd
``never convince'' that they should ``take personal responsibility and
care for their lives.''

Of course, it did not escape American voters that as the very wealthy
scion of a very powerful family, Romney was the epitome of an
``entitled'' man in the original, feudal sense; his entitlement was
hereditary. And he was, above all, in a position to make influential
judgments about what others had, what others wanted, what people
deserved or did not.

\textbf{The entitlements that} infuriate us most, it seems, involve the
sight of other people chasing determinedly after things we'd never dare
to pursue. Many of our most pitched intergenerational arguments revolve
around precisely this issue: the question of what it is reasonable to
expect from the world and what we must learn to bear quietly. Older
people come to believe that they spent their youth working very hard and
expecting little in return, and express awe and outrage over the extent
to which their children expect the world to be made decent or fair or
nice. At the moment, it's millennials who are accused of this kind of
entitlement, despite their shrinking access to nice things --- secure
jobs, employer-provided health care, affordable housing --- that baby
boomers took for granted. But similar complaints echo backward.
Twenty-five years ago, it was Generation X that was criticized for the
same things, in the same terms, sometimes in the same publications.
``Grow Up, Crybabies, You're America's Luckiest Generation,'' sighed a
1993 Washington Post headline. ``They're impatient waiting for job
promotions and want all the perks associated with `paying one's dues' ''
--- a bank employee said that in a 1997 article in The Atlanta
Journal-Constitution. To read a 2013 article in Time magazine calling
millennials ``lazy, entitled, selfish and shallow'' feels like having
dozed off years ago and woken up just in time to see history repeated.

We are clearly galled by people we believe think they deserve something
the rest of us don't. The unsettled issue, of course, is what anyone
deserves in the first place, a question over which we are constantly
erupting into pitched battles. Sometimes the issues are broad and
societal. (Do you deserve health care, or a home, or a lawyer? Should
the government formally entitle you to more of these things?) Sometimes
they are personal or mundane. (Do you deserve a thank-you card or to
have your text responded to? Is it ``entitled'' to expect either?) We
are able to bring a surprising emotional intensity to our fights over
any of these things.

Still: We tend to use ``entitled'' to argue out these questions case by
case and person by person, as if it were possible to settle huge,
overarching questions about the kind of society we want to live in one
situation at a time. We take what could be useful conversations about
rights and values and turn them into ad hominem attacks, accusations of
arrogance and greed. (Calling someone ``entitled'' doesn't suggest they
hold a different conception of social responsibility; it suggests poor
character.) No matter which direction the word points, it tends to serve
as a cudgel. And yet its ubiquity may still be a useful sign, pointing
at deep questions about fairness and equality, about our very different
conceptions about what a person can, or should, expect from life. It
embodies an impasse we've reached --- and it suggests that if we ever
could sort out the details, we might find that we're entitled to far
more than we're getting.

Advertisement

\protect\hyperlink{after-bottom}{Continue reading the main story}

\hypertarget{site-index}{%
\subsection{Site Index}\label{site-index}}

\hypertarget{site-information-navigation}{%
\subsection{Site Information
Navigation}\label{site-information-navigation}}

\begin{itemize}
\tightlist
\item
  \href{https://help.nytimes3xbfgragh.onion/hc/en-us/articles/115014792127-Copyright-notice}{©~2020~The
  New York Times Company}
\end{itemize}

\begin{itemize}
\tightlist
\item
  \href{https://www.nytco.com/}{NYTCo}
\item
  \href{https://help.nytimes3xbfgragh.onion/hc/en-us/articles/115015385887-Contact-Us}{Contact
  Us}
\item
  \href{https://www.nytco.com/careers/}{Work with us}
\item
  \href{https://nytmediakit.com/}{Advertise}
\item
  \href{http://www.tbrandstudio.com/}{T Brand Studio}
\item
  \href{https://www.nytimes3xbfgragh.onion/privacy/cookie-policy\#how-do-i-manage-trackers}{Your
  Ad Choices}
\item
  \href{https://www.nytimes3xbfgragh.onion/privacy}{Privacy}
\item
  \href{https://help.nytimes3xbfgragh.onion/hc/en-us/articles/115014893428-Terms-of-service}{Terms
  of Service}
\item
  \href{https://help.nytimes3xbfgragh.onion/hc/en-us/articles/115014893968-Terms-of-sale}{Terms
  of Sale}
\item
  \href{https://spiderbites.nytimes3xbfgragh.onion}{Site Map}
\item
  \href{https://help.nytimes3xbfgragh.onion/hc/en-us}{Help}
\item
  \href{https://www.nytimes3xbfgragh.onion/subscription?campaignId=37WXW}{Subscriptions}
\end{itemize}
