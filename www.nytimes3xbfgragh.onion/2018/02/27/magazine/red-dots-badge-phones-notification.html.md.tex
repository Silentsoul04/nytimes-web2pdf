Sections

SEARCH

\protect\hyperlink{site-content}{Skip to
content}\protect\hyperlink{site-index}{Skip to site index}

\href{https://myaccount.nytimes3xbfgragh.onion/auth/login?response_type=cookie\&client_id=vi}{}

\href{https://www.nytimes3xbfgragh.onion/section/todayspaper}{Today's
Paper}

How Tiny Red Dots Took Over Your Life

\url{https://nyti.ms/2sUt2ss}

\begin{itemize}
\item
\item
\item
\item
\item
\end{itemize}

Advertisement

\protect\hyperlink{after-top}{Continue reading the main story}

Supported by

\protect\hyperlink{after-sponsor}{Continue reading the main story}

\href{/column/on-technology}{On Technology}

\hypertarget{how-tiny-red-dots-took-over-your-life}{%
\section{How Tiny Red Dots Took Over Your
Life}\label{how-tiny-red-dots-took-over-your-life}}

\includegraphics{https://static01.graylady3jvrrxbe.onion/images/2018/03/04/magazine/04mag-ontech2/04mag-04ontech-t_CA0-articleLarge.jpg?quality=75\&auto=webp\&disable=upscale}

By \href{https://www.nytimes3xbfgragh.onion/by/john-herrman}{John
Herrman}

\begin{itemize}
\item
  Feb. 27, 2018
\item
  \begin{itemize}
  \item
  \item
  \item
  \item
  \item
  \end{itemize}
\end{itemize}

As the ranks of tech-industry critics have expanded, it has become
harder to tell what common ground they occupy. Across various political
divides, there is a sense that Facebook, Twitter and Google exert too
much influence on the national discourse; closely connected to this is
the widespread concern that we users have developed an unhealthful
relationship with our phones, or with the apps on them. But on any more
specific claim than that, consensus becomes impossible. The sudden
arrival of a new class of tech skeptic, the industry apostate, has only
complicated the discussion. Late last year, the co-inventor of the
Facebook ``like,'' Justin Rosenstein, called it a ``bright ding of
pseudopleasure''; in January, the investment firm Jana Partners, a
shareholder in Apple, wrote a letter to the company warning that its
products ``may be having unintentional negative consequences.''

All but conjuring Oppenheimer at White Sands, these critics offer
broadsides, warning about addictive design tricks and profit-driven
systems eroding our humanity. But it's hard to discern a collective
message in their garment-rending: Is it design that needs fixing? Tech?
Capitalism? This lack of clarity may stem from the fact that these
people are not ideologues but reformists. They tend to believe that
companies should be more responsible --- and users must be, too. But
with rare exceptions, the reformists stop short of asking the
uncomfortable questions: Is it possible to reform profit-driven systems
that turn attention into money? In such a business, can you even
separate addiction from success?

Perhaps this is unfair --- the reformists are trying. But while we wait
for a consensus, or a plan, allow me to suggest a starting point: the
dots. Little. Often red. Sometimes with numbers. Commonly seen at the
corners of app icons, where they are known in the trade as badges, they
are now proliferating across once-peaceful interfaces on a steep
epidemic curve. They alert us to things that need to be checked: unread
messages; new activities; pending software updates; announcements;
unresolved problems. As they've spread, they've become a rare
commonality in the products that we --- and the remorseful technologists
--- are so worried about. If not the culprits, the dots are at least
accessories to most of the supposed crimes of addictive app design.

When platforms or services sense their users are disengaged, whether
from social activities, work or merely a continued contribution to
corporate profitability, dots are deployed: outside, inside, wherever
they might be seen. I've met dots that existed only to inform me of the
existence of other dots, new dots, dots with almost no meaning at all; a
dot on my Instagram app led me to another dot within it, which informed
me that something had happened on Facebook: Someone I barely know had
posted for the first time in a while. These dots are omnipresent,
leading everywhere and ending nowhere. So maybe there's something to be
gained by connecting them.

\textbf{The prototypical modern} dot --- stop-sign red, with numbers,
round, maddening --- was popularized with Mac OS X, the first version of
which was released nearly 20 years ago. It was used most visibly as part
of Apple's Mail app, perched atop an icon of a blue postage stamp, in a
new and now ever-present dock full of apps. It contained a number
representing your unread messages. But it wasn't until the launch of the
iPhone, in 2007, that dots transformed from a simple utility into a way
of life --- from a solution into a cause unto themselves.

That year, we got the very first glimpse of the iPhone's home screen, in
Steve Jobs's hand, onstage at MacWorld. It showed three dots, ringed in
white: 1 unread text; 5 calls or voice mail messages; 1 email. Jobs set
about showing off the apps, opening them, eliminating the dots.
Eventually, when the iPhone was opened to outside developers, badge use
accelerated. As touch-screen phones careered toward ubiquity, and as
desktop interfaces and website design and mobile operating systems
huddled together around a crude and adapting set of visual metaphors,
the badge was ascendant.

On Windows desktop computers, they tended to be blue and lived in the
lower right corner. On BlackBerrys, red, with a white asterisk in the
middle. On social media, in apps and on websites, badge design was more
creative, appearing as little speech bubbles or as rectangles. They make
appearances on Facebook and across Google products, perhaps most
notoriously on the ill-fated Google Plus social network, where blocky
badges were filled with inexplicably, desperately high numbers. (This
tactic has since spread, obnoxiously, to news sites and, inexplicably,
to comment sections.) Android itself has remained officially unbadged,
but the next version of the operating system, called Oreo, will include
them by default, completing their invasion.

\includegraphics{https://static01.graylady3jvrrxbe.onion/images/2018/03/04/magazine/04mag-ontech1/04mag-04ontech-t_CA1-articleLarge.jpg?quality=75\&auto=webp\&disable=upscale}

What's so powerful about the dots is that until we investigate them,
they could signify anything: a career-altering email; a reminder that
Winter Sales End Soon; a match, a date, a ``we need to talk.'' The same
badge might lead to word that Grandma's in the hospital or that,
according to a prerecorded voice, the home-security system you don't own
is in urgent need of attention or that, for the 51st time today, someone
has posted in the group chat.

New and flourishing modes of socialization amount, in the most abstract
terms, to the creation and reduction of dots, and the experience of
their attendant joys and anxieties. Dots are deceptively, insidiously
simple: They are either there or they're not; they contain a number, and
that number has a value. But they imbue whatever they touch with a
spirit of urgency, reminding us that behind each otherwise static icon
is unfinished business. They don't so much inform us or guide us as
correct us: You're looking there, but you should be looking here.
They're a lawn that must be mowed. Boils that must be lanced, or at
least scabs that itch to be picked. They're Bubble Wrap laid over your
entire digital existence.

To their credit, the big tech companies seem to be aware of the problem,
at least in the narrow terms of user experience. In Google's guide for
application developers, the company makes a gentle attempt to pre-empt
future senseless dot deployment. ``Don't badge every notification, as
there are cases where badges don't make sense,'' the company suggests.
Apple, in its guidelines, seems a bit more fed up. ``Minimize badging,''
it says. ``Don't overwhelm users by connecting badging with a huge
amount of information that changes frequently. Use it to present brief,
essential information and atypical content changes that are highly
likely to be of interest.''

These companies know better than anyone that dots are a problem, but
they also know that dots work. Late last year, a red badge burbled to
the surface next to millions of iPhone users' Settings apps. It looked
as though it might be an update, but it turned out to be a demand:
Finish adding your credit card to Apple Pay, or the dot stays put. Apple
might as well have said: Give us your credit card number, or we will
annoy you until you do.

The lack of consensus within the mounting resistance to Big Tech can
also be found within the perimeter of the dot. After all, it's where the
most dangerous conflations take place: of what we need, and what we're
told we need; of what purpose our software serves to us, and us to it;
of dismissal with fulfillment. The dot is where ill-gotten attention is
laundered into legitimate-seeming engagement. On this, our most
influential tech companies seem to agree. Maybe our self-appointed
saviors can, too.

Advertisement

\protect\hyperlink{after-bottom}{Continue reading the main story}

\hypertarget{site-index}{%
\subsection{Site Index}\label{site-index}}

\hypertarget{site-information-navigation}{%
\subsection{Site Information
Navigation}\label{site-information-navigation}}

\begin{itemize}
\tightlist
\item
  \href{https://help.nytimes3xbfgragh.onion/hc/en-us/articles/115014792127-Copyright-notice}{©~2020~The
  New York Times Company}
\end{itemize}

\begin{itemize}
\tightlist
\item
  \href{https://www.nytco.com/}{NYTCo}
\item
  \href{https://help.nytimes3xbfgragh.onion/hc/en-us/articles/115015385887-Contact-Us}{Contact
  Us}
\item
  \href{https://www.nytco.com/careers/}{Work with us}
\item
  \href{https://nytmediakit.com/}{Advertise}
\item
  \href{http://www.tbrandstudio.com/}{T Brand Studio}
\item
  \href{https://www.nytimes3xbfgragh.onion/privacy/cookie-policy\#how-do-i-manage-trackers}{Your
  Ad Choices}
\item
  \href{https://www.nytimes3xbfgragh.onion/privacy}{Privacy}
\item
  \href{https://help.nytimes3xbfgragh.onion/hc/en-us/articles/115014893428-Terms-of-service}{Terms
  of Service}
\item
  \href{https://help.nytimes3xbfgragh.onion/hc/en-us/articles/115014893968-Terms-of-sale}{Terms
  of Sale}
\item
  \href{https://spiderbites.nytimes3xbfgragh.onion}{Site Map}
\item
  \href{https://help.nytimes3xbfgragh.onion/hc/en-us}{Help}
\item
  \href{https://www.nytimes3xbfgragh.onion/subscription?campaignId=37WXW}{Subscriptions}
\end{itemize}
