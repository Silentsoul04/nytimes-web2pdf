Sections

SEARCH

\protect\hyperlink{site-content}{Skip to
content}\protect\hyperlink{site-index}{Skip to site index}

\href{https://myaccount.nytimes3xbfgragh.onion/auth/login?response_type=cookie\&client_id=vi}{}

\href{https://www.nytimes3xbfgragh.onion/section/todayspaper}{Today's
Paper}

A Painful Bruise Wouldn't Heal. It Took Several Hospital Visits to
Discover Why.

\url{https://nyti.ms/2F22wix}

\begin{itemize}
\item
\item
\item
\item
\item
\item
\end{itemize}

Advertisement

\protect\hyperlink{after-top}{Continue reading the main story}

Supported by

\protect\hyperlink{after-sponsor}{Continue reading the main story}

\href{/column/diagnosis}{Diagnosis}

\hypertarget{a-painful-bruise-wouldnt-heal-it-took-several-hospital-visits-to-discover-why}{%
\section{A Painful Bruise Wouldn't Heal. It Took Several Hospital Visits
to Discover
Why.}\label{a-painful-bruise-wouldnt-heal-it-took-several-hospital-visits-to-discover-why}}

\includegraphics{https://static01.graylady3jvrrxbe.onion/images/2018/03/04/magazine/04mag-diagnosis1/04mag-04diagnosis-t_CA1-articleLarge.png?quality=75\&auto=webp\&disable=upscale}

By \href{https://www.nytimes3xbfgragh.onion/by/lisa-sanders-md}{Lisa
Sanders, M.D.}

\begin{itemize}
\item
  Feb. 28, 2018
\item
  \begin{itemize}
  \item
  \item
  \item
  \item
  \item
  \item
  \end{itemize}
\end{itemize}

The woman lay on the floor, too weak even to lift the phone to her ear.
She could hear her sister calling her name through the phone's tinny
speaker, but she couldn't reply. A rush of relief flooded over her when
she heard her sister say to someone, ``Call 911.'' And then there was
darkness.

She had been sick for months at that point. She had seen many doctors.
She had been given a variety of diagnoses, but no one could tell her ---
a usually vigorous woman of 39 --- exactly what was wrong.

\hypertarget{a-bruise-that-wont-heal}{%
\subsection{\texorpdfstring{\textbf{A Bruise That Won't
Heal}}{A Bruise That Won't Heal}}\label{a-bruise-that-wont-heal}}

It all seemed to start the previous autumn, when she dropped a can of
paint on her foot. It gave her a big bruise. No surprise, but strangely,
the bruise never went away. Instead, over the next several weeks, the
purple discoloration and swelling snaked up her calf into her thigh and
then over to her other leg. Now both limbs were painful and splashed
with dark bruises.

Dr. Vivek Naranbhai, a doctor in his first year of training at
Massachusetts General Hospital, was assigned to care for the patient
once the ambulance brought her in. He opened her chart and saw that
she'd been in the hospital twice recently. A couple of weeks before, she
was in one closer to her home in the Boston suburbs. And just a few days
earlier, she was seen at and discharged from this same hospital, Mass
General. Each time, she had been worried about the huge bruises on her
legs and the pain and numbness that traveled from foot to thigh when she
walked.

During her visit to Mass General a few days earlier, doctors wanted her
to see their physical therapists to help make walking easier until her
bruising healed. But then something strange happened. As she waited in
the E.R., her red-blood-cell count dropped, leaving her weak and pale;
she was admitted for a transfusion and further evaluation. A CT scan of
her swollen right leg revealed the reason for the drop: In her thigh,
she now had a huge pool of blood that had leaked out of her vessels and
into the muscle of her upper leg. She was bleeding internally. And no
one knew why.

\hypertarget{a-diagnostic-consensus}{%
\subsection{\texorpdfstring{\textbf{A Diagnostic
Consensus}}{A Diagnostic Consensus}}\label{a-diagnostic-consensus}}

Like the patient herself, her doctors assumed all her symptoms were
connected to the can she'd dropped on her foot. Why hadn't that injury
healed? And why was the entire right leg --- and some of the left --- so
painful? None of the tests conducted in her nearly two weeks in the
hospital explained it. Eventually her first Mass General doctors
concluded that she had an unusual disorder called complex regional pain
syndrome (C.R.P.S.). This disorder, which usually affects a limb after
some trauma, is thought to be caused by injury to the nervous system.
That damage in turn causes pain, swelling and changes in skin color and
temperature. No one knows why the body has this extreme overreaction.
Treatment has to focus on reducing pain rather than on treating the
disorder. Recovery takes months, even years. Having made this diagnosis,
the doctors sent the patient home to follow up with specialists to treat
the pain.

In the days after this second hospitalization, the patient worsened. She
felt exhausted and cold all the time. One morning upon waking, she felt
so weak and tired that she couldn't stand. She scooched herself along
the floor toward the bathroom. Halfway there, she was so incapacitated
that she lay down and called her sister. I think I'm dying, she told
her. Her hand dropped weakly to the floor. Her sister asked for 911, and
the E.M.T.s came and took her to the emergency room for the third time.

\hypertarget{loss-of-blood}{%
\subsection{\texorpdfstring{\textbf{Loss of
Blood}}{Loss of Blood}}\label{loss-of-blood}}

A blood test showed that she was bleeding internally again. She had less
than half the blood she should have in her circulatory system. No wonder
she was cold and tired and out of breath --- these are classic signs of
severe anemia. She was given more blood and then transported back again
to Mass General.

As Naranbhai read through the records, he compiled a list of diseases
that could bring this woman to the hospital three times over a month
with severe pain and blood loss. It was a scary collection. At the top
were cancers that could keep her from making blood. Next: diseases that
interfered with her body's ability to form clots and stop bleeding. All
were terrible possibilities.

\includegraphics{https://static01.graylady3jvrrxbe.onion/images/2018/03/04/magazine/04mag-diagnosis2/04mag-04diagnosis-t_CA0-articleLarge.png?quality=75\&auto=webp\&disable=upscale}

Dr. Leigh Simmons, the internist supervising Naranbhai, usually waited
to see patients until after the resident developed his own thoughts
about the case. But it was late, and this patient sounded particularly
sick, so Simmons and Naranbhai went to visit the patient together.

\hypertarget{seeing-the-patient-anew}{%
\subsection{\texorpdfstring{\textbf{Seeing the Patient
Anew}}{Seeing the Patient Anew}}\label{seeing-the-patient-anew}}

She was a small woman, quite thin and pale beneath a dark Mediterranean
complexion. Her 15-year-old daughter stood holding her hand. Simmons
introduced herself and then stepped back to let Naranbhai lead the
investigation. Rather than focus his questions on her foot and leg, he
cast a much wider net. Tell me everything that's going on, he asked.
She'd had her period for nearly a month now, the woman said. That had
never happened before. And, her daughter added, she had these weird dots
on her legs. They just popped up a few weeks before. Naranbhai looked
carefully at the mother's legs. They were covered with tiny freckle-size
dots of blood trapped under the skin at the hair follicles.

Naranbhai looked at Dr. Simmons. Was she thinking what he was thinking?
He looked back at the patient. Did her gums ever bleed when she brushed
her teeth, he asked. All the time, she exclaimed.

Can I see? the young doctor asked. Her gums were swollen and beefy red.
He felt as if they might start bleeding just by looking at them. He
looked up to Simmons, who smiled back encouragingly. He knew what she
had. And so did Simmons.

\hypertarget{clues-in-the-diet}{%
\subsection{\texorpdfstring{\textbf{Clues in the
Diet}}{Clues in the Diet}}\label{clues-in-the-diet}}

What kind of foods do you eat? he asked. Every morning she had two
scrambled eggs. For lunch she had tuna on crackers. And for dinner she
had more scrambled eggs and rice. Did she ever eat any fruits or
vegetables --- especially oranges or lemons? Never, she told him. They
gave her wicked heartburn.

She had something known as gastroparesis, she explained. Her stomach and
intestines didn't move food forward normally, and so food stayed in her
stomach for hours. When food moves that slowly, you have to be careful
that what you eat agrees with you.

No citrus for years. It was clear to Naranbhai that this modern woman
had an ancient disease. She had scurvy --- a disorder caused by a severe
deficiency of vitamin C.

In the mid-18th century, a naval surgeon named James Lind proved that
the juice of oranges and lemons would cure the bony aches, strange
bleeding and sudden death of sailors afflicted with the illness, and the
British Navy later mandated the use of lemon juice on all vessels. But
it wasn't until the 20th century that researchers recognized that the
cause of scurvy was the lack of a certain nutrient, which they named
vitamin C. Without this organic chemical, new connective tissue,
essential for the repair or replacement of damaged or dying cells,
cannot be made, and that causes the bleeding, the bruising, the telltale
little red dots and the terrible fatigue. Our bodies can't make vitamin
C, and so we rely on the foods we eat to provide it. Avoiding these
foods --- as this woman did --- can deplete the body's supply in just a
few months.

\hypertarget{the-miracle-of-vitamin-c}{%
\subsection{\texorpdfstring{\textbf{The Miracle of Vitamin
C}}{The Miracle of Vitamin C}}\label{the-miracle-of-vitamin-c}}

The doctors sent off a blood test to measure her vitamin C, and they
started the woman on large doses of the required vitamin. The
improvement was almost immediate: her gums stopped bleeding within days;
her bruises started to turn yellow and fade. And she started to feel
stronger and less fatigued.

That was two months ago. Now she feels great. She still can't eat
oranges. But she takes her vitamin C tablets every single day.

Advertisement

\protect\hyperlink{after-bottom}{Continue reading the main story}

\hypertarget{site-index}{%
\subsection{Site Index}\label{site-index}}

\hypertarget{site-information-navigation}{%
\subsection{Site Information
Navigation}\label{site-information-navigation}}

\begin{itemize}
\tightlist
\item
  \href{https://help.nytimes3xbfgragh.onion/hc/en-us/articles/115014792127-Copyright-notice}{©~2020~The
  New York Times Company}
\end{itemize}

\begin{itemize}
\tightlist
\item
  \href{https://www.nytco.com/}{NYTCo}
\item
  \href{https://help.nytimes3xbfgragh.onion/hc/en-us/articles/115015385887-Contact-Us}{Contact
  Us}
\item
  \href{https://www.nytco.com/careers/}{Work with us}
\item
  \href{https://nytmediakit.com/}{Advertise}
\item
  \href{http://www.tbrandstudio.com/}{T Brand Studio}
\item
  \href{https://www.nytimes3xbfgragh.onion/privacy/cookie-policy\#how-do-i-manage-trackers}{Your
  Ad Choices}
\item
  \href{https://www.nytimes3xbfgragh.onion/privacy}{Privacy}
\item
  \href{https://help.nytimes3xbfgragh.onion/hc/en-us/articles/115014893428-Terms-of-service}{Terms
  of Service}
\item
  \href{https://help.nytimes3xbfgragh.onion/hc/en-us/articles/115014893968-Terms-of-sale}{Terms
  of Sale}
\item
  \href{https://spiderbites.nytimes3xbfgragh.onion}{Site Map}
\item
  \href{https://help.nytimes3xbfgragh.onion/hc/en-us}{Help}
\item
  \href{https://www.nytimes3xbfgragh.onion/subscription?campaignId=37WXW}{Subscriptions}
\end{itemize}
