Can Venezuela Be Saved?

\url{https://nyti.ms/2t6ZhoA}

\begin{itemize}
\item
\item
\item
\item
\item
\item
\end{itemize}

\includegraphics{https://static01.graylady3jvrrxbe.onion/images/2018/03/04/magazine/04mag-venezuela6-cov/04mag-venezuela6-cov-articleLarge.jpg?quality=75\&auto=webp\&disable=upscale}

Sections

\protect\hyperlink{site-content}{Skip to
content}\protect\hyperlink{site-index}{Skip to site index}

Feature

\hypertarget{can-venezuela-be-saved}{%
\section{Can Venezuela Be Saved?}\label{can-venezuela-be-saved}}

As a nation unwinds, Leopoldo López, the opposition's most prominent
leader, sits under house arrest and contemplates what might still be
possible.

Leopoldo LópezCredit...Diana López for The New York Times

Supported by

\protect\hyperlink{after-sponsor}{Continue reading the main story}

By Wil S. Hylton

\begin{itemize}
\item
  March 1, 2018
\item
  \begin{itemize}
  \item
  \item
  \item
  \item
  \item
  \item
  \end{itemize}
\end{itemize}

\href{https://www.nytimes3xbfgragh.onion/es/2018/03/01/leopoldo-lopez-venezuela-oposicion/}{Leer
en español}

There's a page in a book in a stack on the floor at the house of
Leopoldo López that I think about sometimes. It's a page that López
revisits often; one to which he has returned so many times these past
few years, scribbling new ideas in the margin and underlining words and
phrases in three different colors of ink and pencil, that studying it
today can give you the impression of counting the tree rings in his
political life.

The book is not set out in a way to invite this kind of attention.
Almost nobody is allowed to enter the López house, for one thing, being
surrounded all day and night by the Venezuelan secret police; but also,
for all his flaws and shortcomings, López just isn't the sort to dress
up his library for show. Pretty much every book in the house is piled up
in a stack like this one --- row upon row of stacked-up books rising six
to eight feet from the dark wood floors, these gangly towers of
dog-eared tomes, some of them teetering so precariously that when you
see one of the López children run past, you might involuntarily flinch.

The particular book I have in mind is a collection of political essays
and speeches. It was compiled by the Mexican politician Liébano Sáenz,
with entries on the Mayan prince Nakuk Pech and the French activist
Olympe de Gouges. The chapter that means the most to López begins on
Page 211, under the header ``Carta a los Clérigos de Alabama.'' This is
a mixed-up version of the title you know as ``Letter From Birmingham
Jail,'' which was written by the Rev. Dr. Martin Luther King Jr. in
1963. King was in Birmingham to lead nonviolent protests of the sort
that everybody praises now, but it's helpful to remember that in 1963,
he was catching hell from every quarter. It wasn't just the slithering
goons of the F.B.I. wiretapping his home and office or the ascendant
black-nationalist movement rolling its eyes at his peaceful piety, but a
caucus of his own would-be allies who were happy to talk about civil
rights just as long as it didn't cause any ruckus. A handful of
clergymen in Birmingham had recently issued a statement disparaging King
as an outside agitator whose marches and civil disobedience were
``technically peaceful'' but still broke the law and were likely ``to
incite to hatred and violence.''

\includegraphics{https://static01.graylady3jvrrxbe.onion/images/2018/03/04/magazine/04mag-venezuela2/04mag-04venezuela-t_CA3-articleLarge.jpg?quality=75\&auto=webp\&disable=upscale}

On the page in the book at the López house, King fires back. Writing
from a cramped cell without a mattress or electric light, he scrawled a
response on scraps of paper for his cellmate to smuggle out. Near the
top of the page, López has underlined a passage in pencil where King
condemns the complacency of ``the white moderate'' and the suggestion
that peaceful protesters are responsible for the violent response of
others. ``We who engage in nonviolent direct action are not the creators
of tension,'' he writes in a passage López marked with green: ``We
merely bring to the surface the hidden tension that is already alive.''
King then compares civil disobedience to the lancing of a boil, before
culminating in a passage that López has flagged at least half a dozen
times --- with some words underlined in red, others highlighted pink, a
handful of phrases boxed in green and three large arrows drawn into the
margin beside the words: ``Injustice must be exposed, with all the
tension its exposure creates.''

At a certain level, it's unremarkable when a politician studies King,
and among the people López tries to emulate, I wouldn't put King at the
top. He is more directly influenced by the former Venezuelan president
Rómulo Betancourt or, for that matter, by his own grandfather Eduardo
Mendoza, who was Betancourt's adviser. But when you consider the path
that López has followed these past few years in prison, the choices he's
made, the compromises and blunders, the price he has paid to speak his
mind and the price now of his silence, if you want to understand the
impact of four years in captivity and nine months in solitary
confinement, the message from King in Birmingham is a very good place to
start.

López was arrested in February 2014 after leading a public protest that
turned violent. Prosecutors acknowledged in court that López was
technically peaceful, but they accused him of inciting others to hatred
and violence. Before his arrest, he was among the most prominent and
popular opposition leaders in Venezuela. Polling suggested that he could
defeat President Nicolás Maduro, the unpopular successor to Hugo Chávez,
in a free election. At trial,
\href{https://www.nytimes3xbfgragh.onion/2015/09/11/world/americas/venezuelan-opposition-leader-is-sentenced-to-prison-over-a-protest.html}{he
was sentenced} to 13 years and nine months in prison. Since then, he has
become the most prominent political prisoner in Latin America, if not
the world. His case has been championed by just about every human rights
organization on earth, and he is represented by the attorney Jared
Genser, who is known as ``the extractor'' for his work with political
prisoners like Liu Xiaobo, Mohamed Nasheed and Aung San Suu Kyi. The
list of world leaders who have called on the Venezuelan government to
release López includes Angela Merkel of Germany, Emmanuel Macron of
France, Theresa May of Britain and Justin Trudeau of Canada; it is that
rarest of political causes on which Barack Obama and Donald Trump are in
agreement. In Venezuela, López has become a kind of symbol. His name and
face are emblazoned on billboards, T-shirts and banners --- but there's
widespread disagreement on precisely what he represents. The Venezuelan
government routinely disparages him as a right-wing reactionary from the
ruling class who wants to reverse the social progress of \emph{chavismo}
and restore the landed aristocracy; the Venezuelan right, meanwhile,
considers López a neo-Marxist, whose proposal to distribute the
country's oil wealth among the people would only deepen the
\emph{chavista} agenda.

For three and a half years in prison, López refused to let anyone speak
for him. Though he was prohibited from granting interviews or issuing
public statements and was often denied access to books, paper, pens and
pencils, he managed to scribble messages on scraps of paper for his
family to smuggle out, and he recorded a handful of covert audio and
video messages denouncing the Maduro government. From time to time, he
could even be heard screaming political slogans through the bars of the
concrete tower in the military prison where he was kept in isolation.

López was released to house arrest last July on the condition that he
fall silent. He promptly climbed the fence behind his house to rally a
gathering crowd, then issued a video message asking his followers to
resist the government. Three weeks later, he was back in prison; after
four days, he was released again. Ever since, to the great bewilderment
of his supporters, he has vanished from public view. While the country
descends into an unprecedented crisis --- with the world's highest rate
of inflation, extreme shortages of food and medicine, constant
electrical blackouts, thousands of children dying of malnutrition,
rampant crime in every province, looting and rioting in the streets ---
López has said nothing.

Image

Venezuelan families forced to leave their homes because of economic
conditions now live in Colombia under a bridge that connects to
Venezuela.Credit...Sebastián Liste/NOOR, for The New York Times

Today his critics include not just the hard-line left and right but much
of the Venezuelan majority that once saw him as a future president. They
don't understand what López is doing inside that house, tucked away on
that leafy street in the wealthy suburbs of Caracas, but they suspect
that he has grown comfortable there, reunited with his wife and
children; that his family's wealth insulates him from the economic
crisis; that the secret police who surround his home protect him from
rising crime; and they can't help wondering if Leopoldo López has
finally given up. They know, as he knows, that if he issued a public
statement or released another video message, if he climbed the fence
behind his house to address his followers again, the secret police would
swoop in to haul him back to prison. But López never let the risk of
prison stop him before. He would have at least one chance to speak, and
they wonder why he hasn't.

\textbf{There's a flicker} onscreen whenever López connects, then a blur
of pixelated color as his face comes into view. On different days, at
different times, he can look very different. There are mornings when he
turns up in an old sweatshirt with unkempt hair and a weary smile, and
others when he appears in a pressed oxford shirt with parted hair and
black-rimmed glasses that do nothing whatsoever to obscure the haunted
air of a sleepless night.

I think of a Saturday in October. It was a few minutes after noon. I was
out for a walk with my kids when a message from López popped on my
phone. ``The situation is very delicate,'' he wrote. ``I may be on the
borderline of going back to prison.'' Scrambling home, I opened my
laptop and after a minute, he appeared onscreen. López is 46, 5-foot-10
and fit. He was sitting at the desk in his living room with his hair
poking off at angles, and the cast of his expression was an admixture of
fear, fatigue and fury.

When the audio clicked on, I asked what was happening. López took a deep
breath. He propped an elbow on the desk and rested his head in his hand.
``Last night around 7:30, they came to my house --- more than 30
officers of the political police,'' he said. ``They had more than 10
cars. They closed the entire street. And then they came into my house.''
For more than a decade, López has employed a private security detail, as
political opponents stormed his events with masks and guns, sprayed his
car with bullets and murdered one of his bodyguards. Under house arrest,
he is allowed to maintain a small guard outside. During the raid, López
said, the police took his chief of security into custody, and no one had
heard from him since. ``There was absolutely no reason, legally, they
could take him, and they have not allowed any lawyers to go in to see
him,'' López said. He looked down at his desk and shook his head. ``So
that's the situation,'' he said quietly. ``And I wanted to tell you that
I'm willing to go forward with this, what we're doing.''

At that point, we had a very different sense of what we might be doing.
We had been in contact for only a few weeks. I first reached out to
López through an intermediary in August, not long after his return to
house arrest, and by September we were talking a couple of times a week,
usually for a couple of hours at a time. This was a clear violation of
his release. An order from the Venezuelan Supreme Court specifically
forbid him to speak with the news media, and we didn't expect to get
away with it. At a minimum, it was safe to assume that his house was
bugged for sound, but there were probably hidden cameras as well, and
his computer was surely hacked and his internet activity monitored.

The world is full of byzantine methods to communicate through encrypted
channels, but most of them are obviated for a person who is trapped in a
digital glass house and surrounded by the state security of an
authoritarian regime. We did what little we could to be discreet,
knowing it wasn't much. Rather than connect on Skype or FaceTime, we
used an obscure video service, which seemed at least marginally less
likely to be a platform the police had practice hacking. Whenever we
spoke, López wore a pair of headphones, so a conventional audio bug
would only pick up his side of the conversation, and we adopted the
general posture of old friends catching up. This was not as much of a
stretch as it may seem. López is three years older than I am and a
graduate of Kenyon College, where we briefly overlapped but never met.
From time to time, one of us would mention the school or someone we both
knew there, and our kids would periodically wander onscreen to wave
hello.

All of this seemed hopelessly primitive in the face of state
surveillance, but then again, it seemed as if it might be working. Once
in a while a big white van would appear in front of his house and the
connection would go dark, but within an hour or two, the van would leave
and we'd get back online. Neither of us could explain why, if government
agents were listening, they hadn't shut us down or come inside to arrest
him. There was every reason to believe that they would. In our
conversation that day in October, López mentioned that the agents
raiding his home had given just one reason: They believed he was talking
to a reporter and recording a video message. This led to a curious
moment in our conversation, as the recording system on my end of the
interview captured his denial. ``It's not true,'' he said to anyone
listening. ``I've had no contact with any journalist!''

I don't mean to make light of the situation, but the truth is, we often
did. Venezuela was coming off a summer that promised change. In July,
the opposition movement sponsored a nonbinding referendum on the
government's plan to rewrite the Constitution. With more than seven
million ballots cast, 98 percent of voters opposed the government. Soon
after that, emissaries from the government approached opposition leaders
to begin a formal negotiation, with a primary focus on the release of
political prisoners. Even as we spoke that day in October, the country
was preparing for regional elections in which opposition candidates were
expected to win by a landslide.

There were contradictory signals, of course, but the trajectory was
toward transition. This was a time when headlines everywhere predicted a
``\href{https://www.washingtonpost.com/world/the-venezuelan-government-is-beginning-to-lose-the-poor-its-longtime-base/2017/04/28/c562cb86-2b5d-11e7-9081-f5405f56d3e4_story.html?utm_term=.d7487c5be838}{turning
point}'' for Venezuela, and I think on some unspoken level, López was
counting on it. It was a gamble to speak publicly, but it wasn't crazy
to imagine that by the time this article appeared, the political
landscape could be transformed --- that the opposition, which already
held a supermajority in Congress, could win a similar proportion of
governorships; that a regional victory would carry into the municipal
campaign in December; that they would enter this year's presidential
race with momentum against a deeply unpopular president, polling at 25
percent; that the negotiation for political prisoners might even allow
López to challenge Maduro himself. Polls at one point suggested he could
win the presidency by a margin of 30 percent.

This was the conversation that I think we expected to have last summer:
a look at the next chapter for Venezuela and the role he might play.
Instead, with each passing day, the possibilities grew more slender.
When voting began on the morning after that October call, nothing went
as expected. The polling places for more than 700,000 citizens had
mysteriously moved, in some cases so far that it would take hours for
them to travel on crowded buses. Even so, by evening, officials were
reporting overwhelming turnout and a breathtaking upset: Candidates from
the ruling party had swept all but five of the governor's races. Amid a
global
\href{https://www.wsj.com/articles/venezuelas-latest-election-fraud-1508106069}{outcry
of fraud}, several opposition parties withdrew from the municipal
elections --- and the government responded by invalidating those
parties. The scheduled negotiation with opposition leaders began in
November. In January, the talks collapsed. In February, officials
dissolved the whole coalition of opposition parties.

Political leaders of every stripe were being detained by the secret
police. The second-ranking member of Congress went into hiding at the
Chilean Embassy, and about a dozen mayors fled the country. The state,
the economy and the social fabric were unraveling all at once.
Everywhere, people were leaving Venezuela any way they could. They piled
into ramshackle boats and died at sea. They walked the highway toward
Brazil, collapsing in heat and sun. They poured into Colombia, tens of
thousands each day, a refugee crisis comparable in number to the flight
of the Rohingya to Bangladesh. It seemed that every time I spoke with
López, another friend had taken sanctuary in an embassy, gone to prison
or fled.

Image

Venezuelans crossing into Colombia in February.Credit...Sebastián
Liste/NOOR, for The New York Times

I asked him one day recently how he was managing the pressure. The
secret police had just returned to his house with another order to
arrest him, and he was saying goodbye to his wife, Lilian, who was eight
months pregnant, when one of the agents received a phone call to suspend
the arrest. It wasn't clear when they might return.

``How are you feeling?'' I asked.

``It's tough,'' he said. ``It's tough after what happened. Every day I
think is the last day I have to be with my kids.''

I asked if he ever thought about trying to escape. ``Most people tell me
that I should,'' he said. ``But I believe a commitment to the cause
means that I need to take the risk.'' As he spoke, I realized that what
we had actually been talking about all these months, what he had been
trying to communicate through this portal from his silence, was never
really about the future of Venezuela or the role he hoped to play, and
it wasn't about political ambition or the next chapter in history. It
was something fundamental that kept coming up in offhand remarks. It was
something that he learned in prison about the history we're always in.

\textbf{The line of people} waiting to leave Venezuela begins to form an
hour before dawn. Migrants drift through unlit streets in the border
town of San Antonio to gather at the foot of the Simon Bolívar Bridge,
where they wait beneath a huge red banner that reads, ``Don't speak ill
of Chávez.'' When the checkpoint opens at 6 a.m., they push forward,
moving shoulder to shoulder down the two-lane highway into Colombia.
There will be no tapering off through the day; there is no end to the
people coming. Some have traveled more than a week to get here. It only
takes a glance to see what sort of people they are: every sort, of every
age, from every profession and social stratum --- young families and
older couples and clusters of itinerant boys and solitary young women
looking several weeks overdue. If you stop for a moment on your way
across the bridge, you can almost feel the deflating wind of the
Venezuelan exodus at your back.

Historians have come up with all sorts of arguments about the arc of
Venezuelan history and how things went so wrong. A couple of points
strike me as indispensable to any case: Venezuela is the birthplace of
Latin American independence and sits on the largest proven reserves of
oil in the world. How you interpret the role of these factors in any
given historic event is a matter of personal politics and granular
debate, but you can't have a serious discussion about Venezuela without
taking both into account. For most of the past century, the country has
whipsawed between political movements that court and reflect and
sometimes renounce the legacy of anti-imperialism and the towering glut
of riches.

Like most of its neighbors, Venezuela endured a succession of
\emph{caudillo} strongmen in the early 20th century and responded with a
radical leftist movement in the 1950s and '60s. Unlike its counterparts
in neighboring countries, the Venezuelan left didn't get very far. Some
of them took up arms in the mountains and limped through a series of
gunfights, but by the end of the '60s, most had returned to a marginal
place in conventional politics. One of the few who stayed in the fight
was a guerrilla named Douglas Bravo, who called his nationalist
political ideology ``Bolívarianism.'' Bravo finally settled in Caracas
in the 1980s, where he developed inroads with disaffected citizens and
soldiers in the Venezuelan Army. Two of the acolytes he courted were the
brothers Adán and Hugo Chávez, who welcomed the idea of leading a
Bolívarian coup.

It took about a decade of recruitment and planning, during which time
the Venezuelan establishment seemed to be doing everything possible to
help them. For decades, the two major parties essentially passed the
presidency back and forth in a power-sharing agreement that gave little
heed to the country's swollen underclass. Venezuela's economic fault
lines had become so fraught that in 1989 a rise in bus fares helped
trigger deadly riots. By the time the Chávez brothers were ready to
attempt their coup in 1992, a lot of Venezuelans were just happy to see
the establishment take a hit. Although the coup failed and Chávez spent
two years in prison, he emerged as a minor celebrity. By 1998, he was
running for president.

Image

Gas smugglers from Venezuela on one of the dozens of rural routes that
connect Venezuela and Colombia by crossing the Táchira
River.Credit...Sebastián Liste/NOOR, for The New York Times

It's easy now, with the country in turmoil, to dismiss the whole project
of \emph{chavismo}. But the election of Chávez in 1998 coincided with a
groundswell of social and political movements for whom Chávez, with his
energy and outrage, pledging to crack down on corruption and raise the
minimum wage, seemed a natural ally. Whatever else Chávez became, he
delivered on many of his promises. During his tenure, unemployment fell
by half, the gross domestic product more than doubled, infant mortality
dropped by almost a third and the poverty rate was nearly halved. You
can chalk this up to other factors, like a tenfold increase in the price
of oil that showered his administration with revenue, and you can argue
that Chávez failed miserably to anticipate the next downturn in oil
prices, but you can't really accuse him of making promises to the poor
and then delivering to the rich, or keeping all the money for himself.
Under his watch, income inequality dropped to one of the lowest levels
in the Western Hemisphere. Chávez didn't have to steal elections. He was
wildly popular among the poor and put his proposals up for election
almost every year. He introduced touch-screen voting, with thumbprint
recognition and a printed receipt, an electoral system that Jimmy Carter
described as being, among all the countries he had monitored, ``the best
in the world.''

Chávez also possessed an autocratic impulse that was jarring from the
start. Over the course of 14 years in office, he dismantled the
country's democratic institutions one by one. There's an interesting
debate among political theorists about what to call a leader who
destroys a democracy with democratic support. It's possible to think of
Chávez as a totalitarian or a tyrant for suppressing his opponents while
rejecting the term ``dictator'' to describe a popular president. Chávez
made no secret of his contempt for the country's extant political
system; he couldn't even get through his first inauguration without
ad-libbing, in the middle of the swearing-in ceremony, a promise to
rewrite the Constitution --- which he promptly did, consolidating power
over the Legislature and the courts.

Any limit that Chávez might have been willing to accept on his power
vanished in April 2002, when a junta of military officers and right-wing
leaders tried to oust him in a coup. For about 36 hours, they installed
as president a man named Pedro Carmona, who was the director of
Venezuela's primary business consortium. Carmona's government proceeded
to undermine institutions at a clip that would make even Chávez blush.
In the single day of his presidency, he dissolved the Legislature, the
Supreme Court and the Constitution and began to cleanse the Venezuelan
military of anyone loyal to Chávez. This was too much even for critics
of \emph{chavismo}. The streets of Caracas exploded in protest, and
crowds descended on the presidential palace. Soon Chávez was back in
office, consolidating power more quickly than ever. He persecuted rivals
and stacked the courts and levied so many restrictions on industry that
the private sector essentially disappeared.

You can think of the decade between the coup and his death in 2013 as a
gradual process of bleeding out public resources for public consumption.
At a basic level, Chávez just wasn't very good at managing an economy.
His budget spent the revenue from skyrocketing oil prices, and his
control of the state oil company proved disastrous. Chávez believed that
because oil reserves are a finite resource, it made sense to limit
production and drive up the price of every barrel. This way of thinking
is widely disputed, if not debunked. Producers are constantly developing
new ways to find and access oil; between the American shale revolution
and rising competition from alternative energy, most oil companies today
want to pump as much oil as quickly as they can.

When Chávez took power in Venezuela, the state oil company was producing
about 3.4 million barrels per day. Its leadership planned to almost
double the volume. Instead, through a combination of Chávez's misguided
theories and a general failure to invest in the company and installing
his personal henchmen to run it, production of Venezuelan oil has fallen
by nearly half. Oil prices have also dropped considerably over the past
few years, but the country has little else to sell. According to the
most recent data, oil accounts for about 95 percent of Venezuelan export
earnings. Much of that oil is being shipped to Russia and China in
exchange for help with the national debt, giving both countries
expansive claims on Venezuelan production. The more desperate the Maduro
regime becomes, the more these countries stand to gain.

What you've got then is a domino cascade: less and less oil, at lower
and lower prices, with nothing else to sell, and a dependence on foreign
money at the expense of future income. The final chip in the cascade has
been Venezuela's currency. As national revenue plunged, leaving a gap in
the annual budget, Chávez and Maduro turned to the central bank to print
more money. The number of Venezuelan bolívars has grown exponentially in
recent years. When Maduro took power in 2013, the country's monetary
base was about 250 billion bolívars. Today, it's more than 60 trillion.
For a sense of scale, imagine if you had \$5,000 yesterday, and today it
was \$1.2 million. I don't mean to suggest any meaningful comparison
between your savings account and a national economy, but it's not
difficult to imagine how a huge increase in money distorts the way
people spend it.

Most countries around the world produce official inflation reports. The
Venezuelan government has essentially stopped. One of the world's
leading experts on hyperinflation is a professor at Johns Hopkins
University named Steve Hanke, who has advised governments around the
world on runaway inflation, including Venezuela in 1995 and '96. Hanke
has been tracking the Venezuelan economy closely for the past five
years, producing a daily estimate of the country's annual inflation. As
I write this, his most recent estimate was 5,220 percent. The
International Monetary Fund has predicted that inflation in Venezuela
will reach 13,000 percent this year.

Image

A Venezuelan gas smuggler.Credit...Sebastián Liste/NOOR, for The New
York Times

Image

Joint operations between the police and the Colombian military try to
stop smuggling.Credit...Sebastián Liste/NOOR, for The New York Times

This is what you see in the faces of the people on the bridge leaving
Venezuela. You see people trying to escape a country where basic
supplies are nearly impossible to find and prohibitively expensive,
where the price you paid for a car a few years ago won't buy a loaf of
bread today. You see families with roll-aboard luggage and no plans to
go back, and children who are crossing just for the day with nothing but
a bunch of bananas. They will sell the bananas for a pittance in
Colombian pesos, then return home to convert the cash into a small
fortune of Venezuelan currency --- at least for a few days, when their
money will be worthless again.

\textbf{López was born} to privilege in the wealthy enclaves of
northeastern Caracas. His father, Leopoldo López Gil, was the head of an
international scholarship program who sat on the editorial board of a
center-left newspaper. His mother, Antonieta Mendoza, was a distant
relative of the first president of Venezuela, Cristóbal Mendoza, and of
Simon Bolívar. Each side of the family had long traditions of political
activism and dissent. López grew up hearing about his
great-grandfather's 17 years in prison and his grandfather's role in the
underground resistance. ``We always heard those stories,'' his sister
Diana told me. ``I think it was always in Leo's memory.'' López said he
relished the history in part because it felt so alien, snapshots of a
country that he couldn't quite imagine. ``I saw it as a faraway past of
black-and-white pictures,'' he told me. ``I never thought that in the
21st century, my own reality could be similar.''

The Venezuela that López inhabited was the wealthiest country in Latin
America. It welcomed tens of thousands of immigrants every year and had
been a democracy since 1958. Skateboarding, swimming, crazy for girls,
López at 13 was largely removed from the country's systemic inequities.
He was on a school trip to the rural state of Zulia, passing through the
region's oil fields, when he found himself unexpectedly moved by the
destitution around him. ``I was shocked by the poverty level,'' he
recalled, ``and the fact that below these very humble barrios and
dramatic poverty, we had huge potential.'' Diana told me that López
began making trips into western Caracas ``to try to understand the
dynamics of the city.'' At school, he immersed himself in student
leadership, becoming vice president of the student government and
captain of the swimming team.

After college, López briefly enrolled at Harvard Divinity School but
left after one semester to enroll in the Kennedy School of Government at
Harvard. He completed a master's thesis on the legal and economic
framework of oil production in Venezuela and traveled through Nicaragua
and Bolivia to study the impact of microloans. In 1996, he returned home
for a job in the strategic planning office of the state oil company.

Watching the ascent of Chávez in 1998, López was unimpressed. ``Ever
since the establishment of the Venezuelan republic in 1830, for the most
part we've had military in the government,'' he told me. ``And that has
created a militaristic way of governing.'' I asked López if there was
ever a point when he reconsidered his opinion of Chávez. ``For one
day,'' he said with a laugh. ``When he spoke about microcredits to the
poor.''

As Chávez took office and began making plans to rewrite the
Constitution, López campaigned for a seat in the constitutional
convention. He lost that election but rallied with two other failed
candidates to create a new political party, then he entered the 2000
campaign for mayor of the city's most affluent borough. He won with 51
percent of the vote.

Over the next eight years, López gained international attention as mayor
of Chacao. He began by raising business taxes while offering incentives
for companies to move into the district. With revenue up, he commenced a
series of public works, building health clinics and schools, a theater,
a public market and a recreation center. Still unmarried and in his
early 30s, he was comically hands on, forever rolling up his sleeves at
groundbreaking ceremonies and appearing, in Cory Booker fashion, at
predawn crime scenes to consult with detectives in the blinking red
light. In a city notorious for crime, he implemented policing measures
that were popular in the United States --- think ``zero tolerance'' and
``broken windows'' and ``compstat'' multivariate analysis. His platform,
then, was a heterodox mash-up of initiatives that span the political
spectrum, from lefty measures like raising corporate taxes to
conservative models of policing. Residents loved it. In 2004, he was
re-elected with 81 percent of the vote, and during his second term he
met and married a prominent television personality named Lilian Tintori.
In 2008, López left office with 92 percent approval and a ranking from
the City Mayors Foundation as the
\href{http://www.worldmayor.com/contest_2008/world-mayor-2008-results.html}{third-best
mayor in the world}.

Skimming this résumé, you can see why people often regard López in a
gauzy half-light, but even as he thrived as mayor of Chacao, he was
becoming a polarizing figure. By the end of his second term, he was one
of the most promising young politicians in Venezuela and one of the
least capable of getting along with others. Within the opposition
movement, López represented a radical wing. The word ``radical'' is
often used about López in a misleading way. He favors a mixed economic
model of expansive social services in health care, education and
housing, offset by a large private sector of manufacturing and industry.
On the spectrum of American politics, he would probably land in the
progressive wing of the Democratic Party.

Where you can describe López as a radical is the way he approaches
political activity. He believes that a relentless campaign of street
demonstrations and civil disobedience is essential to challenge an
authoritarian government. On any given day in 2002, a person walking
through Caracas had a good chance of spotting the mayor of Chacao
standing on a bench in some public park, bellowing at a crowd through a
megaphone. How useful this was to the project of building a mature
political party with governing potential was a matter of opinion, but
López believed that the movement would get nowhere by relying on
decorous party mechanics.

One way to measure the success of a strategy is to study its response.
López became a frequent target of physical and administrative attacks.
From 2002 to 2006, there were three major attempts on his life, one of
which left him cradling a bodyguard dying of a gunshot meant for López.
During his tenure as mayor, he was accused by the comptroller's office
of paying municipal expenses from the wrong part of his budget and
barred from seeking public office until 2014. López appealed the
decision and prepared to run for mayor of Caracas. He was leading with
65 percent of the vote when the Supreme Court upheld the comptroller's
decision. The Inter-American Court of Human Rights ruled the ban illegal
and ordered Venezuela to let López run, but the government ignored the
order and López has been forbidden to hold public office ever since.

By 2008, he was also clashing with other opposition leaders. He left the
party he helped found, joined another and soon had issues with its
leadership as well. In August, that party expelled him, and he began
making plans to create yet another. American diplomats in Caracas
weren't sure what to make of López. A classified cable to Washington
described his ``much-publicized rebelliousness,'' noting that López
``will not hesitate to break with his opposition colleagues to get his
way.'' Another referred to him as ``a divisive figure'' who was ``often
described as arrogant, vindictive and power-hungry.''

Image

Venezuelan migrants in Cúcuta.Credit...Sebastián Liste/NOOR, for The New
York Times

In 2012, still barred from running for office, López threw his support
behind an opposition candidate in the presidential election. Chávez beat
that candidate by a 10-point margin but died soon after, opening the
door for a repeat campaign against Maduro. When the electoral board
announced that Maduro had won by a single percentage point, López
suspected fraud. He pushed for the opposition movement to stage a public
demonstration. Most of the other opposition leaders dismissed the idea,
but in January 2014, López called for his supporters to take the
streets.

By February, protests were springing up in every province. On Feb. 12,
López rallied thousands of students at the edge of a park in Caracas.
After his speech, they marched to the office of the attorney general a
mile away. Some of the protesters began throwing rocks at the building.
Security officers emerged, and two protesters were shot. Though López
was gone before the violence began, officials accused him of being the
``intellectual author'' of the skirmish, and the attorney general issued
a warrant for his arrest.

López and Tintori took refuge that night in a friend's apartment. They
recorded a video message for the public. ``I want to say to all
Venezuelans that I do not repent,'' López said. He spent a few days in
hiding, then recorded another video asking his supporters to gather at a
downtown plaza on Feb. 18, dressed in white as a sign of peace, to bear
witness as he turned himself in.

That morning, he climbed on a motorcycle and rode into the city. A large
crowd was gathering, and the police had set up checkpoints to intercept
him. López tried to find a way around the checkpoints but couldn't. He
finally rode up to a cluster of police officers from the Chacao district
and removed his helmet. The officers recognized him, saluted and waved
him through. López saw the crowd extending in every direction. Thousands
upon thousands of people had come dressed in white. He waded through
them to a statue of the Cuban independence hero José Martí and climbed
the pedestal to look over the sea of faces. Someone handed him a
megaphone, and he raised it. ``If my imprisonment helps to awaken a
people,'' he called out, ``then it will be worth the infamous
imprisonment imposed on me.''

After a short speech, he climbed down from the pedestal, where soldiers
were waiting to arrest him. They pulled him inside an armored vehicle,
but the crowd pressed in, rocking it. Minutes passed, then half an hour.
The truck was trapped by the crowd. Someone gave López a handset
connected to the vehicle's outside speakers. He called to the crowd that
he was safe and that they should clear a way for the truck to get
through. Slowly, almost grudgingly, they parted the path for López to
prison.

Image

An opposition demonstration in Caracas in February.Credit...Carlos
Becerra for The New York Times

\textbf{Officials placed López} in a concrete tower on a military base
outside the city, charging him with terrorism, arson and homicide.
Amnesty International condemned his prosecution as ``an affront to
justice'' and ``a politically motivated attempt to silence dissent.''
For his initial arraignment, he was taken from his cell in the middle of
the night and marched outside to face a judge on a bus. The rest of the
proceedings took place at the Palace of Justice in Caracas, a five-story
edifice that sprawls across 1.5 million square feet downtown. Over the
next 19 months, he traveled there nearly 100 times, in a motorcade of
armored S.U.V.s, wearing a bulletproof vest, with his hands shackled
together, wedged between two guards armed with machine guns and two more
behind him. Each time López appeared in court, the Palace of Justice
shut down.

The trial hinged on speech. No one accused López of being violent
himself. Prosecutors scaled back the charges, arguing that he inspired
violence in others. They brought in a linguistic expert to examine
transcripts of his speeches and claimed that his message of peaceful
protest disguised a ``subliminal'' call to violence. They introduced
more than 100 witnesses, some of whom testified that they had received
the subliminal messages. López tried to introduce his own witnesses, but
the judge wouldn't allow it.

A few words about the judge who signed his arrest warrant and a lead
prosecutor and attorney general: They all repent. The judge who signed
the warrant later admitted that she had been forced to do so. The lead
prosecutor, after fleeing the country, denounced the case against López
as ``a farce,'' saying ``100 percent of the investigation was
invented.'' The attorney general, Luisa Ortega, escaped to Colombia last
summer and says that the vice president of Maduro's party instructed her
to pursue López. I tracked Ortega down a few weeks ago, and we met for
coffee in Bogotá. When I asked her about the criminal charges against
López, she shook her head in dismay. ``Without a doubt,'' she said,
``Leopoldo López is a political prisoner.''

Ortega told me it had been illegal to hold a civilian like López in a
military prison. Over the course of three years, his conditions grew
progressively worse. In the early stage, he was allowed to read and
write, and a local university devised a program of study. He read
Venezuelan poets, Ralph Waldo Emerson, the diary of Ho Chi Minh and a
biography of Nguyen Van Thuan. He was consuming several volumes a week,
until officials began to restrict what he could read. Eventually, they
prohibited everything except the Bible. He read it from Genesis to
Revelation. Then they took the Bible, too.

Image

A migration center run by the Catholic Church in
Cúcuta.Credit...Sebastián Liste/NOOR, for The New York Times

Image

Outside the refuge in Cúcuta, Colombia.Credit...Sebastián Liste/NOOR,
for The New York Times

He was moved to a new cell, then another. He spent months in solitary
confinement in a room that was six feet by 10 feet. He would sit in
silence trying to pray or meditate and summon any possible reason for
gratitude: that he could feel himself breathing; that his wife and
children were safe; that through the window he could hear the commotion
of the outside world --- a passing truck, a twisting wind, some emphatic
bird.

Without his books, he reflected on those he had read. He remembered
biographies of nonviolent leaders and the Birmingham letter from King,
and he began to wonder if what they had in common wasn't just a
commitment to resistance but some deeper observation about the character
of history. This came through most clearly in King. His goal was never
just to provoke or confront. It was to locate the elusive fulcrum
between conflict and mediation --- to produce an onslaught of pressure
that forced officials to react while preserving an almost irrational
faith in their capacity for good will.

``I had an illumination moment,'' López recalled. ``One night, I
couldn't sleep, and I was tossing from one side of the bed to the other,
thinking about the son-of-a-bitch director of the prison. I was very,
very angry, and I woke up the next morning, and I said: `What am I
doing? This guy is taking away my tranquillity, my sleep.' '' He
realized that the buildup of anger threatened to distort his thinking.
He began trying to separate his outrage from his fury. He continued to
defy the arbitrary rules of prison --- composing and smuggling a stream
of subversive messages to the outside --- but when the guards would
charge into his cell to look for contraband, shouting and tearing
through his things, he searched for calm. He would stand back, lifting
his hands in a posture of self-defense and say in the most measured tone
he could muster that he would protect himself if necessary. In the hours
between, the interminable stretches of solitude, he tried to be honest
with himself about what anger had cost him. It wasn't just a threat to
his state of mind but to his politics, his movement and the way he
conceived the future.

``In the past, I was in confrontation with different views,'' he told
me. ``Now I understand that everybody is needed in order to reach a way
out of this disaster.'' He thought of books he had read on postwar
Europe and the South African emergence from apartheid, and he realized
that Venezuela would never find stability if it were cleaved into
disparate sectors. It would be necessary to forge, like Mandela with
F.W. de Klerk or King with Lyndon Johnson, some tentative
\emph{confianza} between the opposition and supporters of
\emph{chavismo}. ``A lot of people in the opposition have resentment,
and I understand that,'' he told me. ``But I think our responsibility is
to move beyond the personal resentment. Four years in prison have given
me the possibility of seeing things a different way, of putting rage in
its perspective.''

\textbf{A few nights ago,} I was speaking with López a little before
midnight. His family was asleep, and in the quiet hours he was bracing
for the possibility that appearing in these pages could trigger his
return to prison. This was something we had talked about many times. His
eldest daughter was a toddler when he first went to prison and is now a
little girl. His son had been less than a year old and was just now
getting to know his father. At the end of January, López and Tintori had
a second daughter, and it troubled him to think that years could pass
before he saw any of his children again.

``It's not easy,'' he said quietly. ``It's not easy, but I have the
responsibility to speak my mind. I've been in prison four years now
because of speaking my mind, and if I self-censor, I'm beaten by the
dictatorship.'' López said he still believed that with the right
leadership, Venezuela could rebound. He thought of postwar Japan and
South Korea and Europe. He knew that stabilizing the bolívar could be
accomplished by attaching its value to a foreign currency, and that
under a new government, the private sector would return. He believed the
country's oil production would recover under good management, and he had
been working for nearly a decade on a plan to convert the national oil
company into a kind of Social Security trust, with investment shares
assigned to the public for retirement, education and emergencies.

Image

López with his children, Manuela (left) and Leopoldo, at home in
January.Credit...Diana López for The New York Times

The challenge was to reach a point where any of that work could begin.
As the crisis in Venezuela deepened, the path to a transition seemed
more obscure than ever. Politicians, historians, think-tank
pontificators --- everyone had some sort of proposal, but the problem,
if you studied each of them, was that none had very much chance of
happening, or of working.

Start with the Trump administration, which has lately suggested a
military coup. Speaking in February, Secretary of State Rex Tillerson
mused that in a situation like Venezuela's, ``it's the military that
handles that,'' to which Senator Marco Rubio later added on Twitter that
the Venezuelan military should ``restore democracy by removing a
dictator.'' Apart from the obvious fact that removing a dictator is no
guarantee of democracy, there aren't many people in Venezuela who
consider a coup likely. A few weeks ago, I met up with the leader
installed by the last military coup, Pedro Carmona, who told me the
military has been purged of dissent, with senior officers monitored for
ideological purity by the Cuban intelligence service. ``The G2 has a
facility in Caracas, spying on the Venezuelan military,'' he said. ``So
on the military side, the best I could hope would be for them not to
repress the people.''

Pressure from outside Venezuela has also been slow to coalesce. Critics
accuse the members of the Organization of American States of failing to
constrain the Maduro government, which showers oil bounty on several
member nations. A smaller coalition of Latin American countries has
joined with Canada to create the Lima Group, whose vociferous
condemnation of the political repression has not converted to much
concrete action. American sanctions have been steadily tightening in
recent years. After a protracted debate between the National Security
Council and the State Department, the Obama administration imposed
limited sanctions in 2015, primarily targeting the financial assets of
individual Venezuelan leaders. Mark Feierstein, who assumed
responsibility for N.S.C. policy in the Western Hemisphere later that
year, told me the administration had missed a critical opportunity to
influence a 2016 negotiation between the Maduro government and the
opposition. ``The N.S.C., or at least I, was inclined to move more
quickly,'' he said, ``and I think the negotiations largely failed
because pressure was taken off.'' The Trump administration has expanded
the sanctions program, but how far to deepen sanctions, or expand them,
or restrict the import of Venezuelan oil, is a brutal calculation about
how much of the burden would be carried by the Venezuelan people, and
whether adding to their misery is more likely to inspire an uprising or
simply worsen the humanitarian disaster.

In recent months, there has also been rumbling about war. Trump has made
oblique suggestions of a ``military option'' in Caracas, and even
relatively moderate voices have begun to fantasize about cavalry. In
January, the Harvard scholar Ricardo Hausmann, who served as Venezuela's
minister of planning from 1992 to 1993, published a proposal suggesting
that the Legislature invite a multilateral invasion force to help
support a new government, making a comparison to the liberation of
Europe. I spoke with several opposition leaders who welcome this idea,
but this might say more about the country's desperation than the wisdom
of the proposal. It's difficult to imagine Russia and China, after years
of propping up the Venezuelan economy in exchange for oil, allowing a
foreign invasion to threaten their investment. An even greater concern
is internal: Maduro is polling at about 30 percent approval in a
devastated economy, but nothing would rally former \emph{chavistas} to
his side like an occupying army. Venezuela is a heavily armed society
and increasingly violent. To invite a military intervention is to
welcome civil war.

A few months ago, it was possible to imagine an electoral path to
change, but today nearly all the opposition parties have been
disqualified from running. On the evening of Feb. 15, Maduro took this a
step further, interrupting television and radio broadcasts to announce
that the party López founded in 2009 is not a political organization but
a ``violent fascist group'' operating ``outside the law.'' When I spoke
with López the next morning, he said that 87 party leaders were already
in prison. Those who remained were preparing to convert the party into a
``clandestine organization.'' Soon, he said, they could be reduced to
secret meetings and tossing pamphlets on street corners from unmarked
vans.

But even as conditions spiraled down, I watched López try to incorporate
what he learned in prison to daily life. Unable to speak publicly, he
developed a network of private channels --- reconnecting with leaders of
the political parties from which he'd split, making inroads with members
of the Maduro government and with foreign ministers and heads of state.
During the recent negotiation between opposition leaders and the
government, López was in contact with all sides; even after his party
withdrew from the dialogue, he continued to consult with leaders who
remained at the table. When disputes spilled over among them, he
provided a back channel, an invisible hub to which it seemed as if all
spokes connected.

López was also flexible in his thinking about transition. Through most
of our conversations, he strongly opposed the idea of military action,
but when we spoke late the other night, he said he was beginning to
think differently. An unwelcome mechanism can bring welcome change.

``In 1958, there was a military coup that began the transition to
democracy,'' he said. ``And in other Latin American countries, there
have been coups that called elections. So I don't want to rule anything
out, because the electoral window has been closed. We need to go forward
on many different levels. One is street demonstrations; a second is
coordination with the international community. But this is how I'm
thinking now: We need to increase all forms of pressure. Anything,
anything that needs to happen to produce a free and fair election.''

If it was jarring to hear this from López, it was matched by another
development. For several months, the secret police had been coming to
his front door about four times a day to photograph him with a copy of
the day's newspaper. Lately, López had begun to invite the agents in. He
had recently spoken with one for more than two hours, offering him a
slice of cake from his daughter's birthday and talking about the
inflation crisis and the recent massacre of a small rebel group. ``We've
developed --- I wouldn't say a good relationship, but a relationship,''
he said.

Thinking about these developments together, it seemed to me that López
was trying to strike an increasingly difficult balance. He was willing
to entertain proposals that he found abhorrent six months ago, but he
was also making a greater effort to open the door for dialogue. The
struggle he faced was a heightened version of the tension in all
history. It was to locate the elusive fulcrum between his rage and
faith.

Advertisement

\protect\hyperlink{after-bottom}{Continue reading the main story}

\hypertarget{site-index}{%
\subsection{Site Index}\label{site-index}}

\hypertarget{site-information-navigation}{%
\subsection{Site Information
Navigation}\label{site-information-navigation}}

\begin{itemize}
\tightlist
\item
  \href{https://help.nytimes3xbfgragh.onion/hc/en-us/articles/115014792127-Copyright-notice}{©~2020~The
  New York Times Company}
\end{itemize}

\begin{itemize}
\tightlist
\item
  \href{https://www.nytco.com/}{NYTCo}
\item
  \href{https://help.nytimes3xbfgragh.onion/hc/en-us/articles/115015385887-Contact-Us}{Contact
  Us}
\item
  \href{https://www.nytco.com/careers/}{Work with us}
\item
  \href{https://nytmediakit.com/}{Advertise}
\item
  \href{http://www.tbrandstudio.com/}{T Brand Studio}
\item
  \href{https://www.nytimes3xbfgragh.onion/privacy/cookie-policy\#how-do-i-manage-trackers}{Your
  Ad Choices}
\item
  \href{https://www.nytimes3xbfgragh.onion/privacy}{Privacy}
\item
  \href{https://help.nytimes3xbfgragh.onion/hc/en-us/articles/115014893428-Terms-of-service}{Terms
  of Service}
\item
  \href{https://help.nytimes3xbfgragh.onion/hc/en-us/articles/115014893968-Terms-of-sale}{Terms
  of Sale}
\item
  \href{https://spiderbites.nytimes3xbfgragh.onion}{Site Map}
\item
  \href{https://help.nytimes3xbfgragh.onion/hc/en-us}{Help}
\item
  \href{https://www.nytimes3xbfgragh.onion/subscription?campaignId=37WXW}{Subscriptions}
\end{itemize}
