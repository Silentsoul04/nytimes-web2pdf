Sections

SEARCH

\protect\hyperlink{site-content}{Skip to
content}\protect\hyperlink{site-index}{Skip to site index}

\href{https://www.nytimes3xbfgragh.onion/section/style}{Style}

\href{https://myaccount.nytimes3xbfgragh.onion/auth/login?response_type=cookie\&client_id=vi}{}

\href{https://www.nytimes3xbfgragh.onion/section/todayspaper}{Today's
Paper}

\href{/section/style}{Style}\textbar{}Antoni Porowski Can Cook

\url{https://nyti.ms/2D3SGLm}

\begin{itemize}
\item
\item
\item
\item
\item
\end{itemize}

Advertisement

\protect\hyperlink{after-top}{Continue reading the main story}

Supported by

\protect\hyperlink{after-sponsor}{Continue reading the main story}

\hypertarget{antoni-porowski-can-cook}{%
\section{Antoni Porowski Can Cook}\label{antoni-porowski-can-cook}}

\includegraphics{https://static01.graylady3jvrrxbe.onion/images/2018/03/15/fashion/06antoni1/merlin_134736825_7e0a1415-4557-49a4-b812-4fe717076458-articleLarge.jpg?quality=75\&auto=webp\&disable=upscale}

By \href{https://www.nytimes3xbfgragh.onion/by/bonnie-wertheim}{Bonnie
Wertheim}

\begin{itemize}
\item
  March 6, 2018
\item
  \begin{itemize}
  \item
  \item
  \item
  \item
  \item
  \end{itemize}
\end{itemize}

Antoni Porowski never said he was a chef, but he definitely knows his
way around a kitchen.

As the resident food and wine guy on ``Queer Eye,'' which had its
premiere on Netflix in February, he is responsible for imparting humble
lessons in home cooking to men who are undergoing weeklong
transformations. Mr. Porowski's food demonstrations are just part of a
series of self-improvement seminars the show's subjects attend. They go
shopping with Tan France. They get groomed by Jonathan Van Ness. They
experience ``culture'' with Karamo Brown. And the show's interior
designer, Bobby Berk, makes their homes livable and often gorgeous.

These makeovers are master classes in empathy. The Fab Five pepper their
subjects with compliments, I-know-where-you're-coming-froms and hugs.

But the cooking classes have brought out the worst in viewers, many of
whom feel that Mr. Porowski's dishes are overly simplistic. Their loud
critiques have fueled what The New Yorker's food correspondent Helen
Rosner
\href{https://www.newyorker.com/culture/annals-of-gastronomy/the-exquisite-blankness-and-highly-suspect-guacamole-of-antoni-porowski-from-queer-eye}{called}
a ``culinary conspiracy theory.''

On a recent Wednesday afternoon, at his Brooklyn apartment, Mr. Porowski
whipped up a minimalist spaghetti and meatballs that proved that
sometimes simple is anything but simplistic.

The tomato sauce was an adaptation of
\href{https://cooking.nytimes3xbfgragh.onion/recipes/1015178-marcella-hazans-tomato-sauce}{Marcella
Hazan's recipe}: San Marzano tomatoes; one onion, halved; and a stick of
butter. Mr. Porowski threw in fresh basil and Parmesan rinds for an
added kick.

``It's such a beautiful thing that you don't have to waste,'' he said of
the oft-discarded end of the cheese block.

\includegraphics{https://static01.graylady3jvrrxbe.onion/images/2018/03/15/fashion/06antoni2/merlin_134736831_9fbc49ce-4773-4e8e-8168-3c5ebd1bb923-articleLarge.jpg?quality=75\&auto=webp\&disable=upscale}

While the sauce simmered, Mr. Porowski, wearing a thin white T-shirt and
slim-fit jeans, prepared ground turkey meatballs with crushed fennel,
chili flakes, honey and more cheese. Behind him, a bowl of fresh
spaghetti sat below the mouth of an extruder. He would have to make more
for a dinner party that night, where he was to feed his boyfriend, Joey
Krietemeyer, and three of their close friends.

On ``Queer Eye,'' a reprise of Bravo's ``Queer Eye for the Straight
Guy,'' which ran from 2003 to 2007, Mr. Porowski shares recipes that are
useful for easy entertaining. He teaches Tom, who has been divorced
three times and is trying to rekindle love with his most recent ex, how
to make a creamy guacamole with Greek yogurt. He fries up grilled cheese
for Neal ahead of a release party for his app. He shows Jeremy, a
firefighter, how to dress up hot dogs to serve at a fire station
fund-raiser.

Think pieces criticizing these ``basic'' recipes swiftly followed.
\href{https://www.out.com/television/2018/2/26/every-episode-queer-eye-reboot-ranked-worst-best?pg=full}{Out
magazine} called Mr. Porowski's use of Greek yogurt in guacamole
``absolutely blasphemous.'' Bowen Yang, a writer at Vulture,
\href{http://www.vulture.com/2018/02/queer-eye-antoni-debate.html}{wrote}
that Mr. Porowski prepares ``food a child would make when they're old
enough not to need a sitter. Any queer loves a grilled cheese, but it's
not a revelation to cut it into four triangular pieces.''

Mr. Porowski, 33, said that he tries his best to ignore the noise, but
admitted that it hurts to hear, not least because he knows that an act
as elementary as cutting open a fruit can be eye-opening for some
people.

``Tom Jackson never saw the inside of an avocado before,'' he said. ``We
had all these other components that we made for that food demo that I
wanted to show him how to make if we had a chance to. But when I cut
open that avocado, he looked over in this childlike wonder and was
actually amazed.''

He lamented that negativity was a natural response to a program that
elevates ordinary people to ``hero'' status. ``I think that's why the
show is doing so well,'' he said, noting that ``Queer Eye'' is available
to stream in 190 countries. ``It's about kindness, and we haven't seen
that in a really long time.''

Mr. Porowski was born in Montreal after his parents emigrated from
Warsaw with his two older sisters. They spoke Polish at home, and he
learned English and French simultaneously from television and in school.
At any given time, the family had two miniature dachshunds, including
one his grandmother smuggled into Canada from Poland ``back when you
were able to.''

Mr. Porowski is also a dog lover and has set his sights on a corgi. ``As
Joey pointed out, in this small of an apartment, we can't have an extra
beating heart,'' he said. So while he waits to upgrade to a bigger
space, he keeps corgi magnets on his fridge that display the aphorisms
``I rock the belly flop'' and ``Life is short, so are my legs.''

After studying psychology at Concordia University, Mr. Porowski moved to
New York to pursue acting. He took food service jobs to make rent, and
eventually worked his way up to management at the sushi restaurant
BondSt. All the while he was auditioning for acting gigs with
\href{http://www.imdb.com/name/nm3307410/}{limited success}.

``It's usually like, `I'm Antoni Porowski, 5'11¾.'' Here's Polish
Terrorist No. 2,''' he said. ``And then you scream when you leave the
room, and it's done. And you never hear back.''

His luck changed at a book signing for Ted Allen, who was the original
``Queer Eye'' food and wine expert on the Bravo series. The two became
fast friends, and Mr. Allen hired Mr. Porowski as a personal assistant
and became his mentor. A little over a year later, when one of Mr.
Porowski's friends who works at Untitled Entertainment alerted him to
the Netflix reboot, Mr. Allen was the first person Mr. Porowski called
for advice.

``He's like, `Antoni, do you really want to do this?''' Mr. Porowski
recalled. ``I was like, `I don't know, but I think that I should try.'''
Mr. Allen put in a call to the show's co-creator David Collins, and
after a grueling series of auditions and chemistry reads, Mr. Porowski
was welcomed into the new Fab Five.

Image

Mr. Porowski made turkey meatballs using his own recipe, which includes
fennel, red pepper, Parmesan and honey.Credit...Karsten Moran for The
New York Times

Though the show required him to relocate temporarily to Atlanta, Mr.
Porowski lives with his boyfriend of seven years in Brooklyn. The walls
of their Clinton Hill studio are covered with portraits of three patron
saints of Americana: Bob Dylan, James Dean and Lana Del Rey. He and Mr.
Krietemeyer, who works for the online marketplace
\href{https://www.1stdibs.com/}{1stdibs}, collect vintage modern
furniture. Their first big purchase was a white marble Eero Saarinen
tulip table, which is wedged between two birds of paradise and a banana
tree.

Mr. Porowski also telegraphs his taste through novelty shirts, including
concert tees from the National and the Strokes, and three
Helvetica-ampersand T-shirts that list the names of the main characters
in ``A Little Life,'' the relentlessly sad homosocial novel by Hanya
Yanagihara, the editor of T: The New York Times Style Magazine. Talking
about the book's protagonist, whose dark history is revealed in doses,
Mr. Porowski was reminded of one of his favorite figures in psychology,
Viktor Frankl, who theorized that all people must have a ``will to
meaning,'' or a motivating reason to live.

``With Jude, with all of his circumstances, I think he felt like he
didn't have a reason to continue living,'' Mr. Porowski said of Ms.
Yanagihara's main character. ``He really went through some really
terrible stuff. And to have people like Willem and his friends just show
up,'' he said, fighting back tears, ``just to choose to love him. I will
forever be touched by anybody who has a choice to love someone, and they
make a decision to.''

The story has particular resonance for Mr. Porowski, who describes
himself as ``very boundaried'' but has been forced to contend with a
spotlight on his sexuality since joining the cast.

``For the most part, it was never assumed that I was gay, and I've had
people be sort of surprised that I was gay or act apologetic like they
didn't know, which would just make me really uncomfortable,'' he said.
``And I never had shame for it, but I never felt like introducing myself
as, `I'm Antoni, I'm gay. How are you?'''

Mr. Porowski eschews labels generally and prefers to think of himself as
a student, a human and, if he must, a self-taught home cook.

The pasta was good, by the way.

Advertisement

\protect\hyperlink{after-bottom}{Continue reading the main story}

\hypertarget{site-index}{%
\subsection{Site Index}\label{site-index}}

\hypertarget{site-information-navigation}{%
\subsection{Site Information
Navigation}\label{site-information-navigation}}

\begin{itemize}
\tightlist
\item
  \href{https://help.nytimes3xbfgragh.onion/hc/en-us/articles/115014792127-Copyright-notice}{©~2020~The
  New York Times Company}
\end{itemize}

\begin{itemize}
\tightlist
\item
  \href{https://www.nytco.com/}{NYTCo}
\item
  \href{https://help.nytimes3xbfgragh.onion/hc/en-us/articles/115015385887-Contact-Us}{Contact
  Us}
\item
  \href{https://www.nytco.com/careers/}{Work with us}
\item
  \href{https://nytmediakit.com/}{Advertise}
\item
  \href{http://www.tbrandstudio.com/}{T Brand Studio}
\item
  \href{https://www.nytimes3xbfgragh.onion/privacy/cookie-policy\#how-do-i-manage-trackers}{Your
  Ad Choices}
\item
  \href{https://www.nytimes3xbfgragh.onion/privacy}{Privacy}
\item
  \href{https://help.nytimes3xbfgragh.onion/hc/en-us/articles/115014893428-Terms-of-service}{Terms
  of Service}
\item
  \href{https://help.nytimes3xbfgragh.onion/hc/en-us/articles/115014893968-Terms-of-sale}{Terms
  of Sale}
\item
  \href{https://spiderbites.nytimes3xbfgragh.onion}{Site Map}
\item
  \href{https://help.nytimes3xbfgragh.onion/hc/en-us}{Help}
\item
  \href{https://www.nytimes3xbfgragh.onion/subscription?campaignId=37WXW}{Subscriptions}
\end{itemize}
