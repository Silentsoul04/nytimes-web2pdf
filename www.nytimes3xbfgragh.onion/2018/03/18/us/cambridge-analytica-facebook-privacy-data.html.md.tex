Sections

SEARCH

\protect\hyperlink{site-content}{Skip to
content}\protect\hyperlink{site-index}{Skip to site index}

\href{https://www.nytimes3xbfgragh.onion/section/us}{U.S.}

\href{https://myaccount.nytimes3xbfgragh.onion/auth/login?response_type=cookie\&client_id=vi}{}

\href{https://www.nytimes3xbfgragh.onion/section/todayspaper}{Today's
Paper}

\href{/section/us}{U.S.}\textbar{}Facebook's Role in Data Misuse Sets
Off Storms on Two Continents

\url{https://nyti.ms/2GGTFUY}

\begin{itemize}
\item
\item
\item
\item
\item
\item
\end{itemize}

Advertisement

\protect\hyperlink{after-top}{Continue reading the main story}

Supported by

\protect\hyperlink{after-sponsor}{Continue reading the main story}

\hypertarget{facebooks-role-in-data-misuse-sets-off-storms-on-two-continents}{%
\section{Facebook's Role in Data Misuse Sets Off Storms on Two
Continents}\label{facebooks-role-in-data-misuse-sets-off-storms-on-two-continents}}

\includegraphics{https://static01.graylady3jvrrxbe.onion/images/2018/03/18/us/19CAMBRIDGE/19CAMBRIDGE-articleLarge.jpg?quality=75\&auto=webp\&disable=upscale}

By \href{http://www.nytimes3xbfgragh.onion/by/matthew-rosenberg}{Matthew
Rosenberg} and
\href{https://www.nytimes3xbfgragh.onion/by/sheera-frenkel}{Sheera
Frenkel}

\begin{itemize}
\item
  March 18, 2018
\item
  \begin{itemize}
  \item
  \item
  \item
  \item
  \item
  \item
  \end{itemize}
\end{itemize}

WASHINGTON --- Facebook on Sunday faced a backlash about how it protects
user data, as American and British lawmakers demanded that it explain
how a political data firm with links to President Trump's 2016 campaign
was able to harvest private information from more than 50 million
Facebook profiles without the social network's alerting users.

Senator Amy Klobuchar of Minnesota, a Democratic member of the Senate
Judiciary Committee, went so far as to press for Mark Zuckerberg,
Facebook's chief executive, to appear before the panel to explain what
the social network knew about the misuse of its data ``to target
political advertising and manipulate voters.''

The calls for greater scrutiny followed
\href{https://www.nytimes3xbfgragh.onion/2018/03/17/us/politics/cambridge-analytica-trump-campaign.html?hp\&action=click\&pgtype=Homepage\&clickSource=story-heading\&module=first-column-region\&region=top-news\&WT.nav=top-news}{reports}
on Saturday in The New York Times and
\href{https://www.theguardian.com/news/2018/mar/17/cambridge-analytica-facebook-influence-us-election}{The
Observer}of London that Cambridge Analytica, a political data firm
founded by Stephen K. Bannon and Robert Mercer, the wealthy Republican
donor, had used the Facebook data to develop methods that it claimed
could identify the personalities of individual American voters and
influence their behavior. The firm's so-called psychographic modeling
underpinned its work for the Trump campaign in 2016, though many have
questioned the effectiveness of its techniques.

But Facebook did not inform users whose data had been harvested. The
lack of disclosure could violate laws in Britain and in many American
states.

Damian Collins, a Conservative lawmaker in Britain who is leading a
parliamentary inquiry into fake news and Russian meddling in the
country's referendum to leave the European Union, said this weekend that
he, too, would call on Mr. Zuckerberg or another top executive to
testify. The social network sent executives who handle policy matters to
answer questions
\href{http://data.parliament.uk/writtenevidence/committeeevidence.svc/evidencedocument/digital-culture-media-and-sport-committee/fake-news/oral/78195.html}{at
an earlier hearing} in February.

``It is not acceptable that they have previously sent witnesses who seek
to avoid answering difficult questions by claiming not to know the
answers,'' Mr. Collins said in a statement. ``This also creates a false
reassurance that Facebook's stated policies are always robust and
effectively policed.''

The fallout from the reports added to questions Facebook was already
confronting over the use of its platform by those seeking to spread
Russian propaganda and fake news. The social media giant has grappled
with the criticism over the issue for much of the past year, and
struggled to keep public opinion on its side.

Over the weekend, Facebook was on the defensive. Top executives took to
Twitter to argue that the company's protections had not been breached,
and that Facebook was thus not at fault.

``This was unequivocally not a data breach,''
\href{https://twitter.com/boztank/status/975018461997887494}{tweeted}
Andrew Bosworth, a Facebook executive. ``No systems were infiltrated, no
passwords or information were stolen or hacked.''

The data was obtained in 2014, when Cambridge Analytica, through an
outside researcher, paid users small sums to take a personality quiz and
download an app, which would scrape some private information from their
profiles and from those of their friends --- activity that Facebook
permitted at the time. The approach was based on a technique pioneered
at Cambridge University by data scientists who claimed it could reveal
more about a person than even their parents or romantic partners knew.

The researcher hired by Cambridge Analytica, Alexandr Kogan, told
Facebook and his app's users that he was collecting information for
academic purposes, not for a political data firm owned by a wealthy
conservative. Facebook did nothing to verify how the information was
being used.

Mr. Bosworth argued on Twitter that a violation had been committed only
by Cambridge Analytica and Mr. Kogan, whose app ``did not follow the
data agreements.''

Facebook's chief security officer, Alex Stamos, issued a similar defense
in a
\href{https://twitter.com/aprilaser/status/975078309930311680}{series of
tweets} that have since been deleted.

``The recent Cambridge Analytica stories by the NY Times and The
Guardian are important and powerful, but it is incorrect to call this a
`breach' under any reasonable definition of the term,'' Mr. Stamos
tweeted.

The explanation did little, however, to stem the tide of anger as
independent researchers pointed out that many others could have
similarly misused Facebook data.

``Facebook's platform must protect us from predatory behavior,''
\href{https://twitter.com/evanbaily/status/975020550035779590}{wrote} a
Twitter user named Evan Baily, ``or we can't and shouldn't trust the
platform.''

\includegraphics{https://static01.graylady3jvrrxbe.onion/images/2018/03/19/us/19CAMBRIDGE2/19CAMBRIDGE2-articleLarge.jpg?quality=75\&auto=webp\&disable=upscale}

Jonathan Albright, research director at the Tow Center for Digital
Journalism at Columbia University, wrote that the lack oversight and
transparency into what sort of data Facebook collected on its users
meant that the company's platform could continue to be exploited.

``Unethical people will always do bad things when we make it easy for
them and there are few --- if any --- lasting repercussions,'' Mr.
Albright said.

Paul Grewal, a vice president and deputy general counsel at Facebook,
said in a statement that the company was looking into whether the data
in question still existed. ``That is where our focus lies as we remain
committed to vigorously enforcing our policies to protect people's
information,'' he said.

This month, The Times viewed a set of raw data from the profiles
Cambridge Analytica obtained. And a former employee of the data firm
described having recently seen hundreds of gigabytes of unencrypted data
files on Cambridge servers.

There were also questions from technology experts and others about
Facebook's reaction to the news reports by The Times and The Observer,
especially its decision to suspend the account of Christopher Wylie, a
data expert who oversaw Cambridge Analytica's data harvesting --- but
also spoke out about it to the two news organizations.

On Friday, Facebook threatened to sue The Observer to stop it from
publishing, the newspaper's outgoing editor, John Mulholland, said on
Twitter.

Then, late Friday evening, Facebook posted a statement that expressed
alarm at the data leak. The company promised to take action and
announced that it was suspending the accounts of Cambridge Analytica,
Mr. Kogan and Mr. Wylie.

By then, Facebook had learned that Mr. Wylie, who left Cambridge
Analytica in 2014, was a named source for the news reports.

In a statement on Sunday, Mr. Wylie described himself as ``a curious and
naïve 23-year-old,'' when he first went to work for Cambridge Analytica.

``I feel a sense of regret every day when I see where they have helped
take our world,'' he added. ``I need to make amends, and that's why I'm
coming forward.''

His lawyer, Tamsin Allen, said that last week Mr. Wylie offered to help
Facebook recover the missing data.

Now, though, Facebook said on Sunday, Mr. Wylie is refusing to cooperate
with the company until the suspension is lifted --- a move the social
network is not willing to make because of his role in the data
harvesting.

In both Britain and the United States, lawmakers said that in the light
of the new reports, they wanted fresh answers from both Facebook and
Cambridge Analytica about how the data was obtained and what was done
with it.

Mr. Collins, the British lawmaker, said he planned to call Alexander
Nix, the chief executive of Cambridge Analytica, to return to Parliament
and answer questions about testimony last month in which he claimed that
the company never obtained or used Facebook data.

``It seems clear that he has deliberately misled the committee and
Parliament,'' Mr. Collins said.

In the United States, the attorney general of Massachusetts, Maura
Healey, announced on Saturday that her office was opening an
investigation. ``Massachusetts residents deserve answers immediately
from Facebook and Cambridge Analytica,'' she said in
\href{https://twitter.com/MassAGO/status/975052674818347013}{a Twitter
post} that linked to the Times article.

Also on Saturday, the two top Congressional Democrats leading inquiries
into Russian interference in the 2016 election --- Senator Mark Warner
of Virginia and Representative Adam Schiff of California --- called for
investigations of the Facebook data leak.

``This raises serious questions about the level of detail that Cambridge
Analytica knew about users,'' said Mr. Schiff, who is the ranking
Democrat on the House intelligence committee.

Advertisement

\protect\hyperlink{after-bottom}{Continue reading the main story}

\hypertarget{site-index}{%
\subsection{Site Index}\label{site-index}}

\hypertarget{site-information-navigation}{%
\subsection{Site Information
Navigation}\label{site-information-navigation}}

\begin{itemize}
\tightlist
\item
  \href{https://help.nytimes3xbfgragh.onion/hc/en-us/articles/115014792127-Copyright-notice}{©~2020~The
  New York Times Company}
\end{itemize}

\begin{itemize}
\tightlist
\item
  \href{https://www.nytco.com/}{NYTCo}
\item
  \href{https://help.nytimes3xbfgragh.onion/hc/en-us/articles/115015385887-Contact-Us}{Contact
  Us}
\item
  \href{https://www.nytco.com/careers/}{Work with us}
\item
  \href{https://nytmediakit.com/}{Advertise}
\item
  \href{http://www.tbrandstudio.com/}{T Brand Studio}
\item
  \href{https://www.nytimes3xbfgragh.onion/privacy/cookie-policy\#how-do-i-manage-trackers}{Your
  Ad Choices}
\item
  \href{https://www.nytimes3xbfgragh.onion/privacy}{Privacy}
\item
  \href{https://help.nytimes3xbfgragh.onion/hc/en-us/articles/115014893428-Terms-of-service}{Terms
  of Service}
\item
  \href{https://help.nytimes3xbfgragh.onion/hc/en-us/articles/115014893968-Terms-of-sale}{Terms
  of Sale}
\item
  \href{https://spiderbites.nytimes3xbfgragh.onion}{Site Map}
\item
  \href{https://help.nytimes3xbfgragh.onion/hc/en-us}{Help}
\item
  \href{https://www.nytimes3xbfgragh.onion/subscription?campaignId=37WXW}{Subscriptions}
\end{itemize}
