Sections

SEARCH

\protect\hyperlink{site-content}{Skip to
content}\protect\hyperlink{site-index}{Skip to site index}

\href{https://myaccount.nytimes3xbfgragh.onion/auth/login?response_type=cookie\&client_id=vi}{}

\href{https://www.nytimes3xbfgragh.onion/section/todayspaper}{Today's
Paper}

A Simple Lemon Tart With Sensuous Surprises

\url{https://nyti.ms/2GMwfxu}

\begin{itemize}
\item
\item
\item
\item
\item
\end{itemize}

Advertisement

\protect\hyperlink{after-top}{Continue reading the main story}

Supported by

\protect\hyperlink{after-sponsor}{Continue reading the main story}

\href{/column/on-dessert}{On Dessert}

\hypertarget{a-simple-lemon-tart-with-sensuous-surprises}{%
\section{A Simple Lemon Tart With Sensuous
Surprises}\label{a-simple-lemon-tart-with-sensuous-surprises}}

\includegraphics{https://static01.graylady3jvrrxbe.onion/images/2018/03/25/magazine/25mag-ondessert/25mag-ondessert-articleLarge.jpg?quality=75\&auto=webp\&disable=upscale}

By Dorie Greenspan

\begin{itemize}
\item
  March 20, 2018
\item
  \begin{itemize}
  \item
  \item
  \item
  \item
  \item
  \end{itemize}
\end{itemize}

Sometime during the first year that I was married, when I was teaching
myself to cook, I enrolled in a cake-decorating class. A course on how
not to turn steaks into hockey pucks might have been more useful at that
stage of my life, but I was already in thrall to baking. I bought myself
canvas pastry bags marked ``Made in France'' and outfitted them with a
wardrobe of accessories. Each week, I would take my place in front of a
ruled board, just as I had when I was learning cursive, and spend two
hours squeezing Crisco (cheaper and more forgiving than frosting) into
shapes that always fell short of botanical verisimilitude.

Eventually I developed a repertoire of pansies, ruffled leaves, shells,
swags and roses, never perfect, but nonetheless a source of pride. An
ambitious student, I also set myself the task of learning to construct
the most complicated desserts in the French canon. I worked my way
through the Parisian chef Gaston Lenôtre's first English-language
cookbook, and on any old Tuesday night, I'd serve a gâteau St. Honoré
--- puff pastry, cream puffs, two kinds of pastry cream and caramel,
some spun. I was learning techniques that I would continue to build on,
and I kept at it for a decade or so. Then came the great awakening: I
caught myself soaking a dainty cake with rum syrup, layering it with
coffee buttercream, frosting it, finishing it with piped rosettes ...
and craving a shortbread cookie. I was creating beautiful desserts, but
the more elaborate they were, the less I enjoyed eating them.

My tastes were changing. What I wanted was something trickier to achieve
than a bouquet of sugar-paste flowers: simplicity. Like a minimalist
architect, I began removing the superfluous and judging what I
constructed on its bedrock elements. I always knew that dessert's only
purpose was pleasure. Now I wanted that pleasure to come from the
ingredients and how I worked with them, the juxtapositions of texture,
the combinations of flavors, the aromas that float into the kitchen when
the oven door opens and vanish an instant later.

I haven't entirely given up on the intricate confections of my early
years. I still go over the top on birthday cakes --- it's my frosting
loophole. And I often decorate my desserts, but I try to make sure that
what's outside is never more important than what's within. The desserts
I've come to value are unassuming, yet they slyly call you to attention.
The texture of a well-made poundcake can pull you in like that. Cookies
with intermittent crunch, ice cream with a hard-to-place spice,
chocolate cakes with salt, classics resized or reshaped. They all have
this power, but the exemplar might be the lemon tart that I've been
baking for about 20 years.

Like the best simple desserts, this one is plain, built on unremarkable
ingredients but reliant on each one of them. Nothing about its looks
dazzles: It's a crust and a pale-colored filling that might or might not
have a few cracks on its surface. To me, there's beauty in its
homeliness.

The filling's texture is enigmatic. It has the qualities that make spoon
desserts so seductive: It cuts like pudding, clings tentatively to the
knife like jam and gently shimmies like custard. It's slow to melt in
your mouth, a bit of unexpected sensuousness. And its flavor is bold.
It's exuberantly, sharply, unequivocally, edgily lemony, capturing the
essence of the fruit because it uses every bit of it, peel, pith and
pulp --- only the seeds are left behind. It's the surprise in this
simple dessert. Alone, this mixture of lemon tempered by butter and
sugar is good but incomplete. The crust is as vital to the recipe as the
filling. It adds flavor and texture, balance and contrast; it rounds and
finishes the dessert.

I like to use a sweet dough (almost a cookie dough), and I like to roll
it a bit thicker than usual. A thin crust does a fine job of containing
the filling, but a more substantial crust adds a definitive texture,
creating a lively back-and-forth between its own snap and crumble and
the filling's creaminess. And if it's baked well, to a deep color, then
the crust will also have distinctive flavor --- nutlike from the butter,
which browns as it bakes, and caramel from the sugar. It will have a
personality strong enough to be the filling's true partner. And the tart
will be perfectly balanced. My younger self would have tipped the scale
and piped hillocks of cream over the top; my current self appreciates
the impulse, prefers a dusting of powdered sugar and bows to the wisdom
of less being immeasurably more.

\textbf{Recipe:}
\href{https://cooking.nytimes3xbfgragh.onion/recipes/1019239-whole-lemon-tart}{Whole
Lemon Tart} \textbar{}
\href{https://cooking.nytimes3xbfgragh.onion/recipes/1019240-sweet-tart-crust}{Sweet
Tart Crust}

Advertisement

\protect\hyperlink{after-bottom}{Continue reading the main story}

\hypertarget{site-index}{%
\subsection{Site Index}\label{site-index}}

\hypertarget{site-information-navigation}{%
\subsection{Site Information
Navigation}\label{site-information-navigation}}

\begin{itemize}
\tightlist
\item
  \href{https://help.nytimes3xbfgragh.onion/hc/en-us/articles/115014792127-Copyright-notice}{©~2020~The
  New York Times Company}
\end{itemize}

\begin{itemize}
\tightlist
\item
  \href{https://www.nytco.com/}{NYTCo}
\item
  \href{https://help.nytimes3xbfgragh.onion/hc/en-us/articles/115015385887-Contact-Us}{Contact
  Us}
\item
  \href{https://www.nytco.com/careers/}{Work with us}
\item
  \href{https://nytmediakit.com/}{Advertise}
\item
  \href{http://www.tbrandstudio.com/}{T Brand Studio}
\item
  \href{https://www.nytimes3xbfgragh.onion/privacy/cookie-policy\#how-do-i-manage-trackers}{Your
  Ad Choices}
\item
  \href{https://www.nytimes3xbfgragh.onion/privacy}{Privacy}
\item
  \href{https://help.nytimes3xbfgragh.onion/hc/en-us/articles/115014893428-Terms-of-service}{Terms
  of Service}
\item
  \href{https://help.nytimes3xbfgragh.onion/hc/en-us/articles/115014893968-Terms-of-sale}{Terms
  of Sale}
\item
  \href{https://spiderbites.nytimes3xbfgragh.onion}{Site Map}
\item
  \href{https://help.nytimes3xbfgragh.onion/hc/en-us}{Help}
\item
  \href{https://www.nytimes3xbfgragh.onion/subscription?campaignId=37WXW}{Subscriptions}
\end{itemize}
