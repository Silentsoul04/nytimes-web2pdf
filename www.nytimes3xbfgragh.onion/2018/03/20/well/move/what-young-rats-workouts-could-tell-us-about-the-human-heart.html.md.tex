Sections

SEARCH

\protect\hyperlink{site-content}{Skip to
content}\protect\hyperlink{site-index}{Skip to site index}

\href{https://www.nytimes3xbfgragh.onion/section/well/move}{Move}

\href{https://myaccount.nytimes3xbfgragh.onion/auth/login?response_type=cookie\&client_id=vi}{}

\href{https://www.nytimes3xbfgragh.onion/section/todayspaper}{Today's
Paper}

\href{/section/well/move}{Move}\textbar{}What Young Rats' Workouts Could
Tell Us About the Human Heart

\url{https://nyti.ms/2u4rMUf}

\begin{itemize}
\item
\item
\item
\item
\item
\item
\end{itemize}

Advertisement

\protect\hyperlink{after-top}{Continue reading the main story}

Supported by

\protect\hyperlink{after-sponsor}{Continue reading the main story}

Well

\hypertarget{what-young-rats-workouts-could-tell-us-about-the-human-heart}{%
\section{What Young Rats' Workouts Could Tell Us About the Human
Heart}\label{what-young-rats-workouts-could-tell-us-about-the-human-heart}}

\includegraphics{https://static01.graylady3jvrrxbe.onion/images/2018/03/25/magazine/25mag-well/25mag-25well-t_CA0-articleLarge.jpg?quality=75\&auto=webp\&disable=upscale}

By
\href{https://www.nytimes3xbfgragh.onion/by/gretchen-reynolds}{Gretchen
Reynolds}

\begin{itemize}
\item
  March 20, 2018
\item
  \begin{itemize}
  \item
  \item
  \item
  \item
  \item
  \item
  \end{itemize}
\end{itemize}

The human heart of song and story is a changeable, if not fickle, thing.
But in one crucial sense, the organ has been seen as immutable.
Scientists long thought that mammalian hearts stop producing most of
their new cells shortly after birth, and that when they grow bigger,
they do so primarily because the size of their existing cells increases.
A
\href{https://physoc.onlinelibrary.wiley.com/doi/abs/10.1113/JP275339}{recent
study in The Journal of Physiology}, however, confirms that exercise can
substantially increase the number of cells in the hearts of young lab
rats --- and it also indicates, for the first time, that these
additional cells are still present in mature hearts.

Researchers in Australia took young, healthy male rats and kept some of
them sedentary while making others run on treadmills. This active cohort
was divided into three groups, each of which began exercising at a
different stage of life. In human terms, these starting points
corresponded to childhood, adolescence and adulthood. After the rats
were put through a month of daily hourlong workouts at a moderate pace,
the hearts of some of them were examined microscopically. Exercise was
then curtailed for the rest of the rats, which spent the next several
months (roughly equivalent to 10 human years) inactive. Once they
reached full adulthood, their hearts were also scrutinized.

The treadmillers had bigger hearts than their inactive counterparts, a
finding expected by the researchers. Exercise makes hearts larger, more
efficient and healthier. But the pathways to this change differed
greatly depending on the rats' ages when they started running. Those
that began exercising as adults had bigger cardiac-muscle cells but not
more of them. The hearts of the childhood runners, though, teemed with
about 20 million additional cardiomyocytes --- the type of heart cell
that contracts --- compared with the hearts of the sedentary rats of the
same age. The adolescent runners also showed small increases in the
number of cardiomyocytes, but many fewer than the youngest runners
gained. Perhaps most interesting, the extra heart cells found in both
young and adolescent rats remained in their hearts after they reached
adulthood, despite the cessation of exercise.

The study involved rodents, not people, so the findings shouldn't be
overhyped, cautions Glenn Wadley, an associate professor in the
Institute for Physical Activity and Nutrition at Deakin University, who
conducted the experiment with a graduate student, Yasmin Asif, and
colleagues from the University of Melbourne and Monash University. But
even so, the results do hint at human implications. A heart attack kills
tens of millions of cells in an instant; a reserve of cardiac-muscle
cells built up during childhood could conceivably affect the likelihood
of surviving an attack as an adult, as well as influence how well a
heart functions over the course of a lifetime. ``Exercise is beneficial
for the heart at any age,'' Wadley says. His study provides ``a new
reason'' for people to be active when their hearts are still young and
unburdened.

Advertisement

\protect\hyperlink{after-bottom}{Continue reading the main story}

\hypertarget{site-index}{%
\subsection{Site Index}\label{site-index}}

\hypertarget{site-information-navigation}{%
\subsection{Site Information
Navigation}\label{site-information-navigation}}

\begin{itemize}
\tightlist
\item
  \href{https://help.nytimes3xbfgragh.onion/hc/en-us/articles/115014792127-Copyright-notice}{©~2020~The
  New York Times Company}
\end{itemize}

\begin{itemize}
\tightlist
\item
  \href{https://www.nytco.com/}{NYTCo}
\item
  \href{https://help.nytimes3xbfgragh.onion/hc/en-us/articles/115015385887-Contact-Us}{Contact
  Us}
\item
  \href{https://www.nytco.com/careers/}{Work with us}
\item
  \href{https://nytmediakit.com/}{Advertise}
\item
  \href{http://www.tbrandstudio.com/}{T Brand Studio}
\item
  \href{https://www.nytimes3xbfgragh.onion/privacy/cookie-policy\#how-do-i-manage-trackers}{Your
  Ad Choices}
\item
  \href{https://www.nytimes3xbfgragh.onion/privacy}{Privacy}
\item
  \href{https://help.nytimes3xbfgragh.onion/hc/en-us/articles/115014893428-Terms-of-service}{Terms
  of Service}
\item
  \href{https://help.nytimes3xbfgragh.onion/hc/en-us/articles/115014893968-Terms-of-sale}{Terms
  of Sale}
\item
  \href{https://spiderbites.nytimes3xbfgragh.onion}{Site Map}
\item
  \href{https://help.nytimes3xbfgragh.onion/hc/en-us}{Help}
\item
  \href{https://www.nytimes3xbfgragh.onion/subscription?campaignId=37WXW}{Subscriptions}
\end{itemize}
