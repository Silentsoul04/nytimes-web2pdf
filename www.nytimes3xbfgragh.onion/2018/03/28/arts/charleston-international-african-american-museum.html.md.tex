Sections

SEARCH

\protect\hyperlink{site-content}{Skip to
content}\protect\hyperlink{site-index}{Skip to site index}

\href{https://www.nytimes3xbfgragh.onion/section/arts/design}{Art \&
Design}

\href{https://myaccount.nytimes3xbfgragh.onion/auth/login?response_type=cookie\&client_id=vi}{}

\href{https://www.nytimes3xbfgragh.onion/section/todayspaper}{Today's
Paper}

\href{/section/arts/design}{Art \& Design}\textbar{}Charleston Needs
That African American Museum. And Now.

\url{https://nyti.ms/2utb8xP}

\begin{itemize}
\item
\item
\item
\item
\item
\end{itemize}

Advertisement

\protect\hyperlink{after-top}{Continue reading the main story}

Supported by

\protect\hyperlink{after-sponsor}{Continue reading the main story}

Critic's Notebook

\hypertarget{charleston-needs-that-african-american-museum-and-now}{%
\section{Charleston Needs That African American Museum. And
Now.}\label{charleston-needs-that-african-american-museum-and-now}}

\includegraphics{https://static01.graylady3jvrrxbe.onion/images/2018/03/29/arts/29charleston-museum2/merlin_136057002_3bb58dbe-1456-4922-a03b-3c45424e6e8a-articleLarge.jpg?quality=75\&auto=webp\&disable=upscale}

By \href{http://www.nytimes3xbfgragh.onion/by/michael-kimmelman}{Michael
Kimmelman}

\begin{itemize}
\item
  March 28, 2018
\item
  \begin{itemize}
  \item
  \item
  \item
  \item
  \item
  \end{itemize}
\end{itemize}

CHARLESTON, S.C. --- The unmarked property, beside a big, bland postwar
apartment building, is now an empty grass lot and de facto park. Cabin
cruisers gently bob at a pier.

In this part of Charleston, just north of the historic, postcard
district, industry has increasingly been giving way to boxy condominium
developments with names like The Gadsden, after this city's
Revolutionary War-era patriot, merchant, and sometime slave trader,
\href{http://teachingushistory.org/lessons/Gadsden.htm}{Christopher
Gadsden}.

\includegraphics{https://static01.graylady3jvrrxbe.onion/images/2018/03/29/arts/29charleston-museum1/merlin_136081758_12ceb18a-2f3c-45be-884a-545d664b5d8b-articleLarge.jpg?quality=75\&auto=webp\&disable=upscale}

Justice delayed, as the saying goes.

If it's to be served, that empty plot, still waiting on private
donations and \$11 million in state funding, will be occupied by a
subdued, modernist, 47,000-square-foot pavilion raised above the ground
on thick columns clad in precast oyster-shell tabby.

It will house the \href{https://iaamuseum.org/}{International African
American Museum}.

A graceful project, long discussed and years overdue, the museum has
brought together two very different talents, the veteran architect Harry
Cobb, from Pei Cobb Freed \& Partners, and Walter Hood, the landscape
designer from Oakland, Calif.

Its louvered windows facing the waterfront will direct views past Fort
Sumter toward the Atlantic Ocean --- and Africa. In and around the plaza
created below the lofted building, a memorial garden, planted with
native grasses, will lead toward a shallow tidal pool whose stone floor
is inscribed with the shapes of bodies crammed together, as slaves were,
in the bowels of ships that landed here.

Right here. The spot used to be Gadsden's Wharf. Historians estimate
nearly half of all African slaves brought to America arrived in
Charleston, most of them at Gadsden's Wharf. At 840-feet-long, it was,
two centuries ago, the largest wharf in America. Thousands of Africans
waited in the wharf's warehouses to be auctioned off.

Image

The city of Charleston, S.C., looking across Cooper's River around
1838.Credit...Chronicle

In what has become a parking lot, just inland, 700 of them froze to
death.

For millions of African-Americans today, the site is ``ground zero,'' as
the Harvard scholar Henry Louis Gates, Jr., has put it, for ``blackness,
black culture, the African experience, the African-American experience,
slavery --- however you want to slice it.''

Every era erects, removes, amends --- or ignores --- monuments.
Monuments and historical museums are always mirrors, advertisements,
time bombs. Hardly a street or building in Germany today lacks some sign
or plaque, redressing the past. It was the proposed removal of a Jim
Crow-era statue of Robert E. Lee that became the excuse for the neo-Nazi
rally in Charlottesville, Va., last year, where a white nationalist is
to go on trial late this year in the murder of a protester at the event.

Unlike Virginia, South Carolina hasn't taken down Confederate monuments.
Much has changed here but much has not. The state's most
\href{https://www.postandcourier.com/news/proposed-s-c-social-studies-standards-don-t-mention-mlk/article_8492b92c-075d-11e8-bfaa-0fa934b57487.html}{recent
proposal} for social studies standards in public schools doesn't mention
the Rev. Dr. Martin Luther King Jr., or Rosa Parks.

The Emanuel African Methodist Episcopal Church in Charleston where a
young white man massacred nine black congregants in 2015 is virtually in
the shadow of what's still the city's tallest monument (another Jim Crow
relic) of the antebellum vice president and proud white supremacist John
C. Calhoun.

It has been nearly two decades since Joseph P. Riley Jr., Charleston's
mayor at the time, floated the idea of a museum of African-American
culture and history, on a different site, nearby. A dozen years passed,
then more.

Mr. Riley retired in 2016, after 40 years in office, having been elected
during the 1970s as a racial bridge builder. White racists called him
``L'il Black Joe'' when he appointed a black police chief in 1975.
Charleston prospered over the intervening decades.

But gentrification had its effects. Two-thirds black in the early 1980s,
the population has become 70 percent white. I suggested to Mr. Riley the
other day that Charleston can come across to a visitor as Disneyland for
the Confederacy, still enthralled by its era of slavery, with a monument
on seemingly every downtown corner commemorating some Confederate
soldier, plantation aristocrat or antebellum judge who opposed Lincoln.

``It's a process,'' he replied. ``We worked hard while I was mayor to
avoid alienation, to make this a city where everyone feels welcome. When
I was in school, they didn't teach us about slavery. I really only
learned the truth about how slaves were treated when I had already been
in office for many years. That's when I began to think seriously about
the museum.''

But without enough money or much public enthusiasm, the plan sputtered.
Then excavations turned up traces of Gadsden's Wharf in the muck beneath
the grassy lot. Through the exhibition designer Ralph Appelbaum, Mr.
Riley reached out to Mr. Cobb.

Pretty much the architect's first question: Why not build on the
location of the wharf?

By that point, the city had sold the property to a local restaurateur,
unaware of its history. Mr. Riley spent a tidy sum buying the land back.

``Sometimes you quick-cook something, it's a mistake,'' rationalized the
former mayor, who has taken to calling the museum his ``most important
work,'' especially after the church murders. ``It turned out to be good
that we had a lengthy germination period.''

Now 91, the soft-spoken Mr. Cobb is known for designing the John Hancock
Tower in Boston,
\href{https://www.nytimes3xbfgragh.onion/2015/11/19/arts/design/7-bryant-park-embraces-its-place-in-the-city.html}{7
Bryant Park in New York}, and a variety of big, sleek buildings in
between, the best of which are geometrically eloquent and deceptively
simple. Working here with the structural engineer Guy Nordenson, he
describes this project as an ``unrhetorical work of architecture.''

But that's not quite true. On the edge of the cobblestoned tourist area,
with its ornate Gothic Revival-style churches and Queen Anne houses, the
museum's plain-spoken modernism comes across as almost whisperingly
defiant, a turning of the page, promising a deliverance from history,
modernism's originating goal.

Image

A rendering showing the ``I AM'' hallway that will present audio
prompted by touch with clips of oral history highlights.Credit...Pei
Cobb Freed

Moody Nolan are the architects of record. Slender brick cladding
underscores the pavilion's long horizontal spans and extended
cantilevers on either end. Pointed columns are meant to make the
structure's mass appear to float. Perching the museum on piers will take
account of rising waters. But it's also hard not to see an allusion to a
wharf.

Inside, galleries will document the many diverse cultures Africans
brought to America, and a family center will let visitors trace their
roots to Gadsden's Wharf.

For his part, Mr. Hood has reimagined a constrained and narrow property,
about a football-field long. The late, great Brazilian architect Oscar
Niemeyer was an inspiration. Mr. Hood creates a shaded public plaza, in
the breezy space underneath the raised structure, where people may
congregate around the building's double-sided staircase, so the museum
can become a gathering spot, not just a pilgrimage site.

The memorial garden and tidal pool, at the same time, insure that it's
recognized as hallowed ground, a place for contemplation.

The budget for building the museum is \$75 million. The goal is for
bulldozers to start digging later this year and for construction to
finish in 2020. But there's a hitch. No shovel will be lifted until all
the money is raised. Charleston has committed its \$25 million share,
along with the land, and private donations are approaching the \$25
million goal.

But the South Carolina legislature, after an understanding that it would
contribute \$25 million over five years, allocated \$14 million, and now
won't promise the remaining \$11 million. The clock is ticking. The
current legislature remains in session only until the end of May.

State Representative Brian White, a Republican who heads South
Carolina's House Ways and Means Committee, is one of those holding the
money back. The museum ``is not a state project and we have a lot of
state needs right now that far outweigh a municipality's request,'' he
recently told
\href{https://www.greenvilleonline.com/story/news/2018/03/08/former-charleston-mayors-museum-mission-cant-rest-until-ive-done-my-duty/340719002/}{the
Greenville News}, citing competing priorities like education.

Bobby Hitt, South Carolina's commerce secretary, by contrast, has
pointed out that the museum will help attract businesses to the state.
It adds a work of architectural dignity. And as for educational value,
plainly it fills a gap.

Image

Robert Smalls, the great-great grandfather of Mr. Moore, in 1904. Smalls
commandeered a Confederate ship, turning it over to Union forces and
winning freedom for himself, his crew and his family. During
Reconstruction, he became a state legislator and congressman.Credit...

``This ain't a black project,'' as Bakari Sellers, a former Democrat in
the state legislature, put it to the Greenville News. ``This ain't a
Charleston project. This is an American project.''

Or as James Baldwin said, ``If you know whence you came, there are
absolutely no limitations to where you can go.''

Image

A rendering of the Tide Tribute, a pool whose stone floor is inscribed
with the shapes of bodies crammed together, as slaves were, in the
bowels of ships that landed here.Credit...Pei Cobb Freed

One recent morning I toured the site with Mr. Hood and Michael Boulware
Moore, the museum's president, then we looked out over the harbor. Mr.
Moore said his ancestors were among the slaves who arrived in shackles
at Gadsden's Wharf.

His great-great grandfather was
\href{https://www.smithsonianmag.com/history/thrilling-tale-how-robert-smalls-heroically-sailed-stolen-confederate-ship-freedom-180963689/}{Robert
Smalls}, who commandeered a Confederate ship, turning it over to Union
forces and winning freedom for himself, his family and his crew. Smalls
became a crusading state legislator and United States congressman during
Reconstruction. He brought free public education to South Carolina.

A plaque honoring Smalls was installed on a squat little pillar downtown
not long ago. Mr. Moore showed me a picture of it.

Image

A memorial to Robert Smalls in Waterfront Park in Charleston,
S.C.Credit...Kate Thornton for The New York Times

Think, the Stonehenge set from ``Spinal Tap.'' The memorial looks tiny,
and is periodically obscured by bushes.

Not far away, a big statue on a huge round pedestal, at the tip of the
battery facing Fort Sumter, honors the Confederate Defenders of
Charleston.

Symbols matter. The past is present. The museum would clearly be good
for more than just business.

Advertisement

\protect\hyperlink{after-bottom}{Continue reading the main story}

\hypertarget{site-index}{%
\subsection{Site Index}\label{site-index}}

\hypertarget{site-information-navigation}{%
\subsection{Site Information
Navigation}\label{site-information-navigation}}

\begin{itemize}
\tightlist
\item
  \href{https://help.nytimes3xbfgragh.onion/hc/en-us/articles/115014792127-Copyright-notice}{©~2020~The
  New York Times Company}
\end{itemize}

\begin{itemize}
\tightlist
\item
  \href{https://www.nytco.com/}{NYTCo}
\item
  \href{https://help.nytimes3xbfgragh.onion/hc/en-us/articles/115015385887-Contact-Us}{Contact
  Us}
\item
  \href{https://www.nytco.com/careers/}{Work with us}
\item
  \href{https://nytmediakit.com/}{Advertise}
\item
  \href{http://www.tbrandstudio.com/}{T Brand Studio}
\item
  \href{https://www.nytimes3xbfgragh.onion/privacy/cookie-policy\#how-do-i-manage-trackers}{Your
  Ad Choices}
\item
  \href{https://www.nytimes3xbfgragh.onion/privacy}{Privacy}
\item
  \href{https://help.nytimes3xbfgragh.onion/hc/en-us/articles/115014893428-Terms-of-service}{Terms
  of Service}
\item
  \href{https://help.nytimes3xbfgragh.onion/hc/en-us/articles/115014893968-Terms-of-sale}{Terms
  of Sale}
\item
  \href{https://spiderbites.nytimes3xbfgragh.onion}{Site Map}
\item
  \href{https://help.nytimes3xbfgragh.onion/hc/en-us}{Help}
\item
  \href{https://www.nytimes3xbfgragh.onion/subscription?campaignId=37WXW}{Subscriptions}
\end{itemize}
