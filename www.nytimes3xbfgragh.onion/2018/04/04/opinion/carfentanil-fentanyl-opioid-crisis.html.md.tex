Sections

SEARCH

\protect\hyperlink{site-content}{Skip to
content}\protect\hyperlink{site-index}{Skip to site index}

\href{https://myaccount.nytimes3xbfgragh.onion/auth/login?response_type=cookie\&client_id=vi}{}

\href{https://www.nytimes3xbfgragh.onion/section/todayspaper}{Today's
Paper}

\href{/section/opinion}{Opinion}\textbar{}Ordering Five Million Deaths
Online

\url{https://nyti.ms/2EkMS0r}

\begin{itemize}
\item
\item
\item
\item
\item
\end{itemize}

Advertisement

\protect\hyperlink{after-top}{Continue reading the main story}

Supported by

\protect\hyperlink{after-sponsor}{Continue reading the main story}

\href{/section/opinion}{Opinion}

\hypertarget{ordering-five-million-deaths-online}{%
\section{Ordering Five Million Deaths
Online}\label{ordering-five-million-deaths-online}}

\href{http://www.nytimes3xbfgragh.onion/column/richard-a-friedman}{\includegraphics{https://static01.graylady3jvrrxbe.onion/images/2015/03/16/opinion/Friedman-Richard-circular/Friedman-Richard-circular-thumbLarge-v4.jpg}}

By
\href{http://www.nytimes3xbfgragh.onion/column/richard-a-friedman}{Richard
A. Friedman}

Dr.~Friedman is a contributing opinion writer and the director of the
psychopharmacology clinic at the Weill Cornell Medical College.

\begin{itemize}
\item
  April 4, 2018
\item
  \begin{itemize}
  \item
  \item
  \item
  \item
  \item
  \end{itemize}
\end{itemize}

\includegraphics{https://static01.graylady3jvrrxbe.onion/images/2018/04/05/opinion/05friedman/05friedman-articleLarge.jpg?quality=75\&auto=webp\&disable=upscale}

Medicare officials have announced plans to
\href{https://www.nytimes3xbfgragh.onion/2018/03/27/health/opioids-medicare-limits.html}{crack
down on prescriptions} for opioids in an attempt to limit their use and
thus their damage. But making it harder for people to get pain
medication legally will most likely drive many to seek relief from far
more dangerous and superpotent synthetic opioids. And they are
surprisingly easy to obtain.

Recently, out of curiosity, I typed into Google the terms ``synthetic
opioid and Chinese pharmacy.'' Within minutes, I found a website where I
could purchase the synthetic opioid carfentanil. For just \$750, I could
buy 100 grams of the drug, which would be shipped to me ``overnight by
discreet courier.''

Carfentanil is 100 times more potent than fentanyl, another synthetic
opioid that has already hit the streets, and 10,000 times more powerful
than morphine. This drug is typically used to sedate large animals, like
elephants. It is so dangerous that when veterinarians administer it,
they wear gloves and face masks to avoid exposure.

\includegraphics{https://static01.graylady3jvrrxbe.onion/images/2018/04/05/opinion/05friedman2/05friedman2-articleLarge.jpg?quality=75\&auto=webp\&disable=upscale}

Fentanyl is scary enough.
\href{http://www.sciencemag.org/news/2017/03/underground-labs-china-are-devising-potent-new-opiates-faster-authorities-can-respond}{A
dose of two milligrams} --- a few grains of the substance --- can be
fatal. But with carfentanil, 0.02 milligrams --- hardly more than a
speck of dust --- could be enough to kill a person. That means that for
\$750, I could in theory purchase enough carfentanil for five million
fatal overdoses.

Americans are by far the
\href{https://www.washingtonpost.com/news/wonk/wp/2017/03/15/americans-use-far-more-opioids-than-anyone-else-in-the-world/?utm_term=.3023fbdd455f}{largest
consumers of the world's natural and synthetic opioids}, and
\href{https://www.cdc.gov/drugoverdose/data/statedeaths.html}{116 of us
are dying every day} as a result. Carfentanil could unleash a wave of
mortality that would dwarf these numbers.

The drug naloxone, which can reverse prescription opioid and heroin
overdoses,
\href{http://www.nejm.org/doi/full/10.1056/NEJMsr1706626\#article_references}{may
not be effective against potent synthetics}. So there may be no way to
stop these new opioids from killing people.

We are just beginning to see the leading edge of the synthetic opioid
epidemic. For example, in Ohio,
\href{http://www.news-herald.com/article/HR/20170207/NEWS/170209584}{fentanyl-related
deaths} surged from 75 in 2012 to 1,155 in 2015. Although synthetic
opioids are relatively easy to make in back-alley labs, a majority of
them are coming illegally into the United States from China, where the
chemical and pharmaceutical industries are poorly regulated.

China is the world's largest exporter of active pharmaceutical
ingredients, with more than 160,000 chemical companies
\href{https://www.uscc.gov/sites/default/files/Research/USCC\%20Staff\%20Report_Fentanyl-China\%E2\%80\%99s\%20Deadly\%20Export\%20to\%20the\%20United\%20States020117.pdf}{operating
legally and illegally}. President Trump, who is not afraid to flex his
muscles, should apply a stiff tariff on all Chinese pharmaceuticals to
encourage the government to crack down on the production of illegal
drugs.

It will be hard, if not impossible, to shut down the supply, but we have
to try.

On the demand side, we can do things to try to prevent addiction. But
the new Medicare guidelines, which cover disabled individuals and those
65 and older, would only make things worse.

They would deny coverage to any patient taking more than 90 milligrams
of morphine or the pharmacologic equivalent daily for more than seven
days, except for those with cancer or in hospice (an exemption from this
guideline is possible but burdensome). The Centers for Medicare and
Medicaid Services estimate that some 1.6 million Medicare beneficiaries
are prescribed opioids that meet or exceed this arbitrary threshold, so
the change, if it goes into effect as planned next January, will
suddenly put a very large number of people at risk of severe opioid
withdrawal and the return of pain and suffering. This could well drive
many of them to synthetic opioids.

Instead, we need a rational drug policy both to rein in the excessive
prescribing of opioids and to help the people who are already dependent
on them.

First, we need a national prescription database. The state-level
databases that we have now are not enough. They allow clinicians to
identify patients who ``doctor shop'' and are high consumers of opioids,
but patients can still fill their prescriptions in nearby states, and no
one is the wiser.

We also have to deal with doctors who contribute to the epidemic. The
Drug Enforcement Administration, using that national prescription
database, should identify clinicians, particularly those who aren't pain
specialists, who are outliers in their opioid prescribing patterns,
review their treatments and clamp down on inappropriate and excessive
prescribing.

This is tricky; we do not want to discourage doctors from adequately
treating pain out of fear of legal sanction. But those who adhere to
current standards of care should have little to fear.

Finally, reasonable drug policy has to take account of the fact that
opioid-dependent individuals have different levels of tolerance, which
means there cannot be a one-size-fits-all guideline, like the Medicare
proposal, to limit prescribing.

To be sure, there is solid evidence that nonopioid treatments are safer
and just as effective as opioids for certain types of chronic pain ---
and it's critical that we improve pain education for all health care
professionals so this becomes common knowledge.

But for those who are dependent on opioids, doctors must have the
ability to adjust treatment to the neurobiological and clinical reality.
The fact is that an opioid-dependent brain requires considerable time to
adapt to any change in treatment.

Any opioid policy that ignores this will not just throw an untold number
of people into withdrawal and misery; it could well unleash a synthetic
opioid epidemic of staggering lethality.

Advertisement

\protect\hyperlink{after-bottom}{Continue reading the main story}

\hypertarget{site-index}{%
\subsection{Site Index}\label{site-index}}

\hypertarget{site-information-navigation}{%
\subsection{Site Information
Navigation}\label{site-information-navigation}}

\begin{itemize}
\tightlist
\item
  \href{https://help.nytimes3xbfgragh.onion/hc/en-us/articles/115014792127-Copyright-notice}{©~2020~The
  New York Times Company}
\end{itemize}

\begin{itemize}
\tightlist
\item
  \href{https://www.nytco.com/}{NYTCo}
\item
  \href{https://help.nytimes3xbfgragh.onion/hc/en-us/articles/115015385887-Contact-Us}{Contact
  Us}
\item
  \href{https://www.nytco.com/careers/}{Work with us}
\item
  \href{https://nytmediakit.com/}{Advertise}
\item
  \href{http://www.tbrandstudio.com/}{T Brand Studio}
\item
  \href{https://www.nytimes3xbfgragh.onion/privacy/cookie-policy\#how-do-i-manage-trackers}{Your
  Ad Choices}
\item
  \href{https://www.nytimes3xbfgragh.onion/privacy}{Privacy}
\item
  \href{https://help.nytimes3xbfgragh.onion/hc/en-us/articles/115014893428-Terms-of-service}{Terms
  of Service}
\item
  \href{https://help.nytimes3xbfgragh.onion/hc/en-us/articles/115014893968-Terms-of-sale}{Terms
  of Sale}
\item
  \href{https://spiderbites.nytimes3xbfgragh.onion}{Site Map}
\item
  \href{https://help.nytimes3xbfgragh.onion/hc/en-us}{Help}
\item
  \href{https://www.nytimes3xbfgragh.onion/subscription?campaignId=37WXW}{Subscriptions}
\end{itemize}
