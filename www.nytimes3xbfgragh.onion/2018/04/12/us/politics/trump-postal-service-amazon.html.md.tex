Sections

SEARCH

\protect\hyperlink{site-content}{Skip to
content}\protect\hyperlink{site-index}{Skip to site index}

\href{https://www.nytimes3xbfgragh.onion/section/politics}{Politics}

\href{https://myaccount.nytimes3xbfgragh.onion/auth/login?response_type=cookie\&client_id=vi}{}

\href{https://www.nytimes3xbfgragh.onion/section/todayspaper}{Today's
Paper}

\href{/section/politics}{Politics}\textbar{}Trump, Having Denounced
Amazon's Shipping Deal, Orders Review of Postal Service

\url{https://nyti.ms/2GVKa7S}

\begin{itemize}
\item
\item
\item
\item
\item
\end{itemize}

Advertisement

\protect\hyperlink{after-top}{Continue reading the main story}

Supported by

\protect\hyperlink{after-sponsor}{Continue reading the main story}

\hypertarget{trump-having-denounced-amazons-shipping-deal-orders-review-of-postal-service}{%
\section{Trump, Having Denounced Amazon's Shipping Deal, Orders Review
of Postal
Service}\label{trump-having-denounced-amazons-shipping-deal-orders-review-of-postal-service}}

\includegraphics{https://static01.graylady3jvrrxbe.onion/images/2018/04/13/us/politics/13dc-postal-sub/merlin_136769085_6df39229-a2b2-4e77-9661-cdb86f1554b1-articleLarge.jpg?quality=75\&auto=webp\&disable=upscale}

By \href{http://www.nytimes3xbfgragh.onion/by/michael-d-shear}{Michael
D. Shear}

\begin{itemize}
\item
  April 12, 2018
\item
  \begin{itemize}
  \item
  \item
  \item
  \item
  \item
  \end{itemize}
\end{itemize}

WASHINGTON --- President Trump abruptly issued an executive order on
Thursday demanding an evaluation of the Postal Service's finances,
asserting the power of his office weeks after accusing Amazon, the
online retail giant, of not paying its fair share in postage.

In the executive order, issued just before 9 p.m., Mr. Trump created a
task force to examine the service's ``unsustainable financial path'' and
directed the new panel to ``conduct a thorough evaluation of the
operations and finances of the U.S.P.S.''

The president does not mention Amazon in the order, but it is clear that
he intends for the panel to substantiate his repeated claim that the
financial arrangement between the Postal Service and Amazon, its biggest
shipper of packages, is a money loser.

In December, Mr. Trump railed against the service
\href{https://twitter.com/realdonaldtrump/status/946728546633953285}{on
Twitter} for being ``dumber and poorer'' by losing billions of dollars
and not ``charging MUCH MORE'' to Amazon and other shippers. His Twitter
attacks date back as far as 2013, when he
\href{https://twitter.com/realdonaldtrump/status/299212012121112578}{scoffed
at the service} for planning to eliminate Saturday
\href{https://about.usps.com/news/national-releases/2013/pr13_019.htm}{mail
delivery} --- ``our poor, poor Country,'' he wrote --- and raising the
cost of stamps.

Postal Service experts and even Mr. Trump's own advisers have privately
urged him to back off the accusations, noting that the huge number of
packages shipped by Amazon is actually helping to keep the Postal
Service financially solvent.

While the service has consistently reported net losses for a decade,
much of its financial woes are the result of a prolonged decline in the
volume of marketing mail and first-class mail. The service makes money
on packages, and Amazon is the service's biggest single shipper of
packages.

But the president has refused to believe those arguments, insisting in a
\href{https://twitter.com/realDonaldTrump/status/980063581592047617}{tweet
as recently as March 31} that ``the U.S. Post Office will lose \$1.50 on
average for each package it delivers for Amazon.''

``That amounts to Billions of Dollars,'' he continued.

Mr. Trump's repeated attacks on Amazon have focused in part on the
company's billionaire owner, Jeff Bezos, who also owns The Washington
Post. People close to Mr. Trump have said the president's tirades
against the retailer often come after The Post has published negative
articles about him.

``The \#AmazonWashingtonPost, sometimes referred to as the guardian of
Amazon not paying internet taxes (which they should) is FAKE NEWS!'' he
wrote
\href{https://twitter.com/realdonaldtrump/status/880049704620494848}{last
June}.

``Is Fake News Washington Post being used as a lobbyist weapon against
Congress to keep Politicians from looking into Amazon no-tax monopoly?''
he wondered
\href{https://twitter.com/realdonaldtrump/status/889675644396867584}{the
next month}.

At one point last month, Amazon's stock price tumbled after Mr. Trump's
tweets and a suggestion that he might direct the government to take
action against the company.

Drew Herdener, an Amazon spokesman, declined to comment late Thursday.

But until Thursday night, Mr. Trump had limited his attacks to ones on
Twitter. The White House had said the president did not have immediate
plans to take any action against Amazon and had given no warning that
the executive order was coming.

In the order, the president calls on the task force to examine various
parts of the Postal Service's business, including ones that appear to
directly involve large parts of Amazon's business.

That includes the ``expansion and pricing'' of the package delivery
market and the service's role in competing with other, private delivery
companies. The task force should look at the decline in mail volume and
the implications for the service, Mr. Trump said.

The order also calls for the task force to look at the service's
``universal service obligation,'' which requires the service to deliver
to everyone in the United States, given changes in technology and
e-commerce.

Some parts of the order appear to hint at further privatization of the
Postal Service, indicating that members of the task force should examine
``the U.S.P.S. role in the U.S. economy and in rural areas, communities,
and small towns.''

In the order, Mr. Trump said that the longstanding financial problems at
the Postal Service demand some kind of action.

``A number of factors, including the steep decline in First-Class Mail
volume, coupled with legal mandates that compel the U.S.P.S. to incur
substantial and inflexible costs, have resulted in a structural
deficit,'' Mr. Trump says in the order. ``The U.S.P.S. is on an
unsustainable financial path and must be restructured to prevent a
taxpayer-funded bailout.''

It is unclear how quickly the task force will be assembled, or when its
review might result in changes at the Postal Service that could directly
affect Amazon and other shippers.

In May, Mr. Trump created a similar commission to examine what he said
was evidence of large-scale voter fraud in the 2016 election --- a claim
that was repeatedly debunked by election experts from both major
political parties.

But he dissolved the commission in January after several legal
challenges and a lack of cooperation from Democratic and Republican
election officials in a number of states.

In the case of the Postal Service order on Thursday night, Mr. Trump
demanded that a report of findings be delivered to him by summer's end.
He urged the task force to recommend administrative and legislative
changes that could be made.

Those recommendations could take into account previous studies,
including a
\href{https://ir.citi.com/XInLvxkr5F\%2FJvyPr1NMl\%2FPcIgrn\%2BXqplW8cqbv2ImZxLKrWAiRT\%2BcFMjQe6C\%2BuQT9n1mvCnznGU\%3D}{2017
analysis by Citigroup}, which concluded that the service was charging
below market rates for package delivery.

The report estimated that if the service increased its parcel rates,
Amazon's shipping costs would rise by about \$2.6 billion. Some of that
money would presumably go to the Postal Service, which in 2014 handled
40 percent of Amazon's packages, according to one
\href{https://www.bloomberg.com/news/articles/2017-12-29/trump-says-u-s-post-office-should-charge-amazon-much-more}{2015
estimate}.

Mr. Trump's task force does not address the president's other complaint
about Amazon: that the online retailer fails to pay sales taxes, giving
it an unfair advantage over brick-and-mortar businesses.

In fact, while Amazon used to benefit from tax-free sales, it now
charges sales tax on the sale of virtually all of its own products. Many
third-party merchants who use the service to sell their own goods on
Amazon's sites still do not charge sales tax.

Advertisement

\protect\hyperlink{after-bottom}{Continue reading the main story}

\hypertarget{site-index}{%
\subsection{Site Index}\label{site-index}}

\hypertarget{site-information-navigation}{%
\subsection{Site Information
Navigation}\label{site-information-navigation}}

\begin{itemize}
\tightlist
\item
  \href{https://help.nytimes3xbfgragh.onion/hc/en-us/articles/115014792127-Copyright-notice}{©~2020~The
  New York Times Company}
\end{itemize}

\begin{itemize}
\tightlist
\item
  \href{https://www.nytco.com/}{NYTCo}
\item
  \href{https://help.nytimes3xbfgragh.onion/hc/en-us/articles/115015385887-Contact-Us}{Contact
  Us}
\item
  \href{https://www.nytco.com/careers/}{Work with us}
\item
  \href{https://nytmediakit.com/}{Advertise}
\item
  \href{http://www.tbrandstudio.com/}{T Brand Studio}
\item
  \href{https://www.nytimes3xbfgragh.onion/privacy/cookie-policy\#how-do-i-manage-trackers}{Your
  Ad Choices}
\item
  \href{https://www.nytimes3xbfgragh.onion/privacy}{Privacy}
\item
  \href{https://help.nytimes3xbfgragh.onion/hc/en-us/articles/115014893428-Terms-of-service}{Terms
  of Service}
\item
  \href{https://help.nytimes3xbfgragh.onion/hc/en-us/articles/115014893968-Terms-of-sale}{Terms
  of Sale}
\item
  \href{https://spiderbites.nytimes3xbfgragh.onion}{Site Map}
\item
  \href{https://help.nytimes3xbfgragh.onion/hc/en-us}{Help}
\item
  \href{https://www.nytimes3xbfgragh.onion/subscription?campaignId=37WXW}{Subscriptions}
\end{itemize}
