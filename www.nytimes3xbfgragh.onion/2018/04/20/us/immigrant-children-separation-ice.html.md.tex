Sections

SEARCH

\protect\hyperlink{site-content}{Skip to
content}\protect\hyperlink{site-index}{Skip to site index}

\href{https://www.nytimes3xbfgragh.onion/section/us}{U.S.}

\href{https://myaccount.nytimes3xbfgragh.onion/auth/login?response_type=cookie\&client_id=vi}{}

\href{https://www.nytimes3xbfgragh.onion/section/todayspaper}{Today's
Paper}

\href{/section/us}{U.S.}\textbar{}Hundreds of Immigrant Children Have
Been Taken From Parents at U.S. Border

\url{https://nyti.ms/2vzDl6K}

\begin{itemize}
\item
\item
\item
\item
\item
\end{itemize}

Advertisement

\protect\hyperlink{after-top}{Continue reading the main story}

Supported by

\protect\hyperlink{after-sponsor}{Continue reading the main story}

\hypertarget{hundreds-of-immigrant-children-have-been-taken-from-parents-at-us-border}{%
\section{Hundreds of Immigrant Children Have Been Taken From Parents at
U.S.
Border}\label{hundreds-of-immigrant-children-have-been-taken-from-parents-at-us-border}}

\includegraphics{https://static01.graylady3jvrrxbe.onion/images/2018/04/21/us/21family-separation-p1/merlin_136365531_4fa18cff-0fac-4c81-aa65-1ad3ae301d4b-articleLarge.jpg?quality=75\&auto=webp\&disable=upscale}

By
\href{https://www.nytimes3xbfgragh.onion/by/caitlin-dickerson}{Caitlin
Dickerson}

\begin{itemize}
\item
  April 20, 2018
\item
  \begin{itemize}
  \item
  \item
  \item
  \item
  \item
  \end{itemize}
\end{itemize}

\href{https://www.nytimes3xbfgragh.onion/es/2018/04/25/menores-migrantes-estados-unidos}{Leer
en español}

On Feb. 20, a young woman named Mirian arrived at the Texas border
carrying her 18-month-old son. They had fled their home in Honduras
through a cloud of tear gas, she told border agents, and needed
protection from the political violence there.

She had hoped she and her son would find refuge together. Instead, the
agents ordered her to place her son in the back seat of a government
vehicle, she said later in a sworn declaration to a federal court. They
both cried as the boy was driven away.

For months, members of Congress have been demanding answers about how
many families are being separated as they are processed at stations
along the southwest border, in part because the Trump administration has
in the past said it was considering taking children from their parents
as a way to deter migrants from coming here.

Officials have repeatedly declined to provide data on how many families
have been separated, but suggested that the number was relatively low.

But new data reviewed by The New York Times shows that more than 700
children have been taken from adults claiming to be their parents since
October, including more than 100 children under the age of 4.

The data was prepared by the Office of Refugee Resettlement, a division
of the Department of Health and Human Services that takes custody of
children who have been removed from migrant parents. Senior officials at
the Department of Homeland Security, which processes migrants at the
border, initially denied that the numbers were so high. But after they
were confirmed to The Times by three federal officials who work closely
with these cases, a spokesman for the health and human services
department on Friday acknowledged in a statement that there were
``approximately 700.''

Homeland security officials said the agency does not separate families
at the border for deterrence purposes. ``As required by law, D.H.S. must
protect the best interests of minor children crossing our borders, and
occasionally this results in separating children from an adult they are
traveling with if we cannot ascertain the parental relationship, or if
we think the child is otherwise in danger,'' a spokesman for the agency
said in a statement.

But Trump administration officials have suggested publicly in the past
that they were, indeed, considering a deterrence policy. Last year, John
F. Kelly, President Trump's chief of staff,
\href{https://www.cnn.com/2017/03/06/politics/john-kelly-separating-children-from-parents-immigration-border/}{floated
the idea} while he was serving as homeland security secretary.

If approved, the plan would have closed detention facilities that are
designed to house families and replaced them with separate shelters for
adults and children.
\href{https://www.nytimes3xbfgragh.onion/2017/12/21/us/trump-immigrant-families-separate.html}{The
White House supported the move} and convened a group of officials from
several federal agencies to consider its merits. But the Department of
Homeland Security has said the policy was never adopted.

Children removed from their families are taken to shelters run by
nongovernmental organizations. There, workers seek to identify a
relative or guardian in the United States who can take over the child's
care. But if no such adult is available, the children can languish in
custody indefinitely. Operators of these facilities say they are often
unable to locate the parents of separated children because the children
arrive without proper records.

Once a child has entered the shelter system, there is no firm process to
determine whether they have been separated from someone who was
legitimately their parent, or for reuniting parents and children who had
been mistakenly separated, said a Border Patrol official, who was not
authorized to discuss the agency's policies publicly.

``The idea of punishing parents who are trying to save their children's
lives, and punishing children for being brought to safety by their
parents by separating them, is fundamentally cruel and un-American,''
said Michelle Brané, director of the Migrant Rights and Justice program
at the Women's Refugee Commission, an advocacy group that conducts
interviews and monitoring at immigration detention centers, including
those that house children. ``It really to me is just a horrific
`Sophie's Choice' for a mom.''

Mirian has pinballed across Texas, held at various times in three other
detention centers. She is part of a lawsuit filed by the American Civil
Liberties Union on behalf of many immigrant parents seeking to prohibit
family separations at the border.

Her son's name, along with Mirian's surname, are being withheld for
their safety. But in a declaration she filed in that case, she said she
was never told why her son was being taken away from her. Since
February, the only word she has received about him has come from a case
manager at the facility in San Antonio where he is being held. Her son
asked about her and ``cried all the time'' in the days after he arrived
at the facility, the case worker said, adding that the boy had developed
an ear infection and a cough.

\includegraphics{https://static01.graylady3jvrrxbe.onion/images/2018/04/21/us/21separation-02/merlin_137102136_1968b89c-244f-4249-905f-969ea6632435-articleLarge.jpg?quality=75\&auto=webp\&disable=upscale}

``I had no idea that I would be separated from my child for seeking
help,'' Mirian said in her sworn statement. ``I am so anxious to be
reunited with him.''

Protecting children at the border is complicated because there have,
indeed, been instances of fraud. Tens of thousands of migrants arrive
there every year, and those with children in tow are often released into
the United States more quickly than adults who come alone, because of
restrictions on the amount of time that minors can be held in custody.
Some migrants have admitted they brought their children not only to
remove them from danger in such places as Central America and Africa,
but because they believed it would cause the authorities to release them
from custody sooner.

Others have admitted to posing falsely with children who are not their
own, and Border Patrol officials say that such instances of fraud are
increasing.

As the debate carries on, pressure from the White House to enact a
separation policy has continued. In conversations this month with
Kirstjen Nielsen, the homeland security secretary, Mr. Trump has
repeatedly expressed frustration that the agency has not been aggressive
enough in policing the border, according to a person at the White House
who is familiar with the discussions.

Officials presented Mr. Trump with a list of proposals, including the
plan to routinely separate immigrant adults from their children. The
president urged Ms. Nielsen to move forward with the policies, the
person said.

But even groups that support stricter immigration policies have stopped
short of endorsing a family separation policy. Jessica M. Vaughan, the
director of policy studies for the Center for Immigration Studies, one
such group, said that family separation should only be used as a ``last
resort.''

However, she said that some migrants were using children as ``human
shields'' in order to get out of immigration custody faster.

``It makes no sense at all for the government to just accept these
attempts at fraud,'' Ms. Vaughan said. ``If it appears that the child is
being used in this way, it is in the best interest of the child to be
kept separately from the parent, for the parent to be prosecuted,
because it's a crime and it's one that has to be deterred and
prosecuted.''

Advertisement

\protect\hyperlink{after-bottom}{Continue reading the main story}

\hypertarget{site-index}{%
\subsection{Site Index}\label{site-index}}

\hypertarget{site-information-navigation}{%
\subsection{Site Information
Navigation}\label{site-information-navigation}}

\begin{itemize}
\tightlist
\item
  \href{https://help.nytimes3xbfgragh.onion/hc/en-us/articles/115014792127-Copyright-notice}{©~2020~The
  New York Times Company}
\end{itemize}

\begin{itemize}
\tightlist
\item
  \href{https://www.nytco.com/}{NYTCo}
\item
  \href{https://help.nytimes3xbfgragh.onion/hc/en-us/articles/115015385887-Contact-Us}{Contact
  Us}
\item
  \href{https://www.nytco.com/careers/}{Work with us}
\item
  \href{https://nytmediakit.com/}{Advertise}
\item
  \href{http://www.tbrandstudio.com/}{T Brand Studio}
\item
  \href{https://www.nytimes3xbfgragh.onion/privacy/cookie-policy\#how-do-i-manage-trackers}{Your
  Ad Choices}
\item
  \href{https://www.nytimes3xbfgragh.onion/privacy}{Privacy}
\item
  \href{https://help.nytimes3xbfgragh.onion/hc/en-us/articles/115014893428-Terms-of-service}{Terms
  of Service}
\item
  \href{https://help.nytimes3xbfgragh.onion/hc/en-us/articles/115014893968-Terms-of-sale}{Terms
  of Sale}
\item
  \href{https://spiderbites.nytimes3xbfgragh.onion}{Site Map}
\item
  \href{https://help.nytimes3xbfgragh.onion/hc/en-us}{Help}
\item
  \href{https://www.nytimes3xbfgragh.onion/subscription?campaignId=37WXW}{Subscriptions}
\end{itemize}
