Sections

SEARCH

\protect\hyperlink{site-content}{Skip to
content}\protect\hyperlink{site-index}{Skip to site index}

\href{https://www.nytimes3xbfgragh.onion/section/world/europe}{Europe}

\href{https://myaccount.nytimes3xbfgragh.onion/auth/login?response_type=cookie\&client_id=vi}{}

\href{https://www.nytimes3xbfgragh.onion/section/todayspaper}{Today's
Paper}

\href{/section/world/europe}{Europe}\textbar{}Tiny Islands Make for Big
Tensions Between Greece and Turkey

\url{https://nyti.ms/2HgVcF0}

\begin{itemize}
\item
\item
\item
\item
\item
\end{itemize}

Advertisement

\protect\hyperlink{after-top}{Continue reading the main story}

Supported by

\protect\hyperlink{after-sponsor}{Continue reading the main story}

\hypertarget{tiny-islands-make-for-big-tensions-between-greece-and-turkey}{%
\section{Tiny Islands Make for Big Tensions Between Greece and
Turkey}\label{tiny-islands-make-for-big-tensions-between-greece-and-turkey}}

By \href{https://www.nytimes3xbfgragh.onion/by/patrick-kingsley}{Patrick
Kingsley}

\begin{itemize}
\item
  April 21, 2018
\item
  \begin{itemize}
  \item
  \item
  \item
  \item
  \item
  \end{itemize}
\end{itemize}

\includegraphics{https://static01.graylady3jvrrxbe.onion/images/2018/04/18/world/europe/xxgreece-turkey1/merlin_137001480_2eb1aea1-639c-40e4-92b5-f6b5d0dd345c-articleLarge.jpg?quality=75\&auto=webp\&disable=upscale}

KASTELLORIZO, Greece --- In the narrow Mediterranean strait between the
easternmost islands of Greece and the shoreline of western Turkey,
Kostas Raftis steered his fishing dinghy along the invisible maritime
border dividing the two countries. Usually, this is a placid spot where
Mr. Raftis fishes for red mullet and snapper. Now it is unexpectedly
becoming a geopolitical flash point.

Last week, a low-flying Turkish helicopter had passed provocatively
close to a military base on the nearby Greek island of Ro, drawing
warning shots from soldiers. That incident was followed three days later
by the
\href{https://www.nytimes3xbfgragh.onion/2018/04/12/world/europe/greece-turkey-fighter-jet.html}{death
of a Greek fighter pilot} who crashed, his government said, after
attempting to intercept a Turkish aircraft that had entered the
country's airspace.

In all, the number of incursions by Turkish military ships and jets into
Greek territory has spiked in recent months, according to Greek
officials, stoking concerns of a new military conflict in a region where
Turkey is already embroiled in the war raging in Syria.

The biggest uncertainty involves Turkey's strongman president, Recep
Tayyip Erdogan, and whether his ambitions are fueling renewed claims to
these Greek isles --- particularly after he embarked on Wednesday on an
election campaign in which he is expected to play heavily on
nationalistic sentiment.

``With the people of Turkey, we don't have problems,'' said Mr. Raftis,
58. ``The problem is with Erdogan, with the Turkish government. They
want to make Turkey bigger.''

Indeed, though the border issue has simmered for nearly a century,
analysts worry that the unpredictable nature of Mr. Erdogan makes the
situation more volatile than ever between the countries, nominal NATO
allies, who
\href{https://www.nytimes3xbfgragh.onion/1996/02/01/world/charges-fly-as-the-greeks-and-turks-avert-a-war.html}{almost
fought a war} over an uninhabited island in nearby waters two decades
ago.

In December, to the surprise of his hosts, Mr. Erdogan used the occasion
of the first visit to Greece by a Turkish president in 65 years to call
for
\href{https://www.nytimes3xbfgragh.onion/2017/12/07/world/europe/erdogan-greece-turkey-visit.html}{a
redrawing of the border}. That did not go down well.

In recent years, Mr. Erdogan has often stoked tensions overseas in order
to bolster his domestic standing, insulting several European
governments, deploying troops in Syria, and lashing out at the United
States.

Image

A view of the town of Kastellorizo. Turkey can be seen in the
distance.Credit...Eirini Vourloumis for The New York Times

Image

Army recruits arriving on Kastellorizo. A series of small aggressions by
Turkey has raised tensions between the neighbors.Credit...Eirini
Vourloumis for The New York Times

``Erdogan is a little bit out of control --- he's picking a lot of
fights and there is a lot of uncertainty about how far he's prepared to
go,'' said Nikos Tsafos, who researches the politics of the Eastern
Mediterranean at the Center for Strategic and International Studies, a
Washington-based think tank.

``The odds of something going wrong are increasing on a weekly basis,''
he said.

The border issue has its roots in the collapse of the Ottoman Empire in
the aftermath of World War I, and in subsequent international treaties
that gave many islands that had once belonged to the Ottoman Empire ---
including Kastellorizo, the nearest permanently inhabited island to Ro
--- to other European powers.

Today, Turkey --- which was formed from the rump of the Ottoman Empire
--- does not contest Kastellorizo's sovereignty. But the government
feels it is unfair that Greece should have the right to potentially
exploit energy resources in parts of the Mediterranean seabed that lie
within sight of Turkey but many hundreds of miles from the Greek
mainland.

``At the fundamental level, there is a different perception of how the
Aegean Sea should be treated,'' Mr. Tsafos said.

Other recent developments have compounded the decades-old disagreement.
Talks have
\href{https://www.nytimes3xbfgragh.onion/2017/07/07/world/europe/cyprus-reunification-turkey-talks-fail.html}{broken
down} over the status of the island of Cyprus, which is divided between
a Greek-backed and internationally recognized state in the south, and a
Turkish-backed breakaway state in the north.

Greece declined to extradite
\href{https://www.nytimes3xbfgragh.onion/2017/01/26/world/europe/greece-turkey-soldiers-extradition.html}{eight
Turkish servicemen} who had fled following a failed coup in 2016; and
the Turkish government has arrested two Greek border guards, seemingly
in response.

``The potential for a military conflict between Greece and Turkey has
never seemed as close since the 1990s,'' said Soner Cagaptay, director
of the Turkish Research Program at the Washington Institute for Near
East Policy.

The Turkish government says Greece is to blame for the spike in
tensions.

``The Greeks always want attention,'' said a senior Turkish official who
asked not to be named in accordance with Turkish protocol. ``They're
like babies, and it's always been like that.''

But statistics released by Greece suggest a different narrative.
According to the
\href{http://www.geetha.mil.gr/el/violations-gr.html}{Greek military},
Turkish incursions into Greek airspace rose to 3,317 in 2017 from 1,269
in 2014, while maritime incursions rose to 1,998 from 371 in the same
period.

The Greek and Turkish prime ministers, Alexis Tsipras and Binali
Yildirim, appeared to calm tensions with a phone call after the two
incidents over Ro last week.

But on Monday, the situation worsened again when Turkey said it had
\href{https://www.nytimes3xbfgragh.onion/reuters/2018/04/16/world/europe/16reuters-turkey-greece.html}{sent
coast guards} to remove several Greek flags that had been planted on an
islet in a Greek island group within sight of the Turkish coast.

Less than 24 hours later, Mr. Tsipras had flown to Kastellorizo ---
nominally to open a desalination plant, but in reality to send a strong
signal on Greek sovereignty.

Image

``If you ask me as a Greek islander, the Greek government must give an
answer,'' said Kostas Raftis, a Greek fisherman and restaurateur.
``Nobody wants a war. But enough.''Credit...Eirini Vourloumis for The
New York Times

Image

``Greece can defend its sovereign rights from one end of this country to
the other,'' said Prime Minister Alexis Tsipras on a visit to
Kastellorizo this week.Credit...Eirini Vourloumis for The New York Times

``Greece can defend its sovereign rights from one end of this country to
the other,'' said Mr. Tsipras, as the cliffs of Turkey loomed in the
distance over his right shoulder. ``We won't negotiate, we won't
bargain, we won't cede an inch of Kastellorizo land.''

But Turkey did not seem to get the message. After Mr. Tsipras started
his journey home, his helicopter pilot was radioed by Turkish air
traffic controllers, who accused the pilot of flying into Turkish
airspace, a Greek military official said.

After Mr. Erdogan raised the issue of redrawing the border during his
December visit, the Greek defense minister, Panos Kammenos, accused the
Turkish leadership of stupidity, described its military as enfeebled,
and reminded Turkey of a humiliating Ottoman defeat in the 19th century.

In response, Mr. Yildirim taunted Greece over its retreat from Asia
Minor in 1922, while the leader of the Turkish opposition, Kemal
Kilicdaroglu, attempted to go one better by suggesting that Turkey
\href{http://www.hurriyetdailynews.com/chp-head-slams-greek-defense-minister-vows-to-take-back-18-islands-occupied-by-greece-in-2019-124635}{invade
no less than 18 Greek islands}.

Were such an unlikely scenario to occur, Kastellorizo and Ro would most
likely be on Mr. Kilicdaroglu's list.

Ro is a hallowed place for many Greek patriots: In 1927, a woman from an
old Kastellorizo family, Despina Achladioti, moved there and kept a
Greek flag flying until her death in 1982 --- enshrining her in national
folklore as \href{https://en.wikipedia.org/wiki/Lady_of_Ro}{``the Lady
of Ro.''}

Image

Local girls wearing traditional clothes at the opening of a desalination
plant attended by Mr. Tsipras.Credit...Eirini Vourloumis for The New
York Times

Image

``Until we see a Turkish military boat in the port of Kastellorizo, we
will not be scared,'' said Dimitris Achladiotis, the island's deputy
mayor.Credit...Eirini Vourloumis for The New York Times

But some Turkish nationalists believe these islands are ``so close to
the Anatolian land mass that they should belong to Turkey not Greece,''
said \href{http://carnegieendowment.org/experts/547}{Sinan Ulgen}, a
Turkey analyst at Carnegie Europe, a Brussels-based research group, and
a former Turkish diplomat.

For all the rhetoric, many of Kastellorizo's 300 permanent residents, as
well their Turkish neighbors across the water, feel the tensions have
been exaggerated by the news media --- and by attention-seeking
politicians. For example, none of them saw or heard the helicopter
incident.

``We've had news like this for years, but we've never had an actual
problem,'' said Dimitris Achladiotis, the island's deputy mayor, who is
a great-nephew of the Lady of Ro. ``Until we see a Turkish military boat
in the port of Kastellorizo, we will not be scared.''

Further round the island's horseshoe harbor, a bar owner told the story
of how he met his Turkish wife in Kas, the Turkish town that lies a
short ride across the sea. Many Kastellorizo residents buy their weekly
shopping from Kas's market on Fridays, while a ferry service brings more
than 20,000 people in the other direction every year. And the two
communities cement their friendship with
\href{http://www.bougainville-turkey.com/megisti-kas/}{an annual
swimming race}.

``We all coexist and are similar in lots of respects,'' said Kikkos
Magiafis, the bar owner with a Turkish wife. ``We have a very similar
culture.''

This was a sentiment echoed in Kas, even among Turkish nationalists. The
islanders on Kastellorizo ``are normal people like us, civilians living
their lives like us,'' said Ismail Sah Yilmaz, the head of the local
branch of the
\href{https://www.nytimes3xbfgragh.onion/2018/01/05/world/europe/turkey-aksener-erdogan.html}{Iyi
Party}, a Turkish nationalist group.

But strolling along the quay at Kastellorizo on Tuesday, patting a few
toddlers and listening to their parents' gripes about island life, Mr.
Tsipras appeared to have other ideas.

``You are the guardians of Thermopylae,'' he told several islanders ---
though presumably he did not mean it literally.

At
\href{https://www.britannica.com/event/Battle-of-Thermopylae-Greek-history-480-BC}{the
Battle of Thermopylae} in 480 B.C., a few hundred Greeks held off tens
of thousands of soldiers from the East before, according to myth, being
betrayed and slaughtered.

Image

The island of Ro, left, where a Turkish helicopter entered Greek
airspace this month.Credit...Eirini Vourloumis for The New York Times

Advertisement

\protect\hyperlink{after-bottom}{Continue reading the main story}

\hypertarget{site-index}{%
\subsection{Site Index}\label{site-index}}

\hypertarget{site-information-navigation}{%
\subsection{Site Information
Navigation}\label{site-information-navigation}}

\begin{itemize}
\tightlist
\item
  \href{https://help.nytimes3xbfgragh.onion/hc/en-us/articles/115014792127-Copyright-notice}{©~2020~The
  New York Times Company}
\end{itemize}

\begin{itemize}
\tightlist
\item
  \href{https://www.nytco.com/}{NYTCo}
\item
  \href{https://help.nytimes3xbfgragh.onion/hc/en-us/articles/115015385887-Contact-Us}{Contact
  Us}
\item
  \href{https://www.nytco.com/careers/}{Work with us}
\item
  \href{https://nytmediakit.com/}{Advertise}
\item
  \href{http://www.tbrandstudio.com/}{T Brand Studio}
\item
  \href{https://www.nytimes3xbfgragh.onion/privacy/cookie-policy\#how-do-i-manage-trackers}{Your
  Ad Choices}
\item
  \href{https://www.nytimes3xbfgragh.onion/privacy}{Privacy}
\item
  \href{https://help.nytimes3xbfgragh.onion/hc/en-us/articles/115014893428-Terms-of-service}{Terms
  of Service}
\item
  \href{https://help.nytimes3xbfgragh.onion/hc/en-us/articles/115014893968-Terms-of-sale}{Terms
  of Sale}
\item
  \href{https://spiderbites.nytimes3xbfgragh.onion}{Site Map}
\item
  \href{https://help.nytimes3xbfgragh.onion/hc/en-us}{Help}
\item
  \href{https://www.nytimes3xbfgragh.onion/subscription?campaignId=37WXW}{Subscriptions}
\end{itemize}
