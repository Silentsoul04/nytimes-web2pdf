Sections

SEARCH

\protect\hyperlink{site-content}{Skip to
content}\protect\hyperlink{site-index}{Skip to site index}

\href{https://www.nytimes3xbfgragh.onion/section/business/economy}{Economy}

\href{https://myaccount.nytimes3xbfgragh.onion/auth/login?response_type=cookie\&client_id=vi}{}

\href{https://www.nytimes3xbfgragh.onion/section/todayspaper}{Today's
Paper}

\href{/section/business/economy}{Economy}\textbar{}Public Servants Are
Losing Their Foothold in the Middle Class

\url{https://nyti.ms/2K7SRtG}

\begin{itemize}
\item
\item
\item
\item
\item
\item
\end{itemize}

Advertisement

\protect\hyperlink{after-top}{Continue reading the main story}

Supported by

\protect\hyperlink{after-sponsor}{Continue reading the main story}

\hypertarget{public-servants-are-losing-their-foothold-in-the-middle-class}{%
\section{Public Servants Are Losing Their Foothold in the Middle
Class}\label{public-servants-are-losing-their-foothold-in-the-middle-class}}

\includegraphics{https://static01.graylady3jvrrxbe.onion/images/2018/04/13/business/00PUBLIC-1/merlin_136618272_1770bcfd-7a4c-4e4a-978b-464ce05f4a3c-articleLarge.jpg?quality=75\&auto=webp\&disable=upscale}

By \href{http://www.nytimes3xbfgragh.onion/by/patricia-cohen}{Patricia
Cohen} and
\href{http://www.nytimes3xbfgragh.onion/by/robert-gebeloff}{Robert
Gebeloff}

\begin{itemize}
\item
  April 22, 2018
\item
  \begin{itemize}
  \item
  \item
  \item
  \item
  \item
  \item
  \end{itemize}
\end{itemize}

OKLAHOMA CITY --- The anxiety and seething anger that followed the
disappearance of middle-income jobs in factory towns has helped reshape
the American political map and topple longstanding policies on tariffs
and immigration.

But globalization and automation aren't the only forces responsible for
the loss of those reliable paychecks. So is the steady erosion of the
public sector.

For generations of Americans, working for a state or local government
--- as a teacher, firefighter, bus driver or nurse --- provided a
comfortable nook in
\href{https://www.nytimes3xbfgragh.onion/2015/04/11/business/economy/middle-class-but-feeling-economically-insecure.html}{the
middle class}. No less than automobile assembly lines and steel plants,
the public sector ensured that even workers without a college education
could afford a home, a minivan, movie nights and a family vacation.

In recent years, though, the ranks of state and local employees have
languished even as the populations they serve have grown. They now
account for the
\href{https://fred.stlouisfed.org/graph/?g=ju4l}{smallest share of the
American civilian work force} since 1967.

The 19.5 million workers who remain are finding themselves financially
downgraded.
\href{https://www.nytimes3xbfgragh.onion/2018/04/02/us/teacher-strikes-oklahoma-kentucky.html}{Teachers}
who have been protesting low wages and sparse resources in
\href{https://www.nytimes3xbfgragh.onion/2018/04/12/us/oklahoma-teachers-strike.html}{Oklahoma},
\href{https://www.nytimes3xbfgragh.onion/2018/03/06/us/west-virginia-teachers-strike-deal.html?action=click\&contentCollection=U.S.\&module=RelatedCoverage\&region=Marginalia\&pgtype=article}{West
Virginia} and Kentucky --- and those in Arizona who say they
\href{https://www.npr.org/2018/04/22/604702008/arizona-teachers-plan-to-strike-on-thursday}{plan
to walk out} on Thursday --- are just one thread in that larger skein.

``I was surprised to realize along the way I was no longer middle
class,'' said Teresa Moore, who has spent 30 years investigating
complaints of abused or neglected children, veterans and seniors in
Oklahoma.

She raised two daughters in Alex, a rural dot southwest of the capital,
on her salary. But when she applied for a mortgage nine years ago, the
loan officer casually described her as ``low income.''

At 57, Ms. Moore now earns just over \$43,000, which she supplements
with a part-time job as a computer technician.

\includegraphics{https://static01.graylady3jvrrxbe.onion/images/2018/04/13/business/00PUBLIC-2/merlin_136618278_f8aef56b-fdae-49d8-b061-bfe95f7d2a66-articleLarge.jpg?quality=75\&auto=webp\&disable=upscale}

The private sector has been more welcoming. During 97 consecutive months
of \href{https://www.bls.gov/news.release/empsit.nr0.htm}{job growth},
it created 18.6 million positions, a 17 percent increase.

But that impressive streak comes with an asterisk. Many of the jobs
created --- most in service industries --- lack
\href{https://www.nytimes3xbfgragh.onion/2017/05/31/business/economy/volatile-income-economy-jobs.html}{stability
and security}. They pay little more than the minimum wage and lack
predictable hours, insurance, sick days or parental leave.

The result is that the foundation of the
\href{https://www.nytimes3xbfgragh.onion/2017/09/16/business/economy/bump-in-us-incomes-doesnt-erase-50-years-of-pain.html}{middle
class continues to be gnawed away} even as help-wanted ads multiply.

Reducing state and local payrolls, of course, is a goal that has
champions and detractors. Anti-tax crusaders, concerned about cost and
overreach, have longed for a smaller government that delivers only the
most limited services. Public-sector defenders worry that shortages of
restaurant inspectors, rat exterminators, mental health counselors and
the like will hurt neighborhoods. Pothole-studded roads and unreliable
garbage pickup don't entice businesses, either.

Yet whether one views a diminished public sector as vital to economic
growth or a threat to health and safety, it is undeniable that it has
led to a significant decline in middle-class employment opportunities.

``It's a tough time to be working in government,'' said Neil
Reichenberg, executive director of the International Public Management
Association for Human Resources. Once there were several attractions to
public employment in addition to the mission of making a difference in
your community, he added, but incentives like good health insurance and
retirement benefits have disappeared. ``There's been a lot of cutbacks
that have made government a less competitive employer,'' he said.

From the late 1950s through 1980, the United States added 350,000 new
state and local workers a year. The rate slowed in the mid-1980s through
the early 2000s, but payrolls still grew annually by 300,000 workers.

Government hiring failed to bounce back after the recession in both
Republican- and Democratic-led states, and states continued to shed
workers through 2013. The recovery's slow pace held down revenues at the
same time as baby boomers began retiring and
\href{https://www.nytimes3xbfgragh.onion/2018/04/14/business/pension-finance-oregon.html}{generous
pension and benefit commitments} made in fatter years came due.

``They couldn't pay their obligations,'' said Edwin Benton, a political
scientist at the University of South Florida and the managing editor of
an \href{http://journals.sagepub.com/home/slgb}{academic journal}, State
and Local Government Review. ``The epidemic has grown to almost every
city and state.''

``We're in uncharted waters,'' he added.

In the past 12 months, local and state payrolls grew by 31,000, a
fraction of the historical rate. There are now fewer such workers per
capita than there were three decades ago.

Nonetheless, those combined payrolls dwarf those of the federal
government, which employs about 2.8 million civilians, including postal
workers. That number has shrunk slightly in recent years.

Image

Justin Fortney was among 200 employees laid off this year by the
Oklahoma health department. In his 12 years as a state employee, he came
to realize that ``you will live in a small home and drive a sometimes
unreliable vehicle.''Credit...Brandon Thibodeaux for The New York Times

Short of money, many states have also
\href{https://www.nytimes3xbfgragh.onion/interactive/2016/08/02/business/dealbook/this-is-your-life-private-equity.html}{privatized
services} like managing public water systems, road repair, emergency
services or
\href{https://www.nytimes3xbfgragh.onion/2018/04/10/us/private-prisons-escapes-riots.html}{prisons},
transferring jobs from the public sector to private companies that have
reduced salaries and benefits to increase their profits.

The government employment pinch especially hurts in small and rural
counties, where President Trump and other Republicans are popular. These
areas tend to lack the number and diversity of private employers found
in larger cities, and are therefore more dependent on government jobs.

Oklahoma is one of several Republican-led states where persistent
anti-tax sentiment and severe budget cuts have guided policymaking,
particularly since 2010, when many candidates supported by Tea Party
voters won local offices.

Then, the newly elected governor, Mary Fallin,
\href{https://www.npr.org/2017/06/19/533556306/in-oklahoma-gop-lawmakers-support-tax-increases-to-solve-budget-crisis}{led
the charge} to reduce the state's top income-tax rate and shrink the tax
on oil and gas production to 2 percent from 7 percent for new wells. But
a sharp drop in oil and gas prices in 2014 delivered an unexpected
wallop. Tax revenue evaporated, leaving huge
\href{https://okpolicy.org/budget-trends-outlook-march-2018/}{budget}shortfalls
since then.

Justin Fortney, 41, was one of 200 employees laid off by the state
health department this year. ``It's getting more difficult to be a
public employee --- whether that's a teacher or public health officer
--- and see yourself as part of a thriving middle class,'' he said.

Mr. Fortney, who lives with his wife and son in Guthrie, 30 miles north
of the capital, was forced to start job hunting. ``We always made it
work,'' said Mr. Fortney, who was employed by the state for 12 years and
earned about \$50,000 annually. ``But if you're going to choose to be a
public servant, you have to have in mind that you will live in a small
home and drive a sometimes unreliable vehicle.''

He said he worried that talented workers will opt for the private
sector. Staffing shortages are common in states across the country.

In Houston, pinched by a property tax cap, the police chief has said his
department is
\href{https://www.cnbc.com/2018/02/02/where-the-jobs-are-houstons-police-shortage.html}{short
1,500 to 2,000 officers}. In North Carolina, a federal report blamed a
25 percent job vacancy rate at a state prison in Elizabeth City for
\href{https://www.usnews.com/news/best-states/north-carolina/articles/2018-01-25/report-understaffing-corners-cut-at-prison-where-4-died}{four
deaths that occurred during a breakout attempt}.

Back in Oklahoma, state prisons are at 153 percent capacity, while the
corrections department has lost a tenth of its staff since 2009. ``Our
folks are only armed with their self-defense training, a can of pepper
spray, and a wing and a prayer that someone will come and help them if
they get in trouble,''
\href{http://www.newson6.com/story/37780039/doc-asks-for-15-billion-for-new-prisons-training-and-pay-raises}{Joe
Allbaugh}, the director of the corrections department, has said
publicly.

Since 2009, staffing at the state mental health department in Oklahoma
is down more than 20 percent, and at the Office of Juvenile Affairs by
nearly a quarter. The state Office of Fire Marshal once employed 30
workers, but now has 18.

A
\href{https://www.ok.gov/opm/documents/2017AnnualCompensationReport.pdf}{report
on 2017 state compensation} in Oklahoma found that average salaries were
27 percent lower than for comparable jobs in the private sector.

Many government workers take a second job to make ends meet. Eldon
Johnson, 40, who cares for children with cerebral palsy and autism at a
group home in Norman, works from 2:45 to 10:45 p.m., earning \$12.50 an
hour, less than some clerks at 7-Eleven. He then drives directly to his
better-paying second job at a private mental health center, where he
works until 8 a.m.

Image

Eldon Johnson cares for children with cerebral palsy and autism in his
state job at a group home, then works a better-paying shift at a private
mental health center. ``There's no way I could make it without a second
job,'' he said.Credit...Brandon Thibodeaux for The New York Times

``There's no way I could make it without a second job, unless I lived in
a box, and maybe had a moped,'' said Mr. Johnson, who has worked for the
state for 10 years.

Advocates for disadvantaged groups like foster children or the disabled
have trouble rallying broad support for budget and tax increases, but
public school teachers have been able to recruit additional allies among
families with school-age children.

Parents from affluent Republican suburbs like Bixby and Jenks outside
Tulsa, for instance, car-pooled for a 100-mile trip to the State Capitol
last month to lobby lawmakers for more education funding and raises for
teachers.

``My adjusted gross income is \$28,000,'' said Shala Marshall, a Spanish
teacher at Jenks High School. A 17-year veteran with a master's degree
and a finalist for Oklahoma teacher of the year, Ms. Marshall has two
children. ``I can't support a family on that,'' she said.

So she puts in another 30 hours or so a week tutoring students and
selling online the luminescent pink LipSense gloss she wears. That pays
for soccer cleats, camp and school pictures, as well as her children's
health insurance.

On April 3, Governor Fallin signed
\href{http://www.tulsaworld.com/news/capitol_report/fallin-signs-common-ed-budget-bills-giving-raises-to-school/article_22e39f1b-7d7e-593c-a189-f9aff8a984ab.html}{a
bill} to increase taxes for the first time in 28 years to pay for
teacher raises of roughly \$6,000 and additional funding for schools.
Dozens of Republican lawmakers voted for the measures.

Image

Shala Marshall, right, supplements her teaching salary by tutoring
students and selling lip gloss online. That helps pay for her children's
activities and health insurance.Credit...Brandon Thibodeaux for The New
York Times

``No one wants to raise taxes, but we've got to pay the bills,'' said
Josh West, a freshman Republican from Grove, where his four children
attend public school.

He was one of several lawmakers to meet with Bixby and Jenks parents
over box lunches from Chick-fil-A. An Army veteran, Mr. West said he had
been criticized by conservative groups for refusing to sign a pledge to
never to raise taxes.

``My district just wants to fix the problem,'' Mr. West said. ``They
don't care if you're Republican or Democrat.''

Several Republicans in the administration and Legislature now concede
that tax cutting got out of hand. ``I was rather vocal last year as the
appropriations chair that even as a Republican, we had gone too far and
it was time to start investing again in Oklahoma,'' said
\href{http://www.wbur.org/hereandnow/2018/03/29/oklahoma-tax-cuts}{Leslie
Osborn}, a representative from Mustang.

The recent budget also includes small pay raises for other state
workers. Ms. Moore, for instance, who watches over vulnerable adults, is
slated to get an additional \$750 a year.

But the Department of Human Services, where she works, won't be able to
restore any of the 1,200 jobs eliminated over the past three years,
leaving the agency with 6,109 full-time employees. Those figures don't
include 800 vacancies that cannot be filled because the budget is
overstretched.

So Ms. Moore will still be responsible for roughly a third of the state,
covering 25 counties, 19,000 square miles, and more than 100 long-term
care institutions that care for older or disabled residents as well as
those with dementia.

``That's a lot of windshield time,'' she said.

Because of the cuts, the agency extended the deadline for maltreatment
investigations at nursing homes. Rather than beginning within seven days
of a complaint, they must now start within 30 days.

Ms. Moore's friends and neighbors hold conflicting views of her
taxpayer-funded job. ``The minute they have someone in the nursing home
they perceive to be mistreated, we're the first people they come to,''
she said. ``They want us when they need us. And when they no longer need
us again, they don't want us.''

Some are resentful that they are being asked to pay for benefits that
they themselves struggle to afford.

``I asked my brother, `How do you feel about this pay raise?''' Ms.
Moore recalled. ``He said: `I want you to have it. You deserve it. But
we don't feel like we should pay for it.'''

``Well,'' she said, ``who do you want to pay for it?''

Advertisement

\protect\hyperlink{after-bottom}{Continue reading the main story}

\hypertarget{site-index}{%
\subsection{Site Index}\label{site-index}}

\hypertarget{site-information-navigation}{%
\subsection{Site Information
Navigation}\label{site-information-navigation}}

\begin{itemize}
\tightlist
\item
  \href{https://help.nytimes3xbfgragh.onion/hc/en-us/articles/115014792127-Copyright-notice}{©~2020~The
  New York Times Company}
\end{itemize}

\begin{itemize}
\tightlist
\item
  \href{https://www.nytco.com/}{NYTCo}
\item
  \href{https://help.nytimes3xbfgragh.onion/hc/en-us/articles/115015385887-Contact-Us}{Contact
  Us}
\item
  \href{https://www.nytco.com/careers/}{Work with us}
\item
  \href{https://nytmediakit.com/}{Advertise}
\item
  \href{http://www.tbrandstudio.com/}{T Brand Studio}
\item
  \href{https://www.nytimes3xbfgragh.onion/privacy/cookie-policy\#how-do-i-manage-trackers}{Your
  Ad Choices}
\item
  \href{https://www.nytimes3xbfgragh.onion/privacy}{Privacy}
\item
  \href{https://help.nytimes3xbfgragh.onion/hc/en-us/articles/115014893428-Terms-of-service}{Terms
  of Service}
\item
  \href{https://help.nytimes3xbfgragh.onion/hc/en-us/articles/115014893968-Terms-of-sale}{Terms
  of Sale}
\item
  \href{https://spiderbites.nytimes3xbfgragh.onion}{Site Map}
\item
  \href{https://help.nytimes3xbfgragh.onion/hc/en-us}{Help}
\item
  \href{https://www.nytimes3xbfgragh.onion/subscription?campaignId=37WXW}{Subscriptions}
\end{itemize}
