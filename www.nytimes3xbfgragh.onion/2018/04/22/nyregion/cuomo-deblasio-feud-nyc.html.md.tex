Sections

SEARCH

\protect\hyperlink{site-content}{Skip to
content}\protect\hyperlink{site-index}{Skip to site index}

\href{https://www.nytimes3xbfgragh.onion/section/nyregion}{New York}

\href{https://myaccount.nytimes3xbfgragh.onion/auth/login?response_type=cookie\&client_id=vi}{}

\href{https://www.nytimes3xbfgragh.onion/section/todayspaper}{Today's
Paper}

\href{/section/nyregion}{New York}\textbar{}Inside One of America's
Ugliest Political Feuds: Cuomo vs. de Blasio

\url{https://nyti.ms/2vA4tma}

\begin{itemize}
\item
\item
\item
\item
\item
\item
\end{itemize}

Advertisement

\protect\hyperlink{after-top}{Continue reading the main story}

Supported by

\protect\hyperlink{after-sponsor}{Continue reading the main story}

\hypertarget{inside-one-of-americas-ugliest-political-feuds-cuomo-vs-de-blasio}{%
\section{Inside One of America's Ugliest Political Feuds: Cuomo vs. de
Blasio}\label{inside-one-of-americas-ugliest-political-feuds-cuomo-vs-de-blasio}}

\includegraphics{https://static01.graylady3jvrrxbe.onion/images/2018/04/22/nyregion/00CUOMOBDB-slide-MKBT/00CUOMOBDB-slide-MKBT-articleLarge-v2.jpg?quality=75\&auto=webp\&disable=upscale}

By \href{https://www.nytimes3xbfgragh.onion/by/shane-goldmacher}{Shane
Goldmacher} and
\href{http://www.nytimes3xbfgragh.onion/by/j-david-goodman}{J. David
Goodman}

\begin{itemize}
\item
  April 22, 2018
\item
  \begin{itemize}
  \item
  \item
  \item
  \item
  \item
  \item
  \end{itemize}
\end{itemize}

It was, almost without question, the low point of Andrew M. Cuomo's
political career.

The year was 2002 and Mr. Cuomo was badly trailing in the Democratic
primary for governor and desperately seeking a graceful exit. He needed
a loyal lieutenant, someone to help him salvage his future and negotiate
the delicate terms of political surrender.

Mr. Cuomo turned to a trusted former colleague: Bill de Blasio.

\includegraphics{https://static01.graylady3jvrrxbe.onion/images/2018/04/22/nyregion/00CUOMOBDB-slide-KIJG/00CUOMOBDB-slide-KIJG-articleLarge.jpg?quality=75\&auto=webp\&disable=upscale}

And so, in a weekend of secret shuttle diplomacy, Mr. de Blasio, then a
junior New York City councilman, did just that. Along with a cast that
included President Bill Clinton, Mr. de Blasio was an indispensable
emissary as Mr. Cuomo
\href{https://www.nytimes3xbfgragh.onion/2002/09/04/nyregion/2002-campaign-announcement-cuomo-quits-race-backs-mccall-for-governorship.html}{quit
the race} and endorsed his opponent, H. Carl McCall. It was the start of
a fence-mending mission that would eventually land Mr. Cuomo the
governorship eight years later.

The idea of Mr. de Blasio and Mr. Cuomo ever collaborating on anything
seems almost unfathomable nearly 16 years later. The two Democrats are
now engaged in a feud so nasty, petty and prolonged that even in the
cutthroat politics of New York, few can remember ever seeing anything
quite like it.

The two men have sparred over substance, silliness and everything in
between:
\href{https://www.nytimes3xbfgragh.onion/2018/04/02/nyregion/cuomo-nycha-state-of-emergency.html}{public
housing} and
\href{https://twitter.com/melissadderosa/status/976873129296449538}{private
workout routines},
\href{http://observer.com/2017/10/de-blasio-cuomo-homeless-subway/}{homelessness}
and
\href{https://www.politico.com/states/new-york/city-hall/story/2015/08/esd-joins-fray-against-topless-women-in-times-square-000000}{topless
women} in Times Square,
\href{https://www.nytimes3xbfgragh.onion/2014/01/22/nyregion/cuomo-prekindergarten-proposal.html}{taxing
millionaires} and
\href{https://nypost.com/2016/12/15/deer-who-came-to-manhattan-to-find-a-mate-will-be-put-down/}{euthanizing
a deer},
\href{https://www.wsj.com/articles/cuomo-de-blasio-clash-over-legionnaires-outbreak-1438993349}{a
Legionnaires' disease outbreak} and
\href{https://www.nytimes3xbfgragh.onion/2017/06/26/nyregion/cuomos-deployment-of-troopers-in-city-frustrates-police-leaders.html}{state
troop deployments},
\href{https://www.nytimes3xbfgragh.onion/2017/06/30/nyregion/de-blasio-cuomo-new-york-city.html}{schools},
\href{https://www.nytimes3xbfgragh.onion/2015/01/28/nyregion/new-york-blizzard.html}{snowstorms}
and
\href{https://www.nytimes3xbfgragh.onion/2017/08/06/nyregion/bill-de-blasio-will-push-for-tax-on-wealthy-to-fix-subway.html}{the
subways} --- even naps.

``I'm not a napper, really,''
\href{http://www.nydailynews.com/news/politics/cuomo-doesn-nap-job-swipe-de-blasio-article-1.3398215}{Mr.
Cuomo volunteered} last year after reports of the mayor's alleged
penchant for napping. ``I never have been.''

Both men and their closest aides have dropped any pretense of
cordiality, sniping at each other on Twitter and in interviews; Mr. de
Blasio, in particular, has adopted an Oprah-like confessional tone in
his lamentations.

``I never get that call that says, `How can we help you get the job
done? What would make your life, as the city, work better?''' Mr. de
Blasio said in a recent television interview that people close to him
said captured his frustration. ``A lot of politics, a lot of posturing,
a lot of interference, a lot of red tape, that's what I get.''

The contours of the feud, and its effects, have been puzzled over for
years: Why would two men, whose stated goals often run on similar
tracks, allow their onetime friendship --- ``in the deepest sense of the
word'' as Mr. Cuomo once put it --- to deteriorate into pure
detestation?

This portrait of a relationship fractured is based on interviews with
more than two dozen past and present aides, advisers and officials who
have worked with Mr. Cuomo and Mr. de Blasio over the last two decades.
Many spoke on condition of anonymity for fear of reprisals from either
camp. Mr. Cuomo and Mr. de Blasio both declined to speak on the record.

In their own way, Mr. Cuomo and Mr. de Blasio both are vying to define
the Democratic Party's future in New York and beyond: the mayor as a
progressive beacon for unrepentant liberalism succeeding, the governor
as a deal-cutting Democrat who can actually make good on progressive
promises.

``I believe in action. I believe in results. I believe in making a
difference in people's lives,'' Mr. Cuomo said this year when asked
about the mayor. ``I don't believe it's about giving speeches about
values.''

But the compulsive rivalry makes both look small.

``All rules of political decorum are out the window with these two,''
said Andrew Kirtzman, a New York communications strategist. ``The Cuomo
people genuinely feel that de Blasio is incompetent and the de Blasio
people genuinely feel that Cuomo is pernicious.''

Image

The friction between Mr. Cuomo and Mr. de Blasio has only worsened with
Cynthia Nixon's bid for the Democratic nomination for governor; Ms.
Nixon and Mr. de Blasio are friends, and Mr. Cuomo blames the mayor for
her candidacy.Credit...Stan Honda/Agence France-Presse --- Getty Images

Things have only worsened with the candidacy of Cynthia Nixon, the
actress, education advocate and friend of Mr. de Blasio who is
challenging the governor in the Democratic primary. Mr. Cuomo has
seethed about what he believes is Mr. de Blasio's hidden hand in her
run, and has signaled to allies that he intends to punish the mayor for
it, even against the counsel of his advisers.

The latest flash point: the
\href{https://www.nytimes3xbfgragh.onion/2018/03/31/nyregion/new-york-city-budget-cuomo.html}{recent
state budget}that served as a cudgel to exert his dominance over Mr. de
Blasio. He added new oversight to the city's mayor-run school system. He
forced the mayor to hand over \$418 million for subway repairs,
threatening to garnish property taxes if Mr. de Blasio resisted. He gave
\$250 million to the city's beleaguered public housing system --- but
then declared a state of emergency and ordered an independent monitor.

On the Monday after the state budget passed, Mr. Cuomo held a triumphant
event with the city's top elected leaders to celebrate the new funding
and sign the order. Mr. de Blasio was pointedly not invited. And when
the mayor found out, he pressed at least one elected official not to
attend, according to three people familiar with the efforts.

Later that week, at almost the exact moment that Mr. Cuomo was in a
Manhattan ballroom condemning the city's public housing as
``disgusting,'' Mr. de Blasio was on the roof of a New York City Housing
Authority development in Queens swiping at the governor for acting
``like the great white knight.''

Image

Mr. de Blasio, visiting a New York City Housing Authority property
earlier this month, noted that some politicians like Mr. Cuomo
``suddenly believe it is stylish to visit Nycha.''Credit...Bebeto
Matthews/Associated Press

``There are some politicians who suddenly believe it is stylish to visit
Nycha,'' Mr. de Blasio said. He didn't leave it for people to read
between the lines.

``Of course I'm talking about the governor,'' Mr. de Blasio said when
asked. ``Let's be real.''

\hypertarget{friends-of-convenience}{%
\subsection{Friends of Convenience}\label{friends-of-convenience}}

It is more than a little ironic that the current tug of war between Mr.
Cuomo and Mr. de Blasio is over housing: They came to know each other
when Mr. Cuomo, then the secretary of the federal Department of Housing
and Urban Development, hired Mr. de Blasio to oversee the New York
region.

It was a critical job for Mr. Cuomo, who had clear political ambitions
where his father, Mario M. Cuomo, had served three terms as governor.
Mr. de Blasio became the younger Cuomo's point man in his home state.

Mr. Cuomo had worked the phones to get Mr. de Blasio, then a rising star
in New York politics, to take the job.

``He would say to them, `Tell Bill this is why he should do this job and
why it's important to his career,''' recalled Karen Hinton, who worked
for Mr. Cuomo at HUD and later served as Mr. de Blasio's press secretary
in City Hall.

Image

Two decades ago, when Mr. Cuomo was HUD secretary, he hired Mr. Blasio
to oversee the New York region for the department; the two, while not
close friends, were certainly political allies at the time.Credit...Paul
Hosefros/The New York Times

People who worked with them at the time, and others who have spoken with
Mr. Cuomo in the years since, said he found Mr. de Blasio to be
politically sharp but not particularly substantive. Others say even then
Mr. de Blasio displayed a more ideologically leftward bent than his
boss.

``I was proud to work for him,'' Mr. de Blasio said recently of his
tenure.

They were definitely seen as a pair.

In late 1999, when Mr. Cuomo's department declared that Mayor Rudolph W.
Giuliani could not be trusted to fairly dole out millions in federal
funding for the New York City homeless, Republicans immediately
suspected collusion --- between Mr. Cuomo and Mr. de Blasio, who was
then Hillary Clinton's campaign manager.

``It seems like Cuomo and de Blasio are still working together,'' a
Republican official
\href{https://archive.nytimes3xbfgragh.onion/www.nytimes3xbfgragh.onion/library/politics/camp/122299sen-ny-gop.html}{said
at the time}.

By 2002, when Mr. Cuomo embarked on his quixotic bid for the governor's
mansion by challenging Mr. McCall, widely seen as the Democratic heir
apparent and potentially the state's first black governor, Mr. de Blasio
was among his few backers.

``You could have gotten all of his supporters into a phone booth,'' Mr.
de Blasio joked recently.

But their bond, even then, appeared to many as one of political
convenience.

``Friends might be a stretch,'' said John Marino, another emissary
deployed by Mr. Cuomo in that 2002 weekend when he withdrew from the
governor's race. ``They both respected each other. There was no question
Andrew respected what Bill thought, and vice versa.''

They climbed the ladder of New York politics in parallel steps. Mr. de
Blasio to the City Council in 2001; Mr. Cuomo to state attorney general
in 2006; Mr. de Blasio to city public advocate in 2009; Mr. Cuomo to
governor in 2010; Mr. de Blasio to mayor in 2013.

There was mutual respect, too. Matt Wing, who worked for Mr. de Blasio
as public advocate and later for Mr. Cuomo as governor, still vividly
recalls watching Mr. Cuomo's first budget address in 2011 over junk food
with Mr. de Blasio.

``I remember Bill and I both being impressed and more than a little
inspired,'' Mr. Wing recalled.

Weeks after Mr. de Blasio's inauguration on Jan. 1, 2014, Mr. Cuomo
\href{https://www.nytimes3xbfgragh.onion/2014/01/23/nyregion/cuomo-sweetens-pre-k-deal-whatever-mayor-needs.html}{proclaimed},
``I don't have a better political friend than Bill de Blasio.''

But the problems to come were already apparent.

On the day that Mr. de Blasio's last Democratic opponent bowed out in
September 2013 (Mr. Cuomo had not endorsed his friend until then), aides
to Mr. Cuomo and Mr. de Blasio
\href{https://www.wsj.com/articles/gov-andrew-cuomos-fiercest-defender-goes-on-trial-1516581201}{nearly
came to blows} over the speaking order at a unity rally. It was meant to
be Mr. de Blasio's day of triumph; Mr. Cuomo spoke longer.

The battle to be the alpha male of New York politics had only just
begun.

\hypertarget{v-for-vendetta}{%
\subsection{V for Vendetta}\label{v-for-vendetta}}

This is hardly Andrew Cuomo's first feud. He has quarreled with a parade
of politicians in the last decade:
\href{https://www.nytimes3xbfgragh.onion/2010/09/24/nyregion/24spitzer.html}{Eliot
L. Spitzer}, David A. Paterson,
\href{https://www.nytimes3xbfgragh.onion/2014/01/16/nyregion/cuomo-and-schneiderman-prepare-to-fight-over-jpmorgan-settlement.html}{Eric
T. Schneiderman},
\href{https://www.nytimes3xbfgragh.onion/2011/12/24/nyregion/for-cuomo-and-bloomberg-the-friction-doesnt-let-up.html?_r=2\&hp\&pagewanted=all}{Michael
R. Bloomberg} and
\href{http://www.nydailynews.com/news/politics/cuomo-rips-dinapoli-investing-underperforming-hedge-funds-article-1.2834063}{Thomas
P. DiNapoli}. While he was HUD secretary in the 1990s, Mr. Cuomo's clash
with the department's inspector general was
\href{https://www.washingtonpost.com/archive/politics/2001/05/06/inspector-general-at-hud-to-retire/8b50f131-956a-41b1-9b1d-4f58670e5f1c/?utm_term=.5cc1fc3acbcf}{legendary}.
Before that, he tossed sharp elbows professionally for his father in the
1980s.

Those who have worked closely with the younger Mr. Cuomo over the years
say he has a zero-sum approach to power, especially among Democrats in
New York: The more anyone else has of it, the less there is for him.

Enter Mr. de Blasio into a job where tensions between Albany and New
York City have been longstanding, even among members of the same party.
In the 1980s, it was Mayor Edward I. Koch and Gov. Mario M. Cuomo, two
Democrats. In the 1960s, it was Mayor John V. Lindsay and Gov. Nelson A.
Rockefeller, both Republican moderates.

``They couldn't be in the same room together,'' Sid Davidoff, a lobbyist
who served in Mr. Lindsay's administration, said of his former boss and
Mr. Rockefeller. ``But it wasn't a public spectacle. Now it is.''

If Mr. Cuomo has a penchant for picking fights, Mr. de Blasio has a
notoriously stubborn streak. Some early advisers gave up trying to even
help him out of exhaustion of his obstinance. For example: his
\href{https://www.nytimes3xbfgragh.onion/2016/01/31/nyregion/de-blasio-arrives-in-iowa-to-help-hillary-clinton-in-last-push-before-caucuses.html}{lonely
trip to Iowa} to canvass for Mrs. Clinton in 2016, which some top
advisers counseled against; his daily 12-mile treks to a
\href{https://www.nytimes3xbfgragh.onion/2016/08/03/nyregion/de-blasio-ymca-workout-police-protest.html}{Park
Slope gym}.

Mr. de Blasio's recalcitrance can make it difficult for him to walk away
from fights with the governor.

In some ways, Mr. Cuomo and Mr. de Blasio are more alike than either
would care to acknowledge: two ex-operatives who now serve as the
principals themselves. They are demanding, if not difficult bosses. They
operate with sometimes excruciatingly small circles of advisers, none of
whom they seem to trust as much as themselves, often to a fault.

``He's a lot like Andrew,'' Mario Cuomo had said
\href{https://cityroom.blogs.nytimes3xbfgragh.onion/2009/08/19/snubbing-green-cuomo-endorses-de-blasio/}{in
2009} when he endorsed Mr. de Blasio as public advocate.

What both sides do agree upon is that the seeds of their split were
planted in 2014, Mr. de Blasio's first full year as mayor. They just
cite different episodes.

For Mr. Cuomo, it was the new mayor's insistence on pushing for a tax
increase on millionaires.

Mr. de Blasio had just been elected on a platform to raise taxes on the
wealthy to fund an extra year of prekindergarten --- but New York's
arcane political structure required him to go to Albany for approval.

Even before the election, Mr. Cuomo had told Mr. de Blasio raising taxes
was a nonstarter given Republican control of the State Senate but that
he would provide money for the prekindergarten expansion, people briefed
on those conversations said. But the freshly elected mayor told the
governor that he would be publicly pushing for it anyway.

Mr. de Blasio saw it as fulfilling a key promise to voters. Mr. Cuomo
first saw it as irksome, and eventually disrespectful, as the mayor
rallied the governor's union allies to his cause.

There were competing rallies and hurt feelings. Eventually, the state
approved money for prekindergarten without a tax increase.

For Mr. Cuomo, the experience fueled a belief that he has voiced with
escalating emphasis in recent months: Mr. de Blasio is more concerned
with political rhetoric and talking points than practical results.

``The job of government is to get things done for people. Not to issue
press releases saying `I propose doing this,''' Mr. Cuomo said at a
recent Nycha stop, talking about the mayor without naming him.

People close to the governor say Mr. de Blasio reminds Mr. Cuomo of what
he sees as the failed liberal ideologues of the past.

``For 40 years, Cuomo has consistently said that the prior generation of
Democrats focused too much on poetry and not enough on the prose of
governing,'' said Jon J. Cowan, Mr. Cuomo's chief of staff at HUD.

But Mr. de Blasio's allies have recently cited a different dividing
point late that same year.

When many New York City police officers turned their back on the mayor
after the killing of two police officers in December 2014, Mr. Cuomo
offered the mayor no support. Inside City Hall, the episode felt like an
existential crisis for the administration and Mr. de Blasio had hoped
that his old friend would be of some assistance, according to people in
touch with him at the time.

Instead Mr. Cuomo said he
\href{http://observer.com/2014/12/cuomo-refuses-to-condemn-lynch-for-claiming-de-blasio-has-blood-on-his-hands/}{supported
both} the mayor and the police union chief, who had just said there was
``blood on the hands'' at City Hall.

``There are so many ways for you to be both a responsible actor and a
good friend and he did none of them in that moment,'' one person close
to Mr. de Blasio said.

Image

Through the Metropolitan Transportation Authority, New York State
controls the purse strings over the city's ailing subways, providing yet
another flash point between the mayor and the governor.Credit...Benjamin
Norman for The New York Times

The relationship soured entirely in 2015. The governor shut down the
subways during a snowstorm without first telling the mayor. The
\href{https://www.nytimes3xbfgragh.onion/2015/02/12/nyregion/for-cuomo-and-de-blasio-the-tension-comes-easily.html}{mayor
did not give the governor's office a heads-up} on plans to redevelop a
rail yard in Queens, and the governor quickly squashed the idea.

By June, the governor was knifing the mayor through
\href{http://newyork.cbslocal.com/2015/06/25/cuomo-de-blasio-feud/}{thinly
cloaked} anonymous interviews mocking him as ``Mr. Progressive.'' Then,
Mr. de Blasio went on television to deliver the kind of public broadside
rarely seen between two members of the same political party, let alone
old friends.

``If someone disagrees with him openly,''
\href{https://www.nytimes3xbfgragh.onion/2015/07/01/nyregion/de-blasio-denounces-cuomo-accusing-him-of-hurting-new-york-city.html}{the
mayor said}, ``some kind of revenge or vendetta follows.''

``Vendetta,'' in particular, stuck in Mr. Cuomo's craw, people familiar
with the episode said, in part because of its Mafia connotations in a
fight between two men of Italian heritage.

It was notable, then, that on the day Ms. Nixon emerged as a likely
challenger, a ``Cuomo insider'' used the very same V-word to link Ms.
Nixon to Mr. de Blasio in a statement to multiple news outlets.

``This distraction is clearly an outgrowth of the mayor's vendetta
against the governor,'' the
\href{https://nypost.com/2018/03/06/cynthia-nixon-is-quietly-working-on-her-run-against-cuomo/}{insider}
\href{http://www.nydailynews.com/new-york/cynthia-nixon-reportedly-poised-announce-run-cuomo-article-1.3858946}{said}.

It was a clear message, especially for the two veterans of Mr. de
Blasio's warring with Mr. Cuomo, Bill Hyers and Rebecca Katz, who are
now advising Ms. Nixon. In one of her first interviews, Ms. Nixon called
Mr. Cuomo
\href{https://www.glamour.com/story/cynthia-nixon-new-york-governor-interview}{``famously
vengeful.''} The echo reverberated loudly through the governor's
chambers.

\hypertarget{point-of-no-return}{%
\subsection{Point of No Return}\label{point-of-no-return}}

Of course, the mayor and the governor do talk, often by phone in calls
that are hastily arranged via text message between the two leaders.
Their calls are often not scheduled and, as such, aides are not always
on the line or in the room, especially on Mr. de Blasio's end, according
to a city person with direct knowledge of the routine.

But even when relations appear to momentarily improve --- as after
terrorist strikes --- they quickly worsen again.

From the perspective of City Hall, opinion has hardened that Mr. Cuomo
can no longer be trusted, and some despair of trying to fight back. The
state remains pre-eminent over most city affairs, a power the governor
never shied from using: It is as if the mayor can only bring a knife and
the governor avails himself of military artillery.

``The mayor can't place a monitor on the governor's decrepit state
prisons, or his failing upstate jobs programs, or on the water supply in
Hoosick Falls,'' said Eric F. Phillips, Mr. de Blasio's press secretary.
``New York's governance structure makes this an uneven fight that would
only get worse if you give in to a governor this obsessed with hurting
New York City and a fellow Democrat.''

That the state controls perhaps the city's greatest resource, its
sprawling transit system, has set up a protracted battle over funding
and responsibility as the subway system has faltered.

The mayor and his team have sought, indirectly, to lump Mr. Cuomo in
with a group of old-line Democrats --- the ``descendent Democrats'' ---
facing challenges from what they see as the party's ascendant
progressive wing.

Still, Mr. Cuomo chafes at the primacy given New York City mayors in
times of terrorism, those who know him say: The city is the target and
the mayor controls the New York Police Department, and therefore access
to information.

Mr. Cuomo had seen how Gov. George E. Pataki saw himself diminished by
Mr. Giuliani after Sept. 11, 2001, and would not be similarly
\href{https://www.nytimes3xbfgragh.onion/2002/04/18/nyregion/cuomo-s-criticism-of-pataki-s-role-after-9-11-sets-off-furor.html}{left
in this mayor's shadow}.

And so, the governor deployed the State Police
\href{https://www.dnainfo.com/new-york/20151105/midtown/cuomo-orders-more-state-police-into-nyc-boost-his-presence-here-sources/}{in
the middle of Mr. de Blasio's first term}. ``They never had uniformed
guys patrolling in New York City,'' said a former senior law enforcement
official with direct knowledge of the discussions. The move served no
apparent policing need, the official said, and led to the
\href{https://www.nytimes3xbfgragh.onion/2016/04/09/nyregion/new-york-state-police-leader-is-stepping-down.html}{departure
of the head of the State Police in 2016}.

Image

During Mr. de Blasio's first term, the governor, walking with the mayor
at the site of a terrorist bombing in Manhattan in 2016, began moving
State Police troopers to New York City, citing global terror
threats.Credit...Pool photo by Justin Lane

It was an example of a governor exerting his influence on the city
directly. He has done so increasingly in recent months, from forcing a
state monitor on public housing to an order to close one jail facility
at the troubled complex on Rikers Island.

In response to a request for comment, the Cuomo administration said that
its involvement in municipal matters is a function of state law, along
with responding to tenant requests, federal investigations and security
needs.

Mr. Cuomo has also cited the ``incompetent'' management of Nycha, which
his office notes is controlled by the mayor, and he has mocked the
mayor's timeline to close Rikers.

Even on areas of agreement, there is no comity. Mr. Cuomo's aides
reached out to aides for Mr. de Blasio to suggest a joint push around
bail reform --- an important progressive issue in this year's budget
negotiations. City Hall officials suspected ulterior motives. In the
end, it did not happen.

A majority of New Yorkers, among them Corey Johnson, the new City
Council speaker, say the conflict is
\href{https://twitter.com/QuinnipiacPoll/status/916408472287158272}{hurting
the city}.

``I don't think it's helpful for the city,'' Mr. Johnson said. ``At the
same time, you deal with the cards that you're dealt.''

Nearly everyone has been pressured to pick sides --- few more so than
Mr. Johnson, who has had three face-to-face meetings with the governor
in three months, and weekly sit-downs with the mayor that included a
recent one that stretched to two and a half hours.

``I would just say that it's a little overwhelming,'' Mr. Johnson said.
``I didn't expect to be drawn into this.''

Advertisement

\protect\hyperlink{after-bottom}{Continue reading the main story}

\hypertarget{site-index}{%
\subsection{Site Index}\label{site-index}}

\hypertarget{site-information-navigation}{%
\subsection{Site Information
Navigation}\label{site-information-navigation}}

\begin{itemize}
\tightlist
\item
  \href{https://help.nytimes3xbfgragh.onion/hc/en-us/articles/115014792127-Copyright-notice}{©~2020~The
  New York Times Company}
\end{itemize}

\begin{itemize}
\tightlist
\item
  \href{https://www.nytco.com/}{NYTCo}
\item
  \href{https://help.nytimes3xbfgragh.onion/hc/en-us/articles/115015385887-Contact-Us}{Contact
  Us}
\item
  \href{https://www.nytco.com/careers/}{Work with us}
\item
  \href{https://nytmediakit.com/}{Advertise}
\item
  \href{http://www.tbrandstudio.com/}{T Brand Studio}
\item
  \href{https://www.nytimes3xbfgragh.onion/privacy/cookie-policy\#how-do-i-manage-trackers}{Your
  Ad Choices}
\item
  \href{https://www.nytimes3xbfgragh.onion/privacy}{Privacy}
\item
  \href{https://help.nytimes3xbfgragh.onion/hc/en-us/articles/115014893428-Terms-of-service}{Terms
  of Service}
\item
  \href{https://help.nytimes3xbfgragh.onion/hc/en-us/articles/115014893968-Terms-of-sale}{Terms
  of Sale}
\item
  \href{https://spiderbites.nytimes3xbfgragh.onion}{Site Map}
\item
  \href{https://help.nytimes3xbfgragh.onion/hc/en-us}{Help}
\item
  \href{https://www.nytimes3xbfgragh.onion/subscription?campaignId=37WXW}{Subscriptions}
\end{itemize}
