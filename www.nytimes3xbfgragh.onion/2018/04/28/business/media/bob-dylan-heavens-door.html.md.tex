Sections

SEARCH

\protect\hyperlink{site-content}{Skip to
content}\protect\hyperlink{site-index}{Skip to site index}

\href{https://www.nytimes3xbfgragh.onion/section/business/media}{Media}

\href{https://myaccount.nytimes3xbfgragh.onion/auth/login?response_type=cookie\&client_id=vi}{}

\href{https://www.nytimes3xbfgragh.onion/section/todayspaper}{Today's
Paper}

\href{/section/business/media}{Media}\textbar{}Bob Dylan's Latest Gig:
Making Whiskey

\url{https://nyti.ms/2FoJulP}

\begin{itemize}
\item
\item
\item
\item
\item
\end{itemize}

Advertisement

\protect\hyperlink{after-top}{Continue reading the main story}

Supported by

\protect\hyperlink{after-sponsor}{Continue reading the main story}

\hypertarget{bob-dylans-latest-gig-making-whiskey}{%
\section{Bob Dylan's Latest Gig: Making
Whiskey}\label{bob-dylans-latest-gig-making-whiskey}}

\includegraphics{https://static01.graylady3jvrrxbe.onion/images/2018/04/29/business/29DYLAN03/merlin_137323731_5e4e45b2-1841-4a8e-bf14-bc0cc46f29b3-articleLarge.jpg?quality=75\&auto=webp\&disable=upscale}

By \href{http://www.nytimes3xbfgragh.onion/by/ben-sisario}{Ben Sisario}

\begin{itemize}
\item
  April 28, 2018
\item
  \begin{itemize}
  \item
  \item
  \item
  \item
  \item
  \end{itemize}
\end{itemize}

In late 2015, an unexpected name popped up in the liquor industry press:
Bob Dylan.

A trademark application for the term ``bootleg whiskey'' had been filed
under Mr. Dylan's name. Among those who noticed the news was Marc
Bushala, 52, a lifelong fan and a liquor entrepreneur whose bourbon
brand, Angel's Envy, had just been sold for \$150 million. Mr. Bushala
said he immediately spent weeks ``obsessing over this concept of what a
Dylan whiskey could be.''

So he reached out, and after being vetted by Mr. Dylan's
representatives, Mr. Bushala --- who speaks branding jargon like
``flavor profile'' and ``name exploration'' in an earnest Midwestern
accent --- talked to Mr. Dylan by phone, and proposed working together
on a portfolio of small-batch whiskeys. As he saw it, there was just one
problem: The name ``bootleg,'' while an
\href{https://www.rollingstone.com/music/news/the-10-best-bob-dylan-bootlegs-20110511}{apt
Dylanological pun}, wasn't quite right for a top-shelf liquor. Might Mr.
Dylan, Nobel laureate, be open to some name exploration?

``It was a little bit daunting,'' Mr. Bushala said of his pitch.

But it worked. Next month, he and Mr. Dylan will introduce Heaven's
Door, a collection of three whiskeys --- a straight rye, a straight
bourbon and a ``double-barreled'' whiskey. They are Mr. Dylan's entry
into the booming celebrity-branded spirits market, the latest career
twist for an artist who has spent five decades confounding expectations.

Mr. Dylan is not simply licensing his name. He is a full partner in the
business, Heaven's Door Spirits, which Mr. Bushala said had raised \$35
million from investors.

``We both wanted to create a collection of American whiskeys that, in
their own way, tell a story,'' Mr. Dylan said in a statement to The New
York Times. ``I've been traveling for decades, and I've been able to try
some of the best spirits that the world of whiskey has to offer. This is
great whiskey.''

\emph{{[}}\href{https://www.nytimes3xbfgragh.onion/2018/04/28/business/dylan-whiskey-taste-test.html}{\emph{Read
a Times taster's review of Bob Dylan's whiskeys.}}\emph{{]}}

The marketing of celebrity alcohol tends to lean on the perceived
lifestyle of its mascots. Drink George Clooney's Casamigos tequila, for
example --- sold last year to the beverage giant Diageo for
\href{https://www.nytimes3xbfgragh.onion/2017/06/21/business/george-clooney-tequila-casamigos-diageo.html}{up
to \$1 billion} --- and acquire some of his movie-star glamour. Want to
party like Jay-Z? Buy an
\href{https://www.bloomberg.com/news/articles/2017-04-03/jay-z-s-new-champagne-costs-850-a-bottle-armand-de-brignac-a2}{\$850
Armand de Brignac}.

``It's about fairy dust,'' said Michael Stone, the chairman of the brand
licensing agency Beanstalk, who is not involved with Heaven's Door.
``People are looking for some of the fairy dust to be sprinkled on them
from that celebrity's lifestyle.''

\includegraphics{https://static01.graylady3jvrrxbe.onion/images/2018/04/29/business/29DYLAN04/merlin_137275572_1a4d202e-ae5e-4e2a-acff-1ee3ce978ba1-articleLarge.jpg?quality=75\&auto=webp\&disable=upscale}

Heaven's Door is meant to conjure a broader idea of Mr. Dylan that is
part Renaissance man, part nighthawk. The label design is derived from
his
\href{https://www.nytimes3xbfgragh.onion/2016/09/08/arts/music/bob-dylan-iron-archway-casino.html}{ironwork
sculptures}, with rural iconography --- crows, wagon wheels --- in
silhouette. And in promotional photos lighted like classic movie stills,
a tuxedo-clad Mr. Dylan, 76, gazes off in a dark cocktail lounge or
lonely diner, glass in hand.

Like his recent albums of standards, they portray Mr. Dylan as an urbane
but still gritty crooner --- one who might well wind down his day with a
glass of bourbon.

``Dylan has these qualities that actually work well for a whiskey,'' Mr.
Bushala said. ``He has great authenticity. He is a quintessential
American. He does things the way he wants to do them. I think these are
good attributes for a super-premium whiskey as well.''

Mr. Dylan is entering the craft whiskey market as the business is
exploding. Helped by a craze for classic cocktails, sales of American
whiskey grew 52 percent over the last five years, to \$3.4 billion in
2017, according to data from the Distilled Spirits Council.

But for those who have been listening closely, whiskey has been a
decades-long thread throughout Mr. Dylan's music, going back to the
early outtake ``Moonshiner'' in 1963 and to Mr. Dylan's version of the
song ``Copper Kettle (The Pale Moonlight),'' on the 1970 album ``Self
Portrait,'' which describes the distilling process in detail. (``Get you
a copper kettle, get you a copper coil/Fill it with new made corn mash
and never more you'll toil.'')

Mr. Bushala said that over four or five meetings --- always at Mr.
Dylan's metalworking studio in Los Angeles --- and a number of phone
calls, he had learned that his partner has a sophisticated whiskey
palate.

Yet communication was still a challenge. Mr. Bushala and Ryan Perry, the
chief operating officer, struggled to interpret Mr. Dylan's wishes.
Often they came in the form of enigmatic comments or simply glances.

``Sometimes you just get a long look,'' Mr. Bushala said with a laugh,
``and you're not sure if that's disgust or approval.''

Image

Mr. Dylan performing in Britain in 1962, the year before he recorded the
outtake ``Moonshiner.'' Whiskey has been a musical thread throughout his
career.Credit...Brian Shuel/Redferns

He and Mr. Perry recalled Mr. Dylan's tasting a sample of the
double-barreled whiskey and saying that something was missing. ``It
should feel like being in a wood structure,'' he said.

They struggled to decode the remark. What kind of wooden structure? A
church? A railroad car? A barn? That led Mr. Bushala and Mr. Perry first
to probing discussions about the nose --- the liquor's aroma in the
glass --- and then to experiments in how they toasted the barrels in
which the whiskey is aged.

Months later, the men returned with a sample that they felt embodied
``that sweet, musty smell of a barn,'' Mr. Bushala said, and presented
it to Mr. Dylan, who commented approvingly.

His oblique feedback, Mr. Perry said, ``really helped us think about
barrel finishing in a different way.''

The first batches of Heaven's Door were developed with Jordan Via,
formerly of the Breckenridge Distillery in Colorado. Together, the team
tried various novel finishes. The rye, for example, was aged in
cigar-shaped oak barrels made from wood harvested in the Vosges region
of France.

To preserve Mr. Dylan's original name for the whiskey, the company will
issue an annual Bootleg Series in limited editions, in ceramic bottles
decorated with his oil and watercolor paintings. The first, a
25-year-old whiskey, will be released next year and cost about \$300.
(Heaven's Door's standard line goes for \$50 to \$80 a bottle.)

The idea of Mr. Dylan's being connected to a commercial venture always
activates some level of outrage, as it did in 2014 when fans
\href{http://www.cnn.com/2014/02/03/showbiz/tv/bob-dylan-super-bowl-commercial/index.html}{cried
``sellout''} for his involvement in two Super Bowl TV ads: one for
Chobani yogurt, which used his song ``I Want You,'' and another for
Chrysler, in which Mr. Dylan
\href{https://www.youtube.com/watch?v=zd18am6dc0Y}{recited a patriotic
script} about the car industry.

But Mr. Dylan has never shied from commercial deals, and in the long run
they have barely grazed his reputation. In 1994, he allowed Richie
Havens to sing his anthem ``The Times They Are A-Changin''' in an ad for
the button-down accounting firm Coopers \& Lybrand. Ten years later, Mr.
Dylan was mocked for appearing in a Victoria's Secret commercial (in
which he tossed his black cowboy hat to a supermodel wearing angel
wings). Since then, he has done spots for Apple, Cadillac, Pepsi, IBM
and
\href{https://www.ispot.tv/ad/wh74/google-assistant-make-google-do-it-feat-sia-song-by-bob-dylan}{Google}.

Image

Heaven's Door labels were inspired by Mr. Dylan's ironwork
sculptures.Credit...Lyndon French for The New York Times

Mr. Dylan has also made a novel licensing deal for his full song catalog
to be available for use in a
\href{https://www.nytimes3xbfgragh.onion/2016/04/16/business/media/bob-dylan-inspired-drama-is-in-the-works.html}{television
drama now under development.}

Bill Flanagan, a veteran music journalist who has interviewed Mr. Dylan,
likens him to Hank Williams and Johnny Cash --- self-made entertainers
who saw no conflict in joining the marketplace.

And then there is simply Mr. Dylan's talent for provocation.

``Dylan has always resisted any attempt to fence him in,'' Mr. Flanagan
said. ``As soon as people start calling him king of the folkies, or
patron saint of the counterculture, or beloved anticommercial leftist
icon --- he almost always does something to thwart that.''

Whether Heaven's Door can compete is another question. Mr. Bushala was
one of the founders of Angel's Envy, which was introduced in 2011 and
sold to Bacardi four years later after developing a reputation for
quality and innovation. Yet the whiskey aisle keeps getting more
crowded. According to Nielsen, more than 20,000 kinds of spirits are
sold in the United States, and last year there were 27 percent more
whiskeys on sale than in 2013.

Mr. Bushala said that in their first conversation, he had told Mr. Dylan
that ``whiskey drinkers are a very cynical crowd'' and that the success
of their enterprise would depend on the quality of the product, not Mr.
Dylan's image.

Yet a few months after their first meeting, Mr. Bushala said, he had a
scare when Mr. Dylan was announced as the winner of the Nobel Prize in
Literature --- and then
\href{https://www.nytimes3xbfgragh.onion/2016/10/18/business/media/dylan-newest-nobel-laureate-maintains-his-reticence.html}{waited
weeks} to
\href{https://www.nytimes3xbfgragh.onion/2016/10/29/arts/music/bob-dylan-nobel-prize-comment.html}{acknowledge
the honor}, leading to speculation that he might not accept. ``Oh, no, a
P.R. nightmare!'' Mr. Bushala remembered thinking.

But then he realized that defying expectations was ``very much on
brand'' for Mr. Dylan, and likened the Nobel episode --- ultimately,
\href{https://www.nytimes3xbfgragh.onion/2016/12/10/arts/bob-dylan-skips-nobel-prize-ceremonies.html}{a
success} --- to their whiskey deal.

``For people who are surprised that he did a whiskey,'' Mr. Bushala
said, ``I guess they don't really know Dylan. People who know him expect
him to do things they would never expect.''

Advertisement

\protect\hyperlink{after-bottom}{Continue reading the main story}

\hypertarget{site-index}{%
\subsection{Site Index}\label{site-index}}

\hypertarget{site-information-navigation}{%
\subsection{Site Information
Navigation}\label{site-information-navigation}}

\begin{itemize}
\tightlist
\item
  \href{https://help.nytimes3xbfgragh.onion/hc/en-us/articles/115014792127-Copyright-notice}{©~2020~The
  New York Times Company}
\end{itemize}

\begin{itemize}
\tightlist
\item
  \href{https://www.nytco.com/}{NYTCo}
\item
  \href{https://help.nytimes3xbfgragh.onion/hc/en-us/articles/115015385887-Contact-Us}{Contact
  Us}
\item
  \href{https://www.nytco.com/careers/}{Work with us}
\item
  \href{https://nytmediakit.com/}{Advertise}
\item
  \href{http://www.tbrandstudio.com/}{T Brand Studio}
\item
  \href{https://www.nytimes3xbfgragh.onion/privacy/cookie-policy\#how-do-i-manage-trackers}{Your
  Ad Choices}
\item
  \href{https://www.nytimes3xbfgragh.onion/privacy}{Privacy}
\item
  \href{https://help.nytimes3xbfgragh.onion/hc/en-us/articles/115014893428-Terms-of-service}{Terms
  of Service}
\item
  \href{https://help.nytimes3xbfgragh.onion/hc/en-us/articles/115014893968-Terms-of-sale}{Terms
  of Sale}
\item
  \href{https://spiderbites.nytimes3xbfgragh.onion}{Site Map}
\item
  \href{https://help.nytimes3xbfgragh.onion/hc/en-us}{Help}
\item
  \href{https://www.nytimes3xbfgragh.onion/subscription?campaignId=37WXW}{Subscriptions}
\end{itemize}
