Sections

SEARCH

\protect\hyperlink{site-content}{Skip to
content}\protect\hyperlink{site-index}{Skip to site index}

\href{https://myaccount.nytimes3xbfgragh.onion/auth/login?response_type=cookie\&client_id=vi}{}

\href{https://www.nytimes3xbfgragh.onion/section/todayspaper}{Today's
Paper}

Corrupt Leaders Are Falling Around the World. Will It Boost Economies?

\url{https://nyti.ms/2HKg9sl}

\begin{itemize}
\item
\item
\item
\item
\item
\item
\end{itemize}

Advertisement

\protect\hyperlink{after-top}{Continue reading the main story}

Supported by

\protect\hyperlink{after-sponsor}{Continue reading the main story}

\href{/column/on-money}{On Money}

\hypertarget{corrupt-leaders-are-falling-around-the-world-will-it-boost-economies}{%
\section{Corrupt Leaders Are Falling Around the World. Will It Boost
Economies?}\label{corrupt-leaders-are-falling-around-the-world-will-it-boost-economies}}

\includegraphics{https://static01.graylady3jvrrxbe.onion/images/2018/05/06/magazine/06mag-OnMoney-image2/06mag-OnMoney-image2-articleLarge.png?quality=75\&auto=webp\&disable=upscale}

By Brook Larmer

\begin{itemize}
\item
  May 2, 2018
\item
  \begin{itemize}
  \item
  \item
  \item
  \item
  \item
  \item
  \end{itemize}
\end{itemize}

As the rogues' gallery of fallen world leaders grows, you might be
tempted to conclude that ours is the most corrupt era in history. Early
last month, the former South Korean president Park Geun-hye was
sentenced to 24 years in prison and fined \$17 million for bribery,
extortion and abuse of power. That same day, the former South African
president Jacob Zuma, forced to resign in February by his own party,
appeared in court to face charges over a \$2.5 billion arms deal. The
next day came an even bigger blow: Luiz Inácio Lula da Silva, Brazil's
former president and current front-runner in polls for October's
presidential election, began serving a 12-year jail term on corruption
charges --- in a federal police building he himself inaugurated more
than a decade ago.

Lula is a casualty of what may be the largest corruption scandal ever.
An octopuslike bribery scheme carried out by the Brazilian construction
giant Odebrecht has spawned investigations in more than a dozen
countries, implicating leaders from Chile to Colombia, Mexico to
Mozambique. (The scandal has also led to a \$3.5 billion fine levied by
the United States Justice Department against Odebrecht and inspired a
new Netflix series.) In Peru alone, three former presidents are facing
trial or serving jail time on Odebrecht-related charges.

With the crisis still metastasizing, it was hard not to be cynical when,
in mid-April, the region's presidents gathered in Lima for the Summit of
the Americas to deliver high-minded speeches on this year's theme:
``Democratic Governance Against Corruption.'' Donald Trump skipped the
meeting, sending Vice President Mike Pence in his stead. It was a
measure of the speed of these unfolding scandals that the White House
couldn't keep up. Pence's official schedule initially included ``a
banquet hosted by President Pedro Pablo Kuczynski of Peru.'' The only
problem: Kuczynski was forced to resign three weeks earlier, one day
before he was to face certain impeachment on corruption charges.

Behind all these grim headlines, however, lies a larger and more
encouraging truth: Corruption is being exposed, denounced and prosecuted
more vigorously, and at higher levels, than ever. In Latin America, at
least, these are not just momentary flare-ups at the scandal du jour;
they seem to be the rumblings of a historic shift. ``Judges, journalists
and the public now have the strength to take on high-level corruption,''
says Kevin Casas-Zamora, a former vice president of Costa Rica and
senior fellow at the Inter-American Dialogue. ``It's not that corruption
is getting worse or better, but that societies are more willing and able
to uncover it.''

\textbf{Corruption, of course,} is an ancient temptation. Aristotle
pondered its significance back in the 4th century B.C.E., around the
same time the Indian philosopher Kautiliya enumerated the ``40 ways of
embezzlement.'' Throughout history, there has been a spirited debate
about whether corruption acts like ``grease'' (smoothing the way past
red tape) or ``sand'' (clogging the system with unproductive costs and
scaring off investment). In the 18th century, the Anglo-Dutch economist
Bernard Mandeville contended, in his ``Fable of the Bees,'' that private
vices by the ``dextrous management of skilful politicians'' could be
turned into ``publick benefits.'' More recently, some academics have
argued that corruption can serve as an economic lubricant in societies
that have low levels of trust, weak legal institutions or feckless
governments. As the political scientist Samuel P. Huntington wrote: ``In
terms of economic growth, the only thing worse than a society with a
rigid, overcentralized, dishonest bureaucracy is one with a rigid,
overcentralized, honest bureaucracy.''

\includegraphics{https://static01.graylady3jvrrxbe.onion/images/2018/05/06/magazine/06mag-OnMoney-image1/06mag-OnMoney-image1-articleLarge.png?quality=75\&auto=webp\&disable=upscale}

The ``sand'' argument, however, has gained far more traction in recent
years. ``At this point, the evidence is fairly settled that corruption
is a drag on long-term growth,'' Casas-Zamora says. The World Bank
estimates that bribes siphon off around \$1.5 trillion from the global
economy each year --- about 2 percent of world's gross domestic product
and 10 times the value of overseas development aid. Much of that money,
the bank says, goes into the pockets and foreign bank accounts of elites
rather than into the public investments --- schools, courts, hospitals,
roads --- that are needed to lift economies over the long haul. By
degrading social norms and civic virtues, not to mention the rule of law
on which investment depends, corruption rots the foundations of a
healthy economy.

For citizens in the developing world, this is not an academic debate.
They see the costs of corruption every day: in shoddy public services;
in ``public servants'' suddenly driving luxury cars; in small businesses
pushed into the informal economy out of fear that taxes and laws will be
unfairly applied. Corruption is, in effect, a regressive tax. According
to the World Bank, poor families in developing countries often spend
twice as much of their income on bribes as high-income families. As the
World Bank president, Jim Yong Kim, put it: ``Every dollar that a
corrupt official or a corrupt business person puts in their pocket is a
dollar stolen from a pregnant woman who needs health care; or from a
girl or a boy who deserves an education; or from communities that need
water, roads and schools.'' He concluded: ``In the developing world,
corruption is Public Enemy No. 1.''

Just a few years ago, Peru was growing so fast that it was considered,
with Mexico, Chile and Colombia, one of the ``Pacific Pumas.'' But the
collapse of commodity prices hit hard --- Peru depends heavily on mining
--- and the Odebrecht payoffs in exchange for major infrastructure
projects laid bare the country's vast inequalities and injustices.
Corruption has plagued Latin America since the days of colonial
occupation, but in this case, the cure hurts almost as much as the
disease. The Odebrecht scandal has paralyzed construction ventures
across Peru, says Eduardo Dargent, a political scientist at the
Pontifical Catholic University of Peru, ``but it is also making the
government more prudent for new public investment projects.'' It's
bitter medicine, but if it helps Peru begin to overcome its culture of
corruption, the economy will be far healthier in the long run.

\textbf{Back in the} 1990s, when I lived in South America, Brazilians
used a common refrain, often delivered with a shrug of the shoulders,
that captured their tolerance for corrupt leaders: \emph{``Rouba mas
faz''} --- roughly, ``He steals but he gets things done.'' That passive
acceptance is disappearing fast, in Latin America and around the world.
Part of the change is demographic: Over the last decade, Latin America's
middle class has grown by roughly 40 million people and now outnumbers
those living in poverty. The middle class wants more than jobs or
survival, more than \emph{``rouba mas faz.''} As taxpayers, they want
honest government and are no longer willing to overlook abuses of power.
With the help of another new phenomenon, the mobilizing power of social
media, this rising middle class has led the anticorruption charge into
the streets, not just in Lima, Bogotá and Guatemala City but also in
Baghdad, Bucharest, Tel Aviv and beyond.

A central driver of the anticorruption movement is a young generation of
judges and prosecutors armed with greater independence and, in some
cases, tough new transparency laws. Many countries in Latin America have
adopted stricter accountability laws in the last two decades, though
only Brazil has tested them effectively. Sérgio Moro, the lead
prosecutor in Brazil's biggest corruption cases, has leaned on two other
new tools, plea bargains and whistle-blower protections, to persuade
underlings to testify against top bosses and leaders, including Lula.
``Twenty-five years ago, there were no rules, no conventions, no
jurisdictions against corruption,'' says Delia Ferreira Rubio, chair of
the global anticorruption group Transparency International. It's the
collective responsibility of governments, civil society and businesses,
Ferreira Rubio says, ``to ensure that impunity does not prevail.''

After an earlier wave of corruption scandals led to the fall of
presidents in Brazil and Peru, daily life resumed, and corruption only
worsened. Will this time be different? The social and legal progress
over the last 20 years may signal a tectonic shift. But corruption
fights back, too. In some countries, threatened elites are trying to
curb civil society, weaken prosecutors and attack the press. Even top
prosecutors say corruption cannot be rooted out with only court cases,
even ones that break the spell of impunity. The goal is to build
``systems of integrity'' throughout society. And that requires
commitment greater than the platitudes emanating from the Summit of the
Americas. Such a prospect may be years, even generations, away. But the
biggest psychological obstacle may have fallen already. As Casas-Zamora
says, ``The sense of fatalism about corruption is disappearing.''

Advertisement

\protect\hyperlink{after-bottom}{Continue reading the main story}

\hypertarget{site-index}{%
\subsection{Site Index}\label{site-index}}

\hypertarget{site-information-navigation}{%
\subsection{Site Information
Navigation}\label{site-information-navigation}}

\begin{itemize}
\tightlist
\item
  \href{https://help.nytimes3xbfgragh.onion/hc/en-us/articles/115014792127-Copyright-notice}{©~2020~The
  New York Times Company}
\end{itemize}

\begin{itemize}
\tightlist
\item
  \href{https://www.nytco.com/}{NYTCo}
\item
  \href{https://help.nytimes3xbfgragh.onion/hc/en-us/articles/115015385887-Contact-Us}{Contact
  Us}
\item
  \href{https://www.nytco.com/careers/}{Work with us}
\item
  \href{https://nytmediakit.com/}{Advertise}
\item
  \href{http://www.tbrandstudio.com/}{T Brand Studio}
\item
  \href{https://www.nytimes3xbfgragh.onion/privacy/cookie-policy\#how-do-i-manage-trackers}{Your
  Ad Choices}
\item
  \href{https://www.nytimes3xbfgragh.onion/privacy}{Privacy}
\item
  \href{https://help.nytimes3xbfgragh.onion/hc/en-us/articles/115014893428-Terms-of-service}{Terms
  of Service}
\item
  \href{https://help.nytimes3xbfgragh.onion/hc/en-us/articles/115014893968-Terms-of-sale}{Terms
  of Sale}
\item
  \href{https://spiderbites.nytimes3xbfgragh.onion}{Site Map}
\item
  \href{https://help.nytimes3xbfgragh.onion/hc/en-us}{Help}
\item
  \href{https://www.nytimes3xbfgragh.onion/subscription?campaignId=37WXW}{Subscriptions}
\end{itemize}
