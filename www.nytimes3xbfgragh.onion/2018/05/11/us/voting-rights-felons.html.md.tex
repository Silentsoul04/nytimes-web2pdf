Sections

SEARCH

\protect\hyperlink{site-content}{Skip to
content}\protect\hyperlink{site-index}{Skip to site index}

\href{https://www.nytimes3xbfgragh.onion/section/us}{U.S.}

\href{https://myaccount.nytimes3xbfgragh.onion/auth/login?response_type=cookie\&client_id=vi}{}

\href{https://www.nytimes3xbfgragh.onion/section/todayspaper}{Today's
Paper}

\href{/section/us}{U.S.}\textbar{}They Served Their Time. Now They're
Fighting for Other Ex-Felons to Vote.

\url{https://nyti.ms/2G5de7J}

\begin{itemize}
\item
\item
\item
\item
\item
\item
\end{itemize}

Advertisement

\protect\hyperlink{after-top}{Continue reading the main story}

Supported by

\protect\hyperlink{after-sponsor}{Continue reading the main story}

\hypertarget{they-served-their-time-now-theyre-fighting-for-other-ex-felons-to-vote}{%
\section{They Served Their Time. Now They're Fighting for Other
Ex-Felons to
Vote.}\label{they-served-their-time-now-theyre-fighting-for-other-ex-felons-to-vote}}

\includegraphics{https://static01.graylady3jvrrxbe.onion/images/2018/05/13/us/13EXFELON-01/merlin_137541675_80672f0a-3355-4a94-b410-a727a8637bef-articleLarge.jpg?quality=75\&auto=webp\&disable=upscale}

By \href{https://www.nytimes3xbfgragh.onion/by/farah-stockman}{Farah
Stockman}

\begin{itemize}
\item
  May 11, 2018
\item
  \begin{itemize}
  \item
  \item
  \item
  \item
  \item
  \item
  \end{itemize}
\end{itemize}

Ever since his own three-month stint behind bars, Steve Huerta has
mentored fathers emerging from prison. But it soon dawned on him that
they needed more than advice to break the cycle of joblessness and
incarceration. What they needed, he decided, was political power.

So seven years ago, Mr. Huerta,
\href{http://www.prisonerswithchildren.org/our-projects/allofus-or-none/}{a
community organizer in San Antonio}, began a door-knocking campaign to
encourage former felons to vote, which is their right in Texas as long
as they are no longer on probation or parole. Mr. Huerta has recruited
formerly incarcerated people to head precincts, responsible for getting
their neighbors to the polls. And he meticulously tracks the turnout
rate of 98,000 voters with criminal records.

``This is an entirely new voting bloc,'' said Mr. Huerta, who now
represents his area on a statewide organizing committee for the
Democratic Party in Texas. ``It's a political game-changer for
struggling communities.''

Mr. Huerta is part of a growing
\href{https://ficpmovement.wordpress.com/}{national movement} that is
pushing to politically empower formerly incarcerated people by
encouraging them to vote if they are eligible and pushing to restore
their rights if they are not. Most states curb the
\href{https://www.nytimes3xbfgragh.onion/2020/07/16/us/supreme-court-felons-voting-florida.html}{voting
rights} of former felons to some degree; an
\href{https://www.nytimes3xbfgragh.onion/interactive/2016/10/06/us/unequal-effect-of-laws-that-block-felons-from-voting.html}{estimated
six million people} nationwide are barred from voting because of felony
convictions. But a number of states are now considering whether to get
rid of the disenfranchisement laws that block
\href{https://www.nytimes3xbfgragh.onion/2020/07/16/us/supreme-court-felons-voting-florida.html}{felons
from the polls}.

\includegraphics{https://static01.graylady3jvrrxbe.onion/images/2018/05/02/us/00EXFELON-04/merlin_137541648_1c778e07-3d79-441b-b453-5eb78c3093eb-articleLarge.jpg?quality=75\&auto=webp\&disable=upscale}

In Florida, where
\href{https://www.nytimes3xbfgragh.onion/interactive/2016/10/06/us/unequal-effect-of-laws-that-block-felons-from-voting.html}{10
percent of adults can't vote} because of a felony conviction, a ballot
initiative in November would automatically restore voting rights after a
prison sentence has been completed. In New Jersey, state legislators are
considering a bill that would allow people in prison to vote. It would
be the third state, after Maine and Vermont, to do so.

Supporters say the movement gives former felons hope that they will one
day overcome the stigma of incarceration and be accepted as responsible
citizens, in addition to giving impoverished communities a greater
voice. But many conservative groups fiercely oppose the changes, arguing
that people need to first prove that they are upstanding members of
society before they can vote.

Spearheaded by voting rights activists who have themselves served time
in prison, the movement has racked up successes in recent years. In
2016, Gov. Terry McAuliffe of Virginia restored the voting rights of
\href{https://www.brennancenter.org/analysis/voting-rights-restoration-efforts-virginia}{more
than 150,000} people who had completed their sentences. And last year,
Alabama
\href{https://www.aclualabama.org/en/crimes-moral-turpitude}{passed a
law} that clarified which crimes stripped the right to vote, allowing
thousands of nonviolent offenders to cast a ballot. In New York, Gov.
Andrew M. Cuomo
\href{https://www.nytimes3xbfgragh.onion/2018/04/18/nyregion/felons-pardon-voting-rights-cuomo.html}{recently
announced}that he will grant up to 35,000 parolees the right to vote.

``Rights restoration is all a part of a nationwide struggle to make
America a real democracy,'' said
\href{https://www.motherjones.com/politics/2017/11/ex-felons-voting-for-the-first-time-could-shake-virginias-governors-race/}{Assaddique
Abdul-Rahman}, a 54-year-old Virginia man who had struggled with
homelessness and incarceration since he was 16, when he was sent to
prison for robbery. After his rights were restored by Mr. McAuliffe, he
began to help other formerly incarcerated people register to vote.
Eventually a group called the
\href{https://populardemocracy.org/our-partners/new-virginia-majority}{New
Virginia Majority} hired him as an organizer.

``In prison, they made sure to tell us, `You will never be able to vote,
unless the governor restores your rights,' '' he said. ``I knew that
those who could not vote did not have power. We were the underbelly.''

Image

Dorsey Nunn, third from left, served 10 years for his role in a deadly
liquor store robbery. He now heads a prisoner legal aid office in
California that is pushing to allow low-level felons serving time in
county jails to vote.Credit...Peter DaSilva for The New York Times

It's unclear how these new voters might change the political landscape.
Some political scientists predict that increasing felon turnout would
have a relatively small impact, since it would advantage Democrats in
urban areas where they already hold sway. But that could change as more
formerly incarcerated people flee expensive city centers, said Brandon
Rottinghaus, a political-science professor at the University of Houston.

``As more ex-felons settle in suburbs, the current battleground for so
many political battles, expanding voting rights to felons and active
registration of ex-felons may flip some seats currently held by
Republicans to the Democrats,'' Professor Rottinghaus said. In Texas, he
pointed to potential gains for Democrats in far west Houston, east
Dallas and San Antonio, all areas with competitive congressional races
this fall.

In states with strict voting laws that disenfranchise felons
indefinitely --- like Florida --- increasing turnout would most likely
make a difference in election outcomes, said Christopher Uggen, a
professor of sociology at the University of Minnesota, who
\href{http://users.soc.umn.edu/~uggen/Uggen_Manza_ASR_02.pdf}{estimated
that Democratic votes lost to felon disenfranchisement} would have
changed the outcome of seven Senate races since 1978, as well as the
2000 presidential election of George W. Bush.

The activists insist their work is nonpartisan and say they support
candidates of any party who pledge to expand felons' access to jobs,
student loans, and the polls. But such politicians are rare, Mr. Huerta
said. Democrats and Republicans alike tend to avoid campaigning in
neighborhoods with high concentrations of felons.

The United States is one of only a handful of countries that strips
voting rights from felons even after they have served their time. The
concept dates to the colonial era, when certain criminals were shunned
and stripped of rights, a practice known as
\href{https://scholarship.law.upenn.edu/cgi/viewcontent.cgi?referer=https://www.google.com/\&httpsredir=1\&article=1067\&context=penn_law_review}{civil
death}. But it only began to impact large numbers of people in the wake
of the Civil War, when several Southern states used it
\href{https://www.sentencingproject.org/publications/felony-disenfranchisement-mississippi/}{to
disenfranchise black men} who had recently gained the right to vote.
Today, laws barring felons from voting
\href{https://www.nytimes3xbfgragh.onion/2018/04/21/us/felony-voting-rights-law.html}{vary
by state}. Eligibility can change radically from one governor to the
next, causing widespread confusion.

Image

Mr. Huerta has meticulously tracked the turnout rate of 98,000 voters
with criminal records.Credit...Ilana Panich-Linsman for The New York
Times

The movement to restore felons' voting rights has gotten tangled up in
partisan ideological battles, with Democratic leaders tending to support
expanded access to the ballot and Republicans opposing it.

People who commit serious crimes ``should be required to prove that they
have turned over a new leaf before we invite them back into the fold to
be able to participate in the electoral process,'' said
\href{https://www.heritage.org/staff/jason-snead}{Jason Snead}, a policy
analyst at the Heritage Foundation, a conservative think tank, who
argues for increased scrutiny of felons at the ballot box as part of a
broader campaign against voter fraud.

At least 180 felons have been prosecuted for voting over the past 20
years, according to a \href{https://www.heritage.org/voterfraud}{list of
voting-related convictions} and civil judgments compiled by Mr. Snead.
The list includes over 100 felons who were prosecuted in Minnesota after
a local citizens group, the Minnesota Majority, crosschecked the names
of released felons against the list of people who cast ballots in 2008.

``Voter fraud is a felony,'' said Dan McGrath, a volunteer with the
group, now defunct. ``We think it's a threat to our democracy.''

But many former felons who have been prosecuted for voting say they did
not know they were ineligible, including Crystal Mason, a Texas woman
who recently received
\href{https://www.npr.org/sections/thetwo-way/2018/03/31/598458914/texas-woman-sentenced-to-5-years-for-illegal-voting}{a
five-year prison sentence} for voting in 2016. Ms. Mason, who was on
probation for tax fraud, cast a provisional ballot with the help of a
poll worker.

Image

At an event at the California State Capitol last month, former felons
learned how to lobby lawmakers on bills advocating better rights for
themselves and their families.Credit...Peter DaSilva for The New York
Times

Uncertainty over whether they are eligible and fear of prosecution keep
large numbers of felons from casting ballots, said
\href{https://www.sas.upenn.edu/~marcmere/}{Marc Meredith}, an associate
professor of political science at the University of Pennsylvania. Even
in states that allow felons to vote, he said, their turnout rate lingers
between 10 to 20 percent in a presidential election year, far below the
general population.

``Given that the downsides of voting illegally could be so harsh,
relative to the benefit,'' he said, some felons refuse to take the risk
of voting even if they think they are eligible.

Punishments handed down to those convicted of illegal voting vary
widely, from the payment of court fees to years in prison. In Texas,
judges have sent felons back to prison for violating the terms of their
probation by committing a new crime --- voting while ineligible.

Last year, formerly incarcerated activists put on their first
\href{https://ficpmovement.wordpress.com/2016-ficpfm-national-conference/}{national
conference}, which was attended by about 500 people. It buoyed local
efforts across the country. In Louisiana, Norris Henderson, who spent 27
years in prison for a murder he insists he did not commit, heads Voice
of the Experienced, a group working to expand the franchise to 71,000
people on probation and parole. In California,
\href{http://www.prisonerswithchildren.org/about/staff-directory/dorsey-nunn/}{Dorsey
Nunn}, who served 10 years for his role in a deadly liquor store
robbery, now heads a prisoner legal aid office that is pushing to allow
low-level felons serving time in county jails to vote.

And in Texas, Mr. Huerta presses on with his door-knocking efforts.
Since Ms. Mason's prison sentence, he has revamped his material to
include more prominent warnings against voting while on probation or
parole. When people question whether voting is safe, he assures them it
is not only safe, but vital.

``It's our lifeline,'' he says.

He uses his own 1999 conviction for speeding, drunken driving and drug
possession to show former felons that they can also become voters and
even elected officials.

In San Antonio's City Council District 5, where more than 17 percent of
voters have either a felony or a misdemeanor on their record, Mr.
Huerta's team has reached out to nearly half of all affected households
over a period of years.

Mr. Huerta believes that boosting turnout is crucial to bringing needed
resources into poor neighborhoods.

``No one spends money on people with no voting history,'' he said.

He said felons and their families have already helped elect more
sympathetic judges and a district attorney,
\href{https://www.mysanantonio.com/news/article/Unlawful-past-may-hurt-district-attorney-bid-789407.php}{Nico
LaHood}, who has an arrest record for a youthful drug offense.

In low-turnout local races, Mr. Huerta said, ``We have the ability to
elect justice-impacted people to the school boards that control a
billion-dollar budget with about 600 votes.''

But if he succeeds, he expects a backlash. Given how many Americans have
spent time behind bars, he said, ``People may be thinking, `What if they
all vote?' ''

Advertisement

\protect\hyperlink{after-bottom}{Continue reading the main story}

\hypertarget{site-index}{%
\subsection{Site Index}\label{site-index}}

\hypertarget{site-information-navigation}{%
\subsection{Site Information
Navigation}\label{site-information-navigation}}

\begin{itemize}
\tightlist
\item
  \href{https://help.nytimes3xbfgragh.onion/hc/en-us/articles/115014792127-Copyright-notice}{©~2020~The
  New York Times Company}
\end{itemize}

\begin{itemize}
\tightlist
\item
  \href{https://www.nytco.com/}{NYTCo}
\item
  \href{https://help.nytimes3xbfgragh.onion/hc/en-us/articles/115015385887-Contact-Us}{Contact
  Us}
\item
  \href{https://www.nytco.com/careers/}{Work with us}
\item
  \href{https://nytmediakit.com/}{Advertise}
\item
  \href{http://www.tbrandstudio.com/}{T Brand Studio}
\item
  \href{https://www.nytimes3xbfgragh.onion/privacy/cookie-policy\#how-do-i-manage-trackers}{Your
  Ad Choices}
\item
  \href{https://www.nytimes3xbfgragh.onion/privacy}{Privacy}
\item
  \href{https://help.nytimes3xbfgragh.onion/hc/en-us/articles/115014893428-Terms-of-service}{Terms
  of Service}
\item
  \href{https://help.nytimes3xbfgragh.onion/hc/en-us/articles/115014893968-Terms-of-sale}{Terms
  of Sale}
\item
  \href{https://spiderbites.nytimes3xbfgragh.onion}{Site Map}
\item
  \href{https://help.nytimes3xbfgragh.onion/hc/en-us}{Help}
\item
  \href{https://www.nytimes3xbfgragh.onion/subscription?campaignId=37WXW}{Subscriptions}
\end{itemize}
