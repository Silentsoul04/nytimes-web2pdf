Sections

SEARCH

\protect\hyperlink{site-content}{Skip to
content}\protect\hyperlink{site-index}{Skip to site index}

\href{https://www.nytimes3xbfgragh.onion/section/politics}{Politics}

\href{https://myaccount.nytimes3xbfgragh.onion/auth/login?response_type=cookie\&client_id=vi}{}

\href{https://www.nytimes3xbfgragh.onion/section/todayspaper}{Today's
Paper}

\href{/section/politics}{Politics}\textbar{}Supreme Court Ruling Favors
Sports Betting

\url{https://nyti.ms/2GfabtI}

\begin{itemize}
\item
\item
\item
\item
\item
\item
\end{itemize}

Advertisement

\protect\hyperlink{after-top}{Continue reading the main story}

Supported by

\protect\hyperlink{after-sponsor}{Continue reading the main story}

\hypertarget{supreme-court-ruling-favors-sports-betting}{%
\section{Supreme Court Ruling Favors Sports
Betting}\label{supreme-court-ruling-favors-sports-betting}}

\includegraphics{https://static01.graylady3jvrrxbe.onion/images/2018/05/15/us/politics/15dc-scotus1/15dc-scotus1-articleLarge.jpg?quality=75\&auto=webp\&disable=upscale}

By \href{http://www.nytimes3xbfgragh.onion/by/adam-liptak}{Adam Liptak}
and \href{https://www.nytimes3xbfgragh.onion/by/kevin-draper}{Kevin
Draper}

\begin{itemize}
\item
  May 14, 2018
\item
  \begin{itemize}
  \item
  \item
  \item
  \item
  \item
  \item
  \end{itemize}
\end{itemize}

WASHINGTON --- The Supreme Court
\href{https://www.supremecourt.gov/opinions/17pdf/16-476_dbfi.pdf}{struck
down a 1992 federal law} on Monday that effectively banned commercial
sports betting in most states, opening the door to legalizing the
estimated \$150 billion in illegal wagers on professional and amateur
sports that Americans make every year.

The decision seems certain to result in profound changes to the nation's
relationship with sports wagering. Bettors will no longer be forced into
the black market to use offshore wagering operations or illicit bookies.
Placing bets will be done on mobile devices, fueled and endorsed by the
lawmakers and sports officials who opposed it for so long. A trip to Las
Vegas to wager on March Madness or the Super Bowl could soon seem
quaint.

The law the decision overturned --- the Professional and Amateur Sports
Protection Act --- prohibited states from authorizing sports gambling.
Among its sponsors was Senator Bill Bradley, Democrat of New Jersey and
a former college and professional basketball star. He said the law was
needed to safeguard the integrity of sports.

But the court said the law was unconstitutional. ``It is as if federal
officers were installed in state legislative chambers and were armed
with the authority to stop legislators from voting on any offending
proposals,'' Justice Samuel A. Alito Jr. said, writing for the majority.
``A more direct affront to state sovereignty is not easy to imagine.''

Across the country, state officials and representatives of the casino
industry greeted the ruling with something like glee, nowhere more than
in New Jersey, which anticipated the decision and had been prepared to
quickly take advantage of it.

In 2011, the state's voters passed a constitutional amendment in favor
of legalizing sports betting, and three years later, the Legislature
repealed its law against sports betting. Both were challenged in court.
But now the Legislature only has to pass a law establishing the rules
and regulations for sanctioned sports betting to begin at casinos and
racetracks in the state.

A spokesman for Gov. Philip D. Murphy said his office sent a proposed
bill to the Legislature weeks ago and has been negotiating behind the
scenes in anticipation of a favorable ruling from the court. Stephen M.
Sweeney, the State Senate president, said people in New Jersey would
``definitely'' be able to bet before June 30.

That would give the state a head start in joining Nevada, which was
granted an exemption under the 1992 law, in allowing sports betting. But
five states --- Connecticut, Mississippi, New York, Pennsylvania and
West Virginia --- have recently passed sports betting laws, and similar
legislation has been introduced in at least another dozen states.

``This is a dry constitutional issue about states' rights, but it will
likely change how we have viewed sports for the past 100 years,'' said
Gabriel Feldman, the director of the sports law program at Tulane Law
School.

``It's called the gamblization of sports,'' he added. ``Fans will become
much more focused on gambling than following a team. It will make every
second of every game of every week interesting to fans as it will give
everyone something to root for.''

The American Gaming Association, a trade group that represents casinos,
predicted that the ruling would generate revenue without endangering the
integrity of sports competitions.

``Through smart, efficient regulation, this new market will protect
consumers, preserve the integrity of the games we love, empower law
enforcement to fight illegal gambling and generate new revenue for
states, sporting bodies, broadcasters and many others,'' the group said
in a statement.

The ruling in Murphy v. National Collegiate Athletic Association, No.
16-476, is also likely to be a boon for media and data companies that
have existing relationships with the major sports leagues. They include
television networks like ESPN, which is likely to benefit from more fans
having a more deeply vested interest in the action --- resulting in
higher ratings.

In addition, an entire industry has been created anticipating this kind
of sweeping change. It includes data companies like Sportradar, which
compiles and distributes instant information. Sportradar already has a
relationship with the N.F.L. and the N.B.A., as well as the
International Tennis Federation.

Not everyone was enthusiastic about the decision.

``The court's decision is monumental, with far-reaching implications for
baseball players and the game we love,'' Tony Clark, the executive
director of the Major League Baseball Players Association, said in a
statement. ``From complex intellectual property questions to the most
basic issues of player safety, the realities of widespread sports
betting must be addressed urgently and thoughtfully to avoid putting our
sport's integrity at risk as states proceed with legalization.''

But the ruling confirmed what professional sports leagues like the
N.B.A. and Major League Baseball have come to accept in recent years ---
that no matter how hard they resisted, legalized sports wagering was
inevitable. The leagues and their teams long fought efforts to make it
so, because, among other reasons, they were not assured of being able to
directly tap into the new, vast revenue stream.

Officials across sports have for years complained that legalized
wagering would lead to the corruption of their games through
match-fixing, though there is no indication that is a realistic concern.
Sports betting is legal and wildly popular in Britain, for example, but
the integrity of the Premier League has not suffered. In fact,
legalizing gambling allows companies and leagues to monitor gambling
patterns and flag betting irregularities that could suggest corruption.

In recent years, the professional sports leagues have taken varying
positions. Nominally, they are all against it: When New Jersey repealed
its law against sports betting, the N.B.A., the N.F.L., the N.H.L., and
the M.L.B., as well as the N.C.A.A., which governs college sports,
joined together to sue the state. They were on the losing side of
Monday's ruling.

While the N.F.L. and the N.C.A.A. have been the most steadfast in their
stance against legalized sports betting, the N.B.A. long ago concluded
that public opinion had shifted, that bringing the gray- and
black-market betting into the legal market would be the best way of
preventing match-fixing, and that there is money to be made for the
leagues.

In 2014, Adam Silver, the N.B.A.'s commissioner,
\href{https://www.nytimes3xbfgragh.onion/2014/11/14/opinion/nba-commissioner-adam-silver-legalize-sports-betting.html}{wrote
an Op-Ed} for The New York Times advocating the legalization and
regulation of sports betting. In an appearance for a New York Senate
committee in January, a
\href{https://www.nysenate.gov/sites/default/files/1.24.18_testimony_of_national_basketball_association.pdf}{league
official} laid out the N.B.A.'s opinion on its ideal sports betting
legislation that would, among other things, establish monitoring to
detect unusual betting activity; impose a 1 percent ``integrity'' fee on
bets that would be paid to sports leagues; and authorize digital betting
platforms in addition to brick-and-mortar casinos.

In the months since, the N.B.A. and the M.L.B. have
\href{https://www.legalsportsreport.com/19395/sports-betting-lobby-funded-by-mlb-nba/}{toured
state legislatures} lobbying lawmakers for the rules.

The leagues are not the only stakeholders trying to shape legislation.
Unions representing professional athletes like the baseball players'
association have
\href{https://www.nflpa.com/news/players-assoc-on-sports-betting}{demanded
a seat at the table}, while casinos and gambling trade groups have
opposed any calls for an integrity fee. Native American tribes, which
generate over \$30 billion in casino revenue annually, have mostly taken
a wait-and-see approach to sports betting, but will surely want a say in
how laws are crafted.

Finally, there is always a chance Congress could get involved.

``Congress can regulate sports gambling directly, but if it elects not
to do so, each state is free to act on its own,'' Justice Alito wrote in
his majority opinion.

In statements released after the Supreme Court's ruling, both the N.B.A.
and the N.F.L. called on Congress to pass a federal sports betting law,
and Senator Orrin G. Hatch of Utah, one of the original authors of the
law struck down on Monday,
\href{https://www.hatch.senate.gov/public/index.cfm/releases?ID=02C2FD7A-6D68-40B9-8002-BA458CF4DD4F}{said
he planned to introduce federal legislation} regulating sports betting.

Justice Alito said there were good policy arguments on both sides about
whether to legalize sports betting.

``Supporters argue that legalization will produce revenue for the states
and critically weaken illegal sports betting operations, which are often
run by organized crime,'' he wrote. ``Opponents contend that legalizing
sports gambling will hook the young on gambling, encourage people of
modest means to squander their savings and earnings, and corrupt
professional and college sports.''

But the question for the Supreme Court, Justice Alito wrote, was whether
Congress had crossed a constitutional line in forcing states to do its
bidding. Five justices agreed with every part of his opinion, and
Justice Stephen G. Breyer with much of it. Justice Ruth Bader Ginsburg,
joined by Justice Sonia Sotomayor, dissented, saying the majority had
ruled too broadly.

``The court wields an ax,'' Justice Ginsburg wrote, ``instead of using a
scalpel to trim the statute.''

Advertisement

\protect\hyperlink{after-bottom}{Continue reading the main story}

\hypertarget{site-index}{%
\subsection{Site Index}\label{site-index}}

\hypertarget{site-information-navigation}{%
\subsection{Site Information
Navigation}\label{site-information-navigation}}

\begin{itemize}
\tightlist
\item
  \href{https://help.nytimes3xbfgragh.onion/hc/en-us/articles/115014792127-Copyright-notice}{©~2020~The
  New York Times Company}
\end{itemize}

\begin{itemize}
\tightlist
\item
  \href{https://www.nytco.com/}{NYTCo}
\item
  \href{https://help.nytimes3xbfgragh.onion/hc/en-us/articles/115015385887-Contact-Us}{Contact
  Us}
\item
  \href{https://www.nytco.com/careers/}{Work with us}
\item
  \href{https://nytmediakit.com/}{Advertise}
\item
  \href{http://www.tbrandstudio.com/}{T Brand Studio}
\item
  \href{https://www.nytimes3xbfgragh.onion/privacy/cookie-policy\#how-do-i-manage-trackers}{Your
  Ad Choices}
\item
  \href{https://www.nytimes3xbfgragh.onion/privacy}{Privacy}
\item
  \href{https://help.nytimes3xbfgragh.onion/hc/en-us/articles/115014893428-Terms-of-service}{Terms
  of Service}
\item
  \href{https://help.nytimes3xbfgragh.onion/hc/en-us/articles/115014893968-Terms-of-sale}{Terms
  of Sale}
\item
  \href{https://spiderbites.nytimes3xbfgragh.onion}{Site Map}
\item
  \href{https://help.nytimes3xbfgragh.onion/hc/en-us}{Help}
\item
  \href{https://www.nytimes3xbfgragh.onion/subscription?campaignId=37WXW}{Subscriptions}
\end{itemize}
