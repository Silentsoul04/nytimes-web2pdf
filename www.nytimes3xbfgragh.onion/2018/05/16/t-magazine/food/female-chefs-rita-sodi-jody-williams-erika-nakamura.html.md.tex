Sections

SEARCH

\protect\hyperlink{site-content}{Skip to
content}\protect\hyperlink{site-index}{Skip to site index}

\href{https://myaccount.nytimes3xbfgragh.onion/auth/login?response_type=cookie\&client_id=vi}{}

\href{https://www.nytimes3xbfgragh.onion/section/todayspaper}{Today's
Paper}

The Female Couples Remaking the Restaurant Industry

\url{https://nyti.ms/2GmMLCV}

\begin{itemize}
\item
\item
\item
\item
\item
\end{itemize}

Advertisement

\protect\hyperlink{after-top}{Continue reading the main story}

Supported by

\protect\hyperlink{after-sponsor}{Continue reading the main story}

\href{/column/food-matters}{Food Matters}

\hypertarget{the-female-couples-remaking-the-restaurant-industry}{%
\section{The Female Couples Remaking the Restaurant
Industry}\label{the-female-couples-remaking-the-restaurant-industry}}

\includegraphics{https://static01.graylady3jvrrxbe.onion/images/2018/04/27/t-magazine/27tmag-chefs-slide-6289/27tmag-chefs-slide-6289-articleLarge.jpg?quality=75\&auto=webp\&disable=upscale}

By Ligaya Mishan

\begin{itemize}
\item
  May 16, 2018
\item
  \begin{itemize}
  \item
  \item
  \item
  \item
  \item
  \end{itemize}
\end{itemize}

On a cold February morning, Erika Nakamura, 37, studious in gold
wire-rim glasses, quietly shaved panels of beef with a blade curved like
a scimitar. Standing close, hair tucked under a backward baseball cap,
Jocelyn Guest, 34, alternated between a hook and a short skinny knife,
frenching a cavernous rack of lamb to expose the bones and make it
``prettier,'' she said.

In 2016, the two women opened White Gold Butchers, a meat shop and
restaurant on Manhattan's Upper West Side. They joined an increasing
cohort of American women at the helm of their own establishments. And
they became part of a vanguard of female couples who are changing the
notion of a mom-and-pop. Their emergence comes at a critical moment: In
the past seven months, a number of male chefs and restaurateurs have
been accused by former employees of making unwanted advances and
fostering work environments in which men felt at liberty to grope women
on staff. (Among them was Ken Friedman, whose restaurant group includes
White Gold; no allegations of misconduct have been made there, and he
declined to comment for this piece.)

These stories are part of a larger pattern of brutality built into the
bones of professional kitchens, many of which follow some form of the
brigade de cuisine --- essentially a caste system that was devised by
the 19th-century French chef Auguste Escoffier, an army veteran, to
mirror the military's unyielding chain of command. In the classic
hierarchy, every member of the kitchen was assigned a rank and value,
from apprentice to commis to the chefs de partie, and held accountable
for individual components of a dish, each owing unquestioning fealty to
tiers of superiors, rising upward to the chef. How else could the back
of the house survive the inferno of a Saturday night, with all the
burners on full blast, the printer rat-a-tatting tickets and half the
proteins 86'ed? In battle, souls are forged. Never mind that, along the
way, a cook might get branded by a white-hot spoon pressed into the
flesh.

For much of the brigade's existence, its members have been almost
exclusively male. But this doesn't mean that the system is intrinsically
patriarchal. Men claim no monopoly on abuses of power in the kitchen ---
several high-profile female chefs have been named in lawsuits ---
although men are more likely than women to be given the power to abuse.
And the presence of women at the top has not always protected restaurant
workers from harassment, as was the case at New York's gastropub the
Spotted Pig, where chef April Bloomfield has since apologized for not
knowing the extent of co-owner Friedman's behavior.

But it's worth asking whether there's a difference in restaurants run
entirely by women --- particularly ones run by female couples linked in
life as in work, who consciously model collaboration from the top down.
Might they suggest a better way?

Preeti Mistry, 41, was determined to banish the ghosts of kitchens past
when she opened the Indian pizzeria Navi Kitchen in the Bay Area in 2017
with her wife, Ann Nadeau, 43. No longer would a cook need to stow a
paring knife in her back pocket --- as Mistry did early in her career
--- to whip out if a guy on the line tried to cop a feel. There would be
no hazing in the walk-in, no obscene tirades, no employees going home to
cry their eyes out. Above all, no one would rule by fear, perhaps the
most radical break from the brigade. ``When a 6-foot-2 white guy is red
in the face, screaming expletives,'' Mistry says, ``someone else might
see their football coach or their dad'' --- that is, an exemplar of
tough love. Instead, as a small-statured, queer woman of color, she felt
threatened.

Mentors were few for these lesbian chefs as they came up the ranks; it
seemed that a male chef was more likely to give opportunities to a male
cook in whom he saw a younger version of himself. Some of the women felt
pressure to act ``butch'' in order to fit in. ``It evened the playing
field,'' says Elise Kornack, 31, who until last year ran the tasting
menu restaurant Take Root in Brooklyn with her wife, Anna Hieronimus,
31. For Deborah VanTrece, 58, who owns Twisted Soul Cookhouse \& Pours
in Atlanta with her wife, Lorraine Lane, 50, the situation is now
reversed, with few men applying for jobs in her kitchen: ``They tell me,
the idea of working under a lady chef --- they just can't get with it.''
At the modern Japanese restaurant n/naka in Los Angeles, where Niki
Nakayama, 43, cooks alongside her wife and sous-chef, Carole
Iida-Nakayama, 40, the men who are hired tend to have sisters (so
they're accustomed to following women's orders).

Now mentors themselves, these women are mindful of nourishing their
employees' careers. ``We're just a steppingstone,'' says Jody Williams,
who opened the Mediterranean-inflected Via Carota in Manhattan's West
Village in 2014 with her partner and fellow chef Rita Sodi, both in
their early 50s. Nakayama seeks to preserve a meditative atmosphere at
her restaurant, in which cooks have a chance to ``learn and be bored,
and have that boredom turn into mastery'' --- essential for a kaiseki
meal that unfolds in delicate stanzas and infinitesimal details
including beads of ponzu gel set with the precision of a chemist's
pipette.

When straight couples run restaurants, often it's the man who takes the
mantle of chef while the woman welcomes guests in the front of house or
focuses on pastry. This reflects an industry-wide ratio: Only a fifth of
chefs and head cooks in the U.S. are women, according to 2017 Bureau of
Labor statistics. The dynamic for gay women collaborating at the top is
different, as all roles are open, including those usually ascribed to
men. Yet one difficulty of dual leadership is sorting out who's actually
in charge. At Twisted Soul, Lane, who composes a cocktail list to match
the Southern cuisine, acknowledges the hierarchy with a laugh: ``Here
comes the chef, she's the talent'' --- meaning VanTrece. One partner
might handle equipment and inventory, the other payroll and finances.
Still, employees can be confused when taught opposing techniques --- for
making porchetta at White Gold, say, or shaving Parmesan at Via Carota.
Sodi once arrived at work to find a dish slightly altered from her
instructions. ``I said, `Jody?' and the cook said yes,'' she recalled. A
call home was placed, a compromise reached.

Sodi and Williams are both stubborn enough that they ``try not to be
together in the kitchen,'' Sodi says wryly. Fortunately, each owns her
own restaurant (Williams oversees the French bistro Buvette; Sodi runs
the Tuscan restaurant I Sodi) in addition to Via Carota, where they
schedule themselves on alternating shifts. Still, in emergencies they
sublimate their egos and jump on the line to help each other. ``We have
differences of opinion,'' Williams says. ``But we can finish each
other's plates, sentences, thoughts.'' Before they opened Via Carota,
they rarely saw each other --- a common complaint for couples in the
restaurant industry. By choosing to work together, these women have
found a way to give priority to their family life without curtailing
their ambitions.

``There was once in the western parts of Libya ... a race which was
ruled by women and followed a manner of life unlike that which prevails
among us,'' wrote the Greek historian Diodorus Siculus in the first
century B.C. He was speaking of that tribe of women warriors, the
Amazons, who according to myth founded a dominion that required no input
from men, beyond some brief assistance in propagating the species and
raising children. The story has always sounded improbable: How could
women, whether in the savage ancient world or today's slightly more
civilized one, ever hold all the positions of power and make all the
rules? What would those rules be?

The women who run these restaurants are figuring it out. ``We're trying
to create a different kind of environment that doesn't exist outside our
four walls,'' Nadeau says. None of these places are utopias, and
unlearning behavior is an ongoing struggle. ``You wake one day and
realize, I really yelled at that kid and I don't know why,'' Nakamura
says. At Twisted Soul, ``it isn't all sugar-coated,'' VanTrece says. But
there's a give-and-take that she thinks might not happen in a
predominantly male kitchen: ``If you cry on the line, I'm O.K. with
it.''

Eliminating hierarchy entirely is difficult and perhaps not even
desirable, but productive change begins with the premise that the staff
is an ensemble in which ``nobody is dispensable,'' Williams says; as she
sees it, even her dishwashers are training to one day become chefs.
Nakamura and Guest, who left White Gold in March and moved north of the
city, made a point of hiring a group of peers whom they considered
equals in skill. The shop was an intimate setting --- anybody walking in
the tight space behind the butchers had to warn them first or risk
getting accidentally stabbed --- and Guest hoped the quality of their
work always reflected genuine camaraderie: ``Don't do anything because I
said so but because we're a family.'' It's a spirit that spilled over to
their customers, who brought them canned pâtés from trips abroad and
hand-knit mittens at Christmas.

It once was enough for a restaurant to simply serve great food; no one
had to think about the anonymous, invisible serfs who made it. In
Mistry's first years as a cook, she toiled in a basement assembling
canapés for models and movie stars: ``They don't know you exist, and if
they did, they wouldn't care.'' But now diners do care. Some question
whether they should patronize establishments reputed to have enabled
sexual harassment. Others are choosing to support a different kind of
restaurant --- one where chefs make a safe working environment not a
goal but a baseline; where employees, regardless of gender, orientation
or ethnicity, can thrive. As historian Adrienne Mayor wrote in her 2014
study ``The Amazons,'' looking back on the society in which women ruled,
``The surprise answer to the question of who will be dominated and tamed
is no one.''

Nakamura recalled an afternoon last fall when a father brought his young
daughter into White Gold. Beaming, he waved a hand at the butchers
behind the counter. ``Look!'' he said. But what did he want his child to
see? Two women, smiling, holding knives. Two women with power, who know
how to use it.

Advertisement

\protect\hyperlink{after-bottom}{Continue reading the main story}

\hypertarget{site-index}{%
\subsection{Site Index}\label{site-index}}

\hypertarget{site-information-navigation}{%
\subsection{Site Information
Navigation}\label{site-information-navigation}}

\begin{itemize}
\tightlist
\item
  \href{https://help.nytimes3xbfgragh.onion/hc/en-us/articles/115014792127-Copyright-notice}{©~2020~The
  New York Times Company}
\end{itemize}

\begin{itemize}
\tightlist
\item
  \href{https://www.nytco.com/}{NYTCo}
\item
  \href{https://help.nytimes3xbfgragh.onion/hc/en-us/articles/115015385887-Contact-Us}{Contact
  Us}
\item
  \href{https://www.nytco.com/careers/}{Work with us}
\item
  \href{https://nytmediakit.com/}{Advertise}
\item
  \href{http://www.tbrandstudio.com/}{T Brand Studio}
\item
  \href{https://www.nytimes3xbfgragh.onion/privacy/cookie-policy\#how-do-i-manage-trackers}{Your
  Ad Choices}
\item
  \href{https://www.nytimes3xbfgragh.onion/privacy}{Privacy}
\item
  \href{https://help.nytimes3xbfgragh.onion/hc/en-us/articles/115014893428-Terms-of-service}{Terms
  of Service}
\item
  \href{https://help.nytimes3xbfgragh.onion/hc/en-us/articles/115014893968-Terms-of-sale}{Terms
  of Sale}
\item
  \href{https://spiderbites.nytimes3xbfgragh.onion}{Site Map}
\item
  \href{https://help.nytimes3xbfgragh.onion/hc/en-us}{Help}
\item
  \href{https://www.nytimes3xbfgragh.onion/subscription?campaignId=37WXW}{Subscriptions}
\end{itemize}
