Sections

SEARCH

\protect\hyperlink{site-content}{Skip to
content}\protect\hyperlink{site-index}{Skip to site index}

\href{https://myaccount.nytimes3xbfgragh.onion/auth/login?response_type=cookie\&client_id=vi}{}

\href{https://www.nytimes3xbfgragh.onion/section/todayspaper}{Today's
Paper}

Who Gets to Decide What Belongs in the `Canon'?

\url{https://nyti.ms/2H2ahFn}

\begin{itemize}
\item
\item
\item
\item
\item
\item
\end{itemize}

Advertisement

\protect\hyperlink{after-top}{Continue reading the main story}

Supported by

\protect\hyperlink{after-sponsor}{Continue reading the main story}

\href{/column/first-words}{First Words}

\hypertarget{who-gets-to-decide-what-belongs-in-the-canon}{%
\section{Who Gets to Decide What Belongs in the
`Canon'?}\label{who-gets-to-decide-what-belongs-in-the-canon}}

\includegraphics{https://static01.graylady3jvrrxbe.onion/images/2018/06/03/magazine/03mag-FirstWords-image1/03mag-FirstWords-image1-articleLarge.png?quality=75\&auto=webp\&disable=upscale}

By \href{http://www.nytimes3xbfgragh.onion/by/wesley-morris}{Wesley
Morris}

\begin{itemize}
\item
  May 30, 2018
\item
  \begin{itemize}
  \item
  \item
  \item
  \item
  \item
  \item
  \end{itemize}
\end{itemize}

\href{https://www.nytimes3xbfgragh.onion/es/2018/06/03/fanaticos-canon-arte/}{Leer
en español}

Sometimes, it's not enough to love something. You have to take that
thing --- album, author, song, movie, show --- and do more than love it.
It needs to be placed beyond mere love. You need to take that thing,
wrap it in plastic or put it on a pedestal. You need to dome it under a
force field so that other people's grubby hands, opinions and inferior
fandoms can't stain or disrespect it. You need not only to certify it
but also to forestall decertification. Basically, you need to make it
``canon.''

The phrase didn't originate on the internet but is \emph{of} the
internet and its wing of antidiscursive discourse. It places a work, a
person or an idea beyond reproach. It pre-resolves debate. That is, of
course, what a canon is --- a settled matter. It's established rules and
norms. It's the books of the Bible. It's the approved Catholic saints.
It's Jane Austen, the Beatles, Miles Davis, Andy Warhol and Beyoncé.

Traditionally, the people drawing up our cultural canons have been an
elite group of scholars and critics who embraced a work of art and sent
it aloft to some deifying realm. That consecration has spread from
academia to, say, Reddit, where fans gather around movies, TV dramas,
video games and comic books the way the academy threw its weight behind
Dostoyevsky, Joyce, Faulkner and Updike. ``Battlestar Galactica'' and
``The Simpsons,'' ``Buffy the Vampire Slayer'' and the DC and Marvel
universes --- they're canonical, too. And now ``canon'' has migrated
from noun to adjective, giving the word thunder and muscle and
curatorial certitude.

In this sense, ``canon'' wants to keep something like ``Star Wars''
heresy-free and internally consistent (so yes, there are canons within
canons). The series sprang more than 40 years ago from one man's mind
and a single movie. Now it's an industrial complex whose thematic
integrity desperately matters to its constituents. So when an
installment infuriates fans --- the way, in December, ``The Last Jedi''
did, with its apparent warping of the bylaws and powers of the ``Star
Wars'' galaxy (\emph{this ISN'T how the Force WORKS!!}) --- they don't
simply complain. They say, ``That's not canon.'' Last winter, a
Change.org petition circulated, calling for Disney to ``Strike Star Wars
Episode VIII from the Official Canon'' --- as though it were some kind
of Taco Bell tie-in, and not, as the title clearly states, the eighth
part of a never-ending story --- and more than 104,000 people signed on.
The receptive response to that not-entirely-serious campaign underscores
where we've been for some time with ``canon'': nervous about the unfixed
quality of all kinds of art and unyielding in policing both its meaning
and possibilities.

\textbf{On its face,} canon-making is a fairly human impulse: \emph{I
love this}. Everyone else should, too! Over time a single book becomes a
library; the library becomes a school of thought; the school of thought
becomes a prism through which the world is supposed to see itself. That
enthusiasm hardened, through curriculums, book clubs and great-works
lists, into something more authoritarian, so that canon became taste
hammered into stone tablets.

For many years its Moses has been Harold Bloom, whose ``The Western
Canon: The Books and School of the Ages'' was a best-selling sensation
in 1994, for what it argued was --- and by way of omission \emph{wasn't}
--- canon. In his introduction, Bloom went so far as to pre-emptively
dismiss complaints about his biases as coming from the ``school of
resentment.'' Asked in a 1991 Paris Review interview whom this school
comprised, Bloom explained that it's ``an extraordinary sort of mélange
of latest-model feminists, Lacanians, that whole semiotic cackle.''
These people, he went on to say, ``have no relationship whatever to
literary values.''

But these people --- women, along with nonwhite, nonstraight folks ---
certainly could have shared Bloom's literary values while also applying
prerogatives of their own. Interrogators of both the canon and the
canonizers have been dismissed as identity politicians rather than
critics or scholars. The old guard claims that they're missing the point
of literature, thrusting morality upon an amoral pursuit, sullying the
experience. Often however, they're arguing not for literature's
restriction but for its expansion --- let's include Kafka, obviously,
but also Toni Morrison, Marilynne Robinson and Jhumpa Lahiri no less
obviously.

This questioning of the canon comes from places of lived experience.
It's attuned to how great cultural work can leave you feeling irked and
demeaned. For some readers, loving Herman Melville or Joseph Conrad
requires some peacemaking with the not-quite-human representations of
black people in those texts. Loving Edith Wharton requires the same
reckoning with the insulting way she could describe Jews. Bigotry recurs
in canonical art. And committed engagement leaves us dutybound to
identify it. Shakespeare endures alongside analyses of his flawed
characterizations of all kinds of races, nationalities, religions and
women. Your great works should be strong enough to withstand some
feminist forensics.

But resisting these critiques --- whether it's of ``The House of Mirth''
or the House of Marvel --- with an automatic claim of canon feels like
an act of dominion, the establishment of an exclusive kingdom complete
with moat and drawbridge, which, of course, would make the so-called
resenters a mob of torch-wielding marauders and any challenge to
established ``literary values'' an act of savagery. Insisting that a
canon is settled gives those concerns the ``fake news'' treatment,
denying a legitimate grievance by saying there's no grounds for one.
It's shutting down a conversation, when the longer we go without one,
the harder it becomes to speak.

\textbf{Canon formation, at} its heart, has to do with defending what
you love against obsolescence, but love can tip into zealotry, which can
lead us away from actual criticism into some pretty ugly zones. Our
mutual hypersensitivities might have yanked us away from enlightening,
crucial --- and fun --- cultural detective work (close reading,
unpacking, interpreting) and turned us into beat cops always on patrol,
arresting anything that rankles. That results in a skirmish like the one
during last summer's Whitney Biennial, which culminated in the
insistence that Dana Schutz lose her career for an underwhelming
painting of Emmett Till because, as a white woman, she couldn't possibly
understand this black boy's death. The protests didn't feel like an
aesthetic demand but a post-traumatic lashing out. Canceling her might
be harsh historical justice, but it denies me an understanding of why
the painting fails.

This is to say that fandom and spectatorship, of late, have grown darkly
possessive as the country has become violently divided. Especially in
this moment when certain works of canonical art are in fact at risk of
becoming morally obsolete --- both art that degrades and insults and the
work of men accused of having done the same. There's a camp of fans ---
who tend to be as white and male as the traditional canon makers --- who
don't want that work opened up or repossessed. They don't want a
challenge to tradition --- so please, no women in the writers' room, say
superfans of the animated comedy ``Rick and Morty,'' and no earnest
acknowledgment that Apu is a bothersome South Asian stereotype, say the
makers of ``The Simpsons.'' It's all too canonical to change.

You can see the reactionary urge on every side. We've reached this
comical --- but politically necessary --- place in which nonstraight,
nonwhite, nonmale culture of all kinds has also been placed beyond
reproach. Because it's precious or rare or not meant for the people who
tend to do the canonizing. If Korama Danquah, writing for a site called
Geek Girl Authority, asserts that the sister of Black Panther is more
brilliant than the white billionaire also known as Iron Man, she doesn't
want to hear otherwise. ``Shuri is the smartest person in the Marvel
universe,'' goes the post. ``That's not an opinion, that's canon. She is
smarter than Tony Stark.'' ``Black Panther,'' according to this
argument, is canon not only because it's a Marvel movie but because it
matters too much to too many black people to be anything else.

But that's also made having conversations about the movie in which
somebody leads with, ``I really liked it, but ...'' nearly impossible.
This protectionism makes all the sense in the world for a country that's
failed to acknowledge a black audience's hunger for, say, a black
comic-book blockbuster. But critic-proofing this movie --- making it too
black to dislike --- risks making it less equal to and more fragile than
its white peers.

The intolerance of the traditional gatekeepers might have spurred a kind
of militancy from thinkers (and fans) who've rarely been allowed in.
Bloom's literary paradise is long lost, and now history compels us to
defend Wakanda's. But that leaves the contested art in an equally
perilous spot: not art at all, really, but territory.

Advertisement

\protect\hyperlink{after-bottom}{Continue reading the main story}

\hypertarget{site-index}{%
\subsection{Site Index}\label{site-index}}

\hypertarget{site-information-navigation}{%
\subsection{Site Information
Navigation}\label{site-information-navigation}}

\begin{itemize}
\tightlist
\item
  \href{https://help.nytimes3xbfgragh.onion/hc/en-us/articles/115014792127-Copyright-notice}{©~2020~The
  New York Times Company}
\end{itemize}

\begin{itemize}
\tightlist
\item
  \href{https://www.nytco.com/}{NYTCo}
\item
  \href{https://help.nytimes3xbfgragh.onion/hc/en-us/articles/115015385887-Contact-Us}{Contact
  Us}
\item
  \href{https://www.nytco.com/careers/}{Work with us}
\item
  \href{https://nytmediakit.com/}{Advertise}
\item
  \href{http://www.tbrandstudio.com/}{T Brand Studio}
\item
  \href{https://www.nytimes3xbfgragh.onion/privacy/cookie-policy\#how-do-i-manage-trackers}{Your
  Ad Choices}
\item
  \href{https://www.nytimes3xbfgragh.onion/privacy}{Privacy}
\item
  \href{https://help.nytimes3xbfgragh.onion/hc/en-us/articles/115014893428-Terms-of-service}{Terms
  of Service}
\item
  \href{https://help.nytimes3xbfgragh.onion/hc/en-us/articles/115014893968-Terms-of-sale}{Terms
  of Sale}
\item
  \href{https://spiderbites.nytimes3xbfgragh.onion}{Site Map}
\item
  \href{https://help.nytimes3xbfgragh.onion/hc/en-us}{Help}
\item
  \href{https://www.nytimes3xbfgragh.onion/subscription?campaignId=37WXW}{Subscriptions}
\end{itemize}
