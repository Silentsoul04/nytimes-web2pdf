Sections

SEARCH

\protect\hyperlink{site-content}{Skip to
content}\protect\hyperlink{site-index}{Skip to site index}

\href{https://www.nytimes3xbfgragh.onion/section/obituaries}{Obituaries}

\href{https://myaccount.nytimes3xbfgragh.onion/auth/login?response_type=cookie\&client_id=vi}{}

\href{https://www.nytimes3xbfgragh.onion/section/todayspaper}{Today's
Paper}

\href{/section/obituaries}{Obituaries}\textbar{}Nina Baym, Who Brought
Novels by Women to Light, Dies at 82

\url{https://nyti.ms/2MPdAE7}

\begin{itemize}
\item
\item
\item
\item
\item
\end{itemize}

Advertisement

\protect\hyperlink{after-top}{Continue reading the main story}

Supported by

\protect\hyperlink{after-sponsor}{Continue reading the main story}

\hypertarget{nina-baym-who-brought-novels-by-women-to-light-dies-at-82}{%
\section{Nina Baym, Who Brought Novels by Women to Light, Dies at
82}\label{nina-baym-who-brought-novels-by-women-to-light-dies-at-82}}

\includegraphics{https://static01.graylady3jvrrxbe.onion/images/2018/06/24/obituaries/24BAYM-OBIT/merlin_140063181_c2b03e4c-9b1d-43c6-823b-0ae0e476a67c-articleLarge.jpg?quality=75\&auto=webp\&disable=upscale}

By \href{http://www.nytimes3xbfgragh.onion/by/neil-genzlinger}{Neil
Genzlinger}

\begin{itemize}
\item
  June 22, 2018
\item
  \begin{itemize}
  \item
  \item
  \item
  \item
  \item
  \end{itemize}
\end{itemize}

Nina Baym, a scholar who asked why so few women were represented in the
American literary canon, then spent her career working to correct that
imbalance, died on June 15 in Urbana, Ill. She was 82.

Her daughter, Nancy Baym, said the cause was complications of dementia.

Professor Baym, who taught English at
\href{https://news.illinois.edu/view/6367/663930}{the University of
Illinois at Urbana-Champaign} for more than 40 years, was writing a book
about Nathaniel Hawthorne in 1975 when she began to wonder why
19th-century American literature was so male-dominated. Hawthorne
himself helped pique her curiosity. In 1855 he had
\href{http://debrabrenegan.blogspot.com/2011/07/damned-mob-of-scribbling-women.html}{famously
complained} that ``a damned mob of scribbling women'' was cutting into
his sales.

``I wanted to know where these women were,'' she
\href{https://www.nytimes3xbfgragh.onion/1987/12/06/magazine/literary-feminism-comes-of-age.html}{recalled
in an interview} with The New York Times in 1987.

She went searching through library bookshelves and 19th-century
newspapers and magazines, looking for information about the absent women
writers. She found plenty of novels written by women in the 1800s, and
though they varied in quality, she concluded that many deserved more
than obscurity.

``It's too easy to say they weren't any good,'' Professor Baym said.
``Because it turns out that what's considered good and what isn't is
historically flexible.''

Her work in the mid-1970s coincided with a time of second-wave feminism,
when women were challenging the inherent sexism in political, social,
legal and intellectual spheres that had been dominated by white men.

Her 1978 book, ``Woman's Fiction: A Guide to Novels by and About Women
in America, 1820-1870,'' was a foundational work in the field of
feminist literary history and criticism. It had chapters on writers like
\href{http://utc.iath.virginia.edu/sentimnt/southworthhp.html}{E.D.E.N.
Southworth}, who wrote more than 60 novels, and Maria Jane McIntosh,
whose ``Two Lives'' in 1846 went through seven printings. Professor
Baym, in her introduction, explained why these types of novels were
worthy of study.

``Today we hear of this literature, if at all, chiefly through
detractors who deplore the feminizing --- and hence degradation --- of
the noble art of letters,'' she wrote. ``A segment of literary history
is thus lost to us, a segment that may be of special interest today as
we seek to recover and understand the experiences of women.''

She continued: ``I have not unearthed a forgotten Jane Austen or George
Eliot, or hit upon even one novel that I would propose to set alongside
`The Scarlet Letter.' Yet I cannot avoid the belief that `purely'
literary criteria, as they have been employed to identify the best
American works, have inevitably had a bias in favor of things male ---
in favor, say, of whaling ships rather than the sewing circle as a
symbol of the human community; in favor of satires on domineering
mothers, shrewish wives, or betraying mistresses rather than tyrannical
fathers, abusive husbands, or philandering suitors.''

She was born Nina Zippin on June 14, 1936, in Princeton, N.J., and grew
up in Brooklyn.
\href{https://www.nytimes3xbfgragh.onion/1995/05/20/obituaries/leo-zippin-90-dies-solved-math-puzzle.html}{Her
father}, Leo, was a noted mathematician, and her mother, Frances
(Levinson) Zippin, taught English in New York City public schools.

Professor Baym earned a bachelor's degree at Cornell University and a
master's at Radcliffe College. In the early 1960s she received her Ph.D
at Harvard and joined the University of Illinois faculty in 1963.

Her interests and writings encompassed the traditional canonical
American writers like Emerson and Thoreau. Among her early books was
``Shape of Hawthorne's Career'' (1976). She took up Hawthorne again in
``The Scarlet Letter: A Reading,'' published in 1986.

But it was her work on female writers, and the accompanying exhortation
to re-examine how society judges what writing is significant, that
helped expand the reading lists of high schools and colleges to include
long-ignored women.

Her other books included ``Women Writers of the American West,
1832-1927.'' A book of her essays, ``Feminism and American Literary
History'' (1992), led with a 1981 essay that is still being studied and
quoted: ``Melodramas of Beset Manhood: How Theories of American Fiction
Exclude Women Authors.''

Professor Baym had a chance to have a direct impact on the literature
presented to students when she served as general editor of several
editions of
\href{http://books.wwnorton.com/books/college-subject.aspx?id=4294983309}{The
Norton Anthology of American Literature}, the omnibus work used in many
high schools and colleges. In her preface to the fifth edition, in 1998,
she noted the addition not only of a number of female authors (among
them
\href{http://www.womenhistoryblog.com/2012/01/catharine-maria-sedgwick.html}{Catharine
Maria Sedgwick} and \href{https://fannyfern.org/bio}{Fanny Fern}), but
also of pre-colonial stories by American Indians.

Professor Baym was a Guggenheim fellow in 1975 and a National Endowment
for the Humanities fellow in 1982. She retired as professor emerita at
Illinois in 2004.

Her marriage to Gordon Baym, in 1958, ended in divorce in 1970. In
addition to her daughter, Nancy, she is survived by her husband, Jack
Stillinger, whom she married in 1971; a son, Geoffrey Baym; two
stepdaughters, Susan Stillinger and Mary Stillinger; two stepsons, Tom
Stillinger and Bob Stillinger; two grandsons; and four
step-grandchildren.

Numerous social-media posts from colleagues and former students paid
tribute to Professor Baym, citing both her writing and her excellence as
a teacher. One, on Twitter, was from Catherine Prendergast, a University
of Illinois English professor.

``Please read some American woman's literature in her memory,'' Dr.
Prendergast wrote.

Advertisement

\protect\hyperlink{after-bottom}{Continue reading the main story}

\hypertarget{site-index}{%
\subsection{Site Index}\label{site-index}}

\hypertarget{site-information-navigation}{%
\subsection{Site Information
Navigation}\label{site-information-navigation}}

\begin{itemize}
\tightlist
\item
  \href{https://help.nytimes3xbfgragh.onion/hc/en-us/articles/115014792127-Copyright-notice}{©~2020~The
  New York Times Company}
\end{itemize}

\begin{itemize}
\tightlist
\item
  \href{https://www.nytco.com/}{NYTCo}
\item
  \href{https://help.nytimes3xbfgragh.onion/hc/en-us/articles/115015385887-Contact-Us}{Contact
  Us}
\item
  \href{https://www.nytco.com/careers/}{Work with us}
\item
  \href{https://nytmediakit.com/}{Advertise}
\item
  \href{http://www.tbrandstudio.com/}{T Brand Studio}
\item
  \href{https://www.nytimes3xbfgragh.onion/privacy/cookie-policy\#how-do-i-manage-trackers}{Your
  Ad Choices}
\item
  \href{https://www.nytimes3xbfgragh.onion/privacy}{Privacy}
\item
  \href{https://help.nytimes3xbfgragh.onion/hc/en-us/articles/115014893428-Terms-of-service}{Terms
  of Service}
\item
  \href{https://help.nytimes3xbfgragh.onion/hc/en-us/articles/115014893968-Terms-of-sale}{Terms
  of Sale}
\item
  \href{https://spiderbites.nytimes3xbfgragh.onion}{Site Map}
\item
  \href{https://help.nytimes3xbfgragh.onion/hc/en-us}{Help}
\item
  \href{https://www.nytimes3xbfgragh.onion/subscription?campaignId=37WXW}{Subscriptions}
\end{itemize}
