Sections

SEARCH

\protect\hyperlink{site-content}{Skip to
content}\protect\hyperlink{site-index}{Skip to site index}

\href{https://www.nytimes3xbfgragh.onion/section/opinion/sunday}{Sunday
Review}

\href{https://myaccount.nytimes3xbfgragh.onion/auth/login?response_type=cookie\&client_id=vi}{}

\href{https://www.nytimes3xbfgragh.onion/section/todayspaper}{Today's
Paper}

\href{/section/opinion/sunday}{Sunday Review}\textbar{}The Last of the
Tiger Parents

\url{https://nyti.ms/2MLO0Qt}

\begin{itemize}
\item
\item
\item
\item
\item
\item
\end{itemize}

Advertisement

\protect\hyperlink{after-top}{Continue reading the main story}

Supported by

\protect\hyperlink{after-sponsor}{Continue reading the main story}

\href{/section/opinion}{Opinion}

\hypertarget{the-last-of-the-tiger-parents}{%
\section{The Last of the Tiger
Parents}\label{the-last-of-the-tiger-parents}}

By Ryan Park

Mr. Park is a lawyer~and father of two.

\begin{itemize}
\item
  June 22, 2018
\item
  \begin{itemize}
  \item
  \item
  \item
  \item
  \item
  \item
  \end{itemize}
\end{itemize}

\href{https://cn.nytimes3xbfgragh.onion/opinion/20180625/asian-american-tiger-parents/}{阅读简体中文版}\href{https://cn.nytimes3xbfgragh.onion/opinion/20180625/asian-american-tiger-parents/zh-hant}{閱讀繁體中文版}

\includegraphics{https://static01.graylady3jvrrxbe.onion/images/2018/06/24/opinion/sunday/24park/24park-articleLarge.jpg?quality=75\&auto=webp\&disable=upscale}

In first grade, I arrived at my suburban elementary school as a sort of
academic vaudeville trickster. My classmates stood speechless as I
absorbed thick tomes on medieval history, wrote and presented research
reports, and breezed through fifth-grade math problems like a bored
teenager.

My teachers anointed me a genius, but I knew the truth. My non-Asian
friends hadn't spent hours marching through the snow, reciting
multiplication tables. They hadn't stood at attention at the crack of
dawn reading the newspaper aloud, with each stumble earning a stinging
rebuke. Like a Navy SEAL thrown into a pool of raw conscripts, at 6, I
had spent much of my conscious life training for this moment.

To my authoritarian father, all has gone according to plan. I excelled
in school, attending Amherst College and Harvard Law School. I've
embraced his conventional vision of success: I'm a lawyer. But like many
second-generation immigrant overachievers, I've spent decades struggling
with the paradox of my upbringing. Were the same childhood experiences
that long evoked my resentment also responsible for my academic and
professional achievements? And if so, was the trade-off between
happiness and success worth it?

The way I and other Asian-Americans of my generation answer these
questions could affect American society more broadly. My generation's
academic success has sparked a crisis of sorts in our country's elite
educational institutions. For example, despite having the highest
poverty rate in New York City, Asian-Americans make up a large majority
of students at the city's premier public high schools ---
\href{http://schools.nyc.gov/SchoolPortals/02/M475/AboutUs/Statistics/register.htm}{including
73 percent at the storied Stuyvesant} --- where admission is decided
solely on the basis of a standardized test. Mayor Bill de Blasio has
reacted by proposing to scrap the test to allow more black and Hispanic
students to attend.

Meanwhile, Harvard faces a lawsuit claiming that the university
artificially caps the number of Asian students by emphasizing
non-merit-based factors in admissions, in the same way it deliberately
designed its admissions policies in the 1920s to limit Jewish
enrollment. Harvard itself has found that its share of incoming Asian
students would more than double, to nearly half the class, if it
considered only academic merit in deciding whom to admit.

Efforts to adjust these imbalances may or may not be warranted, but
history also suggests they may naturally abate on their own. If the
children of immigrants are often preternaturally driven, a phenomenon
known as ``second-generation advantage,'' the grandchildren of
immigrants usually experience ``third-generation decline.'' By the third
generation, families absorb American cultural values, lose the feverish
immigrant zeal to succeed and cease being, in any real sense, immigrants
at all.

I've experienced this transition myself, as I've started a family of my
own. When I became a parent, I felt the wonder and uncertainty that
accompany the awesome responsibility of fatherhood. But I was absolutely
sure of one thing: The childhood I devise for my two young daughters
will look nothing like mine. They will feel valued and supported. They
will know home as a place of joy and fun. They will never wonder whether
their father's love is conditioned on an unblemished report card.

I've assumed this means my daughters might someday bring home grades or
make life choices that my father would have regarded as failures. If so,
I embrace the decline.

\textbf{During my constitutional law class,} Akhil Amar --- the only
Asian-American professor I've ever had --- asked for a show of hands:
Whose parents immigrated to the United States after 1965? I joined all
the other Asian students in raising my hand, along with a few white
compatriots with hard-to-pronounce last names. As Mr. Amar explained,
our American story was made possible by the Immigration and Nationality
Act of 1965, a groundbreaking statute that washed away a century of
laws, like the frankly named Chinese Exclusion Act, aimed at making sure
people like us never became Americans.

In the decades that followed, a large wave of Asian immigrants arrived
in the United States. Like my parents, many of these new arrivals
brought two cultural values that would carry their children far: a
near-religious devotion to education as the key to social mobility and a
belief that academic achievement depends mostly on effort rather than
inborn ability. Many (though certainly not all, and probably less than
half) also came armed with the belief that the best way to instill these
values is through harsh methods that other Americans can regard as
cruel.

The results have been striking. Today, Asian-Americans fill the nation's
top universities in staggering numbers, enter elite professions like
medicine at incredible rates (nearly 20 percent of new doctors have
Asian roots) and generally do better in school and make more money than
any other demographic slice. Although overall trends
\href{https://www.americanprogress.org/issues/race/reports/2016/12/20/295359/wealth-inequality-among-asian-americans-greater-than-among-whites/}{mask
vast diversity within our community}, now 20 million strong, as a group
we've broken the curve on standard metrics of success.

Because of pre-1965 immigration restrictions, the third-generation
stories of most Asian-American families have yet to be written. Today,
many second-generation Americans like me are at a parenting crossroads:
Do we replicate the severe, controlling parenting styles many of us were
raised with --- methods that we often assume shaped our own success?

Amy Chua famously answered this question yes. In her memoir, ``Battle
Hymn of the Tiger Mother,'' she explained that her fanatical parenting
choices were driven by the desire to avoid ``family decline.'' But most
second-generation Asian-Americans are not joining her. Rather,
\href{https://onlinelibrary.wiley.com/doi/full/10.1111/famp.12052}{studies
show} that we're largely abandoning traditional Asian parenting styles
in favor of a modern, Western approach focused on developing open and
warm relationships with our children.

My wife is also a second-generation Asian-American overachiever (she's a
doctor, the other immigrant-parent-approved profession), and together
we're trying to instill in our daughters the same grit and reverence for
learning that our upbringings gave us, but in a happy and supportive
home environment. (In this effort, we've followed the example of her
parents, whose unfailing kindness is also common among Asian immigrants,
proving it's possible to have it both ways.) We've also adopted the
relationship-driven mind-set common among young parents today but not
among most immigrant parents, who emphasize discipline. For example,
before my oldest daughter was on an early-morning school schedule, I
freely indulged her disregard for bedtime on a condition: The night was
firmly earmarked for learning. We'd sometimes stay up past midnight,
lying on our stomachs with feet in the air, huddled over a dry-erase
board and a bowl of popcorn, practicing phonics or learning about sea
creatures. My own father, by contrast, strictly policed bedtime, angrily
shutting down my attempts to hide under the sheets with a book and a
flashlight.

Studies on second-generation parenting also show that many of us are
striving to cultivate individuality and autonomy in our children in a
way that we feel was missing from our own childhoods. As
\href{https://onlinelibrary.wiley.com/doi/pdf/10.1111/famp.12052}{the
respondent in one study} explained: ``As a young adult I really
struggled with what I wanted to do. I was always told that I would be a
doctor and so I never had a chance to really look outside of that and if
I did, it wasn't nurtured at all.'' With her own children, she said,
``we try to expose them to everything under the sun and then home in on
the things that excite them, what they like.''

The traditional Asian parenting model is, in theory at least, premised
on imposing pain now to reap meritocratic rewards later. For much of my
life, I accepted this premise and assumed there must be a trade-off
between inculcating academic success and happiness. But as I've learned
since becoming a parent, the
\href{https://www.ncbi.nlm.nih.gov/pmc/articles/PMC3641860/}{research
shows} that children tend to do best, across the board, when parents
command loving respect, not fearful obedience --- when they are both
strict and supportive, directive and kindhearted. By contrast, children
subjected to hostile ``tiger'' parenting methods are more likely to be
depressed, anxious and insecure. And while many tiger cubs run the
gantlet and emerge as academic gladiators, on average, children
subjected to high-pressure parenting actually tend to do worse in
school. In short, a firm hand works best when paired with a warm
embrace. This is the approach I've tried to take with my daughters.

Like all parents, however, my failures stack up alongside my successes.
And I know that the decision to abandon immigrant parenting principles
could backfire. The striving immigrant mind-set, however severe, can
produce results. Every time I snuggle my daughters as they back away
from a challenge --- when my own father would have screamed and spit and
spanked until I prevailed --- I wonder if I'm failing them in a very
different way than he did me.

But I'm temperamentally unable to mimic my father's succeed-at-all-costs
immigrant mind-set, an instinct I share with most of my generation. And
maybe that marks our immigrant parents' ultimate triumph: We have become
American. As part of the American parenting mainstream, I aim to raise
children who are happy, confident and kind --- and not necessarily as
driven, dutiful and successful as the model Asian child. If that means
the next generation will have fewer virtuoso violinists and
neurosurgeons, well, I still embrace the decline.

Advertisement

\protect\hyperlink{after-bottom}{Continue reading the main story}

\hypertarget{site-index}{%
\subsection{Site Index}\label{site-index}}

\hypertarget{site-information-navigation}{%
\subsection{Site Information
Navigation}\label{site-information-navigation}}

\begin{itemize}
\tightlist
\item
  \href{https://help.nytimes3xbfgragh.onion/hc/en-us/articles/115014792127-Copyright-notice}{©~2020~The
  New York Times Company}
\end{itemize}

\begin{itemize}
\tightlist
\item
  \href{https://www.nytco.com/}{NYTCo}
\item
  \href{https://help.nytimes3xbfgragh.onion/hc/en-us/articles/115015385887-Contact-Us}{Contact
  Us}
\item
  \href{https://www.nytco.com/careers/}{Work with us}
\item
  \href{https://nytmediakit.com/}{Advertise}
\item
  \href{http://www.tbrandstudio.com/}{T Brand Studio}
\item
  \href{https://www.nytimes3xbfgragh.onion/privacy/cookie-policy\#how-do-i-manage-trackers}{Your
  Ad Choices}
\item
  \href{https://www.nytimes3xbfgragh.onion/privacy}{Privacy}
\item
  \href{https://help.nytimes3xbfgragh.onion/hc/en-us/articles/115014893428-Terms-of-service}{Terms
  of Service}
\item
  \href{https://help.nytimes3xbfgragh.onion/hc/en-us/articles/115014893968-Terms-of-sale}{Terms
  of Sale}
\item
  \href{https://spiderbites.nytimes3xbfgragh.onion}{Site Map}
\item
  \href{https://help.nytimes3xbfgragh.onion/hc/en-us}{Help}
\item
  \href{https://www.nytimes3xbfgragh.onion/subscription?campaignId=37WXW}{Subscriptions}
\end{itemize}
