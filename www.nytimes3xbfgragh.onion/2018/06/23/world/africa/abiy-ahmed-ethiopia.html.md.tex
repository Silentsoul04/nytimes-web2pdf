Sections

SEARCH

\protect\hyperlink{site-content}{Skip to
content}\protect\hyperlink{site-index}{Skip to site index}

\href{https://www.nytimes3xbfgragh.onion/section/world/africa}{Africa}

\href{https://myaccount.nytimes3xbfgragh.onion/auth/login?response_type=cookie\&client_id=vi}{}

\href{https://www.nytimes3xbfgragh.onion/section/todayspaper}{Today's
Paper}

\href{/section/world/africa}{Africa}\textbar{}Ethiopia's New Leader
Raises Hopes. Now Comes the Hard Part.

\url{https://nyti.ms/2MQM1u5}

\begin{itemize}
\item
\item
\item
\item
\item
\end{itemize}

Advertisement

\protect\hyperlink{after-top}{Continue reading the main story}

Supported by

\protect\hyperlink{after-sponsor}{Continue reading the main story}

\hypertarget{ethiopias-new-leader-raises-hopes-now-comes-the-hard-part}{%
\section{Ethiopia's New Leader Raises Hopes. Now Comes the Hard
Part.}\label{ethiopias-new-leader-raises-hopes-now-comes-the-hard-part}}

\includegraphics{https://static01.graylady3jvrrxbe.onion/images/2018/06/24/world/24ethiopia1/merlin_136289829_0051c41b-1ca6-41a0-882d-74ae335332c9-articleLarge.jpg?quality=75\&auto=webp\&disable=upscale}

By \href{http://www.nytimes3xbfgragh.onion/by/somini-sengupta}{Somini
Sengupta}

\begin{itemize}
\item
  June 23, 2018
\item
  \begin{itemize}
  \item
  \item
  \item
  \item
  \item
  \end{itemize}
\end{itemize}

ADDIS ABABA, Ethiopia --- They call themselves a book club. Usually they
meet one Saturday a month, men and women mostly in their 20s and 30s, to
discuss a literary classic.

Today is unusual. Today, they have decided to discuss the story of their
country. Its protagonist: their prime minister, Abiy Ahmed, whose
\href{https://www.nytimes3xbfgragh.onion/2018/03/28/world/africa/ethiopia-prime-minister-oromo.html?action=click\&module=RelatedCoverage\&pgtype=Article\&region=Footer}{ascension
to the top post in late March} has pulled Ethiopia back from the brink
of a political implosion.

At 41, Mr. Abiy is one of the youngest leaders in all of Africa, itself
the continent with world's youngest population. And he is shaking up
some of the old ways of doing things.

Since taking office, Mr. Abiy has held town hall meetings around the
country and listened to what people had to say. He has apologized for
the killings of protesters by government forces and called for unity
among the country's many ethnic groups.

Perhaps most unusual of all, he has welcomed political differences of
opinion --- almost unheard of in a country where dissidents have often
been imprisoned. On Friday,
\href{https://twitter.com/fitsumaregaa/status/1010095287254372353}{his
office said on Twitter} that it would no longer block 264 websites,
blogs and television stations, many of them pro-opposition.

``He feels our pain,'' said one member of the book club, a university
lecturer named Mekonnen Mengesha, 33. ``Because he's our contemporary.
We have a generation gap with the old leaders.''

``It's refreshing,'' said Makda Getachew, 31, a public policy expert.

Not everyone is cheering the changes. On Saturday,
\href{https://www.nytimes3xbfgragh.onion/2018/06/23/world/africa/ethiopia-explosion-abiy.html}{someone
tossed a grenade into a rally for Mr. Abiy} in Addis Ababa, injuring
several people, according to officials. A spokesman said the prime
minister was ``safe.''

Ethiopia is Africa's second most populous country after Nigeria. And
even for Africa, it is astonishingly young. The median age of its 100
million people is 18.

Satisfying their demands --- both economic and political --- will be Mr.
Abiy's biggest test. Already, he has made some uncommon, politically
savvy moves.

He ordered the
\href{https://www.ft.com/content/d385d31a-6672-11e8-8cf3-0c230fa67aec}{lifting
of emergency rule} earlier than planned. It was imposed for the second
time in less than two years to control the widening, mostly youth-led
anti-government demonstrations that had been roiling the country.

He also
\href{https://www.reuters.com/article/us-ethiopia-politics/ethiopia-pardons-senior-opposition-leader-sentenced-to-death-idUSKCN1IR0AS}{pardoned
one of his country's most high-profile political prisoners}, a British
citizen named Andargachew Tsige who had been sentenced to death in
connection with his role in Ginbot 7, which the government regards as a
terrorist group.

\includegraphics{https://static01.graylady3jvrrxbe.onion/images/2018/06/24/world/24ethiopia2/merlin_136702386_afb7938b-eb37-4a68-baff-3efe95eced0c-articleLarge.jpg?quality=75\&auto=webp\&disable=upscale}

And after the pardon, Mr. Abiy
\href{https://twitter.com/malonebarry/status/1004073615179083776}{posed
for a picture} with Mr. Tsige, both men smiling at the camera.

It was classic Abiy: a big symbolic gesture but without specific steps
toward the things that critics have agitated for such as opening up
space for civil society activities or a national dialogue with
opposition groups.

``Prime Minister Abiy is the kind of guy who is good at saying the right
things to a domestic audience and giving the right gestures to
international development partners,'' said Tamrat Giorgis, editor of
Addis Fortune, an English language weekly paper.

What he has yet to see, Mr. Giorgis said, is whether those words and
gestures add up to a strategy of liberalization.

Still, Mr. Abiy did make a number of bold moves this month.

His government
\href{https://www.nytimes3xbfgragh.onion/2018/06/05/world/africa/ethiopia-eritrea-peace-deal.html?rref=collection\%2Fsectioncollection\%2Fworld\&action=click\&contentCollection=world\&region=rank\&module=package\&version=highlights\&contentPlacement=5\&pgtype=sectionfront}{said
it would honor a peace deal} to settle a bloody border dispute with
Ethiopia's neighbor and rival, Eritrea. This has the potential to end
Mr. Abiy's biggest national security headache.

The government also said it would
\href{https://www.bloomberg.com/news/articles/2018-06-05/ethiopia-moves-to-open-telecoms-airline-to-foreign-investors}{sell
stakes} in two of the biggest state-owned enterprises, opening the doors
for an infusion of cash to solve a foreign currency shortage. And Mr.
Abiy went to Cairo, a rare trip for an Ethiopian leader, in a bid to
ease tensions with Egypt over a hydropower dam his country is building
on the Nile.

I went to Ethiopia in May for the first time in nearly 15 years. In some
ways, it reminded me of India in the early 1990s, where I visited as a
child, when the Indian government, also faced with a foreign currency
crisis, cautiously began to open its economy to the world.

Inflation in Ethiopia is high, more than 11 percent, according to the
\href{http://www.imf.org/en/Countries/ETH\#countrydata}{International
Monetary Fund}. The country owes big debts to foreign creditors and the
currency crunch is so severe that ordinary Ethiopians say they sometimes
can't find basic medicines on the pharmacy shelves.

Addis Ababa, the capital, is on the cusp of change. Where it's headed,
however, is hard to tell.

The population has galloped in recent years to more than three million
inhabitants and the city resembles a construction site. Half-finished
buildings are everywhere. The airport is being revamped with the help of
Chinese investors. A new light rail network snakes across the city,
though Addis is swathed in darkness for hours when the power goes out.

Old Volkswagen Beetles share the streets with late-model Toyota sport
utility vehicles. At a karaoke bar, Coolio and Rihanna share time with
Oromia-language pop.

Buying a new SIM card for a cellphone takes hours at the offices of the
country's one provider --- Ethio Telecom. Credit cards are rarely
accepted in shops. And illegal movie downloads are common and ``guilt
free,'' as one consumer of American movies put it, because there is no
legal way to stream them.

The announced sale of stakes in the country's two most valuable assets
--- Ethio Telecom and Ethiopian Airlines --- could be a boon for foreign
investors, not least the cash-flush Chinese who have nurtured close
relations with the Ethiopian government.

But privatization does not mean opening the floodgates to private
competition --- at least not yet. Mr. Giorgis, the Addis Fortune editor,
described it as ``a desperate response made to the kind of macroeconomic
challenge the country finds itself in today.''

Image

A blast hit a rally in support of Prime Minister Abiy Ahmed in the
Ethiopian capital, Addis Ababa, on Saturday.Credit...EPA, via
Shutterstock

If the economy is the prime minister's most pressing challenge, the
peace deal with Eritrea is the riskiest, said Rashid Abdi, the Horn of
Africa director for the International Crisis Group. That is because it
could invite blowback against Mr. Abiy from the old guard of his party,
which he has unseated.

Beyond that, there is the broader risk of setting Ethiopians up for
disappointment.

``He has raised huge expectations through his bold policy pronouncements
and, inadvertently perhaps, by his rhetoric,'' Mr. Abdi said. ``Managing
them could prove daunting.''

Mr. Abiy is also different from his predecessors by virtue of who he is.
He is part Oromo, one of the country's largest ethnic groups. Its
members have long complained of being marginalized.

Mr. Abiy is no stranger to the Ethiopian establishment, though.

A former military officer, he came up through the ranks of the political
coalition that calls itself the Ethiopian People's Democratic
Revolutionary Front and has a virtual monopoly on power. The party
controls Parliament entirely, along with the justice system. It enjoys
the backing of a powerful military. Civil and political rights are
limited. Land is controlled by the government.

Mr. Abiy represents the younger, more reformist wing of the party,
though it has yet to carry out systematic political or economic reforms.
He has not said anything about negotiating with opposition groups inside
the country and abroad, for instance. And he has done little to assure
ordinary citizens that their institutions --- the police, the judiciary,
the press --- can be independent.

``Is it possible for him to deliver under this system?'' asked Mr.
Mengesha, the university lecturer.

Another member of the book club, Friat Weldekian, 25, praised the prime
minister for what she called his emotional intelligence. ``He touched on
all our issues, especially what the young think,'' she said. She wasn't
sure, however, whether he was doing so simply to quiet them down, like
giving a pacifier to a baby.

Ms. Weldekian used an old Amharic proverb that made the room burst into
laughter. ``Is it like fooling chickens into thinking they're bulls?''
she asked.

As they sipped tea and water and reviewed printouts of one of Mr. Abiy's
speeches, several book club members observed that the new prime minister
seems to steer clear of the stilted, anachronistic jargon of his
predecessors.

``He doesn't use those words,'' Ms. Getachew said. ``He acknowledges, I
think in a very genuine way, what people feel --- the failures of the
party, the failure of government. He really is an Ethiopian, very unity
focused. That's also very new for us.''

I spoke to Ms. Getachew again after the book club meeting. She pointed
out that the prime minister had yet to organize a national dialogue with
the opposition. Nor had he proposed a road map for political reforms
leading to the next elections.

``He is raising a lot of expectations,'' she said. ``At the end of the
day, he is still leading the country with the old party structure, the
old government structure, the old laws.''

Where their own story would go the members of the book club couldn't
say. The protagonist of this story was turning out to be as complicated
as the central figure in the last book they read: ``The Prince,'' by
Machiavelli.

Advertisement

\protect\hyperlink{after-bottom}{Continue reading the main story}

\hypertarget{site-index}{%
\subsection{Site Index}\label{site-index}}

\hypertarget{site-information-navigation}{%
\subsection{Site Information
Navigation}\label{site-information-navigation}}

\begin{itemize}
\tightlist
\item
  \href{https://help.nytimes3xbfgragh.onion/hc/en-us/articles/115014792127-Copyright-notice}{©~2020~The
  New York Times Company}
\end{itemize}

\begin{itemize}
\tightlist
\item
  \href{https://www.nytco.com/}{NYTCo}
\item
  \href{https://help.nytimes3xbfgragh.onion/hc/en-us/articles/115015385887-Contact-Us}{Contact
  Us}
\item
  \href{https://www.nytco.com/careers/}{Work with us}
\item
  \href{https://nytmediakit.com/}{Advertise}
\item
  \href{http://www.tbrandstudio.com/}{T Brand Studio}
\item
  \href{https://www.nytimes3xbfgragh.onion/privacy/cookie-policy\#how-do-i-manage-trackers}{Your
  Ad Choices}
\item
  \href{https://www.nytimes3xbfgragh.onion/privacy}{Privacy}
\item
  \href{https://help.nytimes3xbfgragh.onion/hc/en-us/articles/115014893428-Terms-of-service}{Terms
  of Service}
\item
  \href{https://help.nytimes3xbfgragh.onion/hc/en-us/articles/115014893968-Terms-of-sale}{Terms
  of Sale}
\item
  \href{https://spiderbites.nytimes3xbfgragh.onion}{Site Map}
\item
  \href{https://help.nytimes3xbfgragh.onion/hc/en-us}{Help}
\item
  \href{https://www.nytimes3xbfgragh.onion/subscription?campaignId=37WXW}{Subscriptions}
\end{itemize}
