Sections

SEARCH

\protect\hyperlink{site-content}{Skip to
content}\protect\hyperlink{site-index}{Skip to site index}

\href{/section/business}{Business}\textbar{}Inside Tesla's Audacious
Push to Reinvent the Way Cars Are Made

\url{https://nyti.ms/2lNkq0l}

\begin{itemize}
\item
\item
\item
\item
\item
\end{itemize}

\includegraphics{https://static01.graylady3jvrrxbe.onion/images/2018/07/01/business/01TESLA-1/merlin_139592340_2eb91484-b6bb-4303-ae28-838a4e8feda4-articleLarge.jpg?quality=75\&auto=webp\&disable=upscale}

\hypertarget{inside-teslas-audacious-push-to-reinvent-the-way-cars-are-made}{%
\section{Inside Tesla's Audacious Push to Reinvent the Way Cars Are
Made}\label{inside-teslas-audacious-push-to-reinvent-the-way-cars-are-made}}

Scrambling to turn out its first mass-market electric car, the automaker
set up multiple assembly lines and is changing production processes on
the fly.

Elon Musk, Tesla's chief executive, sees the success of the Model 3 as a
crucial step.Credit...Christie Hemm Klok for The New York Times

Supported by

\protect\hyperlink{after-sponsor}{Continue reading the main story}

By \href{https://www.nytimes3xbfgragh.onion/by/neal-e-boudette}{Neal E.
Boudette}

\begin{itemize}
\item
  June 30, 2018
\item
  \begin{itemize}
  \item
  \item
  \item
  \item
  \item
  \end{itemize}
\end{itemize}

FREMONT, Calif. --- Just outside the north wing of Tesla's sprawling
electric-car plant here, an unusual structure has taken shape in the
last few weeks: a tent, about 50 feet high and several hundred feet
long, its taut gray canvas membrane supported by aluminum columns.

Its purpose is as notable as its hasty construction. The semi-permanent
structure houses a third assembly line --- part of a desperate effort to
speed up production of the Model 3, the car that Elon Musk, Tesla's
chief executive, has said is critical to the company's financial health
and immediate future.

Just two years ago, Mr. Musk envisioned 2018 as a breakthrough moment.
Having established the brand's cachet with high-end offerings --- the
Model S luxury sedan and the Model X sport-utility vehicle --- Tesla
would begin churning out more affordable Model 3 sedans. With a
high-speed, high-tech assembly process, the company's sales would soar
more than fivefold, to half a million vehicles.

It hasn't turned out that way. Tesla had trouble mass-producing both
battery packs and cars. By the end of nearly three months of production
after Tesla started assembling the Model 3 last summer, just 260 had
rolled off the line, and Mr. Musk said the company faced a prolonged
period of ``manufacturing hell.'' He had hoped to produce 20,000 Model
3s a month by December, but a mere 2,425 were completed in the final
three months of 2017.

\includegraphics{https://static01.graylady3jvrrxbe.onion/images/2018/07/01/business/01TESLA-2/merlin_140495487_9bb2f27d-e0ea-495c-9d90-bfad051a945f-articleLarge.jpg?quality=75\&auto=webp\&disable=upscale}

Since then, Tesla has raced to iron out kinks in the assembly process,
mainly by scrapping some complicated robotic machines that proved ill
suited to certain tasks, and hiring hundreds of workers to replace them.
On the factory floor, it's a frantic race to reach Mr. Musk's goals, one
that has taken a toll on some employees. But if the gamble pays off, it
will be a big step toward Tesla's audacious ambitions: not just to be a
mass-market automaker, but also to reinvent the way autos are made.

``We believe in rapid evolution,'' Mr. Musk said in an interview. ``It's
like, find a way or make a way. If conventional thinking makes your
mission impossible, then unconventional thinking is necessary.''

And indeed, Mr. Musk is trying to do things that have never been done.
General Motors, Nissan, BMW, Ford and others have produced electric
cars, but have been unable to shrink costs enough to make them both
affordable and profitable. Mr. Musk, in contrast, has promised investors
and customers that Tesla will be able to produce the Model 3 in high
volumes, sell versions for as little as \$35,000 and ring up hefty
profits.

Once the Model 3 is rolling, Mr. Musk sees Tesla moving on to produce
electric vehicles of all shapes and sizes --- pickups, semitrailer
trucks, and a fast and roomy car for families called the Model Y. The
company's mission, Mr. Musk has said on numerous occasions, is to lead
the transition to emissions-free transportation and to change the world.

A recent daylong tour of the Fremont plant revealed how Tesla is trying
to break with standard auto-industry practices all along the Model 3
assembly lines. It is searching for ways to shorten the time that robots
take to weld parts. It is even making seats, a component most car
companies leave to specialized suppliers. And it is doing this while
trying to root out bottlenecks and glitches in the manufacturing
process.

In the final assembly area, for example, Tesla originally used robotic
arms to install the Model 3 seats. But the machinery was slow and
inconsistent in tightening the bolts that secure the seats and
connecting the wiring that supplies power to them. About a month ago,
company officials said, that work station was modified so that robots
move the seats into place and workers handle bolts and the fitting of
delicate electronic connectors.

Image

Robots can do only so much: Tesla aims to hire about 400 employees a
week to help accelerate Model 3 production.Credit...Christie Hemm Klok
for The New York Times

Mr. Musk doesn't have an office at the plant, but Tesla says he has been
sleeping there --- on the floor in someone else's office, or on a couch
--- while working to streamline Model 3 production. At 3 a.m. on
Thursday, the time Tesla made him available for a telephone interview,
he said he was trying to fix a glitch in the part of the plant where the
Model 3 is painted. ``The carrier that the car is on is coming out of
the paint booth slightly too fast for the sensor to recognize it, and
it's tripping the sensor even though everything is fine,'' he explained.

\href{https://www.nytimes3xbfgragh.onion/2020/07/02/business/tesla-sales-second-quarter.html}{Tesla}
engineers are trying to reprogram the sensor so it can operate at the
accelerated pace. For now, he said, ``we have somebody standing there
just pressing the `O.K.' button to restart it.''

The scramble to ramp up quickly has put a strain on the company. Several
senior executives, some involved in manufacturing, have left. While
investor optimism has remained high --- Tesla's market capitalization
puts it neck and neck with General Motors as the most valuable American
car company --- its bonds are rated in junk territory, and the delay in
bringing in revenue from Model 3 sales has analysts worried Tesla will
continue to use up cash and face the prospect of having to raise
additional capital later this year.

``At some point, investors are going to say, `If you don't have a viable
economic model, we're not going to continue to give you cash,''' Toni
Sacconaghi, an analyst at Sanford C. Bernstein \& Company, said in a
recent conference call with clients.

\hypertarget{the-perils-of-automation}{%
\subsection{The Perils of Automation}\label{the-perils-of-automation}}

Having made his fortune as a Silicon Valley entrepreneur --- including a
nine-figure sum from his early involvement in the online-payment service
PayPal --- Mr. Musk was convinced that technology and vision could
together conquer new frontiers, whether in space exploration (with his
SpaceX venture) or everyday transportation. Replacing gasoline tanks
with batteries was just a start. He was determined to reconceive their
production as well, based on 21st-century advances in automation.

Image

Delays have dogged the Model 3 assembly process. Nearly three months
after production started last summer, only 260 had been
built.Credit...Christie Hemm Klok for The New York Times

Established car companies master the process with assembly-line workers,
and then find ways for machines to take over some of the work. Tesla did
the opposite. It designed a highly automated production line populated
by more than a thousand robots and other assembly machines.

Ron Harbour, a partner at the consulting firm Oliver Wyman, noted that
in his annual survey rating auto plants worldwide, the most efficient
ones use a lot of manual labor. ``The most automated ones are at the
bottom of the list,'' he said.

In some instances, Tesla's gamble on automation has paid off. A separate
production line that makes the Model S and the Model X has a series of
14 stations, with 17 workers, in the area where battery packs and
electric motors are married to the vehicles' underbodies. For the Model
3, that function involves just five work stations and no workers at all,
said Lars Moravy, the company's director of chassis engineering.

For other tasks, the reliance on robots has proved a headache. For
months, Tesla engineers struggled to get a robot to guide a bolt through
a hole accurately to secure part of the rear brake. They found a
maddeningly simple solution: Instead of using a bolt with a flat tip on
its threaded end, engineers switched to a bolt with a tapered point,
known as a ``lead-in,'' that can be guided through the hole even if the
robot is a millimeter off dead center, Mr. Moravy said.

In a very tangible sense, Tesla views its production line as a
laboratory for untested techniques. In recent weeks, company executives
concluded they could produce Model 3 underbodies with fewer spot welds
than they had been using. The car is still held together by about 5,000
welds, but engineers concluded that some 300 were unnecessary and
reprogrammed robots to assemble the steel underbody without them.

Image

Car companies usually start with assembly-line workers, and then find
ways for machines to take over some of the work. Tesla did the opposite,
with highly automated production lines populated by more than a thousand
robots and other assembly machines.Credit...Christie Hemm Klok for The
New York Times

``It's unusual to be doing that at this point in time when the car is
already launched,'' said Mr. Harbour, a veteran manufacturing expert who
has visited most of the world's major auto plants. ``Normally you'd make
changes like that in the prototype stage.''

In another bid to push the limits of technology, Tesla at times pulls
robots off the line and tests them operating at speeds greater than
specified by the supplier, said Charles Mwangi, Tesla's director of body
engineering.

``We are actually breaking them to see what the maximum limit is,'' Mr.
Mwangi said. The idea is to find ways of accelerating production without
spending capital on new machinery. In the future, rather than adding
more machines to increase output, ``we can just dial up our equipment,''
he said.

The willingness to experiment with the production process even as cars
are rolling off the line is perhaps the most significant way Tesla is
defying the industry's conventional wisdom. Automakers like Toyota,
Honda and G.M. engineer manufacturing lines that can churn out cars or
trucks at a rate of about one a minute, and essentially lock in the
basic assembly process once they start production. While they make
tweaks to improve quality or worker safety, they generally make major
changes or introduce new techniques only every few years, when an old
model is phased out and before production of a new one begins.

``The first step in auto quality is stability,'' Mr. Harbour said.
``Once you get a stable process that works, you can go back and make
improvements.''

Tesla, by contrast, is tinkering with its production lines on the fly,
and the tent is a stark illustration of that approach.

Image

The Model 3's third assembly line, under a tent. ``I've never heard
anything like this, ever,'' said Ron Harbour, an expert on auto
plants.Credit...Justin Kaneps for The New York Times

Beneath the peaked canvas, Tesla has hastily set up a third Model 3
production line. Like the other two, it handles final assembly, when
trim and other finishing touches are put on the car. (Tesla did not
include the tent on the tour of the plant.)

Adding a new assembly line, even temporarily, is a rare and risky move
in the auto industry. A line set up hastily, in an untested environment,
might not achieve the quality Tesla promises.

Two assembly lines inside the plant already exist to handle at least
some of those tasks but they have proved troublesome and perform the
work more slowly than Mr. Musk had hoped, in part because Tesla used
robots for tasks that are better left to human workers.

Tesla engineering executives acknowledge that the company overestimated
the rate at which it could produce cars, and designed a production
system that proved to be too complicated --- a problem that Mr. Musk
lamented at the company's June shareholder meeting.

``One of the biggest mistakes we made was trying to automate things that
are super easy for a person to do, but super hard for a robot to do,''
he said. ``And when you see it, it looks super dumb. And you are like,
wow! Why did we do that?''

Image

Automakers typically rely on outside suppliers to make the seats for
their vehicles. Tesla makes its own seats.Credit...Christie Hemm Klok
for The New York Times

Most automakers operate a single line to make two, three or sometimes
four different vehicles, because using a second line would force them to
invest in duplicate tooling and cut into profit margins.

And a third assembly line, outside the walls of a plant? ``I've never
heard anything like this, ever,'' Mr. Harbour said.

Mr. Musk said the capital costs of the line in the tent were minimal
because the company used equipment it already owned. (On Twitter he had
called it ``scrap we had in warehouses.'')

``It does everything that the other assembly lines do but with fewer
people, lower labor costs and much higher uptime,'' he said. ``Our unit
cost for vehicles is lower on that line than on the other lines, and
we're seeing higher initial quality.''

Whether that remains the case as Tesla speeds up production will be
revealed in a few months if the company reports a profit, as Mr. Musk
has promised.

\hypertarget{a-constant-pressure-to-build}{%
\subsection{`A Constant Pressure to
Build'}\label{a-constant-pressure-to-build}}

For many years the Fremont plant was a joint venture of Toyota and
General Motors known as New United Motor Manufacturing Inc., or Nummi.
After G.M.'s bankruptcy, the factory closed in 2010 and the site was
acquired by Tesla.

Today, the four-million-square-foot plant alongside a busy freeway sees
a steady stream of deliveries and tractor-trailers departing with new
vehicles. Each afternoon, line workers in matching black pants with a
white Tesla logo on one leg stream out of the plant and descend upon
packed parking lots in the sprawling, bayside suburb.

Workers feel the pressure to speed output. In interviews away from the
plant, several said they had been putting in 10- and 12-hour days,
sometimes six days a week. They report that turnover among line workers
is high, and that sometimes supervisors join the line during extended
shifts.

Jose Moran, a five-year Tesla veteran who for the last 10 months has
worked as a Model 3 quality-team lead, said the already taxing
production demands of previous models had intensified. ``It's a constant
`How many cars have we built so far?' --- a constant pressure to build,
especially with the Model 3,'' he said. ``It gets desperate sometimes,
especially right now.''

One challenge that workers see is the rapid influx of new workers. The
company aims to hire about 400 employees a week to help accelerate Model
3 production. After Tesla's most recent earnings announcement in early
May, Mr. Musk said he hoped eventually to have three shifts a day,
essentially running the assembly line around the clock.

``Everyone I talk to has only been here for two weeks, a month, and
those people don't last long,'' said Jonathan Galescu, a Model X body
repair technician who has worked at the plant for four years.

Mr. Harbour, the manufacturing expert, says automakers typically give
new workers several weeks of training before putting them into
production work. Bringing in large numbers of new workers can hurt
quality because they may not perform their work properly or fail to
notice when problems crop up.

New workers at the Fremont plant get three days of training before going
to work on a production line. This includes a day of computerized
virtual training on doing their jobs safely, and a day of instruction on
the area to which they will be assigned.

Worker safety at the Fremont plant has come under scrutiny after a
nonprofit news organization, the Center for Investigative Reporting,
\href{https://www.revealnews.org/article/tesla-says-its-factory-is-safer-but-it-left-injuries-off-the-books/}{cataloged
a series of injuries} suffered by Tesla factory workers. California's
job safety watchdog is
\href{https://www.nytimes3xbfgragh.onion/2018/04/20/business/tesla-plant-safety.html}{investigating}
a recent incident that left a worker hospitalized with a broken jaw.

Image

A hydraulic press at the Fremont plant. The scramble to ramp up
production has put a strain on the company. Several senior executives,
some involved in manufacturing, have left.Credit...Christie Hemm Klok
for The New York Times

Michael Catura, a 33-year-old battery-pack line worker who has been with
Tesla for four years, said he had suffered hand, shoulder and elbow
injuries because the company had sometimes dispensed with rotating
workers to different jobs around the factory floor.

``We need to make sure people are thoroughly trained,'' he said, and not
just getting ``cookie-cutter training.''

Mr. Moran, Mr. Galescu and Mr. Catura are involved in efforts by the
United Automobile Workers --- a group reviled by Mr. Musk --- to
organize the plant.

Asked to comment on the intensity and the safety of Tesla's workplace, a
company spokeswoman said, ``We care deeply about the well-being of our
employees.'' Tesla's efforts reduced the injury rate by 25 percent last
year, she said, and ``with each passing month, we improve it further.''

\hypertarget{a-mind-blowing-amount}{%
\subsection{`A Mind-Blowing Amount'}\label{a-mind-blowing-amount}}

In early June, Mr. Musk said Tesla was making 3,500 Model 3 sedans a
week, and vowed to reach 5,000 a week by the end of June. In the
interview on Thursday, he voiced confidence that it was nearing that
elusive goal, the pace he has said is needed for the company to turn a
profit.

Tesla has already spent heavily on the Model 3 assembly process, and
modifications mean machinery purchased for hundreds of millions of
dollars is likely to be discarded. Mr. Musk essentially acknowledged
this point after the earnings announcement when he said he did not
expect the gross margin on the Model 3 --- the share of revenue retained
after the cost of goods sold --- to reach the 25 percent target until
early next year, six to nine months later than previously forecast.

Image

Mr. Musk reaffirmed his confidence that Tesla was nearing the elusive
goal of producing 5,000 Model 3 sedans a week, a pace he has said is
needed for the company to turn a profit.Credit...Christie Hemm Klok for
The New York Times

Max Warburton, an analyst at Sanford C. Bernstein, estimates that Tesla
spent about \$2 billion to set up the Model 3 production line. ``This is
vastly more than we've seen any other car company spend on new
capacity,'' he said, adding, ``\$2 billion is a mind-blowing amount to
spend on a second assembly line at an existing plant.''

For now, Tesla generates most of its revenue from the Model S and the
Model X, which are priced at about \$70,000 and up. Combined, their
global sales add up to about 100,000 vehicles a year --- too few to
offset the billions Tesla has been spending to build its gigantic
battery factory in Nevada, develop new cars and a semi truck, and equip
its car plant.

That means the company's future hinges on the assembly lines Tesla has
set up to produce the Model 3 --- and whether the company can make them
hum.

Advertisement

\protect\hyperlink{after-bottom}{Continue reading the main story}

\hypertarget{site-index}{%
\subsection{Site Index}\label{site-index}}

\hypertarget{site-information-navigation}{%
\subsection{Site Information
Navigation}\label{site-information-navigation}}

\begin{itemize}
\tightlist
\item
  \href{https://help.nytimes3xbfgragh.onion/hc/en-us/articles/115014792127-Copyright-notice}{©~2020~The
  New York Times Company}
\end{itemize}

\begin{itemize}
\tightlist
\item
  \href{https://www.nytco.com/}{NYTCo}
\item
  \href{https://help.nytimes3xbfgragh.onion/hc/en-us/articles/115015385887-Contact-Us}{Contact
  Us}
\item
  \href{https://www.nytco.com/careers/}{Work with us}
\item
  \href{https://nytmediakit.com/}{Advertise}
\item
  \href{http://www.tbrandstudio.com/}{T Brand Studio}
\item
  \href{https://www.nytimes3xbfgragh.onion/privacy/cookie-policy\#how-do-i-manage-trackers}{Your
  Ad Choices}
\item
  \href{https://www.nytimes3xbfgragh.onion/privacy}{Privacy}
\item
  \href{https://help.nytimes3xbfgragh.onion/hc/en-us/articles/115014893428-Terms-of-service}{Terms
  of Service}
\item
  \href{https://help.nytimes3xbfgragh.onion/hc/en-us/articles/115014893968-Terms-of-sale}{Terms
  of Sale}
\item
  \href{https://spiderbites.nytimes3xbfgragh.onion}{Site Map}
\item
  \href{https://help.nytimes3xbfgragh.onion/hc/en-us}{Help}
\item
  \href{https://www.nytimes3xbfgragh.onion/subscription?campaignId=37WXW}{Subscriptions}
\end{itemize}
