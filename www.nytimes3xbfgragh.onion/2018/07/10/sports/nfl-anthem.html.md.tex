Sections

SEARCH

\protect\hyperlink{site-content}{Skip to
content}\protect\hyperlink{site-index}{Skip to site index}

\href{https://www.nytimes3xbfgragh.onion/section/sports}{Sports}

\href{https://myaccount.nytimes3xbfgragh.onion/auth/login?response_type=cookie\&client_id=vi}{}

\href{https://www.nytimes3xbfgragh.onion/section/todayspaper}{Today's
Paper}

\href{/section/sports}{Sports}\textbar{}N.F.L. Players Association Files
Grievance Over Anthem Policy

\url{https://nyti.ms/2Jce83D}

\begin{itemize}
\item
\item
\item
\item
\item
\end{itemize}

Advertisement

\protect\hyperlink{after-top}{Continue reading the main story}

Supported by

\protect\hyperlink{after-sponsor}{Continue reading the main story}

\hypertarget{nfl-players-association-files-grievance-over-anthem-policy}{%
\section{N.F.L. Players Association Files Grievance Over Anthem
Policy}\label{nfl-players-association-files-grievance-over-anthem-policy}}

\includegraphics{https://static01.graylady3jvrrxbe.onion/images/2018/07/11/sports/11nf1l/merlin_139974147_ab8c3f5e-44c7-46e9-97c7-5e4ec7168513-articleLarge.jpg?quality=75\&auto=webp\&disable=upscale}

By \href{http://www.nytimes3xbfgragh.onion/by/ken-belson}{Ken Belson}

\begin{itemize}
\item
  July 10, 2018
\item
  \begin{itemize}
  \item
  \item
  \item
  \item
  \item
  \end{itemize}
\end{itemize}

The N.F.L. Players Association said Tuesday it filed a grievance against
the league for unilaterally changing its policy on the national anthem,
the latest salvo in the nearly two-year old controversy over whether
players should be allowed to kneel during the anthem.

In a statement, the union said the new policy was changed without
consulting the Players Association, something that ``is inconsistent
with the collective bargaining agreement and infringes on player
rights.''

The N.F.L. did not comment Tuesday.

At a meeting in late May, the 32 N.F.L. owners voted to overhaul their
protocol for what players must do during the anthem. In the past,
players were required to be on the sideline during the anthem, but they
were not required to stand. That policy dated back to 2009, when the
N.F.L. signed a marketing deal with the U.S. military.

Under the new policy, players can no longer kneel during the national
anthem without leaving themselves open to punishment. Also their teams
could face possible financial penalties. At the same time, players are
not required to be on the sideline during the playing of the anthem.
\href{https://www.nytimes3xbfgragh.onion/2018/05/23/sports/nfl-anthem-kneeling.html}{They
can remain in their locker room during the pregame ceremony.}

The union immediately objected to the revised policy, saying it was not
consulted.

The players have long contended they are not protesting the flag or the
anthem, but trying to raise awareness to important issues.

The union said it offered to begin confidential discussions with the
league before proceeding to potentially contentious and time-consuming
litigation. The league agreed, the union said.

If talks fail, the union will begin discovery and ultimately can present
its case to an independent arbitrator. It is unclear what remedies the
union is seeking.

The controversy dates back to August 2016 when Colin Kaepernick, then a
quarterback with the San Francisco 49ers, initially sat and then knelt
during the playing of the ``Star-Spangled Banner'' during preseason
games to draw attention to police brutality against African-Americans
and other instances of social injustice. He continued the demonstration
into the regular season, with other players on the 49ers and on other
teams later joining him.

Kaepernick became a free agent after the season, and his inability to
find a new club willing to sign him led to accusations that the owners
were punishing him because of his political views.
\href{https://www.nytimes3xbfgragh.onion/2017/12/08/sports/kaepernick-collusion.html}{Kaepernick
later filed a grievance} against the N.F.L. and the owners, accusing
them of colluding to keep him out of the league.

Last September, the issue exploded into a national controversy when
\href{https://www.nytimes3xbfgragh.onion/2017/09/23/sports/football/trump-nfl-kaepernick.html}{President
Trump said the owners should fire any players who did not stand for the
anthem}. Fans quickly took sides in a debate that in some ways had been
starting to subside. The league struggled to find an adequate response.
In a meeting held several weeks after the president first attacked the
league, a recording of which was obtained by The New York Times, the
owners talked openly about the threat to their businesses, and pleaded
with the players to stop protesting.
\href{https://www.nytimes3xbfgragh.onion/2018/04/25/sports/nfl-owners-kaepernick.html}{The
players said there was no way every player could be controlled or
persuaded not to protest.}

Soon after, the N.F.L. and a coalition of players announced a plan to
donate millions of dollars to groups addressing social injustice.

After the season, the owners began discussing how to tweak their anthem
policy to appease fans who sided with the President, while leaving room
for the players to continue to express themselves.

The new policy, though, makes clear that players could be fined if they
``do not stand and show respect for the flag and the anthem.''

``We want people to be respectful of the national anthem,'' Commissioner
Roger Goodell said at the meeting in May when the policy was changed.

With the start of the season less than two months away, it is unclear
how the union will be able to alter the policy.

There are areas of the policy, though, that could be clarified,
including what fines might be assessed.

Advertisement

\protect\hyperlink{after-bottom}{Continue reading the main story}

\hypertarget{site-index}{%
\subsection{Site Index}\label{site-index}}

\hypertarget{site-information-navigation}{%
\subsection{Site Information
Navigation}\label{site-information-navigation}}

\begin{itemize}
\tightlist
\item
  \href{https://help.nytimes3xbfgragh.onion/hc/en-us/articles/115014792127-Copyright-notice}{©~2020~The
  New York Times Company}
\end{itemize}

\begin{itemize}
\tightlist
\item
  \href{https://www.nytco.com/}{NYTCo}
\item
  \href{https://help.nytimes3xbfgragh.onion/hc/en-us/articles/115015385887-Contact-Us}{Contact
  Us}
\item
  \href{https://www.nytco.com/careers/}{Work with us}
\item
  \href{https://nytmediakit.com/}{Advertise}
\item
  \href{http://www.tbrandstudio.com/}{T Brand Studio}
\item
  \href{https://www.nytimes3xbfgragh.onion/privacy/cookie-policy\#how-do-i-manage-trackers}{Your
  Ad Choices}
\item
  \href{https://www.nytimes3xbfgragh.onion/privacy}{Privacy}
\item
  \href{https://help.nytimes3xbfgragh.onion/hc/en-us/articles/115014893428-Terms-of-service}{Terms
  of Service}
\item
  \href{https://help.nytimes3xbfgragh.onion/hc/en-us/articles/115014893968-Terms-of-sale}{Terms
  of Sale}
\item
  \href{https://spiderbites.nytimes3xbfgragh.onion}{Site Map}
\item
  \href{https://help.nytimes3xbfgragh.onion/hc/en-us}{Help}
\item
  \href{https://www.nytimes3xbfgragh.onion/subscription?campaignId=37WXW}{Subscriptions}
\end{itemize}
