Sections

SEARCH

\protect\hyperlink{site-content}{Skip to
content}\protect\hyperlink{site-index}{Skip to site index}

\href{https://www.nytimes3xbfgragh.onion/section/politics}{Politics}

\href{https://myaccount.nytimes3xbfgragh.onion/auth/login?response_type=cookie\&client_id=vi}{}

\href{https://www.nytimes3xbfgragh.onion/section/todayspaper}{Today's
Paper}

\href{/section/politics}{Politics}\textbar{}Trump Officials Slash Grants
That Help Consumers Get Obamacare

\url{https://nyti.ms/2NF4WZf}

\begin{itemize}
\item
\item
\item
\item
\item
\end{itemize}

Advertisement

\protect\hyperlink{after-top}{Continue reading the main story}

Supported by

\protect\hyperlink{after-sponsor}{Continue reading the main story}

\hypertarget{trump-officials-slash-grants-that-help-consumers-get-obamacare}{%
\section{Trump Officials Slash Grants That Help Consumers Get
Obamacare}\label{trump-officials-slash-grants-that-help-consumers-get-obamacare}}

\includegraphics{https://static01.graylady3jvrrxbe.onion/images/2018/07/11/us/11health_xp/11health_xp-articleLarge.jpg?quality=75\&auto=webp\&disable=upscale}

By \href{https://www.nytimes3xbfgragh.onion/by/robert-pear}{Robert Pear}

\begin{itemize}
\item
  July 10, 2018
\item
  \begin{itemize}
  \item
  \item
  \item
  \item
  \item
  \end{itemize}
\end{itemize}

WASHINGTON --- The Trump administration announced on Tuesday that it was
slashing grants to nonprofit organizations that help people obtain
health insurance under the Affordable Care Act, the latest step in an
escalating attack on the law that threatens to destabilize its insurance
markets.

The cuts are the second round in two years. The government will provide
\$10 million this fall, down from \$36 million last autumn and \$63
million in late 2016 --- a total reduction of more than 80 percent.

Trump administration officials said the insurance counselors, known as
navigators, did not enroll enough people to justify more spending.
Insurance agents and brokers do much better, they said.

The announcement on Tuesday, by Seema Verma, the administrator of the
Centers for Medicare and Medicaid Services, came three days after the
administration
\href{https://www.nytimes3xbfgragh.onion/2018/07/07/us/politics/trump-risk-adjustment-payments-obamacare.html}{suspended
a program that stabilizes health insurance markets} by paying billions
of dollars to insurers that enroll large numbers of unhealthy people
under the Affordable Care Act. Insurers said the freeze would cause
turmoil in insurance markets and drive up premiums.

The administration is not only cutting grants to navigators, but
fundamentally changing their mission. They will, for the first time,
help people enroll in health insurance plans that do not comply with the
consumer protection standards and other requirements of the Affordable
Care Act.

Since they began work in 2013, navigators have helped people enroll in
health plans that comply with the Affordable Care Act. Now the Trump
administration says they should also inform consumers of other options,
like
\href{https://www.nytimes3xbfgragh.onion/2018/06/19/us/politics/trump-affordable-care-act-health-insurance.html}{``association
health plans''} and short-term, limited-duration insurance.

Such plans do not have to provide the standard health benefits like
preventive services, maternity care or prescription drug coverage, but
administration officials say they will also be more affordable to
consumers.

``It's time for the navigator program to evolve, which is why we are
announcing a new direction for the program today,'' Ms. Verma said
Tuesday.

In each of the past two years, she said, navigators enrolled less than 1
percent of the people who signed up for coverage in the federal
marketplace. In the most recent enrollment period, about 8.7 million
people signed up for coverage in states using the federal marketplace,
the administration said.

Senator Ron Wyden of Oregon, the senior Democrat on the Finance
Committee, expressed outrage at the administration's effort to redefine
the purpose of the navigator program.

``This move amounts to federally-funded fraud --- paying groups to sell
unsuspecting Americans on junk plans,'' Mr. Wyden said.

Having failed to persuade Congress to repeal the Affordable Care Act,
the president is now engaged in a ``sabotage crusade'' to wreck the law,
Mr. Wyden said.

Fred Ammons, who supervises the Insure Georgia navigator organization,
said: ``This is a huge cut to navigator programs across the country. It
will virtually eliminate face-to-face in-person assistance. It means
less help, much less help, to underserved, hard-to-reach populations,
people who live in rural areas or have low literacy or don't speak
English as their primary language.''

The House Democratic leader, Nancy Pelosi of California, said, ``Yet
again the Trump administration is trying to trick Americans into buying
junk health insurance plans and making it harder for families to enroll
in real affordable, quality health coverage.''

President Trump declared last fall that the health law was ``dead'' and
``gone,'' but it has proved to be surprisingly durable and evidently
meets a significant need. Nationwide, in federal and state marketplaces,
11.8 million people signed up for coverage in the last open enrollment
period, down from 12.2 million in the prior year but substantially more
than many experts had predicted.

The Trump administration on Tuesday defended its decision to cut grants
to insurance counselors, saying consumers had many other ways to learn
about their options. It said, for example, that insurance companies had
``significantly increased their marketing and promotional spending.''

However, insurance companies typically push their own products, while
navigators are not supposed to favor or recommend a specific company or
product.

In addition, the administration said the insurance exchange was now ``an
established marketplace'' for people seeking coverage. ``Last year,'' it
said, ``we had our most cost-effective and successful open enrollment to
date. As the exchange has grown in visibility and become more familiar
to Americans seeking health insurance, the need for federally funded
navigators has diminished.''

Ms. Verma said grants to navigators would be based on their performance
in past years. Some, she said, had performed poorly.

In 2016-17, she said, 17 navigator groups enrolled fewer than 100 people
each, at an average cost of \$5,000 for each person enrolled.

By contrast, she said, agents and brokers accounted for more than 40
percent of enrollment in the federal exchange for the current year, and
the cost to the government, for training and technical assistance, was
just \$2.40 for each person enrolled.

Agents may receive commissions from insurance companies --- typically
modest payments for marketplace plans --- but navigators are generally
forbidden to accept compensation from insurers.

The Trump administration said it was also eliminating a requirement that
navigator groups have a physical presence in the areas they serve. This
would presumably allow federal grantees to provide aid by telephone or
through web portals, like online insurance brokers.

Navigators can help consumers fill out applications, complete
enrollments and renew coverage online, the administration explained.

Rachel Fleischer, the executive director of Young Invincibles, an
advocacy group for young adults, said she was dismayed by the cuts
announced on Tuesday. Research, she said, has shown the effectiveness of
in-person assistance provided to people shopping for health insurance, a
notoriously complicated product.

The cuts, she said, ``will result in far fewer in-person assisters and
huge swaths of the country lacking any in-person help.''

Advertisement

\protect\hyperlink{after-bottom}{Continue reading the main story}

\hypertarget{site-index}{%
\subsection{Site Index}\label{site-index}}

\hypertarget{site-information-navigation}{%
\subsection{Site Information
Navigation}\label{site-information-navigation}}

\begin{itemize}
\tightlist
\item
  \href{https://help.nytimes3xbfgragh.onion/hc/en-us/articles/115014792127-Copyright-notice}{©~2020~The
  New York Times Company}
\end{itemize}

\begin{itemize}
\tightlist
\item
  \href{https://www.nytco.com/}{NYTCo}
\item
  \href{https://help.nytimes3xbfgragh.onion/hc/en-us/articles/115015385887-Contact-Us}{Contact
  Us}
\item
  \href{https://www.nytco.com/careers/}{Work with us}
\item
  \href{https://nytmediakit.com/}{Advertise}
\item
  \href{http://www.tbrandstudio.com/}{T Brand Studio}
\item
  \href{https://www.nytimes3xbfgragh.onion/privacy/cookie-policy\#how-do-i-manage-trackers}{Your
  Ad Choices}
\item
  \href{https://www.nytimes3xbfgragh.onion/privacy}{Privacy}
\item
  \href{https://help.nytimes3xbfgragh.onion/hc/en-us/articles/115014893428-Terms-of-service}{Terms
  of Service}
\item
  \href{https://help.nytimes3xbfgragh.onion/hc/en-us/articles/115014893968-Terms-of-sale}{Terms
  of Sale}
\item
  \href{https://spiderbites.nytimes3xbfgragh.onion}{Site Map}
\item
  \href{https://help.nytimes3xbfgragh.onion/hc/en-us}{Help}
\item
  \href{https://www.nytimes3xbfgragh.onion/subscription?campaignId=37WXW}{Subscriptions}
\end{itemize}
