Sections

SEARCH

\protect\hyperlink{site-content}{Skip to
content}\protect\hyperlink{site-index}{Skip to site index}

\href{https://myaccount.nytimes3xbfgragh.onion/auth/login?response_type=cookie\&client_id=vi}{}

\href{https://www.nytimes3xbfgragh.onion/section/todayspaper}{Today's
Paper}

When It Comes to a Trade War, China Takes the Long View

\url{https://nyti.ms/2OQwdZd}

\begin{itemize}
\item
\item
\item
\item
\item
\end{itemize}

Advertisement

\protect\hyperlink{after-top}{Continue reading the main story}

Supported by

\protect\hyperlink{after-sponsor}{Continue reading the main story}

\href{/column/on-money}{On Money}

\hypertarget{when-it-comes-to-a-trade-war-china-takes-the-long-view}{%
\section{When It Comes to a Trade War, China Takes the Long
View}\label{when-it-comes-to-a-trade-war-china-takes-the-long-view}}

\includegraphics{https://static01.graylady3jvrrxbe.onion/images/2018/08/12/magazine/12OnMoney_1/12OnMoney_1-articleLarge-v2.jpg?quality=75\&auto=webp\&disable=upscale}

By Brook Larmer

\begin{itemize}
\item
  Aug. 7, 2018
\item
  \begin{itemize}
  \item
  \item
  \item
  \item
  \item
  \end{itemize}
\end{itemize}

\href{https://cn.nytimes3xbfgragh.onion/business/20180808/when-it-comes-to-a-trade-war-china-takes-the-long-view/}{阅读简体中文版}\href{https://cn.nytimes3xbfgragh.onion/business/20180808/when-it-comes-to-a-trade-war-china-takes-the-long-view/zh-hant/}{閱讀繁體中文版}

Ye Fangsu, a retired Shanghai schoolteacher, doesn't need any lessons
about the dangers of trade wars. For nearly 60 years, she has lived in
the former French Concession, a leafy part of Shanghai whose name itself
carries the humiliation of China's biggest trade war. The ``concession''
was one of many slices of territory, including Hong Kong and parts of
other port cities, that China was forced to hand over to foreign powers
after its defeat in the mid-19th-century Opium Wars. ``China was so weak
and backward then,'' Ye said, shaking her head as she offered me slices
of apple and pear on a hot July afternoon. ``We had to give in.''

The first salvos of today's trade war have barely been felt yet in
China. But for many Chinese, there's a sense of history repeating
itself. The Opium Wars, as every Chinese schoolchild is reminded, began
as a British attempt to pry open the Chinese market. Much as it does
today, China in the 17th and 18th centuries ran a huge trade surplus
with the West, exporting large quantities of tea, porcelain and silk but
importing little in return. (It balanced its current-account surplus by
buying up loads of Latin American silver; these days, Beijing has piled
up \$1.2 trillion in United States government securities.) By hooking
China on opium, British and American merchants redressed the trade
imbalance even as they weakened the country's social fabric. The Chinese
revolted, but they were no match for Western gunboats --- leading to the
unequal treaties that have fueled China's sense of historical grievance
and patriotic ambitions ever since.

Ye Fangsu was a teenager in 1949 when Mao Zedong's Communist troops
marched into Shanghai, the vanguard of a revolution that vowed to end
China's ``century of humiliation.'' Now 84 and widowed, Ye says she was
``angry'' when she learned from state-run media about the United States'
punitive trade tariffs on Chinese products. ``It just seems like the
foreigners are bullying us again,'' she told me. But this time will be
different, she said, pointing out that rather than panic or surrender,
China's leaders announced a reciprocal ``counterattack'' aimed at
products, like soybeans and pork, meant to hit the heart of President
Donald Trump's rural base. ``We've become strong now, and our leaders
are more tenacious. They won't back down.''

\textbf{American officials insist} that China started this conflict with
its long-term pattern of unfair trade practices: manipulating its
currency, raising barriers around its domestic industries, stealing
intellectual property from foreign companies. Even if Americans differ
on the wisdom of a trade war, there is widespread agreement that China
needs to be pushed to play by free-market rules. That rationale, though,
is hard to find behind China's Great Firewall, where censorship and
state-controlled media reign. For Chinese observers like Ye, then, the
American tariffs look like an unprovoked act of aggression against their
innocent homeland --- yet another Western attempt to contain China and
prevent its rise as a superpower.

It's easy to understand why Trump and his advisers believed China might
blink first. The country exports nearly four times more taxable goods to
the United States than it imports (\$505 billion compared with \$130
billion in 2017), so it could never keep up with Washington in a
tit-for-tat tariff war. Moreover, the Chinese economy is slowing,
weighed down by debt and unemployment and dependent on the American
market. Still, in targeting China's high-tech industry, Washington
seemed to forget that Beijing has more than a trillion dollars in
reserves to cushion the blow and that 21st-century supply chains are
truly global. The Peterson Institute for International Economics
calculates that 87 percent of the high-tech Chinese products hit by
United States tariffs get some of their parts or financing from
companies based outside China. American tariffs, in other words, could
inflict even greater pain on allies like South Korea, Japan and Taiwan
--- and on the United States itself. As the Ministry of Commerce
spokesman Gao Feng put it, ``The U.S. is opening fire on the world and
on itself, too.''

China's retaliatory tariffs have been less scattershot. By taking aim at
soybeans and pork, Beijing struck at Trump country in the Midwest, while
going after a smaller export like Kentucky bourbon hit the district of
the Senate majority leader, Mitch McConnell. ``China targets Trump's
interest groups with surgical precision,'' a former government official
in Shanghai told me. ``That's China's institutional advantage. It
understands America, but the United States doesn't really understand how
China operates politically.'' So far, the United States is feeling the
brunt of the trade war: Soy and pork exports to China have collapsed,
and China has scuppered one major tech deal: Qualcomm's proposed \$44
billion takeover of the Dutch chip maker NXP Semiconductors. Meanwhile,
in Shanghai, residents seem almost oblivious to the trade war, as if it
were a storm on a distant horizon. ``I don't know much about it,'' says
a waitress at a local dumpling shop. ``But the price of soy sauce hasn't
gone up yet.''

The United States would be wise not to underestimate China's resolve.
President Xi Jinping, the most powerful Chinese leader since Mao, has
staked his legitimacy on restoring China's greatness. Backing down is
not an option for him; he can't afford to look weak in a face-off with
China's biggest rival. Xi may hope the trade war eases up after the
midterm elections, but he seems ready to dig in. ``If it comes down to a
war of attrition, China will win,'' says James L. McGregor, chairman of
the greater China region for the consulting firm APCO Worldwide. ``At
the end of day, we may end up making China great again.''

\includegraphics{https://static01.graylady3jvrrxbe.onion/images/2018/08/12/magazine/12OnMoney_2/12OnMoney_2-articleLarge-v2.jpg?quality=75\&auto=webp\&disable=upscale}

Chinese leaders have so far been careful not to rouse much nationalist
fervor, however. In disputes with other countries --- France, Japan and,
most recently, South Korea --- Beijing ramped up the hostility,
orchestrated street protests and even encouraged consumer boycotts. The
trade war has evoked a more measured response. Some Chinese-language
articles online have highlighted China's historical grievances ---
``Never Forget the Eight Lessons of the Opium War,'' one reads --- but
the official tack has been to calm public opinion. ``Don't attack
Trump's vulgarity,'' a government guidance memo instructed media outlets
earlier this summer, according to the website China Digital Times.
``Don't make this a war of insults.''

If the trade war persists, China could make life even more difficult for
American companies by adding to the web of nontariff barriers (red tape,
for example) or by supporting a consumer boycott. American-owned
subsidiaries in China sold \$280 billion worth of goods and services
there in 2017, more than double the amount of United States imports, and
famous brands like Apple and Starbucks make about 20 percent of their
revenue in China. As vulnerable as those companies might be, China has
no interest, yet, in alienating its biggest market. ``At some point,
Chinese leaders may want to get nationalistic and stir up anger against
U.S. companies, but that's a dangerous road,'' McGregor says. ``Right
now, Chinese leaders are just studying the battlefield. They are not
running around with shock and awe.''

As the trade war escalates, in fact, China is seeking to shift the blame
and portray itself suddenly as a champion of openness. In the past
month, Beijing has approved two major deals --- with Tesla and the
German chemical giant BASF --- that give those companies what the
foreign business community has long desired: full ownership without the
need for a local joint-venture partner. Xi Jinping also held a rare
meeting with American and European business executives, though his words
on the trade war were defiant. In the West, he reportedly told them,
``you turn the other cheek. In our culture, we punch back.''

Back in Shanghai's former French Concession, Ye Fangsu walked me to the
gate of her lane. Across the street stands an old factory that has been
transformed into a cluster of hip new restaurants and marketplaces,
selling everything from Japanese tapas to French pastries. Two young
Chinese men with fancy fade haircuts parked their BMW and wandered in.
``China is rich now,'' Ye said. ``People's lives are better.'' Her
neighborhood, once a symbol of China's weakness, is now evidence of its
economic vitality. By her street-level calculus, that means China may
have the strength, this time, to stand up to the West.

Advertisement

\protect\hyperlink{after-bottom}{Continue reading the main story}

\hypertarget{site-index}{%
\subsection{Site Index}\label{site-index}}

\hypertarget{site-information-navigation}{%
\subsection{Site Information
Navigation}\label{site-information-navigation}}

\begin{itemize}
\tightlist
\item
  \href{https://help.nytimes3xbfgragh.onion/hc/en-us/articles/115014792127-Copyright-notice}{©~2020~The
  New York Times Company}
\end{itemize}

\begin{itemize}
\tightlist
\item
  \href{https://www.nytco.com/}{NYTCo}
\item
  \href{https://help.nytimes3xbfgragh.onion/hc/en-us/articles/115015385887-Contact-Us}{Contact
  Us}
\item
  \href{https://www.nytco.com/careers/}{Work with us}
\item
  \href{https://nytmediakit.com/}{Advertise}
\item
  \href{http://www.tbrandstudio.com/}{T Brand Studio}
\item
  \href{https://www.nytimes3xbfgragh.onion/privacy/cookie-policy\#how-do-i-manage-trackers}{Your
  Ad Choices}
\item
  \href{https://www.nytimes3xbfgragh.onion/privacy}{Privacy}
\item
  \href{https://help.nytimes3xbfgragh.onion/hc/en-us/articles/115014893428-Terms-of-service}{Terms
  of Service}
\item
  \href{https://help.nytimes3xbfgragh.onion/hc/en-us/articles/115014893968-Terms-of-sale}{Terms
  of Sale}
\item
  \href{https://spiderbites.nytimes3xbfgragh.onion}{Site Map}
\item
  \href{https://help.nytimes3xbfgragh.onion/hc/en-us}{Help}
\item
  \href{https://www.nytimes3xbfgragh.onion/subscription?campaignId=37WXW}{Subscriptions}
\end{itemize}
