How the Trump Administration Is Remaking the Courts

\url{https://nyti.ms/2nZlAXX}

\begin{itemize}
\item
\item
\item
\item
\item
\item
\end{itemize}

\includegraphics{https://static01.graylady3jvrrxbe.onion/images/2018/08/26/magazine/26mag-judges-image1/26mag-judges-image1-articleLarge-v4.png?quality=75\&auto=webp\&disable=upscale}

Sections

\protect\hyperlink{site-content}{Skip to
content}\protect\hyperlink{site-index}{Skip to site index}

Feature

\hypertarget{how-the-trump-administration-is-remaking-the-courts}{%
\section{How the Trump Administration Is Remaking the
Courts}\label{how-the-trump-administration-is-remaking-the-courts}}

Thanks to ruthless discipline --- and a plan long in the making --- the
G.O.P is carrying out a sweeping transformation of the federal
judiciary.

Credit...Illustration by Tracy Ma. Source photograph of chair from
Leathercraft Furniture.

Supported by

\protect\hyperlink{after-sponsor}{Continue reading the main story}

By Jason Zengerle

\begin{itemize}
\item
  Aug. 22, 2018
\item
  \begin{itemize}
  \item
  \item
  \item
  \item
  \item
  \item
  \end{itemize}
\end{itemize}

Donald F. McGahn, the White House counsel, stood in the gilded ballroom
of Washington's Mayflower Hotel last November to address the annual
meeting of the Federalist Society. He seemed humbled, even a bit awed to
be delivering the Barbara K. Olson Memorial lecture, named after the
conservative lawyer who died in the Sept. 11 attacks. Noting some of the
legal giants who gave the Olson lecture in years past, McGahn reflected,
``You hear names like Scalia, Roberts and Gorsuch and then me; one of
those names really is different than the rest.'' Unlike previous
speakers --- to say nothing of many of those to whom he was now speaking
--- McGahn, himself a member of the Federalist Society, hadn't attended
an Ivy League law school; he went to Widener University, a ``second
tier'' law school in Pennsylvania. He had never held a tenured
professorship or boasted an appellate practice, much less a judgeship,
that required him to think deeply about weighty constitutional issues;
he specialized in the comparably mundane and technical field of campaign
finance and election law. ``But here we are,'' McGahn said to the
audience, almost apologetically. In 2015, Donald Trump hired McGahn to
be the lawyer for his long-shot presidential campaign. Then, after Trump
shockingly won the election, he tapped McGahn, who had proved his talent
and loyalty during the campaign, to be White House counsel. Trump, in
other words, had made McGahn's wildest dreams come true. Now, McGahn
told the Federalist Society, Trump was going to make their wildest
dreams come true, too.

The Federalist Society for Law and Public Policy Studies, as it is
officially known, has played a crucial role in
\href{https://www.nytimes3xbfgragh.onion/2020/06/29/us/politics/supreme-court-trump-biden.html}{putting
conservative jurists on the bench}. As White House counsel, McGahn is
responsible for helping Trump select his judicial nominees. And, as he
explained in his speech that November afternoon, he had drawn up two
lists of potential judicial appointments. The first list consisted of
``mainstream folks, not a big paper trail, the kind of folks that will
get through the Senate and will make us feel good that we put some
pragmatic folks on the bench.'' The second list was made up of ``some
folks that are kind of too hot for prime time, the kind that would be
really hot in the Senate, probably people who have written a lot, we
really get a sense of their views --- the kind of people that make some
people nervous.'' The first list, McGahn said, Trump decided to ``throw
in the trash.'' The second list Trump resolved ``to put before the U.S.
Senate'' for a confirmation vote. The president, McGahn assured his
audience, was ``very committed to what we are committed to here, which
is nominating and appointing judges that are committed originalists and
textualists.''

As White House counsel, McGahn has exercised an unprecedented degree of
control over judicial appointments. In previous White Houses, both
Republican and Democrat, judicial nominations were typically
crowdsourced among officials from different parts of the administration.
Under George W. Bush, for instance, there was a judicial-selection
committee made up of people from the offices of the White House counsel,
political affairs and legislative affairs, as well as officials from the
Justice Department. This tended to produce a leveling effect. ``You
killed nominees by committee,'' says one Republican involved in judicial
confirmations. Under Trump, the job belongs exclusively to the White
House Counsel's Office, with McGahn and his deputy, Robert Luther, and
about 10 associate counsels identifying and then scrutinizing
candidates. This process is unique in White House history. Instead of
engaging in the typical legislative horse-trading for nominating judges
--- promising a senator, for instance, that the president will support
the nomination of the lawyer who served as the senator's
campaign-finance chairman in exchange for a yes vote on the
administration's agriculture bill --- the Trump White House has given
the counsel's office near-absolute authority. In a White House known for
chaos and dysfunction, the counsel's office, under McGahn, is generally
viewed as an island of competence. ``The White House is like a Dante's
`Inferno'-strange comedy,'' says one leading conservative lawyer who
requested anonymity for fear of reprisal, ``but the people in the
counsel's office are like the A-Team.'' That many of the lawyers in the
counsel's office are also Federalist Society members --- as elite
Republican lawyers today often are --- has given McGahn a handy rebuttal
to the complaint that Trump has outsourced his judicial-selection
process to the group. ``Frankly,'' McGahn has said, ``it seems like it's
been insourced.''

While Trump has lagged behind other presidents in political
appointments, the streamlining of the judicial-selection process has
helped him deliver a historic number of judges to the federal bench. In
2017, the Senate confirmed 12 of Trump's appeals court picks --- the
most for any president in his first year in office. This year, the
Senate has already confirmed 12 appellate judges and, according to a
Republican Judiciary Committee aide, hopes to confirm at least four
more. The White House refers to every new batch of judicial appointees
Trump selects as ``waves'' --- in early June, it announced the
``Fifteenth Wave of Judicial Nominees''--- as if they're soldiers
landing on the beaches of Normandy.

Trump's appointees have tended to be unusually well credentialed and
conservative. Republicans like to emphasize their academic and
professional bona fides --- the summa cum laudes, the Phi Beta Kappas,
the Supreme Court clerks --- and jokingly celebrate their ``deep bench''
of candidates. Democrats, for their part, prefer to focus on the
appointees' ideology. ``If someone had said or written something half as
controversial as these Trump nominees, they never would have been picked
by President George W. Bush,'' says Kristine Lucius, a former Democratic
Senate Judiciary Committee aide and now the executive vice president for
policy at the Leadership Conference on Civil and Human Rights. ``What
once would have been disqualifying'' --- a nominee's stated views on
contraception or gay rights or consumer protection --- ``is now
motivating this president.'' Or perhaps, the nominees' views are what's
motivating many conservatives to go along with Trump's presidency ---
which is what's motivating Trump.

When it comes to Trump's judicial appointments, the public has been
understandably focused on the Supreme Court, with first Neil Gorsuch and
now Brett Kavanaugh receiving most of the attention. When one of Trump's
lower-court nominees has managed to penetrate public consciousness, it
has usually been an outlier, like Brett Talley, whom Trump picked last
year for an Alabama Federal District Court judgeship. Talley, who had
never tried a case and whom the American Bar Association rated
unanimously ``not qualified,'' ultimately withdrew his nomination after
it was discovered that he was a member of a ghost-hunting group and had
apparently defended the honor of the early Ku Klux Klan on an Alabama
Crimson Tide football fan message board.

More representative of Trump's judicial appointees are judges like James
C. Ho. Born in Taiwan, Ho moved to the United States as a toddler. He
graduated from Stanford and the University of Chicago law school before
going on to clerk for Clarence Thomas at the Supreme Court. After
working in George W. Bush's Justice Department, he succeeded Ted Cruz as
Texas solicitor general. Ho is as pure a product as exists of the
conservative legal movement created by the Federalist Society. Last
October, Trump nominated Ho to the United States Court of Appeals for
the Fifth Circuit. In December, he was confirmed by the Senate. And in
April, Ho issued his first opinion --- a blistering dissent in a
campaign-finance case after a Fifth Circuit appellate panel ruled, 12 to
2, that the City of Austin, Tex., could prohibit individuals from
donating more than \$350 per election to municipal candidates. Ho used
his dissent not only to voice his disapproval of campaign-finance laws
but also to criticize those that regulated gun purchases and protected
abortion; he even threw in a swipe, in a citation, at the Supreme
Court's Obamacare ruling. Lamenting a government that has grown so large
that it ``would be unrecognizable to our founders,'' Ho wrote: ``If
there is too much money in politics, it's because there's too much
government. The size and scope of government makes such spending
essential.''

To be sure, Ho's was a dissenting opinion, but what so cheered members
of the conservative legal movement is that it was likely the first of
many, because Ho is only 45. And because there will be more and more
judges like Ho on the federal bench, it's only a matter of time before
such opinions will no longer be dissents. Indeed, after just 18 months,
Trump has ``flipped'' two circuits --- the Sixth and Seventh --- from
what Trump's supporters in the conservative legal movement consider
``liberal'' to more properly conservative. Two more --- the Eighth and
the 11th --- are on the verge of tipping. Even circuits that are
decidedly liberal are undergoing significant changes. ``It'll be really
important for the Second and the Ninth Circuits to have between two and
four really good, high-octane intellectual conservative jurists,''
explains a person close to the judicial-nominations process, ``because
dissents provide a signaling function to the U.S. Supreme Court, and
those are very important circuits.''

In short, a radically new federal judiciary could be with us long after
Trump is gone. Brian Fallon, a veteran Democratic operative who leads
Demand Justice, a group formed to help Democrats with research and
communications in the judicial wars, says, ``We can win back the House
this November, we can defeat Trump in 2020 and we'll still be dealing
with the lingering effects of Trumpism for the next 30 or 40 years
because of the young Trump-appointed judges.''

And if Trump is re-elected? Newt Gingrich, who during the 2016 campaign
began emphasizing the importance of judges to Trump, posits: ``He could,
by the end of his time in office, be the most important president since
Franklin Delano Roosevelt in shaping the judiciary.''

\textbf{Like most members} of the Republican establishment, Leonard Leo,
the executive vice president of the Federalist Society, was initially
skeptical --- if not contemptuous --- of Trump's political aspirations.
A few months before Trump announced his presidential bid in 2015, Sam
Nunberg, then a political adviser to Trump's campaign, tried to arrange
a sit-down between his boss and Leo. ``I told Leonard, `Mr. Trump is a
conservative now on these issues, and you're not going to believe how
good he is going to be for you,' '' Nunberg recalls. ``He sat there like
regular D.C. and listened, and who the hell knows what he said after I
left. He probably said, `Gee, I feel bad for that kid.' '' The meeting
between Trump and Leo never happened.

But by the time the Republican presidential primaries began in early
2016, Leo's thinking about Trump had evolved. Trump, by then, had
established himself as a plausible candidate, maybe even a front-runner.
That March, Leo was part of a small group of Washington Republicans,
including Gingrich and the Heritage Foundation president at the time,
Jim DeMint, who met with Trump for a lunch at the Jones Day law firm,
where McGahn was a partner. When the meal was over, Leo, McGahn and
Trump broke off from the larger group for a private meeting. Reaching
into his suit-jacket pocket, Leo presented Trump with a list of
potential Supreme Court nominees that McGahn had asked him to bring ---
a combination of federal judges and State Supreme Court justices who Leo
believed would be suitable successors to Justice Antonin Scalia, who
died the previous month.

``I was really hoping for 12,'' Trump told him.

``Well, you've got eight,'' Leo replied.

``Can't we find more?'' Trump asked.

``We can try,'' Leo pledged.

Trump, Leo recalls, had a question about the State Supreme Court
justices on Leo's list: Do they make ``the final decision''? Leo
explained that sometimes they do and sometimes they don't, depending on
whether the issue before them involves the United States Constitution.
He asked why Trump wanted to know. ``Because when you have to make a
final decision, and it is the real final decision, you own it,'' Trump
explained. ``And like a businessman, when you own something that you do,
you take it very seriously, and it has consequences.''

\includegraphics{https://static01.graylady3jvrrxbe.onion/images/2018/08/26/magazine/26mag-judges-image2/26mag-26judges-t_CA1-articleLarge.jpg?quality=75\&auto=webp\&disable=upscale}

``It was an interesting insight that I hadn't really thought of,'' Leo
told me. We were sitting at Morton's, the venerable Washington
steakhouse, where Leo, an owlish man in his 50s who wears a pocket
watch, keeps a wine locker. ``It was, I think, the best conversation I
have ever had in my professional life with a Republican presidential
candidate on the issue of judges,'' he marveled.

It's hard to take that claim at face value. In 2016 alone, Leo discussed
judges with Jeb Bush --- who, as governor of Florida, appointed more
than 100 of them --- and Ted Cruz, who clerked for Chief Justice William
H. Rehnquist, argued nine cases in front of the Supreme Court and
himself is a member of the Federalist Society. A mere two weeks after
Trump shared his insights about the judiciary with Leo, he told an
interviewer that he planned to appoint Supreme Court justices who
``would look very seriously at her'' --- Hillary Clinton's --- ``email
disaster.'' A few months after that, Trump promised a group of
congressional Republicans that he would protect ``Article XII'' of the
Constitution --- an article that doesn't exist.

Trump might not have known much about the law, but he needed, as
Gingrich told me, to create the impression that he ``would be reliable
in terms of conservative judges, because that would calm down and
consolidate a very large bloc of his coalition.'' That is, what mattered
to the Federalist Society --- and the Heritage Foundation --- was that
Trump take their advice on judicial nominees. In an interview with
Breitbart in June 2016, Trump pledged, ``We're going to have great
judges, conservative, all picked by Federalist Society.''

Leo also figured out what mattered to Trump. ``Leonard is smart,'' says
David Lat, the founding editor of the influential legal website Above
the Law and a former Federalist Society member. ``He knows the way to
Trump's heart is through his ego.'' And perhaps his pocketbook. In May,
McClatchy reported that a mysterious \$1 million donation to Trump's
inaugural committee in December 2016 --- made by a company called BH
Group L.L.C. that was apparently formed to hide the source of the
donation --- was tied to Leo. (When I asked him about the McClatchy
report, Leo declined to comment.) ``Leonard had an instinct,'' Gingrich
says, ``that this could be the great opportunity to redevelop
conservatism on the courts.''

\textbf{The Federalist Society} was founded in 1982 by a small cadre of
conservative law students at Yale and the University of Chicago. Its
first faculty advisers were Robert H. Bork at Yale and Antonin Scalia at
Chicago. The group quickly spread to other campuses, and within a few
years it had received an infusion of cash from conservative donors,
including the Koch brothers. Ever since then, if you were a law student
with conservative leanings, it was more than likely you became a
Federalist Society member and were absorbed into a sprawling network of
law school chapters, practice groups, publications and seminars that
could nurture you for your entire career. Today, the Federalist Society
boasts more than 70,000 members.

For most of the organization's first three decades, its dominant
philosophical emphasis was on judicial restraint: the idea that judges
shouldn't overrule majority-passed, democratically enacted laws --- that
they shouldn't, as Amanda Hollis-Brusky, a Pomona College professor and
the author of the 2015 book ``Ideas With Consequences: The Federalist
Society and the Conservative Counterrevolution,'' puts it, ``move the
law too far, too fast.'' This philosophy emerged largely as a reaction
to liberal rulings by the Warren and Burger courts --- as well as those
of lower-court judges --- who, conservatives complained, tried to
``legislate from the bench'' on civil rights and civil liberties. But
within the Federalist Society and the larger conservative legal
movement, there was an emerging faction that favored a more aggressive
approach. These libertarian legal theorists, led by the Georgetown law
professor Randy Barnett, subscribed to the judicial philosophies of
originalism and textualism, which hold that judges should interpret the
Constitution according to the meaning of its plain text, instead of its
intent or purpose, and, more important, should not hesitate to overturn
any law that deviates from that text.

Originalists and textualists gave intellectual and theoretical ballast
to this approach in the academy. But they didn't achieve critical mass
in the larger conservative legal movement until 2012, when the Supreme
Court upheld the Affordable Care Act, with Chief Justice John G. Roberts
Jr. writing the majority opinion. ``Conservatives were so disappointed
they had to stop and think, How did this happen?'' says Barnett, who
helped mount the challenge to the constitutionality of Obamacare by
invoking the commerce clause. ``And how did it happen at the hands of a
chief justice who was a Bush appointee and who had been signed off on by
the Federalist Society?'' The conservative legal movement's long-held
devotion to judicial restraint began to founder. ``Now the situation has
reversed itself,'' Barnett told me. ``The originalism side, and
invalidating laws if they're unconstitutional, has the upper hand.''

Critics of the Federalist Society contend that the group actually favors
judicial activism: Judges who will take a stance on social issues,
particularly on abortion. Many of the group's members question the legal
basis for Roe v. Wade and whether a right to privacy exists in the
Constitution, as Roe held it does. Leo was hailed by the conservative
legal activist and writer Ed Whelan in National Review in 2016 for being
``more dedicated to the enterprise of building a Supreme Court that will
overturn Roe v. Wade'' than anyone else in the United States. Yet Leo
accuses Democrats of ``scare tactics'' when they charge that Trump seeks
to appoint judges who will outlaw abortion. Similarly, Trump, who at one
point during the 2016 campaign pledged that Roe would be overturned
because he would put ``pro-life justices on the court,'' maintains that
he did not bring up the topic of abortion in his interviews with Brett
Kavanaugh.

The Federalist Society --- and the Trump administration --- are more
forthright about the ways in which they hope originalism and textualism
may apply to other arenas, particularly government regulation. ``The
greatest threat to the rule of law in our modern society is the
ever-expanding regulatory state,'' McGahn declared in his November
speech to the Federalist Society, ``and the most effective bulwark
against that threat is a strong judiciary.'' He added, ``Regulatory
reform and judicial selection are so deeply connected.'' This idea is
now at the heart of the Federalist Society, whose members believe that
federal agencies have become an unaccountable ``fourth branch'' of
government --- and that their bureaucrats, oftentimes experts in their
fields, should no longer be shown any deference by the courts in how
they apply laws enacted by Congress, but should instead be restrained
from doing anything beyond what the law, as Congress wrote it,
stipulates. The originalists and textualists now favored by the
Federalist Society and the Trump administration are decidedly
disinclined to defer to executive-branch agencies, whether it's the
Environmental Protection Agency or the Food and Drug Administration or
the Occupational Safety and Health Administration, when it comes to
interpreting arguably (and often necessarily) ambiguous statutes about
the environment or public health or workplace safety. Unless Congress
explicitly mandates it, originalists and textualists believe, agencies
can't do it.

Gorsuch is said to have risen to the top of Trump's Supreme Court list
in large part because of a 2016 concurring opinion he wrote as a judge
on the United States Court of Appeals for the 10th Circuit, in which he
forcefully attacked what's known as ``Chevron deference'' --- a term
that stems from a 1984 Supreme Court case, Chevron U.S.A. Inc. v.
Natural Resources Defense Council Inc., that instructed courts to grant
policymaking flexibility to government agencies. Similarly, Don Willett,
whom Trump appointed to the United States Court of Appeals for the Fifth
Circuit and who is on the president's Supreme Court list, became a
Federalist Society favorite largely because of a 2015 concurring opinion
he wrote as a justice on the Texas Supreme Court. In Patel v. Texas
Department of Licensing and Regulation, Willett struck down a state
licensing requirement that mandated 750 hours of training for eyebrow
threaders, denouncing what he described as a ``nonsensical government
encroachment'' on ``occupational freedom'' and economic liberty. And
when the White House rolled out Kavanaugh's nomination to the Supreme
Court in early July, it circulated a memo to business groups that,
according to Politico, praised the 75 times Kavanaugh, as a Court of
Appeals judge, overruled federal regulators on cases involving issues
like clean air, consumer protection and net neutrality in order to
protect ``American businesses from illegal job-killing regulation.''
``The Federalist Society has embraced judicial activism,'' Hollis-Brusky
says. ``They're just calling it a different name.''

The ascendant wing of the Federalist Society has, according to critics,
effectively managed to change how Washington operates, by shifting power
away from the executive and legislative branches and toward the courts.
It also represents something of a long-term strategy by the Republican
Party. ``By appointing judges who'll narrowly interpret congressional
regulations and statutes,'' Hollis-Brusky says, ``you're gambling that
you won't be in power politically but that your judges will be on the
bench and take a more active role in shaping laws over the next 30
years.''

The appellate courts are especially important in this effort. Although
the Supreme Court is the highest court in the land, its caseload, which
was not huge to begin with, has become even smaller in recent years ---
declining from about 150 cases a term in 1980 to just 79 in the term
that ended in June. The appeals courts, by contrast, collectively hear
and decide thousands of cases each year. ``The Courts of Appeals are the
regional Supreme Courts of the nation,'' says Sheldon Goldman, a
University of Massachusetts at Amherst professor and a scholar of the
American judiciary, ``and are of greater importance, in many respects,
than the Supreme Court.'' Nan Aron, the president of the liberal
judicial group Alliance for Justice, says liberals don't always
recognize the centrality of the appeals courts. ``They're making the law
of the land in critically important areas,'' she says. ``It's something
that's not lost on the Republicans. That's why they have their eyes on
the prize.''

And never have they been as focused as they have been under Trump, who,
according to Randy Barnett, ``has made as good a selection of judges as
any Republican president in my lifetime.'' Even Leo, who has enjoyed
unrivaled influence for more than two decades, seems impressed by his
--- and the conservative legal movement's --- good fortunes in the Trump
era. ``This administration,'' he said that afternoon at Morton's, ``is
trying to hit as many triples and home runs as possible.''

\textbf{When Trump Took office,} he inherited not just an open Supreme
Court seat but 107 additional judicial vacancies. Ronald Reagan, by
contrast, had 35 unfilled judgeships; Obama had 54. ``There's a million
qualified conservative lawyers out there,'' says J.Scott Jennings, a
Republican strategist close to Senator Mitch McConnell, the majority
leader. ``The hard part was securing the vacancies and actually having a
place to put them all. That was the spade work done by Mitch McConnell
in the Obama years.''

From the moment Obama entered the White House, McConnell led Senate
Republicans in a disciplined, sustained, at times underhanded campaign
to deny the Democratic president the opportunity to appoint federal
judges. McConnell's first move came six weeks after Obama's
inauguration, in the form of a letter, signed by all 41 Republican
senators, which warned the new president that if he did not consult with
--- and, more crucial, receive the approval of --- home-state senators
for his judicial nominees, then the Republicans would filibuster,
insisting on 60 votes to end debate. ``They were very clear from the
beginning that they were going to make this as difficult and as partisan
as possible,'' says Christopher Kang, who worked on judicial nominations
in the Obama White House Counsel's Office.

Republican approval would be conveyed by one of the Senate's many
cherished procedural instruments known as a ``blue slip.'' The blue slip
is literally a slip of paper that a senator returns to the Judiciary
Committee signaling that a given nominee in his or her state should
receive a hearing. First introduced in 1917, the blue slip has been
accorded varying weights by different Judiciary Committee chairmen, but
when Obama took office, the committee's chairman was Senator Patrick
Leahy, a Vermont Democrat. Leahy, a strong institutionalist and
protector of the Senate's prerogatives, viewed the blue slip as
something akin to a Holy Writ. If a home-state senator, a Republican or
a Democrat, did not want a judicial nominee to have a hearing, Leahy
would not schedule one --- essentially putting a hold on the nomination.
In doing so, Leahy told me, he was giving ``real meaning to `advise and
consent' '' and ensuring that the Senate kept ``its institutional
independence and didn't become a rubber stamp.''

Even without Leahy's strict blue-slip policy, Obama would probably still
have sought Republican approval for his judges. He was less interested
in making the judiciary more liberal than in making it more diverse. It
was important, Obama once told The New Yorker, for minorities ``to see
folks in robes that look like them.'' Surely, there were
African-American, Latino, Asian, gay and female judges with moderate
records and temperaments whom Republican senators could support. ``We
were in the business of picking judges,'' says Michael Zubrensky, a
former Department of Justice official who worked on judicial nominations
in the Obama administration, ``not picking fights.'' The fundamental
battles Obama wanted to wage with Republicans involved legislation, not
a long game with the courts.

When Republican senators dragged out their consultations with the Obama
administration on judicial nominees, the Obama White House did not press
them; when a blue slip was finally returned and the Judiciary Committee
held a hearing and voted a nominee out of committee, Senate Democrats
would often take their time before scheduling a floor vote, which
Republicans would usually insist couldn't be held until after the
maximum 30 hours of debate. ``Judges at that time were sort of an
afterthought,'' recalls Brian Fallon, who was then an aide to Senator
Chuck Schumer of New York. ``It was viewed as something you got around
to scheduling when you were in between big pieces of legislation and you
needed some filler on the floor.''

The combination of Republican intransigence --- ``You had all of these
nominees piling up on the calendar,'' Kang says, ``because Republicans
felt any day without an Obama judge in place was a better day for them''
--- and Democratic dawdling meant that even though Democrats enjoyed a
Senate majority, Obama, several months into his second term, had more
than 60 unfilled judicial vacancies that lacked even a nominee. ``In
retrospect, I think we all didn't react with enough alarm when it was
happening,'' Fallon says. ``We indulged it for far too long in the Obama
years, and now our chickens are coming home to roost.''

The issue came to a head in the fall of 2013 over three vacancies in
particular --- all of them on the United States Court of Appeals for the
District of Columbia Circuit. Senate Republicans had been unable to
prevent the nominees from receiving a hearing, because there were no
home-state senators to withhold blue slips. But when the Senate's
majority leader at the time, Harry Reid, brought each of the three
nominees to the floor, Republican senators, who by then numbered 45,
refused to give them a vote. The Republicans didn't object to the
nominees themselves; all three were considered moderate and eminently
qualified. Rather, the Republicans argued that the District of Columbia
Circuit's caseload was so meager that the judgeships should be
eliminated. Reid decided to invoke what was known as the ``nuclear
option,'' doing away with filibusters for most nominations by
presidents, including those to the lower courts.

With the filibuster for lower-court judicial nominations eliminated,
Obama was able to score more than 100 judicial confirmations in just
over a year. Then and now, Republicans denounced Reid's triggering of
the nuclear option. ``You'll regret this,'' Mitch McConnell warned in
2013, ``and you may regret this a lot sooner than you think.'' But with
Leahy leading the Judiciary Committee, Republican senators, even without
the filibuster, still maintained some leverage --- and whenever
possible, they used their blue slips to bottle up Obama's judicial
nominations. Saxby Chambliss and Johnny Isakson, both Georgia
Republicans, held up Jill Pryor's nomination to the United States Court
of Appeals for the 11th Circuit for two years, finally returning their
blue slips only after Obama agreed to nominate one of their picks to
another 11th Circuit vacancy and three of their picks to district court
judgeships. Democrats might have played hardball with the nuclear
option, but they still shrank from a fight, either because they agreed
with Leahy that the blue slip was an important procedural safeguard or
because they didn't have the stomach to pressure him to change his
stance.

Republicans weren't as squeamish. After taking back the Senate in the
2014 midterm elections, McConnell, now the majority leader, began a near
blockade of Obama's judicial appointments. In Reagan's final two years
in office, 66 of his district court and 15 of his appeals court nominees
were confirmed. Clinton managed 57 and 13 in his last two years. George
W. Bush had 58 and 10. In Obama's final two years, 18 of his district
nominees and just one of his appellate court nominees were confirmed ---
the lowest number since Harry Truman was president.

\textbf{One of the earliest,} and ultimately most prolonged, battles in
McConnell's fight for Republican control of the judiciary began when a
routine vacancy opened up on the United States Court of Appeals for the
Seventh Circuit in Wisconsin in 2010. Wisconsin had long been considered
a model of bipartisanship in filling vacancies in the federal judiciary.
In 1979, the state's senators, William Proxmire and Gaylord Nelson,
established a commission to depoliticize the judicial-selection process.
Known as Wisconsin's Federal Nominating Commission, it was composed of
11 legal experts --- some appointed by the senators, others by the state
bar --- and led by the dean of the law school at either Marquette or the
University of Wisconsin. It solicited applications, vetted candidates
and ultimately came up with a list of four to six individuals that the
senators would review and forward to the White House for consideration.

Although Proxmire and Nelson were both Democrats and a Democrat was
president when the commission was created, the process also worked in
times of divided government. In 2003, when Senate Democrats and George
W. Bush were battling over his judicial nominees, Wisconsin's two
Democratic senators, Herb Kohl and Russell Feingold, both supported
Diane Sykes, Bush's choice for a Seventh Circuit seat. ``There are a
number of topics on which we do not see eye to eye,'' Feingold said in
introducing Sykes at her Judiciary Committee hearing, but her
``nomination is the result of a collaborative bipartisan process.''
(Today, Sykes is on Trump's list of Supreme Court candidates.)

To fill the Seventh Circuit vacancy in 2010, the commission recommended
six candidates to replace Terence Evans, who was taking senior status.
Kohl and Feingold forwarded the names to Obama for consideration. In
July, Obama nominated Victoria Nourse, a University of Wisconsin law
professor and former Senate Judiciary Committee staff member for Joe
Biden. She was a fairly typical Obama nominee, in that she was a woman
(42 percent of Obama's judges were women, the highest percentage of any
president) and a moderate (besides working for Biden, Nourse worked on
the Judiciary Committee with Orrin Hatch, a Utah Republican). She was
also typical in that the Obama administration and Senate Democrats
didn't seem to afford her nomination much urgency. At the end of 2010,
Nourse still had not been scheduled for a confirmation hearing.

This would prove to be a problem when, in November of that year,
Feingold lost his re-election campaign to Ron Johnson, a Republican
businessman and Tea Party candidate. Although Feingold had returned his
blue slip for Nourse, as well as for a nominee to a district court,
Johnson, upon joining the Senate, essentially took them back. In
Wisconsin, legal experts chalked up Johnson's move to inexperience.
``Everybody just assumed that once he got up to speed, he would see
these were quality nominees and would support them,'' says Michelle
Behnke, a former state bar president who served on the commission. But
Johnson was simply adhering to the Washington Republican playbook,
outlined by McConnell in his original letter about judges to Obama two
years earlier. Nourse says she repeatedly sought a meeting with Johnson
so he could review her credentials, but he rebuffed her entreaties and
refused to return her blue slip. In early 2012, after 18 months of
waiting, Nourse withdrew her nomination. Johnson, meanwhile, said he
wanted Wisconsin to come up with a new system for recommending judges,
but he and Kohl couldn't agree on what it would look like. The Obama
administration declined to nominate anyone until they could.

In 2012, Representative Tammy Baldwin, a Democrat, was elected to
succeed Kohl in the Senate. One of her first acts was to meet with
Johnson to start filling Wisconsin's judicial vacancies. Johnson
insisted on a new nominating commission with a different structure. No
longer would the state bar or the state's law school deans participate,
nor would a senator whose party was in the White House be able to
appoint more members. Instead, the commission would consist of six
members, three appointed by each senator. Under the commission's new
rules, a judicial candidate needed five votes to be recommended for a
judgeship; the commission was required to recommend four to six
candidates. Baldwin agreed to Johnson's stipulations, and in April 2013,
the commission was formed.

The commission began soliciting applications for the Court of Appeals
for the Seventh Circuit seat in 2014, which by then had been vacant for
more than four years. By the end of 2014, the commission had reviewed
numerous applications for the position and interviewed eight candidates,
but only two, a Madison lawyer named Don Schott and a Milwaukee Circuit
Court judge named Rick Sankovitz, received the requisite five votes.
Months later, when the commission was still at an impasse, Baldwin sent
the White House the names of the eight candidates interviewed. Johnson
and his commissioners cried foul. ``In our view, there's no sense in
which they were finalists,'' says Rick Esenberg, one of Johnson's
commissioners.

Rather than take advantage of the deadlock to nominate a ``finalist'' of
whom the commission's Republican members disapproved, the Obama
administration sought to defuse tensions and quickly settled on Schott,
one of two candidates who had received five votes, as its likely
nominee. In late July 2015, the White House initiated Schott's F.B.I.
background investigation and American Bar Association evaluation, which
were concluded by early September. It stood to reason that Johnson would
support Schott's nomination --- after all, he had been recommended by
the commission Johnson established --- but the White House wanted to
make sure before it went ahead with it. Johnson dragged his feet. First,
according to a former government official familiar with the process,
Johnson asked to see Schott's F.B.I. file. ``I don't remember any other
instance in which a Senate office made that request'' before a
nomination was made, the official says. When the Obama administration
wouldn't give Johnson Schott's F.B.I. file, Johnson insisted on
interviewing Schott himself. But then he didn't schedule an interview.
Finally, Schott flew to Washington in early November to meet with
Johnson. In January 2016, six years after the Seventh Circuit seat
became vacant, Johnson told the White House that Schott was acceptable,
and Schott was quickly nominated.

Then came another round of delays. First, Johnson didn't return his blue
slip for Schott until March. Next, Charles E. Grassley, the Iowa
Republican and chairman of the Judiciary Committee, stalled on
scheduling a confirmation hearing. Finally, in May, Schott was given a
hearing. Baldwin appeared to introduce him and speak on his behalf to
the committee. Johnson did not attend. Although he had returned his blue
slip, he refused to offer Schott any support. ``I have recommended the
committee consider it,'' Johnson explained. ``What I am not going to do
is publicly go out and make any other statements beyond that.''
Nonetheless, in June, the Judiciary Committee voted 13 to 7 to advance
Schott's nomination. By now, however, there were only five months until
the presidential election, and with McConnell already refusing to give
Merrick Garland a hearing for his Supreme Court nomination, it seemed
unlikely that he would schedule a floor vote for Schott. The White House
and Baldwin pressed him; Johnson did nothing. In November, when Trump
was elected and Johnson was re-elected, Schott still hadn't been given a
vote. His nomination was dead.

\textbf{Even before Trump} was sworn in as president, Don McGahn,
Leonard Leo and other members of Trump's transition team began vetting
potential judicial candidates to fill all the empty seats on the bench.
Together with McConnell, McGahn and transition officials devised a
strategy to speed confirmations through the Senate: Trump would
prioritize appellate judges, rather than district court ones, and
initially fill vacancies from states with two Republican senators or
from states with Democratic senators that had been won by Trump. Like
George W. Bush, Trump wouldn't allow the American Bar Association to vet
potential judges before they were nominated. But under the new way of
business, the Senate wouldn't necessarily wait for the association to
complete its vetting before the nominees were given hearings. The
Judiciary Committee would also more regularly take the unusual step of
holding confirmation hearings for two appellate nominees at a time.

The most crucial procedural maneuver, however, involved the blue slip.
When Grassley became chairman of the Judiciary Committee in 2015 after
Republicans took back the Senate, he publicly indicated (in a column in
The Des Moines Register) and privately said (in a conversation with
Leahy, according to Leahy) that he would afford blue slips the same
weight as his Democratic predecessor. If both home-state senators didn't
return their blue slips, the nominee wouldn't receive a hearing. Sure
enough, in Obama's final two years in the White House, Grassley denied
hearings to four appellate court and five district court nominees who
didn't receive blue slips. (A Grassley spokesman maintains that the
nominees didn't receive hearings ``solely'' because of unreturned blue
slips.) Although this benefited Republicans, it was viewed not as
partisan but as principled; Grassley, in his sixth term and having just
turned 80, was, like Leahy, an avowed institutionalist.

Then, last November, 10 months into Trump's presidency, Grassley took to
the Senate floor to announce that he had a new, more nuanced view of the
blue slip. While he would be ``less likely'' to grant a hearing to a
district court nominee who didn't have support of their home-state
senators, he would no longer allow a home-state senator to ``wield veto
power'' over appellate nominees. ``Circuit courts cover multiple
states,'' Grassley said. ``There's less reason to defer to the views of
a single state's senator for such nominees.''

Grassley's decision to disregard blue slips worries Leahy, who contends
that it eviscerates the Senate's ``advise and consent'' role. If a
home-state senator can no longer put a hold on a nominee for an
appellate judgeship, what's to stop a president from nominating a judge
who isn't even from that state? ``Depending on who's in the majority and
who's president,'' Leahy told me, ``they might decide, `We've got to
make this Texas court a little bit different, so we have this New Yorker
who'd make a good judge down there,' and nominate them and get them
confirmed.'' As to whether he regrets adhering to that principle, Leahy
said: ``The only regret I have is that the Republicans haven't stuck to
the position they claimed was the right position when there was a
Democratic president. I'm old-school. I believe in senators sticking to
their commitments.''

Grassley maintains that he still takes the Senate's advise-and-consent
role seriously. ``My blue-slip policy is consistent with its traditional
application as a way of promoting consultation with home-state
senators,'' he told me in a statement. ``I find it a bit ironic that the
same senators who opted to change the Senate's filibuster rule in 2013
to silence the voices of 41 senators are now calling for the ability of
a single senator to obstruct the Senate's mere consideration of judicial
nominees.'' According to a Republican Judiciary Committee aide, Grassley
has required that McGahn show him consultation logs --- a ticktock of
every communication the White House Counsel's Office has with home-state
senators about judicial nominations --- to be assured that there's
meaningful home-state consultation. But as Senate Democrats note,
``meaningful consultation'' is in the eye of the beholder. ``We've never
relied on a chairman's view of whether a senator was consulted before,''
one Democratic aide says. ``It was up to the senator about whether they
were consulted.'' Democrats charge that Grassley is not really concerned
about the opinions of home-state senators. ``If it's what Donald Trump
wants, they're going to go along with it,'' Leahy says. ``That seems to
be the standard.''

While Grassley and the White House have sought the input of some
Democratic senators on judicial nominations from their states ---
notably those in the Democratic leadership, like Chuck Schumer, Richard
Durbin and Dianne Feinstein, the ranking Democrat on the Judiciary
Committee --- they have, more often than not, steamrolled the rest of
them.

When Senator Al Franken of Minnesota announced in September that he
wouldn't return his blue slip for David Stras, Trump's first choice for
a seat on the Court of Appeals for the Eighth Circuit, McGahn informed
him that Stras would be nominated anyway. Then Grassley held a hearing
for Stras, clearing the way for his confirmation. (Stras was confirmed
in January, after Franken resigned; Franken's successor, Tina Smith,
voted against him.)

Similarly, Oregon's Democratic senators, Jeff Merkley and Ron Wyden,
both refused to return their blue slips for Ryan Bounds, Trump's nominee
for a judgeship on the Court of Appeals for the Ninth Circuit. Bounds
had concealed newspaper columns he wrote as a Stanford undergraduate in
the 1990s, in which he railed against ``race-focused groups'' on campus
and likened the university's multicultural efforts to Nazi Germany.
Nonetheless, Bounds received a Judiciary Committee hearing in June and
was voted out of committee on an 11-to-10 party-line vote. His
nomination was withdrawn minutes before a floor vote, when Senator Tim
Scott of South Carolina, the only African-American Republican in the
Senate, announced that he wouldn't vote for Bounds on account of those
columns.

During the transition, Trump's advisers turned their attention to
Wisconsin's Court of Appeals for the Seventh Circuit vacancy. Leonard
Leo and several others recommended Mike Brennan for the spot. A
Milwaukee lawyer and a founding member of that city's Federalist Society
chapter, Brennan had been the chairman of an advisory committee that
helped Gov. Scott Walker select his own state-level judges, many of whom
had won plaudits from conservatives. In March 2017, seven weeks into
Trump's presidency, the White House Counsel's Office interviewed Brennan
for the appellate judgeship. This was somewhat awkward, because Johnson
and Baldwin had intended to use their beleaguered commission to help
fill the vacancy. Johnson prevailed upon the White House to hold off on
making a nomination until the commission could review candidates, and in
April it began accepting applications.

Brennan applied. So, surprisingly, did Schott. According to those
familiar with the commission's deliberations, which are confidential,
Schott initially received the same five votes he did two years earlier,
but when the Republican commissioners realized that Brennan had fallen
short of the required five votes --- receiving just four --- two
commissioners changed their votes, and Schott finished with just three.
If in 2014 the commission was able to give Johnson and Baldwin only two
names, this time it came up with zero. In June, Schott was invited to
Washington to interview with the White House Counsel's Office. Five
weeks later, Trump nominated Brennan for the vacancy.

Johnson hailed the move, but Baldwin cried foul, noting that the
commission hadn't recommended any candidates for the Seventh Circuit,
nor did she believe that she had been sufficiently consulted by the
White House Counsel's Office about Brennan. She said she wouldn't return
her blue slip. But under Grassley's new blue-slip policy, that didn't
matter. In January, the Judiciary Committee held a confirmation hearing
for Brennan. Two weeks later, the committee approved Brennan's
nomination 11 to 10 on a party-line vote. And in May, Brennan's
nomination came before the full Senate, which approved it 49 to 46.
After 3,044 days of sitting vacant, Terence Evans's Seventh Circuit seat
was finally filled.

\textbf{It remains difficult} to parse Trump's own legal views. In the
wake of Antonin Scalia's death during the Republican presidential
primaries, he apparently became close with Scalia's widow, and he now
views the Supreme Court justice as a judicial role model. Perhaps in
Scalia, Trump saw something of himself: a Queens kid who likes to mix it
up. ``Whether or not he gets every nuance of textualism and
originalism,'' says one Republican lawyer who has been involved in the
Trump administration's judicial-selection process, ``he gets that Scalia
was courageous. He probably liked that Scalia was pugilistic in
public.''

Leo, McGahn and others have done a remarkable job of persuading the
president that their intellectual judicial philosophy of originalism and
textualism is in perfect sync with his visceral preferences that judges
be ``courageous'' and ``not weak.'' It's easy to see how, in Trump's
mind, declaring war on ``the administrative state'' might dovetail
neatly with his desire to go after the ``deep state.'' ``A lot of the
things that make Trump so loathsome as a person and a politician,''
David Lat notes, ``are why he's been nominating judges who are such
great conservatives.''

So far, Trump appears to be pleased with the decisions of the judges he
has appointed. In June, after the Supreme Court decided 5 to 4, with
Gorsuch in the majority, that the president does have the authority to
ban travelers from certain majority-Muslim countries, reversing two
lower-court rulings, Trump tweeted: ``SUPREME COURT UPHOLDS TRUMP TRAVEL
BAN. Wow!''

Yet despite Trump's record on judicial appointments, some in the
conservative legal movement remain uneasy. ``He's been great, and
everything's good,'' one prominent conservative legal activist says,
``but what happens if the Senate goes 50-50? What happens if Don McGahn
gets replaced by Judge Napolitano?'' Or what happens when an originalist
judge does something that goes against Trump? When Neil Gorsuch was
awaiting confirmation to the Supreme Court, he told a Democratic senator
that Trump's attacks on the federal judge who temporarily blocked his
travel ban were ``demoralizing.'' According to The Washington Post,
Trump contemplated withdrawing his nomination because Gorsuch was not
``loyal.'' But after reading Gorsuch's note thanking him for the
nomination --- ``Your address to Congress was magnificent,'' Gorsuch
wrote --- he decided to stay the course.

For the moment, Trump may believe that originalism and textualism cut in
his favor, but there is no guarantee this will always be the case. While
a president with an intellectual commitment to originalism and
textualism would most likely be philosophical about a ruling from a
like-minded judge that runs counter to his political or personal
interests, this doesn't describe Trump. He would almost certainly
interpret such a ruling as evidence of a judge's ``cowardice'' and
``weakness.'' Yet Trump can't simply fire the offending judge the way he
fires a secretary of state; these are lifetime appointments and thus,
unlike so many others whom Trump has ushered into power, judges are
protected from his capriciousness. Earlier this year, McConnell, looking
back on Trump's achievements in 2017, noted that the tax bill was
``hugely important,'' but added that once Democrats took back control of
the White House or Congress, they would revisit the tax code. By
contrast, he said, ``the thing that will last the longest is the
courts'' --- whether Trump ultimately wants them to or not.

Advertisement

\protect\hyperlink{after-bottom}{Continue reading the main story}

\hypertarget{site-index}{%
\subsection{Site Index}\label{site-index}}

\hypertarget{site-information-navigation}{%
\subsection{Site Information
Navigation}\label{site-information-navigation}}

\begin{itemize}
\tightlist
\item
  \href{https://help.nytimes3xbfgragh.onion/hc/en-us/articles/115014792127-Copyright-notice}{©~2020~The
  New York Times Company}
\end{itemize}

\begin{itemize}
\tightlist
\item
  \href{https://www.nytco.com/}{NYTCo}
\item
  \href{https://help.nytimes3xbfgragh.onion/hc/en-us/articles/115015385887-Contact-Us}{Contact
  Us}
\item
  \href{https://www.nytco.com/careers/}{Work with us}
\item
  \href{https://nytmediakit.com/}{Advertise}
\item
  \href{http://www.tbrandstudio.com/}{T Brand Studio}
\item
  \href{https://www.nytimes3xbfgragh.onion/privacy/cookie-policy\#how-do-i-manage-trackers}{Your
  Ad Choices}
\item
  \href{https://www.nytimes3xbfgragh.onion/privacy}{Privacy}
\item
  \href{https://help.nytimes3xbfgragh.onion/hc/en-us/articles/115014893428-Terms-of-service}{Terms
  of Service}
\item
  \href{https://help.nytimes3xbfgragh.onion/hc/en-us/articles/115014893968-Terms-of-sale}{Terms
  of Sale}
\item
  \href{https://spiderbites.nytimes3xbfgragh.onion}{Site Map}
\item
  \href{https://help.nytimes3xbfgragh.onion/hc/en-us}{Help}
\item
  \href{https://www.nytimes3xbfgragh.onion/subscription?campaignId=37WXW}{Subscriptions}
\end{itemize}
