Sections

SEARCH

\protect\hyperlink{site-content}{Skip to
content}\protect\hyperlink{site-index}{Skip to site index}

\href{https://myaccount.nytimes3xbfgragh.onion/auth/login?response_type=cookie\&client_id=vi}{}

\href{https://www.nytimes3xbfgragh.onion/section/todayspaper}{Today's
Paper}

There's Less to Portraits Than Meets the Eye, and More

\url{https://nyti.ms/2BFe529}

\begin{itemize}
\item
\item
\item
\item
\item
\end{itemize}

Advertisement

\protect\hyperlink{after-top}{Continue reading the main story}

Supported by

\protect\hyperlink{after-sponsor}{Continue reading the main story}

\href{/column/on-photography}{On Photography}

\hypertarget{theres-less-to-portraits-than-meets-the-eye-and-more}{%
\section{There's Less to Portraits Than Meets the Eye, and
More}\label{theres-less-to-portraits-than-meets-the-eye-and-more}}

\includegraphics{https://static01.graylady3jvrrxbe.onion/images/2018/08/26/magazine/26mag-onphoto-slide-TPQC/26mag-onphoto-slide-TPQC-articleLarge.png?quality=75\&auto=webp\&disable=upscale}

By \href{https://www.nytimes3xbfgragh.onion/by/teju-cole}{Teju Cole}

\begin{itemize}
\item
  Aug. 23, 2018
\item
  \begin{itemize}
  \item
  \item
  \item
  \item
  \item
  \end{itemize}
\end{itemize}

Portraiture existed long before photography was invented. And for more
than a dozen years after photography's invention, it was practically
impossible to make a photographic portrait: the required exposure times
were too long. But the two eventually came together, and now their
pairing seems so natural that it's as though photography was invented
for making portraits.

One of the first photographic portraits, if not the first, was a
self-portrait daguerreotype made by a 30-year-old amateur chemist from
Philadelphia named Robert Cornelius. Cornelius held his pose for several
minutes in the bright October sun in 1839. His dark coat has a high
collar, and his hair is tousled. The catalog text at the Library of
Congress adds that he is ``peering uncertainly into the camera.'' But is
that true? How would we verify it?

We tend to interpret portraits as though we were reading something
inherent in the person portrayed. We talk about strength and
uncertainty; we praise people for their strong jaws and pity them their
weak chins. High foreheads are deemed intelligent. We easily link the
people's facial features to the content of their character. This is odd.
After all, we no longer believe you can determine someone's personality
by measuring their skull with a pair of calipers. Phrenology has rightly
been consigned to the dustbin of history. But physiognomy, the idea that
faces carry meanings, still haunts the interpretation of portraiture.

The reason for the temptation is obvious: Faces are malleable. A smile
is intentional and might indeed indicate happiness, just as a furrowed
brow might be proof of a melancholic temperament. But we also know that
emotion is fleeting and can be faked. We thus shouldn't really trust
whatever it is a photographic portrait seems to be telling us.

\includegraphics{https://static01.graylady3jvrrxbe.onion/images/2018/08/26/magazine/26mag-onphoto-slide-OY11/26mag-onphoto-slide-OY11-articleLarge.png?quality=75\&auto=webp\&disable=upscale}

This is not to deny any of the wonder or gratitude you feel before a
superb portrait. Sometimes this response is amplified when it's a
portrait of someone not famous, a face that isn't burdened with
predetermined knowledge. I'm looking at one such image in Dawoud Bey's
magnificent career retrospective, ``Seeing Deeply'' (2018). In the book,
this black-and-white photograph is given a full page. The format invites
contemplation, and this should be mentioned because what we see in a
photograph is connected to its material circumstances: An exhibition
print of the same image would give one impression, a magazine
reproduction would be another, a digital file meant to be seen on a
computer or hand-held device is something else again. The warm tone and
low gloss of this photograph in this book are calming. A boy stands
alone before a tent and some chairs. We don't know who he is, and the
caption doesn't help much: ``Young Man at a Tent Revival, Brooklyn, NY,
1989.'' The surprising detail there is the date, as this picture looks
as if it could have been taken at any point in the past century. It is
strangely timeless, with his attire somewhere between formal and casual,
the slim dark tie and serious black pants contrasting with the baggy
pale-colored plaid shirt.

I want to fall back on old ways and say that the gentle arch of the
boy's left eyebrow seems to mark him as an ironic sort, or that the
symmetry of his features make him both trusting and trustworthy. But
really, that would be projecting. What we can really say is that there's
something poignant about the way the skinny tie is tucked into the
skinny belt and the way the numerous verticals in the picture --- the
tent poles, the ropes of its rigging, the legs of the chairs in the
background, the tie, the lines of the shirt and finally the boy himself
--- all seem to be tilting just off true.

The picture wavers in tremulous equilibrium. Even the boy's head is
cocked to the side. Quizzically? Or is he simply at his ease? I don't
know. But the cumulative effect is endearing. There's a boy, and his
appearance is dense with a life that we can only guess at. There's faith
in it (it's a revival, after all); there's probably hope, too. But what
we can be surer of is that there's love: the love with which Dawoud Bey
has seen the elements of the moment and captured them for posterity, and
the love with which, almost three decades later, I am looking at this
portrait in a book.

\textbf{The rise of} portrait photography made immortality of a new kind
available to ordinary people. Picture-making establishments in New York,
Boston and San Francisco displayed countless photographs of
seamstresses, servants, soldiers, laborers, lawyers and even the
recently dead. A wide swath of society owned treasured likenesses of
themselves that they displayed at home, kept in specially made cases,
sent to their lovers or bequeathed to their descendants. And that
abundance has become, in our time, positively torrential. There must be
very few people on Earth who have not been photographed.

But something truly strange has also happened: Automation is playing an
outsize role in the creation and dissemination of photographic
portraits. Machines are making images of people for other machines to
see and analyze. We are photographed when we cross international
borders. Cameras in public places scan and collect the faces of
passers-by. We rouse our mobile phones with our faces. Even the cameras
on our computers cannot be trusted not to spy on us. Our faces are
spirited away in the name of societal stability, crime prevention,
corporate profit or national security.

Surveillance is nothing new, but with storage getting cheaper and
analytical tools more ferocious, a dystopian future is closer than it
has ever been. In many parts of China, ubiquitous facial data collection
is already an everyday reality. Facial-recognition technology is giving
the government there powerful tools to control and discipline its
populace. Unsurprisingly, religious minorities and political activists,
in addition to petty offenders and hardened criminals, are already
bearing the brunt of these initiatives. To be ethnically Uighur in China
today, for example, is to be under tremendous restriction. Other
governments will follow, and arguments about the right to privacy or
freedom will lag behind.

Machines take advantage of the particularity of each person's appearance
to flatten out our collective individuality. A machine sees without
sympathy. And yet our individual particularities might themselves serve
as a comfort in this machine-driven age. The shape of my lips, the shine
on my nose, the corners of my eyes, the breadth of my forehead: the same
features that allow machines to track me are also dear to the people who
love me (not because those features are objectively special but because
they are mine). And those features also say something to people who
don't know me: that I am not disembodied, that I am not abstraction.
Physiognomy is of limited use: I am not my face. But a set of features
retains affect, as in a cistern, and from this something more subtle can
be retrieved.

A photographic portrait records a human encounter. The photographer's
intent and the sitter's agreement, and vice versa, are made visible. The
portrait also contains the tacit hope that a third party, the viewer,
will be able to register the traces of that previous encounter. Better
if it's printed out and held in the hand, vibrant to the touch. This was
the experience of those who bought the small, inexpensive
cardboard-mounted photographs known as \emph{cartes de visite} in the
19th century.

Image

Carte de visite of Sojourner Truth, around 1864.Credit...From the
American Antiquarian Society

Perhaps the most famous usage of the American c\emph{arte de visite} was
by Sojourner Truth, in the 1860s. Truth escaped from slavery in Ulster
County, New York, in 1826 and became a noted abolitionist activist. She
was a gifted orator who, as one contemporary noted, ``poured forth a
torrent of natural eloquence which swept everything before it.''
Illiteracy did not prevent her from producing (with the help of
intermediaries) a large number of letters, speeches, petitions and
autobiographical texts. And, particularly during the years of the Civil
War, she also sat for numerous photographers, leaving behind at least 28
different photographs. Most of these were printed as \emph{cartes de
visite} and sold to support her abolitionist work.

One of the senses of ``shadow'' at that time was ``photograph,'' and
from 1864 onward, Truth's \emph{cartes de visite} included a caption
text and her name: ``I Sell the Shadow to Support the Substance.
Sojourner Truth.'' She was not the photographer of these images, but so
insistent was her control over how she was seen that these are
practically self-portraits.

Like Frederick Douglass (with whom she had a mutual antipathy), Truth
knew how powerful a photographic presence could be in the struggle to
make white Americans see black American humanity. Her photograph was not
herself --- it was a shadow, and as an ex-slave that distinction must
have been one she sensed especially keenly --- but she knew it did
convey some indelible news of her reality. The photos show a tall and
somewhat gaunt woman in her 60s, in modest dress and with a white shawl
and cap, sometimes sitting, sometimes standing. Her skin is dark and
smooth, and her expression might be read either as serious or neutral.
Though she tends to look directly at the camera, her eyes are usually
obscured behind glasses. Truth's photographs did not have the
cosmopolitan and occasionally conceited air that Douglass's did but,
like his, they reminded others that she did have a real self and that
her dignity was not negotiable, and this reminder was a challenge to the
conscience of all who saw, held or bought the ``shadow.''

A portrait is an open door. It can remind us of our ethical duty to the
other. ``The face speaks to me, and thereby invites me to a relation,''
as the philosopher Emmanuel Lévinas puts it. Unlike machines, we see
with sympathy. (This is why a mere portrait of a despot can be
dangerously effective propaganda. The portrait humanizes the person
depicted in ways we can't quite control. Inhuman behavior is rarely
apparent on a human face.) A photograph by Berenice Abbott, Seydou
Keïta, Gordon Parks, Dawoud Bey or any of the greats in the history of
photography, a portrait of Sojourner Truth, Frederick Douglass or an
unnamed boy standing in front of a tent in Brooklyn presents us with the
face of the other and restores us to ourselves. Some magic happens
there, a magic as old and reliable as the portraits painted on the
Fayyum funerary boards 2,000 years ago. Not all portraits are created
equal: To be great, they must contain presence, tension, a finely
balanced amalgam of feeling and craft. ``This is human,'' is the final
meaning of a great portrait, ``and I am human, and this is worth
defending.''

Advertisement

\protect\hyperlink{after-bottom}{Continue reading the main story}

\hypertarget{site-index}{%
\subsection{Site Index}\label{site-index}}

\hypertarget{site-information-navigation}{%
\subsection{Site Information
Navigation}\label{site-information-navigation}}

\begin{itemize}
\tightlist
\item
  \href{https://help.nytimes3xbfgragh.onion/hc/en-us/articles/115014792127-Copyright-notice}{©~2020~The
  New York Times Company}
\end{itemize}

\begin{itemize}
\tightlist
\item
  \href{https://www.nytco.com/}{NYTCo}
\item
  \href{https://help.nytimes3xbfgragh.onion/hc/en-us/articles/115015385887-Contact-Us}{Contact
  Us}
\item
  \href{https://www.nytco.com/careers/}{Work with us}
\item
  \href{https://nytmediakit.com/}{Advertise}
\item
  \href{http://www.tbrandstudio.com/}{T Brand Studio}
\item
  \href{https://www.nytimes3xbfgragh.onion/privacy/cookie-policy\#how-do-i-manage-trackers}{Your
  Ad Choices}
\item
  \href{https://www.nytimes3xbfgragh.onion/privacy}{Privacy}
\item
  \href{https://help.nytimes3xbfgragh.onion/hc/en-us/articles/115014893428-Terms-of-service}{Terms
  of Service}
\item
  \href{https://help.nytimes3xbfgragh.onion/hc/en-us/articles/115014893968-Terms-of-sale}{Terms
  of Sale}
\item
  \href{https://spiderbites.nytimes3xbfgragh.onion}{Site Map}
\item
  \href{https://help.nytimes3xbfgragh.onion/hc/en-us}{Help}
\item
  \href{https://www.nytimes3xbfgragh.onion/subscription?campaignId=37WXW}{Subscriptions}
\end{itemize}
