Sections

SEARCH

\protect\hyperlink{site-content}{Skip to
content}\protect\hyperlink{site-index}{Skip to site index}

\href{https://www.nytimes3xbfgragh.onion/section/world/asia}{Asia
Pacific}

\href{https://myaccount.nytimes3xbfgragh.onion/auth/login?response_type=cookie\&client_id=vi}{}

\href{https://www.nytimes3xbfgragh.onion/section/todayspaper}{Today's
Paper}

\href{/section/world/asia}{Asia Pacific}\textbar{}Myanmar Sentences
Reuters Journalists to 7 Years in Prison

\url{https://nyti.ms/2NIrxUk}

\begin{itemize}
\item
\item
\item
\item
\item
\end{itemize}

Advertisement

\protect\hyperlink{after-top}{Continue reading the main story}

Supported by

\protect\hyperlink{after-sponsor}{Continue reading the main story}

\hypertarget{myanmar-sentences-reuters-journalists-to-7-years-in-prison}{%
\section{Myanmar Sentences Reuters Journalists to 7 Years in
Prison}\label{myanmar-sentences-reuters-journalists-to-7-years-in-prison}}

\includegraphics{https://static01.graylady3jvrrxbe.onion/images/2018/09/04/world/04myanmar-2-print/merlin_143217135_91cb7336-d5e8-4627-80bf-232bd047c75c-articleLarge.jpg?quality=75\&auto=webp\&disable=upscale}

By
\href{https://www.nytimes3xbfgragh.onion/by/richard-c-paddock}{Richard
C. Paddock}

\begin{itemize}
\item
  Sept. 3, 2018
\item
  \begin{itemize}
  \item
  \item
  \item
  \item
  \item
  \end{itemize}
\end{itemize}

\href{https://www.nytimes3xbfgragh.onion/es/2018/09/03/birmania-reuters-rohinya}{Leer
en español}

BANGKOK --- The two reporters met the police corporal at his insistence,
joining him at a restaurant in Yangon. Confused when the conversation
did not live up to the policeman's initial urgency, the two got up to
leave after the meal ended, only to have him hand off two rolled-up
pieces of paper with no explanation.

The journalists, U Wa Lone and U Kyaw Soe Oo, reporters for Reuters,
barely made it out of the restaurant before they were arrested, the
papers confiscated before they had any chance to look at them, they
testified.

But on Monday, those papers --- despite testimony in April by another
police official that higher-ups had ordered them to be planted on the
reporters --- were at the heart of a judge's rationale in convicting and
sentencing the two journalists to seven years in prison for violating
Myanmar's colonial-era Official Secrets Act.

Their case, which has stretched over almost nine months of court
hearings, has become the most notable blow in Myanmar's intensified
crackdown on the press, as officials seek to deny or obscure atrocities
against the country's Rohingya Muslim minority. A United Nations mission
recently called for Myanmar military leaders to be
\href{https://www.nytimes3xbfgragh.onion/2018/08/27/world/asia/myanmar-rohingya-genocide.html}{tried
for genocide against the Rohingya}.

The reporters' defenders, including news organizations and rights groups
from around the world, say their only crime was committing journalism,
documenting the mass killings and ethnic cleansing by soldiers and
Buddhist mobs in Rakhine State that began in August last year.

Their lawyers argued that all the documents cited by the judge in
convicting and sentencing Mr. Wa Lone and Mr. Kyaw Soe Oo --- including
the rolled-up papers, as well as files and phone numbers captured from
their cellphones and homes --- were already public at the time of their
arrest, and that having phone numbers was part of their job.

The judge, U Ye Lwin, ruled that the journalists intended to harm the
country by sharing its secrets.

``It cannot be said that they were doing normal journalistic work,'' he
said in announcing the verdict. ``And the top-secret documents they were
holding can be useful to the enemies of the country or the ones who
oppose the country.''

The judge's verdict and harsh sentence outraged rights activists and
dealt another blow to the legacy of Aung San Suu Kyi, the Nobel Peace
laureate who heads the civilian government, as a former symbol of the
fight for democracy and human rights.

It was her government that pursued the charges against the two
journalists, despite her own history of serving 15 years under house
arrest at the hands of an earlier military government in Myanmar, then
known as Burma.

``It's deeply troubling for everybody who has struggled so hard here for
media freedom,'' said Scot Marciel, the United States ambassador to
Myanmar, who attended the court hearing.

Mr. Wa Lone, 32, and Mr. Kyaw Soe Oo, 28, were targeted by the police
while they were investigating a massacre of 10 Rohingya Muslim villagers
in Rakhine State.

The massacre took place a year ago, during a broader wave of arson, rape
and killing by soldiers that drove more than 700,000 Rohingya into
neighboring Bangladesh, in what has been widely condemned as
\href{https://www.nytimes3xbfgragh.onion/2017/11/22/us/politics/tillerson-myanmar-rohingya-ethnic-cleansing.html}{ethnic
cleansing}. The Rohingya have lived in Myanmar for generations, but many
Burmese consider them to be illegal immigrants from Bangladesh.

The two journalists found evidence that members of the military and
Buddhist civilians killed 10 Rohingya males in the village of Inn Din.
\href{https://www.nytimes3xbfgragh.onion/2018/02/10/world/asia/reuters-myanmar-massacre-rohingya.html}{Their
report,} with photographs of the victims tied up and kneeling before
their executions, and evidence of the mass grave where they were buried,
was published after the reporters' arrest.

\includegraphics{https://static01.graylady3jvrrxbe.onion/images/2018/09/04/world/04myanmar-1-print/merlin_143217069_2f5a6fc7-7a93-420f-951f-844e11555572-articleLarge.jpg?quality=75\&auto=webp\&disable=upscale}

Despite pressure from the international community, including recent
sanctions by the United States against some of Myanmar's top generals,
Ms. Aung San Suu Kyi has refused to criticize the military campaign
against the Rohingya.

Western governments, human rights groups and press freedom organizations
around the world had urged Myanmar to free the Reuters journalists as a
sign of commitment to the establishment of democracy and freedom of
expression.

On Monday, the small courtroom was packed with family members,
journalists, rights advocates and foreign diplomats.

``The decision is very disappointing,'' said one of the reporters'
lawyers, Khin Maung Zaw, after the hearing. ``This is bad for the rule
of law, bad for freedom of expression. This decision is against
democracy.''

The Reuters president and editor in chief, Stephen J. Adler, called the
verdict an ``injustice'' and urged Ms. Aung San Suu Kyi's government to
step in and free them.

``These two admirable reporters have already spent nearly nine months in
prison on false charges designed to silence their reporting and
intimidate the press,'' he said. ``Without any evidence of wrongdoing
and in the face of compelling evidence of a police setup, today's ruling
condemns them to the continued loss of their freedom and condones the
misconduct of security forces.''

Mr. Wa Lone and Mr. Kyaw Soe Oo, who have repeatedly denounced the
authorities' actions, said after the hearing that they were disappointed
by the verdict and the prison sentence, and insisted again that they had
done nothing wrong.

At trial, the defense argued that it was a clear case of entrapment by
the police, and that none of the prosecution's 17 witnesses had produced
evidence of a crime.

One prosecution witness who said he was present during the arrests
admitted under cross-examination that he had written the location on his
hand so he would not forget it while he was testifying.

An officer admitted that he burned his notes of the arrest. Yet another
police witness acknowledged that the information in the supposedly
secret documents had been published in newspaper reports before the
arrests.

A police captain who told the court in April that
\href{https://www.nytimes3xbfgragh.onion/2018/05/02/world/asia/myanmar-journalists-police-captain.html}{the
arrests had been a setup} was punished for his testimony with a year in
prison. ``I am revealing the truth, because police of any rank must
maintain their own integrity,'' the captain, Moe Yan Naing told
reporters after he testified. ``It is true that they were set up.''

In his ruling, Judge Ye Lwin rejected the defense arguments and found
that the journalists illegally possessed confidential documents. He said
that the reporters had been consistently trying to obtain government
secrets and share them, which could be useful to enemies of the state.

In addition, the judge said that phone numbers they possessed, such as
the number of a contact with the rebel Arakan Army, was further evidence
of their intention to undermine the government.

For many journalists, finding secrets and making them public is the
essence of the job.

``This decision is made politically, not because they did something
wrong,'' said Myint Kyaw, a member of the independent Myanmar Press
Council. ``This decision is a warning that no journalist can report
freely about the Rakhine issue.''

Diplomats and rights advocates were outspoken in their criticism of the
verdict.

Sean Bain, legal adviser in Yangon for the International Commission of
Jurists, said the trial was ``grossly unfair'' and that punishing
journalists for exposing human rights violations undermines the rule of
law.

``The outrageous convictions of the Reuters journalists show Myanmar
courts' willingness to muzzle those reporting on military atrocities,''
said Brad Adams, Asia director at Human Rights Watch. ``These sentences
mark a new low for press freedom and further backsliding on rights under
Aung San Suu Kyi's government.''

Advertisement

\protect\hyperlink{after-bottom}{Continue reading the main story}

\hypertarget{site-index}{%
\subsection{Site Index}\label{site-index}}

\hypertarget{site-information-navigation}{%
\subsection{Site Information
Navigation}\label{site-information-navigation}}

\begin{itemize}
\tightlist
\item
  \href{https://help.nytimes3xbfgragh.onion/hc/en-us/articles/115014792127-Copyright-notice}{©~2020~The
  New York Times Company}
\end{itemize}

\begin{itemize}
\tightlist
\item
  \href{https://www.nytco.com/}{NYTCo}
\item
  \href{https://help.nytimes3xbfgragh.onion/hc/en-us/articles/115015385887-Contact-Us}{Contact
  Us}
\item
  \href{https://www.nytco.com/careers/}{Work with us}
\item
  \href{https://nytmediakit.com/}{Advertise}
\item
  \href{http://www.tbrandstudio.com/}{T Brand Studio}
\item
  \href{https://www.nytimes3xbfgragh.onion/privacy/cookie-policy\#how-do-i-manage-trackers}{Your
  Ad Choices}
\item
  \href{https://www.nytimes3xbfgragh.onion/privacy}{Privacy}
\item
  \href{https://help.nytimes3xbfgragh.onion/hc/en-us/articles/115014893428-Terms-of-service}{Terms
  of Service}
\item
  \href{https://help.nytimes3xbfgragh.onion/hc/en-us/articles/115014893968-Terms-of-sale}{Terms
  of Sale}
\item
  \href{https://spiderbites.nytimes3xbfgragh.onion}{Site Map}
\item
  \href{https://help.nytimes3xbfgragh.onion/hc/en-us}{Help}
\item
  \href{https://www.nytimes3xbfgragh.onion/subscription?campaignId=37WXW}{Subscriptions}
\end{itemize}
