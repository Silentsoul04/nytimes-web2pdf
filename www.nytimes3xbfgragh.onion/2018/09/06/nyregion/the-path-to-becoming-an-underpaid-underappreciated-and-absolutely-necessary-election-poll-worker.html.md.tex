Sections

SEARCH

\protect\hyperlink{site-content}{Skip to
content}\protect\hyperlink{site-index}{Skip to site index}

\href{https://www.nytimes3xbfgragh.onion/section/nyregion}{New York}

\href{https://myaccount.nytimes3xbfgragh.onion/auth/login?response_type=cookie\&client_id=vi}{}

\href{https://www.nytimes3xbfgragh.onion/section/todayspaper}{Today's
Paper}

\href{/section/nyregion}{New York}\textbar{}The Path to Becoming an
Underpaid, Underappreciated and Absolutely Necessary Election Poll
Worker

\url{https://nyti.ms/2NURD6E}

\begin{itemize}
\item
\item
\item
\item
\item
\end{itemize}

Advertisement

\protect\hyperlink{after-top}{Continue reading the main story}

Supported by

\protect\hyperlink{after-sponsor}{Continue reading the main story}

\hypertarget{the-path-to-becoming-an-underpaid-underappreciated-and-absolutely-necessary-election-poll-worker}{%
\section{The Path to Becoming an Underpaid, Underappreciated and
Absolutely Necessary Election Poll
Worker}\label{the-path-to-becoming-an-underpaid-underappreciated-and-absolutely-necessary-election-poll-worker}}

\includegraphics{https://static01.graylady3jvrrxbe.onion/images/2018/09/09/nyregion/09polling01/09polling01-articleLarge.jpg?quality=75\&auto=webp\&disable=upscale}

By Spenser Mestel

\begin{itemize}
\item
  Sept. 6, 2018
\item
  \begin{itemize}
  \item
  \item
  \item
  \item
  \item
  \end{itemize}
\end{itemize}

On the morning of the 2016 presidential election, a line was forming
outside of the Julia Richman Educational Complex, a polling site on the
Upper East Side of Manhattan. It was Trump vs. Clinton, and everyone had
expected high turnout, but when a few of the site's scanners jammed, a
crowd started to fill the auditorium as well.

Diane Burrows, a poll worker, wasn't particularly concerned. She and the
other temporary workers had hours of training, and they also had the
poll worker manual on hand if any problems arose. But then, after the
rush had died down, a man approached her and started ranting about his
wife.

``My wife was here at 8 o'clock this morning trying to vote,'' she
remembers him saying, ``and she had to wait two hours, so she just threw
her ballot in the trash can and left.'' When Ms. Burrows retells this
part of the story, her eyes go wide in re-enacted shock, and for good
reason. Every ballot, whether it's cast, voided, or left blank, is
tracked on Election Day, and when the polls close at 9 p.m. --- after
poll workers have already been on the clock for at least 16 hours ---
each one must be accounted for.

Under ideal circumstances, that process can be as brief as 30 minutes.
However, any discrepancy that can't be resolved requires poll workers to
recheck the used ballot stubs, the voided ballots, and the printed
results from the scanners. What isn't part of the protocol is sifting
through the poll site's trash cans, which is how Ms. Burrows and her
team found the woman's ballot after her husband's tirade. ``She had no
concept of what she'd done,'' Ms. Burrows said, laughing. ``I mean ---
we would still be there.''

\includegraphics{https://static01.graylady3jvrrxbe.onion/images/2018/09/09/nyregion/09polling3/09polling3-articleLarge.jpg?quality=75\&auto=webp\&disable=upscale}

As the midterms approach amid reports of
\href{https://www.nytimes3xbfgragh.onion/2018/08/23/us/randolph-county-georgia-voting.html}{voter
suppression} and
\href{https://www.nytimes3xbfgragh.onion/2018/08/24/us/politics/cia-russia-midterm-elections.html}{foreign
interference}, it's easy to lose sight of the humble poll worker who is
at the mechanical level of the city's electoral process.

***

In many ways, Ms. Burrows is a model poll worker. After 27 years of
teaching (five in high school English, the other 22 with fifth
-graders), she's habitually patient, and she wasn't attracted to the job
simply for the pay, which works out to about \$14 an hour. ``I had a
list of things I wanted to do when I retired,'' she said. ``I wanted to
pursue the League of Women Voters, and I wanted to learn how to be a
poll worker.''

Not all New Yorkers are as civic-minded. To fill the 34,000 or so
vacancies for each election (roughly equal to the number of officers in
the New York Police Department), the Board of Elections advertises
throughout the city: on the subway, in local newspapers, through high
school guidance counselors, and so on. Yet according to Michael Ryan,
the executive director of the Board of Elections, recruitment is a
perennial problem. For the upcoming primary, on Sept. 13, the board
still must fill about 6,400 vacancies.

It's a process that must be repeated every cycle. Though as many as 15
percent of poll workers fail to show up for duty on Election Day, the
biggest obstacle seems to be the mandatory four-hour training session
before each election. Roughly 70 percent of its work force drops out
between recruitment and actually working the polls. .

At her first training, Ms. Burrows was surprised that the instructor did
little more than read the poll worker manual aloud to the class. ``I
guess because I'm a teacher, I'm used to a certain structure,'' she
said. While she had a more engaging session this year, not all students
get hands-on experience, even with the scanners, which are notorious for
jamming, or the Ballot Marking Device, which helps voters with
disabilities navigate the ballot. She left the first training session
feeling less than optimistic. ``I had no picture in my mind of how this
was going to work,'' she said.

In 2013, following a report that the Board of Elections had ``wasted at
least \$2.4 million in city funds by failing to consolidate election
districts during the November 2011 off-year elections,'' New York City's
Department of Investigation again took aim at the board and recommended
that it incorporate more hands-on training. The agency responded by
saying that the \$100 compensation for trainees was ``woefully
inadequate.'' Under this kind of educational triage, the board's
curriculum is best summarized by an instructor at a recent training in
Bushwick, Brooklyn: ``I don't want you to remember this,'' he said. ``I
want you to know how to use this manual to solve any problem you may
have.''

Image

Typical signs that go up at polling sites around the city on Election
Day.Credit...Lily Landes for The New York Times

At the end of every class, prospective poll workers must pass an exam.
It's open book, 20 multiple choice and true-false questions --- and the
quiz even gives the page number where the answer can be found. But
according to prospective poll workers interviewed across Manhattan and
Brooklyn and the 2013 report by the Department of Investigation, some
trainees still lack the basic literacy skills to pass.

In the past eight years, an average of 8 percent of prospective poll
workers have failed the test. And in 2015, it was 20 percent, or 1,031
people. This happened despite the fact that, according to the Department
of Investigation report from 2013, investigators ``observed trainers
telling trainees the specific subjects to be covered on the exam before
trainees took the exam, trainers effectively giving answers to trainees
during the exam, or trainees cheating on the exam.''

In order to recruit more high-caliber workers, experts have suggested
alternatives, but according to Mr. Ryan, each has its drawbacks. While
college students' schedules may seem ideal to accommodate the sporadic,
all-day staffings, elections often fall during CUNY's midterms. While
municipal workers have already been vetted by the city and also get
Election Day off, their recruitment partly depends on the cooperation of
other government agencies, which is also the obstacle for another
proposal --- to waive jury duty for those who work the polls.

Every year for the past eight years, the Board of Elections has asked
the State Assembly to increase compensation for poll workers --- this
year by paying poll workers \$100 for the four-hour training and \$300
for the roughly 17 hours of work on Election Day. The proposal has never
passed.

The result is an unusually integrated working environment, according to
Jan Combopiano, who woke up at 3:30 for the 2017 mayoral election to
travel from her home in Downtown Brooklyn to her assigned location in
Greenpoint. She says that some of her co-workers were professionals,
like her, who had taken the day off. ``And then there were people who
were out of work, who were like: `I want to do this to get paid. This
money is a big deal to me.''' Regardless of income levels, though, she
felt that everyone was united around a common goal: to help the voter.

That diversity was a selling point for David Iscoe, who would put
himself in both of Ms Combopiano's categories. ``I'm motivated by civic
duty to some sense,'' he said. ``But I'm also a freelance writer, and
you need money.'' After hearing about the job from a neighbor, he worked
his first election in Park Slope in 2017 and recommends that all his
friends give it a try. ``You meet people who are not in the same social
scene or day job,'' he said, ``so I think that's good in just getting a
sense of who's in the community.''

Advertisement

\protect\hyperlink{after-bottom}{Continue reading the main story}

\hypertarget{site-index}{%
\subsection{Site Index}\label{site-index}}

\hypertarget{site-information-navigation}{%
\subsection{Site Information
Navigation}\label{site-information-navigation}}

\begin{itemize}
\tightlist
\item
  \href{https://help.nytimes3xbfgragh.onion/hc/en-us/articles/115014792127-Copyright-notice}{©~2020~The
  New York Times Company}
\end{itemize}

\begin{itemize}
\tightlist
\item
  \href{https://www.nytco.com/}{NYTCo}
\item
  \href{https://help.nytimes3xbfgragh.onion/hc/en-us/articles/115015385887-Contact-Us}{Contact
  Us}
\item
  \href{https://www.nytco.com/careers/}{Work with us}
\item
  \href{https://nytmediakit.com/}{Advertise}
\item
  \href{http://www.tbrandstudio.com/}{T Brand Studio}
\item
  \href{https://www.nytimes3xbfgragh.onion/privacy/cookie-policy\#how-do-i-manage-trackers}{Your
  Ad Choices}
\item
  \href{https://www.nytimes3xbfgragh.onion/privacy}{Privacy}
\item
  \href{https://help.nytimes3xbfgragh.onion/hc/en-us/articles/115014893428-Terms-of-service}{Terms
  of Service}
\item
  \href{https://help.nytimes3xbfgragh.onion/hc/en-us/articles/115014893968-Terms-of-sale}{Terms
  of Sale}
\item
  \href{https://spiderbites.nytimes3xbfgragh.onion}{Site Map}
\item
  \href{https://help.nytimes3xbfgragh.onion/hc/en-us}{Help}
\item
  \href{https://www.nytimes3xbfgragh.onion/subscription?campaignId=37WXW}{Subscriptions}
\end{itemize}
