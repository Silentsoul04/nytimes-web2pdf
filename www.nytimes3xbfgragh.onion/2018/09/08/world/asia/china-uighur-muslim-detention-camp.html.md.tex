Sections

SEARCH

\protect\hyperlink{site-content}{Skip to
content}\protect\hyperlink{site-index}{Skip to site index}

\href{https://www.nytimes3xbfgragh.onion/section/world/asia}{Asia
Pacific}

\href{https://myaccount.nytimes3xbfgragh.onion/auth/login?response_type=cookie\&client_id=vi}{}

\href{https://www.nytimes3xbfgragh.onion/section/todayspaper}{Today's
Paper}

\href{/section/world/asia}{Asia Pacific}\textbar{}China Is Detaining
Muslims in Vast Numbers. The Goal: `Transformation.'

\url{https://nyti.ms/2O0wPuq}

\begin{itemize}
\item
\item
\item
\item
\item
\end{itemize}

Advertisement

\protect\hyperlink{after-top}{Continue reading the main story}

Supported by

\protect\hyperlink{after-sponsor}{Continue reading the main story}

\hypertarget{china-is-detaining-muslims-in-vast-numbers-the-goal-transformation}{%
\section{China Is Detaining Muslims in Vast Numbers. The Goal:
`Transformation.'}\label{china-is-detaining-muslims-in-vast-numbers-the-goal-transformation}}

\includegraphics{https://static01.graylady3jvrrxbe.onion/images/2018/09/04/world/00xinjiang-6/merlin_143252649_d0056165-7ead-4eb6-ba4a-a04b6f36e7a5-articleLarge.jpg?quality=75\&auto=webp\&disable=upscale}

By \href{https://www.nytimes3xbfgragh.onion/by/chris-buckley}{Chris
Buckley}

\begin{itemize}
\item
  Sept. 8, 2018
\item
  \begin{itemize}
  \item
  \item
  \item
  \item
  \item
  \end{itemize}
\end{itemize}

\href{https://cn.nytimes3xbfgragh.onion/china/20180908/china-builds-a-vast-network-of-detention-camps-to-transform-muslims/}{阅读简体中文版}\href{https://cn.nytimes3xbfgragh.onion/china/20180908/china-builds-a-vast-network-of-detention-camps-to-transform-muslims/zh-hant/}{閱讀繁體中文版}

HOTAN, China --- On the edge of a desert in far western China, an
imposing building sits behind a fence topped with barbed wire. Large red
characters on the facade urge people to learn Chinese, study law and
acquire job skills. Guards make clear that visitors are not welcome.

Inside, hundreds of ethnic Uighur Muslims spend their days in a
high-pressure indoctrination program, where they are forced to listen to
lectures, sing hymns praising the Chinese Communist Party and write
``self-criticism'' essays, according to detainees who have been
released.

The goal is to remove any devotion to Islam.

Abdusalam Muhemet, 41, said the police detained him for reciting a verse
of the Quran at a funeral. After two months in a nearby camp, he and
more than 30 others were ordered to renounce their past lives. Mr.
Muhemet said he went along but quietly seethed.

``That was not a place for getting rid of extremism,'' he recalled.
``That was a place that will breed vengeful feelings and erase Uighur
identity.''

This camp outside Hotan, an ancient oasis town in the Taklamakan Desert,
is one of hundreds that China has built in the past few years. It is
part of a campaign of breathtaking scale and ferocity that has swept up
hundreds of thousands of Chinese Muslims for weeks or months of what
critics describe as brainwashing, usually without criminal charges.

Though limited to China's western region of Xinjiang, it is the
country's most sweeping internment program since the Mao era --- and the
focus of a growing chorus of international criticism.

China has sought for decades to restrict the practice of Islam and
maintain an iron grip in Xinjiang, a region almost as big as Alaska
where more than half the population of 24 million belongs to
\href{http://theasiadialogue.com/2016/03/07/spatial-results-of-the-2010-census-in-xinjiang/}{Muslim
ethnic minority groups}. Most are Uighurs, whose religion, language and
culture, along with a history of independence movements and resistance
to Chinese rule, have long unnerved Beijing.

After a succession of violent antigovernment attacks
\href{https://www.nytimes3xbfgragh.onion/2014/03/04/world/asia/han-uighur-relations-china.html}{reached
a peak in 2014}, the Communist Party chief, Xi Jinping, sharply
escalated the crackdown, orchestrating an unforgiving drive to turn
ethnic Uighurs and other Muslim minorities into loyal citizens and
supporters of the party.

\includegraphics{https://static01.graylady3jvrrxbe.onion/images/2018/09/04/world/00xinjiang-1/merlin_143059680_1ea354ae-320f-4ca6-8688-4cbd5bd76c79-articleLarge.jpg?quality=75\&auto=webp\&disable=upscale}

``Xinjiang is in an active period of terrorist activities, intense
struggle against separatism and painful intervention to treat this,''
Mr. Xi told officials, according to
\href{http://cpc.people.com.cn/n1/2017/0420/c64102-29225251.html}{reports
in the state news media} last year.

In addition to the mass detentions, the authorities have intensified the
use of informers and expanded police surveillance, even installing
cameras in some people's homes. Human rights activists and experts say
the campaign has traumatized Uighur society, leaving behind fractured
communities and families.

``Penetration of everyday life is almost really total now,'' said
\href{https://crawford.anu.edu.au/people/academic/michael-clarke}{Michael
Clarke}, an expert on Xinjiang at Australian National University in
Canberra. ``You have ethnic identity, Uighur identity in particular,
being singled out as this kind of pathology.''

China has categorically denied reports of abuses in Xinjiang. At a
meeting of a United Nations panel in Geneva last month, it said it does
not operate re-education camps and described the facilities in question
as
\href{https://www.nytimes3xbfgragh.onion/2018/08/13/world/asia/china-xinjiang-un.html}{mild
corrective institutions} that provide job training.

``There is no arbitrary detention,'' Hu Lianhe, an official with a role
in Xinjiang policy,
\href{http://webtv.un.org/en/ga/watch/consideration-of-china-contd-2655th-meeting-96th-session-committee-on-elimination-of-racial-discrimination/5821422267001/?term=\&lan=english}{told
the U.N. Committee on the Elimination of Racial Discrimination}. ``There
is no such thing as re-education centers.''

The committee pressed Beijing to disclose how many people have been
detained and free them, but the Ministry of Foreign Affairs dismissed
the demand as having ``no factual basis'' and said China's security
measures were comparable to those of other countries.

The government's business-as-usual defense, however, is contradicted by
overwhelming evidence, including official directives, studies, news
reports and construction plans that have surfaced online, as well as the
eyewitness accounts of a growing number of former detainees who have
fled to countries such as Turkey and Kazakhstan.

The government's own documents describe a vast network of camps ---
usually called ``transformation through education'' centers --- that has
expanded without public debate, specific legislative authority or any
system of appeal for those detained.

The New York Times interviewed four recent camp inmates from Xinjiang
who described physical and verbal abuse by guards; grinding routines of
singing, lectures and self-criticism meetings; and the gnawing anxiety
of not knowing when they would be released. Their accounts were echoed
in interviews with more than a dozen Uighurs with relatives who were in
the camps or had disappeared, many of whom spoke on condition of
anonymity to avoid government retaliation.

The Times also discovered reports online written by teams of Chinese
officials who were assigned to monitor families with detained relatives,
and \href{http://www.zjdata.net/literature/detail/657545.html}{a study
published last year} that said officials in some places were
indiscriminately sending ethnic Uighurs to the camps to meet numerical
quotas.

The study, by Qiu Yuanyuan, a scholar at the Xinjiang Party School,
where officials are trained, warned that the detentions could backfire
and fan radicalism. ``Recklessly setting quantitative goals for
transformation through education has been erroneously used'' in some
areas, she wrote. ``The targeting is imprecise, and the scope has been
expanding.''

Image

A satellite image taken over Hotan in late August showed that the
internment camp, center, had expanded. Credit...Planet Labs Inc.

\hypertarget{eradicating-a-virus}{%
\subsection{Eradicating a `Virus'}\label{eradicating-a-virus}}

The long days in the re-education camp usually began with a jog.

Nearly every morning, Mr. Muhemet recalled, he and dozens of others ---
college graduates, businessmen, farmers --- were told to run around an
assembly ground. Impatient guards sometimes slapped and shoved the
older, slower inmates, he said.

Then they were made to sing rousing patriotic hymns in Chinese, such as
``Without the Communist Party, There Would Be No New China.'' Those who
could not remember the words were denied breakfast, and they all learned
the words quickly.

Mr. Muhemet, a stocky man who ran a restaurant in Hotan before fleeing
China this year, said he spent seven months in a police cell and more
than two months in the camp in 2015 without ever being charged with a
crime. Most days, he said, the camp inmates assembled to hear long
lectures by officials who warned them not to embrace Islamic radicalism,
support Uighur independence or defy the Communist Party.

The officials did not ban Islam but dictated very narrow limits for how
it should be practiced, including a prohibition against praying at home
if there were friends or guests present, he said. In other sessions, the
inmates were forced to memorize laws and write essays criticizing
themselves.

``In the end, all the officials had one key point,'' he said. ``The
greatness of the Chinese Communist Party, the backwardness of Uighur
culture and the advanced nature of Chinese culture.''

After two months, Mr. Muhemet's family was finally allowed to visit the
camp, located near
``\href{https://www.thepaper.cn/newsDetail_forward_1266895}{New Harmony
Village},'' a settlement built as a symbol of friendship between ethnic
Uighurs and the majority Han Chinese. ``I couldn't say anything,'' he
recalled. ``I just held my two sons and wife, and cried and cried.''

The Xinjiang government issued
\href{http://xj.people.com.cn/n2/2017/0330/c186332-29942874.html}{``deradicalization''
rules} last year that gave vague authorization for the camps, and many
counties now run several of them, according to government documents,
including requests for bids from construction companies to build them.

Image

Police outposts and checkpoints dot the streets of Hotan every few
hundred yards. President Xi Jinping, seen on the screen above, has
overseen a security crackdown across Xinjiang.Credit...Ng Han
Guan/Associated Press

Some facilities are designed for inmates who are allowed to go home at
night. Others can house thousands around the clock. One camp outside
Hotan has grown in the past two years from a few small buildings to
facilities on at least 36 acres, larger than Alcatraz Island, and work
appears to be underway to expand it further, according to satellite
photos.

In government documents, local officials sometimes liken inmates to
patients requiring isolation and emergency intervention.

``Anyone infected with an ideological `virus' must be swiftly sent for
the `residential care' of transformation-through-education classes
before illness arises,''
\href{http://www.pinlue.com/article/2017/04/1221/211147275246.html}{a
document} issued by party authorities in Hotan said.

The number of Uighurs, as well as Kazakhs and other Muslim minorities,
who have been detained in the camps is unclear. Estimates range from
\href{https://jamestown.org/program/evidence-for-chinas-political-re-education-campaign-in-xinjiang/}{several
hundred thousand} to
\href{https://www.nytimes3xbfgragh.onion/2018/08/10/world/asia/china-xinjiang-un-uighurs.html}{perhaps
a million}, with exile Uighur groups saying the number is even higher.

About 1.5 percent of China's total population lives in Xinjiang. But the
region accounted for more than 20 percent of arrests nationwide last
year,
\href{https://www.nchrd.org/2018/07/criminal-arrests-in-xinjiang-account-for-21-of-chinas-total-in-2017/}{according
to official data} compiled by Chinese Human Rights Defenders, an
advocacy group. Those figures do not include people in the re-education
camps.

Residents said people have been sent to the camps for visiting relatives
abroad; for possessing books about religion and Uighur culture; and even
for wearing a T-shirt with a Muslim crescent. Women are sometimes
detained because of transgressions by their husbands or sons.

One official directive warns people to look for
\href{http://www.cssn.cn/zjx/zjx_zjsj/201412/t20141224_1454905.shtml}{75
signs of ``religious extremism,''} including behavior that would be
considered unremarkable in other countries: growing a beard as a young
man, praying in public places outside mosques or even abruptly trying to
give up smoking or drinking.

Image

Chinese military police at a rally last year in Hotan. Schools,
hospitals and other facilities in the city are ringed by barbed
wire.Credit...Agence France-Presse --- Getty Images

\hypertarget{we-are-in-trouble}{%
\subsection{`We Are in Trouble'}\label{we-are-in-trouble}}

Hotan feels as if under siege by an invisible enemy. Fortified police
outposts and checkpoints dot the streets every few hundred yards.
Schools, kindergartens, gas stations and hospitals are garlanded in
barbed wire. Surveillance cameras sprout from shops, apartment entrances
and metal poles.

``It's very tense here,'' a police officer said. ``We haven't rested for
three years.''

This \href{http://www.hts.gov.cn/xinxigongkai/show.php?itemid=809}{city
of 390,000} underwent a Muslim revival about a decade ago. Most Uighurs
have adhered to relatively relaxed forms of Sunni Islam, and a
significant number are secular. But budding prosperity and growing
interaction with the Middle East fueled interest in stricter Islamic
traditions. Men grew long beards, while women wore hijabs that were not
a part of traditional Uighur dress.

Now the beards and hijabs are gone, and posters warn against them.
Mosques appear poorly attended; people must register to enter and
worship under the watch of surveillance cameras.

The government shifted to harsher policies in 2009 after protests in
Xinjiang's capital, Urumqi, spiraled into rioting and left
\href{https://www.webcitation.org/5p3pHXm0l?url=http://news.xinhuanet.com/english/2009-07/18/content_11727782.htm}{nearly
200 people dead}. Mr. Xi and his regional functionaries went further,
adopting methods reminiscent of Mao's draconian rule --- mass rallies,
public confessions and ``work teams'' assigned to ferret out dissent.

They have also wired dusty towns across Xinjiang with an array of
technology that has put the region on the cutting edge of programs for
surveillance cameras as well as facial and voice recognition.
\href{http://www.xinjiangnet.com.cn/2018/0203/2044552.shtml}{Spending on
security in Xinjiang} has soared, with nearly \$8.5 billion allocated
for the police, courts and other law enforcement agencies last year,
nearly double the previous year's amount.

The campaign has polarized Uighur society. Many of the ground-level
enforcers are Uighurs themselves, including police officers and
officials who staff the camps and security checkpoints.

Ordinary Uighurs moving about Hotan sometimes shuffle on and off buses
several times to pass through metal detectors, swipe their identity
cards or hand over and unlock their mobile phones for inspection.

Image

On patrol in Hotan. ``It's very tense here,'' one police officer said.
``We haven't rested for three years.''Credit...Ben Dooley/Agence
France-Presse --- Getty Images

A resident or local cadre is assigned to monitor every 10 families in
Xinjiang, reporting on comings and goings and activities deemed
suspicious, including praying and visits to mosques, according to
residents and
\href{http://kzls.xjkunlun.cn/www.xjkzdj.cn/xwsd/bddt/2017/5502935.htm}{government
reports}. Residents said the police sometimes search homes for forbidden
books and suspect items such as prayer mats, using special equipment to
check walls and floors for hidden caches.

The authorities are also gathering biometric data and
\href{https://www.hrw.org/news/2017/12/13/china-minority-region-collects-dna-millions}{DNA}.
Two Uighurs, a former official and a student, said they were ordered to
show up at police buildings where officers recorded their voices, took
pictures of their heads at different angles and collected hair and blood
samples.

The pressure on Uighur villages intensifies when party ``work teams''
arrive and take up residence, sometimes living in local homes. The teams
ask villagers to inform on relatives, friends and neighbors, and they
investigate residents' attitudes and activities, according to government
reports published online.

One account published last year described how the authorities in one
village arranged for detainees accused of ``religious extremism'' to be
\href{http://www.cqvip.com/read/read.aspx?id=7000154900}{denounced by
their relatives} at a public rally, and encouraged other families to
report similar activities.

``More and more people are coming forward with information,'' Cao Lihai,
an editor for a party journal, wrote in the report. ``Some parents have
personally brought in their children to give themselves up.''

A Uighur woman in her 20s who asked to be identified only by her
surname, Gul, said she came under scrutiny after wearing an Islamic head
wrap and reading books about religion and Uighur history. Local
officials installed cameras at her family's door --- and inside their
living room.

``We would always have to be careful what we said and what we did and
what we read,'' she said.

Every week, Ms. Gul added, a neighborhood official visited and spent at
least two hours interrogating her. Eventually, the authorities sent her
to a full-time re-education camp.

Ms. Gul, who fled China after being released, later tried to contact her
brother to find out if he was in trouble. He sent a wordless reply, an
emoticon face in tears.

Afterward, Ms. Gul's mother sent her another message: ``Please don't
call us again. We are in trouble.''

Image

Walking past a mosque in the city of Kashgar. Muslims throughout
Xinjiang are under intense scrutiny. ``Penetration of everyday life is
almost really total now,'' one expert said.Credit...Gilles Sabrié for
The New York Times

\hypertarget{broken-families}{%
\subsection{Broken Families}\label{broken-families}}

The Chinese government says it is winning a war against Islamic
extremism and separatism, which it blames for attacks that have killed
hundreds in recent years. Information about such violence is censored
and incomplete, but incidents appear to have fallen off sharply since
2014, when the ``deradicalization'' push began.

Still, many who have emerged from the indoctrination program say it has
hardened public attitudes against Beijing.

``It was of absolutely no use,'' said Omurbek Eli, a Kazakh businessman,
of his time
\href{https://www.rfa.org/english/news/uyghur/kazakh-01302018161655.html}{held
in a camp in 2017}. ``The outcome will be the opposite. They will become
even more resistant to Chinese influence.''

For many families, the disappearance of a loved one into the camps can
be devastating, both emotionally and economically --- a point reflected
in reports posted online by the party's ``work teams.''

Some of these reports describe Uighur families
\href{https://www.meipian.cn/12nasvul}{unable to harvest crops} on their
own because so many members have been taken away, and one mentioned a
mother \href{https://www.jianshu.com/p/35d75dd53642}{left to care for
five children}. In another report, an official near Hotan
\href{https://www.meipian.cn/j1ja93p}{described holding a village
meeting} to calm distraught relatives of those sent to the camps.

The mass internments also break Uighur families by forcing members to
disown their kin or by separating small children from their parents. So
many parents have been detained in Kashgar, a city in western Xinjiang,
that it has \href{http://www.xjks.gov.cn/Item/41128.aspx}{expanded}
boarding schools to take custody of older, ``troubled'' children.

``Whether consciously or unconsciously, authorities in Xinjiang have
recognized the power of families as an alternative source of
authority,'' said
\href{http://cas.loyno.edu/history/bios/rian-thum}{Rian Thum}, a
professor at Loyola University in New Orleans who has
\href{https://www.nytimes3xbfgragh.onion/2018/05/15/opinion/china-re-education-camps.html}{followed
the detentions}. ``The kind of extreme party loyalty they want has no
room for that.''

Ms. Gul said the camp she was in was ramshackle enough that children who
lived nearby sometimes crept up to a window late at night and called out
to their mothers inside. ``Their children would come and say, `Mother, I
miss you,''' she said.

``We didn't say anything,'' she added, ``because there was a camera
inside the cell.''

Advertisement

\protect\hyperlink{after-bottom}{Continue reading the main story}

\hypertarget{site-index}{%
\subsection{Site Index}\label{site-index}}

\hypertarget{site-information-navigation}{%
\subsection{Site Information
Navigation}\label{site-information-navigation}}

\begin{itemize}
\tightlist
\item
  \href{https://help.nytimes3xbfgragh.onion/hc/en-us/articles/115014792127-Copyright-notice}{©~2020~The
  New York Times Company}
\end{itemize}

\begin{itemize}
\tightlist
\item
  \href{https://www.nytco.com/}{NYTCo}
\item
  \href{https://help.nytimes3xbfgragh.onion/hc/en-us/articles/115015385887-Contact-Us}{Contact
  Us}
\item
  \href{https://www.nytco.com/careers/}{Work with us}
\item
  \href{https://nytmediakit.com/}{Advertise}
\item
  \href{http://www.tbrandstudio.com/}{T Brand Studio}
\item
  \href{https://www.nytimes3xbfgragh.onion/privacy/cookie-policy\#how-do-i-manage-trackers}{Your
  Ad Choices}
\item
  \href{https://www.nytimes3xbfgragh.onion/privacy}{Privacy}
\item
  \href{https://help.nytimes3xbfgragh.onion/hc/en-us/articles/115014893428-Terms-of-service}{Terms
  of Service}
\item
  \href{https://help.nytimes3xbfgragh.onion/hc/en-us/articles/115014893968-Terms-of-sale}{Terms
  of Sale}
\item
  \href{https://spiderbites.nytimes3xbfgragh.onion}{Site Map}
\item
  \href{https://help.nytimes3xbfgragh.onion/hc/en-us}{Help}
\item
  \href{https://www.nytimes3xbfgragh.onion/subscription?campaignId=37WXW}{Subscriptions}
\end{itemize}
