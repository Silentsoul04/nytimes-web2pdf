Sections

SEARCH

\protect\hyperlink{site-content}{Skip to
content}\protect\hyperlink{site-index}{Skip to site index}

Americans Want to Believe Jobs Are the Solution to Poverty. They're Not.

\url{https://nyti.ms/2MkTNLc}

\begin{itemize}
\item
\item
\item
\item
\item
\item
\end{itemize}

\includegraphics{https://static01.graylady3jvrrxbe.onion/images/2018/09/16/magazine/16Poverty1/16Poverty1-articleLarge.png?quality=75\&auto=webp\&disable=upscale}

Feature

\hypertarget{americans-want-to-believe-jobs-are-the-solution-to-poverty-theyre-not}{%
\section{Americans Want to Believe Jobs Are the Solution to Poverty.
They're
Not.}\label{americans-want-to-believe-jobs-are-the-solution-to-poverty-theyre-not}}

U.S. unemployment is down and jobs are going unfilled. But for people
without much education, the real question is: Do those jobs pay enough
to live on?

The house of Vanessa Solivan's mother, right of garden.Credit...Devin
Yalkin for The New York Times

Supported by

\protect\hyperlink{after-sponsor}{Continue reading the main story}

By Matthew Desmond

\begin{itemize}
\item
  Sept. 11, 2018
\item
  \begin{itemize}
  \item
  \item
  \item
  \item
  \item
  \item
  \end{itemize}
\end{itemize}

Vanessa Solivan and her three children fled their last place in June
2015, after a young man was shot and killed around the corner. They
found a floor to sleep on in Vanessa's parents' home on North Clinton
Avenue in East Trenton. It wasn't a safer neighborhood, but it was a
known one. Vanessa took only what she could cram into her station wagon,
a 2004 Chrysler Pacifica, letting the bed bugs have the rest.

At her childhood home, Vanessa began caring for her ailing father. He
had been a functional crack addict for most of her life, working as a
landscaper in the warmer months and collecting unemployment when
business slowed down. ``It was something you got used to seeing,''
Vanessa said about her father's drug habit. ``My dad was a junkie, but
he never left us.'' Vanessa, 33, has black hair that is usually pulled
into a bun and wire-framed glasses that slide down her nose; a shy smile
peeks out when she feels proud of herself.

Vanessa's father died a year after Vanessa moved in. The family erected
a shrine to him in the living room, a faded, large photo of a younger
man surrounded by silk flowers and slowly sinking balloons. Vanessa's
mother, Zaida, is 62 and from Puerto Rico, as was her husband. She uses
a walker to get around. Her husband's death left her with little income,
and Vanessa was often broke herself. Her health failing, Zaida could
take only so much of Vanessa's children, Taliya, 17, Shamal, 14, and
Tatiyana, 12. When things got too loud or one of her grandchildren gave
her lip, she would ask Vanessa to take her children somewhere else.

If Vanessa had the money, or if a local nonprofit did, she would book a
motel room. She liked the Red Roof Inn, which she saw as ``more
civilized'' than many of the other motels she had stayed in. It looked
like a highway motel: two stories with doors that opened to the outside.
The last time the family checked in, the kids carried their homework up
to the room as Vanessa followed with small grocery bags from the food
pantry, passing two men sipping Modelos and apologizing for their loud
music. Inside their room, Vanessa placed her insulin in the minifridge
as her children chose beds, where they would sleep two to a mattress.
Then she slid into a small chair, saying, ``Y'all don't know how tired
Mommy is.'' After a quiet moment, Vanessa reached over and rubbed
Shamal's back, telling him, ``I wish we had a nice place like this.''
Then her eye spotted a roach feeling its way over the stucco wall.

``Op! Not too nice,'' Vanessa said, grinning. With a flick, she sent the
bug flying toward Taliya, who squealed and jerked back. Laughter burst
from the room.

When Vanessa couldn't get a motel, the family spent the night in the
Chrysler. The back of the station wagon held the essentials: pillows and
blankets, combs and toothbrushes, extra clothes, jackets and
nonperishable food. But there were also wrinkled photos of her kids. One
showed Taliya at her eighth-grade graduation in a cream dress holding
flowers. Another showed all three children at a quinceañera --- Shamal
kneeling in front, with a powder blue clip-on bow tie framing his baby
face, and Tatiyana tucked in back with a deep-dimpled smile.

\includegraphics{https://static01.graylady3jvrrxbe.onion/images/2018/09/16/magazine/16Poverty3/16Poverty3-articleLarge.png?quality=75\&auto=webp\&disable=upscale}

So that the kids wouldn't run away out of anger or shame, Vanessa
learned to park off Route 1, in crevices of the city that were so still
and abandoned that no one dared crack a door until daybreak. Come
morning, Vanessa would drive to her mother's home so the kids could get
ready for school and she could get ready for work.

In May, Vanessa finally secured a spot in public housing. But for almost
three years, she had belonged to the ``working homeless,'' a
now-necessary phrase in today's low-wage/high-rent society. She is a
home health aide, the same job her mother had until her knees and back
gave out. Her work uniform is Betty Boop scrubs, sneakers and an ID
badge that hangs on a red Bayada Home Healthcare lanyard. Vanessa works
steady hours and likes her job, even the tougher bits like bathing the
infirm or hoisting someone out of bed with a Hoyer lift. ``I get to help
people,'' she said, ``and be around older people and learn a lot of
stuff from them.'' Her rate fluctuates: She gets \$10 an hour for one
client, \$14 for another. It doesn't have to do with the nature of the
work --- ``Sometimes the hardest ones can be the cheapest ones,''
Vanessa said --- but with reimbursement rates, which differ according to
the client's health care coverage. After juggling the kids and managing
her diabetes, Vanessa is able to work 20 to 30 hours a week, which earns
her around \$1,200 a month. And that's when things go well.

\textbf{These days,} we're told that the American economy is strong.
Unemployment is down, the Dow Jones industrial average is north of
25,000 and millions of jobs are going unfilled. But for people like
Vanessa, the question is not, Can I land a job? (The answer is almost
certainly, Yes, you can.) Instead the question is, What kinds of jobs
are available to people without much education? By and large, the answer
is: jobs that do not pay enough to live on.

In recent decades, the nation's tremendous economic growth has not led
to broad social uplift. Economists call it the ``productivity-pay gap''
--- the fact that over the last 40 years, the economy has expanded and
corporate profits have risen, but real wages have remained flat for
workers without a college education. Since 1973, American productivity
has increased by 77 percent, while hourly pay has grown by only 12
percent. If the federal minimum wage tracked productivity, it would be
more than \$20 an hour, not today's poverty wage of \$7.25.

American workers are being shut out of the profits they are helping to
generate. The decline of unions is a big reason. During the 20th
century, inequality in America decreased when unionization increased,
but economic transformations and political attacks have crippled
organized labor, emboldening corporate interests and disempowering the
rank and file. This imbalanced economy explains why America's poverty
rate has remained consistent over the past several decades, even as per
capita welfare spending has increased. It's not that safety-net programs
don't help; on the contrary, they lift millions of families above the
poverty line each year. But one of the most effective antipoverty
solutions is a decent-paying job, and those have become scarce for
people like Vanessa. Today, 41.7 million laborers --- nearly a third of
the American work force --- earn less than \$12 an hour, and almost none
of their employers offer health insurance.

The Bureau of Labor Statistics defines a ``working poor'' person as
someone below the poverty line who spent at least half the year either
working or looking for employment. In 2016, there were roughly 7.6
million Americans who fell into this category. Most working poor people
are over 35, while fewer than five in 100 are between the ages of 16 and
19. In other words, the working poor are not primarily teenagers bagging
groceries or scooping ice cream in paper hats. They are adults --- and
often parents --- wiping down hotel showers and toilets, taking food
orders and bussing tables, eviscerating chickens at meat-processing
plants, minding children at 24-hour day care centers, picking berries,
emptying trash cans, stacking grocery shelves at midnight, driving taxis
and Ubers, answering customer-service hotlines, smoothing hot asphalt on
freeways, teaching community-college students as adjunct professors and,
yes, bagging groceries and scooping ice cream in paper hats.

America prides itself on being the country of economic mobility, a place
where your station in life is limited only by your ambition and grit.
But changes in the labor market have shrunk the already slim odds of
launching yourself from the mailroom to the boardroom. For one, the job
market has bifurcated, increasing the distance between good and bad
jobs. Working harder and longer will not translate into a promotion if
employers pull up the ladders and offer supervisory positions
exclusively to people with college degrees. Because large companies now
farm out many positions to independent contractors, those who buff the
floors at Microsoft or wash the sheets at the Sheraton typically are not
employed by Microsoft or Sheraton, thwarting any hope of advancing
within the company. Plus, working harder and longer often isn't even an
option for those at the mercy of an unpredictable schedule. Nearly 40
percent of full-time hourly workers know their work schedules just a
week or less in advance. And if you give it your all in a job you can
land with a high-school diploma (or less), that job might not exist for
very long: Half of all new positions are eliminated within the first
year. According to the labor sociologist Arne Kalleberg, permanent
terminations have become ``a basic component of employers' restructuring
strategies.''

Home health care has emerged as an archetypal job in this new, low-pay
service economy. Demand for home health care has surged as the
population has aged, but according to the latest data from the Bureau of
Labor Statistics, the 2017 median annual income for home health aides in
the United States was just \$23,130. Half of these workers depend on
public assistance to make ends meet. Vanessa formed a rapport with
several of her clients, to whom she confided that she was homeless. One
replied, ``Oh, Vanessa, I wish I could do something for you.'' When
Vanessa told her supervisor about her situation, he asked if she wanted
time off. ``No!'' Vanessa said. She needed the money and had been
picking up fill-in shifts. The supervisor was prepared for the moment;
he'd been there before. He reached into a drawer and gave her a \$50 gas
card to Shell and a \$100 grocery card to ShopRite. Vanessa was grateful
for the help. She thought Bayada was a generous and sympathetic
employer, but her rate hadn't changed much in the three years she had
worked there. Vanessa earned \$9,815.75 in 2015, \$12,763.94 in 2016 and
\$10,446.81 last year.

To afford basic necessities, the federal government estimates that
Vanessa's family would need to bring in \$29,420 a year. Vanessa is not
even close --- and she is one of the lucky ones, at least among the
poor. The nation's safety net now strongly favors the employed, with
benefits like the earned-income tax credit, a once-a-year cash boost
that applies only to people who work. Last year, Vanessa received a tax
refund of around \$5,000, which included earned-income and child tax
credits. They helped raise her income, but not above the poverty line.
If the working poor are doing better than the nonworking poor, which is
the case, it's not so much because of their jobs per se, but because
their employment status provides them access to desperately needed
government help. This has caused growing inequality below the poverty
line, with the working poor receiving much more social aid than the
abandoned nonworking poor or the precariously employed, who are plunged
into destitution.

When life feels especially grinding, Vanessa often rings up Sheri
Sprouse, her best friend since middle school. ``She's like me,'' Vanessa
said. ``She's strong.'' Sheri is a reserve of emotional support and
perspective, often encouraging her friend to be patient and grateful for
what she has. But Sheri herself is also just scraping by, raising two
daughters on a fixed disability check. And because Sheri's housing is
subsidized through a federally administered voucher, it is also
monitored. ``With Section 8, you can't have people staying with you,''
Vanessa said. ``So I wouldn't want to mess that up.'' When Vanessa was
homeless, Sheri couldn't offer her much else besides love.

Vanessa received some help last year, when her youngest child, Tatiyana,
was approved for Supplemental Security Income because of a learning
disability. Vanessa began receiving a monthly \$766 disability check.
But when the Mercer County Board of Social Services learned of this
additional money, it sent Vanessa a letter announcing that her
Supplemental Nutrition Assistance Program benefits would be reduced to
\$234 from \$544. Food was a constant struggle, and this news didn't
help. A 2013 study by Oxfam America found that two-thirds of working
poor people worry about being able to afford enough food. When Vanessa
stayed at a hotel, her food options were limited to what she could heat
in the microwave; when she slept in her car, the family had to settle
for grab-and-go options, which tend to be more expensive. Sometimes
Vanessa stopped by a bodega and ordered four chicken-and-rice dishes for
\$15. Sometimes her kids went to school hungry. ``I just didn't have
nothing,'' Vanessa told me one morning. For dinner, she planned to stop
by a food pantry, hoping they still had the mac-and-cheese that Shamal
liked.

\emph{\textbf{In America,}} \emph{if you work hard, you will succeed. So
those who do not succeed have not worked hard.} It's an idea found deep
in the marrow of the nation. William Byrd, an 18th-century Virginia
planter, wrote of poor men who were ``intolerable lazy'' and ``Sloathful
in everything but getting of Children.'' Thomas Jefferson advocated
confinement in poorhouses for vagabonds who ``waste their time in idle
and dissolute courses.'' Leap into the 20th century, and there's Barry
Goldwater saying that Americans with little education exhibit ``low
intelligence or low ambition'' and Ronald Reagan disparaging ``welfare
queens.'' In 2004, Bill O'Reilly said of poor people: ``You gotta look
people in the eye and tell 'em they're irresponsible and lazy,'' and
then continued, ``Because that's what poverty is, ladies and
gentlemen.''

Americans often assume that the poor do not work. According to a 2016
survey conducted by the American Enterprise Institute, nearly two-thirds
of respondents did not think most poor people held a steady job; in
reality, that year a majority of nondisabled working-age adults were
part of the labor force. Slightly over one-third of respondents in the
survey believed that most welfare recipients would prefer to stay on
welfare rather than earn a living. These sorts of assumptions about the
poor are an American phenomenon. A 2013 study by the sociologist Ofer
Sharone found that unemployed workers in the United States blame
themselves, while unemployed workers in Israel blame the hiring system.
When Americans see a homeless man cocooned in blankets, we often wonder
how he failed. When the French see the same man, they wonder how the
state failed him.

If you believe that people are poor because they are not working, then
the solution is not to make work pay but to make the poor work --- to
force them to clock in somewhere, anywhere, and log as many hours as
they can. But consider Vanessa. Her story is emblematic of a larger
problem: the fact that millions of Americans work with little hope of
finding security and comfort. In recent decades, America has witnessed
the rise of bad jobs offering low pay, no benefits and little certainty.
When it comes to poverty, a willingness to work is not the problem, and
work itself is no longer the solution.

Image

Vanessa in the living room of her mother's house with
Tatiyana.Credit...Devin Yalkin for The New York Times

Until the late 18th century, poverty in the West was considered not only
durable but desirable for economic growth. Mercantilism, the dominant
economic theory of the early modern period, held that hunger
incentivized work and kept wages low. Wards of public charity were
jailed and required to work to eat. In the current era, politicians and
their publics have continued to demand toil and sweat from the poor. In
the 1980s, conservatives wanted to attach work requirements to food
stamps. In the 1990s, they wanted to impose work requirements on
subsidized-housing programs. Both proposals failed, but the impulse has
endured.

Advocates of work requirements scored a landmark victory with welfare
reform in the mid-1990s. Proposed by House Republicans, led by Speaker
Newt Gingrich, and signed into law by President Bill Clinton, welfare
reform affixed work requirements and time limits to cash assistance.
Caseloads fell to 4.5 million in 2011 from 12.3 million in 1996. Did
``welfare to work'' in fact work? Was it a major success in reducing
poverty and sowing prosperity? Hardly. As Kathryn Edin and Laura Lein
showed in their landmark book, ``Making Ends Meet,'' single mothers
pushed into the low-wage labor market earned more money than they did on
welfare, but they also incurred more expenses, like transportation and
child care, which nullified modest income gains. Most troubling, without
guaranteed cash assistance for the most needy, extreme poverty in
America surged. The number of Americans living on only \$2 or less per
person per day has more than doubled since welfare reform. Roughly three
million children --- which exceeds the population of Chicago --- now
suffer under these conditions. Most of those children live with an adult
who held a job sometime during the year.

A top priority for the Trump administration is expanding work
requirements for some of the nation's biggest safety-net programs. In
January, the federal government announced that it would let states
require that Medicaid recipients work. A dozen states have formally
applied for a federal waiver to affix work requirements to their
Medicaid programs. Four have been approved. In June, Arkansas became the
first to implement newly approved work requirements. If all states
instated Medicaid work requirements similar to that of Arkansas, as many
as four million Americans could lose their health insurance.

In April, President Trump issued an executive order mandating that
federal agencies review welfare programs, from the Supplemental
Nutrition Assistance Program to housing assistance, and propose new
standards. Although SNAP already has work requirements, in June the
House passed a draft farm bill that would deny able-bodied adults SNAP
benefits for an entire year if they did not work or engage in
work-related activities (like job training) for at least 20 hours a week
during a single month. Falling short a second time could get you barred
for three years. The Senate's farm bill, a bipartisan effort, removed
these rules and stringent penalties, setting up a showdown with the
House, whose version Trump has endorsed. The Congressional Budget Office
estimates that work requirements could deny 1.2 million people a benefit
that they use to eat.

Work requirements affixed to other programs make similar demands.
Kentucky's proposed Medicaid requirements are satisfied only after 80
hours of work or work-related training each month. In a low-wage labor
market characterized by fluctuating hours, tenuous employment and
involuntary part-time work, a large share of vulnerable workers fall
short of these requirements. Nationally representative data from the
Survey of Income and Program Participation show that among workers who
qualify for Medicaid, almost 50 percent logged fewer than 80 hours in at
least one month.

In July, the White House Council of Economic Advisers issued a report
enthusiastically endorsing work requirements for the nation's largest
welfare programs. The council favored ``negative incentives,'' tying aid
to labor-market effort, and dismissed ``positive incentives,'' like tax
benefits for low-income workers, because the former is cheaper. The
council also claimed that America's welfare policies have brought about
a ``decline in self-sufficiency.''

Is that true? Researchers set out to study welfare dependency in the
1980s and 1990s, when this issue dominated public debate. They didn't
find much evidence of it. Most people started using cash welfare after a
divorce or separation and didn't stay long on the dole, even if they
returned to welfare periodically. One study found that 90 percent of
young women on welfare stopped relying on it within two years of
starting the program, but most of them returned to welfare sometime down
the road. Even at its peak, welfare did not function as a dependency
trap for a majority of recipients; rather, it was something people
relied on when they were between jobs or after a family crisis. A 1988
review in Science concluded that ``the welfare system does not foster
reliance on welfare so much as it acts as insurance against temporary
misfortune.''

Image

Vanessa and her client Laura at Laura's home in Hamilton,
N.J.Credit...Devin Yalkin for The New York Times

Today as then, the able-bodied, poor and idle adult remains a rare
creature. According to the Brookings Institution, in 2016 one-third of
those living in poverty were children, 11 percent were elderly and 24
percent were working-age adults (18 to 64) in the labor force, working
or seeking work. The majority of working-age poor people connected to
the labor market were part-time workers. Most couldn't take on many more
hours either because of caregiver responsibilities, as with Vanessa, or
because their employer didn't offer this option, rendering them
involuntary part-time workers. Among the remaining working-age adults,
12 percent were out of the labor force owing to a disability (including
some enrolled in federal programs that limit work), 15 percent were
either students or caregivers and 3 percent were early retirees. That
leaves 2 percent of poor people who did not fit into one of these
categories. That is, among the poor, two in 100 are working-age adults
disconnected from the labor market for unknown reasons. The nonworking
poor person getting something for nothing is a lot like the cheat
committing voter fraud: pariahs who loom far larger in the American
imagination than in real life.

\textbf{When Vanessa was} not working for Bayada, she was running after
her kids. Vanessa worried over Shamal the most. At more than six feet
tall, his size made him both a tool and a target in the neighborhood.
Smaller kids wanted him to be their enforcer or trouble-starter. Harder
kids saw him as a threat. Last year, Shamal was suspended twice for
fighting. As punishment, Vanessa made him shave off his prized Afro. But
she also set her children's outbursts against a larger backdrop. ``How's
their behavior supposed to be when we're out here on these streets?''
she asked me in frustration. Shamal once told me that outsiders
``probably think I'm selling drugs. But I'm not. I'm just a cool person
that likes hanging out and making people laugh.'' He wanted to become a
chef. Vanessa wondered if she could get Shamal a police-issued ankle
bracelet, which would track his movements. It was impossible, of course,
but Shamal liked the idea. ``It could help me when my friends want me to
go somewhere,'' he told me. That is, the bracelet would give him a good
excuse to back down when his friends nudged him toward a risky path.

Shamal and Tatiyana's father had recently moved back to Trenton,
``carrying a sack like a hobo,'' Vanessa remembered. Other than erratic
child-support payments and a single trip to Chuck E. Cheese's, he
doesn't play much of a role in his children's lives. Taliya's father
went to prison when she was 1. He was released when she was 8 and was
killed a few months later, shot in the chest. Sometimes Vanessa's three
kids teased one another about their fathers. ``Your dad is dead,''
Tatiyana would say. ``Yeah? Your dad's around, but he don't give a crap
about you,'' Taliya would shoot back.

Other times, though, the siblings offered soft reassurances that their
fathers' absence wasn't their fault. ``I don't have time for him,''
Tatiyana said once, as if it were her choice. ``I have time for my real
friends.'' Taliya looked at her baby sister and replied: ``Watch. When
you're doing good, he gonna start coming around.''

If Vanessa clocked more hours, it would be difficult to keep up with all
the ways she manages her family: doing the laundry, arranging dentist
appointments, counseling the children about sex, studying their deep
mysteries to extract their gifts and troubles. Yet our political leaders
tend to refuse to view child care as work. During the early days of
welfare reform, some local authorities thought up useless jobs for
single mothers receiving the benefit. In one outrageous case, recipients
were made to sort small plastic toys into different colors, only to have
their supervisor end the day by mixing everything up, so the work could
start anew the next morning. This was thought more important than
keeping children safe and fed.

Caring for a sick or dying parent doesn't count either. Vanessa spooned
\emph{arroz con gandules} into her ailing father's mouth, refilled his
medications and emptied his bedpan. But only when she does these things
for virtual strangers, as a Bayada employee, does she ``work'' and
therefore become worthy of concern. As Evelyn Nakano Glenn argues in her
2010 book, ``Forced to Care,'' industrialization caused American
families to become increasingly reliant on wages, which had the effect
of reducing tasks that usually fell to women (homemaking, cooking, child
care) to ``moral and spiritual vocations.'' ``In contrast to men's paid
labor,'' Glenn writes, ``women's unpaid caring was simultaneously
priceless and worthless --- that is, not monetized.'' She continues:
``To add insult to injury, because they could not live up to the ideal
of full-time motherhood, poor women of color were seen as deficient
mothers and caregivers.''

Vanessa attributed her own academic setbacks --- a good student in
middle school, she began cutting class and courting trouble in high
school --- to the fact that her parents were checked out. At a critical
juncture when Vanessa needed guidance and discipline, her father was
using drugs and her mother seemed always to be at work. She didn't want
to make the same mistake with her kids. Vanessa's life revolved around a
small routine: drop the kids off at school; work; try finding an
apartment that rents for less than \$1,000 a month; pick the kids up;
feed them; sleep. She didn't spend her money on extras, including
cigarettes and alcohol. She was trying to save ``the little money that I
got,'' she told me, ``so when we do get a place, I can get the kids
washcloths and towels.''

Image

Vanessa and Laura going to buy groceries.Credit...Devin Yalkin for The
New York Times

\textbf{We might think} that the existence of millions of working poor
Americans like Vanessa would cause us to question the notion that
indolence and poverty go hand in hand. But no. While other
inequality-justifying myths have withered under the force of collective
rebuke, we cling to this devastatingly effective formula. Most of us
lack a confident account for increasing political polarization, rising
prescription drug costs, urban sprawl or any number of social ills. But
ask us why the poor are poor, and we have a response quick at the ready,
grasping for this palliative of explanation. We have to, or else the
national shame would be too much to bear. How can a country with such a
high poverty rate --- higher than those in Latvia, Greece, Poland,
Ireland and all other member countries of the Organization for Economic
Cooperation and Development --- lay claim to being the greatest on
earth? Vanessa's presence is a judgment. But rather than hold itself
accountable, America reverses roles by blaming the poor for their own
miseries.

Here is the blueprint. First, valorize work as the ticket out of
poverty, and debase caregiving as not work. Look at a single mother
without a formal job, and say she is not working; spot one working part
time and demand she work more. Transform love into laziness. Next, force
the poor to log more hours in a labor market that treats them as
expendables. Rest assured that you can pay them little and deny them
sick time and health insurance because the American taxpayer will step
in, subsidizing programs like the earned-income tax credit and food
stamps on which your work force will rely. Watch welfare spending
increase while the poverty rate stagnates because, well, you are
hoarding profits. When that happens, skirt responsibility by blaming the
safety net itself. From there, politicians will invent new ways of
denying families relief, like slapping unrealistic work requirements on
aid for the poor.

Democrats may scoff at Republicans' work requirements, but they have yet
to challenge the dominant conception of poverty that feeds such
meanspirited politics. Instead of offering a counternarrative to
America's moral trope of deservedness, liberals have generally submitted
to it, perhaps even embraced it, figuring that the public will not
support aid that doesn't demand that the poor subject themselves to the
low-paying jobs now available to them. Even stalwarts of the progressive
movement seem to reserve economic prosperity for the full-time worker.
Senator Bernie Sanders once declared, echoing a long line of Democrats
who have come before and after him, ``Nobody who works 40 hours a week
should be living in poverty.'' Sure, but what about those who work 20 or
30 hours, like Vanessa?

Because liberals have allowed conservatives to set the terms of the
poverty debate, they find themselves arguing about radical solutions
that imagine either a fully employed nation (like a jobs guarantee) or a
postwork society (like a universal basic income). Neither plan has the
faintest hope of being actually implemented nationwide anytime soon,
which means neither is any good to Vanessa and millions like her. When
so much attention is spent on far-off, utopian solutions, we neglect the
importance of the poverty fixes we already have. Safety-net programs
that help families confront food insecurity, housing unaffordability and
unemployment spells lift tens of millions of people above the poverty
line each year. By itself, SNAP annually pulls over eight million people
out of poverty. According to a 2015 study, without federal tax benefits
and transfers, the number of Americans living in deep poverty (half
below the poverty threshold) would jump from 5 percent to almost 19
percent. Effective social-mobility programs should be championed,
expanded and stripped of draconian work requirements.

While Washington continues to require more of vulnerable workers, it has
required little from employers in the form of living wages or job
security, creating a labor market in which the biggest disincentive to
work is not welfare but the lousy jobs that are available. Judging from
the current state of the nation's poverty agenda, it appears that most
people creating federal and state policy don't know many people like
Vanessa. ``Half of the people in City Hall don't even live in Trenton,''
Vanessa once told me, flustered. ``They don't even know what goes on
here.'' Meanwhile, this is the richest Congress on record, with one in
13 members belonging to the top 1 percent. From such a high perch,
poverty appears a smaller problem, something less gutting, and work
appears a bigger solution, something more gratifying. But when we shrink
the problem, the solution shrinks with it; when small solutions are
applied to a huge problem, they don't work; and when weak antipoverty
initiatives don't work, many throw up their hands and argue that we
should stop tossing money at the problem altogether. Cheap solutions
only cheapen the problem.

This month, I had dinner with first-year honors students at a university
in Massachusetts. Some leaned right, others left. But all of them were
united in their inability to explain poverty in a way that didn't
somehow hold the poor responsible for their predicament. Poor people
lacked work ethic, they told me, or maybe a strong backbone or a
commitment to a better life. I began to regret that alcohol hadn't been
served when one student brought up the movie ``The Pursuit of
Happyness,'' in which Will Smith's character performs superhumanly well
at his job to leap from homelessness to affluence. The student was no
senator's son: He told us that times were lean after his parents
divorced. As I watched this young man identify with Smith's character,
it dawned on me that what his parents, preachers, teachers, coaches and
guidance counselors had told him for motivation --- ``Study hard, stick
to it, dream big and you will be successful'' --- had been internalized
as a theory of life.

We need a new language for talking about poverty. ``Nobody who works
should be poor,'' we say. That's not good enough. Nobody in America
should be poor, period. No single mother struggling to raise children on
her own; no formerly incarcerated man who has served his time; no young
heroin user struggling with addiction and pain; no retired bus driver
whose pension was squandered; nobody. And if we respect hard work, then
we should reward it, instead of deploying this value to shame the poor
and justify our unconscionable and growing inequality. ``I've worked
hard to get where I am,'' you might say. Well, sure. But Vanessa has
worked hard to get where she is, too.

Advertisement

\protect\hyperlink{after-bottom}{Continue reading the main story}

\hypertarget{site-index}{%
\subsection{Site Index}\label{site-index}}

\hypertarget{site-information-navigation}{%
\subsection{Site Information
Navigation}\label{site-information-navigation}}

\begin{itemize}
\tightlist
\item
  \href{https://help.nytimes3xbfgragh.onion/hc/en-us/articles/115014792127-Copyright-notice}{©~2020~The
  New York Times Company}
\end{itemize}

\begin{itemize}
\tightlist
\item
  \href{https://www.nytco.com/}{NYTCo}
\item
  \href{https://help.nytimes3xbfgragh.onion/hc/en-us/articles/115015385887-Contact-Us}{Contact
  Us}
\item
  \href{https://www.nytco.com/careers/}{Work with us}
\item
  \href{https://nytmediakit.com/}{Advertise}
\item
  \href{http://www.tbrandstudio.com/}{T Brand Studio}
\item
  \href{https://www.nytimes3xbfgragh.onion/privacy/cookie-policy\#how-do-i-manage-trackers}{Your
  Ad Choices}
\item
  \href{https://www.nytimes3xbfgragh.onion/privacy}{Privacy}
\item
  \href{https://help.nytimes3xbfgragh.onion/hc/en-us/articles/115014893428-Terms-of-service}{Terms
  of Service}
\item
  \href{https://help.nytimes3xbfgragh.onion/hc/en-us/articles/115014893968-Terms-of-sale}{Terms
  of Sale}
\item
  \href{https://spiderbites.nytimes3xbfgragh.onion}{Site Map}
\item
  \href{https://help.nytimes3xbfgragh.onion/hc/en-us}{Help}
\item
  \href{https://www.nytimes3xbfgragh.onion/subscription?campaignId=37WXW}{Subscriptions}
\end{itemize}
