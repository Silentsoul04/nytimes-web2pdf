Sections

SEARCH

\protect\hyperlink{site-content}{Skip to
content}\protect\hyperlink{site-index}{Skip to site index}

\href{https://myaccount.nytimes3xbfgragh.onion/auth/login?response_type=cookie\&client_id=vi}{}

\href{https://www.nytimes3xbfgragh.onion/section/todayspaper}{Today's
Paper}

Letter of Recommendation: Phyllis Rose's `Parallel Lives'

\url{https://nyti.ms/2N1o5Dd}

\begin{itemize}
\item
\item
\item
\item
\item
\item
\end{itemize}

Advertisement

\protect\hyperlink{after-top}{Continue reading the main story}

Supported by

\protect\hyperlink{after-sponsor}{Continue reading the main story}

\href{/column/letter-of-recommendation}{Letter of Recommendation}

\hypertarget{letter-of-recommendation-phyllis-roses-parallel-lives}{%
\section{Letter of Recommendation: Phyllis Rose's `Parallel
Lives'}\label{letter-of-recommendation-phyllis-roses-parallel-lives}}

\includegraphics{https://static01.graylady3jvrrxbe.onion/images/2018/09/30/magazine/30Mag-LOR/30Mag-LOR-articleLarge.jpg?quality=75\&auto=webp\&disable=upscale}

By Haley Mlotek

\begin{itemize}
\item
  Sept. 25, 2018
\item
  \begin{itemize}
  \item
  \item
  \item
  \item
  \item
  \item
  \end{itemize}
\end{itemize}

This summer I asked almost everyone I met to tell me the story of his or
her life. (We were in the right season for gossip --- confidences rise
and fall with the humidity index.) I was not so concerned with honesty.
``True story'' is already something of an oxymoron. Instead I wanted the
character arcs people wrote for themselves, tales that ended with the
moment they sat down on a chair in front of me. The good gossips among
them knew to deflect --- ``Tell me the story of your life'' --- but no
matter who went first, we always started, or claimed to start, at the
beginning.

I began asking people this question after reading the book ``Parallel
Lives: Five Victorian Marriages,'' by Phyllis Rose. It was published, to
critical acclaim, in 1983 --- the second of several books by Rose, now a
professor emerita of English at Wesleyan University. Nora Ephron once
said she read it every four or five years, an impulse I understand but a
frequency that feels insufficient: Since my first reading, I've been
averaging every four or five \emph{months}. I would read a sentence
once, considering it insightful but obvious enough; then, on a second or
fourth or seventh reading, the same sentence would uncoil itself,
revealing something almost unsettlingly perceptive.

For example: ``It is, of course, one of life's persistent
disappointments that a great moral crisis in my life is nothing but
matter for gossip in yours.''

That sentence made me gasp the first time I read it and makes me want to
lie down every time I remember it. It's so clearly true: People have a
skill for reducing the shame and sadness of others into easy witticisms.
I had always suspected there were two kinds of people in the world:
those who will admit to gossiping about the end of a marriage with the
same gravity they would bring to ordering appetizers, and liars. Rose
crystallizes both why we gossip and why we don't want to admit to it.
We're revealing our standards, pretending to judge vice in others while
demonstrating the quality of our virtues.

``Parallel Lives'' is a group biography of several notable Victorians
and their marriages, including Charles Dickens, George Eliot, John
Stuart Mill and various much-beloved or long-suffering or
less-remembered spouses and paramours. It follows the ways in which
people without easy access to divorce got very creative (which is to say
very weird) in negotiating their breakups and makeups. Rose chose
writers not because they are more inventive in living --- far from it
--- but because they tend to be diligent reporters of their lives,
leaving us more material to work with. Some of the famous figures have
reputations that precede them, which Rose is careful to correct. Did the
art critic John Ruskin turn celibate after seeing an actual adult woman
naked for the first time? Well, not exactly, but his marriage did remain
unconsummated. Did Dickens have an affair with his wife's sister? Well,
not exactly, but he was definitely a jerk, issuing a news release
denying ``whispered rumors'' about his ``domestic trouble'' even as he
sent his wife away and began an affair with a young actress. (As far as
the public was concerned, the statement was a confirmation beyond rumor
of his infidelity.)

The thrill of Rose's book is seeing how those scandalous details became
the foundation of actual societal change. To Rose, marriage is the
primary political experience of adulthood, as intimate a contract with
society as it is with your partner. Every society arranges strictures
around the family, and often it's by looking at individual relationships
that we see how and where those strictures failed and what might replace
them. It would be flattering to believe such codes were an antiquated
mania we've long been free of, but that's not quite true. It is, after
all, still just five years since gay marriage became nationally
available, eight since New York became the last state with no-fault
divorce, 51 since the Supreme Court struck down bans on interracial
marriage.

Comparing the stories of how people live is how we start cracking these
codes --- how we discover new ideas about what a good life can look
like. The search for more complex plots is the search for more ways of
being a person. Marriages fail and couples split for many reasons, but
this narration is always there: As long as we have our story straight,
the couple says, we have an us. A good gossip knows that to hear these
tales is not, on its own, invasive --- that the story is as much for the
couple as their presumed audience. Bad gossips, on the other hand, are
just like bad readers: inattentive and unimaginative. They believe that
stories exist to be solved and that behavior should be rated bad or
good. I cannot help such people and have learned the hard way not to
try.

This is what ``Parallel Lives'' will teach you: Intimacy is easy.
Honesty is much harder. Gossip --- where we reveal what we think is true
about love and lust, power and politics, beginnings and endings --- is
what happens in between. I sometimes wonder what my life would be like
if I had found the book earlier, but this is only a fantasy; reading it
sooner would not have made me less stupid in marriage or less sad in
divorce. It would have been nice, of course, to have more time with the
permission Rose gave me to love gossip --- more seasons to reread it and
remember how the same stories change each time they're told. But at
least I have the rest of my life to tell people how much I hate Charles
Dickens.

Advertisement

\protect\hyperlink{after-bottom}{Continue reading the main story}

\hypertarget{site-index}{%
\subsection{Site Index}\label{site-index}}

\hypertarget{site-information-navigation}{%
\subsection{Site Information
Navigation}\label{site-information-navigation}}

\begin{itemize}
\tightlist
\item
  \href{https://help.nytimes3xbfgragh.onion/hc/en-us/articles/115014792127-Copyright-notice}{©~2020~The
  New York Times Company}
\end{itemize}

\begin{itemize}
\tightlist
\item
  \href{https://www.nytco.com/}{NYTCo}
\item
  \href{https://help.nytimes3xbfgragh.onion/hc/en-us/articles/115015385887-Contact-Us}{Contact
  Us}
\item
  \href{https://www.nytco.com/careers/}{Work with us}
\item
  \href{https://nytmediakit.com/}{Advertise}
\item
  \href{http://www.tbrandstudio.com/}{T Brand Studio}
\item
  \href{https://www.nytimes3xbfgragh.onion/privacy/cookie-policy\#how-do-i-manage-trackers}{Your
  Ad Choices}
\item
  \href{https://www.nytimes3xbfgragh.onion/privacy}{Privacy}
\item
  \href{https://help.nytimes3xbfgragh.onion/hc/en-us/articles/115014893428-Terms-of-service}{Terms
  of Service}
\item
  \href{https://help.nytimes3xbfgragh.onion/hc/en-us/articles/115014893968-Terms-of-sale}{Terms
  of Sale}
\item
  \href{https://spiderbites.nytimes3xbfgragh.onion}{Site Map}
\item
  \href{https://help.nytimes3xbfgragh.onion/hc/en-us}{Help}
\item
  \href{https://www.nytimes3xbfgragh.onion/subscription?campaignId=37WXW}{Subscriptions}
\end{itemize}
