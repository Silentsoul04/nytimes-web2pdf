Sections

SEARCH

\protect\hyperlink{site-content}{Skip to
content}\protect\hyperlink{site-index}{Skip to site index}

\href{https://myaccount.nytimes3xbfgragh.onion/auth/login?response_type=cookie\&client_id=vi}{}

\href{https://www.nytimes3xbfgragh.onion/section/todayspaper}{Today's
Paper}

The Man Who Taught a Generation of Black Artists Gets His Own
Retrospective

\url{https://nyti.ms/2NLn07y}

\begin{itemize}
\item
\item
\item
\item
\item
\end{itemize}

Advertisement

\protect\hyperlink{after-top}{Continue reading the main story}

Supported by

\protect\hyperlink{after-sponsor}{Continue reading the main story}

\hypertarget{the-man-who-taught-a-generation-of-black-artists-gets-his-own-retrospective}{%
\section{The Man Who Taught a Generation of Black Artists Gets His Own
Retrospective}\label{the-man-who-taught-a-generation-of-black-artists-gets-his-own-retrospective}}

When he died in 1979, Charles White had been influential, both in and
outside of the art world. Now, a coming show at MoMA resurrects the
American master.

\includegraphics{https://static01.graylady3jvrrxbe.onion/images/2018/09/28/t-magazine/art/charles-white-slide-6DDK/charles-white-slide-6DDK-articleLarge.jpg?quality=75\&auto=webp\&disable=upscale}

By \href{https://www.nytimes3xbfgragh.onion/by/m-h-miller}{M.H. Miller}

\begin{itemize}
\item
  Sept. 28, 2018
\item
  \begin{itemize}
  \item
  \item
  \item
  \item
  \item
  \end{itemize}
\end{itemize}

AT THE TIME of his death in 1979, Charles White was the most famous
black artist in the country. As the painter Benny Andrews said
\href{https://www.nytimes3xbfgragh.onion/1979/10/06/archives/charles-w-white-is-dead-at-61-artist-with-work-in-49-museums.html}{in
Whiteʼs obituary} in The New York Times, ``People who didnʼt know his
name knew and recognized his work.'' He was a public figure who ranked
in the imagination of black Americans alongside such figures as Harry
Belafonte --- a friend, collector and portrait subject --- and Sidney
Poitier, who eulogized the artist at his memorial service.

To begin a discussion about White like this, with the ending so to
speak, is strange but appropriate. He was not a morbid or melancholy
artist --- quite the opposite, in fact: His images are passionately
alive. But there is a sense, in his work, that time itself is not linear
and history is not inevitable. His drawings and paintings include
figures ranging from Harriet Tubman and Nat Turner to Langston Hughes
and Sammy Davis Jr. (in character from the 1958 film ``Anna Lucasta'')
to anonymous street figures all the more captivating for their stoic
mystery. In his 1943 mural, ``The Contribution of the Negro to Democracy
in America,'' which remains to this day at the Clarke Hall auditorium at
Hampton University in Virginia, White depicts a transhistorical scene
that spans centuries, showing black Union soldiers marching alongside
the folk singer Leadbelly, captured in the midst of performance, while
George Washington Carver works away in his lab. ``Black Pope'' (1973),
perhaps his most famous painting, casts a bearded black man with
sunglasses, wearing a sandwich board and flashing a peace sign, as an
ecclesiastical figure. His divinity is neither forced nor satirical, it
just is, and though the painting, with its tremendous grandeur and
respect for its subject, isn't a self-portrait, it's tempting to see it
as one. But Whiteʼs project, in general, was bigger than himself,
nothing less than the presentation of a history too long ignored:
``Because the white man does not know the history of the Negro, he
misunderstands him,'' he said in 1940.

\includegraphics{https://static01.graylady3jvrrxbe.onion/images/2018/09/28/t-magazine/art/charles-white-slide-KYXJ/charles-white-slide-KYXJ-articleLarge.jpg?quality=75\&auto=webp\&disable=upscale}

Image

``The Contribution of the Negro to Democracy in America,''
1943.Credit...Collection of the Hampton University Museum, Hampton, Va.

Following his death, White's fame faded somewhat. Part of this had to do
with the fact that throughout the '80s and '90s, artists of color were
rare in an industry that endorses the work of white men at the exclusion
of everyone else, an imbalance that was certainly the case during
Whiteʼs lifetime and remains to this day, despite more recent efforts to
correct it. It also didn't help that, during his lifetime and in the
decades following, the figurative art that White championed was
overshadowed by a more abstract or conceptual style, or as Belafonte put
it in the foreword to a 1967 monograph on White, ``Many artists have
deserted reality for the various schools of nonobjectivity.'' Following
a 1982 retrospective show at the Studio Museum in Harlem, Whiteʼs work,
aside from his murals, most notably ``The Contribution of the Negro to
Democracy in America,'' could be difficult to see firsthand.

Next month, the Museum of Modern Art is staging Whiteʼs first, and long
overdue, \href{https://www.moma.org/calendar/exhibitions/3930}{major
museum retrospective} of the century, an exhibition that has traveled
from the Art Institute of Chicago, in White's hometown, where the artist
used to sketch with a drawing pad as a child. In the time between his
show in Harlem and this one, White has taken on the status of a folk
hero not unlike some of the subjects of his paintings: an American
master, who made mysterious, almost metaphysical images of
African-American dignity and, as a teacher at the Otis Art Institute in
Los Angeles from 1965 until his death, became a role model to an entire
generation of younger black disciples. For many of these artists ---
Kerry James Marshall, David Hammons, Alonzo Davis among them --- the
rarefied space they carved out for themselves in the art world was
previously unimaginable outside of Whiteʼs own paintings.

Image

``Bessie Smith,'' 1950.Credit...Private collection. © The Charles White
Archives. Photo: © Museum Associates/LACMA

WHITE'S UPBRINGING WAS, in many respects, typical of a black
working-class childhood in the years between the wars. He was defined
more by what he was denied, and his early life demonstrates just how
limited a path was available for any African-American with artistic
ambitions. Born in Chicago in 1918, White was raised by his mother, who
had migrated north some years earlier. She never married his absent
father and, with no day care options, would often drop White at the
Chicago Public Library while she worked. As a child, White became
interested in Harlem Renaissance artists and writers like W. E. B. Du
Bois, and when he reached high school, he was awarded a scholarship to
take Saturday classes at the Art Institute of Chicago. Already a
remarkable talent, he would soon receive scholarships to the Chicago
Academy of Fine Arts and the Frederic Mizen Academy of Art; both schools
would rescind these scholarships upon learning of his race. (In 1937, he
earned a scholarship to support one year of study at the School of the
Art Institute of Chicago, where he took courses in art history,
composition and figure drawing.)

Chicago was where he came of age, but his frequent trips to the South to
visit his mother's family greatly affected him. He became active in
politics early on. While still a teenager, White was the staff artist of
the Chicago-based National Negro Congress, which fought for black
liberation. His early social realist style reflected this activism,
which he refined as an artist for the Works Progress Administration and
while living in New York in the '40s and '50s, inspired in part by the
frescoes of Diego Rivera. In later years, he was an important figure
within the civil rights movement. Along with Belafonte --- with whom
White was a member of the Struggle for Freedom in the South, a group
founded in support of Martin Luther King Jr. --- he would also count
among his admirers Emory Douglas, the Black Panthers' Minister of
Culture, who liked White's work, even though it didn't inspire viewers
to ``go out and kill pigs,'' according to the historian Ilene Susan Fort
in the retrospective's catalog.

Image

A life drawing class taught by White at the South Side Community Center
in Chicago, circa 1940.Credit...Holger Cahill papers, 1910-1993, bulk
1910-1960. Archives of American Art, Smithsonian Institution.

Long plagued by health problems --- he lost one lung to tuberculosis and
the other became infected in 1956 --- White moved to Southern California
that year at the urging of his doctor. He lived near Poitier, a friend
he first met in New York, and the two spent a great deal of time
together, including on the set of Poitier's 1958 film ``The Defiant
Ones.'' White remained an active public figure --- he introduced James
Baldwin at an event for the Pasadena chapter of the Congress of Racial
Equality --- and began showing at the Heritage Gallery in the Pacific
Palisades, which opened in 1961 to show artists of color and gave White
multiple solo shows throughout the rest of his life. His work appeared
on album and book covers --- Belafonte even used White's drawings as
backdrops for his television show ``Tonight With Belafonte'' in 1959.
But White did not obtain steady teaching work in California until 1965,
when he was offered a job at the Otis Art Institute.

IN SOUTHERN CALIFORNIA, in the ʼ60s and ʼ70s, there were two main places
a person could receive a classical training in the fine arts: One was
the Walt Disney-owned Chouinard Art Institute, which began as a kind of
feeder school for Disney's production company and in 1969 would be
renamed CalArts; the other was Otis. According to the MoMA show's
co-curator Esther Adler, at Otis, which was known in particular for its
life drawing classes, White gave assignments like, ``From a
representational study of the figure, create an abstract study.'' White
also had the opportunity to teach students about his own work, a great
validation for an artist who constantly fought for institutional
recognition. He developed a more cosmic later style at the school, still
rooted in realism, but mining an almost unnamable vastness that
underscored the scope of White's career: His figures, carrying the same
grace, were now pictured amid a background of swirling abstraction, or a
floating seashell, or a bloody hand print.

Image

``Sojourner Truth and Booker T. Washington (Study for Contribution of
the Negro to Democracy in America),'' 1943.Credit...Collection of the
Newark Museum, purchase 1944 Sophronia Anderson Bequest Fund. © The
Charles White Archives. Photo courtesy of Michael Rosenfeld Gallery LLC,
New York, N.Y.

White became a sought-after figure at the school, a generous and kind
mentor to artists of color, like David Hammons, who took night and
weekend classes with him. In 1971, Hammons had his first major
exhibition at the Los Angeles County Museum of Art alongside his former
instructor, as well as the artist Timothy Washington, another student of
Whiteʼs, in a show called ``Three Graphic Artists.'' This was an
extraordinary moment in many ways, a conversation between ``the doyen of
American black artists,'' as the museum described White, and his younger
pupils, who were at the time largely unknown but already making vital
work. ``Three Graphic Artists'' included Hammonsʼs now-canonical body
prints, his works most clearly indebted to White, in which the artist
used his own body to apply paint to the canvas, leaving shadowlike
impressions. These were shown alongside a number of White's greatest
works, including the drawing ``Seed of Love,'' a stately portrait of a
pregnant black woman, the curve of her belly covered in a long frock
rendered in staggering detail, and which the museum would later acquire
for its permanent collection.

Most importantly, though, the show was a serious investigation of
contemporary black artists at a time when most institutions ignored
their existence. The exhibition catalog reinforces just how remarkable
the appearance of these three artists in a museum was at the time,
positioning White as an important but nonetheless exceptional figure in
American culture. In an interview, Hammons describes seeking out White
at Otis particularly because ``I never knew there were `black' painters,
or artists, or anything until I found out about him.''

Image

Left: ``Folksinger,'' 1957. Right: ``Mahalia,'' 1955.Credit...Collection
of Pamela and Harry Belafonte. © 1957 The Charles White Archives. Photo:
Christopher Burke Studios.

Image

``J'Accuse \#7,'' 1966.Credit...Private collection. © The Charles White
Archives. Photo courtesy of Michael Rosenfeld Gallery LLC, New York,
N.Y.

As the '70s continued, White's fondness for realism became more of an
outlier, as contemporary art began reassessing itself, throwing out
classical modes of thinking and delving deeper into a theoretical
framework of how and why artists create meaning. This affected
curriculum as well. In January 1977, the painter Kerry James Marshall
enrolled full-time at Otis in order to study with White. (Like so many
black artists of the '60s and '70s, White was Marshall's first exposure
to any artist of stature, let alone an artist of color.) By the time of
Marshall's arrival, the school had introduced an ``intermedia''
department, where students experimented with the still-nascent fields of
conceptual art, video and installation, dismantling the more traditional
art practices at the school, both figuratively and, in at least one
instance, literally. In an essay called
``\href{https://www.theparisreview.org/blog/2018/05/31/a-black-artist-named-white/}{A
Black Artist Named White},'' Marshall describes the chair of the
intermedia department at Otis personally tearing down a medieval bronze
statue on the campus quad that depicted ``the she-wolf sucking Romulus
and Remus.'' He tied a rope to it and attached it to the bumper of his
truck.

IN A 1971 INTERVIEW, Hammons described what he found so appealing about
White's work: ``There aren't too many people smiling, and I like that in
his things. There's always an agonized kind of look, I think, because
there aren't many pleasant things in his past. He's gone through a lot
of hell. I know he has.''

Going through hell, and surviving, was indeed a kind of through-line of
much of his work. In recent years, it may not have been as easy to see
White's own work as it has been to see that of his students, which is a
testament both to White's skills as a teacher and to cultural
vicissitudes. (Marshall, for instance, is the top-selling living black
artist in history.) But one can see his imprint in surprising places.
After Aretha Franklin died in August, The New Yorker
\href{https://www.newyorker.com/culture/cover-story/cover-story-2018-08-27}{featured
a cover} by the artist Kadir Nelson called ``The Queen of Soul (After
Charles White's `Folksinger'),'' a riff on White's extraordinary 1957
portrait of Belafonte, which Belafonte and his wife lent to the recent
retrospective. The ink drawing falls into the artist's more spiritual
later style: Belafonte, his back arched, his eyes shut, is not so much a
popular singer in this image as he is some time-displaced traveler, his
face seeming to communicate all the anguish of an American past, but
also shrouded in light, pointing upward, toward something better.

``Charles White: A Retrospective'' is on view from Oct. 7, 2018, to Jan.
13, 2019, at MoMA, 11 West 53rd St., New York,
\href{https://www.moma.org/calendar/exhibitions/3930}{moma.org}.

Advertisement

\protect\hyperlink{after-bottom}{Continue reading the main story}

\hypertarget{site-index}{%
\subsection{Site Index}\label{site-index}}

\hypertarget{site-information-navigation}{%
\subsection{Site Information
Navigation}\label{site-information-navigation}}

\begin{itemize}
\tightlist
\item
  \href{https://help.nytimes3xbfgragh.onion/hc/en-us/articles/115014792127-Copyright-notice}{©~2020~The
  New York Times Company}
\end{itemize}

\begin{itemize}
\tightlist
\item
  \href{https://www.nytco.com/}{NYTCo}
\item
  \href{https://help.nytimes3xbfgragh.onion/hc/en-us/articles/115015385887-Contact-Us}{Contact
  Us}
\item
  \href{https://www.nytco.com/careers/}{Work with us}
\item
  \href{https://nytmediakit.com/}{Advertise}
\item
  \href{http://www.tbrandstudio.com/}{T Brand Studio}
\item
  \href{https://www.nytimes3xbfgragh.onion/privacy/cookie-policy\#how-do-i-manage-trackers}{Your
  Ad Choices}
\item
  \href{https://www.nytimes3xbfgragh.onion/privacy}{Privacy}
\item
  \href{https://help.nytimes3xbfgragh.onion/hc/en-us/articles/115014893428-Terms-of-service}{Terms
  of Service}
\item
  \href{https://help.nytimes3xbfgragh.onion/hc/en-us/articles/115014893968-Terms-of-sale}{Terms
  of Sale}
\item
  \href{https://spiderbites.nytimes3xbfgragh.onion}{Site Map}
\item
  \href{https://help.nytimes3xbfgragh.onion/hc/en-us}{Help}
\item
  \href{https://www.nytimes3xbfgragh.onion/subscription?campaignId=37WXW}{Subscriptions}
\end{itemize}
