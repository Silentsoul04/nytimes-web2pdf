Sections

SEARCH

\protect\hyperlink{site-content}{Skip to
content}\protect\hyperlink{site-index}{Skip to site index}

\href{https://www.nytimes3xbfgragh.onion/section/movies}{Movies}

\href{https://myaccount.nytimes3xbfgragh.onion/auth/login?response_type=cookie\&client_id=vi}{}

\href{https://www.nytimes3xbfgragh.onion/section/todayspaper}{Today's
Paper}

\href{/section/movies}{Movies}\textbar{}Farewell, FilmStruck: A
Bittersweet Guide to the Movies to Catch Before It's Gone

\url{https://nyti.ms/2CNEtpg}

\begin{itemize}
\item
\item
\item
\item
\item
\item
\end{itemize}

Advertisement

\protect\hyperlink{after-top}{Continue reading the main story}

Supported by

\protect\hyperlink{after-sponsor}{Continue reading the main story}

\hypertarget{farewell-filmstruck-a-bittersweet-guide-to-the-movies-to-catch-before-its-gone}{%
\section{Farewell, FilmStruck: A Bittersweet Guide to the Movies to
Catch Before It's
Gone}\label{farewell-filmstruck-a-bittersweet-guide-to-the-movies-to-catch-before-its-gone}}

\includegraphics{https://static01.graylady3jvrrxbe.onion/images/2018/10/30/arts/30filmstruck-grid/30filmstruck-grid-articleLarge.jpg?quality=75\&auto=webp\&disable=upscale}

By The New York Times

\begin{itemize}
\item
  Oct. 29, 2018
\item
  \begin{itemize}
  \item
  \item
  \item
  \item
  \item
  \item
  \end{itemize}
\end{itemize}

On Friday, classic movie lovers and art house buffs received some
devastating news: \href{https://www.filmstruck.com/us/}{FilmStruck}, the
streaming service partnership between Turner Classic Movies and the
Criterion Collection library, is shutting down for good on Nov. 30.

Since its debut two years ago, FilmStruck has offered its subscribers a
wealth of cinephile delights: carefully curated retrospectives and
themed collections; bonus features, including movie introductions from
TCM hosts; and guest programming from the likes of Barry Jenkins and
Rebecca Miller. To say that it will be missed by its ``loyal'' but
``niche'' fan base --- as Turner and Warner Bros. Digital described
subscribers
\href{https://variety.com/2018/digital/news/filmstruck-shutdown-warnermedia-turner-1202998364/}{when
announcing their decision} --- is an understatement.

{[}\href{https://www.nytimes3xbfgragh.onion/watching}{\emph{Filmstruck
might be leaving us, but you can find recommendations for hundreds of
movies to stream, on Watching.}}{]}

If, like us, you are in mourning, the best way to cope is by taking full
advantage now of those films you've had sitting in your queue, waiting
to be discovered (or re-discovered). Want recommendations? We reached
out to a bunch of our reporters, editors and regular contributors to
find out which films they would miss most. Below, they bid farewell and
pick their favorites --- all of which are currently on FilmStruck but
not on other major services. Don't
\href{https://twitter.com/aliciamalone/status/1056955367878348800}{stress
yourself out too much} as you binge these next few weeks. (Where
relevant, dates reflect the year of a film's foreign release.) \emph{---
AISHA HARRIS}

\hypertarget{what-price-hollywood-1932}{%
\subsection{`What Price Hollywood?'
(1932)}\label{what-price-hollywood-1932}}

\includegraphics{https://static01.graylady3jvrrxbe.onion/images/2018/10/30/arts/30filmstruck-whatprice/merlin_146051625_d28d8ce0-571c-484f-ad76-e75def21b885-articleLarge.jpg?quality=75\&auto=webp\&disable=upscale}

FilmStruck has never been simply a repository of classics from the
Criterion Collection and the TCM archive, and it hasn't relied on a
janky algorithm to anticipate what viewers want. The service is
thoughtfully curated, making it possible to create and screen your own
personal film festivals at home and put the work of certain directors,
actors and movements in context.

If you saw the new Bradley Cooper-directed
``\href{https://www.nytimes3xbfgragh.onion/2018/10/03/movies/a-star-is-born-review-lady-gaga-bradley-cooper.html}{A
Star Is Born},'' for example, you could not only compare the three
previous versions on FilmStruck, you could also look at their precursor,
``What Price Hollywood?,'' which is more or less a dry run for the same
story. Directed by George Cukor, who went on to make the definitive ``A
Star is Born'' with Judy Garland 22 years later, the film is about a
Brown Derby waitress (Constance Bennett) with the moxie to persuade a
hard-drinking director (Lowell Sherman) to give her a shot as an
actress. She rockets to stardom as he sinks into addiction. Future
versions put more meat on the bone, but Bennett's zesty performance is
notable for being more confident and assertive than that of others in
the role. She needs the door opened only a crack. \emph{--- SCOTT
TOBIAS}

\hypertarget{zero-for-conduct-1933}{%
\subsection{`Zero for Conduct' (1933)}\label{zero-for-conduct-1933}}

Image

A scene from ``Zero for Conduct.''Credit...Comptoir Français de
Distribution de Films Franfilmdis

When I was a misfit movie-mad kid from Jersey, my wonderful aunt Peggy
would invite me to ``the city'' and take me to museums and cinemas. With
her, at the MoMA bookshop in 1972, when I was 12, I bought my first
hardcover movie book: a biography of the French renegade filmmaker Jean
Vigo. His pictures, only four in all, anarchic and tender, subversive
and romantic, had not yet been seen by me. They were rarely shown in my
neck of the woods. So I just read about them, obsessively.

Then a library a few towns away announced a screening of ``Zero For
Conduct,'' and I begged my dad to take me. We got there and learned the
print was too damaged to run; another film was substituted. I burst into
tears. Jeez, my poor dad.

Today, on FilmStruck's Criterion Channel, you can watch all of Vigo's
filmography, beautifully restored, at the touch of a button. How spoiled
my 12-year-old self would have felt! And he would've been even angrier
than I am now that some bean counter has deemed FilmStruck an expendable
corporate asset. \emph{--- GLENN KENNY}

\hypertarget{floating-clouds-1955}{%
\subsection{`Floating Clouds' (1955)}\label{floating-clouds-1955}}

Image

Masayuki Mori, left, and Hideko Takamine in ``Floating
Clouds.''Credit...Toho Company

Forget what's streaming elsewhere --- when FilmStruck goes, so will
movies that are tough to find on physical video, including Rainer Werner
Fassbinder's ``Querelle'' and Ernst Lubitsch's ``Cluny Brown.'' Credit
to the indispensable classic-movie maven Farran Smith Nehme
\href{https://twitter.com/selfstyledsiren/status/1055844790049427456}{for
noting} that the films of the great Japanese director Mikio Naruse will
be one of the biggest voids. Catch his work now, or you may wait years.

Naruse's ``Floating Clouds,'' never released on DVD in the United
States, is an ideal introduction to his unusually pure brand of despair.
It begins in 1946, when Yukiko (Hideko Takamine), repatriated to Japan,
returns to Tomioka (Masayuki Mori), with whom she had an affair in
Indochina during the war. She continues to love him, despite his refusal
to divorce his wife, his drunkenness, his negligible work prospects and
his willingness to cheat on and exploit her. And perhaps in their shared
hopelessness, they are well matched. Immaculately lit and performed, the
movie captures the anguish of the postwar milieu so piercingly that you
can almost feel it in your bones. \emph{--- BEN KENIGSBERG}

\hypertarget{mr-arkadin-1955}{%
\subsection{`Mr. Arkadin' (1955)}\label{mr-arkadin-1955}}

Image

Orson Welles in ``Mr. Arkadin.''Credit...Filmorsa

I've been aiming to be an Orson Welles completist. Wish me luck: So much
went unfinished in the director's career, and so many of his films were
cut and cut again in wildly different ways, that much of his art was
left in ``Is this what he wanted?'' fragments.

{[}\href{https://www.nytimes3xbfgragh.onion/2018/10/31/movies/orson-welles-netflix.html}{\emph{A
new Orson Welles movie is coming to Netflix. Read about his films
streaming right now.}}{]}

A blind spot for me until recently was his 1955 film ``Mr. Arkadin,''
which seemed the ultimate example of a feature that was in constant flux
and remix. It concerns a wealthy amnesiac (Welles, hidden behind makeup
and a fancy beard) who hires a smuggler (Robert Arden) to investigate
his past. As the FilmStruck countdown clock was ticking, I finally
caught up with it. It's a gorgeously shot and invigorating journey down
a Cold War rabbit hole. If you truly want to dive off the deep end,
FilmStruck is the only place offering Criterion's assemblage of three
cuts of the movie, including the ``Comprehensive Version'' I watched,
for which film scholars and archivists tried to assemble it to Welles's
wishes, based on later statements he made about it, among other factors.
\emph{--- MEKADO MURPHY}

\hypertarget{daisies-1966}{%
\subsection{`Daisies' (1966)}\label{daisies-1966}}

Image

Jitka Cerhova, left, and Ivana Karbanova in ``Daisies.''Credit...Filmové
studio Barrandov

In my teens and twenties, I was the kind of film buff who haunted
repertory theaters, worked at a video store and obsessively watched
Turner Classic Movies. But even I didn't hear about Vera Chytilova's
social satire ``Daisies'' until 2012, when the Criterion Collection put
it on DVD as part of its ``Pearls of the Czech New Wave'' box set.

A breezy 76 minutes of mostly plotless yet breathtakingly imaginative
avant-garde comedy, ``Daisies'' follows two women, both named Marie
(played by Jitka Cerhova and Ivana Karbanova) as they spend their days
giggling together, changing clothes, spreading mischief and baffling
men. Chytilova doesn't indulge in free-form quirkiness for its own sake.
The movie is a puckish poke at authoritarianism of all stripes, from the
patriarchy to the Iron Curtain bureaucracy. It is also a salute to
sisterhood, celebrating the bond between two best friends who can
withstand any oppression so long as they're side by side. It's the kind
of little-known world cinema classic FilmStruck was created to spotlight
--- not only making it available, but putting it into context. \emph{---
NOEL MURRAY}

\hypertarget{mouchette-1967}{%
\subsection{`Mouchette' (1967)}\label{mouchette-1967}}

Image

Nadine Nortier in ``Mouchette.''Credit...Argos Films

I came to Robert Bresson in my mid-30s, late (for now) in my cinematic
education. But maybe it was better that way. How wonderful to discover
as an adult, and on one's own, such a singular and mature vision! The
experience changed my understanding of cinema, which I hadn't thought
possible. And knowing Bresson's work felt like knowing a sacred name
that only the initiates of some secret order are permitted to pronounce:
To meet a fellow admirer was to meet someone who understood something
essential, and not only about film.

Jean-Luc Godard wrote that ``Robert Bresson is French cinema, as
Dostoyevsky is the Russian novel and Mozart is German music.'' He wasn't
wrong. But Bresson would no doubt reject Godard's comparisons --- and
definitely my own analogies --- because he was interested in presenting
things only as they are, stripped to their barest, most essential forms.
``Mouchette'' is Bresson at the height of his powers, released the year
after his ``Au Hasard Balthazar'' practically reinvented cinema. Like
``Balthazar,'' it is a cleareyed tragedy about the suffering of women
among cruel and small-minded provincials. And speaking of tragedy, it is
streaming only on FilmStruck. \emph{--- AUSTIN CONSIDINE}

\hypertarget{day-for-night-1973}{%
\subsection{`Day for Night' (1973)}\label{day-for-night-1973}}

Image

Jean-Pierre Léaud in ``Day for Night.''Credit...Les Films du Carrosse

I was 11 years old the first time I saw François Truffaut's ``Day For
Night.'' It was airing late one night on A\&E, which once stood for
``Arts \& Entertainment,'' and was programmed accordingly (not that
``Storage Wars'' isn't art). I recognized the title from Roger Ebert's
``Movie Home Companion''; he had given it four stars and called it ``not
only the best movie ever made about the movies'' but ``also a great
entertainment.''

That was about all I needed to hear, and although I had never watched a
foreign film before, I programmed the VCR (I'm aware of how much that
phrase dates me) and gave it a look after school the next day. There was
something so warm and welcoming about Truffaut's chronicle of the making
of a movie that the subtitles didn't even matter; it spoke a language I
understood, which was the love of cinema. And in that way, it's the
quintessential FilmStruck movie --- foreign but approachable, decades
old but fresh and alive, pulsing with affection for the art of movies
and all of their possibilities. \emph{--- JASON BAILEY}

\hypertarget{alambrista-1977}{%
\subsection{`Alambrista!' (1977)}\label{alambrista-1977}}

Image

Domingo Ambriz in ``Alambrista!''Credit...Filmhaus

I'm going to miss how easy it was to discover on FilmStruck the movies I
had previously only read about. Alongside the most recognizable names in
international and independent cinema were lesser-known gems from around
the world and from our own backyard, as is the case with Robert M.
Young's ``Alambrista!''

A low-budget film, ``Alambrista'' follows Roberto (Domingo Ambriz), a
young Mexican man who crosses the border to find work in the United
States to support his new family. He faces the Sisyphean task of
scraping together money and evading law enforcement officers looking to
deport undocumented laborers. Roberto's journey is often heartbreaking,
only occasionally rewarding. ``Alambrista!'' won the Caméra d'Or at the
1978 Cannes Film Festival for its hand-held camerawork, and it opened
the door for other Chicano movies like Gregory Nava's ``El Norte'' and
Luis Valdez's ``Zoot Suit.'' I'm sorry to lose a source for rarities
like ``Alambrista!'' \emph{--- MONICA CASTILLO}

\hypertarget{withnail-and-i-1987}{%
\subsection{`Withnail and I' (1987)}\label{withnail-and-i-1987}}

Image

Paul McGann, left and Richard E. Grant in ``Withnail and
I.''Credit...Cineplex-Odeon Films

Like many before me, I first discovered the 1987 dark comedy ``Withnail
and I'' back in the VHS days, after hearing countless people mimicking
Richard E. Grant's gloriously slurred belligerence. (``We want the
finest wines available to humanity! We want them here and we want them
now!'') The role of Withnail --- a flamboyant alcoholic and failing
actor --- launched Grant's career, and it's still a knockout.

This permanently snockered man-child is on the verge of 30, but he is
hopelessly incapable of meeting life's baseline requirements: paying his
rent, feeding himself, even changing his underwear. Drinks, however, are
always a possibility. (No one can match his unquenchable, suicidal
thirst, although many tried in college drinking games, subbing vinegar
for lighter fluid in one particular scene.) At the end of this hilarious
lost weekend in the country, we realize, along with his former
partner-in-crime Marwood (Paul McGann), that we must leave the Withnails
of the world behind if we ever hope to grow up. First, though, one more
rewatch. \emph{--- JENNIFER VINEYARD}

\hypertarget{naked-1993}{%
\subsection{`Naked' (1993)}\label{naked-1993}}

Image

David Thewlis and Lesley Sharp in ``Naked.''Credit...Thin Man Films

I don't think I've seen ``Naked'' more than three times. And yet,
``Naked'' is one of my favorite films. How can both statements be true?
Because like Johnny, the human vortex of misanthropy at the heart of
this scathing, haunting film from Mike Leigh, ``Naked'' arrives
unexpectedly and does enough psychic damage to mark you for life.

Played by David Thewlis in his breakout role, Johnny is a shuffling,
shaggy-haired native of Manchester, now down-and-out in London after
fleeing the consequences of the sexual assault that opens the film. (The
merciless tone is established from the start.) With his cruel
intelligence, dizzying monologues and trademark black trench coat, he
upends the lives of old friends, acquaintances and total strangers
alike.

The film's devastating final shot casts Johnny as a sad-sack Satan
wandering the world, unwilling to accept either punishment or
forgiveness for his sins. When FilmStruck vanishes from the internet, it
will take this unforgettable portrait of humanity as a failed state with
it for now --- but the film will remain lodged in my mind forever.
\emph{--- SEAN T. COLLINS}

\hypertarget{in-the-mood-for-love-2000}{%
\subsection{`In the Mood for Love'
(2000)}\label{in-the-mood-for-love-2000}}

Image

Maggie Cheung and Tony Leung Chiu-wai in ``In the Mood for
Love.''Credit...Block 2 Pictures

The first time I experienced ``In the Mood for Love,'' I was a graduate
student in cinema studies several years ago. I believe it was in my
class on film form and style, one of our required courses. At the time,
I considered it stunning, if a bit slow for my tastes.

It's one I always meant to revisit but just never did, though it's been
sitting in my FilmStruck queue for months. After learning that my
beloved streaming service was leaving, however, I took the chance this
past weekend to return to it. How glad I am that I did: Watching Wong
Kar-wai's gorgeous meditation on secrets, yearning and heartbreak
between two lonely souls (played by Tony Leung Chiu-wai and Maggie
Cheung) in 1960s Hong Kong is even better than I remembered, and hardly
a drag. Perhaps it's with more life experience that I can connect to it
on a deeper level than I did in my early 20s. Whatever the reason, I
fully understand why my professor considered it required viewing.
\emph{--- AISHA HARRIS}

\hypertarget{forza-bastia-2002}{%
\subsection{`Forza Bastia' (2002)}\label{forza-bastia-2002}}

Image

A scene from ``Forza Bastia.''Credit...Specta Films

The 26-minute documentary ``Forza Bastia 78'' may not immediately catch
your eye on FilmStruck, but it is a delightful U.F.O. in the career of
Jacques Tati, the French filmmaker famous for such droll comedies as
``Mon Oncle.''

In April 1978, Tati flew to Corsica to cover the first leg of the UEFA
Cup Final, which pitted the scrappy Bastia team against the much larger,
much more powerful Dutch club PSV Eindhoven. I was a kid living in
Corsica an hour from Bastia back then, and I distinctly remember our
soccer fever.

Tati's commentary-free doc shows maybe two minutes of the game itself.
In its place, we mostly see fans of all ages decorate shops and the
local church with the team colors, wave flags, honk car horns. We hear
chants and firecrackers, and the thunderstorm that wrecked the field.
Cue the surreal images of employees mopping up the puddles. For whatever
reason, Tati's Corsican footage was shelved until his daughter, Sophie
Tatischeff, unearthed it decades later and put it all together. I can't
watch it without tearing up. And I still have my team scarf. \emph{---
ELISABETH VINCENTELLI}

Advertisement

\protect\hyperlink{after-bottom}{Continue reading the main story}

\hypertarget{site-index}{%
\subsection{Site Index}\label{site-index}}

\hypertarget{site-information-navigation}{%
\subsection{Site Information
Navigation}\label{site-information-navigation}}

\begin{itemize}
\tightlist
\item
  \href{https://help.nytimes3xbfgragh.onion/hc/en-us/articles/115014792127-Copyright-notice}{©~2020~The
  New York Times Company}
\end{itemize}

\begin{itemize}
\tightlist
\item
  \href{https://www.nytco.com/}{NYTCo}
\item
  \href{https://help.nytimes3xbfgragh.onion/hc/en-us/articles/115015385887-Contact-Us}{Contact
  Us}
\item
  \href{https://www.nytco.com/careers/}{Work with us}
\item
  \href{https://nytmediakit.com/}{Advertise}
\item
  \href{http://www.tbrandstudio.com/}{T Brand Studio}
\item
  \href{https://www.nytimes3xbfgragh.onion/privacy/cookie-policy\#how-do-i-manage-trackers}{Your
  Ad Choices}
\item
  \href{https://www.nytimes3xbfgragh.onion/privacy}{Privacy}
\item
  \href{https://help.nytimes3xbfgragh.onion/hc/en-us/articles/115014893428-Terms-of-service}{Terms
  of Service}
\item
  \href{https://help.nytimes3xbfgragh.onion/hc/en-us/articles/115014893968-Terms-of-sale}{Terms
  of Sale}
\item
  \href{https://spiderbites.nytimes3xbfgragh.onion}{Site Map}
\item
  \href{https://help.nytimes3xbfgragh.onion/hc/en-us}{Help}
\item
  \href{https://www.nytimes3xbfgragh.onion/subscription?campaignId=37WXW}{Subscriptions}
\end{itemize}
