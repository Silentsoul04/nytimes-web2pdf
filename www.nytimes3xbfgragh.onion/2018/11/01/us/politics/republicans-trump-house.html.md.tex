Sections

SEARCH

\protect\hyperlink{site-content}{Skip to
content}\protect\hyperlink{site-index}{Skip to site index}

\href{https://www.nytimes3xbfgragh.onion/section/politics}{Politics}

\href{https://myaccount.nytimes3xbfgragh.onion/auth/login?response_type=cookie\&client_id=vi}{}

\href{https://www.nytimes3xbfgragh.onion/section/todayspaper}{Today's
Paper}

\href{/section/politics}{Politics}\textbar{}Trump's Nationalism Is
Breaking Point for Some Suburban Voters, Risking G.O.P. Coalition

\url{https://nyti.ms/2CSEmbY}

\begin{itemize}
\item
\item
\item
\item
\item
\item
\end{itemize}

Advertisement

\protect\hyperlink{after-top}{Continue reading the main story}

Supported by

\protect\hyperlink{after-sponsor}{Continue reading the main story}

\hypertarget{trumps-nationalism-is-breaking-point-for-some-suburban-voters-risking-gop-coalition}{%
\section{Trump's Nationalism Is Breaking Point for Some Suburban Voters,
Risking G.O.P.
Coalition}\label{trumps-nationalism-is-breaking-point-for-some-suburban-voters-risking-gop-coalition}}

\includegraphics{https://static01.graylady3jvrrxbe.onion/images/2018/11/02/us/politics/02housetrump1/merlin_144816846_599bcd3f-5b9f-4131-bddc-eaa5e42b13bd-articleLarge.jpg?quality=75\&auto=webp\&disable=upscale}

By \href{https://www.nytimes3xbfgragh.onion/by/jonathan-martin}{Jonathan
Martin} and
\href{https://www.nytimes3xbfgragh.onion/by/alexander-burns}{Alexander
Burns}

\begin{itemize}
\item
  Nov. 1, 2018
\item
  \begin{itemize}
  \item
  \item
  \item
  \item
  \item
  \item
  \end{itemize}
\end{itemize}

HOUSTON --- Two years ago, the presidential election hinged in large
part on a rightward shift among working-class whites who deserted
Democrats.

Tuesday's House election may turn on an equally significant and opposite
force: a generational break with the Republican Party among educated,
wealthier whites --- especially women --- who like the party's
pro-business policies but recoil from President Trump's divisive
language on race and gender.

Rather than seeking to coax voters like these back into the Republican
coalition, Mr. Trump appears to have all but written them off, spending
the final days of the campaign delivering a scorching message about
preoccupations like birthright citizenship and a migrant ``invasion''
from Mexico that these voters see through as alarmist.

In Republican-leaning districts that include diverse populations or abut
cities that do --- from bulwarks of Sunbelt conservatism like Houston
and Orange County, Calif., to the well-manicured bedroom communities
outside Philadelphia and Minneapolis --- the party is in danger of
losing its House majority next week because Mr. Trump's racially-tinged
nationalism has alienated these voters who once made up a dependable
constituency.

One of those disenchanted voters is J. Mark Metts, a 60-year-old partner
at one of this city's prestigious law firms. Mr. Metts had never voted
for a Democratic presidential candidate until 2016. Now he and some of
his neighbors in the moneyed River Oaks enclave of Houston are about to
oppose a Republican once again, to register their disapproval of
President Trump.

``With Congress not really standing up to Trump, this election is
becoming a referendum,'' Mr. Metts said, explaining why he would no
longer support the re-election of Representative John Culberson, an
eight-term Republican.

Mr. Culberson is now running roughly even with the Democratic candidate,
Lizzie Pannill Fletcher, according to
\href{https://www.nytimes3xbfgragh.onion/interactive/2018/upshot/elections-poll-tx07-3.html}{a
New York Times/Siena College poll last week} --- an extraordinary
development in a district that has not elected a Democrat since before
an oilman named George H.W. Bush won here in 1966, and one that
illustrates how difficult Mr. Trump has made it for his party to retain
control of the House.

The president amplified his fear-peddling Wednesday night with
\href{https://twitter.com/realDonaldTrump/status/1057728445386539008}{an
online video} that is being widely condemned as racist, showing a
Mexican man convicted of killing two California deputies with a
voice-over saying ``Democrats let him into the country.''

Traditional Republicans warn that Mr. Trump's conduct is further
narrowing his party's appeal on the eve of the election, catering to a
rural base in conservative states like Missouri, North Dakota and
Montana that will decide control of the Senate at the possible expense
of the Republicans' House majority and crucial governorships.

\includegraphics{https://static01.graylady3jvrrxbe.onion/images/2018/11/02/us/politics/02housetrump-jp-2-print/merlin_146175999_602569b2-4e1b-4f95-9d21-0d882c155c48-articleLarge.jpg?quality=75\&auto=webp\&disable=upscale}

**``**The divisiveness may play well in some parts of the country but it
doesn't play everywhere,'' said the speaker of the Texas House, Joe
Straus, who has sought to keep his party from drifting too far right.
``It's hard to grow a party when your whole approach is to incite the
base.''

To see incumbent Republicans like Mr. Culberson or Representative Pete
Sessions, whose district is in an affluent part of the Dallas area,
locked in difficult re-elections ``would have been unthinkable just a
few years ago,'' said Mr. Straus.

More ominous for the G.O.P. is that the desertion of educated whites
following Mr. Trump's 2016 win could establish a new Democratic
coalition in future elections, one that would certainly return to the
polls in 2020. That would represent the mirror opposite of 1964, when
Barry Goldwater lost the presidential race but made inroads into
traditionally Democratic precincts among culturally conservative and
economically prosperous voters --- presaging Republican success further
down the ballot in the years to come.

Just as Goldwater began unmooring conservative whites away from their
Democratic roots, it is easy to see which demographic could shift most
fundamentally on Election Day: college-educated white women, who were
once fairly reliable Republican supporters. The impact of Mr. Trump with
these voters is unmistakable: they supported Mitt Romney over President
Barack Obama by six percentage points in 2012, before backing Hillary
Clinton by seven points four years later.

College-educated white women now say they prefer Democrats to control
Congress by 18 points, according to a survey by Marist College and NPR.

In moderate areas, the Republican coalition has long depended on upscale
whites casting aside their more liberal views on issues like gun control
and abortion to support G.O.P. economic policies. Mr. Trump's national
message does virtually nothing to accommodate those voters.

``I'm not hearing anything helpful at all,'' said Gene DiGirolamo, a
moderate Republican state legislator from Bucks County, outside
Philadelphia, where Republicans are struggling to hold on to a House
seat and hold back Democratic gains in state races.

In his area, Mr. DiGirolamo said, Mr. Trump's support ``among
independents has slipped dramatically from when he was first elected.''

Perhaps nowhere has Mr. Trump's persistent use of inflammatory language
become as much of an issue as in Pennsylvania, where Republicans were
already bracing to suffer losses in some newly drawn House districts
before a gunman fixated on immigration massacred 11 worshipers at a
Pittsburgh synagogue Saturday.

\href{https://www.nytimes3xbfgragh.onion/interactive/2018/10/24/us/elections/2018-battle-for-congress.html}{}

\includegraphics{https://static01.graylady3jvrrxbe.onion/images/2018/10/24/us/2018-battle-for-congress-promo-1540395475116/2018-battle-for-congress-promo-1540395475116-articleLarge.png}

\hypertarget{the-battle-for-congress-is-close-heres-the-state-of-the-race}{%
\subsection{The Battle for Congress Is Close. Here's the State of the
Race.}\label{the-battle-for-congress-is-close-heres-the-state-of-the-race}}

The math currently favors the Democrats in the House and the Republicans
in the Senate.

At a gathering in a tavern outside Philadelphia on Monday evening,
supporters of Scott Wallace, a Democrat running in the state's most
hotly contested House race, denounced Mr. Trump for his ``cruelty'' and
alluded repeatedly to the president's rhetoric on race and national
identity. Addressing a tightly packed crowd, former Representative
Patrick Murphy, a Democrat who used to represent the area, warned that
``people who hate feel so emboldened to act on it.''

The suburbs around Philadelphia used to be a reliable Republican
bastion. But Shelley Howland, a Republican who attended the pro-Wallace
event, said Mr. Trump represented a breaking point.

A supporter of abortion rights and gun control, Ms. Howland voted two
years ago for Hillary Clinton over Mr. Trump, but stayed loyally
Republican in the congressional election, supporting Mr. Wallace's
opponent, Representative Brian Fitzpatrick, who is now seeking his
second term. She said she would not support Mr. Fitzpatrick again.

``This year, it's going to be a straight Democratic ticket,'' said Ms.
Howland, 65, lamenting ``this whole movement to the alt-right, Steve
Bannon in the White House, Trump in the White House.''

Mr. Wallace, an investor whose grandfather served as vice president,
cast his campaign as an opportunity for Bucks County to repudiate a
president who has unleashed a ``Pandora's box'' of dangerous social
turmoil.

``The tone that the president has set is absolutely toxic to relations
between people of different faiths and different races and different
sexual orientations,'' Mr. Wallace said.

Mr. Fitzpatrick, whose campaign did not respond to emails and phone
calls seeking comment, has sought distance from Mr. Trump and brands
himself as an ``independent'' lawmaker in campaign materials. Like other
Bucks County Republicans, he has collected support from labor unions and
endorsed policies like gun control that break with the national
Republican agenda.

But former Representative Phil English, a Pennsylvania Republican, said
his party was now grappling with what amounts to a mortal political
threat. Alluding to ``challenges right now with the brand,'' Mr. English
said his party would likely face painful setbacks in precincts that
evoke silk stocking Republicanism, such as Philadelphia's Main Line.

``Southeastern Pennsylvania has clearly made the transition from being
one of the mainstays of the Pennsylvania Republican statewide base, and
a significant part of the national Republican source of support, to
being an enormous challenge,'' Mr. English said, noting that national
cultural divisions had driven away swing voters.

Image

Scott Wallace, a Democrat running in Pennsylvania's First District, cast
his campaign as an opportunity for voters in Bucks County to repudiate
the president.Credit...Bryan Anselm for The New York Times

In Mr. Culberson's well-heeled district, where even the restaurants with
ample parking offer valet services, Mr. Trump is as polarizing as he was
when he narrowly lost the seat in 2016.

``I'm staying focused on John Culberson and who I am,'' Mr. Culberson
said in an interview when asked whether Mr. Trump was an asset or
liability here, repeating a variation of the same phrase multiple times.

But the lawmaker acknowledged that this year is ``unusual'' because of
what he termed the ``infinite'' amount of money flowing in for Ms.
Fletcher, who has raised nearly twice what he has.

Ms. Fletcher, a corporate attorney who grew up attending the same
Episcopal Church as George H.W. Bush, said voters here were attracted to
the G.O.P. that the former president exemplifies.

``They were people who saw Republicans as the party of good government
and moderation and I think they're not seeing that now,'' she said,
scorning Mr. Culberson for not standing up to Mr. Trump.

Like every race in Texas, this House contest has been overshadowed by
Representative Beto O'Rourke's challenge of Senator Ted Cruz. And Mr.
Cruz's allies quietly concede that having Mr. Trump come to Houston last
month hurt them, and by extension Mr. Culberson, with moderate voters in
Harris County, where the president was trounced. (The congressman was
notably absent from the Trump rally.)

But the deeper structural problem for Texas Republicans, one that may
outlast this year's Senate race, is that their long-running fears about
Hispanics consolidating behind Democrats may prove to be less worrisome
than the prospect of an even more reliable bloc of voters finding a new
political home.

``The more explosive element is college educated white women,'' said
Democratic strategist Paul Begala. ``They are not itinerant voters like
a lot of the Democratic base --- they're rooted and they always vote.''

\href{https://www.nytimes3xbfgragh.onion/interactive/2018/09/28/us/politics/the-campaign-reporter-ul.html?src=hpPromoHeadline}{}

\hypertarget{sign-up-for-the-campaign-reporter}{%
\subsection{Sign up for The Campaign
Reporter}\label{sign-up-for-the-campaign-reporter}}

\includegraphics{https://int.graylady3jvrrxbe.onion/newsgraphics/push-interactive/projects/campaign-reporter/avatars/alex_burns.png}

Hey, I'm Alex Burns, a politics correspondent for The Times. Send me
your questions using the NYT app. I'll give you the latest intel from
the campaign trail.

Sign up via push alert

It is almost eerily symmetrical, the possibility that Texas Republicans
could see their iron grip on the state loosened because of a political
realignment in the state's population centers. That was where Texas
Democrats first saw their own supremacy challenged.

``The areas that were the first to break away from the Democrats decades
ago are now showing signs they could break away from Republicans now,''
said Mr. Straus, whose family helped build the modern Republican Party
in San Antonio. And he did not hesitate to identify the proximate cause
of the shift.

Mr. Trump, he said, has ``changed the Republican Party in ways that are
just less appealing to the traditional Republicans and independents
we've always relied on.''

Advertisement

\protect\hyperlink{after-bottom}{Continue reading the main story}

\hypertarget{site-index}{%
\subsection{Site Index}\label{site-index}}

\hypertarget{site-information-navigation}{%
\subsection{Site Information
Navigation}\label{site-information-navigation}}

\begin{itemize}
\tightlist
\item
  \href{https://help.nytimes3xbfgragh.onion/hc/en-us/articles/115014792127-Copyright-notice}{©~2020~The
  New York Times Company}
\end{itemize}

\begin{itemize}
\tightlist
\item
  \href{https://www.nytco.com/}{NYTCo}
\item
  \href{https://help.nytimes3xbfgragh.onion/hc/en-us/articles/115015385887-Contact-Us}{Contact
  Us}
\item
  \href{https://www.nytco.com/careers/}{Work with us}
\item
  \href{https://nytmediakit.com/}{Advertise}
\item
  \href{http://www.tbrandstudio.com/}{T Brand Studio}
\item
  \href{https://www.nytimes3xbfgragh.onion/privacy/cookie-policy\#how-do-i-manage-trackers}{Your
  Ad Choices}
\item
  \href{https://www.nytimes3xbfgragh.onion/privacy}{Privacy}
\item
  \href{https://help.nytimes3xbfgragh.onion/hc/en-us/articles/115014893428-Terms-of-service}{Terms
  of Service}
\item
  \href{https://help.nytimes3xbfgragh.onion/hc/en-us/articles/115014893968-Terms-of-sale}{Terms
  of Sale}
\item
  \href{https://spiderbites.nytimes3xbfgragh.onion}{Site Map}
\item
  \href{https://help.nytimes3xbfgragh.onion/hc/en-us}{Help}
\item
  \href{https://www.nytimes3xbfgragh.onion/subscription?campaignId=37WXW}{Subscriptions}
\end{itemize}
