Sections

SEARCH

\protect\hyperlink{site-content}{Skip to
content}\protect\hyperlink{site-index}{Skip to site index}

\href{https://www.nytimes3xbfgragh.onion/section/politics}{Politics}

\href{https://myaccount.nytimes3xbfgragh.onion/auth/login?response_type=cookie\&client_id=vi}{}

\href{https://www.nytimes3xbfgragh.onion/section/todayspaper}{Today's
Paper}

\href{/section/politics}{Politics}\textbar{}G.O.P. Sees Trump's Playbook
as Best Hope in Some Tight Races

\url{https://nyti.ms/2D104dR}

\begin{itemize}
\item
\item
\item
\item
\item
\end{itemize}

Advertisement

\protect\hyperlink{after-top}{Continue reading the main story}

Supported by

\protect\hyperlink{after-sponsor}{Continue reading the main story}

\hypertarget{gop-sees-trumps-playbook-as-best-hope-in-some-tight-races}{%
\section{G.O.P. Sees Trump's Playbook as Best Hope in Some Tight
Races}\label{gop-sees-trumps-playbook-as-best-hope-in-some-tight-races}}

\includegraphics{https://static01.graylady3jvrrxbe.onion/images/2018/11/04/us/politics/04gop1p/04gop1p-articleLarge-v2.jpg?quality=75\&auto=webp\&disable=upscale}

By \href{https://www.nytimes3xbfgragh.onion/by/jeremy-w-peters}{Jeremy
W. Peters}

\begin{itemize}
\item
  Nov. 3, 2018
\item
  \begin{itemize}
  \item
  \item
  \item
  \item
  \item
  \end{itemize}
\end{itemize}

ALBERT LEA, Minn. --- The Republican attack ads targeting George Soros
and \href{https://youtu.be/9-22S-kFong}{Colin Kaepernick} were the first
to arrive in southern Minnesota last month, so closely echoing President
Trump that he could have written them himself.

Then came the caravan.

This latest ad --- part of a multimillion-dollar blitz from Republican
groups in this battleground House district --- warns of ``a caravan full
of illegal immigrants marching on America,'' bringing with it ``gang
members and criminals.'' Grainy video shows Latin American men pumping
fists in the air.

(MN-01) CaravanCredit...CreditVideo by CLFSuperPAC

As Republicans scramble ahead of Tuesday's election to try to save their
majorities in the House and Senate, many party officials and candidates
like Jim Hagedorn, the nominee here in Minnesota's First District, have
concluded that their best shot at victory is embracing the Trump
political playbook of demonization.

The goal --- through overt, frontal attacks on prominent liberals,
minorities and immigrants --- is to stoke an us vs. them narrative about
the country's security, culture and heritage, in hopes of getting
conservatives to see the election as a battle to save the nation's
future.

Beyond rural Minnesota, in districts across Ohio and New York,
Republicans believe the strategy is resonating, and may help them win
some races Tuesday as their 23-seat majority comes under attack.

Mr. Hagedorn is making his fourth run for the seat, and was once widely
regarded as a weak candidate because of a long record of sexist and
insensitive remarks about a number of minority groups --- the
conservative Washington Examiner called him
``\href{https://www.washingtonexaminer.com/opinion/jim-hagedorn-the-worst-republican-candidate-in-america}{the
worst Republican candidate in America}.'' Yet polling indicates a tight
race, giving the G.O.P. one of its few decent chances of flipping a
Democratic-held seat this year.

Republicans are also defending dozens of other seats that the party now
holds, but they are finding that in many places the Trump playbook can
go only so far before it backfires.

In the affluent, educated suburbs where much of the political
battleground is, Republicans are framing immigration less as a cultural
issue with implications for the country's identity and more as a
question of security by drawing attention to sanctuary cities, which
limit cooperation between local law enforcement and federal immigration
officials, and MS-13, the violent transnational gang.

In the handful of more rural, conservative-leaning districts where the
party has a better shot of winning, the message is blunter.

In Minnesota, ads from groups like the Congressional Leadership Fund and
the National Republican Congressional Committee try to link Mr.
Hagedorn's opponent, Dan Feehan, to people and themes that conservatives
have portrayed as a threat, like Mr. Soros, the liberal philanthropist
who has been smeared with anti-Semitic attacks; Mr. Kaepernick, the
black football player famous for kneeling during the national anthem;
and now the migrant caravan.

In one, a doctored image appears to show Mr. Feehan, an Army veteran,
\href{https://www.youtube.com/watch?v=9-22S-kFong\&feature=youtu.be}{saluting}
Mr. Kaepernick.
\href{https://www.congressionalleadershipfund.org/new-ad-in-mn-01-dan-feehans-liberal-resistance/}{Another}
shows Mr. Kaepernick alongside two other favorite villains of
conservatives --- Nancy Pelosi, the House Democratic leader, and Keith
Ellison, the Minnesota congressman who is black and Muslim and
\href{https://www.nytimes3xbfgragh.onion/2018/09/26/us/politics/ellison-abuse-accusation.html}{is
now facing abuse allegations} as he runs for state attorney general. Mr.
Feehan's face appears on the screen, and the announcer intones, ``He's
one of them.''

Ads like these have aired repeatedly across the district's 12,000 square
miles of farmland and small towns, making it what Republicans say is the
most active political laboratory for the bald appeals to white racial
and cultural anxieties that President Trump and many Republicans are
using to drive conservatives to the polls.

The First District, which is 90 percent white, fits the demographic
profile of many of the places where Mr. Trump has been confined to
campaigning by Republicans who do not want him in more diverse
districts. Given Mr. Trump's self-described ``nationalist'' views, some
Republicans believe their path to staying in power in the Trump era is
increasingly narrow and heavily dependent on the parts of the country
that are the most white and rural.

Jason Hulburt, a production line supervisor from Albert Lea, said he
planned to vote for Mr. Hagedorn on Tuesday, a choice he made in part
because of how important he thinks it is for Mr. Trump to have
like-minded Republicans in Congress.

``Donald Trump knows what he wants, and the Democrats are afraid of the
issues Trump talks about,'' he said Friday night as he finished a brandy
and water at Eddie's, a local bar. ``They're afraid to offend certain
people. Trump is not afraid.''

Republican strategists argued that such pinpoint tactics worked for Mr.
Trump in 2016 and could work again.

``They are going to all the right places,'' said Scott Reed, the chief
political strategist for the United States Chamber of Commerce. ``That's
what has me optimistic. The president's energy level and where they land
the plane could be the difference on Election Day.''

Mr. Trump's playbook represents an extraordinary break with recent
Republican presidents like the Bushes and nominees like Mitt Romney and
John McCain. They sought to grow the party with appeals, however
limited, to Hispanic voters, women and culturally conservative
immigrants.

\includegraphics{https://static01.graylady3jvrrxbe.onion/images/2018/11/04/us/politics/04gop-2/04gop-2-articleLarge.jpg?quality=75\&auto=webp\&disable=upscale}

Mr. Trump, who promised after his victory in 2016 to be a president for
``all Americans,'' has been fixated this year on visiting states that
were critical to his Electoral College win and doubling down on
nurturing his homogeneous base in those places. Many Republicans
privately worry that in terms of the future health of their party, the
outreach and agenda they are pursuing feels a lot like the president's
travel footprint: provincial and small.

``No one has repealed the long-term demographic trends in the country,''
said Whit Ayres, a prominent Republican pollster. ``At some point,
Republicans are going to have to reach out beyond the base if they hope
to win a majority of the popular vote in the future.''

When Air Force One touched down in West Virginia on Friday, for
instance, it was Mr. Trump's eighth visit to the state that delivered
him his largest share of the vote anywhere --- 68 percent --- and in a
congressional district that is 93 percent white. The districts he plans
to visit through Monday night have very similar compositions: Indiana's
Third District around Fort Wayne is 83 percent white, while Tennessee's
Third in Chattanooga is 82 percent white.

Even on his trips over the last few days to Florida, a state that is 45
percent nonwhite, Mr. Trump held his rally in a district near Fort Myers
(70 percent white) and one encompassing Pensacola (73 percent).

The president visited Minnesota's First last month to campaign for Mr.
Hagedorn. Through a spokesman, Mr. Hagedorn declined to be interviewed.

At his rallies lately Mr. Trump has laid out a clear, if misleading,
picture of what Democratic control of Congress would look like. It
includes, in no particular order, ``caravan after caravan''; crime;
chaos; sanctuary cities; ``birth tourism,'' where immigrants can have
children who are automatically citizens; and an empowered and emboldened
Maxine Waters, the black Democratic congresswoman he has often insulted
as ``low I.Q.''

A vote for Republicans, by contrast, he said the other night at a rally
in central Missouri, represents greatness; ``standing up for our
national anthem''; more jobs; and the promise ``to win, win, win.''

\href{https://www.nytimes3xbfgragh.onion/interactive/2018/10/24/us/elections/2018-battle-for-congress.html}{}

\includegraphics{https://static01.graylady3jvrrxbe.onion/images/2018/10/24/us/2018-battle-for-congress-promo-1540395475116/2018-battle-for-congress-promo-1540395475116-articleLarge.png}

\hypertarget{the-battle-for-congress-is-close-heres-the-state-of-the-race}{%
\subsection{The Battle for Congress Is Close. Here's the State of the
Race.}\label{the-battle-for-congress-is-close-heres-the-state-of-the-race}}

The math currently favors the Democrats in the House and the Republicans
in the Senate.

In interviews, his supporters rejected any suggestion that Mr. Trump
traffics in racist or divisive views, saying that the issues the
president raised were merely matters of law and order and made good
fiscal sense.

``We are a nation of immigrants, but we are a nation of laws, too,''
said Stan Hale of Oakville, Mo., who drove two hours for the Trump rally
in Columbia on Thursday night. ``The other side is so fake when they
talk about the little guy, the working class. They don't care. They only
care about votes.''

His wife, Melita, insisted: ``Race has nothing to do with it. Race
doesn't even enter the mind.'' For her the issue with immigrants was
``vetting,'' she said. ``You don't know who they are. You just don't
know.''

Mr. Trump's narrower approach, compared with past Republican leaders, is
especially evident in his travel schedule. During the final month before
the midterm elections of 1990 and 2002, George H.W. Bush and George W.
Bush traveled to far more states than Mr. Trump has, according to data
compiled by Brendan Doherty, a political scientist at the United States
Naval Academy.

``The big difference with his travel,'' said Mr. Doherty, ``is President
Trump has been spending more time in states that he won quite
comfortably.''

The elder Mr. Bush visited 20 states between Oct. 1 and Election Day in
1990; the younger Mr. Bush went to 29 states over the same period in his
first term. Strategists familiar with Mr. Trump's plans said that as of
Monday, the president had visited 19 states since Labor Day, which was
two months ago.

Mr. Trump's appeal is already more limited than any other modern chief
executive's, a worrisome fact for Republicans who fear that he will
leave the party in a woefully uncompetitive position whenever he leaves
office. No president in the history of Gallup's polling has averaged as
low an approval rating as he has during the first two years in office.
And he is the only president whose approval in Gallup's surveys has
never exceeded 50 percent at this point into the first term, even for a
brief period of time.

Even one of the polling statistics Mr. Trump is proudest of --- his
approval rating among Republicans, which has been near 90 percent for
months --- is not quite what it seems.

Americans who identify as Republicans are the smallest part of the
electorate,
\href{https://news.gallup.com/poll/15370/party-affiliation.aspx}{currently
below 30 percent}, public opinion surveys from Gallup, the Pew Research
Center and others show.

``Having 90 percent of the smallest party ID in the country is not
necessarily something to brag about,'' said Elaine Kamarck, a senior
fellow at the liberal-leaning Brookings Institution. ``And any way you
look at it,'' she added, ``Trump is not in control of a big piece of the
electorate.''

\href{https://www.nytimes3xbfgragh.onion/interactive/2018/09/28/us/politics/the-campaign-reporter-ul.html?src=hpPromoHeadline}{}

\hypertarget{sign-up-for-the-campaign-reporter}{%
\subsection{Sign up for The Campaign
Reporter}\label{sign-up-for-the-campaign-reporter}}

\includegraphics{https://int.graylady3jvrrxbe.onion/newsgraphics/push-interactive/projects/campaign-reporter/avatars/alex_burns.png}

Hey, I'm Alex Burns, a politics correspondent for The Times. Send me
your questions using the NYT app. I'll give you the latest intel from
the campaign trail.

Sign up via push alert

One thing the midterm elections will be a test of is whether the most
vocal and active slice of Mr. Trump's voters --- a minority of a
minority --- can swing close elections after two years of a chaotic
presidency. Mr. Hagedorn and his campaign staff hope that is the case.

Gregg Peppin, a consultant working for Mr. Hagedorn, recalled an
exchange he witnessed Mr. Hagedorn have at a McDonald's on Friday
morning.

``Jim went up to a table of five guys and a woman,'' Mr. Peppin said.
``And the first words out of the guy's mouth was, `What are you going to
do about the caravan?'''

Advertisement

\protect\hyperlink{after-bottom}{Continue reading the main story}

\hypertarget{site-index}{%
\subsection{Site Index}\label{site-index}}

\hypertarget{site-information-navigation}{%
\subsection{Site Information
Navigation}\label{site-information-navigation}}

\begin{itemize}
\tightlist
\item
  \href{https://help.nytimes3xbfgragh.onion/hc/en-us/articles/115014792127-Copyright-notice}{©~2020~The
  New York Times Company}
\end{itemize}

\begin{itemize}
\tightlist
\item
  \href{https://www.nytco.com/}{NYTCo}
\item
  \href{https://help.nytimes3xbfgragh.onion/hc/en-us/articles/115015385887-Contact-Us}{Contact
  Us}
\item
  \href{https://www.nytco.com/careers/}{Work with us}
\item
  \href{https://nytmediakit.com/}{Advertise}
\item
  \href{http://www.tbrandstudio.com/}{T Brand Studio}
\item
  \href{https://www.nytimes3xbfgragh.onion/privacy/cookie-policy\#how-do-i-manage-trackers}{Your
  Ad Choices}
\item
  \href{https://www.nytimes3xbfgragh.onion/privacy}{Privacy}
\item
  \href{https://help.nytimes3xbfgragh.onion/hc/en-us/articles/115014893428-Terms-of-service}{Terms
  of Service}
\item
  \href{https://help.nytimes3xbfgragh.onion/hc/en-us/articles/115014893968-Terms-of-sale}{Terms
  of Sale}
\item
  \href{https://spiderbites.nytimes3xbfgragh.onion}{Site Map}
\item
  \href{https://help.nytimes3xbfgragh.onion/hc/en-us}{Help}
\item
  \href{https://www.nytimes3xbfgragh.onion/subscription?campaignId=37WXW}{Subscriptions}
\end{itemize}
