Sections

SEARCH

\protect\hyperlink{site-content}{Skip to
content}\protect\hyperlink{site-index}{Skip to site index}

\href{https://www.nytimes3xbfgragh.onion/section/fashion}{Fashion}

\href{https://myaccount.nytimes3xbfgragh.onion/auth/login?response_type=cookie\&client_id=vi}{}

\href{https://www.nytimes3xbfgragh.onion/section/todayspaper}{Today's
Paper}

\href{/section/fashion}{Fashion}\textbar{}A New Fashion Star is Crowned,
and He's Not Afraid of Controversy

\url{https://nyti.ms/2yTE7uK}

\begin{itemize}
\item
\item
\item
\item
\item
\end{itemize}

Advertisement

\protect\hyperlink{after-top}{Continue reading the main story}

Supported by

\protect\hyperlink{after-sponsor}{Continue reading the main story}

\hypertarget{a-new-fashion-star-is-crowned-and-hes-not-afraid-of-controversy}{%
\section{A New Fashion Star is Crowned, and He's Not Afraid of
Controversy}\label{a-new-fashion-star-is-crowned-and-hes-not-afraid-of-controversy}}

Kerby Jean-Raymond of Pyer Moss wins the CFDA/Vogue Fashion Fund prize.

\includegraphics{https://static01.graylady3jvrrxbe.onion/images/2018/11/08/fashion/06OTR1/merlin_143479212_73d92a28-4e5c-4fac-b93d-3878b52d6db1-articleLarge.jpg?quality=75\&auto=webp\&disable=upscale}

\href{https://www.nytimes3xbfgragh.onion/by/vanessa-friedman}{\includegraphics{https://static01.graylady3jvrrxbe.onion/images/2018/06/12/multimedia/vanessa-friedman/vanessa-friedman-thumbLarge.png}}

By \href{https://www.nytimes3xbfgragh.onion/by/vanessa-friedman}{Vanessa
Friedman}

\begin{itemize}
\item
  Nov. 6, 2018
\item
  \begin{itemize}
  \item
  \item
  \item
  \item
  \item
  \end{itemize}
\end{itemize}

The evening before Americans went to the polls to cast their votes in
the midterm elections, some of the most powerful gatekeepers in American
fashion cast a vote of their own. For change.

Monday night, the CFDA/Vogue Fashion Fund --- the most prestigious and
lucrative of the awards for emerging designers in the United States ---
was handed to
\href{https://www.nytimes3xbfgragh.onion/2018/10/11/magazine/kerby-jean-raymond-is-expanding-the-fashion-canon.html}{Kerby
Jean-Raymond of Pyer Moss}, a Haitian-American designer who has
effectively become the sartorial bard of the African-American cultural
experience. He beat nine other finalists.

``At a time when our world faces so many challenges, I'm impressed by
this year's winners,'' said Anna Wintour, editor in chief of Vogue and
artistic director of Condé Nast. ``Their work highlights a high degree
of creativity and a deep-rooted commitment to the notion of community,''
she added, calling it an expression of ``the optimism and inclusivity of
the very best American fashion.''

The two runners-up were Emily Adams Bode, whose men's wear brand Bode
focuses on sustainability and craft, and Jonathan Cohen, whose namesake
women's line has its roots in painterliness and print.

``This is unexpected,'' Mr. Jean-Raymond said gleefully from the
makeshift stage in a warehouse at the Brooklyn Navy Yards, where the
event was held. (Another sign of change: the willingness of fashion to
embrace a new borough). He was wearing black leather pants and a
matching cropped leather jacket from
\href{https://www.nytimes3xbfgragh.onion/slideshow/2018/02/10/fashion/runway-womens/pyer-moss-fall-2018.html}{his
fall collection}, which celebrated the history of the black cowboy.

The award may have been a surprise for the designer, but it wasn't for
most of the audience, which gave him a standing ovation.

Mr. Jean-Raymond founded his label in 2013 as a men's wear brand, and
first started using the runway as a means to explore activism in 2015
with a ``Black Lives Matter'' collection (never sold and currently in
the archive of the National Museum of African-American History and
Culture). That was followed by a show that juxtaposed the ideas of
Bernie Madoff and Bernie Sanders. But it wasn't until late last year,
when he bought his company back from his financers and became the sole
owner, that he truly found a unique and multilayered voice --- as well
as a deal with Reebok that provided a certain amount of security.

He has since shied away from calling his company a fashion brand, saying
he thinks of it more as an art project or a social experiment, though it
also involves very good clothes. He uses it almost as a freewheeling
exploration of various aspects of black history, from the Negro Motorist
Green Book (a travel guide used by African-Americans to navigate the Jim
Crow South) to once-defining brands like Cross Colours and FUBU.

\includegraphics{https://static01.graylady3jvrrxbe.onion/images/2018/11/06/world/06OTR2/merlin_143476797_a1cc4070-4158-4f97-99a1-d604b454011b-articleLarge.jpg?quality=75\&auto=webp\&disable=upscale}

His most recent show, held at the Weeksville Heritage Center in Brooklyn
on a rainy evening with very grumbling attendees, imagined what
``\href{https://www.nytimes3xbfgragh.onion/2018/09/09/fashion/pyer-moss-black-life-fashion-eckhaus-latta.html}{black
American leisure}'' might look like in a world where there was no threat
of the police being called on a black man creating a community garden,
and was so accomplished that all the damp, cranky guests went away
practically bouncing on their toes with glee.

Earlier Monday, as fashion folk and their celeb friends like Emily Blunt
(who presented the prize after everyone had finished eating --- or not
--- their chicken potpies) and La La Anthony (who was wearing Pyer Moss)
and Georgina Chapman of Marchesa (who continues her
\href{https://www.nytimes3xbfgragh.onion/2018/05/10/fashion/marchesa-harvey-weinstein.html}{tiptoe
back into the world}) mingled and sipped Champagne and speculated on who
might nab the prize, Mr. Jean-Raymond attributed his success to having a
team around him that came from the arts, but not, necessarily, fashion.

``I'm a designer --- I don't need another designer around me,'' he said.
``I need different perspectives. I need an architect! They all love
fashion, but none of them are jaded about fashion.''

It's tempting to be jaded about the fashion fund itself, which was
celebrating its 15th anniversary (fashion loves an anniversary).
Established to support the American fashion industry in the aftermath of
the Sept. 11, 2001, attacks, it has unquestionably nurtured some
designer names that have become full-fledged brands core to the New York
Fashion identity: Proenza Schouler, Thom Browne, Prabal Gurung, Joseph
Altuzarra. But as a booklet left on every plate listing all winners,
runners-up and finalists since 2004 perhaps inadvertently demonstrated,
several previously celebrated names have also dropped away: Behnaz
Sarafpour, Trovata, Doo.Ri, Rogan and Sophie Theallet, to name a few.

Image

La La Anthony wore Pyer Moss to the awards event.Credit...Evan
Agostini/Invision, via Associated Press

It's not an award's job to guarantee success, of course. The \$400,000
prize and a yearlong mentorship with an established fashion executive
does not a secure business make, though it helps. (Runners-up each get
\$150,000 and mentorship.) But it does reveal that talent and a moment
in the spotlight isn't enough. Most of all, a designer needs a unique
point of view and the patience to break through. Mr. Jean-Raymond has
that.

He also other qualities that help with success: a sense of humor, and
some perspective on it all. Accepting his award, he noted of the
famously intimidating Ms. Wintour, ``I'm going to say something that
Anna probably doesn't want me to say: She's a really nice person.''

It was a striking sentiment to hear from a podium in the current
political climate.

Advertisement

\protect\hyperlink{after-bottom}{Continue reading the main story}

\hypertarget{site-index}{%
\subsection{Site Index}\label{site-index}}

\hypertarget{site-information-navigation}{%
\subsection{Site Information
Navigation}\label{site-information-navigation}}

\begin{itemize}
\tightlist
\item
  \href{https://help.nytimes3xbfgragh.onion/hc/en-us/articles/115014792127-Copyright-notice}{©~2020~The
  New York Times Company}
\end{itemize}

\begin{itemize}
\tightlist
\item
  \href{https://www.nytco.com/}{NYTCo}
\item
  \href{https://help.nytimes3xbfgragh.onion/hc/en-us/articles/115015385887-Contact-Us}{Contact
  Us}
\item
  \href{https://www.nytco.com/careers/}{Work with us}
\item
  \href{https://nytmediakit.com/}{Advertise}
\item
  \href{http://www.tbrandstudio.com/}{T Brand Studio}
\item
  \href{https://www.nytimes3xbfgragh.onion/privacy/cookie-policy\#how-do-i-manage-trackers}{Your
  Ad Choices}
\item
  \href{https://www.nytimes3xbfgragh.onion/privacy}{Privacy}
\item
  \href{https://help.nytimes3xbfgragh.onion/hc/en-us/articles/115014893428-Terms-of-service}{Terms
  of Service}
\item
  \href{https://help.nytimes3xbfgragh.onion/hc/en-us/articles/115014893968-Terms-of-sale}{Terms
  of Sale}
\item
  \href{https://spiderbites.nytimes3xbfgragh.onion}{Site Map}
\item
  \href{https://help.nytimes3xbfgragh.onion/hc/en-us}{Help}
\item
  \href{https://www.nytimes3xbfgragh.onion/subscription?campaignId=37WXW}{Subscriptions}
\end{itemize}
