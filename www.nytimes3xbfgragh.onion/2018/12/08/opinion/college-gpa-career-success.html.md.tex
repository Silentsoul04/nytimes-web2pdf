Sections

SEARCH

\protect\hyperlink{site-content}{Skip to
content}\protect\hyperlink{site-index}{Skip to site index}

\href{https://www.nytimes3xbfgragh.onion/section/opinion/sunday}{Sunday
Review}

\href{https://myaccount.nytimes3xbfgragh.onion/auth/login?response_type=cookie\&client_id=vi}{}

\href{https://www.nytimes3xbfgragh.onion/section/todayspaper}{Today's
Paper}

\href{/section/opinion/sunday}{Sunday Review}\textbar{}What Straight-A
Students Get Wrong

\url{https://nyti.ms/2G7GakS}

\begin{itemize}
\item
\item
\item
\item
\item
\item
\end{itemize}

Advertisement

\protect\hyperlink{after-top}{Continue reading the main story}

\href{/section/opinion}{Opinion}

Supported by

\protect\hyperlink{after-sponsor}{Continue reading the main story}

\hypertarget{what-straight-a-students-get-wrong}{%
\section{What Straight-A Students Get
Wrong}\label{what-straight-a-students-get-wrong}}

If you always succeed in school, you're not setting yourself up for
success in life.

\href{https://www.nytimes3xbfgragh.onion/column/adam-grant}{\includegraphics{https://static01.graylady3jvrrxbe.onion/images/2015/03/16/opinion/Grant-Adam-circular/Grant-Adam-circular-thumbLarge-v4.jpg}}

By \href{https://www.nytimes3xbfgragh.onion/column/adam-grant}{Adam
Grant}

Dr. Grant is an organizational psychologist and a contributing opinion
writer.

\begin{itemize}
\item
  Dec. 8, 2018
\item
  \begin{itemize}
  \item
  \item
  \item
  \item
  \item
  \item
  \end{itemize}
\end{itemize}

\includegraphics{https://static01.graylady3jvrrxbe.onion/images/2018/12/09/opinion/sunday/09grant/09grant-articleLarge.jpg?quality=75\&auto=webp\&disable=upscale}

\href{https://www.nytimes3xbfgragh.onion/es/2018/12/11/obsesion-calificaciones/?}{Leer
en español}

A decade ago, at the end of my first semester teaching at Wharton, a
student stopped by for office hours. He sat down and burst into tears.
My mind started cycling through a list of events that could make a
college junior cry: His girlfriend had dumped him; he had been accused
of plagiarism. ``I just got my first A-minus,'' he said, his voice
shaking.

Year after year, I watch in dismay as students obsess over getting
straight A's. Some sacrifice their health; a few have even tried to sue
their school after falling short. All have joined the cult of
perfectionism out of a conviction that top marks are a ticket to elite
graduate schools and lucrative job offers.

I was one of them. I started college with the goal of graduating with a
4.0. It would be a reflection of my brainpower and willpower, revealing
that I had the right stuff to succeed. But I was wrong.

The evidence is clear: Academic excellence is
\href{https://link.springer.com/article/10.1007/BF00974070}{not a strong
predictor} of
\href{http://journals.sagepub.com/doi/abs/10.3102/00028312021002311}{career
excellence}. Across industries,
\href{http://psycnet.apa.org/record/1997-41278-008}{research shows} that
the correlation between grades and job performance is modest in the
first year after college and trivial within a handful of years. For
example, at Google, once employees are two or three years out of
college, their grades have
\href{https://www.amazon.com/Work-Rules-Insights-Inside-Transform/dp/1455554790}{no
bearing} on their performance. (Of course, it must be said that if you
got D's, you probably didn't end up at Google.)

\href{https://www.nytimes3xbfgragh.onion/newsletters/sunday-best?action=click\&module=Intentional\&pgtype=Article}{\emph{{[}Never
be uninteresting. Read the most thought-provoking, funny, delightful and
raw stories from The New York Times Opinion section.{]}}}

Academic grades
\href{https://www.nytimes3xbfgragh.onion/2016/09/11/opinion/sunday/why-we-should-stop-grading-students-on-a-curve.html}{rarely
assess} qualities like creativity, leadership and teamwork skills, or
social, emotional and political intelligence. Yes, straight-A students
master cramming information and regurgitating it on exams. But career
success is rarely about finding the right solution to a problem --- it's
more about finding the right problem to solve.

In a classic 1962 study, a team of psychologists tracked down America's
most creative architects and compared them with their technically
skilled but less original peers. One of the factors that distinguished
the creative architects was a record of spiky grades. ``In college our
creative architects earned about a B average,'' Donald MacKinnon
\href{http://psycnet.apa.org/record/1963-04959-001}{wrote}. ``In work
and courses which caught their interest they could turn in an A
performance, but in courses that failed to strike their imagination,
they were quite willing to do no work at all.'' They paid attention to
their curiosity and prioritized activities that they found intrinsically
motivating --- which ultimately served them well in their careers.

Getting straight A's requires conformity. Having an influential career
demands originality. In a
\href{https://www.amazon.com/Lives-Promise-Valedictorians-Fourteen-year-Achievement/dp/0787901466}{study}
of students who graduated at the top of their class, the education
researcher Karen Arnold found that although they usually had successful
careers, they rarely reached the upper echelons. ``Valedictorians aren't
likely to be the future's visionaries,'' Dr. Arnold
\href{http://time.com/money/4779223/valedictorian-success-research-barking-up-wrong/}{explained}.
``They typically settle into the system instead of shaking it up.''

This might explain why Steve Jobs finished high school with a
\href{https://www.theatlantic.com/technology/archive/2012/02/what-was-steve-jobss-high-school-gpa-not-40-or-even-30/252828/}{2.65
G.P.A.}, J.K. Rowling graduated from the University of Exeter with
\href{https://www.thetimes.co.uk/article/got-a-22-like-to-get-stoned-youre-hired-3wxrlczdmtx}{roughly
a C average,} and the Rev. Dr. Martin Luther King Jr. got
\href{https://kinginstitute.stanford.edu/sites/default/files/morehouse_years.pdf}{only
one A} in his four years at Morehouse.

If your goal is to graduate without a blemish on your transcript, you
end up taking easier classes and staying within your comfort zone. If
you're willing to tolerate the occasional B, you can learn to program in
Python while struggling to decipher ``Finnegans Wake.'' You gain
experience coping with failures and setbacks, which
\href{https://reason.com/archives/2017/10/26/the-fragile-generation}{builds
resilience}.

Straight-A students also miss out socially. More time studying in the
library means less time to start lifelong friendships, join new clubs or
volunteer. I know from experience. I didn't meet my 4.0 goal; I
graduated with a 3.78. (This is the first time I've shared my G.P.A.
since applying to graduate school 16 years ago. Really, no one cares.)
Looking back, I don't wish my grades had been higher. If I could do it
over again, I'd study less. The hours I wasted memorizing the inner
workings of the eye would have been better spent trying out improv
comedy and having more midnight conversations about the meaning of life.

So universities: Make it easier for students to take some intellectual
risks. Graduate schools can be clear that they don't care about the
difference between a 3.7 and a 3.9. Colleges could just report letter
grades without pluses and minuses, so that any G.P.A. above a 3.7
appears on transcripts as an A. It might also help to stop the madness
of grade inflation, which creates an academic arms race that encourages
too many students to strive for meaningless perfection. And why not let
students wait until the end of the semester to declare a class
pass-fail, instead of forcing them to decide in the first month?

Employers: Make it clear you value skills over straight A's. Some
recruiters are already on board: In a 2003
\href{https://onlinelibrary.wiley.com/doi/abs/10.1111/j.1744-6570.2003.tb00241.x}{study}
of over 500 job postings, nearly 15 percent of recruiters actively
selected against students with high G.P.A.s (perhaps questioning their
priorities and life skills), while more than 40 percent put no weight on
grades in initial screening.

Straight-A students: Recognize that underachieving in school can prepare
you to overachieve in life. So maybe it's time to apply your grit to a
new goal --- getting at least one B before you graduate.

\href{https://www.theatlantic.com/author/adam-grant/}{Adam Grant}, an
organizational psychologist at Wharton and contributing opinion writer,
is the author of
``\href{http://www.adamgrant.net/originals}{Originals}'' and
``\href{http://www.adamgrant.net/give-and-take}{Give and Take}'' and is
the host of the podcast
``\href{http://www.applepodcasts.com/worklife}{WorkLife}.''

\emph{Follow The New York Times Opinion section on}
\href{https://www.facebookcorewwwi.onion/nytopinion}{\emph{Facebook}}\emph{,}
\href{http://twitter.com/NYTOpinion}{\emph{Twitter (@NYTopinion)}}
\emph{and}
\href{https://www.instagram.com/nytopinion/}{\emph{Instagram}}\emph{.}

Advertisement

\protect\hyperlink{after-bottom}{Continue reading the main story}

\hypertarget{site-index}{%
\subsection{Site Index}\label{site-index}}

\hypertarget{site-information-navigation}{%
\subsection{Site Information
Navigation}\label{site-information-navigation}}

\begin{itemize}
\tightlist
\item
  \href{https://help.nytimes3xbfgragh.onion/hc/en-us/articles/115014792127-Copyright-notice}{©~2020~The
  New York Times Company}
\end{itemize}

\begin{itemize}
\tightlist
\item
  \href{https://www.nytco.com/}{NYTCo}
\item
  \href{https://help.nytimes3xbfgragh.onion/hc/en-us/articles/115015385887-Contact-Us}{Contact
  Us}
\item
  \href{https://www.nytco.com/careers/}{Work with us}
\item
  \href{https://nytmediakit.com/}{Advertise}
\item
  \href{http://www.tbrandstudio.com/}{T Brand Studio}
\item
  \href{https://www.nytimes3xbfgragh.onion/privacy/cookie-policy\#how-do-i-manage-trackers}{Your
  Ad Choices}
\item
  \href{https://www.nytimes3xbfgragh.onion/privacy}{Privacy}
\item
  \href{https://help.nytimes3xbfgragh.onion/hc/en-us/articles/115014893428-Terms-of-service}{Terms
  of Service}
\item
  \href{https://help.nytimes3xbfgragh.onion/hc/en-us/articles/115014893968-Terms-of-sale}{Terms
  of Sale}
\item
  \href{https://spiderbites.nytimes3xbfgragh.onion}{Site Map}
\item
  \href{https://help.nytimes3xbfgragh.onion/hc/en-us}{Help}
\item
  \href{https://www.nytimes3xbfgragh.onion/subscription?campaignId=37WXW}{Subscriptions}
\end{itemize}
