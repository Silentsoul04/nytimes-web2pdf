Sections

SEARCH

\protect\hyperlink{site-content}{Skip to
content}\protect\hyperlink{site-index}{Skip to site index}

\href{https://www.nytimes3xbfgragh.onion/section/reader-center}{Reader
Center}

\href{https://myaccount.nytimes3xbfgragh.onion/auth/login?response_type=cookie\&client_id=vi}{}

\href{https://www.nytimes3xbfgragh.onion/section/todayspaper}{Today's
Paper}

\href{/section/reader-center}{Reader Center}\textbar{}How to Make a
Times Crossword Puzzle

\url{https://nyti.ms/2GtfTxl}

\begin{itemize}
\item
\item
\item
\item
\item
\item
\end{itemize}

Advertisement

\protect\hyperlink{after-top}{Continue reading the main story}

Supported by

\protect\hyperlink{after-sponsor}{Continue reading the main story}

Understanding The Times

\hypertarget{how-to-make-a-times-crossword-puzzle}{%
\section{How to Make a Times Crossword
Puzzle}\label{how-to-make-a-times-crossword-puzzle}}

The New York Times crossword editors reveal their process for evaluating
and editing a puzzle submission.

\includegraphics{https://static01.graylady3jvrrxbe.onion/images/2018/12/30/reader-center/xx-utt-crosswords/xx-utt-crosswords-articleLarge.jpg?quality=75\&auto=webp\&disable=upscale}

By Will Shortz and Joel Fagliano

\begin{itemize}
\item
  Dec. 18, 2018
\item
  \begin{itemize}
  \item
  \item
  \item
  \item
  \item
  \item
  \end{itemize}
\end{itemize}

\emph{In an effort to shed more light on how we work, The Times is
running a series of short posts explaining some of our journalistic
practices.}
\href{https://www.nytimes3xbfgragh.onion/series/understanding-the-times?module=inline}{\emph{Read
more from this series here}}\emph{.}

Ever wondered how your daily crossword puzzle gets to the pages of The
New York Times or to your puzzle app? How do the editors Will Shortz and
Joel Fagliano evaluate constructors' submissions and decide which day of
the week it should run? What changes, if any, do they make to the
author's creation?

\href{https://www.nytimes3xbfgragh.onion/2018/09/14/crosswords/how-to-make-a-crossword-puzzle-the-series.html}{Wordplay},
the crossword column, pulled back the curtain recently in a series of
articles in which four successive pairs of Times constructors
collaborated on a
\href{https://static01.graylady3jvrrxbe.onion/packages/other/crossword/HTMAC.pdf}{sample
crossword puzzle} --- focusing on theme, grid design, fill and clues.
Below, in a condensed version of the final installment of the series,
Mr. Shortz and Mr. Fagliano review the sample puzzle, divulging their
editorial process along the way.

To make it easier for you to follow along, the theme here is song
synonyms consistently in the second position of two-word phrases: i.e.,
PUZZLE PIECE, TENURE TRACK, CALL NUMBER and MACBOOK AIR.

Take it away, guys!

\emph{--- Deb Amlen}

\begin{center}\rule{0.5\linewidth}{\linethickness}\end{center}

\hypertarget{the-envelope-please}{%
\subsection{The Envelope, Please}\label{the-envelope-please}}

WILL SHORTZ: Typically, when we accept a puzzle, we file it for the day
of the week for which we think it is most naturally suited. Monday has
an easy theme with easy vocabulary. Thursday tends to have the trickiest
themes. Tuesday and Wednesday are in the middle. Friday and Saturday are
the hardest, and usually themeless.

Then when we come to editing, we select a week's worth of puzzles at a
time, trying to vary the themes and the puzzle makers.

JOEL FAGLIANO: So if Tuesday's theme has circles, we might avoid a
puzzle with circles on Monday and Wednesday. If the theme involves puns,
we'd be unlikely to run more than one other puzzle during the same week
with puns.

SHORTZ: I think of the Times crossword as being like a three-ring
circus. What brings joy and awe is being surprised. I like every day's
puzzle to have a little surprise.

\hypertarget{the-theme}{%
\subsection{The Theme}\label{the-theme}}

SHORTZ: When I first saw the theme of this puzzle, honestly, I wasn't
sure it was my cup of tea. But I'm liking it better now that I've seen
the clues that go with it.

FAGLIANO: Yeah, what I think is done well in the theme is the
parallelism. The synonyms are all placed as the last words, which allows
the solver to have some expectation --- O.K., the next theme answer is
going to involve some synonym of ``song'' at the end and another word at
the beginning.

Another asset of the theme is that the dictionary meaning of each phrase
is pretty far from how it's clued. For example, TENURE TRACK has nothing
to do with a song, which is good.

SHORTZ: In an ideal puzzle, all the key words in the answer are not
referred to in the clue. Here, PUZZLE PIECE --- clued as ``crossword
enthusiast's favorite song'' --- is still about a PUZZLE, although it's
changed a bit from a jigsaw puzzle, so there's a little twist there.
CALL NUMBER is maybe my favorite theme example, because both key words
in the answer are used playfully in the clue: ``telemarketer's favorite
song.''

\hypertarget{the-grid-design}{%
\subsection{The Grid Design}\label{the-grid-design}}

SHORTZ: So, this grid (designed around the puzzle's themed answers) has
76 words. The maximum we allow typically in weekday crosswords is 78.
It's nice that the word count here is two under our maximum, which means
the answers average a little longer than usual. Something else I like
about the grid: It has great flow. No corner or section is isolated from
the rest, hanging on by a single square, say. Once you start solving,
you can keep moving around the grid. If you get stuck, there are several
ways to get unstuck.

\hypertarget{the-fill}{%
\subsection{The Fill}\label{the-fill}}

SHORTZ: If we like a theme and grid well enough, then we look at the
puzzle's fill (the words, peripheral to the theme, with which the rest
of the boxes in the grid have been filled).

We ask for submissions on paper rather than by email because it's easier
for us to examine the whole grid at once, and to mark up the manuscript
with pluses, minuses and other comments.

FAGLIANO: We look at all the Across answers first, and then all the
Downs, making minus marks for answers we think are subpar, check marks
for answers we like, exclamation points for ``Wow!,'' question marks for
things to be looked up and sometimes written comments. When we're done,
this helps us to visualize potential issues: ``O.K., there are a lot of
minus marks in this one corner --- this is an area that needs to be
revised.''

SHORTZ: Looking at the Downs \ldots{} ANTE UP \ldots{} wasn't there
another answer with UP in the puzzle? Yes, MEET UP. That doesn't bother
me, though. UP is an inconspicuous word.

And OLIVIA MUNN \ldots{} Well, I'm going to expose my ignorance, but I
don't know who she is.

FAGLIANO: She's an actress. On ``The Newsroom'' and other things.

SHORTZ: Hmm, and next to that answer is DANA SCULLY. I do know her, of
course, but that could be troublesome for some solvers.

FAGLIANO: Yeah, and there's a third name to the right of it: SPACEK. In
this case, we'd really need to check the crossings to make sure we're
not setting up solvers to get stuck.

SHORTZ: Over all, the fill looks good to me. But I would want to clue
the upper right of the grid on the easy side, for solvers who don't know
all those names.

\hypertarget{the-clues}{%
\subsection{The Clues}\label{the-clues}}

SHORTZ: Often we'll edit the theme clues first, because those are the
most important ones. Then we'll return to the top.

We ask constructors to send us manuscripts with
\href{https://www.nytimes3xbfgragh.onion/crosswords/submissions}{the
clues typed on the left, double spaced, with the answer words on the far
right}. This is for our convenience when editing. We'll go through the
clues one by one.

The most important thing is \textbf{accuracy}. It doesn't matter how
interesting or clever a clue is if it's wrong. So anything we aren't 100
percent certain of, we will verify.

Besides accuracy, we edit clues for the \textbf{level of difficulty}
appropriate to the day of the week on which the puzzle will appear; then
for \textbf{colorfulness, freshness, sense of fun}.

On average, about half the clues are changed in the editing process. The
number can be as low as 5 percent for someone who writes terrific clues,
and as high as 95 percent for someone who has a great theme and grid but
isn't necessarily an experienced clue writer.

\textbf{Brevity} is important, too. Partly it's for reasons of space on
the printed page, which is limited. But even online, where space is not
a real consideration, it's nicer to have generally shorter clues. It's
like that old saying from ``Hamlet'': ``Brevity is the soul of wit.''

\hypertarget{on-using-databases}{%
\subsubsection{On using databases}\label{on-using-databases}}

SHORTZ: There have been some constructors we've noticed who have taken
most or all of their clues from a database, such as XWord Info. That
turns me off. Part of the process of making a great crossword is writing
original clues, and it makes for more interesting solving. \emph{Some}
of the clues, of course, may repeat old ones. There are only so many
ways to clue certain words. But I'd like constructors to make an honest
effort.

\hypertarget{on-repeating-words-that-are-in-the-grid}{%
\subsubsection{On repeating words that are in the
grid}\label{on-repeating-words-that-are-in-the-grid}}

FAGLIANO: Our basic rule is that no answer in its entirety should be
repeated as a clue, and no clue in its entirety should be repeated as an
answer. But if just part of a clue appears as part of an answer, we
usually don't mind.

For example, we would avoid ``ice cream'' as a clue if ICE was already
an answer on its own in the grid. But if a clue said ``Eat some ice
cream'' and there was also ICE SKATING in the grid, that would be fine.

\hypertarget{on-say-clues-and-question-mark-clues}{%
\subsubsection{On ``say'' clues and question mark
clues}\label{on-say-clues-and-question-mark-clues}}

SHORTZ: Clues shouldn't have too many waffle words. It's O.K. to use
qualifiers like ``perhaps,'' ``maybe'' and ``say'' once in a while, but
if they're used too much a solver may get frustrated: ``Just tell me
what the damn thing means!''

FAGLIANO: I would extend that same sentiment to question mark clues.
When done well, a clue with a question mark or a joke can brighten a
whole corner of a puzzle. But if every single clue is trying to mislead
you --- even on a Friday or Saturday --- that can become annoying. So
even in our hardest puzzles, we try to provide plenty of straight
definitions for the solver to work with.

\hypertarget{on-fill-in-the-blank-clues}{%
\subsubsection{On fill-in-the-blank
clues}\label{on-fill-in-the-blank-clues}}

SHORTZ: Fill-in-the-blank clues should be used sparingly, and they
should be interesting. ``\_\_\_ circus'' (with FLEA as the answer), for
instance, really is not an interesting clue.

FAGLIANO: Exactly. FLEA is a perfectly interesting word on its own, so
we'd probably want to clue it straight.

\hypertarget{on-the-consistency-between-clues-and-answers}{%
\subsubsection{On the consistency between clues and
answers}\label{on-the-consistency-between-clues-and-answers}}

SHORTZ: For the answer C'MON (47D), the clue was ``You're pulling my
leg, right?'' The problem here is that the clue is in the form of a
question, while the answer is a statement. I'd rephrase the clue to be
in the form of a statement as well.

\hypertarget{on-brand-names}{%
\subsubsection{On brand names}\label{on-brand-names}}

SHORTZ: At 1D, the clue is ``Sleeveless Victoria's Secret purchase,
informally.'' I have no problem with commercial names in puzzles, but I
don't include them gratuitously. In this case, there's no strong reason
to mention Victoria's Secret. I'd probably change the clue to
``Sleeveless women's undergarment, informally.''

\hypertarget{the-tone}{%
\subsection{The Tone}\label{the-tone}}

FAGLIANO: If there was an answer of MURDER, we might clue it as ``Group
of crows'' rather than choosing any of the numerous more grisly ways to
clue it.

SHORTZ: Or we could say ``Topic for Agatha Christie.'' Often putting
something in a fictional context can lighten it.

My feeling is that the crossword should reflect everything in life,
positive and negative. Not everything in life is peaches and cream.
While I wouldn't want a whole puzzle theme that's depressing, a single
downbeat word here and there is fine.

FAGLIANO: In the case of the word SERF, for instance, there aren't
likely to be any serfs doing our crossword, so it's hard to imagine
someone taking personal offense. That said, crosswords should entertain
and uplift --- and generally, a light tone is desirable.

\emph{If you're thinking about submitting a puzzle, check out our}
\href{https://www.nytimes3xbfgragh.onion/crosswords/submissions}{\emph{crossword
puzzle submission guidelines}}\emph{.}

A note to readers who are not subscribers: This article from the
\href{https://www.nytimes3xbfgragh.onion/section/reader-center}{Reader
Center} does not count toward your monthly free article limit.

Follow the \href{https://twitter.com/readercenter}{@ReaderCenter} on
Twitter for more coverage highlighting your perspectives and experiences
and for insight into how we work.

Advertisement

\protect\hyperlink{after-bottom}{Continue reading the main story}

\hypertarget{site-index}{%
\subsection{Site Index}\label{site-index}}

\hypertarget{site-information-navigation}{%
\subsection{Site Information
Navigation}\label{site-information-navigation}}

\begin{itemize}
\tightlist
\item
  \href{https://help.nytimes3xbfgragh.onion/hc/en-us/articles/115014792127-Copyright-notice}{©~2020~The
  New York Times Company}
\end{itemize}

\begin{itemize}
\tightlist
\item
  \href{https://www.nytco.com/}{NYTCo}
\item
  \href{https://help.nytimes3xbfgragh.onion/hc/en-us/articles/115015385887-Contact-Us}{Contact
  Us}
\item
  \href{https://www.nytco.com/careers/}{Work with us}
\item
  \href{https://nytmediakit.com/}{Advertise}
\item
  \href{http://www.tbrandstudio.com/}{T Brand Studio}
\item
  \href{https://www.nytimes3xbfgragh.onion/privacy/cookie-policy\#how-do-i-manage-trackers}{Your
  Ad Choices}
\item
  \href{https://www.nytimes3xbfgragh.onion/privacy}{Privacy}
\item
  \href{https://help.nytimes3xbfgragh.onion/hc/en-us/articles/115014893428-Terms-of-service}{Terms
  of Service}
\item
  \href{https://help.nytimes3xbfgragh.onion/hc/en-us/articles/115014893968-Terms-of-sale}{Terms
  of Sale}
\item
  \href{https://spiderbites.nytimes3xbfgragh.onion}{Site Map}
\item
  \href{https://help.nytimes3xbfgragh.onion/hc/en-us}{Help}
\item
  \href{https://www.nytimes3xbfgragh.onion/subscription?campaignId=37WXW}{Subscriptions}
\end{itemize}
