Sections

SEARCH

\protect\hyperlink{site-content}{Skip to
content}\protect\hyperlink{site-index}{Skip to site index}

\href{https://myaccount.nytimes3xbfgragh.onion/auth/login?response_type=cookie\&client_id=vi}{}

\href{https://www.nytimes3xbfgragh.onion/section/todayspaper}{Today's
Paper}

One of America's Most Vital Exports, Education, Never Goes Abroad, but
It Still Faces Threats

\url{https://nyti.ms/2RsZDRb}

\begin{itemize}
\item
\item
\item
\item
\item
\end{itemize}

Advertisement

\protect\hyperlink{after-top}{Continue reading the main story}

Supported by

\protect\hyperlink{after-sponsor}{Continue reading the main story}

\href{/column/on-money}{On Money}

\hypertarget{one-of-americas-most-vital-exports-education-never-goes-abroad-but-it-still-faces-threats}{%
\section{One of America's Most Vital Exports, Education, Never Goes
Abroad, but It Still Faces
Threats}\label{one-of-americas-most-vital-exports-education-never-goes-abroad-but-it-still-faces-threats}}

\includegraphics{https://static01.graylady3jvrrxbe.onion/images/2019/01/06/magazine/06OnMoney_1/06OnMoney_1-articleLarge.jpg?quality=75\&auto=webp\&disable=upscale}

By Brook Larmer

\begin{itemize}
\item
  Jan. 3, 2019
\item
  \begin{itemize}
  \item
  \item
  \item
  \item
  \item
  \end{itemize}
\end{itemize}

It's no mere coincidence that Jeffrey R. Brown, the dean of the Gies
College of Business, at the University of Illinois at Urbana-Champaign,
is also a scholar of risk management. At his first faculty meeting four
years ago, Brown fretted that his school had become, like many American
universities, overly dependent on a single source of money --- roughly a
fifth of tuition revenue came from Chinese students. ``I saw our
reliance on China as a risk,'' says Brown, noting that 800 or so of the
university's nearly 5,800 Chinese students attend the business school.
``At the time, I was concerned that China could pull the plug on
students coming to America. But then the U.S. political landscape
shifted, and we were vulnerable to changing visa and immigration
policies here, too. Both were threats we could not control.''

Brown approached the problem as a risk manager, which meant doing what
no school has been known to do before: buying insurance to protect
against a sudden drop in Chinese enrollment. The three-year policy
requires the university to pay \$424,000 annually for up to \$60 million
in coverage. The insurer, Lloyd's of London, will pay out a claim if a
specific incident --- a visa ban because of a government action, for
example --- causes the number of Chinese students in the colleges of
business and engineering to decline by 18.5 percent over a 12-month
period. The political risks have only grown since the policy went into
effect. The trade war with China, visa restrictions and the
anti-immigrant rhetoric coming from the White House are making it more
complicated for international students to come to America. ``It's a
tough environment right now,'' Brown says. But the insurance gives him
peace of mind. The school continues to invest in China, he says, but now
``with the confidence that we are not doubling down on risk.''

Over the past decade, the explosion in the number of international
students has turned education, almost by stealth, into one of the most
vital American exports. The idea that a student taking classes in Iowa
City or Ann Arbor can be counted as an export might seem strange. In
economic terms, however, the student's situation is not so different
from, say, a Japanese company buying American soybeans: Foreign money
flows into the United States from abroad --- except that in this case,
the product doesn't leave the country.

Nearly 1.1 million international students attended American colleges and
universities in 2017. They generated \$42.4 billion in export revenue,
more than double the amount eight years ago, according to the Bureau of
Economic Analysis. (Because far fewer Americans study abroad, the United
States ran a \$34.2 billion surplus in education in 2017.) Nafsa, a
nonprofit group that supports international education, estimates that
students from abroad created or sustained more than 455,000 jobs in the
United States, almost nine times the number of American coal miners. The
value of education is almost double the revenue from America's top
agricultural export in 2017, soybeans (\$21.6 billion). When other
student spending is factored in --- food, cars, clothes --- education's
total export value rivals that of pharmaceuticals (\$51 billion) and
automobiles (\$53 billion). ``In the public at large, there's little
awareness that higher education is one of America's biggest exports,''
says Rajika Bhandari, senior adviser for research and strategy at the
Institute of International Education. ``Or that this export drives
American competitiveness.''

\textbf{America's reputation as} a beacon for the world's students,
however, is now faltering. In 2016, for the first time in decades, new
enrollments of international students in United States colleges and
universities fell, by 3.3 percent, according to an Institute of
International Education report. In 2017, the decrease steepened, to 6.6
percent. An 8.8 percent drop in graduate students from India led the
decline, while China's torrid pace of enrollments slowed. The downturn,
which the institute reports continued in the fall of 2018, albeit at a
lesser rate, is not solely in response to American politics. The first
drop, after all, reflected decisions made in 2015, when Donald Trump was
just one of 17 Republicans battling for the party nomination. Other
factors have also been altering the calculus for international students:
rising tuition costs in the United States; growing competition from
other countries like Australia and Canada; heavy investment in higher
education in their home countries; and a fear of American gun violence.

\includegraphics{https://static01.graylady3jvrrxbe.onion/images/2019/01/06/magazine/06OnMoney_2/06OnMoney_2-articleLarge.jpg?quality=75\&auto=webp\&disable=upscale}

Still, the ``Trump effect'' acts as a powerful deterrent. Shortly before
the 2016 election, I spoke with Chinese students at the University of
Iowa who were so convinced that Trump's travel ban would extend to
Chinese nationals that they canceled trips home out of fear that they
would not be let back into the United States. Some of their friends in
China opted to study in England to avoid the uncertainty. Their fears
seemed unfounded at first. But as the trade war has escalated, along
with reports of espionage and intellectual-property theft, Chinese
students find themselves squarely in Trump's cross hairs. In November,
according to The Financial Times, the White House even briefly discussed
the idea of imposing a total ban on Chinese students --- a radical
option that inevitably brings back memories of the Chinese Exclusion
Act, which kept out most Chinese for nearly 60 years before the outbreak
of World War II.

As a practical matter, the tightening of American visa restrictions hits
students harder than dark grumblings. Beyond generating more visa
denials and delays --- the issuance of F-1 student visas declined 17
percent from 2016 to 2017 --- Washington is targeting foreign students
who apply for graduate work or training in technological areas. In June,
citing national-security concerns, the government announced that visas
for Chinese students in robotics, aviation and high tech could be
reduced to just one year from five years. Last month, 65 colleges and
universities signed a letter supporting a legal challenge to a new
United States government policy that would make it easier to ban
international students for overstaying their visas, calling
international students ``essential to the fabric'' of higher
education.'' Esther Brimmer, executive director and chief executive of
Nafsa, laments the hostile turn: ``It could take us years to rebuild the
reputation of America as a nation that welcomes all to our campus
communities.''

\textbf{Some universities} are trying to buck the trend. Chief among
them is the University of Illinois, where a commitment to welcoming
Chinese students stretches back more than a century. In 1906, when the
policy of excluding Chinese was still in force, the president of the
university, Edmund James, wrote a letter to President Theodore Roosevelt
proposing a scholarship program to bring some of the first Chinese
students to American universities. ``China is upon the verge of a
revolution,'' he explained. ``The nation which succeeds in educating the
young Chinese of the present generation will be the nation which for a
given expenditure of effort will reap the largest possible returns in
moral, intellectual and commercial influence.'' Roosevelt agreed. Over
the next half century, the University of Illinois received roughly a
third of all Chinese students who came to the United States.

Today, again, Illinois is leading a pack of universities trying to
``reap the largest possible returns'' from international students ---
and, with its insurance policy, to avoid any sudden losses. The number
of Chinese students at Illinois, 5,797, remains one of the highest in
the country, but in 2018, for the first time in decades, Chinese
enrollment there dropped, by 2.2 percent. (The business school, Brown
says, suffered no decline.) Even as national policy tightens,
universities are intensifying recruitment efforts beyond China and India
and trying to make students feel more welcome. In recent years, Purdue
University has begun to reduce its intake of Chinese students in a move
that will diversify its student body.

It's shortsighted to think of this issue solely in terms of revenue:
International students and scholars also spur American innovation and
growth. Today, nearly a quarter of the founders of billion-dollar United
States start-up companies first came here as international students,
according to the National Foundation for American Policy. And the
National Science Foundation says that more than a third of the
postdoctoral researchers in science, engineering and health in American
labs are temporary visa holders. ``We don't have enough Americans
trained for these jobs in engineering, computer science, even
economics,'' Richard Startz, an economics professor at the University of
California, Santa Barbara, says.

For all the growing concerns about national security, the United States
can't afford to close the pipeline of talent and tuition that supports
its education system and drives its economic future. As Yu He, a postdoc
research scholar at the Stanford Institute for Materials and Energy
Sciences, wrote in the online magazine ChinaFile, ``If the secret sauce
to the success of the United States is the ability to attract and retain
`the best and the brightest,' why is it shooting itself in the foot
today?''

Advertisement

\protect\hyperlink{after-bottom}{Continue reading the main story}

\hypertarget{site-index}{%
\subsection{Site Index}\label{site-index}}

\hypertarget{site-information-navigation}{%
\subsection{Site Information
Navigation}\label{site-information-navigation}}

\begin{itemize}
\tightlist
\item
  \href{https://help.nytimes3xbfgragh.onion/hc/en-us/articles/115014792127-Copyright-notice}{©~2020~The
  New York Times Company}
\end{itemize}

\begin{itemize}
\tightlist
\item
  \href{https://www.nytco.com/}{NYTCo}
\item
  \href{https://help.nytimes3xbfgragh.onion/hc/en-us/articles/115015385887-Contact-Us}{Contact
  Us}
\item
  \href{https://www.nytco.com/careers/}{Work with us}
\item
  \href{https://nytmediakit.com/}{Advertise}
\item
  \href{http://www.tbrandstudio.com/}{T Brand Studio}
\item
  \href{https://www.nytimes3xbfgragh.onion/privacy/cookie-policy\#how-do-i-manage-trackers}{Your
  Ad Choices}
\item
  \href{https://www.nytimes3xbfgragh.onion/privacy}{Privacy}
\item
  \href{https://help.nytimes3xbfgragh.onion/hc/en-us/articles/115014893428-Terms-of-service}{Terms
  of Service}
\item
  \href{https://help.nytimes3xbfgragh.onion/hc/en-us/articles/115014893968-Terms-of-sale}{Terms
  of Sale}
\item
  \href{https://spiderbites.nytimes3xbfgragh.onion}{Site Map}
\item
  \href{https://help.nytimes3xbfgragh.onion/hc/en-us}{Help}
\item
  \href{https://www.nytimes3xbfgragh.onion/subscription?campaignId=37WXW}{Subscriptions}
\end{itemize}
