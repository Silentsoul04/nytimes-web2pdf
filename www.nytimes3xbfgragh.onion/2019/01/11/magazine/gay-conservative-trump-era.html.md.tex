For Gay Conservatives, the Trump Era is the Best and Worst of Times

\url{https://nyti.ms/2H7Lh55}

\begin{itemize}
\item
\item
\item
\item
\item
\item
\end{itemize}

\includegraphics{https://static01.graylady3jvrrxbe.onion/images/2019/01/20/magazine/20mag-gayconservatives-slide-7GD1/20mag-gayconservatives-slide-7GD1-articleLarge.jpg?quality=75\&auto=webp\&disable=upscale}

Sections

\protect\hyperlink{site-content}{Skip to
content}\protect\hyperlink{site-index}{Skip to site index}

Feature

\hypertarget{for-gay-conservatives-the-trump-era-is-the-best-and-worst-of-times}{%
\section{For Gay Conservatives, the Trump Era is the Best and Worst of
Times}\label{for-gay-conservatives-the-trump-era-is-the-best-and-worst-of-times}}

Inside the emboldened, if hardly unified, ranks of the L.G.B.T. right.

Ben Holden: ``Demographics shouldn't be destiny.''Credit...Peyton
Fulford for The New York Times

Supported by

\protect\hyperlink{after-sponsor}{Continue reading the main story}

By
\href{https://www.nytimes3xbfgragh.onion/by/benoit-denizet-lewis}{Benoit
Denizet-Lewis}

\begin{itemize}
\item
  Jan. 11, 2019
\item
  \begin{itemize}
  \item
  \item
  \item
  \item
  \item
  \item
  \end{itemize}
\end{itemize}

Hannity is a buffoon,'' Ben Holden said, perhaps a bit too loudly.
Holden was drinking disappointing sangria with a friend at the bar of
the Trump International Hotel in Washington, where he had come last
February more out of curiosity than reverence for the president. He was
in town for his first Conservative Political Action Conference (CPAC),
an event that he took seriously enough to dress up for (dark suit,
American-flag tie) but that he was also interested in for its
anthropological weirdness. A 23-year-old student at Suffolk University
in Boston who is gay and ``leans conservative,'' Holden planned to take
copious notes and write a gonzo-style journalistic piece about a
political gathering known as much for its raucous parties as its
provocative speakers.

Holden wasn't the only young L.G.B.T. person in the Trump lobby that
night. A few feet away, several conservative gay and bisexual
journalists and activists reclined on couches. Among them was Charlie
Nash, a tweed-wearing 21-year-old British reporter for Breitbart who
described himself to me as a pagan, an absurdist and a right-wing
environmentalist. Next to Nash was Lucian Wintrich, the 30-year-old
former White House correspondent for The Gateway Pundit, a
conspiracy-peddling far-right website founded by another gay man, Jim
Hoft, to ``expose the wickedness of the left.'' Wintrich is perhaps best
known for his
\href{https://broadly.vice.com/en_us/article/ypaab7/make-america-hairless-again-sensual-photos-of-twinks-for-trump}{Twinks4Trump
photo series}, in which he photographed lithe young men wearing Make
America Great Again baseball caps.

At the bar, Holden and a fellow Suffolk student were joined by a
heavyset man in a colorful checkered shirt. Before telling them his name
(and asking that I not use it), the man introduced himself by way of a
toast: ``We're going to build that wall! We're going to make America
great again!''

Holden's friend challenged the man to an arm-wrestling contest before
having second thoughts. ``Actually, my masculinity is not worth sweating
over in a zero-sum situation,'' he said.

``That's nonsense!'' the man told him. ``There's an economic benefit to
masculinity.''

This led to some back-and-forth about economics and gender theory before
Holden's friend relented and assumed an arm-wrestling position at the
bar. The showdown didn't go his way. ``I think you got help from the
Russians,'' he said.

``Collusion!'' the man shouted with delight.

Before long, it became clear why he had joined the students in
conversation: to hit on Holden, who is tall and broad-shouldered and has
big, protruding ears that add to an aura of youthful affability. But
even as the man flirted he confided that he was deeply closeted and, in
fact, saw his same-sex attractions as a kind of affliction. Still, he
wanted Holden's phone number.

Holden couldn't relate to someone with shame about his sexual
orientation, nor to those he called the ``loud gay Trump fanboys,''
referring to people like Wintrich and the former Breitbart senior editor
Milo Yiannopoulos, who are both categorized by the Anti-Defamation
League as ``alt lite,'' a designation given to those the organization
says ``are in step with the alt-right in their hatred of feminists and
immigrants, among others.'' Wintrich and Yiannopoulos have made careers
out of social media trolling and incendiary campus speeches tailored to
outrage progressive students. (Wintrich titled a 2017 talk ``It's O.K.
to Be White.'') Holden saw them as gay minstrels producing a kind of
garish, campy performance art meant more to shock than to make a cogent
argument. He wasn't sure what they actually believed.

For his part, Holden said he believed his sexual orientation was one of
the least interesting things about him. ``Being gay is not an
accomplishment in and of itself,'' he told me, ``so I'm reluctant to
lead with it or believe that it should dictate how I think about health
care.'' Holden was increasingly skeptical of tribalism and extremist
elements of both parties; he seemed almost traumatized after attending
the live CPAC taping of ``Hannity,'' describing some in the crowd as
``maladjusted and mindless'' and ``dredged up from the savage American
hinterlands.''

Though he said he is liberal on most social issues and wishes the
Republican Party would take climate change seriously, Holden aligns
himself with conservatives and libertarians in many other ways --- he's
anti-abortion, free-market-oriented and skeptical of big government. But
perhaps above all else, Holden rejects what he considers a bedrock of
contemporary liberalism: that, as he put it, your ``immutable
characteristics'' --- race, ethnicity, sexual orientation --- ``should
determine what your position is on every political issue, or what you're
allowed to express an opinion about.'' He added that he feels alienated
from progressives on his campus and across the country, many of whom he
believes are unwilling to debate issues ``without resorting to shaming
or name-calling.''

Holden certainly didn't endear himself to most students on his campus
when he showed up to classes wearing a MAGA hat a week before the 2016
presidential election. In retrospect, he said, he wasn't proud of his
support for Trump. ``I think I did it mostly out of spite,'' he told me.
``It was a kind of `F you' to the left and the Democratic Party, which
is doubling down on intersectionality and identity politics.''

After barely an hour at the bar, Holden and his friend returned to
CPAC's host hotel, the Gaylord National Resort in National Harbor, Md.
Holden considered the Gaylord a fitting name for a conference with many
openly gay attendees, including Log Cabin Republicans (a conservative
L.G.B.T. group founded in 1977), Fox News analysts, transgender women
and students from across the country.

Gregory T. Angelo, a 40-year-old longtime communications specialist who
until recently was the president of the Log Cabin Republicans, told me
that he had never seen so many openly gay conservatives at CPAC.
``They're coming out in recent years in a way they have not before,'' he
said. Guy Benson, 33, a conservative writer and Fox News on-air
contributor who came out publicly in 2015, told me that the conservative
gay movement has become diverse enough in the past few years ``to have
multiple constituencies with vastly different priorities and political
styles.''

Many L.G.B.T. conservatives say they feel newly relevant and accepted in
the Republican Party, which has long opposed L.G.B.T. rights. And,
perhaps counterintuitive, some attribute this in part to Trump himself.
``The narrative on the left tends to be that Trump is horrible for
L.G.B.T. people in every way imaginable, but that's not how many gays on
the right see it,'' Benson told me. ``As a candidate, Trump signaled
that L.G.B.T.-related culture wars are not ones the G.O.P. needs to be
fighting anymore, and much of the base noticed. As flawed as Trump is,
and despite some of his unfortunate policy moves on this front, he might
actually represent a fulcrum point within the party on gay issues.''

Some gay conservatives feel so emboldened, in fact, that they ``shout
about their love of the president and their L.G.B.T. identity from
rooftops,'' Angelo told me. (By ``rooftops,'' he mostly meant Twitter.)
Standing in front of the Log Cabin booth at CPAC, next to a poster
affirming the organization's support for the Second Amendment, Angelo
didn't shout, but he did beam as he showed me a letter Trump wrote in
2017 congratulating the group on its 40th anniversary. Trump is the
first sitting Republican president to publicly commend the organization.

What a difference three years can make. In 2015, CPAC
\href{https://www.politico.com/story/2015/02/log-cabin-republicans-cpac-115336}{wouldn't
even let the Log Cabin Republicans set up a booth} at the conference.
But now here they were, snapping pictures in front of their booth and
poster (``This will be sure to trigger my entire school at once,'' a Log
Cabin intern said) and basking in enthusiastic thumbs-ups from
convention attendees. Though Angelo conceded the Republican Party
``still has work to do'' on L.G.B.T. issues, he insisted the future has
never looked brighter. ``It's a good time to be a gay conservative,'' he
said.

The reality, I would come to learn, is a bit more complicated than that.

\textbf{There have historically} been few good times to be a gay
conservative. Gay Republicans have spent the better part of several
decades being excoriated from all sides, largely rejected by their party
and alternately mocked and reviled by many in the L.G.B.T. community.
When I asked Rob Smith --- a 36-year-old Iraq War veteran and former
Democrat who is now a conservative --- about the longtime narrative
associated with gays in the G.O.P., he didn't hesitate. ``Self-hating
queens,'' he said.

\includegraphics{https://static01.graylady3jvrrxbe.onion/images/2019/01/20/magazine/20mag-gayconservatives-slide-KJVS/20mag-gayconservatives-slide-KJVS-articleLarge.jpg?quality=75\&auto=webp\&disable=upscale}

Gay conservatives have offered endless fodder for comedians. David
Letterman took a shot during the 2004 Republican National Convention:
``You know the Log Cabin Republicans --- they hate Hillary Clinton, but
they love what she's done with her hair.'' Jimmy Dore, co-host of the
Young Turks' ``The Aggressive Progressives'' web series, joked during a
2007 standup routine, ``They're gay Republicans --- they're people who
are gay and, on purpose, are Republicans.''

When not mocking gay conservatives, comedians --- as well as many in the
L.G.B.T. community --- have delighted in the sex scandals of closeted
gay Republican lawmakers across the country, who often voted against gay
rights even as they solicited gay sex in restrooms, hired male escorts
or hooked up with men in their congressional offices. But gay
Republicans have also long been seen by many in the L.G.B.T. community
as no laughing matter. They're routinely denounced for supporting a
party that only 4 percent of L.G.B.T. people view as ``friendly'' toward
the L.G.B.T. community,
\href{http://www.pewsocialtrends.org/2013/06/13/a-survey-of-lgbt-americans/}{according
to a 2013 Pew poll.}

Gay Republicans have typically offered two reasons for remaining loyal
to a party that offers little reciprocation. The first is that while
they wish the party were better on L.G.B.T. issues, they prioritize
other concerns more. ``Why should I be a Democrat when I disagree with
Democrats on most issues?'' Sarah Longwell, the 38-year-old chairwoman
of the Log Cabin Republicans, asked me. ``I became interested in
conservative ideas, particularly economic ideas, in high school. I knew
I was conservative before I knew I was gay.''
\href{https://www.prageru.com/videos/im-gayconservativeso-what}{In a
video} on the home page of PragerU, a conservative video site, Guy
Benson explained his political priorities: ``I'm a Christian, a
patriotic American and a free-market, shrink-the-government conservative
who also happens to be gay.''

But gay conservatives also speak of their party affiliation as a kind of
public service. Many have insisted for decades that their presence in
the G.O.P. (their ``place at the table,'' as some put it) has helped it
evolve, however slowly, on L.G.B.T. rights. In recent years, gay and
lesbian conservatives have been especially eager to take partial credit
for the legalization of same-sex marriage. ``You weren't going to have
the cultural shift on gay marriage without Republicans talking to
Republicans about gay dignity and why gay marriage is important,''
Longwell said.

In a new documentary about the Log Cabin Republicans produced by the
organization, longtime members also champion their 2004 lawsuit to
overturn ``don't ask, don't tell,'' the Clinton administration policy on
gay, bisexual and lesbian service members, which the organization
opposed because it required service members to conceal their sexual
orientation. ``It was Clinton and the Democratic Party that passed
`don't ask, don't tell,' '' a Log Cabin member says on camera. ``We
fought that for 20 years.''

Listening to gay Republicans take credit for gay civil rights victories
is a mind-bending exercise for many L.G.B.T. people. The writer and
sex-advice columnist Dan Savage, who has publicly called gay Republicans
``house faggots,'' told me that ``the G.O.P. continues to be an
anti-queer political movement, and these useful idiots continue to let
themselves be used by the party to inoculate itself against charges of
homophobia and transphobia.''

Though L.G.B.T. activists have never had particularly nice things to say
about gay Republicans, the rhetoric has been dialed up in the Trump era.
Kevin Sessums, a magazine writer and author who prolifically rails
against Trump and Republicans on his popular Facebook page, has called
gay Trump supporters ``Vichy gays'' for what he describes as their
``collaboration with a fascist and deeply homophobic regime.'' Recently,
when a gay and formerly liberal power couple from New York were profiled
in The Times as Trump supporters, the reaction was fierce. ``These
people are vile, despicable gay men,'' the writer and gay activist
Michelangelo Signorile wrote on Twitter.

\href{https://www.nytimes3xbfgragh.onion/2018/11/26/nyregion/gay-trump-supporters.html}{{[}Read
how a gay and liberal couple became two of N.Y.'s biggest Trump
supporters.{]}}

Savage, Sessums and Signorile don't lack for evidence when it comes to
the Republican Party's continued L.G.B.T. problem. Though
\href{https://www.nytimes3xbfgragh.onion/2016/06/15/us/politics/trump-immigration-rally-lgbt.html}{Trump
claimed} while a candidate that he would be a ``better friend'' to
L.G.B.T. people than Hillary Clinton would, gay rights advocates insist
that he has failed to govern that way. ``The coordinated, systematic
onslaught of attacks on L.G.B.T.Q. civil rights has been unprecedented
in scale and scope,'' says Chad Griffin, the president of the Human
Rights Campaign, an L.G.B.T. civil rights organization, adding that in
Trump's first year alone, ``there were dozens of rollbacks, rescissions
and executive orders attacking basic rights and protections.''

Transgender Americans have borne the brunt of those efforts.
\href{https://mashable.com/2017/07/26/donald-trump-transgender-military-tweet/}{Trump
has tried to block transgender people from serving in the military} and
reversed several Obama-era policies that protected transgender Americans
from discrimination in workplaces, schools and prisons. But gays and
lesbians haven't escaped unscathed. In addition to more symbolic
gestures, like failing to recognize L.G.B.T. Pride Month, Trump has
taken a plethora of anti-gay actions ``to pacify the intolerant base of
his party,'' says Jimmy LaSalvia, a longtime gay conservative activist
who left the G.O.P. in 2014. On the same day as the transgender military
ban announcement, for example,
\href{https://www.nytimes3xbfgragh.onion/2017/07/27/us/politics/white-house-lgbt-rights-military-civil-rights-act.html}{the
Trump administration landed two other blows against L.G.B.T. rights}:
The Justice Department argued that the 1964 Civil Rights Act's ban on
sex discrimination doesn't protect American workers on the basis of
sexual orientation, and Trump nominated a longtime gay rights foe, Sam
Brownback, as his ambassador at large for international religious
freedom, a State Department position. (As governor of Kansas, Brownback
signed an executive order in 2015 prohibiting the state government from
penalizing religious groups that deny services to married same-sex
couples.)

Still, many of Trump's L.G.B.T. supporters dispute that Trump is bad for
gay people; at CPAC, a Log Cabin Republicans flyer boasted of ``fighting
the `fake news' about our president.'' Gay conservatives like to cite
Trump's nomination of the openly gay Richard Grenell as ambassador to
Germany as evidence that Trump has ``no personal animus toward L.G.B.T.
people,'' as Angelo put it.

Critics of the party's positions on L.G.B.T. issues have other targets
besides the Trump administration. The most obvious is the G.O.P.'s
breathtakingly anti-L.G.B.T. 2016 platform, which implicitly affirms
conversion therapy for minors, claims that allowing transgender people
to use the restroom matching their gender identity is ``dangerous'' and
argues for the superiority of heterosexual households.
\href{https://abcnews.go.com/Politics/log-cabin-republicans-gop-party-platform-anti-lgbt/story?id=40564850}{Angelo
called it} ``the most anti-L.G.B.T. platform in the party's 162-year
history.''

And yet, many L.G.B.T. conservatives --- including Angelo --- insist the
party today is no longer an inhospitable place for gay people. Some,
like Lucian Wintrich, go so far as to say that ``it's liberal propaganda
to suggest that the right today is anti-gay.'' Others are more cleareyed
about their party's shortcomings but say the platform, which is voted on
by a committee dominated by social conservatives, is, as Angelo told me,
``functionally meaningless'' and ``doesn't represent the views of the
Republicans I know.''

Angelo, who said the Log Cabin Republicans had a spike in membership and
social media followers in 2016, believes that this greater G.O.P.
openness largely explains why increasing numbers of young conservatives
are coming out of the closet and ``speaking their minds.'' But other gay
conservatives told me that Trump has simultaneously had an opposite
effect. Andrew Sullivan, arguably the most influential (and
controversial) conservative gay voice of the last three decades, told me
he knows many politically moderate gay conservatives who have decided to
``keep their heads down'' during the Trump era. ``Because they know that
during this period of the Great A\emph{woke}ning, opposing Trump is not
enough to satisfy the far left,'' said Sullivan, who still considers
himself center-right politically even though he has supported Democratic
presidential candidates since 2000. ``Anything less than completely
accepting the far left's worldview will get you attacked as racist, or
misogynistic, or ableist, or whatever slur the mob settles on.''

Considering how much criticism L.G.B.T. conservatives face from outside
their ranks, I was surprised by how often I heard them disparage one
another. The assimilationist-minded Log Cabin Republicans, the Trump
critics like Sullivan, the deliberately trollish Yiannopoulos acolytes
and the conservative-leaning college students coming of age in an era of
greater social acceptance have seemingly little in common besides their
sexual orientation --- and their oft-stated distaste for identity
politics. I routinely heard conservative gays criticize other
conservative gays as ineffective, boring or empty vessels. ``What I see
right now in the conservative L.G.B.T. community are a lot of Twitter
trolls and some social media celebrities,'' Rob Smith, the Iraq vet,
told me. ``What I don't see are a lot of movement leaders.''

\textbf{Not long before} CPAC last year, I asked Doug Hattaway, a gay
Democratic strategist who was a senior adviser to Hillary Clinton's 2008
presidential campaign, if he counted any gay conservatives as good
friends. He did not, he told me, though he recently had gone on a Tinder
date with a Trump appointee. ``It did not go well,'' he said. But
Hattaway was friends with a gay former conservative --- a 32-year-old
named Ryan Newcomb, who worked in the White House during the George W.
Bush administration and whom Hattaway describes as now being a ``raging
progressive.'' Hattaway invited Newcomb to join us that night for a
drink at a bar in Washington's Logan Circle neighborhood.

The same day, I spoke by phone with a longtime gay conservative who
served in Trump's presidential campaign. (Though he is out of the
closet, he asked me not to use his name so he could speak freely about
his personal life. I'll call him C., the first letter of his first
name.) C. was having a rough week. His liberal boyfriend of about a
year, whom C. was ``head over heels for,'' had decided, after much
consternation, that he couldn't continue seeing a Trump-supporting
Republican. Though C. was devastated, he said he'd had plenty of
practice being rejected by gay Democrats. He'd had men storm out of
first dates with him, yell at him in bars and pour drinks on his head.

I didn't expect anything that dramatic to happen when I invited C. to
join Hattaway, Newcomb and me at the bar. As we waited for C. to arrive,
Newcomb reclined in his seat with a drink and scrolled through his
cellphone contacts, amazed at how many right-leaning gays he knew. I
heard something similar from Tim Miller, a gay former communications
staff member for Jeb Bush, who told me he was surprised by how quickly a
community of mostly young, openly gay conservative men has formed in
recent years in Washington. (Conservative lesbians often have less luck
finding community. Sarah Longwell told me that she personally knew only
a handful of conservative lesbians, and that her spouse and all her
close lesbian friends are Democrats.)

Image

Colton Buckley: ``I'm one of the most conservative gay people you'll
ever meet.''Credit...Peyton Fulford for The New York Times

When C. finally arrived, it didn't take long for talk to turn to Trump.
``I still can't believe he's president,'' Newcomb said, shaking his head
in disbelief.

``Why?'' C. asked.

``Because he's not worthy of the title.''

``Well, he \emph{won,''} C. said, annoyed.

Awkward silence ensued. Before long, C. left. ``I can stomach gay
Republicans,'' Hattaway said once C. was out the door. ``But a gay Trump
supporter? They know it's indefensible, so off they go.''

But many gay Trump supporters aren't so quick to run from a fight. In
late 2017 I visited Chadwick Moore, a 35-year-old former liberal and
writer for the national gay magazine Out who is now one of the most
combative L.G.B.T. conservatives on social media and on Fox News, where
he is Tucker Carlson's go-to gay on the supposed hysterics of the gay
left. During a June segment about a Huffington Post piece calling for a
boycott of Chick-fil-A for its past donations to groups opposing
L.G.B.T. rights, Moore gleefully drank from a Chick-fil-A cup as he
mocked ``pearl-clutching lefty gays'' he deemed ``desperate for
villains'' because they have ``no one left to hate.''

Moore --- who has repeatedly defended the Proud Boys, a far-right men's
group of self-identified ``Western chauvinists'' that was banned on
Facebook and Instagram after 10 members were arrested on charges of riot
and attempted assault in New York in October --- insists that the real
threat to gay people comes from Islam. A strain of Islamophobia is
common among some gay conservatives here and abroad, including in
France's far-right National Rally party (formerly called the National
Front), which, though it opposes same-sex marriage,
\href{https://www.buzzfeednews.com/article/lesterfeder/frances-nationalist-party-is-winning-gay-support}{reportedly}
had more gay people in leadership roles in 2017 than any other major
party in the country. ``Pray for Le Pen,'' Moore tweeted in support of
National Rally candidate Marine Le Pen before last year's French
presidential election.

This current iteration of Moore would likely come as a surprise to the
old version, who voted for Hillary Clinton. Moore ``came out'' as a
conservative not long after he wrote an October 2016
\href{https://www.out.com/out-exclusives/2016/9/21/send-clown-internet-supervillain-milo-doesnt-care-you-hate-him}{Out
cover story about Yiannopoulos} that was harshly criticized as too
sympathetic by many L.G.B.T. journalists. When I met with Moore on the
patio of a bar in Williamsburg, Brooklyn, to talk about his political
metamorphosis, he had come directly from a taping of Carlson's show and
was still on an adrenaline high. Though he was friendlier and more
introspective in person than he is on social media, it was difficult to
take him seriously when he said things like ``David Duke is actually a
leftist'' and ``What's not to love about Trump? He's a drag queen. He's
a cartoon character. He's fabulous. He's a Kardashian!''

I was curious how much of Moore's persona was Yiannopoulos-inspired
performance art that he didn't actually believe but that was gaining him
more notoriety than he enjoyed as a writer for Out. I also wondered
whether Moore's schoolyard mocking of the gay left (sample tweet to
Glaad, a group focused on L.G.B.T. media coverage: ``Grow a pair,
ladies'') was retribution for being publicly rebuked by his L.G.B.T.
colleagues and eventually shunned by his longtime gay friends.

Unsurprisingly, Moore rejected both theories, insisting that as a member
of the mainstream media, I couldn't possibly understand him or portray
him positively. ``I like myself so much more and am so much happier'' as
a conservative, he said, but that's ``going to be left out of your
article, because it's too uplifting.'' He no longer supports Democrats,
he explained, because the contemporary left is dishonest, hysterical and
obsessed with policing speech. Worse yet, the left is no fun anymore.
``If you love mischief, if you love upsetting delicate people, I don't
know where else you would be right now than the gay right,'' he told me.

Though Moore and Lucian Wintrich rarely passed up an opportunity to
throw shade at each other when I spoke with them --- Moore calls
Wintrich ``the dumbest person on the internet,'' while Wintrich says
Moore is ``stealing Milo's tired act'' --- they share a belief that
their contemporary brand of conservatism is channeling a subversive,
old-school gay spirit.

``Being gay used to be about being transgressive and pushing the
culture,'' Wintrich told me in late 2017 in the Washington apartment he
lived in at the time, which was decorated with huge framed Twinks4Trump
photographs. Wintrich, who attended Bard College and could even now pass
for a brooding student at the famously liberal school, smoked a
cigarette near an open kitchen window. ``When did gay men get so
boring?''

\textbf{In April} I traveled to northwest Oklahoma to meet Colton
Buckley, a 24-year-old gay cowboy in the midst of a Republican primary
campaign for a seat in the Oklahoma House. A self-described
``God-fearing, gun-toting gay,'' Buckley hoped to represent Ellis
County, a sparsely populated area that may have more feral pigs than
Hillary Clinton supporters. Of the 1,766 county residents who voted in
the 2016 presidential election, only 155 backed Clinton.

That was good news for Buckley, one of the youngest Trump delegates to
the 2016 Republican convention and one of more than 20 Republican
L.G.B.T. candidates who competed in federal or local races in the 2018
election, according to the LGBTQ Victory Fund, a political action
committee. (Five of these candidates won.)

Buckley, who came out publicly after the deadly 2016 terrorist shooting
at the Pulse nightclub in Orlando, told me that his primary opponents
were trying to use his sexual orientation against him. ``There's a
whisper campaign going on,'' he said, as he drove around Ellis County in
his pickup truck, wearing jeans and a cowboy hat. Buckley told me he
opposes both same-sex marriage --- ``for biblical reasons,'' he said ---
and what he calls ``the homosexual agenda.'' (When I asked for sample
agenda items, Buckley said it was less of an actual list and more of a
``catchall phrase for a liberal doctrine.'') Buckley summarized his
political beliefs this way: ``I'm one of the most conservative gay
people you'll ever meet.''

Buckley lives in Arnett, a small, barren town with only one place to get
a beer --- a dive bar called the Longhorn with signage you might expect
to find at the after-party for a women's rodeo (``Cowgirl Motto: Party
'til He's Cute''). The Longhorn's chatty owner, Stacy McCartor, had also
outfitted the place with Buckley campaign signs. She didn't care that he
was gay, she said, though she worried that others would. ``If only you
were a lesbian --- guys can wrap their heads around that!'' she told
him.

I watched Buckley give a short version of his stump speech to three men
in their 30s sitting around a table drinking. Then Buckley pulled out
his phone to play a video from his campaign website, in which he shoots
an AR-15 rifle after defiantly asking, ``What part of `shall not be
infringed' do you not understand?''

The most talkative of the three men didn't know that Buckley was gay,
and finally he asked why I was there taking notes. ``This is a
journalist,'' Buckley told him, ``and he wants to know how a young man
who lives in rural Oklahoma and who is running for office as a
Republican is also a fag.'' (Buckley told me he often used derogatory
terms ``to disarm voters who would potentially shut down based on my
sexual orientation.'')

The man looked confused. ``I'm going to need a couple more beers,'' he
said finally.

After gathering his thoughts, he told Buckley that he was ``in the wrong
area to be doing this. People around here ain't gonna vote for you.'' He
said he didn't personally have a problem with gay people before
suggesting, a few minutes later, that Buckley might eventually learn to
appreciate the opposite sex. ``You don't have no interest in a woman?''
he asked.

``Nope,'' Buckley said, adding that he didn't choose to be gay. ``Why
would I live in a rural area and be a Republican and a Christian and
choose something where everybody's gonna hate me?''

Image

Jennifer Williams: ``I was a Republican long before I was
transgender.''Credit...Peyton Fulford for The New York Times

``I don't hate you,'' the man said. Before long, in fact, he almost
seemed ready to play matchmaker. ``Do you have any interest in anyone
here in town? Any fellas?''

Buckley offered him a choice. Would he prefer a candidate who is
straight but who wants to raise taxes, as Buckley suggested one of his
opponents did? Or would he prefer ``a faggot that's going to fight for
your gun rights and make sure your taxes don't get raised''? The man
didn't hesitate. ``The faggot,'' he said. Buckley turned toward me.
``See? That's why I'm going to win this race.''

Buckley turned out to be wrong about that --- he finished in third place
with just 26 percent of the vote. When I texted him after the primary to
ask if he thought he would have made the runoff had he not been openly
gay, he didn't hesitate. ``Yes,'' he wrote back. But Buckley didn't
regret coming out. ``The fact that I'm running honestly, bringing all of
myself to the table, is a testament to how things are changing in this
country for gay people,'' he told me. He suspected that had he been born
five or 10 years earlier, he would have run as a closeted candidate.
``That's what most gay conservatives did until now,'' he said. ``Or they
didn't run at all.''

\textbf{If an openly} gay cowboy running for office was a surprise to
Republicans in Ellis County, conservative transgender activists were an
equally unexpected sight at last year's CPAC. Three transgender women,
including Jennifer Williams, a 50-year-old government contractor from
Trenton, walked around the Gaylord holding an L.G.B.T.-pride flag and
small signs that read: ``Proud to be Conservative. Proud to be
Transgender. Proud to be American. \#SameTeam.''

They knew they had their work cut out for them. Williams, who is running
for the state's General Assembly, told me that while most mainstream
Republicans wouldn't dare be openly contemptuous of gays and lesbians
anymore, there's no similar reprieve for transgender people. She
described an endless barrage of antitransgender rhetoric from
conservatives, including from some gay men (both Wintrich and Moore used
the transgender slur ``tranny'' in their conversations with me) and from
prominent voices like Ben Shapiro, who has called transgenderism a
``mental disorder.'' At CPAC, Shapiro told the crowd that ``you don't
get to tell little boys they can become little girls just to avoid
offending people.''

The day after Shapiro's speech, I watched the transgender women engage
in a lengthy discussion with several young men, including Ben Holden and
another conservative gay college student. In what occasionally felt like
a debate, Williams tried to get them to understand that transgender
people face many of the same smears --- that they're mentally unstable
and a threat to children in restrooms --- that were aimed at gay men not
long ago. Their conversation was momentarily interrupted when a young
white nationalist walked between them, handing out his business card and
suggesting that his organization ``is going to be the future, because we
have stuff like this'' --- meaning transgender people --- ``we have to
deal with.'' Though jarring, the disruption offered Williams and the
students something they could agree on: White nationalism is bad.

I could think of few lonelier identities than that of transgender
conservative activist, and I wondered whether Williams considered
leaving the party after she transitioned in 2015. She had, she said, but
she decided against it partly because ``I was a Republican long before I
was transgender,'' adding that her politics --- including limited
government, a strong military and free-market policies --- still align
her more closely with Republicans.

Like many L.G.B.T. conservatives, she also held out hope that her party
might change. Jimmy LaSalvia, the longtime gay conservative who left the
party in 2014, told me that he had watched several waves of gay
conservatives have similar hopes dashed over the decades: ``I've seen so
many fight the good fight, then become disillusioned that the party
isn't changing and become independents or Democrats,'' he said. ``Then a
new group of young gay conservatives appears, and they know almost
nothing of this history, and they again insist that the party will
change.''

Williams's initial optimism in 2016 was shared by many L.G.B.T.
conservatives, who watched as candidate Trump ``made rather
unprecedented public moves for a Republican to declare himself on the
side of L.G.B.T. voters,'' recalls Patrick J. Egan, a political
scientist at N.Y.U. who researches L.G.B.T. voting behavior. Trump
hawked ``LGBTQ for Trump'' T-shirts on his campaign website, held up a
pride flag during a campaign event and presided over what Angelo, the
former Log Cabin president, called ``the most gay-friendly convention in
G.O.P. history.'' That's a low bar, to be sure, but for some Log Cabin
members who witnessed Pat Buchanan's virulently anti-gay speech at the
1992 Republican convention, Trump's willingness to say the term
``L.G.B.T.Q.'' from the stage and to offer the PayPal co-founder and
openly gay conservative Peter Thiel a prime speaking slot was ``deeply
meaningful,'' Angelo said.

But it wasn't meaningful enough to earn Williams's support --- or that
of many L.G.B.T. people. Trump received just 14 percent of the
community's vote, according to exit polling, significantly less than the
22 percent who backed Mitt Romney in 2012. One reason, Williams said,
was Trump's selection of Mike Pence, who has a long history of opposing
L.G.B.T. rights, including suggesting that same-sex marriage might cause
``societal collapse,'' as his running mate.

Still, Trump's announcement as president that he would block transgender
people from serving in the military came as a surprise to Williams. ``It
felt like somebody sucker-punched me,'' she said. But many gay
conservatives I spent time with played down the importance of Trump's
record on transgender rights. ``I think the trans issue gets more
attention than it warrants,'' says Jamie Kirchick, a center-right gay
writer and visiting fellow at the Brookings Institution who opposed
Trump's military ban but who believes ``the gay movement has been
overtaken by transgender issues affecting a minuscule percentage of the
population.'' Rob Smith, the Iraq veteran, channeled the feelings of
many gay conservatives I spoke to about transgender rights when he
tweeted: ``A `good' gay in 2018 must: Diffuse his masculinity at all
costs. Never question a trans person. Ever.''

The unwillingness of many gay conservatives to prioritize the struggle
of transgender people comes as little surprise to Richard Goldstein, a
gay former executive editor for The Village Voice who published
``Homocons,'' a scathing book about gay conservatives, 17 years ago.
Though Goldstein doesn't view them with the same scorn he once did (he
sees their ability to live openly gay lives as proof of ``the gay left's
success making it possible for every gay person to be themselves''), he
remains disappointed by what he sees as their inability to empathize
with marginalized communities. ``These are mostly white gay men who are
pretty comfortable and who can't seem to understand that many in the
L.G.B.T. community are still not safe and need protection,'' Goldstein
said.

That seeming lack of compassion also struck Alexander Chalgren, who for
a time was arguably the most famous young Trump supporter in America.
The former deputy director of Students for Trump, Chalgren, now a
21-year-old student at Cornell,
\href{https://www.thisamericanlife.org/580/thats-one-way-to-do-it}{was
featured} on a 2016 episode of ``This American Life,'' during which he
said that just because he was gay and black didn't mean he had to be a
Democrat. But by the time I met Chalgren at CPAC, he had begun to sour
on the president. He was particularly disheartened by Trump's reaction
to the 2017 white-supremacist marches and violence in Charlottesville
and by his transgender military ban, which Chalgren called needlessly
cruel. ``I don't have respect for a draft dodger who won't allow other
people to serve,'' he told me.

When I talked to Chalgren again in November, he said he had lost all
faith in Trump --- and was disgusted by the Republicans' ``complete
capitulation'' to him. ``I don't consider myself a Republican anymore,''
he told me. ``I'll be voting for Democrats in 2020.''

\textbf{Among some L.G.B.T.} conservatives there's a contention that
Chalgren's experience is rare, and that the real movement is among
people --- both straight and gay --- fleeing the Democratic Party,
though the only evidence for this is anecdotal. In the 2018 midterms, in
fact, 82 percent of L.G.B.T. voters supported their Democratic candidate
for the House of Representatives, an increase over the three previous
midterm election cycles,
\href{https://www.nbcnews.com/feature/nbc-out/record-lgbt-support-democrats-midterms-nbc-news-exit-poll-shows-n934211}{according
to NBC News exit polling}. The same polls show a decline since 2014 in
Republican Party identification among L.G.B.T. voters, though the
proportion who identify as ``conservative'' has held steady at 14
percent.

A leading proponent of the Democratic-flight theory is Brandon Straka, a
gay 41-year-old hairstylist and longtime liberal from New York who
became disillusioned with the Democratic Party and announced in a
YouTube video last May that he was walking away from it. The \#WalkAway
hashtag became a sensation on right-wing social media, and Straka
organized a \#WalkAway march and rally in Washington 10 days before the
midterm elections.

I met Straka the morning of the event at the Trump International Hotel;
he had come from an appearance on ``Fox \& Friends,'' which apparently
caught the attention of Trump, who promptly tweeted about the march.
Though it was raining, about 500 people (the crowd would later at least
quadruple, by my estimate) gathered for a premarch rally at a park. Some
came bearing signs. One read ``I never dreamed I'd grow up to be a
deplorable, but here I am killing it. \#WalkAway,'' while another read
``Not a Bot,'' a reference to reports contending the movement's
popularity was inflated by Russian social media accounts and other
agents of disinformation.

Image

Rob Smith: ``What I see right now in the conservative L.G.B.T. community
are a lot of Twitter trolls and some social media
celebrities.''Credit...Peyton Fulford for The New York Times

Many of those in attendance at the premarch rally said they were
longtime conservatives --- or ``WalkWiths,'' as they called themselves.
But I also met longtime Democrats and formerly ``closeted
conservatives,'' people like Lynzee and Michelle Domanico, a married
lesbian couple who in 2018 launched The Closet on the Right, a website
for ``people living in fear'' of being ``shunned, abandoned and
vilified'' for their conservative beliefs. As I spoke with Lynzee and
Michelle, another lesbian walked by and said: ``More lesbians for Trump.
We love Daddy!''

The most interesting conversation I had that morning was with a married
lesbian couple in their 60s who had until recently lived in San
Francisco. The quieter of the pair, Jill, seemed surprised and not
altogether comfortable that her recent political metamorphosis (from
``San Francisco liberal'' to political independent) had brought her
here, only feet from a man holding a sign critical of Planned
Parenthood. ``I'm walking away --- I'm just not sure what I'm walking
away toward,'' Jill told me. ``I'm not a Democrat anymore, but I'm not
ready to embrace Trump or to align myself with Republicans. I'm a Jew,
I'm pro-choice. The evangelical wing of the party would keep me away.''

Attending the rally had been the idea of Jill's wife, Ann, who expressed
frustration with a contemporary Democratic Party she insists has lost
its mind --- and its priorities --- in the Trump era. ``I don't hear any
coherent vision for what the Democratic leadership believes in --- most
of what I hear is constant demonizing of Trump and his supporters,'' she
said. ``I told Jill: `Let's say I had a MAGA hat on. I wouldn't, but
let's say I did. How far do you think I'd get down the street in New
York, San Francisco or Berkeley before somebody spit on me or hit me?'
That's not my Democratic Party. Old-school Democrats --- we fought for
the right of people we disagreed with to be able to speak, even when we
thought their positions were offensive and wrong.''

Among the gays and lesbians I spoke with at the rally, there was a
prevailing belief that while the L.G.B.T. community's loyalty to the
Democratic Party may have made sense in the past, it doesn't now and
won't in 2020. As many gay conservatives see it, most L.G.B.T. people
are now fully assimilated and are as secure as any other Americans.

Whether L.G.B.T. people feel secure in this country could have profound
implications on the future of the community's vote, says Patrick Egan,
the N.Y.U. political scientist. He believes that as L.G.B.T. people feel
increasingly assimilated, they could go the way of one or the other of
two traditionally Democratic constituencies: Jewish voters, who have by
and large remained loyal to the Democratic Party as they have
assimilated, or non-Hispanic Catholics, who gradually shed their
partisanship. He suspects that will depend partly on the degree to which
L.G.B.T. people continue to see themselves as outsiders.

Egan notes that marginalized groups can feel insecure even when
protected by law, as L.G.B.T. people increasingly are.
\href{https://www.washingtonpost.com/news/monkey-cage/wp/2015/06/29/will-marriage-turn-gay-people-into-republicans-not-anytime-soon/?noredirect=on\&utm_term=.ac915d72c9e0}{In
a 2015 Washington Post article}, he proposed asking ``any legally
married gay couple this question: Where do you feel comfortable holding
your spouse's hand in public?'' For most gay couples, he suggested, the
list of safe places is a short one. Until that changes, Egan suspects
L.G.B.T. voting behavior won't.

There's another factor that could curb any meaningful L.G.B.T. migration
toward the Republicans. The L.G.B.T. community, Egan says, has been
``deeply infused with the notion of coalitions with other
disenfranchised groups. There's a sense among many rank-and-file voters
that these fates are linked.''

Egan suspects that's partly why Trump got so little L.G.B.T. support.
Longwell, the Log Cabin chairwoman, agrees. ``For many L.G.B.T. people,
it didn't matter how positive candidate Trump's posture was on gay
issues,'' she says. ``It couldn't compensate for the alarming way Trump
talked about women and minority and immigrant communities,'' adding that
many L.G.B.T. people are also members of those groups. Though Longwell
can envision the day L.G.B.T. young people don't automatically vote
Democratic, she told me that will depend on what the Republican Party
looks like after Trump. ``I don't believe that the party at this moment
is compelling for many young people, gay or straight. In fact, I worry
that we're losing a generation.''

\textbf{In October,} Ben Holden sat with three other conservative
students at a table in a student center on the campus of Suffolk
University behind a banner promoting their chapter of Young Americans
for Freedom (Y.A.F.), a national conservative student activist
organization.

The four club members were white men, a stark contrast to the diverse
students at tables around them. Holden recognized the optics problem
while also lamenting that he had to think that way. What you look like
shouldn't make your argument any more or less valid, he said:
``Demographics shouldn't be destiny.''

During the hour I sat with the club members, only one person --- a
reporter for The Suffolk Voice, the school's online student publication
--- stopped to chat with them. ``You guys causing drama?'' the young
reporter said with a smile. (Y.A.F. is used to ruffling feathers. In
April the group invited Christina Hoff Sommers, a critic of contemporary
feminism, to campus.)

``Not enough, unfortunately,'' Holden told her. ``We're just here
showing people we're still alive.''

``Oh, \emph{they know,''} she said. ``Especially after your little
Twitter escapade.''

The reporter was referencing a short video Holden and Y.A.F. posted to
Twitter on Oct. 11, National Coming Out Day, during which Holden said he
was ``coming out as a conservative.'' Almost as an afterthought he
added, ``Also, I'm gay. Not that that really matters anymore.'' The
video drew some outrage on Twitter, with one young woman writing that
``this parasite is mocking a day that was created to spread awareness
about a community that is oppressed every single day.''

Despite making the Coming Out Day video, Holden played down the
relevance of his sexual orientation to his politics. Most conservatives
his age ``couldn't care less that I'm gay,'' he told me. Though he
conceded that ``the left is responsible for most of the progress on gay
issues,'' he believes that ``now it's more a generational issue than a
left vs. right one.''

Holden was keener to discuss his favorite class this semester, ``Fyodor
Dostoyevsky and the Great Philosophical Novel.'' He said that the course
was causing him to re-evaluate his motivations and those of everyone
else in the world. It was also making him even more suspicious of
conversations (political or otherwise) ``where one or both people have a
predetermined conclusion,'' he said.

He reminded me that he's only 23, and that though he leans politically
conservative at this moment in his life (and plans to return to CPAC in
late February), he doesn't want to be forever wedded to one ideology.
``But we live at a time when you're expected to pick a side, and to
stick to it without giving an inch or admitting that the opposing side
might not have malicious intent,'' he said. Nearly every time we spoke
over the past year, Holden lamented this polarization, which he said had
an impact on students on his campus, cable-news commentators and
seemingly everyone else. Holden had hoped to put together a ``smart and
nuanced'' panel at Suffolk about immigration policy, for example, but he
wasn't sure such a thing was possible.

``Trying to engage people in a thoughtful debate about ideas during the
Donald Trump era seems like something very few people want to do,'' he
said. ``I spend a lot more time thinking about how to exist during this
time of political lunacy than I do about being a gay conservative.''

Advertisement

\protect\hyperlink{after-bottom}{Continue reading the main story}

\hypertarget{site-index}{%
\subsection{Site Index}\label{site-index}}

\hypertarget{site-information-navigation}{%
\subsection{Site Information
Navigation}\label{site-information-navigation}}

\begin{itemize}
\tightlist
\item
  \href{https://help.nytimes3xbfgragh.onion/hc/en-us/articles/115014792127-Copyright-notice}{©~2020~The
  New York Times Company}
\end{itemize}

\begin{itemize}
\tightlist
\item
  \href{https://www.nytco.com/}{NYTCo}
\item
  \href{https://help.nytimes3xbfgragh.onion/hc/en-us/articles/115015385887-Contact-Us}{Contact
  Us}
\item
  \href{https://www.nytco.com/careers/}{Work with us}
\item
  \href{https://nytmediakit.com/}{Advertise}
\item
  \href{http://www.tbrandstudio.com/}{T Brand Studio}
\item
  \href{https://www.nytimes3xbfgragh.onion/privacy/cookie-policy\#how-do-i-manage-trackers}{Your
  Ad Choices}
\item
  \href{https://www.nytimes3xbfgragh.onion/privacy}{Privacy}
\item
  \href{https://help.nytimes3xbfgragh.onion/hc/en-us/articles/115014893428-Terms-of-service}{Terms
  of Service}
\item
  \href{https://help.nytimes3xbfgragh.onion/hc/en-us/articles/115014893968-Terms-of-sale}{Terms
  of Sale}
\item
  \href{https://spiderbites.nytimes3xbfgragh.onion}{Site Map}
\item
  \href{https://help.nytimes3xbfgragh.onion/hc/en-us}{Help}
\item
  \href{https://www.nytimes3xbfgragh.onion/subscription?campaignId=37WXW}{Subscriptions}
\end{itemize}
