Sections

SEARCH

\protect\hyperlink{site-content}{Skip to
content}\protect\hyperlink{site-index}{Skip to site index}

\href{https://myaccount.nytimes3xbfgragh.onion/auth/login?response_type=cookie\&client_id=vi}{}

\href{https://www.nytimes3xbfgragh.onion/section/todayspaper}{Today's
Paper}

All the President's Memes

\url{https://nyti.ms/2RSpwdK}

\begin{itemize}
\item
\item
\item
\item
\item
\item
\end{itemize}

Advertisement

\protect\hyperlink{after-top}{Continue reading the main story}

Supported by

\protect\hyperlink{after-sponsor}{Continue reading the main story}

\href{/column/first-words}{First Words}

\hypertarget{all-the-presidents-memes}{%
\section{All the President's Memes}\label{all-the-presidents-memes}}

\includegraphics{https://static01.graylady3jvrrxbe.onion/images/2019/01/20/magazine/20mag-FirstWords-1/20mag-FirstWords-1-articleLarge.png?quality=75\&auto=webp\&disable=upscale}

By Willy Staley

\begin{itemize}
\item
  Jan. 14, 2019
\item
  \begin{itemize}
  \item
  \item
  \item
  \item
  \item
  \item
  \end{itemize}
\end{itemize}

On the 12th day of the federal government shutdown, the 45th president
of the United States of America posted a meme on his Instagram account:
an image of his half-glowering, half-smirking visage, hovering
gigantically above the Southwestern desert, dwarfing the picture's
centerpiece --- a rendering of his signature campaign promise --- and,
in a familiar font, some explanatory text: ``The Wall Is Coming.'' It's
an image that makes you think, That's from HBO's hit series ``Game of
Thrones'' --- \emph{sort of}, and then makes you think about the unique
privileges and burdens of living in this moment in history.

There are so many unusual aspects of Donald Trump's presidency that his
willingness to communicate with the public through internet memes is
often overshadowed. Typically, he retweets images made by his most
enthusiastic backers --- in November he shared one of the Clintons,
Barack Obama, Huma Abedin, Robert Mueller, his own deputy attorney
general Rod Rosenstein and others, all locked up in a prison cell
together --- but the wall meme appears to be a White House original. It
is also the second ``Game of Thrones'' meme the president has shared in
the last two months. He does this sort of thing so often that the Senate
minority leader, Chuck Schumer, recently felt compelled to tweet, like
an exasperated high school teacher, ``Enough with the memes.''

It's impossible to overstate how peculiar it is that the most powerful
man in the world, who will turn 73 in June, posts memes. It's a behavior
more often associated with youth, irreverence and a surfeit of free time
--- though certainly plenty of old, aggrieved people have picked up the
habit in recent years. In 2016, the Trump campaign united message-board
trolls and Facebook boomers, and together they disseminated so many
memes that some of them began to believe --- both jokingly and not ---
that their ``meme magic'' had helped Trump win the election.

In current usage, ``meme'' refers most often to an image with text
overlay, designed for distribution online. They're like the bumper
stickers of the digital realm, in that any one concept can be endlessly
remixed to convey just about any sort of sentiment (Calvin can pee on
anything). What began more than a decade ago as a fun way to imagine how
cats might talk has evolved into a surprisingly fertile mode of
political communication. The online database Know Your Meme has
confirmed the existence of some 4,066 successful memes in the wild ---
including newcomers like Big Chungus, a fake series of video games
starring an obese rabbit, and classics like Doge, which featured a shiba
inu speaking broken English. The actual number is certainly much higher.

You might find this very silly, and you wouldn't be wrong. But keep in
mind: The \emph{president} posts them.

\textbf{All of this} represents a long fall from the meme's origins. The
word was coined by the British ethologist Richard Dawkins in his 1976
book, ``The Selfish Gene,'' as a way to conceptualize the transmission
of culture in biological terms. For Dawkins, a meme, shortened from the
Greek mimeme --- ``an imitated thing'' --- was a unit of culture, a
building block of our mental architecture. ``Just as genes propagate
themselves in the gene pool by leaping from body to body via sperms or
eggs,'' Dawkins wrote, ``memes propagate themselves in the meme pool by
leaping from brain to brain.'' They could be as small and short-lived as
a tune or catchphrase, he explained, or as large and consequential as
``God'' and ``eternal damnation.''

Genes, Dawkins argued, do not aim to propagate a species; they seek only
to propagate themselves. Memes, he believed, were similarly selfish.
Dawkins spends most of his chapter on memetics fixating on religion, and
because he has since become such an aggressively outspoken atheist,
reading the book today it is easy to get the sense that he considers
religion almost parasitic. The concept of ``God,'' he figured, endures
because it offers a psychological salve to people, while ``eternal
damnation'' survives because it is useful as a means of social control.
Thus both live on, copied from generation to generation, latching on to
humanity and perpetuating themselves too effectively to be got rid of.
``Selection favors memes that exploit their cultural environment to
their own advantage,'' Dawkins wrote.

Ideas, in this view, have lives of their own, and the environment in
which they struggle for survival is the human mind --- our limited
processing power means that only the toughest will persevere. ``If a
meme is to dominate the attention of a human brain,'' Dawkins wrote,
``it must do so at the expense of `rival' memes.'' He conceded that a
meme would also have to compete for airtime on the radio or TV,
billboard space, column inches and book pages. But, writing back in the
1970s, he had no reason to consider what would happen if those scarcity
conditions vanished. If they did, you would find yourself in a
terrifyingly fecund primordial soup in which all sorts of ideas could
develop, mutate, cross-pollinate, do battle, die off and be reborn. You
would find yourself, well, online.

And while this primordial soup has brought forth many novel concepts,
and resuscitated some old, corrosive beliefs, the things we call
``memes'' today are largely just joke formats --- mechanisms for the
efficient production of humor. They develop less like new ideas and more
like algal blooms, spreading until they block out the sun and consume
all the oxygen, before dying out naturally (people get sick of them) or
getting hit with bleach (explainer journalists write about them).
Individually, these memes leave little mark on our culture. Worse than
being forgettable, they become, within a year or two, embarrassing to
think back on for even one second.

But taken as a whole, this swarm of cultural mayflies represents a
meaningful shift in our culture. Joke-making, a sometimes cruel
enterprise, has been mechanized and democratized. Humor now emerges from
the ether, authorless or, more accurate, authored and improved upon by
everyone. Jokes are communal now, and \emph{constant}. Online,
everything that happens all day --- in politics, in culture, in the news
--- is rapidly repurposed for laughs, by everyone, all at once.

For the most part, this is harmless. After all, what could possibly go
wrong in a culture where all anyone wants is to be perpetually amused?

\textbf{Before Trump's border} wall was the cause of a government
shutdown, it was a mnemonic device --- less a policy proposal than a
string tied around the finger. According to a recent article in The
Times, the wall was a ``memory trick for an undisciplined candidate.''
Trump's advisers Sam Nunberg and Roger Stone knew that getting tough on
immigration would play well to a right-wing audience, but they also knew
the man they were dealing with. He has a mind for the tactile, so they
gave him something gigantic to hang onto: an 1,800-mile-long slab of
concrete.

But Trump's talking points were never just talking points. They were
more like bits. His campaign rallies were rambling, unscripted affairs,
almost like an open-mic comedy set: Not a fearsome Nuremberg rally, but
an aging showman road-testing material, seeing what caught the
audience's attention. Early on, his speeches were ``all over the
place,'' the NBC reporter Katy Tur told ``Frontline,'' but as time
passed, ``he started to really hone his message, and he started to
remember what lines worked.'' In the same episode, the writer Marc
Fisher said Trump told him that he would simply wait to see the red
lights on the TV cameras in the press box turn on, indicating he was
live, and then he would say ``whatever it took to keep the red light
on.''

The border wall kept the lights on. At a 2016 rally in Burlington, Vt.,
Trump mentioned the wall to tremendous, wonderful applause, then paused
and asked his audience, ``And who's gonna pay for the wall?'' The crowd
roared back, ``MEXICO!'' They --- he and his crowd --- did this two more
times together, then Trump laughed. ``I've never done it before, I
swear,'' he said, throwing his arms up as if surprised it had worked.
``That was pretty cool. We're gonna have to use that.''

This incentive structure, in which an easily distracted person says a
bunch of stuff he kind of means to an assembled audience, slowly
learning what generates a reaction and what doesn't, is familiar: It's
like posting online. This is the process that nudged the wall ever
closer to reality, despite the fact that it was only ever supposed to be
a metaphor, a shorthand, a catchphrase. It is an idea with no real owner
or creator, passed from person to person, from lectern to grandstand to
TV and Twitter and back again, copying itself and growing and mutating
until it became big, beautiful and tipped with spikes forged from
American steel. The border wall is, in the truest sense, a meme: an idea
that persists not because it will benefit us but simply because it
thrives in our environment. It was so effective at doing whatever it did
that it couldn't be contained, spilling out of the president's brain and
spreading throughout our entire body politic, cooling and hardening like
bacon grease, until it finally brought everything to a standstill. And I
hate to admit it, but that is a little funny.

Advertisement

\protect\hyperlink{after-bottom}{Continue reading the main story}

\hypertarget{site-index}{%
\subsection{Site Index}\label{site-index}}

\hypertarget{site-information-navigation}{%
\subsection{Site Information
Navigation}\label{site-information-navigation}}

\begin{itemize}
\tightlist
\item
  \href{https://help.nytimes3xbfgragh.onion/hc/en-us/articles/115014792127-Copyright-notice}{©~2020~The
  New York Times Company}
\end{itemize}

\begin{itemize}
\tightlist
\item
  \href{https://www.nytco.com/}{NYTCo}
\item
  \href{https://help.nytimes3xbfgragh.onion/hc/en-us/articles/115015385887-Contact-Us}{Contact
  Us}
\item
  \href{https://www.nytco.com/careers/}{Work with us}
\item
  \href{https://nytmediakit.com/}{Advertise}
\item
  \href{http://www.tbrandstudio.com/}{T Brand Studio}
\item
  \href{https://www.nytimes3xbfgragh.onion/privacy/cookie-policy\#how-do-i-manage-trackers}{Your
  Ad Choices}
\item
  \href{https://www.nytimes3xbfgragh.onion/privacy}{Privacy}
\item
  \href{https://help.nytimes3xbfgragh.onion/hc/en-us/articles/115014893428-Terms-of-service}{Terms
  of Service}
\item
  \href{https://help.nytimes3xbfgragh.onion/hc/en-us/articles/115014893968-Terms-of-sale}{Terms
  of Sale}
\item
  \href{https://spiderbites.nytimes3xbfgragh.onion}{Site Map}
\item
  \href{https://help.nytimes3xbfgragh.onion/hc/en-us}{Help}
\item
  \href{https://www.nytimes3xbfgragh.onion/subscription?campaignId=37WXW}{Subscriptions}
\end{itemize}
