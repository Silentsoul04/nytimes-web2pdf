Sections

SEARCH

\protect\hyperlink{site-content}{Skip to
content}\protect\hyperlink{site-index}{Skip to site index}

\href{https://www.nytimes3xbfgragh.onion/section/us}{U.S.}

\href{https://myaccount.nytimes3xbfgragh.onion/auth/login?response_type=cookie\&client_id=vi}{}

\href{https://www.nytimes3xbfgragh.onion/section/todayspaper}{Today's
Paper}

\href{/section/us}{U.S.}\textbar{}Los Angeles Teachers Strike,
Disrupting Classes for 500,000 Students

\url{https://nyti.ms/2RQDpc0}

\begin{itemize}
\item
\item
\item
\item
\item
\item
\end{itemize}

Advertisement

\protect\hyperlink{after-top}{Continue reading the main story}

Supported by

\protect\hyperlink{after-sponsor}{Continue reading the main story}

\hypertarget{los-angeles-teachers-strike-disrupting-classes-for-500000-students}{%
\section{Los Angeles Teachers Strike, Disrupting Classes for 500,000
Students}\label{los-angeles-teachers-strike-disrupting-classes-for-500000-students}}

\includegraphics{https://static01.graylady3jvrrxbe.onion/images/2019/01/16/us/15strike-print/15strike-altsub-videoSixteenByNine3000.jpg}

By \href{https://www.nytimes3xbfgragh.onion/by/jennifer-medina}{Jennifer
Medina}, \href{https://www.nytimes3xbfgragh.onion/by/tim-arango}{Tim
Arango},
\href{https://www.nytimes3xbfgragh.onion/by/dana-goldstein}{Dana
Goldstein} and Louis Keene

\begin{itemize}
\item
  Jan. 14, 2019
\item
  \begin{itemize}
  \item
  \item
  \item
  \item
  \item
  \item
  \end{itemize}
\end{itemize}

LOS ANGELES --- More than 30,000 Los Angeles public-school teachers
began a strike on Monday, the first in three decades in the district.
Holding plastic-covered signs on rain-drenched picket lines across the
city, they demanded higher pay, smaller classes and more support staff
in schools.

The strike effectively shut down learning for roughly 500,000 students
at 900 schools in the district, the second-largest public school system
in the nation. The schools remained open, staffed by substitutes hired
by the city, but many parents chose to keep their children at home,
either out of support for the strike or because they did not want them
inside schools with a skeletal staff.

With negotiations apparently at a standstill, the strike could last days
or even weeks.

The decision to walk off the job came after months of negotiations
between the teachers' union, United Teachers Los Angeles, and the Los
Angeles Unified School District. Although educators on all sides agree
California should spend more money on education, the union and the
district are locked in a bitter feud about how Los Angeles should use
the money it already gets.

Although district officials have agreed to come closer to meeting some
of the union's demands, they say fulfilling all of them would bankrupt
the system, which is already strained by rising health care and pension
costs.

Several Democratic politicians voiced their support for the strike
Monday, including
\href{https://twitter.com/KamalaHarris/status/1084862005767557120}{Senator
Kamala Harris} of California, who is considering a run for president.
But Gov. Gavin Newsom, who took office just a week ago, was more
measured.

``This impasse is disrupting the lives of too many kids and their
families,'' he said in a statement. ``I strongly urge all parties to go
back to the negotiating table and find an immediate path forward that
puts kids back into classrooms and provides parents certainty.''

The sprawling district goes far beyond the Los Angeles city limits,
stretching some 720 square miles from wealthy coastal areas like Pacific
Palisades to working-class southeast suburbs like Montebello. It is
overwhelmingly low income; more than 80 percent of students qualify for
free or reduced-price lunch. And Latinos make up roughly 75 percent of
all students, while whites and African-Americans each account for less
than 10 percent of enrollment.

On Monday, we went to schools across the district to talk with teachers
on the picket lines, as well as the students and parents who joined
them.

\hypertarget{inside-schools-ipads-and-crowded-auditoriums}{%
\subsection{Inside Schools, iPads and Crowded
Auditoriums}\label{inside-schools-ipads-and-crowded-auditoriums}}

At Robert F. Kennedy Community Schools, a large campus in Koreatown,
hundreds of teachers and supporters gathered before sunrise, holding
signs declaring ``We demand respect'' and ``Striking for our students.''
By 8 a.m., dozens of students and parents had gathered along the
sidewalk to support the teachers, but there was also a steady stream of
students entering the campus, which houses several small schools.

Sophie Chiang, a 10th-grade student, arrived well before the first bell.
``Oh God, it's really happening,'' she said as she approached the line
of teachers in red ponchos shouting ``Whose schools? Our schools!'' The
campus security guard who is usually at the entrance was not at his
post.

At nearby John Burroughs Middle School, the classrooms mostly sat empty
Monday. Roughly 40 percent of the school's 1,700 students showed up in
the morning and were sent to the gymnasium, auditorium and multipurpose
rooms to work independently on their school-provided iPads. A school
administrator and substitute teacher were stationed in each large space,
trying to keep order for hundreds of children.

\includegraphics{https://static01.graylady3jvrrxbe.onion/images/2019/01/15/us/15strike-03/merlin_149177055_e90af58f-5763-4692-9cb2-5fcb90743e39-articleLarge.jpg?quality=75\&auto=webp\&disable=upscale}

``We're trying to make the best learning environment with what we
have,'' said Steve Martinez, the school's principal. ``It's not an ideal
situation.''

Mr. Martinez had asked the district for 65 substitute teachers, but
received only six. By lunchtime, the rooms were rowdy but under control
--- though several iPads and phones clearly displayed games that would
not be considered educational.

Still, Matias Garcia, a sixth-grade student, texted his mother asking
her to come get him by 11 a.m.

``Nobody is listening, it's loud and we spent like an hour taking
attendance,'' Matias said.

His mother, Lilly Santaniello, works part-time as a lawyer and hired a
relative to come take care of her other sons. Matias said he had a plan
for the afternoon: video games and sleep. \emph{--- JENNIFER MEDINA}

\hypertarget{priced-out-of-the-neighborhood}{%
\subsection{Priced Out of the
Neighborhood}\label{priced-out-of-the-neighborhood}}

In a downpour, dozens of teachers, wearing ponchos and waving green and
red placards --- ``We stand with L.A. teachers'' --- picketed outside
the Paul Revere Charter Middle School in the Pacific Palisades
neighborhood. One teacher waved a large American flag, others held
bullhorns.

Steven Bilek, a union representative and math and science teacher at
Paul Revere, said that for some teachers the main issue was pay, but
others were just as concerned with class size or the lack of support
staff like librarians or counselors.

``The idea that teachers just want raises is just not true,'' he said.

Paul Revere is a charter school affiliated with the district, and
students attend from more than a hundred ZIP codes around Los Angeles,
he said. The school is in one of the city's wealthiest areas, and most
if not all of the teachers cannot afford to live nearby.

``I don't think any of our teachers live in this area,'' Mr. Bilek said.
``We're lucky to live within an hour.''

Some parents and students joined the teachers on the picket line. ``We
are here fighting for the future of public education,'' said Dennise
Weir, whose children, Olivia and Edward, were by her side.

Ms. Weir lives in Beverly Hills and can afford private school, but she
sends her children to public school because, she said, ``they are the
last civic institutions that are available to everyone.''

There are about 2,100 students enrolled at Paul Revere, but on Monday
about a third showed up for class.

Each grade was gathered in large assemblies, overseen by the few adults
on duty --- substitute teachers, campus aides and technicians. On a
normal day the school has 95 teachers, but on Monday there were only 10
adults on hand with teaching credentials. --- \emph{TIM ARANGO}

Image

A picket line outside Robert F. Kennedy Community Schools, a large
campus in Koreatown.Credit...Jenna Schoenefeld for The New York Times

\hypertarget{how-does-teaching-in-la-stack-up}{%
\subsection{How Does Teaching in L.A. Stack
Up?}\label{how-does-teaching-in-la-stack-up}}

Union leaders say teaching in the city is unsustainable, with salaries
far outpaced by a high cost of living, large class sizes and not enough
resources to help struggling students. Here's a look at how Los Angeles
Unified compares nationally:

\textbf{Teacher pay:} The average teacher salary in Los Angeles Unified
was \$75,000 during the 2017-2018 school year, according to the
\href{https://www.cde.ca.gov/ds/fd/cs/documents/j90summary1718.pdf}{California
Department of Education} --- far higher than the national average of
\href{https://nces.ed.gov/programs/digest/d17/tables/dt17_211.60.asp?current=yes}{around
\$59,000}.

But Los Angeles is an expensive place to live. When you compare the city
with other high-priced urban centers, its teacher pay no longer looks so
extraordinary. The average salary in San Francisco last year was
\$73,000, while New York City teachers earned an average salary of
\$88,000, according to their union.

Beginning salaries matter, too, for attracting and retaining young
talent. In Los Angeles, the entry-level teacher salary was \$44,000 last
year, compared with \$47,000 in San Francisco and
\href{https://www.schools.nyc.gov/careers/working-at-the-doe/benefits-and-pay}{\$57,000}
in New York City.

Los Angeles Unified has, so far, offered teachers a 6 percent raise.

\textbf{Class size:} An independent
\href{https://www.lataco.com/wp-content/uploads/utla-lausd-fact-finding-report.pdf}{report}
noted that the two sides can't agree on how to calculate class size, but
it is clear that classes in Los Angeles are big. The district has
offered to cap classes at 35 students in grades 4-6; 39 students in
middle- and high-school English and math; and 32 students at elementary
schools that serve many low-income children. Nationally, average class
sizes in urban schools ranged between 16 and 28 students, depending on
grade level and how the school was organized, according to the
\href{https://nces.ed.gov/surveys/ntps/tables/ntps_7t_051617.asp}{National
Teacher and Principal Survey} for 2015-2016.

\textbf{Professional staff beyond the classroom:} One of the union's
main demands is for the district to hire more guidance counselors,
nurses and librarians. With increased pressure over the last two decades
to raise standardized test scores, many public schools have funneled
funds into math and reading instruction, and suffer from a dearth of
such professional staff.

California's situation is worse than the national average, with more
than 600 students per counselor across the state and more than 500 per
counselor in Los Angeles County, according to an
\href{https://www.kidsdata.org/topic/126/pupilsupportpersonnel-type/table\#fmt=2391\&loc=2,364\&tf=95\&ch=276,278,280,277,279,807,1136\&sortColumnId=0\&sortType=asc}{analysis}
from the Lucile Packard Foundation for Children's Health. There are
nearly 2,000 students per every school nurse in the county.

Los Angeles Unified has offered to add an additional academic counselor
to each district high school and to make sure each elementary school has
daily nursing services. The district also offered to ensure library
services at each middle school. --- \emph{DANA GOLDSTEIN}

\hypertarget{making-noise}{%
\subsection{Making Noise}\label{making-noise}}

As the final bell sounded at Gerald A. Lawson Academy of the Arts,
Mathematics and Science in South Los Angeles on Monday afternoon,
striking teachers formed a circle outside the exit and chanted
call-and-response songs in the rain. A handful of students lingered with
their parents, singing along.

Susie Chavez ducked under a teacher's umbrella, belting out ``We are the
students! The mighty, mighty students'' with her arms around her
children.

Ms. Chavez said she would keep bringing them to school no matter how
long the strike went on because ``the teachers are on strike, not my
kids.''

In the car pool line, students in their raincoats packed snugly into
Cristina Aguilar's S.U.V. While she said she supported the strike, Ms.
Aguilar wanted to avoid creating a truancy record for her younger
children, in kindergarten and third grade. Her eldest had stayed home.

She was taking a wait-and-see approach to school attendance during the
strike. ``If there's no teachers, there's no point in bringing them,''
Ms. Aguilar said. --- LOUIS KEENE

Advertisement

\protect\hyperlink{after-bottom}{Continue reading the main story}

\hypertarget{site-index}{%
\subsection{Site Index}\label{site-index}}

\hypertarget{site-information-navigation}{%
\subsection{Site Information
Navigation}\label{site-information-navigation}}

\begin{itemize}
\tightlist
\item
  \href{https://help.nytimes3xbfgragh.onion/hc/en-us/articles/115014792127-Copyright-notice}{©~2020~The
  New York Times Company}
\end{itemize}

\begin{itemize}
\tightlist
\item
  \href{https://www.nytco.com/}{NYTCo}
\item
  \href{https://help.nytimes3xbfgragh.onion/hc/en-us/articles/115015385887-Contact-Us}{Contact
  Us}
\item
  \href{https://www.nytco.com/careers/}{Work with us}
\item
  \href{https://nytmediakit.com/}{Advertise}
\item
  \href{http://www.tbrandstudio.com/}{T Brand Studio}
\item
  \href{https://www.nytimes3xbfgragh.onion/privacy/cookie-policy\#how-do-i-manage-trackers}{Your
  Ad Choices}
\item
  \href{https://www.nytimes3xbfgragh.onion/privacy}{Privacy}
\item
  \href{https://help.nytimes3xbfgragh.onion/hc/en-us/articles/115014893428-Terms-of-service}{Terms
  of Service}
\item
  \href{https://help.nytimes3xbfgragh.onion/hc/en-us/articles/115014893968-Terms-of-sale}{Terms
  of Sale}
\item
  \href{https://spiderbites.nytimes3xbfgragh.onion}{Site Map}
\item
  \href{https://help.nytimes3xbfgragh.onion/hc/en-us}{Help}
\item
  \href{https://www.nytimes3xbfgragh.onion/subscription?campaignId=37WXW}{Subscriptions}
\end{itemize}
