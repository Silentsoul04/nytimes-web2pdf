Is Ancient DNA Research Revealing New Truths --- or Falling Into Old
Traps?

\url{https://nyti.ms/2RUI4tM}

\begin{itemize}
\item
\item
\item
\item
\item
\item
\end{itemize}

\includegraphics{https://static01.graylady3jvrrxbe.onion/images/2019/01/20/magazine/20mag-dna-slide-GV9W/20mag-dna-slide-GV9W-articleLarge.jpg?quality=75\&auto=webp\&disable=upscale}

Sections

\protect\hyperlink{site-content}{Skip to
content}\protect\hyperlink{site-index}{Skip to site index}

\hypertarget{is-ancient-dna-research-revealing-new-truths--or-falling-into-old-traps}{%
\section{Is Ancient DNA Research Revealing New Truths --- or Falling
Into Old
Traps?}\label{is-ancient-dna-research-revealing-new-truths--or-falling-into-old-traps}}

Geneticists have begun using old bones to make sweeping claims about the
distant past. But their revisions to the human story are making some
scholars of prehistory uneasy.

A skull found at a prehistoric burial site near Teouma Bay, on the
island nation of Vanuatu.Credit...David Maurice Smith for The New York
Times

Supported by

\protect\hyperlink{after-sponsor}{Continue reading the main story}

By Gideon Lewis-Kraus

\begin{itemize}
\item
  Jan. 17, 2019
\item
  \begin{itemize}
  \item
  \item
  \item
  \item
  \item
  \item
  \end{itemize}
\end{itemize}

\textbf{PART I}

\hypertarget{1-the-ghosts-of-teouma}{%
\subsection{\texorpdfstring{\textbf{1. The Ghosts of
Teouma}}{1. The Ghosts of Teouma}}\label{1-the-ghosts-of-teouma}}

A faint aura of destiny seems to hover over Teouma Bay. It's not so much
the landscape, with its ravishing if boilerplate tropical splendor ---
banana and mango trees, coconut and pandanus palms, bougainvillea, the
apprehensive trill of the gray-eared honeyeater --- as it is the shape
of the harbor itself, which betrays, in the midst of such organic
profusion, an aspect of the unnatural. The bay, on the island of Efate
in the South Pacific nation Vanuatu, is long, symmetrical and briskly
rectangular. In the expected place of wavelets is a blue so calm and
unbroken that the sea doesn't so much crash on the land as neatly abut
it. From above, it looks as though a safe harbor had been engraved in
the shoreline by some celestial engineer.

In late 2003, while clearing land just above the seaside, a bulldozer
driver found a broken piece of pottery in the rubble. The villagers of
Vanuatu often happen upon shards of timeworn ceramic, which spark an
idly mythical curiosity; they're said to be fragments of Noah's Ark, or
the original Ten Commandments, or the burst water vessels of powerful
ancestral spirits. These shards are often left alone, but word in this
particular case traveled quickly, and the artifact soon found its way to
the Vanuatu Cultural Center and National Museum, where Stuart Bedford, a
New Zealand archaeologist who had studied local pot shards for years,
was called in to inspect it. He immediately recognized its distinctive
pattern --- ``dentate stamping,'' an ancient technique so named because
it looked as though some tiny-toothed creature had bitten an intricate
pattern into the ceramic --- and understood that this pottery coincided
with the very first movement of ancient peoples into the South Seas.

Bedford rushed to the site of the discovery, an old colonial coconut
plantation that the bulldozer had been clearing for use as a prawn farm.
Further burrowing turned up not only more pottery but also tools of
obsidian and a great cache of human bones, which had lain undisturbed
and unusually well preserved over thousands of years. The site was soon
identified as the oldest and largest prehistoric cemetery ever found in
the Pacific. Everything at the site indicated a founding colony ---
first arrivals to the shores of uninhabited islands. Teouma was,
according to Bedford, ``unlike anything anyone had ever seen, or was
likely to see, in this part of the world ever again.''

Archaeologists hoped the bones might help provide a clue to the abiding
mystery of how anybody had gotten to these far-off coastlines in the
first place. Vanuatu is a volcanic archipelago of more than 80 islands
littered in an extended slingshot shape across an 800-mile arc of the
South Pacific. Europeans first heard of its existence in 1606, when a
Portuguese navigator stopped through on a brief but violent imperial
errand for the Spanish crown. The islands were largely left to their own
devices until the end of the 18th century, when French and British ships
arrived to plant their own flags. The two countries ruled the
archipelago as a joint colony, called the Condominium of the New
Hebrides, until independence was achieved in 1980. National coherence
remains a work in progress. By some measures, Vanuatu is per capita the
most linguistically diverse country on the planet: Its quarter-million
citizens, predominantly the native ni-Vanuatu, speak as many as 140
different indigenous languages and maintain an astonishing variety of
cultural practices. A meaningful national identity has been constructed
from a common appreciation of ceremonial pig-tusk bracelets and the
taking of kava, a very mild narcotic root that looks like primordial pea
soup and tastes like a fine astringent dirt. Above all, however, the
ni-Vanuatu are bound together by the fact of the country's nautical
isolation: Their nearest neighbors are hundreds of miles in any
direction.

It is the peculiar geography of this isolation that made the Teouma site
so significant. Many of the islands of the South Pacific are much
farther-flung: Easter Island makes Vanuatu look like an Australian
exurb. But with one very small exception --- the tiny eastern outliers
of the Solomon Islands --- Vanuatu offers the first solid ground on the
far side of a major but invisible maritime boundary. On the west side of
that border is a string of archipelagoes called Near Oceania: islands
chained to one another (and to the rest of the world) by lines of sight.
Prehistoric peoples, after tens of thousands of years of travel by foot
from Africa, had arrived at the end of Southeast Asia and hopscotched
their way forward via short sea outings, presumably crossing the narrow
channels they encountered on crude watercraft. Finally, however, some
40,000 years ago, their path was decisively blocked by open ocean. In
front of them, across more than 200 miles of empty sea, was the vast
aquatic wilderness of Remote Oceania.

\href{https://www.nytimes3xbfgragh.onion/2019/01/17/magazine/ancient-dna-research.html}{{[}5
takeaways from this report on ancient DNA research.{]}}

That border marked the absolute limit of human expansion for tens of
thousands of years, until at last someone sailed out across the naval
event horizon and into the unknown. This first traversal was one of the
greatest and most courageous passages in human prehistory. The peopling
of Remote Oceania --- an obscure exodus that easily ranks among the
signal triumphs of the ancient world --- has inspired awe and vexation
for generations. In the mid-20th century, archaeologists came to
identify these first voyagers with a set of jars and tools unique to the
region, the ``Lapita cultural complex,'' and determined that they
crossed the boundary into Remote Oceania some 3,000 years ago. Further
details were presumed lost to history.

But in 2014, Bedford got another surprise call, this time from a
researcher affiliated with a genetics team at Harvard. A small group of
pioneering lab scientists had found ways to isolate and analyze DNA from
ancient bones, methods potent enough to inspire a wholesale revision of
our knowledge about ancient peoples. The Harvard operation, which was
then preparing a landmark paper about European origins, now intended to
visit their attention upon the South Pacific, and they wanted to know
whether Bedford might facilitate access to the Teouma remains. Bedford
agreed, and over the next four years, the Harvard team used the DNA they
found to present a radical new story about Remote Oceania's first
settlers.

Bedford and I met last summer in the hilly and sedate capital of
Port-Vila, outside the towering thatched A-frame of the national museum.
He is tall and friendly, with a square head, short brown hair, a
rancher's open gait and the incessant squint of someone in perpetual
communion with the near-hopeless complication of human affairs. We
climbed into his white Land Cruiser and drove to a tidy village compound
outside town. There, Bedford embraced the local chief, Silas Alben, who
led us through village gardens of banana and tuber to a high limestone
cliff with a sprawling view of the Teouma site.

\includegraphics{https://static01.graylady3jvrrxbe.onion/images/2019/01/20/magazine/20mag-dna-slide-02B9/20mag-dna-slide-02B9-articleLarge.png?quality=75\&auto=webp\&disable=upscale}

As we shared the sweating neon flesh of a machete-split papaya, Bedford,
now affiliated with the Australian National University, ran through all
the reasons that the sheltered cove far below --- just then rippling
beneath a late-afternoon rainbow --- would have made an inviting stage
for the encounter of an ancient people with a primeval place. For
whoever arrived in those first canoes, these empty islands offered a
bounty of unfished reefs, unoccupied land and naïve, slow-moving animal
prey; for those who now studied those first colonists, their arrival
represented an important inflection point in human expansion and
development. And now, the science of ``paleogenomics'' had coaxed new
stories of ancient lives from the Teouma bones.

\hypertarget{2-prehistory-a-history}{%
\subsection{\texorpdfstring{\textbf{2. Prehistory: A
History}}{2. Prehistory: A History}}\label{2-prehistory-a-history}}

For most of human history, our beliefs about our origins drew upon oral
traditions or the evidence found in ancient texts. One 17th-century
scholar calculated, on the basis of biblical genealogies, that the
creation happened in 4004 B.C.; subsequent refinements settled on the
date of Oct. 23. Sir Isaac Newton criticized the ancient Egyptians for
the ``vanity'' of their own calendrical reckoning, which placed the
beginning of their monarchy before the existence of the world. As the
pre-eminent British archaeologist Colin Renfrew once put it, ``For an
educated man in the 17th or even the 18th century, any suggestion that
the human past extended back further than 6,000 years was a vain and
foolish speculation.''

It wasn't long before a series of scientific interventions pried open
human prehistory to methodical study. Two great advances of 1859 helped
cement the view that 4004 B.C. was not, in fact, the starting point of
all human activity. The first was the argument, made by a geologist and
an antiquarian, that animal remains found alongside stone tools in
Britain and France proved the antiquity of the human race. The second
was the publication of Darwin's ``On the Origin of Species,'' which was
incompatible with both the specifics of biblical creationism and the
more general proposition that the world was only a few thousand years
old. It was all of a sudden widely plausible that stuff in the ground
had been there for an unimaginably long time.

Before anyone could even begin to tell an ordered story about what might
have happened, however, there needed to be a way to differentiate what
happened sooner from what happened later. In the early 20th century,
geologists and archaeologists began to draw upon contemporary
observations of regular sedimentary deposits to project elementary
prehistorical ``clocks'' backward in time. The end of the last ice age,
for example, was set at about 10,000 years ago. Archaeologists then
realized that they could cross-reference these geological clocks with
the earliest written documents, ancient Egyptian and Mesopotamian
records that reached back 4,000 or 5,000 years. If geological time could
be roughly calibrated everywhere, and if even a smattering of places had
left behind calendars, recorded history could be tied to sedimentary
chronology and true dates derived from the ground.

This was heralded as a magnificent advance. The trouble, as it turned
out, was that an emphasis on written records from Egypt and the Middle
East prompted scholars to take for granted the cultural superiority of
those early civilizations and to make major assumptions on that basis
--- Stonehenge, for example, simply had to have followed the majesty of
the Great Pyramids.

In 1949, the invention of radiocarbon dating, by the American physical
chemist Willard F. Libby, turned the whole field upside down. By giving
cosmically certain dates rather than cross-referenced estimations,
radiocarbon dating undermined virtually all of archaeology's basic
premises. (Stonehenge could not have been patterned after the Great
Pyramids if it was built at the same time as Giza.) There was stubborn
resistance to the new lab results. These dates, pronounced one vaunted
Edinburgh archaeologist with a now-notorious sniff, are
``archaeologically unacceptable.'' By the early 1960s, they could no
longer be ignored, and a new generation of archaeologists gutted the
discipline and rebuilt it with very different assumptions --- ones that
did not rely on the idea that a few peoples of first-rate culture and
pedigree had been responsible for humanity's major steps forward.

If prehistorians had learned one hard lesson from chemists, their
colleagues in biology departments were slowly laying the groundwork for
another. In 1967, the molecular biologist Allan Wilson at the University
of California, Berkeley, along with one of his students, Vincent Sarich,
demonstrated that evolutionary relationships between species could be
determined not only from fossils but also, via a quantitative analysis
of blood proteins, from living specimens. Humans and apes, Wilson found,
diverged only five million years ago --- far more recently than
previously believed.

Within the decade, researchers trained in the discipline of population
genetics would get in on the historical act. Every contemporary genome
is a mosaic of individual tiles passed along from thousands of
ancestors; each of us thus contains not only our ``own'' ancestry but
those of multitudes. With each new generation, random mutations, like
misspellings, are introduced into a population; some of these will
disappear over time, but others will increase in frequency until they
are common enough to become a statistically significant part of a
population's genetic signature. If two populations have been distinct
for a long time --- that is, if people from one don't tend to mate with
people from the other --- they will share fewer of these mutations; if
they encountered each other and were fruitful, their mutation
frequencies will overlap. These insights could be made relevant to
prehistorians insofar as they could demonstrate that modern human
populations were forged in the mixture of ancient ones. It was still
mostly impossible, though, to conclude anything about when these groups
might have mixed, or where, or how.

Image

A storage room in the Vanuatu Cultural Center and National
Museum.Credit...David Maurice Smith for The New York Times

The answers to those questions required not just contemporary genetic
data but actual prehistoric DNA. The idea that it might be preserved in
old specimens has been around since 1984, when Wilson announced that his
lab had extracted DNA from the salted skin of a quagga, an extinct
equine species with the head of a zebra and the haunches of a donkey.
The further possibilities suggested by ancient DNA were awarded a
special place in the public imagination by the 1993 release of Steven
Spielberg's ``Jurassic Park.'' But even as the journal Nature
capitalized on the premiere with a paper that sequenced the DNA of an
amber-trapped weevil --- a study rendered dubious after widespread
speculation that the sample had been contaminated with the researchers'
own DNA --- observers wondered whether the sequencing of ancient genomes
was just a neat trick or research of actual value.

Over the past few years, a growing cohort of scientists has at last
produced a fantastic answer. Ancient DNA, they believe, not only allows
us to cut through what scholars once wrote off as ``wrapped in a thick
fog'' of ``heathendom.'' It promises nothing less than what the Harvard
geneticist David Reich has called ``the genome revolution in the study
of the human past.''

\hypertarget{3-the-revisionist}{%
\subsection{\texorpdfstring{\textbf{3. The
Revisionist}}{3. The Revisionist}}\label{3-the-revisionist}}

David Reich's lab is folded into a corner of a glassy, long-corridored
labyrinth at Harvard Medical School. The only exterior advertisements of
the nature of his research are large mounted maps of landforms all
around the world. One afternoon last fall, as I stood and examined a
continent, Reich materialized beside me. He is a long-limbed man with a
lithe, almost balletic figure, and he wore a closefitting pullover and
fading coral chinos. Though his hairline has receded and the curls
behind his ears are graying, a boyish precocity makes him seem much
younger than his 44 years. He led me swiftly past a confab of postdocs
and into his windowed office. There was very little in the way of
adornment, save a ghostly, truncated branch of the Indo-European
language tree (``Greek,'' ``Armenian'') that someone had sketched out,
on the wall over his desk, with what looked a permanent marker.

In his recent book, Reich ranks the ``ancient-DNA revolution'' with the
invention of the microscope. Ancient DNA, his research suggests, can
explain with more certainty and detail than any previous technique the
course of human evolution, history and identity --- as he puts it in the
book's title, ``Who We Are and How We Got Here.'' Though Reich works
with samples that are thousands or tens of thousands of years old, the
phrase ``ancient DNA'' encompasses any old genetic material that has
been heavily degraded, and Reich's work has been made possible only by a
series of technological and procedural advances. Researchers in the
field ship or hand-carry the bones to Harvard, where clean-suited
technicians expose them to ultraviolet light to prevent contamination,
then bore holes in them with dental drills. These skeletal remains are
often rare --- one pinkie-finger fragment that researchers in a lab in
Leipzig used to demonstrate the existence of a long-extinct form of
archaic humans was one of only four such bones ever found. Minuscule
portions of genetic code are isolated and enriched, then read by
expensive sequencers; statistical techniques then plot the relationship
between this particular sample and thousands more in enormous data sets.

Reich inherited from his parents a humanistic bent: His mother, Tova, is
a novelist of some renown; his father, Walter, is a psychiatrist who was
the first director of the United States Holocaust Memorial Museum in
Washington. He entered Harvard with an inclination toward social
studies, but halfway through, in pursuit of greater rigor, he switched
to physics; after graduation, he went to Oxford, where he studied
biochemistry with the idea that he might go on to medical school. The
impression he gives when talking about these years is one of restless
intellectual ambition in search of a commensurate object. He eventually
returned to Oxford to complete a doctorate, in zoology, where he at last
found a sense of belonging in the lineage of Luca Cavalli-Sforza, a
population geneticist who spearheaded efforts to make historical inquiry
resemble a hard science.

After abandoning medical school at Harvard for a postdoc at M.I.T.,
Reich returned to Harvard to establish his own medical-genetics lab. His
chief interest lay in the effort to design novel statistical approaches
to better explain how populations were related to one another. He
showed, for example, on the basis of contemporary genetic data, that
modern Indians are in fact a product of two highly distinct groups, one
that had been on the subcontinent for thousands of years and another
that formed more recently.

He got his first opportunity to study ancient DNA when Svante Paabo ---
a Swedish geneticist who had worked with Wilson --- enlisted Reich in
his efforts, based out of a lab in Leipzig, to sequence the entirety of
the Neanderthal genome. Reich's analysis helped demonstrate that most
living humans, with the general exception of sub-Saharan Africans, have
some Neanderthal ancestry. ``It was clear with the sequencing of the
Neanderthal,'' Reich told me in his office, ``that this was obviously
the best data in the world in any type of science.'' It didn't just tell
you that Indians were a mixed group; it could, in theory, specify the
moment where and when that mixture began.

So in 2013, Reich, along with a veteran of Paabo's lab and a longtime
mathematician collaborator, retooled his shop at Harvard Medical School
as one of the country's first dedicated ancient-DNA labs. The idea, he
writes in his book, ``was to make ancient DNA industrial --- to build an
American-style genomics factory'' that would liberate such fields as
archaeology, history and anthropology from hitherto insoluble debates.

He was more successful than even he anticipated. By the end of 2010,
only five ancient genomes had been sequenced in total, but in 2014, 38
were done in one year. Soon the number will be close to 2,000. Reich's
lab alone is responsible for at least half of the published output,
which doesn't include some 5,500 more bones in the process of being
analyzed and 3,000 more in storage. ``Ancient DNA and the genome
revolution,'' he declares in his book's introductory overture, ``can now
answer a previously unresolvable question about the deep past: the
question of \emph{what happened}.''

\hypertarget{4-what-happened}{%
\subsection{\texorpdfstring{\textbf{4. What
Happened?}}{4. What Happened?}}\label{4-what-happened}}

Everybody pretty much agrees that the story of what happened began in
Africa, with the evolution of modern humans; later, as of 50,000 to
100,000 years ago, the human story continued on the other continents. As
Reich sees it, the study of ancient DNA has disproved our conjectures
about what happened next. One longtime premise is that as these early
humans spread out in all directions over the land, groups of them
encountered places that struck their fancy, pitched their tents and more
or less stayed ``home'' for the duration of prehistory. This is not just
a pet theory of academic prehistorians but the natural way that human
beings have tended, over the millenniums, to connect their identities to
where they live. The ni-Vanuatu, for example, take for granted their
eternal ties to the archipelago; their oral traditions ascribe their
origins to some nonhuman feature of the landscape, their first ancestors
having emerged from a stone, say, or a coconut tree. Nonindigenous
people seek the same rootedness in consumer ancestry services like
23andMe, which declare that they're ``Spanish'' or ``Yoruba.''

Reich believes he has proved, to the contrary, that human history is
marked not by stasis and purity but by movement and cross-pollination.
People who live in a place today often bear no genetic resemblance to
people who lived there thousands of years ago, so the idea that
something in your blood makes you meaningfully Spanish is absurd. Paabo
had shown that early humans mated with Neanderthals, but that was only
one small part of the swirling ``admixture'' that characterized human
interbreeding. Even after the Neanderthals became extinct, roughly
40,000 years ago, the archaic human populations of the earth --- Reich
gives them names like Ancient North Eurasians --- were utterly unlike
the populations we see today.

Image

David Reich, who runs an ancient DNA lab at Harvard Medical
School.Credit...Kayana Szymczak for The New York Times

While Paabo continued to work on the Neanderthal period, Reich devoted
his energy to obtaining samples from the last 10,000 or so years --- the
historical domain of archaeologists. Ancient DNA's ``big bang,'' as more
than one geneticist described it to me, came with the
\href{https://www.nature.com/articles/nature14317}{2015 publication}, in
Nature, of a Reich paper called ``Massive Migration From the Steppe Was
a Source for Indo-European Languages in Europe.'' On the basis of
genetic information culled from 69 ancient individuals dug up by
collaborating archaeologists in Scandinavia, Western Europe and Russia,
the paper argued that Europeans aren't quite who they thought they were.
About 5,000 years ago, a ``relatively sudden'' mass migration of nomadic
herders from the east --- the steppes of eastern Ukraine and southern
Russia --- swept in and almost entirely replaced existing communities of
hunter-gatherers and early farmers in Central and Northern Europe. These
newcomers were known to exploit many of the cutting-edge technologies of
the time: the domestication of horses, the wheel and, perhaps most
salient, axes and spearheads of copper. (Their corpses sometimes
featured cutting-edge wounds.)

The Reich team inferred that the major source of contemporary European
ancestry --- and probably Indo-European languages as well --- was not,
in fact, from Europe but from far to the east. And this discovery,
confirmed by the near-simultaneous publication of almost identical
results from a competing ancient-DNA lab in Denmark, had monumental
implications for science's understanding of the whole ancient world.
Great migration events --- like the movement of Siberian peoples into
North America or the spread of voyagers into the Pacific --- were not
outliers but the norm. After Europe and India, there were similar mass
migrations identified in Africa, the Middle East and Southeast Asia. No
one ever expected that we could possibly amass so much new evidence
about the human past. And no one was producing this work at the pace and
throughput of David Reich and his genomics factory. Most scientists felt
lucky if they published one or at the most two Nature papers in a
lifetime. Reich was publishing three or four a year.

There was an obvious pattern to the great migratory arrows freshly drawn
across world geography, which were often coincident with the spread of
technology or agricultural practices. Earlier paleogenomic results
established thousands of years of heady mixture among long-forgotten
ancient populations. With the relatively recent rise of everything we
associate with ``culture'' --- technologies like agriculture, metallurgy
and eventually writing --- much of this continuous ``admixture'' began
to give way, it seemed, to discontinuous episodes better characterized
as ``replacement'' or ``turnover.'' That is, about 5,000 to 9,000 years
ago, human history was, at least in a few crucial places, less about
various groups coming together and more about some groups blotting out
their neighbors.

This was not only relevant as an eccentricity of prehistoric demography,
but broadly consequential for the ongoing study of culture itself --- of
where new ideas come from and how they proliferate. When we thought of
populations as stationary and largely stable, we assumed that whatever
evolutionary progress they made, from toolmaking to agriculture,
reflected either a native innovation or the incorporation of some
adjacent group's avant-garde practice. Now it seemed as though culture
was less about the invention and spread of new ideas and more about the
mass movements of particular peoples --- and the resulting integration,
outcompetition or extermination of the communities they overran.
Previously, it was possible to think about prehistory as a kind of grand
bazaar. Now the operative metaphor (as multiple science journalists
observed) was more like Risk, or even ``Game of Thrones.''

\hypertarget{5-looking-for-the-lapita}{%
\subsection{\texorpdfstring{\textbf{5. Looking for the
Lapita}}{5. Looking for the Lapita}}\label{5-looking-for-the-lapita}}

The ancient-DNA revolution seemed unlikely to have anything to say about
Oceania, where the heat and humidity made the preservation of DNA
implausible. But in 2014, Stuart Bedford got that second surprise call,
from a Dublin-based archaeologist named Ron Pinhasi, a frequent Reich
collaborator and procurer of samples. Pinhasi had discovered that the
inner ear's petrous bone, one of the densest in the body, often
preserved vast quantities of genetic material. Could he and Reich
examine the skulls of Teouma? In Vanuatu, human remains are often
associated with ancestral spirits and are thus taboo --- understandably,
Bedford emphasized to me, explaining that he wouldn't be comfortable
digging up and boring into ``Granddad.'' But in this case, the
ni-Vanuatu expressed no reservations: Local oral traditions contained no
sacred reference to the Teouma dead, and Chief Alben gave his blessing.
One of Bedford's colleagues opened the skulls in a workshop warren
behind the national museum, extracted the nubbins of petrous bone and
shipped them to Dublin, where they were sandblasted. There turned out to
be DNA in three of the samples. It was the first to be found in the
tropics and suggested the opening of wide new fronts in ancient-DNA
research.

The skulls of Teouma were particularly interesting to paleogenomicists
not only because they produced the first ancient DNA in the Pacific but
because their genetic evidence could be brought to bear on an
outstanding debate in the region. The pivotal moment in Pacific
archaeological history happened in 1952, when a team of researchers
found a cache of dentate-stamped pots at a place called Lapita in New
Caledonia, a French collectivity to the southwest of Vanuatu. More than
200 sites eventually turned up nearly duplicate versions of this
innovation across an enormous span of the region. The pots were often
found with particular varieties of preserved plants and nuts, as well as
stone adzes. Whoever made those pots some 3,000 years ago had traveled
across more than 2,000 miles of ocean --- from near Papua New Guinea to
Tonga and Samoa --- in perhaps as little as 10 generations. As Patrick
V. Kirch, the dean of American archaeology in the Pacific, once put it,
``Without a doubt, the Lapita colonization of Remote Oceania ranks as
one of the great sagas of world prehistory.''

Where had this ``Lapita'' culture come from, and who were the people
associated with it? Over the last 50 years, a collaboration among
archaeologists, linguists, botanists, ecologists, geologists and more
had produced some form of consensus. A population of early farmers
departed from Taiwan about 5,000 years ago, with the help of the newly
developed outrigger canoe. They moved down through the Philippines and
the Spice Islands, along the northern coasts of New Guinea and
eventually out to the Bismarck Archipelago, more or less the limit of
Near Oceania; the ``tracer dye'' for their path was the language family
they left behind, one known as Austronesian. Along the way, they
encountered populations of ``Papuans'' --- a generic shorthand for
highly distinct groups of people who had been in the Papua New Guinea
region for 40,000 years. The interactions between the incoming
``Austronesians,'' another shorthand for whoever was presumably
spreading those languages, and the indigenous Papuans created the
constellation of practices that would become known as Lapita. Finally,
the people now associated with Lapita sailed into the blankness of the
open ocean for the first time, crossing the Remote Oceania divide to
Vanuatu and, from there, outward to the farthest reaches of the Pacific.

Archaeologists differed, often bitterly, on the details, but as Reich
describes it in his book, the prevailing opinion was that ``the Lapita
archaeological culture was forged during a period of intense exchange
between people ultimately originating in the farming center of China
(via Taiwan) and New Guineans.'' This certainly made intuitive sense.
The people of contemporary Vanuatu are black, like the Papuan people of
New Guinea, but they speak Austronesian languages that can ultimately be
traced to Asia. Reich believed that the existing consensus was the
perfect sort of hypothesis to put to the ancient-DNA test. The
Austronesians and the Papuans had been separated by at least 40,000
years of genetic differentiation, which meant that it would be very easy
to discriminate by genetic signature. Would the samples taken from the
skulls at Teouma show a closer relationship to the people of nearby
Papua or the people of distant Asia?

In October 2016, \href{https://www.nature.com/articles/nature19844}{the
paper} --- with such well-regarded Pacific archaeologists as Stuart
Bedford and his mentor, Matthew Spriggs of the Australian National
University, among the 31 authors --- was published in Nature as
``Genomic Insights Into the Peopling of the Southwest Pacific.'' The
analysis of ancient DNA from three 3,000-year-old skulls from Teouma,
along with one skull dated a few hundred years later from Tonga,
appeared to provide unambiguous confirmation of Lapita heritage. The
First Remote Oceanians, as the paper calls them, were not, after all, a
heterogenous group; they were of unmixed Asian descent.

Image

A mural in Port-Vila depicting life in Vanuatu before Western influence
set in.Credit...David Maurice Smith for The New York Times

The paper suggested that the old archaeological consensus --- that the
Lapita advances reflected the joint contributions of Austronesian and
Papuan peoples --- could be replaced by a much starker story. The
genetic record can be more ``parsimoniously explained,'' the authors
remark, by at least two \emph{separate} migrations to Vanuatu: first,
the Austronesians, with their East Asian ancestry, and then, hundreds of
years later, the Papuans. This wasn't a story of ``admixture'' but one
of successive waves of migratory ``turnover.''

These results were published two years earlier, but as we sat in his
silent office, Reich still betrayed some enduring wonder at his
revelation. He reminded me that he hadn't been trained as an
archaeologist and had thus come to these debates as an outsider. In the
broadest conceptual terms, though, he saw the lessons of this
once-enigmatic Lapita migration to be exceedingly profound. ``I think
the important finding for archaeologists and for historians and
sociologists and anthropologists is that this group moved thousands of
kilometers over many hundreds of years, through a region occupied by
long-established, sophisticated people, and hardly mixed with them.'' He
observed that ``essentially everybody was surprised.'' They were
surprised, in part, because archaeologists since the 1960s had been
trained never to assume the purity or coherence of a people, a slippery
slope to the conclusion that certain peoples came by their advantages
``naturally.''

But the data seemed indisputable. ``Now we can establish that
definitively. That's what this technology allows us to do. And then
they'' --- meaning all those other disciplines, which heretofore had
overseen the study of prehistory --- ``can get on with answering what
really matters, which is try to interpret what happened.''

He paused. ``But I think that basically everybody, almost without
exception --- except for very extreme people --- is excited about this
data in archaeology.''

\textbf{PART II}

\hypertarget{6-the-storm-beneath-the-surface}{%
\subsection{\texorpdfstring{\textbf{6. The Storm Beneath the
Surface}}{6. The Storm Beneath the Surface}}\label{6-the-storm-beneath-the-surface}}

The primary characteristic of the deepest reaches of the past,
especially for the sort of observer whose paramount concerns are those
of the present, is the accommodating silence found there. The quieter an
epoch on its own terms, the more loudly it can be made to speak, in the
way of a ventriloquist's dummy, for ours. The study of ancient peoples
--- or of the ``primitive'' ones often taken to stand in as their
contemporary proxies --- has been framed by our preference for simple,
just-so stories of origin and trajectory. Archaeologists, who feel as
though they learned this lesson long ago, thus survey the rapid rise of
ancient DNA with an overwhelming sense of déjà vu. By once again giving
``migration'' pride of place in the story of prehistory, paleogenomics
has resurrected some old intellectual ghosts.

By the time radiocarbon dating had come of age, in the postcolonial
ferment of the 1960s, archaeology was already primed to relinquish its
emphasis on narratives of migration. In the 1910s, a German named Gustaf
Kossinna turned his personal fixation on heightened Proto-Germanic
Barbarian activity after the fall of Rome into a theory, ``settlement
archaeology,'' that emphasized the glory of the German nation. For
Kossinna, a given material culture --- a uniformity in pottery style,
say --- was the index of a coherent people, and it was the destiny of
the Germans, greater than even that of the Romans, to extend their
superior Aryan culture from their homeland to the ends of the earth. The
Nazis were more than happy to put these claims into service, and even
academics with better politics became convinced that the driver of human
progress was the roving exogenous shock of migratory adventures. As
Colin Renfrew described it: ``Prehistory was seen as a kind of global
chessboard, with the various cultures as pieces shifting from square to
square. The task of the archaeologist was simply to plot the moves ---
or, in other words, trace the path of the `influence' as new ideas were
diffused.''

The brutal ramifications of settlement archaeology, when combined with
the dramatic introduction of radiocarbon dating, shook the entire
discipline to its foundations. The disruption was so major that the
profession felt it had to rehabilitate itself as the ``New
Archaeology.'' This new generation of practitioner agreed that just
because similar pots were found in various places didn't mean they were
all made by one homogeneous group of people. Instead, archaeologists
retreated to a much more modest and fine-comb preoccupation with what
they called the ``processual'': very particular inquiries into very
particular societal dynamics. They paid much closer attention to how
individual cultures appeared to change and grow over time and much less
attention to how Culture Had Changed --- to the fantasy that some
special key will unlock the secrets of history. This left a big-picture
vacuum that paleogenomicists like Reich have been eager to fill.

The resulting schism has been easy to caricature as the old struggle
between hard scientists and humanists --- a suspicion of all geneticists
as quantitative imperialists, a derision of all archaeologists as
sentimental Luddites --- but that isn't quite accurate. Many
archaeologists are thrilled about the arrival of the first genuinely new
form of prehistoric data in generations. The more meaningful division is
between two alternate intellectual attitudes: those bewitched by grand
historical narratives, who believe that there is something both detailed
and definitive to say about the very largest questions, and those who
wearily warn that such adventures rarely end well.

Archaeologists shouldn't necessarily care. They remain theoretically
free to continue doing things at processual pace, paying thorough
attention to historic context and indigenous sensitivities. Even those
who are enthusiastic about ancient-DNA research --- not only for the new
data it provides but for the vigor it has brought to their field ---
could in principle choose to partner only with geneticists who respect
their priorities and expertise; after all, they are the ones who dig the
samples out of the ground, and nobody is forcing them to surrender their
treasures at gunpoint. Collaborations between geneticists and
archaeologists on an even footing have produced well-received studies
that plot, say, the family trees in a medieval cemetery.

Image

A human petrous bone being analyzed at the Max Planck Institute for the
Science of Human History in Jena, Germany.Credit...David Maurice Smith
for The New York Times

But in practice, the paleogenomicists have totally altered the
environment in which prehistory is being studied by everyone. The
landscape is dominated by four well-funded, well-connected labs, three
of which --- Paabo's in Leipzig, along with those of two of his
protégés, Reich at Harvard and Johannes Krause, who runs a newer outfit
in the small German city Jena --- collaborate closely with one another,
to the point that some critics accuse them of collusion. The power of
these top labs extends to samples, data and even technology: Proprietary
chemical reagents let them isolate and enrich ancient samples much more
accurately and cost-effectively than other labs can. One geneticist
compared competing with the big labs to battling an entire navy ``with a
little dinghy, armed with a small knife.'' Another told me: ``A small
lab focusing on a particular site would not be able to place their work
in the context of the bigger picture. The only way I can get access to
that data is if I give my bone to David or Johannes and wait until they
process it --- and bury me in the list of contributors to their paper.''

The selective pressure to collaborate with this state-of-the-art
oligopoly is extremely strong, not only because of their advantages in
funding, speed and operational scale but also because of the
relationships they enjoy with the top-tier journals. Publication in a
title like Nature or Science can utterly transform a young scholar's
career, enhancing both the prospect of tenure and the ability to secure
grant funding. The rush to corner the market on old bones in some
``understudied'' place or time period has placed a high premium on
virtually all samples, creating perverse incentives for researchers to
procure these scarce, nonrenewable resources. The only entry fee for a
27th or 53rd author slot in this ``free-for-all bonanza of Nature
papers'' (as one geneticist described it to me) is the cost of a bone
shipment and a minimal account of its basic archaeological provenance.
Multiple researchers told me that it wasn't unusual for junior authors
to be given just days to review a finished manuscript, with little input
into its broader framing. (Reich, Krause and Paabo all dispute this,
saying they couldn't think of any examples in which co-authors were
given such a short time to review a manuscript.)

There thus reigns, in the world of ancient DNA, an atmosphere of intense
suspicion, anxiety and paranoia, among archaeologists and geneticists
alike. In dozens of interviews with practitioners of both disciplines,
almost everyone requested anonymity for fear of professional reprisal.
Many archaeologists described a ``smash and grab'' culture in which
hopeful co-authors source their bones by any means necessary. Among
teams at work on any given excavation, it takes only a single colleague
to deliver a bone to one of the industrial giants for the entire group
to lose control of their findings. Museums, too, are being swept up by
the perverse incentives: One of the geneticists told me stories about
having brokered an agreement to sample a particular collection, only to
arrive and discover that someone else showed up the previous day to
claim the same bones under a false pretense. The weaker the institutions
of the country, the harder it is for local researchers to have a
fighting chance. Scientists in Turkey and Mexico told me that museum
curators routinely had to explain that they had promised their native
bone collections elsewhere. As one ancient-DNA researcher in Turkey put
it to me, ``Certain geneticists see the rest of world as the
19th-century colonialists saw Africa --- as raw-material opportunities
and nothing else.''

(Reich, Krause and Paabo strenuously denied the characterization of
their labs as colluding in a manner that harms competitors. Krause noted
that his lab employs students and scientists from 30 nations and
supports foreign researchers. Reich commented via email: ``The fact that
the substantial majority of the world's human ancient DNA data has been
produced by a small number of laboratories is not because of any special
access to samples, but rather because of the high quality of work these
laboratories deliver.'')

It has not gone unnoticed that the stunning, magisterial sweep of
genetic revisionism, on the one hand, and a genetic emphasis on radical
prehistoric migrations, on the other, bear more than a little in common.
Some anthropologists and archaeologists accept this analogy with gallows
humor. One told me that I should model this article after the format of
the standard Nature paper: ``Ancient DNA Reveals Massive Population
Turnovers in the Humanities,'' she suggested as a title, and proposed
this as an abstract: ``The aristocratic lab scientists arrived with
their superior technology and displaced the pre-existing researchers and
their primitive truth-implements and overcomplicated belief systems.''

Others saw less to laugh at. Some archaeologists who had collaborated on
the 2015 paper about Indo-European invasions withdrew their names to
protest conclusions they saw as echoes of Kossinna --- the mass
migrations of advanced Indo-Europeans into Central Europe. (Reich got
the critics back on board by adding a note, on Page 138 of their paper's
141-page supplementary materials, that said their work in fact
contradicted Kossinna, not because he was wrong about mass migration but
on a technicality: The European ancestral homeland had, in fact, been
far to the east, near the Caucasus and nowhere near present-day
Germany.) The analogue was hard to counter. Geneticists had indeed swept
down from their laboratory enclaves to extend their sovereignty over
what had always been the terrain of archaeology. And no single
individual had as much influence or power as Reich.

\hypertarget{7-the-postage-stamp-problem}{%
\subsection{\texorpdfstring{\textbf{7. The Postage-Stamp
Problem}}{7. The Postage-Stamp Problem}}\label{7-the-postage-stamp-problem}}

Migration in the Pacific had never been quite as fraught as it was
elsewhere; the people had obviously shown up from someplace. Or rather,
this had been obvious to outsiders, if not to the locals. Upon our
return from the Teouma overlook, Bedford went off to catch up on village
gossip, and I sat with Chief Alben in the shade of a stout, leggy banyan
tree, its exposed root system rising from the earth like a half-exhumed
skeleton. Alben is a hale and jovial older man with a round paunch and a
push-broom mustache. For years, he has participated in a volunteer
fieldworker program that trains the ni-Vanuatu to record and preserve
their local traditions amid the creep of global monoculture and to pay
attention to the sorts of archaeological finds they might otherwise
ignore.

I asked him about how the concept of Lapita migration to empty islands
had been received by people whose oral traditions said they came from a
stone or a coconut tree. After the Teouma find, the national post issued
a special commemorative stamp --- ``Lapita People: The Pacific's
Original Explorers'' --- with an artist's recreation of a colonial Eden
that showed men and women, drawn black to resemble the ni-Vanuatu,
cleaning fish and making camp, and Bedford printed pamphlets for
schoolchildren that explained that the Lapita were the grandfathers of
grandfathers of grandfathers. Now Reich's research had raised the
prospect that they bore not even a passing resemblance to Vanuatu's
earliest settlers.

In the wake of the initial discoveries at Teouma, Alben replied, he
explained to his villagers that there was nothing surprising in the fact
that the grandfathers of grandfathers of grandfathers had once come from
someplace else. ``Our \emph{kastom} teaches us that people moved from
place to place to place,'' he told me. \emph{Kastom} is an expansive
concept that includes tradition, history, land rights and social norms;
local \emph{kastom} varies tremendously across the more than 80 islands
of Vanuatu, but the notion itself has become sacrosanct for the
continuity and authority it provided in the aftermath of colonial
occupation.

Image

A technician at the Planck lab in Jena.Credit...David Maurice Smith for
The New York Times

Alben told me he had been stymied by the practicalities, though. ``Maybe
these Lapita people came from Asia! How? How?! How can these people come
here?''

He waited for me to answer, but it wasn't clear what he meant; I
shrugged and ventured a timid, ``Canoe?''

He shook with laughter at such a painfully obvious answer. His question
was not about what they used to cross the water but how they founded a
way of life that endured until today. ``They took the coconut'' --- he
pointed off to his left --- ``and they took the breadfruit'' --- he
pointed off to his right --- ``and they put it into the canoe. When the
canoe lands, they plant.'' The people, in other words, were tied to the
land by what they had brought with them. On the road out of Port-Vila,
I'd made an idle remark to Bedford about the primeval greenery around
us; he corrected me to say that what looked like jungle was actually
under heavy cultivation.

The ni-Vanuatu were not accustomed to thinking about history for its own
sake, instead expecting that any story you told about the past
necessarily gave form and guidance to the present. If \emph{kastom} told
you that your people came from a stone near the lagoon, that was
relevant for ongoing disputes about who now deserved to till that land.
The idea that in some abstract, scientific way they were ``really'' from
somewhere else didn't mean anything unless there was a direct
contemporary moral.

They did know, however, that what had often been presented to them as
abstract scientific knowledge routinely concealed some practical agenda.
The first European explorers in the region, even if they weren't quite
so forthrightly instrumental about it, also interpreted the history of
the South Seas to suit their own contemporary concerns --- both imperial
and philosophical. Pacific Islanders, marooned in what were seen as the
natural laboratories of primal isolation, were enlisted as the ``noble
savages'' of Enlightenment fantasy. The question of who they were and
where they had come from became lively topics. Some ventured that they
were refugees from the Lost Continent of Mu. Others tried to classify
them in a way that would accord with their own pet-scientific notions of
cultural evolution. The French explorer Dumont d'Urville, who first
sighted Vanuatu in the 1820s, proposed a tripartite scheme that
unfortunately endured: There were the Polynesians (``many islands''),
the lighter-skinned people who inhabited an enormous triangle of the
Eastern Pacific bounded by Hawaii, New Zealand and Easter Island; the
Micronesians (``small islands''), who lived on the atolls of the Western
Pacific north of the Equator; and, always finally, the Melanesians
(``dark islands''), the dark-skinned people east of New Guinea who
spanned the divide between Near and Remote Oceania. Europeans fixated on
the differences between the Melanesians and the Polynesians, imagining
the Polynesians as a kind of laggard aristocracy, comparable to the
ancient Greeks, and the Melanesians as naturally backward black people.

And so, when it came to the question of how ancient peoples had
populated the Pacific, the most persistent proposals rested on racial
typologies. The Melanesians obviously came from in and around Papua,
which was relatively nearby and inhabited by ``savage'' black people,
whereas the lighter-skinned and more ``advanced'' Polynesians probably
sojourned via heroic open-sea navigation from Asia. Anything
``superior'' --- technology or social structure --- was linked to the
migratory intervention of exceptional groups from distant shores. The
European colonial enterprise was thus justified as part of the natural
relationship of incoming enlightenment and indigenous savagery.

The ni-Vanuatu are not unaware of the region's racialized history, or of
its racialized present. As Bedford and I drove back to town, the only
visible graffiti was a reminder of regional Melanesian pride: ``Free
West Papua,'' a show of racial solidarity with the black residents of
the western half of New Guinea, a persecuted colonial possession of
Indonesia since the early 1960s. A new sort of colonial anxiety,
meanwhile, is manifesting itself about the Chinese, who have been
investing heavily in the country. The old shops on Port-Vila's main
harborside drag have been replaced by cheap Chinese joints hawking
souvenir ukuleles, and the new luxury-condominium developments above
downtown advertise ``Hong Kong Apartment-Style Life'' over images of
white and Asian people in infinity pools.

Bedford and his archaeologist colleagues on Vanuatu are known for their
long tenure in the country and their keen acquaintance with local
sensitivities, and it was only on their bond that the Teouma petrous
bones were sent abroad for sandblasting. Now their names were on a
genetics paper arguing that the ni-Vanuatu's ancestors were not Lapita
after all, but latecomers to an archipelago first settled by purely
Asian expeditions.

\hypertarget{8-the-ghosts-of-peer-review}{%
\subsection{\texorpdfstring{\textbf{8. The Ghosts of Peer
Review}}{8. The Ghosts of Peer Review}}\label{8-the-ghosts-of-peer-review}}

As it happens, this radical claim was not as definitively accepted as
the published paper seemed to show: Serious challenges to its soundness
were laid out during Nature's peer-review process. And yet, in a highly
unusual move, the paper was accepted over steadfast opposition from two
of the three original peer reviewers on its anonymous panel.
Confidential documents made available to me reveal deep concerns with
the paper's methods and its conclusions.

Among the two objecting reviewers, the methodological critiques --- both
on the level of archaeological context and that of data analysis ---
were paramount. ``It seems to me that a significant question,'' Reviewer
Two writes, ``is whether these individuals were actually `Lapita
people,' assuming that such a thing exists.'' The paper listed six of
the nine skulls found at Teouma, though the team had only successfully
extracted ancient DNA from three of them. ``In addition,'' Reviewer Two
continues, ``it seems clear that these skulls were not related to the
100+ individuals excavated from the Teouma site. That is, the skulls do
not fit the bodies. Clearly there was a complex set of traditions around
these burials including decapitation at some time before or after
initial burial. I am curious as to whether these skulls might have been
kept by relatives and only later (perhaps much later) (re)buried at
Teouma,'' a tradition among some indigenous groups.

Even if the skulls were the same age as the rest of the bones at the
cemetery, there was still the matter of how oddly they had been interred
--- one inside a jar, the others arrayed like a shield across another
skeleton's chest. ``This seems to suggest,'' Reviewer Two adds, ``that
the three people were special in some way. Hence I am concerned about
drawing too many conclusions from such a small number of individuals
plus individuals who were certainly not a random sample of the
population.'' Shouldn't the collaborating archaeologists have pointed
all this out?

The study's authors, the objecting reviewers insist, had made
disproportionate or even wholly unwarranted claims on the basis of both
the archaeological and genetic evidence they had provided. Yes, the
Teouma skulls came from an important site, and yes, the new data they
provided was a fascinating additional piece of evidence. But they still
just represented three samples from one site on one island, and the
objecting reviewers noted that Reich's inferences could have been skewed
by what one of them called ``bias in the method'' --- the set of
assumptions necessitated by his complex statistical models. Meanwhile,
the contemporary samples they used for historical comparison weren't
even from Vanuatu, but from potentially unrelated regional communities
used as proxies. ``In my opinion,'' Reviewer Three wrote, ``this paper
does not merit a significant advancement over current studies and the
lack of detail regarding basic data description is frustrating.''

The paper's purchase on significance, then, seemed to have less to do
with its originality than with its certainty. The title of the first
submission was ``Ancient DNA Documents Multiple Human Migrations Into
the South Pacific,'' and it presumed to offer the final word on the
history and ancestry of an entire region. Three contemporaneous samples
might be sufficient for a modest paper about the Teouma site, but modest
papers about one archaeological site in Vanuatu are not the sort of
thing Nature is in the business of publishing. ``In the light of these
various comments,'' an editor wrote to the reviewers, ``we have declined
publication of this study.'' There is a clear distinction, at Nature and
elsewhere, between a rejection and a call to revise and resubmit.
``Rejection means rejection,'' one geneticist told me, ``and rejection
is final.''

Yet the Reich team proceeded to revise. They were aided in this by their
colleagues at the Max Planck Institute for the Science of Human History,
in Jena. Its director of archaeogenetics, Johannes Krause, had worked
alongside Reich in Paabo's lab. When the Jena team heard that the
Oceania paper had been found wanting for further regional samples ---
samples that would allow them to expand their claims beyond simply
Vanuatu --- one doctoral candidate remembered that their inventory
contained a stray petrous bone from a site in Tonga, one already found
to contain readable DNA. It was, fortuitously, highly relevant to
Reich's Oceania work, and the data was forwarded along in due course.

Image

Johannes Krause, who runs the Planck lab in Jena.Credit...David Maurice
Smith for The New York Times

On the basis of this single additional ancient bone, the Reich lab
resubmitted their paper, and a fourth reviewer was added to the panel.
The revision addressed very few of the objecting reviewers' concerns,
and the changes it did provide struck those reviewers --- who were
asked, to their surprise, to review the revision --- as perfunctory and
weak. ``The analysis and data generation presented herein, in my
opinion,'' Reviewer Three ultimately concluded, ``simply does not merit
a Nature-level manuscript.'' Nevertheless, the paper was accepted.

When pressed about the peer-review process, Reich told me his reply to
the initial round of concerns had been ``the most robust, powerful,
compelling response we've ever given to a set of reviews. We completely
answered absolutely every question very robustly; there was not a single
point in those Reviewer Two and Three comments that had any validity and
that we were not able to fully and powerfully answer.'' When I noted
that the objecting reviewers had not been convinced by their
counterarguments, he said: ``The fact that a person who sends a review
doesn't feel like their arguments have been answered doesn't mean that
they haven't been answered. I felt that those reviews were not
compelling reviews, didn't make sense, didn't take into account the
actual evidence that we had brought to bear properly and were completely
addressed by our response, and the journal agreed.''

He acknowledged that it was rare for journal editors to overrule their
referees. ``This was a case where the reviewers were making egregious
errors,'' Reich said. ``These were problematic reviews that should have
been discounted because of their problematic nature, and we were able to
successfully make that case on very good grounds, and the editor agreed
with that in the course of the review process. And it's a very rare
thing.'' (A spokesperson for Nature said in a statement, ``For
confidentiality reasons, we cannot discuss the editorial history or
review process of any Nature paper with anyone other than the
authors.'')

At the end of our conversation, Reich returned to his Vanuatu effort,
waxing unsolicitedly about his personal attachment to the finding.
``That paper is such an important paper. It's such an important
observation, such an important measurement --- it's exactly the type of
thing that needs to be published in that type of journal. It's in the
class of an unrejectable paper.''

\hypertarget{9-the-seductions-of-migration}{%
\subsection{\texorpdfstring{\textbf{9. The Seductions of
`Migration'}}{9. The Seductions of `Migration'}}\label{9-the-seductions-of-migration}}

In Reich's view, quibbles about which skull did or did not fit which
skeleton in an ancient tropical cemetery in a land he had never visited
were entirely beside the point. He was doing large-scale, broad-brush
work, and it was up to the archaeologists to add their fine filigree of
detail. Even if you accepted the paper's broad-brush results, however,
most archaeologists find this distinction misleading. The problem wasn't
that he was explaining too much on the basis of too little, but that he
wasn't ultimately explaining anything at all; it was all well and good
to put ``migration'' back on the table, but the concept itself did
little to clarify what was actually going on. For example, it was a
still a mystery that secondary Papuan migrants had replaced the original
settlers but somehow adopted their Austronesian language.

The Jena outfit, evenly split between geneticists, archaeologists and
linguists, was set up to address questions of this order, in studies
designed to include each discipline's contributors as full partners. The
edifice itself is an architectural bricolage, a vaguely Bauhaus-inspired
white building conjoined via metal tube to a stately 19th-century villa.
The head of the institute's department of linguistic and cultural
evolution had decided that his team's flagship project would be a
fine-grained 10-year investigation of the ``Galápagos of language
evolution'' that made Vanuatu a ``microcosm of all those forces that
have generated human diversity.''

A young Irish anthropologist, Heidi Colleran, was brought on to help
lead the relevant ethnographic field research; just before she left, she
and her partner, a British population geneticist named Adam Powell (who
also happened to be her collaborator on the project), were asked if they
might try to collect spit from the groups she planned to work with, for
the purposes of a proper contemporary baseline. Reich had used other
modern Oceanic groups as rough proxies in part because no one imagined
that any ni-Vanuatu would ever assent to such a study. Stuart Bedford,
who had been brought in on the Jena project, believed that it wouldn't
happen in a million years. If outsiders said that spit held secrets
about the past, the ni-Vanuatu might worry that those secrets --- if
these foreigners said they were ``actually'' from elsewhere, indeed
latecomers to their own nation --- could nullify their rights to the
land. After the publication of Reich's paper, the indigenous Kanaks of
``neighboring'' New Caledonia declared a three-year moratorium on any
genetic research, for fear that their limited sovereignty might be
undermined.

Image

Villagers from Alpalak, a community participating in a study of Vanuatu
languages being run by the Planck lab in Jena.Credit...David Maurice
Smith for The New York Times

The Jena team sought the ethical oversight of an institutional review
board. Once in Vanuatu, Colleran, along with Powell and Kaitip Kami, the
curator at the national museum, pitched their project as a way for
villagers to understand where they fit in the family tree of the
Pacific; they also promised that, in accordance with best ethical
practices, they would return to present the results to the participant
communities. To their great delight, they were deluged with willing
volunteers. Over the next year, the researchers back in Jena put these
results together with data they had retrieved from 19 new ancient
samples; after a review period of six weeks,
\href{https://www.nature.com/articles/s41559-018-0498-2}{their paper}
appeared in Nature Ecology \& Evolution on Feb. 27, 2018.

While so much of Reich's work has conjured the notion of sweeping,
wholesale replacements by one population of another, the Jena paper
proposed instead a much more gradual process. Their samples demonstrated
not a single decisive turnover event but at least 500 years of ongoing
traffic between Papuans and Austronesians --- plenty of time to explain
how the former had managed to pick up the latter's languages, for one
thing. Whatever happened in that period was clearly complex, but it
seemed to them inaccurate to describe it as the one-off snuffing out of
one group by another. ``The idea that one day there were tons of people
in canoes,'' Krause told me when I met him in Jena, ``that's not how we
should see it.'' What Reich was wont to attribute to simple
``migration'' was just a restatement of the problem of what happened.
The actual causal mechanism could have been malaria, or warfare, or
volcanic activity, or some competitive advantage in agriculture.

A thought experiment might help to illustrate this. Imagine that the
written history of our current era were lost to time, and
paleogenomicists of the future were trying to explain the peopling of
North America on the basis of a few bones that dated from between the
16th and 20th centuries. If these bones included the descendants of
British, Spanish and French colonists as well as those of Yoruba slaves,
the researchers might conclude that European migrants arrived together
with African migrants and that their ``sex-biased admixture'' created
the people known henceforth as Americans. From our perspective, those
geneticists wouldn't exactly be wrong about all this --- but nobody
would accuse them of being right, either.

There's no particularly good reason to believe that the past was
significantly simpler than the present, and archaeologists have come to
believe that the more digging they do, the more complexity they uncover.
It makes sense that they would resist simple explanations. From Reich's
perspective, this archaeological truculence represents a stubborn
attachment to the old, complicated stories in the face of new molecular
data --- just as some archaeologists held fast to their tall tales
despite what Renfrew called the ``mysterious boffinry'' of radiocarbon
dating. The analogy, however, doesn't quite work. This time the
scientists have arrived with their advanced technologies not to
dismantle theories of coherent ``cultures'' who ``migrated'' from
``homelands'' but to revive them --- without any disciplinary memory of
the traps involved or the stakes of failure.

Over the course of 2017, Reich was working on his own competing
follow-up, though by the time the Jena team submitted its completed
paper to journals he had barely begun to compose his own. (Reich, who
told me he could not remember the specific timeline, said, ``The whole
analysis was mature; we basically had the key findings already --- we
were just slow in writing it up, because we were overextended.'') The
two labs had briefly contemplated collaborating, but in May 2017, the
Jena team vetoed the idea, one of its leaders told me, because Reich
wanted too much control. So the projects advanced separately. Reich
tried unsuccessfully to get contemporary ni-Vanuatu spit from other
researchers until he learned of some blood samples, drawn decades ago by
medical researchers and now held in trust at a repository at Oxford. The
Reich team obtained permission to resequence the old samples for their
own purposes --- even though in gray-area cases like these it is never
at all clear who holds the authority to retroactively license the use of
vital fluids taken when ethical protocols were considerably more lax. He
also had 11 new ``ancient'' samples, though six of them were from only
about 150 years ago.

Reich submitted his manuscript to the journal Current Biology on the
same day that Jena's paper was accepted by Nature Ecology \& Evolution.
One week later, on Feb. 19,
\href{https://www.cell.com/current-biology/fulltext/S0960-9822(18)30236-7?innerTabgraphical_S0960982218302367=}{the
paper} was accepted and given an online publication date of Feb. 28 ---
one day after the online publication date the Jena team had been given.
Peer review and acceptance of a paper in a week was in itself an
unprecedented feat; not a single person I talked to in the field could
think of a similar case. Reich conceded that it was uncommon. ``It was
the fastest review we ever had,'' he told me, ``but it was actually a
very high-quality review. It was better than most reviews we got. It was
actually a serious review, a very serious review.'' Some other
geneticists doubted it; one said to me: ``There's no way there's proper
peer review there. That's an egregious violation of scientific norms.''
(``The Reich paper was properly reviewed by three relevant experts, all
of whom recommended publication with minor requests for revision,'' the
Current Biology editor in chief, Geoffrey North, said in an emailed
statement, crediting the turnaround to reviewers who made the paper a
priority.) Even so, publication on successive days was apparently not a
satisfactory outcome. On Feb. 19, Reich's paper appeared in
\href{https://www.biorxiv.org/content/early/2018/02/19/268037}{preprint
form on the web}, eight days before the Jena effort came out.

Image

A technician collecting a petrous bone from a UV light box at the Planck
lab in Jena. The items are bathed in UV light to eliminate foreign DNA
contamination.Credit...David Maurice Smith for The New York Times

While the Jena samples showed at least 500 years of Austronesian-Papuan
mixture, Reich's follow-up argued --- on the basis of a single sample
from a single island --- that the First Remote Oceanians had been
replaced by at least one wave of belated Papuans. Otherwise, the paper
had little to add. Reich had, however, updated his analysis of the
original skulls with improved, ``higher-resolution'' statistical
techniques. One new data point, which Reich saw as a refinement, struck
some critics as a significant revision: While Reich emphasized to me in
his office that the first paper conclusively demonstrated no mixture
between the Austronesians and the people they encountered, the updated
analysis showed that Teouma's ``Lapita individuals had a nonzero
proportion of Papuan-related ancestry.'' It ``remains striking,'' the
new paper remarked, that these first migrants were only ``minimally
admixed'' --- but admixed they were.

\hypertarget{10-the-archaeologists-dilemma}{%
\subsection{\texorpdfstring{\textbf{10. The Archaeologist's
Dilemma}}{10. The Archaeologist's Dilemma}}\label{10-the-archaeologists-dilemma}}

Day's end in Port-Vila is colored by the process of selecting which kava
bar to patronize; each imports its kava from a different island, and
friendly arguments about kava strength and quality are common. On our
return from Teouma, Bedford and I met up with an extended crew from the
national museum for kava grown on the volcanic slopes of the northern
island of Ambae, where an eruption threatened to stop shipments. Kava is
a cloudy green tonic, served in little miso bowls meant to resemble
coconut shells. The custom is to collect your shell, retire alone to the
cover of a nearby shadow, take the entirety at one draft and then spit
the particulate remnants; by nightfall, when even the city is blanketed
in thick dark, the only regular sounds are the screech of the fruit bat
and the hock of spit.

I sat in the dark next to Frederique Valentin, a French bioarchaeologist
who was an author on Reich's original Vanuatu paper; it was she who made
the final contribution that rescued the effort, the Tongan petrous bone.
As it turns out, in 2015 she submitted a manuscript to Nature that made
an almost identical argument to Reich's. She had reached the same
conclusions upon examination of the cranial morphology of the exact same
skulls, which she believed more closely resembled those of Asians than
those of Papuans. But her paper was rejected by Nature. As far as she or
many others could tell, the only difference between her conclusions and
Reich's were those of methods --- hers old, theirs shiny and new --- and
rhetorical grandeur. I asked if she thought that Reich's definitive
statements about Lapita origins were warranted.

``A small sample,'' she replied, ``is only representative of itself.''

The controversy over paleogenomics was becoming a near-ubiquitous
presence in archaeology journals, and Bedford, as an author on all three
Vanuatu papers, had recently written the introduction to
\href{https://onlinelibrary.wiley.com/doi/full/10.1002/arco.5165}{an
academic forum on the subject}, in the journal Archaeology in Oceania.
The evident differences between the two competing follow-ups put him in
a bit of a bind, because his name was on both of them. ``Both papers,''
Bedford maintained, ``arrive at a similar conclusion,'' that initial
Austronesian settlement was followed by a Papuan gene flow. But as the
introduction continued, it became increasingly clear that he could not,
in fact, at all believe that both could be right, and he tipped his hand
in favor of the Jena paper, with its emphasis on an ``incremental and
complex'' process that accorded much better with the artifactual record
as he had spent his career understanding it.

The contradictions of Bedford's introduction --- in which he said that
both papers could be right but that the complicated one was probably
more right than the simplistic one --- felt less like an equivocation
than it did a form of subtle apology. As one contributor to Bedford's
forum observed, archaeologists had told the ni-Vanuatu for decades that
they were the descendants of the Lapita voyagers; now they had to go
back and advise them to alter the commemorative postage stamps to
feature not black people but Taiwanese aboriginals. A national
self-image was not something to take lightly. ``One can only feel,'' one
forum contributor wrote, ``a collective sense of betrayal in all of
this.''

Some critics believed that any association with Reich represented a
betrayal, too, not only of the ni-Vanuatu but of anyone who believed
that culture was as powerful a human determinant as the gene. Shortly
before the publication of his book, Reich wrote an
\href{https://www.nytimes3xbfgragh.onion/2018/03/23/opinion/sunday/genetics-race.html}{Op-Ed
in The New York Times} in which he warned that the future was likely to
demonstrate some meaningful genetic differences among populations and
that we needed to be honest about such truths, lest they be abused by
racist pseudoscience. He was careful to differentiate the idea of a
genetic population from the old idea of race, which he agreed was a
social rather than biological fact. But he nonetheless gave comfort to
those who maintain that on the deepest of all levels our destiny is
written into our genetic signature. It was hard not to see that
conviction reflected in the findings of Reich's papers, which seemed to
blithely recapitulate discredited theories of Pacific expansion, making
categorical claims not only about four individual skulls but about the
shape of human history --- claims that were essentially
indistinguishable from the racialized notions of the swashbuckling
imperial era.

Younger scholars who don't think that the big, powerful labs are
exhibiting proper respect, sensitivity and historical consciousness ---
including anthropologists like Heidi Colleran, who went to great lengths
for ni-Vanuatu spit --- are thus put into impossible positions of tragic
compromise; they face the decision to spend their careers as access
mercenaries, to work with smaller outfits that get pushed aside or
scooped, or to get out of the field entirely. As Colleran would put it
to me later: ``When any fieldworker talks to collaborating communities
in the field, they are putting their professional and personal integrity
on the line, their own legitimacy, often in a completely different line
of work, for these samples. And once they are out of your hands, you
have very little control. That's a gamble for anyone doing long-term
fieldwork.'' She expressed reluctance to take part in any future studies
in which the paleogenomicists alone set the pace. Her
population-geneticist partner, Adam Powell, feels the same way. ``I
really wanted for us to do things differently,'' he told me, ``but
didn't think it would be this hard. I'm now going to focus my energies
elsewhere.''

The day after our night out at the kava bar, Colleran booked us on an
Air Vanuatu flight to the northern island of Malekula to visit a remote
village called Alpalak, which was roughly translated for me as ``the
place where if you go you will definitely die.'' There she introduced me
to Chief Jimmyson Sanhambath, who sat and drank kava with me in the
shifting shadows of a mango tree, heavy with unripe fruit. Sanhambath is
an exceptionally vital man in his late 50s, with a slender, wiry
physique, a thickly corded neck, and a long, smooth forehead and sharply
angled jaws knitted together by a trim graying mustache. Asked about the
spit that his people had given Colleran, he told me he had come to
believe that there couldn't be anything in it; spit evaporates to
nothing, after all. He admired Colleran and didn't want to trifle with
her work, he insisted, but he continued with a mischievous grin, ``They
must just be making it up.''

The next day, he took us through bamboo thickets to see some of the
oldest cave art in the region. We crouched down through a dark opening,
followed a short slope and emerged into a large, well-lit chamber; a
single banyan had snaked its way up and through a skylight opening high
above. As we passed into the midday twilight of the rear of the chamber,
Sanhambath pointed out dark handprints of a mossy jade color high up on
the smooth walls. Nearby were crisp figures with the heads of dogs and
pigs and the bodies of men; they wore unmistakable versions of the penis
sheaths associated even today with Sanhambath's community.

Archaeologists said they were made by men who ate charcoal, chewed it up
and spat it back onto the walls. The oldest dated back 2,600 years and
looked at once hauntingly archaic and vividly recent. ``They're not
Lapita,'' Sanhambath said, gesturing at the drawings, which had been
dated by radiocarbon to shortly after the Lapita period ended. ``But so
what?'' Besides, as much faith as he had in what the archaeologists said
about pottery or bones, he just couldn't bring himself to believe them
when they said these paintings were made by ancient men.

``These paintings,'' he said quietly in the cave dark, ``were made by
the spirits.''

Advertisement

\protect\hyperlink{after-bottom}{Continue reading the main story}

\hypertarget{site-index}{%
\subsection{Site Index}\label{site-index}}

\hypertarget{site-information-navigation}{%
\subsection{Site Information
Navigation}\label{site-information-navigation}}

\begin{itemize}
\tightlist
\item
  \href{https://help.nytimes3xbfgragh.onion/hc/en-us/articles/115014792127-Copyright-notice}{©~2020~The
  New York Times Company}
\end{itemize}

\begin{itemize}
\tightlist
\item
  \href{https://www.nytco.com/}{NYTCo}
\item
  \href{https://help.nytimes3xbfgragh.onion/hc/en-us/articles/115015385887-Contact-Us}{Contact
  Us}
\item
  \href{https://www.nytco.com/careers/}{Work with us}
\item
  \href{https://nytmediakit.com/}{Advertise}
\item
  \href{http://www.tbrandstudio.com/}{T Brand Studio}
\item
  \href{https://www.nytimes3xbfgragh.onion/privacy/cookie-policy\#how-do-i-manage-trackers}{Your
  Ad Choices}
\item
  \href{https://www.nytimes3xbfgragh.onion/privacy}{Privacy}
\item
  \href{https://help.nytimes3xbfgragh.onion/hc/en-us/articles/115014893428-Terms-of-service}{Terms
  of Service}
\item
  \href{https://help.nytimes3xbfgragh.onion/hc/en-us/articles/115014893968-Terms-of-sale}{Terms
  of Sale}
\item
  \href{https://spiderbites.nytimes3xbfgragh.onion}{Site Map}
\item
  \href{https://help.nytimes3xbfgragh.onion/hc/en-us}{Help}
\item
  \href{https://www.nytimes3xbfgragh.onion/subscription?campaignId=37WXW}{Subscriptions}
\end{itemize}
