Sections

SEARCH

\protect\hyperlink{site-content}{Skip to
content}\protect\hyperlink{site-index}{Skip to site index}

\href{https://www.nytimes3xbfgragh.onion/section/world/asia}{Asia
Pacific}

\href{https://myaccount.nytimes3xbfgragh.onion/auth/login?response_type=cookie\&client_id=vi}{}

\href{https://www.nytimes3xbfgragh.onion/section/todayspaper}{Today's
Paper}

\href{/section/world/asia}{Asia Pacific}\textbar{}Chinese Rights Lawyer
Swept Up in Xi's Crackdown Gets More Than 4 Years in Prison

\url{https://nyti.ms/2S8LM38}

\begin{itemize}
\item
\item
\item
\item
\item
\end{itemize}

Advertisement

\protect\hyperlink{after-top}{Continue reading the main story}

Supported by

\protect\hyperlink{after-sponsor}{Continue reading the main story}

\hypertarget{chinese-rights-lawyer-swept-up-in-xis-crackdown-gets-more-than-4-years-in-prison}{%
\section{Chinese Rights Lawyer Swept Up in Xi's Crackdown Gets More Than
4 Years in
Prison}\label{chinese-rights-lawyer-swept-up-in-xis-crackdown-gets-more-than-4-years-in-prison}}

\includegraphics{https://static01.graylady3jvrrxbe.onion/images/2019/01/29/world/29china-lawyer-1/merlin_149888445_c298b695-9726-4ff6-b983-cd6081bfadfe-articleLarge.jpg?quality=75\&auto=webp\&disable=upscale}

By
\href{https://www.nytimes3xbfgragh.onion/by/javier-c-hernandez}{Javier
C. Hernández}

\begin{itemize}
\item
  Jan. 28, 2019
\item
  \begin{itemize}
  \item
  \item
  \item
  \item
  \item
  \end{itemize}
\end{itemize}

\href{https://cn.nytimes3xbfgragh.onion/china/20190129/china-wang-quanzhang-human-rights/}{阅读简体中文版}\href{https://cn.nytimes3xbfgragh.onion/china/20190129/china-wang-quanzhang-human-rights/zh-hant/}{閱讀繁體中文版}

BEIJING --- An outspoken Chinese human rights lawyer was sentenced to
four and a half years in prison on Monday, the last to be prosecuted
among hundreds of legal activists who had been rounded up in a crackdown
in 2015.

The lawyer, Wang Quanzhang, was found guilty of ``subversion of state
power,'' the No. 2 Intermediate People's Court of Tianjin
\href{http://tj2zy.chinacourt.org/article/detail/2019/01/id/3716862.shtml}{said
on its website}. That charge is usually applied to critics of the ruling
Communist Party who are accused of organizing political challenges.

Other lawyers and activists who had been picked up in
\href{https://www.nytimes3xbfgragh.onion/2015/07/23/world/asia/china-crackdown-human-rights-lawyers.html?action=click\&module=RelatedCoverage\&pgtype=Article\&region=Footer}{an
expansive campaign} by the government that began in the summer of 2015
were either released or put on trial and sentenced. But Mr. Wang, 42,
had been held for nearly three and a half years before he faced charges
\href{http://tj2zy.chinacourt.org/article/detail/2019/01/id/3716862.shtml}{in
a closed trial} in Tianjin on Dec. 26 that even his wife was barred from
attending.

The crackdown, which included
\href{https://www.nytimes3xbfgragh.onion/2016/08/06/world/asia/china-trial-activists-lawyers.html}{televised
show trials} in which lawyers confessed to plotting to overthrow the
government and working on behalf of foreign forces, is a part of
President Xi Jinping's efforts to obliterate threats to the party's
control.

Mr. Wang's wife, Li Wenzu, condemned the decision on Monday, accusing
the police, prosecutors and judge of committing crimes by detaining her
husband for more than three years.

``I will continue to fight for the rights of Wang Quanzhang,'' Ms. Li
said in a
\href{https://twitter.com/709liwenzu/status/1089752511773270016}{Twitter
post}. ``I will look after our child and wait for Wang Quanzhang to come
home.''

Calls to Ms. Li's mobile phone went unanswered. It was unclear whether
Mr. Wang would appeal the decision. Since Mr. Wang has been in custody
since August 2015, it is possible he could be released next year, legal
experts said.

Mr. Wang was one of most prominent figures to be swept up in the
campaign against rights activists. As a lawyer at the Beijing Fengrui
law firm, he represented members of a banned religious group,
\href{https://www.nytimes3xbfgragh.onion/topic/organization/falun-gong}{Falun
Gong}, and helped train legal activists.

\includegraphics{https://static01.graylady3jvrrxbe.onion/images/2018/12/27/world/29china-lawyer-3/merlin_148497558_e223097d-047e-49e2-9467-d71237544ea3-articleLarge.jpg?quality=75\&auto=webp\&disable=upscale}

Prosecutors had accused Mr. Wang of ``stirring up trouble'' and
colluding with foreign-funded groups.

Human rights advocates and Western governments have raised concerns over
the Chinese government's handling of the case. Mr. Wang's relatives and
friends have not been allowed to see him.

Germany's human rights commissioner, Bärbel Kofler, on Monday called the
sentencing ``completely incomprehensible.''

``Wang Quanzhang has done nothing but advocate within the legal
framework as a lawyer in China for disenfranchised citizens and
politically persecuted people,'' she said in a statement.

There are also concerns that he might have faced abuse while in prison.
Several lawyers who were detained in 2015 have said they were abused
while in custody.

``People are saying that he is too stubborn for his own good and that
he's not going to cave in,'' said Eva Pils, a professor of law at King's
College London who studies Chinese rights lawyers and knows Mr. Wang.
``You just have to worry about the basic state of his mental and
physical health.''

The government has continued its campaign against human rights
advocates, disbarring lawyers who take on cases the party sees as a
threat and keeping others under surveillance.

Experts said that Mr. Wang's sentencing would most likely deepen
concerns among China's legal rights advocates, a small but daring group
of lawyers who help dissidents, religious leaders, aggrieved farmers and
others fight everyday injustices.

Doriane Lau, a China researcher at Amnesty International in Hong Kong,
called Mr. Wang's trial a ``sham.''

``He is being imprisoned solely for doing his job,'' she said. ``This
will have a chilling effect on the many human rights lawyers in China
who are still fighting very hard for justice.''

Advertisement

\protect\hyperlink{after-bottom}{Continue reading the main story}

\hypertarget{site-index}{%
\subsection{Site Index}\label{site-index}}

\hypertarget{site-information-navigation}{%
\subsection{Site Information
Navigation}\label{site-information-navigation}}

\begin{itemize}
\tightlist
\item
  \href{https://help.nytimes3xbfgragh.onion/hc/en-us/articles/115014792127-Copyright-notice}{©~2020~The
  New York Times Company}
\end{itemize}

\begin{itemize}
\tightlist
\item
  \href{https://www.nytco.com/}{NYTCo}
\item
  \href{https://help.nytimes3xbfgragh.onion/hc/en-us/articles/115015385887-Contact-Us}{Contact
  Us}
\item
  \href{https://www.nytco.com/careers/}{Work with us}
\item
  \href{https://nytmediakit.com/}{Advertise}
\item
  \href{http://www.tbrandstudio.com/}{T Brand Studio}
\item
  \href{https://www.nytimes3xbfgragh.onion/privacy/cookie-policy\#how-do-i-manage-trackers}{Your
  Ad Choices}
\item
  \href{https://www.nytimes3xbfgragh.onion/privacy}{Privacy}
\item
  \href{https://help.nytimes3xbfgragh.onion/hc/en-us/articles/115014893428-Terms-of-service}{Terms
  of Service}
\item
  \href{https://help.nytimes3xbfgragh.onion/hc/en-us/articles/115014893968-Terms-of-sale}{Terms
  of Sale}
\item
  \href{https://spiderbites.nytimes3xbfgragh.onion}{Site Map}
\item
  \href{https://help.nytimes3xbfgragh.onion/hc/en-us}{Help}
\item
  \href{https://www.nytimes3xbfgragh.onion/subscription?campaignId=37WXW}{Subscriptions}
\end{itemize}
