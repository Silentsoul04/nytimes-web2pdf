Sections

SEARCH

\protect\hyperlink{site-content}{Skip to
content}\protect\hyperlink{site-index}{Skip to site index}

\href{https://www.nytimes3xbfgragh.onion/section/books}{Books}

\href{https://myaccount.nytimes3xbfgragh.onion/auth/login?response_type=cookie\&client_id=vi}{}

\href{https://www.nytimes3xbfgragh.onion/section/todayspaper}{Today's
Paper}

\href{/section/books}{Books}\textbar{}February's Book Club Pick: `The
Wife,' by Meg Wolitzer

\url{https://nyti.ms/2HCH3SX}

\begin{itemize}
\item
\item
\item
\item
\item
\end{itemize}

Advertisement

\protect\hyperlink{after-top}{Continue reading the main story}

Supported by

\protect\hyperlink{after-sponsor}{Continue reading the main story}

Now Read This

\hypertarget{februarys-book-club-pick-the-wife-by-meg-wolitzer}{%
\section{February's Book Club Pick: `The Wife,' by Meg
Wolitzer}\label{februarys-book-club-pick-the-wife-by-meg-wolitzer}}

\includegraphics{https://static01.graylady3jvrrxbe.onion/images/2019/01/30/books/NowReadThis-February-Wolitzer/merlin_135621447_fd6da841-97de-4797-b240-364a3e4837d4-articleLarge.jpg?quality=75\&auto=webp\&disable=upscale}

By Claire Dederer

\begin{itemize}
\item
  Jan. 30, 2019
\item
  \begin{itemize}
  \item
  \item
  \item
  \item
  \item
  \end{itemize}
\end{itemize}

\textbf{THE WIFE}\\
By Meg Wolitzer

\emph{{[}This is an excerpt from the original book review,}
\href{https://www.nytimes3xbfgragh.onion/2003/04/20/books/in-the-shadow-of-the-big-boys.html}{\emph{``In
the Shadow of the Big Boys.''}}\emph{{]}}

Here are three words that land with a thunk: ``gender,'' ``writing'' and
``identity.'' Yet in ``The Wife,'' Meg Wolitzer has fashioned a
light-stepping, streamlined novel from just these dolorous,
bitter-sounding themes. Maybe that's because she's set them all
smoldering: Rage might be the signature emotion of the powerless, but in
Wolitzer's hands, rage is also very funny.

As the book opens, Joe and Joan Castleman are on a plane to Helsinki,
where Joe is to receive a prestigious literary prize. Joan, the
narrator, tells us that her husband is ``one of those men who own the
world'' and describes him with a nice mixture of wifely regard and
satirical distance: ``There are many varieties of this kind of man: Joe
was the writer version, a short, wound-up, slack-bellied novelist who
almost never slept, who loved to consume runny cheeses and whiskey and
wine \ldots{} who derived much of his style from `The Dylan Thomas
Handbook of Personal Hygiene and Etiquette.' ``

Image

``The Wife'' is also a movie for which Glenn Close has received an Oscar
nomination for best actress.

The story of the Castleman marriage is told in a series of flashbacks.
Joan, painfully alive to the hackneyed nature of their match, recalls
their early days: ``It kills me to say it, but I was his student when we
met. There we were in 1956, a typical couple, Joe intense and focused
and tweedy, me a fluttering budgie circling him again and again.'' The
entire novel, in fact, is a kind of paean to the notion that clichés are
clichés because they're often true. The pathetic thing about the younger
version of Joan is not that her story is unique; it's the fact that
there were --- and still are --- so many Joans, circling like so many
budgies.

A promising writer, Joan abandons her own career in the service of her
husband's. Joe, meanwhile, roars through life. He chases other women,
drinks vats of booze, torments himself over his literary stature and
happily ignores his children. In relating all this, Wolitzer deploys a
calm, seamless humor not found in her previous novels. The jokes don't
barge in and tap us on the shoulder as they did in ``This Is Your Life''
or ``Surrender, Dorothy.'' Instead, they gradually accumulate, creating
a rueful, sardonic atmosphere. ``Wives,'' Joan tells us in a typical
aside, ``are the sad sacks of any writers' conference.'' She is just as
sharp on Joe's self-involvement:

``The men who own the world don't get to do that by being magnanimous
and overly interested in other people. They get to do it by taking care
of themselves along the way. They stoke the fire of their own
reputations, and sometimes other people come by, asking: What's that
you're doing there?

``Oh, stoking the fire of my reputation.

``Can I help?

``Certainly. Go get some wood.''

Eventually, Joan lets us in on the Castlemans' secret. And once we know
the truth, we want to go back and examine the carapace of justification,
blind-eye-turning and bitter regret that is Joan's history as a wife.
The book represents a real step forward for Wolitzer, and its success
lies in its reticence. Joan defiantly leaves us wanting more, whereas
Wolitzer's other heroines left us wanting maybe a teensy bit less. As a
portrait of deception, this small, intelligently made novel rivals ``The
Dangerous Husband,'' by Jane Shapiro, and John Lanchester's ``Debt to
Pleasure.''

But if ``The Wife'' is a puzzle and an entertainment, it's also a near
heartbreaking document of feminist realpolitik. In the modernist milieu
the Castlemans inhabit, to be a woman writer is automatically to be
lesser, to produce work faintly praised as ``powerful in its own
right.'' Oh, there are exceptions, notably Mary McCarthy. She appears
here as a kind of Lady Writer fetish object that the male writers finger
when they want to demonstrate an appreciation of the weaker sex. ``But
what,'' Joan asks, ``happened to the talented women who lacked sharp
cheekbones or an ease in the universe?'' She herself is the answer to
this question. The central event of the book is a nonevent: the moment
when Joan Castleman gave up her own writing to be a wife.

Advertisement

\protect\hyperlink{after-bottom}{Continue reading the main story}

\hypertarget{site-index}{%
\subsection{Site Index}\label{site-index}}

\hypertarget{site-information-navigation}{%
\subsection{Site Information
Navigation}\label{site-information-navigation}}

\begin{itemize}
\tightlist
\item
  \href{https://help.nytimes3xbfgragh.onion/hc/en-us/articles/115014792127-Copyright-notice}{©~2020~The
  New York Times Company}
\end{itemize}

\begin{itemize}
\tightlist
\item
  \href{https://www.nytco.com/}{NYTCo}
\item
  \href{https://help.nytimes3xbfgragh.onion/hc/en-us/articles/115015385887-Contact-Us}{Contact
  Us}
\item
  \href{https://www.nytco.com/careers/}{Work with us}
\item
  \href{https://nytmediakit.com/}{Advertise}
\item
  \href{http://www.tbrandstudio.com/}{T Brand Studio}
\item
  \href{https://www.nytimes3xbfgragh.onion/privacy/cookie-policy\#how-do-i-manage-trackers}{Your
  Ad Choices}
\item
  \href{https://www.nytimes3xbfgragh.onion/privacy}{Privacy}
\item
  \href{https://help.nytimes3xbfgragh.onion/hc/en-us/articles/115014893428-Terms-of-service}{Terms
  of Service}
\item
  \href{https://help.nytimes3xbfgragh.onion/hc/en-us/articles/115014893968-Terms-of-sale}{Terms
  of Sale}
\item
  \href{https://spiderbites.nytimes3xbfgragh.onion}{Site Map}
\item
  \href{https://help.nytimes3xbfgragh.onion/hc/en-us}{Help}
\item
  \href{https://www.nytimes3xbfgragh.onion/subscription?campaignId=37WXW}{Subscriptions}
\end{itemize}
