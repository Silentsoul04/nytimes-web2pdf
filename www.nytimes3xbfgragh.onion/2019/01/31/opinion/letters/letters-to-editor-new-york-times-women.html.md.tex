Sections

SEARCH

\protect\hyperlink{site-content}{Skip to
content}\protect\hyperlink{site-index}{Skip to site index}

\href{https://myaccount.nytimes3xbfgragh.onion/auth/login?response_type=cookie\&client_id=vi}{}

\href{https://www.nytimes3xbfgragh.onion/section/todayspaper}{Today's
Paper}

\href{/section/opinion}{Opinion}\textbar{}A Woman's Plea: Let's Raise
Our Voices!

\url{https://nyti.ms/2SfEsCN}

\begin{itemize}
\item
\item
\item
\item
\item
\item
\end{itemize}

Advertisement

\protect\hyperlink{after-top}{Continue reading the main story}

\href{/section/opinion}{Opinion}

Supported by

\protect\hyperlink{after-sponsor}{Continue reading the main story}

letter

\hypertarget{a-womans-plea}{%
\section{A Woman's Plea:}\label{a-womans-plea}}

Let's Raise Our Voices!

A reader points out that letters to the editor ``skew male.'' She's
right. We are determined to publish diverse voices and views, but we
need your help.

Jan. 31, 2019

\begin{itemize}
\item
\item
\item
\item
\item
\item
\end{itemize}

\includegraphics{https://static01.graylady3jvrrxbe.onion/images/2019/02/03/opinion/03LWomen/03LWomen-articleLarge-v2.gif?quality=75\&auto=webp\&disable=upscale}

\textbf{To the Editor:}

In 1855, Nathaniel Hawthorne wrote to his publisher, ``America is now
wholly given over to a damned mob of scribbling women.'' Although he was
referring specifically to sentimental novelists, his letter expressed
the larger belief that women's writing was not worth reading or
publishing, that their words and ideas didn't matter, and that their
work was, to use the language of Hawthorne, ``trash.''

As a historian, I see this playing out not only in the antebellum
period, but also in the postwar era when I read letters to the editor.
As I scan through various national newspapers, day after day, year after
year, I find myself hoping that someday, \emph{eventually,} women will
be represented proportionally. I am always disappointed; they always
skew male.

Perhaps Hawthorne's disdain for scribbling women is not such distant
history.

This problem is especially concerning because unlike an Op-Ed --- where
the writer presumably has some expertise in the subject matter ---
anybody can submit a letter to the editor. It is, I'd argue, the most
democratic section of the paper because children and adults, billionaire
philanthropists and minimum-wage workers, and people of all genders can
contribute. Each has an equal opportunity to express her or his thoughts
and participate in a robust debate in the public sphere. Therefore, I'm
troubled that in 2019, The New York Times struggles to find women's
letters that are worthy of publication.

When I first inquired as to why so few women were writing, I was told
that there aren't formal statistics on the number of women submitting
letters, but that a large majority come from men. Gail Collins provided
a similar explanation when she became the first woman editor of the
editorial page at The Times in 2001 and started looking into this
problem. She found that in letters to the editor and Op-Ed submissions,
``the preponderance of men was off the charts.''

But still, causality remains murky. Are women not writing because they
don't see themselves represented? What role does implicit bias play? In
the absence of formal research, it's hard to know.

The Times ** could put in place a quota for women's letters, ensuring
that the number of women published each week is roughly proportional to
the number of women in the population. But given the overwhelming
backlash against affirmative action, I am not optimistic that this is a
realistic possibility.

So while I would like to see more institutional changes, in the short
term I want to encourage women to write more letters to the editor. The
poet Audre Lorde described writing as a political act, the way ``we
predicate our hopes and dreams toward survival and change.''

Similarly, submitting a letter to the editor says that in a society that
refuses to acknowledge your full humanity, you insist on it. It is
asserting that your ideas and words deserve an audience in a world that
has historically devalued them. It is accepting that you most likely
will never receive external validation for your efforts save for an
automated email thanking you for your letter.

You will never know if your letter wasn't published because you were
Kimberly and not Karl, or if your letter was boring, or if it had
absolutely nothing to do with the merits of what you wrote. As Ta-Nehisi
Coates reflected in ``We Were Eight Years in Power'': ``My reasons for
writing had to be my own, divorced from expectation. There would be no
reward.''

I used to think the reward would be the individual accomplishment of an
editor selecting my letter as worthy of publication. But now I know the
reward would be for tomorrow and the next day and the next, to open up
The New York Times --- in fact, letters pages in any national newspaper
--- and to see nasty, scribbling women from all over the country sharing
their ideas and having their thoughts equally represented.

\href{https://americanstudies.columbian.gwu.edu/kimberly-probolus}{KIMBERLY
PROBOLUS}\\
Washington\\
\emph{The writer is a Ph.D. candidate in American studies at George
Washington University.}

\hypertarget{the-editors-respond-we-hear-you}{%
\subsection{The Editors Respond: We Hear
You}\label{the-editors-respond-we-hear-you}}

Ms. Probolus is right. Even before we received her note, we'd wrestled
with the fact that women have long been underrepresented on the letters
page. By our rough estimate, women account for a quarter to a third of
submissions --- although women do tend to write in greater numbers about
issues like education, health, gender and children.

This gender disparity problem is not unique to the letters page. Online
comments on our articles and the unsolicited Op-Ed submissions we
receive skew heavily male. Nor is this issue unique to The Times.

The lack of women's voices is an industrywide phenomenon, as documented
in a
\href{https://www.poynter.org/reporting-editing/2011/why-women-dont-contribute-to-opinion-pages-as-often-as-men-what-we-can-do-about-it/}{2011
article in Poynter.org} (``Why women don't contribute to opinion pages
as often as men \& what we can do about it''). It is reflected in
efforts such as the \href{https://www.theopedproject.org/}{Op-Ed
Project}, founded in 2008 ``to increase the number of women thought
leaders in key commentary forums to a tipping point.''

As for our letters page, we make our selections regardless of gender.
But we are sensitive to gender imbalance, and as editors of a space
dedicated to readers' voices, we are determined to have it reflect more
closely society as a whole. Going forward, we're committing ourselves to
work toward a goal of parity on a weekly basis. We'll report back on our
progress in February 2020.

But we need your help. So we want to urge women --- and anyone else who
feels underrepresented --- to write in (here is a
\href{https://help.nytimes3xbfgragh.onion/hc/en-us/articles/115014925288-How-to-submit-a-letter-to-the-editor}{guide}).

Traditionally we have chosen letters that were sent by email
(\href{mailto:letters@NYTimes.com}{\nolinkurl{letters@NYTimes.com}}) or
postal mail. From now on, in addition to those sources, we'll seek a new
pool of writers by reaching out in newsletters, the
\href{https://www.nytimes3xbfgragh.onion/section/reader-center}{Reader
Center}, Facebook groups and other social media.

Just as we were thrilled to hear from Kimberly Probolus, we'd love to
hear from you.

\href{https://www.nytimes3xbfgragh.onion/2004/05/23/opinion/23READ.html}{THOMAS
FEYER}, \emph{Letters Editor}\\
SUSAN MERMELSTEIN, \emph{Staff Editor}

\hypertarget{please-write-to-us}{%
\subsection{Please Write to Us}\label{please-write-to-us}}

We'd like to hear your thoughts about why more women don't write letters
and comments, and how to remedy this. Let us know in the comments
section of this article.

Or send an email to
\href{mailto:letters@NYTimes.com}{\nolinkurl{letters@NYTimes.com}}\emph{.}
Please keep your letters to 200 words or less. Include your name,
city/state and contact information, and put ``women'' in the subject
line.

Advertisement

\protect\hyperlink{after-bottom}{Continue reading the main story}

\hypertarget{site-index}{%
\subsection{Site Index}\label{site-index}}

\hypertarget{site-information-navigation}{%
\subsection{Site Information
Navigation}\label{site-information-navigation}}

\begin{itemize}
\tightlist
\item
  \href{https://help.nytimes3xbfgragh.onion/hc/en-us/articles/115014792127-Copyright-notice}{©~2020~The
  New York Times Company}
\end{itemize}

\begin{itemize}
\tightlist
\item
  \href{https://www.nytco.com/}{NYTCo}
\item
  \href{https://help.nytimes3xbfgragh.onion/hc/en-us/articles/115015385887-Contact-Us}{Contact
  Us}
\item
  \href{https://www.nytco.com/careers/}{Work with us}
\item
  \href{https://nytmediakit.com/}{Advertise}
\item
  \href{http://www.tbrandstudio.com/}{T Brand Studio}
\item
  \href{https://www.nytimes3xbfgragh.onion/privacy/cookie-policy\#how-do-i-manage-trackers}{Your
  Ad Choices}
\item
  \href{https://www.nytimes3xbfgragh.onion/privacy}{Privacy}
\item
  \href{https://help.nytimes3xbfgragh.onion/hc/en-us/articles/115014893428-Terms-of-service}{Terms
  of Service}
\item
  \href{https://help.nytimes3xbfgragh.onion/hc/en-us/articles/115014893968-Terms-of-sale}{Terms
  of Sale}
\item
  \href{https://spiderbites.nytimes3xbfgragh.onion}{Site Map}
\item
  \href{https://help.nytimes3xbfgragh.onion/hc/en-us}{Help}
\item
  \href{https://www.nytimes3xbfgragh.onion/subscription?campaignId=37WXW}{Subscriptions}
\end{itemize}
