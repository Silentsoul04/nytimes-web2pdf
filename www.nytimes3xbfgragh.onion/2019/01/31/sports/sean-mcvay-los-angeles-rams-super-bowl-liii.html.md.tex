Sections

SEARCH

\protect\hyperlink{site-content}{Skip to
content}\protect\hyperlink{site-index}{Skip to site index}

\href{/section/sports}{Sports}\textbar{}Sean McVay's Age Is Just a
Number. A Small One.

\url{https://nyti.ms/2SdEATn}

\begin{itemize}
\item
\item
\item
\item
\item
\end{itemize}

\includegraphics{https://static01.graylady3jvrrxbe.onion/images/2019/02/01/sports/01mcvay-web1/merlin_149915511_af277fe1-7eaa-415f-a205-00bef37f4665-articleLarge.jpg?quality=75\&auto=webp\&disable=upscale}

Super Bowl liii

\hypertarget{sean-mcvays-age-is-just-a-number-a-small-one}{%
\section{Sean McVay's Age Is Just a Number. A Small
One.}\label{sean-mcvays-age-is-just-a-number-a-small-one}}

At 33, McVay will be the youngest head coach in Super Bowl history. To
put it mildly, the rest of the N.F.L. has noticed.

Coach Sean McVay has gone a combined 24-8 in his first two seasons with
the Los Angeles Rams.Credit...Kirby Lee/USA Today Sports, via Reuters

Supported by

\protect\hyperlink{after-sponsor}{Continue reading the main story}

\href{https://www.nytimes3xbfgragh.onion/by/benjamin-hoffman}{\includegraphics{https://static01.graylady3jvrrxbe.onion/images/2018/10/17/multimedia/author-benjamin-hoffman/author-benjamin-hoffman-thumbLarge.png}}

By
\href{https://www.nytimes3xbfgragh.onion/by/benjamin-hoffman}{Benjamin
Hoffman}

\begin{itemize}
\item
  Jan. 31, 2019
\item
  \begin{itemize}
  \item
  \item
  \item
  \item
  \item
  \end{itemize}
\end{itemize}

ATLANTA --- Mozart was composing music at 5. Sergey Karjakin became a
chess grandmaster at 12. Blaise Pascal invented the mechanical
calculator as a teenager. And Sean McVay led the Los Angeles Rams to the
Super Bowl at 33.

One of those four does not really qualify as a prodigy. But that has not
stopped virtually everyone in the football world from throwing the word
around in the last few weeks in reference to McVay, a second-year head
coach who has become the toast of the N.F.L. in the lead-up to
\href{https://www.nytimes3xbfgragh.onion/2019/02/01/sports/tom-brady-patriots.html?action=click\&module=Top+Stories\&pgtype=Homepage}{Super
Bowl LIII}.

As with most things Super Bowl-related, McVay-as-prodigy is a narrative
born of something true and then expanded to an uncomfortable degree.

The rush to replicate McVay's success began in earnest in January, when
a league of teams that have often relied on retread coaches and
established hierarchies suddenly was in need of its next boy genius.

The Green Bay Packers created a McVay coaching tree, of sorts, by hiring
his former offensive coordinator, Matt LaFleur, who is 39, to be their
head coach. The Cleveland Browns hired Freddie Kitchens, a 44-year-old
offensive guru, to lead their staff. And in an extreme example of McVay
mania, the Arizona Cardinals' website, in its official announcement
about the hiring of the 39-year-old Kliff Kingsbury as the team's new
head coach, noted that Kingsbury was friends with McVay.

That the two had never coached together seemed not to matter, though
after some blowback, the team
\href{https://www.azcardinals.com/news/cardinals-hire-kliff-kingsbury-as-head-coach}{amended
the announcement} to remove the part about their friendship while still
noting that McVay is a young ``offensive genius who has become the
blueprint of many of the new coaching hires around the N.F.L.'' and had
recently offered Kingsbury a chance to join the Rams as an offensive
consultant.

``It's certainly humbling and flattering,'' McVay said of the hiring
trend, ``but I think more than anything it's a reflection of the success
the Rams have had.''

In the end, teams may discover that replicating McVay will be difficult.

Young, offensive-minded coaches are ``the vogue thing right now,'' said
Brian Billick, the former Baltimore Ravens coach. ``Some will be
successful, and some won't. Eighty percent will probably fail.''

Very few coaches have engineered the kind of turnaround that the Rams
have made in two seasons under McVay, who, despite having coached in the
N.F.L. since 2008, is younger than one of his players, and two of the
Patriots.

He began his head coaching career by taking the Rams from a 4-12 record
with a league-low average of 14 points a game in 2016 to 11-5 with a
league-leading average of 29.9 points in 2017. Then McVay proved that
was no fluke by improving a bit more this season, to 13-3 with an
average of 32.9 points a game.

\includegraphics{https://static01.graylady3jvrrxbe.onion/images/2019/02/01/sports/01mcvay-web2/merlin_148889823_82899aff-bed2-4e9f-a353-a1781e39712a-articleLarge.jpg?quality=75\&auto=webp\&disable=upscale}

\hypertarget{we-not-me}{%
\subsection{`We, Not Me'}\label{we-not-me}}

McVay, who has already become a master of coaching catchphrases and
clichés, has regularly tried to expand the spotlight to the other
coaches around him. He frequently uses the phrase ``we, not me'' to
describe his team's approach.

``One of the things that's so important is surrounding yourself with
great people,'' he said, ``specifically in those other areas where they
have leadership roles.''

His call for inclusiveness extends to coaches of all ages. His first
hire for the Rams was Wade Phillips, a 71-year-old defensive coordinator
who had been coaching in the N.F.L. for 10 years when McVay was born.

Phillips said he was somewhat familiar with McVay, then the offensive
coordinator of the Washington Redskins, before his hiring, but that it
was a call from Phillips's son, Wes, who had replaced McVay as
Washington's tight ends coach, that set the whole thing in motion. Wes
Phillips asked his father if he would be interested in joining a McVay
staff should the opportunity arise.

``I said, `You know he's 30 years old; you think he's going to get
one?''' Wade Phillips said on Thursday. ``He said, `Dad, if he
interviews for one, he's going to get it.'''

He added that the first call from McVay was straightforward: ``He said,
`Would you interested in going with me if I got a job?' and I said,
`Yeah, sure.'''

But even with the Phillips hiring serving as a nod to his elders, McVay
seems content to play up his own youthfulness. Short and fit by N.F.L.
coaching standards and naturally baby-faced, he spikes his hair up with
product and trims his beard into a perfect 5 o'clock shadow.

He talks glowingly about watching Tom Brady win the 2002 Super Bowl ---
when McVay was 16 --- and revels in getting congratulatory text messages
from Patriots Coach Bill Belichick, who is twice his age.

And in what served as the perfect visual for the N.F.L.'s only
millennial head coach, a video circulated recently in which Ted Rath, a
Rams assistant, repeatedly pulled McVay out of the way of officials on
the sideline, with the whole thing set to music. McVay is hardly the
only coach to have a spotter charged with making sure he is out of the
way, but the coach's seeming obliviousness to Rath's maneuvers created a
scene that, depending on your age, reflected either a lack of
situational awareness or a remarkable ability to focus.

\hypertarget{the-prodigies-who-came-before-him}{%
\subsection{The Prodigies Who Came Before
Him}\label{the-prodigies-who-came-before-him}}

When McVay signed on to become head coach of the Rams, he was 30 years
11 months old, which made him the youngest person in that role in the
N.F.L. since 1938, when Art Lewis, at 27, was hired by the Cleveland
Rams (a team that would eventually move to Los Angeles, then to St.
Louis, and then back to Los Angeles).

Before McVay, the youngest N.F.L. head coach of the modern era was Lane
Kiffin, who was hired by the Oakland Raiders at 31 years 8 months old
(and then fired 20 months later), and the youngest head coach to lead
his team to a Super Bowl was Mike Tomlin of the Pittsburgh Steelers, who
was 36. For an idea of how unusual it is for coaches to reach the
league's pinnacle that early, the previous seven Super Bowls have been
won by coaches who were 50 and older.

Billick is one of the few people who can relate to McVay's experience of
being the N.F.L.'s hot new thing. Hired to coach the Ravens in 1999,
after coordinating a Minnesota Vikings offense that averaged an
eye-popping 34.8 points a game, Billick turned the hype into results by
winning a Super Bowl in his second season, a feat McVay could match on
Sunday.

Billick called McVay a brilliant coach, praising his ability to change
up looks at the line of scrimmage. But he could not quite stop himself
from diluting the praise by pointing out that there is relatively little
difference between the offense McVay runs and the one employed by
Belichick and New England's offensive coordinator, Josh McDaniels.

McDaniels also took a turn as the N.F.L.'s coveted young coach. Ten
years ago, he left the Patriots at 32 to become the Denver Broncos' head
coach, a job he lost 12 games into his second season.

The similarities between him and McVay led to an amusing scene on Monday
involving a reporter and McDaniels.

``Everyone continues to talk about Sean McVay and his young mind, and
they seem to have forgotten about you,'' the reporter said before
asking, ``What kind of advice do you have for him?''

McDaniels, who is still just 42, clearly had anticipated questions about
McVay, even ones as harshly worded as that. He jumped right in with
praise.

``There's nothing you can say about what Sean's done that's not in some
way shape or form a superlative,'' McDaniels said. ``He's pretty
special. I think we need to appreciate him for what he is and who he is
and what he's done at an early age.''

Image

McVay spent seven seasons as a coach with the Washington Redskins,
serving as the team's offensive coordinator from 2014 to
2016.Credit...Alex Brandon/Associated Press

\hypertarget{the-road-to-los-angeles}{%
\subsection{The Road to Los Angeles}\label{the-road-to-los-angeles}}

The grandson of John McVay, who was a Giants head coach and an executive
who helped shape the dynastic San Francisco 49ers of the 1980s and
1990s, Sean McVay was a star quarterback in high school --- he was named
Georgia's 4A offensive player of the year over Calvin Johnson in 2003
--- but a transition to wide receiver at Miami University did not lead
to any professional opportunities.

His first stop in the N.F.L. was as an assistant wide receivers coach
for the Tampa Bay Buccaneers in 2008. He worked under Jon Gruden and
Gruden's younger brother, Jay, who was an offensive assistant. Aside
from his grandfather, McVay said, those two men have had the largest
impact on his career.

``Jon Gruden taught me the foundation of what I know about this game,''
he said. ``Took me under his arm, taught me to see it from a big-picture
perspective.''

McVay went on to coach tight ends for the Redskins, and eventually
worked again for Jay Gruden, who became Washington's head coach in 2014.
He promptly promoted McVay, at 28, to offensive coordinator.

After developing Kirk Cousins as a quarterback, McVay was hired by the
Rams to revamp their moribund franchise.

In the announcement of McVay's hiring, Stan Kroenke, the Rams' owner,
set the bar extremely high for success.

``We are confident in his vision to make this team a consistent
winner,'' Kroenke said, ``and we will all continue to work together to
achieve our ultimate goal --- bringing a Rams Super Bowl championship
home to Los Angeles.''

Come Sunday, McVay will either be the youngest head coach to win a Super
Bowl or the youngest head coach to lose one. But regardless of whether
you want to call him a prodigy, an offensive guru, or simply a coach,
McVay will have to watch his back. In a copycat league, some other team
can always go younger.

\emph{Naila-Jean Meyers contributed reporting.}

Advertisement

\protect\hyperlink{after-bottom}{Continue reading the main story}

\hypertarget{site-index}{%
\subsection{Site Index}\label{site-index}}

\hypertarget{site-information-navigation}{%
\subsection{Site Information
Navigation}\label{site-information-navigation}}

\begin{itemize}
\tightlist
\item
  \href{https://help.nytimes3xbfgragh.onion/hc/en-us/articles/115014792127-Copyright-notice}{©~2020~The
  New York Times Company}
\end{itemize}

\begin{itemize}
\tightlist
\item
  \href{https://www.nytco.com/}{NYTCo}
\item
  \href{https://help.nytimes3xbfgragh.onion/hc/en-us/articles/115015385887-Contact-Us}{Contact
  Us}
\item
  \href{https://www.nytco.com/careers/}{Work with us}
\item
  \href{https://nytmediakit.com/}{Advertise}
\item
  \href{http://www.tbrandstudio.com/}{T Brand Studio}
\item
  \href{https://www.nytimes3xbfgragh.onion/privacy/cookie-policy\#how-do-i-manage-trackers}{Your
  Ad Choices}
\item
  \href{https://www.nytimes3xbfgragh.onion/privacy}{Privacy}
\item
  \href{https://help.nytimes3xbfgragh.onion/hc/en-us/articles/115014893428-Terms-of-service}{Terms
  of Service}
\item
  \href{https://help.nytimes3xbfgragh.onion/hc/en-us/articles/115014893968-Terms-of-sale}{Terms
  of Sale}
\item
  \href{https://spiderbites.nytimes3xbfgragh.onion}{Site Map}
\item
  \href{https://help.nytimes3xbfgragh.onion/hc/en-us}{Help}
\item
  \href{https://www.nytimes3xbfgragh.onion/subscription?campaignId=37WXW}{Subscriptions}
\end{itemize}
