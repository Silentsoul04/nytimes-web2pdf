\href{/section/books}{Books}\textbar{}Valeria Luiselli, at Home in Two
Worlds

\url{https://nyti.ms/2DZJ3RB}

\begin{itemize}
\item
\item
\item
\item
\item
\end{itemize}

\includegraphics{https://static01.graylady3jvrrxbe.onion/images/2019/02/08/books/08luiselli3/merlin_149909922_8f3bde97-ea48-4900-b990-630d517cf447-articleLarge.jpg?quality=75\&auto=webp\&disable=upscale}

Sections

\protect\hyperlink{site-content}{Skip to
content}\protect\hyperlink{site-index}{Skip to site index}

Profile

\hypertarget{valeria-luiselli-at-home-in-two-worlds}{%
\section{Valeria Luiselli, at Home in Two
Worlds}\label{valeria-luiselli-at-home-in-two-worlds}}

The Mexican writer has made a name for herself with experimental books
and essays. Her latest, ``Lost Children Archive,'' is a road trip novel
that shows off her intellectual sensibilities.

``I have never written a novel that just sort of springs from the head
of Zeus, from an absolute space of fiction,'' said Valeria Luiselli. ``I
always begin my work documenting my everyday.''Credit...Devin Yalkin for
The New York Times

Supported by

\protect\hyperlink{after-sponsor}{Continue reading the main story}

By
\href{https://www.nytimes3xbfgragh.onion/by/concepcion-de-leon}{Concepción
de León}

\begin{itemize}
\item
  Feb. 7, 2019
\item
  \begin{itemize}
  \item
  \item
  \item
  \item
  \item
  \end{itemize}
\end{itemize}

\href{https://www.nytimes3xbfgragh.onion/es/2019/02/07/valeria-luiselli-entrevista/}{Leer
en español}

A few days before Christmas, a group of 20 or so people crowded into the
Mexican novelist Valeria Luiselli's living room. Luiselli dipped in and
out among her guests, serving mulled wine as school-aged children from
Still Waters in a Storm, an educational center in Brooklyn focused on
reading and writing, prepared to perform an original musical adapted
from Cervantes' ``Don Quixote.'' The children had worked with the
center's founder, Stephen Haff, to translate the book from the Spanish
and write songs reinterpreting the story with a chorus of migrant
children. By the time Haff told the children to find their spots on the
small makeshift stage, the seats had filled up, so Luiselli sat on the
floor next to her 9-year-old daughter, Maia. It wasn't the first time
Luiselli had seen the show, but she still cried, as did her daughter,
when the children sang songs with lyrics like, ``Innocence needs a
home.''

The experiences of asylum-seeking children from Latin America have
preoccupied Luiselli for several years now and serve as a central theme
in her latest book about a family road trip across the United States.
``Lost Children Archive,'' which will be published by Knopf next week,
is Luiselli's fifth book, and the first of her novels to be written in
English.

\includegraphics{https://static01.graylady3jvrrxbe.onion/images/2019/02/08/books/08LUISELLI-BOOKCOVER/08LUISELLI-BOOKCOVER-articleLarge.jpg?quality=75\&auto=webp\&disable=upscale}

{[}
\href{https://www.nytimes3xbfgragh.onion/2019/02/11/books/review-lost-children-archive-valeria-luiselli.html}{\emph{Read
our review of ``Lost Children Archive.''}} {]}

When Luiselli, 35, started writing ``Lost Children Archive'' in the
summer of 2014, she struggled with using it ``as a loudspeaker for all
of my political rage.'' She had volunteered as a court translator for
child refugees from Latin America and was therefore familiar with the
migration crisis. She set aside the novel and wrote
\href{https://www.nytimes3xbfgragh.onion/2017/04/28/books/review/tell-me-how-it-ends-valeria-luiselli.html}{``Tell
Me How it Ends: An Essay in 40 Questions,''} a meditation on the
children's stories and the circumstances that brought them to the United
States. It was formatted after the questionnaire the court had her use
to interview the children and was a finalist for the National Book
Critics Circle Award in Criticism in 2017. Afterward, Luiselli said, she
was able to return to her novel and offer ``more open questions and open
ends instead of political stances that are too loud and obvious by
themselves.''

The formal inventiveness of ``Tell Me How It Ends'' is characteristic of
all of Luiselli's former work. Diego Rabasa, who has edited Luiselli's
books with Sexto Piso, an independent press in Mexico City, said her
first book, ``Papeles Falsos'' (translated as ``Sidewalks'' in English),
contained elements of literary, personal and travel essays. ``There has
always been a distinct aura of brilliance and intelligence surrounding
her,'' said Rabasa. ``What dazzled us was the audacity of a young writer
who was starting on such an original path.''

Luiselli characterized her first book, which was rejected by several
publishing houses before it was acquired by Sexto Piso, as an attempt to
``write myself into my mother tongue'' after a lifetime living away from
Mexico. She had always studied in English, so when it came to Spanish,
``I never had the inflections of the people my age. It didn't get
renewed with slang and street talk.''

Image

School-aged children perform an original musical adapted from Cervantes'
famous novel, ``Don Quixote'' at Valeria Luiselli's
apartment.Credit...Devin Yalkin for The New York Times

Luiselli first left Mexico at the age of 2, when her father moved the
family to Madison, Wis., to complete his doctorate. From there, her
father's work as a diplomat took them to Costa Rica, South Korea and
South Africa, where they arrived in 1994, shortly after Mandela's
historic election. By then, her mother had left the family to join the
Zapatista movement in Mexico. ``I come from a matriarchal line of women
who have always been very involved politically and socially,'' she said,
referring also to her grandmother, who worked with indigenous
communities in Puebla, Mexico. After attending boarding school in India,
she decided, as she puts it, ``I need to go back to Mexico and become
Mexican,'' Luiselli recalled. She was 19 when she enrolled at the
National Autonomous University of Mexico to major in philosophy and
began writing there.

Luiselli followed ``Papeles Falsos'' in 2011 with a novel, ``Los
Ingrávidos'' (``Faces in the Crowd''). It has been translated into 20
languages. Rabasa noted that the book remains one of Sexto Piso's most
popular and is reprinted every year. ``The Story of my Teeth,''
Luiselli's second novel, was the product of a collaboration with Jumex
factory workers in Mexico, in which she sent them chapters, and they
helped her shape the plot. Her books always reflect a deep plunge into
her sensibilities --- books and cultural references, and even real
people and places where she's been. In ``Lost Children Archive,'' for
instance, Still Waters in a Storm is mentioned, as are Haff, obscure
Italian writers and Ezra Pound.

``I have never written a novel that just sort of springs from the head
of Zeus, from an absolute space of fiction,'' Luiselli said. ``I always
begin my work documenting my everyday.''

Image

Guests listen to the children performing their musical version of ``Don
Quixote.''Credit...Devin Yalkin for The New York Times

``She's really wrestling with a certain strand of Latino and Latin
American identity in the U.S. in this political moment,'' said the
novelist Daniel Alarcón, who has sat on panels with Luiselli. In ``Tell
Me How It Ends,'' in particular, he said, ``she confronts directly
questions of privilege and gaze, at the same time wrestling with a
political moment that affects not just all Latinos but all Americans.''

``Without knowing it or planning to, she has opened doors,'' said her
friend, the writer Laia Jufresa, who told me she never imagined a
Mexican writer could plausibly achieve the number of translations,
awards or critical acclaim Luiselli has in her early career.

Both the novelist Francisco Goldman and Rabasa mentioned the ---
``insufferable,'' in Goldman's words --- macho culture of Mexican
literature. Goldman rattled off a list of woman writers, besides
Luiselli, who have been causing a seismic shift in the country's
literary world over the last decade: Gabriela Jauregui, Guadalupe
Nettel, Verónica Gerber Bicecci, Jufresa and Fernanda Melchor.

Image

Valeria Luiselli has a foot in Mexico and the United States, which is
perhaps why her literature is charged with the lucidness of
estrangement, said the Argentine novelist Samanta
Schweblin.Credit...Devin Yalkin for The New York Times

The Argentine novelist Samanta Schweblin, whose second book, ``Mouthful
of Birds,'' was published in January, said Luiselli's vision is a cross
between the Latin American and the North American views of the world.
``Both visions are so nostalgic, critical, loving and painful at the
same time. Valeria belongs to both territories and therefore understands
their signals, but at the same time she seems to always understand
herself as a foreigner,'' wrote Schweblin in a recent email. Perhaps,
she said, having one foot in each world is why Luiselli's literature is
charged with the lucidness of estrangement.

Luiselli is currently exploring different art forms altogether. She
recently received an Art for Justice fellowship to research and write
about mass incarceration in the United States, with an emphasis on
detention centers. ``The same companies own immigration jails and normal
jails,'' she said, and many don't understand that ``immigration
detention and mass incarceration are exactly the same thing.''

She started a literary program to teach creative writing to girls in a
detention center in upstate New York. She is also working on a
performance piece related to mass incarceration and violence against
women with the poet Natalie Diaz. In the fall, she will begin a two-year
residency at Bard College.

Luiselli lives in the Bronx and jokes that she is raising her daughter,
whose father is the Mexican novelist Álvaro Enrigue, in a household full
of women (her niece lives with her and her mother visits often). ``I
feel my female bonds stronger than ever in my life, the way that women
can group and discuss and think politically, and also just how friends
can get together and be a network of support,'' Luiselli said.

``Lost Children Archive'' was in large part a response to seeing her
daughter try to interpret the current migration crisis. ``Children can
add a tinge of bizarreness to what is possibly accepted as normal but
actually is not,'' said Luiselli. Her approach is to discuss issues with
her daughter ``in a way that she is not scared --- that she finds the
right balance between a certain rage or outrage and clarity to imagine
possible change.''

Advertisement

\protect\hyperlink{after-bottom}{Continue reading the main story}

\hypertarget{site-index}{%
\subsection{Site Index}\label{site-index}}

\hypertarget{site-information-navigation}{%
\subsection{Site Information
Navigation}\label{site-information-navigation}}

\begin{itemize}
\tightlist
\item
  \href{https://help.nytimes3xbfgragh.onion/hc/en-us/articles/115014792127-Copyright-notice}{©~2020~The
  New York Times Company}
\end{itemize}

\begin{itemize}
\tightlist
\item
  \href{https://www.nytco.com/}{NYTCo}
\item
  \href{https://help.nytimes3xbfgragh.onion/hc/en-us/articles/115015385887-Contact-Us}{Contact
  Us}
\item
  \href{https://www.nytco.com/careers/}{Work with us}
\item
  \href{https://nytmediakit.com/}{Advertise}
\item
  \href{http://www.tbrandstudio.com/}{T Brand Studio}
\item
  \href{https://www.nytimes3xbfgragh.onion/privacy/cookie-policy\#how-do-i-manage-trackers}{Your
  Ad Choices}
\item
  \href{https://www.nytimes3xbfgragh.onion/privacy}{Privacy}
\item
  \href{https://help.nytimes3xbfgragh.onion/hc/en-us/articles/115014893428-Terms-of-service}{Terms
  of Service}
\item
  \href{https://help.nytimes3xbfgragh.onion/hc/en-us/articles/115014893968-Terms-of-sale}{Terms
  of Sale}
\item
  \href{https://spiderbites.nytimes3xbfgragh.onion}{Site Map}
\item
  \href{https://help.nytimes3xbfgragh.onion/hc/en-us}{Help}
\item
  \href{https://www.nytimes3xbfgragh.onion/subscription?campaignId=37WXW}{Subscriptions}
\end{itemize}
