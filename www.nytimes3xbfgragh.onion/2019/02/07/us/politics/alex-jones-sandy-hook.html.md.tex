Sections

SEARCH

\protect\hyperlink{site-content}{Skip to
content}\protect\hyperlink{site-index}{Skip to site index}

\href{https://www.nytimes3xbfgragh.onion/section/politics}{Politics}

\href{https://myaccount.nytimes3xbfgragh.onion/auth/login?response_type=cookie\&client_id=vi}{}

\href{https://www.nytimes3xbfgragh.onion/section/todayspaper}{Today's
Paper}

\href{/section/politics}{Politics}\textbar{}Sandy Hook Families Gain in
Defamation Suits Against Alex Jones

\url{https://nyti.ms/2UN5ezX}

\begin{itemize}
\item
\item
\item
\item
\item
\item
\end{itemize}

Advertisement

\protect\hyperlink{after-top}{Continue reading the main story}

Supported by

\protect\hyperlink{after-sponsor}{Continue reading the main story}

\hypertarget{sandy-hook-families-gain-in-defamation-suits-against-alex-jones}{%
\section{Sandy Hook Families Gain in Defamation Suits Against Alex
Jones}\label{sandy-hook-families-gain-in-defamation-suits-against-alex-jones}}

\includegraphics{https://static01.graylady3jvrrxbe.onion/images/2019/02/08/us/politics/08dc-alexjones/merlin_148005867_23f79707-4530-4d5c-9104-3e9a99d927a1-articleLarge.jpg?quality=75\&auto=webp\&disable=upscale}

By
\href{https://www.nytimes3xbfgragh.onion/by/elizabeth-williamson}{Elizabeth
Williamson}

\begin{itemize}
\item
  Feb. 7, 2019
\item
  \begin{itemize}
  \item
  \item
  \item
  \item
  \item
  \item
  \end{itemize}
\end{itemize}

Families of the Sandy Hook school shooting victims have won a series of
victories in their defamation suits against the conspiracy theorist Alex
Jones that would open Mr. Jones's business records to them and compel
him to speak under oath.

Ten families are pursuing lawsuits against Mr. Jones over his role in
spreading bogus claims about the shooting, including that the victims'
families were actors in a plot to confiscate firearms from Americans.
The families have endured death threats, stalking and online abuse.

Mr. Jones, a far-right provocateur and the owner of
\href{https://www.nytimes3xbfgragh.onion/2019/12/09/us/politics/owen-shroyer-infowars.html}{Infowars},
a radio show and website on which he sells diet supplements, survivalist
gear and gun paraphernalia, has come under growing scrutiny over the
past year and has lost access to much of his online audience. Facebook,
Twitter, Apple and YouTube have all
\href{https://www.nytimes3xbfgragh.onion/2018/08/06/technology/infowars-alex-jones-apple-facebook-spotify.html?module=inline}{banned
him}, and a recent deal for his show to stream on
\href{https://www.nytimes3xbfgragh.onion/2019/01/15/us/politics/alex-jones-infowars-roku.html}{Roku
was revoked} last month after public outrage.

The suits by the Sandy Hook families have advanced on several fronts in
recent weeks.

A Texas judge on Jan. 25 ordered Mr. Jones and representatives of his
company to submit to questioning by lawyers for Scarlett Lewis, the
mother of Jesse Lewis, one of the 20 children killed at Sandy Hook
Elementary School in Newtown, Conn., in 2012. The Texas judge also
granted access to Mr. Jones's relevant business records, and denied his
lawyer's motion to keep the records sealed.

In Connecticut, a judge ordered Infowars representatives and business
partners to testify, and a ruling is expected as soon as mid-February on
the families' request to depose Mr. Jones, and several Infowars
``reporters'' and associates.

They include Wolfgang Halbig, a former school administrator and Infowars
contributor who for years has deluged Newtown officials with open
records requests, demanding, among other things, records from the
cleanup of ``bodily fluids, brain matter, skull fragments and around
45-60 gallons of blood.''

Some families of the victims have been subjected to years of harassment
from people who have embraced the bogus crisis-actor theory promoted by
Mr. Jones.

Robbie and Alissa Parker's eldest daughter, 6-year-old Emilie, died in
the Dec. 14, 2012 shooting that killed six adults as well as the 20
first graders.

In October 2016 in Seattle, Mr. Parker was walking to meet his family at
a hotel, he said, when a middle-age man, dressed in khakis and a sport
coat, approached him. The man, Mr. Parker said, asked if he had lost a
child at Sandy Hook.

``Yes, my daughter,'' Mr. Parker responded, offering his hand. The man
ignored it, and instead spewed obscenities at him, Mr. Parker said.

The man trailed him, ``jabbering in my ear,'' Mr. Parker recalled in an
interview, as he walked for blocks, trying to put distance between the
harasser and his family.

``It was absolute venom,'' said Mr. Parker, who is a plaintiff in one of
three defamation suits against Mr. Jones. ``He was absolutely disgusted
with the person that he believed that I was.''

Mr. Jones did not respond to requests for comment. His lawyer, Marc
Randazza, acknowledged the series of decisions in favor of the families,
but said, ``If you're keeping score here, this is just the coin toss.''

Mr. Jones's role in spreading baseless conspiracies about the families
began within days of the shooting.

The night after the shooting, Mr. Parker had agreed to meet a news crew
in front of a Newtown church to share a statement about Emilie.
Surprised to find a sea of waiting reporters and cameras, he gasped out
a nervous laugh before stepping forward to speak.

``This world is a better place because she has been in it,'' he said,
his voice cracking.

The footage ricocheted among conspiracy theorists, and Mr. Jones seized
on it. ``He's laughing, and then he goes over and starts basically
breaking down and crying,'' Mr. Jones told his followers, according to
court documents. ``This needs to be investigated. They're clearly using
this to go after our guns.''

On another show, Mr. Jones mocked Mr. Parker's emotions as ``method
acting.''

Mr. Randazza said the issue was not one of taste but of Mr. Jones's
First Amendment rights.

"If we're going to reset the rules about what you're allowed to say,
then let's reset them for everyone,'' Mr. Randazza said. ``And I don't
think anybody would be happy with that.''

Mr. Jones, once an obscure radio personality, gained national visibility
in 2015, when Donald J. Trump appeared on his show as a presidential
candidate, praising his ``amazing'' reputation. Mr. Jones has invited a
number of Sandy Hook conspiracy theorists on his show, shared video
footage and some families' photos and personal information.

Mr. Trump has echoed Mr. Jones's claims that establishment media
companies have conspired to silence conservative voices online, and in
November, the White House press secretary, Sarah Huckabee Sanders,
\href{https://www.nytimes3xbfgragh.onion/2018/11/08/business/media/infowars-white-house-jim-acosta-cnn.html}{used
a misleading Infowars video} to bolster the administration's claims that
a CNN reporter had manhandled a White House aide.

As the lawsuits inch toward trial, Mr. Jones is fighting to shield his
records and himself. Last week the Connecticut Supreme Court rejected
his appeal, letting stand the lower court's ruling granting access to
his records. Also last week, Mr. Jones's lawyers and Mr. Halbig, who is
representing himself, filed a request to change the location of an
eventual jury trial from Fairfield County, which encompasses Newtown, to
a Connecticut county more distant from the scene of the shooting.

The families are asserting that the case is about preventing harmful
falsehoods from proliferating in the ``post-truth'' ecosystem of the
internet.

``This was not journalism. It was marketing; Jones used these lies to
sell his Infowars brand products for profit,'' the families' Connecticut
lawyers wrote in court documents. The families are represented by
Koskoff Koskoff \& Bieder in Connecticut, and by Farrar \& Ball in
Texas.

Though Infowars and its affiliates are private and do not report
financial results, a
\href{https://www.nytimes3xbfgragh.onion/2018/09/07/us/politics/alex-jones-business-infowars-conspiracy.html?module=inline}{New
York Times investigation} last year found that in 2014 Infowars' revenue
was more than \$20 million a year, according to testimony Mr. Jones
provided in an unrelated court case. Mr. Jones recently claimed revenue
of up to \$50 million, saying on his show that it ``all goes to
lawyers.''

Last week, the police arrested a man suspected of stalking Emilie
Parker's family sporadically for years, sending letters that terrified
them, because they realized he knew where they lived.

Authorities said the man, Kevin Purfield, 51, of Portland, Ore., was
arrested on stalking charges after he repeatedly called and wrote the
editor of The Oregonian, asserting that its reporting on Sandy Hook and
other mass shootings was false, because they never happened. Mr.
Purfield was imprisoned in 2013 for harassing victims' families after
the 2012 mass shooting at a movie theater in Aurora, Colo. When his
probation ended, he resumed harassing the Parkers, Mr. Parker said.

Mr. Parker and his family joined the legal effort against Mr. Jones only
after speaking with a family whose daughter died in the Marjory Stoneman
Douglas High School shooting in Parkland, Fla., last year.

``The mom said, `My husband did a media interview, and now he just gets
attacked online by all these conspiracy theorists,''' Mr. Parker said.

``I thought, `These people aren't going away,''' he said. ```It's time
for us to see what we can do about it.'''

Advertisement

\protect\hyperlink{after-bottom}{Continue reading the main story}

\hypertarget{site-index}{%
\subsection{Site Index}\label{site-index}}

\hypertarget{site-information-navigation}{%
\subsection{Site Information
Navigation}\label{site-information-navigation}}

\begin{itemize}
\tightlist
\item
  \href{https://help.nytimes3xbfgragh.onion/hc/en-us/articles/115014792127-Copyright-notice}{©~2020~The
  New York Times Company}
\end{itemize}

\begin{itemize}
\tightlist
\item
  \href{https://www.nytco.com/}{NYTCo}
\item
  \href{https://help.nytimes3xbfgragh.onion/hc/en-us/articles/115015385887-Contact-Us}{Contact
  Us}
\item
  \href{https://www.nytco.com/careers/}{Work with us}
\item
  \href{https://nytmediakit.com/}{Advertise}
\item
  \href{http://www.tbrandstudio.com/}{T Brand Studio}
\item
  \href{https://www.nytimes3xbfgragh.onion/privacy/cookie-policy\#how-do-i-manage-trackers}{Your
  Ad Choices}
\item
  \href{https://www.nytimes3xbfgragh.onion/privacy}{Privacy}
\item
  \href{https://help.nytimes3xbfgragh.onion/hc/en-us/articles/115014893428-Terms-of-service}{Terms
  of Service}
\item
  \href{https://help.nytimes3xbfgragh.onion/hc/en-us/articles/115014893968-Terms-of-sale}{Terms
  of Sale}
\item
  \href{https://spiderbites.nytimes3xbfgragh.onion}{Site Map}
\item
  \href{https://help.nytimes3xbfgragh.onion/hc/en-us}{Help}
\item
  \href{https://www.nytimes3xbfgragh.onion/subscription?campaignId=37WXW}{Subscriptions}
\end{itemize}
