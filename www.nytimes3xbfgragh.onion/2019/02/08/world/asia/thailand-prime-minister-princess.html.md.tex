Sections

SEARCH

\protect\hyperlink{site-content}{Skip to
content}\protect\hyperlink{site-index}{Skip to site index}

\href{https://www.nytimes3xbfgragh.onion/section/world/asia}{Asia
Pacific}

\href{https://myaccount.nytimes3xbfgragh.onion/auth/login?response_type=cookie\&client_id=vi}{}

\href{https://www.nytimes3xbfgragh.onion/section/todayspaper}{Today's
Paper}

\href{/section/world/asia}{Asia Pacific}\textbar{}Thailand's King
Rejects His Sister's Candidacy for Prime Minister

\url{https://nyti.ms/2UHMrGi}

\begin{itemize}
\item
\item
\item
\item
\item
\end{itemize}

Advertisement

\protect\hyperlink{after-top}{Continue reading the main story}

Supported by

\protect\hyperlink{after-sponsor}{Continue reading the main story}

\hypertarget{thailands-king-rejects-his-sisters-candidacy-for-prime-minister}{%
\section{Thailand's King Rejects His Sister's Candidacy for Prime
Minister}\label{thailands-king-rejects-his-sisters-candidacy-for-prime-minister}}

\includegraphics{https://static01.graylady3jvrrxbe.onion/images/2019/02/09/world/09thailand-1sub/09thailand-1sub-articleLarge.jpg?quality=75\&auto=webp\&disable=upscale}

By \href{https://www.nytimes3xbfgragh.onion/by/hannah-beech}{Hannah
Beech}

\begin{itemize}
\item
  Feb. 8, 2019
\item
  \begin{itemize}
  \item
  \item
  \item
  \item
  \item
  \end{itemize}
\end{itemize}

BANGKOK --- The sister of Thailand's king was nominated on Friday as a
candidate for prime minister, but by day's end, the political foray was
rebuked by the monarch as ``inappropriate'' behavior that violated the
nation's constitutional monarchy, apparently ending her candidacy.

In a kingdom where the royal family is considered above the volcanic
eruptions of Thai politics, the prospect of the king's sister running
for office, followed by the public airing of a disagreement between the
royal siblings, upended the political landscape.

Ubolratana Rajakanya Sirivadhana Varnavadi, 67, the elder sister of King
Maha Vajiralongkorn Bodindradebayavarangkun, was nominated by a party
associated with the Shinawatra political family, which includes two
fugitive former prime ministers who have been accused of subverting the
power of Thailand's royal institutions.

Calling Ms. Ubolratana ``an educated and skilled person'' who was the
``most suitable choice,'' Preechapol Pongpanich, the leader of the Thai
Raksa Chart Party, announced her candidacy on Friday but cautioned that
the choice still had to be accepted by Thailand's election commission.

But in the highly unusual palace statement, King Vajiralongkorn, 66,
made clear that he disapproved of his sister's candidacy. The late-night
statement appeared to preclude any need for the commission to try to
adjudicate an unprecedented question of royalty and politics.

``Involvement of a high-ranking member of the royal family in politics,
in any way, is against the nation's traditions, customs and culture and
is therefore considered improper and highly inappropriate,'' the royal
statement said.

King Vajiralongkorn's royal command effectively invalidated Ms.
Ubolratana's brief political career, analysts said.

``The withdrawal of her candidacy will cool the political temperature,
because it would have renewed tensions and polarization,'' said Thitinan
Pongsudhirak, a political scientist at Chulalongkorn University in
Bangkok.

Thailand has been under military rule
\href{https://www.nytimes3xbfgragh.onion/2014/05/23/world/asia/thailand-military-coup.html?module=inline}{since
a coup in 2014} unseated forces loyal to Thaksin Shinawatra, a brash
billionaire who challenged the country's traditional power structure.
The country is now led by Prime Minister Prayuth Chan-ocha, a former
general and junta chief who has fashioned himself as
\href{https://www.nytimes3xbfgragh.onion/2014/05/27/world/asia/thailand.html}{a
fierce defender of Thailand's monarchy}.

On Friday, Mr. Prayuth also announced his candidacy for prime minister,
backed by the military's proxy Palang Pracharat Party.

``Although I have served as a soldier for all my life, I am willing to
sacrifice myself in order to protect Thailand,'' Mr. Prayuth said in a
statement.

National elections are scheduled for March 24, after repeated delays by
Thailand's junta. A military-drafted Constitution ensures that the
country's next prime minister will be chosen by a Parliament in which
many members will be appointed by the military.

The junta has restricted freedom of speech and assembly, sending
perceived political opponents to so-called attitude adjustment camps.

Thailand also has stringent lèse-majesté laws that criminalize insults
to the monarchy, and prosecutions of this crime have increased
significantly in recent years.

\includegraphics{https://static01.graylady3jvrrxbe.onion/images/2019/02/09/world/09thailand-2/merlin_148892286_8e4160bf-2622-4ee1-acfb-5163efc76e98-articleLarge.jpg?quality=75\&auto=webp\&disable=upscale}

Even though Ms. Ubolratana officially gave up her royal titles when she
married an American in 1972, the king's statement made clear that she
still represented the royal family and was therefore to stay out of the
political realm.

``Even though she relinquished her royal title in writing in line with
royal rules, she still maintains her status and life as a member of the
Chakri dynasty,'' the statement said, referring to Thailand's royal
family.

The royal communiqué on Friday departed from previous convention in one
important respect. While the notion that the king, queen and heir
apparent are ``above politics'' has been a pillar of Thailand's many
constitutions, the king's statement broadened the definition of those
who should stay away from the political realm to include other members
of the royal family who are close to the monarch.

On Friday, five prominent analysts of Thai politics declined to comment
about Ms. Ubolratana's candidacy. Several political activists who on
social media had initially criticized her candidacy for unnecessarily
muddying an already complicated political field quickly deleted their
comments.

The deputy prime minister, Wissanu Krea-ngam, told reporters on Friday
that he had no comment on Ms. Ubolratana's candidacy. ``If I could
answer you, I would,'' he said. ``But I can't.''

Since Thailand's absolute monarchy was abolished in 1932, immediate
members of the royal family have not run for high office. The country's
political system has for decades involved a cut and thrust between
powerful political forces committed to elections and a military that has
at times deemed the ballot box harmful to the country. The military has
staged a dozen successful coups.

On Friday afternoon, Paiboon Nititawan, the head of the military-linked
party that nominated Mr. Prayuth, said that he had asked the election
commission to investigate whether Ms. Ubolratana's nomination
contravened regulations that ban political parties from using the royal
family for election campaigning.

But the king's late-night statement appeared to preclude any need for
the election commission to weigh in on matters of the monarchy and
statecraft.

In an Instagram post on Friday afternoon, Ms. Ubolratana maintained that
she was a ``commoner'' and that she enjoyed ``no privilege over the Thai
people in accordance with the Constitution.''

``I have conducted this action with sincerity and the willingness to
sacrifice to have the opportunity to lead the country to prosperity,''
she wrote.

Rumors about Ms. Ubolratana's close ties with the Shinawatra family
intensified last year when she was pictured with Mr. Thaksin, a former
prime minister, and Yingluck Shinawatra, his sister and another former
prime minister.

Both brother and sister have been convicted of corruption-linked crimes
in absentia and are living in exile. A 2006 military putsch ended the
tenure of Mr. Thaksin, whose political base came from Thailand's rural
poor. Every election this century has been won by forces loyal to Mr.
Thaksin.

But some of those close to Mr. Thaksin characterized Ms. Ubolratana's
candidacy as a dangerous blurring of lines.

``In a democratic system, we have to make clear: What is power that
comes from outside the system and what is power that comes from the
people?'' Anon Nampha, a lawyer critical of the junta, said in a
Facebook post. ``This is not a democracy.''

Late last month, Ms. Ubolratana --- an actress who is still sometimes
referred to as a princess despite having given up her royal titles ---
made a high-profile pilgrimage to nine Buddhist temples, something that
politicians often do to pray for good fortune in electoral contests.

Her father,
\href{https://www.nytimes3xbfgragh.onion/2016/10/14/world/asia/thai-king-bhumibol-adulyadej-dies.html?module=inline}{King
Bhumibol Adulyadej}, was the world's longest-reigning monarch until his
death in 2016. Although he had no formal political role, during his
seven decades on the throne, King Bhumibol was seen as a unifying force
for a nation frequently troubled by coups and deadly political violence.

Advertisement

\protect\hyperlink{after-bottom}{Continue reading the main story}

\hypertarget{site-index}{%
\subsection{Site Index}\label{site-index}}

\hypertarget{site-information-navigation}{%
\subsection{Site Information
Navigation}\label{site-information-navigation}}

\begin{itemize}
\tightlist
\item
  \href{https://help.nytimes3xbfgragh.onion/hc/en-us/articles/115014792127-Copyright-notice}{©~2020~The
  New York Times Company}
\end{itemize}

\begin{itemize}
\tightlist
\item
  \href{https://www.nytco.com/}{NYTCo}
\item
  \href{https://help.nytimes3xbfgragh.onion/hc/en-us/articles/115015385887-Contact-Us}{Contact
  Us}
\item
  \href{https://www.nytco.com/careers/}{Work with us}
\item
  \href{https://nytmediakit.com/}{Advertise}
\item
  \href{http://www.tbrandstudio.com/}{T Brand Studio}
\item
  \href{https://www.nytimes3xbfgragh.onion/privacy/cookie-policy\#how-do-i-manage-trackers}{Your
  Ad Choices}
\item
  \href{https://www.nytimes3xbfgragh.onion/privacy}{Privacy}
\item
  \href{https://help.nytimes3xbfgragh.onion/hc/en-us/articles/115014893428-Terms-of-service}{Terms
  of Service}
\item
  \href{https://help.nytimes3xbfgragh.onion/hc/en-us/articles/115014893968-Terms-of-sale}{Terms
  of Sale}
\item
  \href{https://spiderbites.nytimes3xbfgragh.onion}{Site Map}
\item
  \href{https://help.nytimes3xbfgragh.onion/hc/en-us}{Help}
\item
  \href{https://www.nytimes3xbfgragh.onion/subscription?campaignId=37WXW}{Subscriptions}
\end{itemize}
