Sections

SEARCH

\protect\hyperlink{site-content}{Skip to
content}\protect\hyperlink{site-index}{Skip to site index}

\href{https://www.nytimes3xbfgragh.onion/section/world/americas}{Americas}

\href{https://myaccount.nytimes3xbfgragh.onion/auth/login?response_type=cookie\&client_id=vi}{}

\href{https://www.nytimes3xbfgragh.onion/section/todayspaper}{Today's
Paper}

\href{/section/world/americas}{Americas}\textbar{}Humanitarian Aid
Stalls, Testing Venezuela's Opposition

\url{https://nyti.ms/2UT886g}

\begin{itemize}
\item
\item
\item
\item
\item
\end{itemize}

Advertisement

\protect\hyperlink{after-top}{Continue reading the main story}

Supported by

\protect\hyperlink{after-sponsor}{Continue reading the main story}

\hypertarget{humanitarian-aid-stalls-testing-venezuelas-opposition}{%
\section{Humanitarian Aid Stalls, Testing Venezuela's
Opposition}\label{humanitarian-aid-stalls-testing-venezuelas-opposition}}

\includegraphics{https://static01.graylady3jvrrxbe.onion/images/2019/02/14/world/14venezuela1-print/merlin_150598467_47ae2005-333b-4443-8c55-8d5a06997678-articleLarge.jpg?quality=75\&auto=webp\&disable=upscale}

By \href{https://www.nytimes3xbfgragh.onion/by/nicholas-casey}{Nicholas
Casey} and Anatoly Kurmanaev

\begin{itemize}
\item
  Feb. 13, 2019
\item
  \begin{itemize}
  \item
  \item
  \item
  \item
  \item
  \end{itemize}
\end{itemize}

\href{https://www.nytimes3xbfgragh.onion/es/2019/02/14/venezuela-frontera-ayuda-humanitaria/}{Leer
en español}

CÚCUTA, Colombia --- The battle over the legitimate leadership of
Venezuela --- which has included rallies of thousands, international
diplomacy and oil sanctions --- is now focused on a single heavily
guarded shipment of humanitarian aid.

Venezuela's opposition, which has relished a month of victories in its
effort to challenge President Nicolás Maduro and take over as the
country's legitimate government, brought the donated supplies of food
and medical kits to the country's border with Colombia.

Its goal was to bring the supplies into Venezuela, forcing a
confrontation with Mr. Maduro, who has refused the help. This would cast
Mr. Maduro in a bad light, opposition leaders said, and display their
ability to set up a government-like relief system in a nation where the
crumbling economy has left many starving, sick and without access to
medicine.

But there was no dramatic confrontation.

Instead, Mr. Maduro's administration erected a crude, but effective
blockade across the border bridge with Colombia. The move brought the
relief effort to a halt, and left the opposition and its leader, Juan
Guaidó, at a standstill, aware that each passing day dampens its
considerable momentum toward winning the trust of Venezuelans and the
recognition of other governments. A delay could also mean reverting back
to the status quo, in which Mr. Maduro retains control.

``The whole country is waiting to see what Mr. Guaidó does next,'' said
Carlos Andrés Taborda, an opposition organizer in the small Venezuelan
border town of Ureña, as he marched to demand the release of
humanitarian aid. ``Whether this remains a massive movement depends
largely on him.''

\includegraphics{https://static01.graylady3jvrrxbe.onion/images/2019/02/14/world/14venezuela4/merlin_150010104_d7f4fe92-b803-4267-802c-694d89602837-articleLarge.jpg?quality=75\&auto=webp\&disable=upscale}

On Wednesday, Mr. Guaidó heightened the stakes, telling supporters that
he would open a ``humanitarian corridor'' to allow aid to flow into the
country by Feb. 23.

The pledge increased tensions at the border, raising expectations on
both sides and setting a deadline to meet them. But the obstacles ahead
were clear.

At the heavily guarded warehouse in Cúcuta, where supplies have sat for
nearly a week, workers packed bags with medical kits or with vegetable
oil, flour, lentils and rice. More donations were being prepared in
Miami and Houston for deployment.

A short drive away, Mr. Maduro's improvised barrier spread across the
bridge's lanes, blocking passage.

Freddy Bernal, Mr. Maduro's envoy to the Colombian border region of
Táchira, called the aid ``trash'' that ``can't even feed a small
shantytown.''

Surrounded by six bodyguards with bullet-resistant vests, he repeated
claims that the aid delivery was a ploy to destabilize Mr. Maduro's
government. Mr. Bernal said that there was no standoff at the border.

Image

Venezuelans marching against the government on Tuesday in Pedro Maria
Urena.Credit...Meridith Kohut for The New York Times

``There's complete normality here --- there's peace and folk music,'' he
said.

Gaby Arellano, an opposition lawmaker in charge of the shipment in
Colombia, said one of the goals was to force the military, which has
remained loyal to the government, to choose between Mr. Maduro and
feeding the Venezuelan people.

``Popular pressure to break the military --- this is what we're working
toward,'' she said.

In recent days, opposition lawmakers have traveled to the United States,
Brazil and a second location in Colombia to talk with local authorities
about setting up similar warehouses ahead of the push on Feb. 23. On
Wednesday, an official from the government of the Caribbean island of
Curacao said he would participate.

In Cúcuta, members of the opposition say they are considering options to
physically force the shipment into Venezuela.

Omar Lares, a former opposition mayor in exile in Cúcuta, said
organizers want people to surround an aid truck on the Colombian side
and accompany it to the bridge. A crowd of thousands would be gathered
on the other side to push through a security cordon, move the containers
blocking the bridge, and accompany the aid into Venezuela.

``One group over there, one over here, and we'll make one large human
chain,'' he said.

Lorena Valero, an activist on the Venezuelan side of the border, staged
a protest two years ago to call for the flow of food and supplies. She
said she's willing to participate again should Mr. Guaidó call to storm
the bridge.

``We're not afraid. We are certain that it will enter,'' she said.

Still, all the uncertainty has some observers questioning the
consequences if the opposition cannot make good on its promises.

Image

Shipping containers blocked the Las Tienditas Bridge connecting
Venezuela and Colombia.Credit...Meridith Kohut for The New York Times

``The opposition has created immense expectations, and it's not at all
clear they have a plan for actually fulfilling them,'' said David
Smilde, a Venezuela analyst at the Washington Office on Latin America.
``Furthermore, the opposition and the U.S. have not been clear that this
aid, even if allowed in, will make a significant dent in Venezuela's
humanitarian crisis.''

Some Venezuelans have even put off buying medication, expecting that the
American donations will arrive across the border soon, Mr. Smilde said.

The heightened expectations have managed to bring together what has been
a fractured opposition, giving Mr. Maduro the first challenge to his
rule in years. Mr. Guaidó --- now recognized as Venezuela's legitimate
president by dozens of countries --- has emerged as a leader. And the
aid corridor has given the opposition a common project to promote.

Still, using a food shipment to challenge Mr. Maduro has concerned the
same nongovernmental groups that would normally assist in such an
effort. Caritas, the charitable arm of the Catholic Church, and the
International Committee of the Red Cross have declined to participate,
saying they must remain politically neutral.

Some diplomats and even opposition strategists have questioned the
viability of organizing such a complex sea and land operation in 10
days. So far, Mr. Guaidó and his advisers have remained quiet about
their plans to get the aid over the border. They have said the help
would include \$200 million in aid donations from friendly governments
and the private sector, in addition to the American shipment.

For those on the border, a sense of urgency prevailed.

At an opposition rally on Tuesday in Ureña, spirits remained high, but
protesters were becoming impatient for concrete results.

``We can't let those containers sit there for so long,'' said Linda
Acosta, an Ureña resident.

Advertisement

\protect\hyperlink{after-bottom}{Continue reading the main story}

\hypertarget{site-index}{%
\subsection{Site Index}\label{site-index}}

\hypertarget{site-information-navigation}{%
\subsection{Site Information
Navigation}\label{site-information-navigation}}

\begin{itemize}
\tightlist
\item
  \href{https://help.nytimes3xbfgragh.onion/hc/en-us/articles/115014792127-Copyright-notice}{©~2020~The
  New York Times Company}
\end{itemize}

\begin{itemize}
\tightlist
\item
  \href{https://www.nytco.com/}{NYTCo}
\item
  \href{https://help.nytimes3xbfgragh.onion/hc/en-us/articles/115015385887-Contact-Us}{Contact
  Us}
\item
  \href{https://www.nytco.com/careers/}{Work with us}
\item
  \href{https://nytmediakit.com/}{Advertise}
\item
  \href{http://www.tbrandstudio.com/}{T Brand Studio}
\item
  \href{https://www.nytimes3xbfgragh.onion/privacy/cookie-policy\#how-do-i-manage-trackers}{Your
  Ad Choices}
\item
  \href{https://www.nytimes3xbfgragh.onion/privacy}{Privacy}
\item
  \href{https://help.nytimes3xbfgragh.onion/hc/en-us/articles/115014893428-Terms-of-service}{Terms
  of Service}
\item
  \href{https://help.nytimes3xbfgragh.onion/hc/en-us/articles/115014893968-Terms-of-sale}{Terms
  of Sale}
\item
  \href{https://spiderbites.nytimes3xbfgragh.onion}{Site Map}
\item
  \href{https://help.nytimes3xbfgragh.onion/hc/en-us}{Help}
\item
  \href{https://www.nytimes3xbfgragh.onion/subscription?campaignId=37WXW}{Subscriptions}
\end{itemize}
