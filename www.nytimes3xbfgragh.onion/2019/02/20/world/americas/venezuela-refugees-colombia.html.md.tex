Sections

SEARCH

\protect\hyperlink{site-content}{Skip to
content}\protect\hyperlink{site-index}{Skip to site index}

\href{https://www.nytimes3xbfgragh.onion/section/world/americas}{Americas}

\href{https://myaccount.nytimes3xbfgragh.onion/auth/login?response_type=cookie\&client_id=vi}{}

\href{https://www.nytimes3xbfgragh.onion/section/todayspaper}{Today's
Paper}

\href{/section/world/americas}{Americas}\textbar{}A Staggering Exodus:
Millions of Venezuelans Are Leaving the Country, on Foot

\url{https://nyti.ms/2GzHtsf}

\begin{itemize}
\item
\item
\item
\item
\item
\item
\end{itemize}

Advertisement

\protect\hyperlink{after-top}{Continue reading the main story}

Supported by

\protect\hyperlink{after-sponsor}{Continue reading the main story}

\hypertarget{a-staggering-exodus-millions-of-venezuelans-are-leaving-the-country-on-foot}{%
\section{A Staggering Exodus: Millions of Venezuelans Are Leaving the
Country, on
Foot}\label{a-staggering-exodus-millions-of-venezuelans-are-leaving-the-country-on-foot}}

\includegraphics{https://static01.graylady3jvrrxbe.onion/images/2019/02/21/world/21venez-walkers-p1/merlin_150254385_767dc384-e5c1-4d26-80f9-a2f3ebbe101c-articleLarge.jpg?quality=75\&auto=webp\&disable=upscale}

By \href{https://www.nytimes3xbfgragh.onion/by/nicholas-casey}{Nicholas
Casey} and Jenny Carolina González

\begin{itemize}
\item
  Feb. 20, 2019
\item
  \begin{itemize}
  \item
  \item
  \item
  \item
  \item
  \item
  \end{itemize}
\end{itemize}

\href{https://www.nytimes3xbfgragh.onion/es/2019/02/20/venezuela-migrantes-cucuta/?}{Leer
en español}

PAMPLONA, Colombia --- The walking began before dawn: before the clouds
broke against the mountaintops, before the trucks took over the highway,
even before anyone in the town woke up to check the vacant lot where
scores of Venezuelan refugees had been huddling through the night.

Children, grandmothers, teachers, nurses, oil workers and the jobless
had all sprawled there together --- bound by a collective will to put as
many miles as possible between themselves and the collapsing country
they had fled.

All but Yoxalida Pimentel. She could not take another step.

``After so many hours of walking, after days, nights, sun, cold, rain
--- I lost my baby,'' she said, crying alone the morning after her
miscarriage.

The economic crisis that has engulfed Venezuela under President Nicolás
Maduro has set off a staggering exodus. The economic damage is among the
worst in Latin American history, researchers say, with more than three
million people leaving the country in recent years --- largely on foot.

They are fleeing dangerous shortages of food, water, electricity and
medicine, as well as the government's political crackdowns, in which
more than 40 people have been killed in the last few weeks alone.

Rolling suitcases behind them, some walk along highways, their salaries
so obliterated by Venezuela's hyperinflation that bus tickets are out of
reach. Others try to hitchhike for thousands of miles until they reach
Ecuador or Peru.

\includegraphics{https://static01.graylady3jvrrxbe.onion/images/2019/02/21/world/21venez-walkers-1-print/21venez-walkers-1-print-articleLarge-v3.jpg?quality=75\&auto=webp\&disable=upscale}

But no matter their destination, the vast majority come through these
treacherous roads in Colombia: a 125-mile journey over a 12,000-foot
pass here in the Andes Mountains.

``It's the coldest place I've known in my life,'' said Fredy Rondón, who
had come from Venezuela's capital, Caracas, with a single bag of
belongings. Now, he was breathless at 10,500 feet, with a treeless
steppe before him.

``I thought I could take the cold, but this is too, too much,'' he said.

His willingness to travel these twisting mountain roads speaks to the
desperation in Venezuela. The country is experiencing its deepest
political unrest in a generation, with
\href{https://www.nytimes3xbfgragh.onion/2019/02/04/world/americas/venezuela-maduro-guaido-legitimate.html}{two
men claiming the presidency simultaneously}.

Here in the Colombian mountains, Venezuelan refugees now murmur about
\href{https://www.nytimes3xbfgragh.onion/2019/01/22/world/americas/juan-guaido-facts-history-bio.html}{Juan
Guaidó, the opposition leader who declared himself Venezuela's
legitimate leader} last month, inspiring many Venezuelans to rally to
his side. The opposition and Mr. Maduro are at
loggerheads\href{https://www.nytimes3xbfgragh.onion/2019/02/13/world/americas/aid-to-venezuela.html}{over
the delivery of humanitarian aid}, which Mr. Maduro's government has
blockaded at the border with Colombia, a short distance from where their
journey begins.

``We are all scared it will get ugly between Maduro and Guaidó,'' said
Norma López, who walked with her five children and 6-day-old infant. Her
neighbors, she said, told her the government was ``going to take away
their teenagers to defend Maduro.''

On hearing the rumor, Ms. López said, she moved up her plans to leave
the country.

For most Venezuelans, the exodus first takes them to Cúcuta, a sprawling
Colombian border city where thousands arrive by footbridge every day.

Image

Damaris Olivares, right, who crossed the border into Colombia with nine
small children, the youngest just 6 days old.Credit...Federico Rios
Escobar for The New York Times

On the outskirts of the city is a parking lot where volunteers gather
around 6 a.m. to offer the migrants a place to shower, a bowl of oatmeal
and jackets for the children, if they don't have any.

``I'm lost, disoriented,'' said Edwin Villareal, 25, who said he would
walk to the Colombian city of Medellín with his wife and three children,
one of whom had asthma. The five had only 10,000 Colombian pesos among
them, about \$3.

``Perhaps someone gives us a ride,'' he said. ``We have no money for the
bus.''

Few people were offering rides on Route 55, a two-lane highway that
weaves into the eastern mountain range known as the Cordillera Oriental.
Instead, scores of Venezuelans ascended the road on foot, at a snail's
pace.

At 7,700 feet above sea level, Martha Socorro Duque has spent months
watching the migrants file past her living room in the Colombian town of
Pamplona. They search for food and shelter in a town with little to
offer.

``People came arriving with their shoes totally broken and destroyed,''
she said. ``But the hardest wasn't seeing their shoes, it was seeing
their feet: the lacerations, the blisters that were filled with blood.''

So Ms. Duque decided to improvise a shelter of her own. She opened up
the lot across the street to the strangers who came, and collected
donations from neighbors to offer them food. Sixty now camp here on a
given night, the men in blankets on the floor outside, the women and
children on makeshift bedding in a shed beside a stream.

Image

A nurse treating the blisters on a 9-year-old girl.Credit...Federico
Rios Escobar for The New York Times

She offered Ms. Pimentel a place to stay before the young woman gave
birth to her stillborn child.

Heartbroken, Ms. Pimentel explained that her mother had already left
Venezuela, walking and hitching rides to Chile with hopes of sending
money home. But when no jobs could be found there, Ms. Pimentel crossed
the border too, leaving behind three children she had been unable to
feed.

``Out of sheer desperation I've decided to walk,'' she said, ``so I can
take care of my children back there who are still alive.''

The road continued on from Pamplona, past an abandoned home where the
roof had caved in. Above 10,000 feet, Alexis Ron and his brother-in-law
walked with backpacks slung over their shoulders, a mile ahead of their
wives, who carried more suitcases and supplies. They had left Venezuela
months before, but they said it was the miserable life they had
encountered in Colombia that had spurred them to keep walking.

Mr. Ron, 40, said he used to fix high-end cars in Caracas. ``I could
take apart a car from front to back,'' he said. ``And I could give it
back to you assembled again without a screw missing.''

But most of those cars had been off the roads for ages in Venezuela,
where even car tires can be hard to find, so he left. In Cúcuta, he
washed cars instead of repairing them, earning a few dollars a day.

His boss stiffed him on his pay. Colombians spat on him and others in
the street, saying Venezuelans were stealing their jobs, he said. But
his breaking point came when a man offered to pay to sleep with his wife
--- and in desperation she took the man's cellphone number to make the
arrangements.

Image

Migrants outside an improvised shelter in Pamplona that houses 60 people
on a given night.Credit...Federico Rios Escobar for The New York Times

``He would give her 20,000 pesos,'' said Mr. Ron, the equivalent of \$6.

He decided it was time to leave.

His wife caught up with the men an hour later and confirmed the story,
looking at the ground. Her brother put an arm over her shoulder and, for
a while, no one spoke.

Miles ahead, the road flattened out at 11,000 feet, revealing a vast
plateau where only sedges grew. Génesis Zambrano, 20 years old and eight
months pregnant, stood holding her infant daughter and caught her
breath.

``My back,'' she said, pointing to where her pain was.

She had wanted to rest in Cúcuta before making the journey to the
capital, Bogotá, to find her father there. But instead, she spent the
days watching her infant daughter, Yeanis, slowly fade: There was food
to buy in Colombia, but no money to feed her daughter anything other
than a bottle filled with water and rice.

``The girl cried, but she had no tears,'' she said.

Yeanis spent nine days in a hospital in Cúcuta being treated for anemia
and a respiratory infection. But when her daughter's vitals returned to
normal, Ms. Zambrano said she decided it was time to leave, setting out
for Bogotá on foot.

The road seemed endless. But not far from the summit, a miracle
happened: A giant, empty truck pulled over.

Image

Migrants climbing into a truck whose driver offered them a ride. ``When
you have no load, you have to take them,'' the driver said. ``But the
truth is you risk your livelihood too, if the company finds out or the
police stop you.''Credit...Federico Rios Escobar for The New York Times

``When you have no load, you have to take them,'' said the driver, who
asked not to be named. ``But the truth is you risk your livelihood too,
if the company finds out or the police stop you.''

With a sudden \emph{whoosh,} the truck picked up speed and the tundra
landscape --- which had seemed unchanging at a walker's pace ---
suddenly transformed as cow pastures, creeks and road signs flashed by.

Inside, the masses huddled for warmth. There was Marian Jiménez, who had
sprained her foot. Jeremy Hidalgo, who had been walking four days.

Roberto Javier Tovar, who had left his wife and child behind in
Venezuela, pulled in his jacket and praised the driver loudly, even
though no one knew where the vehicle would take them.

``Almost no one has helped us but this man,'' he said.

The sun began to settle and the back of the flatbed grew crowded as the
driver took on dozens more migrants.

By evening, more than 100 adults and children were packed inside,
leaving a silent, empty road behind.

``We must give praise to God Almighty for this blessing,'' someone
shouted when the vehicle stopped.

Night fell and stars came out. The temperature fell, too, but the truck
was at last descending and the lights of Bucaramanga were visible, still
thousands of feet below.

Daniel Bermúdez, who had left his family behind and had been walking for
the last five days, looked out at the unknown city.

``My 6-year-old son, he saw me with my suitcase, and he said, `You're
not coming back,''' Mr. Bermúdez said as he began to cry in the icy
wind.

He paused. ``Yes, I'll come back. But look at me now, I am so far from
home.''

Advertisement

\protect\hyperlink{after-bottom}{Continue reading the main story}

\hypertarget{site-index}{%
\subsection{Site Index}\label{site-index}}

\hypertarget{site-information-navigation}{%
\subsection{Site Information
Navigation}\label{site-information-navigation}}

\begin{itemize}
\tightlist
\item
  \href{https://help.nytimes3xbfgragh.onion/hc/en-us/articles/115014792127-Copyright-notice}{©~2020~The
  New York Times Company}
\end{itemize}

\begin{itemize}
\tightlist
\item
  \href{https://www.nytco.com/}{NYTCo}
\item
  \href{https://help.nytimes3xbfgragh.onion/hc/en-us/articles/115015385887-Contact-Us}{Contact
  Us}
\item
  \href{https://www.nytco.com/careers/}{Work with us}
\item
  \href{https://nytmediakit.com/}{Advertise}
\item
  \href{http://www.tbrandstudio.com/}{T Brand Studio}
\item
  \href{https://www.nytimes3xbfgragh.onion/privacy/cookie-policy\#how-do-i-manage-trackers}{Your
  Ad Choices}
\item
  \href{https://www.nytimes3xbfgragh.onion/privacy}{Privacy}
\item
  \href{https://help.nytimes3xbfgragh.onion/hc/en-us/articles/115014893428-Terms-of-service}{Terms
  of Service}
\item
  \href{https://help.nytimes3xbfgragh.onion/hc/en-us/articles/115014893968-Terms-of-sale}{Terms
  of Sale}
\item
  \href{https://spiderbites.nytimes3xbfgragh.onion}{Site Map}
\item
  \href{https://help.nytimes3xbfgragh.onion/hc/en-us}{Help}
\item
  \href{https://www.nytimes3xbfgragh.onion/subscription?campaignId=37WXW}{Subscriptions}
\end{itemize}
