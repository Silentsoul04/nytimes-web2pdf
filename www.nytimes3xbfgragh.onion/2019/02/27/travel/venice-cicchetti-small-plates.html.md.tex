Sections

SEARCH

\protect\hyperlink{site-content}{Skip to
content}\protect\hyperlink{site-index}{Skip to site index}

\href{https://www.nytimes3xbfgragh.onion/section/travel}{Travel}

\href{https://myaccount.nytimes3xbfgragh.onion/auth/login?response_type=cookie\&client_id=vi}{}

\href{https://www.nytimes3xbfgragh.onion/section/todayspaper}{Today's
Paper}

\href{/section/travel}{Travel}\textbar{}When in Venice, Eat Like a
Venetian

\url{https://nyti.ms/2IDH75r}

\begin{itemize}
\item
\item
\item
\item
\item
\item
\end{itemize}

Advertisement

\protect\hyperlink{after-top}{Continue reading the main story}

Supported by

\protect\hyperlink{after-sponsor}{Continue reading the main story}

\hypertarget{when-in-venice-eat-like-a-venetian}{%
\section{When in Venice, Eat Like a
Venetian}\label{when-in-venice-eat-like-a-venetian}}

When they want a bite, locals head to their favorite bacaro for
cicchetti, the Venetian version of tapas. Here are seven of the best
places to find them.

\includegraphics{https://static01.graylady3jvrrxbe.onion/images/2019/03/03/travel/03venice-cicchetti/03test-articleLarge.jpg?quality=75\&auto=webp\&disable=upscale}

By Steven Raichlen

\begin{itemize}
\item
  Feb. 27, 2019
\item
  \begin{itemize}
  \item
  \item
  \item
  \item
  \item
  \item
  \end{itemize}
\end{itemize}

Each year, 20 million tourists visit Venice. The vast majority will pay
too much for indifferent food eaten mostly in the company of other
tourists. But there's one way to eat great Venetian food that's
thrilling, filling and authentic. You'll find it at a place where you're
almost certain to rub and bend elbows with locals. Visit a bacaro.

Like Spain's tapas bars, the bacaro ** serves infinitely varied,
kaleidoscopically colorful small plates at prices even a budget traveler
can afford. What makes the Venetian version unique is that the menu
changes not only seasonally (you're in Italy after all), but day by day
and hour by hour.

Venetians call these small plates cicchetti (pronounced ``chi-KET-tee'')
--- said to derive from the Latin ``ciccus,'' meaning ``little'' or
``nothing.'' The term embraces a broad range of dishes: polpette (fried
meatballs), crostini (small open-faced sandwiches), panini (small
sandwiches on crusty rolls), tramezzini (triangular white bread
sandwiches) --- and a scintillating array of pickled, baked, stuffed or
sauced seafoods and vegetables.

You find cicchetti at a bacaro ** (wine bar), but also at a botegòn,
cantina, cicchetteria*,* enoteca ** and ** osteria --- confused yet? And
likely at your neighborhood bar*.* Depending on whom you ask, bacaro **
comes from the Venetian word for ``wine'' or ``a good bar,'' or even
from the ancient Roman god of wine, Bacchus*.*

Venetians eat cicchetti ** at breakfast time, for lunch, dinner and a
midnight snack --- mostly with their fingers. It's look-and-point food:
no special mastery of Italian required. And you don't need to wait to be
seated to enjoy it. (That's not even an option at some bacari*.\emph{)
Cicchetti are cheap, costing on average a couple of euros --- or
dollars, for that matter --- apiece (a bit more for more substantial
seafood or meat cicchetti). Six to eight make a meal, and the local
wines served by the glass are affordable, too. There are few ways more
delectable or fun to get to know La Serenissima than by embarking on a
bar crawl. Here are some of my favorites}.*

\hypertarget{cantine-del-vino-giuxe0-schiavi}{%
\subsection{Cantine del Vino già
Schiavi}\label{cantine-del-vino-giuxe0-schiavi}}

\includegraphics{https://static01.graylady3jvrrxbe.onion/images/2019/03/03/travel/03venice-cicchetti-02/03venice-cicchetti-02-articleLarge.jpg?quality=75\&auto=webp\&disable=upscale}

Let Jiro dream of sushi. Alessandra De Respinis dreams of cicchetti. The
septuagenarian owner of \href{http://www.cantinaschiavi.com/}{this
hobbit-size b}a\href{http://www.cantinaschiavi.com/}{car}o ** in the
artsy Dorsoduro District turns out these diminutive open-faced
sandwiches by the hundreds. And she constantly invents new ones, like a
cicchetto di castagna that plays the earthy sweetness of chestnut purée
against the creamy funk of robiola cheese. Or her gamberi in saór --- a
shrimp riff on classic Venetian sweet and sour sardines piled on a
crusty slice of baguette.

``My customers would revolt if I stopped serving the tartare di tonno,''
said Ms. De Respinis of her caper- and brandy-laced chopped tuna dusted
with unsweetened cocoa powder. That surprising combination won her a
prize at a culinary contest in Mexico City. But Ms. De Respinis is no
globe-trotting chef: day in, day out, you'll find her behind the bar,
white hair swept back, striped apron tied around her waist, with a gold
fork pinned to her blouse --- a gift from a customer in homage to her
preferred cooking utensil. On a given day, già Schiavi serves 25 wines
--- mostly from the Veneto --- by the glass. I'm partial to the
minerally white Orto di Venezia, grown on the nearby island of
Sant'Erasmo. Just don't expect to enjoy it sitting down. As Ms. De
Respinis cautioned about a meal at her cantina: ``You eat, drink and pay
standing up.''

\emph{Cantine del Vino già Schiavi, Dorsoduro 992, Fondamenta Nani;
\href{http://www.cantinaschiavi.com}{www.cantinaschiavi.com}}

Image

The current owners took over All'Arco, found in a tangle of alleyways
near the Rialto fish market, in 1996, but it has been a bacaro for more
than a century.Credit...Andrea Wyner for The New York Times

\hypertarget{allarco}{%
\subsection{All'Arco}\label{allarco}}

Matteo Pinto grates a fresh horseradish root so hard, you can hear it
rasp as the fiery shreds fall on bread slices topped with pink slices of
ham and caramelized onions. His father, Francesco Pinto, pours an ombra
(``shadow,'' literally --- glass of wine in local parlance) for a Rialto
fish merchant speaking Veneziano (a dialect quite distinct from Italian)
with a woman rocking a child in a baby carriage. If the Piazza San Marco
is Europe's drawing room (to quote Napoleon),
\href{https://www.google.com/search?q=All\%E2\%80\%99Arco\&oq=All\%E2\%80\%99Arco\&aqs=chrome..69i57j0l5.1342j0j8\&sourceid=chrome\&ie=UTF-8}{All'Arco}
is its neighborhood tavern. Situated in a maze of alleyways behind the
Rialto fish market, this has been a bacaro ** for more than a century,
explains All'Arco founder Francesco Pinto, who took over the one-room
bar in 1996. He started with a dozen cicchetti --- classic porchetta **
sandwiches*,* for example, and crostini smeared with gorgonzola and
anchovies.

Today, you'll find more than 30 in an ever-changing repertory of
hundreds. ``We make our crostini by the minute, not by the hour,''
explained Matteo Pinto. ``The freshness must be apparent in each bite.''
You don't get much fresher than the season's first canocchie,
supernaturally sweet mantis shrimp fresh from the Lagoon, served atop
tiny arugula leaves and tomatoes. In the winter, you'll find heavier
fare, such as veal stracotto (stew) --- piled on a crusty roll, ** like
some Venetian version of pulled pork. All'Arco takes its name from the
ancient stone arch facing the bar --- said to symbolize the matrimonial
union of two neighboring households centuries ago. The bacaro fills
quickly, soustomers spill out to small tables lining the sidewalk. The
ambience is joyous and jovial --- the perfect embodiment of a Venetian
institution designed not only to slake your thirst and assuage your
appetite, but also to build your sense of community. ``We've lost so
much of our city,'' Mr. Pinto said. ``This is one of the places where
Venetians come to feel Venetian.''

\emph{All'Arco,} \emph{San Polo 436, Calle de l'Ochialer;
011-39-041-520-5666.}

Image

The cicchetti at Osteria al Squero is made with fish like tuna and
salmon, and there are vegetarian selections, including one with
mushrooms, pumpkin and smoked ricotta.Credit...Andrea Wyner for The New
York Times

\hypertarget{osteria-al-squero}{%
\subsection{Osteria al Squero}\label{osteria-al-squero}}

Squero refers to a boatyard where gondolas are built or repaired. This
appropriately named osteria ** faces the Squero San Trovaso --- one of
the last such working boatyards in Venice. Founded by two brothers
(Alessandro Vio runs the front of the house; Cristiano makes the
cicchetti), \href{https://osteriaalsquero.wordpress.com/}{Al Squero} **
draws an animated crowd of art students from the nearby Academia Museum
and tourists from around the planet. Wine bottles line the walls and
cicchetti sparkle in the showcase.

Of course, they serve the ubiquitous baccalà mantecato ** (salt cod
simmered in milk and whipped with oil to a snowy mousse) and sarde in
saór (sweet and sour sardines). But you'll also find such decidedly
untraditional cicchetti as tissue-thin slices of lardo perfumed with
honey, rosemary and pink peppercorns, and crostini heaped with roasted
pumpkin, porcini and ricotta. There are meatless polpette ** for
vegetarians, and in a nod to the ecological concerns of young Venetians,
the cicchetti come on biodegradable plates. The Vio brothers specialize
in wines from northern Italy, including a particularly refreshing J.
Hoffstatter Gewürztraminer from the Alto Adige.

They've also upscaled Venice's indispensable cocktail, the spritz, here
made with your choice of electric orange Aperol, ruby red Campari, or
bracingly bitter artichoke-based Cynar. If you really want to seem in
the know, order a mezzo e mezzo --- prepared with half Aperol and half
Campari. All come festooned with a salty olive on a skewer instead of
the traditional orange slice.

\emph{Osteria al Squero,} \emph{Dorsoduro 943/944, Fondamenta Nani}**;**
\emph{osteriaalsquero.wordpress.com}

Image

Cantina Do Spade is in a storefront that has housed an osteria since
1488. It is known for its meatballs, made with fiery Calabrian
sausage..Credit...Andrea Wyner for The New York Times

\hypertarget{cantina-do-spade}{%
\subsection{Cantina Do Spade}\label{cantina-do-spade}}

To embark on a cicchetti crawl in Venice without trying Do Spade's
polpetta di spianata calabra ** would be like visiting San Marco and
overlooking the basilica. It's a meatball, but, oh, what a meatball:
fiery Calabrian sausage mashed with smoked cheese ** and potatoes, and
lightly breaded and fried. ``We wanted to open a cicchetteria that
serves more than open-faced sandwiches,'' explained Francesco Munarini,
a former bank executive who opened \href{https://cantinadospade.com/}{Do
Spade} a decade ago with his wife, Pilar, and sister, Giovanna. (The
storefront has housed an osteria since 1488.) Inspired by the Rialto
fish market nearby, the Munarini family decided to specialize in seafood
seasoned with the big-flavored spices that made the fortunes of Venetian
traders for centuries.

In quick succession, I downed calamari ripieni (tender squid stuffed
with olives and bread crumbs), fiori de zucca farciti con baccalà
mantecato ** (fried squash flowers filled with creamed codfish),
moscardini in umido ** (stewed baby octopus), la buzara ** (scampi
simmered in ginger- and pepper-piqued tomato sauce) and what may well be
the best sarde in saór in Venice. The local-leaning wine list is
ecumenical enough to include bottles from Istria, now part of Croatia,
but which belonged to Italy before World War II. Recognizable by the
crossed sabers in the window and spillover crowds in the alleyway, Do
Spade (``Two Swords'') offers seating in a warren of simply decorated
blue rooms, but most customers prefer to eat standing by the open
kitchen or in the street.

\emph{Cantina Do Spade,} \emph{San Polo 859, Calle do Spade}**;**
\emph{cantinadospade.com}

\hypertarget{osteria-bancogiro}{%
\subsection{Osteria Bancogiro}\label{osteria-bancogiro}}

For half a millennium or so, Venice dominated Europe's international
commerce, so it should come as no surprise that two modern financial
instruments originated here: the bancarotta (bankruptcy''---``broken
bench,'' literally) and the bancogiro, bank transfer (named for the
world's first publicly funded bank, founded in Venice in 1587). A
lengthy introduction to one of the most scenic barcari in Venice. Housed
in a former vegetable depot,
\href{https://www.osteriabancogiro.it/en/}{Bancogiro} is part wine bar
and part osteria (restaurant). Unlike most bacari, there's outdoor
seating on a wide terrace situated directly on the Grand Canal. (At
lunch and dinner time, these tables are reserved for people who order a
full meal, so arrive early.) If the water were any closer, you'd have to
dine in a gondola (more on that in a minute).

Here, too, seafood figures prominently, from a luscious crostino of
piovra, lardo e melanzana ** (octopus, lardo and eggplant) to
Bancogiro's signature ricotta salata con gamberi al curry (salted
ricotta and curried shrimp over a rectangle of creamy squid ink polenta)
--- the latter popular with the gluten-free crowd. If cold cuts are your
thing, you'll find artisanal mortadella from Bologna dotted with toasted
sweet pistachios, and crostini carpeted with lacy coppa ** (shoulder
ham) cured with Amarone wine.

On any given day, Bancogiro offers 17 wines by the glass, including a
house white blended from Garganega and Durella grapes. After your meal,
follow the signs to the nearby Traghetto Santa Sofia ** for a ride on
what I call a poor man's gondola. Two euros gets you on an oversize
gondola across the Grand Canal in the company of Venetians with their
market bags. Gentlemen take note: It's considered good manners for the
male passengers to remain standing.

\emph{Osteria Bancogiro,} \emph{San Polo 122, Campo San Giacometto;}
******
\emph{\href{http://www.osteriabancogiro.it}{www.osteriabancogiro.it}}

Image

The cicchetti at Basegò are made, variously, with mortadella, anchovies
and salame.Credit...Andrea Wyner for The New York Times

Image

Wine barrels serve as tables at Basegò, and the day's wine selections
are written on a blackboard.Credit...Andrea Wyner for The New York Times

\hypertarget{baseguxf2}{%
\subsection{Basegò}\label{baseguxf2}}

Like most Venetians, Tobia Lenarda deplores mass tourism. So the
one-time conservatory pianist, recognizable by his salt-and-pepper beard
and red sneakers, chose a singular way to fight back: He opened a
new-school bacaro. ``I toured the tapas bars in Spain and Portugal to
get ideas,'' he said. ``I researched sushi and Mexican street food.'' He
called his venture Basegò, the Venetian dialect word for basil. ``It's
clean, it's bright, it's fresh --- just like basil,'' he said. Basegò
doesn't look like your typical bacaro, not with pin spots casting a
discrete light on clean walls of exposed brick, natural wood and white
plaster. Wine barrels and wall shelves serve as tables, with the day's
selection of wines written on a blackboard. The cicchetti are as fresh
as the décor. Tonno afumicato (smoked tuna) comes with avocado
``mayonnaise.'' (``Think of it as Venetian guacamole,'' said Mr.
Lenarda.) Wasabi lights up a salmon cicchetto. Gorgonzola comes with
balsamic vinegar-marinated strawberries.

Equal care goes into Basegò's wines. ``We try to work with heroic
vintners, who grow varieties and make wines no one bothers with any
more.'' One such wine, Calzo della Vignia, comes from Giglio Island in
Tuscany. ``The hills are so steep, they have to harvest the grapes on
foot and by hand,'' Mr. Lenarda said. The result: a golden wine with an
earth taste so rich, you can almost chew it. Mr. Lenarda summed up his
view of customer service this way: ``If you treat you customers politely
--- and that includes tourists --- they pay you back with courtesy.'' He
thinks for a minute and paraphrases John F. Kennedy. ``Ask not what
Venice can do for you. Ask what you can do for Venice.'' The philosophy
earns high praise from the locals. The morning I was there, the only
languages I heard were Italian and Venetian.

\emph{Basegò, San Polo 2863, Calle del Scaleter; 011-39-041-850-0299.}

Image

Bar 5000 has a 120-bottle wine list, including organic, biodynamic and
vegan selections.Credit...Andrea Wyner for The New York Times

Image

There are six to eight types of cicchetti in Bar 5000's showcase daily,
with ingredients like mortadella, goat cheese and
pistachio.Credit...Andrea Wyner for The New York Times

\hypertarget{bar-5000}{%
\subsection{Bar 5000}\label{bar-5000}}

Bacari specialize in wine, of course, and
\href{http://www.lunasentada.it/home}{Bar 5000} takes that mandate
seriously, offering an impressive selection of vino bio, vino
biodinámico and vino vegano. The first is organic wine (made from grapes
grown without chemical fertilizers or fungicides), while the second are
wines vinified without supplemental yeast or other additives. As for
vegan wines, they're clarified without gelatin, a fining agent derived
from animal bones. And all three are available on a 120-bottle list at
this new-school wine bar, located on the tranquil Campo San Severo in
the Castello District near Piazza San Marco. Gone, the mosh pit crowds
of the Rialto bacari. The clean modern interior runs to up-lit brick
walls, polished concrete floors, and a chandelier blown by the Murano
glass master Fabio Fornasier. Weather permitting, you can sit at one of
a handful of tables along the quiet Severno canal.

When it comes to the cicchetti, Bar 5000 may lack the jaw-dropping
variety of All'Arco or già Schiavi, but the six to eight daily
selections in the showcase are thoughtfully chosen and well prepared. A
plump salty sun-dried tomato crowns a crostino of sopressata cut
paper-thin on a Berker meat slicer. Fresh oranges and mostarda (fruit
jam) counterpoint a tiny wedge of Monte Veronese cheese. The pickles
come from vegetables grown on Sant'Erasmo Island. ``We bake our own
bread daily,'' said Micael Nordio. co-owner of Bar 5000. ``We don't have
a freezer, so you know our food is fresh.'' Twice a month, Bar 5000
stages wine dinners, often with live music. If you're still hungry after
the cicchetti, you can go for a proper meal at the sister restaurant,
Luna Sentada, next door.

\emph{Bar 5000,} \emph{Castello 5000, Campo San Severo}***;***
\emph{\href{http://www.lunasentada.it}{www.lunasentada.it}}

\begin{center}\rule{0.5\linewidth}{\linethickness}\end{center}

\emph{Steven Raichlen is a longtime food and travel journalist with an
abiding passion for Italy. Two of the 32 books he has written are on
Italian cooking, and he hosts ``Steven Raichlen Grills Italy,'' a
television show on Gambero Rosso, the Italian food network.}

\begin{center}\rule{0.5\linewidth}{\linethickness}\end{center}

\emph{Follow} \href{https://twitter.com/nytimestravel}{\emph{NY
Times}}\href{https://twitter.com/nytimestravel}{\emph{Travel on
Twitter}}\emph{,}
\href{https://www.instagram.com/nytimestravel/}{\emph{Instagram}}
\emph{and}
\href{https://www.facebookcorewwwi.onion/nytimestravel/}{\emph{Facebook}}\emph{.}
\href{https://www.nytimes3xbfgragh.onion/newsletters/traveldispatch}{\emph{Get
weekly updates from our Travel Dispatch newsletter, with tips on
traveling smarter, destination coverage and photos from all over the
world.}}

Advertisement

\protect\hyperlink{after-bottom}{Continue reading the main story}

\hypertarget{site-index}{%
\subsection{Site Index}\label{site-index}}

\hypertarget{site-information-navigation}{%
\subsection{Site Information
Navigation}\label{site-information-navigation}}

\begin{itemize}
\tightlist
\item
  \href{https://help.nytimes3xbfgragh.onion/hc/en-us/articles/115014792127-Copyright-notice}{©~2020~The
  New York Times Company}
\end{itemize}

\begin{itemize}
\tightlist
\item
  \href{https://www.nytco.com/}{NYTCo}
\item
  \href{https://help.nytimes3xbfgragh.onion/hc/en-us/articles/115015385887-Contact-Us}{Contact
  Us}
\item
  \href{https://www.nytco.com/careers/}{Work with us}
\item
  \href{https://nytmediakit.com/}{Advertise}
\item
  \href{http://www.tbrandstudio.com/}{T Brand Studio}
\item
  \href{https://www.nytimes3xbfgragh.onion/privacy/cookie-policy\#how-do-i-manage-trackers}{Your
  Ad Choices}
\item
  \href{https://www.nytimes3xbfgragh.onion/privacy}{Privacy}
\item
  \href{https://help.nytimes3xbfgragh.onion/hc/en-us/articles/115014893428-Terms-of-service}{Terms
  of Service}
\item
  \href{https://help.nytimes3xbfgragh.onion/hc/en-us/articles/115014893968-Terms-of-sale}{Terms
  of Sale}
\item
  \href{https://spiderbites.nytimes3xbfgragh.onion}{Site Map}
\item
  \href{https://help.nytimes3xbfgragh.onion/hc/en-us}{Help}
\item
  \href{https://www.nytimes3xbfgragh.onion/subscription?campaignId=37WXW}{Subscriptions}
\end{itemize}
