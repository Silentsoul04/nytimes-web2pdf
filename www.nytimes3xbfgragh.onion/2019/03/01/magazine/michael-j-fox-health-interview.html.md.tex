Sections

SEARCH

\protect\hyperlink{site-content}{Skip to
content}\protect\hyperlink{site-index}{Skip to site index}

\href{https://myaccount.nytimes3xbfgragh.onion/auth/login?response_type=cookie\&client_id=vi}{}

\href{https://www.nytimes3xbfgragh.onion/section/todayspaper}{Today's
Paper}

Michael J. Fox on His Health and Career: Six Things We Learned From Our
Interview

\url{https://nyti.ms/2VoPT98}

\begin{itemize}
\item
\item
\item
\item
\item
\end{itemize}

Advertisement

\protect\hyperlink{after-top}{Continue reading the main story}

Supported by

\protect\hyperlink{after-sponsor}{Continue reading the main story}

\hypertarget{michael-j-fox-on-his-health-and-career-six-things-we-learned-from-our-interview}{%
\section{Michael J. Fox on His Health and Career: Six Things We Learned
From Our
Interview}\label{michael-j-fox-on-his-health-and-career-six-things-we-learned-from-our-interview}}

\includegraphics{https://static01.graylady3jvrrxbe.onion/images/2019/03/03/magazine/03mag-talk-slide-QIIM/03mag-talk-slide-QIIM-articleLarge.png?quality=75\&auto=webp\&disable=upscale}

By The New York Times Magazine

\begin{itemize}
\item
  March 1, 2019
\item
  \begin{itemize}
  \item
  \item
  \item
  \item
  \item
  \end{itemize}
\end{itemize}

In 1985, Michael J. Fox was one of the biggest stars in the world:
``Back to the Future'' and ``Teen Wolf'' occupied the top two spots at
the box office, and ``Family Ties,'' the long-running sitcom on which he
played Alex P. Keaton, was at its peak. ``I wanted to be a rock star,''
\href{https://www.nytimes3xbfgragh.onion/interactive/2019/03/01/magazine/michael-j-fox-parkinsons-acting.html}{Fox
told David Marchese in this week's Talk interview}. ``That's what I
thought being famous was. But I wasn't a rock star. I was kind of an
idiot. I missed the point.'' Six years later, Fox found out he had
Parkinson's disease, which he credits with getting him on track. Fox
still acts occasionally --- later roles include those on ``Scrubs'' and
``Boston Legal'' --- but has dedicated his life to advocating a cure for
Parkinson's. Fox and Marchese talked about acting with Parkinson's,
taking the wrong roles and staying positive: ``Until it's not funny
anymore, it \emph{is} funny.''

\emph{{[}Read our full}
\href{https://www.nytimes3xbfgragh.onion/interactive/2019/03/01/magazine/michael-j-fox-parkinsons-acting.html}{\emph{interview
with Michael J. Fox}}\emph{.{]}}

\textbf{He still believes in a cure for Parkinson's.}

His foundation, which has raised \$800 million to combat the disease,
has a working relationship with the government. (Still, Fox found
himself upset when President Trump appeared to mock Serge Kovaleski, a
reporter for The New York Times who has the joint condition
arthrogryposis. But he didn't feel the need to say anything in response:
``People already know Trump is an {[}expletive{]}.'') He's still hopeful
for a cure. In the interim, he's optimistic about a new drug that acts
like a rescue inhaler when people freeze, but knows that prophylactics
are not a cure --- but they're better than nothing.

\textbf{He worried about presenting a ``false hope'' about his
diagnosis.}

``I'd made peace with the disease but presumed others had that same
relationship when they didn't,'' he told Marchese. But then his
experience got worse: He found himself in a position where he couldn't
walk and had round-the-clock aides, and was unsure if he could still
keep up his hopeful disposition. He's working on a new book about wading
through all this.

\textbf{His response to the diagnosis was to make a number of comedic
films: ``For Love or Money,'' ``Life With Mikey'' and ``Greedy,'' which
were critical and commercial flops.}

``I was so scared,'' Fox told Marchese. He wishes, instead, that he had
opted for quality over quantity: to have made as many \emph{good} movies
as possible, instead of just going along as quickly as he could. It took
him about five years to fully accept his diagnosis --- around the time
he started onwhen he did ``Spin City.''

\emph{{[}Read our full}
\href{https://www.nytimes3xbfgragh.onion/interactive/2019/03/01/magazine/michael-j-fox-parkinsons-acting.html}{\emph{interview
with Michael J. Fox}}\emph{.{]}}

\textbf{He left ``Spin City'' when he realized that his particular
acting charm was starting to fade.}

``I used a lot of high-level muggery,'' he said. ``I could pull a face;
I could do a double take. And one of the reasons I left `Spin City' was
that I felt my face hardening.'' In the last few seasons, he'd have to
anchor himself against a wall or piece of furniture --- until it got too
difficult.

\textbf{He knows why ``The Michael J. Fox Show'' didn't succeed.}

Fox played a news anchor dealing with his own Parkinson's diagnosis on
the short-lived sitcom. He said he didn't have the energy to keep the
show on track, but he also puts some of the blame on NBC, wondering if
the network was prepared to actually work with him --- and his disease.
``This is probably unfair, but I feel like one day they woke up and
said, `Oh, he really has Parkinson's.' Like somebody saw me tremoring in
rehearsal and said: `What's wrong with him?' `Uh, he has Parkinson's
remember? It's the premise of the show.' ''

\textbf{He thinks there's a ``millennial sense of humor.''}

The difference between an older sitcom and a more contemporary one? The
amount of time to set up a joke. Millennial humor is abstract and
piecemeal: ``It's like memes: You don't have to tell the whole joke.''
On ``Family Ties,'' he could look at a co-star, go to the fridge, get a
drink, take off his coat, sit down and \emph{then} deliver his punch
line. ``That doesn't exist anymore because people are more
sophisticated.''

Advertisement

\protect\hyperlink{after-bottom}{Continue reading the main story}

\hypertarget{site-index}{%
\subsection{Site Index}\label{site-index}}

\hypertarget{site-information-navigation}{%
\subsection{Site Information
Navigation}\label{site-information-navigation}}

\begin{itemize}
\tightlist
\item
  \href{https://help.nytimes3xbfgragh.onion/hc/en-us/articles/115014792127-Copyright-notice}{©~2020~The
  New York Times Company}
\end{itemize}

\begin{itemize}
\tightlist
\item
  \href{https://www.nytco.com/}{NYTCo}
\item
  \href{https://help.nytimes3xbfgragh.onion/hc/en-us/articles/115015385887-Contact-Us}{Contact
  Us}
\item
  \href{https://www.nytco.com/careers/}{Work with us}
\item
  \href{https://nytmediakit.com/}{Advertise}
\item
  \href{http://www.tbrandstudio.com/}{T Brand Studio}
\item
  \href{https://www.nytimes3xbfgragh.onion/privacy/cookie-policy\#how-do-i-manage-trackers}{Your
  Ad Choices}
\item
  \href{https://www.nytimes3xbfgragh.onion/privacy}{Privacy}
\item
  \href{https://help.nytimes3xbfgragh.onion/hc/en-us/articles/115014893428-Terms-of-service}{Terms
  of Service}
\item
  \href{https://help.nytimes3xbfgragh.onion/hc/en-us/articles/115014893968-Terms-of-sale}{Terms
  of Sale}
\item
  \href{https://spiderbites.nytimes3xbfgragh.onion}{Site Map}
\item
  \href{https://help.nytimes3xbfgragh.onion/hc/en-us}{Help}
\item
  \href{https://www.nytimes3xbfgragh.onion/subscription?campaignId=37WXW}{Subscriptions}
\end{itemize}
