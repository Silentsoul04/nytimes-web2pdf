Sections

SEARCH

\protect\hyperlink{site-content}{Skip to
content}\protect\hyperlink{site-index}{Skip to site index}

\href{https://www.nytimes3xbfgragh.onion/section/style}{Style}

\href{https://myaccount.nytimes3xbfgragh.onion/auth/login?response_type=cookie\&client_id=vi}{}

\href{https://www.nytimes3xbfgragh.onion/section/todayspaper}{Today's
Paper}

\href{/section/style}{Style}\textbar{}How to Talk to Your Kids About
Drugs When Everyone Is Doing Them

\url{https://nyti.ms/2HiJnwN}

\begin{itemize}
\item
\item
\item
\item
\item
\end{itemize}

Advertisement

\protect\hyperlink{after-top}{Continue reading the main story}

Supported by

\protect\hyperlink{after-sponsor}{Continue reading the main story}

Rites of passage

\hypertarget{how-to-talk-to-your-kids-about-drugs-when-everyone-is-doing-them}{%
\section{How to Talk to Your Kids About Drugs When Everyone Is Doing
Them}\label{how-to-talk-to-your-kids-about-drugs-when-everyone-is-doing-them}}

At least my ``this is your brain on drugs'' generation understood fear.
Now, narcotics are the new boxed wine.

\includegraphics{https://static01.graylady3jvrrxbe.onion/images/2019/03/07/style/07Rites-drugs-1/07Rites-drugs-1-articleLarge.jpg?quality=75\&auto=webp\&disable=upscale}

By David Hochman

\begin{itemize}
\item
  March 7, 2019
\item
  \begin{itemize}
  \item
  \item
  \item
  \item
  \item
  \end{itemize}
\end{itemize}

The assignment was to write a four-page research paper on any biology
topic. My son, Sebastian, is a high school freshman, and it was his
first real chance to shine. I expected him to pick something like
photosynthesis. He went with psychedelic drugs instead.

Let me tell you what would have happened if I had made that choice as a
ninth-grader: I would have been grounded until graduation. In
northeastern Pennsylvania, where I grew up, my mother worked for the
county commission on drug and alcohol abuse, and she could literally
smell stoned people. The breath of a pothead, she warned, as if hunting
dragons, has the odorousness of burned rope.

One night she shook me awake after finding a tiny tube of Krazy Glue
under the seats of her Buick Skyhawk.

``Are you sniffing this stuff to get high?'' she said.

I wasn't. I didn't even know that was a thing. I cried.

In my mom's defense, the narc approach was more common back then. In
1984, when I was in high school, Nancy Reagan first uttered the phrase
``Just say no'' after a student asked what to do if friends offered her
funny cigarettes. Later, the public ad campaign
``\href{https://www.youtube.com/watch?v=3FtNm9CgA6U}{This Is Your Brain
on Drugs}'' made us believe our teenage minds would be fried --- sunny
side up --- if we even thought about weed.

None of it worked --- have you seen ``Pineapple Express''? --- but at
least my generation understood fear. Today in California, where my
family lives, drugs are the new boxed wine, ho-hum even in suburbia.

The gleaming MedMen marijuana dispensary near us has touch screens and
roving geniuses like an Apple store. Friends of mine with actual jobs
and offspring spend their weekends vomiting on psychiatrist-led
ayahuasca retreats. In Silicon Valley, microdosing LSD for focus and
creativity is as common as bringing your Labradoodle to work. And dinner
parties aren't dinner parties lately until the conversation shifts to
books by
\href{https://www.nytimes3xbfgragh.onion/2018/12/24/books/review/psychedelics-how-to-change-your-mind.html}{Michael
Pollan} and
\href{https://www.nytimes3xbfgragh.onion/2017/01/07/style/microdosing-lsd-ayelet-waldman-michael-chabon-marriage.html}{Ayelet
Waldman} on how dropping acid is a cure-all for depression, anxiety, and
middle-class, middle-age ennui --- turn on, tune in and binge watch
``The Marvelous Mrs. Maisel.''

Having the drug talk can be a lonely task at a time when getting baked
seems fine with everyone and their mother. Although, not my mother.
She's still not imbibing, and I'm holding on to some of that
generational concern.

My struggle is to update the proactive messaging for the age of the
legal buzz. I don't care that every other billboard in Los Angeles makes
casual drug use look as harmless as hailing a Lyft; I don't want my
child messing with his still-developing brain and body. I don't want him
self-medicating to escape. I definitely don't want him
\href{https://www.nytimes3xbfgragh.onion/2018/04/07/style/the-juul-is-too-cool.html}{Juuling},
snorting, dosing or dabbing while driving. I also don't want to wait
until his grades tank, and I catch him gorging on raw cookie dough and
``Dark Side of the Moon'' before stepping in. (I know this makes me
sound very old-fashioned.)

That's why I created ``prehab.''

If rehab is a way to recover from drug addiction, prehab is a program
designed (by me, at least) to avert the whole mess in the first place.
It's like pulling a Marty McFly with your future party self, pre-empting
poor decisions way before you're drunk texting your ex and sleeping
through Mondays. So I enrolled Sebastian as the test case.

Step one: not turning into my mother. I wasn't going to storm into
Sebastian's room in riot gear or hold his eyelids open as I flashed
before-and-after images of meth addicts. Tone mattered, and so did
timing. I didn't want to submit him to the claustrophobia of a car-ride
disquisition on the way to school (``Hugs not drugs, kid''). So, I
started the way nearly all conversations do these days between fathers
and sons: with both of us on the couch, slumped over our phones.

``You know I really, really care about you, right?''

``Um, yesss,'' Sebastian said, and already I was failing.

``It's just that your choice of LSD as a science topic scared me a
little and I want to make sure that you're, you know, fine.''

Oh God, oh God, Oh God.

``I'm fine,'' he said.

``Can you say what interests you about psychedelics?''

``What's interesting is that they're interesting.''

O.K. It was a start, and it led to step two: a quick course in drug
education. Mine, not his. Sebastian told me about ``this scientist guy
named Albert Hofmann'' who in the 1930s studied rye or, more
specifically, a certain fungus that grows on rye called ergot. As I
worked to restrain all muscle movements in my face, Sebastian read off
his screen from his paper in progress:

``Hofmann synthesized LSD-25, or Lysergic Acid Diethylamide, as part of
his studies and put it on a shelf for several years. Those years ended
abruptly when he ingested a dose of the substance and immediately had
one of the weirdest experiences of his life, saying that the devil had
taken over his body.'' Hofmann, incidentally, died in 2008, at age 102.
What a long, strange trip it was.

Sebastian followed that with a mini-TED Talk on dopamine, glutamate and
serotonin receptors, covering the various pleasure sensations unleashed
when certain neurons are activated by tiny tabs of acid-blotted paper.

Interesting for sure. I was both impressed and terrified. It seemed like
two minutes ago that the boy was giggling in the back seat over
``Phineas and Ferb.'' My impulse was to immediately commence home
schooling.

Instead, I asked if he had come across anything confusing or scary that
I could help clarify (``Not so far''), and, just for the record, if he
knew how I felt about him taking drugs now or in the future.

``Bad?'' he said.

``It's more complicated than that,'' I said. ``I can't force you not to
do drugs even though I think you should hold out as long as possible for
health and safety reasons. There are going to be a lot of temptations,
and maybe there already are, but you can avoid them, and we can talk
about how. You also might find it weird that pot is now legal for
adults. But so is alcohol, and they're both illegal until you're 21.''

I could tell he was drifting, so I let him get back to YouTube. I said I
would be available anytime to answer questions, even crazy ones, as he
dug deeper into the research.

Step three happened over the next couple weeks. I took Sebastian to see
``Beautiful Boy,'' in which Steve Carell essentially plays me in my
panic vision of a teenage son's addiction run amok.

I told him about the graduate school friend of mine who got so stupid
while smoking hashish on a camping trip that he spent a half-hour
looking for his flashlight. With his flashlight.

I made a dad joke about not lighting up, but Sebastian wasn't laughing.
He looked at me and said, ``I actually understand everything you're
telling me. If it makes any difference to you, I'm not really interested
in trying drugs. But that could change someday. I really don't know. For
now, can we be done talking about this?''

It hit me in that moment that Sebastian wasn't the one needing prehab. I
was. I'd been so focused on finding the formula for guiding him down the
right path that I forgot that he's the only one who can make these
decisions. My task isn't to steer him around every bong and backyard keg
gathering for the rest of his childhood. It's to be there for him, to
set basic guidelines. I also want to support him in figuring out what he
truly enjoys, what he dreams about for his future, what he needs more of
--- and less of --- from me. I'd do anything to help him make the most
of this one shot he's got.

Above all, prehab is about letting go. I can talk to Sebastian about
drugs until the psychedelic cows come home but I can't control his
future, and why would I want to? Even if he ends up experimenting, I
need to have the faith that he'll be fine. Would it have done John
Lennon's father any good to worry about his kid's toking habits? Would I
be typing on this Apple computer if Steve Jobs just said no to LSD?

As a parent, I certainly don't have all the answers. But a few weeks
into prehab, treatment appeared to be going well, for both of us.
Sebastian had a hilarious part in a musical. He was writing a column for
the student newspaper. He had a sideline gig as his high school's
mascot, J.J. the Jaguar. I tell him I love him (without a D.A.R.E.
lecture) every night, and on one of those nights he was thrilled to
report he had gotten an A on that science essay.

As they say in prehab, one day at a time.

David Hochman is a writer in Los Angeles.

\emph{Rites of Passage is a weekly-ish column from Styles and The Times
Gender Initiative. For information on how to submit an essay,}
\href{https://www.nytimes3xbfgragh.onion/2018/07/06/style/how-to-submit-a-rites-of-passage-essay.html?module=inline}{\emph{click
here}}\emph{. ​To read past essays,}
\href{https://www.nytimes3xbfgragh.onion/column/rites-of-passage}{\emph{check
out this page}}\emph{.}

Advertisement

\protect\hyperlink{after-bottom}{Continue reading the main story}

\hypertarget{site-index}{%
\subsection{Site Index}\label{site-index}}

\hypertarget{site-information-navigation}{%
\subsection{Site Information
Navigation}\label{site-information-navigation}}

\begin{itemize}
\tightlist
\item
  \href{https://help.nytimes3xbfgragh.onion/hc/en-us/articles/115014792127-Copyright-notice}{©~2020~The
  New York Times Company}
\end{itemize}

\begin{itemize}
\tightlist
\item
  \href{https://www.nytco.com/}{NYTCo}
\item
  \href{https://help.nytimes3xbfgragh.onion/hc/en-us/articles/115015385887-Contact-Us}{Contact
  Us}
\item
  \href{https://www.nytco.com/careers/}{Work with us}
\item
  \href{https://nytmediakit.com/}{Advertise}
\item
  \href{http://www.tbrandstudio.com/}{T Brand Studio}
\item
  \href{https://www.nytimes3xbfgragh.onion/privacy/cookie-policy\#how-do-i-manage-trackers}{Your
  Ad Choices}
\item
  \href{https://www.nytimes3xbfgragh.onion/privacy}{Privacy}
\item
  \href{https://help.nytimes3xbfgragh.onion/hc/en-us/articles/115014893428-Terms-of-service}{Terms
  of Service}
\item
  \href{https://help.nytimes3xbfgragh.onion/hc/en-us/articles/115014893968-Terms-of-sale}{Terms
  of Sale}
\item
  \href{https://spiderbites.nytimes3xbfgragh.onion}{Site Map}
\item
  \href{https://help.nytimes3xbfgragh.onion/hc/en-us}{Help}
\item
  \href{https://www.nytimes3xbfgragh.onion/subscription?campaignId=37WXW}{Subscriptions}
\end{itemize}
