Sections

SEARCH

\protect\hyperlink{site-content}{Skip to
content}\protect\hyperlink{site-index}{Skip to site index}

Railing Against India's Right-Wing Nationalism Was a Calling. It Was
Also a Death Sentence.

\url{https://nyti.ms/2u7IhfA}

\begin{itemize}
\item
\item
\item
\item
\item
\item
\end{itemize}

\includegraphics{https://static01.graylady3jvrrxbe.onion/images/2019/03/17/magazine/17mag-Lankesh-image1/17mag-Lankesh-image1-articleLarge-v3.jpg?quality=75\&auto=webp\&disable=upscale}

Feature

\hypertarget{railing-against-indias-right-wing-nationalism-was-a-calling-it-was-also-a-death-sentence}{%
\section{Railing Against India's Right-Wing Nationalism Was a Calling.
It Was Also a Death
Sentence.}\label{railing-against-indias-right-wing-nationalism-was-a-calling-it-was-also-a-death-sentence}}

How the journalist Gauri Lankesh became a casualty of India's
increasingly intolerant politics.

Gauri Lankesh in 2015.Credit...Esha Lankesh

Supported by

\protect\hyperlink{after-sponsor}{Continue reading the main story}

By Rollo Romig

\begin{itemize}
\item
  March 14, 2019
\item
  \begin{itemize}
  \item
  \item
  \item
  \item
  \item
  \item
  \end{itemize}
\end{itemize}

Gauri Lankesh usually worked late on Tuesday nights. The exuberantly
leftist weekly newspaper she edited, Gauri Lankesh Patrike, went to
press on Wednesdays, and she had to finalize the articles. But on
Tuesday, Sept. 5, 2017, she drove home early, around 7:45 p.m.; she had
an evening appointment with cable repairmen to fix her TV.

The last person she spoke to before leaving the office was Satish, the
paper's I.T. manager. Money was always tight because of her refusal to
allow advertisements in the newspaper, which she felt was necessary to
shield it from the corruption and outside pressure that often compromise
the Indian press. Gauri Lankesh Patrike ran on subscriptions and
newsstand sales, supplemented by a book-publishing sideline. But the
paper's financial situation had become so dire that she had decided, for
the first time, to run ads in a forthcoming special holiday issue. She
asked Satish (who goes by a single name) to start soliciting them the
next day.

At its peak, Gauri Lankesh Patrike's circulation numbered only in the
high four digits, and Lankesh mostly wrote in Kannada, a regional
language understood by only 3.6 percent of Indians (though in
hyper-populous India, that is 48 million people, more than the total
population of Spain). But her political activism and her lively social
media presence extended her reach far beyond the paper's print run. At a
time of intense vitriol against the press in India, she was a fearless,
sometimes reckless critic of the right-wing, Hindu-nationalist Bharatiya
Janata Party, or B.J.P., which has held power in India since 2014. Her
paper was a tabloid in every sense, gleefully sensational and
indifferent to decorum. But the vehemence and humor of her polemics in
defense of pluralism and minority rights had made her a beloved figure
to an increasingly embattled opposition.

She was more vulnerable than she sounded on the page. She reminded one
friend of a sparrow: her head topped with a feathery whorl of short gray
hair, bursting with noisy argument but fundamentally gentle. At 55 she
was five feet and a half inch tall --- she always insisted on the half
inch, her ex-husband said --- and skinny, possibly because of her heavy
smoking and her tendency to work through mealtimes.

She lived alone in an unusually quiet pocket of Bangalore, the capital
city of the south Indian state of Karnataka. Her lone concession to
friends and family concerned about her safety was a few closed-circuit
TV cameras she installed half a year earlier --- cameras that captured
some of what happened on the night of Sept. 5.

Just after 8 p.m. she parked her car, a compact white Toyota, at an
indifferent angle, then jumped out to open the gate. From the camera
footage, it appeared that she hadn't noticed the motorcycle with two
riders that had followed her home. The moment she got her gate open, the
motorcycle's passenger rushed up and shot her with a crude pistol. Two
bullets hit her in the abdomen, one passing through her liver.

Lankesh turned to run, and the third shot missed her and struck a wall.
A fourth bullet hit her in the back, passing through a lung and grazing
her heart before exiting through the left cup of her bra. The whole
encounter lasted about five seconds. Within a minute, the cable
repairmen pulled up and found her splayed across the entryway to her
house in a pool of blood.

About 20,000 people attended a Bangalore rally in her honor a week
later. Her friends marveled not only at the number of supporters but at
their variety: writers, students, activists, members of the marginalized
Dalit and Adivasi communities, transgender women, rickshaw drivers,
landless farmers, Muslims, Christians. Large ``I Am Gauri''
demonstrations arose nationwide in outrage at the increasing attacks,
rhetorical and physical, on Indian journalists. Narendra Modi, the prime
minister, routinely tweets condolences after airplane crashes in foreign
countries but made no comment about Lankesh's murder.

\includegraphics{https://static01.graylady3jvrrxbe.onion/images/2019/03/17/magazine/17mag-Lankesh-image2/17mag-Lankesh-image2-articleLarge.jpg?quality=75\&auto=webp\&disable=upscale}

The Committee to Protect Journalists has been keeping track of 35 cases
of Indian journalists murdered specifically for their work since 1992,
and only two of these cases have resulted in a successful conviction.
``There seems to be a license that people feel to beat up or attack
journalists in India,'' Steven Butler, the coordinator of the
committee's Asia program, told me when I met him in December 2017. As we
spoke, his phone buzzed: Another journalist had been arrested in
Kashmir.

India's newspaper culture has long been among the most varied and
vigorous in the world, which the country's free-speech laws help enable.
But the protections offered by those laws have always been as tenuous as
they are broad. The country has no explicit constitutional protection of
freedom of the press, and the laws that do exist are easily curtailable
in the interest of security, public decency or religious sentiment. Its
sluggish judicial system can be exploited to harass journalists, and
endemic corruption forever threatens to compromise their work.

The situation has unquestionably deteriorated over the past several
years --- a fact that owes much to the ascent of the B.J.P. In the 2014
elections, the party won 282 of the 545 seats in the lower house of
India's Parliament, which determines the prime ministership. The
Congress Party, which has led nearly every Indian government since
independence, won only 44.

Political pressure on journalists is nothing new in India, but the
current government is the first in many years to treat them as an
ideological enemy. Since he took office in 2014, Modi has not held a
single news conference in India. Among B.J.P. politicians, a popular
term for journalists is ``presstitutes.'' A dispatch on Indian
journalism last year by the Committee to Protect Journalists described
an unprecedented climate of self-censorship and fear, reporting, ``The
media is in the worst state India has ever seen.''

In these circumstances, Lankesh's audacity and integrity were all the
more notable. And her murder has deepened the chill. The anonymous
author of Humans of Hindutva, a popular Facebook page satirizing the
religious right wing, abruptly shut it down twice in 2017 after posting
about receiving death threats (though the page has since returned). ``I
have no desire to end up like Gauri Lankesh,'' the author wrote. A young
investigative reporter named Aruna Chandrasekhar told me that Lankesh's
example had been particularly inspiring to Indian women freelance
journalists, and that when she found herself feeling vulnerable while
reporting a story alone in an unfamiliar place, the thought of Lankesh's
fearlessness used to embolden her. ``Gauri's murder shook me,'' she
said.

By the end of May, national elections will determine if Modi and the
B.J.P. are elected to another five years. It is likely to be the ugliest
campaign season in India's history. Hostility toward journalists and
opposition figures is intensifying as voting day approaches. The
investigative journalist Rana Ayyub, best known for her investigation
into B.J.P. complicity in religious riots (which Lankesh had published
in a Kannada translation), wrote in a Times Op-Ed last year that she has
been the target of an unrelenting online assault by right-wing
activists: her face was grafted on a pornographic video; her home
address and phone number were circulated; there were threats of gang
rape.

In August, near a public event in New Delhi called Freedom From Fear, an
unknown gunman tried to shoot Umar Khalid, a student activist who was
close to Lankesh. The police last year arrested 11 opposition activists
and lawyers on what appear to have been flimsy charges of instigating
violence at an event in Maharashtra, and in February arrested, on
apparently even thinner evidence, the prominent caste scholar Anand
Teltumbde. Tensions have further risen since Feb. 14, when a suicide
bomber killed 40 Indian soldiers in Kashmir, setting off a series of
skirmishes with Pakistan that are likely to politically benefit the
B.J.P.

Jignesh Mevani, a legislator and an activist from Gujarat, fears that if
the B.J.P. is re-elected, its extremist supporters will be emboldened.
``Every year they will kill 10 to 15 of our kind of people and put 10 to
15 of our kind of people in jail,'' he told me at a July meeting in
Bangalore in Lankesh's honor. ``So by the time they are in power for a
decade, the major faces of the progressive civil rights movements of
this country will be gone.''

Image

The body of Gauri Lankesh is brought to the Ravindra Kalakshetra
cultural center as people gather to pay their respects on Sept. 6,
2017.Credit...Manjunath Kiran/Agence France-Presse --- Getty Images

Lankesh's murder seemed to fit what was by then an unmistakable pattern
of assassinations of intellectuals who opposed the fundamentalist-Hindu
ideology that animates the B.J.P., all of which remained unsolved.
Between 2013 and 2015, three religiously freethinking Indian writers and
activists were shot dead near their homes by assailants who escaped on
motorcycles: the doctor Narendra Dabholkar, in Pune; the politician
Govind Pansare, in Kolhapur; and the scholar M.M. Kalburgi, in Dharwad.
After Kalburgi's murder, scores of Indian writers returned their awards
from the National Academy of Letters to protest both the lack of
progress in the murder investigations and the B.J.P.'s silence over
rising intolerance, to no effect. There was much anxious speculation
over who might be the next writer to die. But few thought it would be
Lankesh, in part simply because she lived in Bangalore.

Situated on a plateau at the center of India's southern triangle,
Bangalore has a mild climate year-round, a condition that seems to have
nourished the city's reputation as an easygoing, tolerant place. It is
often said that the city's slogan is the Kanglish phrase ``swalpa adjust
maadi'' --- or, ``please adjust a little.'' Bangalore reflects India's
diversity --- its mélange of cultures, languages, religions and
histories --- more than most places. It is a city that attracts migrants
from all over India, few of whom speak Kannada, the official language of
Karnataka, as their primary tongue. India's science-research efforts
have centered on Bangalore for more than a century as has, in recent
decades, its information-technology industry, and the city consequently
has one of the world's most educated work forces. By some accounts, its
most intractable problem is traffic. According to the Karnataka Police,
a year can pass in Bangalore without a single instance of a gun used in
a crime.

To many Bangaloreans, Lankesh's murder felt like the violent
announcement of the end of an era --- an era that had arguably sprung
from the imagination of Lankesh's father, P. Lankesh. A commanding
figure with huge eyeglasses and a generous mustache, Lankesh was a
compulsively productive, endlessly quarrelsome English professor,
fiction writer, poet, playwright, filmmaker, essayist and journalist. He
dominated and in many ways dictated the cultural and political discourse
in the state of Karnataka to a degree unimaginable before or after the
20 years in which he edited Lankesh Patrike, the tabloid he founded in
1980.

Gauri Lankesh's ex-husband, the journalist Chidanand Rajghatta,
describes Lankesh Patrike --- the name, in Kannada, simply means
``Lankesh's newspaper'' --- as ``a weird mixture of high literary essay
combined with low political tattle,'' like an unlikely merger of The New
Yorker and The New York Post, but with a delightful idiom all its own.
``There was everything in that paper: great politics, great literature,
great gossip,'' the journalist Sugata Srinivasaraju told me. ``He
translated Baudelaire, he translated Rimbaud, then he talked about
Barthes. You are sitting in this corner of Karnataka and you are being
introduced to the world.''

Gauri Lankesh grew up in her father's shadow, and at first she kept her
distance from the Kannada literary scene he personified. She and
Rajghatta took jobs in Delhi, far from Karnataka, and wrote exclusively
in English. ``We were completely Anglicized,'' Rajghatta told me.
``Deracinated. We'd forgotten our roots.'' When P. Lankesh died in 2000,
it was unthinkable that anyone could fill his shoes --- least of all his
daughter, who was then barely literate in Kannada. But her family legacy
proved irresistible, and she moved back to Bangalore to serve as the
paper's editor.

It was an impossible job, but Lankesh found she loved it. She never
approached her father's literary talents in Kannada but was his equal in
pluck. Shortly after she assumed the editorship, a journalist
interviewing her noted that her father had often been threatened and
insulted by his critics. ``I am not afraid of physical attacks at all,''
she said. ``Being a woman is my security right now.''

The job radicalized her. After spending much of her adult life removed
from Karnataka, she suddenly found herself immersed in its problems: the
labor complaints of Bangalore's municipal sanitation workers or the
persistence of retrograde local superstitions such as made snana,
wherein lower-caste Hindus roll on the ground over leftover food from a
ceremonial meal eaten by Brahmins. (The practice was finally outlawed in
Karnataka in 2017.) The experience transformed her into a leftist and an
activist, and Lankesh Patrike transformed with her. Its new direction
led to an ideological rift with the paper's owner and publisher, her
brother Indrajit. In 2005, she left the paper, and the next week she
started a new tabloid of her own: Gauri Lankesh Patrike.

In a cave high in the mountains of central Karnataka there is a
religious shrine called Baba Budangiri. It is named for Baba Budan, a
Muslim Sufi saint who lived there in the 16th century and who is
credited in legend with introducing coffee to India. The shrine has
functioned for centuries as a place of worship for Muslims and Hindus
alike. Hinduism has always resisted any universally satisfactory
definition, and syncretic sites like Baba Budangiri are the religion's
frontiers, where Hinduism's porousness is most evident. In any religion,
the regulation of such sites is the surest sign of a hard shift toward
orthodoxy --- toward an attempt to rigidly define doctrine and heresy.
Around 30 years ago, right-wing activists began organizing large,
festive religious rallies at Baba Budangiri, eventually demanding that
it should be declared an exclusively Hindu site.

In 2001, Lankesh visited the site as part of a delegation of literary
figures on a fact-finding mission about the controversy. Soon she was
not simply reporting on the situation but diving into it, headlining
counterdemonstrations and making political connections for the
activists. Shiva Sundar, her closest colleague in her journalism and in
her activism, told me that in 2003 the police had refused permission for
a protest in the nearest town. But Lankesh was determined to be
arrested, so she sneaked into town wearing a burqa, then threw it off
when she reached the police station and shouted slogans until she was
hauled into custody.

The dispute over Baba Budangiri was the latest in a long series of
battles over two rival ideas of India. One idea is the pluralist,
multireligious, multicultural vision on which the country was founded in
1947. The other is known as Hindutva: a fundamentalist, majoritarian
movement that seeks to codify and enforce orthodox Hinduism and to
define India as an explicitly Hindu country (despite the fact that India
has the second-largest Muslim population in the world). The most
important Hindutva organization is the Rashtriya Swayamsevak Sangh, or
R.S.S., a powerful Hindu-nationalist paramilitary group ---
rank-and-file members line up daily to perform physical training drills
in white-and-brown uniforms --- that was founded in 1925 and reportedly
has millions of members. The Hindutva groups affiliated with the R.S.S.
are known collectively as the Sangh Parivar. One of them is the
Bharatiya Janata Party.

Many of India's worst internal conflicts have occurred along the
pluralist-Hindutva fault line. In 1948, a former R.S.S. member named
Nathuram Godse assassinated Mohandas Gandhi over what he felt was
Gandhi's preferential treatment of Muslims. In 1992, a crowd of Hindutva
activists, accompanied by B.J.P. politicians, tore down a 450-year-old
mosque in the north Indian city of Ayodhya; ensuing nationwide riots
left approximately 1,000 people dead, most of them Muslims. In 2002, a
train car carrying Hindutva activists from Ayodhya to the state of
Gujarat caught fire under circumstances that remain highly disputed,
killing 59; riots in Gujarat killed around 1,000 people, again mostly
Muslims. (Narendra Modi, who was then chief minister of Gujarat, has
been accused of allowing the riots; before he was elected prime
minister, he was denied a visa to visit the United States on those
grounds.) It was in this context that Sangh Parivar and B.J.P. leaders
began talking about making Baba Budangiri ``the Ayodhya of the south.''

The Congress Party, whose politics are generally secular and social
democratic, has undoubtedly been guilty at times of suppressing the
press and of condoning the mass slaughter of religious minorities. But
many Indian liberals fear that the B.J.P.'s overwhelming victory in 2014
marks the most profound threat to India's democracy and pluralism since
its founding. The B.J.P. had controlled the prime ministership before,
for six years, after breaking the Congress Party's longtime hold on the
office in the 1998 elections, but only as part of a coalition government
that required it to tamp down its hard-line positions. In 2014, it won
in a landslide, and a B.J.P. re-election this year would be seen as a
mandate to fully implement the party's ideology. In the B.J.P.'s
rhetoric, being Indian is equated with being Hindu, and religious
minorities are spoken of as if they were foreigners. Critics are branded
as ``anti-national.'' Advocates of a secular Indian state --- which the
Indian constitution calls for in its very first sentence --- are called
``sickulars.''

Such talk has already emboldened a surge of vigilantism. Since the
B.J.P. took power, what is known as ``cow protection'' has become
increasingly a matter of national politics --- the cow holds religious
importance to many Hindus --- and lynch mobs have murdered scores of
people, largely Muslims, suspected of slaughtering or selling cattle. In
July of last year, a B.J.P. minister invited to his home eight men who
had been convicted in such a lynching and presented them with garlands
and sweets. In January 2018, after an 8-year-old Muslim girl was
repeatedly raped and then murdered in a Hindu temple, two B.J.P.
ministers attended a rally in support of the men accused of raping and
murdering her.

By the time the B.J.P. won in 2014, Lankesh had, for nearly a decade,
been using her own newspaper to thrust herself into the center of local
debates over Hindu nationalism. Gauri Lankesh Patrike mostly jettisoned
the literary entertainments and ideological unclassifiability that
characterized her father's paper and evolved into a single-minded
political broadside against the right wing. Shiva Sundar described Gauri
Lankesh Patrike as ``a weekly threat to their philosophy. Every page.
Even the film page had something to say about egalitarian values and to
condemn these people.''

Its mission was earnest, but its tone was typically puckish (often in
ways that defy translation into English). The cover image on the issue
published the week before Lankesh was murdered depicted the bald head of
Amit Shah, the president of the B.J.P., under the headline ``The Story
of a Saffron Egg.'' (Saffron is Hindutva's chosen color, and the
headline nodded to a popular movie at the time, ``Story of an Egg.'')
``Everything on the cover was harsh,'' the journalist Sugata
Srinivasaraju said. ``A lot of times it was below the belt.''

Lankesh sometimes got death threats at the office, either by phone or by
mail. ``She would ignore it,'' her colleague Satish said. ``She would
say, `Who will shoot me?' We didn't take it seriously.'' Like her
father, she often treated political argument like sport. ``She loved
it,'' Lankesh's sister, Kavitha Lankesh, said. ``She loved fighting, she
loved voicing her views, she took great pleasure in standing up for
people. She would make a joke, saying, `I am on the hit list,' and she
felt proud to say that.''

Image

Lankesh, in Paris, where she was studying journalism.Credit...Kavitha
Lankesh

More than once, her subjects reported her to the police for criminal
defamation and libel. Such charges rarely hold up in Indian courts, but
they are effective in harassing journalists because the accused must
show up in court wherever the charge is filed. Lankesh's opponents would
file cases all over the state, which ate up her time and resources. She
took the opportunity to make connections. When she had to appear before
a judge in some distant town, she would often schedule a political
meeting there. Her friends say she learned the best places to eat all
over Karnataka. ``All these guys did in harassing her actually helped
her,'' her lawyer, Venkatesh Bubberjung, said. ``Her sphere of influence
increased multifold.''

Still, he would advise Lankesh to be more careful in her words. ``She'd
say: `I am going to call a scoundrel a scoundrel! It's your job to
defend me,' '' he said. In November 2016 she was finally convicted in a
criminal defamation case over a story she published eight years earlier
claiming that several B.J.P. leaders had defrauded a jeweler and was
sentenced to six months in jail. (The sentence was immediately
suspended, and when she was killed, she was awaiting appeal.)

I asked Venkatesh if Lankesh's rhetoric went overboard at times.
``Frequently, not at times!'' he said. ``Whenever you put her on a stage
to speak, you don't know what's going to get into her. She said Hinduism
is not a religion at all. Her speech was sometimes very intemperate.''
In one example that particularly offended her opponents, in response to
a campaign to mail sanitary napkins to Modi to protest a new tax on
menstrual hygiene products, she suggested on Twitter that women mail
napkins that had already been used.

But Lankesh had defenders among mainstream Indian liberals too, like the
historian Ramachandra Guha. ``There is no such thing as overboard,'' he
insisted, pointedly paraphrasing an adage that had been a favorite of
the former B.J.P. prime minister Atal Bihari Vajpayee: ``The answer to a
piece of writing is another piece of writing. It's not murdering
someone.''

We were sitting in Koshy's, a cozy old restaurant that has long been the
favored watering hole for Bangalore writers. Guha said he had run into
Lankesh several times there. ``Certainly Gauri was killed because of
what she said --- and because she's a woman,'' he said. ``Patriarchal
societies cannot abide independent-minded women. And we are an extremely
patriarchal society still.''

The day after Lankesh was murdered, Guha said in a video interview that
it was very likely that her murderers came from the Sangh Parivar, the
family of Hindutva organizations. The B.J.P.'s youth wing sent him a
letter, written by a former B.J.P. state attorney general, demanding
that Guha apologize for the statement or face defamation charges. ``Of
course it's all part of an attempt to silence and intimidate,'' Guha
told me. ``The B.J.P., which is a cadre-based, ideological party, is
increasingly a party of thugs and vigilantes. And that spreads. And so
instead of making speeches, you intimidate and threaten. And of course
there's acquiescence from the top leadership. They never say anything.
Amit Shah and Modi say nothing if violence is committed in the name of
Hindutva --- never.''

Throughout the past five years of national B.J.P. rule, the party and
its allies have controlled a majority of state governments, too:
currently 16 out of India's 29 states, down from its peak of 21 last
year. (India's political system is parliamentary and federalist, with
powers distributed between the central and state governments.) South
India is the only region where the Hindutva party has never had much
luck. ``Communal, radical, hard-line right-wing politics is an import to
Karnataka,'' Srinivasaraju told me.

But Karnataka is the southern state where the B.J.P. may have fought
hardest to gain a foothold. The party likes to call Karnataka its
``gateway to the south.'' It's the only southern state the B.J.P. has
ever governed, from 2008 to 2013. And it nearly took power again in
state elections last year, eight months after Lankesh's murder.

Image

Police officials hold up sketches of three of the suspects in murder of
Lankesh.Credit...Bangalore-Arun Kumar Rao/Press Trust of India, via PTI
photo

In the state Legislative Assembly elections (which take place every five
years) in 2008, the B.J.P. ran and won in Karnataka on bread-and-butter
issues; religious ideology took a back seat, as it usually does in
southern elections. But in the 2018 state election season, the B.J.P.
opposition leader of Karnataka's Assembly promised that the party's
first bill after victory would be a statewide ban on cattle slaughter.
This election ``is not about roads, drinking water or gutters,'' the
B.J.P. legislator Sanjay Patil said at a rally in April 2018. ``This
election is about a battle between Hindus and Muslims.'' It was one of
the most religiously divisive election campaigns any southern state had
ever seen, and it won a plurality of the state's 222 seats, just nine
short of a majority, though the B.J.P. failed to form a coalition that
would put it in power.

One afternoon in January 2018, a few months before that election, I went
to the B.J.P. headquarters in Bangalore to discuss Lankesh's murder with
five local party leaders. We met in the building's library, and as we
spoke a growing assemblage of B.J.P. members crowded against its glass
door to catch a glimpse of Malavika Avinash, a popular Kannada-language
actress who moonlights as a B.J.P. spokeswoman. Outside the room, party
members chanted party slogans.

``See, there are two versions to this story,'' Avinash said. ``Everyone
has their own conspiracy theory about who might have killed her or who
would have benefited by killing her.'' Many observers had noted that
Lankesh, like the three previous assassination victims, Dabholkar,
Pansare and Kalburgi, was particularly critical of Hindutva. But the
theory Avinash pointed to, as did every other Hindutva adherent I met,
was that underground Maoist revolutionaries had killed Lankesh because
she helped some of their comrades negotiate re-entry into society.
``There were allegations that she perhaps, in a sense, sold them to the
state government,'' Avinash said. ``But nobody knows who did it yet.''

At first the B.J.P. representatives spoke carefully to me about Lankesh,
but soon their complaints began tumbling out. They repeatedly accused
her of yellow journalism, of Hindu-bashing and of ``character
assassination'' against them --- an unfortunate choice of words about
the victim of a literal assassination. ``She was extremely scathing,''
Avinash said. ``Language that was unbecoming of a journalist.'' Anytime
they sensed they were piling on too much, they added the caveat that
murder, of course, was wrong. ``She did live in a very remote place,''
Avinash said. ``She lived alone and didn't care, really. She should have
perhaps cared for her own safety.''

The slogan-chanting outside the room grew louder. After our interview
concluded, I followed the noise downstairs and found a crowd of men
festively hoisting a newly minted B.J.P. legislator on their shoulders;
they were celebrating the recent defections to the B.J.P. of several
politicians from rival parties. In the crowd I met a friendly
middle-aged journalist named S.A. Hemantha Kumar who introduced me to
Sabitha Rao, a B.J.P. supporter who used to work for a mainstream
newspaper called the Deccan Herald. When Kumar learned I was writing
about Lankesh, he excitedly gave me a copy of an issue of the magazine
he writes for, a right-wing weekly called Uday India, with a cover story
on Lankesh. Kumar's own article described her as ``a so-called
journalist with a devious agenda, dubious character \& malicious
intent.''

``She had a concern for the poor, no doubt about it,'' Kumar said of
Lankesh. ``She was a very passionate person, eccentric and perverse.
Pervert thinking. She had no children. A strange marriage. But she had a
lot of boyfriends. That has nothing to do with it, just telling you. She
was taking drugs as of late.'' (There is no evidence that this was the
case.)

``Substances,'' Rao said.

``She was drinking, she was smoking, she had taken to drugs,'' Kumar
continued. ``She lived alone. Huge house.'' (It is actually fairly
modest, and her mother owned it.) ``She was not a good journalist.''

``Very coarse,'' Rao said.

``Very coarse,'' Kumar agreed. ``But ultimately, killing is sad. Killing
is not acceptable. You cannot justify it.''

Image

The cover of Gauri Lankesh Patrike following her murder.

This seemed like the final word until Rao added: ``She behaved like a
16-year-old. She was always living on the edge. Reckless, I'd say. She
paid for it.''

\textbf{For nearly half} a year after Lankesh's murder, there were no
arrests, and nearly everyone following the case seemed to be resigned to
the fact that this would be just another unsolved assassination. But
then, in May, the Karnataka Police's special investigation team filed a
charge sheet against a Hindutva activist named K.T. Naveen Kumar,
running to some 650 pages and accusing him of criminal conspiracy, among
other things. Fifteen more suspects have been arrested and charged in
the months since then; all are in jail awaiting trial and are expected
to plead not guilty. Police are still searching for two more.

The accused include a young utensil salesman named Parashuram Waghmare,
who the police say confessed to pulling the trigger. The police also say
that Waghmare wasn't familiar with Lankesh when the conspirators asked
him to kill her, so they showed him YouTube videos of her speeches to
persuade him to commit the murder. They gave him 10,000 rupees, or
around \$150. Members of a Hindutva group called Sri Ram Sene started a
Facebook fund-raising campaign to support his family. (The group's
leader, Pramod Muthalik, later denied any connection to Waghmare.)

According to the police, forensics indicated that the gun that killed
Lankesh was potentially also used in two of the three other unsolved
assassinations that seemed to fit the same pattern. The police suspect
that the accused are part of an apparently nameless, multistate
right-wing assassination network with at least 60 members. Many of the
accused have connections with a small, secretive Hindutva group called
the Sanatan Sanstha, members of which have previously been arrested as
suspects in four separate bombings of public places. (The cases are
ongoing; two Sanatan Sanstha members were convicted of one blast but are
out on bail awaiting appeal.)

The more established Hindutva organizations, including the R.S.S. (the
Hindu-nationalist paramilitary group) and B.J.P., have tried to distance
themselves from such groups and have raised legal complaints against
those who have tried to connect them to violence perpetrated by the
Hindutva fringe. In February, a magistrate ruled that Rahul Gandhi, the
president of the Congress Party, would stand trial for defamation for
implying a link between the R.S.S. and Lankesh's murder.

Late one night I met with N.P. Amruthesh, the lawyer for four of the
accused men, who is himself a proud follower of the Sanatan Sanstha. An
affable man, seemingly indifferent to appearances, he wore a worn orange
dhoti and white shirt with a blue ink stain billowing out beneath the
pocket. While we spoke, a news segment about Lankesh's case appeared on
his TV: The R.S.S., it was reported, had issued a statement saying that
the latest man arrested, Mohan Nayak, who is not represented by
Amruthesh, was not a member of the organization. Amruthesh laughed. ``In
my opinion, personal opinion, that is not correct,'' he said. ``When any
person is working for Hindutva, it is your duty to give protection to
that person. ... They're claiming that he's not our member, but I came
to know that he always goes to R.S.S. activities and everything. These
organizations, they don't want to take the responsibility.'' Such
disavowals, he said, were bad for morale.

Narendra Modi, meanwhile, has kept his silence. He has never publicly
mentioned Lankesh's name or referred to her case. ``Why should Prime
Minister Modi react?'' Muthalik, the Sri Ram Sene leader, said in a
public speech. ``Do you expect Modi to respond every time a dog dies in
Karnataka?''

Perhaps the most extraordinary discovery the police have made in their
investigation of Lankesh's murder is a detailed diary recovered from the
home of a leading suspect. In it were two lists, ostensibly of people
the conspirators wanted dead, reportedly including Veerabhadra
Chennamalla, a liberal-minded Hindu priest, and K.S. Bhagavan, an
outspokenly atheist Shakespeare scholar. First on one of the lists was
Girish Karnad, who is perhaps the greatest living Kannada playwright.
All have been particularly forthright in their criticism of Hindutva.

Second on one list was Lankesh. In the months since she was shot, some
of her friends and colleagues have grown more cautious about what they
write and say and post to social media, even as this year's unusually
fraught and uncertain Election Day approaches. Others have found
themselves speaking out almost compulsively where they wouldn't have
before. Prakash Raj, a popular film actor and friend of Lankesh's who
had previously been quiet on politics, is now running for office on what
could be called the Gauri platform. ``When we buried Gauri, we were
actually sowing her,'' he said at a literary festival in January. ``They
thought she could be silenced, but she lives through us. And if I end up
in the Parliament, it will be Gauri's voice that will be heard there.''
When the B.J.P. came to national power in the past, it seemed to have
won despite its ideology, campaigning on less divisive issues. But this
year's election feels like a referendum on Hindutva: Is India primarily
a country for Hindus, or, as Lankesh insisted, for everyone who's
Indian?

The last two people to have a real conversation with Lankesh were two
old friends, Madhu Bhushan and Kalpana Chakravarthy, who dropped by the
newspaper office on the afternoon of the day she was murdered to search
the archive of Lankesh Patrike, her father's newspaper, for poems that
Chakravarthy's husband used to submit. They ended up sitting and talking
for two and a half hours, as if time had stopped and none of them had
anything to do, even though Lankesh's paper was supposed to go to press
the next day.

I met Bhushan, a feminist activist, four months later at Hotel Chalukya,
whose restaurant is famous for its big, red triangular dosas. As she
ate, she marveled at the vitality, the appetite for life and fight and
fun that Lankesh had displayed just hours before she died. I asked what
they talked about. ``What didn't we talk about?'' Bhushan said. ``It was
an incredible conversation. We were catching up on 20 years.'' They
talked about their shared college days, about the era of P. Lankesh, but
most of all, ``nice, juicy gossip.'' Friends had been urging Lankesh to
get police protection, but Bhushan recalled Lankesh's telling her:
``I've had one marriage. I don't need a policeman who will replace my
husband.''

They talked and laughed until around 6 p.m. As I often saw when
Lankesh's friends spoke of her, Bhushan's eyes glowed as she recounted
the time she spent with her, as though the pleasure of her company still
lingered. ``She was a very, very genuine human being,'' she said. ``I
guess that's the most radical thing one can be.''/•/

Advertisement

\protect\hyperlink{after-bottom}{Continue reading the main story}

\hypertarget{site-index}{%
\subsection{Site Index}\label{site-index}}

\hypertarget{site-information-navigation}{%
\subsection{Site Information
Navigation}\label{site-information-navigation}}

\begin{itemize}
\tightlist
\item
  \href{https://help.nytimes3xbfgragh.onion/hc/en-us/articles/115014792127-Copyright-notice}{©~2020~The
  New York Times Company}
\end{itemize}

\begin{itemize}
\tightlist
\item
  \href{https://www.nytco.com/}{NYTCo}
\item
  \href{https://help.nytimes3xbfgragh.onion/hc/en-us/articles/115015385887-Contact-Us}{Contact
  Us}
\item
  \href{https://www.nytco.com/careers/}{Work with us}
\item
  \href{https://nytmediakit.com/}{Advertise}
\item
  \href{http://www.tbrandstudio.com/}{T Brand Studio}
\item
  \href{https://www.nytimes3xbfgragh.onion/privacy/cookie-policy\#how-do-i-manage-trackers}{Your
  Ad Choices}
\item
  \href{https://www.nytimes3xbfgragh.onion/privacy}{Privacy}
\item
  \href{https://help.nytimes3xbfgragh.onion/hc/en-us/articles/115014893428-Terms-of-service}{Terms
  of Service}
\item
  \href{https://help.nytimes3xbfgragh.onion/hc/en-us/articles/115014893968-Terms-of-sale}{Terms
  of Sale}
\item
  \href{https://spiderbites.nytimes3xbfgragh.onion}{Site Map}
\item
  \href{https://help.nytimes3xbfgragh.onion/hc/en-us}{Help}
\item
  \href{https://www.nytimes3xbfgragh.onion/subscription?campaignId=37WXW}{Subscriptions}
\end{itemize}
