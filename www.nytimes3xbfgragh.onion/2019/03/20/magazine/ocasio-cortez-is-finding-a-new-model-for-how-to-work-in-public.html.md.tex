Sections

SEARCH

\protect\hyperlink{site-content}{Skip to
content}\protect\hyperlink{site-index}{Skip to site index}

\href{https://myaccount.nytimes3xbfgragh.onion/auth/login?response_type=cookie\&client_id=vi}{}

\href{https://www.nytimes3xbfgragh.onion/section/todayspaper}{Today's
Paper}

Ocasio-Cortez Is Finding a New Model for How to Work in Public

\url{https://nyti.ms/2USqO6o}

\begin{itemize}
\item
\item
\item
\item
\item
\item
\end{itemize}

Advertisement

\protect\hyperlink{after-top}{Continue reading the main story}

Supported by

\protect\hyperlink{after-sponsor}{Continue reading the main story}

\href{/column/screenland}{Screenland}

\hypertarget{ocasio-cortez-is-finding-a-new-model-for-how-to-work-in-public}{%
\section{Ocasio-Cortez Is Finding a New Model for How to Work in
Public}\label{ocasio-cortez-is-finding-a-new-model-for-how-to-work-in-public}}

\includegraphics{https://static01.graylady3jvrrxbe.onion/images/2019/03/17/magazine/17mag-screenland-ocasio/17mag-screenland-ocasio-articleLarge-v2.gif?quality=75\&auto=webp\&disable=upscale}

By Carina Chocano

\begin{itemize}
\item
  March 20, 2019
\item
  \begin{itemize}
  \item
  \item
  \item
  \item
  \item
  \item
  \end{itemize}
\end{itemize}

The level of attention surrounding Alexandria Ocasio-Cortez is, by any
measure, extraordinary, bordering on Kardashian-grade. This goes for
both the positive and the negative attention, each of which can feel
more like a self-serving projection than an accurate representation of
who Ocasio-Cortez is or what she's doing.

The congresswoman has become a handy and versatile symbol --- the
representative representative. For the leftward and younger wings of the
Democratic Party, she serves as a figurehead and a hero; for
conservative media outlets, as reliable outrage bait. Everyone seems
willing to let her be a lightning rod --- including, notably,
Ocasio-Cortez herself.

It's not easy to absorb so much static. ``At first, it was really,
really, really hard,'' she recently told Vanity Fair. ``I felt like I
was being physically ripped apart in those first two to three months.''
But Ocasio-Cortez's own acknowledgment of the attention, which she has
incorporated into her public persona as a kind of running metacommentary
on fame, spin and bias, is also a handy way to channel it in useful
directions. Take, for instance, the lowly clip of routine congressional
testimony --- from a House Committee on Oversight and Reform hearing on
H.R. 1, a bill involving campaign finance and ethics rules --- that
recently went viral, generating well over 40 million views.

There was nothing especially dramatic about it: no bombshell
revelations, no combative exchanges, no emotional outbursts. On the
contrary, Ocasio-Cortez was calm, upbeat and pleasant, framing her
questions as a kind of game --- a ``lightning round'' on the limits of
campaign-finance rules. Casting herself as a hypothetical villain and
enlisting witnesses as her co-conspirators, she described a legislative
system reduced by special interests to a corrupt, zero-sum competition.
``I'm gonna be the bad guy,'' she says in the video. ``Which I'm sure
half the room would agree with anyway. And I want to get away with as
much bad things as possible, ideally to enrich myself and advance my
interests, even if that means putting my interests ahead of the American
people.''

\includegraphics{https://static01.graylady3jvrrxbe.onion/images/2019/03/24/magazine/24mag-screenland-vid-promo/24mag-screenland-vid-promo-videoSixteenByNineJumbo1600.png}

Criticism of Ocasio-Cortez has often come wrapped in dismissiveness;
early attacks, especially, hinged on the ease with which Americans might
be persuaded to see a young woman and political outsider as unserious,
unprepared, even vapid. At the start of this year, a clip of her dancing
in the style of ``The Breakfast Club'' --- taken from a video made when
she was an undergraduate --- was circulated gleefully, as if an image of
her younger self dancing would undermine her legitimacy as a legislator.
Until quite recently, the protocol for a woman subjected to such attacks
was to rise above them, brushing off all the negative attention (and
even some of the positive), projecting an air of being occupied in
stolid, competent work. But Ocasio-Cortez has, thus far, found more
interesting reactions. Her response to the circulation of that video
wasn't to strike a more mature pose; it was to have herself filmed, as a
legislator, dancing in front of her office.

Part of what makes her congressional questioning on H.R. 1 interesting
is that she doesn't try to play down the qualities that are used to mock
or attack her. If anything, she harnesses them for rhetorical effect.
Being dismissed as young, inexperienced and female turns out to be
something she's quite good at.

It's not that the mode of her questioning is especially unusual. She
wants to demonstrate how easy it can be, under current law, for
unscrupulous people to hijack the legislative process, and like
countless legislators and attorneys before her, she uses friendly
witnesses to make the argument for her. She asks lengthy, essentially
rhetorical questions and occasionally prompts the experts to confirm
that what she's saying is correct. This method is familiar enough to
anyone who has ever watched television, as are its rhetorical flourishes
--- feigned naïveté, false modesty, hand-wringing solemnity and other
devices, all enacted for the benefit of the audience.

And yet the tone here is completely different. Ocasio-Cortez doesn't
pretend to take the moral high ground or strive for a clinical,
prosecutorial demeanor. She has fun with it. At least one reason this
video has been watched more than 40 million times is that she has opted
for the pretense of a ``game,'' embracing youthful speech and a breezy,
chipper tone:

\begin{quote}
So, green light for hush money. I can do all sorts of terrible things.
It's totally legal right now for me to pay people off, and that is
considered speech. That money is considered speech. So I use my
special-interest, dark-money-funded campaign to pay off folks that I
need to pay off and get elected. So now I'm elected. Now I'm in. I've
got the power to draft, lobby and shape the laws that govern the United
States of America. Fabulous.
\end{quote}

By playing a game, she exposes the game --- the way the law allows for
behavior the average person might consider corrupt on its face. ``It's
already super legal, as we've seen, for me to be a pretty bad guy,'' she
concludes, having prompted witnesses to confirm this. And on top of this
game, of course, is a trickier one: The rest of the committee is being
asked to indulge a what-if scenario that describes the actual rules
under which they were elected. But citing real-world details would break
the spell, plunging the whole thing into the mire of accusations. With a
low-stakes hypothetical --- hey, guys, let's pretend someone,
incredibly, \emph{did} want to exploit the system --- Ocasio-Cortez
bends any air of wide-eyed innocence to her advantage. If you really
were frivolous or an unwelcome interloper in serious affairs, the upside
is that you'd be excused from having to pretend that the rules of
serious affairs actually work. You'd be free to be the 2019 version of
Elle Woods from ``Legally Blonde,'' charmingly turning people's
underestimation of you against them.

It's a strange artifact of Ocasio-Cortez's attention-magnetism that her
hypotheticals about corruption might have ended up garnering more
attention and emotion than the corruption itself. But if there's
anything she has shown amazing skill at, it's knowing how to manage and
channel that attention, to let the air out of it and turn it back on
whoever paid it. It makes the old model --- of keeping your head down
and rising above --- seem distinctly old. Amid the furious echo chambers
of modern media, failing to acknowledge the most absurd image of
yourself --- failing to laugh at it or own it or hit back at it online
--- can hurt more than it helps. A caricature of Hillary Clinton, for
instance, seemed to obscure her entirely, feeding on her silence. There
aren't too many models for how a woman, in particular, can respond to
attacks and antipathy without being looked on as fragile, or shrill, or
weak, or vain, or full of grievance.

Ocasio-Cortez is finding one. Today's fray is too vicious to stay above;
it is too handy with Photoshop and social media, too full of noise, too
undivided between legitimate sources and illegitimate ones. It's a mosh
pit. There is no dignified attending to your own business in a mosh pit.
You keep your elbows ready, and you dance.

Advertisement

\protect\hyperlink{after-bottom}{Continue reading the main story}

\hypertarget{site-index}{%
\subsection{Site Index}\label{site-index}}

\hypertarget{site-information-navigation}{%
\subsection{Site Information
Navigation}\label{site-information-navigation}}

\begin{itemize}
\tightlist
\item
  \href{https://help.nytimes3xbfgragh.onion/hc/en-us/articles/115014792127-Copyright-notice}{©~2020~The
  New York Times Company}
\end{itemize}

\begin{itemize}
\tightlist
\item
  \href{https://www.nytco.com/}{NYTCo}
\item
  \href{https://help.nytimes3xbfgragh.onion/hc/en-us/articles/115015385887-Contact-Us}{Contact
  Us}
\item
  \href{https://www.nytco.com/careers/}{Work with us}
\item
  \href{https://nytmediakit.com/}{Advertise}
\item
  \href{http://www.tbrandstudio.com/}{T Brand Studio}
\item
  \href{https://www.nytimes3xbfgragh.onion/privacy/cookie-policy\#how-do-i-manage-trackers}{Your
  Ad Choices}
\item
  \href{https://www.nytimes3xbfgragh.onion/privacy}{Privacy}
\item
  \href{https://help.nytimes3xbfgragh.onion/hc/en-us/articles/115014893428-Terms-of-service}{Terms
  of Service}
\item
  \href{https://help.nytimes3xbfgragh.onion/hc/en-us/articles/115014893968-Terms-of-sale}{Terms
  of Sale}
\item
  \href{https://spiderbites.nytimes3xbfgragh.onion}{Site Map}
\item
  \href{https://help.nytimes3xbfgragh.onion/hc/en-us}{Help}
\item
  \href{https://www.nytimes3xbfgragh.onion/subscription?campaignId=37WXW}{Subscriptions}
\end{itemize}
