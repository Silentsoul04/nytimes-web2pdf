The Daily Miracle: Finding Magic Inside The Times's Printing Plant

\url{https://nyti.ms/2TW0gF4}

\begin{itemize}
\item
\item
\item
\item
\item
\item
\end{itemize}

\includegraphics{https://static01.graylady3jvrrxbe.onion/images/2019/03/24/magazine/24collegepoint-image-slide-UOL0/24collegepoint-image-slide-UOL0-articleLarge.jpg?quality=75\&auto=webp\&disable=upscale}

Sections

\protect\hyperlink{site-content}{Skip to
content}\protect\hyperlink{site-index}{Skip to site index}

\hypertarget{the-daily-miracle-finding-magic-inside-the-timess-printing-plant}{%
\section{The Daily Miracle: Finding Magic Inside The Times's Printing
Plant}\label{the-daily-miracle-finding-magic-inside-the-timess-printing-plant}}

The photographer Christopher Payne spent two years shooting The Times's
printing plant in College Point, Queens. He captured the craft,
precision, and unexpected beauty of the newspaper printing process.

Credit...

Supported by

\protect\hyperlink{after-sponsor}{Continue reading the main story}

Photographs by Christopher Payne

Introduction by Luc Sante

\begin{itemize}
\item
  March 26, 2019
\item
  \begin{itemize}
  \item
  \item
  \item
  \item
  \item
  \item
  \end{itemize}
\end{itemize}

\textbf{Consider the newspaper,} the physical object, printed on
processed wood pulp, shot upward on rollers at high speed as ink is
applied, gathered and folded and bundled, dropped off at newsstands and
bodegas or delivered to doorsteps. Nowadays it is a minority choice, as
a majority of consumers around the globe opt to get the latest word from
their screens as they zoom from one place to the next. But that minority
of readers is substantial, and fierce. They savor the thrill of the
first hit of newsprint in the morning, with its slightly acrid odor and
its ironclad association with the first cup of coffee. They have
mastered the origami skills required to read the paper on the subway.
They appreciate the staggered hierarchy of the front page --- the
significance of the top left corner, the relative point sizes of major
and minor headlines, the weight of an obituary (say) that lies above the
fold, the nuance of features heralded in the bottom third --- not
because they are lesser but because their front-page appearance is
merely the tip of a much larger body within. And they can foretell how
their copy will end up: folded to the crossword puzzle, decorated with
various stains, missing a recipe or a travel tip, ready to return unto
pulp.

\includegraphics{https://static01.graylady3jvrrxbe.onion/images/2019/03/24/magazine/24collegepoint-image-slide-DECH/24collegepoint-image-slide-DECH-articleLarge.jpg?quality=75\&auto=webp\&disable=upscale}

Unlike its digital counterpart, the physical newspaper is not just a
transient occurrence along an ongoing stream; it is a singular
phenomenon, which happens every day. The city may no longer feature
vendors shouting the main headline at passing commuters, but that
headline still carries a sense of urgency, can still thrill or appall
when caught unexpectedly in a sidelong glance. A newspaper is a measure
of days, an index of passing time. Its front page compresses the
significance of a unique date into a rectangular field of words and
images, its jarring list of specifics unforeseeable until press time.
Why else would Picasso and Braque have chosen the newspaper as both
their model and their source when they began making collages in 1912?
How else could filmmakers convey the galloping speed of history but with
a process-shot montage of spinning newspaper front pages? How else would
kidnappers prove that their victims were still alive on a given day
other than by photographing them holding that day's Late Final?

Online you are encouraged to compose your own newspaper, according to
your personal algorithmic specifications. The paper model ensures that
although you can ignore entire sections (at your peril), you will at
least skim the principal news part from stem to stern, reading the
headlines. Along the way you will become absorbed in stories no
mathematical formula would have selected for you. And if your flight is
grounded, you will end up reading the entire paper, down to the small
ads, and possibly come away with several new interests as a consequence.

The newspaper in its 19th-century flowering was not just a novel and
practical way of transmitting current political and military doings; it
was also a great herald and artifact of modernism. It exuded the
teeming, democratic randomness of the city --- see the immense
broadsheets of the time, with their acres of close-set columns, each one
crammed with stories, each one clamoring for attention like orphans at a
train window. Eventually, editors and compositors learned how to marshal
their contents, to deploy them strategically around the page, to let the
reader's eye cascade from upper left to lower right and then circle
around again. Most of the leading newspapers devised their own design
languages, promoting a look that, although it might be austere, was very
particular. A glance at a torn fragment, maybe less than an inch long,
maybe affixed to the sole of a boot, would tell the observer just which
paper it came from.

Image

Jason Erikson, pressman.

The newspaper is a veritable helper, with many functions in the course
of the day. It wakes you in the morning, tells you things on the way to
work, comes to hand when you have 10 minutes to spare or need to clear
your head, sits with you during your solo lunch, ladles out the soothing
stuff you've kept for the ride home. It can serve as a veil when you're
at a cafe terrace and wish to avoid attention, as a fan when you're
relaxing in the park on a hot day, as a pillow when you consequently cop
a snooze, as a cushion for transporting pottery or glassware, as a fire
starter for the hearth, as a floor covering when painting or sanding, as
a liner for your cockatoo's cage.

Oddly enough, newspapers also come in handy when you are being stalked
by paparazzi. Stars, crooks and indicted officials seem to have lately
overlooked this handy source of portable foliage. But fame is ever
fleeting in the world of news. In the 1937 comedy ``Nothing Sacred,''
Carole Lombard plays a young woman apparently dying of a rare disease.
Just before we find out she really isn't, the camera lingers for a
moment on the front page of some gazette, on which appears an overripe
ode to her courage --- and then a fish lands on top and hands wrap it
up.

Image

These are perilous times for newspapers, which all across the nation are
being bought up by speculators and stripped for parts, from major papers
in secondary markets to small-town bugles that sometimes have survived
for a century or more. Their reportorial and editorial staffs are
drastically cut, their page counts decreased, the local interest and
focus that once lay at their hearts yanked in favor of wire-service
summaries of national news already being covered much better everywhere
else. Sometimes these papers are taken online altogether, where with
their minimized resources they can be mistaken for URL place holders or
the output of Romanian troll farms. For a paper to lose its print
edition is to become ghostlike, immaterial.

There are plenty of fine online periodicals, of course, but those were
designed with the logic of the internet in mind. The local paper, on the
other hand, was meant to draw a picture of a physical community, with
its customs, rituals, manners, lingo and collective memory inscribed in
a thousand minute ways. Even if you get it only for the yard-sale ads or
to hate-read the letters column, its presence itself is reassuring, part
of the sometimes thin fabric that holds you and your neighbors together.
But if you can't buy it at the gas station along with your six pack and
your night crawlers, it might as well not exist.

Nevertheless, even as the physical newspaper is diminished in numbers,
it bows to no product in the modernity and total awesomeness of its
production. As these spectacular photos by Christopher Payne show, this
paper is produced in College Point, Queens, on machines the size of
large houses, which do everything from the unwinding of the paper rolls
to the folding of the complete sections. Each tower prints 28 spreads at
once, front and back, in cyan, magenta, yellow and black, at ridiculous
speed --- 80,000 copies an hour can be produced this way. And then you
can hold it in your hand, fold it, tear it, use it as a rain hat --- a
voluminous paper object with visual dazzle and hundreds of thousands of
words, representing the collected information of that moment: news,
opinion, analysis, testimony, critique, charts, graphs, photos,
displays. And it happens every day, over and over again. Small wonder
they call it The Daily Miracle. --- \emph{Luc Sante}

\emph{Luc Sante is a writer and critic. His books include ``Low Life,''
``Kill All Your Darlings'' and ``The Other Paris.''}

\begin{center}\rule{0.5\linewidth}{\linethickness}\end{center}

\hypertarget{a-newspaper-from-start-to-finish}{%
\subsection{A Newspaper, From Start to
Finish}\label{a-newspaper-from-start-to-finish}}

Mike Connors, who manages The Times's printing plant in College Point,
Queens, started working for the Times in 1976, long before this plant
was built in 1997. Connors is Fourth-generation with the Times; his
family has been working for the organization for 126 years. He often
works the night shift. ``Urgency is second nature. You've got one chance
at night, one inning.''

Image

James Hommel, electrician.

The paper rolls are moved from storage to the presses with the help of
three robotic cranes. Each roll of paper weighs 2,000 pounds and
unspools 10 miles.

Finished newspaper pages are sent digitally from the newsroom to the
plant and burned with lasers onto thin aluminum sheets called plates. A
pressman like William Toohey, pictured below, bends each plate around a
press cylinder. ``Putting the plates in place requires a little bit of a
knack,'' he says. ``There are a lot of musicians in my family. You need
some dexterity in your fingers to get the plates all lined up. And
fast.''

Image

Vincent Dwyer, paper handler.

Image

William Toohey, pressman.

The facility has 8,000-gallon vats of black, oil-based ink and
2,500-gallon vats of synthetic cyan, magenta and yellow ink. Tanker
trucks deliver ink every week.

Image

Pressmen calibrate each press carefully before a print run. Electricians
and machinists are also on hand during this setup, to address any
problems before the run begins. That includes calibrating the folder and
slitter and trimmer that will process the pages once ink is applied.

Image

John Matishka, pressman.

Each of the seven printing presses at the plant is several stories tall.
They can print as many as 80,000 newspapers each hour of a run. College
Point prints 1.7 million copies of The Times each week. There are 25
other Times print sites in the U.S., but College Point is the biggest.

Image

Inevitably, news breaks during a press run. If it's significant enough,
the newsroom calls to ``Stop the presses!'' This happened after the
recent tornadoes in Alabama, when Osama bin Laden was killed and during
the 2016 election. Often the issues already printed aren't scrapped, but
the rest of the night's run is dedicated to the updated version.

Each press is shut down for one eight-hour shift each week, when press
staff climb inside to remove ink buildup and wipe down crucial parts.

Image

Image

John Machado, pressman.

The Times isn't the only newspaper the staff encounter most nights.
College Point prints USA Today, Newsday and AM New York too.

Image

Hugette Watson, pressman.

Sections of the Sunday Times, like Arts \& Leisure, printed earlier in
the week are kept on giant rolls (with the Muller Martini storage
system), then inserted in the weekend paper after it has been printed.

Image

Image

The sound of the running presses and working plant is overwhelming.
There are 14 miles of gripper conveyor belts overhead. Machines insert
copies of the magazines and preprinted sections, and then finished
papers move via conveyor to bundling machines. The bundles are stacked
on pallets and fork-lifted into waiting delivery trucks.

Image

Steve Costello, mailer.

By 3 a.m. on a typical night, all print runs are complete, and
maintenance and electrical teams are moving around the plant to prepare
for the next morning. College Point has a fleet of two dozen bicycles
that workers use to get around the cavernous building, which feels even
larger when the presses are quiet. --- \emph{Caitlin Roper}

Image

Loba Hubbard, former machinist.

\begin{center}\rule{0.5\linewidth}{\linethickness}\end{center}

\hypertarget{about-the-photographer}{%
\subsection{About the Photographer}\label{about-the-photographer}}

\textbf{Christopher Payne's} first job, at 13, was selling newspapers on
a street corner in downtown Boston. He would weave between cars stopped
at a light, hawking The Boston Globe. (The paper cost 25 cents then, and
he made a nickel on each copy he sold.) Growing up near The Globe's
printing plant, he could see the pressroom lit up at night while the
rest of the city slept.

In 2017, Payne started photographing The Times's College Point plant. He
has gone 40 times, often losing track of time and staying late into the
night as he watched the pressmen to understand their work and to
anticipate where to set his camera. It's the most challenging place he
has ever photographed, he says: ``It is vast, chaotic and visually
overwhelming. Every press run was unique, so I never knew what to
expect. Sometimes I would walk around for hours, only to leave
empty-handed.'' Shooting the presses in motion was especially tricky
because of the intense vibration, but he found the deafening noise
exhilarating. ``When I'm climbing around the presses, I feel like I'm
inside a giant engine,'' he says.

Trained as an architect, Payne has published books of photographs on the
Steinway \& Sons piano factory, New York City's power substations, state
mental hospitals and an uninhabited island of ruins in the East River.
For The Times Magazine, he has documented a Colombian candy factory, the
American textile industry and a 130-year-old pencil factory in Jersey
City, N.J. He lives in New York with his wife and daughter.

Image

Advertisement

\protect\hyperlink{after-bottom}{Continue reading the main story}

\hypertarget{site-index}{%
\subsection{Site Index}\label{site-index}}

\hypertarget{site-information-navigation}{%
\subsection{Site Information
Navigation}\label{site-information-navigation}}

\begin{itemize}
\tightlist
\item
  \href{https://help.nytimes3xbfgragh.onion/hc/en-us/articles/115014792127-Copyright-notice}{©~2020~The
  New York Times Company}
\end{itemize}

\begin{itemize}
\tightlist
\item
  \href{https://www.nytco.com/}{NYTCo}
\item
  \href{https://help.nytimes3xbfgragh.onion/hc/en-us/articles/115015385887-Contact-Us}{Contact
  Us}
\item
  \href{https://www.nytco.com/careers/}{Work with us}
\item
  \href{https://nytmediakit.com/}{Advertise}
\item
  \href{http://www.tbrandstudio.com/}{T Brand Studio}
\item
  \href{https://www.nytimes3xbfgragh.onion/privacy/cookie-policy\#how-do-i-manage-trackers}{Your
  Ad Choices}
\item
  \href{https://www.nytimes3xbfgragh.onion/privacy}{Privacy}
\item
  \href{https://help.nytimes3xbfgragh.onion/hc/en-us/articles/115014893428-Terms-of-service}{Terms
  of Service}
\item
  \href{https://help.nytimes3xbfgragh.onion/hc/en-us/articles/115014893968-Terms-of-sale}{Terms
  of Sale}
\item
  \href{https://spiderbites.nytimes3xbfgragh.onion}{Site Map}
\item
  \href{https://help.nytimes3xbfgragh.onion/hc/en-us}{Help}
\item
  \href{https://www.nytimes3xbfgragh.onion/subscription?campaignId=37WXW}{Subscriptions}
\end{itemize}
