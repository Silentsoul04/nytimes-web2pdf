Sections

SEARCH

\protect\hyperlink{site-content}{Skip to
content}\protect\hyperlink{site-index}{Skip to site index}

\href{https://www.nytimes3xbfgragh.onion/section/theater}{Theater}

\href{https://myaccount.nytimes3xbfgragh.onion/auth/login?response_type=cookie\&client_id=vi}{}

\href{https://www.nytimes3xbfgragh.onion/section/todayspaper}{Today's
Paper}

\href{/section/theater}{Theater}\textbar{}Review: A Smashing `Oklahoma!'
Is Reborn in the Land of Id

\url{https://nyti.ms/2I2dF8p}

\begin{itemize}
\item
\item
\item
\item
\item
\item
\end{itemize}

Advertisement

\protect\hyperlink{after-top}{Continue reading the main story}

Supported by

\protect\hyperlink{after-sponsor}{Continue reading the main story}

Critic's Pick

\hypertarget{review-a-smashing-oklahoma-is-reborn-in-the-land-of-id}{%
\section{Review: A Smashing `Oklahoma!' Is Reborn in the Land of
Id}\label{review-a-smashing-oklahoma-is-reborn-in-the-land-of-id}}

\includegraphics{https://static01.graylady3jvrrxbe.onion/images/2019/04/07/arts/07oklahoma/merlin_152353413_727cbcf6-df8b-463d-b04f-72f63c4cb9d4-articleLarge.jpg?quality=75\&auto=webp\&disable=upscale}

\begin{itemize}
\tightlist
\item
  Oklahoma!\\
  **NYT Critic's Pick Broadway, Musical 2 hrs. and 45 min Closing Date:
  Jan. 19, 2020 Circle in the Square, 235 W 50th St. 212-239-6200
\end{itemize}

By \href{https://www.nytimes3xbfgragh.onion/by/ben-brantley}{Ben
Brantley}

\begin{itemize}
\item
  April 7, 2019
\item
  \begin{itemize}
  \item
  \item
  \item
  \item
  \item
  \item
  \end{itemize}
\end{itemize}

How is it that the coolest new show on Broadway in 2019 is a 1943
musical usually regarded as a very square slice of American pie? The
answer arrives before the first song is over in Daniel Fish's
wide-awake, jolting and altogether wonderful production of
\href{https://oklahomabroadway.com/?gclid=EAIaIQobChMI-bj_vbO54QIViI3ICh2_HA6aEAAYASAAEgJEdvD_BwE\&gclsrc=aw.ds}{``Rodgers
and Hammerstein's Oklahoma!,'' which opened on Sunday night at the
Circle in the Square Theater}.

``Oh, What a Beautiful Mornin''' is the title and the opening line of
this familiar number, a paean to a land of promisingly blue skies and
open spaces. But Curly, the cowboy who sings it, isn't cushioned by the
expected lush orchestrations. Nor is the actor playing him your usual
solid slab of beefcake with a strapping tenor.

As embodied by the excellent Damon Daunno, this lad of the prairies is
wiry and wired, so full of unchanneled sexual energy you expect him to
implode. There's the hint of a wobble in his cocky strut and voice.

Doing his best to project a confidence he doesn't entirely feel, to the
accompaniment of a down-home guitar, he seems so palpably young. As is
often true of big boys with unsettled hormones, he also reads as just a
little dangerous.

He's a lot like the feisty, ever-evolving nation he's so proud to belong
to. That would be the United States of America, then and now.

\href{https://www.nytimes3xbfgragh.onion/2019/03/27/theater/daniel-fish-oklahoma.html}{Making
his Broadway debut as a director, Mr. Fish} has reconceived a work often
seen as a byword for can-do optimism as a mirror for our age of doubt
and anxiety. This is ``Oklahoma!'' for an era in which longstanding
American legacies are being examined with newly skeptical eyes.

Such a metamorphosis has been realized with scarcely a changed word of
Oscar Hammerstein II's original book and lyrics. This isn't an act of
plunder, but of reclamation. And a cozy old friend starts to seem like a
figure of disturbing --- and exciting --- depth and complexity.

Mr. Fish's version isn't the first ``Oklahoma!'' to elicit the shadows
from within the play's sunshine. Trevor Nunn and Susan Stroman's
\href{https://www.nytimes3xbfgragh.onion/2002/03/22/movies/theater-review-this-time-a-beautiful-mornin-with-a-dark-side.html}{interpretation
for London's National Theater of nearly two decades ago}, while more
traditionally staged, also scaled up the disquieting erotic elements.

But this latest incarnation goes much further in digging to a core of
fraught ambivalence. To do so, it strips ``Oklahoma!'' down to its
skivvies, discarding the picturesque costumes and swirling
orchestrations, and revealing a very human body that belongs to our
conflicted present as much as it did to 1943 or to 1906, the year in
which the show (based on
\href{http://thislandpress.com/2014/04/30/broadways-forgotten-man/}{Lynn
Riggs's ``Green Grow the Lilacs'')} takes place.

{[}\emph{What's new onstage and
off:\href{https://www.nytimes3xbfgragh.onion/newsletters/theater-update?module=inline}{Sign
up for our Theater Update newsletter.}}{]}

Laura Jellinek's set suggests a small-town community center that might
double as a polling station, decorated with festive banners, colored
lights --- and a full arsenal of guns on the walls. It's made clear that
we the audience are part of this community. The house lights stay on for
much of the show, in a homogenizing brightness, that is occasionally and
abruptly changed for pitch darkness. (Scott Zielinski is the first-rate
lighting designer.)

There's chili cooking on the refectory tables onstage, for the
audience's consumption at intermission. A seven-member hootenanny-style
band sits in plain view. The well-known melodies they play have been
reimagined --- by the brilliant orchestrator and arranger Daniel Kluger
--- with the vernacular throb and straightforwardness of country and
western ballads.

The cast members --- wearing a lot of good old, form-fitting denim
(Terese Wadden did the costumes) --- are just plain folks. Singing with
conversational ease, they occasionally flirt and joke with the audience
seated on either side of the stage. We are all, it would appear, in this
together.

\includegraphics{https://static01.graylady3jvrrxbe.onion/images/2019/04/08/arts/07oklahoma2/merlin_152353416_99a2f89b-4567-44d8-879c-276ec30c7ee3-articleLarge.jpg?quality=75\&auto=webp\&disable=upscale}

Though the cast has been whittled down to 11 speaking parts (and one
dancer), the key characters are very much present. They include our
scrapping leading lovers, Curly McLain and Laurey Williams (Rebecca
Naomi Jones); their comic counterparts, Will Parker (James Davis) and
Ado Annie (Ali Stroker); that bastion of homespun wisdom and stoicism,
Aunt Eller (Mary Testa) and the womanizing peddler Ali Hakim (Will
Brill).

Oh, I almost forgot poor old Jud Fry (Patrick Vaill), the slightly,
well, weird handyman who's sweet, in a sour way, on Laurey. Everybody
forgets Jud, or tries to. Not that this is possible, with Mr. Vaill
lending a charismatic, hungry loneliness to the part that's guaranteed
to haunt your nightmares.

These people --- in some cases nontraditionally yet always perfectly
cast --- intersect much as they usually do in ``Oklahoma!'' They court
and spark, fight and reunite. They also show off by picking up guitars
and microphones and dancing like prairie bacchantes. (John Heginbotham
did the spontaneous-feeling choreography.) They use household chores,
like shucking corn, to memorably annotative effect.

Ms. Stroker's boy-crazy, country siren-voiced Ado Annie, who rides a
wheelchair as if it were a prize bronco, and Mr. Davis's deliciously
dumb Will emanate a blissful endorphin haze. Mr. Brill is a refreshingly
unmannered Ali Hakim, and Ms. Testa is a splendid, wryly authoritative
Aunt Eller.

But there's an abiding tension. This is especially evident in Ms.
Jones's affectingly wary Laurey, who regards her very different suitors,
Curly and Jud, with a confused combination of desire and terror.

That her fears are not misplaced becomes clear in an encounter in Jud's
dank hovel of a home. Curly sings ``Pore Jud,'' in which he teasingly
imagines his rival's funeral with an ominous breathiness.

The scene occurs in darkness, with a simulcast video in black and white
of the two men face to face. (Joshua Thorson did the projection design.)
And the lines between sex and violence, already blurred in this
gun-toting universe, melt altogether.

I first saw Mr. Fish's ``Oklahoma!''
\href{https://www.nytimes3xbfgragh.onion/2015/07/07/theater/review-oklahoma-preserves-a-classic-while-adding-punch.html}{at
Bard College in 2015}, and again
\href{https://www.nytimes3xbfgragh.onion/2018/10/07/theater/oklahoma-review.html}{at
St. Ann's Warehouse in Brooklyn last year.} It was an exciting work from
the get-go, but it just keeps getting better. The performances are
looser and bigger; they're Broadway-size now, with all the infectious
exuberance you expect from a great musical.

At the same time, though, this production reminds us that such raw
energy can be harnessed to different ends, for ill as well as for good.
In the earlier versions, I had problems with its truly shocking
conclusion --- the scene that takes the most liberties with the
original. In its carefully retooled rendering, it's disturbing for all
the right reasons.

The other significant change here involves the dream ballet, which in
this version begins the second act and has been newly varied and paced.
It is performed by one dancer (the exquisite Gabrielle Hamilton) with a
shaved head and a glittering T-shirt that reads ``Dream Baby Dream.''

What she does is a far cry from the same sequence as immortalized by
Agnes de Mille, the show's legendary original choreographer. But on its
own, radically reconceptualized terms, it achieves the same effect.

As she gallops, slithers and crawls the length of the stage, casting
wondering and seductive glances at the front row, Ms. Hamilton comes to
seem like undiluted id incarnate, a force that has always been rippling
beneath the surface here.

She's as stimulating and frightening --- and as fresh --- as last
night's fever dream. So is this astonishing show.

Advertisement

\protect\hyperlink{after-bottom}{Continue reading the main story}

\hypertarget{site-index}{%
\subsection{Site Index}\label{site-index}}

\hypertarget{site-information-navigation}{%
\subsection{Site Information
Navigation}\label{site-information-navigation}}

\begin{itemize}
\tightlist
\item
  \href{https://help.nytimes3xbfgragh.onion/hc/en-us/articles/115014792127-Copyright-notice}{©~2020~The
  New York Times Company}
\end{itemize}

\begin{itemize}
\tightlist
\item
  \href{https://www.nytco.com/}{NYTCo}
\item
  \href{https://help.nytimes3xbfgragh.onion/hc/en-us/articles/115015385887-Contact-Us}{Contact
  Us}
\item
  \href{https://www.nytco.com/careers/}{Work with us}
\item
  \href{https://nytmediakit.com/}{Advertise}
\item
  \href{http://www.tbrandstudio.com/}{T Brand Studio}
\item
  \href{https://www.nytimes3xbfgragh.onion/privacy/cookie-policy\#how-do-i-manage-trackers}{Your
  Ad Choices}
\item
  \href{https://www.nytimes3xbfgragh.onion/privacy}{Privacy}
\item
  \href{https://help.nytimes3xbfgragh.onion/hc/en-us/articles/115014893428-Terms-of-service}{Terms
  of Service}
\item
  \href{https://help.nytimes3xbfgragh.onion/hc/en-us/articles/115014893968-Terms-of-sale}{Terms
  of Sale}
\item
  \href{https://spiderbites.nytimes3xbfgragh.onion}{Site Map}
\item
  \href{https://help.nytimes3xbfgragh.onion/hc/en-us}{Help}
\item
  \href{https://www.nytimes3xbfgragh.onion/subscription?campaignId=37WXW}{Subscriptions}
\end{itemize}
