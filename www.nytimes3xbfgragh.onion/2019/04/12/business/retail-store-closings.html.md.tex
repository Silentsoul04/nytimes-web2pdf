Sections

SEARCH

\protect\hyperlink{site-content}{Skip to
content}\protect\hyperlink{site-index}{Skip to site index}

\href{https://www.nytimes3xbfgragh.onion/section/business}{Business}

\href{https://myaccount.nytimes3xbfgragh.onion/auth/login?response_type=cookie\&client_id=vi}{}

\href{https://www.nytimes3xbfgragh.onion/section/todayspaper}{Today's
Paper}

\href{/section/business}{Business}\textbar{}U.S. Retail Stores' Planned
Closings Already Exceed 2018 Total

\url{https://nyti.ms/2Zbbg06}

\begin{itemize}
\item
\item
\item
\item
\item
\item
\end{itemize}

Advertisement

\protect\hyperlink{after-top}{Continue reading the main story}

Supported by

\protect\hyperlink{after-sponsor}{Continue reading the main story}

\hypertarget{us-retail-stores-planned-closings-already-exceed-2018-total}{%
\section{U.S. Retail Stores' Planned Closings Already Exceed 2018
Total}\label{us-retail-stores-planned-closings-already-exceed-2018-total}}

\includegraphics{https://static01.graylady3jvrrxbe.onion/images/2019/04/13/business/13retailstores/12retailstores-sub-articleLarge.jpg?quality=75\&auto=webp\&disable=upscale}

By \href{https://www.nytimes3xbfgragh.onion/by/sapna-maheshwari}{Sapna
Maheshwari}

\begin{itemize}
\item
  April 12, 2019
\item
  \begin{itemize}
  \item
  \item
  \item
  \item
  \item
  \item
  \end{itemize}
\end{itemize}

As an executive vice president at Great American Group, a firm that
helps liquidate the merchandise, clothing racks and mannequins at stores
that are closing, Ryan Mulcunry has been watching booms and busts in the
retail industry for almost two decades.

Companies like his have been busy in recent years, but lately one thing
has been missing.

``In all the other cycles, including 2008, a lot of people would come in
and buy racking, circular racks and so on,'' Mr. Mulcunry said. ``They'd
buy it all and warehouse it and wait until somebody wanted to reopen a
store and sell it back to them. Those people have gone away.''

He added, ``People don't think retail is going to grow again from a
bricks-and-mortar perspective.''

As the internet continues to change shopping habits, stores across the
United States continue to close. Less than halfway through April,
American retailers have announced plans this year to shut 5,994 stores,
exceeding the 5,854 announced in all of 2018, according to data from
Coresight Research.

Retailers in good financial shape are paring locations as their leases
expire, while brands like
\href{https://www.nytimes3xbfgragh.onion/2019/02/16/business/payless-shoes-stores.html}{Payless
ShoeSource} and Charlotte Russe are filing for bankruptcy and shutting
hundreds of stores within months. Payless and Gymboree --- which both
filed for bankruptcy this year for a second time --- account for almost
half of the announced closings.

``For a long time, companies have talked about the squeeze in the middle
of retail, but then you see the closure of a Payless,'' said John
Mercer, a senior analyst at Coresight, a research and advisory firm.
``There's just so much choice now that it's not so much always the
middle.''

Stores that are surviving tend to offer consumers more compelling
experiences and better complement online shopping options, Mr. Mercer
added.

\includegraphics{https://static01.graylady3jvrrxbe.onion/images/2019/04/12/business/12retailstores-2/12retailstores-2-articleLarge.jpg?quality=75\&auto=webp\&disable=upscale}

The announced closings still have a ways to go before they reach the
2017 record of more than 8,000. And
\href{https://www.nytimes3xbfgragh.onion/2018/09/03/business/retail-walmart-amazon-economy.html}{openings}
and renovations are still taking place. Coresight has tracked
announcements of 2,641 store openings by retailers in the United States
this year, compared with 3,239 for all of 2018. Many of this year's
openings are dollar stores and other discount chains --- areas that are
less threatened by e-commerce right now. Online retailers like Warby
Parker are also opening stores, though on a small scale.

Struggling stores can slog on for years, as shopper traffic declines and
their 40-percent-off sales begin to feel permanent. But when companies
file for bankruptcy, closings often move at lightning speed. In the past
year, liquidation sales have happened at Bon-Ton, Toys ``R'' Us,
Charlotte Russe, Gymboree and Payless, shaking up the lives of
employees.

Great American Group and Tiger Capital Group are among a handful of
businesses that have managed store liquidations for years, hired to
wring as much value from them as they can. Both worked on the Bon-Ton
and
\href{https://www.nytimes3xbfgragh.onion/2018/06/30/business/toys-r-us-closing.html}{Toys
``R'' Us} closings last year.

Usually, the money they extract will help pay creditors. Tiger Capital's
\href{https://www.tigergroup.com/services/retail-dispositions/}{website}
said it had sold \$5 billion in assets from chainwide liquidations and
strategic retail closings in 2018.

``When we're doing a liquidation sale, it's all about recovery versus
time --- the longer you are open, the more expensive it is,'' said
Michael McGrail, chief operating officer of Tiger Capital. ``It's all
about getting this thing moving, moving it fast, starting on a certain
day and being out by a certain day.''

Mall-order clothing retailers can shut down in as little as five weeks,
furniture stores may take 12 weeks, and jewelry stores can last 20
weeks, Mr. McGrail said. Tiger Capital uses different models to optimize
sales, with the goal of leaving nothing in the store, including
furniture, jewelry cases and shelves.

Mr. Mulcunry of Great American said it was ``like we're running a
retailer for the short term.'' The hardest part is keeping workers, he
said, especially with low unemployment rates over all. The firm tries to
maintain a sense of culture at stores, offer resources for new jobs and
potentially pay bonuses to staff who stay roughly 60 days after the
closing is announced, he said.

The firms also prepare staff for the ``Christmas in July'' effect of
going-out-of-business sales, Mr. McGrail said. ``You're dealing with
people ripping apart shelves, and employees are not used to that,'' he
said. ``So we're in there, we're helping them, we're coaching them and
walking them through this process.''

Image

In the past year, liquidation sales have happened at Bon-Ton, Toys ``R''
Us, Charlotte Russe, Gymboree and Payless, shaking up employees'
lives.Credit...Emma Howells/The New York Times

Sydney Blair, 18, was excited to be working at a Charlotte Russe store
in the Asheville Mall in North Carolina when the company filed for
bankruptcy in February. In early March, she was told that the store
would be liquidated, a process that was completed by the end of the
month. Charlotte Russe, which was unable to find a buyer after its
bankruptcy filing,
\href{https://www.sb360.com/sb360-to-conduct-going-out-of-business-sales-at-all-416-charlotte-russe-stores/}{announced}
closing sales at more than 400 locations.

``It happened very quickly,'' Ms. Blair said, adding that the store was
packed with customers in its final weeks. ``I had a few customers
complaining about my attitude not being 100 percent, and I was just kind
of like: We're all losing our jobs. I can't help but walk in here and
slowly see it being emptied out and just sort of feel sad.''

And even during the liquidation process, e-commerce looms large.

``The biggest thing we've had to do is figure out how to sell stuff
online,'' Mr. Mulcunry said. Great American has been exploring ways to
maximize sales through company websites, ads on Facebook and Instagram,
and by offers on Amazon.

``A going-out-of-business sale is exciting: People want to go, it's fun
and drives energy, and that's what we made money on for a long period of
time,'' he said. ``But if the customer changes, so must the
liquidator.''

Mr. McGrail views the power shifts in retail as a ``slow and steady''
evolution. ``Even though we're seeing this bump of store closures now,
it'll slow down a bit and then we'll see another wave,'' he said. Still,
he said that he expected retail square footage to continue to shrink and
a
\href{https://www.nytimes3xbfgragh.onion/2015/01/04/business/the-economics-and-nostalgia-of-dead-malls.html}{widening
of the gap} between the best malls and more mediocre locations.

Many retailers have been exiting their least profitable mall locations.
Gap recently
\href{https://www.nytimes3xbfgragh.onion/2019/02/28/business/gap-old-navy-spinoff.html}{said
that} it would close 230 stores in the next two years, mostly as leases
expire, as it tries to balance its online, outlet and regular sales.
Victoria's Secret plans to close 53 stores in North America this year,
up from its usual culling of about 15 stores annually. The chain will
still have more than 900 stores.

Payless plans to close 2,300 stores in North America by the end of May,
in what is expected to be the
\href{https://www.payless.com/on/demandware.static/-/Sites-payless-Library/default/dwdd7f9484/documents/payless-restructure-pr-store-closing-sales-02-28-2019.pdf}{biggest}
liquidation of a retailer by store count. Gymboree started closing 749
Gymboree and Crazy 8 stores in the United States in January. The
personalized engraving retailer Things Remembered, which filed for
bankruptcy this year, has closed more than 200 stores.

``It's not really a recession-driven or, even to an extent,
management-driven change --- it's a change in the way people are
buying,'' Mr. Mulcunry said. ``Retail is not dying. It's just changing,
so we're a part of that change.''

Advertisement

\protect\hyperlink{after-bottom}{Continue reading the main story}

\hypertarget{site-index}{%
\subsection{Site Index}\label{site-index}}

\hypertarget{site-information-navigation}{%
\subsection{Site Information
Navigation}\label{site-information-navigation}}

\begin{itemize}
\tightlist
\item
  \href{https://help.nytimes3xbfgragh.onion/hc/en-us/articles/115014792127-Copyright-notice}{©~2020~The
  New York Times Company}
\end{itemize}

\begin{itemize}
\tightlist
\item
  \href{https://www.nytco.com/}{NYTCo}
\item
  \href{https://help.nytimes3xbfgragh.onion/hc/en-us/articles/115015385887-Contact-Us}{Contact
  Us}
\item
  \href{https://www.nytco.com/careers/}{Work with us}
\item
  \href{https://nytmediakit.com/}{Advertise}
\item
  \href{http://www.tbrandstudio.com/}{T Brand Studio}
\item
  \href{https://www.nytimes3xbfgragh.onion/privacy/cookie-policy\#how-do-i-manage-trackers}{Your
  Ad Choices}
\item
  \href{https://www.nytimes3xbfgragh.onion/privacy}{Privacy}
\item
  \href{https://help.nytimes3xbfgragh.onion/hc/en-us/articles/115014893428-Terms-of-service}{Terms
  of Service}
\item
  \href{https://help.nytimes3xbfgragh.onion/hc/en-us/articles/115014893968-Terms-of-sale}{Terms
  of Sale}
\item
  \href{https://spiderbites.nytimes3xbfgragh.onion}{Site Map}
\item
  \href{https://help.nytimes3xbfgragh.onion/hc/en-us}{Help}
\item
  \href{https://www.nytimes3xbfgragh.onion/subscription?campaignId=37WXW}{Subscriptions}
\end{itemize}
