Sections

SEARCH

\protect\hyperlink{site-content}{Skip to
content}\protect\hyperlink{site-index}{Skip to site index}

\href{https://www.nytimes3xbfgragh.onion/section/business/media}{Media}

\href{https://myaccount.nytimes3xbfgragh.onion/auth/login?response_type=cookie\&client_id=vi}{}

\href{https://www.nytimes3xbfgragh.onion/section/todayspaper}{Today's
Paper}

\href{/section/business/media}{Media}\textbar{}Pulitzer Prize: 2019
Winners List

\url{https://nyti.ms/2XdBQEf}

\begin{itemize}
\item
\item
\item
\item
\item
\end{itemize}

Advertisement

\protect\hyperlink{after-top}{Continue reading the main story}

Supported by

\protect\hyperlink{after-sponsor}{Continue reading the main story}

\hypertarget{pulitzer-prize-2019-winners-list}{%
\section{Pulitzer Prize: 2019 Winners
List}\label{pulitzer-prize-2019-winners-list}}

From meritorious public service for coverage of the deadly shooting
rampage at Marjory Stoneman Douglas High School to a special citation to
the ``Queen of Soul,'' here is the full list of winners and finalists.

By The New York Times

\begin{itemize}
\item
  April 15, 2019
\item
  \begin{itemize}
  \item
  \item
  \item
  \item
  \item
  \end{itemize}
\end{itemize}

\includegraphics{https://static01.graylady3jvrrxbe.onion/images/2019/04/15/business/15PULITZER-sunsentinel-staff/merlin_153568356_a4464a3e-a20f-4df9-9977-7e3122268952-articleLarge.jpg?quality=75\&auto=webp\&disable=upscale}

PUBLIC SERVICE

\hypertarget{the-south-florida-sun-sentinel}{%
\subsection{The South Florida Sun
Sentinel}\label{the-south-florida-sun-sentinel}}

The Sun Sentinel's sweeping coverage of the causes and consequences of
the deadly mass shooting at Marjory Stoneman Douglas High School in
Parkland, Fla., shaped the national gun safety debate and prompted
changes in local policies. The paper addressed a
\href{https://www.sun-sentinel.com/local/broward/parkland/florida-school-shooting/fl-florida-school-shooting-discipline-20180510-story.html}{culture
of leniency} at Broward County schools, blunders by the sheriff's office
in
\href{http://projects.sun-sentinel.com/2018/sfl-parkland-school-shooting-critical-moments/}{responding
to the attack} and attempts by officials to
\href{https://www.sun-sentinel.com/local/broward/parkland/florida-school-shooting/fl-florida-school-shooting-district-secrecy-20181112-story.html}{mask
their failures}.

\textbf{Finalists} ProPublica, The Washington Post

BREAKING NEWS REPORTING

\hypertarget{staff-of-the-pittsburgh-post-gazette}{%
\subsection{Staff of The Pittsburgh
Post-Gazette}\label{staff-of-the-pittsburgh-post-gazette}}

``I don't think there's a single person on the staff who didn't
contribute'' to covering the shooting deaths of 11 worshipers at the
Tree of Life synagogue, said Keith Burris, the paper's executive editor.
A
\href{https://newsinteractive.post-gazette.com/pittsburgh-squirrel-hill-synagogue-massacre/}{visceral
account} of the attack began with the first words of the Mourners'
Kaddish, a Jewish prayer for the dead, rendered in Hebrew. Other
articles examined the victims' lives, the
\href{https://www.post-gazette.com/news/crime-courts/2018/10/29/tree-of-life-witness-barry-werber-shooting-pittsburgh/stories/201810290101}{harrowing
experiences} of survivors and the quick reactions of
\href{https://www.post-gazette.com/news/crime-courts/2018/11/01/tree-of-life-synagogue-shooting-allegheny-county-911-dispatcher-pittsburgh/stories/201811010157}{911
center workers}.

\textbf{Finalists} The Chico Enterprise-Record; South Florida Sun
Sentinel

INVESTIGATIVE REPORTING

\hypertarget{matt-hamilton-harriet-ryan-and-paul-pringle}{%
\subsection{Matt Hamilton, Harriet Ryan and Paul
Pringle}\label{matt-hamilton-harriet-ryan-and-paul-pringle}}

of The Los Angeles Times

The Los Angeles Times was honored for a
\href{https://www.latimes.com/local/lanow/la-me-times-pulitzer-george-tyndall-usc-20190414-story.html}{series
of articles} that exposed a pattern of abuse by a University of Southern
California gynecologist, Dr. George Tyndall, who was accused of sexually
abusing hundreds of students at a campus clinic over the course of three
decades The allegations led to the resignation of the university's
president and prompted local, state and federal investigations.

\textbf{Finalists} David Barstow, Susanne Craig and Russ Buettner of The
New York Times; Kathleen McGrory and Neil Bedi of The Tampa Bay Times

Image

Susanne Craig addressing The New York Times newsroom after she, Russ
Buettner, to her right, and David Barstow, to Mr. Buettner's right, won
the Pulitzer Prize for explanatory reporting. Credit...Hiroko
Masuike/The New York Times

EXPLANATORY REPORTING

\hypertarget{david-barstow-susanne-craig-and-russ-buettner}{%
\subsection{David Barstow, Susanne Craig and Russ
Buettner}\label{david-barstow-susanne-craig-and-russ-buettner}}

of The New York Times

The
\href{https://www.nytimes3xbfgragh.onion/interactive/2018/10/02/us/politics/donald-trump-tax-schemes-fred-trump.html?module=inline}{18-month
investigation} of President Trump's finances debunked his claims of
self-made wealth and revealed a business empire "riddled with tax
dodges," the Pulitzer committee said. The series led city and state
officials in New York to open investigations into whether Mr. Trump and
his family had underpaid taxes on his father's real estate empire and
participated in fraudulent tax schemes. This is the fourth Pulitzer for
Mr. Barstow; Ms. Craig, 51, and Mr. Buettner, 57, have been finalists.

\textbf{Finalists} Kyra Gurney, Nicholas Nehamas, Jay Weaver and Jim
Wyss of The Miami Herald; Aaron Glanz and Emmanuel Martinez of
Reveal/The Center for Investigative Reporting; Staff of The Washington
Post

LOCAL REPORTING

\hypertarget{staff-of-the-advocate-baton-rouge-la}{%
\subsection{Staff of The Advocate, Baton Rouge,
La.}\label{staff-of-the-advocate-baton-rouge-la}}

The Advocate was honored for its
\href{https://www.theadvocate.com/new_orleans/news/courts/article_a03f02d2-fcf5-11e8-937c-fb85ceb6d9d3.html}{investigation}
of a Louisiana law that allowed juries to convict defendants without
unanimous verdicts. In the series'
\href{https://www.theadvocate.com/new_orleans/news/courts/article_16fd0ece-32b1-11e8-8770-33eca2a325de.html}{first
installation}, the reporters revealed how the framers of the state's
Constitution noted explicitly that the law would perpetuate ``the
supremacy of the Anglo-Saxon race in Louisiana." After the series
appeared, Louisiana voters amended their Constitution to require
unanimous verdicts. "It seemed to help turn the tide," said Gordon
Russell, one of the reporters on the series.

\textbf{Finalists} Barbara Laker, Wendy Ruderman, Dylan Purcell and
Jessica Griffin of The Philadelphia Inquirer; Brandon Stahl, Jennifer
Bjorhus, MaryJo Webster and Renée Jones Schneider of the Star Tribune of
Minneapolis, Minn.

NATIONAL REPORTING

\hypertarget{staff-of-the-wall-street-journal}{%
\subsection{Staff of The Wall Street
Journal}\label{staff-of-the-wall-street-journal}}

The Wall Street Journal won the national reporting prize for its
\href{https://www.wsj.com/articles/trumps-hush-money-11545349433?mod=article_inline}{articles
on hush money payments} made to two women who claimed during the 2016
presidential campaign that they had sexual encounters with Donald J.
Trump. ``To see the story unfold the way it did and reach the president
was just incredible,'' said Michael Rothfeld, one of The Journal
reporters who contributed to the coverage. ``For me, it's just the story
of a lifetime.''

\textbf{Finalists} Staff of Associated Press; Staff of The New York
Times with contributions from Carole Cadwalladr of The Guardian/The
Observer

INTERNATIONAL REPORTING

\hypertarget{maggie-michael-maad-al-zikry-and-nariman-el-mofty}{%
\subsection{Maggie Michael, Maad al-Zikry and Nariman
El-Mofty}\label{maggie-michael-maad-al-zikry-and-nariman-el-mofty}}

of Associated Press; Staff of Reuters, with notable contributions from
Wa Lone and Kyaw Soe Oo

The Reuters staff and two reporters specifically were honored for
``\href{https://www.reuters.com/investigates/section/myanmar-rohingya/}{expertly
exposing} the military units and Buddhist villagers responsible for the
systematic expulsion and murder of Rohingya Muslims from Myanmar.'' The
reporters, Wa Lone, 33, and Kyaw Soe Oo, 29, were arrested in December
2017 and later sentenced to
\href{https://www.nytimes3xbfgragh.onion/2018/09/03/world/asia/myanmar-reuters-journalists-sentenced-trial.html}{seven
years in prison} for reporting on the atrocities. Stephen J. Adler,
Reuters' top editor, said he was ``thrilled'' for the recognition, but
``deeply distressed'' that they reporters were still imprisoned.

Ms. Michael, Mr. al-Zikry and Ms. El-Mofty won for
\href{https://apnews.com/YemenDirtyWar}{their series} ``detailing the
atrocities of the war in Yemen, including theft of food aid, deployment
of child soldiers and torture of prisoners.'' The journalists
documented, in text, video and photography, the civilian deaths caused
by United States drones and interviewed torture victims in Yemen's civil
war. Ms. Michael
\href{https://apnews.com/44f2f36b0c7b46b19b2fa32e6904b06c}{told} The
Associated Press that she and her colleagues were ``very happy to be
able to draw some attention'' to the story.

\textbf{Finalist} Rukmini Callimachi of The New York Times

FEATURE WRITING

\hypertarget{hannah-dreier-of-propublica}{%
\subsection{Hannah Dreier of
ProPublica}\label{hannah-dreier-of-propublica}}

Ms. Dreier's detailed portraits of
\href{https://www.propublica.org/series/ms-13-on-long-island}{Salvadoran
immigrants} were cited for exposing how their lives had been destroyed
``by a botched federal crackdown on the international criminal gang
MS-13.'' After Ms. Dreier, 32, heard President Trump tie immigration to
gang violence, her reporting revealed that immigrants were often victims
of the crime groups. ``What was so cruel was that this population was
being preyed upon,'' she said. The series was published jointly with The
New York Times Magazine, Newsday and New York magazine.

\textbf{Finalists} Deanna Pan and Jennifer Berry Hawes of The Post and
Courier, Charleston, S.C., and Elizabeth Bruenig of The Washington Post

COMMENTARY

\hypertarget{tony-messenger-of-the-st-louis-post-dispatch}{%
\subsection{Tony Messenger of The St. Louis
Post-Dispatch}\label{tony-messenger-of-the-st-louis-post-dispatch}}

Mr. Messenger, The Post-Dispatch's metro columnist, was cited for a
\href{https://www.stltoday.com/news/local/metro/post-dispatch-columnist-tony-messenger-wins-pulitzer-prize/article_ef22909b-12a9-510c-802f-48d0a66cb16c.html}{series
of pieces} that exposed how poor people convicted of misdemeanor crimes
were charged fees for their time in jail, sometimes leading to years of
debt and further imprisonment. The columns resulted in a ruling by the
Missouri Supreme Court that the practice was illegal, although Mr.
Messenger, 52, said it persists. In an interview, he called it ``a story
of human tragedy.''

\textbf{Finalists} Caitlin Flanagan of The Atlantic*;* Melinda
Henneberger of The Kansas City Star

CRITICISM

\hypertarget{carlos-lozada-of-the-washington-post}{%
\subsection{Carlos Lozada of The Washington
Post}\label{carlos-lozada-of-the-washington-post}}

Mr. Lozada, 47, The Post's
\href{https://www.washingtonpost.com/people/carlos-lozada}{nonfiction
book critic}, won for reviews and essays on politics, truth, immigration
and American identity in the Trump era. In an interview, he called it
``an irony of the time'' that a president with seemingly few literary
interests had fueled such a publishing explosion of topical work. ``It's
a really rich time to dig in to books as means to understand what is
going on right now,'' he said.

\textbf{Finalists} Manohla Dargis of The New York Times*;* Jill Lepore
of The New Yorker

Image

Brent Staples in The New York Times newsroom on Monday after he won the
Pulitzer Prize for editorial writing. Credit...Hiroko Masuike/The New
York Times

EDITORIAL WRITING

\hypertarget{brent-staples-of-the-new-york-times}{%
\subsection{Brent Staples of The New York
Times}\label{brent-staples-of-the-new-york-times}}

Mr. Staples, a member of The Times's editorial board, was cited for his
writing on racial justice and culture, including pieces about how the
suffrage movement betrayed black women, Southern newspapers' role in
lynchings and the Afrofuturism behind the movie ``Black Panther.''

In
\href{https://www.nytimes3xbfgragh.onion/2019/04/15/opinion/brent-staples-pulitzer-prize.html}{a
collection of his work} published on Monday, The Times wrote that Mr.
Staples, 67, ``has sought to correct the parts of the national narrative
on race that have been sanitized and distorted.''

\textbf{Finalists} The **** editorial board of The Advocate in Baton
Rouge, La., and the editorial board of The Capital Gazette in Annapolis,
Md.

EDITORIAL CARTOONING

\hypertarget{darrin-bell-freelancer}{%
\subsection{Darrin Bell, freelancer}\label{darrin-bell-freelancer}}

Mr. Bell, whose \href{http://darrinbell.com/}{work} is syndicated by
King Features, won for cartoons that addressed racial injustice and
political turmoil surrounding the Trump administration. He was cited for
``calling out lies, hypocrisy and fraud.'' He is the first
African-American winner in the category, a Pulitzer spokeswoman
confirmed. ``All the nights I called home and told my wife and kids I
had to stay at the office to cover something that just happened were not
for nothing,'' he said in an email.

\textbf{Finalists} Ken Fisher, freelancer, Rob Rogers, freelancer

Image

One of the pictures that earned the photography staff of Reuters the
Pulitzer Prize for breaking news photography for coverage of the flow of
migrants from Central and South America.Credit...Adrees Latif/Reuters

BREAKING NEWS PHOTOGRAPHY

\hypertarget{photography-staff-of-reuters}{%
\subsection{Photography Staff of
Reuters}\label{photography-staff-of-reuters}}

The photography staff of Reuters won the prize for breaking news
photography for its
\href{https://www.reuters.com/news/picture/reuters-wins-pulitzer-prize-for-migrant-idUSRTX6RS1A}{series
of photos} titled ``On the Migrant Trail to America.'' Taken by 11
photographers, the winning images, inventive and heart-wrenching in
equal measure, conveyed the violent conditions that Central American
migrants were leaving behind in their homelands, and the harsh response
they received from the authorities when they got to the United States.

\textbf{Finalists} Noah Berger, John Locher and Ringo H.W. Chiu of
Associated Press; Staff of Associated Press

Image

Lorenzo Tugnoli of The Washington Post, the winner of the Pulitzer Prize
for feature photography. Credit...The Pulitzer Prizes, via Associated
Press

FEATURE PHOTOGRAPHY

\hypertarget{lorenzo-tugnoli-of-the-washington-post}{%
\subsection{Lorenzo Tugnoli of The Washington
Post}\label{lorenzo-tugnoli-of-the-washington-post}}

Some of the pictures Lorenzo Tugnoli, 39,
\href{https://www.washingtonpost.com/graphics/2018/world/amp-stories/photos-of-war-in-yemen/}{has
taken in Yemen} could be from anywhere: a young girl having her height
measured, veiled women in a classroom. That these images alternate with
photos of starving infants and armed men in camouflage only heightens
their impact. In an interview, Mr. Tugnoli said he had tried to balance
depictions of the extent and immediacy of the tragedy with pictures that
displayed deep respect for their subjects' everyday lives.

\textbf{Finalists} Craig F. Walker of The Boston Globe; ** Maggie Steber
and Lynn Johnson of National Geographic

FICTION

\hypertarget{the-overstory-by-richard-powers-ww-norton}{%
\subsection{The Overstory by Richard Powers (W.W.
Norton)}\label{the-overstory-by-richard-powers-ww-norton}}

Mr. Powers, 61, is known as a brainy novelist, but
``\href{https://www.nytimes3xbfgragh.onion/2018/04/09/books/review/overstory-richard-powers.html}{The
Overstory}'' tested even his intellectual capacity: Its central
characters are trees. The environmental plot involves humans, but also
communication occurring in nature. Mr. Powers called the Pulitzer Prize
``an astonishing recognition,'' and he said the reception to the novel
proved to him that readers were ``hungry for a story that takes the
nonhuman seriously, and tries to reconnect the human to a world we're so
alienated from.''

\textbf{Finalists} ``The Great Believers,'' by Rebecca Makkai (Viking
Books); ``There There,'' by Tommy Orange (Alfred A. Knopf)

DRAMA

\hypertarget{fairview-by-jackie-sibblies-drury}{%
\subsection{Fairview by Jackie Sibblies
Drury}\label{fairview-by-jackie-sibblies-drury}}

Ms. Drury's
\href{https://www.nytimes3xbfgragh.onion/2018/06/17/theater/review-theater-as-sabotage-in-the-dazzling-fairview.html}{dazzling
play} appears at first to be a family comedy, but subverts expectations
with a head-spinning shift in content and form that forces theatergoers
to think in new ways about race and the white gaze. Ms. Drury, 37, said
she was trying ``to communicate to white people what it's like walking
around in a body of color and having people judge you.'' The play, which
debuted last year at Soho Rep, will return in June
\href{https://www.tfana.org/current-season/fairview/overview}{at Theater
for a New Audience}.

\textbf{Finalists} ``Dance Nation,'' by Claire Barron; ``What the
Constitution Means to Me,'' by Heidi Schreck.

Image

"Frederick Douglass: Prophet of Freedom," by David W. Blight, winner of
the Pulitzer Prize for history.Credit...Simon \& Schuster, via
Associated Press

HISTORY

\hypertarget{frederick-douglass-prophet-of-freedom-by-david-w-blight-simon--schuster}{%
\subsection{Frederick Douglass: Prophet of Freedom by David W. Blight
(Simon \&
Schuster)}\label{frederick-douglass-prophet-of-freedom-by-david-w-blight-simon--schuster}}

Mr. Blight, a professor at Yale University, was cited for ``a
breathtaking history'' that traced Douglass's transformation from
runaway slave to one of the most profound thinkers on race, equality and
American identity. \href{http://www.davidwblight.com/}{Mr. Blight}, 70,
began the book 12 years ago after encountering a private collection of
Douglass manuscripts assembled by a retired African-American surgeon in
Savannah, Ga. ``What does it mean to be an American?'' Professor Blight
said. ``Nobody has had more to say about that than Douglass.''

\textbf{Finalists} ``Civilizing Torture: An American Tradition,'' by W.
Fitzhugh Brundage (Harvard University Press); ``American Eden: David
Hosack, Botany, and Medicine in the Garden of the Early Republic,'' by
Victoria Johnson (Liveright/W.W. Norton).

BIOGRAPHY

\hypertarget{the-new-negro-the-life-of-alain-locke-by-jeffrey-c-stewart-oxford-university-press}{%
\subsection{The New Negro: The Life of Alain Locke by Jeffrey C. Stewart
(Oxford University
Press)}\label{the-new-negro-the-life-of-alain-locke-by-jeffrey-c-stewart-oxford-university-press}}

Mr. Stewart, a professor of black studies at the University of
California, Santa Barbara, was honored for offering ``a panoramic view
of the personal trials and artistic triumphs of the father of the Harlem
Renaissance and the movement he inspired.'' Locke, whose protégés
included Langston Hughes and Zora Neale Hurston, believed artistic
expression was crucial to racial progress. ``We don't often celebrate
people who create room and space for others,'' Mr. Stewart, 69, said.
``That's one thing I really love about Locke.''

\textbf{Finalists} ``The Road Not Taken: Edward Lansdale and the
American Tragedy in Vietnam,'' by Max Boot (Liveright/W.W. Norton), and
``Proust's Duchess: How Three Celebrated Women Captured the Imagination
of Fin-de-Siècle Paris,'' by Caroline Weber (Alfred A. Knopf).

POETRY

\hypertarget{be-with-by-forrest-gander-new-directions}{%
\subsection{Be With by Forrest Gander (New
Directions)}\label{be-with-by-forrest-gander-new-directions}}

The poems in \href{https://forrestgander.com/}{Forrest Gander}'s latest
collection deal with, among other things, grief. His wife, the acclaimed
poet C. D. Wright, died suddenly in 2016 at 67. ``I didn't write for
almost two years,'' Gander, 63, said of the period after her death. ``We
were always great readers of each other's work.'' When he started
writing again, the poems ``just came out,'' requiring less revising than
usual. He was glad readers felt, as he did, that ``writing about the
darkness might be transformative.''

\textbf{Finalists} ``feeld,'' by jos charles (Milkweed); ``Like,'' by A.
E. Stallings (Farrar, Straus and Giroux).

Image

"Amity and Prosperity: One Family and the Fracturing of America" by
Eliza Griswold, the winner of the Pulitzer Prize for nonfiction.
Credit...Farrar, Straus and Giroux, via Associated Press

GENERAL NONFICTION

\hypertarget{amity-and-prosperity-one-family-and-the-fracturing-of-america-by-eliza-griswold-farrar-straus-and-giroux}{%
\subsection{Amity and Prosperity: One Family and the Fracturing of
America by Eliza Griswold (Farrar, Straus and
Giroux)}\label{amity-and-prosperity-one-family-and-the-fracturing-of-america-by-eliza-griswold-farrar-straus-and-giroux}}

\href{https://www.elizagriswoldauthor.com/}{Ms. Griswold}, 46, observed
one family in Amity, Penn., over seven years for this book about how
hydraulic fracturing, better known as fracking, affected the town.
``What Griswold depicts is a community, like the earth, cracked open,''
\href{https://www.nytimes3xbfgragh.onion/2018/06/13/books/review-amity-prosperity-fracking-eliza-griswold.html}{our
reviewer wrote}. ``Every book takes a village, that's the nature of
books, but this one took two,'' said Ms. Griswold, adding that she felt
``deeply grateful'' to the people of Amity.

\textbf{Finalists} ``Rising: Dispatches From the New American Shore,''
by Elizabeth Rush (Milkweed); ``In a Day's Work,'' by Bernice Yeung (The
New Press).

Image

Ellen Reid, winner of the Pulitzer Prize for music. Credit...James
Matthew Daniel

MUSIC

\hypertarget{prism-by-ellen-reid-premiered-by-the-los-angeles-opera-on-nov-29-2018}{%
\subsection{``prism'' by Ellen Reid, premiered by the Los Angeles Opera
on Nov. 29,
2018}\label{prism-by-ellen-reid-premiered-by-the-los-angeles-opera-on-nov-29-2018}}

In this haunting opera, a mother and daughter grapple with trauma and
guilt after a sexual assault.
The\href{https://www.pulitzer.org/winners/ellen-reid}{Pulitzer board
praised} the way Ms. Reid, 36, uses ``sophisticated vocal writing and
striking instrumental timbres to confront difficult subject matter: the
effects of sexual and emotional abuse.'' Ms. Reid said in an interview
that she had worked on the piece for five years. ``There were times when
I didn't want to finish the piece,'' she said. Roxie Perkins wrote the
elliptical, poetic libretto.

\textbf{Finalists} ``Sustain,'' by Andrew Norman, premiered by the Los
Angeles Philharmonic (Schott Music), and ``Still,'' by James Romig,
released by New World Records.

SPECIAL CITATION

\hypertarget{the-capital-gazette-annapolis-md}{%
\subsection{The Capital Gazette, Annapolis,
Md.}\label{the-capital-gazette-annapolis-md}}

Employees of the Capital Gazette were recognized for
\href{https://www.capitalgazette.com/news/annapolis/bs-md-gazette-shooting-20180628-story.html}{reporting
and publishing news} even as they dealt with a shooting in their
newsroom that left five colleagues dead. After the attack, the deadliest
against journalists in United States history, some journalists gathered
in a garage across the street to file articles and photographs. ``We are
\href{https://twitter.com/chaseacook/status/1012465236195061766}{putting
out a damn paper},'' one said on Twitter that day. The Pulitzer board
awarded the paper a \$100,000 grant to support its work.

Image

Aretha Franklin, who died last year, received a special citation for
what the Pulitzer board called her "indelible contribution to American
music and culture for more than five decades.''Credit...Chris
Pizzello/Reuters

Special citation

\hypertarget{aretha-franklin}{%
\subsection{Aretha Franklin}\label{aretha-franklin}}

When the ``Queen of Soul"
\href{https://www.nytimes3xbfgragh.onion/2018/08/16/obituaries/aretha-franklin-dead.html}{died
last year at 76}, she had won 18 Grammy Awards and placed more than 100
singles on the Billboard charts (``Respect'' and ``Think'' among them).
The Pulitzer board recognized Franklin, who bridged gospel traditions
with secular music, ``for her indelible contribution to American music
and culture for more than five decades.'' In receiving the posthumous
honor, she joins such other popular music titans as Hank Williams (2010)
and Duke Ellington (1999).

Advertisement

\protect\hyperlink{after-bottom}{Continue reading the main story}

\hypertarget{site-index}{%
\subsection{Site Index}\label{site-index}}

\hypertarget{site-information-navigation}{%
\subsection{Site Information
Navigation}\label{site-information-navigation}}

\begin{itemize}
\tightlist
\item
  \href{https://help.nytimes3xbfgragh.onion/hc/en-us/articles/115014792127-Copyright-notice}{©~2020~The
  New York Times Company}
\end{itemize}

\begin{itemize}
\tightlist
\item
  \href{https://www.nytco.com/}{NYTCo}
\item
  \href{https://help.nytimes3xbfgragh.onion/hc/en-us/articles/115015385887-Contact-Us}{Contact
  Us}
\item
  \href{https://www.nytco.com/careers/}{Work with us}
\item
  \href{https://nytmediakit.com/}{Advertise}
\item
  \href{http://www.tbrandstudio.com/}{T Brand Studio}
\item
  \href{https://www.nytimes3xbfgragh.onion/privacy/cookie-policy\#how-do-i-manage-trackers}{Your
  Ad Choices}
\item
  \href{https://www.nytimes3xbfgragh.onion/privacy}{Privacy}
\item
  \href{https://help.nytimes3xbfgragh.onion/hc/en-us/articles/115014893428-Terms-of-service}{Terms
  of Service}
\item
  \href{https://help.nytimes3xbfgragh.onion/hc/en-us/articles/115014893968-Terms-of-sale}{Terms
  of Sale}
\item
  \href{https://spiderbites.nytimes3xbfgragh.onion}{Site Map}
\item
  \href{https://help.nytimes3xbfgragh.onion/hc/en-us}{Help}
\item
  \href{https://www.nytimes3xbfgragh.onion/subscription?campaignId=37WXW}{Subscriptions}
\end{itemize}
