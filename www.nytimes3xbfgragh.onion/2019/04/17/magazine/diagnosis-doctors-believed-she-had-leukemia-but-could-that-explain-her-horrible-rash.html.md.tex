Sections

SEARCH

\protect\hyperlink{site-content}{Skip to
content}\protect\hyperlink{site-index}{Skip to site index}

\href{https://myaccount.nytimes3xbfgragh.onion/auth/login?response_type=cookie\&client_id=vi}{}

\href{https://www.nytimes3xbfgragh.onion/section/todayspaper}{Today's
Paper}

The Doctors Believed That the Woman Had Leukemia. But Could That Explain
Her Horrible Rash?

\url{https://nyti.ms/2Uoy35d}

\begin{itemize}
\item
\item
\item
\item
\item
\item
\end{itemize}

Advertisement

\protect\hyperlink{after-top}{Continue reading the main story}

Supported by

\protect\hyperlink{after-sponsor}{Continue reading the main story}

\href{/column/diagnosis}{Diagnosis}

\hypertarget{the-doctors-believed-that-the-woman-had-leukemia-but-could-that-explain-her-horrible-rash}{%
\section{The Doctors Believed That the Woman Had Leukemia. But Could
That Explain Her Horrible
Rash?}\label{the-doctors-believed-that-the-woman-had-leukemia-but-could-that-explain-her-horrible-rash}}

\includegraphics{https://static01.graylady3jvrrxbe.onion/images/2019/04/21/magazine/21mag-Diagnosis-1/21mag-Diagnosis-1-articleLarge.png?quality=75\&auto=webp\&disable=upscale}

By \href{https://www.nytimes3xbfgragh.onion/by/lisa-sanders-md}{Lisa
Sanders, M.D.}

\begin{itemize}
\item
  April 17, 2019
\item
  \begin{itemize}
  \item
  \item
  \item
  \item
  \item
  \item
  \end{itemize}
\end{itemize}

The woman watched the streetlights flash by as the ambulance carried her
through the midnight roads of her hometown, Bay City, Mich., to Karmanos
Cancer Institute in Detroit. Her husband was in the front seat talking
with the driver. He wasn't allowed to ride in the back with her. But he
had been at her side for the past four weeks, ever since she found out
she had ulcerative colitis.

She was 35 and completely healthy until just after the Fourth of July,
when she had several days of terrible cramping pain and unrelenting
diarrhea. Her family pressured her into going to the local community
hospital, where a colonoscopy provided her with the diagnosis of U.C.
After a few days, she was able to go home and, a couple of weeks out,
was finally starting to feel better. Fortunately, it was summer, and the
students at the school where she worked as an administrator wouldn't be
back for several weeks. Thanks to the medications she was taking,
mesalamine and sulfasalazine, she was able to eat again and even put on
some of the weight she lost.

\textbf{Another Round of Illness}

She was just starting to feel like herself when she came down with what
felt like the flu. She had a fever, and her whole body ached. She called
her doctor, who agreed that it probably was some virus. The next day,
she woke to find a sprinkle of itchy red pinpoints dotting her chest.
Her doctor thought that was also a symptom of the virus. He wasn't
worried, and neither was she.

That weekend, she gave a goodbye party for her sister, who was moving
out of state. Between the fever and the rash, which now covered her
upper body, she could barely drag herself through the event. She
collapsed in bed once everyone left and was burning up when her husband
came to join her. When her temperature went over 103, he insisted that
they go back to the hospital --- her second visit in less than a month.

She called her gastroenterologist from the car to let him know she was
sick. Stop the medications, he told her. You could be allergic to one of
them. In the emergency room, her fever was too high to measure. They
gave her antibiotics and sent her to the intensive-care unit, where she
spent the rest of the night under an icy blanket to bring her
temperature down.

\textbf{Something More Serious?}

When her fever was under control, she was transferred out of intensive
care and awoke to find herself on the floor dedicated to patients with
cancer. A doctor delivered the frightening news --- it looked as though
she had leukemia. A leukemia is an abnormal proliferation of the
earliest cells that develop and grow in the bone marrow --- those that
would normally become either red or, in this patient's case, white blood
cells. If the cancer isn't treated, the immature blood cells multiply
unchecked, taking over the bone marrow and, eventually, the entire blood
stream. Routine blood tests often show high levels of these early white
blood cells and not enough red blood cells or platelets.

When they tested the patient's blood, the number of red blood cells and
platelets was average, but her white-blood-cell count was extremely high
--- nearly 90,000, about nine times what is usually seen. And the cells
themselves looked very abnormal. So, the doctor explained, it seemed
likely that she had some kind of leukemia. They had to determine which
kind and treat her; to do that, they were transferring her to a hospital
in Detroit that specializes in cancer.

\textbf{An Alternative Explanation}

The next night in the ambulance traveling to Karmanos hospital, 100
miles away, she couldn't believe she was going to a cancer hospital. She
didn't know what she had but was certain it couldn't be cancer. The
following morning, the oncologist and his resident came by. Her family
surrounded her bed as if their mere presence could shield her from
whatever came next. The older doctor explained that they would be
running tests to find out exactly what kind of cancer she had.

\includegraphics{https://static01.graylady3jvrrxbe.onion/images/2019/04/21/magazine/21mag-Diagnosis-2/21mag-Diagnosis-2-articleLarge.png?quality=75\&auto=webp\&disable=upscale}

Could she see a dermatologist, she asked? The only thing bothering her,
other than the idea that she had cancer, was this terrible rash. It had
gone from itchy to sore and blistering. Her arms were a deep crimson.
Her face felt tight and tender. Her hands were nearly purple and hugely
swollen. The doctor agreed to send a dermatologist. The rash could be
from an allergic reaction, he acknowledged, but it could also be caused
by her leukemia.

\textbf{The Story Provides a Clue}

Dr. Benjamin Workman, the dermatology resident on call, reviewed her
chart before going to see her. He had been struck by her elevated white
count --- who wouldn't be? --- but wasn't sure what to make of it.
Workman sat at the bedside and listened as the patient recounted the
horrors of her summer so far: Before this diagnosis, she had a diagnosis
of ulcerative colitis and had to take daily medications. And then three
weeks later, this terrible rash and fever.

He looked at her hands. They did look painful --- swollen to nearly
twice their normal size. The skin there was dark and weeping in places.
And the rest of her --- her arms, her face, even her neck glowed with
what looked like the worst sunburn ever. Her face seemed swollen as
well. Suddenly, he knew what she had.

He couldn't say for sure that she didn't have cancer, he told her. But
he suspected that she had a severe form of a drug allergy. The rash and
fever appearing just weeks after she started taking new medications was
classic for what used to be called Dress syndrome: drug rash with
eosinophilia and systemic symptoms, now more commonly known as
drug-induced hypersensitivity. Sulfasalazine was in a class of drugs
known to, on rare occasion, trigger these terrible and sometimes
life-threatening allergic reactions.

He had given a presentation on this very topic a few weeks earlier, but
she was the first patient he had ever seen with the ailment. This type
of severe drug allergy is still not well understood. It was first
identified in 1950 in a patient who developed a high fever and a rash
after taking anti-seizure medication. Since then, it has also been seen
in patients taking a variety of medications ranging from
anti-inflammatories to antidepressants. There may be a genetic
predisposition in many cases. The incidence varies with the specific
drug but is thought to be on average around one in 100,000 patients.

\textbf{Relief}

Cancer was still possible, the dermatologist explained. The robust
white-cell proliferation, especially to the extent seen in this patient,
is far more common in a leukemia than in Dress --- which can also show
an elevated white-blood-cell count, though a much lower one. If it was
Dress, a course of high-dose steroids would help her. But they would
still need to be certain she didn't have any form of leukemia. The next
day, she had a bone-marrow biopsy; that test would show for certain if
cancer was present.

She was in the bathroom the next morning when the oncologist came in
with the results. She could hear him speaking to her parents but
couldn't make out what he was saying. She quickly finished washing her
hands and rushed into the room to see her parents and the doctor
smiling. They were the first smiles she had seen since she got sick.

``It's not cancer,'' the resident announced. And suddenly they were all
crying. She had never accepted the diagnosis, but hearing at last what
she hoped and believed all along felt like a miracle. The high-dose
steroids prescribed by Workman were effective, and a couple of days
later she finally went home.

The rash improved slowly. It took her months before she was willing to
try another medication for her ulcerative colitis. When she finally did,
it helped. And with the advice of a nutritionist, she revamped her diet.
The two interventions worked so well that now, after two years of
intense treatment, she has been off her medication and in total
remission for the past year.

Advertisement

\protect\hyperlink{after-bottom}{Continue reading the main story}

\hypertarget{site-index}{%
\subsection{Site Index}\label{site-index}}

\hypertarget{site-information-navigation}{%
\subsection{Site Information
Navigation}\label{site-information-navigation}}

\begin{itemize}
\tightlist
\item
  \href{https://help.nytimes3xbfgragh.onion/hc/en-us/articles/115014792127-Copyright-notice}{©~2020~The
  New York Times Company}
\end{itemize}

\begin{itemize}
\tightlist
\item
  \href{https://www.nytco.com/}{NYTCo}
\item
  \href{https://help.nytimes3xbfgragh.onion/hc/en-us/articles/115015385887-Contact-Us}{Contact
  Us}
\item
  \href{https://www.nytco.com/careers/}{Work with us}
\item
  \href{https://nytmediakit.com/}{Advertise}
\item
  \href{http://www.tbrandstudio.com/}{T Brand Studio}
\item
  \href{https://www.nytimes3xbfgragh.onion/privacy/cookie-policy\#how-do-i-manage-trackers}{Your
  Ad Choices}
\item
  \href{https://www.nytimes3xbfgragh.onion/privacy}{Privacy}
\item
  \href{https://help.nytimes3xbfgragh.onion/hc/en-us/articles/115014893428-Terms-of-service}{Terms
  of Service}
\item
  \href{https://help.nytimes3xbfgragh.onion/hc/en-us/articles/115014893968-Terms-of-sale}{Terms
  of Sale}
\item
  \href{https://spiderbites.nytimes3xbfgragh.onion}{Site Map}
\item
  \href{https://help.nytimes3xbfgragh.onion/hc/en-us}{Help}
\item
  \href{https://www.nytimes3xbfgragh.onion/subscription?campaignId=37WXW}{Subscriptions}
\end{itemize}
