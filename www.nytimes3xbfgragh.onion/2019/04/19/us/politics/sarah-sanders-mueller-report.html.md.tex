Sections

SEARCH

\protect\hyperlink{site-content}{Skip to
content}\protect\hyperlink{site-index}{Skip to site index}

\href{https://www.nytimes3xbfgragh.onion/section/politics}{Politics}

\href{https://myaccount.nytimes3xbfgragh.onion/auth/login?response_type=cookie\&client_id=vi}{}

\href{https://www.nytimes3xbfgragh.onion/section/todayspaper}{Today's
Paper}

\href{/section/politics}{Politics}\textbar{}Sanders's `Slip of the
Tongue' Would Be a Problem in Some White Houses. Not Trump's.

\url{https://nyti.ms/2veBgtY}

\begin{itemize}
\item
\item
\item
\item
\item
\end{itemize}

Advertisement

\protect\hyperlink{after-top}{Continue reading the main story}

Supported by

\protect\hyperlink{after-sponsor}{Continue reading the main story}

\hypertarget{sanderss-slip-of-the-tongue-would-be-a-problem-in-some-white-houses-not-trumps}{%
\section{Sanders's `Slip of the Tongue' Would Be a Problem in Some White
Houses. Not
Trump's.}\label{sanderss-slip-of-the-tongue-would-be-a-problem-in-some-white-houses-not-trumps}}

\includegraphics{https://static01.graylady3jvrrxbe.onion/images/2019/04/20/us/politics/20dc-Sanders-print/merlin_151924818_bbd91406-c4ea-4d56-af0a-37577b9b6482-articleLarge.jpg?quality=75\&auto=webp\&disable=upscale}

By \href{https://www.nytimes3xbfgragh.onion/by/annie-karni}{Annie Karni}
and \href{https://www.nytimes3xbfgragh.onion/by/maggie-haberman}{Maggie
Haberman}

\begin{itemize}
\item
  April 19, 2019
\item
  \begin{itemize}
  \item
  \item
  \item
  \item
  \item
  \end{itemize}
\end{itemize}

PALM BEACH, Fla. --- After admitting to investigators for the special
counsel, Robert S. Mueller III, that she delivered a false statement
from the White House podium,
\href{https://www.nytimes3xbfgragh.onion/2019/06/13/us/politics/sarah-sanders-leaving-white-house.html}{Sarah
Huckabee Sanders}, the White House press secretary, defended herself in
Trumpian fashion on Friday morning. She counterattacked.

\href{https://www.nytimes3xbfgragh.onion/interactive/2019/04/18/us/politics/mueller-report-document.html}{The
Mueller report} revealed that
\href{https://www.nytimes3xbfgragh.onion/2019/06/13/us/politics/sarah-sanders-leaving-white-house.html}{Ms.
Sanders} had acknowledged that her repeated claim in 2017 that she had
personally communicated with ``countless'' F.B.I. officials who told her
they were happy with
\href{https://www.nytimes3xbfgragh.onion/2019/06/13/us/politics/sarah-sanders-leaving-white-house.html}{President
Trump's} decision to fire James B. Comey as the agency's director was a
``slip of the tongue'' and not founded on any facts.

Asked on ``Good Morning America'' if the report had damaged her
credibility, Ms. Sanders responded that she had made the statement in
the heat of the moment, and that it was not ``a scripted talking
point.''

But then she added, ``I'm sorry that I wasn't a robot like the Democrat
Party that went out for two and a half years and repeated time and time
again that there was definitely Russian collusion between the president
and his campaign.''

It has been a hallmark of the
\href{https://www.nytimes3xbfgragh.onion/2019/06/13/us/politics/sarah-sanders-leaving-white-house.html}{Trump
White House} never to admit a mistake, never to apologize and never to
cede a point. This case was no different. ``The White House staff will
never be lectured on truth-telling from the media that pushed a flat-out
lie about Donald Trump for two years,'' Hogan Gidley, a White House
spokesman, said in an email.

Ms. Sanders, who has often taken news media outlets to task for what
they write and report and has accused them of spreading ``fake news''
about Mr. Trump, was just a footnote in Mr. Mueller's 448-page report.
But because of her public role, the anecdote involving her false
statement loomed large in the broader portrait by the special counsel of
a White House
\href{https://www.nytimes3xbfgragh.onion/2019/04/18/us/politics/white-house-mueller-report.html?action=click\&module=Spotlight\&pgtype=Homepage}{defined
by a culture of dishonesty}.

And it remains to be seen how the incident and Ms. Sanders's response
will affect her credibility with reporters, with whom she has had fewer
interactions than many of her predecessors since the White House quietly
phased out the daily press briefing over the past nine months.

More often, Ms. Sanders speaks for the president on friendly programs
like ``Fox \& Friends.'' She has also come to view her role as a person
who defends her colleagues and the president, rather than someone who
delivers a message to the press about the work that is underway at the
White House.

Some of Mr. Trump's aides and allies acknowledged on Friday that it was
problematic for the president's chief spokeswoman to spend airtime
defending her own credibility. But White House officials --- some of
whom think Ms. Sanders is taking an unfair beating in the press --- do
not expect Mr. Trump to be fazed by the controversy. Unlike previous
administrations, in which officials feared blows to their credibility in
public, Mr. Trump's press aides are generally performing for an audience
of one --- the president.

For months, Ms. Sanders, who declined to comment, has toyed with the
idea of moving on from a position she has held for almost two years as
part of a White House press and communications team that has often
seemed dysfunctional.

The job of communications director has remained vacant since Bill Shine,
the fifth person to hold the position,
\href{https://www.nytimes3xbfgragh.onion/2019/03/08/us/politics/bill-shine-resigns.html}{resigned
in March}. Along with Kellyanne Conway, the White House counselor, and
Mercedes Schlapp, a top communications adviser, Ms. Sanders is now
viewed internally as part of a three-headed messaging operation, and it
is not clear who is really in charge, other than Mr. Trump.

\includegraphics{https://static01.graylady3jvrrxbe.onion/images/2017/05/11/us/11factcheck/11factcheck-videoSixteenByNine3000.jpg}

Ms. Sanders's predecessor, Sean Spicer, suffered his own reputational
damage on Day 1 of the administration, when Mr. Trump ordered him to
falsely tell reporters that the crowd at his inauguration was ``the
largest audience to ever witness an inauguration, period, both in person
and around the globe.'' Mr. Spicer has since said he regretted that
comment and his belligerent appearance in the briefing room that day,
which also enraged the president.

But for now, Ms. Sanders is not willing to admit any misstep.

In an interview with Fox News on Thursday night, she told Sean Hannity
that ``I acknowledged that I had a slip of the tongue when I used the
word `countless,' but it's not untrue'' because her broader point was
that ``a number of both current and former F.B.I. agents agreed with the
president.''

In her interview with George Stephanopoulos on ``Good Morning America,''
he also grilled her about her false statement that Mr. Trump did not
dictate the administration's statement on a meeting that his eldest son,
Donald Trump Jr., took with a Russian operative at Trump Tower in June
2016. Mr. Mueller's report revealed that the president's lawyers
acknowledged that he did, in fact, dictate the statement that was
released.

``I'm not denying that he had involvement in what the statement said,''
Ms. Sanders said on Friday. ``That was the information I was given at
the time.''

In previous administrations, when it was revealed that press secretaries
had delivered false statements from the White House podium, there was
more of a reaction and often some soul-searching.

``She is in a credibility bind that in most White Houses would be
disqualifying, but probably not in this one,'' said David Axelrod, a
former senior adviser to President Barack Obama. ``No matter how
honorable you were coming in, when you sign up as the spokesman for
someone who habitually lies, you, by necessity, become a habitual
liar.''

In 2003, Scott McClellan, President George W. Bush's press secretary,
told reporters that two senior administration officials, Karl Rove and
I. Lewis Libby Jr., were ``not involved'' in a leak about Valerie Plame,
the C.I.A. operative.

Later, in his memoir, Mr. McClellan delivered a mea culpa, admitting
that he had been given false information to disseminate to the press.
``I had unknowingly passed along false information,'' Mr. McClellan
wrote, adding that five of the highest-ranking White House officials,
including the president, had misinformed him.

Others have survived accusations that they lied. In 1983, Larry Speakes,
a press aide to President Ronald Reagan, told reporters that the notion
that the United States would invade Grenada was ``preposterous.'' The
invasion took place the next day, and Mr. Speakes said he learned of it
belatedly.

Jay Carney, Mr. Obama's press secretary, took heat for helping to
propagate the president's message on the Affordable Care Act that ``if
you like your health care plan, you can keep it'' after it became clear
that the administration was unable to deliver on the promise.

But for veteran aides of the Trump White House, the furor over Ms.
Sanders's remark has only added to a bunker mentality shared by aides
who feel they are constantly defending themselves, and the president,
from unfair attacks.

And on the president's favorite show, Ms. Sanders had the last word.
``James Comey was a disgraced leaker who tried to politicize and
undermine the very agency he was supposed to run,'' she told Mr.
Hannity. ``Firing James Comey remains one of the best decisions that the
president made.''

Advertisement

\protect\hyperlink{after-bottom}{Continue reading the main story}

\hypertarget{site-index}{%
\subsection{Site Index}\label{site-index}}

\hypertarget{site-information-navigation}{%
\subsection{Site Information
Navigation}\label{site-information-navigation}}

\begin{itemize}
\tightlist
\item
  \href{https://help.nytimes3xbfgragh.onion/hc/en-us/articles/115014792127-Copyright-notice}{©~2020~The
  New York Times Company}
\end{itemize}

\begin{itemize}
\tightlist
\item
  \href{https://www.nytco.com/}{NYTCo}
\item
  \href{https://help.nytimes3xbfgragh.onion/hc/en-us/articles/115015385887-Contact-Us}{Contact
  Us}
\item
  \href{https://www.nytco.com/careers/}{Work with us}
\item
  \href{https://nytmediakit.com/}{Advertise}
\item
  \href{http://www.tbrandstudio.com/}{T Brand Studio}
\item
  \href{https://www.nytimes3xbfgragh.onion/privacy/cookie-policy\#how-do-i-manage-trackers}{Your
  Ad Choices}
\item
  \href{https://www.nytimes3xbfgragh.onion/privacy}{Privacy}
\item
  \href{https://help.nytimes3xbfgragh.onion/hc/en-us/articles/115014893428-Terms-of-service}{Terms
  of Service}
\item
  \href{https://help.nytimes3xbfgragh.onion/hc/en-us/articles/115014893968-Terms-of-sale}{Terms
  of Sale}
\item
  \href{https://spiderbites.nytimes3xbfgragh.onion}{Site Map}
\item
  \href{https://help.nytimes3xbfgragh.onion/hc/en-us}{Help}
\item
  \href{https://www.nytimes3xbfgragh.onion/subscription?campaignId=37WXW}{Subscriptions}
\end{itemize}
