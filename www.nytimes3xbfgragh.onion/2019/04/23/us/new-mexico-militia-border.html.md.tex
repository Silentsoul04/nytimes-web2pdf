Sections

SEARCH

\protect\hyperlink{site-content}{Skip to
content}\protect\hyperlink{site-index}{Skip to site index}

\href{https://www.nytimes3xbfgragh.onion/section/us}{U.S.}

\href{https://myaccount.nytimes3xbfgragh.onion/auth/login?response_type=cookie\&client_id=vi}{}

\href{https://www.nytimes3xbfgragh.onion/section/todayspaper}{Today's
Paper}

\href{/section/us}{U.S.}\textbar{}Militia Defiant in New Mexico: `It's
My God-Given Right to Be Here'

\url{https://nyti.ms/2Vn7Mss}

\begin{itemize}
\item
\item
\item
\item
\item
\item
\end{itemize}

Advertisement

\protect\hyperlink{after-top}{Continue reading the main story}

Supported by

\protect\hyperlink{after-sponsor}{Continue reading the main story}

\hypertarget{militia-defiant-in-new-mexico-its-my-god-given-right-to-be-here}{%
\section{Militia Defiant in New Mexico: `It's My God-Given Right to Be
Here'}\label{militia-defiant-in-new-mexico-its-my-god-given-right-to-be-here}}

\includegraphics{https://static01.graylady3jvrrxbe.onion/images/2019/04/23/us/23militia1/merlin_153880782_2f31c0be-b25b-4491-8242-ca9f94188a07-articleLarge.jpg?quality=75\&auto=webp\&disable=upscale}

By \href{https://www.nytimes3xbfgragh.onion/by/simon-romero}{Simon
Romero}

\begin{itemize}
\item
  April 23, 2019
\item
  \begin{itemize}
  \item
  \item
  \item
  \item
  \item
  \item
  \end{itemize}
\end{itemize}

{[}\emph{Read the}
\href{https://www.nytimes3xbfgragh.onion/2019/04/18/us/mexico-border-deaths.html?searchResultPosition=1}{\emph{latest
edition of Crossing the Border}}\emph{, a limited-run newsletter about
life where the United States and Mexico meet.}
\emph{\href{https://www.nytimes3xbfgragh.onion/newsletters/crossing-the-border?module=inline}{Sign
up here to receive the next issue}} \emph{in your inbox.}{]}

SUNLAND PARK, N.M. --- Their
\href{https://www.nytimes3xbfgragh.onion/2019/04/22/us/militia-border-new-mexico.html}{commander}
is in jail. The authorities are giving them until Friday to clear out
and leave. But the United Constitutional Patriots, the right-wing
militia under scrutiny over detaining migrant families at the border
with Mexico, is digging in.

``It's my God-given right to be here,'' said one balaclava-clad militia
member who gave his name only as Viper. Chafing at the hostile reactions
to the militia's actions, he said that he was an Army veteran and that
he expected his group, if pushed out, to set up camp in another location
along the border.

``The guys in Washington say one thing about not wanting us on the
ground but no one from the Border Patrol here has ever told me they
don't want our help,'' he said, squinting under the midday sun. ``We're
here to protect Americans from the illegals violating our sovereignty.''

The militia's encampment on Tuesday was little more than a trailer and a
few pickup trucks next to a newly installed ``No Trespassing'' sign. It
appeared to reflect the impasse these armed vigilantes now find
themselves in: under the magnifying glass of the F.B.I.,
\href{https://gizmodo.com/paypal-gofundme-yank-accounts-for-far-right-militia-ro-1834193555}{cut
off} from funding, defending their actions to the public and torn
asunder by the
\href{https://www.nytimes3xbfgragh.onion/2019/04/20/us/militia-arrest-border-new-mexico.html}{arrest
of their leader}, a resident of northwest New Mexico and a three-time
felon who went by the alias Johnny Horton Jr. but whose real name is
Larry Hopkins.

Tempers were on edge in the camp, which is next to railroad tracks and a
dusty road where the existing wall on the border comes to an abrupt end.
That is where militia members have been filming their activities, and
where, on several occasions, they have
\href{https://www.nytimes3xbfgragh.onion/2019/04/18/us/new-mexico-militia.html}{confronted
and detained groups of migrants} who have crossed the border into the
United States.

These migrants, like others who have crossed the border in recent
months, have been largely Central Americans. In sharp contrast to
previous inflows of migrants, most of these new arrivals routinely seek
to surrender to Border Patrol agents in order to legally request asylum.

\includegraphics{https://static01.graylady3jvrrxbe.onion/images/2019/04/23/us/23militia2/merlin_153880986_e2ddf39e-89ba-48b8-adf9-6aeb6b950abc-articleLarge.jpg?quality=75\&auto=webp\&disable=upscale}

Still, the militiamen and those who support them have seen their work as
necessary.

Armando Gonzalez, 52, said he drove to Sunland Park, which sits on New
Mexico's borders with Mexico and Texas, from his home in Tulsa, Okla.,
to lend a hand to the United Constitutional Patriots. He said he
believed that the news media had distorted the group's work and the
reality of life along the border.

``If you ask me this is all about politics,'' said Mr. Gonzalez, adding
that he was a disabled Army veteran with post-traumatic stress disorder.
``The Democrats want illegal immigration because that means more votes
for them.''

``But I took an oath to protect my country and what's happening on the
border is an invasion threatening our people,'' said Mr. Gonzalez, who
is planning to sleep in his 2001 Chevrolet Suburban. ``These men are
patriots and I'm proud to stand alongside them.''

Mr. Gonzalez, who was carrying a 9-millimeter handgun in a holster
strapped to his belt, said he considered himself part of the militia.
But the two other men at the spartan camp said they didn't think Mr.
Gonzalez was part of their group.

``He just showed up today,'' said the man called Viper.

At one point on Tuesday, there were more reporters milling about the
camp than militiamen. Some of the journalists broadcast in Spanish to
reach viewers on both sides of the border.

Judith Sierra, the owner of Tortilleria Sierra in Sunland Park, chuckled
at the thought of armed men traveling to the border to chase after women
and children. It's not uncommon for migrants to pass through her
property, she said, lately in large groups.

``We offer them water or tortillas,'' she said, adding that the Border
Patrol is never far behind.

``Even with a fence or whatever else it's not going to stop people,''
said Ms. Sierra, who was attending to a steady stream of customers in a
black apron dusted with flour. ``They'll come over or under, somehow,
they'll find a way to cross.''

Meanwhile, the authorities in Sunland Park, whose population of about
15,600 is more than 90 percent Latino, have made it clear that the men
in the desert are testing their patience.

Image

Jesus Hernandez, 70, at Elena Memorial Park in Sunland Park. Mr.
Hernandez said he opposed the civilian militia camped out near the
border.Credit...Ryan Christopher Jones for The New York Times

The Sunland Park Police Department is evicting the group, telling its
members that they want them gone from the camp site by the end of the
week. Union Pacific, which operates the railroad near the militia's
camp, warned the armed men that they were trespassing to reach their
camp.

``These outsiders talk about an invasion when they are the ones invading
our peace and quiet,'' said Jesus Hernandez, 70, who lives in Sunland
Park and works in nearby El Paso shining shoes. ``I have some advice for
them: Get a job and leave us alone."

What's next for the United Constitutional Patriots? Their ranks, while
never numerous, seem to be thinning. Mr. Hopkins is in jail on a felony
weapons charge and Jim Benvie, the group's self-described spokesman, was
away from the camp on Tuesday.

An older militia member known as "Pops" used vulgar language when a
reporter asked him a question, making it clear that he doesn't care for
journalists. He also warned against trying to take his picture, telling
reporters he didn't want publicity.

Still, officials are increasing scrutiny of the group. Three Democratic
members of Congress --- Deb Haaland and Ben Ray Luján of New Mexico and
Veronica Escobar of Texas --- wrote to F.B.I. Director Christopher Wray
requesting an investigation of the United Constitutional Patriots.

``The right to stop and detain should remain reserved for law
enforcement,'' they said in the letter. ``As a nation, we must end this
xenophobic behavior.''

Elsewhere in Sunland Park, residents are wondering when the militia will
leave. Claudio Alvarado, 69, a retired foreman with the Texas Gas
Company, was on a morning walk through his neighborhood with his two
young grandchildren.

Mr. Alvarado said that he had lived in the town since he was 12 and that
his son-in-law had worked as a Border Patrol agent for a decade. Mr.
Alvarado made it clear that he doesn't like the idea of militia members
patrolling the border.

``It makes me angry,'' Mr. Alvarado said, ``because that's not their
job.''

Advertisement

\protect\hyperlink{after-bottom}{Continue reading the main story}

\hypertarget{site-index}{%
\subsection{Site Index}\label{site-index}}

\hypertarget{site-information-navigation}{%
\subsection{Site Information
Navigation}\label{site-information-navigation}}

\begin{itemize}
\tightlist
\item
  \href{https://help.nytimes3xbfgragh.onion/hc/en-us/articles/115014792127-Copyright-notice}{©~2020~The
  New York Times Company}
\end{itemize}

\begin{itemize}
\tightlist
\item
  \href{https://www.nytco.com/}{NYTCo}
\item
  \href{https://help.nytimes3xbfgragh.onion/hc/en-us/articles/115015385887-Contact-Us}{Contact
  Us}
\item
  \href{https://www.nytco.com/careers/}{Work with us}
\item
  \href{https://nytmediakit.com/}{Advertise}
\item
  \href{http://www.tbrandstudio.com/}{T Brand Studio}
\item
  \href{https://www.nytimes3xbfgragh.onion/privacy/cookie-policy\#how-do-i-manage-trackers}{Your
  Ad Choices}
\item
  \href{https://www.nytimes3xbfgragh.onion/privacy}{Privacy}
\item
  \href{https://help.nytimes3xbfgragh.onion/hc/en-us/articles/115014893428-Terms-of-service}{Terms
  of Service}
\item
  \href{https://help.nytimes3xbfgragh.onion/hc/en-us/articles/115014893968-Terms-of-sale}{Terms
  of Sale}
\item
  \href{https://spiderbites.nytimes3xbfgragh.onion}{Site Map}
\item
  \href{https://help.nytimes3xbfgragh.onion/hc/en-us}{Help}
\item
  \href{https://www.nytimes3xbfgragh.onion/subscription?campaignId=37WXW}{Subscriptions}
\end{itemize}
