Sections

SEARCH

\protect\hyperlink{site-content}{Skip to
content}\protect\hyperlink{site-index}{Skip to site index}

\href{https://myaccount.nytimes3xbfgragh.onion/auth/login?response_type=cookie\&client_id=vi}{}

\href{https://www.nytimes3xbfgragh.onion/section/todayspaper}{Today's
Paper}

\href{/section/upshot}{The Upshot}\textbar{}They Want It to Be Secret:
How a Common Blood Test Can Cost \$11 or Almost \$1,000

\begin{itemize}
\item
\item
\item
\item
\item
\item
\end{itemize}

Advertisement

\protect\hyperlink{after-top}{Continue reading the main story}

Upshot

Supported by

\protect\hyperlink{after-sponsor}{Continue reading the main story}

\hypertarget{they-want-it-to-be-secret-how-a-common-blood-test-can-cost-11-or-almost-1000}{%
\section{They Want It to Be Secret: How a Common Blood Test Can Cost
\$11 or Almost
\$1,000}\label{they-want-it-to-be-secret-how-a-common-blood-test-can-cost-11-or-almost-1000}}

Huge price discrepancies like that are unimaginable in other industries.
Also unusual: not knowing the fee ahead of time.

\href{https://www.nytimes3xbfgragh.onion/by/margot-sanger-katz}{\includegraphics{https://static01.graylady3jvrrxbe.onion/images/2019/12/13/reader-center/author-margot-sanger-katz/author-margot-sanger-katz-thumbLarge.png}}

By
\href{https://www.nytimes3xbfgragh.onion/by/margot-sanger-katz}{Margot
Sanger-Katz}

\begin{itemize}
\item
  April 30, 2019
\item
  \begin{itemize}
  \item
  \item
  \item
  \item
  \item
  \item
  \end{itemize}
\end{itemize}

It's one of the most common tests in medicine, and it is performed
millions of times a year around the country. Should a metabolic blood
panel test cost \$11 or \$952?

Both of these are real, negotiated prices, paid by health insurance
companies to laboratories in Jackson, Miss., and El Paso in 2016.
\href{https://www.healthcostinstitute.org/blog/entry/hmi-2019-service-prices}{New
data}, analyzing the health insurance claims of 34 million Americans
covered by large commercial insurance companies, shows that enormous
swings in price for identical services are common in health care. In
just one market --- Tampa, Fla. --- the most expensive blood test costs
40 times as much as **** the least expensive one.

If you're a patient seeking a metabolic blood panel, good luck finding
out what it will cost. Although hospitals are
\href{https://www.nytimes3xbfgragh.onion/2019/01/13/us/politics/hospital-prices-online.html}{now
required to publish} a list of the prices they would like patients to
pay for their services, the amounts that medical providers actually
agree to accept from insurance companies tend to remain closely held
secrets. Some insurance companies provide consumers with tools to help
steer them away from the \$450 test, but in many cases you won't know
the price your insurance company agreed to until you get the bill. If
you have an insurance deductible, a \$400 --- or even a \$200 --- bill
for a blood test can be an unpleasant surprise.

Outside of health care, a swing of prices as huge as the one for blood
tests in Tampa is unheard-of. Recent studies of the retail prices of
\href{https://www.semanticscholar.org/paper/THE-MORPHOLOGY-OF-PRICE-DISPERSION-Kaplan/fe959cc8917c4843ef926a5c1fd5e480317a588e}{ketchup}
and
\href{https://pubs.aeaweb.org/doi/pdfplus/10.1257/mic.20170130}{drywall},
for example, showed much less variation. A bottle of Heinz ketchup in
the most expensive store in a given market could cost **** six times as
much as it would in the least expensive store. But most bottles of
ketchup tended to cost around the same. And, in every case, you would
know the price of your ketchup before buying it.

``It's shocking,'' said Amanda Starc, an associate professor at the
Kellogg School of Management at Northwestern, who has studied the issue.
``The variation in prices in health care is much greater than we see in
other industries.''

In some cities, the blood test prices look more like the prices for
consumer goods. Most tests in Baltimore cost around \$30. Most in
Portland, Ore., cost around \$20. But if you live in Miami or Los
Angeles, the price becomes much harder to predict.

Hospitals and insurers negotiate over prices in private, and they don't
want competitors to know about the deals they've been able to cut. The
data in this article comes from the Health Care Cost Institute, which
pools bills from three large insurance companies. (Even the institute
can't say which insurers and providers are attached to the different
prices, and it has eliminated certain markets with less competition
where it might be easy to guess.)

The Trump administration may eliminate this secrecy, making numbers like
the ones in these charts more common and easier to find. As The Wall
Street Journal has
\href{https://www.wsj.com/articles/trump-administration-weighs-publicizing-hospital-rates-negotiated-with-insurers-11551990505}{reported},
the administration has asked for comments on a proposal to require
doctors and hospitals to
\href{https://www.nytimes3xbfgragh.onion/2019/03/08/us/politics/trump-health-care-rates.html}{publish
negotiated prices}.

\href{https://www.healthcostinstitute.org/blog/entry/hmi-2019-service-prices}{The
institute examined} several common procedures and observed two kinds of
pricing differences. Prices vary considerably between markets. And, in
many metro areas, they range widely between one health care provider and
another.

Because these are prices paid by insurance companies, many experts say
the differences between markets matter more, because they affect
insurance premiums that all those with insurance in that area pay, even
if they don't get a blood test or an
\href{https://www.nytimes3xbfgragh.onion/interactive/2015/12/15/upshot/the-best-places-for-better-cheaper-health-care-arent-what-experts-thought.html}{operation}.
On average, a cesarean section birth in the Bay Area costs more than
three times as much as one near Louisville, Ky., according to the
institute's data.

The average hotel room in the San Francisco area last year cost only
around double the average hotel room in Louisville, according to STR,
which tracks the industry.

The swing within markets increasingly matters for patients, too, as the
share of employer plans with sizable deductibles
\href{https://www.nytimes3xbfgragh.onion/2016/09/15/business/health-insurance-analysis-kaiser.html}{keeps
rising}. That means that choosing a provider where your insurance
company has failed to strike a good deal could mean significant
out-of-pocket costs.

In some cases, prices may be higher because the quality of services or
the cost of doing business in a given market is higher. More influential
is market power, that of either insurers or hospitals, research shows.

Sherry Glied, a health economist who is the dean of the Wagner School of
Public Service at New York University, said a bigger factor was probably
how many patients your insurer sent to a given hospital. Popular places
are likely to offer better prices, because the insurance company
negotiates a bulk discount. The most expensive providers tend to be the
ones where the insurance company has little negotiating leverage --- and
where the service is so rarely used it doesn't mind the higher price.

``One person buys one hamburger, and another buys 1,000,'' she said.
``And it completely makes sense that the guy who buys 1,000 hamburgers
gets a better price.''

That sort of market power can work in the opposite direction, too. In
markets where there is a dominant hospital chain, or a powerful hospital
that many patients insist on using, insurers tend to face high prices,
with less leverage to bargain the hospitals down. Martin Gaynor, a
professor of health economics at Carnegie Mellon University, was a
co-author of a
\href{https://academic.oup.com/qje/article-abstract/134/1/51/5090426?redirectedFrom=fulltext}{recent
study} showing that in markets where fewer hospitals competed for
patients, the hospitals tended to be paid more.

``Some of these really simple diagnostic tests --- what the heck?'' Mr.
Gaynor said. ``It does mean, in a sense, the market is broken in terms
of problems with market power.''

The prices that hospitals and doctors charge to patients who are not in
their insurance networks also range widely, and are typically (though
not always) higher than the prices that insurers pay. The Obama
administration began publishing these list prices for some of the most
common medical services on a
\href{https://www.cms.gov/Research-Statistics-Data-and-Systems/Statistics-Trends-and-Reports/Medicare-Provider-Charge-Data/index.html}{government
website}. The Trump administration recently began requiring hospitals to
also publish a comprehensive list of prices on their own sites, though
the data
\href{https://www.nytimes3xbfgragh.onion/2019/01/13/us/politics/hospital-prices-online.html}{can
be challenging} to use.

For years, Jeanne Pinder, who runs the consumer-oriented website
\href{https://clearhealthcosts.com/}{Clear Health Costs}, has been
collecting the cash prices for medical procedures around the country.
She said the only health care services with predictable pricing were the
cash-only treatments that insurance doesn't cover, like Lasik eye
surgery, Botox and tooth whitening.

``When you get into M.R.I.s, ultrasounds and blood tests, they are
crazy,'' she said. ``The secrecy in pricing all over this marketplace
encourages this behavior.''

\begin{center}\rule{0.5\linewidth}{\linethickness}\end{center}

The data from the Health Care Cost Institute shows real, negotiated
prices for services in metropolitan areas among patients with private
employer insurance through Aetna, Humana and UnitedHealthcare. The
prices range from the 10th percentile to the 90th percentile, but
eliminate the lowest and highest prices from the range. For outpatient
services, the price is the cost for a single C.P.T. code. For inpatient
services, the number represents all payments from an admission
associated with the relevant D.R.G. code, so some of the variation
reflects differences in care as well as price.

Advertisement

\protect\hyperlink{after-bottom}{Continue reading the main story}

\hypertarget{site-index}{%
\subsection{Site Index}\label{site-index}}

\hypertarget{site-information-navigation}{%
\subsection{Site Information
Navigation}\label{site-information-navigation}}

\begin{itemize}
\tightlist
\item
  \href{https://help.nytimes3xbfgragh.onion/hc/en-us/articles/115014792127-Copyright-notice}{©~2020~The
  New York Times Company}
\end{itemize}

\begin{itemize}
\tightlist
\item
  \href{https://www.nytco.com/}{NYTCo}
\item
  \href{https://help.nytimes3xbfgragh.onion/hc/en-us/articles/115015385887-Contact-Us}{Contact
  Us}
\item
  \href{https://www.nytco.com/careers/}{Work with us}
\item
  \href{https://nytmediakit.com/}{Advertise}
\item
  \href{http://www.tbrandstudio.com/}{T Brand Studio}
\item
  \href{https://www.nytimes3xbfgragh.onion/privacy/cookie-policy\#how-do-i-manage-trackers}{Your
  Ad Choices}
\item
  \href{https://www.nytimes3xbfgragh.onion/privacy}{Privacy}
\item
  \href{https://help.nytimes3xbfgragh.onion/hc/en-us/articles/115014893428-Terms-of-service}{Terms
  of Service}
\item
  \href{https://help.nytimes3xbfgragh.onion/hc/en-us/articles/115014893968-Terms-of-sale}{Terms
  of Sale}
\item
  \href{https://spiderbites.nytimes3xbfgragh.onion}{Site Map}
\item
  \href{https://help.nytimes3xbfgragh.onion/hc/en-us}{Help}
\item
  \href{https://www.nytimes3xbfgragh.onion/subscription?campaignId=37WXW}{Subscriptions}
\end{itemize}
