Sections

SEARCH

\protect\hyperlink{site-content}{Skip to
content}\protect\hyperlink{site-index}{Skip to site index}

\href{https://www.nytimes3xbfgragh.onion/section/business}{Business}

\href{https://myaccount.nytimes3xbfgragh.onion/auth/login?response_type=cookie\&client_id=vi}{}

\href{https://www.nytimes3xbfgragh.onion/section/todayspaper}{Today's
Paper}

\href{/section/business}{Business}\textbar{}Facebook Objects to Chris
Hughes Essay Calling for Its Breakup

\url{https://nyti.ms/2Lz3MBE}

\begin{itemize}
\item
\item
\item
\item
\item
\end{itemize}

Advertisement

\protect\hyperlink{after-top}{Continue reading the main story}

Supported by

\protect\hyperlink{after-sponsor}{Continue reading the main story}

\hypertarget{facebook-objects-to-chris-hughes-essay-calling-for-its-breakup}{%
\section{Facebook Objects to Chris Hughes Essay Calling for Its
Breakup}\label{facebook-objects-to-chris-hughes-essay-calling-for-its-breakup}}

\includegraphics{https://static01.graylady3jvrrxbe.onion/images/2019/05/09/multimedia/09xp-Facebook/merlin_154613076_0b61f737-845f-4076-8623-f6e505288e13-articleLarge.jpg?quality=75\&auto=webp\&disable=upscale}

By \href{https://www.nytimes3xbfgragh.onion/by/heather-murphy}{Heather
Murphy}

\begin{itemize}
\item
  May 9, 2019
\item
  \begin{itemize}
  \item
  \item
  \item
  \item
  \item
  \end{itemize}
\end{itemize}

Facebook pushed back Thursday after Chris Hughes, a billionaire
co-founder of the company, argued in a New York Times Op-Ed essay that
the company should be broken up and regulated.

``Facebook accepts that with success comes accountability,'' Nick Clegg,
Facebook's vice president for global affairs and communication, wrote in
a statement. ``But you don't enforce accountability by calling for the
breakup of a successful American company.''

The statement followed a
\href{https://www.nytimes3xbfgragh.onion/2019/05/09/opinion/sunday/chris-hughes-facebook-zuckerberg.html}{lengthy
Op-Ed by Mr. Hughes} published online Thursday morning arguing that the
social media giant be subjected to extensive government oversight and
separated into multiple companies, notably by spinning off the
photo-sharing app Instagram and the messenger service WhatsApp. The
essay will appear in print on Sunday.

Mr. Hughes was a co-founder of Facebook with Mark Zuckerberg 15 years
ago when they were undergraduates at Harvard. He
\href{http://www.pbs.org/wnet/need-to-know/the-daily-need/the-facebook-co-founder-who-got-away/3847/}{left
the company in 2007} to work with the Obama campaign and more recently
has focused on the issue of income inequality. In the essay, he said
repeatedly that Mr. Zuckerberg is a good, kind person but simply has too
much power for any one individual.

``The government must hold Mark accountable,'' Mr. Hughes wrote in the
5,800-word column, arguing that the social media giant has grown far too
powerful.

``From our earliest days, Mark used the word `domination' to describe
our ambitions, with no hint of irony or humility,'' he wrote. ``Over a
decade later, Facebook has earned the prize of domination. It is worth
half a trillion dollars and commands, by my estimate, more than 80
percent of the world's social networking revenue. It is a powerful
monopoly, eclipsing all of its rivals and erasing competition from the
social networking category.''

\emph{{[}Read Chris Hughes's}
\href{https://www.nytimes3xbfgragh.onion/2019/05/09/opinion/sunday/chris-hughes-facebook-zuckerberg.html?action=click\&module=Intentional\&pgtype=Article}{\emph{argument
for breaking up Facebook}}\emph{.{]}}

Though others have made similar arguments over the past year, the essay
by a longtime confidant of Mr. Zuckerberg drew intense interest, given
Mr. Hughes's role in the start of the company and his once close-knit
relationship with Mr. Zuckerberg.

Naturally, the debate also played out on Facebook itself and other
social media platforms. Some said Mr. Hughes's argument rang hollow
because he had made hundreds of millions of dollars from Facebook.
Others said that a breakup would not fix the core issues posed by a few
companies with so much power over the flow of information.

The specific prescription Mr. Hughes offered was government
intervention. He wrote that the Federal Trade Commission and the Justice
Department ``should enforce antitrust laws by undoing the Instagram and
WhatsApp acquisitions and banning future acquisitions for several
years.'' He called the decision to allow Facebook to buy these two major
competitors ``the F.T.C.'s biggest mistake.''

He also argued that simply breaking up Facebook would not be sufficient.
``We need a new agency, empowered by Congress to regulate tech
companies,'' he wrote, saying that ``its first mandate should be to
protect privacy.''

One person who embraced the essay was Senator Elizabeth Warren, the
Massachusetts Democrat who has made the breakup of Facebook, Google and
Amazon a cornerstone of her presidential campaign. She
\href{https://twitter.com/ewarren/status/1126493176406081537}{expressed}
her support on Twitter and
\href{https://medium.com/@teamwarren/heres-how-we-can-break-up-big-tech-9ad9e0da324c}{on
Medium}.

``Weak antitrust enforcement has led to a dramatic reduction in
competition and innovation in the tech sector,'' Ms. Warren wrote.
``Venture capitalists are now hesitant to fund new start-ups to compete
with these big tech companies because it's so easy for the big companies
to either snap up growing competitors or drive them out of business.''

\href{https://www.law.columbia.edu/faculty/timothy-wu}{Tim Wu,} a
professor of antitrust law at Columbia Law School and the author of
``\href{https://www.amazon.com/Curse-Bigness-Antitrust-New-Gilded/dp/0999745468}{The
Curse of Bigness: Antitrust in the New Gilded Age},'' also applauded the
proposal. ``It would be part of an American tradition of breaking up
some of the largest tech companies that has produced positive results,''
he said.

He pointed to the
\href{https://www.nytimes3xbfgragh.onion/1995/09/22/business/at-t-move-is-a-reversal-of-course-set-in-1980-s.html}{case
of AT\&T}, the telephone monopoly that was disassembled into eight
smaller companies in 1984. ``A lot of what we now call the internet
revolution was a byproduct of the breakup,'' he said.

But Kent Lassman, the president of the\href{http://cei.org/}{Competitive
Enterprise Institute}, said industry-specific regulations can backfire.
Established business interests will use them to ward off competition, he
said. ``It is absurd for innovative companies to require constant
permission from regulators like the F.T.C. to adapt to a dynamic
market.''

As a practical matter, a Facebook breakup won't happen anytime soon,
said \href{https://www.law.gwu.edu/william-e-kovacic}{Bill Kovacic}, a
professor of global competition at George Washington Law School. He said
Facebook could argue that it had grown WhatsApp and Instagram into
successful businesses with satisfied customers: ``How do we know they
would have turned out to be an independent and effective competitor?''

He said that Mr. Hughes's essay ``dramatically underestimates the
difficulty for the government to prevail in such a case.''

Advertisement

\protect\hyperlink{after-bottom}{Continue reading the main story}

\hypertarget{site-index}{%
\subsection{Site Index}\label{site-index}}

\hypertarget{site-information-navigation}{%
\subsection{Site Information
Navigation}\label{site-information-navigation}}

\begin{itemize}
\tightlist
\item
  \href{https://help.nytimes3xbfgragh.onion/hc/en-us/articles/115014792127-Copyright-notice}{©~2020~The
  New York Times Company}
\end{itemize}

\begin{itemize}
\tightlist
\item
  \href{https://www.nytco.com/}{NYTCo}
\item
  \href{https://help.nytimes3xbfgragh.onion/hc/en-us/articles/115015385887-Contact-Us}{Contact
  Us}
\item
  \href{https://www.nytco.com/careers/}{Work with us}
\item
  \href{https://nytmediakit.com/}{Advertise}
\item
  \href{http://www.tbrandstudio.com/}{T Brand Studio}
\item
  \href{https://www.nytimes3xbfgragh.onion/privacy/cookie-policy\#how-do-i-manage-trackers}{Your
  Ad Choices}
\item
  \href{https://www.nytimes3xbfgragh.onion/privacy}{Privacy}
\item
  \href{https://help.nytimes3xbfgragh.onion/hc/en-us/articles/115014893428-Terms-of-service}{Terms
  of Service}
\item
  \href{https://help.nytimes3xbfgragh.onion/hc/en-us/articles/115014893968-Terms-of-sale}{Terms
  of Sale}
\item
  \href{https://spiderbites.nytimes3xbfgragh.onion}{Site Map}
\item
  \href{https://help.nytimes3xbfgragh.onion/hc/en-us}{Help}
\item
  \href{https://www.nytimes3xbfgragh.onion/subscription?campaignId=37WXW}{Subscriptions}
\end{itemize}
