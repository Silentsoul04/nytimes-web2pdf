Sections

SEARCH

\protect\hyperlink{site-content}{Skip to
content}\protect\hyperlink{site-index}{Skip to site index}

\href{https://myaccount.nytimes3xbfgragh.onion/auth/login?response_type=cookie\&client_id=vi}{}

\href{https://www.nytimes3xbfgragh.onion/section/todayspaper}{Today's
Paper}

\href{/section/opinion}{Opinion}\textbar{}Nike Told Me to Dream Crazy,
Until I Wanted a Baby

\url{https://nyti.ms/2vSfYCT}

\begin{itemize}
\item
\item
\item
\item
\item
\item
\end{itemize}

Advertisement

\protect\hyperlink{after-top}{Continue reading the main story}

\href{/section/opinion}{Opinion}

Supported by

\protect\hyperlink{after-sponsor}{Continue reading the main story}

\hypertarget{nike-told-me-to-dream-crazy-until-i-wanted-a-baby}{%
\section{Nike Told Me to Dream Crazy, Until I Wanted a
Baby}\label{nike-told-me-to-dream-crazy-until-i-wanted-a-baby}}

Being a mother and a champion was a crazy dream. It didn't have to be.

By Alysia Montaño

Video by Max Cantor and
\href{https://www.nytimes3xbfgragh.onion/by/taige-jensen}{Taige Jensen}

Written and Produced by
\href{https://www.nytimes3xbfgragh.onion/by/lindsay-crouse}{Lindsay
Crouse}

Alysia Montaño is an Olympic runner and three-time U.S. national
champion.

\begin{itemize}
\item
  May 12, 2019
\item
  \begin{itemize}
  \item
  \item
  \item
  \item
  \item
  \item
  \end{itemize}
\end{itemize}

\href{https://www.nytimes3xbfgragh.onion/es/2019/05/14/nike-patrocinio-mujeres/}{Leer
en español}

\includegraphics{https://static01.graylady3jvrrxbe.onion/images/2019/05/12/opinion/maternity-leave-video/maternity-leave-video-videoSixteenByNine3000.jpg}

\emph{\textbf{Update: Following this report, after broad}}
\textbf{\href{mailto:https://www.si.com/olympics/2019/05/24/nike-maternity-protection-sponsorships-contract-allyson-felix-alysia-montano}{\emph{public
outcry}}} \emph{\textbf{and a}}
\textbf{\href{mailto:https://herrerabeutler.house.gov/uploadedfiles/05_17_19_letter_to_nike.pdf}{\emph{congressional
inquiry}}\emph{, Nike}}
\textbf{\href{https://www.washingtonpost.com/sports/2019/08/16/under-fire-nike-expands-protections-pregnant-athletes/}{\emph{announced
a new maternity policy}}} \emph{\textbf{for all sponsored athletes on
Aug. 12. The new contract guarantees an athlete's pay and bonuses for 18
months around pregnancy. Three other athletic apparel companies added
maternity protections for sponsored athletes.}}

Many athletic apparel companies, including Nike, claim to elevate female
athletes. A commercial
\href{http://thesource.com/2019/02/25/serena-williams-narrates-new-nike-dream-crazier-commercial/}{released
in February} received widespread acclaim for spotlighting women at all
stages of their careers, from childhood to motherhood. On Mother's Day
this year, Nike
\href{https://www.youtube.com/watch?v=IHcWPVbDArU}{released a video}
promoting gender equality.

But that's just advertising.

The economics of sports like track and field are different than those of
professional sports like basketball or soccer. In track, athletes aren't
paid a salary by a league. Instead, their income comes almost
exclusively from sponsorship deals inked with apparel companies like
Nike and Asics.

The best of the best can supplement that income with prize money from
winning races outright. But the majority of athletes --- who are often
the breadwinners for their families --- sign exclusive five- or
six-figure deals that keep them bound to a single company.

For the vast majority of athletes, their sport is a way to earn a decent
living by doing what they love and excel at. They don't get rich.

Sports take a heavy toll on the human body, and sponsors accommodate
this with time off for injuries. But rarely do they offer enough time
off to have a child.

The four Nike executives who negotiate contracts for track and field
athletes are all men.

``Getting pregnant is the kiss of death for a female athlete,'' said
Phoebe Wright, who was a runner sponsored by Nike from 2010 through
2016. ``There's no way I'd tell Nike if I were pregnant.''

More than a dozen track athletes, agents and others familiar with the
business describe a multi-billion-dollar industry that praises women for
having families in public --- but doesn't guarantee them a salary during
pregnancy and early maternity.

For the Olympian Kara Goucher, the most difficult part of motherhood
wasn't resuming training just a week after childbirth in 2010. It wasn't
even when her doctor told her she must choose: run 120 miles each week
or breastfeed her son. Her body couldn't do both.

The toughest moment was when Ms. Goucher learned that Nike would stop
paying her until she started racing again. But she was already pregnant.
So, she scheduled a half-marathon three months after she had her son,
Colt. Then her son got dangerously ill. Ms. Goucher had to choose again:
be with her son or prepare for the race that she hoped would restart her
pay.

She kept training. ``I felt like I had to leave him in the hospital,
just to get out there and run, instead of being with him like a normal
mom would,'' Ms. Goucher said, crying at the memory. ``I'll never
forgive myself for that.''

Nike acknowledged in a statement that some of its sponsored athletes
have had their sponsorship payments reduced because of pregnancies. But
the company says it changed its approach in 2018 so that athletes are no
longer penalized. Nike declined to say if it wrote those changes into
its contracts.

According to a 2019 Nike track and field contract shared with The Times,
Nike can still reduce an athlete's pay ``for any reason'' if the athlete
doesn't meet a specific performance threshold, for example a top five
world ranking. There are no exceptions for childbirth, pregnancy or
maternity.

Most people who spoke to The Times requested anonymity because they
feared retribution, or had signed nondisclosure agreements, which may
help explain why these arrangements have persisted.

\href{https://parenting.nytimes3xbfgragh.onion/}{\emph{{[}The topics new
parents are talking about. Evidence-based guidance. Personal stories
that matter. Visit NYT Parenting for everything you need to raise
thriving babies and kids.{]}}}

Many American laws protect the rights of pregnant employees --- they
can't be fired, for instance. But, since professional athletes are more
like independent contractors, those protections don't apply.

When Alysia Montaño ran in the 2014 United States Championships while
eight months pregnant, she was celebrated as
``\href{https://www.flotrack.org/articles/5067150-that-pregnant-runner-alysia-montano-runs-221-in-usa-800-prelims}{the
pregnant runner}.'' Privately, she had to fight with her sponsor to keep
her paycheck.

Sponsors do sometimes pay new mothers --- Serena Williams is branded as
a famous example. But those who do get paid often have to beg for the
money.

Ms. Goucher made more than a dozen unpaid appearances on behalf of Nike
during her high-risk pregnancy. She had to wait more than four months to
disclose that she was pregnant, so that
\href{https://www.nytimes3xbfgragh.onion/2010/05/09/sports/09marathon.html}{Nike
could announce it in The Times for Mother's Day.}

These kinds of pressures can lead to health complications. Ms. Goucher,
for instance, has suffered from chronic hip injuries ever since she
raced the Boston Marathon seven months after childbirth.

``It took such a toll on me mentally and physically, for myself and for
my child,'' said Ms. Goucher. ``Returning to competition so quickly was
a bad choice for me. And looking back and knowing that I wasn't the kind
of mother that I want to be --- it's gut wrenching.''

New mothers don't just deal with their sponsors. Top athletes receive
health insurance from The United States Olympic Committee and U.S.A.
Track \& Field. But that insurance can vanish if women don't place in
the top tier of the nation's most competitive races. Ms. Goucher and Ms.
Montaño both lost their health insurance because they were unable to
compete at that level while having their children.

``Some people think women are racing pregnant for themselves,'' said Ms.
Wright. ``It sometimes is, but it's also because there's a baby to
feed.''

Advertisement

\protect\hyperlink{after-bottom}{Continue reading the main story}

\hypertarget{site-index}{%
\subsection{Site Index}\label{site-index}}

\hypertarget{site-information-navigation}{%
\subsection{Site Information
Navigation}\label{site-information-navigation}}

\begin{itemize}
\tightlist
\item
  \href{https://help.nytimes3xbfgragh.onion/hc/en-us/articles/115014792127-Copyright-notice}{©~2020~The
  New York Times Company}
\end{itemize}

\begin{itemize}
\tightlist
\item
  \href{https://www.nytco.com/}{NYTCo}
\item
  \href{https://help.nytimes3xbfgragh.onion/hc/en-us/articles/115015385887-Contact-Us}{Contact
  Us}
\item
  \href{https://www.nytco.com/careers/}{Work with us}
\item
  \href{https://nytmediakit.com/}{Advertise}
\item
  \href{http://www.tbrandstudio.com/}{T Brand Studio}
\item
  \href{https://www.nytimes3xbfgragh.onion/privacy/cookie-policy\#how-do-i-manage-trackers}{Your
  Ad Choices}
\item
  \href{https://www.nytimes3xbfgragh.onion/privacy}{Privacy}
\item
  \href{https://help.nytimes3xbfgragh.onion/hc/en-us/articles/115014893428-Terms-of-service}{Terms
  of Service}
\item
  \href{https://help.nytimes3xbfgragh.onion/hc/en-us/articles/115014893968-Terms-of-sale}{Terms
  of Sale}
\item
  \href{https://spiderbites.nytimes3xbfgragh.onion}{Site Map}
\item
  \href{https://help.nytimes3xbfgragh.onion/hc/en-us}{Help}
\item
  \href{https://www.nytimes3xbfgragh.onion/subscription?campaignId=37WXW}{Subscriptions}
\end{itemize}
