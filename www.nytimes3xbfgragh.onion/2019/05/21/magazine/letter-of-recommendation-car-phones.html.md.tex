Sections

SEARCH

\protect\hyperlink{site-content}{Skip to
content}\protect\hyperlink{site-index}{Skip to site index}

\href{https://myaccount.nytimes3xbfgragh.onion/auth/login?response_type=cookie\&client_id=vi}{}

\href{https://www.nytimes3xbfgragh.onion/section/todayspaper}{Today's
Paper}

Letter of Recommendation: Car Phones

\url{https://nyti.ms/2ElpYJm}

\begin{itemize}
\item
\item
\item
\item
\item
\end{itemize}

Advertisement

\protect\hyperlink{after-top}{Continue reading the main story}

Supported by

\protect\hyperlink{after-sponsor}{Continue reading the main story}

\href{/column/letter-of-recommendation}{Letter of Recommendation}

\hypertarget{letter-of-recommendation-car-phones}{%
\section{Letter of Recommendation: Car
Phones}\label{letter-of-recommendation-car-phones}}

\includegraphics{https://static01.graylady3jvrrxbe.onion/images/2019/05/26/magazine/26mag-LOR-image1/f9ee85a53dfd4394866f1d5893ea1d4a-articleLarge.jpg?quality=75\&auto=webp\&disable=upscale}

By Devin Friedman

\begin{itemize}
\item
  May 21, 2019
\item
  \begin{itemize}
  \item
  \item
  \item
  \item
  \item
  \end{itemize}
\end{itemize}

One of the accepted truths of the contemporary human condition is that
talking on the phone is awful. No one likes it. If you answer your
phone, people might be scared and wonder what's wrong with you. And what
kind of deranged individual would actually leave a voice mail message,
let alone listen to one? A voice mail message is just a prerecorded
one-sided phone call you're passively subjected to. If the internet age
has largely delivered only disappointments and a broken social contract,
at least it made it so you don't have to talk on phones, except with
airlines. After all, what are technology ``disrupters'' selling when
they say that now you can order food or see a doctor without even
putting on your pants? They're saying you won't have to be on the phone.

I am among those who believed talking on the phone was a subhuman
activity reserved for the desperate and disturbed. But then last summer,
after I moved with my family from New York to Los Angeles, I found
myself driving around and making phone calls. And liking it. Not just
liking it: looking forward to it. Calling people, taking calls, waiting
for calls. I call everyone now: old friends; older friends; my parents,
all the time, almost every day; people I work with and like; people I
work with and don't like. It's beautiful. To hear them talk! To have
them listen. When I get home, I sit in my car in front of my house and
find excuses to call someone else. It lowers my blood pressure, makes my
skin feel better, reduces my anxiety about death. The phone! Try it! You
do it with your mouth!

It's largely about the car. Think about it: What is a car now? A
four-wheeled method of warming the Earth until it destroys humanity?
Yes. But not just that. It's a phone that you sit in. That's
climate-controlled. That doesn't require a slab of microwave-heated
glass plastered to your face, because you don't have to hold anything.
To be on the phone in New York is to wander the streets raging to
yourself while wearing earbuds; to hold a little computer in front of
your mouth in a coffee shop; to sit on a train saying: ``I'm on the
train! I'm on the train!'' In Los Angeles, your phone has seat heaters.

I think my embrace of the car phone has something to do with the fact
that, scientifically speaking, we are growing more and more alone and
isolated. You know how the universe is expanding? At great speed?
(Specifically, it's expanding at a rate of about 72 kilometers per
second, per megaparsec, more or less; which means that depending on
where you are, the universe is expanding faster than the speed of
light.) My middle-school science teacher did a demonstration with a
Sharpie and a balloon to illustrate how it worked. He blew up the
balloon and made a series of dots on it with the marker. Then he blew it
up some more, and we watched the dots get farther and farther from each
other. When the universe expands, it's not merely that everything gets
farther away from some central point; it's that everything is getting
farther and farther from everything else.

Are we not all, as we proceed through our lives, solitary Earths? Are we
not only receding from our young selves, but kind of, also, everything?
The very culture we live in, which as we accrue more and more days
begins to feel foreign to us? Our friends and our jobs and the internet?
The people I knew in high school and my friends from New York and the
people I knew in college, all of them flee from me and one another and
themselves at sickening speeds. I see their little updates on Facebook
(which, for younger readers, is a message board used by the elderly to
discuss the crimes of immigrant gangs) with their aging faces that
astonish me and their grown children who terrify me --- they're nothing
but little marker dots of humanity getting farther away on that
metaphorical balloon that Mr. Brenner blew up for us those many years
ago.

I had no idea that cars are the best phones, because in New York we're
freaks and pariahs who are never in cars alone and don't understand how
normal Americans live. It's pleasing to travel the vast California
highway system, insulated by tempered glass and plastic with the ghost
voice of a friend traveling with you in the front seat. The timbre of
the human voice is pleasure. The crackle of the 5G ether during times
when no one really has anything to say. The expulsion of all the
thoughts that you had kicking around in your head that you didn't even
know were there until you opened your mouth and started talking. There
used to be \href{https://www.youtube.com/watch?v=emzK9dfL0qM}{an AT\&T
commercial}with the tagline ``Reach out and touch someone.'' It showed
heartwarming images of American immigrant children calling their
grandparents overseas, soldiers calling their wives, phone cables
collapsing time and space so people could be together. It's kind of a
nuts ad campaign if you think about it now --- \emph{Hey, use the
telephone! The telephone is good for talking to people!} But I guess
that's what I'm saying here. Talking to people is good. It makes us feel
better. Tagline: \emph{The phone, it's actually not just a small
computer!}

A warning: It will be uncomfortable at first. People will get ornery
when you call. A lot of them will immediately ask you what you want, why
you're calling, so it's best to be ready for that. But after a shorter
time than you would expect, people will get used it. And they will kind
of like it. And before long, you won't need a reason. There will be a
tacit understanding that you are two people seeking human connection,
corporeal texture and warmth, the way we like when cats and cows nuzzle
each other on the internet. All the while just in the car, hanging out,
by yourself.

Advertisement

\protect\hyperlink{after-bottom}{Continue reading the main story}

\hypertarget{site-index}{%
\subsection{Site Index}\label{site-index}}

\hypertarget{site-information-navigation}{%
\subsection{Site Information
Navigation}\label{site-information-navigation}}

\begin{itemize}
\tightlist
\item
  \href{https://help.nytimes3xbfgragh.onion/hc/en-us/articles/115014792127-Copyright-notice}{©~2020~The
  New York Times Company}
\end{itemize}

\begin{itemize}
\tightlist
\item
  \href{https://www.nytco.com/}{NYTCo}
\item
  \href{https://help.nytimes3xbfgragh.onion/hc/en-us/articles/115015385887-Contact-Us}{Contact
  Us}
\item
  \href{https://www.nytco.com/careers/}{Work with us}
\item
  \href{https://nytmediakit.com/}{Advertise}
\item
  \href{http://www.tbrandstudio.com/}{T Brand Studio}
\item
  \href{https://www.nytimes3xbfgragh.onion/privacy/cookie-policy\#how-do-i-manage-trackers}{Your
  Ad Choices}
\item
  \href{https://www.nytimes3xbfgragh.onion/privacy}{Privacy}
\item
  \href{https://help.nytimes3xbfgragh.onion/hc/en-us/articles/115014893428-Terms-of-service}{Terms
  of Service}
\item
  \href{https://help.nytimes3xbfgragh.onion/hc/en-us/articles/115014893968-Terms-of-sale}{Terms
  of Sale}
\item
  \href{https://spiderbites.nytimes3xbfgragh.onion}{Site Map}
\item
  \href{https://help.nytimes3xbfgragh.onion/hc/en-us}{Help}
\item
  \href{https://www.nytimes3xbfgragh.onion/subscription?campaignId=37WXW}{Subscriptions}
\end{itemize}
