5 Writers on the Summers They Will Never Forget

\url{https://nyti.ms/2wy2UTr}

\begin{itemize}
\item
\item
\item
\item
\item
\item
\end{itemize}

\includegraphics{https://static01.graylady3jvrrxbe.onion/images/2019/05/28/magazine/mag-beach-pattern/mag-beach-pattern-articleLarge.jpg?quality=75\&auto=webp\&disable=upscale}

Sections

\protect\hyperlink{site-content}{Skip to
content}\protect\hyperlink{site-index}{Skip to site index}

\hypertarget{5-writers-on-the-summers-they-will-never-forget}{%
\section{5 Writers on the Summers They Will Never
Forget}\label{5-writers-on-the-summers-they-will-never-forget}}

In the pool, by the sea, tales of long summer days on the water.

Credit...

Supported by

\protect\hyperlink{after-sponsor}{Continue reading the main story}

\emph{Longer, warmer days are finally here, and the beach beckons. But
what makes a beach trip truly unforgettable? We asked five writers for
their favorite memories about sand, surf and summertime.}

\begin{center}\rule{0.5\linewidth}{\linethickness}\end{center}

\includegraphics{https://static01.graylady3jvrrxbe.onion/images/2019/05/27/magazine/mag-beach-05/mag-beach-05-articleLarge.jpg?quality=75\&auto=webp\&disable=upscale}

\hypertarget{michael-paterniti-in-pursuit-of-an-underwater-goal-at-the-town-pool}{%
\subsection{Michael Paterniti in pursuit of an underwater goal at the
town
pool.}\label{michael-paterniti-in-pursuit-of-an-underwater-goal-at-the-town-pool}}

\textbf{It's the summer} of 1975. I'm 10, and the biggest goal in my
life is to swim two 25-meter laps of the pool without a breath: 50
meters underwater, at the end of swim practice, what's called, aptly
enough, ``no-breathers.'' To be honest, I've spent most of the summer
underwater already. I seem to be in training for whalehood, or
something. If you were an anthropologist hanging around our little
summer swim club, your notes might read, ``presents as a normal suburban
boy spending an abnormal amount of time underwater.''

Why?

Maybe it feels good down here on hot, humid, dead-air days, dawdling in
the cool eddies. I feel more alive in the pool, everything glassy and
gravity-less and bubbles. I like it when friends come down with me,
their hair haloed hilariously around their head, and we act out some
important conversation because you can't hear anything but soft
whooshing in that underwater room, not even the adults jabbering up on
the deck.

This summer has included a first girlfriend, with whom I communicate by
terrified avoidance; Patty Hearst, who's still out there somewhere,
wearing her beret; and the movie ``Jaws.'' We live in a town on Long
Island Sound, and a 25-foot shark munching people, even if mechanical,
has left its impression. No one swims in the sound this summer. Or if
they must, they swim as fast as they can back to shore. When I jump from
the pier, I pretend to be very unhurried, then unable to overcome
visions of a spurting, bloody end, sprint like hell for the dock, all
splash and panic.

Swim practice is the one place where I feel most safe. I love our coach,
Coach Sangster, even the way he'll break the occasional clipboard when
things don't go our way in a meet, the cloud of disappointment that
passes like a sudden thunderstorm. Even at 10, I must be dimly aware
that adult disappointment might hover closer than I know, and that water
must symbolize something important, perhaps both freedom and danger. In
fact, my cousin Cindy, who had been my age, drowned in a crick. Well, it
was a creek, but that's what they called it up in the Adirondacks,
``crick.'' And in the aftermath, our extended family exhibited a new
carefulness, as if it might bring her back.

The pool is not a crick but a universe unto itself: the snack bar, the
locker room, the woods and fields, the orange Popsicles. All the while
the pool sits serenely at the center, inviting, nonjudgmental. When my
father makes his cameos here, he milks the moment by swimming two laps
underwater, moving in a thrilling blur that keeps his four sons riveted.
As the oldest, and because I'm 10 now and have made myself an expert in
submersion, it's time that I at least try.

And so I do, at the innocuous end of practice one morning near the
innocuous end of summer, when most of the team has left the pool after a
grueling workout, and Coach Sangster is picking up stray kickboards, and
someone else is beginning to roll up the lane lines a final time. I
hyperventilate quickly, supercharging my lungs, plunge underwater and
push off. There's a quick infrared of doubt, that initial awkwardness of
trying to find a rhythm to hold on to. Like the beginning of summer
itself.

But then instinct kicks in, literally, frog kicks and long pulls, the
finer feeling of propulsion through sunlight and molecules. The mind
steadies as the body calms. A balancing point asserts itself between
movement and stillness. Halfway down the first lap, your lung tickles
for air, but the warm sun throws glitter through which you swim. And
you're pretty sure you could swim like this forever, all thoughts and
rabble obliterated. Perhaps you will.

At the wall, you execute an underwater turn by touching halfway up,
balling your knees to your chest as you pivot, and then planting both
feet and exploding off again as you elongate and make yourself an
aerodynamic missile. When the momentum subsides and you begin to stroke,
your lungs suddenly scream for oxygen, on fire, irrationally so, from
the exertion. You can feel the sizzle of hypoxia running the telephone
lines of your body, the war in your head, your muscles turning heavy.

And this is where you would surface, just to gulp fresh air. To end the
searing of your alveoli. The reptilian brain's reflex, its will to live.
But it's your legs now that decide: They keep frogging, and your arms
keep pulling, overriding all other urges until that rhythm reasserts
itself. Everything turns on memory, until the last strokes come easily,
and there is no doubt you will reach the wall. Now you are just water
within water, pure being.

These many years later, that's what I remember somehow, the absolute
satisfaction, the final euphoria of that summer, in 1975, above all
others, just before touching the wall. I probably hesitated before doing
so, bittersweet knowing that I'd made it and that by surfacing, summer
was officially over. Of course, time set itself ticking again when I
did, the universe inside the pool flickered and faded. I got out and
toweled off, leaves now flaring in the trees' top reach, realizing no
one had seen the small wonder of what I'd done.

\begin{center}\rule{0.5\linewidth}{\linethickness}\end{center}

Image

\hypertarget{heidi-julavits-on-the-phosphorescence-off-the-coast-of-maine}{%
\subsection{Heidi Julavits on the phosphorescence off the coast of
Maine.}\label{heidi-julavits-on-the-phosphorescence-off-the-coast-of-maine}}

\textbf{When they} are here, my family acts like vandals. We wait until
dark. We drive to the water with stockpots and lobster pots and pails
and fill them with rocks from the shore. Then we walk to the end of the
town dock. We slam the rocks at the water and watch the light explosions
with awe. We are by turns aggressive and devotional, and must appear, to
those unfamiliar with our tradition, as if we're trying to viciously
kill something, while stricken by remorse between attacks. Because we
know nothing about ``the phosphorescence,'' it's possible that they only
light up as they die. But I don't think that's what's happening. I think
they are saying, as we are with our rocks, ``Hello.''

Every summer I return to Maine, my childhood home. For a few days or
weeks in August, the ocean, at night, when disturbed, lights up and
glows green. To the extent that I can explain this phenomenon, it
involves tiny organisms, maybe bacteria or plankton. My family and
friends could bother with the science, to demystify their magic and
better pinpoint their arrival, but why? We might learn we'd been calling
them by the wrong name all these years. (They are not phosphorescent;
they are bioluminescent.) The ocean should have its privacy, and we
prefer to be taken unawares. Then we can run around town at the first
sighting and announce: ``They're here! They're here!''

Sometimes the phosphorescence are so bright I find them shockingly
unnatural. They pull from me a gasp. Stick an oar in the water, and it's
like churning up electric glitter. Or it's like someone threw a
plugged-in appliance into the sea. Sometimes I like to kneel on the dock
edge and ease a rock into the water, then let go, watching it spark
toward the bottom as it sinks, until I can't see it anymore. One summer
night, when the sky was clear and starry, the moon absent, the air cold,
my friend and I decided to swim in the phosphorescence. We stroked out
far enough to escape the lights from the rental house. We made a quick
fuss and then kept our bodies quiet. We let the phosphorescence swarm
around us. The starry sky became the ocean became the starry sky became
the ocean. It was hard to know which substance was keeping us afloat and
which one we were meant to breathe.

When they stopped blinking, we flailed around again.

As a child, I spent summer weekends camping on a boat with my family.
The ocean was our kitchen sink and our bathtub and our washing machine
and our toilet and our highway and our accomplice and sometimes, too,
our foe. When we washed our dishes from dinner, we leaned over the stern
in the dark, and hanging upside-down, the blood rushing to our heads, we
pointlessly swished and swished. The plates stayed greasy, and we could
never get our hands chalky-clean before bed, because saltwater and soap,
like old food, is sticky. Except when the phosphorescence were out, this
chore was a punishment. But when they were there, we let loose the
trails of a hundred comets with our dirty plates. The sky never
delivered such heavens.

Sometimes my husband and my children and I imagine the phosphorescence
have arrived when they haven't. Following the rumor of a sighting, we'll
go to the harbor with our stockpots, and we'll throw rocks and reassure
one another that we saw the glow, even though we didn't. But it feels
unlucky not to see the phosphorescence, as if the universe decided we
deserved no magic that year. We will not receive a visitation. And so,
by force of will, I make myself and everyone else see them. If the four
of us can agree on a deception, we can pass it down through the
generations. When the universe forgets us, at least we are a family
connected by tiny lights.

\begin{center}\rule{0.5\linewidth}{\linethickness}\end{center}

Image

\hypertarget{concepciuxf3n-de-leuxf3n-on-her-familys-beach-days-in-the-bronx}{%
\subsection{Concepción de León on her family's beach days in the
Bronx.}\label{concepciuxf3n-de-leuxf3n-on-her-familys-beach-days-in-the-bronx}}

\textbf{For years,} every single weekend of every summer, my family
trekked 45 minutes to Orchard Beach in the Bronx. The choice was odd: We
lived in Queens, and there was an (arguably better) beach in Far
Rockaway. Still, my family always chose Orchard. On beach days, the
aunts woke up early to cook --- plantains and yucca, rice and beans ---
while my dad or an uncle went to buy Coronas, Cokes, chicken and
\emph{chicharrón}. The kids slept later and got ready slowly. We packed
into three cars with sleep still in our eyes.

Orchard Beach is part of a park, and there is a vast picnic area where
my dad usually picked a spot for us to set up under a tree's shadow.
Puerto Rican families sitting nearby played Tito Puente, and the sound
of Spanish overtook the distant waves. The adults typically stayed on
the grass, while the kids --- there were about a dozen of us ---
immediately stripped down to our bathing suits and ran to the
never-warm-enough water. The boys would dunk the girls, and we swam for
hours, returning only for lunch. By the time we got back, our skin had
caramelized around our bathing suits. We wrapped ourselves in towels and
collapsed onto the grass, drinking soda and untangling one another's
curly hair with conditioner while our parents had conversations we did
not understand.

I always found it strange that our beach days weren't like the ones on
TV, with towels laid out in the sand for sunbathing or sand-castle
building. As kids, all my cousins and I wanted was to be in the water,
and sitting on the grass put a few extra yards between us and the ocean.
Why, we wondered, do our parents make us sit so far away?

Then, my first summer back from college, I started staying behind, too.
``Aren't you going to bathe?'' one of my younger cousins would
inevitably ask, using the Spanish \emph{bañar} to ask if I was going for
a swim. I was on the brink of adulthood and suddenly did not feel
compelled to rush to the beach. I drank Coronas out of Coke cans with my
dad, aunts and uncles and learned the rhythm of their banter. We danced
\emph{bachata} and merengue (if someone had remembered a speaker), and
my aunt and I made up dance moves. I listened to them tell stories of
life back home in the Dominican countryside. I learned the names of
relatives I had never met: Cabo, Confesor, Julia, Zijinango.

After I graduated, I started taking trips to the Dominican Republic and
experienced firsthand life in their small town nestled between
mountains. My parents grew up within walking distance of a river hugged
by trees. It's where they washed their clothes and bathed. It's where
they drank Presidente beer and told stories while sitting on oversize
plantain leaves and cooked meals over firewood. Now my cousins and I do
the same whenever I visit.

I think I finally understand the reason the older generation took us to
Orchard Beach all those summer weekends: They were trying to recreate
their lives back home. My mother and father were from the mountains,
more accustomed to dirt than sand, freshwater than salt. They grew up
far from the pristine blue waters of Punta Cana. My father did not see
the ocean until he was 15. Many in his town live and die without ever
having seen it at all.

At first, returning to the island where I was born felt like landing in
the middle of my father's memories, continuing a story he cut short to
start a new one in New York. It turns out I have always been tethered to
his old life. He brought pieces of home with him and created a patchwork
life in this strange new land.

``Maybe that's it,'' my dad said when I brought up the connection
recently. Or maybe, he said, it's that the sun is a reminder of the
labor of country life: waking up early to fish, working the land with
his father, fetching breadfruit for his mother. ``Being from the
countryside, we're tired of taking in the sun,'' he said. They seek out
the shade, because now they can.

I can count on one hand the number of times I've been to the beach with
friends, but when I have, we've done it the American way. We bring
sandwiches, and we lie in the sun for hours to tan. Once, when we were
preteens at Orchard Beach, a cousin found a patch of grass and said she
was tanning. My family still retells that story more than 15 years
later.

We've found other beaches to visit, but always those with large parks
attached, where we sit far enough from the water to forget we're at the
beach. I still get in the water, eventually. But I'm reluctant to leave
the shade.

\emph{{[}}\href{https://www.nytimes3xbfgragh.onion/2019/05/31/travel/25-writers-on-their-favorite-beach-vacations.html}{\emph{Read
25 writers on their favorite beach vacations.}}\emph{{]}}

\begin{center}\rule{0.5\linewidth}{\linethickness}\end{center}

Image

\hypertarget{karen-russell-on-the-wild-freedom-of-biking-around-miami-at-13}{%
\subsection{Karen Russell on the wild freedom of biking around Miami at
13.}\label{karen-russell-on-the-wild-freedom-of-biking-around-miami-at-13}}

In Miami, the sky is also a body of water. Lakes of rain fell on my best
friends and me during the summer of 1995. That July and August we were
always wet, pedaling through spectacular afternoon thundershowers along
Biscayne Bay, the 35-mile inlet of the Atlantic that indents the
southeast coastline of Florida. We rode our bicycles everywhere. For a
trio of 13-year-old girls, this was a wild and unusual autonomy. We'd
make landline plans to meet, then set out from our respective homes,
promising our parents we'd stay on the bike path. The speed at which our
promise became a lie: mere seconds, as we went bumping off the curb
toward U.S. 1. Nobody knew where we were for hours at a stretch --- our
location was a secret that traveled with us.

In my memory, the thrill of disappearing in tandem is inseparable from
the sudden flashes of saltwater behind the mangroves, the heat pulling
at our spines. We rode through neighborhoods we'd known since birth but
had never seen alone, unchaperoned and unsurveilled. What watched us,
instead, was the ocean. A great, lidless eye, inhumanly blue, following
us along Main Highway through the Grove toward Miami Beach; waiting for
us if we chose to pedal in the opposite direction, down Old Cutler Road
to Matheson Hammock, a park with a man-made coral atoll pool bordering
the real bay. In sunny weather, you could see the chrome stalagmites of
downtown Miami; on a gray day, you could believe the land had been
erased entirely.

The ocean haunted us even when we could not see it: infiltrating our
senses as a warm tarry pungency at low tide, or a ticklish, ionic charge
gathering under the banyan leaves before a 3 p.m. storm. Private estates
barricaded their waterfront views, yet we could turn down any cul-de-sac
and find Biscayne Bay waiting for us.

We hugged its rocky coastline, waiting for it to soften into sand.
Pumping our legs, gasping for air, scaling bridges. We wore our bathing
suits under our clothes so that we could strip and swim. One afternoon,
we ate soggy KFC biscuits in the rain on a wealthy, empty private beach
just off the Rickenbacker Causeway, skipping the tollbooths and pushing
our bikes through the sea grapes, our leg muscles still spasming from
climbing the steep bridge that connects mainland Miami to the island of
Key Biscayne. This was our longest ride; we had traveled, at most, 11
miles from our homes. I'm not sure why buying fried chicken with our
parents' money and eating it on the damp sand felt like the epitome of
freedom, but it did.

Manatees in Dinner Key Marina. ``Beware of Alligators'' signs at
Matheson Hammock. The stink of low tide, the tiny crabs waving their
pitchfork arms at us from the exposed rocks like a mob of irate
villagers. Somewhere, deep in the back of our brains, we could hear our
parents yelling: ``Slow down! Slow down!'' Instead we let our arms
extend into a fixed-wing soar over the handlebars, echoing the hundred
gulls wheeling above us. Sometimes our parents' flickery omniscience
detected a breach in the system, and they'd notice, for example, that
we'd returned home penniless with soaking hair. But ``riding bikes''
sounds so innocent, and we still had the round eyes of children. We had
the bodies of women, which meant that men had begun to holler at us from
passing cars, words that drew butcher-shop lines around us and made us
consider ourselves as an assemblage of parts: breasts, asses, thighs,
faces. During rush-hour traffic, we had to pedal through this uglier
sort of thundershower, our faces burning. At certain intersections we
knew to sit and hunch over the handlebars, our eyes on the pavement.

But once we made it over the bridge, the huge, blue solvent of the bay
erased whatever hideous self-consciousness we'd felt while riding along
the highway. To get to the beach, we had to stand to pedal, and then the
fire left our faces and came from inside us, from our lungs and calves
--- I discovered how strong my body was on those rides, pushing uphill.
We went shrieking downhill toward the wavy tarpaulin of the bay. At
last, we could relax into the sea, with its beautiful elasticity, its
deep and generous amnesia. Like us, Biscayne Bay could forget a violent
storm in an instant. We swam through smooth water, hidden up to our
necks, buoyed inside the happy silence that follows great physical
exertion.

Under that moody, aqueous sky, my two best friends and I turned 14. I
didn't know then that I was coasting through the best summer of my life.
In my memory, that summer is a suspension bridge over the water,
connecting the worlds of childhood and adulthood. Fall came, and we
started high school, a violent eviction from the freedom of those
afternoons.

There are plenty of places that you can get to by bicycle, even in a
city as vast as Miami. The gas-station convenience store that sells to
minors. Tattoo parlors with a financial stake in believing that you are
18. The houses of male strangers willing to extend to you this same line
of credit. The Planned Parenthood. When I think about the dark straits
that young women have to travel, I remember racing the waves on either
side of a winnowing road. The Biscayne Bay I've written about here is
not a place to which I can return; in the past decade, some 25,000 acres
of its sea-grass meadows --- more than 90 percent in one part of the bay
--- have died, and its famous aquamarine color belies the devastation of
raw sewage, chemical runoff, global warming and acidification, toxic
blooms of algae. We three were not amnesiacs, as it turns out, and
neither is the ocean; the damage we sustain lives on inside us. So does
this memory: the bridge to a blue expanse of dreaming time that girls
deserve, and not only for a summer.

\begin{center}\rule{0.5\linewidth}{\linethickness}\end{center}

Image

\hypertarget{rowan-ricardo-phillips-on-spains-costa-brava}{%
\subsection{Rowan Ricardo Phillips on Spain's Costa
Brava.}\label{rowan-ricardo-phillips-on-spains-costa-brava}}

Before you and I get too far into this story about one summer day in a
tiny coastal town called Tamariu, you should know that Tamariu is about
a 90-minute drive north of Barcelona in the northeast of Catalonia,
which itself is in the northeast of Spain, and just south of the French
border.

Before I ever step foot in Tamariu my wife, who was born and raised in
Barcelona, spends a good chunk of every summer of her childhood there.
As soon as she's old enough to choose to go or not to go, she doesn't
return for years. She's looking for change and change is hard to come by
in Tamariu which is neither good nor bad, it just is.

Before you think about coming to Tamariu know that there are two other
coastal towns nearby --- Lllafranc and Calella --- also under the
administrative aegis of Palafrugell and those two are larger, busier,
more finished if you will; Tamariu is good for a swim, a page-turner,
and a walk along the esplanade; it never seems more itself than when
it's empty and quiet.

Before my wife has a change of heart years later and we walk down la
Riera ---~Tamariu's pipe-cleaner of a main street that floods when
tempests come ---~and arrive at the small, semicircular beach at the
base of town that glistens like the shiny half-moon at the base of your
thumbnail we decide to skip the beach.

Before the crowds start to clot on the beach and bake in the sun, before
they start to splash in the seawater, the inescapable summer sounds and
colorful phalanxes of parasols looking like polka dots in the heat's
hazy distance as we tiptoe around them all, crowd-averse as we are, cut
across the beach and head toward the rocky hills that curve back toward
the bay, away from the beach, and out of sight.

Before we start to climb an improvised footpath worn into the rocky
seaside embankment toward the isolated bays behind Tamariu with their
vertiginous views of themselves, the sea, and little else.

Before we finish climbing the cliff, we see someone else already on the
cliff: an old man fidgeting with something on the high rocks and decked
out in a blue wet suit.

Before we begin to pretend again that summer is a rational thing and you
corner me on a cliff and say I have to choose what came first and what
came next, find the brightest spot in this song of the summer and stay
with it.

Before one summer blurs into several summers in the same place at once
separated only by the perforations that age make on your memory.

Before I know what this is about, I know \emph{when} this is about: that
moment in summer when summer seems like it will never end and becomes an
eternal present, or a singular moment made of many moments transformed
into a constant feeling with no beginning or end.

Before I think to tell you that this isn't a story ---~not an arc with a
beginning and an end ---~I remember that summer is never really a story,
no matter how many summer stories we think we tell; summer is a texture.

Before we go further, try to remember your favorite texture, portrait,
tableau: close your eyes and see it now in front of you, remember how
its many scenes and multiple moments happen together at once, the
foreground and background unfolding rhythmically in their own time like
when you stare into the sun and slowly another shapes appear.

Before the texture of last summer fades, I remember who I wanted to tell
you about.

Before we reach the peak of our favorite cliff he's already there,
anxious to enter the sea. At his side are a boy and a girl who watch the
old man wind his way down the cliffside and slowly ease himself into the
water with an absent look on his face and a knife in his hand that makes
me remember my own grandfather and that the few times I saw him back in
Antigua were almost exclusively at sea dazzled by a type of dreaming I
couldn't fathom.

Before we go too far into this summer portrait or not far enough.

Before his grandchildren, left to their own devices, leap fifty feet
down into the shining water and climb back up, and leap back down into
the shimmering water twice so happy that when they squeal nothing comes
out of their mouths, I wonder if I'll tell them to be careful or if in
doing so, in just introducing the idea of being careful, I'll fear
becoming the monster who ruined their summer.

Before I consider leaping fifty feet down into the shining water,
climbing back up, and leaping down into the shimmering water again,
having been left to my own devices by my wife due to my own laziness.

Before I contemplate taking a nap, I look down to see my wife in the
water with two of our friends, wading far away in the depths of the
Mediterranean, chatting in Catalan, occasionally slipping on a pair of
goggles they're sharing between them to marvel at the life passing
beneath their feet; its beauty, and that some of it is still even there.

Before the grandfather's two grandchildren lose sight of him, and I lose
sight of him, and they weave around me on those high rocks in the way
that children do when they're distressed and try their best not to seem
distressed in front of a stranger as they call out for him and hear back
only the laughter of the lapping water.

Before I decide that, although we're strangers, we shouldn't be
strangers now and I ask the children if they know where their
grandfather is, and they say no, no they don't.

Before I turn the cliff into a lookout tower and scan as far as I can
see for their grandfather and look down watch them my wife and friends
swim from one end of the bay to another, searching for him, finding no
one, and then ---~after they look up to the cliff to make sure the
children aren't looking at them ---~deep dive down into the sea to make
sure his body hasn't been dragged down by something.

Before my wife and two friends swim from one of Tamariu's bays to
another, turning out of sight, all the way back to the main beach we'd
hiked so far away from and find the grandfather there, scouring the
seabed for urchins, lost in the thought of the richness of their flesh.

For him there was nothing before that: not the calls of his desperate
grandchildren; not dangers of the sea; and certainly not this thing
they'll say in the future was a simple figment of our imagination, a
splinter of sun in the mind, a temporary little flinch in the timeline
that we once called summer.

Advertisement

\protect\hyperlink{after-bottom}{Continue reading the main story}

\hypertarget{site-index}{%
\subsection{Site Index}\label{site-index}}

\hypertarget{site-information-navigation}{%
\subsection{Site Information
Navigation}\label{site-information-navigation}}

\begin{itemize}
\tightlist
\item
  \href{https://help.nytimes3xbfgragh.onion/hc/en-us/articles/115014792127-Copyright-notice}{©~2020~The
  New York Times Company}
\end{itemize}

\begin{itemize}
\tightlist
\item
  \href{https://www.nytco.com/}{NYTCo}
\item
  \href{https://help.nytimes3xbfgragh.onion/hc/en-us/articles/115015385887-Contact-Us}{Contact
  Us}
\item
  \href{https://www.nytco.com/careers/}{Work with us}
\item
  \href{https://nytmediakit.com/}{Advertise}
\item
  \href{http://www.tbrandstudio.com/}{T Brand Studio}
\item
  \href{https://www.nytimes3xbfgragh.onion/privacy/cookie-policy\#how-do-i-manage-trackers}{Your
  Ad Choices}
\item
  \href{https://www.nytimes3xbfgragh.onion/privacy}{Privacy}
\item
  \href{https://help.nytimes3xbfgragh.onion/hc/en-us/articles/115014893428-Terms-of-service}{Terms
  of Service}
\item
  \href{https://help.nytimes3xbfgragh.onion/hc/en-us/articles/115014893968-Terms-of-sale}{Terms
  of Sale}
\item
  \href{https://spiderbites.nytimes3xbfgragh.onion}{Site Map}
\item
  \href{https://help.nytimes3xbfgragh.onion/hc/en-us}{Help}
\item
  \href{https://www.nytimes3xbfgragh.onion/subscription?campaignId=37WXW}{Subscriptions}
\end{itemize}
