Sections

SEARCH

\protect\hyperlink{site-content}{Skip to
content}\protect\hyperlink{site-index}{Skip to site index}

\href{https://www.nytimes3xbfgragh.onion/section/politics}{Politics}

\href{https://myaccount.nytimes3xbfgragh.onion/auth/login?response_type=cookie\&client_id=vi}{}

\href{https://www.nytimes3xbfgragh.onion/section/todayspaper}{Today's
Paper}

\href{/section/politics}{Politics}\textbar{}Biden's First Run for
President Was a Calamity. Some Missteps Still Resonate.

\url{https://nyti.ms/2JTOFRG}

\begin{itemize}
\item
\item
\item
\item
\item
\item
\end{itemize}

\begin{itemize}
\item
  \href{https://www.nytimes3xbfgragh.onion/live/2020/09/11/us/trump-vs-biden?action=click\&pgtype=Article\&state=default\&region=TOP_BANNER\&context=storylines_menu}{Election
  Updates}
\item
  \href{https://www.nytimes3xbfgragh.onion/interactive/2020/us/elections/election-states-biden-trump.html?action=click\&pgtype=Article\&state=default\&region=TOP_BANNER\&context=storylines_menu}{Paths
  to 270}
\item
  \href{https://www.nytimes3xbfgragh.onion/interactive/2019/us/elections/2020-presidential-election-calendar.html?action=click\&pgtype=Article\&state=default\&region=TOP_BANNER\&context=storylines_menu}{Key
  Dates}
\item
  \href{https://www.nytimes3xbfgragh.onion/interactive/2020/08/31/us/politics/vote-by-mail-deadlines.html?action=click\&pgtype=Article\&state=default\&region=TOP_BANNER\&context=storylines_menu}{Voting
  by Mail}
\item
  \href{https://www.nytimes3xbfgragh.onion/newsletters/politics?action=click\&pgtype=Article\&state=default\&region=TOP_BANNER\&context=storylines_menu}{Politics
  Newsletter}
\end{itemize}

Advertisement

\protect\hyperlink{after-top}{Continue reading the main story}

Supported by

\protect\hyperlink{after-sponsor}{Continue reading the main story}

The long run

\hypertarget{bidens-first-run-for-president-was-a-calamity-some-missteps-still-resonate}{%
\section{Biden's First Run for President Was a Calamity. Some Missteps
Still
Resonate.}\label{bidens-first-run-for-president-was-a-calamity-some-missteps-still-resonate}}

In 1988, Joe Biden was prone to embellishment. Hints of that linger
today. But unlike then, his message to voters is clear: He's a
stabilizing statesman in a tumultuous time.

\includegraphics{https://static01.graylady3jvrrxbe.onion/images/2019/05/30/us/politics/biden88-video/biden88-video-videoSixteenByNine3000.jpg}

\href{https://www.nytimes3xbfgragh.onion/by/matt-flegenheimer}{\includegraphics{https://static01.graylady3jvrrxbe.onion/images/2018/10/02/multimedia/author-matt-flegenheimer/author-matt-flegenheimer-thumbLarge.png}}

By \href{https://www.nytimes3xbfgragh.onion/by/matt-flegenheimer}{Matt
Flegenheimer}

\begin{itemize}
\item
  June 3, 2019
\item
  \begin{itemize}
  \item
  \item
  \item
  \item
  \item
  \item
  \end{itemize}
\end{itemize}

Joe Biden was riffing again --- an R.F.K. anecdote, a word about ``civil
wrongs,'' a meandering joke about the baseball commissioner --- and
aides knew enough to worry a little.

``When I marched in the civil rights movement, I did not march with a
12-point program,'' Mr. Biden thundered, testing his presidential
message in February 1987 before a New Hampshire audience. ``I marched
with tens of thousands of others to change attitudes. And we changed
attitudes.''

More than once, advisers had gently reminded Mr. Biden of the problem
with this formulation: He had not actually marched during the civil
rights movement. And more than once, Mr. Biden assured them he
understood --- and kept telling the story anyway.

By that September, his recklessness as a candidate had caught up with
him. He was accused of plagiarizing in campaign speeches. He had
inflated his academic record. Reporters began calling out his
exaggerated youth activism.

``I've done some dumb things,'' Mr. Biden conceded at a
stop-the-bleeding news conference at the Capitol. ``And I'll do dumb
things again.''

He vowed that day to fight on. He quit the race within a week.

Thirty-two years later, as Mr. Biden seeks the presidency for a third
time, his disastrous campaign for the 1988 Democratic nomination offers
a revealing look at the personal tics and political flaws of the
front-runner in the 2020 race --- traits that, in many ways, continue to
color Mr. Biden's public life.

\includegraphics{https://static01.graylady3jvrrxbe.onion/images/2019/06/03/us/politics/03biden88-2/merlin_155262297_87fd027d-8c37-49bf-b3f5-66f11b347679-articleLarge.jpg?quality=75\&auto=webp\&disable=upscale}

Mr. Biden was, and remains, a ``gut politician,'' as he has long told
associates --- swaggering, ad-libbing, liable to get carried away in
front of a crowd. Already this year, he has
\href{https://www.cbsnews.com/news/biden-raises-more-than-100000-at-south-carolina-fundraiser/}{boasted
of his purportedly peerless foreign policy knowledge}, comparing himself
favorably to Henry A. Kissinger, the former secretary of state. He has
suggested, implausibly, that he has
\href{https://abcnews.go.com/Politics/video/biden-progressive-records-running-62735484}{the
most progressive record in the 2020 field}. He has muddled through
explanations of his treatment of Anita Hill when she accused Justice
Clarence Thomas of sexual harassment, at times
\href{https://www.nytimes3xbfgragh.onion/2019/04/26/us/politics/biden-the-view.html}{stopping
himself midsentence} to abandon a line of defense.

{[}\href{https://www.nytimes3xbfgragh.onion/newsletters/politics?smid=rd?action=click\&module=Intentional\&pgtype=Article}{Sign
up for our politics newsletter and join the conversation around the 2020
presidential race.}{]}

Biden allies insist this run will succeed where his others failed. His
discipline has improved, they say. He is now widely known and admired in
the Democratic Party, affording him more latitude for slip-ups. For the
first time, he enters the race as a genuine favorite, requiring no
introduction.

But interviews with top advisers and confidants from then and now help
explain how Mr. Biden came to see himself as presidential material in
the first place, and suggest that the central tensions and
vulnerabilities laid bare during Biden '88 remain the most urgent
questions at the core of Biden 2020:

Can he credibly present himself as a man in step with the times without
sounding off-key or stretching the truth, as he did while gilding his
1960s-era biography?

Can he win while mounting another campaign premised as much on personal
characteristics --- his decency, his integrity, his presumed
electability --- as any particular policy platform?

In both the 1988 race and today, Mr. Biden has seemed to see the nation
at a turning point, in need of a particular kind of leader.

During his first run, he liked to say that presidential history ran in
cycles: bursts of progress and upheaval, followed by periods of
correction in which voters choose a candidate who can ``let America
catch its breath.''

His implication then, as a 44-year-old senator from Delaware, was that
he belonged to the first group of political figures: the sprightly
agitators. His pitch this time, as a septuagenarian two-term vice
president, places him firmly in the second camp: He is now the
stabilizing statesman, in his telling, poised to deliver the nation from
the Trumpian tumult.

``It's kind of funny in retrospect,'' said Mike Lux, a top Biden aide in
Iowa in 1988. ``A lot of the message was based on sort of `time for
generational change.' Now, he is sort of the opposite of the changing of
the guard.''

These days, Mr. Biden, whose campaign declined to make him available for
an interview, keeps an understated schedule, holding far fewer events
than most rivals. But in his first race, his candidacy could feel like
an exercise in performative stamina --- sustained by an uncommon talent
for talk-until-they-leave speechifying and an oversize bottle of Tylenol
that helped ease foreboding headaches on the road.

Storming across Iowa in a maroon and gray campaign van, Mr. Biden asked
his team to blast the ``Les Misérables'' cassette (\emph{``One day
morrrrrrre / Another day, another destiny''}) because it helped him
think. At events, he would smile almost mockingly at staff members
signaling for him to wrap up, long after they had handed reporters
prewritten text with a semi-wry warning in capital letters atop the
page: ``Senator may stray from prepared remarks.''

The downsides of this high-intensity approach became clear in time.
Oversubscribed and running himself ragged --- presidential contender,
Senate committee chairman, father of teenagers --- Mr. Biden began
making the mistakes that would shape his enduring reputation for
carelessness in speech: loose talk, citation-free borrowing, outright
misstatements. (Twenty years later, he reinforced the trope
\href{https://www.nytimes3xbfgragh.onion/2007/02/01/us/politics/01biden.html}{almost
immediately} upon entering the 2008 presidential race, giving an
interview in which he inelegantly called Barack Obama ``the first
mainstream African-American who is articulate and bright and clean and a
nice-looking guy.'')

Other old habits have likewise persisted. Long before he hesitated on
campaign decisions before the 2016 and 2020 elections, Mr. Biden spent
much of his 1988 run doubting its very wisdom: Was the team ready? Was
he?

At one point before his official announcement, he asked his son Hunter
what to do.

``If you don't do it now, I couldn't see you doing it some other time,''
Hunter reasoned, according to a Time magazine article in 1987.

``Yeah,'' Mr. Biden agreed. ``That's the thing.''

Image

Mr. Biden was, and remains, ``a gut politician,'' as he has long told
associates --- swaggering, ad-libbing, liable to get carried away in
front of a crowd.Credit...Keith Meyers/The New York Times

\hypertarget{i-decided-i-could-beat-them}{%
\subsection{`I decided I could beat
them'}\label{i-decided-i-could-beat-them}}

For a while, Mr. Biden's mouth was his best asset.

He had flirted with a run in 1984, establishing himself as an estimable
orator with a speech in Atlantic City invoking the deaths of President
John F. Kennedy and the Rev. Dr. Martin Luther King Jr. ``Just because
our political heroes were murdered,'' he said, months before the 1984
primaries, ``does not mean that the dream does not still live, buried
deep in our broken hearts.''

The flourish seemed to signal the kind of candidate Mr. Biden hoped to
be four years later, when he decided to explore a run more seriously:
aspirational, of the times --- but also a bit of a throwback, a bridge
between political moments.

While many Democrats weary of Reagan-era Republicanism envisioned their
party racing to the left, Mr. Biden chafed as some advisers nudged him
toward a kind of zealous populism, an imperfect fit for the candidate's
within-the-system sensibility. ``I wore \emph{sport coats},'' Mr. Biden
told reporters once, explaining his limited involvement in antiwar
fervor. ``I was not part of that.''

Presaging his current bet on political centrism, Mr. Biden appeared wary
of catering to the party's base. Asked in February 1987 if Democrats'
success in the 1986 midterms showed that the country was becoming more
liberal, Mr. Biden gave an unqualified no. ``I think my party would make
a big mistake if they read it that way,'' he said.

Less clear to Mr. Biden was precisely what he did want to say. In his
2007 book, ``Promises to Keep,'' Mr. Biden wrote that his early message
felt ``a bit opaque, like audiences were hearing me through a veil.''

He struggled to verbalize a campaign rationale that felt true to him, in
``words that felt absolutely authentic.'' He went ahead anyway.

``I started looking at the race through the wrong prism,'' he wrote. ``I
looked around, judged myself against the other potential candidates for
the nomination, and by the beginning of 1987 I decided I could beat
them.''

For a time, he had a case. On raw ability, few in the underwhelming
field --- known derisively as ``the seven dwarfs'' in the political
press --- could hope to match Mr. Biden, who quickly outpaced many of
them in crowd size and early fund-raising.

Image

The Democratic presidential candidates in 1987 were, from left, Senator
Al Gore, Representative Richard Gephardt, Gov. Michael Dukakis, Mr.
Biden, Rev. Jesse Jackson, Gov. Bruce Babbitt and Senator Paul
Simon.Credit...Steve Kagan/The LIFE Images Collection, via Getty Images

Rival camps took notice of Mr. Biden's progress, viewing him as a
growing force in a primary race that included Gov. Michael Dukakis,
Representative Richard A. Gephardt and the Rev. Jesse Jackson. (Gary
Hart, a former senator and the early front-runner,
\href{https://www.nytimes3xbfgragh.onion/1987/05/09/us/hart-drops-race-for-white-house-in-a-defiant-mood.html}{dropped
out} in the spring amid allegations of an extramarital affair.)

``He was going to be a problem,'' Joe Trippi, a strategist who worked
for Mr. Hart and later Mr. Gephardt, said of Mr. Biden. ``He always had
that ability to connect.''

Ever the tactile politician, Mr. Biden seemed to relish finding the
holdouts in an audience and trying to flip them. At one event in Iowa,
Mr. Biden came upon a woman who refused to turn to face him. He
approached from behind and continued the speech with his hands on her
shoulders. ``The woman looked like she'd swallowed her tongue,'' the
journalist Richard Ben Cramer wrote in ``What It Takes,'' his
magisterial book about the 1988 campaign.

Mr. Biden also engendered fierce loyalty on his team, growing close to
several aides who remain part of his extended political family. Staff
nicknames were assigned as a matter of course, ``almost like `Animal
House,''' said John Anzalone (``Zo'' to the candidate), a pollster who
worked on the 1988 campaign and has advised the current one.

But for all the enthusiasm on the ground, there were early signs of
indiscipline, starting at the top.

In February of 1987, Mr. Biden borrowed from a Robert Kennedy speech
without attribution, later saying the remarks had been written for him;
the incident initially went unreported. In April --- as one of his
headaches flared, Mr. Biden would later recall --- a camera crew caught
him berating a New Hampshire man who asked about his academic history.

``I probably have a much higher I.Q. than you do, I suspect,'' Mr. Biden
shot back, before exaggerating his record in school. The clip did not
initially find wide circulation.

By June, on the eve of his planned kickoff, Mr. Biden was still unsure
if he wanted to run at all.

``He said, `I just don't feel right about this,''' recalled Ted Kaufman,
a longtime adviser and close friend who briefly succeeded Mr. Biden in
the Senate when he became vice president.

It was left to Mr. Biden's wife, Jill, to talk him down: Too many
people, she said, had already committed too much to the cause.

The campaign summoned reporters to Wilmington, Del., for a special
announcement.

Image

Mr. Biden, then chairman of the Senate Judiciary Committee, oversaw the
hearings when Judge Robert H. Bork was a nominee for the Supreme Court
in September 1987.Credit...Jose R. Lopez/The New York Times

\hypertarget{hints-of-a-troubling-pattern}{%
\subsection{Hints of a troubling
pattern}\label{hints-of-a-troubling-pattern}}

Much of Mr. Biden's reluctance owed to concerns about balancing his
duties in Washington. He was right to be worried.

Within weeks of Mr. Biden's official entry, President Ronald Reagan
\href{https://www.nytimes3xbfgragh.onion/1987/07/02/us/bork-picked-for-high-court-reagan-cites-his-restraint-confirmation-fight-looms.html}{nominated}
Judge Robert H. Bork, an unflinching conservative, to the Supreme Court.
Mr. Biden, leading the Senate committee overseeing the confirmation
process, was gifted a high-profile showcase, with the court's balance at
stake.

Quickly, though, the drawbacks of this ostensible political blessing
became clear: His Senate work was dividing his attention and limiting
his preparation for campaign events.

The first major primary debate came just after Mr. Bork's nomination,
and Mr. Biden has said he was plainly distracted. Reviews were unkind.
``Unless we ditch television for the remainder of the campaign,''
\href{https://www.washingtonpost.com/archive/lifestyle/1987/07/03/the-diverting-democrats/7d31e08c-fcd3-4cd7-8006-f0ab5b135ea5/?utm_term=.ff508010ed4e}{wrote
Tom Shales} of The Washington Post, ``Biden will never be president.''

A debate in Iowa the next month would prove far more damaging. On the
drive to the venue, Mr. Biden found himself without a closing statement.
An aide, David Wilhelm, suggested he repurpose something that had been
working well on the trail: a refrain --- credited to Neil Kinnock, the
British Labour Party leader --- about giving citizens ``a platform upon
which to stand,'' with anecdotes from Mr. Kinnock's family of Welsh coal
miners.

Not long before, Mr. Biden had received a videotape of Mr. Kinnock's
remarks from Bill Schneider, a writer and political commentator. The
candidate instantly admired their emotional power. At campaign events,
Mr. Biden had repeatedly attributed Mr. Kinnock's words properly. At the
debate, compressing his closing argument into the allotted time, he did
not.

``It was an unremarkable moment to me,'' Mr. Wilhelm said. ``Shows how
much I know.''

Rivals knew better. With a nudge from the Dukakis campaign, several news
outlets, including
\href{https://www.nytimes3xbfgragh.onion/1987/09/12/us/biden-s-debate-finale-an-echo-from-abroad.html}{The
New York Times}, The Des Moines Register and NBC News, reported weeks
later on Mr. Biden's lifted passages.

He has suggested his whole life might have changed if he had said two
words on the debate stage: ``Like Kinnock \ldots{}''

But the criticism ran deeper than finger-wagging about proper citation.
The burgeoning scandal seemed to confirm a persistent critique of Mr.
Biden: that he lacked his own vision, his own story to tell.

Image

Mr. Biden with his son Beau at the Iowa State Fair in
1987.Credit...Shelly Katz/The LIFE Images Collection, via Getty Images

``He talked as if the details of Kinnock's background were details of
his own,'' Mr. Schneider said. At the debate, Mr. Biden had implied that
he descended from coal miners, like Mr. Kinnock did. (Years later, as
Mr. Biden sought the 2008 Democratic presidential nomination, Mr.
Schneider recalled encountering him shortly after a primary debate. Mr.
Biden had not forgotten Mr. Schneider's role in the fateful Kinnock
affair. ``Got any hot videotapes?'' the candidate joked.)

Perhaps most concerning for his 1988 campaign, news accounts began to
hint at a troubling pattern in Mr. Biden's behavior. The San Jose
Mercury News flagged his failure to cite Robert Kennedy. Newsweek wrote
about the video of Mr. Biden insulting a voter while misstating his
academic record. And that record, according to a rash of new reports,
included an episode of plagiarism in law school.

``It was a tsunami,'' Valerie Biden Owens, his sister and longtime top
campaign adviser, said in a recent interview.

Aides believed that reporters --- perhaps emboldened by the sensational
fall of Mr. Hart --- were overhyping Mr. Biden's misdeeds. Top advisers
insist to this day that their candidate's sins were minor, and that he
was wronged.

On Sept. 17, Mr. Biden
\href{https://www.nytimes3xbfgragh.onion/1987/09/18/us/biden-admits-plagiarism-in-school-but-says-it-was-not-malevolent.html}{called
a news conference} at the Capitol to explain himself. By turns contrite
and simmering, he said the law school citation issue was not
``malevolent.'' He took responsibility for not knowing that he was using
Mr. Kennedy's words in the February speech. He acknowledged that he had
forgotten to cite Mr. Kinnock at the debate. And he seemed to seethe at
reporters' questions about the extent of his activism on civil rights
and Vietnam.

``I find y'all going back and saying, `Well, where were you, Senator
Biden, at the time?' --- you know, I think it's bizarre,'' Mr. Biden
said, adding: ``Other people marched. I ran for office.''

He said he had no plans to drop out.

``I am in this race to stay, I am in this race to win,'' he said. ``And
here I come.''

Image

Mr. Biden announced the end of his presidential campaign at a news
conference in September 1987.Credit...Jose R. Lopez/The New York Times

\hypertarget{an-excruciating-decision}{%
\subsection{An excruciating decision}\label{an-excruciating-decision}}

And there he went.

Initially, Mr. Biden convinced himself there was a way to do it all:
defend his name, win the Bork fight, charge into primary season.

``This would have passed,'' said Mr. Kaufman, Mr. Biden's longtime
adviser. ``Most of it had no substance to it.''

Others were less certain. Congressional supporters asked Mr. Biden if
they should expect more shoes to drop. ``I just, honest to God, don't
know,'' he said, according to Mr. Cramer's ``What It Takes.''

As the campaign team convened in Mr. Biden's Wilmington living room,
most saw no choice but to quit. Mr. Biden's eldest son, Beau, then a
college freshman, argued the other side, saying that exiting the race
would validate the allegations.

Mark Gitenstein, a top aide on the Judiciary Committee, feared that Mr.
Biden's flagging reputation might imperil the Senate vote if he did not
end his candidacy: ``If we win Bork, it will be in spite of us,'' he
told Mr. Biden, according to Mr. Biden's book. ``If we lose now, it's
going to be because of us.''

Mr. Biden had his advisers draft a statement. Reporters were told to
gather at the Capitol for an announcement, six days after the last one.

``With incredible reluctance,'' Mr. Biden told them, he was
\href{https://www.nytimes3xbfgragh.onion/1987/09/24/us/biden-withdraws-bid-for-president-in-wake-of-furor.html}{ending
his campaign}.

``It's time for me to assess my mistakes and make sure I don't make them
again,'' he said, while also lamenting ``the environment of presidential
politics'' that had allowed his biography to be reduced to a few
missteps.

He collected himself. ``Lest I say something that might be somewhat
sarcastic,'' Mr. Biden said, ``I should go to the Bork hearing.''

Within weeks, he would
\href{https://www.nytimes3xbfgragh.onion/1987/10/24/politics/borks-nomination-is-rejected-5842-reagan-saddened.html}{help
defeat the Bork nomination}, paving the way for a more moderate choice,
Justice Anthony Kennedy, to join the court.

Within months, he was hitting the speaking circuit again, cautiously
going about repairing his image, even as he smarted over how the
campaign had collapsed.

That is how Mr. Biden found himself in Rochester in February 1988,
settling into his hotel room after a speech. Only snippets remain
memorable to him now: mulling a late-night pizza delivery, waking up on
the floor some time later, straining to understand the impossible
pounding in his head --- ``something like lightning flashing,'' Mr.
Biden wrote in his book.

An aneurysm.

Mr. Biden made it back to Wilmington, weak and gray, before it was
diagnosed. He was hustled to a hospital, where family members gathered.
A priest was called in to administer last rites.

That would prove premature. And after surgery, the family developed a
grand theory of it all --- the campaign, the scare, the cosmic plan for
Joe Biden.

In his book, Mr. Biden recalled a conversation with his wife as he was
recovering. The emergency had come in the middle of the primary
calendar, she reminded him. He would have been going full blast across
New Hampshire, campaigning like his life depended on it.

``Would I have stopped long enough for treatment?'' Mr. Biden wrote.
``Would I have tried to push through the pain?''

Both of them seemed to know the answer: One loss had prevented a bigger
one.

\emph{Kitty Bennett contributed research.}

\hypertarget{our-2020-election-guide}{%
\section{Our 2020 Election Guide}\label{our-2020-election-guide}}

Updated ~Sept. 11, 2020

\begin{center}\rule{0.5\linewidth}{\linethickness}\end{center}

\begin{itemize}
\item ~
  \hypertarget{the-latest}{%
  \subsection{The Latest}\label{the-latest}}

  \begin{itemize}
  \item
    Joe Biden and President Trump put
    \href{https://www.nytimes3xbfgragh.onion/2020/09/11/us/politics/shanksville-trump-biden.html?action=click\&pgtype=Article\&state=default\&region=BELOW_MAIN_CONTENT\&context=storylines_guide}{hostilities
    on hold today to travel to ground zero and then to Shanksville, Pa.,
    where they separately honored 9/11 victims}.
  \end{itemize}
\item ~
  \hypertarget{how-to-win-270}{%
  \subsection{How to Win 270}\label{how-to-win-270}}

  \begin{itemize}
  \item
    Joe Biden and Donald Trump need 270 electoral votes to reach the
    White House. Try building
    \href{https://www.nytimes3xbfgragh.onion/interactive/2020/us/elections/election-states-biden-trump.html?action=click\&pgtype=Article\&state=default\&region=BELOW_MAIN_CONTENT\&context=storylines_guide}{your
    own coalition of battleground states}~to see potential outcomes.
  \end{itemize}
\item ~
  \hypertarget{voting-by-mail}{%
  \subsection{Voting by Mail}\label{voting-by-mail}}

  \begin{itemize}
  \item
    Will you have enough time to vote by mail in your state? Yes, but
    it's risky to procrastinate.
    \href{https://www.nytimes3xbfgragh.onion/interactive/2020/08/31/us/politics/vote-by-mail-deadlines.html?action=click\&pgtype=Article\&state=default\&region=BELOW_MAIN_CONTENT\&context=storylines_guide}{Check
    your state's deadline.}
  \item
    \href{https://www.nytimes3xbfgragh.onion/interactive/2020/us/elections/joe-biden.html?action=click\&pgtype=Article\&state=default\&region=BELOW_MAIN_CONTENT\&context=storylines_guide}{}

    \hypertarget{joe-biden}{%
    \section{Joe Biden}\label{joe-biden}}

    \hypertarget{democrat}{%
    \subsection{Democrat}\label{democrat}}

    \href{https://www.nytimes3xbfgragh.onion/interactive/2020/us/elections/donald-trump.html?action=click\&pgtype=Article\&state=default\&region=BELOW_MAIN_CONTENT\&context=storylines_guide}{}

    \hypertarget{donald-trump}{%
    \section{Donald Trump}\label{donald-trump}}

    \hypertarget{republican}{%
    \subsection{Republican}\label{republican}}
  \end{itemize}
\item
  \hypertarget{keep-up-with-our-coverage}{%
  \subsection{Keep Up With Our
  Coverage}\label{keep-up-with-our-coverage}}

  \begin{itemize}
  \item
    Get an
    \href{https://www.nytimes3xbfgragh.onion/newsletters/politics?action=click\&pgtype=Article\&state=default\&region=BELOW_MAIN_CONTENT\&context=storylines_guide}{email}~recapping
    the day's news
  \item
    Download our mobile app on
    \href{https://apps.apple.com/us/app/nytimes/id284862083?ls=1\&mat_click_id=5c79ae7455014fd1bd66b5610c05b8f2-20191112-16948\&referrer=mat_click_id\%3D5c79ae7455014fd1bd66b5610c05b8f2-20191112-16948\%26link_click_id\%3D722930677036718082}{iOS}~and
    \href{http://a.localytics.com/android?id=com.nytimes.android\&referrer=utm_source\%3Dother_nyt_mobile_web\%26utm_medium\%3DWeb\%2520page\%26utm_term\%3DGeneral\%2520Mobile\%2520Page\%26utm_campaign\%3DNYT\%2520Mobile\%2520General\%2520Page}{Android}~and
    turn on Breaking News and Politics alerts
  \end{itemize}
\end{itemize}

Advertisement

\protect\hyperlink{after-bottom}{Continue reading the main story}

\hypertarget{site-index}{%
\subsection{Site Index}\label{site-index}}

\hypertarget{site-information-navigation}{%
\subsection{Site Information
Navigation}\label{site-information-navigation}}

\begin{itemize}
\tightlist
\item
  \href{https://help.nytimes3xbfgragh.onion/hc/en-us/articles/115014792127-Copyright-notice}{©~2020~The
  New York Times Company}
\end{itemize}

\begin{itemize}
\tightlist
\item
  \href{https://www.nytco.com/}{NYTCo}
\item
  \href{https://help.nytimes3xbfgragh.onion/hc/en-us/articles/115015385887-Contact-Us}{Contact
  Us}
\item
  \href{https://www.nytco.com/careers/}{Work with us}
\item
  \href{https://nytmediakit.com/}{Advertise}
\item
  \href{http://www.tbrandstudio.com/}{T Brand Studio}
\item
  \href{https://www.nytimes3xbfgragh.onion/privacy/cookie-policy\#how-do-i-manage-trackers}{Your
  Ad Choices}
\item
  \href{https://www.nytimes3xbfgragh.onion/privacy}{Privacy}
\item
  \href{https://help.nytimes3xbfgragh.onion/hc/en-us/articles/115014893428-Terms-of-service}{Terms
  of Service}
\item
  \href{https://help.nytimes3xbfgragh.onion/hc/en-us/articles/115014893968-Terms-of-sale}{Terms
  of Sale}
\item
  \href{https://spiderbites.nytimes3xbfgragh.onion}{Site Map}
\item
  \href{https://help.nytimes3xbfgragh.onion/hc/en-us}{Help}
\item
  \href{https://www.nytimes3xbfgragh.onion/subscription?campaignId=37WXW}{Subscriptions}
\end{itemize}
