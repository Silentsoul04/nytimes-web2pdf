Sections

SEARCH

\protect\hyperlink{site-content}{Skip to
content}\protect\hyperlink{site-index}{Skip to site index}

\href{https://www.nytimes3xbfgragh.onion/section/world/americas}{Americas}

\href{https://myaccount.nytimes3xbfgragh.onion/auth/login?response_type=cookie\&client_id=vi}{}

\href{https://www.nytimes3xbfgragh.onion/section/todayspaper}{Today's
Paper}

\href{/section/world/americas}{Americas}\textbar{}Brazil Fails to
Replace Cuban Doctors, Hurting Health Care of 28 Million

\url{https://nyti.ms/2Wu0G1V}

\begin{itemize}
\item
\item
\item
\item
\item
\item
\end{itemize}

Advertisement

\protect\hyperlink{after-top}{Continue reading the main story}

Supported by

\protect\hyperlink{after-sponsor}{Continue reading the main story}

\hypertarget{brazil-fails-to-replace-cuban-doctors-hurting-health-care-of-28-million}{%
\section{Brazil Fails to Replace Cuban Doctors, Hurting Health Care of
28
Million}\label{brazil-fails-to-replace-cuban-doctors-hurting-health-care-of-28-million}}

\includegraphics{https://static01.graylady3jvrrxbe.onion/images/2019/06/03/world/00brazil-doctors1/merlin_154541508_992d5e0e-ea1f-48e7-93dd-c1686e241e6e-articleLarge.jpg?quality=75\&auto=webp\&disable=upscale}

By Shasta Darlington and Letícia Casado

\begin{itemize}
\item
  June 11, 2019
\item
  \begin{itemize}
  \item
  \item
  \item
  \item
  \item
  \item
  \end{itemize}
\end{itemize}

\href{https://www.nytimes3xbfgragh.onion/es/2019/06/11/medicos-cubanos-brasil-bolsonaro/}{Leer
en español}

EMBU-GUAÇU, Brazil --- The shiny plastic chairs all sat empty in a
public health clinic, and the patients who staggered in were told to
come back Thursday --- the only day of the week now when a doctor is
there.

This small Brazilian city, Embu-Guaçu, home to 70,000 people, recently
lost eight of its 18 public-sector doctors, a devastating loss for the
city's network of free clinics, forcing hard choices about who gets care
and when.

``It's heartbreaking,'' said Fernanda Kimura, a doctor who coordinates
the assignment of physicians to the clinics for the local health
department. ``Like choosing which child to feed.''

The sick and the injured turned away that day in a working-class
neighborhood of Embu-Guaçu represent only a tiny fraction of the
estimated 28 million people across Brazil whose access to health care
has been sharply curtailed, according to the National Confederation of
Municipalities, following a confrontation between Brazil's new
president, Jair Bolsonaro, and Cuba.

In November, Cuba announced it was
\href{https://www.nytimes3xbfgragh.onion/2018/11/14/world/americas/brazil-cuba-doctors-jair-bolsonaro.html}{recalling
the 8,517 doctors} it had deployed to poor and remote regions of Brazil,
a response to the tough stance against Cuba that Mr. Bolsonaro had vowed
to take when he was elected in October.

The abrupt departure of thousands of doctors has presented Mr. Bolsonaro
with one of his first major policy challenges --- and has tested his
ability to deliver on a promise to find homegrown substitutions quickly.

\includegraphics{https://static01.graylady3jvrrxbe.onion/images/2019/06/03/world/00brazil-doctors2/merlin_154541484_980a100f-5815-45fa-8875-a64914f95357-articleLarge.jpg?quality=75\&auto=webp\&disable=upscale}

``We are graduating, I am certain, around 20,000 doctors a year, and the
trend is to increase that
number,''\href{http://agenciabrasil.ebc.com.br/politica/noticia/2018-11/bolsonaro-diz-que-programa-mais-medicos-nao-sera-suspenso}{Mr.
Bolsonaro said in November}. ``We can solve this problem with these
doctors.''

But six months into his presidential term, which started in January,
Brazil is struggling to replace the departed Cuban doctors with
Brazilian ones: 3,847 public-sector medical positions in almost 3,000
municipalities remained unfilled as of April, according to the most
recent figures available.

``In several states, health clinics and their patients don't have
doctors,'' said Ligia Bahia, a professor at the Federal University of
Rio de Janeiro. ``It's a step backward. It impedes early diagnoses, the
monitoring of children, pregnancies and the continuation of treatments
that were already underway.''

\hypertarget{our-cuban-brothers-will-be-freed}{%
\subsection{`Our Cuban brothers will be
freed'}\label{our-cuban-brothers-will-be-freed}}

During his campaign for the presidency, Mr. Bolsonaro, a right-wing
populist, committed to making major changes to the Mais Médicos program,
an initiative begun in 2013 when a leftist government was in power. The
program sent doctors into Brazil's small towns, indigenous villages and
violent, low-income urban neighborhoods.

About half of the Mais Médicos doctors were from Cuba, and they were
deployed to 34 remote indigenous villages and the poorer quarters of
more than 4,000 towns and cities, places that established Brazilian
physicians largely shun.

``The willingness of Cuban doctors to work in difficult conditions
became a cornerstone of the public health system,'' said Ms. Bahia, the
professor.

Brazil paid millions of dollars a month to Cuba for the doctors, making
them
\href{https://www.nytimes3xbfgragh.onion/2019/03/17/world/americas/venezuela-cuban-doctors.html}{a
vital export for the island's coffers}. But most of the money went
directly to Cuba's Communist government, an arrangement Mr. Bolsonaro
warned he would change.

Image

A group of Cuban doctors who returned to Cuba in November waiting to
meet the island's president, Miguel Diaz-Canel.Credit...Desmond
Boylan/Associated Press

Cuban doctors
\href{https://www.nytimes3xbfgragh.onion/2017/09/29/world/americas/brazil-cuban-doctors-revolt.html}{have
long complained about getting only a small cut} of the money for their
work, and Mr. Bolsonaro said they would have to be allowed to keep their
entire salaries and to bring their families with them to Brazil. They
would also have to pass equivalency exams to prove their qualifications.

``Our Cuban brothers will be freed,'' Mr. Bolsonaro said in an official
campaign proposal presented to electoral authorities. ``Their families
will be allowed to migrate to Brazil. And, if they pass the
revalidation, they will begin to receive the entire amount that was
being robbed by the Cuban dictators!''

Two weeks after Mr. Bolsonaro
\href{https://www.nytimes3xbfgragh.onion/2018/10/28/world/americas/jair-bolsonaro-brazil-election.html}{won
the presidency in October}, Cuba ordered all its doctors out.

\hypertarget{37000-young-children-at-risk-of-death}{%
\subsection{37,000 young children at risk of
death}\label{37000-young-children-at-risk-of-death}}

Access to free health care is a right under Brazilian law, and Mais
Médicos was enacted in 2013 by President Dilma Rousseff in a bid to
provide medical care to communities that were not being served by the
public health system. Through a network of free clinics, the program
provided 60 million Brazilians with access to a family doctor in their
community for the first time.

In the first four years of Mais Médicos, the percentage of Brazilians
receiving primary care rose to 70 percent from 59.6 percent, according
to\href{https://www.paho.org/bra/index.php?option=com_content\&view=article\&id=5809:opas-lanca-relatorio-30-anos-de-sus-que-sus-para-2030-e-destaca-importancia-de-atencao-primaria-e-mais-medicos\&Itemid=843}{a
report} by the Pan-American Health Organization, which coordinated
Cuba's participation in the program.

The withdrawal of Cuban doctors could reverse that trend, with the
consequences especially severe for those under 5, potentially leading to
the deaths of up to 37,000 young children by 2030,
\href{https://www.terra.com.br/vida-e-estilo/saude/saida-de-cubanos-pode-levar-a-aumento-de-37-mil-mortes-diz-opas,c9da3a2fd30f623b7f4a90f607913279de24ust3.html}{warned
Dr. Gabriel Vivas}, an official with the Pan-American Health
Organization.

In February, it looked as if Mr. Bolsonaro would fulfill his promise:
the national Health Ministry announced that all of the positions left
vacant by Cuba's withdrawal had been filled with Brazilian doctors. But
by April, thousands of the new recruits had either quit or failed to
show up for work in the first place.

More than 2,000 Cuban doctors have chosen to remain in Brazil, defying
the call to return home. But with the special arrangement with Cuba
terminated, they are now ineligible to practice medicine until they pass
an exam --- which the Brazilian government has not offered since 2017
and for which the Health Ministry has set no date.

Luiz Henrique Mandetta, Brazil's health minister, said the new
government was working on a bill to ensure the goals of Mais Médicos
were achieved and the doctors replaced.

``Even if the program has various problems, it has a positive side,
which is, precisely, diminishing the inequality in health care
neglect,'' he said.

But Mr. Mandetta initially said the bill would be sent to Congress
between April and May. Now, the ministry says it will be introduced by
the end of June.

Karel Sánchez was one of four Cuban doctors sent to the remote region of
Cachoeira do Arari in the Brazilian Amazon. He waited there for five
months after his government ordered the withdrawal of all Cuban doctors,
with the expectation that Mr. Bolsonaro would respect his campaign
pledge to provide an exam so he could continue to work and receive his
full salary.

Image

Dr. Karel Sánchez decided not to return to Cuba, expecting to be able to
keep working in Brazil after taking a revalidation exam. That has not
been the case.Credit...Maira Erlich for The New York Times

``I was happy when Bolsonaro said he wouldn't support a dictatorship,''
Dr. Sánchez said.

In April, Dr. Sánchez gave up and moved to São Paulo, where he scrapes
together money by selling homemade sweets and working as a baggage
handler at an airport.

``Now he doesn't talk about us at all, just silence,'' Dr. Sánchez said.

\hypertarget{i-told-people-to-think-about-that-before-they-voted}{%
\subsection{`I told people to think about that before they
voted'}\label{i-told-people-to-think-about-that-before-they-voted}}

In Embu-Guaçu, Dr. Santa Cobas, the Cuban doctor who had been serving
residents at the clinic now only open on Thursdays, was still nearby and
eager to work.

But Dr. Cobas is unemployed, and the 4,000 people she once cared for
don't have access to a local doctor six days a week.

``Now we end up doing triage all day --- deciding who needs to rush to
another hospital, who gets to see the visiting doctor on Thursday and
who will just have to wait,'' said Erica Toledo, the head nurse at the
clinic, Jardim Campestre, which was opened in 2015.

``Dr. Santa was here from the first day, and it was the first time
people felt taken care of by their `own' doctor,'' Ms. Toledo said.
``They really love her.''

The health secretary of Embu-Guaçu, Dr. Maria Dalva, said she was
frustrated that 63 percent of the city had voted for Mr. Bolsonaro,
despite his overt antipathy for Mais Médicos.

``The child mortality rate here dropped to 7 percent from 17 percent in
five years thanks to Mais Médicos,'' said Dr. Dalva. ``I told people to
think about that before they voted.''

Image

A nurse in an otherwise empty corridor at the clinic in
Embu-Guaçu.Credit...Maira Erlich for The New York Times

Advertisement

\protect\hyperlink{after-bottom}{Continue reading the main story}

\hypertarget{site-index}{%
\subsection{Site Index}\label{site-index}}

\hypertarget{site-information-navigation}{%
\subsection{Site Information
Navigation}\label{site-information-navigation}}

\begin{itemize}
\tightlist
\item
  \href{https://help.nytimes3xbfgragh.onion/hc/en-us/articles/115014792127-Copyright-notice}{©~2020~The
  New York Times Company}
\end{itemize}

\begin{itemize}
\tightlist
\item
  \href{https://www.nytco.com/}{NYTCo}
\item
  \href{https://help.nytimes3xbfgragh.onion/hc/en-us/articles/115015385887-Contact-Us}{Contact
  Us}
\item
  \href{https://www.nytco.com/careers/}{Work with us}
\item
  \href{https://nytmediakit.com/}{Advertise}
\item
  \href{http://www.tbrandstudio.com/}{T Brand Studio}
\item
  \href{https://www.nytimes3xbfgragh.onion/privacy/cookie-policy\#how-do-i-manage-trackers}{Your
  Ad Choices}
\item
  \href{https://www.nytimes3xbfgragh.onion/privacy}{Privacy}
\item
  \href{https://help.nytimes3xbfgragh.onion/hc/en-us/articles/115014893428-Terms-of-service}{Terms
  of Service}
\item
  \href{https://help.nytimes3xbfgragh.onion/hc/en-us/articles/115014893968-Terms-of-sale}{Terms
  of Sale}
\item
  \href{https://spiderbites.nytimes3xbfgragh.onion}{Site Map}
\item
  \href{https://help.nytimes3xbfgragh.onion/hc/en-us}{Help}
\item
  \href{https://www.nytimes3xbfgragh.onion/subscription?campaignId=37WXW}{Subscriptions}
\end{itemize}
