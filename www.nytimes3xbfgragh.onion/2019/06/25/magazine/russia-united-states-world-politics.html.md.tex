What Does Putin Really Want?

\url{https://nyti.ms/2ZQYFz3}

\begin{itemize}
\item
\item
\item
\item
\item
\item
\end{itemize}

\includegraphics{https://static01.graylady3jvrrxbe.onion/images/2019/06/30/magazine/30mag-russia-image1/30mag-russia-image1-articleLarge-v3.jpg?quality=75\&auto=webp\&disable=upscale}

Sections

\protect\hyperlink{site-content}{Skip to
content}\protect\hyperlink{site-index}{Skip to site index}

\hypertarget{what-does-putin-really-want}{%
\section{What Does Putin Really
Want?}\label{what-does-putin-really-want}}

Russia is dead set on being a global power. But what looks like grand
strategy is often improvisation --- amid America's retreat.

Credit...Photo illustration by Delcan \& Company

Supported by

\protect\hyperlink{after-sponsor}{Continue reading the main story}

By Sarah A. Topol

\begin{itemize}
\item
  June 25, 2019
\item
  \begin{itemize}
  \item
  \item
  \item
  \item
  \item
  \item
  \end{itemize}
\end{itemize}

\emph{This story was supported by the Pulitzer Center.}

On a warm late-May afternoon, I took a taxi to the outskirts of the
Russian capital to the Moscow State Institute of International
Relations, known by its Russian acronym, Mgimo. Flags marked the
entrance to the campus, a stately Soviet behemoth with a
hammer-and-sickle on a panel above the doors. Students in skinny jeans,
button-down shirts and thick black glasses gathered in gaggles by the
flagpoles, checking their phones and chatting. I signed in as a visitor
at the security desk and wandered past a buzzing cafeteria, into the
institute's gift shop, with its rainbow of sweatshirts, coffee mugs and
notebooks emblazoned with the Mgimo logo.

Since 1944, Mgimo has trained legions of diplomats; its 53 language
offerings --- including Afrikaans, Amharic and Vietnamese --- serve as a
reminder of the Soviet Union's global ambitions. As much as ninety-five
percent of Russia's foreign ministry is made up of Mgimo alumni, while
those who graduate with honors and pass a language test become attachés,
complete with a green diplomatic passport. They are then sent forth, as
Vladimir Putin himself put it, ``to protect Russian interests'' in the
rest of the world. Alumni include the president of Azerbaijan, the
foreign-affairs ministers of Slovakia and Mongolia and Russia's own
foreign minister, Sergey Lavrov, who regularly returns to give the
commencement address.

Mgimo is run by the foreign ministry, so Andrey Baykov, its vice rector,
is a hybrid --- part academic, part representative of Russian diplomacy.
I asked him a question that I would spend a long time trying to
understand: What does Russia \emph{really} want?

From behind a large dark wooden desk, Baykov --- young and fresh-faced,
dressed smartly in a suit and slender tie --- answered in nearly
flawless British-accented English: ``To be an autonomous player, to
uphold its identity of a great power which is strategically
independent.'' Russia, he explained, did not want to dismantle the
trans-Atlantic world order by splintering NATO and demolishing the
European Union, as was frequently suggested by the Western press using
headlines like
``\href{https://www.vanityfair.com/news/2016/12/is-vladimir-putin-master-plan-only-beginning}{Is
Putin's Master Plan Only Beginning}?'' (Vanity Fair);
``\href{https://www.theatlantic.com/international/archive/2018/02/trump-manafort-mueller-indictment-putin-russia-ukraine-germany-lobbying/554210/}{The
Dark Arts of Foreign Influence-Peddling}'' (The Atlantic);
``\href{https://slate.com/technology/2016/12/why-russia-is-using-the-internet-to-undermine-western-democracy.html}{Why
Russia Is Using the Internet to Undermine Western Democracy}'' (Slate).
Instead, he spoke about the importance of Russia's national identity and
its territorial sovereignty.

``Nationalism comes in different disguises,'' he told me. ``America's
nationalism comes in the disguise of universalism, but this universalism
is basically the American national model being expanded.'' Russia's
nationalism, he went on, is inward-looking. Students ``all come here
with the idea that Russia is a great power. It has all the rights it
deserves, but it's being mistreated,'' he said. ``There is a huge sense
of being betrayed.''

\href{https://www.nytimes3xbfgragh.onion/2018/05/08/magazine/the-quiet-americans-behind-the-us-russia-imbroglio.html}{\emph{{[}Read
about the quiet Americans behind the U.S.-Russia imbroglio.{]}}}

Baykov's sentiment was one I heard repeatedly while speaking to
analysts, academics and journalists in Moscow. My trip came before the
release of the Mueller report, and Russians alternately laughed or
dismissed as hysterical what they saw as the Western press's assertions
of Putin's grandiose manipulations. ``The American media made the
Kremlin the third player in the U.S. election, which is great,'' joked
Andrei Soldatov, a Russian investigative journalist who specializes in
cybersecurity. ``Like, you think to yourself: `We are such a great
country, we can interfere with world elections!' '' His real point was
serious, though: that Americans looking for a master plan fundamentally
misunderstood the Russian leadership's mentality. ``When you are trained
by the K.G.B., it means you see the world in terms of threats,'' he
explained. ``That's the only way you see it. The thing about threats is
that when you see threats, you do not have strategy; you rely on
tactics. Because you don't know what the next threat might be, you only
respond.''

On another afternoon in Moscow, I entered the Russian Academy of
Sciences' Institute for U.S. and Canadian Studies, a crumbling yellow
building, and climbed a stairwell under a massive chandelier to reach
the office of its current director, Valery Garbuzov. The institute was
founded in 1967 after the Cuban missile crisis, perhaps the nadir of
U.S.-Russian relations. Garbuzov told me that the institute was formed
to provide the Soviet regime with detailed analysis of its adversaries
to help direct the U.S.S.R.'s responses. ``That doesn't mean that the
Soviet leaders did what was advised!'' he told me mirthfully. ``Rather,
they did the opposite.''

\includegraphics{https://static01.graylady3jvrrxbe.onion/images/2019/07/03/magazine/03mag-russiapix2/eb50421b0d474de18f23dbf9c8c298a7-articleLarge.jpg?quality=75\&auto=webp\&disable=upscale}

Garbuzov suggested that little had changed --- the Kremlin did not
understand America and did not listen to those who did. The United
States was no different. ``We have an image of America as the country
that foments revolutions around the world. The Americans have the image
of Russia as a country that wants to revive the Soviet Union by any
means,'' he went on. ``Both parties deeply misunderstand the motives of
each other's behavior.'' He concluded by saying: ``This is a very sad
thing, the mutual misunderstanding we couldn't overcome during the
decades of the Cold War and can't overcome now.''

When I met Ruslan Pukhov, director of the Center for Analysis of
Strategies and Technologies, a military think tank, at the elegant Cafe
Pushkin, he did not mince words. ``Every time some Western observer says
`Russians did this, Russia did that,' I say: `You describe Russians like
they are Germans and Americans. We are not.' I also ask: `Do you know
the word \emph{bardak}?' '' I did. ``If you don't know the word
\emph{bardak}, you are an idiot and not an analyst of Russia. Because
\emph{bardak} is disorder, it's fiasco.'' Pukhov's point about
\emph{bardak} --- which technically means ``mess'' but is also used
colloquially to describe utter chaos --- was that Russia's political
system isn't a streamlined, top-down dictatorship. Only naïveté,
paranoia or both could convince you that the system functioned
efficiently enough to execute a grand global anything.

Russia has long been a canvas on which Americans project their thoughts
or fears --- of the Red menace, and of Putin's quest for world
domination. This tradition only accelerated after the 2016 election,
when it seemed as if everyone were an expert on Putin's agenda. There
wasn't an election he didn't hack, a border he wouldn't violate or an
American ally he couldn't manipulate. The very word ``Putin'' has come
to symbolize a coherent, systematic destruction of the post-Cold War
international order. But no one I spoke with who had an intimate
knowledge of Russia saw that as anything but fiction. Instead, they
talked about Russia's strides back onto the world stage as improvised
reactions, tactics, gambles that were at times more worrisome than
masterful.

Because a nation's foreign policy is in part built on its perceptions of
itself, magnified to the world stage, I came to Moscow to understand how
Russians saw themselves as much as how they saw the world. On and off
for over two years, I visited other countries in the Middle East and
Europe --- historical allies of the United States that were portrayed in
the press as pivoting to Russia --- to do the same. If Americans tried
to see the world as the Russians did, and as our allies did, could we
better understand what any of these countries were doing? And if we
understood what they really wanted, could we better understand the world
ourselves?

\textbf{To comprehend contemporary} Russian thinking about the West, I
was told to start at the beginning. Yet even identifying the beginning
of the post-Cold War international order is a fraught exercise. Russian
policymakers often set the start date in 1989, when General Secretary
Mikhail Gorbachev willingly dismantled Russia's political and military
dominance over Eastern Europe. After such a magnanimous gesture, Moscow
believed it would be treated as an equal partner of the United States,
rather than as a rival, with the right to retain influence over
countries in what it considered its neighborhood.

Western observers, on the other hand, date the dawn of the American
hegemonic age as 1991, when the Soviet Union was roundly defeated and
collapsed, costing Russia any say over its neighboring countries. That
is, each side would come to blame the other for reneging on a post-Cold
War compact that the other side never agreed on or perhaps even really
understood. As the academics Andrej Krickovic and Yuval Weber noted in a
2016 article in the journal Russia in Global Affairs: ``The basic
disagreement becomes clear: Was the status quo set in 1989, making the
U.S. a revisionist hegemon, or was it set in 1991, making Russia a
revisionist challenger?''

The 1999 war in Kosovo provided the first clear indication that the
Russian view would not be reciprocated. Under President Boris Yeltsin,
Russia had joined the Council of Europe in 1996 and the G7 in 1998. It
sought special status with NATO and even flirted with joining the
European Union. The Russians were furious when NATO forces launched a
military campaign in Kosovo without United Nations Security Council
authorization. The Kremlin viewed Yugoslavia as within its sphere of
influence. Prime Minister Yevgeny Primakov was over the Atlantic, en
route to Washington, when Vice President Al Gore called to inform him
that airstrikes had commenced. In a show of anger, Primakov turned his
plane around.

When Putin assumed the presidency in 2000, he remained ``convinced that
he could build good relations with the West, in particular with the
United States,'' the Russian journalist Mikhail Zygar writes in ``All
the Kremlin's Men: Inside the Court of Vladimir Putin.'' He took pains
to court Tony Blair and George W. Bush, and he was the first leader to
call Bush after the Sept. 11 attacks. Russia was fighting the second
Chechen war, and Putin sought to portray Chechen separatists as
terrorists. He mistakenly believed the attacks on Sept. 11 would align
the two countries' world views around the war on terror.

Zygar writes that before the Americans began their bombing campaign in
Afghanistan, Washington reached out to Moscow for approval to construct
a temporary air base in Kyrgyzstan, promising that the occupation would
last a year at most. Russia agreed. But by 2002, as the campaign
appeared to be dying down, Alexander Voloshin, the main political
strategist of Putin's first term, asked National Security Adviser
Condoleezza Rice when the United States would leave. ``You know what?''
Zygar writes that Rice responded. ``It turns out we really need this
base, like, permanently.'' That year, the United States unilaterally
withdrew from the landmark 1972 Anti-Ballistic Missile Treaty, despite
Russian protests. The treaty was signed to slow the nuclear arms race
and was referred to by Russians as ``the cornerstone of strategic
stability.''

The next offense came in 2003, when Bush circumvented United Nations
authorization and invaded Iraq. Russia maintained historic and economic
ties to Iraq, and the Kremlin publicly disputed the White House's claim
that Baghdad possessed weapons of mass destruction. Meanwhile, from 2003
to 2005, a wave of protests against Soviet-era rulers spread across
Georgia, Ukraine and Kyrgyzstan, leading to the establishment of
pro-Western governments. Nations in what was once considered the Soviet
sphere joined NATO as a way to protect themselves from Russia. The
Kremlin perceived these shifts as a threat to its own territorial
sovereignty. Russians now understood clearly that the West saw them as a
``de facto defeated country,'' Fyodor Lukyanov, chairman of the
Presidium on the Council of Foreign and Defense Policy, told me, ``which
had no right to claim to be on the same footing as Americans or
Europeans.''

The difference in perspectives slowly became intractable. By 2007, Putin
voiced his displeasure at the Munich Security Conference, an annual
assembly of global elites, but it's unclear if anyone understood the
depth of his discontent. ``The United States has overstepped its
national borders in every way,'' he said. ``This is visible in the
economic, political, cultural and educational policies it imposes on
other nations. Well, who likes this? Who is happy about this?'' He went
on: ``No one feels safe! Because no one can feel that international law
is like a stone wall that will protect them.''

The Kremlin took particular issue with what Pyotr Stegny, a Russian
academic writing in Russia in Global Affairs, called a kind of
democratic fundamentalism emanating from the United States, including
its insistence on liberal values like women's, minorities' and L.G.B.T.
rights without considering the local context. Russian observers could
find no other logic in America's decision-making around the Arab Spring,
when the Obama administration supported the resignation of President
Hosni Mubarak of Egypt, a 30-year ally. Moscow preferred stability at
all costs, while the United States ``mercilessly and mindlessly betrays
allies for the sake of theoretical dogmatism,'' Yevgeny Satanovsky, a
Russian academic, wrote in the same journal, noting that these policies
often lead to support of Islamic fundamentalists, who are even more at
odds with America's purported values.

The 2014 Winter Olympics in Sochi, intended as Putin's crowning glory,
were overshadowed by boycott threats over Russia's draconian ban on
``gay propaganda'' (the country was later slammed by the United Nations
Human Rights Committee for violating an international covenant on human
rights). The Kremlin viewed the condemnation as a geopolitical insult
driven by the American government. ``People were attacking Russia,
starting with the pro-L.G.B.T. campaign in the run-up to the Olympics
and with, from the Russian standpoint, a very clear objective of
scuttling the Olympics information space,'' Dmitri Trenin, director of
the Carnegie Moscow Center, told me.

In February 2014, mass protests led President Viktor Yanukovych of
Ukraine to flee to Russia. Putin promptly sent soldiers without insignia
to take over Crimea, redrawing the borders of continental Europe. ``What
did Russia do in March 2014?'' asked Timofey Bordachev, director of the
Center for Comprehensive European and International Studies in Moscow.
``Russia violated the United States' monopoly on breaking international
law,'' he continued. ``When the United States did not respect the
international law, it was relatively O.K. for everybody, but when Russia
followed, it created such a tense attitude.''

The West responded by imposing multiple sanctions on Moscow, and then
expanding them as the violence in Ukraine increased, including the
downing of a civilian airplane by Russian-backed rebels that killed 298
people. Though it is debatable whether the sanctions had their intended
economic effect, Russians saw them as hypocritical and demeaning. (They
also took great offense when Obama called their country a ``regional
power that is threatening some of its immediate neighbors, not out of
strength but out of weakness.'') Putin imposed his own countersanctions,
but those, too, had little effect. Russia was now an international
pariah, and the isolation seemed to only embolden Putin. ``After Ukraine
in particular,'' Trenin told me, ``Putin, in my view, decided to move on
from standing defense to active defense.''

Image

Putin with President Bashar al-Assad of Syria in Sochi, Russia, in
2017.Credit...Kremlin Press Office/Handout/Andalou Agency/Getty Images

\textbf{It was in Syria} where Putin challenged his country's post-Cold
War identity, as well as how the West had perceived it for so long. His
decision to commit Russian forces has been portrayed as the first step
in an effort to realign the region, but the strategy was largely a
result of luck and timing, its tactics born partly of a lack of
resources.

After protests against the government of President Bashar al-Assad began
in 2011, Moscow blocked United Nations resolutions that would have paved
the way for future intervention and continued shipping weapons to the
Syrian Army. Russia's drive to protect Assad dovetailed with the
American administration's regional disengagement. Obama had no interest
in entangling America in another war. He said the ``red line'' would be
the use of chemical weapons, yet as evidence of their use piled up,
Washington did little. After the emergence of the Islamic State in 2013,
the United States quickly became fixated on fighting the group itself,
with little more than vague words of support for the Syrian opposition.

By 2015, Assad's regime was on the precipice of collapse, losing
territory to ISIS and the anti-government militias simultaneously. The
Middle East is far closer to Moscow than to Washington; Syria's nearest
border with Russia is roughly as far away as Washington is from Boston.
Moscow feared the spread of unrest and terrorism like a contagion. On
Sept. 30, 2015, Assad sent a formal request for assistance based on a
1980 military cooperation treaty, which the Russian Parliament
rubber-stamped. Russian armed forces officially went in. (Many Russians
I spoke to quickly pointed out that because of this process, Putin's
military presence in Syria, unlike the U.S. invasion of Iraq, is
permissible.)

After years of ``covert'' war by Russia in Ukraine and the legacy of a
decade-long Soviet embroilment in Afghanistan, which killed 14,000
Soviet troops and one million civilians, the Russian population was
genuinely wary about the intervention in Syria. In a poll by the Levada
Center, Russia's only major independent polling center, 69 percent of
Russians were against direct military involvement. In terms of technical
support, opinions were split --- only 43 percent believed Russia should
advise and arm Assad, while 41 percent were against it. Analysts and
opposition politicians pointed out a host of risks from Russian
``adventurism.'' Dmitry Gudkov, Russia's most vocal opposition member of
Parliament, warned about potential repercussions. ``It is not known how
this will end,'' he cautioned during an interview with Radio Free
Europe/Radio Liberty. Russia has its own large Muslim minority, roughly
10 percent of the population, and ex-Soviet countries contributed the
largest cadre of foreign ISIS fighters, estimated at up to 8,500 people.
Gudkov cautioned that troops on the ground in Syria risked inflaming
ethnic tensions within Russia's own borders.

The Kremlin cares deeply about domestic opinion and set about selling
the intervention on television and through pro-war op-eds. ``The entire
Middle East, due to internal reasons and the stupid, unceremonious and
irresponsible intervention of the West, entered a period of decades of
instability,'' Sergei Karaganov, honorary chairman of the Council on
Foreign and Defense Policy, explained on a popular Saturday
political-affairs show. ``And this instability will have to be
controlled.''

\href{https://www.nytimes3xbfgragh.onion/2017/09/13/magazine/rt-sputnik-and-russias-new-theory-of-war.html}{\emph{{[}How
the Kremlin built one of the most powerful information weapons of the
21st century.{]}}}

The Russian incursion began with a volley of airstrikes ``targeting
ISIS'' in an area where there were no records of its activity. It was
instead home to anti-Assad militias and at least one C.I.A.-trained
group. The following day, Lavrov clarified that the Russian air campaign
was targeting ``all terrorists.'' Over the coming weeks, Russia
surprised Western officials with its modernized military. After the
country's disastrous war with Georgia in 2008, Putin had embarked on an
extensive modernization program. Russia deployed 215 new weapons,
showcasing them to potential buyers. Strikes utilized the relatively
untested Sukhoi Su-34 strike fighter and a ship-based cruise missile
fired more than 900 miles from the Caspian Sea, which,
\href{https://www.nytimes3xbfgragh.onion/2015/10/15/world/middleeast/russian-military-uses-syria-as-proving-ground-and-west-takes-notice.html}{according
to some analysts}, exceeded American capabilities. Putin was open about
Russia's intended message. ``It is one thing for the experts to be aware
that Russia supposedly has these weapons, and another thing for them to
see for the first time that they do really exist,'' he said on state
television.

The payoff was immediate: Russia's international isolation was over. The
afternoon Russia struck, Secretary of State John Kerry and Lavrov met at
the United Nations and agreed to begin talks on avoiding unintended
confrontations. ``Obama didn't give a hoot about Syria by 2015,'' Robert
Ford, America's former ambassador to Syria, told me. ``All he was
interested in was the fight against ISIS.'' He went on: ``I think the
Americans at that point, the White House, had washed their hands of it,
and Kerry, kind of operating almost solo, was pleading with the
Russians.''

Russia's official military death toll has remained relatively low
throughout the conflict. The brunt of the casualty count, which the
state suppresses, has been borne by private contractors, including the
Wagner Group, a firm said to be run by Putin's close friend Yevgeny
Prigozhin, known as ``Putin's chef,'' though mercenaries are technically
illegal in Russia.

As Michelle Dunne, director of the Middle East program at the Carnegie
Endowment for International Peace, told me: ``Putin has played a clever
hand on the cheap, without really investing that much in it.'' Russia's
economy is 10 times smaller than America's, and Trenin estimated the
cost at roughly \$4 million per day, which is ``reasonably affordable''
for the state budget. (By comparison, the United States' mission in
Syria, part of Operation Inherent Resolve, cost \$54 billion from 2014
to 2019, or \$25 million a day.)

By 2017, Russia had embarked on its own diplomatic solution to the
conflict, outside the United Nations process, holding conferences in
Astana and Sochi alongside Turkey (a NATO member) and Iran. ``The
Russians transformed the Syrian rebellion from one actively fighting
Bashar al-Assad to one giving up on the fall of Bashar al-Assad and
trying to preserve their areas of influence,'' Hassan Hassan, director
of the nonstate actors program at the Center for Global Policy, told me.
Hassan explained that Turkey's place alongside Moscow in the Astana
conference gave Russia credibility in the eyes of the anti-Assad
militias. Yet they knew full well whom they were dealing with. Under the
guise of supporting Assad against ISIS, Moscow assisted and looked the
other way as the regime dropped barrel bombs on hospitals, starved
civilian population centers and set up massive domestic detention and
torture facilities. More than half a million people have been killed in
the fighting. Russia has also successfully obfuscated the most egregious
Assad crimes --- the use of chemical weapons. In Sochi and Astana, those
fighting the dictator are now sitting down with the same power breathing
life into his despotic government. ``The fighters trusted Turkey more
than they trusted Russia,'' Hassan told me. ``But Russia was the
leader.''

Image

Putin with President Recep Tayyip Erdogan of Turkey at the Kremlin in
2017.Credit...Mikhail Svetlov/Getty Images

In many ways, Moscow's assertive foreign policy came as a result of
earlier decisions made in Washington. Both Putin and the Obama
administration were responding to the same thing: George W. Bush's
aggressive foreign policy. America's withdrawal from the Middle East and
elsewhere was the result of an American imperial hangover. Russia and
the United States had moved in opposite directions, creating an
appearance of one power rising and the other falling. The outcome was
ultimately the same: ``Whatever people think about Russia's role,
everybody acknowledges that it is the key state there,'' Lukyanov told
me. ``It's not because of Russian strength; it's because of American
weakness, it's because of European geopolitical collapse. But that's a
fact of life now.''

\textbf{Russia's success in} Syria has inspired the Kremlin to sell
itself as a neutral moderator in other Middle Eastern conflicts --- the
fight among factions in Libya, the war in Yemen and the
Israeli-Palestinian quagmire. Russia has now signed arms deals with all
sides of the region's complicated rivalries, including the United Arab
Emirates and Qatar. King Salman became the first Saudi monarch to visit
Russia when he arrived in October 2017, a 1,500-person entourage in tow.
Riyadh and Moscow now coordinate on energy policy. Russia is working
with the Saudis' archrival, Iran, on nuclear power and increasing trade
to help Tehran survive American sanctions. In Iraq, Russia has opened a
military-intelligence-sharing center, signed arms deals and invested in
an oil pipeline in Kurdistan. Prime Minister Benjamin Netanyahu of
Israel has made 10 public visits to Moscow in the last five years.
Across the region, United States allies are often seen as ``pivoting''
--- as if on a magnetic axis --- toward Putin. But when I visited the
region, I found something very different was happening.

Image

Vladimir Putin with President Hassan Rouhani of Iran in
2017.Credit...Sergei Karpukhin/Agence France-Presse/Getty Images

In his sleek, large office inside the Turkish Parliament in Ankara,
Ahmet Berat Conkar of the ruling Justice and Development Party (A.K.P.)
sat across from me, anxiously clutching index cards of notes. Along the
wall he had three bright red Turkish flags and, displayed prominently on
his desk, a Russian teacup. It was a gift, he said, from the
Russian-Turkish Civic Forum, of which he has been a chairman since 2014.
Conkar told me that far from choosing Moscow, Turkey had been obliged to
work with Moscow on Syria only because the United States had left its
NATO ally in the cold.

When the Syrian protests began, President Recep Tayyip Erdogan threw his
support behind protesters against Assad. He kept Turkey's borders open
to support anti-regime militias, which also allowed foreign fighters to
slip across to join the growing ISIS caliphate, about which the United
States repeatedly telegraphed its displeasure. In early 2015, the United
States and Turkey started a three-year, \$500 million ``train and
equip'' program to create a 5,000-strong army to fight ISIS, but it was
a complete failure. Washington called it off later that year, saying in
a statement: ``Given the complexities of the situation, we're going to
take an operational pause.'' It then began working directly with Syrian
Kurdish militias.

Ankara saw America's Kurdish partnership as an existential betrayal ---
the Turkish state considers the group to be an offshoot of a domestic
separatist group, the P.K.K., which the United States and Turkey have
labeled a terrorist organization. Conkar likened it to a situation in
which Turkey would be providing arms to an Al Qaeda base in Mexico that
was trying to separate Texas from the United States --- how would
America feel then? ``With respect to Turkey's fight against terrorism
groups, Russia seems to be more supportive and more understanding of
Turkey's concerns,'' Conkar said. It is up to the United States, he
continued, ``to work on regaining Turkish hearts and minds and regaining
the Turkish trust for cooperation.''

In many ways, I was told repeatedly, Turkey's relationship with Russia
was a kind of rebound, a fling undertaken after being jilted by the
Americans. In recent years, Turkey and the United States have disagreed
on multiple issues, including the extradition of a Turkish preacher
living in America, democratic backsliding, American citizens jailed in
Turkey and U.S.-imposed steel tariffs. In July 2016, Erdogan purged the
military after a coup attempt and claimed that Turkey needed a new
air-defense system to replace the Patriot missiles the United States
withdrew in 2015. In December 2017, Ankara signed a reported \$2.5
billion agreement to buy the Russian S-400 air-defense system.

``Technically, Turkey doesn't need the S-400,'' a retired Turkish
military officer I spoke with explained, requesting anonymity to speak
freely about the deal. Not only is the range Turkey plans to purchase
severely limited, but the system itself is designed to counter NATO
threats. ``Turkey is playing a really complicated game of trying to
increase its strategic value in the eyes of Western powers by showing
them it has alternatives,'' Berk Esen, assistant professor of
International Relations at Bilkent University in Ankara, told me. ``Both
Erdogan and Putin are very pragmatic leaders, and as long as they think
that they will benefit from this cooperation, they will continue to
exploit this opportunity.''

A year of negotiations ensued. The Trump administration offered to
return the Patriot missiles to stop the S-400 deal and threatened to
stop exports of F-35 advanced fighter jets. But Erdogan merely dug in
deeper. Earlier this month, then acting Defense Secretary Patrick
Shanahan sent an ultimatum to his Turkish counterpart. The letter
explained that if Turkey went ahead with the S-400, the United States
would suspend Turkish participation in the F-35 program by July 31.

Conkar and others I spoke with grew exasperated when describing the
brinkmanship American allies felt subjected to after being accused of
crossing the United States. Relations often felt zero-sum, with any
movement toward Russia, especially, answered by an American threat.
``The U.S. is oblivious to how its allies feel about it,'' Hassan told
me. ``Russia comes across as a power to be trusted by America's
traditional allies, even when differences exist, while the U.S. is often
seen as overweening despite the shared interests.'' Russia's aims, taken
at face value, are far less expansive than America's, its interests more
narrow and steady. And notably, they do not extend into questions of
human rights or democracy.

At the same time, more countries have embraced the ideology of
realpolitik. In Egypt, historically America's largest recipient of
foreign assistance after Israel, the United States partly withheld
military aid after Abdel Fattah el-Sisi seized power in a 2013 coup
whose aftermath killed more than 800 people. Russia was one of the first
countries to lend international legitimacy to the government. Putin and
el-Sisi have since staged grandiose announcements about trade
agreements, an industrial zone in the Suez Canal and the construction of
a nuclear power plant. Egypt agreed to purchase \$3.5 billion in Russian
arms. Both authoritarians fed off the pomp of their publicity stunts to
a fawning, censored domestic media to reify their stature.

Image

Putin with President Abdel Fattah el-Sisi of Egypt at the Kremlin in
2015.Credit...Sasha Mordovets/Getty Images

I met Gen. Nasr Salem, a retired military officer who worked on weapons
procurement, in his living room shortly after reports of a preliminary
agreement for Russian use of Egyptian air bases, which had been
\href{https://www.nytimes3xbfgragh.onion/2017/11/30/world/middleeast/russia-egypt-air-bases.html}{portrayed
in the Western media} as a sign that Egypt was snubbing Washington in
favor of Moscow. Papers were stacked on the dining table, ornate rugs
blanketed the floor and a dagger hung on the wall. Seated on a plush
golden armchair, Salam chuckled good-naturedly at many of my questions.
He explained the Egyptians felt they had no choice but to go to Russia
when the United States suspended military aid.

``The important thing about the weapon is not the weapon itself, but the
supplies and the spare parts,'' he said. ``If you stop the supplies and
the spare parts, you are hanging me.'' He continued: ``Russia at that
time, they opened their arms to us.'' I asked Salam if there was concern
within the Egyptian military that el-Sisi was jeopardizing the U.S.
alliance with the base agreement. ``Do you know what sort of facilities
we are giving the American Army?'' he responded, his voice rising ever
so slightly. ``Why are you always putting it this way, that if we open
the link with Russia, then that will affect the relationship with
America? While, for example, Israel has very, very strong relations with
both sides,'' he said. ``Why are you always putting us in that corner?''

Image

Putin with Prime Minister Benjamin Netanyahu of Israel in
2016.Credit...Mikhail Svetlov/Getty Images

It may appear from the outside that Russia was orchestrating
authoritarian governments and nativist movements, but the reality was
more nuanced. The governments Putin was reportedly cultivating, I found,
had their own reasons for courting him right back. Under the Trump
administration, America's policies in the Middle East have appeared
inconsistent and indecipherable --- be it flip-flops on troops in Syria
or a potential march toward war with Iran. Aberrations in routine
diplomatic protocol have become the norm. Since 2017, the United States
has been without an ambassador in Egypt and in Turkey. I heard the same
assessment from Arab and Turkish former ambassadors and diplomats: The
Russian diplomatic core was pragmatic. It moved slowly and deliberately
when it comes to making deals. Those decisions could be trusted.

I encountered mild amusement from everyone I spoke to in the region
about Russia's supposed interference in the West, side-smiles of payback
for all of America's policies that had upended the Middle East for
decades. ``When the West today complains that Russia is trying to
influence internal affairs, I don't have hard evidence, but that's what
you've been doing for generations,'' a former government official told
me. ``It was actually your tool before it was the Russian tool.'' He
added: ``The rest of the world doesn't really understand what for you
guys is new here? You've been doing this all along.''

Russia is certainly positioning itself for a greater international role.
But there is a danger of giving Putin too much credit without looking at
the context. Evidence of Russia's increased presence on the ground in
Egypt is scant. Russia has been a chief source of tourism in Egypt, but
direct flights from Russia to Red Sea beaches were suspended after a
Russian airliner was bombed over the Sinai Peninsula in 2015. Monthly
losses after the incident were worth \$173 million, according to the
tourism department. Cairo has been begging Moscow to resume these
flights, and both Russia and Egypt have issued statements that it would
happen soon --- but nothing has. They flaunted Russia's \$7 billion
investment in an industrial zone, but other than signing agreement after
agreement, there didn't appear to be anything happening on the ground.
``All what you heard about promise of investing in this and that is just
a statement to the media, to the press,'' Ezzat Saad, Egypt's former
ambassador to Moscow, told me last year. ``Nothing else.''

\textbf{The same populist} forces that reshaped Turkey and Egypt have
altered the landscape in Europe. In Germany, another American ally,
Russian interference was initially blamed for this rightward tilt. I met
Stefan Meister of the German Council on Foreign Relations at his sparse
office. Meister was one of the first analysts in Berlin to raise the
alarm of a new Russian threat. In 2016, a year before a pivotal German
parliamentary election, he named civil-society groups, lobbying groups
and politicians as ``networks of influence,'' working to advance the
Russian cause to destabilize the European Union. ``In the campaign for
the next federal elections in Germany,'' he then predicted, ``Russia
will play a prominent role.''

In those elections, the German far-right party Alternative for Germany
(AfD) picked up more seats than it had ever held, but the polls came and
went with no serious disinformation campaign, no leaks from a Parliament
hack (which German intelligence services had pegged to Russian-backed
hackers), no evidence of meddling. After the polls, an independent
N.G.O. did a broad scrape of the German internet and found that Russian
trolls did not seem to be significantly involved in the creation of the
most viral ``fake news'' stories around the election. Instead, they came
from local AfD politicians and their supporters, who themselves had
merely recast news headlines and some stories on hot-button issues like
immigration to make them more incendiary. ``It's not that this is a
Russian conspiracy; these false stories were a very important mobilizing
tactic for the right,'' Stefan Heumann, a member of the N.G.O.'s board,
told me. ``They pick this stuff up, and they spin it further.'' He
suggested that AfD had probably learned more tactics from the Trump
campaign than from Russia.

Still, after the election, Western journalists kept calling Meister,
waiting to hear what Russia was really up to: Was Putin devising a new
strategy? Had his meddling just gone undetected? The alarm Meister
helped raise had become deafening. ``My task for the last year is to get
the balance right --- to say that Russia is not the problem, the problem
is us. We are opening opportunities for Russian actors to strengthen
some narratives which already exist in our societies,'' Meister told me.
``We shouldn't overestimate Russia because that makes Russia stronger
than it is; then we overestimate its capabilities and we don't fix our
own problems. It distracts from our own domestic problems, and it's
useful also for our elites.''

It had been incredibly hard to speak to someone from Russia's Ministry
of Foreign Affairs --- one reason the Russia position is often absent
from Western media. But after months of chasing diplomats, I had a
chance to sit down with Denis Mikerin, the Russian press attaché in
Berlin, in March 2018. (He has since returned to work with the ministry
in Moscow, and when we spoke in mid-June, he confirmed that the Russian
position remained unchanged.) The Russian Embassy was a beautiful
complex protected by an imposing white stone wall and wrought-iron
double gates. Inside, we walked past a stained-glass wall depicting a
rainbow over the Kremlin. Interviews with diplomats are often boring
recitations of the country's talking points, but in the case of Russia,
even this is novel. Our two-hour conversation ambled among Crimea,
Europe, Russia and the Middle East. It went on so long that we had to
move from what looked like an ornate tearoom to one resembling a hunting
lodge, with wood-paneled walls decked with trophy heads with antlers.

I asked him if Russia was willing to take on a greater global role and
all the unwanted criticism that would entail. (In a Pew 2018 poll of 25
countries, many saw Russia playing a more important international role
compared with 10 years ago, but views of Putin had grown more
unfavorable.)

He suggested Russia was ready. Moscow would do things differently, he
said. He pointed to Syria: ``Russia has all the legal basis to be there
--- in response to official request from the government of Syria. With
Geneva negotiations completely stalled, the Astana format with Turkey
and Iran appears to be an efficient platform. We did not come in alone
and say now we are deciding. On the contrary, we are trying to join the
efforts of all those committed to preserving Syria's territorial
integrity.'' He seemed to be drawing a distinction between the Astana
format and the American-proposed way forward in Iraq, but the two
versions --- countries spearheading a ``coalition of the willing'' to
work outside the United Nations --- did not feel dissimilar to me.
Still, the strain of trying to spearhead constructive policy was
becoming evident. He claimed that the Americans had sent low-level
officials to the first peace conference in Sochi in an attempt to
undercut their efforts. ``They were sending the signals to those whom
they had in fact wanted, not to sabotage, but to avoid taking part.''
The Russians were falling into the same trap the United States had for
so long --- looking for others to blame for the difficulties of
constructing policy.

In contrast to Western condescension, he explained, when Russia works
with other countries it's about finding common ground and pragmatic
interests. ``It's ridiculous to presume that some countries are lobbying
Russian interests,'' he said. ``They are lobbying their own interests in
the first place.'' He said the Russians were tired of the United States
and the European Union ``mentoring'' them. ``We kept trying to find this
very high road in relations between Russia and the Western world in
general. We treat everyone equally and want to be treated the same way.
But it came to nothing at all. The West said: `All right, guys, you have
certain limits you can come to, but those are your limits and you may
not exceed them.' This is arrogant, to say the least. We know exactly
what's good and bad for us. We totally comply with the international
law. That is solid and indisputable. The rest of it is subject for
negotiations.'' Our interview was polite, friendly even, like two people
genuinely attempting to communicate. The issue of whether Russia had
broken international law in Crimea was one of the few topics where we
were completely stuck, as if we were discussing two different realities.

Over all, the diplomat seemed earnestly baffled when I told him
Americans believed Putin had a master plan he was slyly executing. And
on this point, I didn't disagree. It didn't seem to me that Russia was
pushing a grand strategy so much as responding to opportunities in order
to do exactly what Baykov said the country would: ``to be an autonomous
player, to uphold its identity of a great power which is strategically
independent.'' If we look at the world through Russian eyes, the plan is
working, but it isn't the plan we thought it was. Russia did not break
the back of the international world order, as much as it recognized the
opportunities created by American withdrawal and the new era of global
\emph{bardak}.

Advertisement

\protect\hyperlink{after-bottom}{Continue reading the main story}

\hypertarget{site-index}{%
\subsection{Site Index}\label{site-index}}

\hypertarget{site-information-navigation}{%
\subsection{Site Information
Navigation}\label{site-information-navigation}}

\begin{itemize}
\tightlist
\item
  \href{https://help.nytimes3xbfgragh.onion/hc/en-us/articles/115014792127-Copyright-notice}{©~2020~The
  New York Times Company}
\end{itemize}

\begin{itemize}
\tightlist
\item
  \href{https://www.nytco.com/}{NYTCo}
\item
  \href{https://help.nytimes3xbfgragh.onion/hc/en-us/articles/115015385887-Contact-Us}{Contact
  Us}
\item
  \href{https://www.nytco.com/careers/}{Work with us}
\item
  \href{https://nytmediakit.com/}{Advertise}
\item
  \href{http://www.tbrandstudio.com/}{T Brand Studio}
\item
  \href{https://www.nytimes3xbfgragh.onion/privacy/cookie-policy\#how-do-i-manage-trackers}{Your
  Ad Choices}
\item
  \href{https://www.nytimes3xbfgragh.onion/privacy}{Privacy}
\item
  \href{https://help.nytimes3xbfgragh.onion/hc/en-us/articles/115014893428-Terms-of-service}{Terms
  of Service}
\item
  \href{https://help.nytimes3xbfgragh.onion/hc/en-us/articles/115014893968-Terms-of-sale}{Terms
  of Sale}
\item
  \href{https://spiderbites.nytimes3xbfgragh.onion}{Site Map}
\item
  \href{https://help.nytimes3xbfgragh.onion/hc/en-us}{Help}
\item
  \href{https://www.nytimes3xbfgragh.onion/subscription?campaignId=37WXW}{Subscriptions}
\end{itemize}
