Sections

SEARCH

\protect\hyperlink{site-content}{Skip to
content}\protect\hyperlink{site-index}{Skip to site index}

\href{https://myaccount.nytimes3xbfgragh.onion/auth/login?response_type=cookie\&client_id=vi}{}

\href{https://www.nytimes3xbfgragh.onion/section/todayspaper}{Today's
Paper}

Biden's Gaffes, and Why We Hate Politicians Talking About Us Behind Our
Backs

\url{https://nyti.ms/31Ualms}

\begin{itemize}
\item
\item
\item
\item
\item
\item
\end{itemize}

\begin{itemize}
\item
  \href{https://www.nytimes3xbfgragh.onion/live/2020/09/08/us/trump-vs-biden?action=click\&pgtype=Article\&state=default\&region=TOP_BANNER\&context=storylines_menu}{Election
  Updates}
\item
  \href{https://www.nytimes3xbfgragh.onion/interactive/2020/us/elections/election-states-biden-trump.html?action=click\&pgtype=Article\&state=default\&region=TOP_BANNER\&context=storylines_menu}{Paths
  to 270}
\item
  \href{https://www.nytimes3xbfgragh.onion/interactive/2020/08/31/us/politics/vote-by-mail-deadlines.html?action=click\&pgtype=Article\&state=default\&region=TOP_BANNER\&context=storylines_menu}{Voting
  by Mail}
\item
  \href{https://www.nytimes3xbfgragh.onion/interactive/2019/us/elections/2020-presidential-election-calendar.html?action=click\&pgtype=Article\&state=default\&region=TOP_BANNER\&context=storylines_menu}{Key
  Dates}
\item
  \href{https://www.nytimes3xbfgragh.onion/newsletters/politics?action=click\&pgtype=Article\&state=default\&region=TOP_BANNER\&context=storylines_menu}{Politics
  Newsletter}
\end{itemize}

Advertisement

\protect\hyperlink{after-top}{Continue reading the main story}

Supported by

\protect\hyperlink{after-sponsor}{Continue reading the main story}

\href{/column/screenland}{Screenland}

\hypertarget{bidens-gaffes-and-why-we-hate-politicians-talking-about-us-behind-our-backs}{%
\section{Biden's Gaffes, and Why We Hate Politicians Talking About Us
Behind Our
Backs}\label{bidens-gaffes-and-why-we-hate-politicians-talking-about-us-behind-our-backs}}

\includegraphics{https://static01.graylady3jvrrxbe.onion/images/2019/06/30/magazine/30mag-screenland-1/ff7e3317fe8d40ed9581afa25e920411-articleLarge.jpg?quality=75\&auto=webp\&disable=upscale}

By \href{https://www.nytimes3xbfgragh.onion/by/john-herrman}{John
Herrman}

\begin{itemize}
\item
  June 26, 2019
\item
  \begin{itemize}
  \item
  \item
  \item
  \item
  \item
  \item
  \end{itemize}
\end{itemize}

Joe Biden is suited, floating in a black void of stage, sitting
cross-legged across from a glamorous woman in a striking red jacket,
speaking to an invisible crowd. ``The younger generation now tells me
how tough things are,'' he
\href{https://www.latimes.com/politics/95641832-132.html}{says}. ``Give
me a break. No, no, I have no empathy for it. Give me a break. Because
here's the deal, guys: We decided we were going to change the world. And
we did.'' He delivers the line in high avuncular style, stating an
opinion with the tone and cadence of a meaningful story. The crowd,
whoever they are, laughs. They're not the ``guys'' being told what the
``deal'' is. Those people are somewhere else.

The clip is from 2018. Seen in context and with maximum generosity,
Biden seems to be saying that there have been riskier times to agitate
for change --- he mentions Kent State --- and that people did it anyway.
But to many members of ``the younger generation,'' who circulate this
quote whenever Biden says something they see as dismissive, this is an
out-of-touch Biden caught red-handed, demeaning them in their absence.
Their cohort, they point out, is quantitatively worse off than its
parents were, facing down not just uncertainty about the future but, in
2019, a raging gerontocracy doing everything in its considerable power
to deny them influence.

Who really had it tougher is almost beside the point. It's the feeling
of being talked \emph{about}, rather than talked to, that is already
emerging as a powerful force in this election. To the wrong viewer, that
Biden quote is damning, the kind of thing that should derail a campaign.
The kind of thing that \emph{has}. Remember Mitt Romney in 2012?
\href{https://www.motherjones.com/politics/2012/09/secret-video-romney-private-fundraiser/}{The
candidate was recorded speaking to ``wealthy contributors''} about his
campaign against Barack Obama. ``There are 47 percent of the people who
will vote for the president no matter what,'' he said. ``All right,
there are 47 percent who are with him, who are dependent upon
government, who believe that they are victims, who believe the
government has a responsibility to care for them, who believe that they
are entitled to health care, to food, to housing, to you-name-it.'' (A
head in the front row nodded.) ``My job is not to worry about those
people. I'll never convince them that they should take personal
responsibility and care for their lives.''

It was instantly clear, even to those who may have sympathized with
Romney, that the emergence of this video, filmed by a bartender at the
event, was a disaster. It aligned with a core Democratic message about
Romney, which was that he was more in touch with the ultrarich than with
regular people. What made that message singularly effective wasn't
Romney's recitation of what was, at the time, Republican boilerplate ---
it was where he recited it and to whom. Peeking behind closed doors,
millions of voters saw an anxiety dream on film: \emph{them} talking to
\emph{one another} about \emph{us}.

In the 2018 clip, however, Biden was not speaking at a rich donor's home
or flattering attendees at a fund-raiser. He was onstage, at an event
open to the public, promoting a book. His quote didn't have to be
smuggled out by an angry bartender to assume an illicit quality. It only
had to move from the official event video to tweets --- from its
intended audience to one that wouldn't normally watch an ``Ideas
Exchange'' talk whose tickets start at \$60. In the Romney video, voters
who suspected that the powerful privately disdained them had rare and
valuable evidence. In the Biden video --- and countless more like it,
from across the political spectrum, multiplying across social media ---
voters see something even more crazy-making: them, talking about us
behind our backs, right in front of our faces.

To engage with politics in 2019, it turns out, is to be able to access
the thrill and humiliation of seeing people talk about you as if you're
not there, whenever you want. Every Twitter feed but your own is a
secret room with a very thin door. Every Facebook argument can also be a
show for a much larger audience. Every old-media outlet is either for
you or about the problem of you. This sensation has long been imposed on
those poorly served by the powerful or the media that covers them. Now
it can be enjoyed by everyone, on demand.

A private conversation made public has power, and genuine snippets of
eavesdropping remain common enough in politics, in ever-evolving forms:
a cache of email conversations about running a campaign, a chat-room
transcript revealing plans for violence at a rally, a hot-mic recording
of a future presidential candidate and an ``Access Hollywood'' host
discussing what they're entitled to do to women. But what happens when
``public'' conversations are so numerous, and so undifferentiated, that
they start to seem, to the people in them, almost private? We're shown
such conversations constantly now, pulled from the deafening backdrop of
media and held up as shocking proof of what's really happening just
outside our awareness. Biden is a constant source of this, habitually
trying to level with the audience in front of him --- say, reminding
people about his record of reaching across the aisle, even, long ago, to
open segregationists --- in ways that will be profoundly alienating
anywhere else.

This process can also be orchestrated, as often happens with the
voluminous content created by Representative Alexandria Ocasio-Cortez.
Her social-media videos --- made for public consumption --- are
routinely seized on by the right with the urgency of hidden-camera
scoops, glimpses of what she secretly wants and believes. This sort of
energizing context-switch can even be recycled. In November 2018, a
\href{https://video.foxnews.com/v/5966650393001/\#sp=show-clips}{segment}
by the Fox News host Laura Ingraham presented a list of ``WACKY NEW
DEMOCRATIC IDEAS,'' followed by recordings of four new congresswomen,
including Ocasio-Cortez, publicly espousing their positions. This, in
turn, was turned into a
\href{https://twitter.com/AOC/status/1062731138031120384?ref_src=twsrc\%5Etfw\%7Ctwcamp\%5Etweetembed\%7Ctwterm\%5E1062731138031120384\&ref_url=https\%3A\%2F\%2Fwww.commondreams.org\%2Fnews\%2F2018\%2F11\%2F14\%2Ffox-news-once-again-forgets-radical-new-democratic-ideas-are-really-popular}{viral
Twitter post} by Ocasio-Cortez, who captioned a screenshot: ``Oh no!
They discovered our vast conspiracy to take care of children and save
the planet.''

Two popular critiques of social media --- that it sorts people into
bubbles and that it fosters constant conflict --- can feel as though
they're at odds with each other. But they are easily reconciled by
considering this kind of eavesdropping. People may spend their time
online among presorted clans, but the platforms that contain those
groups also make certain that they're constantly aware of one another,
and always able to look with horror or mockery at what everyone else is
up to. A nation constantly peering at itself through a peephole camera
is not, strictly, a nation more divided. If social media democratized
certain forms of speech, it has also made universal the anxious,
helpless sensation of listening to voices that don't represent you but
seem perfectly comfortable discussing you --- what you want, what you
need, what must be done for you, what must be done about you.

\hypertarget{our-2020-election-guide}{%
\section{Our 2020 Election Guide}\label{our-2020-election-guide}}

Updated ~Sept. 8, 2020

\begin{center}\rule{0.5\linewidth}{\linethickness}\end{center}

\begin{itemize}
\item ~
  \hypertarget{the-latest}{%
  \subsection{The Latest}\label{the-latest}}

  \begin{itemize}
  \item
    The campaign
    \href{https://www.nytimes3xbfgragh.onion/live/2020/09/08/us/trump-vs-biden?action=click\&pgtype=Article\&state=default\&region=BELOW_MAIN_CONTENT\&context=storylines_guide}{shifts
    to a higher gear this week}, with President Trump set to visit
    Florida and North Carolina today and Joseph R. Biden heading to
    Michigan tomorrow.
  \end{itemize}
\item ~
  \hypertarget{how-to-win-270}{%
  \subsection{How to Win 270}\label{how-to-win-270}}

  \begin{itemize}
  \item
    Joe Biden and Donald Trump need 270 electoral votes to reach the
    White House. Try building
    \href{https://www.nytimes3xbfgragh.onion/interactive/2020/us/elections/election-states-biden-trump.html?action=click\&pgtype=Article\&state=default\&region=BELOW_MAIN_CONTENT\&context=storylines_guide}{your
    own coalition of battleground states}~to see potential outcomes.
  \end{itemize}
\item ~
  \hypertarget{voting-by-mail}{%
  \subsection{Voting by Mail}\label{voting-by-mail}}

  \begin{itemize}
  \item
    Will you have enough time to vote by mail in your state? Yes, but
    it's risky to procrastinate.
    \href{https://www.nytimes3xbfgragh.onion/interactive/2020/08/31/us/politics/vote-by-mail-deadlines.html?action=click\&pgtype=Article\&state=default\&region=BELOW_MAIN_CONTENT\&context=storylines_guide}{Check
    your state's deadline.}
  \item
    \href{https://www.nytimes3xbfgragh.onion/interactive/2020/us/elections/joe-biden.html?action=click\&pgtype=Article\&state=default\&region=BELOW_MAIN_CONTENT\&context=storylines_guide}{}

    \hypertarget{joe-biden}{%
    \section{Joe Biden}\label{joe-biden}}

    \hypertarget{democrat}{%
    \subsection{Democrat}\label{democrat}}

    \href{https://www.nytimes3xbfgragh.onion/interactive/2020/us/elections/donald-trump.html?action=click\&pgtype=Article\&state=default\&region=BELOW_MAIN_CONTENT\&context=storylines_guide}{}

    \hypertarget{donald-trump}{%
    \section{Donald Trump}\label{donald-trump}}

    \hypertarget{republican}{%
    \subsection{Republican}\label{republican}}
  \end{itemize}
\item
  \hypertarget{keep-up-with-our-coverage}{%
  \subsection{Keep Up With Our
  Coverage}\label{keep-up-with-our-coverage}}

  \begin{itemize}
  \item
    Get an
    \href{https://www.nytimes3xbfgragh.onion/newsletters/politics?action=click\&pgtype=Article\&state=default\&region=BELOW_MAIN_CONTENT\&context=storylines_guide}{email}~recapping
    the day's news
  \item
    Download our mobile app on
    \href{https://apps.apple.com/us/app/nytimes/id284862083?ls=1\&mat_click_id=5c79ae7455014fd1bd66b5610c05b8f2-20191112-16948\&referrer=mat_click_id\%3D5c79ae7455014fd1bd66b5610c05b8f2-20191112-16948\%26link_click_id\%3D722930677036718082}{iOS}~and
    \href{http://a.localytics.com/android?id=com.nytimes.android\&referrer=utm_source\%3Dother_nyt_mobile_web\%26utm_medium\%3DWeb\%2520page\%26utm_term\%3DGeneral\%2520Mobile\%2520Page\%26utm_campaign\%3DNYT\%2520Mobile\%2520General\%2520Page}{Android}~and
    turn on Breaking News and Politics alerts
  \end{itemize}
\end{itemize}

Advertisement

\protect\hyperlink{after-bottom}{Continue reading the main story}

\hypertarget{site-index}{%
\subsection{Site Index}\label{site-index}}

\hypertarget{site-information-navigation}{%
\subsection{Site Information
Navigation}\label{site-information-navigation}}

\begin{itemize}
\tightlist
\item
  \href{https://help.nytimes3xbfgragh.onion/hc/en-us/articles/115014792127-Copyright-notice}{©~2020~The
  New York Times Company}
\end{itemize}

\begin{itemize}
\tightlist
\item
  \href{https://www.nytco.com/}{NYTCo}
\item
  \href{https://help.nytimes3xbfgragh.onion/hc/en-us/articles/115015385887-Contact-Us}{Contact
  Us}
\item
  \href{https://www.nytco.com/careers/}{Work with us}
\item
  \href{https://nytmediakit.com/}{Advertise}
\item
  \href{http://www.tbrandstudio.com/}{T Brand Studio}
\item
  \href{https://www.nytimes3xbfgragh.onion/privacy/cookie-policy\#how-do-i-manage-trackers}{Your
  Ad Choices}
\item
  \href{https://www.nytimes3xbfgragh.onion/privacy}{Privacy}
\item
  \href{https://help.nytimes3xbfgragh.onion/hc/en-us/articles/115014893428-Terms-of-service}{Terms
  of Service}
\item
  \href{https://help.nytimes3xbfgragh.onion/hc/en-us/articles/115014893968-Terms-of-sale}{Terms
  of Sale}
\item
  \href{https://spiderbites.nytimes3xbfgragh.onion}{Site Map}
\item
  \href{https://help.nytimes3xbfgragh.onion/hc/en-us}{Help}
\item
  \href{https://www.nytimes3xbfgragh.onion/subscription?campaignId=37WXW}{Subscriptions}
\end{itemize}
