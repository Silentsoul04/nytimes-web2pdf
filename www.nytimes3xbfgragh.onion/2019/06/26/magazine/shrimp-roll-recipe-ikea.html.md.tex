Sections

SEARCH

\protect\hyperlink{site-content}{Skip to
content}\protect\hyperlink{site-index}{Skip to site index}

\href{https://myaccount.nytimes3xbfgragh.onion/auth/login?response_type=cookie\&client_id=vi}{}

\href{https://www.nytimes3xbfgragh.onion/section/todayspaper}{Today's
Paper}

A Shrimp Roll Inspired by Ikea

\url{https://nyti.ms/2FzhFKs}

\begin{itemize}
\item
\item
\item
\item
\item
\end{itemize}

Advertisement

\protect\hyperlink{after-top}{Continue reading the main story}

Supported by

\protect\hyperlink{after-sponsor}{Continue reading the main story}

\href{/column/magazine-eat}{Eat}

\hypertarget{a-shrimp-roll-inspired-by-ikea}{%
\section{A Shrimp Roll Inspired by
Ikea}\label{a-shrimp-roll-inspired-by-ikea}}

\includegraphics{https://static01.graylady3jvrrxbe.onion/images/2019/06/30/magazine/30mag-eat/48bcfb92f92f4965a83b73fa9ad61f13-articleLarge.jpg?quality=75\&auto=webp\&disable=upscale}

By Gabrielle Hamilton

\begin{itemize}
\item
  June 26, 2019
\item
  \begin{itemize}
  \item
  \item
  \item
  \item
  \item
  \end{itemize}
\end{itemize}

I still briefly flounder when I meet someone --- at the dog run, at the
laundry or in line at the bodega --- and as the conversation turns to
work and I say I have a restaurant, they ask: What kind of restaurant is
it?

The question strikes me as anachronistic, like when you have to fill out
those forms at the doctor's office and they ask you for your three phone
numbers: daytime, evening, office.

And my answer even more so, given that we now have restaurants of such
micro-specificity as ``Isan cooking of northeastern Thailand.''
``Creative American?'' I offer with a wobbly question mark, falling back
on the neatly supplied category it was given by the Zagat guide after we
opened in 1999; 20 years ago that was a statement meant to distinguish
you in a landscape of Italian or French.

We serve a pretty straightforward shrimp roll at lunch right now, but we
are not by any stretch a seafood restaurant. It comes with a side of
French fries, but we are as far away from a Maine-inflected lobster
shack as you can get. Still, ``creative American'' feels vague and
inadequate. Other times, more confidently, I've said, ``It's tiny, very
small.'' Hoping to sum it up, I've also tried: ``I'm a salt-and-pepper
cook. It's all olive oil, parsley and lemon.''

I've found it helps a lot to just name the actual menu items:
``Sometimes we run a whole grilled fish with toasted fennel oil, and in
the winter, braised lamb shoulder with preserved lemons. When we can get
it, monkfish liver on buttered toast. Same with the rabbit kidneys.''

If there's time and the conversation gets going, I have sometimes
explained: ``It's a personal restaurant. The food reflects a lot of my
own experiences and appetites.'' But I take extreme care with that,
since lately the ``narrative'' aspect of eating out in a restaurant can
often take an absurd turn, with the waiter standing there explaining the
menu to you. ``So, every dish here tells a story,'' he begins, taking
you hostage, and you immediately start looking for a magic getaway taxi
to pull up in front honking for you. As Anton Chekhov said: Don't tell
me the moon is shining; show me the glint of light on the broken glass.

Take that shrimp roll, for instance. We use rock shrimp --- a kind of
``poor man's lobster'' --- and there's a sliced hard-boiled egg shingled
on top with crosswise sliced coins of bitter endive. I love the clean
crunch of that endive so much. The bun is griddled, not just on the
outside but split and griddled on the inside too --- getting the most
possible surface area of the best part of the bun: the warm, sweet,
buttery part. We pile the salad --- plenty of mayonnaise, plenty of rock
shrimp, very little onion and celery --- onto the bun so generously that
this could be a fork-and-knife deal.

The rock shrimp --- so called as their shells are rock-hard, unlike the
brown, pink and tiger varieties, whose shells more closely resemble
flimsy plastic --- have sweet meats and a texture that resembles
uncannily the tail meat of lobsters. They come from deep cold waters in
Florida and the harvest practices are monitored --- no coral damage, no
overfishing --- and the shrimp are happily affordable.

\includegraphics{https://static01.graylady3jvrrxbe.onion/images/2019/06/30/magazine/30mag-eat-02/48bcfb92f92f4965a83b73fa9ad61f13-1-articleLarge.jpg?quality=75\&auto=webp\&disable=upscale}

If there's one sure way I've always described the restaurant, if there's
one constant through line to the story here, it's about being thrifty.
The restaurant has always had expensive tastes but modest means. I've
spent my career making excellent use of the low cuts, the discards, the
crumbs, the castoffs. Our roasted marrow bones used to be sent by the
butcher free --- for our dogs, he thought! Now they are \$3.95 a pound.

While the negroni we serve comes from that late-afternoon piazza in
Rome, with all the Italians sitting around in their cashmere sweaters
and suede loafers, and the Calvados omelet on our dessert menu comes
from a summer trip to France, this rock-shrimp roll comes, in a way,
from the Ikea in Brooklyn. There's a bike ride I loved to take all the
way out to the maritime wonder that is Red Hook, Brooklyn --- not for a
bookcase or a bed frame or a stylish affordable pendant lamp --- but
just to go to the cafeteria and get the open-faced shrimp-salad sandwich
on rye bread and then to sit on a bench just past the parking lot,
looking out at the industrial mouth of New York Harbor, those
mesmerizing colossal ocean liners and oil tankers and cargo ships being
tugged in and out. But that back story need never make its way to the
table. The piled-up rock-shrimp roll speaks for itself.

\textbf{Recipe:}
\href{https://cooking.nytimes3xbfgragh.onion/recipes/1020322-rock-shrimp-roll}{Rock-shrimp
roll}

Advertisement

\protect\hyperlink{after-bottom}{Continue reading the main story}

\hypertarget{site-index}{%
\subsection{Site Index}\label{site-index}}

\hypertarget{site-information-navigation}{%
\subsection{Site Information
Navigation}\label{site-information-navigation}}

\begin{itemize}
\tightlist
\item
  \href{https://help.nytimes3xbfgragh.onion/hc/en-us/articles/115014792127-Copyright-notice}{©~2020~The
  New York Times Company}
\end{itemize}

\begin{itemize}
\tightlist
\item
  \href{https://www.nytco.com/}{NYTCo}
\item
  \href{https://help.nytimes3xbfgragh.onion/hc/en-us/articles/115015385887-Contact-Us}{Contact
  Us}
\item
  \href{https://www.nytco.com/careers/}{Work with us}
\item
  \href{https://nytmediakit.com/}{Advertise}
\item
  \href{http://www.tbrandstudio.com/}{T Brand Studio}
\item
  \href{https://www.nytimes3xbfgragh.onion/privacy/cookie-policy\#how-do-i-manage-trackers}{Your
  Ad Choices}
\item
  \href{https://www.nytimes3xbfgragh.onion/privacy}{Privacy}
\item
  \href{https://help.nytimes3xbfgragh.onion/hc/en-us/articles/115014893428-Terms-of-service}{Terms
  of Service}
\item
  \href{https://help.nytimes3xbfgragh.onion/hc/en-us/articles/115014893968-Terms-of-sale}{Terms
  of Sale}
\item
  \href{https://spiderbites.nytimes3xbfgragh.onion}{Site Map}
\item
  \href{https://help.nytimes3xbfgragh.onion/hc/en-us}{Help}
\item
  \href{https://www.nytimes3xbfgragh.onion/subscription?campaignId=37WXW}{Subscriptions}
\end{itemize}
