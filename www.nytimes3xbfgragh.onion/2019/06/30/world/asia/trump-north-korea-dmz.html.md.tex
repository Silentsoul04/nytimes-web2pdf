Sections

SEARCH

\protect\hyperlink{site-content}{Skip to
content}\protect\hyperlink{site-index}{Skip to site index}

\href{https://www.nytimes3xbfgragh.onion/section/world/asia}{Asia
Pacific}

\href{https://myaccount.nytimes3xbfgragh.onion/auth/login?response_type=cookie\&client_id=vi}{}

\href{https://www.nytimes3xbfgragh.onion/section/todayspaper}{Today's
Paper}

\href{/section/world/asia}{Asia Pacific}\textbar{}Trump Steps Into North
Korea and Agrees With Kim Jong-un to Resume Talks

\url{https://nyti.ms/2Xd8VQi}

\begin{itemize}
\item
\item
\item
\item
\item
\item
\end{itemize}

Advertisement

\protect\hyperlink{after-top}{Continue reading the main story}

Supported by

\protect\hyperlink{after-sponsor}{Continue reading the main story}

\hypertarget{trump-steps-into-north-korea-and-agrees-with-kim-jong-un-to-resume-talks}{%
\section{Trump Steps Into North Korea and Agrees With Kim Jong-un to
Resume
Talks}\label{trump-steps-into-north-korea-and-agrees-with-kim-jong-un-to-resume-talks}}

\includegraphics{https://static01.graylady3jvrrxbe.onion/images/2019/07/01/world/30prexy04/30prexy04-videoSixteenByNine3000.jpg}

By \href{https://www.nytimes3xbfgragh.onion/by/peter-baker}{Peter Baker}
and \href{https://www.nytimes3xbfgragh.onion/by/michael-crowley}{Michael
Crowley}

\begin{itemize}
\item
  June 30, 2019
\item
  \begin{itemize}
  \item
  \item
  \item
  \item
  \item
  \item
  \end{itemize}
\end{itemize}

SEOUL, South Korea --- President Trump on Sunday became the first
sitting American commander in chief to set foot in North Korea as he met
Kim Jong-un, the country's leader, at the heavily fortified
Demilitarized Zone, and the two agreed to restart negotiations on a
long-elusive nuclear agreement.

Greeted by a beaming Mr. Kim, the president stepped across a low
concrete border marker at 3:46 p.m. local time and walked 20 paces to
the base of a building on the North Korean side for an unprecedented,
camera-friendly demonstration of friendship intended to revitalize
stalled talks.

``It is good to see you again,'' an exuberant Mr. Kim told the president
through an interpreter. ``I never expected to meet you in this place.''

``Big moment, big moment,'' Mr. Trump told him.

After about a minute on officially hostile territory, Mr. Trump escorted
Mr. Kim back over the line into South Korea, where the two briefly
addressed a scrum of journalists before slipping inside the building
known as Freedom House for a private conversation along with President
Moon Jae-in of South Korea. Mr. Trump said he would invite Mr. Kim to
visit him at the White House.

``This has a lot of significance because it means that we want to bring
an end to the unpleasant past and try to create a new future,'' Mr. Kim
told reporters. ``So it's a very courageous and determined act.''

``Stepping across that line was a great honor,'' Mr. Trump replied. ``A
lot of progress has been made, a lot of friendships have been made, and
this has been in particular a great friendship.''

A showman by nature and past profession, Mr. Trump delighted in the
drama of the moment, which he had arranged with
\href{https://www.nytimes3xbfgragh.onion/2019/06/29/world/asia/trump-kim-dmz.html}{a
surprise invitation via Twitter}barely 24 hours earlier. Never before
had American and North Korean leaders gotten together at the military
demarcation line, where heavily armed forces have faced off across a
tense divide for 66 years since the end of fighting in the Korean War.

\includegraphics{https://static01.graylady3jvrrxbe.onion/images/2019/06/30/world/30prexy05/20f650c54b6c448a9a8c73a5f3c3221d-articleLarge.jpg?quality=75\&auto=webp\&disable=upscale}

The encounter in Panmunjom had been cast as a brief handshake, not a
formal negotiation, but the two ended up together for a little more than
an hour. After emerging from their conversation, Mr. Trump said he and
Mr. Kim had agreed to designate negotiators to resume talks in the next
few weeks, four months after they collapsed at a summit meeting in
Hanoi, Vietnam.

The American team will still be headed by Stephen Biegun, the special
envoy, but it remained unclear who would be on the North Korean side
after
\href{https://www.nytimes3xbfgragh.onion/2019/06/02/world/asia/north-korean-purge-kim.html}{reports
of a purge of Mr. Kim's team}. Asked later if North Korean negotiators
were still alive, Mr. Trump said: ``I think they are. I can tell you who
the main person is. And I would hope the rest are, too.''

Mr. Trump was already scheduled to make an unannounced visit to the DMZ
during his trip to South Korea, and he portrayed the idea of meeting Mr.
Kim as a spontaneous one, although he had been musing out loud about it
for days. Still, it caught even his own staff by surprise and forced an
extraordinary scramble to arrange logistics and security, a task that
would ordinarily take days or weeks.

Mr. Trump gambled that the show of amity could crack the nuclear logjam,
underscoring his faith in the power of his own personal diplomacy ---
even with brutal strongmen like Mr. Kim --- to achieve what past
presidents could not. More than halfway through his term, Mr. Trump is
eager for a resolution to the longstanding dispute, seeing it as a
signature element of the legacy he hopes to forge and a potential boost
to his re-election campaign.

Even in this symbolic moment of reconciliation, Mr. Trump seemed to
toggle back and forth between glory and grievance, reveling one minute
in the history of the day and then the next griping that he was not
getting enough credit for reducing friction with North Korea.

He seemed acutely defensive about criticism that he has yet to reach an
agreement to eliminate North Korea's nuclear arsenal despite summit
meetings with Mr. Kim in Singapore in June 2018 and in Hanoi in
February. Almost every time he saw a microphone, he complained that his
achievement had not been appreciated.

``There was great conflict here prior to our meeting in Singapore,'' he
said. ``Tremendous conflict and death all around them. And it's now been
extremely peaceful. It's been a whole different world.''

``That wouldn't necessarily have been reported, but they understand it
very well,'' he added**,** referring to the news media. ``I keep saying
that for the people who say nothing has been accomplished. So much has
been accomplished.''

Image

Mr. Trump said he and Mr. Kim had agreed to designate negotiators to
resume conversations in the next few weeks.Credit...Erin Schaff/The New
York Times

Since Mr. Trump took office, North Korea has suspended nuclear tests,
released detained Americans and sent back to the United States the
remains of some American soldiers killed in the war. South Korean
officials and others in the region say tension has eased significantly,
and Mr. Moon praised Mr. Trump as ``the peacemaker of the Korean
Peninsula.''

But American intelligence agencies have concluded that North Korea ``is
unlikely'' to give up its nuclear arsenal, as Mr. Trump has demanded,
and even amid the rapprochement with the president, the North has
produced enough fuel for a half-dozen additional nuclear weapons,
\href{https://www.reuters.com/article/us-northkorea-usa-nuclear-study/north-korea-may-have-made-more-nuclear-bombs-but-threat-reduced-study-idUSKCN1Q10EL}{according
to one study}. In May, it
\href{https://www.nytimes3xbfgragh.onion/2019/05/04/world/asia/north-korea-missiles-trump.html?module=inline}{launched
short-range missiles} in violation of United Nations resolutions.

Critics called the DMZ greeting an overhyped photo opportunity by a
president who himself ratcheted up the conflict with North Korea in his
first year in office by vowing to unleash ``fire and fury'' if it
threatened American security.

``Today is a victory for South Korea's middle-power diplomacy and
President Moon's peace agenda,'' said Leif-Eric Easley, associate
professor of international studies at Ewha Womans University in Seoul.
``But tomorrow, North Korea will still have nuclear weapons, and the
U.S. will still maintain sanctions.''

\href{https://www.nytimes3xbfgragh.onion/2018/06/12/world/asia/north-korea-summit.html}{Mr.
Trump's meeting with Mr. Kim in Singapore} was the first time sitting
American and North Korean leaders had met anywhere, and it produced
vague promises to eliminate Pyongyang's nuclear arsenal. Their second
meeting, in Hanoi,
\href{https://www.nytimes3xbfgragh.onion/2019/02/28/world/asia/trump-kim-vietnam-summit.html?module=inline}{ended
in failure} when Mr. Kim made an offer that fell far short of that.

North Korean officials went dark after the collapse of the talks,
refusing to respond to either the Americans or the South Koreans. In
recent weeks, however, North Korea re-emerged on the world stage as Mr.
Kim
\href{https://www.nytimes3xbfgragh.onion/2019/06/22/world/asia/north-korea-letter.html}{exchanged
letters} with Mr. Trump in what was seen as a signal of its interest in
resuming diplomacy.

American officials have said they did not think a third meeting should
be arranged unless a substantive agreement could be negotiated
beforehand. But Mr. Trump was seized with the idea of an encounter at
the DMZ.

Panmunjom, which straddles the North-South border, is called the ``truce
village'' because the two sides signed an armistice there in 1953 to
halt the three-year Korean War. The two-and-a-half-mile-wide DMZ is a
no-man's zone, but Panmunjom is an exception, a ``joint security area''
where border guards face off with no buffer between them.

Image

The encounter in the truce village of Panmunjom was cast as a brief
greeting, not a formal negotiation, but the two leaders ended up
together for a little more than an hour.Credit...Erin Schaff/The New
York Times

Mr. Kim
\href{https://www.nytimes3xbfgragh.onion/2018/04/26/world/asia/korea-kim-moon-summit.html}{crossed
the DMZ} in April 2018 to meet with Mr. Moon, becoming the first North
Korean leader to step over the line since the war. Former Presidents
Jimmy Carter and Bill Clinton each visited North Korea, flying into its
capital, Pyongyang, but only after they left office. Sitting presidents,
including Ronald Reagan, Mr. Clinton, George W. Bush and Barack Obama,
visited the DMZ, but were never greeted by North Korea's leader.

Mr. Trump, wearing a dark suit, emerged from Freedom House on the South
Korean side, walked along gravel between two blue huts to the
demarcation line and stopped there to wait for Mr. Kim to approach. Mr.
Kim, wearing his traditional Mao suit, bounded forward to greet him.

They shook hands and Mr. Trump patted the younger man's arm before they
stepped across the barrier and strode across a dirt field. The two
turned and shook hands again for the cameras, then walked back to the
border marker, posed again, and finally headed toward Freedom House.

The scene was fairly chaotic. North Korean security personnel were
particularly aggressive, pushing and pulling journalists and even White
House staff members, including the new press secretary, Stephanie
Grisham, as she tried to get American media into the room. The jostling
made television images from the scene look frenzied.

Accompanying the president were Secretary of State Mike Pompeo, the
acting chief of staff, Mick Mulvaney, and other top aides, including
Ivanka Trump and Jared Kushner. Asked how North Korea was, the
president's daughter answered, ``Surreal.''

Mr. Kim said he knew nothing about a possible meeting until the
president's tweet. ``I don't think this kind of surprise meeting would
have happened without the excellent personal relationship between your
excellency and me,'' he told Mr. Trump in Freedom House.

Mr. Trump expressed relief that Mr. Kim came. ``If he didn't show up,
the press was going to make me look very bad,'' he said. ``So you made
us both look good, and I appreciate it.''

After their private conversation, which lasted about 50 minutes, Mr.
Trump escorted Mr. Kim back to the demarcation line and then watched as
the North Korean headed back to his country.

``Certainly, this was a great day; this was a very legendary, very
historic day,'' Mr. Trump exulted afterward, before adding a cautionary
note. ``It'll be even more historic if something comes out of it.''

Advertisement

\protect\hyperlink{after-bottom}{Continue reading the main story}

\hypertarget{site-index}{%
\subsection{Site Index}\label{site-index}}

\hypertarget{site-information-navigation}{%
\subsection{Site Information
Navigation}\label{site-information-navigation}}

\begin{itemize}
\tightlist
\item
  \href{https://help.nytimes3xbfgragh.onion/hc/en-us/articles/115014792127-Copyright-notice}{©~2020~The
  New York Times Company}
\end{itemize}

\begin{itemize}
\tightlist
\item
  \href{https://www.nytco.com/}{NYTCo}
\item
  \href{https://help.nytimes3xbfgragh.onion/hc/en-us/articles/115015385887-Contact-Us}{Contact
  Us}
\item
  \href{https://www.nytco.com/careers/}{Work with us}
\item
  \href{https://nytmediakit.com/}{Advertise}
\item
  \href{http://www.tbrandstudio.com/}{T Brand Studio}
\item
  \href{https://www.nytimes3xbfgragh.onion/privacy/cookie-policy\#how-do-i-manage-trackers}{Your
  Ad Choices}
\item
  \href{https://www.nytimes3xbfgragh.onion/privacy}{Privacy}
\item
  \href{https://help.nytimes3xbfgragh.onion/hc/en-us/articles/115014893428-Terms-of-service}{Terms
  of Service}
\item
  \href{https://help.nytimes3xbfgragh.onion/hc/en-us/articles/115014893968-Terms-of-sale}{Terms
  of Sale}
\item
  \href{https://spiderbites.nytimes3xbfgragh.onion}{Site Map}
\item
  \href{https://help.nytimes3xbfgragh.onion/hc/en-us}{Help}
\item
  \href{https://www.nytimes3xbfgragh.onion/subscription?campaignId=37WXW}{Subscriptions}
\end{itemize}
