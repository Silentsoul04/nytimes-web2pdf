Sections

SEARCH

\protect\hyperlink{site-content}{Skip to
content}\protect\hyperlink{site-index}{Skip to site index}

\href{https://www.nytimes3xbfgragh.onion/section/world/middleeast}{Middle
East}

\href{https://myaccount.nytimes3xbfgragh.onion/auth/login?response_type=cookie\&client_id=vi}{}

\href{https://www.nytimes3xbfgragh.onion/section/todayspaper}{Today's
Paper}

\href{/section/world/middleeast}{Middle East}\textbar{}To Evade
Sanctions on Iran, Ships Vanish in Plain Sight

\url{https://nyti.ms/2LBvcFr}

\begin{itemize}
\item
\item
\item
\item
\item
\item
\end{itemize}

Advertisement

\protect\hyperlink{after-top}{Continue reading the main story}

Supported by

\protect\hyperlink{after-sponsor}{Continue reading the main story}

\hypertarget{to-evade-sanctions-on-iran-ships-vanish-in-plain-sight}{%
\section{To Evade Sanctions on Iran, Ships Vanish in Plain
Sight}\label{to-evade-sanctions-on-iran-ships-vanish-in-plain-sight}}

\includegraphics{https://static01.graylady3jvrrxbe.onion/images/2019/06/28/multimedia/00ghostships-01/merlin_157170555_2aeb2b41-068d-41fd-9559-b208944a792f-articleLarge.jpg?quality=75\&auto=webp\&disable=upscale}

By \href{https://www.nytimes3xbfgragh.onion/by/michael-forsythe}{Michael
Forsythe} and
\href{https://www.nytimes3xbfgragh.onion/by/ronen-bergman}{Ronen
Bergman}

\begin{itemize}
\item
  July 2, 2019
\item
  \begin{itemize}
  \item
  \item
  \item
  \item
  \item
  \item
  \end{itemize}
\end{itemize}

A week ago, a small tanker ship approached the Persian Gulf after a
19-day voyage from China. The captain, as required by
\href{http://www.imo.org/en/OurWork/Safety/Navigation/Pages/AIS.aspx}{international
rules}, reported the ship's position, course, speed and another key
detail: It was riding high in the water, meaning it was probably empty.

Then the Chinese-owned ship, the Sino Energy 1, went silent and
essentially vanished from the grid.

It reported in again on Sunday, near the spot where it had vanished six
days earlier, only now it was heading east, away from the Strait of
Hormuz near Iran. If past patterns hold, the captain will soon report
that it is riding low in the water, meaning its tanks are most likely
full.

As the Trump administration's sanctions on Iranian oil and petrochemical
products have taken hold, some of the world's shipping fleets have
defied the restrictions by ``going dark'' when they pick up cargo in
Iranian ports, according to commercial analysts who track shipping data
and intelligence from authorities in Israel, a country that backs the
Trump crackdown.

{[}\href{https://www.nytimes3xbfgragh.onion/2019/07/01/world/middleeast/iran-uranium-enrichment-limit.html}{\emph{Iran
breached}}
\href{https://www.nytimes3xbfgragh.onion/2019/07/01/world/middleeast/iran-uranium-enrichment-limit.html}{\emph{a
nuclear fuel limit in what it said was a response to the reimposition of
sanctions by the Trump administration.}}{]}

``They are hiding their activity,'' said Samir Madani, co-founder of
TankerTrackers.com, a company that uses satellite imagery to identify
tankers calling on Iranian ports. ``They don't want to broadcast the
fact that they have been in Iran, evading sanctions. It's that simple.''

\includegraphics{https://static01.graylady3jvrrxbe.onion/images/2019/07/02/multimedia/02ghostships/02ghostships-articleLarge.jpg?quality=75\&auto=webp\&disable=upscale}

A
\href{http://www.imo.org/en/OurWork/Safety/Navigation/Pages/AIS.aspx}{maritime
treaty} overseen by a United Nations agency requires ships of 300 tons
or more that travel international routes to have an automatic
identification system. The gear helps avoid collisions and aids in
search-and-rescue operations. It also allows countries to monitor
shipping traffic.

It is not illegal under international law to buy and haul Iranian oil or
related products. The Trump administration's sanctions, which
\href{https://www.treasury.gov/resource-center/sanctions/programs/pages/iran.aspx}{went
into effect last November} after the United States pulled out of the
Iran nuclear agreement, are unilateral.

But foreign companies doing business with American companies or banks
risk being punished by the United States. Actions can include banning
American banks from working with them, freezing assets and barring
company officials from traveling to the United States, said
\href{https://energypolicy.columbia.edu/richard-nephew}{Richard Nephew},
a research scholar at Columbia University who oversaw Iran policy on the
National Security Council during the Obama administration.

``We have sanctioned dozens of Chinese state-owned enterprises for
nuclear, missile, arms and other forms of proliferation,'' Mr. Nephew
said. ``But it is not entered into lightly.''

A State Department spokeswoman said, ``We do not comment on intelligence
matters.''

\hypertarget{chinese-tankers-keep-disappearing-in-the-persian-gulf}{%
\subsection{Chinese Tankers Keep Disappearing in the Persian
Gulf}\label{chinese-tankers-keep-disappearing-in-the-persian-gulf}}

\includegraphics{https://static01.graylady3jvrrxbe.onion/newsgraphics/2019/06/24/sinochem/97a89ec8f05c8448c0af4867a1c32a2b282cfcd5/fallback/en/1.jpg}

The SC Mercury, an oil and chemical tanker owned by Sinochem until April
2019, sails regularly from Chinese ports into the Persian Gulf.

\includegraphics{https://static01.graylady3jvrrxbe.onion/newsgraphics/2019/06/24/sinochem/97a89ec8f05c8448c0af4867a1c32a2b282cfcd5/fallback/en/2.jpg}

On the morning of Jan. 27, 2018, it disappeared. The Mercury's A.I.S.
transponder --- a device that broadcasts a ship's location continuously,
required by an international maritime treaty --- fell silent.

Several days later, the transponder came back to life, tracking the
Mercury as it sailed toward ports in India. Having deposited its cargo,
it turned back toward the gulf.

On Feb. 15, 2018, the ship went dark again as it navigated the Strait of
Hormuz, reappearing days later to begin a weekslong journey back to
Shanghai.

All ships 300 tons or greater on international journeys are required to
broadcast their location, course and speed on the system, but sometimes,
to hide their activities from competitors, ships ``go dark,'' analysts
say.

\includegraphics{https://static01.graylady3jvrrxbe.onion/newsgraphics/2019/06/24/sinochem/97a89ec8f05c8448c0af4867a1c32a2b282cfcd5/fallback/en/3.jpg}

The Persian Gulf isn't the only place in the world where ships go
silent. It also happens in the South China Sea, but there, one analyst
said, the reason may be because the sheer number of ships overwhelms the
system.

\includegraphics{https://static01.graylady3jvrrxbe.onion/newsgraphics/2019/06/24/sinochem/97a89ec8f05c8448c0af4867a1c32a2b282cfcd5/fallback/en/4.jpg}

In the case of the Mercury, outages appeared to be more selective. In
April and May 2018, the ship's transponder stayed active as it visited
ports in Saudi Arabia, Bahrain and the United Arab Emirates.

When a ship goes dark in the Persian Gulf, it may be related to dodging
sanctions, not technical problems, said Samir Madani of
\href{https://tankertrackers.com/}{TankerTrackers.com}, which uses
satellite technology to monitor ships. Countries and companies that
import Iranian oil risk punishment from the United States.

\includegraphics{https://static01.graylady3jvrrxbe.onion/newsgraphics/2019/06/24/sinochem/97a89ec8f05c8448c0af4867a1c32a2b282cfcd5/fallback/en/5.jpg}

In the past 18 months, the five ships, which regularly sail between
China and the Persian Gulf, made only two port visits in Iran, according
to information from their A.I.S. data. In contrast, those ships made
close to 50 stops in ports in Bahrain, Oman, Saudi Arabia and the U.A.E.
In another 28 instances, the ships vanished in the region for days or
weeks.

By Rich Harris and Derek Watkins. Source: VesselsValue

debug 1458: waiting for message.......

Brian Hook, the United States special representative for Iran, told
reporters in London on Friday that the United States would punish any
country importing Iranian oil. Mr. Hook was responding to a question
about reports of Iranian oil going to Asia, according to
\href{https://www.reuters.com/article/us-mideast-iran-usa-diplomat-idUSKCN1TT1IW}{the
Reuters news agency}.

President Trump's efforts to halt Iranian oil and petrochemical exports
are at the heart of rising tensions between the two countries. Last
month, he imposed
\href{https://www.nytimes3xbfgragh.onion/2019/06/24/us/politics/iran-sanctions.html}{new
sanctions} on Iran's leaders after it downed an
\href{https://www.nytimes3xbfgragh.onion/interactive/2019/06/21/world/middleeast/map-us-iran-drone-attack.html}{American
surveillance drone} and nearly precipitated a counterstrike that was
\href{https://www.nytimes3xbfgragh.onion/2019/06/20/world/middleeast/iran-us-drone.html}{called
off} at the last minute. The attack on the drone came a week after the
United States accused Iran of being responsible for explosions that had
crippled two tankers near the Strait of Hormuz.

American and Israeli intelligence agencies say the country's Islamic
Revolutionary Guard Corps is deeply entwined with its petrochemical
industry, using oil revenues to swell its coffers. Mr. Trump has labeled
the military group a terrorist organization.

Iran has been trying to work around the American sanctions by offering
``significant reductions'' in price for its oil and petrochemical
products, said Gary Samore, a professor at Brandeis University who
worked on weapons issues in the Obama administration.

Image

Brian Hook, left, the United States special representative for Iran, has
said the American government would punish any country importing Iranian
oil.Credit...Yasser Al-Zayyat/Agence France-Presse --- Getty Images

When shipping companies defy the sanctions, they weaken their
effectiveness, especially if the companies --- or the countries where
they are based --- see no consequences, analysts said. Some shipping
companies with direct Iranian ties do not try to hide their movements,
according to data collected by the commercial tracking sites.

Last month, the Salina, an Iranian-flagged oil tanker under American
sanctions, docked in Jinzhou Bay, a port in northeastern China,
according to data from VesselsValue, a website that analyzes global
shipping information. The Salina regularly reported its position, course
and speed via the automatic identification system.

Oil tankers like the Salina, which can transport as much as a million
barrels of crude, or about 5 percent of the
\href{https://www.eia.gov/tools/faqs/faq.php?id=33\&t=6}{daily
consumption} of the United States, are so big that they can call on only
a limited number of ports. They are also more easily spotted by
satellites than smaller ships like the Sino Energy 1.

That vessel, and its more than 40 sister ships, are far more difficult
to track when they go off the grid. They were owned until April by a
subsidiary of Sinochem, a state-owned company in China that is one of
the world's biggest chemical manufacturers.

Sinochem has extensive business ties in the United States. It has an
office in Houston and works with big American companies including Boeing
and Exxon Mobil. In March, it
\href{http://www.sinochem.com/en/s/1569-4171-126554.html}{signed an
agreement} with Citibank to ``deepen the partnership'' between the two
companies, Sinochem said. In 2013, a United States subsidiary of
Sinochem bought a 40 percent stake in a Texas shale deposit for \$1.7
billion.

In April, it sold a controlling share in its shipping fleet to a private
company, Inner Mongolia Junzheng Energy \& Chemical Group Co., whose
\href{http://static.sse.com.cn/disclosure/listedinfo/announcement/c/2019-04-30/601216_2019_1.pdf}{biggest
shareholder} is Du Jiangtao, a Chinese billionaire who made his fortune
in medical equipment, chemicals and coal-generated power.

A person answering the phone at Junzheng's investor relations office was
not familiar with the newly acquired shipping business. For now,
Junzheng owns 40 percent of Sinochem's former shipping fleet, with the
rest owned by two Beijing companies.

Frank Ning, the chairman of Sinochem, speaking in a brief interview in
Dalian, China, said that shipping had not been central to the company's
business. In a statement, the company said it had ``adopted strict
compliance policies and governance on export control and sanctions,''
though a former employee who had helped manage the shipping business,
speaking on the condition of anonymity, said the company had shipped
petrochemicals from Iran for years.

The tracking data also show that some of the Sinochem ships made trips
to Iran before the fleet was sold, and both before and after the
American sanctions went into effect.

In April 2018, for example, one of the ships, the SC Brilliant, was
moored at Asalouyeh, a major Iranian petrochemical depot on the Persian
Gulf, according to data from VesselsValue. The SC Brilliant's voyage was
easy to plot. Its captain made constant reports via the automatic
identification system, broadcasting its course, speed and destination.

But after Mr. Trump's announcement last August that he would reimpose
sanctions on Iran's petroleum industry, the SC Brilliant's voyages
became less transparent.

In late September and early October, shortly before the sanctions took
effect, the ship went off the grid for 10 days in the same stretch of
the Strait of Hormuz where the Sino Energy 1 disappeared last week. When
the SC Brilliant went off the grid, it appeared empty; when it
re-emerged, it appeared full.

The pattern was repeated in February, with the ship disappearing for
four days, according to the tracking data.

That month, another Sinochem ship, the SC Neptune, stopped transmitting
its position when it approached the Strait of Hormuz, the tracking data
show. Four days later, for a brief period, it appeared back on the grid,
transmitting its location from an export terminal on Iran's Kharg
Island. It then went quiet for another 24 hours, reappearing on its way
out of the strait.

Image

Iran's Kharg Island (pictured in a screenshot from Google Maps), where a
Chinese ship called SC Neptune briefly reported its position in February
after going off the grid.

In some parts of the world, including the South China Sea, it is not
uncommon for ships to go silent because the automatic identification
system may be overloaded by the volume of vessels, said Court Smith, a
former officer in the United States Coast Guard who is now an analyst at
VesselsValue. Sometimes they do so for competitive reasons, he added.

But in the Persian Gulf, where traffic is lighter, Mr. Smith said,
vessels generally do not turn off the system, known in the industry as
A.I.S.

``If the A.I.S. signal is lost, it is almost certainly because the
A.I.S. transponder has been disabled or turned off,'' Mr. Smith said of
ships in the Persian Gulf. ``The captain has decided to turn off the
A.I.S.''

Another possible clue that Iran-bound ships are disabling their
reporting systems is that ships making trips to countries on the western
part of the gulf are not going off the grid.

The SC Mercury, another of the Sinochem ships, disappeared for about
nine days at the end of December and into January, vanishing close to
where the Sino Energy 1 disappeared last week, the tracking data show.
But in early April, the ship's course through the Persian Gulf had no
interruptions in its signal. The destination that time was the United
Arab Emirates.

Advertisement

\protect\hyperlink{after-bottom}{Continue reading the main story}

\hypertarget{site-index}{%
\subsection{Site Index}\label{site-index}}

\hypertarget{site-information-navigation}{%
\subsection{Site Information
Navigation}\label{site-information-navigation}}

\begin{itemize}
\tightlist
\item
  \href{https://help.nytimes3xbfgragh.onion/hc/en-us/articles/115014792127-Copyright-notice}{©~2020~The
  New York Times Company}
\end{itemize}

\begin{itemize}
\tightlist
\item
  \href{https://www.nytco.com/}{NYTCo}
\item
  \href{https://help.nytimes3xbfgragh.onion/hc/en-us/articles/115015385887-Contact-Us}{Contact
  Us}
\item
  \href{https://www.nytco.com/careers/}{Work with us}
\item
  \href{https://nytmediakit.com/}{Advertise}
\item
  \href{http://www.tbrandstudio.com/}{T Brand Studio}
\item
  \href{https://www.nytimes3xbfgragh.onion/privacy/cookie-policy\#how-do-i-manage-trackers}{Your
  Ad Choices}
\item
  \href{https://www.nytimes3xbfgragh.onion/privacy}{Privacy}
\item
  \href{https://help.nytimes3xbfgragh.onion/hc/en-us/articles/115014893428-Terms-of-service}{Terms
  of Service}
\item
  \href{https://help.nytimes3xbfgragh.onion/hc/en-us/articles/115014893968-Terms-of-sale}{Terms
  of Sale}
\item
  \href{https://spiderbites.nytimes3xbfgragh.onion}{Site Map}
\item
  \href{https://help.nytimes3xbfgragh.onion/hc/en-us}{Help}
\item
  \href{https://www.nytimes3xbfgragh.onion/subscription?campaignId=37WXW}{Subscriptions}
\end{itemize}
