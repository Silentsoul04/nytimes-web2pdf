Sections

SEARCH

\protect\hyperlink{site-content}{Skip to
content}\protect\hyperlink{site-index}{Skip to site index}

\href{https://www.nytimes3xbfgragh.onion/section/health}{Health}

\href{https://myaccount.nytimes3xbfgragh.onion/auth/login?response_type=cookie\&client_id=vi}{}

\href{https://www.nytimes3xbfgragh.onion/section/todayspaper}{Today's
Paper}

\href{/section/health}{Health}\textbar{}Judge Blocks Trump Rule
Requiring Drug Companies to List Prices in TV Ads

\url{https://nyti.ms/32fQuOM}

\begin{itemize}
\item
\item
\item
\item
\item
\item
\end{itemize}

Advertisement

\protect\hyperlink{after-top}{Continue reading the main story}

Supported by

\protect\hyperlink{after-sponsor}{Continue reading the main story}

\hypertarget{judge-blocks-trump-rule-requiring-drug-companies-to-list-prices-in-tv-ads}{%
\section{Judge Blocks Trump Rule Requiring Drug Companies to List Prices
in TV
Ads}\label{judge-blocks-trump-rule-requiring-drug-companies-to-list-prices-in-tv-ads}}

\includegraphics{https://static01.graylady3jvrrxbe.onion/images/2019/06/14/science/08drugads1/merlin_156460728_50887551-85ac-46ec-8396-02fc74f83ee5-articleLarge.jpg?quality=75\&auto=webp\&disable=upscale}

By \href{https://www.nytimes3xbfgragh.onion/by/katie-thomas}{Katie
Thomas} and
\href{https://www.nytimes3xbfgragh.onion/by/katie-rogers}{Katie Rogers}

\begin{itemize}
\item
  July 8, 2019
\item
  \begin{itemize}
  \item
  \item
  \item
  \item
  \item
  \item
  \end{itemize}
\end{itemize}

A federal judge ruled on Monday that the Trump administration cannot
force pharmaceutical companies to disclose the list price of their drugs
in television ads, dealing a blow to one of the president's most visible
efforts to pressure drug companies to lower their prices.

Judge Amit P. Mehta, of the United States District Court in the District
of Columbia, ruled that the Department of Health and Human Services
exceeded its regulatory authority by seeking to require all drugmakers
to include in their television commercials the list price of any drug
that costs more than \$35 a month. The rule was to take effect this
week.

With the 2020 presidential election race underway, the Trump
administration has searched for ways to appeal to Americans burdened by
the high cost of health care and prescription drugs.

The Affordable Care Act was once a reliable campaign trail villain for
President Trump, but leading Republicans in Congress have become
reluctant to revisit repealing the federal health care law. An appeals
court in New Orleans on Tuesday is set to hear
\href{https://www.nytimes3xbfgragh.onion/2019/05/01/health/unconstitutional-trump-aca.html}{oral
arguments on the constitutionality of Obamacare}.

In some ways, rising drug prices have provided a more populist issue for
the president and members of Congress. Politicians in both parties have
clamored to show they are doing something, but little has changed and
many companies have continued to raise their prices.

The administration's effort to provide transparency in drug pricing was
seen as largely symbolic --- a way to hold drugmakers accountable for
their prices, even if it did not directly do anything to lower costs and
even if those prices were not what consumers usually paid.

On Monday night, Judd Deere, a spokesman for the White House, said: ``It
is outrageous that an Obama-appointed judge sided with big pharma to
keep high drug prices secret from the American people, leaving patients
and families as the real victims.''

And Caitlin Oakley, a spokeswoman for H.H.S., said the administration
was disappointed and was consulting with the Justice Department on what
to do next. ``Although we are not surprised by the objections to
transparency from certain special interests,'' she said, ``putting drug
prices in ads is a useful way to put patients in control and lower
costs.''

A spokeswoman for the Justice Department did not immediately respond to
phone calls and emails requesting comment on whether the administration
would immediately appeal the ruling.

David Mitchell, the founder of Patients for Affordable Drugs, which
advocates lower drug prices, said his group never thought the
television-ad rule would get drugmakers to reduce their prices. ``But if
you take that away, at least it was something visible they could point
to that they'd done,'' he said.

Last week, the president said he would be issuing an
\href{https://www.nytimes3xbfgragh.onion/2019/07/05/upshot/trump-drug-prices-executive-order.html}{executive
order on drug pricing}, but the breadth of the order remained unclear.
His administration has proposed other moves, including allowing older
adults to more directly benefit from drug rebates in Medicare, and tying
the cost of some drugs to their price in other countries.

Republicans and Democrats in Congress
\href{https://www.nytimes3xbfgragh.onion/2019/06/16/health/drug-prices-congress-trump.html}{have
also put forward a range of legislation} that would address the issue,
from limiting out-of-pocket costs for people covered by Medicare to
allowing the federal government to directly negotiate the price of
drugs.

\href{https://www.nytimes3xbfgragh.onion/2019/06/14/health/drug-prices-tv-ads.html}{Merck,
Eli Lilly and Amgen had sued to block the television-ad rule} in June,
arguing that forcing companies to disclose their list prices was beyond
the reach of the federal government as well as a violation of the First
Amendment. The companies also said many patients have health insurance
that lowers their out-of-pocket costs, and seeing the higher list price
might lead them to stop taking drugs they needed.

The Trump administration, including Secretary Alex M. Azar II of Health
and Human Services, had argued that requiring such disclosure could
shame the drugmakers into lowering their prices.

In a statement, Lilly said it was pleased with the ruling. ``We are
committed to working with stakeholders across the health care system to
find better solutions for the larger issue, namely, lowering
out-of-pocket costs for Americans who still struggle to pay for their
medicines,'' the company said.

AARP, which represents older Americans, expressed disappointment in the
judge's decision. ``Today's ruling is a step backward in the battle
against skyrocketing drug prices and providing more information to
consumers,'' the group said. ``Americans should be trusted to evaluate
drug price information and discuss any concerns with their health care
providers.''

Judge Mehta,
\href{https://www.dcd.uscourts.gov/content/district-judge-amit-p-mehta}{who
was nominated to his position by President Barack Obama in 2014}, did
not delve into whether the proposed rule violated the First Amendment.
He relied instead on whether the Department of Health and Human Services
had overstepped its bounds because it sought to issue the rule under the
authority of the Social Security Act.

While saying the court did not question the agency's motives, he wrote:
``Nor does it take any view on the wisdom of requiring drug companies to
disclose prices. That policy very well could be an effective tool in
halting the rising cost of prescription drugs. But no matter how vexing
the problem of spiraling drug costs may be, H.H.S. cannot do more than
what Congress has authorized. The responsibility rests with Congress to
act in the first instance.''

Last year, Senator Charles Grassley of Iowa, the Republican chairman of
the Senate Finance Committee, and Senator Dick Durbin of Illinois, a
Democrat, proposed legislation that was similar to the Trump
administration's proposal. It passed the Senate in August 2018, and in
May,
\href{https://www.grassley.senate.gov/news/news-releases/grassley-durbin-statement-hhs-rule-require-disclosure-prescription-drug-prices-tv}{the
senators said they were still pursuing the legislation}.

Mr. Trump has faced hurdles --- some of his own making --- as he has
sought to make changes either unilaterally or with the help of
Democrats. In May, the president said during a speech in the Roosevelt
Room that his administration would work with Democrats to eliminate
surprise medical billing --- the practice of billing patients with
undisclosed costs at the time of care.

He also singled out the drug-price disclosure rule.

The rule was ``going to be something, I think, very special,'' Mr. Trump
said. ``You may have heard about it. Maybe not. But it's the beginning
of a plan of transparency.''

Advertisement

\protect\hyperlink{after-bottom}{Continue reading the main story}

\hypertarget{site-index}{%
\subsection{Site Index}\label{site-index}}

\hypertarget{site-information-navigation}{%
\subsection{Site Information
Navigation}\label{site-information-navigation}}

\begin{itemize}
\tightlist
\item
  \href{https://help.nytimes3xbfgragh.onion/hc/en-us/articles/115014792127-Copyright-notice}{©~2020~The
  New York Times Company}
\end{itemize}

\begin{itemize}
\tightlist
\item
  \href{https://www.nytco.com/}{NYTCo}
\item
  \href{https://help.nytimes3xbfgragh.onion/hc/en-us/articles/115015385887-Contact-Us}{Contact
  Us}
\item
  \href{https://www.nytco.com/careers/}{Work with us}
\item
  \href{https://nytmediakit.com/}{Advertise}
\item
  \href{http://www.tbrandstudio.com/}{T Brand Studio}
\item
  \href{https://www.nytimes3xbfgragh.onion/privacy/cookie-policy\#how-do-i-manage-trackers}{Your
  Ad Choices}
\item
  \href{https://www.nytimes3xbfgragh.onion/privacy}{Privacy}
\item
  \href{https://help.nytimes3xbfgragh.onion/hc/en-us/articles/115014893428-Terms-of-service}{Terms
  of Service}
\item
  \href{https://help.nytimes3xbfgragh.onion/hc/en-us/articles/115014893968-Terms-of-sale}{Terms
  of Sale}
\item
  \href{https://spiderbites.nytimes3xbfgragh.onion}{Site Map}
\item
  \href{https://help.nytimes3xbfgragh.onion/hc/en-us}{Help}
\item
  \href{https://www.nytimes3xbfgragh.onion/subscription?campaignId=37WXW}{Subscriptions}
\end{itemize}
