Sections

SEARCH

\protect\hyperlink{site-content}{Skip to
content}\protect\hyperlink{site-index}{Skip to site index}

\href{https://www.nytimes3xbfgragh.onion/section/arts/design}{Art \&
Design}

\href{https://myaccount.nytimes3xbfgragh.onion/auth/login?response_type=cookie\&client_id=vi}{}

\href{https://www.nytimes3xbfgragh.onion/section/todayspaper}{Today's
Paper}

\href{/section/arts/design}{Art \& Design}\textbar{}Where Are All the
Bob Ross Paintings? We Found Them.

\url{https://nyti.ms/2xHTtBk}

\begin{itemize}
\item
\item
\item
\item
\item
\end{itemize}

Advertisement

\protect\hyperlink{after-top}{Continue reading the main story}

Supported by

\protect\hyperlink{after-sponsor}{Continue reading the main story}

\hypertarget{where-are-all-the-bob-ross-paintings-we-found-them}{%
\section{Where Are All the Bob Ross Paintings? We Found
Them.}\label{where-are-all-the-bob-ross-paintings-we-found-them}}

Bob Ross painted more than 1,000 landscapes for his television show ---
so why are they so hard to find? We solve one of the internet's favorite
little mysteries.

\includegraphics{https://static01.graylady3jvrrxbe.onion/images/2019/07/13/arts/13video/bob-ross-cover-videoSixteenByNineJumbo1600-v5.png}

By \href{https://www.nytimes3xbfgragh.onion/by/larry-buchanan}{Larry
Buchanan}, \href{https://www.nytimes3xbfgragh.onion/by/aaron-byrd}{Aaron
Byrd},
\href{https://www.nytimes3xbfgragh.onion/by/alicia-desantis}{Alicia
DeSantis} and
\href{https://www.nytimes3xbfgragh.onion/by/emily-rhyne}{Emily Rhyne}

\begin{itemize}
\item
  July 12, 2019
\item
  \begin{itemize}
  \item
  \item
  \item
  \item
  \item
  \end{itemize}
\end{itemize}

Watched the video? Here are a few more details.

\begin{center}\rule{0.5\linewidth}{\linethickness}\end{center}

\hypertarget{what-will-be-in-the-smithsonian}{%
\subsubsection{What will be in the
Smithsonian?}\label{what-will-be-in-the-smithsonian}}

Bob Ross made three versions of each painting that appeared on ``The Joy
of Painting.'' The first was made before the show, to be used as a
reference. He painted the second during the 26-minute taping, sometimes
with last-minute improvisations. The third was made afterward, for
instructional books.

The donation to the Smithsonian includes the book version of ``Blue
Ridge Falls,'' from Season 30 (1994):

\includegraphics{https://static01.graylady3jvrrxbe.onion/images/2019/07/13/arts/bob-ross-oak_blue-ridge-falls/bob-ross-oak_blue-ridge-falls-articleLarge-v2.jpg?quality=75\&auto=webp\&disable=upscale}

As well as all three versions of the painting ``On a Clear Day,'' from
Season 14 (1988):

Image

Credit...Bob Ross Inc.

Other items include a converted stepladder that was used as an easel
used during the first season of the show, and two handwritten notebooks
that were used to plan the production of Seasons 2 and 3.

\hypertarget{how-were-the-objects-chosen}{%
\subsubsection{How were the objects
chosen?}\label{how-were-the-objects-chosen}}

``The hardest part was choosing the paintings,'' said Eric Jentsch, the
entertainment and sports curator for the National Museum of American
History. Mr. Jentsch and his colleague Ryan Lintelman visited the
offices of Bob Ross Inc. in Herndon, Va., to find the images and
materials that best exemplified Mr. Ross's lifetime of work.

The Smithsonian also acquired fan letters sent to Mr. Ross, including
some written after he died of lymphoma in 1995 at 52. ``These letters
help reveal the significant impact Ross has had on diverse individuals
and communities, helping them to express and feel better about
themselves,'' Mr. Jentsch said.

The paintings and other objects officially became part of the museum's
permanent collection on March 22.

For now, the Smithsonian has no plans to display the paintings.

\hypertarget{how-many-bob-ross-paintings-are-there-exactly}{%
\subsubsection{\texorpdfstring{\textbf{How many Bob Ross paintings are
there
exactly?}}{How many Bob Ross paintings are there exactly?}}\label{how-many-bob-ross-paintings-are-there-exactly}}

We don't know.

According to an analysis by the website
\href{https://fivethirtyeight.com/features/a-statistical-analysis-of-the-work-of-bob-ross/}{FiveThirtyEight},
Mr. Ross painted in 381 of the 403 episodes of the show (the rest
featured a guest). If three versions were made of each of those
paintings, at least 1,143 originals would exist. Bob Ross Inc. estimates
that it has 1,165 paintings stored on site.

But Mr. Ross also painted as an instructor, as well as for public events
and for charity, so there may be additional paintings out there.

\hypertarget{how-much-does-one-cost-}{%
\subsubsection{\texorpdfstring{\textbf{How much does one cost?}
}{How much does one cost? }}\label{how-much-does-one-cost-}}

In the rare cases when a Bob Ross painting does surface, it depends who
is buying. Joan Kowalski, president of Bob Ross Inc., said she has seen
authentic Ross paintings sell online for \$8,000 to \$10,000 in recent
years.

After we set out on our quest, a three-panel painting described as a
``Bob Ross Original Oil Painting Triptych Mountain Landscape'' surfaced
on eBay. It is listed at \$55,000:

Image

Credit...eBay

\hypertarget{how-do-i-have-a-painting-authenticated-}{%
\subsubsection{\texorpdfstring{\textbf{How do I have a painting
authenticated?}
}{How do I have a painting authenticated? }}\label{how-do-i-have-a-painting-authenticated-}}

Bob Ross Inc. will authenticate paintings that are sent to be inspected
in person by Annette Kowalski, Joan Kowalski's mother and the woman who
discovered Mr. Ross. (The company will not certify images that can be
viewed only as scans or digital files.)

Annette Kowalski said that in addition to the brushwork and other signs
of Mr. Ross's hand, she looks for a specific detail in the quality of
his signature that she declined to describe:

Image

Credit...Bob Ross Inc.

If a painting is certified as an original Bob Ross, the owner will be
provided with documentation attesting its authenticity.

\hypertarget{can-i-visit-bob-ross-inc}{%
\subsubsection{\texorpdfstring{\textbf{Can I visit Bob Ross
Inc.?}}{Can I visit Bob Ross Inc.?}}\label{can-i-visit-bob-ross-inc}}

Bob Ross Inc. is not open to visitors. Some of the original paintings
are displayed at the \href{https://www.bobrossartworkshop.com/}{Bob Ross
Art Workshop}\href{https://www.bobrossartworkshop.com/}{\& Gallery} in
New Smyrna Beach, Fla. Starting next year, people will be able to visit
the studio in Muncie, Ind., where the show was taped.

\hypertarget{is-it-true-that-there-are-no-people-in-the-paintings}{%
\subsubsection{\texorpdfstring{\textbf{Is it true that there are no
people in the
paintings?}}{Is it true that there are no people in the paintings?}}\label{is-it-true-that-there-are-no-people-in-the-paintings}}

In the 11 years that Mr. Ross painted on television, there are only a
few known instances when he included a human figure in his landscapes.
In ``Morning Walk'' (Series 17, Episode 11, from 1989), two people
stroll through the woods:

Image

Credit...Bob Ross Inc.

In ``Campfire'' (Series 3, Episode 10, 1984), a figure in a hat leans
against a tree:

Image

Credit...Bob Ross Inc.

According to Annette Kowalski, ``Campfire'' was among Mr. Ross's least
favorite paintings.

Though cabins often appear in Mr. Ross's landscapes, they are rarely
depicted with chimneys (another sign of people).

\hypertarget{how-did-the-kowalskis-come-to-own-the-company}{%
\subsubsection{\texorpdfstring{\textbf{How did the Kowalskis come to own
the
company?}}{How did the Kowalskis come to own the company?}}\label{how-did-the-kowalskis-come-to-own-the-company}}

Originally Mr. Ross and his wife, Jane, shared ownership of the company
with Annette and Walt Kowalski, who had helped to finance Mr. Ross's
early career. Jane Ross died in 1992; when Mr. Ross died in 1995, the
company was left to the Kowalskis alone.

\hypertarget{whats-the-name-of-the-kowalskis-dog}{%
\subsubsection{\texorpdfstring{\textbf{What's the name of the Kowalskis'
dog?}}{What's the name of the Kowalskis' dog?}}\label{whats-the-name-of-the-kowalskis-dog}}

Cricket.

Image

Credit...Emily Rhyne/The New York Times

\hypertarget{what-were-the-names-of-bob-rosss-squirrels}{%
\subsubsection{\texorpdfstring{\textbf{What were the names of Bob Ross's
squirrels?}}{What were the names of Bob Ross's squirrels?}}\label{what-were-the-names-of-bob-rosss-squirrels}}

Mr. Ross had several pet squirrels, a number of which he featured on his
show. One was named Bobette --- a combination of Bob and Annette.
Bobette appeared in several episodes in Series 18 (1989). Another
squirrel, Peapod, appeared in Series 22 and 23 (1991). Peapod Jr. joined
in Series 30 and 31 (1993-94).

\hypertarget{whats-the-story-with-the-hair}{%
\subsubsection{\texorpdfstring{\textbf{What's the story with the
hair?}}{What's the story with the hair?}}\label{whats-the-story-with-the-hair}}

Bob Ross did not always have a perm:

Image

Credit...Bob Ross Inc.

According to Annette Kowalski, Mr. Ross originally chose to perm his
hair because it was cheaper than getting frequent haircuts.

Later, she said, he disliked the hairstyle but did not feel he could
change it because it was depicted in the company logo:

Image

Credit...Bob Ross Inc.

\hypertarget{who-was-bill-alexander}{%
\subsubsection{\texorpdfstring{\textbf{Who was Bill
Alexander?}}{Who was Bill Alexander?}}\label{who-was-bill-alexander}}

William Alexander was the creator of ``The Magic of Oil Painting,''
which aired on PBS from 1974 to 1982. In 1984, he symbolically handed
over his brush to Mr. Ross in a marketing campaign.

They later had a falling out. In a
\href{https://timesmachine.nytimes3xbfgragh.onion/timesmachine/1991/12/22/128191.html?action=click\&contentCollection=Archives\&module=LedeAsset\&region=ArchiveBody\&pgtype=article\&pageNumber=175}{1991
interview} with The New York Times, Mr. Alexander said, ``He betrayed
me.'' ``I \emph{invented} `wet on wet,''' he added. ``I trained him, and
he is copying me --- what bothers me is not just that he betrayed me,
but that he thinks he can do it better.''

\hypertarget{did-bob-ross-want-his-paintings-to-be-shown}{%
\subsubsection{\texorpdfstring{\textbf{Did Bob Ross want his paintings
to be
shown?}}{Did Bob Ross want his paintings to be shown?}}\label{did-bob-ross-want-his-paintings-to-be-shown}}

In 1994, the talk show host Phil Donahue asked Mr. Ross to ``say out
loud your work will never hang in a museum.''

``Well, maybe it will,'' Mr. Ross replied. ``But probably not the
Smithsonian.''

When The Times asked Mr. Ross
\href{https://timesmachine.nytimes3xbfgragh.onion/timesmachine/1991/12/22/128191.html?action=click\&contentCollection=Archives\&module=LedeAsset\&region=ArchiveBody\&pgtype=article\&pageNumber=175}{about
his legacy} in 1991, he gave a similar answer:

Image

Advertisement

\protect\hyperlink{after-bottom}{Continue reading the main story}

\hypertarget{site-index}{%
\subsection{Site Index}\label{site-index}}

\hypertarget{site-information-navigation}{%
\subsection{Site Information
Navigation}\label{site-information-navigation}}

\begin{itemize}
\tightlist
\item
  \href{https://help.nytimes3xbfgragh.onion/hc/en-us/articles/115014792127-Copyright-notice}{©~2020~The
  New York Times Company}
\end{itemize}

\begin{itemize}
\tightlist
\item
  \href{https://www.nytco.com/}{NYTCo}
\item
  \href{https://help.nytimes3xbfgragh.onion/hc/en-us/articles/115015385887-Contact-Us}{Contact
  Us}
\item
  \href{https://www.nytco.com/careers/}{Work with us}
\item
  \href{https://nytmediakit.com/}{Advertise}
\item
  \href{http://www.tbrandstudio.com/}{T Brand Studio}
\item
  \href{https://www.nytimes3xbfgragh.onion/privacy/cookie-policy\#how-do-i-manage-trackers}{Your
  Ad Choices}
\item
  \href{https://www.nytimes3xbfgragh.onion/privacy}{Privacy}
\item
  \href{https://help.nytimes3xbfgragh.onion/hc/en-us/articles/115014893428-Terms-of-service}{Terms
  of Service}
\item
  \href{https://help.nytimes3xbfgragh.onion/hc/en-us/articles/115014893968-Terms-of-sale}{Terms
  of Sale}
\item
  \href{https://spiderbites.nytimes3xbfgragh.onion}{Site Map}
\item
  \href{https://help.nytimes3xbfgragh.onion/hc/en-us}{Help}
\item
  \href{https://www.nytimes3xbfgragh.onion/subscription?campaignId=37WXW}{Subscriptions}
\end{itemize}
