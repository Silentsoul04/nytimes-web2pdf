Sections

SEARCH

\protect\hyperlink{site-content}{Skip to
content}\protect\hyperlink{site-index}{Skip to site index}

\href{https://www.nytimes3xbfgragh.onion/section/politics}{Politics}

\href{https://myaccount.nytimes3xbfgragh.onion/auth/login?response_type=cookie\&client_id=vi}{}

\href{https://www.nytimes3xbfgragh.onion/section/todayspaper}{Today's
Paper}

\href{/section/politics}{Politics}\textbar{}House Condemns Trump's
Attack on Four Congresswomen as Racist

\url{https://nyti.ms/2k8fBSs}

\begin{itemize}
\item
\item
\item
\item
\item
\item
\end{itemize}

Advertisement

\protect\hyperlink{after-top}{Continue reading the main story}

Supported by

\protect\hyperlink{after-sponsor}{Continue reading the main story}

\hypertarget{house-condemns-trumps-attack-on-four-congresswomen-as-racist}{%
\section{House Condemns Trump's Attack on Four Congresswomen as
Racist}\label{house-condemns-trumps-attack-on-four-congresswomen-as-racist}}

\includegraphics{https://static01.graylady3jvrrxbe.onion/images/2019/07/16/us/politics/16dc-cong1-sub3/16dc-cong1-sub3-videoSixteenByNine3000.jpg}

By
\href{https://www.nytimes3xbfgragh.onion/by/julie-hirschfeld-davis}{Julie
Hirschfeld Davis}

\begin{itemize}
\item
  July 16, 2019
\item
  \begin{itemize}
  \item
  \item
  \item
  \item
  \item
  \item
  \end{itemize}
\end{itemize}

WASHINGTON --- The House voted on Tuesday to condemn as racist President
Trump's attacks against four congresswomen of color, but only after the
debate over the president's language devolved into a bitterly partisan
brawl that showcased deep rifts over race, ethnicity and political
ideology in the age of Trump.

The measure, the first House rebuke of a president in more than 100
years, passed nearly along party lines, 240 to 187, after one of the
most polarizing exchanges on the floor in recent times. Only four
Republicans and the House's lone independent, Representative Justin
Amash of Michigan, voted with all Democrats to condemn the president.

\emph{{[}Update:}
\href{https://www.nytimes3xbfgragh.onion/2019/07/17/us/politics/house-impeachment-trump.html}{\emph{House
votes to kill Trump impeachment resolution}}\emph{.{]}}

``I know racism when I see it, I know racism when I feel it, and at the
highest level of government, there's no room for racism,'' said
Representative John Lewis, Democrat of Georgia, an icon of the civil
rights movement.

Some Republicans were just as adamant in their defense of Mr. Trump:
``What has really happened here is that the president and his supporters
have been forced to endure months of allegations of racism,'' said
Representative Dan Meuser, Republican of Pennsylvania. ``This ridiculous
slander does a disservice to our nation.''

\emph{{[}Read}
\href{https://www.nytimes3xbfgragh.onion/2019/07/16/us/politics/house-resolution-condemning-trump.html}{\emph{the
text of the resolution}}\emph{.{]}}

Republicans ground the proceedings to a halt shortly before the House
was to vote on
\href{https://www.congress.gov/bill/116th-congress/house-resolution/489/text?q=\%7B\%22search\%22\%3A\%5B\%22racist\%22\%2C\%22racist\%22\%5D\%7D\&r=2\&s=3}{the
nonbinding resolution}, which calls Mr. Trump's tweets and verbal
volleys ``racist comments that have legitimized increased fear and
hatred of new Americans and people of color.'' It was the Democrats'
response to Mr. Trump's attacks on Representatives Alexandria
Ocasio-Cortez of New York, Ilhan Omar of Minnesota, Ayanna S. Pressley
of Massachusetts and Rashida Tlaib of Michigan, who he said should ``go
back'' to their countries,
\href{https://www.nytimes3xbfgragh.onion/2019/07/16/us/politics/aoc-trump-tlaib-omar-pressley.html}{an
insult} that he has continued to employ in the days since.

``There's no excuse for any response to those words but a swift and
strong, unified condemnation,'' Speaker Nancy Pelosi said as the House
debated the resolution. ``Every single member of this institution,
Democratic and Republican, should join us in condemning the president's
racist tweets.''

As Republicans rose to protest, Ms. Pelosi turned toward them on the
House floor and picked up her speech, her voice rising as she added,
``To do anything less would be a shocking rejection of our values and a
shameful abdication of our oath of office to protect the American
people.''

\includegraphics{https://static01.graylady3jvrrxbe.onion/images/2019/07/16/us/politics/16dc-cong1-sub/merlin_158029704_159d542f-4529-4b92-92c7-04af3259cb9a-articleLarge.jpg?quality=75\&auto=webp\&disable=upscale}

Representative Doug Collins of Georgia, the top Republican on the
Judiciary Committee, made a formal objection to the remarks, charging
that they had violated the rules of decorum, which call for lawmakers to
avoid impugning the motives of their colleagues or the president. It was
a stunning turn in a debate about Mr. Trump's own incendiary language.

Mr. Trump on Tuesday denied that his tweets were racist and implored
House Republicans to reject the measure. The president raged on Twitter,
calling the House resolution a ``con game'' as he renewed his harsh
criticism of the congresswomen.

``Those Tweets were NOT Racist,''
\href{https://twitter.com/realDonaldTrump/status/1151129281134768128}{Mr.
Trump wrote}. ``I don't have a Racist bone in my body! The so-called
vote to be taken is a Democrat con game. Republicans should not show
`weakness' and fall into their trap.''

Late Tuesday,
\href{https://twitter.com/realDonaldTrump/status/1151327083110510594}{the
president praised how the Republicans voted}, tweeting, ``So great to
see how unified the Republican Party was on today's vote concerning
statements I made about four Democrat Congresswomen.''

The vote was a show of unity for Democrats --- who had been squabbling
for weeks --- and a test of Republican principles. In the end, the only
Republicans to cross party lines were Representative Fred Upton of
Michigan, Susan W. Brooks of Indiana, Brian Fitzpatrick of Pennsylvania
and Will Hurd of Texas, the House's only black Republican.

But as the debate played out, the scene devolved into a spectacle.

At one point, Representative Emanuel Cleaver II, Democrat of Missouri,
who was presiding in the House when Republicans challenged Ms. Pelosi's
words, banged the gavel, rose from the marble dais and stormed off the
floor. ``We aren't ever, ever going to pass up, it seems, an opportunity
to escalate, and that's what this is,'' Mr. Cleaver said, his voice
rising in frustration. ``We want to just fight.''

For their part, Republicans took to the floor not to defend the
president's remarks but to condemn Democrats for what they called a
breach of decorum.

Image

Senator Mitch McConnell, Republican of Kentucky and the majority leader,
said, ``The president is not a racist.''Credit...Erin Schaff/The New
York Times

Ultimately, it was left to Representative Steny H. Hoyer, the majority
leader, to recite the official ruling that Ms. Pelosi had, in fact,
violated a House rule against characterizing an action as ``racist.''
The move by Republicans to have her words stricken from the record then
failed along party lines, and Ms. Pelosi was unrepentant.

``I stand by my statement,'' she said as she strode through the Capitol.
``I'm proud of the attention being called to it because what the
president said was completely inappropriate.''

While Democrats were publicly unanimous in their support of the
resolution, some moderate lawmakers from Republican-leaning districts
that backed Mr. Trump in 2016 privately voiced their discomfort. They
said that while the president's comments had been racist, the party was
playing into his hands by spending so much time condemning his remarks,
according to centrist lawmakers and senior aides who spoke on the
condition of anonymity to describe internal discussions.

They were particularly angry about being asked to vote to condone Ms.
Pelosi's breach of the rules, which two of them described as throwing
moderate lawmakers ``under the bus'' in order to help the speaker shore
up support among progressives who had been alienated by her feud with
Ms. Ocasio-Cortez and her allies. One lawmaker described the upshot of
the extraordinary episode as ``another week burned on his terms instead
of ours.''

The scene underscored the intensity of feeling prompted by Mr. Trump's
latest comments. Republicans spent the day arguing that Democrats,
particularly Ms. Ocasio-Cortez's so-called Squad, were no better.

``In those tweets, I see nothing that references anybody's race --- not
a thing --- I don't see anyone's name being referenced in the tweets,
but the president's referring to people, congresswomen, who are
anti-American,'' said Representative Sean P. Duffy, Republican of
Wisconsin. ``And lo and behold, everybody in this chamber knows who he's
talking about.''

Mr. Duffy's comments prompted an angry response from Representative
Pramila Jayapal, Democrat of Washington, who sought to register an
official objection. She said the use of the word ``anti-American'' was
``completely inappropriate'' but was not allowed to formally ask to have
the words stricken.

\includegraphics{https://static01.graylady3jvrrxbe.onion/images/2019/07/15/us/politics/15dc-trump-sub/15dc-trump-sub-videoSixteenByNine3000.jpg}

At a closed-door meeting of House Democrats on Tuesday morning, Ms.
Pelosi set the stage for the debate, calling the four freshman
congresswomen ``our sisters,'' and saying Mr. Trump's insults echoed
hurtful and offensive remarks he makes every day.

``So this is a resolution based in who we are as a people, as well as a
recognition of the unacceptability of what his goals were,'' Ms. Pelosi
told Democrats, according to an aide present for the private meeting who
described her remarks on condition of anonymity. ``This is, I hope, one
where we will get Republican support. If they can't support condemning
the words of the president, well, that's a message in and of itself.''

A smattering of Republicans have denounced Mr. Trump's performance,
including Gov. Charlie Baker of Massachusetts. The president's comments
``were shameful, they were racist,''
\href{https://www.wbur.org/news/2019/07/15/baker-republican-trump-tweets-pressley}{he
told WBUR in Boston} on Monday, ``and they bring a tremendous amount of,
sort of, disgrace to public policy and public life, and I condemn them
all.''

But most Republican leaders refrained from criticizing Mr. Trump, at
least directly, and top House Republicans lobbied their colleagues to
oppose the resolution.

Representative Kevin McCarthy of California, the House Republican leader
and a close ally of the president's, said he would oppose the measure,
and when asked whether Mr. Trump's tweets were racist, he replied
flatly: ``No.''

Senator Mitch McConnell, Republican of Kentucky and the majority leader,
said that all politicians should dial back their rhetoric. But he did
not take issue with Mr. Trump, saying that ``the president's not a
racist.''

Earlier, Mr. Trump tried to shift the focus to what he called
``HORRIBLE'' things said by the four liberal freshman congresswomen, who
have been among the most outspoken in their criticisms of him. On
Monday, they described Mr. Trump as racist, xenophobic, misogynistic and
criminal.

\href{https://www.nytimes3xbfgragh.onion/interactive/2019/07/18/us/politics/trump-racist-tweet-evolution.html}{}

\includegraphics{https://static01.graylady3jvrrxbe.onion/images/2019/07/18/us/trump-racist-tweet-evolution-promo-1563501582520/trump-racist-tweet-evolution-promo-1563501582520-articleLarge-v3.png}

\hypertarget{how-trumps-twitter-attack-against-democrats-evolved-into-send-her-back-chant}{%
\subsection{How Trump's Twitter Attack Against Democrats Evolved Into
`Send Her Back'
Chant}\label{how-trumps-twitter-attack-against-democrats-evolved-into-send-her-back-chant}}

On Sunday, the president tweeted about four congresswomen in messages
denounced as racist. By Wednesday, a crowd of his supporters had a
rallying cry.

``This should be a vote on the filthy language, statements and lies told
by the Democrat Congresswomen, who I truly believe, based on their
actions, hate our Country,''
\href{https://twitter.com/realDonaldTrump/status/1151129281919102976}{Mr.
Trump tweeted}.

While some Democrats had pressed for a stronger resolution of censure,
House leaders opted instead for a narrower measure based on Mr. Trump's
latest remarks, in an effort to generate a unanimous vote in their
party.

During the meeting on Tuesday morning, Representative Jim McGovern,
Democrat of Massachusetts and the chairman of the Rules Committee,
warned members to take care with their language during the debate,
including checking with the official in charge of enforcing floor
procedures to make sure their speeches would not violate House rules
against making personal references to the president on the floor.

Ms. Pelosi advised Democrats to focus on how Mr. Trump's ``words were
racist,'' which would keep them in compliance with the rules. Later,
after Mr. Collins objected to her speech, Ms. Pelosi shot back that she
had cleared them in advance to ensure they were within bounds.

It is virtually unheard-of for Congress to rebuke a sitting president.
The last one to be challenged was William Howard Taft, who served from
1909 to 1913. He was accused of having tried to influence a disputed
Senate election, but in the end, the Senate passed a watered-down
resolution and the phrase ``ought to be severely condemned'' was
removed.

While the vote on Tuesday was symbolic and nonbinding, the debate
dramatized the conflict between Democrats and a president who has
organized his agenda and his re-election campaign around stoking racial
controversy and casting the group of progressive stars as dangerous
extremists to be feared.

Among other things, the resolution declares that the House ``believes
that immigrants and their descendants have made America stronger,'' that
``those who take the oath of citizenship are every bit as American as
those whose families have lived in the United States for many
generations,'' and that the House ``is committed to keeping America open
to those lawfully seeking refuge and asylum from violence and
oppression, and those who are willing to work hard to live the American
Dream, no matter their race, ethnicity, faith, or country of origin.''

One after another, Republicans rose to reject the criticism of Mr.
Trump, arguing that it was Ms. Ocasio-Cortez and her colleagues --- who
have sometimes used coarse language to describe Mr. Trump and his
policies --- who should be rebuked and punished for their words and
conduct.

``When we consider the power of this chamber to legislate for the common
good,'' Mr. Collins said, ``I wonder why my colleagues have become so
eager to attack the president they are willing to sacrifice the rules,
precedent and the integrity of the people's house for an unprecedented
vote that undercuts its very democratic processes.''

The Democratic unity on the vote could prove short-lived. Moments after
the measure passed, Representative Al Green, Democrat of Texas, went to
the House floor to reintroduce his articles of impeachment against the
president. If Mr. Green can force a debate, the divisions between
liberals and more moderate Democrats will almost certainly re-emerge.

Advertisement

\protect\hyperlink{after-bottom}{Continue reading the main story}

\hypertarget{site-index}{%
\subsection{Site Index}\label{site-index}}

\hypertarget{site-information-navigation}{%
\subsection{Site Information
Navigation}\label{site-information-navigation}}

\begin{itemize}
\tightlist
\item
  \href{https://help.nytimes3xbfgragh.onion/hc/en-us/articles/115014792127-Copyright-notice}{©~2020~The
  New York Times Company}
\end{itemize}

\begin{itemize}
\tightlist
\item
  \href{https://www.nytco.com/}{NYTCo}
\item
  \href{https://help.nytimes3xbfgragh.onion/hc/en-us/articles/115015385887-Contact-Us}{Contact
  Us}
\item
  \href{https://www.nytco.com/careers/}{Work with us}
\item
  \href{https://nytmediakit.com/}{Advertise}
\item
  \href{http://www.tbrandstudio.com/}{T Brand Studio}
\item
  \href{https://www.nytimes3xbfgragh.onion/privacy/cookie-policy\#how-do-i-manage-trackers}{Your
  Ad Choices}
\item
  \href{https://www.nytimes3xbfgragh.onion/privacy}{Privacy}
\item
  \href{https://help.nytimes3xbfgragh.onion/hc/en-us/articles/115014893428-Terms-of-service}{Terms
  of Service}
\item
  \href{https://help.nytimes3xbfgragh.onion/hc/en-us/articles/115014893968-Terms-of-sale}{Terms
  of Sale}
\item
  \href{https://spiderbites.nytimes3xbfgragh.onion}{Site Map}
\item
  \href{https://help.nytimes3xbfgragh.onion/hc/en-us}{Help}
\item
  \href{https://www.nytimes3xbfgragh.onion/subscription?campaignId=37WXW}{Subscriptions}
\end{itemize}
