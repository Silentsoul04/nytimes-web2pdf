Sections

SEARCH

\protect\hyperlink{site-content}{Skip to
content}\protect\hyperlink{site-index}{Skip to site index}

\href{https://myaccount.nytimes3xbfgragh.onion/auth/login?response_type=cookie\&client_id=vi}{}

\href{https://www.nytimes3xbfgragh.onion/section/todayspaper}{Today's
Paper}

How to Make an Ancient Grain Taste Better Than Mac and Cheese

\url{https://nyti.ms/2JCD2gA}

\begin{itemize}
\item
\item
\item
\item
\item
\end{itemize}

Advertisement

\protect\hyperlink{after-top}{Continue reading the main story}

Supported by

\protect\hyperlink{after-sponsor}{Continue reading the main story}

\href{/column/magazine-eat}{Eat}

\hypertarget{how-to-make-an-ancient-grain-taste-better-than-mac-and-cheese}{%
\section{How to Make an Ancient Grain Taste Better Than Mac and
Cheese}\label{how-to-make-an-ancient-grain-taste-better-than-mac-and-cheese}}

\includegraphics{https://static01.graylady3jvrrxbe.onion/images/2020/05/03/magazine/21mag-eat/21mag-eat-articleLarge.jpg?quality=75\&auto=webp\&disable=upscale}

By Samin Nosrat

\begin{itemize}
\item
  July 17, 2019
\item
  \begin{itemize}
  \item
  \item
  \item
  \item
  \item
  \end{itemize}
\end{itemize}

For the first decade of my career, I took cooking incredibly seriously.
As a restaurant cook, not to mention a nerd, I loved steeping myself in
the history of a dish and all its traditions. I obsessed over pasta
shapes, flour types and varieties of tomatoes and their specific,
appropriate uses. I wanted my food to echo hundreds, or even thousands,
of years of tradition --- or at the very least to be something a
\emph{nonna} would recognize and enjoy.

Last December, in preparation for an appearance on the ``Today'' show, I
spent a week of frantically practicing --- and failing at --- making my
favorite pasta, \emph{cacio e pepe.} While the classic Roman dish is
simple --- made with only pasta and its cooking water; \emph{cacio,} or
Pecorino Romano; and abundant freshly ground black pepper --- it's not
always easy to make.

I first fell in love with \emph{cacio e pepe} as a restaurant cook. The
trick to getting a thick, creamy sauce lies in combining the cheese and
pepper with starchy pasta water in just the right way; this usually
requires a lot of erratic stirring and sweat. A professional kitchen, an
industrial stove with powerful burners, a seemingly bottomless pot of
starchy pasta water and the knowledge that a cleaning crew would be
wiping down the walls later gave me everything I needed to practice pan
after pan of pasta until I was satisfied with the satiny sheen of the
melted-cheese sauce, proud that I hadn't needed to rely on butter, oil
or cream to bring it together. But years later, practicing the dish in
my home kitchen late at night, I couldn't get it right. No matter what I
did, the cheese wouldn't melt properly, instead leaving behind a curdled
mess.

Panicking, I asked for advice from colleagues, all of whom essentially
said, ``Stir harder.'' But no matter how hard I stirred, I couldn't get
it to work. Desperately, I turned to YouTube. A quick search for
\emph{``cacio e pepe''} yielded a video from Elizabeth Minchilli, an
American food writer in Rome and one of my most trusted sources on the
city's cuisine. In the three-minute video, Minchilli visits the kitchen
of Flavio de Maio, a master of \emph{cucina Romana} and producer of one
of Rome's most beloved bowls of \emph{cacio e pepe.} I watched in a
state of semi-disbelief as he used an immersion blender to combine
grated cheese, ground pepper and a few spoonfuls of cold water into a
creamy paste before tossing it with hot, just-cooked pasta. He then used
a little bit of pasta water to thin it out to the right consistency. It
looked glorious, but I couldn't help thinking that the Roman shepherds
thought to be the progenitors of this dish weren't zipping the cheese
into a paste with a blender.

My desire to adhere to tradition was being tested. I briefly considered
using the method, before realizing that I couldn't go on national
television and teach something I'd learned in a panic at the 11th hour
from an internet video that another food writer had posted about a
restaurant chef in Rome. Instead, I asked to make another dish for the
TV segment. But in the meantime, I couldn't stop thinking about that
cheese paste.

A few weeks later, I was at the home of my dear friend Greta Caruso
while she was cooking dinner using her favorite ingredient: farro, an
ancient grain. ``It has the virtue of brown rice and the pleasure of
pasta,'' she loves to exclaim when introducing farro to a newbie. When
she asked me how I'd like to eat it that night, I replied, ``With lots
of cheese.'' But then she added an obscene amount of pepper, and we
called it \emph{farro e pepe}. It was chewy, salty and entirely
satisfying. And it made me wonder what farro might be like when coated
in de Maio's cheese paste.

Soon after, I bought a big bag of farro. While it simmered in a pot, I
made the cheese paste with an immersion blender, which was as simple to
prepare as I'd hoped. When the grains were fluffy and split, I drained
them, reserving some of the starchy water. In a big bowl, I stirred the
farro with a generous spoonful of the cheese paste, watching with glee
as it melted like butter to coat the grains in a layer of salt, pepper,
richness and tang. Even though to an untrained eye it might have looked
like porridge, I proudly sent a photo of it to Caruso, the only other
person I knew who'd get such a thrill from a bowl of whole grains. She
responded with expletives and exclamation points, and when I told her
that it tasted as good as I could have imagined --- like mac and cheese,
but for a sophisticated palate --- she excitedly wrote, ``All I want is
for farro to have its day in the sun!''

If anything could turn farro from a specialty ingredient into a
universal pantry item, it's blended cheese. I quickly used up my
leftover paste --- stirring a spoonful into grits, tossing it with
boiled green beans and, of course, for making \emph{cacio e pepe} ---
and started daydreaming about using the technique with other hard
cheeses. Asiago, Parmesan and even clothbound Cheddar would make for
fantastic versions. ``Keep a jar of it in the fridge,'' I've been
telling anyone who will listen, ``and you'll never need to make mac and
cheese from a box again!''

There was a time when I would have scoffed at the idea of bringing an
electric appliance near a bowl of pasta; at the thought of making
\emph{cacio e pepe} with any pasta shape besides the traditional
\emph{tonnarelli,} let alone farro; or at the notion of blending cheese
with water and then keeping it in the refrigerator to use later in the
week. I'm still a traditionalist. I'm just more of a pragmatist now too.
And I can see that ``stirring harder'' isn't the only path to
satisfaction for me anymore. So with the unexpected delights that tend
to appear when I relax, I'm going to take a break from this column. I'll
travel, learn, eat and cook --- and when I return next year, I'll share
it all with you. In the meantime, though, please make yourself some
\emph{farro e pepe,} and think of me.

\textbf{Recipe:}
\href{https://cooking.nytimes3xbfgragh.onion/recipes/1020356-farro-e-pepe}{Farro
e Pepe}

Advertisement

\protect\hyperlink{after-bottom}{Continue reading the main story}

\hypertarget{site-index}{%
\subsection{Site Index}\label{site-index}}

\hypertarget{site-information-navigation}{%
\subsection{Site Information
Navigation}\label{site-information-navigation}}

\begin{itemize}
\tightlist
\item
  \href{https://help.nytimes3xbfgragh.onion/hc/en-us/articles/115014792127-Copyright-notice}{©~2020~The
  New York Times Company}
\end{itemize}

\begin{itemize}
\tightlist
\item
  \href{https://www.nytco.com/}{NYTCo}
\item
  \href{https://help.nytimes3xbfgragh.onion/hc/en-us/articles/115015385887-Contact-Us}{Contact
  Us}
\item
  \href{https://www.nytco.com/careers/}{Work with us}
\item
  \href{https://nytmediakit.com/}{Advertise}
\item
  \href{http://www.tbrandstudio.com/}{T Brand Studio}
\item
  \href{https://www.nytimes3xbfgragh.onion/privacy/cookie-policy\#how-do-i-manage-trackers}{Your
  Ad Choices}
\item
  \href{https://www.nytimes3xbfgragh.onion/privacy}{Privacy}
\item
  \href{https://help.nytimes3xbfgragh.onion/hc/en-us/articles/115014893428-Terms-of-service}{Terms
  of Service}
\item
  \href{https://help.nytimes3xbfgragh.onion/hc/en-us/articles/115014893968-Terms-of-sale}{Terms
  of Sale}
\item
  \href{https://spiderbites.nytimes3xbfgragh.onion}{Site Map}
\item
  \href{https://help.nytimes3xbfgragh.onion/hc/en-us}{Help}
\item
  \href{https://www.nytimes3xbfgragh.onion/subscription?campaignId=37WXW}{Subscriptions}
\end{itemize}
