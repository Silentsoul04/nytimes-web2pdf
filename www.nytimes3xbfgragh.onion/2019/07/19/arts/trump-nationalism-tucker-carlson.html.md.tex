Sections

SEARCH

\protect\hyperlink{site-content}{Skip to
content}\protect\hyperlink{site-index}{Skip to site index}

\href{https://www.nytimes3xbfgragh.onion/section/arts}{Arts}

\href{https://myaccount.nytimes3xbfgragh.onion/auth/login?response_type=cookie\&client_id=vi}{}

\href{https://www.nytimes3xbfgragh.onion/section/todayspaper}{Today's
Paper}

\href{/section/arts}{Arts}\textbar{}Polishing the Nationalist Brand in
the Trump Era

\url{https://nyti.ms/2M2DNRu}

\begin{itemize}
\item
\item
\item
\item
\item
\item
\end{itemize}

Advertisement

\protect\hyperlink{after-top}{Continue reading the main story}

Supported by

\protect\hyperlink{after-sponsor}{Continue reading the main story}

\hypertarget{polishing-the-nationalist-brand-in-the-trump-era}{%
\section{Polishing the Nationalist Brand in the Trump
Era}\label{polishing-the-nationalist-brand-in-the-trump-era}}

Conservative thinkers are trying to bring intellectual coherence to the
Trumpian moment under the banner of nationalism. But can it be cleansed
of its darker currents?

\includegraphics{https://static01.graylady3jvrrxbe.onion/images/2019/07/19/arts/19conservative1/19conservative1-articleLarge.jpg?quality=75\&auto=webp\&disable=upscale}

\href{https://www.nytimes3xbfgragh.onion/by/jennifer-schuessler}{\includegraphics{https://static01.graylady3jvrrxbe.onion/images/2018/02/16/multimedia/author-jennifer-schuessler/author-jennifer-schuessler-thumbLarge-v2.png}}

By
\href{https://www.nytimes3xbfgragh.onion/by/jennifer-schuessler}{Jennifer
Schuessler}

\begin{itemize}
\item
  July 19, 2019
\item
  \begin{itemize}
  \item
  \item
  \item
  \item
  \item
  \item
  \end{itemize}
\end{itemize}

WASHINGTON --- Ever since Donald J. Trump laid waste to its ideological
shibboleths with his victory at the polls, the conservative intellectual
class has been scrambling to keep up with him.

And earlier this week, at the first major gathering dedicated to
wresting a coherent ideology out of the chaos of the Trumpist moment,
the president was upending their efforts again.

On Sunday evening, some 500 policy thinkers, theorists, journalists and
students gathered in a ballroom at the Ritz-Carlton here for the start
of the \href{https://nationalconservatism.org/}{National Conservatism
Conference}, a three-day event dedicated to charting a new path for
conservatism under the banner of nationalism.

And not the kind associated with tiki torches and Nazi salutes, the
conference was at pains to make clear.

``We are nationalists, not white nationalists,'' David Brog, one of the
organizers, said in his welcoming remarks, calling any equation of the
two ``a slander.'' He then pointed to the door and invited anyone who
``defines our American nation in terms of race'' who had slipped through
the conference's careful screening to leave.

But inconveniently, just a few hours earlier, President Trump had let
loose with tweets calling for four freshman congresswomen of color to
\href{http://www.apple.com/}{``go back}'' to the ``broken and crime
infested'' countries they came from, throwing an awkward wrench into the
messaging.

Not that Mr. Trump's name was mentioned in the program or the
\href{https://nationalconservatism.org/about/}{mission statement} for
the event, which was organized by the Edmund Burke Foundation, a newly
formed public affairs institute. It featured headlining speeches by
Tucker Carlson, John Bolton and Peter Thiel, as well as some three dozen
speakers on panels covering topics like immigration, foreign policy and
economic nationalism. The names of Burke and Lincoln may have been
uttered as much as the president's.

Conservatives have always prided themselves on being driven by ideas,
and the big idea here was that nationalism --- shorn of its darker
associations --- could provide an intellectual banner now that the
conservatism based on free trade, libertarian economics and military
interventionism that held sway for decades has run out of gas.

``Today is our independence day,'' Yoram Hazony, an Israeli political
theorist, author of the recent book
\href{https://www.nationalreview.com/magazine/2018/10/01/in-defense-of-nations-book-review/}{``The
Virtue of Nationalism''} and the conference's intellectual prime mover,
declared in his fiery opening remarks. ``We declare independence from
neoconservatism. We declare independence from neoliberalism, from
libertarianism, from what they call classical liberalism.''

``There is something that unites everyone in this room,'' he continued.
``We are national conservatives.''

Those in attendance may not have all agreed. They included
\href{https://www.nytimes3xbfgragh.onion/2014/07/06/magazine/can-the-gop-be-a-party-of-ideas.html}{reform
conservatives} and religious traditionalists, ardent Trumpists and
former Never-Trumpers, and more than a few unconverted free-marketeers
and others who were keeping a skeptical eye on the proceedings.

Geoffrey Kabaservice, a
\href{https://global.oup.com/academic/product/rule-and-ruin-9780199768400?cc=us\&lang=en\&}{historian
of conservatism} and director of political studies at the Niskanen
Center, described the gathering as part of an ongoing effort by
conservatives to unite ``under an ideological banner that Trump himself
doesn't carry.''

``They are trying to find a way to retroactively justify their support
of Trumpism under a broader conservative movement,'' he said. ``But
that's a tricky assignment.''

\hypertarget{detoxifying-nationalism}{%
\subsection{Detoxifying nationalism}\label{detoxifying-nationalism}}

Just how tricky was suggested by those tweets from the president, and
the muted response to them at the conference.

In the hotel bar, the national uproar over the tweets unspooled
continuously on the television (at least until it was switched to Fox
News). But in the conference sessions, there was virtually no reference
to them, and little appetite among those chattering in the halls to
offer more than tepid criticism, if that.

``They were bad,'' Rich Lowry, the editor of National Review (and a
recovering Never-Trumper), said a bit grimly, when asked about the
tweets. ``His trolling at its worst. Unproductive. Indefensible.''

Mr. Hazony, caught in the hallway between sessions, waved the question
away. ``It's a great honor to be running the intellectual part of
political conservatism,'' he said. ``We just don't have to deal with
that stuff.''

Helen Andrews, the managing editor of The Washington Examiner and
\href{https://www.firstthings.com/article/2019/08/our-socialist-moment}{a
contributor} to various conservative publications, looked puzzled when
asked on Monday about the tweets, and said she hadn't seen them.

As for nationalism, she said she saw ``no downside'' to embracing it.
``I don't think it's a word that needs to be detoxified, even as the
term conservatism sometimes needs to get detoxified,'' she said.

\includegraphics{https://static01.graylady3jvrrxbe.onion/images/2019/07/19/arts/19conservative4/merlin_158022771_738b653f-48cc-433d-9352-af00d1eb5536-articleLarge.jpg?quality=75\&auto=webp\&disable=upscale}

But some others expressed reservations about the new political brand
being road-tested.

Yuval Levin, the editor of National Affairs and a speaker at the
conference, said that the label ``national conservatism'' captured some
of his own interest in a conservatism that focuses on social health,
rather than just the market.

``But I don't think we can just go around saying nationalism is the
answer to our problems,'' he said. He added, ``People are not crazy to
worry when they hear that term.''

\hypertarget{soil-but-not-blood}{%
\subsection{Soil, but not blood}\label{soil-but-not-blood}}

When it came to defining who belonged to the nation, there was lots of
talk of soil and rootedness, alongside repeated disavowals of blood, or
its modern equivalent, DNA.

In a talk called ``Why America Is Not an Idea,'' Mr. Lowry, the author
of the forthcoming book
\href{https://www.harpercollins.com/9780062839640/the-case-for-nationalism/}{``The
Case for Nationalism,''} took aim at ``one of our most honored
clichés'': that the essence of Americanism lies only in its ideals.

The problem with this ``overintellectualized understanding of America,''
he said, is ``it slights the absolutely indispensable influence of
culture.''

Even the phrase ``city on a hill,'' an emblem of American universalism,
he said, comes from East Anglia, and is rooted in ``a particular soil, a
particular place, a particular way of thinking.'' ****

We should insist, Mr. Lowry said, ``on the assimilation of immigrants
into a common culture.'' A panel on immigration happening simultaneously
echoed that theme of culture, but with a much harder, racially
exclusionary edge. Amy Wax, a law professor at the University of
Pennsylvania who was removed from teaching first-year students last year
\href{https://www.vox.com/policy-and-politics/2018/3/21/17143150/conservative-scientific-racism-national-review}{after
writing an article} questioning the abilities of black students, offered
what she called ``the cultural case'' for reduced immigration.

She defended
\href{https://www.nytimes3xbfgragh.onion/2018/01/11/us/politics/trump-shithole-countries.html}{President
Trump's vulgar comment}last year disparaging immigration from certain
countries, to laughter and applause. And she dismissed the idea that
immigrants somehow became American simply by living here, which Ms. Wax
(borrowing a term used by white nationalists and self-described ``race
realists'') mocked as the ``magic dirt'' argument.

There's no reason that ``people who come here will quickly come to
think, live and act just like us.'' she said. Immigration policy, she
said, should take into account ``cultural compatibility.''

``In effect,'' she said, this ``means taking the position that our
country will be better off with more whites and fewer nonwhites.''

\hypertarget{vote-for-tucker}{%
\subsection{Vote for Tucker?}\label{vote-for-tucker}}

The conference was full of attacks on identity politics and
``wokeness,'' and culture-war chestnuts tended to get the biggest
applause. But the most electric response was for Mr. Carlson, who since
delivering a blistering on-air monologue in January denouncing the
``priorities of the ruling class'' has become the
\href{https://www.nytimes3xbfgragh.onion/2019/01/12/opinion/sunday/tucker-carlson-fox-news-republicans.html?module=inline}{de
facto intellectual leader}of Trumpist economic populism.

The title was ``Big Business Hates Your Family,'' and the antic Mr.
Carlson ---~who, by the way, raised then dismissed the idea that he
might run for president --- hit the theme hard.

Image

The most electric response at the conference was for Mr.
Carlson.Credit...Justin T. Gellerson for The New York Times

``The main threat to your ability to live your life as you choose, does
not come from the government, it comes from the private sector,'' he
declared. ``I can't believe I'm saying that!''He took a fresh whack at
Representative Ilhan Omar, one of the four Democratic congresswomen
targeted in Mr. Trump's tweets, whom Mr. Carlson, a few days before the
conference, attacked on air as ``living proof that the way we practice
immigration has become dangerous to this country.''

An audience member asked if he saw Sen. Elizabeth A. Warren, whose
economic plan Mr. Carlson has praised, as a ``potential ally for
national conservatism.'' He called her ``a human tragedy'' and a
``joke,'' but said her 2003 book, ``The Two-Income Trap,'' was ``one of
the best books I've ever read on economic policy.''

``The single biggest change to our society, and it got almost no press,
was the moment where it became impossible for the average person to
support a family on one income,'' he said.

\hypertarget{questioning-the-market}{%
\subsection{Questioning the market}\label{questioning-the-market}}

The once-heretical notion that the free market may not be conservatism's
friend was discussed with less shouting, and more wonkish detail, in
other sessions.

In a mock-parliamentary debate on Monday evening, Mr. Hazony called
``the House of Conservatism'' to order to consider the proposition
``America needs an industrial policy.''

Image

``There is something that unites everyone in this room,'' Yoram Hazony,
one of the organizers, said. ``You are all national
conservatives.''Credit...Justin T. Gellerson for The New York Times

Oren Cass, a senior fellow at the Manhattan Institute and the author of
\href{https://www.manhattan-institute.org/theonceandfutureworker}{``The
Once and Future Worker,''} argued the affirmative. ``Market economies
are not going to give us what they want on their own,'' he said.

His opponent was Richard Reinsch, the editor of Law \& Liberty, a
website dedicated to the classical liberal tradition.

The debate unfolded in a dense blizzard of references to economic
indicators and quotes from the Federalist papers. In his summation, Mr.
Reinsch gave a flash of irritated incredulity at the idea, increasingly
expressed by conservatives, that ``the market doesn't work.''

``Why do you think
\href{https://www.nytimes3xbfgragh.onion/2018/03/26/us/politics/trump-judges-courts-administrative-state.html}{the
administrative state} is going to deliver, given the way it functions
now?'' he said. ``I thought we were conservatives.''

In a vote, Mr. Cass's side won handily, 99 in favor and 51 against.

The argument was picked up again at a panel called ``What Is Economic
Nationalism?'' Julius Krein, the editor of the journal
\href{https://www.nytimes3xbfgragh.onion/2017/03/08/arts/american-affairs-journal-donald-trump.html}{American
Affairs}, mixed arguments about supply chains and productivity with jabs
at ``some of the dimmer bulbs in the Koch network,'' and the free-market
purism they espouse.

J.D. Vance, the author of ``Hillbilly Elegy,'' who himself delivered an
emotional, stump-speech-like critique of libertarianism earlier in the
day, called Mr. Krein's talk a high point of the conference.

``That's not the sort of thing you're used to hearing at a gathering of
conservatives,'' Mr. Vance said. ``His speech more than anything just
goes to show how much different thinking there is on the right these
days.''

\hypertarget{guarding-the-bridge}{%
\subsection{Guarding the bridge}\label{guarding-the-bridge}}

The closing dinner brought a lone emissary from the world of electoral
politics: Josh Hawley of Missouri, the youngest member of the Senate,
who delivered a stentorian attack on ``the cosmopolitan agenda that has
powered both left and right.''

He denounced the corporate greed that had hollowed out the heartland and
the ``mountain of debt'' heaped on young people by ``the
higher-education monopoly.'' And he ended with the story of Horatius, a
soldier from the early days of the Roman republic who guarded a bridge
into Rome against the invading Etruscans, allowing the city to reset its
defenses, thus saving the republic.

``Let us stand with the conviction of Horatius,'' Mr. Hawley said. ``Who
will stand on either hand and keep this bridge with me?''

Mr. Hazony took the stage for his closing speech, which included a long
disquisition tracing the idea of nationalism to the Book of Genesis.
Among those who stepped out for a break was Charles Kesler, the editor
of the Claremont Review of Books, one of the few elite conservative
publications
\href{https://www.nytimes3xbfgragh.onion/2017/02/20/arts/charge-the-cockpit-or-you-die-behind-an-incendiary-case-for-trump.html}{to
publish pro-Trump arguments before the election}.

``I'm a little more impressed with the market,'' he said when asked
about Mr. Hawley's speech. As for whether national conservatism could
hold the bridge, he called the conference ``a promising start.''

``It's an attempt to make something thoughtful and lasting out of this
strange, remarkable, three years, one term maybe, Trump,'' he said.

He laughed heartily. ``We'll see.''

Advertisement

\protect\hyperlink{after-bottom}{Continue reading the main story}

\hypertarget{site-index}{%
\subsection{Site Index}\label{site-index}}

\hypertarget{site-information-navigation}{%
\subsection{Site Information
Navigation}\label{site-information-navigation}}

\begin{itemize}
\tightlist
\item
  \href{https://help.nytimes3xbfgragh.onion/hc/en-us/articles/115014792127-Copyright-notice}{©~2020~The
  New York Times Company}
\end{itemize}

\begin{itemize}
\tightlist
\item
  \href{https://www.nytco.com/}{NYTCo}
\item
  \href{https://help.nytimes3xbfgragh.onion/hc/en-us/articles/115015385887-Contact-Us}{Contact
  Us}
\item
  \href{https://www.nytco.com/careers/}{Work with us}
\item
  \href{https://nytmediakit.com/}{Advertise}
\item
  \href{http://www.tbrandstudio.com/}{T Brand Studio}
\item
  \href{https://www.nytimes3xbfgragh.onion/privacy/cookie-policy\#how-do-i-manage-trackers}{Your
  Ad Choices}
\item
  \href{https://www.nytimes3xbfgragh.onion/privacy}{Privacy}
\item
  \href{https://help.nytimes3xbfgragh.onion/hc/en-us/articles/115014893428-Terms-of-service}{Terms
  of Service}
\item
  \href{https://help.nytimes3xbfgragh.onion/hc/en-us/articles/115014893968-Terms-of-sale}{Terms
  of Sale}
\item
  \href{https://spiderbites.nytimes3xbfgragh.onion}{Site Map}
\item
  \href{https://help.nytimes3xbfgragh.onion/hc/en-us}{Help}
\item
  \href{https://www.nytimes3xbfgragh.onion/subscription?campaignId=37WXW}{Subscriptions}
\end{itemize}
