Sections

SEARCH

\protect\hyperlink{site-content}{Skip to
content}\protect\hyperlink{site-index}{Skip to site index}

\href{https://myaccount.nytimes3xbfgragh.onion/auth/login?response_type=cookie\&client_id=vi}{}

\href{https://www.nytimes3xbfgragh.onion/section/todayspaper}{Today's
Paper}

\href{/section/opinion}{Opinion}\textbar{}The Who-Can-Beat Trump Test
Leads to Kamala Harris

\url{https://nyti.ms/2T0dzR4}

\begin{itemize}
\item
\item
\item
\item
\item
\item
\end{itemize}

Advertisement

\protect\hyperlink{after-top}{Continue reading the main story}

\href{/section/opinion}{Opinion}

Supported by

\protect\hyperlink{after-sponsor}{Continue reading the main story}

\hypertarget{the-who-can-beat-trump-test-leads-to-kamala-harris}{%
\section{The Who-Can-Beat Trump Test Leads to Kamala
Harris}\label{the-who-can-beat-trump-test-leads-to-kamala-harris}}

Bringing the energy and hope to stare down Trump and his movement.

\href{https://www.nytimes3xbfgragh.onion/by/roger-cohen}{\includegraphics{https://static01.graylady3jvrrxbe.onion/images/2014/11/01/opinion/cohen-circular/cohen-circular-thumbLarge-v6.png}}

By \href{https://www.nytimes3xbfgragh.onion/by/roger-cohen}{Roger Cohen}

Opinion Columnist

\begin{itemize}
\item
  Aug. 2, 2019
\item
  \begin{itemize}
  \item
  \item
  \item
  \item
  \item
  \item
  \end{itemize}
\end{itemize}

\includegraphics{https://static01.graylady3jvrrxbe.onion/images/2019/08/05/opinion/02Cohen-web/merlin_157871103_29929c0e-64e7-4b72-b1f1-7e5fd2b3a6ac-articleLarge.jpg?quality=75\&auto=webp\&disable=upscale}

Nations, like people, may change somewhat, but not in their essential
characteristics. The United States is defined by space and hope. It is
an optimistic country of can-do strivers. They took the risk of coming
to a new land. They are suspicious of government, inclined to
self-reliance. Europeans ask where you came from. Americans ask what you
can do.

The Declaration of Independence posited a universal idea, that human
beings are created equal, that they are endowed with certain inalienable
rights, and that among these are ``life, liberty and the pursuit of
happiness.'' Americans, then, embraced an idea, however flawed in
execution, when they became a nation. Their government, whatever else it
does, exists to safeguard and further that idea, in the United States
and beyond.

President Trump, in the name of making American great again, has
trampled on America's essence. He is angry, a stranger to
\emph{happiness}, angrier still for not knowing the source of his rage.
He is less interested in \emph{liberty} than the cash of his autocratic
cronies. As for \emph{life}, he views it as a selective right, to which
the white Christian male has priority access, with women, people of
color and the rest of humanity trailing along behind for scraps.

Adherents to an agenda of ``national conservatism'' held a conference
last month in Washington dedicated, as
\href{https://www.nytimes3xbfgragh.onion/2019/07/19/arts/trump-nationalism-tucker-carlson.html?rref=collection\%2Fbyline\%2Fjennifer-schuessler\&action=click\&contentCollection=undefined\&region=stream\&module=stream_unit\&version=latest\&contentPlacement=1\&pgtype=collection}{my
colleague Jennifer Schuessler put it}, ``to wresting a coherent ideology
out of the chaos of the Trumpist moment.''

Good luck with that. One of the meeting's leading lights was Rich Lowry,
the editor of National Review. Lowry's forthcoming book is called ``The
Case for Nationalism.'' Enough said. The endpoint of that ``case'' is on
display at military cemeteries across Europe.

Nationalism, self-pitying and aggressive, seeks to change the present in
the name of an illusory past in order to create a future vague in all
respects except its glory. Trump is a self-styled nationalist. The
``U.S.A.! U.S.A.! U.S.A.!'' chants at his rallies have chilling echoes.

Lowry holds that ``America is not an idea'' and to
\href{https://twitter.com/richlowry/status/952210040462741505?lang=en}{call
it so is a ``lazy cliché.}'' This argument denies the essence of the
country --- an essence palpable at every naturalization ceremony across
the United States. Becoming American is a process that involves the
inner absorption of the nation's founding idea.

The gravest thing Trump has done is to empty this idea of meaning. His
has been an assault on honesty, decency, dignity, tolerance and
civility. On this president's wish list, every right is alienable. He
leads a movement more than he does a nation, and so depends on fear to
mobilize people. Any victorious Democratic Party candidate in 2020 has
to counter that negative energy with a positive energy that lifts
Americans from Trump's web.

I watched the Democratic Party debates among presidential contenders
through a single prism: Who can beat Trump? In the end, nothing else
matters because another five and a half years of this will drag
Americans into an abyss of moral collapse.

Yes, how far left, how moderate that candidate may be is of some
significance, but can he or she bring the heat and the hope to stare
Trump down and topple him is all I care about. That's the bouncing ball
all eyes should be on, with no illusions as to how vicious and devious
Trump will be between now and November 2020.

With reluctance, because he is a good and honorable man of great
personal courage, I do not believe that Joe Biden has the needed energy,
mental agility and nimbleness. Nor do I believe that the nation of
can-do strivers I described above is ready for Bernie Sanders's
``democratic socialism.'' Forms of socialism work in Europe, and the
word is widely misunderstood in America, but socialism and America's
essence are incompatible.

Elizabeth Warren's couching of a campaign for radical change as
``economic patriotism'' is a much smarter way to go, and her energetic
advocacy of ideas to redress the growing injustices in American life has
been powerful. Still, I am not convinced that enough Americans are ready
to move as far left as she proposes or that she passes the critical
commander in chief test.

Kamala Harris does that for me. The California senator is a work in
progress, with uneven debate performances, and policies, notably health
care, that she has zigzagged toward defining. But she's tough, broadly
of the center, has a great American story, is passionate on issues
including immigrants, African-Americans and women, and has proved she is
not averse to risk. She has a former prosecutor's toughness and the
ability to slice through Trump's self-important bluster.

Last month Harris said
\href{https://www.usatoday.com/story/news/politics/elections/2019/07/03/kamala-harris-trump-predator/1644504001/}{Trump
was a ``predator.''} She continued: ``The thing about predators you
should know, is that they prey on the vulnerable. They prey on those who
they do not believe are strong. And the thing you must importantly know,
predators are cowards.''

Those were important words. It's early days, but Trump's biggest
electoral vulnerability is to women. They have seen through his misogyny
at last, and they know just where the testosterone of nationalism leads.

\emph{The Times is committed to publishing}
\href{https://www.nytimes3xbfgragh.onion/2019/01/31/opinion/letters/letters-to-editor-new-york-times-women.html}{\emph{a
diversity of letters}} \emph{to the editor. We'd like to hear what you
think about this or any of our articles. Here are some}
\href{https://help.nytimes3xbfgragh.onion/hc/en-us/articles/115014925288-How-to-submit-a-letter-to-the-editor}{\emph{tips}}\emph{.
And here's our email:}
\href{mailto:letters@NYTimes.com}{\emph{letters@NYTimes.com}}\emph{.}

\emph{Follow The New York Times Opinion section on}
\href{https://www.facebookcorewwwi.onion/nytopinion}{\emph{Facebook}}\emph{,}
\href{http://twitter.com/NYTOpinion}{\emph{Twitter (@NYTopinion)}}
\emph{and}
\href{https://www.instagram.com/nytopinion/}{\emph{Instagram}}\emph{.}

Advertisement

\protect\hyperlink{after-bottom}{Continue reading the main story}

\hypertarget{site-index}{%
\subsection{Site Index}\label{site-index}}

\hypertarget{site-information-navigation}{%
\subsection{Site Information
Navigation}\label{site-information-navigation}}

\begin{itemize}
\tightlist
\item
  \href{https://help.nytimes3xbfgragh.onion/hc/en-us/articles/115014792127-Copyright-notice}{©~2020~The
  New York Times Company}
\end{itemize}

\begin{itemize}
\tightlist
\item
  \href{https://www.nytco.com/}{NYTCo}
\item
  \href{https://help.nytimes3xbfgragh.onion/hc/en-us/articles/115015385887-Contact-Us}{Contact
  Us}
\item
  \href{https://www.nytco.com/careers/}{Work with us}
\item
  \href{https://nytmediakit.com/}{Advertise}
\item
  \href{http://www.tbrandstudio.com/}{T Brand Studio}
\item
  \href{https://www.nytimes3xbfgragh.onion/privacy/cookie-policy\#how-do-i-manage-trackers}{Your
  Ad Choices}
\item
  \href{https://www.nytimes3xbfgragh.onion/privacy}{Privacy}
\item
  \href{https://help.nytimes3xbfgragh.onion/hc/en-us/articles/115014893428-Terms-of-service}{Terms
  of Service}
\item
  \href{https://help.nytimes3xbfgragh.onion/hc/en-us/articles/115014893968-Terms-of-sale}{Terms
  of Sale}
\item
  \href{https://spiderbites.nytimes3xbfgragh.onion}{Site Map}
\item
  \href{https://help.nytimes3xbfgragh.onion/hc/en-us}{Help}
\item
  \href{https://www.nytimes3xbfgragh.onion/subscription?campaignId=37WXW}{Subscriptions}
\end{itemize}
