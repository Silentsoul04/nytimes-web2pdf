Sections

SEARCH

\protect\hyperlink{site-content}{Skip to
content}\protect\hyperlink{site-index}{Skip to site index}

\href{https://www.nytimes3xbfgragh.onion/section/style}{Style}

\href{https://myaccount.nytimes3xbfgragh.onion/auth/login?response_type=cookie\&client_id=vi}{}

\href{https://www.nytimes3xbfgragh.onion/section/todayspaper}{Today's
Paper}

\href{/section/style}{Style}\textbar{}Why Sexually Transmitted
Infections Can't Shake Their Stigma

\begin{itemize}
\item
\item
\item
\item
\item
\item
\end{itemize}

Advertisement

\protect\hyperlink{after-top}{Continue reading the main story}

Supported by

\protect\hyperlink{after-sponsor}{Continue reading the main story}

The Cycle

\hypertarget{why-sexually-transmitted-infections-cant-shake-their-stigma}{%
\section{Why Sexually Transmitted Infections Can't Shake Their
Stigma}\label{why-sexually-transmitted-infections-cant-shake-their-stigma}}

We live in an era of sex positivity --- until we get positive test
results. And that's unfortunate, because S.T.I.s are on the rise.

\includegraphics{https://static01.graylady3jvrrxbe.onion/images/2019/08/25/fashion/13Cycle-sti-1/13Cycle-sti-1-articleLarge.jpg?quality=75\&auto=webp\&disable=upscale}

By Jen Gunter

\begin{itemize}
\item
  Aug. 13, 2019
\item
  \begin{itemize}
  \item
  \item
  \item
  \item
  \item
  \item
  \end{itemize}
\end{itemize}

As an obstetrician and gynecologist, I get a lot of panicked calls and
direct messages on social media --- sometimes from acquaintances,
sometimes from complete strangers --- about gynecological matters. Pap
smear results, missed periods and hot flashes are my bread and butter.
Also common: abortion complications, lost condoms and
\href{https://www.nytimes3xbfgragh.onion/2018/06/11/style/dont-put-this-up-there.html}{misadventures
with food}.

I think people contact me directly because they don't have access to
health care, because they are embarrassed and think (erroneously) they
are the only ones to experience such a misadventure, because they worry
their providers will be judgmental (sadly, this is not uncommon),
because there is so much misinformation online that it is hard to
separate the good from the bad, or simply because they're really scared
and it's 3 a.m. and just maybe I'll answer.

I'm proud that people confide in me, not just as a medical professional,
but also as a nonjudgmental person. It seems that almost nothing is off
the table.

Except one thing: sexually transmitted infections (S.T.I.s).

In almost 30 years of specializing in gynecology and obstetrics, I have
been asked only twice about them outside of the office.

It seems S.T.I.s are one of the last taboos.

(Incidentally, some people --- including the C.D.C. --- still use the
term ``sexually transmitted diseases.'' However, ``sexually transmitted
infectious diseases'' and ``sexually transmitted infections'' are also
considered acceptable. I feel the word ``disease'' in this context
sounds stigmatizing, which is why I don't use it.)

I don't need to tell you that we live in an era of easy access to sex:
Pornography is a click away and often free. Full frontal nudity and
graphically simulated sex scenes are part of many television shows and
movies. Popular magazines for women have
\href{https://www.cosmopolitan.com/sex-love/advice/a6676/anal-sex-beginners-guide/}{explicit
articles} on fellatio and anal sex. People are very specific on dating
apps about looking for hookups.

I want to be very clear that none of this is wrong. You like what you
like, whether that is how you choose to be entertained, your fantasies
or the way you engage sexually. If it's all consensual, it's all good.

We are also able to talk more publicly about other aspects of sex, like
contraception choices and abortion (\#ShoutYourAbortion is one of many
hashtags). And yet there seems to be a hard line at S.T.I.s.

What makes this especially surprising is that S.T.I.s are so ubiquitous.
Consider that
\href{http://www.ashasexualhealth.org/stdsstis/statistics/}{50 percent
of sexually active people} will have at least one S.T.I. by age 25 (HPV
is the most common) and there are
\href{https://www.cdc.gov/std/stats/sti-estimates-fact-sheet-feb-2013.pdf}{over
110 million new and existing S.T.I.} cases each year in the United
States. People are clearly not shy sharing with me, so the only logical
conclusion is the sexual revolution stopped short of liberating people
from the shame and stigma of sexually transmitted infections.

I see this reflected in my day-to-day work. No diagnosis, apart from
cancer, can as reliably bring a woman to tears as an S.T.I. Especially
when it's herpes. (For some reason, and I don't know the definitive
answer, there is a lot of stigma around herpes. In 1982, Time magazine
\href{https://www.amazon.com/Magazine-August-Todays-Scarlet-Letter/dp/B000LD6V4K}{touted
it} as ``Today's Scarlet Letter'' on a cover.)

\textbf{\href{https://www.nytimes3xbfgragh.onion/newsletters/wait}{\emph{Sign
up here to receive Wait ---}}\emph{, a newsletter that brings you
stories about money, power, sex and scrunchies.}}

Not much has changed in the last four decades. I have been a specialist
in women's health with a focus on infectious diseases for 24 years, and
the conversations I have about herpes (or trichomoniasis or gonorrhea or
chlamydia --- other common S.T.I.s) have not changed.

An S.T.I. somehow makes many woman feel as if they are damaged goods. In
many ways our society still thinks of women as ``loose'' when they have
sex before a certain age or if they have multiple partners. That
construct never seems to apply to straight men. It is also women who
bear the fertility and pregnancy ramifications of S.T.I.s.

The viral S.T.I.s that persist seem to hit people the hardest. The idea
of an infection that you can't get rid is very challenging for many
people to accept.

Consider the contrasting reactions to infection with human papilloma
virus (HPV), the cause of cervical cancer and genital warts (as well as
some vaginal, vulvar, oral and anal cancers), versus the diagnosis of
Epstein Barr Virus (EBV), the cause of infectious mononucleosis.

Biologically the viruses are very similar. They can persist for years,
hibernating in cells. They can both reactivate --- wake up, if you will
--- and this causes people to shed the virus and hence spread it
unknowingly.

Many people may never know they even had the infection to begin with, or
if they did they do not know they are shedding the virus and hence are
infectious. This phenomenon is called asymptomatic transmission, and is
how we mostly pass these infections to other people.

(As an aside, people typically don't transmit S.T.I.s knowingly. Most
people have the decency not to have sex when they have a visible sore,
for instance, just as they know not to shake hands when they have a
killer cold.)

Despite the basic biological similarities between HPV and EBV, it is
only \href{https://www.ncbi.nlm.nih.gov/pmc/articles/PMC5763398/}{HPV
that is associated with shame}. It is HPV that is the virus transmitted
by sex, genital contact and oral sex, while EBV is close contact and
kissing.

Why should it be any more shameful to catch an infection from sex than
it is from shaking hands, a kiss or being coughed upon? Why is it
shameful to have genital herpes, even though more people have oral
herpes and everyone can see those outbreaks?

I suspect it is because shame and stigma are effective weapons of
control that have been used throughout history to marginalize women,
people of color and the L.G.B.T.Q.+ community. S.T.I.s are generally
higher in these groups: a combination of biology, as transmission to the
cervix, vagina and rectum is easiest for most S.T.I.s, and traditionally
people in these groups have less access to health care because of
economic marginalization or prejudice, which leads to less access to
screening and treatment.

In the 1800s when an indigent person had the ``pox'' (either syphilis or
gonorrhea; diagnostics to distinguish reliably between the two did not
exist at the time), a public declaration was required for assistance
from the churchwardens. Men and women with the same afflictions were
sent to different places ---
women\href{https://journalofethics.ama-assn.org/article/hivaids-stigma-historical-perspectives-sexually-transmitted-diseases/2005-10}{to
a workhouse} and men to a hospital.

Those with money could, of course, avoid public disclosure altogether
and obtain medical care.

In many ways the public shame and stigma has remained unchanged.
Economic disadvantage prevents many from screening and treatment. People
wishing to protect themselves from S.T.I.s through safer sex practices
--- especially those who are not cis gendered white heterosexual men ---
can be falsely labeled promiscuous or dirty (whatever that means). Or
both.

Those words can be thrown at them by their partners,
\href{https://www.ncbi.nlm.nih.gov/pmc/articles/PMC4566537/}{their
community}, strangers online and even via stigmatizing interactions with
health care professionals.

The consequence is that many S.T.I.s are on the rise.
\href{https://www.cdc.gov/media/releases/2018/p0828-increases-in-stds.html}{Between
2013 and 2017} (the last year for data) the number of cases of gonorrhea
have increased by 67 percent and syphilis by 76 percent.

And while the rates of new diagnosis of H.I.V. are stable, there are
still
\href{https://www.cdc.gov/hiv/statistics/overview/ataglance.html}{over
30,000 cases of new H.I.V. infection}s acquired sexually each year in
the United States, almost every one of them preventable with
\href{https://www.cdc.gov/hiv/risk/prep/index.html}{pre-exposure
prophylaxis} (PrEP). The only bright spot is the reduction in
\href{https://www.nytimes3xbfgragh.onion/2019/06/27/health/hpv-vaccine-warts-cancer.html}{HPV-related
infections} because of vaccination.

Having an S.T.I. should have the same stigma as having influenza,
meaning none. Making people ashamed or judging them for their choices
simply means they are less likely to be screened, treated and get the
care that can prevent infections and save lives.

I don't want that for anyone.

\begin{center}\rule{0.5\linewidth}{\linethickness}\end{center}

Dr. Jen Gunter is an obstetrician and gynecologist practicing in
California. She writes
\href{https://www.nytimes3xbfgragh.onion/column/the-cycle}{The Cycle}, a
column on women's health that appears regularly in Styles. She is also
the author of a forthcoming book called
``\href{https://www.kensingtonbooks.com/book.aspx/37809}{The Vagina
Bible}.''

Advertisement

\protect\hyperlink{after-bottom}{Continue reading the main story}

\hypertarget{site-index}{%
\subsection{Site Index}\label{site-index}}

\hypertarget{site-information-navigation}{%
\subsection{Site Information
Navigation}\label{site-information-navigation}}

\begin{itemize}
\tightlist
\item
  \href{https://help.nytimes3xbfgragh.onion/hc/en-us/articles/115014792127-Copyright-notice}{©~2020~The
  New York Times Company}
\end{itemize}

\begin{itemize}
\tightlist
\item
  \href{https://www.nytco.com/}{NYTCo}
\item
  \href{https://help.nytimes3xbfgragh.onion/hc/en-us/articles/115015385887-Contact-Us}{Contact
  Us}
\item
  \href{https://www.nytco.com/careers/}{Work with us}
\item
  \href{https://nytmediakit.com/}{Advertise}
\item
  \href{http://www.tbrandstudio.com/}{T Brand Studio}
\item
  \href{https://www.nytimes3xbfgragh.onion/privacy/cookie-policy\#how-do-i-manage-trackers}{Your
  Ad Choices}
\item
  \href{https://www.nytimes3xbfgragh.onion/privacy}{Privacy}
\item
  \href{https://help.nytimes3xbfgragh.onion/hc/en-us/articles/115014893428-Terms-of-service}{Terms
  of Service}
\item
  \href{https://help.nytimes3xbfgragh.onion/hc/en-us/articles/115014893968-Terms-of-sale}{Terms
  of Sale}
\item
  \href{https://spiderbites.nytimes3xbfgragh.onion}{Site Map}
\item
  \href{https://help.nytimes3xbfgragh.onion/hc/en-us}{Help}
\item
  \href{https://www.nytimes3xbfgragh.onion/subscription?campaignId=37WXW}{Subscriptions}
\end{itemize}
