Sections

SEARCH

\protect\hyperlink{site-content}{Skip to
content}\protect\hyperlink{site-index}{Skip to site index}

\href{https://www.nytimes3xbfgragh.onion/section/us}{U.S.}

\href{https://myaccount.nytimes3xbfgragh.onion/auth/login?response_type=cookie\&client_id=vi}{}

\href{https://www.nytimes3xbfgragh.onion/section/todayspaper}{Today's
Paper}

\href{/section/us}{U.S.}\textbar{}Fighting Hunger in the Klamath River
Basin

\url{https://nyti.ms/2H5iVpC}

\begin{itemize}
\item
\item
\item
\item
\item
\end{itemize}

Advertisement

\protect\hyperlink{after-top}{Continue reading the main story}

Supported by

\protect\hyperlink{after-sponsor}{Continue reading the main story}

California Today

\hypertarget{fighting-hunger-in-the-klamath-river-basin}{%
\section{Fighting Hunger in the Klamath River
Basin}\label{fighting-hunger-in-the-klamath-river-basin}}

Wednesday: Native Americans have some of the nation's highest rates of
food insecurity. Also: Another round of California v. Trump; and a
special Berkeley cafe.

\href{https://www.nytimes3xbfgragh.onion/by/jill-cowan}{\includegraphics{https://static01.graylady3jvrrxbe.onion/images/2018/12/10/multimedia/author-jill-cowan/author-jill-cowan-thumbLarge.png}}

By \href{https://www.nytimes3xbfgragh.onion/by/jill-cowan}{Jill Cowan}

\begin{itemize}
\item
  Aug. 14, 2019
\item
  \begin{itemize}
  \item
  \item
  \item
  \item
  \item
  \end{itemize}
\end{itemize}

\includegraphics{https://static01.graylady3jvrrxbe.onion/images/2018/08/21/us/14klamathcatoday/merlin_141878241_37bbbca5-1915-4b67-bd5c-01d96ff26004-articleLarge.jpg?quality=75\&auto=webp\&disable=upscale}

\emph{Good morning.}

\emph{(If you don't get California Today by email,}
\href{https://www.nytimes3xbfgragh.onion/newsletters/california-today}{\emph{here's
the sign-up}}\emph{.)}

It worked out well that Lisa Hillman was at the tiny airport in
McKinleyville, Humboldt County, when I called her to talk recently. She
usually doesn't have cell service.

That's because Ms. Hillman, a member of the
\href{http://www.karuk.us/}{Karuk Tribe} who works as program manager
for its Píkyav Field Institute, lives two hours away in the town of
Orleans, along the Klamath River.

Also a long drive away from her home: The nearest grocery store.

``They just opened up a supermarket in Hoopa, which is 40 minutes
away,'' she told me. ``But they're all small and they're all super
expensive.''

The tribes, like the Karuk, who live in the vast, towering forests of
the Klamath River Basin --- who have for centuries hunted deer and
gathered acorns, who knew how to weave baskets to catch once-plentiful
salmon --- now face food shortages at higher rates than almost anywhere
else in the country.

\emph{{[}Read more about how}
\href{https://www.nytimes3xbfgragh.onion/2018/09/04/us/klamath-river-california-tribes-heroin.html}{\emph{a
sick river and heroin abuse have plagued communities}} \emph{in the
Klamath River Basin.{]}}

While 11.8 percent of households nationally experience some level of
food insecurity,
\href{https://link.springer.com/article/10.1007/s12571-019-00925-y}{a
recent federally funded five-year study} found that 92 percent of the
households in the Klamath Basin suffer from some kind of food
insecurity. Almost 65 percent rely on food assistance, compared with 12
percent nationally.

But the research, which Ms. Hillman worked on along with academics from
U.C. Berkeley, also found that those community members lacked access to
indigenous foods --- and those could better feed those communities
today.

Native foods, according to the federal definition, are plants or animals
that are hunted, harvested, gathered, grown or prepared using
traditional Native American methods. Such foods can be wild or
cultivated and they're specific to locations and distinct cultures.

Ms. Hillman said her grandparents were taught to be ashamed to be Native
American. They were sent to a faraway boarding school.

``They did the best they could,'' she said. ``And they taught their kids
how to survive in white culture.''

Subsequent generations, as a result, haven't gotten the lifelong
education in the kind of traditional practices they'd need to cultivate
a diet on native foods.

Then there's the fact that while the tribes once had unfettered access
to millions of acres, over decades, that land has been effectively
closed off to the people who were its first stewards by the federal
government.

\emph{{[}Read more about}
\href{https://www.nytimes3xbfgragh.onion/2019/06/19/us/newsom-native-american-apology.html?smid=nytcore-ios-share}{\emph{Gov.
Gavin Newsom's apology to the state's Native Americans}} \emph{and about
tribal leaders' responses.{]}}

Climate change and poor forest management have made the lands that are
left less fertile for food sources. As my colleague Jose Del Real
\href{https://www.nytimes3xbfgragh.onion/2018/09/04/us/klamath-river-california-tribes-heroin.html}{reported
last year}, salmon runs have declined.

On top of all that, there aren't as many employment opportunities in
remote areas, so all people can afford, Ms. Hillman said, are unhealthy
commodity foods: white flour, processed sugar and milk in communities
where most people are lactose intolerant.

``It's absolutely a dead ringer for diabetes and heart disease and
obesity,'' Ms. Hillman said.

But Ms. Hillman said she sees promise for
\href{https://civileats.com/2019/07/24/indigenous-food-security-is-dependent-on-food-sovereignty/}{Native
American agriculture in the most recent Farm Bill}.

She said that when she sits down to eat with her six kids, they have
vegetables from a garden that Ms. Hillman said is bursting right now,
along with venison and jarred acorns.

\hypertarget{heres-what-else-were-following}{%
\subsection{Here's what else we're
following}\label{heres-what-else-were-following}}

\emph{(We often link to sites that limit access for nonsubscribers. We
appreciate your reading Times coverage, but we also encourage you to
support local news if you can.)}

Image

Kenneth T. Cuccinelli II, acting director of the United States
Citizenship and Immigration Services, said a new rule was meant to favor
immigrants ``who can stand on their own two feet.''Credit...T.J.
Kirkpatrick for The New York Times

--- \textbf{San Francisco and Santa Clara Counties sued} in an attempt
to block the Trump administration from implementing a rule that would
deny permanent residency to immigrants if they were deemed likely to use
government benefits.
{[}\href{https://www.nytimes3xbfgragh.onion/2019/08/13/us/trump-green-card-lawsuit.html}{The
New York Times}{]}

\emph{Also: Janet Napolitano, president of the University of
California,}
\href{https://www.universityofcalifornia.edu/press-room/uc-statement-final-rule-regarding-public-charge}{\emph{said
in a statement}} \emph{that the rule would potentially scare off the
world's ``best and brightest'' scholars, who she said conduct important
research and ``contribute substantially to the economy.''}

--- In another battle, California and more than two dozen other states
and cities \textbf{sued the Trump administration over its plan to roll
back restrictions on coal-burning power plants}. The case could have
far-reaching ramifications for how the federal government can fight
climate change in the future.
{[}\href{https://www.nytimes3xbfgragh.onion/2019/08/13/climate/states-lawsuit-clean-power-ace.html}{The
New York Times}{]}

---
\href{https://www.pe.com/2019/08/13/slain-riverside-area-chp-officer-andre-moye-jr-remembered-as-very-giving-and-caring-person/}{Andre
Moye Jr.}, the California Highway Patrol officer who was \textbf{killed
in what officials described as a ``long and horrific'' gun battle, was
remembered for his ``service heart.''}
{[}\href{https://www.pe.com/2019/08/13/eastridge-avenue-still-closed-for-investigation-into-fatal-shooting-of-chp-officer/}{The
Press-Enterprise}{]}

--- \textbf{California is considering making ethnic studies mandatory.}
But details about the curriculum are sparking debates.
{[}\href{https://www.latimes.com/california/story/2019-08-12/california-ethnic-studies-curriculum}{The
Los Angeles Times{]}}

--- \textbf{Plácido Domingo has been placed under investigation by the
Los Angeles Opera}, which he helped found and has led, following
\href{https://www.apnews.com/c2d51d690d004992b8cfba3bad827ae9}{a report
by The Associated Press} that he sexually harassed women over years. The
San Francisco Opera canceled a concert with him in October.
{[}\href{https://www.nytimes3xbfgragh.onion/2019/08/13/arts/music/placido-domingo-sexual-harassment-opera.html}{The
New York Times}{]}

--- At Beautycon in L.A., Priyanka Chopra probably expected to field
questions about her self-care routine and female empowerment. Instead,
another beauty influencer asked her about a tweet in which \textbf{she
referenced the tensions between India and Pakistan}.
{[}\href{https://www.nytimes3xbfgragh.onion/2019/08/13/style/priyanka-chopra-beautycon-india-pakistan.html}{The
New York Times}{]}

--- After \textbf{San Diego paid out \$1.7 million in a legal settlement
related to a Segway injury}, the city is imposing new safety and
insurance requirements on tour companies.
{[}\href{https://www.sandiegouniontribune.com/communities/san-diego/story/2019-08-12/san-diego-cracking-down-on-segways-to-curb-injury-payouts}{The
San Diego Union-Tribune}{]}

\emph{Also, here's more about the emerging realm of e-scooter liability.
{[}}\href{https://www.nytimes3xbfgragh.onion/2019/02/06/us/california-today-scooter-lawsuits-liability.html}{\emph{The
New York Times}}\emph{{]}}

--- \textbf{Snoop Dogg made good on a promise to throw an after party
for the 30th reunion for Long Beach Poly High} class of 1989. Attendees
said it was as fun as it sounds.
{[}\href{https://www.presstelegram.com/2019/08/09/snoop-dogg-gives-fellow-long-beach-poly-high-class-of-89-alums-the-party-of-a-lifetime-for-30-year-reunion/}{The
Press-Telegram}{]}

\hypertarget{and-finally-}{%
\subsection{And Finally \ldots{}}\label{and-finally-}}

Image

Alicia Adams-Potts serves a platter of oyster mushrooms and onions
roasted in walnut oil.Credit...Jason Henry for The New York Times

It's not a complete fix for food insecurity among Native Americans, but
it could be a start: My colleague
\href{https://www.nytimes3xbfgragh.onion/by/tejal-rao}{Tejal Rao}
recently feasted at Berkeley's \href{https://www.makamham.com/}{Cafe
Ohlone} and came away recommending that we do, too.

It's a ``small, enchanting restaurant that pops up a few times a week
behind a bookshop,''
\href{https://www.nytimes3xbfgragh.onion/2019/08/12/dining/cafe-ohlone-review-berkeley.html}{she
wrote.} It's where a member of the Muwekma Ohlone Tribe of the Bay Area
lovingly recovers native cuisine and uses its ingredients as inspiration
for new dishes.

Cafe Ohlone is worth a visit, Tejal wrote, ``not only to eat, but to
listen.''

\emph{California Today goes live at 6:30 a.m. Pacific time weekdays.
Tell us what you want to see:}
\href{mailto:CAtoday@NYTimes.com}{\emph{CAtoday@NYTimes.com}}\emph{.
Were you forwarded this email?}
\href{https://www.nytimes3xbfgragh.onion/newsletters/california-today?module=inline}{Sign
up for California Today here.}

\emph{Jill Cowan grew up in Orange County, went to school at U.C.
Berkeley and has reported all over the state, including the Bay Area,
Bakersfield and Los Angeles --- but she always wants to see more. Follow
along here or on Twitter,}
\href{https://twitter.com/JillCowan}{\emph{@jillcowan}}\emph{.}

\emph{California Today is edited by Julie Bloom, who grew up in Los
Angeles and graduated from U.C. Berkeley.}

\begin{center}\rule{0.5\linewidth}{\linethickness}\end{center}

\begin{center}\rule{0.5\linewidth}{\linethickness}\end{center}

Advertisement

\protect\hyperlink{after-bottom}{Continue reading the main story}

\hypertarget{site-index}{%
\subsection{Site Index}\label{site-index}}

\hypertarget{site-information-navigation}{%
\subsection{Site Information
Navigation}\label{site-information-navigation}}

\begin{itemize}
\tightlist
\item
  \href{https://help.nytimes3xbfgragh.onion/hc/en-us/articles/115014792127-Copyright-notice}{©~2020~The
  New York Times Company}
\end{itemize}

\begin{itemize}
\tightlist
\item
  \href{https://www.nytco.com/}{NYTCo}
\item
  \href{https://help.nytimes3xbfgragh.onion/hc/en-us/articles/115015385887-Contact-Us}{Contact
  Us}
\item
  \href{https://www.nytco.com/careers/}{Work with us}
\item
  \href{https://nytmediakit.com/}{Advertise}
\item
  \href{http://www.tbrandstudio.com/}{T Brand Studio}
\item
  \href{https://www.nytimes3xbfgragh.onion/privacy/cookie-policy\#how-do-i-manage-trackers}{Your
  Ad Choices}
\item
  \href{https://www.nytimes3xbfgragh.onion/privacy}{Privacy}
\item
  \href{https://help.nytimes3xbfgragh.onion/hc/en-us/articles/115014893428-Terms-of-service}{Terms
  of Service}
\item
  \href{https://help.nytimes3xbfgragh.onion/hc/en-us/articles/115014893968-Terms-of-sale}{Terms
  of Sale}
\item
  \href{https://spiderbites.nytimes3xbfgragh.onion}{Site Map}
\item
  \href{https://help.nytimes3xbfgragh.onion/hc/en-us}{Help}
\item
  \href{https://www.nytimes3xbfgragh.onion/subscription?campaignId=37WXW}{Subscriptions}
\end{itemize}
