Sections

SEARCH

\protect\hyperlink{site-content}{Skip to
content}\protect\hyperlink{site-index}{Skip to site index}

\href{https://myaccount.nytimes3xbfgragh.onion/auth/login?response_type=cookie\&client_id=vi}{}

\href{https://www.nytimes3xbfgragh.onion/section/todayspaper}{Today's
Paper}

\href{/section/business/dealbook}{DealBook}\textbar{}How Shareholder
Democracy Failed the People

\url{https://nyti.ms/2OZJ4fF}

\begin{itemize}
\item
\item
\item
\item
\item
\item
\end{itemize}

Advertisement

\protect\hyperlink{after-top}{Continue reading the main story}

DealBook Business and Policy

Supported by

\protect\hyperlink{after-sponsor}{Continue reading the main story}

Dealbook

\hypertarget{how-shareholder-democracy-failed-the-people}{%
\section{How Shareholder Democracy Failed the
People}\label{how-shareholder-democracy-failed-the-people}}

Shareholder democracy seemed like a good idea at the time. What we got
was shareholder supremacy.

\includegraphics{https://static01.graylady3jvrrxbe.onion/images/2019/08/19/business/19db-sorkin1/merlin_138863691_2826cd85-f647-4524-bbd4-12027ad91511-articleLarge.jpg?quality=75\&auto=webp\&disable=upscale}

\href{https://www.nytimes3xbfgragh.onion/by/andrew-ross-sorkin}{\includegraphics{https://static01.graylady3jvrrxbe.onion/images/2018/07/16/multimedia/author-andrew-ross-sorkin/author-andrew-ross-sorkin-thumbLarge.png}}

By
\href{https://www.nytimes3xbfgragh.onion/by/andrew-ross-sorkin}{Andrew
Ross Sorkin}

\begin{itemize}
\item
  Published Aug. 20, 2019Updated Aug. 21, 2019
\item
  \begin{itemize}
  \item
  \item
  \item
  \item
  \item
  \item
  \end{itemize}
\end{itemize}

Democracy is a messy thing. Shareholder democracy may be even messier.

For nearly a half-century, corporate America has prioritized, almost
maniacally, profits for its shareholders. That single-minded devotion
overran nearly every other constituent, pushing aside the interests of
customers, employees and communities.

That philosophy was rooted in an idea that has an air of nobility about
it. Shareholder democracy was the name given to investors asserting
themselves in corporate governance. The idea was that investors would
wrest control of companies from entrenched managers, letting the actual
owners set their corporate priorities. But what we really got was
something else: an era of shareholder primacy.

That may have a chance --- a chance --- of changing now that 181 chief
executives have lent their signatures to a new ``Statement on the
Purpose of a Corporation'' that was published by the Business Roundtable
on Monday. The statement from the leaders of companies including
JPMorgan Chase, Apple, Amazon and Walmart affirms that the nation's
largest companies have a ``fundamental commitment'' to all their
stakeholders: putting employees, suppliers and communities on a pedestal
that once belonged only to shareholders.

The companies' statement is a significant shift and a welcome one. For
years, businesses have resisted calls --- including from this column ---
to rethink their responsibility to society. In response, corporations
typically dismissed hot-button topics like income inequality, climate
change, gun violence and more as political issues unrelated to them.

Some will doubt the sincerity of these business leaders' words, and it
remains an open question whether their companies will be held
accountable --- and by whom. But what we may be at the start of is less
a new era and more a return to the past.

For nearly 50 years --- following the publication of a seminal academic
treatise in 1932 called ``The Modern Corporation and Private Property''
by Adolf A. Berle Jr. and Gardiner C. Means --- corporations, for the
most part, were run for all stakeholders. It was a time defined by
organized labor, corporate pension programs, gold-watch retirements and
charitable gifts from companies that invested heavily in their
communities and the kind of research that promised future growth.

It is a period often referred to --- sometimes derisively --- as
``managerialism.''

But by the 1970s, managerialism became synonymous in investment circles
with immovable executives who were running bloated businesses more for
their own benefit than for their shareholders.

It also coincided with the ascent of Milton Friedman, the University of
Chicago economist who preached a gospel of profits-as-purpose and mocked
anyone who thought that businesses should do anything else.

\includegraphics{https://static01.graylady3jvrrxbe.onion/images/2017/01/29/podcasts/the-daily-album-art/the-daily-album-art-articleInline-v2.jpg?quality=75\&auto=webp\&disable=upscale}

\hypertarget{listen-to-the-daily-what-american-ceos-are-worried-about}{%
\subsubsection{Listen to `The Daily': What American C.E.O.s Are Worried
About}\label{listen-to-the-daily-what-american-ceos-are-worried-about}}

Nearly 200 executives tried this week to redefine the role of a
corporation in society. We look at what's behind that move.

transcript

Back to The Daily

bars

0:00/25:22

-25:22

transcript

\hypertarget{listen-to-the-daily-what-american-ceos-are-worried-about-1}{%
\subsection{Listen to `The Daily': What American C.E.O.s Are Worried
About}\label{listen-to-the-daily-what-american-ceos-are-worried-about-1}}

\hypertarget{hosted-by-michael-barbaro-produced-by-alexandra-leigh-young-eric-krupke-and-paige-cowett-and-edited-by-lisa-tobin-and-marc-georges}{%
\subsubsection{Hosted by Michael Barbaro, produced by Alexandra Leigh
Young, Eric Krupke and Paige Cowett, and edited by Lisa Tobin and Marc
Georges}\label{hosted-by-michael-barbaro-produced-by-alexandra-leigh-young-eric-krupke-and-paige-cowett-and-edited-by-lisa-tobin-and-marc-georges}}

\hypertarget{nearly-200-executives-tried-this-week-to-redefine-the-role-of-a-corporation-in-society-we-look-at-whats-behind-that-move}{%
\paragraph{Nearly 200 executives tried this week to redefine the role of
a corporation in society. We look at what's behind that
move.}\label{nearly-200-executives-tried-this-week-to-redefine-the-role-of-a-corporation-in-society-we-look-at-whats-behind-that-move}}

\begin{itemize}
\item
  michael barbaro\\
  From The New York Times, I'm Michael Barbaro. This is ``The Daily.''

  Today: For five decades, American corporations have prized profits for
  shareholders above all else. Now, America's most powerful chief
  executives say it's time to do things differently. Andrew Ross Sorkin
  on what's driving that change.

  It's Wednesday, August 21.

  Andrew, tell me what happened in your world on Monday morning.
\item
  andrew ross sorkin\\
  On Monday morning, 5:00 a.m. in the morning, the Business Roundtable,
  which is probably the most powerful and influential lobbying
  organization for the nation's biggest companies --- think Apple, think
  Amazon, Walmart, JPMorgan, all of them --- came out and said that
  shareholders were going to be just one piece of a larger puzzle.
\item
  {[}music{]}
\item
  archived recording 1\\
  From the Business Roundtable, 181 of the top C.E.O.s in the country
  have agreed now that maximizing profits in all situations cannot
  necessarily be the main goal of corporations.
\item
  archived recording 2\\
  A statement signed by almost 200 C.E.O.s, including JPMorgan's Jamie
  Dimon, says companies should focus on all stakeholders.
\end{itemize}

andrew ross sorkin

For as long as I covered the world of business, every C.E.O. in America
said they had a fiduciary duty to shareholders. Everything was in the
name of profits.

\begin{itemize}
\tightlist
\item
  archived recording\\
  So after decades of explicitly saying that shareholders were the
  highest end of a corporation, they point out the corporation's duties
  to their customers, to their suppliers, to their communities. And then
  they get to their shareholders. A massive change.
\end{itemize}

andrew ross sorkin

And so the idea that any of these other stakeholders are even being
acknowledged as part of the equation is a major shift.

But in many ways, it's a return to an earlier era, an era almost a
century ago, when these other stakeholders mattered in a way that they
haven't for so many years.

michael barbaro

And what is that era?

andrew ross sorkin

If you go back to the 1930s, possibly even earlier, the biggest
corporations in America genuinely thought about the full plethora of
constituents. Employees mattered, customers mattered, suppliers
mattered. The profits mattered, but there was clearly a larger social
compact that had been reached between companies and the rest of society.

michael barbaro

And what did that look like? And what's an example of a company that
reflected that?

andrew ross sorkin

You could even go back to the early 1900s and look at Henry Ford and the
Ford Company, and his decision to raise the average pay from \$2.25 to
\$5.

\begin{itemize}
\tightlist
\item
  archived recording\\
  Back in 1914, Ford had revolutionized assembly line production. And to
  keep his workers from quitting, he announced he would raise their pay
  to a generous \$5 a day, twice what they earned before, and twice what
  they could earn at any other auto company.
\end{itemize}

andrew ross sorkin

He believed that it was important for his employees not only to have a
fair wage, but to have a wage that might give them an opportunity to
actually buy the car that he was selling.

\begin{itemize}
\tightlist
\item
  archived recording\\
  It was a simple American bargain, security and high wages in exchange
  for hard work.
\end{itemize}

andrew ross sorkin

And then you can look even at so many companies, whether they be IBM,
General Electric, so many big American companies between 1930 and 1970
that created defined pension programs for their employees, that gave
huge amounts of corporate charity to their communities, that became
connected in a way that made the companies intertwined with the
community that they lived in.

michael barbaro

And what exactly is the motivation for these companies to conduct
themselves in this way, as community-minded corporations investing in
their employees and in their communities?

andrew ross sorkin

It was ultimately good for business. It was the idea that if you could
attract great employees and you could keep those employees, often for
life, that you would have a better product, that you would have a better
company, and that they were all inextricably tied. The backdrop of all
of this is a post-World War II world, in which the United States, in
truth, is a monopoly power.

\begin{itemize}
\tightlist
\item
  archived recording\\
  Factories were churning out products to satisfy the growing consumer
  appetite in America and to meet the needs of a postwar Europe. The
  defense industry kept military supplies flowing in reaction to the
  Cold War. And the nation was building, straight up in the cities and
  far out into the country. America's economy was the biggest in the
  world.
\end{itemize}

andrew ross sorkin

There really is no international competition. We are it. And that meant
that there was a limited workforce, and that meant that you had to be
good to your people.

michael barbaro

Because they could go shop around for a different job.

andrew ross sorkin

Exactly. Among the world of academics, this period was really defined as
managerialism, the idea that you were managing the company for the
people that were in it.

michael barbaro

And when does that start to change?

andrew ross sorkin

In the 1970s, the idea of managerialism went from being a good thing to
being a very bad thing.

michael barbaro

Why?

andrew ross sorkin

Because investors, the shareholders, started to raise their hand and
say, ah, over here, we're actually the people who own you. And we think
that you're mismanaging the company, that you're spending too much money
on your own people, that you're fat and happy. There was a view that
managerialism had been perverted and abused. And at the same time, you
had the rise of Milton Friedman.

\begin{itemize}
\tightlist
\item
  archived recording (milton friedman)\\
  First of all, tell me, is there some society you know that doesn't run
  on greed? You think Russia doesn't run on greed? You think China
  doesn't run on greed? What is greed? Of course, none of us are greedy,
  it's only the other fellow who's greedy. The world runs on individuals
  pursuing their separate interests.
\end{itemize}

andrew ross sorkin

He's an economist at the University of Chicago who really becomes one of
most popular figures of this time, not just in the world of corporate
America, but throughout the country. And it's in large part because he
has a provocative view about the way we do business. And he pens this
famous piece, actually in The New York Times Magazine, with the headline
``The Social Responsibility of Business Is to Increase Its Profits.''
Let me read you what he wrote. He wrote, ``What does it mean to say that
business has responsibilities?'' He almost asked it rhetorically. And
then he writes, ``Businessmen who talk this way are unwitting puppets of
the intellectual forces that have been undermining the basis of a free
society these past decades.'' It's effectively a rebuke of the way
business has been managed. It's a rebuke of managerialism.

michael barbaro

And what does Friedman believe will happen if corporations don't see
social responsibility as part of their job, if they just focus on what
he says they should, which is profits and shareholders?

andrew ross sorkin

He fundamentally believes that if you focus on profits, everything else
will come with --- that a company that is not as profitable as it
humanly can be will ultimately lose out to other companies that are. And
you need a strong company to employ people who will pay taxes to the
community, who will give charitable giving to others down the line.

michael barbaro

So all those functions of the traditional corporation that came before,
they get served through a healthy company that profitably serves its
shareholders.

andrew ross sorkin

I don't think they called it trickle-down then, but they would now.

\begin{itemize}
\tightlist
\item
  archived recording (milton friedman)\\
  What are you going to do for the people who are out of work when the
  public at large decides it's not going to go in big cars, it's going
  to go in little cars? I don't want to do a thing. I want to let the
  private market work. The private market system is a system of profit
  and loss. And the loss part is just as essential as the profit part.
  It is a disgrace ---
\end{itemize}

andrew ross sorkin

He was a free marketeer. This was all part of a larger free market
theory that profits above everything else would ultimately win the day
and make the country stronger. So you have this confluence of these two
ideas --- Milton Friedman on one side, and shareholders that are
starting to look at companies and saying, maybe they're a little too fat
and happy. And that really brings about a new era in corporate America
where the shareholder becomes the top priority.

{[}music{]}

michael barbaro

We'll be right back. Andrew, once the idea of the shareholder takes hold
in the U.S., how does that actually play out? How does corporate
behavior change?

andrew ross sorkin

It manifests itself first in the form of what became known as corporate
raiders, investors who basically started knocking on the door of
companies, saying, you need to make more profits. And if you don't,
we're going to take you over.

michael barbaro

And what period is this, roughly?

andrew ross sorkin

This is the 1980s.

\begin{itemize}
\tightlist
\item
  archived recording\\
  It takes a certain breed of stock market investor, the kind with lots
  of money and lots of guts, to thrive in queasy times like these.
\end{itemize}

andrew ross sorkin

This is greed is good. This is in the midst of this sort of rush around
Wall Street, around capitalism.

\begin{itemize}
\tightlist
\item
  archived recording\\
  Carl Icahn is one of that breed. He has a knack for turning someone
  else's loss into profit for himself.
\end{itemize}

andrew ross sorkin

And you had some very early investors, like Carl Icahn ---

\begin{itemize}
\tightlist
\item
  archived recording (carl icahn)\\
  I was always good at making money. I always was good.
\end{itemize}

andrew ross sorkin

--- go to some of the biggest companies in the world, that T.W.A.s of
the world, one of the biggest airlines in the country, go to companies
like U.S. Steel, and say, I'm going to buy you.

\begin{itemize}
\tightlist
\item
  archived recording\\
  Got some breaking news for you. This time, Carl Icahn is at it again.
  He has offered to buy Commercial Metals for \$15 a share. He already
  owns about 10 percent.
\end{itemize}

andrew ross sorkin

I'm going to take you over. I'm going to throw your C.E.O. out. I'm
going to lay off scores of employees. I'm going to undo all the benefit
programs. And I'm going to manage this company in a much leaner way.
That was the euphemism, leaner.

michael barbaro

Because leaner was, in their minds, stronger.

andrew ross sorkin

Leaner was more profitable.

michael barbaro

So these corporate raiders were emboldened by this new guiding
philosophy that it was good and right to cut the fat, cut the excess,
and increase profits, and that that was actually the socially
responsible thing to do, no matter how ruthless it might have seemed.

andrew ross sorkin

Absolutely, and Milton Friedman had almost turned it into a moral
argument, so that these investors had a moral underpinning for what they
were doing.

michael barbaro

And how does that era of corporate raids affect the larger American
business world?

andrew ross sorkin

These C.E.O.s start to really internalize what they're seeing in the
headlines with these corporate raiders. They don't want to be the next
target of these guys. All of a sudden, C.E.O.s that historically might
have been a little bit looser with the purse, say to themselves, you
know what, we should maybe cut back on some of these employees. We don't
need all of these people. We need higher profits. Maybe instead of
investing in research and development, we should start buying back our
stock, or dividending out money to our shareholders. Maybe we should
rethink our defined pension contributions and move towards 401(k) plans,
which will cost us less. Maybe that charitable budget that we had for
the community, maybe we should scale that back. Over the next 20 or 30
years, you saw a massive restructuring of corporate America that put the
shareholder first, the shareholder over the stakeholder. You saw scores
of layoffs. Millions of people were laid off over this period. You saw
charitable contributions by companies fall in half during this period.
You saw pension funds and retirement funds diminish materially. All of
this leads to a mindset in the corner office among the C.E.O. world of
being very short-term oriented. They all want to hit their quarterly
numbers. Their bonuses become tied to the quarterly numbers. Everything
is now around the stock price. How high is the stock price? Everybody's
getting compensated in stock. And in some ways, that's supposed to
incentivize managers to do the right thing, to align their interests
with the shareholders. But at the same time, it often pitted them
against their own colleagues.

michael barbaro

Andrew, is there a sense that this Miltonian, shareholder-first system,
that it works in this period?

andrew ross sorkin

Some people loved Milton Friedman. Some people thought he was absolutely
wrong. But within the world of corporate America, it became a mantra.
This had permeated the brains of the C.E.O. community so much so that by
1997, the Business Roundtable actually changed their mission statement
then and said, quote, ``The paramount duty of management and of boards
of directors is to the corporations' stockholders.''

michael barbaro

So Andrew, how do we get to this week, this statement from this group of
America's most powerful C.E.O.s rejecting this philosophy that you've
just described as basically accepted wisdom in American business, that
shareholders should be first, why would they suddenly reject that?

andrew ross sorkin

I'd point back 10 years ago, to the financial crisis ---

\begin{itemize}
\tightlist
\item
  archived recording\\
  For Wall Street, it was another case of whiplash. The markets haven't
  been this volatile in almost 80 years.
\end{itemize}

andrew ross sorkin

--- to a moment where so much of this crystallized in a national
conversation ---

\begin{itemize}
\tightlist
\item
  archived recording (george w. bush)\\
  The market is not functioning properly. There has been a widespread
  loss of confidence.
\end{itemize}

andrew ross sorkin

--- around the role of businesses, around the role of banks, which had
taken on these short-term interests at the expense of the entire
country, where questions about capitalism were raised.

\begin{itemize}
\item
  archived recording 1\\
  All across the country, plants are closing, and employees are being
  laid off.
\item
  archived recording 2\\
  For every job opening, there are six people looking to fill it.
\end{itemize}

andrew ross sorkin

And when we were living at a time of unemployment of 10 percent, it
really changed the narrative about what a company does. And people felt
it. They felt it in their bones, because there were so many layoffs.

\begin{itemize}
\tightlist
\item
  archived recording\\
  We are the 99 percent. We are the 99 percent.
\end{itemize}

andrew ross sorkin

And it became a political story. The ascendancy of Elizabeth Warren.

\begin{itemize}
\tightlist
\item
  archived recording (elizabeth warren)\\
  People feel like the system is rigged against them. And here's the
  painful part: They're right.
\end{itemize}

andrew ross sorkin

The ascendancy of Bernie Sanders.

\begin{itemize}
\tightlist
\item
  archived recording (bernie sanders)\\
  Yeah, corporate greed is running this country. And corporate greed is
  destroying the dreams and aspirations of millions of American people.
\end{itemize}

andrew ross sorkin

And so much of the country started to ask real questions. And I think
that the C.E.O. community has had a realization that if they don't
change their ways, if they don't at least nod to these issues, that
capitalism itself, that the system itself that they've been living in,
will change, that the political forces in this country will change them
for them.

michael barbaro

So this evolution, this statement, is about shifting public opinion,
not, again, altruism. These C.E.O.s are reading the tea leaves. They're
looking at the polls and the politics, and that is telling them that
it's good business to change the way that they're doing business.

andrew ross sorkin

This is straight survival instinct. They're doing this because they
think it's good for their business.

michael barbaro

O.K., so let's talk about this statement and the people who put it out.
I wonder what would actually change about the behavior of corporations
if they put into practice what they're saying here, if they actually
mean it? Like, how does the C.E.O. of JPMorgan --- one of the people who
signed it --- Jamie Dimon's job change if he puts into practice this
change in approach that this document outlines, where shareholders are
just one of a dozen people he now thinks of his corporation as serving.

andrew ross sorkin

I'm going to give you my hopefully skeptical but not cynical view. I
think there's some element of progress here, because it changes the
conversation. It provides for an allowance, if you will, for a board of
directors or C.E.O. to say, you know what, let's raise the minimum wage,
let's actually spend the money on this plant, let's increase our
research and development budget. You know what, in this community, maybe
we should give a little bit more and increase our charitable giving
budget. You know what, we're not going to nail our profit number next
quarter, because we're going to invest in these other things.

michael barbaro

And it's O.K.

andrew ross sorkin

And that's O.K.

michael barbaro

Because before, there wouldn't have been an allowance for that.

andrew ross sorkin

There might not have been an allowance for that. In certain boardrooms
in America, there was no allowance for not hitting your profit number.
Now, there may be. That would be the positive view of this.

michael barbaro

That's not all that optimistic.

andrew ross sorkin

Well, the negative view of this is that they're words on a page, and
that's all they are. Politicians will look at this, maybe give them
credit for it, maybe not, and what does it cost them? Their signature on
a piece of paper. They got a front-page story in The New York Times out
of it. They get a ``Daily'' podcast. There's safety in numbers here.
That's probably the best that can be said about this.

michael barbaro

Mm-hmm. I don't hear you saying that you think this is representing a
fundamental change in how corporations see themselves or function.

andrew ross sorkin

I still think that ultimately, if these companies are not profitable,
that these executives are going to lose their jobs, full stop. I still
think the investment community is very short-term. I think we are, over
the long term, on a journey where social responsibility is going to be a
central piece, at least a piece, of this larger puzzle. I think it's
almost impossible that it's not going to be. And I think you're seeing
it in the voices of politicians, in the voices of the public, in the
voices of regulators. And as a function of that, companies are
listening.

michael barbaro

But only because social responsibility is also good for business, and
good for profits, and good for shareholders.

andrew ross sorkin

At the end of the day, C.E.O.s are only going to do things that are
ultimately profitable. And in this moment, thinking about all of these
other stakeholders may be profitable.

Ultimately, if you're looking for big social change, I don't think
you're going to look to corporate leaders for that. I don't think that's
where it's going to come from. Companies ultimately have to be
profitable entities. If they're not profitable, they don't exist, and
they can't serve any of these other purposes, which is to some degree
what Milton Friedman was trying to say.

michael barbaro

Andrew, thank you very much.

andrew ross sorkin

Thank you very much.

michael barbaro

We'll be right back.

Here's what else you need to know today.

\begin{itemize}
\tightlist
\item
  archived recording (giuseppe conte)\\
  {[}SPEAKING ITALIAN{]}
\end{itemize}

michael barbaro

On Tuesday, Italian Prime Minister Giuseppe Conte resigned, after his
government, a 14-month-old coalition of populists and nationalists who
are skeptical of the European Union, collapsed. His resignation was
triggered by one of Conte's own ministers, Matteo Salvini, an
increasingly popular right-wing figure, who called for a vote of no
confidence in Conte's government, and who has now plunged the country
into political uncertainty. And ---

\begin{itemize}
\item
  archived recording (donald trump)\\
  Yes, any questions?
\item
  archived recording\\
  Mr. President, what sort of contingency steps or plans is the White
  House thinking about to stave off any kind of economic slowdown? What
  are you looking at?
\item
  archived recording (donald trump)\\
  We're looking at various tax reductions, but I'm looking at that all
  the time anyway, tax reductions.
\end{itemize}

michael barbaro

President Trump said he's weighing a set of tax cuts to stimulate the
U.S. economy amid growing fears it may be entering a recession.

\begin{itemize}
\tightlist
\item
  archived recording (donald trump)\\
  Payroll tax is something that we think about. And a lot of people
  would like to see that. And that very much affects the workers of our
  country.
\end{itemize}

michael barbaro

Trump focused on the possibility of cutting the country's payroll taxes,
the percentage of a paycheck withheld by employers to comply with tax
laws, which would immediately put money into the hands of consumers. The
Times reports that the president is anxious about the possibility of a
recession occurring in the middle of his presidential campaign and is
eager to find ways to stave off a downturn.

That's it for ``The Daily.'' I'm Michael Barbaro. See you tomorrow.

``What does it mean to say that `business' has responsibilities?''
\href{https://timesmachine.nytimes3xbfgragh.onion/timesmachine/1970/09/13/223535702.html}{Mr.
Friedman wrote} in this newspaper in 1970. ``Businessmen who talk this
way are unwitting puppets of the intellectual forces that have been
undermining the basis of a free society these past decades.''

That began the rise of shareholder democracy, an idea that the public
and news media embraced. Shareholders and, in turn, a new class of
investors known as corporate raiders convinced executives to slash any
and all fat from their budgets or risk being taken over or voted out.
Layoffs increased, research and development budgets were cut, and
pension programs were traded for 401(k)s. There was a rush of mergers
driven by ``cost savings'' that grabbed headlines while profits soared
and dividends increased.

And here we are. Americans mistrust companies to such an extent that the
very idea of capitalism is now being debated on the political stage.
Populism has been embraced on both ends of the political spectrum,
whether in the trade protectionism of President Trump or the social-net
supremacy of Senator Bernie Sanders.

It is against that backdrop that the Business Roundtable released its
statement on Monday. The group should be commended for coming around ---
and no one wants to criticize progress --- but it is undeniably late.

Make no mistake, it wasn't shareholder democracy that created this new
enlightened moment. Public outrage pushed this forward. So did anger in
Washington and regulatory scrutiny that is finally coming to bear.

Shareholders --- with some exceptions --- did not come around until they
had no choice but to realize that these forces could have an impact on
their investments.

And in an echo of managerialism, there are some corporate executives who
deserve credit for this change.

Larry Fink, the chairman of BlackRock, deserves to be doing laps for
putting these ideas into his annual letters years ago, when some of
those who signed Monday's statement laughed at the idea.

Credit should go, too, to Howard Schultz, the former chief executive of
Starbucks, whose company embraced its employees as stakeholders from the
beginning. And companies like Patagonia and Ben and Jerry's, which are
so-called B Corporations, committed to community principles early.

The investor Paul Tudor Jones II has been talking about this for years.
So has Judith F. Samuelson, an executive director at the Aspen Institute
who has pressed corporate leaders to embrace a view of service to
society, and told me about a dinner where she and others leaned on Jamie
Dimon, the JPMorgan chief executive and chairman of the Business
Roundtable, to change the group's mission statement.

And there was Prof. Klaus Schwab, who founded the World Economic Forum,
drafting the Davos Manifesto of 1973: ``The purpose of professional
management is to serve clients, shareholders, workers and employees, as
well as societies, and to harmonize the different interests of the
stakeholders.''

If you suspect that the Business Roundtable's statement changes little,
there may be reason for skepticism. Some big companies didn't sign on,
including the Blackstone Group, General Electric and Alcoa.

And the \href{https://www.cii.org/aug19_brt_response}{Council of
Institutional Investors} --- which represents many of the same companies
as Business Roundtable and many of the nation's largest pension funds
--- distributed a response that forcefully disavowed the ideas set forth
in the roundtable's statement.

``Accountability to everyone means accountability to no one,'' the
council said. ``It is government, not companies, that should shoulder
the responsibility of defining and addressing societal objectives with
limited or no connection to long-term shareholder value.''

For whatever progress may have been made Monday, it is hardly clear the
debate is over. In fact, the fight for corporate identity is just
beginning.

Advertisement

\protect\hyperlink{after-bottom}{Continue reading the main story}

\hypertarget{site-index}{%
\subsection{Site Index}\label{site-index}}

\hypertarget{site-information-navigation}{%
\subsection{Site Information
Navigation}\label{site-information-navigation}}

\begin{itemize}
\tightlist
\item
  \href{https://help.nytimes3xbfgragh.onion/hc/en-us/articles/115014792127-Copyright-notice}{©~2020~The
  New York Times Company}
\end{itemize}

\begin{itemize}
\tightlist
\item
  \href{https://www.nytco.com/}{NYTCo}
\item
  \href{https://help.nytimes3xbfgragh.onion/hc/en-us/articles/115015385887-Contact-Us}{Contact
  Us}
\item
  \href{https://www.nytco.com/careers/}{Work with us}
\item
  \href{https://nytmediakit.com/}{Advertise}
\item
  \href{http://www.tbrandstudio.com/}{T Brand Studio}
\item
  \href{https://www.nytimes3xbfgragh.onion/privacy/cookie-policy\#how-do-i-manage-trackers}{Your
  Ad Choices}
\item
  \href{https://www.nytimes3xbfgragh.onion/privacy}{Privacy}
\item
  \href{https://help.nytimes3xbfgragh.onion/hc/en-us/articles/115014893428-Terms-of-service}{Terms
  of Service}
\item
  \href{https://help.nytimes3xbfgragh.onion/hc/en-us/articles/115014893968-Terms-of-sale}{Terms
  of Sale}
\item
  \href{https://spiderbites.nytimes3xbfgragh.onion}{Site Map}
\item
  \href{https://help.nytimes3xbfgragh.onion/hc/en-us}{Help}
\item
  \href{https://www.nytimes3xbfgragh.onion/subscription?campaignId=37WXW}{Subscriptions}
\end{itemize}
