Sections

SEARCH

\protect\hyperlink{site-content}{Skip to
content}\protect\hyperlink{site-index}{Skip to site index}

\href{https://www.nytimes3xbfgragh.onion/section/books/review}{Book
Review}

\href{https://myaccount.nytimes3xbfgragh.onion/auth/login?response_type=cookie\&client_id=vi}{}

\href{https://www.nytimes3xbfgragh.onion/section/todayspaper}{Today's
Paper}

\href{/section/books/review}{Book Review}\textbar{}How Fast Fashion Is
Destroying the Planet

\begin{itemize}
\item
\item
\item
\item
\item
\item
\end{itemize}

Advertisement

\protect\hyperlink{after-top}{Continue reading the main story}

Supported by

\protect\hyperlink{after-sponsor}{Continue reading the main story}

Nonfiction

\hypertarget{how-fast-fashion-is-destroying-the-planet}{%
\section{How Fast Fashion Is Destroying the
Planet}\label{how-fast-fashion-is-destroying-the-planet}}

\includegraphics{https://static01.graylady3jvrrxbe.onion/images/2019/10/13/books/review/schlossberg01/merlin_70552169_4c9a0a6c-f2cd-47b8-a8a7-e20811bd7848-articleLarge.jpg?quality=75\&auto=webp\&disable=upscale}

Buy Book ▾

\begin{itemize}
\tightlist
\item
  \href{https://www.amazon.com/gp/search?index=books\&tag=NYTBSREV-20\&field-keywords=Fashionopolis+Dana+Thomas}{Amazon}
\item
  \href{https://du-gae-books-dot-nyt-du-prd.appspot.com/buy?title=Fashionopolis\&author=Dana+Thomas}{Apple
  Books}
\item
  \href{https://www.anrdoezrs.net/click-7990613-11819508?url=https\%3A\%2F\%2Fwww.barnesandnoble.com\%2Fw\%2F\%3Fean\%3D9780735224018}{Barnes
  and Noble}
\item
  \href{https://www.anrdoezrs.net/click-7990613-35140?url=https\%3A\%2F\%2Fwww.booksamillion.com\%2Fp\%2FFashionopolis\%2FDana\%2BThomas\%2F9780735224018}{Books-A-Million}
\item
  \href{https://bookshop.org/a/3546/9780735224018}{Bookshop}
\item
  \href{https://www.indiebound.org/book/9780735224018?aff=NYT}{Indiebound}
\end{itemize}

When you purchase an independently reviewed book through our site, we
earn an affiliate commission.

By
\href{https://www.nytimes3xbfgragh.onion/by/tatiana-schlossberg}{Tatiana
Schlossberg}

\begin{itemize}
\item
  Sept. 3, 2019
\item
  \begin{itemize}
  \item
  \item
  \item
  \item
  \item
  \item
  \end{itemize}
\end{itemize}

\textbf{FASHIONOPOLIS}\\
\textbf{The Price of Fast Fashion and the Future of Clothes}\\
By Dana Thomas

There is that old saying, usually attributed to Yves Saint Laurent:
``Fashion fades, style is eternal.''

Literally speaking, that actually may no longer be true, especially when
it comes to fast fashion. Fast-fashion brands may not design their
clothing to last (and they don't), but as artifacts of a particularly
consumptive era, they might become an important part of the fossil
record.

More than 60 percent of fabric fibers are now synthetics, derived from
fossil fuels, so if and when our clothing ends up in a landfill (about
85 percent of textile waste in the United States goes to landfills or is
incinerated), it will not decay.

Nor will the synthetic microfibers that end up in the sea, freshwater
and elsewhere, including the deepest parts of the oceans and the highest
glacier peaks. Future archaeologists may look at landfills taken over by
nature and discover evidence of Zara.

\includegraphics{https://static01.graylady3jvrrxbe.onion/images/2019/08/26/books/review/schlossberg02/merlin_151596036_ea40852f-97bf-47f2-9cbb-1661f30923b7-articleLarge.jpg?quality=75\&auto=webp\&disable=upscale}

And it is Zara and other brands like it that have helped plant flags on
the farthest reaches of the planet. In ``Fashionopolis,'' Dana Thomas, a
veteran style writer, convincingly connects our fast-fashion wardrobes
to global economic and climate patterns and crises, rooting the current
state of the fashion biosphere as a whole --- production methods, labor
practices and environmental impacts --- in the history of the garment
industry.

Her narrative is broken up into three manageable sections. The first
focuses on today's global fast-fashion and regular fashion industries
and how they came to be so enormous, voracious, so seemingly
uncontainable. It includes a fascinating account of how NAFTA made
possible the international success of fast fashion. The second presents
alternative, even opposite, approaches to making clothing that Thomas
terms ``slow fashion'': locally grown materials, often domestically
manufactured or sourced on a relatively small scale, like the farmer and
entrepreneur Sarah Bellos's American-grown indigo. Lastly, she meets
people who are trying to reform the system entirely, from the materials
we use to how clothes are produced and the ways we shop.

Throughout, Thomas reminds us that the textile industry has always been
one of the darkest corners of the world economy. The defining product of
the Industrial Revolution, textiles were crucial to the development of
our globalized capitalist system, and its abuses today are built on a
long history. Slave labor in the American South supplied factories in
both England, where they were notorious for child labor and other
horrors, and the United States, where factory fires took the lives of
recent immigrants at the turn of the 20th century. Thomas reports that
there are immigrant workers in Los Angeles today who are victims of wage
theft and exploitation, not to mention the Bangladeshi, Chinese,
Vietnamese and other laborers who face working conditions that are at
best grim and at worst inhumane. Fashion is an industry that has
depended on the toil of the powerless and the voiceless, and on keeping
them that way.

In one of the most powerful parts of the book, Thomas recounts the
tragedy of the 2013 Rana Plaza factory collapse in Bangladesh, told
through the harrowing experiences of two survivors. The explosion killed
1,100 people and injured another 2,500. And this was not a one-off:
``Between 2006 and 2012, more than 500 Bangladeshi garment workers died
in factory fires.'' And, she notes, none of this news --- the Rana Plaza
catastrophe was widely covered --- diminished Americans' appetites for
cheap clothing. In fact, Thomas writes, that same year Americans ``spent
\$340 billion on fashion,'' and ``much of it was produced in Bangladesh,
some of it by Rana Plaza workers in the days leading up to the
collapse.''

Not all of the book is this pessimistic: There is plenty of bubbliness
and glamour for fashion lovers to get excited about. Thomas displays her
skills as a culture and style reporter as she visits the visionaries who
are attempting to remake the industry, if not from whole cloth, then
maybe from lab-grown or recycled fibers of some kind. She conjures a
pastoral idyll, for instance, in her depiction of the designer Natalie
Chanin and her business, Alabama Chanin, a line of cotton clothing
produced almost entirely in Florence, Ala., once the ``Cotton T-Shirt
Capital of the World.'' In Thomas's telling, these garments are both
environmentally sustainable and humane, though with a revenue of just
over \$3 million last year, the 30-person company is no replacement for
mass production when it comes to dressing seven billion people.

Image

Credit...Tomas Munita for The New York Times

Among the book's delights are Thomas's sketches of her individual
subjects. I can't get her description of a woman as ``peaches-and-cream
pretty'' out of my head; I know exactly what she looks like. The author
also has a gift for bringing luxury to life: She conjures Moda
Operandi's London showroom so vividly that I felt as though I'd moved
in.

In the last section, Thomas marvels at the ingenuity of those trying to
``disrupt'' fashion. She makes a strong argument for the importance of
science applied to (what are often seen as) the frivolities of fashion,
especially if we want to move away from the unartful excesses of mass
production.

Stella McCartney gets a disproportionate amount of attention here, and
for good reason. McCartney has long been committed to sustainable
practices, in her own business and others'. As the head designer at
Chloé in the late 1990s, she refused to include leather or fur in her
collections, which many executives then considered a death wish (some
still do). She made it work, and has amplified those practices in her
eponymous company, using, for instance, only ``reclaimed'' cashmere,
refusing to use polyvinyl chloride or untraceable rayon.

However, it is in contextualizing this single industry from a broader
climate perspective that the book falls short. Some statistics are
exaggerated: Livestock are not responsible for ``at least half of all
global greenhouse gas emissions,'' but rather closer to 15 percent of
them; nor is fashion production alone consuming water at a rate that, if
maintained, ``will surpass the world's supply by 40 percent by 2030''
(not even the world's total water demand necessarily will). And much of
the discussion of new materials and production methods raises further
questions. What are the differences between organic, conventional and
``Better Cotton''? (Organic cotton is periodically touted as a
sustainable alternative, though it currently makes up only about 0.4
percent of the cotton market, making it nearly impossible for any
company to rely on now or in the near future.) Another: Does the
landfilling of non-synthetic clothing matter? Thomas doesn't say, but in
fact it does, because it contributes to global emission of methane, a
potent heat-trapping gas.

A lot of faith is placed here in the idea of ``a circular --- or
closed-loop --- system, in which products are continually recycled,
reborn, reused. Nothing, ideally, should go in the trash.'' But the
practical considerations --- cost, efficiency, resource limitations ---
are often left unaddressed. Ultimately, Thomas finds that renting
clothing is the most sustainable model, and that feels like a more
realistic solution than the futuristic materials she describes at
length. In the end I was left wondering: If the fashion industry is this
damaging, and none of these developments alone will fix the problem,
shouldn't governments be regulating production beyond enacting stricter
pollution standards?

That may be a question for another book; it is not the goal of
``Fashionopolis'' to provide all the answers. Thomas has succeeded in
calling attention to the major problems in the \$2.4-trillion-a-year
industry, in a way that will engage not only the fashion set but also
those interested in economics, human rights and climate policy. Her
portraits of the figures who are transforming a field that hasn't
changed all that much in the last century or more sound at once like
messages from the future and like nostalgic reveries of life in a
smaller, simpler world. If we can combine them, this book suggests, the
envisioned ``fashionopolis'' could transform from an urban nightmare
into a shining city on a hill.

Advertisement

\protect\hyperlink{after-bottom}{Continue reading the main story}

\hypertarget{site-index}{%
\subsection{Site Index}\label{site-index}}

\hypertarget{site-information-navigation}{%
\subsection{Site Information
Navigation}\label{site-information-navigation}}

\begin{itemize}
\tightlist
\item
  \href{https://help.nytimes3xbfgragh.onion/hc/en-us/articles/115014792127-Copyright-notice}{©~2020~The
  New York Times Company}
\end{itemize}

\begin{itemize}
\tightlist
\item
  \href{https://www.nytco.com/}{NYTCo}
\item
  \href{https://help.nytimes3xbfgragh.onion/hc/en-us/articles/115015385887-Contact-Us}{Contact
  Us}
\item
  \href{https://www.nytco.com/careers/}{Work with us}
\item
  \href{https://nytmediakit.com/}{Advertise}
\item
  \href{http://www.tbrandstudio.com/}{T Brand Studio}
\item
  \href{https://www.nytimes3xbfgragh.onion/privacy/cookie-policy\#how-do-i-manage-trackers}{Your
  Ad Choices}
\item
  \href{https://www.nytimes3xbfgragh.onion/privacy}{Privacy}
\item
  \href{https://help.nytimes3xbfgragh.onion/hc/en-us/articles/115014893428-Terms-of-service}{Terms
  of Service}
\item
  \href{https://help.nytimes3xbfgragh.onion/hc/en-us/articles/115014893968-Terms-of-sale}{Terms
  of Sale}
\item
  \href{https://spiderbites.nytimes3xbfgragh.onion}{Site Map}
\item
  \href{https://help.nytimes3xbfgragh.onion/hc/en-us}{Help}
\item
  \href{https://www.nytimes3xbfgragh.onion/subscription?campaignId=37WXW}{Subscriptions}
\end{itemize}
