The Strange and Beautiful Universe of Walter Van Beirendonck

\url{https://nyti.ms/2NIikye}

\begin{itemize}
\item
\item
\item
\item
\item
\end{itemize}

\includegraphics{https://static01.graylady3jvrrxbe.onion/images/2019/09/03/t-magazine/03tmag-beirendonck-slide-WJNK/03tmag-beirendonck-slide-WJNK-articleLarge.jpg?quality=75\&auto=webp\&disable=upscale}

Sections

\protect\hyperlink{site-content}{Skip to
content}\protect\hyperlink{site-index}{Skip to site index}

\hypertarget{the-strange-and-beautiful-universe-of-walter-van-beirendonck}{%
\section{The Strange and Beautiful Universe of Walter Van
Beirendonck}\label{the-strange-and-beautiful-universe-of-walter-van-beirendonck}}

For more than three decades, the Belgian designer has been offering
gleefully irreverent alternatives to the staid men's wear of old.

Walter Van Beirendonck in a jacket from his spring 2019 collection, Wild
is the Wind, in front of looks from the spring 2012 Cloud \#9
collection.Credit...Mark Peckmezian

Supported by

\protect\hyperlink{after-sponsor}{Continue reading the main story}

By
\href{https://www.nytimes3xbfgragh.onion/by/thessaly-la-force}{Thessaly
La Force}

\begin{itemize}
\item
  Published Sept. 3, 2019Updated Sept. 7, 2019
\item
  \begin{itemize}
  \item
  \item
  \item
  \item
  \item
  \end{itemize}
\end{itemize}

THE COLLECTION WAS called Wonder and was shown in Paris's Bataclan
theater, in the 11th Arrondissement, in late June of 2009. Large,
hirsute men --- bears, in the lexicon of gay male culture --- stomped
down the runway. A dizzying tempo of electronic music blared on the
speakers. The bears had a variety of facial hair and wore a variety of
looks: wide cotton cargo pants with zip pockets in the shape of dildos
and clouds paired with electric blue and pink blazers fashioned from a
lightweight cloqué material; acid pink and neon green PVC ponchos over
sneakers with knee-high white socks scrawled with words like ``bear''
and ``pleasure.'' It was seemingly the end when the designer, Walter Van
Beirendonck --- himself a large, hirsute Belgian man, a chicer version
of Santa Claus --- emerged to do his own walk down the runway. As a
finale, the bears assembled onstage, only to be upstaged by another
group of bears, this one wearing white briefs, with a ``W'' sewn in red
across the crotch, who arrived to stand defiantly in front of them,
claiming the runway as their own. The crowd clapped. The show was over.

\href{http://www.waltervanbeirendonck.com/}{Van Beirendonck}, who is 62,
has been designing improbable, idiosyncratic men's fashion for over four
decades. He is one of the so-called
\href{https://www.nytimes3xbfgragh.onion/2013/06/18/fashion/a-rare-reunion-for-the-antwerp-six.html}{Antwerp
Six}, a group of Belgian designers who in the mid-80s helped transform
the city, only Belgium's second largest, into an unlikely fashion
center. He is also the director of the fashion school at Antwerp's
well-regarded
\href{https://www.ap.be/en/royal-academy-fine-arts-antwerp}{Royal
Academy of Fine Arts}, from which he graduated, and is considered a
mentor by many designers, including
\href{https://www.nytimes3xbfgragh.onion/2019/01/09/t-magazine/craig-green-mens-fashion-week.html}{Craig
Green},
\href{https://www.nytimes3xbfgragh.onion/2015/06/29/t-magazine/a-paris-designer-sets-up-a-creative-commune-in-the-hollywood-hills.html}{Bernhard
Willhelm} and Kris Van Assche, the
\href{https://www.nytimes3xbfgragh.onion/2018/04/03/fashion/berluti-kris-van-assche-lvmh.html}{creative
director of Berluti}. He's a contemporary and collaborator of artists as
diverse as the Japanese founder of
\href{https://www.nytimes3xbfgragh.onion/2018/09/03/t-magazine/rei-kawakubo-comme-des-garcons-menswear.html}{Comme
des Garçons},
\href{https://www.nytimes3xbfgragh.onion/topic/person/rei-kawakubo}{Rei
Kawakubo}; the Austrian conceptual artist
\href{https://www.nytimes3xbfgragh.onion/2014/03/09/t-magazine/erwin-wurm-shape-shifter.html}{Erwin
Wurm}; the French artist
\href{https://www.nytimes3xbfgragh.onion/1993/11/21/style/a-portrait-in-skin-and-bone.html}{Orlan};
the Irish rock band
\href{https://www.nytimes3xbfgragh.onion/topic/organization/u2}{U2}; and
the Australian industrial designer
\href{https://www.nytimes3xbfgragh.onion/2010/03/14/t-magazine/02well-newsonhouse.html}{Marc
Newson}. He is often credited with pioneering a new aesthetic: When he
started his career in 1982, at the age of 25, European men's wear was
generally limited to suiting and separates cut in traditional tweeds,
wools and cottons. Van Beirendonck was one of the lone disrupters of
this code of masculinity and formality, melding technically futuristic
fabrics --- often those used exclusively in sportswear --- with the
couture-like craftsmanship he learned at the Royal Academy, while
overlaying a uniquely sunny, humorous, almost childlike filter on his
anarchistic silhouettes, which were otherwise informed by darker
allusions to B.D.S.M. and punk rock culture.

\emph{{[}Coming soon: the T List newsletter, a weekly roundup of what T
Magazine editors are noticing and coveting.}
\href{https://www.nytimes3xbfgragh.onion/newsletters/t-list?module=inline}{\emph{Sign
up here}}\emph{.{]}}

He is also, in an age in which fashion has become a multibillion-dollar
business --- less a forum for artistic expression than one for
potentially enormous profits --- defiant in not only his singular
sensibility but in his apparent lack of interest in the commercial and
the compromised. These days, the runway is, with increasing frequency, a
place for political provocations, for expressions of gender fluidity,
kink and queerness. But this is a recent development, and one that
arguably finds its roots in Van Beirendonck's shows. He is the last
punk, one whose work can today feel predictive in its obsessions and
transgressions, and to which contemporary fashion owes a debt. ``It was
easy for me to do what I do when I started in the early 2000s,'' says
the American fashion designer
\href{https://www.nytimes3xbfgragh.onion/topic/person/thom-browne}{Thom
Browne}, known for his own upending of traditional men's wear codes.
``In the early '80s, it was brave, what Walter was doing. You have to
cite real revolutionaries like him for leading the way. He's inspiring
to all of us who are designing now.''

Image

A look from Sado, Van Beirendonck's first collection, for spring 1983.

Image

From the Worlds of Sun and Moon collection, for Walter Van Beirendonck,
fall 2018.Credit...Courtesy of Walter Van Beirendonck/KINOPROBY

``WHAT WOULD YOU like to know about me?'' Van Beirendonck asks. It is
evening in the village of Zandhoven, a 30-minute drive east of Antwerp,
where Van Beirendonck lives in a large 19th-century house, which sits on
an expansive property covered by wild grass and old trees. He is about
to travel to Tokyo to collaborate with Kawakubo on a new T-shirt.

Van Beirendonck was born in 1957 and grew up in the same small village
where he lives today. His parents owned an auto-repair garage and gas
station, and Van Beirendonck was mostly raised by his grandmother and
his oldest sister. At 12, he was sent to boarding school in Lier,
southeast of Antwerp, where he kept to himself, sketching and writing in
his diary. He knew he was gay at 14, and came out to his family without
much of an issue. It was the early '70s then, and
\href{https://www.nytimes3xbfgragh.onion/topic/person/david-bowie}{David
Bowie} had created (with the help of the Japanese designer
\href{https://www.nytimes3xbfgragh.onion/2017/07/24/t-magazine/fashion/kansai-yamamoto-revival.html}{Kansai
Yamamoto}) his
\href{https://www.nytimes3xbfgragh.onion/slideshow/2016/01/11/t-magazine/david-bowie-style/s/11tmag-bowie-slide-77O9.html}{Ziggy
Stardust} persona, with his plume of red hair, graphic makeup and
narrow, androgynous jumpsuits. For teenagers like Van Beirendonck, who
``couldn't play football'' and were living in polite, bourgeois
villages, Bowie and his compatriots ---
\href{https://www.nytimes3xbfgragh.onion/2016/11/03/t-magazine/iggy-pop-life-class-nude-drawing-brooklyn-museum.html}{Iggy
Pop}, with his snakelike torso and skintight pants; Lou Reed, with his
wild hair and kohl-lined eyes --- were a revelation: Here was another
way of being, another way of living. At 18, he picked up a Dutch
magazine called Avenue, which described the fashion department at the
Royal Academy of Fine Arts in Antwerp; he enrolled the next year, in
1976. Founded in 1663, the academy was known for its classical pedagogy,
emphasizing art over commercialism --- its fashion department, created
in 1963, was a relatively new addition.

At the Royal Academy, Van Beirendonck discovered the rigors and ambition
of design, as well as a coterie of like-minded friends. In his class was
the prodigious
\href{https://www.nytimes3xbfgragh.onion/2018/03/08/t-magazine/fashion/martin-margiela-history-fall-winter-2000-show.html}{Martin
Margiela}, who a decade later would be working in Paris for the designer
\href{https://www.nytimes3xbfgragh.onion/topic/person/jean-paul-gaultier}{Jean
Paul Gaultier} while launching his namesake line. A year later,
\href{https://www.nytimes3xbfgragh.onion/2017/10/16/t-magazine/dries-van-noten.html}{Dries
Van Noten}, Ann Demeulemeester, Dirk Van Saene, Dirk Bikkembergs and
Marina Yee would join the class below him. Antwerp only had a population
of around half a million in the 1970s, but a particularly outré
avant-garde movement had blossomed around the art gallery Wide White
Space, which represented conceptual artists such as
\href{https://www.nytimes3xbfgragh.onion/1986/01/25/obituaries/joseph-beuys-sculptor-is-dead-at-64.html}{Joseph
Beuys} and
\href{https://www.nytimes3xbfgragh.onion/2019/07/15/t-magazine/most-important-contemporary-art.html}{Marcel
Broodthaers}. There was also a burgeoning experimental theater scene led
by the young director
\href{https://www.nytimes3xbfgragh.onion/2018/11/20/t-magazine/rufus-wainwright-ivo-van-hove-conversation.html}{Ivo
van Hove}. Like its European neighbors to the east and north, stolid
Belgium --- with its
\href{https://www.nytimes3xbfgragh.onion/topic/person/rene-magritte}{Magritte}-gray
skies and cobblestone streets --- was rapidly transforming. Different in
their aesthetic leanings but similar in their drive, Van Beirendonck and
his schoolmates formed a clique. Van Beirendonck says, ``When Ann did
something, then Dries wanted to do it better, and then I wanted to do it
better.'' The emergence of new designers such as
\href{https://www.nytimes3xbfgragh.onion/2018/11/19/t-magazine/giorgio-armani-home-broni-italy.html}{Giorgio
Armani},
\href{https://www.nytimes3xbfgragh.onion/1997/07/16/style/gianni-versace-50-the-designer-who-infused-fashion-with-life-and-art.html}{Gianni
Versace},
\href{https://www.nytimes3xbfgragh.onion/2019/03/01/t-magazine/thierry-mugler-1995-show.html}{Thierry
Mugler} and Claude Montana --- who piled big shoulders on ultrasexy
feminine silhouettes and alluded to contemporary culture --- was
inspiring, too. They were proof that one didn't need to build a couture
house to make fashion that people wanted to wear.

\includegraphics{https://static01.graylady3jvrrxbe.onion/images/2019/09/03/t-magazine/fashion/walter-van-beirendonck-slide-TEAK/walter-van-beirendonck-slide-TEAK-articleLarge.jpg?quality=75\&auto=webp\&disable=upscale}

Though Belgium may have been waking from its slumber, Van Beirendonck's
real lodestar was across the sea. From the '70s through the mid-80s,
London was the primary seat of dissident and anti-establishment design,
not just in fashion but across disciplines: Britain was in the midst of
protracted financial recessions, the Conservative prime minister
\href{https://www.nytimes3xbfgragh.onion/topic/person/margaret-thatcher}{Margaret
Thatcher} had been elected, the performance artist
\href{https://www.nytimes3xbfgragh.onion/1995/01/07/obituaries/leigh-bowery-33-artist-and-model.html}{Leigh
Bowery} had opened his underground nightclub Taboo, which fueled the
polysexual and New Romantic scenes, and all of young artistic life was
in revolt. What they made thrilled a generation of young designers, who
saw in their abandonment of old mores a chance for reinvention. ``Walter
was really focused on London,'' recalls Van Noten, who was also
enthralled by the city's raw energy, riotous night life and
anti-establishment punk rock bands and fashion trends. ``He was a big
fan of I-D and
\href{https://www.nytimes3xbfgragh.onion/2019/03/26/style/the-face-magazine.html}{The
Face} and these types of magazines, and of punk clothes made by
\href{https://katharinehamnett.com/}{Katharine Hamnett} and
\href{https://www.nytimes3xbfgragh.onion/topic/person/vivienne-westwood}{Vivienne
Westwood}, and of Adam and the Ants; he was following that scene very
closely, so when there was a concert or anything like that, we all went
together, completely dressed up. Ann Demeulemeester was more on the
darker side. \ldots{} I had more of a traditional upbringing, so for me,
it was quality fabrics, couture silks. Everybody brought a different
element in the group and everybody learned from everybody.'' The friends
traveled to London and Paris, sneaking into fashion shows on forged
invitations.

Upon graduating in 1980, Van Beirendonck began styling and designing for
the Belgian raincoat company Barston. His classmates took on commercial
jobs as well. But they were still putting everything they earned back
into their own collections; Van Beirendonck debuted his first, Sado, in
1982 (he titles all of his collections). ``We really were desperate to
be in foreign magazines and be picked up by press, and that's why we had
to leave Belgium. We did things, but nobody was talking about them
outside of Belgium,'' Van Beirendonck says. In 1986, the group --- Van
Beirendonck, Van Noten, Van Saene, Bikkembergs and Yee (Demeulemeester,
nine months pregnant, stayed home) --- traveled to London with Geert
Bruloot, who ran a boutique in Antwerp. Bruloot showcased their designs
as a collective at a London fashion fair; it drew interest from buyers
in New York and Paris. The group was nicknamed the Antwerp (or Belgian)
Six to avoid the difficulty of pronouncing their Flemish names.

Image

Looks from the Cosmic Culture Clash collection, for Wild \& Lethal
Trash, fall 1994.Credit...Courtesy of Walter Van Beirendonck/Ronald
Stoops

Image

Cloud \#9, Walter Van Beirendonck in collaboration with Erwin Wurm,
spring 2012.Credit...Courtesy of Walter Van Beirendonck/Ronald Stoops

Suddenly, it seemed as though the center of the fashion world was, if
not moving away, then expanding beyond Paris and London --- New York,
Milan, Tokyo and even Antwerp were offering new models for how to be a
designer, for new clothes. Kawakubo showed her collection for the first
time in Paris in 1981 and was shocking people with her colorless,
strange and experimental designs.
\href{https://www.nytimes3xbfgragh.onion/topic/person/yohji-yamamoto}{Yohji
Yamamoto} would show a year later. Fashion was shifting from
traditionally feminine and masculine shapes and silhouettes toward
something new and undefined.

For a young designer, then, there was suddenly space --- both
existentially and physically --- to take risks. It was both the
recklessness and, paradoxically, repressiveness of those years that made
Van Beirendonck's generation \emph{want} to act out, made them
\emph{want} to choose ecstasy over safety. Who knew what tomorrow would
bring? There was also, incidentally, real financial support by way of a
program launched in the early '80s by Belgium's minister of economics,
who was hoping to find a way to reinvigorate the country's once-thriving
linen industry. A campaign called Fashion: It's Belgian was established,
along with a competition called the Golden Spindle, both with the goal
of promoting Belgian fashion designers. Unsurprisingly, nearly all of
the Antwerp Six won the Golden Spindle within the first decade of its
creation.

This was also the period in which Van Beirendonck fell in love. He and
Van Saene, a designer and artist, had become romantically involved at
school, and, as Van Beirendonck says, ``we never split up again.'' They
married last year, and Van Saene now works in a studio next to Van
Beirendonck's. Together they have formed one of those rare artistic
partnerships in which they are each present to witness the other. ``He
knows my strengths and my weaknesses,'' Van Beirendonck says. ``He is
the one I can talk with about my ideas, and he reacts in a very open and
very clear way.''

Image

From the Paradise Pleasure Productions collection, for Wild \& Lethal
Trash, fall 1995.Credit...Courtesy of Walter Van Beirendonck/Photography
by Jean-Baptiste Mondino

BUT IF THE '80s was a time of rebellion, it was, in part, a rebellion
against death, brought on by AIDS. ``It was a tough and severe period,''
Van Beirendonck tells me, ``when several people around me got sick and
died. So did so many of the artists and creative people who we looked up
to.'' And yet, even at the height of fear, even when there was little
hope of a cure, even when the shame and metaphor of the virus had become
intertwined with gay culture, Van Beirendonck remained joyful about sex.
His work from those years, such as 1995's fall collection, Paradise
Pleasure Productions, recognizes and celebrates various subcultures and
codes of gay sex: There are latex suits dripping with flaccid phalluses
and bondage masks made to look like cheap blowup dolls. They have been
leitmotifs ever since --- the fall 2012 collection, Lust Never Sleeps,
featured more bondage masks, made from a tan leather and tweed,
evocative of a British dandy --- half respectable, half subversive.

The frankness of these collections makes them mesmerizing now, but when
they were first shown, when many governments were actively trying to
stigmatize gay sex, drawing a line between it and the disease, they
would have been shocking. Their bravery is in their lack of apology. And
yet Van Beirendonck doesn't consider himself a provocateur; his
intention has never been to shock, only to present. And he persists: The
fall 2018 collection, Worlds of Sun and Moon, had pink, yellow and black
ponchos, silk bomber jackets and jumpsuits, all punctuated with glory
holes. ``I know that certain things I am doing could be shocking for
certain people,'' he said in a 2011
\href{https://www.showstudio.com/projects/in_fashion/walter-van-beirendonck}{interview
with ShowStudio}. ``But they are definitely not for me.''

This dedication to depict, unflinchingly, a certain subculture and to
explore the toll AIDS had taken on his community reached its apotheosis
in his 1996 spring collection, Killer / Astral Travel / 4D-Hi-D. Held in
the Lido nightclub in Paris, the show presented an intergalactic world
filled with neon-colored clothes, moon boots and enormous space-age
topiary-like wigs, along with a narrative about a young girl named Heidi
who befriends a space goat. The show's Heidi character was a direct
reference to the work of the American artists
\href{https://www.nytimes3xbfgragh.onion/2017/03/08/t-magazine/art/mike-kelley-mobile-homestead.html}{Mike
Kelley} and
\href{https://www.nytimes3xbfgragh.onion/topic/person/paul-mccarthy}{Paul
McCarthy}, both of whom Van Beirendonck cites as major influences in
confronting the violence and oppression of childhood (Kelley's work was
more abject, even plaintive; McCarthy's remains more aggressive and
sexually threatening). In 1992, the two collaborated on a video work
called ``\href{https://www.eai.org/titles/heidi}{Heidi},'' inspired by
the Swiss writer Johanna Spyri's 1881 novel about an orphan sent to live
in the woods with her grandfather. Kelley and McCarthy were attracted to
the allegory of Heidi's story: an innocent girl living with an old man
she wants to please. But in Van Beirendonck's interpretation, projected
onto a screen at the show, Heidi's sweet little mountain goat is
transmogrified into a devil, who represents AIDS. He called their
encounter a fatal attraction: the story of something that looks benign
suddenly transforming into something deadly.

Image

A look from Van Beirendonck's first Wild \& Lethal Trash collection, for
spring 1993.Credit...Courtesy of Walter Van Beirendonck/Ronald Stoops

Image

Van Beirendonck wearing a fall 1994 Walter Van Beirendonck
sweater.Credit...Courtesy of Walter Van Beirendonck/Ronald Stoops

IN 1992, VAN BEIRENDONCK began working with Mustang, a German denim
brand. He was given big budgets to mount elaborate shows under the label
\href{https://www.nytimes3xbfgragh.onion/1995/10/03/style/IHT-belgian-designers-walk-on-the-urban-wild-side.html}{W.\&
L.T.} (an acronym for ``Wild \& Lethal Trash'')
---\href{https://www.nytimes3xbfgragh.onion/1998/02/08/style/out-there-a-provocateur-of-the-paris-runways.html}{huge,
theatrical productions} --- that attracted an enormous amount of
attention and press. But the '90s also heralded the arrival of
minimalism, and Van Beirendonck didn't --- and couldn't --- conform to
its restrained, colorless aesthetic and cool affectlessness. He parted
ways with Mustang in 1999, and he has remained independent ever since,
free from the confines of a co-ownership with a fashion conglomerate
like LVMH or Kering. This independence explains in large part why Van
Beirendonck is lesser known than his peers and even some of his
students.

Another factor is his dedication to Antwerp, which he has never truly
left. Van Beirendonck joined the faculty at the Royal Academy in 1985
and took over the fashion department in 2007. (Van Saene joined the
faculty in 2009.) The Antwerp Six's legacy is everywhere; the school now
has an international student body, and its graduates helm some of the
most influential European fashion houses
(\href{https://www.veroniquebranquinho.com/}{Veronique Branquinho} and
\href{https://www.nytimes3xbfgragh.onion/2017/06/14/t-magazine/fashion/balenciaga-how-to-wear.html}{Balenciaga}'s
\href{https://www.nytimes3xbfgragh.onion/2016/04/11/t-magazine/gucci-alessandro-michele-balenciaga-vetements-demna-gvasalia.html}{Demna
Gvasalia} are both alumni). Van Beirendonck has a reputation as a
purist, for treating his students' collections as works of art --- and
for trying to protect them from the industry's obsession with profits.
Last year, though, a student's suicide raised questions about the
academy's rigor and isolationist tendencies. ``The suicide of our
student was an incredibly sad incident,'' Van Beirendonck says. ``We
care a lot about our students and we work with them all the time --- we
know them rather personally and form a strong bond. Since it happened,
we've rethought certain exercises and tasks to lower the pressure.''

Image

From the Wonder collection, for Walter Van Beirendonck, spring
2010.Credit...Courtesy of Walter Van Beirendonck

But if Van Beirendonck's work can be challenging to actually find --- he
sells to around 45 buyers worldwide, including the online retailer
Farfetch, which recently reissued designs from his 2,000-piece archive
--- his admirers, though small in number, are devout. The New York-based
artist \href{http://www.naylandblake.net/}{Nayland Blake} is one such
customer, who says that as a 300-pound, six-foot-two self-described
bear, wearing a Van Beirendonck poncho decorated with evil eyes and butt
plugs is a ``monstrously joyful'' experience. ``So much of our life is
spent in pursuit of a kind of conformity,'' he says, ``and to be willing
to put on a show for each other is, to me, a really exciting prospect.''

Monstrous joy may not have been what Van Beirendonck had in mind when he
began his line. He has increasingly begun emphasizing his considerable
skill as a technical designer, forcing a consideration of the
couture-level skeleton that has always lurked beneath his clothing's
brash exterior: the fine tailoring, the intricate beading, the
manipulation of traditionally fine fabrics like brocade, Harris tweed
and organza. He claims that he's never been especially adventurous in
love or life. But he has also never minded being adopted by the club
kids, the ravers, the fetishists and the freaks --- all the people who
occupy the fringes, who want to look different because looking different
makes them finally feel seen. Van Beirendonck's creations can make him
seem like a more glamorous, outsize, wilder person than he really is,
but his fashion never denies who he is, which is what fashion ultimately
does to so many. It denies our bodies, it denies our hunger, it denies
our weird and varied beauty. Van Beirendonck may be one of very few
designers left who is radically true to himself. ``In the beginning,
when I worked on my collections, it allowed me to escape from all this
darkness, and I was also trying to give some brightness to the world,''
he says. ``That's exactly what we have now, a dark world, and we want to
at least have some hope.''

\begin{center}\rule{0.5\linewidth}{\linethickness}\end{center}

Hair by Louis Ghewy at M\&A using Oribe. Grooming by Florence Teerlinck
using MAC Cosmetics. Production: Mindbox. Photo assistants: Willy
Cuylits and Tim Coppens. Hair assistant: Nina Stenger. Grooming
assistant: Gwen de Vylder

Advertisement

\protect\hyperlink{after-bottom}{Continue reading the main story}

\hypertarget{site-index}{%
\subsection{Site Index}\label{site-index}}

\hypertarget{site-information-navigation}{%
\subsection{Site Information
Navigation}\label{site-information-navigation}}

\begin{itemize}
\tightlist
\item
  \href{https://help.nytimes3xbfgragh.onion/hc/en-us/articles/115014792127-Copyright-notice}{©~2020~The
  New York Times Company}
\end{itemize}

\begin{itemize}
\tightlist
\item
  \href{https://www.nytco.com/}{NYTCo}
\item
  \href{https://help.nytimes3xbfgragh.onion/hc/en-us/articles/115015385887-Contact-Us}{Contact
  Us}
\item
  \href{https://www.nytco.com/careers/}{Work with us}
\item
  \href{https://nytmediakit.com/}{Advertise}
\item
  \href{http://www.tbrandstudio.com/}{T Brand Studio}
\item
  \href{https://www.nytimes3xbfgragh.onion/privacy/cookie-policy\#how-do-i-manage-trackers}{Your
  Ad Choices}
\item
  \href{https://www.nytimes3xbfgragh.onion/privacy}{Privacy}
\item
  \href{https://help.nytimes3xbfgragh.onion/hc/en-us/articles/115014893428-Terms-of-service}{Terms
  of Service}
\item
  \href{https://help.nytimes3xbfgragh.onion/hc/en-us/articles/115014893968-Terms-of-sale}{Terms
  of Sale}
\item
  \href{https://spiderbites.nytimes3xbfgragh.onion}{Site Map}
\item
  \href{https://help.nytimes3xbfgragh.onion/hc/en-us}{Help}
\item
  \href{https://www.nytimes3xbfgragh.onion/subscription?campaignId=37WXW}{Subscriptions}
\end{itemize}
