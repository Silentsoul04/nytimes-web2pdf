Sections

SEARCH

\protect\hyperlink{site-content}{Skip to
content}\protect\hyperlink{site-index}{Skip to site index}

\href{https://www.nytimes3xbfgragh.onion/section/world/asia}{Asia
Pacific}

\href{https://myaccount.nytimes3xbfgragh.onion/auth/login?response_type=cookie\&client_id=vi}{}

\href{https://www.nytimes3xbfgragh.onion/section/todayspaper}{Today's
Paper}

\href{/section/world/asia}{Asia Pacific}\textbar{}Hong Kong's Leader,
Carrie Lam, to Withdraw Extradition Bill That Ignited Protests

\url{https://nyti.ms/2PGyJWz}

\begin{itemize}
\item
\item
\item
\item
\item
\item
\end{itemize}

Advertisement

\protect\hyperlink{after-top}{Continue reading the main story}

Supported by

\protect\hyperlink{after-sponsor}{Continue reading the main story}

\hypertarget{hong-kongs-leader-carrie-lam-to-withdraw-extradition-bill-that-ignited-protests}{%
\section{Hong Kong's Leader, Carrie Lam, to Withdraw Extradition Bill
That Ignited
Protests}\label{hong-kongs-leader-carrie-lam-to-withdraw-extradition-bill-that-ignited-protests}}

\includegraphics{https://static01.graylady3jvrrxbe.onion/images/2019/09/04/world/04hongkong1-sub/04hongkong1-sub-videoSixteenByNine3000.jpg}

By \href{https://www.nytimes3xbfgragh.onion/by/austin-ramzy}{Austin
Ramzy} and Elaine Yu

\begin{itemize}
\item
  Published Sept. 4, 2019Updated Sept. 5, 2019
\item
  \begin{itemize}
  \item
  \item
  \item
  \item
  \item
  \item
  \end{itemize}
\end{itemize}

\href{https://cn.nytimes3xbfgragh.onion/china/20190904/hong-kong-carrie-lam-extradition-bill/}{阅读简体中文版}\href{https://cn.nytimes3xbfgragh.onion/china/20190904/hong-kong-carrie-lam-extradition-bill/zh-hant/}{閱讀繁體中文版}

HONG KONG --- Carrie Lam, Hong Kong's chief executive, said Wednesday
that the government would withdraw a contentious extradition bill that
ignited months of protests in the city, moving to quell the worst
political crisis since the former British colony returned to Chinese
control 22 years ago.

The move responds to a major demand of the protesters, who feared China
would exploit the measure to extradite suspects for prosecution in
China's opaque judicial system. But it was unclear if the concession
would be enough to bring an end to intensifying demonstrations, which
are now driven by multiple grievances with the government.

``Incidents over these past two months have shocked and saddened Hong
Kong people,'' she said in an eight-minute televised statement broadcast
shortly before 6 p.m. ``We are all very anxious about Hong Kong, our
home. We all hope to find a way out of the current impasse and
unsettling times.''

\includegraphics{https://static01.graylady3jvrrxbe.onion/images/2019/09/03/world/03hk-lam4/03hk-lam4-videoSixteenByNine3000.jpg}

Her decision, which was met with skepticism by some pro-democracy
figures in Hong Kong, comes as the protests near their three-month mark
and show little sign of abating, roiling a city known for its
orderliness and hurting its economy.

It also came as something of a surprise: Just a day before, China's Hong
Kong and Macau Affairs Office had signaled an uncompromising stance
toward the protests. Yang Guang, a spokesman for the office, said at a
briefing in Beijing that there could be ``no middle ground, no hesitance
and no dithering, when it comes to stopping the violence and controlling
riots in Hong Kong.''

A possible hint of a change in Beijing's stand, however, came from the
country's leader, Xi Jinping. In a speech on Tuesday to the Party School
of the Communist Party's Central Committee, Mr. Xi called on rising
party officials to show resolve for a long struggle but suggested that
the leadership could adjust its tactics to achieve its aims.

``On matters of principle, not an inch will be yielded,'' Mr. Xi said,
``but on matters of tactics there can be flexibility.''

{[}\href{https://www.nytimes3xbfgragh.onion/2019/09/04/world/asia/hong-kong-carrie-lam-protests.html}{\emph{Hong
Kong's leader partly relents. Will the protests continue?}}{]}

\href{http://gis.hkbu.edu.hk/staff/cabestan.html}{Jean-Pierre Cabestan},
a professor at Hong Kong Baptist University and the author of ``China
Tomorrow: Democracy or Dictatorship?'' suggested on Wednesday evening
that Beijing had asked Mrs. Lam to make the decision as a tactical
calculation ahead of the Oct. 1 anniversary of the establishment of the
People's Republic of China.

The aim, he said, was ``to calm down the movement's moderates'' and
``weaken and isolate the radicals.''

``Maybe it is a good calculation on the part of Beijing,'' he added,
``but it may also fail.''

By early Thursday, the Chinese government had not made any public
comment on Mrs. Lam's announcement, and state-run Chinese media gave her
withdrawal of the extradition bill relatively brief, muted coverage.
Absent was any of the fiery commentary that has dominated mainland
Chinese reporting of Hong Kong's turmoil, suggesting that Communist
Party officials may still be working out a reaction designed to counter
accusations of retreat.

\includegraphics{https://static01.graylady3jvrrxbe.onion/images/2019/09/04/world/04hongkong2-sub/merlin_156527316_4ee5d551-40c9-42ab-8334-b792f7a00b5e-articleLarge.jpg?quality=75\&auto=webp\&disable=upscale}

Mrs. Lam had
\href{https://www.nytimes3xbfgragh.onion/2019/06/15/world/asia/hong-kong-protests-extradition-law.html}{suspended
the bill in June} and later
\href{https://www.nytimes3xbfgragh.onion/2019/07/08/world/asia/carrie-lam-hong-kong.html}{said
that it was ``dead,''} but demonstrators have long been suspicious of
her government's refusal to formally withdraw the bill and feared that
it could be revived at a later date.

Withdrawal of the bill has remained at the top of the list of
protesters' demands. But the list has grown to include an independent
investigation into the police response, amnesty for arrested protesters
and direct elections for all lawmakers and the chief executive.

{[}\emph{\href{https://www.nytimes3xbfgragh.onion/2019/08/08/world/asia/hong-kong-protests-explained.html}{What's
going on in Hong Kong? What to know about the
protests}}\href{https://www.nytimes3xbfgragh.onion/2019/08/08/world/asia/hong-kong-protests-explained.html}{.}{]}

Michael Tien, a moderate pro-Beijing lawmaker, said withdrawal alone
might have been enough to calm the protests in mid-June. But since then,
``with the accumulation of so much resentment, so many accusations and
so many disputes,'' the establishment of an independent inquiry ``is 100
percent necessary,'' Mr. Tien said.

At least some of the hard-line, pro-Beijing camp in Hong Kong expressed
skepticism on Wednesday evening about Mrs. Lam's overture, seeing it not
as a clever gambit to ease pressure but rather as a sign of political
weakness that would only encourage further protests.

One hard-liner, who insisted on anonymity because of political
sensitivities, said that the initial hostility to the overture from
democracy advocates showed that the hard-liners' worries about
concessions were being vindicated.

Mrs. Lam described the withdrawal as a step to initiate dialogue. She
also said she would add two members to an existing police review board,
but that step was far short of calls for an independent investigation.

Claudia Mo, a pro-democracy lawmaker, described Mrs. Lam's announcement
as a ``political performance.''

``That it took her three months to formally use the word `withdraw' is
truly too little, too late,'' Ms. Mo told reporters. ``A big mistake has
been made.''

This summer has seen peaceful marches involving hundreds of thousands of
people, as well as street protests by smaller groups who have become
increasingly violent in recent weeks, throwing bricks and firebombs at
the police. More than 1,100 people have been arrested since early June.
The police, who have used batons, rubber bullets and tear gas against
protesters, have faced allegations of excessive force.

Months of protests have started to ripple through the economy, hurting
some small businesses and the tourism industry. Many economists believe
the city's economy is now slipping into recession.

Hong Kong's Hang Seng Index closed up 3.9 percent on Wednesday as the
prospect of Mrs. Lam's news conference began circulating. Cathay
Pacific, the Hong Kong-based airline that has faced criticism from the
Chinese government for its employees' participation in the protests,
climbed more than 7 percent. After the market closed, the
\href{https://www.nytimes3xbfgragh.onion/2019/09/04/business/cathay-pacific-chairman-resigns.html}{company
announced the resignation} of its chairman.

Withdrawal of the extradition bill was the initial demand of protesters,
and the rallying cry when, by organizers' estimates,
\href{https://www.nytimes3xbfgragh.onion/2019/06/09/world/asia/hong-kong-extradition-protest.html}{more
than one million people marched} on June 9 and
\href{https://www.nytimes3xbfgragh.onion/2019/06/16/world/asia/carrie-lam-hong-kong-protests.html}{nearly
two million marched a week later}, more than one in every four people in
Hong Kong.

Image

Saturday saw some of the most intense clashes since the protests began,
including the use of firebombs by demonstrators.Credit...Laurel Chor for
The New York Times

Withdrawal of the bill ``will help to an insignificant extent,'' said
Willy Lam, an adjunct professor at the Chinese University of Hong Kong's
Center for China Studies.

The concession ``might pacify a small sector of the population but it
will not have any impact on whether the waves of protests would
subside,'' he added.

On LIHKG, an online forum popular with protesters, several posts on
Wednesday repeated longstanding calls not to compromise until all
demands are met. The Civil Human Right Front, which has organized huge
protest marches against the bill, said it would also continue its
campaign until the protesters' calls were satisfied.

Full withdrawal of the extradition bill has long been seen as the
easiest compromise that the government could make. But mainland Chinese
officials had objected to that possibility, saying that doing so would
suggest that the original intentions behind the legislation were
mistaken.

Chinese officials had also said that any independent inquiry into the
police's conduct and other aspects of the unrest could not be started
until the protests died down.

Over weeks of protests, state news outlets and other commentators on the
mainland unleashed scathing criticism of the protesters, portraying them
as rioters and, in some cases, suggesting they were traitors or
terrorists. One analyst said Beijing may have been motivated to allow a
concession in order to begin reining in the nationalistic rhetoric to
some degree.

``Beijing loves the nationalistic sentiment, but only to the point that
it's controllable,'' said Samson Yuen, an assistant professor of
political science at Lingnan University in Hong Kong. ``At some point
Beijing wants to rein that in for fear that these nationalistic
sentiments will point toward the government.''

As the protests dragged on, pro-Beijing lawmakers had expressed concern
that the anger at the government would hurt their camp in district
council elections in November and legislative elections next year.

Saturday, the fifth anniversary of a decision by China's legislature to
put limits on direct elections in Hong Kong,
\href{https://www.nytimes3xbfgragh.onion/2019/08/31/world/asia/hong-kong-protest.html}{saw
some of the most intense clashes} since the protests began. After a
march by tens of thousands, some protesters gathered around the main
government offices, hurling rocks and firebombs. Riot police officers
fired tear gas and pumped blue-dyed water from trucks at protesters.

Protesters built barricades and set fires, and the police later pursued
them across several neighborhoods, arresting dozens. In a subway station
in the Prince Edward neighborhood, officers from the police's Special
Tactical Squad entered a stopped train, using batons to hit people who
were crouching on the floor and dousing them with pepper spray.

The Chinese government was initially silent on this summer's protests,
then began to condemn them in increasingly strident tones, warning that
the military could be called in. Images of Chinese police officers and
paramilitary troops conducting anti-riot drills in Shenzhen, a mainland
city near Hong Kong, were given regular coverage by state media outlets.

On Friday, the police in Hong Kong
\href{https://www.nytimes3xbfgragh.onion/2019/08/29/world/asia/joshua-wong-hong-kong.html}{arrested
several prominent activists and three pro-democracy lawmakers} as a
crackdown on the opposition intensified.

Advertisement

\protect\hyperlink{after-bottom}{Continue reading the main story}

\hypertarget{site-index}{%
\subsection{Site Index}\label{site-index}}

\hypertarget{site-information-navigation}{%
\subsection{Site Information
Navigation}\label{site-information-navigation}}

\begin{itemize}
\tightlist
\item
  \href{https://help.nytimes3xbfgragh.onion/hc/en-us/articles/115014792127-Copyright-notice}{©~2020~The
  New York Times Company}
\end{itemize}

\begin{itemize}
\tightlist
\item
  \href{https://www.nytco.com/}{NYTCo}
\item
  \href{https://help.nytimes3xbfgragh.onion/hc/en-us/articles/115015385887-Contact-Us}{Contact
  Us}
\item
  \href{https://www.nytco.com/careers/}{Work with us}
\item
  \href{https://nytmediakit.com/}{Advertise}
\item
  \href{http://www.tbrandstudio.com/}{T Brand Studio}
\item
  \href{https://www.nytimes3xbfgragh.onion/privacy/cookie-policy\#how-do-i-manage-trackers}{Your
  Ad Choices}
\item
  \href{https://www.nytimes3xbfgragh.onion/privacy}{Privacy}
\item
  \href{https://help.nytimes3xbfgragh.onion/hc/en-us/articles/115014893428-Terms-of-service}{Terms
  of Service}
\item
  \href{https://help.nytimes3xbfgragh.onion/hc/en-us/articles/115014893968-Terms-of-sale}{Terms
  of Sale}
\item
  \href{https://spiderbites.nytimes3xbfgragh.onion}{Site Map}
\item
  \href{https://help.nytimes3xbfgragh.onion/hc/en-us}{Help}
\item
  \href{https://www.nytimes3xbfgragh.onion/subscription?campaignId=37WXW}{Subscriptions}
\end{itemize}
