Sections

SEARCH

\protect\hyperlink{site-content}{Skip to
content}\protect\hyperlink{site-index}{Skip to site index}

\href{https://myaccount.nytimes3xbfgragh.onion/auth/login?response_type=cookie\&client_id=vi}{}

\href{https://www.nytimes3xbfgragh.onion/section/todayspaper}{Today's
Paper}

Brunello Cucinelli, Renaissance Man

\url{https://nyti.ms/2LtlmUh}

\begin{itemize}
\item
\item
\item
\item
\item
\end{itemize}

Advertisement

\protect\hyperlink{after-top}{Continue reading the main story}

Supported by

\protect\hyperlink{after-sponsor}{Continue reading the main story}

Profile in Style

\hypertarget{brunello-cucinelli-renaissance-man}{%
\section{Brunello Cucinelli, Renaissance
Man}\label{brunello-cucinelli-renaissance-man}}

The designer, best known for what he calls ``sportivo chic,'' has
maintained roots in his native Italy, improving and restoring the town
where his brand is based.

By \href{https://www.nytimes3xbfgragh.onion/by/lindsay-talbot}{Lindsay
Talbot}

\begin{itemize}
\item
  Published Sept. 6, 2019Updated Sept. 9, 2019
\item
  \begin{itemize}
  \item
  \item
  \item
  \item
  \item
  \end{itemize}
\end{itemize}

Even as a child growing up in the rural Umbrian village of Castel
Rigone, \href{http://www.brunellocucinelli.com/en/}{Brunello Cucinelli}
had strong opinions about clothes. ``When I was 7, my mother gave me a
pair of green velvet trousers for Christmas,'' says the designer. ``I
buried them in the garden behind our house because I've never been too
fond of green.'' A decade later, he dropped out of engineering school
and, in 1978, launched an eponymous line of women's sweaters in punchy
shades inspired by United Colors of Benetton but woven in fine cashmere
and Shetland wool.

Today, Cucinelli designs ready-to-wear collections for men, women and,
as of this month, children, as well as home décor, shoes and handbags.
The 66-year-old oversees a factory and some 1,000 workers in the hamlet
of
\href{https://www.nytimes3xbfgragh.onion/2018/09/21/fashion/brunello-cucinelli-italy.html}{Solomeo}
(population 500), which is nine miles from
\href{https://www.nytimes3xbfgragh.onion/interactive/2017/09/14/travel/what-to-do-36-hours-in-perugia-italy.html}{Perugia}
and 15 from his hometown. He's spent decades improving and restoring
Solomeo --- repaving roads, planting vineyards, erecting a
16th-century-style theater and founding an arts and crafts school ---
and over time it's become nearly synonymous with the label itself, which
is best known for what Cucinelli describes as ``sportivo chic.'' That
means plush and drapey cardigans and crew necks, softly tailored
trousers and cashmere-covered collectibles (soccer balls, waste bins) in
a serene earth-toned palette reflective of the landscape that surrounds
those that work, eat and live in Cucinelli's domain. The designer's
understated wares are popular with the Silicon Valley crowd ---
\href{https://www.nytimes3xbfgragh.onion/topic/person/steve-jobs}{Steve
Jobs}'s black mock
\href{https://www.nytimes3xbfgragh.onion/2016/01/03/magazine/can-the-turtleneck-ever-be-cool-again.html}{turtlenecks}
were custom Cucinelli, as are
\href{https://www.nytimes3xbfgragh.onion/topic/person/mark-zuckerberg}{Mark
Zuckerberg}'s signature gray T-shirts --- and yet the designer himself
is proudly old-fashioned. ``I want everything I create to be totally
timeless, much like a good book,'' says Cucinelli. ``It should be
something you keep forever and go back to over and over.''

\emph{{[}}\href{https://www.nytimes3xbfgragh.onion/newsletters/t-list?module=inline}{\emph{Sign
up here}} \emph{for the T List newsletter, a weekly roundup of what T
Magazine editors are noticing and coveting now.{]}}

\includegraphics{https://static01.graylady3jvrrxbe.onion/images/2019/09/06/t-magazine/06tmag-brunello-slide-L1SP/06tmag-brunello-slide-L1SP-articleLarge.jpg?quality=75\&auto=webp\&disable=upscale}

``I love the colors in this picture, which shows me on a walk through
Solomeo. I'm with my Lab, Viola, whose coat is the perfect shade of
dirty white. Everything I'm wearing I designed, from the sneakers to the
denim to the coat, which I've had for 23 years.''

\begin{center}\rule{0.5\linewidth}{\linethickness}\end{center}

Image

Credit...From left: Weichia Huang; courtesy of Brunello Cucinelli

\emph{Left:} ``Anytime I wear a jacket --- so almost every day --- I
choose an accompanying pocket square. For me, it's mandatory. Say you
see a guy in a restaurant: You notice his jacket, his watch and his
pocket square. There's not much else men have to distinguish themselves.
I tend to wear the same ones again and again.''

\emph{Right:} ``My dear friend Stefano Giannoni is a sculptor who works
in the style of
\href{https://www.nytimes3xbfgragh.onion/topic/person/michelangelo-buonarroti}{Michelangelo}
and keeps this studio on Rome's Via Col di Nava. You can see marble
carvings of Marcus Aurelius, Dionysus, Hadrian and Pericles. I have his
pieces in my home and have given his marble busts as gifts.''

\begin{center}\rule{0.5\linewidth}{\linethickness}\end{center}

Image

Credit...Courtesy of Brunello Cucinelli

\emph{Left:} ``This is cashmere in its original state, before it's made
into yarn. I have about a hundred of these sorts of samples in my office
and mix the different colors to create custom shades, blending a red and
an ecru, say. All of our cashmere comes from Mongolia and Inner
Mongolia, where I've traveled and helped shear the goats myself.''

\emph{Right:} ``I never went to art school and am not able to draw
myself, but I am able to explain exactly what I want after playing with
fabric and test-wearing things --- a jacket should sit just close enough
to the chest, and the length of a hem is paramount. This sketch shows a
look I came up with for our fall 2019 men's wear collection: a burgundy
velvet suit worn with a single-pleated pant and a suede desert boot.''

\begin{center}\rule{0.5\linewidth}{\linethickness}\end{center}

Image

Credit...From left: © Everyman History, 1945; courtesy of Brunello
Cucinelli

\emph{Left:} ``Thomas More's
``\href{https://www.bl.uk/learning/timeline/item126618.html}{Utopia}''
(1516), about the social and political makeup of an imaginary country,
is among the 15 or 20 pieces of literature that have defined my life. To
me, More is one of the great geniuses of humanity, and a great teacher.
Every year, I give my employees a book that's made a big impression on
me. Other past selections have been
``\href{https://www.penguinrandomhouse.com/books/286572/don-quixote-by-miguel-de-cervantes-saavedra/9780142437230/}{Don
Quixote}'' (1605) and
``\href{https://www.penguinrandomhouse.com/books/306130/the-analects-by-confucius/9780143106852/}{The
Analects of Confucius}'' (500 B.C.).''

\emph{Right:} ``A Brunello Cucinelli ad from 1992. It was the first year
we focused on more leisurely clothes that you could wear around the
house. The premise for the photo was that I'd invited my friends to
dinner and everyone had arrived dressed in roomy trousers and elevated
sweatpants. I'm on the far left, and I have to say, I think I look
pretty good. After all, I was only in my 40s back then.''

\begin{center}\rule{0.5\linewidth}{\linethickness}\end{center}

Image

Credit...Courtesy of Brunello Cucinelli

\emph{Left:} ``My wife, Federica, and I met when we were teenagers ---
she would ride the school bus in the morning, and I'd follow behind on
my scooter in the hopes that she might become interested. I can't
imagine how much exhaust I must have inhaled. In this picture of us
visiting a Roman temple in Northern Italy, I think she's 18 and I'm 19,
making it somewhere around 1973. We're wearing flared jeans that were so
tight we had to lie down to zip them.''

\emph{Right:} ``They say that Italian food should have a maximum of
three strong flavors. This
\href{https://cooking.nytimes3xbfgragh.onion/68861692-nyt-cooking/807163-our-greatest-pizza-recipes}{pizza}
was made by one of the cooks in the brand's cafeteria, where we serve
800 people a day and always try for the freshest ingredients: Here,
you've got tomatoes, mozzarella and a touch of basil.''

\begin{center}\rule{0.5\linewidth}{\linethickness}\end{center}

Image

Credit...Courtesy of Brunello Cucinelli

\emph{Left:} ``Our family villa in Solomeo dates back to the late 1500s.
Beneath the Renaissance-style arches of the dining room you'll see
stacks of books that are very dear to me, a statue of Dionysus and a
17th-century oil painting depicting the annunciation of Abraham.''

\emph{Right:} ``When I moved to Solomeo, it was in rough shape --- the
village was founded in the 14th century, but by the 1950s, few people
and fewer jobs remained. Over the last 30 years, I've helped to restore
it by building a theater, a winery and, of course, our factories, which
I hope will be around for another century or two. I suppose I feel a bit
like a custodian for this place, which is surrounded by beautiful
countryside on all sides.''

\begin{center}\rule{0.5\linewidth}{\linethickness}\end{center}

Image

Credit...From left: Courtesy of Brunello Cucinelli; Giotto di Bondone,
``St. Francis Revives the Unatoned Woman to Facilitate Her Confession,''
fresco, circa 1297-99, Assissi, Italy © Raffaello Bencini/Bridgeman
Images

\emph{Left:} ``As a teenager, I practiced jujitsu. I found it gave me
self-confidence and serenity. The best takeaway was the idea that if
someone offends you, you should count to 10 before you respond. To this
day, I've never really had big arguments with anyone, and I think that
must be why. Because the photo is black and white, you can't tell that
my hair was dyed bright orange --- I was a model for a hairdresser at
the time.''

\emph{Right:}
``\href{https://www.nytimes3xbfgragh.onion/1937/01/17/archives/giotto-600-years-after-sexcentenary-of-death-of-the-painter-to-be.html}{Giotto
di Bondone} was one of the most important Italian artists of the late
Middle Ages --- he was the first to paint humans realistically and with
depth. We in Solomeo are lucky that his frescoes at the Basilica of
Saint Francis of Assisi are only a few towns away. This one is called
`St. Francis Revives the Unatoned Woman to Facilitate Her Confession'
(circa 1298). I remember seeing it as a child and marveling at its
beauty and precision even then.''

Advertisement

\protect\hyperlink{after-bottom}{Continue reading the main story}

\hypertarget{site-index}{%
\subsection{Site Index}\label{site-index}}

\hypertarget{site-information-navigation}{%
\subsection{Site Information
Navigation}\label{site-information-navigation}}

\begin{itemize}
\tightlist
\item
  \href{https://help.nytimes3xbfgragh.onion/hc/en-us/articles/115014792127-Copyright-notice}{©~2020~The
  New York Times Company}
\end{itemize}

\begin{itemize}
\tightlist
\item
  \href{https://www.nytco.com/}{NYTCo}
\item
  \href{https://help.nytimes3xbfgragh.onion/hc/en-us/articles/115015385887-Contact-Us}{Contact
  Us}
\item
  \href{https://www.nytco.com/careers/}{Work with us}
\item
  \href{https://nytmediakit.com/}{Advertise}
\item
  \href{http://www.tbrandstudio.com/}{T Brand Studio}
\item
  \href{https://www.nytimes3xbfgragh.onion/privacy/cookie-policy\#how-do-i-manage-trackers}{Your
  Ad Choices}
\item
  \href{https://www.nytimes3xbfgragh.onion/privacy}{Privacy}
\item
  \href{https://help.nytimes3xbfgragh.onion/hc/en-us/articles/115014893428-Terms-of-service}{Terms
  of Service}
\item
  \href{https://help.nytimes3xbfgragh.onion/hc/en-us/articles/115014893968-Terms-of-sale}{Terms
  of Sale}
\item
  \href{https://spiderbites.nytimes3xbfgragh.onion}{Site Map}
\item
  \href{https://help.nytimes3xbfgragh.onion/hc/en-us}{Help}
\item
  \href{https://www.nytimes3xbfgragh.onion/subscription?campaignId=37WXW}{Subscriptions}
\end{itemize}
