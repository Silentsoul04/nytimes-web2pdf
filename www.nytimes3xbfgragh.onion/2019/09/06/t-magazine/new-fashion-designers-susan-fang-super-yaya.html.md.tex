Sections

SEARCH

\protect\hyperlink{site-content}{Skip to
content}\protect\hyperlink{site-index}{Skip to site index}

\href{https://myaccount.nytimes3xbfgragh.onion/auth/login?response_type=cookie\&client_id=vi}{}

\href{https://www.nytimes3xbfgragh.onion/section/todayspaper}{Today's
Paper}

6 Emerging Designers to Watch This Fashion Month

\url{https://nyti.ms/2HPzczp}

\begin{itemize}
\item
\item
\item
\item
\item
\item
\end{itemize}

Advertisement

\protect\hyperlink{after-top}{Continue reading the main story}

Supported by

\protect\hyperlink{after-sponsor}{Continue reading the main story}

\hypertarget{6-emerging-designers-to-watch-this-fashion-month}{%
\section{6 Emerging Designers to Watch This Fashion
Month}\label{6-emerging-designers-to-watch-this-fashion-month}}

Meet the rising talents who will be showing their collections in New
York, London, Paris and Milan.

By \href{https://www.nytimes3xbfgragh.onion/by/angela-koh}{Angela Koh}

\begin{itemize}
\item
  Published Sept. 6, 2019Updated Sept. 9, 2019
\item
  \begin{itemize}
  \item
  \item
  \item
  \item
  \item
  \item
  \end{itemize}
\end{itemize}

\includegraphics{https://static01.graylady3jvrrxbe.onion/images/2019/09/06/t-magazine/06tmag-new-designers-slide-GO7F/06tmag-new-designers-slide-GO7F-articleLarge.jpg?quality=75\&auto=webp\&disable=upscale}

\hypertarget{fall-risk-by-john-targon}{%
\subsection{Fall Risk by John Targon}\label{fall-risk-by-john-targon}}

\emph{John Targon, 36, New York}

John Targon is already a familiar name in the fashion world. As a
co-founder of the New York-based **** brand Baja East and the former
creative director of Marc Jacobs's contemporary ventures, the
36-year-old Chicago native has been in the industry's limelight on and
off for half a decade. In 2017, after leaving Baja East, Targon took a
step back from fashion. ``I wanted to shut the door on shame and bring
forward my life's learnings,'' he says. This ultimately led to his new
label, \href{https://www.instagram.com/fallriskinc/}{Fall Risk} by John
Targon, which he launched in April of this year. The name is lifted from
the wristbands given to medical patients, which caught Targon's eye when
he was looking out for an intoxicated friend at a hospital in New York.
``Life is unexpected and these chapters of our story are both humbling
and hilarious --- your clothes should reflect that,'' he says. Instead
of showing seasonal collections, Targon is selling his garments ---
which are inspired by retro sportswear and streetwear --- directly to
his customers while creating buzz through social media. A knitwear
expert, Targon seeks to create comfortable yet thoughtful clothing that
is seasonless and uncomplicated. ``My clothes should be a second skin so
that when you're comfortable in them, well then, they're yours,'' he
says. His latest collection, which will be Fall Risk's fourth, is
inspired by the idea of remix, and it references the '70s and '90s as
well as pieces from his own past collections. There's also a subtle
sports motif, seen in an argyle polo shirt inspired by boarding school
uniforms and a chunky powder-blue sleeveless sweater with a shrunken
fit.

\emph{{[}}\href{https://www.nytimes3xbfgragh.onion/newsletters/t-list?module=inline}{\emph{Sign
up here}} \emph{for the T List newsletter, a weekly roundup of what T
Magazine editors are noticing and coveting now.{]}}

\begin{center}\rule{0.5\linewidth}{\linethickness}\end{center}

Image

Credit...Photo and portrait by Richie Shazam.

\hypertarget{puppets-and-puppets}{%
\subsection{Puppets and Puppets}\label{puppets-and-puppets}}

\emph{Carly Mark, 31, and Ayla Argentina, 27, New York}

Carly Mark and Ayla Argentina met in 2016 through a mutual friend in New
York City. Mark was working as a contemporary artist at the time but
considering a career change (she had previously interned at Versace and
Marc Jacobs before working as a gallery assistant at Gavin Brown's
Enterprise). ``Once I realized fashion didn't have to be corporate, and
the art world was more corporate than I had anticipated, I moved back
toward clothes again. This time feeling very free about it,'' she says.
Argentina (who uses the pronoun ``they''), had a background in fashion
design, having interned and worked for brands such as Ralph Lauren and
TSE Cashmere. Mark asked Argentina to collaborate on costumes for a
video she was working on for a solo exhibition. Shortly after, Argentina
joined Mark's art studio as her costume assistant and they continued to
create garments for her work. In 2018, they debuted a collection of
their pieces --- patchwork suits and dramatic fuzzy dusters inspired by
medieval garb --- during New York Fashion Week, and the brand
\href{https://www.puppetsandpuppets.com/}{Puppets and Puppets} was born.
For the spring 2020 season, the designers will present their second
collection, which Mark describes as ``Romanov meets `American Psycho.'''
Expect to see tailored suits evocative of Wall Street businessmen mixed
with Russian-inspired outerwear and hoop skirts. ``Fashion is so
corporate here that there is a pushback happening,'' Mark says. ``Young
designers are doing it their way, on their terms, whatever way they
want.''

\begin{center}\rule{0.5\linewidth}{\linethickness}\end{center}

Image

Credit...Photo by Anna Victoria Best. Portrait by Flora Karamolegkou.

\hypertarget{eftychia}{%
\subsection{Eftychia}\label{eftychia}}

\emph{Eftychia Karamolegkou, London}

Eftychia Karamolegkou always had an eye for design. She grew up in
Greece surrounded by paintings and hand-carved furniture made by her
grandfather, a trained architect and self-taught artist, which continue
to define her minimal aesthetic. Later, she worked in graphic design
before enrolling in the fashion design master's course at Central Saint
Martins and interning for the London-based brands Mary Katrantzou and
Marques Almeida. Her graduate collection, shown in 2017, centered on
suiting and quickly caught the attention of stores such as Opening
Ceremony and
\href{https://www.machine-a.com/collections/designer-eftychia}{Machine
A}, prompting her to found her own namesake label that same year. The
brand still focuses on innovative tailoring; Karamolegkou takes
inspiration from office wear, which she then transforms with baggy cuts
and neutral colors. ``I think I gravitate toward tailoring because it is
a kind of secret language,'' she says. ``It has so many codes that by
changing details, you can reveal different messages.'' For her next
collection, she mined business meetings and corporate hierarchies for
inspiration, creating imaginary characters that informed her pieces. The
archetypal chairman, for example, who doesn't need to prove himself and
can adopt a more relaxed look, inspired a clean loose fitted shirt
paired with relaxed trousers, while for a C.E.O. she designed a full
suit. Karamolegkou's signature two-tone technique, in which she mixes
tonal browns in one suiting fabric, will reoccur in this collection
along with new styles such as a wool mohair Harrington jacket.

\begin{center}\rule{0.5\linewidth}{\linethickness}\end{center}

Image

Credit...Photo courtesy of Paula Canovas del Vas. Portrait by Coco
Capitán.

\hypertarget{paula-canovas-del-vas}{%
\subsection{Paula Canovas del Vas}\label{paula-canovas-del-vas}}

\emph{Paula Canovas del Vas, 28, London}

\href{https://paulacanovasdelvas.com/}{Paula Canovas del Vas} was born
and raised in Murcia, Spain, where she spent much of her childhood at
her mother's bridal dress store learning from the atelier's seamstresses
and patternmakers. In 2013, she moved to London to freelance at the
British brand Ashish and later interned at Gucci in Rome and Margiela in
Paris --- but she always wanted to design her own collections and began
working on her own label while studying for her master's at Central
Saint Martins. In 2018, her graduate collection of fantastically
patterned neon jackets was picked up by Dover Street Market in Tokyo.
``My tendency is to gravitate toward shieldlike volumes,'' she says of
her oversize, eye-catching shapes. ``There is something very comforting
about wearing them.'' This month, Vas will present her third collection
in London via a virtual reality installation that will be open to the
public. The idea was partly inspired by the idea of voyeurism and the
personal broadcasting made possible by social media. ``I wanted to offer
an alternative to a fashion catwalk and encapsulate the experience in a
way that would be long-lasting and democratized,'' she says. Vas also
wanted to give viewers insight into how she sources her upcycled
materials and works with artisans in southern Spain to develop her
fabrics, such as synthetic patent leather and whimsically embossed
knits. One particular dress in the new collection, which was constructed
from 10 meters of pink and green dead-stock organza, took two weeks to
make.

\begin{center}\rule{0.5\linewidth}{\linethickness}\end{center}

Image

Credit...Simonas Berkutis

\hypertarget{susan-fang}{%
\subsection{Susan Fang}\label{susan-fang}}

\emph{Susan Fang, 26, Milan}

\href{https://www.susanfangofficial.com/}{Susan Fang} began designing
clothes at age 5, for the girls in her comic books. Born in Yuyao,
China, she moved from place to place during her childhood, from China to
England, Canada and the U.S. ``It was a lot of changes,'' she says.
``But it made me extremely interested in the difference between people's
perceptions and perspectives.'' Fang became passionate about exploring
the arts, culture and fashion of the places in which she lived, and she
eventually settled in London, where she studied fashion at Central Saint
Martins. After graduating, she spent two years gaining experience at
Celine and Stella McCartney before founding her eponymous label in 2018.
Characterized by naturalistic and geometric motifs, her otherworldly
pieces include bags made from bubblelike glass beads and sheer pastel
dresses, but perhaps her most notable invention is the ``airweave,'' a
garment made from strips of featherlight fabric (such as chiffon, yarn)
that shape-shifts as the wearer moves, creating the impression, in
Fang's words, that her garments ``swim between two and three
dimensions.'' Although she is still based in London, Fang will show her
spring 2020 collection in Milan, at the invitation of Sara Maino, Editor
of Vogue Talents, a platform dedicated to emerging designers. Fang was
also shortlisted for the LVMH Prize this year. Her new collection will
be filled with optical illusions and unexpected materials, she says:
``It will be very surreal and ethereal.''

\begin{center}\rule{0.5\linewidth}{\linethickness}\end{center}

Image

Credit...Photo courtesy of Super Yaya. Portrait by~Alice Neale.

\hypertarget{super-yaya}{%
\subsection{Super Yaya}\label{super-yaya}}

\emph{Rym Beydoun, 29, Paris}

For her placement year at Central Saint Martins, the designer Rym
Beydoun decided to go back home to Abidjan, Ivory Coast, and take a step
back from the fashion scene. ``I wanted to be in Africa and reconnect
with people and the culture,'' she explains. While there, she started to
teach herself about the different textile weaving, dying and printing
techniques of ethnic groups across the continent. Later that year, she
interned at Uniwax, a wax print manufacturer in West Africa where she
worked with street tailors to make custom suits, and at the
Abidjan-based clothing label Laurenceairline. After graduating, she then
moved to Beirut to work on her own pieces and in 2017 launched
\href{http://www.super-yaya.com/}{Super Yaya}, a line of colorful
clothes made from fabrics inspired by the different places where she's
lived and traveled. For her next collection, she looked to Indonesia and
the tradition of bati, a technique of wax-resist dyeing on fabrics. ``I
conducted research and gathered old photographs depicting Indonesian
dress and compared them to the ones I had from Abidjan, Bamako and
Dakar,'' she says. ``I go to places for inspiration, specifically
markets where I can exchange and learn from traders. I usually need to
build my own research by taking photographs of the people and
environment.'' Her new garments will feature a lot of patchworks made of
hand-dyed bazin and wax fabric (both colorful African fabrics made of
cotton). These fabrics are manipulated together to create an explosion
of color as well as transparency, creating a sensual yet modest feel.

Advertisement

\protect\hyperlink{after-bottom}{Continue reading the main story}

\hypertarget{site-index}{%
\subsection{Site Index}\label{site-index}}

\hypertarget{site-information-navigation}{%
\subsection{Site Information
Navigation}\label{site-information-navigation}}

\begin{itemize}
\tightlist
\item
  \href{https://help.nytimes3xbfgragh.onion/hc/en-us/articles/115014792127-Copyright-notice}{©~2020~The
  New York Times Company}
\end{itemize}

\begin{itemize}
\tightlist
\item
  \href{https://www.nytco.com/}{NYTCo}
\item
  \href{https://help.nytimes3xbfgragh.onion/hc/en-us/articles/115015385887-Contact-Us}{Contact
  Us}
\item
  \href{https://www.nytco.com/careers/}{Work with us}
\item
  \href{https://nytmediakit.com/}{Advertise}
\item
  \href{http://www.tbrandstudio.com/}{T Brand Studio}
\item
  \href{https://www.nytimes3xbfgragh.onion/privacy/cookie-policy\#how-do-i-manage-trackers}{Your
  Ad Choices}
\item
  \href{https://www.nytimes3xbfgragh.onion/privacy}{Privacy}
\item
  \href{https://help.nytimes3xbfgragh.onion/hc/en-us/articles/115014893428-Terms-of-service}{Terms
  of Service}
\item
  \href{https://help.nytimes3xbfgragh.onion/hc/en-us/articles/115014893968-Terms-of-sale}{Terms
  of Sale}
\item
  \href{https://spiderbites.nytimes3xbfgragh.onion}{Site Map}
\item
  \href{https://help.nytimes3xbfgragh.onion/hc/en-us}{Help}
\item
  \href{https://www.nytimes3xbfgragh.onion/subscription?campaignId=37WXW}{Subscriptions}
\end{itemize}
