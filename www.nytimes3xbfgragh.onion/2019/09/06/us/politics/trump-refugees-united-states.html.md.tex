Sections

SEARCH

\protect\hyperlink{site-content}{Skip to
content}\protect\hyperlink{site-index}{Skip to site index}

\href{https://www.nytimes3xbfgragh.onion/section/politics}{Politics}

\href{https://myaccount.nytimes3xbfgragh.onion/auth/login?response_type=cookie\&client_id=vi}{}

\href{https://www.nytimes3xbfgragh.onion/section/todayspaper}{Today's
Paper}

\href{/section/politics}{Politics}\textbar{}Trump Administration
Considers a Drastic Cut in Refugees Allowed to Enter U.S.

\url{https://nyti.ms/34vqC2b}

\begin{itemize}
\item
\item
\item
\item
\item
\item
\end{itemize}

Advertisement

\protect\hyperlink{after-top}{Continue reading the main story}

Supported by

\protect\hyperlink{after-sponsor}{Continue reading the main story}

\hypertarget{trump-administration-considers-a-drastic-cut-in-refugees-allowed-to-enter-us}{%
\section{Trump Administration Considers a Drastic Cut in Refugees
Allowed to Enter
U.S.}\label{trump-administration-considers-a-drastic-cut-in-refugees-allowed-to-enter-us}}

\includegraphics{https://static01.graylady3jvrrxbe.onion/images/2019/09/06/us/politics/06dc-refugees1/merlin_135861474_5b0862b8-4526-4900-8b4d-cec10405edc0-articleLarge.jpg?quality=75\&auto=webp\&disable=upscale}

By
\href{https://www.nytimes3xbfgragh.onion/by/julie-hirschfeld-davis}{Julie
Hirschfeld Davis} and
\href{https://www.nytimes3xbfgragh.onion/by/michael-d-shear}{Michael D.
Shear}

\begin{itemize}
\item
  Sept. 6, 2019
\item
  \begin{itemize}
  \item
  \item
  \item
  \item
  \item
  \item
  \end{itemize}
\end{itemize}

\href{https://www.nytimes3xbfgragh.onion/es/2019/09/09/espanol/mundo/trump-refugiados-recorte.html}{Leer
en español}

WASHINGTON --- The White House is considering a plan that would keep
most refugees who are fleeing war, persecution and famine out of the
United States, significantly cutting back a decades-old program,
according to current and former administration officials.

One option that top officials are weighing would cut refugee admissions
by half or more, to 10,000 to 15,000 people, but reserve most of those
spots for people from a few countries or from groups with special
status, such as Iraqis and Afghans who work alongside American troops,
diplomats and intelligence operatives abroad. Another option, proposed
by a top administration official, would reduce refugee admissions to
zero, while leaving the president with the ability to admit some in an
emergency.

Both options would all but end the United States' status as a leader in
accepting refugees from around the world.

The issue is expected to come to a head on Tuesday, when White House
officials plan to convene a high-level meeting to discuss the annual
number of refugee admissions for the coming year, as determined by
President Trump.

``At a time when the number of refugees is at the highest level in
recorded history, the United States has abandoned world leadership in
resettling vulnerable people in need of protection,'' said Eric
Schwartz, the president of Refugees International. ``The result is a
world that is less compassionate and less able to deal with future
humanitarian challenges.''

For two years, Stephen Miller, Mr. Trump's top immigration adviser, has
used his considerable influence in the West Wing to reduce the refugee
ceiling to its lowest levels in history, capping the program at 30,000
this year. That is a more than 70 percent cut from its level when
President Barack Obama left office.

The move has been part of Mr. Trump's broader effort to reduce the
number of documented and undocumented immigrants entering the United
States, including numerous restrictions on asylum seekers, who, like
refugees, are fleeing persecution but cross into the United States over
the border with Mexico or Canada.

Now, Mr. Miller and allies from the White House whom he placed at the
Departments of State and Homeland Security are pushing aggressively to
shrink the program even further, according to one senior official
involved in the discussions and several former officials briefed on
them, who spoke on the condition of anonymity to detail the private
deliberations.

White House officials did not respond to a request for comment.

John Zadrozny, a top official at United States Citizenship and
Immigration Services, made the argument for simply lowering the ceiling
to zero, a stance that
\href{https://www.politico.com/story/2019/07/18/trump-officials-refugee-zero-1603503}{was
first reported by Politico}. Others have suggested providing
``carveouts'' for certain countries or populations, such as the Iraqis
and Afghans, whose work on behalf of the American government put both
them and their families at risk, making them eligible for special status
to come to the United States through the refugee program.

Advocates of the nearly 40-year-old refugee program inside and outside
the administration fear that approach would effectively starve the
operation out of existence, making it impossible to resettle even those
narrow populations.

``Pulling the rug out from under refugees and the resettlement program,
as is reported, is unfair, inhumane and strategically flawed for the
United States,'' said Nazanin Ash, the vice president for global policy
and advocacy for the International Rescue Committee. ``This is a program
that is reserved for, and vital to, the most vulnerable refugees.''

Now, officials at the advocacy groups say the fate of the program
increasingly hinges on an unlikely figure:
\href{https://www.nytimes3xbfgragh.onion/2019/07/23/us/politics/mark-esper-secretary-defense.html}{Mark
T. Esper, the secretary of defense}, who they are hoping will save the
program by protesting the cut and recommending that Mr. Trump set a
higher refugee ceiling.

Barely two months into his job as Pentagon chief, Mr. Esper, a former
lobbyist and defense contracting executive, is the newest voice at the
table in the annual debate over how many refugees to admit. But while
Mr. Esper's predecessor, Jim Mattis, had taken up the refugee cause with
an almost missionary zeal, repeatedly declining to embrace large cuts
because of the potential effect he said they would have on American
military interests around the world, Mr. Esper's position on the issue
is unknown.

The senior military leadership at the Defense Department has been
urgently pressing Mr. Esper to follow his predecessor's example and be
an advocate for the refugee program, according to people familiar with
the conversations in the Pentagon.

But current and former senior military officials said the defense
secretary had not disclosed to them whether he would fight for higher
refugee admissions at the White House meeting next week. One former
general described Mr. Esper as in a ``foxhole defilade'' position, a
military term for the infantry's effort to remain shielded or concealed
from enemy fire.

\includegraphics{https://static01.graylady3jvrrxbe.onion/images/2019/09/06/us/politics/06dc-refugees2/merlin_156824220_b0f4dcc4-05c8-477e-8e23-3c948c177a83-articleLarge.jpg?quality=75\&auto=webp\&disable=upscale}

A senior Defense Department official said that Mr. Esper had not decided
what his recommendation would be for the refugee program this year. As a
result, an intense effort is underway by a powerful group of retired
generals and humanitarian aid groups to persuade Mr. Esper to pick up
where Mr. Mattis left off.

In a
\href{https://int.graylady3jvrrxbe.onion/data/documenthelper/1694-generals-letter-refugee/46f652adbbef5a13c2c0/optimized/full.pdf\#page=1}{letter
to Mr. Trump} on Wednesday, some of the nation's most distinguished
retired military officers implored the president to reconsider the cuts,
taking up the national security argument that Mr. Mattis made when he
was at the Pentagon. They called the refugee program a ``critical
lifeline'' to people who help American troops, diplomats and
intelligence officials abroad, and warned that cutting it off risked
greater instability and conflict.

``We urge you to protect this vital program and ensure that the refugee
admissions goal is robust, in line with decades-long precedent, and
commensurate with today's urgent global needs,'' wrote the military
brass, including Admiral William H. McRaven, the former commander of
United States Special Operations; General Martin E. Dempsey, the former
chairman of the Joint Chiefs of Staff; and Lt. General Mark P. Hertling,
the former commanding general of Army forces in Europe.

They said that even the current ceiling of 30,000 was ``leaving
thousands in harm's way.''

Gen. Joseph L. Votel, who retired this year after overseeing the
American military's command that runs operations in the Middle East,
also signed the letter. In an interview, he noted that the flows of
refugees leaving war-torn countries like Syria was one of the driving
forces of instability in the region.

``We don't do anything alone,'' General Votel said of American military
operations overseas, which are regularly helped by Iraqi citizens who
become persecuted refugees. ``This is not just the price we pay but an
obligation.''

Mr. Mattis privately made the same arguments in 2018 and 2019 as he
tried to fight back efforts by Mr. Miller to cut the refugee cap, which
had already been reduced to 50,000 by Mr. Trump's travel ban executive
order.

Joined by Rex W. Tillerson, who was then the secretary of state, and
Nikki R. Haley, the United Nations ambassador at the time, Mr. Mattis
succeeded in keeping the cap at 45,000 for 2018. The next year, Mr.
Miller tried to persuade Mr. Mattis to support a lower number by
promising to ensure the program for the Iraqi and Afghans would not be
affected. But Mr. Mattis refused, pushing for the program to remain at
45,000 refugees. But with Mr. Tillerson gone, Mr. Miller succeeded in
persuading the president to drop the ceiling to 30,000.

In his announcement last year, Secretary of State Mike Pompeo argued
that because of a recent surge of asylum seekers at the southwestern
border, there was less of a need for the United States to accept
refugees from abroad.

``This year's refugee ceiling reflects the substantial increase in the
number of individuals seeking asylum in our country, leading to a
massive backlog of outstanding asylum cases and greater public
expense,'' Mr. Pompeo said at the time.

Now, a year later, Mr. Miller and his allies have repeatedly made that
same argument in urging that the number go even lower.

Barbara Strack, who retired last year as chief of the Refugee Affairs
Division at the federal Citizenship and Immigration Services, said the
United States used to be a model for other countries by accepting
refugees from all over the globe. After America began accepting
Bhutanese refugees from Nepal, she said, other countries followed suit.

``Very often, that leadership matters,'' she said. ``That is something
that is just lost in terms of who the United States is in the world and
how other governments see us.''

The State Department was once the main steward and champion of the
refugee resettlement program, but under Mr. Trump, that has changed, as
the president and Mr. Miller have made clear that they view it with
disdain. The top State Department official now in charge of refugees is
Andrew Veprek, a former aide of Mr. Miller's at the White House Domestic
Policy Council who --- with Mr. Zadrozny --- was a central player in
2017 in efforts to scale back refugee resettlement as much as possible.

That has left the Defense Department as the last agency that could
potentially preserve the refugee program. Its proponents inside the
administration say they feel a sense of desperation waiting to see
whether Mr. Esper will become its advocate.

``The strength of D.O.D.'s argument would really make a difference,''
Ms. Strack said. ``There just needs to be an acknowledgment that this
administration would be walking away from a longstanding, bipartisan
tradition of offering refuge to the most vulnerable people around the
world.''

That sense of foreboding has intensified in recent weeks, as Mr. Miller
has locked down the process for determining the refugee ceiling, to
guard against leaks and cut down on opportunities for officials to
intervene to save it. Normally, cabinet-level officials would be
informed in advance of the options to be discussed at a meeting like the
one scheduled on Tuesday.

This time, officials have been informed that their bosses will learn
what numbers the White House is proposing only when they sit down at the
table and are asked to weigh in.

Advertisement

\protect\hyperlink{after-bottom}{Continue reading the main story}

\hypertarget{site-index}{%
\subsection{Site Index}\label{site-index}}

\hypertarget{site-information-navigation}{%
\subsection{Site Information
Navigation}\label{site-information-navigation}}

\begin{itemize}
\tightlist
\item
  \href{https://help.nytimes3xbfgragh.onion/hc/en-us/articles/115014792127-Copyright-notice}{©~2020~The
  New York Times Company}
\end{itemize}

\begin{itemize}
\tightlist
\item
  \href{https://www.nytco.com/}{NYTCo}
\item
  \href{https://help.nytimes3xbfgragh.onion/hc/en-us/articles/115015385887-Contact-Us}{Contact
  Us}
\item
  \href{https://www.nytco.com/careers/}{Work with us}
\item
  \href{https://nytmediakit.com/}{Advertise}
\item
  \href{http://www.tbrandstudio.com/}{T Brand Studio}
\item
  \href{https://www.nytimes3xbfgragh.onion/privacy/cookie-policy\#how-do-i-manage-trackers}{Your
  Ad Choices}
\item
  \href{https://www.nytimes3xbfgragh.onion/privacy}{Privacy}
\item
  \href{https://help.nytimes3xbfgragh.onion/hc/en-us/articles/115014893428-Terms-of-service}{Terms
  of Service}
\item
  \href{https://help.nytimes3xbfgragh.onion/hc/en-us/articles/115014893968-Terms-of-sale}{Terms
  of Sale}
\item
  \href{https://spiderbites.nytimes3xbfgragh.onion}{Site Map}
\item
  \href{https://help.nytimes3xbfgragh.onion/hc/en-us}{Help}
\item
  \href{https://www.nytimes3xbfgragh.onion/subscription?campaignId=37WXW}{Subscriptions}
\end{itemize}
