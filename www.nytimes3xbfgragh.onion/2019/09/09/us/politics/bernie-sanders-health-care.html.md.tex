Sections

SEARCH

\protect\hyperlink{site-content}{Skip to
content}\protect\hyperlink{site-index}{Skip to site index}

\href{/section/politics}{Politics}\textbar{}Bernie Sanders Went to
Canada, and a Dream of `Medicare for All' Flourished

\url{https://nyti.ms/318WOq7}

\begin{itemize}
\item
\item
\item
\item
\item
\item
\end{itemize}

\begin{itemize}
\item
  \href{https://www.nytimes3xbfgragh.onion/live/2020/09/08/us/trump-vs-biden?action=click\&pgtype=Article\&state=default\&region=TOP_BANNER\&context=storylines_menu}{Election
  Updates}
\item
  \href{https://www.nytimes3xbfgragh.onion/interactive/2020/us/elections/election-states-biden-trump.html?action=click\&pgtype=Article\&state=default\&region=TOP_BANNER\&context=storylines_menu}{Paths
  to 270}
\item
  \href{https://www.nytimes3xbfgragh.onion/interactive/2020/08/31/us/politics/vote-by-mail-deadlines.html?action=click\&pgtype=Article\&state=default\&region=TOP_BANNER\&context=storylines_menu}{Voting
  by Mail}
\item
  \href{https://www.nytimes3xbfgragh.onion/interactive/2019/us/elections/2020-presidential-election-calendar.html?action=click\&pgtype=Article\&state=default\&region=TOP_BANNER\&context=storylines_menu}{Key
  Dates}
\item
  \href{https://www.nytimes3xbfgragh.onion/newsletters/politics?action=click\&pgtype=Article\&state=default\&region=TOP_BANNER\&context=storylines_menu}{Politics
  Newsletter}
\end{itemize}

\includegraphics{https://static01.graylady3jvrrxbe.onion/images/2019/09/06/us/00sandershealthcare-congress/00sandershealthcare-congress-articleLarge.jpg?quality=75\&auto=webp\&disable=upscale}

\hypertarget{bernie-sanders-went-to-canada-and-a-dream-of-medicare-for-all-flourished}{%
\section{Bernie Sanders Went to Canada, and a Dream of `Medicare for
All'
Flourished}\label{bernie-sanders-went-to-canada-and-a-dream-of-medicare-for-all-flourished}}

The Vermont senator has staked his presidential campaign, and much of
his political legacy, on transforming health care in America. His
mother's illness, and a trip he made to study the Canadian system, help
explain why.

Bernie Sanders in his campaign office while running for Congress in
1988. By that time, he was speaking about health care in the kind of
dogmatic terms he uses today.Credit...Steve Liss/The LIFE Images
Collection, via Getty Images

Supported by

\protect\hyperlink{after-sponsor}{Continue reading the main story}

\href{https://www.nytimes3xbfgragh.onion/by/sydney-ember}{\includegraphics{https://static01.graylady3jvrrxbe.onion/images/2018/06/12/multimedia/author-sydney-ember/author-sydney-ember-thumbLarge.png}}

By \href{https://www.nytimes3xbfgragh.onion/by/sydney-ember}{Sydney
Ember}

\begin{itemize}
\item
  Sept. 9, 2019
\item
  \begin{itemize}
  \item
  \item
  \item
  \item
  \item
  \item
  \end{itemize}
\end{itemize}

BURLINGTON, Vt. --- In July 1987, Bernie Sanders, then the mayor of
Burlington, Vt., arrived in Ottawa convinced he was about to see the
future of health care.

Years earlier, as his mother's health declined and his family struggled
to pay for medical treatment, he was spending more time attending to her
than in classes at Brooklyn College, suffering through what his brother
called ``a wrecked year'' leading to her death. Over time, he had come
to believe that the American health care system was flawed and
inherently unfair. In Canada, he wanted to observe firsthand the
government-backed, universal model that he strongly suspected was
better.

Amid tours of community centers and meetings with health care providers,
Mr. Sanders more than liked what he saw.

``He was thrilled,'' recalled Beth Mintz,
\href{https://www.uvm.edu/cas/sociology/profiles/beth-mintz}{a professor
of sociology at the University of Vermont} and a member of a task force
that accompanied Mr. Sanders. ``It gave him much more confidence in the
possibility of the single-payer system as a solution.''

Decades before ``Medicare for all'' would propel his presidential
campaigns, Mr. Sanders's expedition to Ottawa helped forge his
determination to transform the American health care system. His views
burst onto the national political scene during his 2016 presidential
run, when he championed a single-payer program alongside many other
liberal policy ideas. Now, as he seeks the Democratic presidential
nomination for a second time, he has made ``Medicare for all'' the most
important issue of his campaign and set the agenda for the ideological
discussion in the primary.

Health care dominated the first two Democratic debates this summer and
will most likely be prominent again during the third debate on Thursday
in Houston. Other candidates support ``Medicare for all,'' but it is Mr.
Sanders who has become singularly identified with it ---
\href{https://www.nytimes3xbfgragh.onion/2019/07/30/us/politics/bernie-sanders-debate.html}{``I
wrote the damn bill!''} he proclaimed in July's debate.

A review of hundreds of pages of documents from the first chapters of
his political career --- including speeches, correspondences and
newspaper clippings --- as well as interviews with those who have known
him throughout his life, show that while his democratic socialist
worldview underpins his ``Medicare for all'' pitch, he was also guided
by other factors. Chief among them were his mother's illness and death,
which instilled in him the desire to ensure everyone had access to
medical care, and the adjacency of Vermont to Canada, which afforded him
a blueprint for universal health care.

Together, they help explain why he has staked not only his campaign, but
also much of his political legacy, on promoting ``Medicare for all.''

``You can't overstate the impact that Vermont's proximity to Canada had
on Bernie's thinking about how to approach reforming the American health
care system,'' said Jeff Weaver, who has worked with Mr. Sanders since
the 1980s and remains one of his closest advisers. The pull of Canada
remains strong: In July, Mr. Sanders took a bus trip from Detroit
\href{https://www.nytimes3xbfgragh.onion/2019/07/28/us/politics/bernie-sanders-prescription-drug-prices.html}{to
Windsor, Ontario}, with diabetes patients to highlight lower drug prices
in Canada.

\includegraphics{https://static01.graylady3jvrrxbe.onion/images/2019/09/09/us/politics/09sandershealthcare-mfa/merlin_150943836_d722bb5c-483c-448a-ab26-7e8059252e1e-articleLarge.jpg?quality=75\&auto=webp\&disable=upscale}

In an interview on Sunday, Mr. Sanders described how seeing the Canadian
system up close significantly shaped his own views on health care.

``It was kind of mind blowing to realize that the country 50 miles away
from where I live --- that people could go to the doctor whenever they
wanted and not have to take out their wallet,'' he said.

``That was just a profound lesson that I learned,'' he said.

He also criticized the American system as ``barbaric.'' And he vowed ---
as he often does in his stump speeches --- ``to take on the greed and
the corruption of the health care industry.''

Mr. Sanders's health care proposal has attracted legions of supporters
fed up with the rising costs of the current system, and it sets him
apart from more centrist candidates like Joseph R. Biden Jr. But his
uncompromising position also threatens to alienate voters who are
pleased with the Affordable Care Act, or who do not want to give up
their private insurance. His own state of Vermont so far does not have a
single-payer program.

Despite skepticism about his views, however, Mr. Sanders has
consistently resolved to reform the health care system, even before
being elected to public office. In 1972, when he was running for Senate
as a candidate from Vermont's left-wing Liberty Union Party, The
Bennington Banner, a local newspaper, reported him taking an
uncompromising stance: ``There is absolutely no rational reason, in the
United States of America today, we could not have full and total free
medical care for all.''

\hypertarget{the-challenge-of-paying-medical-bills}{%
\subsection{The challenge of paying medical
bills}\label{the-challenge-of-paying-medical-bills}}

The first seeds of Mr. Sanders's concern were sown in Brooklyn.

A high-school track and cross country star with an emerging political
streak, Mr. Sanders had wanted to go to Harvard, friends said. But by
his senior year, his mother, Dorothy Sanders, had become sick, her heart
damaged from having rheumatic fever as a child.

As her health declined, her illness consumed him. He stopped going to
track practice. To be closer to her, he began his freshman year at
Brooklyn College.

Mr. Sanders describes his family as lower middle class. His father, an
immigrant from Poland, was a paint salesman. He has said his parents
frequently argued about money.

Image

Mr. Sanders, second from left, with his parents, Dorothy and Eli, and
brother, Larry.Credit...Bernie Sanders campaign, via Associated Press

When his mother fell ill, his family moved her into a charity hospital
in New Jersey. After a failed heart surgery, she died in March 1960,
when she was in her mid-40s. ``Bernard actually spent much more time
with her than he did in class,'' his brother, Larry, recalled in an
interview in February. ``It was really a kind of wrecked year and a very
unhappy year.''

Then, as now, Mr. Sanders avoided speaking of his mother's death. On
Sunday, he declined to discuss his personal life, but said that his
family had ``struggled economically, and that's it.''

In a
\href{https://video.vermontpbs.org/video/political-profiles-bernie-sanders-om6tgp/}{2006
interview with Vermont PBS}, he offered a glimpse into how her illness
shaped his thinking.

``When you talk about money and family, how do you get the money for the
medical treatment that my mother needed?'' he said. ``I won't go into
the whole long song and dance of it. But trust me, it was something that
I also have not forgotten about --- the right of people to have health
care, which was a little bit difficult in our family situation.''

It would still be some years, though, before health care became his
political hallmark.

Mr. Sanders transferred to the University of Chicago, where he spent
hours in the library reading progressive publications that would
influence his political views. There, he turned his energy toward civil
rights.

``We didn't talk about health care,'' said one of his roommates, Ivan
Light. ``It was not on the political agenda at that time. Civil rights
was on the agenda.''

After moving to Vermont, he became active in politics. A perennial
candidate with the Liberty Union Party in the 1970s, he focused on
issues like the tax structure.

But he also began to study health care seriously. Included in a
collection of papers from those days are pamphlets, articles and other
material related to medical care. One publication he saved from March
1972 was titled, ``Health Rights News;'' its slogan was ``Health care is
a human right.''

\href{https://www.nytimes3xbfgragh.onion/news-event/2020-election}{Election
2020 ›}

\hypertarget{live-updates}{%
\subsection{\texorpdfstring{\href{https://www.nytimes3xbfgragh.onion/live/2020/09/08/us/trump-vs-biden}{Live
Updates}}{Live Updates}}\label{live-updates}}

\href{https://www.nytimes3xbfgragh.onion/live/2020/09/08/us/trump-vs-biden\#pence-and-harris-vied-for-wisconsin-as-trump-vented-from-the-white-house}{}

Sept. 8, 2020, 8:40 a.m. ET

\href{https://www.nytimes3xbfgragh.onion/live/2020/09/08/us/trump-vs-biden\#pence-and-harris-vied-for-wisconsin-as-trump-vented-from-the-white-house}{Pence
and Harris vied for Wisconsin as Trump vented from the White
House.}\href{https://www.nytimes3xbfgragh.onion/live/2020/09/08/us/trump-vs-biden\#with-labor-day-behind-them-the-campaigns-take-flight-literally}{}

Sept. 8, 2020, 8:18 a.m. ET

\href{https://www.nytimes3xbfgragh.onion/live/2020/09/08/us/trump-vs-biden\#with-labor-day-behind-them-the-campaigns-take-flight-literally}{With
Labor Day behind them, the campaigns take flight ---
literally.}\href{https://www.nytimes3xbfgragh.onion/live/2020/09/08/us/trump-vs-biden\#the-trump-campaigns-lavish-spending-eroded-its-head-start-a-times-analysis-shows}{}

Sept. 8, 2020, 8:18 a.m. ET

\href{https://www.nytimes3xbfgragh.onion/live/2020/09/08/us/trump-vs-biden\#the-trump-campaigns-lavish-spending-eroded-its-head-start-a-times-analysis-shows}{The
Trump campaign's lavish spending eroded its head start, a Times analysis
shows.}

That research soon began to take hold: In October 1976, when he was the
Liberty Union candidate for governor, he told The Burlington Free Press
that the delivery of medical care was ``basically a national problem''
and that he supported ``public ownership of the drug companies and
placing doctors on salaries.''

``I believe in socialized medicine,'' he said.

Image

As mayor of Burlington, Mr. Sanders became somewhat fixated on Canada's
health care system.Credit...Irene Fertik/Burlington Free Press-USA Today
Network

John Bloch, who has known Mr. Sanders since the 1970s when they were
active in Vermont politics, said he thought Mr. Sanders's views were
influenced in part by the people he lived near in the rural town of
Stannard, Vt., many of whom were in desperate need of health care.

``He didn't just come to this as Johnny-come-lately,'' Mr. Bloch said.

In the interview Sunday, Mr. Sanders said he was particularly affected
at the time by a young boy who lived across the road whose teeth, he
said, were rotting in his mouth.

Deb Richter, a Vermont physician and longtime advocate for single-payer
health care, who has worked with Mr. Sanders on the issue for 20 years,
said he had always felt that health care was a human right.

``You ask Vermonters, `How long has Bernie been talking about
single-payer health care for all?' and nobody can remember a time he
wasn't talking about it,'' she said.

\hypertarget{turning-his-sights-toward-canada}{%
\subsection{Turning his sights toward
Canada}\label{turning-his-sights-toward-canada}}

After Mr. Sanders was elected mayor of Burlington in 1981, he largely
emphasized local issues, like property taxes and affordable housing. ``I
was the mayor of a city of 40,000 people,'' Mr. Sanders said in the
interview. ``Talking about national health care is not exactly what you
talk to the board of aldermen about.''

By then, he had also become somewhat fixated on Canada. In September
1981, he invited the director of the Quebec Insurance Board to speak
about the province's health insurance plan. Later, he demanded more
accountability from the state's health insurance company and encouraged
a review of hospital budgets.

Image

Mr. Sanders returned to Canada in July with a group seeking to purchase
insulin at a pharmacy in Windsor, Ontario.Credit...Brittany Greeson for
The New York Times

As he pondered higher office, his focus on health care intensified. Even
before he announced his 1986 campaign for governor, he said he planned
to run in part on controlling medical costs, according to an article in
the Vermont newspaper The Times Argus.

He lost that race but gathered valuable information in the process:
During his campaign, his team had polled Vermont residents on issues.
``To my surprise,'' Mr. Sanders said in 1987, ``the issue that
Vermonters felt most strongly about was the rapidly rising costs of
health care.''

That finding served to galvanize his actions on health care. He quickly
set up a task force and charged it with studying how to make the system
more affordable.

Soon Mr. Sanders and the task force --- which included an expert on the
Cuban health care system, professors and a minister --- were traveling
to Ottawa, which had implemented a government-supported, single-payer
system.

Jed Lowy, who went on the trip, recalled touring a public hospital,
visiting a neighborhood community health center and speaking with
physicians.

``It was interesting to see another way that health care was provided,''
Mr. Lowy said.

That trip, and a later one to Montreal, reinforced Mr. Sanders's idea
that Vermont's northern neighbor had effectively put into practice the
kind of accessible, affordable system he had long sought.

At a news conference after the Ottawa visit, the task force suggested
Burlington could model its health care system after Canada's. And in
unequivocal tones, Mr. Sanders said it would be ``absolutely negligent''
not to examine at least some aspects of the Canadian model.

In March 1988, the task force released a report recommending the
creation of a national health care system.

Mr. Sanders's focus on health care policy met some resistance at home
from city employees reluctant to give up benefits they had earned.

Mr. Sanders forcefully rebutted the criticism.

``You may regard this as `propaganda','' he wrote tersely in response to
a letter from an angry constituent in December 1982. ``I expect that you
may not have talked to citizens who are taking their food money to pay
for medical care.''

Image

Mr. Sanders has helped set the agenda for the Democratic primary with
his focus on ``Medicare for all.''Credit...Sarah Rice for The New York
Times

By the time Mr. Sanders was mounting his 1988 congressional run, he was
speaking about health care in the kind of dogmatic terms he uses today,
and he was broadening his vision beyond Vermont. He praised the National
League of Cities for adopting a resolution to establish a national
health system.

Soon after formally announcing his congressional campaign, he set forth
his premier agenda item, one that he had imagined since his mother's
death some three decades earlier.

``I want to make it emphatically clear,'' he said in April 1988, ``that
I will make health care reform a top priority as a United States
congressman from the state of Vermont.''

Alexander Burns contributed reporting. Kitty Bennett and Alain
Delaquérière contributed research.

\hypertarget{our-2020-election-guide}{%
\section{Our 2020 Election Guide}\label{our-2020-election-guide}}

Updated ~Sept. 8, 2020

\begin{center}\rule{0.5\linewidth}{\linethickness}\end{center}

\begin{itemize}
\item ~
  \hypertarget{the-latest}{%
  \subsection{The Latest}\label{the-latest}}

  \begin{itemize}
  \item
    The campaign
    \href{https://www.nytimes3xbfgragh.onion/live/2020/09/08/us/trump-vs-biden?action=click\&pgtype=Article\&state=default\&region=BELOW_MAIN_CONTENT\&context=storylines_guide}{shifts
    to a higher gear this week}, with President Trump set to visit
    Florida and North Carolina today and Joseph R. Biden heading to
    Michigan tomorrow.
  \end{itemize}
\item ~
  \hypertarget{how-to-win-270}{%
  \subsection{How to Win 270}\label{how-to-win-270}}

  \begin{itemize}
  \item
    Joe Biden and Donald Trump need 270 electoral votes to reach the
    White House. Try building
    \href{https://www.nytimes3xbfgragh.onion/interactive/2020/us/elections/election-states-biden-trump.html?action=click\&pgtype=Article\&state=default\&region=BELOW_MAIN_CONTENT\&context=storylines_guide}{your
    own coalition of battleground states}~to see potential outcomes.
  \end{itemize}
\item ~
  \hypertarget{voting-by-mail}{%
  \subsection{Voting by Mail}\label{voting-by-mail}}

  \begin{itemize}
  \item
    Will you have enough time to vote by mail in your state? Yes, but
    it's risky to procrastinate.
    \href{https://www.nytimes3xbfgragh.onion/interactive/2020/08/31/us/politics/vote-by-mail-deadlines.html?action=click\&pgtype=Article\&state=default\&region=BELOW_MAIN_CONTENT\&context=storylines_guide}{Check
    your state's deadline.}
  \item
    \href{https://www.nytimes3xbfgragh.onion/interactive/2020/us/elections/joe-biden.html?action=click\&pgtype=Article\&state=default\&region=BELOW_MAIN_CONTENT\&context=storylines_guide}{}

    \hypertarget{joe-biden}{%
    \section{Joe Biden}\label{joe-biden}}

    \hypertarget{democrat}{%
    \subsection{Democrat}\label{democrat}}

    \href{https://www.nytimes3xbfgragh.onion/interactive/2020/us/elections/donald-trump.html?action=click\&pgtype=Article\&state=default\&region=BELOW_MAIN_CONTENT\&context=storylines_guide}{}

    \hypertarget{donald-trump}{%
    \section{Donald Trump}\label{donald-trump}}

    \hypertarget{republican}{%
    \subsection{Republican}\label{republican}}
  \end{itemize}
\item
  \hypertarget{keep-up-with-our-coverage}{%
  \subsection{Keep Up With Our
  Coverage}\label{keep-up-with-our-coverage}}

  \begin{itemize}
  \item
    Get an
    \href{https://www.nytimes3xbfgragh.onion/newsletters/politics?action=click\&pgtype=Article\&state=default\&region=BELOW_MAIN_CONTENT\&context=storylines_guide}{email}~recapping
    the day's news
  \item
    Download our mobile app on
    \href{https://apps.apple.com/us/app/nytimes/id284862083?ls=1\&mat_click_id=5c79ae7455014fd1bd66b5610c05b8f2-20191112-16948\&referrer=mat_click_id\%3D5c79ae7455014fd1bd66b5610c05b8f2-20191112-16948\%26link_click_id\%3D722930677036718082}{iOS}~and
    \href{http://a.localytics.com/android?id=com.nytimes.android\&referrer=utm_source\%3Dother_nyt_mobile_web\%26utm_medium\%3DWeb\%2520page\%26utm_term\%3DGeneral\%2520Mobile\%2520Page\%26utm_campaign\%3DNYT\%2520Mobile\%2520General\%2520Page}{Android}~and
    turn on Breaking News and Politics alerts
  \end{itemize}
\end{itemize}

Advertisement

\protect\hyperlink{after-bottom}{Continue reading the main story}

\hypertarget{site-index}{%
\subsection{Site Index}\label{site-index}}

\hypertarget{site-information-navigation}{%
\subsection{Site Information
Navigation}\label{site-information-navigation}}

\begin{itemize}
\tightlist
\item
  \href{https://help.nytimes3xbfgragh.onion/hc/en-us/articles/115014792127-Copyright-notice}{©~2020~The
  New York Times Company}
\end{itemize}

\begin{itemize}
\tightlist
\item
  \href{https://www.nytco.com/}{NYTCo}
\item
  \href{https://help.nytimes3xbfgragh.onion/hc/en-us/articles/115015385887-Contact-Us}{Contact
  Us}
\item
  \href{https://www.nytco.com/careers/}{Work with us}
\item
  \href{https://nytmediakit.com/}{Advertise}
\item
  \href{http://www.tbrandstudio.com/}{T Brand Studio}
\item
  \href{https://www.nytimes3xbfgragh.onion/privacy/cookie-policy\#how-do-i-manage-trackers}{Your
  Ad Choices}
\item
  \href{https://www.nytimes3xbfgragh.onion/privacy}{Privacy}
\item
  \href{https://help.nytimes3xbfgragh.onion/hc/en-us/articles/115014893428-Terms-of-service}{Terms
  of Service}
\item
  \href{https://help.nytimes3xbfgragh.onion/hc/en-us/articles/115014893968-Terms-of-sale}{Terms
  of Sale}
\item
  \href{https://spiderbites.nytimes3xbfgragh.onion}{Site Map}
\item
  \href{https://help.nytimes3xbfgragh.onion/hc/en-us}{Help}
\item
  \href{https://www.nytimes3xbfgragh.onion/subscription?campaignId=37WXW}{Subscriptions}
\end{itemize}
