Sections

SEARCH

\protect\hyperlink{site-content}{Skip to
content}\protect\hyperlink{site-index}{Skip to site index}

\href{https://myaccount.nytimes3xbfgragh.onion/auth/login?response_type=cookie\&client_id=vi}{}

\href{https://www.nytimes3xbfgragh.onion/section/todayspaper}{Today's
Paper}

This Artist Lived Like a Victorian During the 1980s. Now, He Reflects on
His Own Past.

\url{https://nyti.ms/2V0vCaW}

\begin{itemize}
\item
\item
\item
\item
\item
\item
\end{itemize}

Advertisement

\protect\hyperlink{after-top}{Continue reading the main story}

Supported by

\protect\hyperlink{after-sponsor}{Continue reading the main story}

Artist's Questionnaire

\hypertarget{this-artist-lived-like-a-victorian-during-the-1980s-now-he-reflects-on-his-own-past}{%
\section{This Artist Lived Like a Victorian During the 1980s. Now, He
Reflects on His Own
Past.}\label{this-artist-lived-like-a-victorian-during-the-1980s-now-he-reflects-on-his-own-past}}

Ahead of the release of his memoir, Peter McGough looks back on 40 years
as half of the time-traveling artist duo McDermott \& McGough.

\includegraphics{https://static01.graylady3jvrrxbe.onion/images/2019/09/26/t-magazine/art/Mcgough-slide-CZ1O/Mcgough-slide-CZ1O-articleLarge.jpg?quality=75\&auto=webp\&disable=upscale}

By Jameson Fitzpatrick

\begin{itemize}
\item
  Published Sept. 16, 2019Updated Sept. 24, 2019
\item
  \begin{itemize}
  \item
  \item
  \item
  \item
  \item
  \item
  \end{itemize}
\end{itemize}

On an intermittently sunny morning in August, the artist Peter McGough
is in his Bushwick, Brooklyn, studio contemplating mortality. ``People
don't want to talk about death, but when I talk about it, I start
living,'' he explains. ``I think, I'm going to die. Wow! What a great
opportunity to enjoy myself.'' Ebullient and dressed in a blue plaid
linen suit --- which, he says, ``certainly gets a lot of attention'' ---
he gives the impression of someone who takes fun seriously.

For 40 years, McGough, 61, has been in an artistic partnership with his
collaborator --- and former longtime boyfriend --- David McDermott, who
has lived in Ireland since the 1990s. As
\href{http://www.mcdermottandmcgough.com/}{McDermott \& McGough}, the
two have shown at museums and galleries around the world and worked in a
number of media, though they are best known for their paintings ---
vibrant celebrations of gay sensibility and desire --- and for their
photographs documenting their experiments in period living. When they
rose to fame in the 1980s, the duo dressed, made work and lived as
though it were the Victorian era, occupying a townhouse on Avenue C
without electricity or modern appliances. They wore top hats and
backdated their paintings to the late 19th and early 20th century. ``But
our work was not just about the past,'' McGough says. ``It was about
time, and how time is fleeting.'' These days, his taste favors the
1930s, as the style of his suit reflects, and he has relaxed the
strictures of his early years with McDermott. McGough has an iPhone, but
limits his use and laments how smartphones increasingly dominate
people's lives: ``Life just disappears --- and now it's gonna disappear
through a phone? No way!''

\emph{{[}}\href{https://www.nytimes3xbfgragh.onion/newsletters/t-list?module=inline}{\emph{Sign
up here}} \emph{for the T List newsletter, a weekly roundup of what T
Magazine editors are noticing and coveting now.{]}}

\includegraphics{https://static01.graylady3jvrrxbe.onion/images/2019/09/26/t-magazine/art/Mcgough-slide-X9HO/Mcgough-slide-X9HO-articleLarge.jpg?quality=75\&auto=webp\&disable=upscale}

Image

Of his new series, McGough says, ``I wanted them to look like twee
little Victorian paintings.'' This is one of several iterations of the
McDermott quote that gives McGough's memoir its name.Credit...Nicholas
Calcott

Image

The painted wood sculpture ``Sacred Love and Pain, 1960'' (2006) from
McDermott \& McGough's 2006 show at Cheim \& Read gallery, ``A True
Story Based on Lies.''Credit...Nicholas Calcott

McGough's musings are occasioned by the release of his memoir,
``\href{https://www.penguinrandomhouse.com/books/561291/ive-seen-the-future-and-im-not-going-by-peter-mcgough/9781524747046/}{I've
Seen the Future and I'm Not Going: The Art Scene and Downtown New York
in the 1980s}\emph{,''} due out from Pantheon Books on Sept. 17. The
book begins, in medias res, in 1998 and takes the form of a
near-deathbed confession: A year after learning he has AIDS, McGough has
dropped to 100 pounds and expects to die. The narrative then pivots back
to his boyhood in a Syracuse suburb, tracing his journey to New York
City and his introduction to the eccentric downtown personality whose
theories of parallel timelines and passion for the past would change his
life. From there, McGough offers an intimate inside look at the New York
art world, his tumultuous relationship with McDermott, the AIDS crisis
and his own illness and treatment.

In person, McGough speaks with the same chatty candor that characterizes
his book. This morning, he's been working in colored pencil, on
reproductions of McDermott \& McGough paintings ``the Europeans bought
and didn't show'' so that, as drawings, they might enjoy a second act.
In the next room stands a row of small bright blue canvases, embellished
with spider webs and flowers, that feature cheeky phrases (many of them
obscene) in a delicate Gothic script in which all the letters have been
painted to resemble twigs. Two in-progress works from the series rest on
easels opposite each other: One displays the title of McGough's memoir
--- a saying of McDermott's that also served as the title of the pair's
2017 retrospective at Dallas Contemporary and has appeared in past
paintings --- while the other bears the title of their 2006 exhibition
at Cheim \& Read gallery, ``A True Story Based on Lies.'' Surrounded by
images illustrating McDermott \& McGough's prolific career together, and
with the occasional interjection from his longhaired Chihuahua, Queenie,
McGough answered T's Artist's Questionnaire.

\textbf{What is your day like? How much do you sleep, and what's your
work schedule?}

I keep the curtains open in my bedroom, so that I wake up early. I go to
bed before 11. And my days, I come here. I'm lucky: I experience joy
every working day. And the joy I experience is incredible. I think
that's what a day is supposed to feel like.

\textbf{How many hours of creative work do you think you do in a day?}

All of them. Being an artist, one lives a creative life. You're looking
at something --- you go see a museum, you see a film, you see the world,
you look out, you see the tree --- that's creative action.

\textbf{What's the first piece of art you ever made?}

A watercolor of a clown in a hoop, when my mother put me in school at
the Everson Museum. And she still has it! It's in this weird plastic
frame.

\textbf{What's the worst studio you ever had?}

All our studios were so wonderful. You could live in a slum, but
McDermott knew how to turn your slum into a beautiful place with
nothing, a can of paint.

\textbf{What's the first work you ever sold? For how much?}

I'd say the first work we ever sold --- that I didn't make any money off
of --- was the painting for McDermott's mother. She supported him, so
she wanted compensation. It was our first painting together.

Image

As McGough recounts in his memoir, an art critic once mocked McDermott
\& McGough for wearing painters smocks and using easels. McGough still
does.Credit...Nicholas Calcott

Image

The artist stores his photographs in the drawers below what he calls his
``curiosity cabinet.''Credit...Nicholas Calcott

\textbf{When you start a new piece, where do you begin? What's the first
step?}

The idea. I have so many ideas I'll never, ever do them all. And when
people say, ``I don't have any ideas,'' I'm like: Are you kidding me?
The world is your ideas!

\textbf{How do you know when you're done?}

Picasso said a painting's never done. You could go on and on and on.

\textbf{How many assistants do you have?}

I have two part-time assistants.

\textbf{When did you first feel comfortable saying you're a professional
artist?}

The difference between the professional and the amateur is the
professional makes money and the amateur doesn't. I was always an
artist. There's art --- the idea of art, the irrational purity of art,
the insanity of it --- and then there's product-making. I wasn't a
6-year-old kid making a drawing, thinking, I'm gonna sell this. Money
comes later.

Image

McDermott \& McGough's collaborative process often began by preparing
canvases. ``We'd stretch the canvases, glue them, gesso them, and then
they'd sit,'' McGough says. ``We'd line them all up and then move from
each painting to the next. You make a lot and then edit down.'' Behind
him, fresh canvases await.Credit...Nicholas Calcott

Image

One of McGough's new drawings of past paintings. ``You could buy this
portrait and then have your face put in --- that's the theory,'' he
explains.Credit...Nicholas Calcott

Image

The artist says he has so many ideas he ``could have five studios.''
Pictured here are just a few of his quickly scribbled
notes.Credit...Nicholas Calcott

\textbf{What do you pay for rent?}

A couple thousand or so. You know, I had grand studios in the '80s. Oh,
it was glorious. I could have bought it for nothing, but --- who cares?
Who am I going to leave all these things to? I don't care if I own
anything. I care about the art --- the paintings living on, being
protected.

\textbf{What's the weirdest object in your studio?}

The two things that don't go along with my aesthetic are
air-conditioners and computers. Necessary evils.

\textbf{How often do you talk to other artists?}

Well, my friends are artists. Artists like to be around artists. And
artists are out of their minds! They are not normal people. To sit there
and look at a glass of water and want to draw that glass of water, you
gotta be bonkers. Because you're thinking abstractly, you're looking at
the world upside down. And McDermott's the Mad Hatter of art.

\textbf{How often do you talk to McDermott?}

He has no phone, he has no computer, he hates cellphones --- he says
it's like talking into a chocolate bar. So I last saw him in London at
the Oscar Wilde Temple {[}an immersive exhibition installed in a former
chapel{]} in the spring. He has a friend who I contact him through. When
he's with his friend, I talk to him.

\textbf{What's the last thing that made you cry?}

Writing the book. That was so sad at times. When I wrote about the
deaths of my friends from AIDS, I cried. I had to stop writing and I
sobbed. And those
\href{https://www.nytimes3xbfgragh.onion/2019/07/02/us/politics/border-center-migrant-detention.html}{children
in}\href{https://www.nytimes3xbfgragh.onion/2019/07/02/us/politics/border-center-migrant-detention.html}{cages}
--- that made me cry, when I saw them, crying out for their mothers and
fathers.

\textbf{What embarrasses you?}

The most embarrassing thing is to be ignorant. Not to understand, and to
be cruel. How can you hate gay people as a group? How can you hate black
people as a group? Latinos? Etc., etc., etc. It's complete ignorance.

\textbf{What are you reading?}

Maupassant. Colette --- ``The Vagabond'' is a dream of a book. It was
based on her life as a performer and it opens with her looking at her
face in a mirror while the dancers are above her onstage. This great
anthology, ``The Stonewall Reader.'' I know a couple of the people in
it, like Martin Boyce and Tommy Lanigan-Schmidt --- and Marsha P.
Johnson I knew through a friend of mine. What's so fantastic about
reading is you get to fall into someone else's view.

\textbf{What's your favorite artwork by someone else?}

I'm sentimentally attached to the Hudson River School, especially Thomas
Cole, Frederic Church. They were capturing creation, and showing the
greatest, greatest artwork: nature. Nothing tops it, in all its
weirdness.

\emph{This interview has been condensed and edited.}

Advertisement

\protect\hyperlink{after-bottom}{Continue reading the main story}

\hypertarget{site-index}{%
\subsection{Site Index}\label{site-index}}

\hypertarget{site-information-navigation}{%
\subsection{Site Information
Navigation}\label{site-information-navigation}}

\begin{itemize}
\tightlist
\item
  \href{https://help.nytimes3xbfgragh.onion/hc/en-us/articles/115014792127-Copyright-notice}{©~2020~The
  New York Times Company}
\end{itemize}

\begin{itemize}
\tightlist
\item
  \href{https://www.nytco.com/}{NYTCo}
\item
  \href{https://help.nytimes3xbfgragh.onion/hc/en-us/articles/115015385887-Contact-Us}{Contact
  Us}
\item
  \href{https://www.nytco.com/careers/}{Work with us}
\item
  \href{https://nytmediakit.com/}{Advertise}
\item
  \href{http://www.tbrandstudio.com/}{T Brand Studio}
\item
  \href{https://www.nytimes3xbfgragh.onion/privacy/cookie-policy\#how-do-i-manage-trackers}{Your
  Ad Choices}
\item
  \href{https://www.nytimes3xbfgragh.onion/privacy}{Privacy}
\item
  \href{https://help.nytimes3xbfgragh.onion/hc/en-us/articles/115014893428-Terms-of-service}{Terms
  of Service}
\item
  \href{https://help.nytimes3xbfgragh.onion/hc/en-us/articles/115014893968-Terms-of-sale}{Terms
  of Sale}
\item
  \href{https://spiderbites.nytimes3xbfgragh.onion}{Site Map}
\item
  \href{https://help.nytimes3xbfgragh.onion/hc/en-us}{Help}
\item
  \href{https://www.nytimes3xbfgragh.onion/subscription?campaignId=37WXW}{Subscriptions}
\end{itemize}
