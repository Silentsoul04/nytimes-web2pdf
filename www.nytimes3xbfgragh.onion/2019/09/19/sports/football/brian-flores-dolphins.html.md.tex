Sections

SEARCH

\protect\hyperlink{site-content}{Skip to
content}\protect\hyperlink{site-index}{Skip to site index}

\href{https://www.nytimes3xbfgragh.onion/section/sports/football}{Pro
Football}

\href{https://myaccount.nytimes3xbfgragh.onion/auth/login?response_type=cookie\&client_id=vi}{}

\href{https://www.nytimes3xbfgragh.onion/section/todayspaper}{Today's
Paper}

\href{/section/sports/football}{Pro Football}\textbar{}The Dolphins Are
Awful. Brian Flores Is Fine.

\url{https://nyti.ms/2Q8QhJp}

\begin{itemize}
\item
\item
\item
\item
\item
\end{itemize}

Advertisement

\protect\hyperlink{after-top}{Continue reading the main story}

Supported by

\protect\hyperlink{after-sponsor}{Continue reading the main story}

\hypertarget{the-dolphins-are-awful-brian-flores-is-fine}{%
\section{The Dolphins Are Awful. Brian Flores Is
Fine.}\label{the-dolphins-are-awful-brian-flores-is-fine}}

Brian Flores, one of the few N.F.L. coaches from New York City, is
trying to revive the sinking Miami Dolphins. It's not going well.

\includegraphics{https://static01.graylady3jvrrxbe.onion/images/2019/09/18/sports/18flores/18flores-articleLarge.jpg?quality=75\&auto=webp\&disable=upscale}

By \href{https://www.nytimes3xbfgragh.onion/by/jere-longman}{Jeré
Longman}

\begin{itemize}
\item
  Published Sept. 19, 2019Updated Sept. 22, 2019
\item
  \begin{itemize}
  \item
  \item
  \item
  \item
  \item
  \end{itemize}
\end{itemize}

DAVIE, Fla. --- Before agreeing to become the head coach of the Miami
Dolphins last winter, Brian Flores consulted his former high school
coach in Brooklyn. The response was, sure you'll have to face New
England twice a year, but even Bill Belichick has to retire at some
point.

``They won't be great forever,'' Dino Mangiero, who coached Flores at
Poly Prep Country Day School, advised him. ``Miami might be a really
good place to land.''

And at some point for Flores, it might be. It is impossible to judge an
N.F.L. head coaching career that consists of two games with a franchise
that has gutted its veteran talent and is rebuilding with fragile youth
and the hoarding of draft picks.

But after losing to New England by 43-0 on Sunday --- the Super Bowl
champion and A.F.C. East rival for whom Flores worked the previous 15
seasons --- the winless Dolphins have been outscored in two games,
102-10. The team might not be merely bad, but historically futile.

To many observers, Miami's front office seems to be tanking to secure
the first overall pick in the 2020 draft. Safety Minkah Fitzpatrick, the
team's No. 11 draft selection in 2018, asked out and was traded to
Pittsburgh on Monday. Every position, Flores said recently, is up for
evaluation.

On Sunday, the team's owner, Stephen Ross,
\href{https://www.miamiherald.com/sports/nfl/miami-dolphins/article235100657.html}{told}\href{https://www.miamiherald.com/sports/nfl/miami-dolphins/article235100657.html}{reporters}
that he remained committed to rebuilding for long-term success. The
Dolphins
have\href{https://www.miamiherald.com/article235137292.html}{five
first-round picks}and four second-round picks over the next two drafts.
Flores, 38, has a five-year contract.

``I think he's the right guy to lead us through these times,'' Chris
Grier, Miami's general manager, said Tuesday.

\includegraphics{https://static01.graylady3jvrrxbe.onion/images/2019/09/18/sports/18flores8/18flores8-articleLarge.jpg?quality=75\&auto=webp\&disable=upscale}

Still, black head coaches tend to have the most precarious hold on jobs
with the most vulnerable teams and the most limited opportunities for a
second chance helming a staff elsewhere. Last season,
\href{https://www.nytimes3xbfgragh.onion/2018/12/31/sports/nfl-black-coaches-fired.html}{five
African-American head coaches were fired.}

So far, Flores has shown a rare ability to remain even-keeled during one
of the rockiest starts to an N.F.L. head coaching career, with no
outward sign of anguish or regret. That stoicism is fitting perhaps for
someone whose life has been built on a refusal to despair. He is the son
of Honduran immigrants, born to a family who lived in the frayed
Brownsville neighborhood of Brooklyn, where
\href{https://www.nytimes3xbfgragh.onion/2019/07/28/nyregion/brooklyn-shooting-brownsville-park.html}{violent
crime has declined but where official neglect, gang feuds and ruthless
poverty have been corrosive.}

It is one thing to lose football games. It is another to grow
accustomed, as Flores has said he did, to helping his mother carry
groceries up 20 flights of stairs when the elevators failed at the
Glenmore Plaza housing project.

``I'm very prepared for difficult moments,'' Flores said Monday. ``I
learned resiliency at a very early age.''

Flores and three of his four brothers have master's degrees. And Brian
appears to have become only the eighth N.F.L. head coach in the modern
era from New York City, --- no one's idea of a football hotbed ---
according to the Elias Sports Bureau, the league's official
statistician.

Flores possesses a singular identity in professional football --- black
and Latino at a time when there are only two other African-American head
coaches (Mike Tomlin of the Pittsburgh Steelers and Anthony Lynn of the
Los Angeles Chargers) in the N.F.L. and one other Hispanic coach (Ron
Rivera of Carolina).

While Belichick, Flores' mentor, is the epitome of a gruff, taciturn
coach who reveals little, Flores possesses a blunt candor. During a
training camp practice, he played
\href{https://ftw.usatoday.com/2019/08/dolphins-coach-brian-flores-kenny-stills-8-jay-z-songs-practice}{eight
consecutive songs} by Brooklyn-born Jay-Z as a rebuttal to then-Dolphins
receiver Kenny Stills, a social activist who
\href{https://www.si.com/nfl/2019/08/19/kenny-stills-criticizes-jay-z-nfl-partnership-dolphins}{criticized
the rapper} as being tone-deaf after
he\href{https://www.nytimes3xbfgragh.onion/2019/08/13/sports/nfl-jay-z.html}{formed
an entertainment and social justice partnership} with the N.F.L.

Image

Flores, left, with his mentor, Bill Belichick, climbed from the
personnel office to New England's top defensive coach.Credit...Steven
Senne/Associated Press

But Flores also gave
\href{https://theshadowleague.com/dolphins-head-coach-brian-flores-i-support-the-player-protests/}{an
impassioned defense}, rarely done by the league's coaches, of the right
of Stills and the ostracized quarterback Colin Kaepernick to kneel
during the national anthem in protest against racial inequality and
police brutality.

``They're bringing attention to my story,'' he said. ``I'm the son of
immigrants. I'm black. I grew up poor. I grew up in New York during the
stop-and-frisk era. I've been stopped because I fit a description
before. So everything these guys protest, I've lived it, I've
experienced it.''

The Flores family story reflects the classic American immigrant
experience. Yet his ascent in America's most popular sport comes as the
Trump Administration
\href{https://www.nytimes3xbfgragh.onion/2019/09/14/world/europe/trump-america-asylum-migration.html}{attempts
to bar most Hondurans} leaving a Central American country overwhelmed by
poverty and violence from seeking asylum in the United States. The
administration has also tried to end the protected status of some 57,000
Honduran immigrants, many of whom have been in the U.S. for more than 20
years.

``What's interesting about Flores is that he's part of multiple
identities,'' said Danielle Clealand, an associate professor of politics
and international relations at Florida International University who
studies Afro-Latinos in Miami.

As an N.F.L. coach in a sport fundamental to American identity, Flores
has challenged the notion in a divisive political climate that
immigrants do not belong in the United States, Clealand said.

``We have to think of the diversity in those communities and how they
have integrated into our society, '' she said.

Asked what he thought of the President Trump's plan to severely restrict
Hondurans from entering the United States, Flores said Thursday through
a team spokesman, ``My journey is the answer to that question.''

With Flores on the sidelines, the Dolphins, are the only N.F.L. team
with a black head coach and a black general manager, Grier. Ross, the
team owner, is the founder of a nonprofit called RISE --- the Ross
Initiative in Sports For Equality --- whose mission is to use sport to
help improve race relations.

But Ross's reputation for progressiveness grew complicated in August
when he held a re-election fund-raiser in the Hamptons for President
Trump. Stills, the receiver,
\href{https://www.nytimes3xbfgragh.onion/2019/08/07/sports/football/kenny-stills-dolphins-trump-ross.html}{criticized
Ross via Twitter}, writing, ``You can't have a nonprofit with this
mission statement then open your doors to Trump.''

When Stills also criticized Jay-Z and Flores responded with his
calculated playlist, the move drew mixed reaction. Mangiero, who coached
Flores in high school, said he chuckled at Flores's feistiness.

Image

Flores plans to continue coaching his way. ``If anybody's got a problem
with that, we've just got a problem.''Credit...Wilfredo Lee/Associated
Press

But the
\href{https://www.miamiherald.com/opinion/editorials/article234224472.html}{Miami
Herald responded harshly on its editorial page}, saying that Flores's
musical choice was insensitive and ``looked like a smirking taunt,
giving the back of his hand to a real-life American plague.''

Flores said he was challenging Stills to perform at a higher level and
to not become distracted by events outside the team. Whatever scrutiny
he received, Flores said at the time, he would continue to coach his own
way. ``If anybody's got a problem with that, we've just got a problem,''
he said. ``We're going to agree to disagree.''

Days later, the Dolphins traded Stills, an extremely popular player, and
Laremy Tunsil, an emerging star at left tackle, to Houston. Asked if the
trade was personal or political, Flores told reporters, ``Not at all.''
The compensation received by the Dolphins, which included two
first-round draft picks and a second-round pick, ``was something we
couldn't turn down,'' Flores said.

He seemed taken aback by the widespread attention paid to the
Stills/Ross/Jay-Z controversy. To Richard Lapchick, the founder and
director the Institute for Diversity and Ethics in Sport at the
University of Central Florida, Flores attempted to perform a delicate
balancing act. In playing the Jay-Z songs, Prof. Lapchick said, Flores
appeared to be ``toeing the company line.''

But Flores's plea for social justice was something few coaches outside
of the N.B.A. ever address, Lapchick said, excepting the
\href{https://www.nytimes3xbfgragh.onion/2017/09/24/sports/nfl-trump-anthem-protests.html}{mass
demonstration of solidarity} that occurred across the N.F.L. on Sept.
24, 2017, after President Trump criticized protests during the national
anthem.

``He realized, `My players do have opinions,' and if he wants to
successfully coach them, he can't be dismissive, as the playlist seemed
to indicate he was,'' Lapchick said.

As a New Yorker, Flores is another sort of rarity in the modern N.F.L.
His only current compatriot is Jacksonville's Doug Marrone. Other New
Yorkers who have coached include the legendary Vince Lombardi and the
less than legendary former Jets and Eagles coach Rich Kotite, with his
career record of 41-57.

Flores's parents --- Raul and Maria --- immigrated from Honduras in the
1970s, speaking no English, seeking a better life, and his father spent
as many as 10 months each year away as a merchant seaman. An uncle,
Darrel Patterson, then a Brooklyn firefighter, became a father figure,
taking the Flores brothers bowling and on trips to a video arcade.
Traveling home one evening when Brian was 12, he said he spotted a Pop
Warner game and asked his uncle if could play.

Image

Flores and Jacksonville's Doug Marrone, left, are a rarity in their
ranks --- New Yorkers.~Credit...Mark Brown/Getty Images

Patterson, 66, and now a fire safety educator, remembers the football
origin story somewhat differently: He visited the family's apartment in
Brownsville on a beautiful fall weekend, only to find the brothers
watching television. When asked why they were indoors, Brian or one of
his siblings, replied, ``Mom doesn't want us outside; she thinks it's
too dangerous.''

Patterson said he took the brothers in his station wagon to a youth
league game in Howard Beach, Queens. Brian ran an impressive 40-yard
dash and was pointed to the team equipment van, where he grabbed a
helmet and shoulder pads. But no one in his family had ever played
football and the shoulder pads felt awkward.

``He had the pads on backward,'' Patterson said. ``We turned them around
and from there he excelled.''

Flores received a scholarship to Poly Prep Country Day, an elite
academic and football school, commuting more than an hour across
Brooklyn by bus and subway. He struck Mangiero, his coach, as Flores
strikes many people --- serious, driven.

At Boston College, Flores played safety and linebacker, but a leg injury
in 2003 ended any slim chance of playing professionally. So Flores
famously wrote to every N.F.L. team looking for a job. He took an
entry-level post in the Patriots' personnel department in 2004. His
duties included getting coffee and picking up dry cleaning. He slept on
an air mattress in a friend's attic for a time. He climbed from scout to
assistant coach, to the de facto defensive coordinator last season as
New England won its sixth Super Bowl.

Miami players describe Flores as New England players did. Quarterback
Ryan Fitzpatrick: ``He's been great being upfront.'' Linebacker Vince
Biegel: ``Steady Eddie.''

Flores often recalls his mother, Maria, who died of breast cancer in
March, shortly after the Dolphins named him head coach, forcing him to
practice his reading when he was little and wanted to cut the lessons
short. She would grab him by the ear and tell him, ``We're going to do
this right now.'' So that is how he plans to rebuild the Dolphins: Move
forward. Persevere.

``You always know that if you put your head down and work hard,'' he
said, ``things normally turn around and get better.''

Alain Delaqueriere contributed research.

Advertisement

\protect\hyperlink{after-bottom}{Continue reading the main story}

\hypertarget{site-index}{%
\subsection{Site Index}\label{site-index}}

\hypertarget{site-information-navigation}{%
\subsection{Site Information
Navigation}\label{site-information-navigation}}

\begin{itemize}
\tightlist
\item
  \href{https://help.nytimes3xbfgragh.onion/hc/en-us/articles/115014792127-Copyright-notice}{©~2020~The
  New York Times Company}
\end{itemize}

\begin{itemize}
\tightlist
\item
  \href{https://www.nytco.com/}{NYTCo}
\item
  \href{https://help.nytimes3xbfgragh.onion/hc/en-us/articles/115015385887-Contact-Us}{Contact
  Us}
\item
  \href{https://www.nytco.com/careers/}{Work with us}
\item
  \href{https://nytmediakit.com/}{Advertise}
\item
  \href{http://www.tbrandstudio.com/}{T Brand Studio}
\item
  \href{https://www.nytimes3xbfgragh.onion/privacy/cookie-policy\#how-do-i-manage-trackers}{Your
  Ad Choices}
\item
  \href{https://www.nytimes3xbfgragh.onion/privacy}{Privacy}
\item
  \href{https://help.nytimes3xbfgragh.onion/hc/en-us/articles/115014893428-Terms-of-service}{Terms
  of Service}
\item
  \href{https://help.nytimes3xbfgragh.onion/hc/en-us/articles/115014893968-Terms-of-sale}{Terms
  of Sale}
\item
  \href{https://spiderbites.nytimes3xbfgragh.onion}{Site Map}
\item
  \href{https://help.nytimes3xbfgragh.onion/hc/en-us}{Help}
\item
  \href{https://www.nytimes3xbfgragh.onion/subscription?campaignId=37WXW}{Subscriptions}
\end{itemize}
