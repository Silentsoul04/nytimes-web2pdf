Sections

SEARCH

\protect\hyperlink{site-content}{Skip to
content}\protect\hyperlink{site-index}{Skip to site index}

\href{https://www.nytimes3xbfgragh.onion/section/business}{Business}

\href{https://myaccount.nytimes3xbfgragh.onion/auth/login?response_type=cookie\&client_id=vi}{}

\href{https://www.nytimes3xbfgragh.onion/section/todayspaper}{Today's
Paper}

\href{/section/business}{Business}\textbar{}Forever 21 Bankruptcy
Signals a Shift in Consumer Tastes

\url{https://nyti.ms/2otepLf}

\begin{itemize}
\item
\item
\item
\item
\item
\item
\end{itemize}

Advertisement

\protect\hyperlink{after-top}{Continue reading the main story}

Supported by

\protect\hyperlink{after-sponsor}{Continue reading the main story}

\hypertarget{forever-21-bankruptcy-signals-a-shift-in-consumer-tastes}{%
\section{Forever 21 Bankruptcy Signals a Shift in Consumer
Tastes}\label{forever-21-bankruptcy-signals-a-shift-in-consumer-tastes}}

\includegraphics{https://static01.graylady3jvrrxbe.onion/images/2019/10/01/business/29forever1-sub/29forever1-sub-articleLarge.jpg?quality=75\&auto=webp\&disable=upscale}

By \href{https://www.nytimes3xbfgragh.onion/by/sapna-maheshwari}{Sapna
Maheshwari}

\begin{itemize}
\item
  Sept. 29, 2019
\item
  \begin{itemize}
  \item
  \item
  \item
  \item
  \item
  \item
  \end{itemize}
\end{itemize}

Forever 21, the California retailer that helped popularize fast fashion
in the United States with its bustling stores and \$5 tops, said on
Sunday night that it would file for bankruptcy, a sign of the eroding
power of shopping malls and the shifting tastes of young consumers.

The private, family-held company capped months of
\href{https://www.nytimes3xbfgragh.onion/2019/09/12/business/forever-21-bankruptcy.html}{speculation}
about its restructuring efforts by saying that it would cease operations
in 40 countries, including Canada and Japan, as part of a Chapter 11
filing. It will close up to 178 stores in the United States and up to
350 over all.

Forever 21 said that it would continue to operate its website and
hundreds of stores in the United States, where it is a major tenant for
mall owners, as well as stores in Mexico and Latin America.

``What we're hoping to do with this process is just to simplify things
so we can get back to doing what we do best,'' Linda Chang, the chain's
executive vice president, said in an interview. Ms. Chang's parents, Do
Won and Jin Sook Chang, who still run the chain, founded Forever 21 in
the 1980s after immigrating to California from South Korea.

The bankruptcy is a blow to a company that prided itself on embodying
the American dream, as well as a reminder of how quickly the retail
landscape is transforming. Forever 21 experienced big success in the
early 2000s with its troves of merchandise that imitated of-the-moment
designer styles at rock-bottom prices. It joined Zara and H\&M in making
fast, disposable fashion widely available to American shoppers,
especially young women, who were exposed to new wares seemingly every
time they entered a store. But the company expanded too aggressively
just as technology was beginning to upend its business.

``We went from seven countries to 47 countries within a
less-than-six-year time frame and with that came a lot of complexity,''
Ms. Chang said. At the same time, she said, ``the retail industry is
obviously changing --- there has been a softening of mall traffic and
sales are shifting more to online.''

Forever 21, which said e-commerce made up 16 percent of its sales, saw
its revenue drop to \$3.3 billion last year, down from \$4.4 billion in
2016. It expects the restructured company to bring in \$2.5 billion in
annual sales. The company employs about 32,800 people, down from 43,000
in 2016.

Mr. Chang, the company's chief executive, said in a
\href{http://edition.cnn.com/TRANSCRIPTS/1209/21/ta.01.html}{2012
interview} that the chain was named Forever 21 because it targeted
20-somethings and because ``old people wanted to be 21 again, and young
people wanted to be 21 forever.'' A large part of the company's base is
minorities, Ms. Chang said, and customer studies have suggested that 40
percent of Forever 21 shoppers are between the ages of 25 and 40. She
said the company would still aim to keep merchandise below \$50.

\includegraphics{https://static01.graylady3jvrrxbe.onion/images/2019/09/29/business/29forever-promo/merlin_161681970_4cc0ad08-4258-4fac-b531-7a3dfae90a09-articleLarge.jpg?quality=75\&auto=webp\&disable=upscale}

Image

In 2010, Michael Bloomberg, then mayor, appeared with Forever 21's
co-founder, Do Won Chang, and his daughter Linda at the company's Times
Square store, which occupies four floors and about 90,000 square
feet.Credit...Jamie Mccarthy/Getty Images for Forever 21

Forever 21's bankruptcy puts a spotlight on the widening chasm between
America's lower-quality malls,
\href{https://www.nytimes3xbfgragh.onion/2015/01/04/business/the-economics-and-nostalgia-of-dead-malls.html}{which
are losing customers and anchor tenants}, and its top shopping centers,
which continue to draw foot traffic.

In the years before and after the recession, Forever 21 opened stores at
a rapid clip --- they also served as the company's main marketing
vehicle --- and bigger was often better. While teenage and 20-something
women were the core customer base, Forever 21 believed that it could
sell to the whole family. It moved into spaces vacated by bankrupt
chains like Mervyn's and Gottschalks and opened huge flagships in major
cities, including a Times Square colossus in 2010 that was
\href{https://www.wsj.com/articles/SB10001424052748704629804575325121769810944}{around}
90,000 square feet and still spans four floors. (The company said it is
in discussions with the landlord of that store about its future.)

The retailer, which did not pay rent on its stores in September in order
to preserve capital, believes it can renegotiate many of the leases on
its United States stores after the filing, said Jon Goulding, an
executive at the consultancy Alvarez \& Marsal who will be Forever 21's
chief restructuring officer during the proceedings. He said liquidations
might begin Oct. 31 for the stores that are closing and that he
anticipated the final count to be below 178.

``A number of these folks don't want boxes back of the size we have with
what's going on in the mall space,'' he said of the chain's landlords.
While the company did not have specific data available, Mr. Goulding
said that underperforming stores were likely located in lower-quality
malls and those that had lost other bankrupt retailers, like Sears.

Forever 21 continued to add more merchandise as it grew and did not seem
to anticipate the rise of digitally-savvy competitors like Asos and
Fashion Nova. It introduced F21 Red in 2014 with a plan to sell Forever
21 ``basics'' like \$1.90 camisoles and \$7.90 jeans, while Riley Rose,
a beauty brand created by Linda Chang and her sister, Esther, opened in
2017. The Riley Rose stores will likely close and become part of
existing Forever 21 locations, while F21 Red will continue to operate
some stand-alone locations.

Ms. Chang said that the company still saw promise in areas like men's
and girls' merchandise, but that it planned to pare down other areas
like home décor, electronics and cosmetics.

Forever 21's struggles have provoked questions around the appeal of fast
fashion more broadly. The industry has faced backlash surrounding the
environmental impact of quickly disposable clothes and concerns about
worker safety
\href{https://www.nytimes3xbfgragh.onion/2015/06/02/world/asia/bangladesh-rana-plaza-murder-charges.html}{in
the wake of} the Rana Plaza building collapse in Bangladesh in 2013 that
killed more than 1,100 garment workers.

Younger shoppers have increasingly turned to consigned goods and brands
that claim sustainability as a value, said Wendy Liebmann, chief
executive of the consultancy WSL Strategic Retail.

Forever 21 ``placed their bets on this notion that fast fashion was
going to continue the same way it had for the last decade or so, and
that they just needed to be in the right locations and create newness
with some of the spinoffs they were playing around with,'' Ms. Liebmann
said. ``The emotional and physical aesthetic of it is not something that
the current shopper wants as much.''

Mark A. Cohen, the director of retail studies at Columbia Business
School, said that he believed fast fashion was as popular as ever,
pointing to the success of Zara, but that Forever 21 had expanded far
too quickly ``without regard to a reasonable outlook.''

``It's a self-inflicted catastrophe,'' he said. ``This is a bonanza for
the competition that Forever 21 has and it's another death knell for the
malls they're in that have already lost a Sears, Macy's, Penney's, and
are struggling with footsteps diminishing every day.''

When asked whether Forever 21's challenges were from declining mall
traffic or a waning interest in fast fashion, Ms. Chang said she thought
it was ``a little of both.''

``You hear a lot of conversations about the rental market or the resale
market and things like that, so I think there are definitely shifts
there that are happening,'' she said. ``It's still a massive market but
we do want to make sure we get ahead of things and that we're not just
staying still while the consumers are changing.''

While trying to quickly mimic designer wares for its customers, Forever
21 has been the subject of multiple copyright and trademark lawsuits
over the years, including a
\href{https://www.nytimes3xbfgragh.onion/2019/09/03/arts/music/ariana-grande-forever-21.html}{recent
complaint} from the singer Ariana Grande that the company used a
``look-alike model'' to make it seem like she was endorsing its goods.
The company said it could not comment on how ongoing litigation may be
handled during its reorganization.

The information set to emerge in a bankruptcy will be of interest for
the retail industry. Forever 21 has maintained a tight-knit corporate
culture even as it spread across shopping malls in the United States and
expanded to other countries. Mr. Chang and his wife rarely give
interviews, though they nod to their faith by having ``John 3:16,'' a
reference to the Bible verse, printed on every one of Forever 21's
bright yellow shopping bags. The elder Changs have long planned to pass
the company on to their two daughters.

Ms. Chang said that she and her sister intended to keep working for the
brand, but could not speak to whether they would still take it over
someday.

``My parents built an amazing brand,'' she said. ``When you think of
fast fashion, there's really only a handful of names that come top of
mind for most people, and to be in that top list is a pretty amazing
feat.''

Advertisement

\protect\hyperlink{after-bottom}{Continue reading the main story}

\hypertarget{site-index}{%
\subsection{Site Index}\label{site-index}}

\hypertarget{site-information-navigation}{%
\subsection{Site Information
Navigation}\label{site-information-navigation}}

\begin{itemize}
\tightlist
\item
  \href{https://help.nytimes3xbfgragh.onion/hc/en-us/articles/115014792127-Copyright-notice}{©~2020~The
  New York Times Company}
\end{itemize}

\begin{itemize}
\tightlist
\item
  \href{https://www.nytco.com/}{NYTCo}
\item
  \href{https://help.nytimes3xbfgragh.onion/hc/en-us/articles/115015385887-Contact-Us}{Contact
  Us}
\item
  \href{https://www.nytco.com/careers/}{Work with us}
\item
  \href{https://nytmediakit.com/}{Advertise}
\item
  \href{http://www.tbrandstudio.com/}{T Brand Studio}
\item
  \href{https://www.nytimes3xbfgragh.onion/privacy/cookie-policy\#how-do-i-manage-trackers}{Your
  Ad Choices}
\item
  \href{https://www.nytimes3xbfgragh.onion/privacy}{Privacy}
\item
  \href{https://help.nytimes3xbfgragh.onion/hc/en-us/articles/115014893428-Terms-of-service}{Terms
  of Service}
\item
  \href{https://help.nytimes3xbfgragh.onion/hc/en-us/articles/115014893968-Terms-of-sale}{Terms
  of Sale}
\item
  \href{https://spiderbites.nytimes3xbfgragh.onion}{Site Map}
\item
  \href{https://help.nytimes3xbfgragh.onion/hc/en-us}{Help}
\item
  \href{https://www.nytimes3xbfgragh.onion/subscription?campaignId=37WXW}{Subscriptions}
\end{itemize}
