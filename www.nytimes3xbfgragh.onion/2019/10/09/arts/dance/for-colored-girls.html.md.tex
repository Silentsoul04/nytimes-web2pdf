Sections

SEARCH

\protect\hyperlink{site-content}{Skip to
content}\protect\hyperlink{site-index}{Skip to site index}

\href{https://www.nytimes3xbfgragh.onion/section/arts/dance}{Dance}

\href{https://myaccount.nytimes3xbfgragh.onion/auth/login?response_type=cookie\&client_id=vi}{}

\href{https://www.nytimes3xbfgragh.onion/section/todayspaper}{Today's
Paper}

\href{/section/arts/dance}{Dance}\textbar{}`For Colored Girls' Is a
Choreopoem. What's a Choreopoem?

\url{https://nyti.ms/31ZwSxz}

\begin{itemize}
\item
\item
\item
\item
\item
\item
\end{itemize}

Advertisement

\protect\hyperlink{after-top}{Continue reading the main story}

Supported by

\protect\hyperlink{after-sponsor}{Continue reading the main story}

\hypertarget{for-colored-girls-is-a-choreopoem-whats-a-choreopoem}{%
\section{`For Colored Girls' Is a Choreopoem. What's a
Choreopoem?}\label{for-colored-girls-is-a-choreopoem-whats-a-choreopoem}}

This landmark work, ``a whole new form of theater,'' originated with a
poet, Ntozake Shange, dancing.

\includegraphics{https://static01.graylady3jvrrxbe.onion/images/2019/10/13/arts/13colored-girls8/merlin_162314124_9975ccf4-d2d2-47cf-8253-175865cdf8a0-articleLarge.jpg?quality=75\&auto=webp\&disable=upscale}

\href{https://www.nytimes3xbfgragh.onion/by/brian-seibert}{\includegraphics{https://static01.graylady3jvrrxbe.onion/images/2019/04/03/multimedia/author-brian-seibert/author-brian-seibert-thumbLarge.png}}

By \href{https://www.nytimes3xbfgragh.onion/by/brian-seibert}{Brian
Seibert}

\begin{itemize}
\item
  Oct. 9, 2019
\item
  \begin{itemize}
  \item
  \item
  \item
  \item
  \item
  \item
  \end{itemize}
\end{itemize}

``For Colored Girls Who Have Considered Suicide/When the Rainbow Is
Enuf'' is not a play. Or that's not what the breakthrough work was
called by its author, Ntozake Shange. Her word was ``choreopoem,'' and
any production of ``For Colored Girls,''
\href{https://www.nytimes3xbfgragh.onion/2019/09/13/theater/for-colored-girls-returns-as-a-celebration-and-as-a-weapon.html?searchResultPosition=1}{like
the major revival now in previews at the Public Theater}, has to figure
out what the term means.

To the revival's director, Leah C. Gardiner, the definition is not that
complicated. ``A choreopoem,'' she said, ``is a combination of all forms
of theater storytelling.'' Which isn't to say that it's simple to
realize. In the carefully chosen words of the revival's in-demand
choreographer, Camille A. Brown, ``combining text with movement is very
complex.''

For Paula Moss, the choreographer of the original 1976 production at the
Public, ``choreopoem'' summons a memory of San Francisco in 1973.
\href{https://www.nytimes3xbfgragh.onion/2018/10/28/obituaries/ntozake-shange-is-dead-at-70.html}{Shange}\href{https://www.nytimes3xbfgragh.onion/2018/10/28/obituaries/ntozake-shange-is-dead-at-70.html}{,
who died last year,} had just met Ms. Moss in a dance class and invited
her to a poetry reading.

``She stood up and started to read a few of her poems,'' Ms. Moss
recalled in a telephone interview from her home in Rome. ``Normally,
poets stand there very stiff, but as she read, she starting dancing.
Everyone was shocked.''

``When she sat down,'' Ms. Moss continued, ``she turned to me and said,
`I can't do this anymore.' I thought she was talking about writing. But
she was talking about standing still.''

From then on, whenever Shange recited her poetry, she moved. Often, Ms.
Moss would be beside her, improvising in dance. ``She told me to listen
to her words and the sound of her voice and interpret them,'' Ms. Moss
recalled. ``She had the flow of a musician, and she could communicate
the most intimate and secret feelings that women were going through. And
movement and words came from the same breath. There could be no
separation.''

\includegraphics{https://static01.graylady3jvrrxbe.onion/images/2019/10/13/arts/13coloredgirls7/merlin_145987200_34b9aa71-1d4f-48ee-ac89-adba1c452d53-articleLarge.jpg?quality=75\&auto=webp\&disable=upscale}

Both in their 20s, they were quickly joined by other young women, and
sometimes by musicians, as they performed the choreopoetry that would
become ``For Colored Girls'' in women's bars, cafes and Fillmore
District clubs. After Shange and Ms. Moss drove to New York City, the
work acquired a director and a more fixed shape. Eventually, at the
Public Theater and then on Broadway, it became a landmark production of
American theater, its script a classic of African-American literature.
But it is not always remembered that it all started with a poet dancing.

Shange, for her part, persistently emphasized the importance of dance in
her life and work, the way it connected her to her body and her
African-American heritage. ``It's how we remember what cannot be said,''
she once wrote.

In her essay-poem ``Why I Had to Dance,'' she recalled how, in the years
she was developing ``For Colored Girls,'' dance became as important to
her as writing. She acquired the habit of going directly from dance
class to a coffee shop with her journal, as ``the endorphin high''
helped her ``get to the truth.'' More broadly, ``movements propelled the
language and/or the language propelled the dance.''

After Shange relocated to New York, her favorite dance classes were
those she took as a scholarship student at Sounds in Motion, the Harlem
studio of \href{http://www.diannemcintyre.com/}{the choreographer Dianne
McInytre}. Over the phone from Cleveland, Ms. McIntyre recalled ``the
beautiful flow'' and flexibility of Shange in class: ``She cut through
space like a gazelle.''

Image

Okwui Okpokwasili, center, with D.Woods, left, and Adrienne C. Moore,
rehearsing the Public Theater's new production.Credit...Josefina Santos
for The New York Times

Image

From left, Celia Chevalier, Ms. Woods, Ms. Moore and Ms.
Okpokwasili.Credit...Josefina Santos for The New York Times

For the rest of Shange's life, Ms. McIntyre often worked with her as a
choreographer and a dancer. What Shange wanted from her, she said, was
not literal translation: ``You respond to the energy of her words, the
meaning. You're going deep to the place where she's coming from.'' There
was ``total collaboration,'' she said, among Shange, a choreographer and
a director. Similarly, it was ``sometimes not easy to tell the
difference between dancer, actor, singer. Everybody does everything.
Everybody is thinking motion.''

\href{http://www.akukadogo.com/}{Aku Kadogo}, another McIntyre student,
joined Shange's crew, improvising in bars then becoming an original cast
member of ``For Colored Girls.'' Now the chair of theater and
performance at Spelman College, she remembers how dance was ``more
public'' back then: ``Dance was around us, integral to our culture, and
Ntozake captured that. She caught it with language.''

References to dance abound in the text of ``For Colored Girls,'' in both
the poetry (``we gotta dance to keep from cryin/we gotta dance to keep
from dyin'') and the stage directions. Yet some audience members, Ms.
Moss recalled, wondered where the choreography was. ``That wasn't the
point,'' she said. ``This was a whole new form of theater. The
choreography was within the poem and the performer. We worked hard to be
free in our bodies as we recited the words.''

It was a form of theater that Shange often had to defend. Calling
herself a poet rather than a playwright, she railed against theater
conventions, arguing that for black artists, plays without dancers and
musicians were a waste of an ``interdisciplinary culture.'' Explaining
her collaborations with choreographers and musicians, she wrote how in
her work, as in traditional theater, ``everyone's efforts are directed
toward the exploration and integrity of the text.'' The difference in
her collaborations: ``The text grows.''

Therein lies the promise and some of the perils facing the ``For Colored
Girls'' revival. Ms. Brown knows something about the borderline between
theater and dance. She is one of very few choreographers to maintain a
successful contemporary dance troupe (hers performs at the Joyce Theater
Nov. 9-10) and a thriving career in theater. (She choreographed
\href{https://www.youtube.com/watch?v=35lO5Bu-UOE}{the current
Metropolitan Opera production of ``Porgy and Bess}'' and
\href{https://www.nytimes3xbfgragh.onion/2019/02/13/arts/dance/speaking-in-dance-choir-boy.html?rref=collection\%2Fcolumn\%2Fspeakingindance\&action=click\&contentCollection=dance\&region=stream\&module=stream_unit\&version=latest\&contentPlacement=24\&pgtype=collection}{her
work on ``Choir Boy''} was nominated for a Tony Award earlier this
year.)

Image

The choreographer and dancer Camille A. Brown: ``What's beautiful about
Ntozake's work is that movement is not seen as a distraction. It's part
of the story.''Credit...Josefina Santos for The New York Times

Interviewed before a recent rehearsal, Ms. Brown said that
choreographing a choreopoem wasn't so different from her usual practice.
Much of her work is rooted in social dance and gesture. The children's
games --- hand clapping, double Dutch --- important to both the original
production and to Ms. Gardiner's vision, are a facet of African-American
culture that Ms. Brown has already explored in her own
``\href{https://www.nytimes3xbfgragh.onion/2015/09/24/arts/dance/review-in-a-new-work-camille-a-brown-plays-with-empowerment.html?searchResultPosition=3}{Black
Girl: Linguistic Play}.''

It's a connection that Shange recognized. Before her death, Shange
interviewed Ms. Brown for a book she was writing about dance. (``I
should be interviewing \emph{you},'' Ms. Brown recalls thinking.)
Planning the ``For Colored Girls'' revival, she told Ms. Gardiner that
Ms. Brown was her choice for choreographer.

Ms. Brown admits some difficulties, like the danger of falling into
clichés of spoken-word performance. ``How can we honor that tradition
but still challenge people to think differently?'' she asked. ``How far
can we stretch the abstraction and leave things a little more
ambiguous?''

But, especially as a choreographer working in theater, Ms. Brown said
that she is grateful for the precedent set by Shange: ``What's beautiful
about Ntozake's work is that movement is not seen as a distraction. It's
part of the story. When I'm given the space to serve the story through
movement, that feels good.''

From her director's perspective, Ms. Gardiner described the making of
this choreopoem as a process of unusually pervasive collaboration. ``You
need everybody in the room at all times,'' she said. ``Even when I've
directed musicals, everyone had a set role,'' she added, whereas in
``For Colored Girls'' rehearsals, the team of black women --- she and
Ms. Brown and the composer, Martha Redbone --- were ``all in the pot,
cooking it up together.''

Working that way is costly, Ms. Gardiner noted. But it was Shange's way,
and Ms. Gardiner sees a connection between Shange's example, a current
\href{https://www.nytimes3xbfgragh.onion/2019/06/11/arts/dance/raja-feather-kelly.html?searchResultPosition=1}{upswing
in dance and music in plays} and
\href{https://www.nytimes3xbfgragh.onion/2019/04/25/theater/black-playwrights-theater.html?searchResultPosition=5}{a
surge in black playwrights} ``working and writing out of our
tradition.'' It is, she said, ``an exciting time in American theater.''

A time that recalls an earlier time. Ms. Moss, thinking of the beginning
in San Francisco, remembered a painful feeling, as she danced to
Shange's poems, of something holding her back, like being at the edge of
a cliff unable to jump. ``Until one day, I jumped,'' she said, ``and I
felt this rush of freedom, and we drove that car across the country,
saying to ourselves, ``We have something.'''

Advertisement

\protect\hyperlink{after-bottom}{Continue reading the main story}

\hypertarget{site-index}{%
\subsection{Site Index}\label{site-index}}

\hypertarget{site-information-navigation}{%
\subsection{Site Information
Navigation}\label{site-information-navigation}}

\begin{itemize}
\tightlist
\item
  \href{https://help.nytimes3xbfgragh.onion/hc/en-us/articles/115014792127-Copyright-notice}{©~2020~The
  New York Times Company}
\end{itemize}

\begin{itemize}
\tightlist
\item
  \href{https://www.nytco.com/}{NYTCo}
\item
  \href{https://help.nytimes3xbfgragh.onion/hc/en-us/articles/115015385887-Contact-Us}{Contact
  Us}
\item
  \href{https://www.nytco.com/careers/}{Work with us}
\item
  \href{https://nytmediakit.com/}{Advertise}
\item
  \href{http://www.tbrandstudio.com/}{T Brand Studio}
\item
  \href{https://www.nytimes3xbfgragh.onion/privacy/cookie-policy\#how-do-i-manage-trackers}{Your
  Ad Choices}
\item
  \href{https://www.nytimes3xbfgragh.onion/privacy}{Privacy}
\item
  \href{https://help.nytimes3xbfgragh.onion/hc/en-us/articles/115014893428-Terms-of-service}{Terms
  of Service}
\item
  \href{https://help.nytimes3xbfgragh.onion/hc/en-us/articles/115014893968-Terms-of-sale}{Terms
  of Sale}
\item
  \href{https://spiderbites.nytimes3xbfgragh.onion}{Site Map}
\item
  \href{https://help.nytimes3xbfgragh.onion/hc/en-us}{Help}
\item
  \href{https://www.nytimes3xbfgragh.onion/subscription?campaignId=37WXW}{Subscriptions}
\end{itemize}
