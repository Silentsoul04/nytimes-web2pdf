Sections

SEARCH

\protect\hyperlink{site-content}{Skip to
content}\protect\hyperlink{site-index}{Skip to site index}

\href{https://www.nytimes3xbfgragh.onion/section/sports}{Sports}

\href{https://myaccount.nytimes3xbfgragh.onion/auth/login?response_type=cookie\&client_id=vi}{}

\href{https://www.nytimes3xbfgragh.onion/section/todayspaper}{Today's
Paper}

\href{/section/sports}{Sports}\textbar{}Eliud Kipchoge Breaks Two-Hour
Marathon Barrier

\url{https://nyti.ms/2MyCx7p}

\begin{itemize}
\item
\item
\item
\item
\item
\end{itemize}

Advertisement

\protect\hyperlink{after-top}{Continue reading the main story}

Supported by

\protect\hyperlink{after-sponsor}{Continue reading the main story}

\hypertarget{eliud-kipchoge-breaks-two-hour-marathon-barrier}{%
\section{Eliud Kipchoge Breaks Two-Hour Marathon
Barrier}\label{eliud-kipchoge-breaks-two-hour-marathon-barrier}}

In Vienna, the Kenyan achieved a milestone once believed to be
unattainable. But his time, 1:59:40, will not be recognized as a world
record.

\includegraphics{https://static01.graylady3jvrrxbe.onion/images/2019/10/20/world/18running-print/12marathon-sub-articleLarge.jpg?quality=75\&auto=webp\&disable=upscale}

By \href{https://www.nytimes3xbfgragh.onion/by/andrew-keh}{Andrew Keh}

\begin{itemize}
\item
  Published Oct. 12, 2019Updated Oct. 14, 2019
\item
  \begin{itemize}
  \item
  \item
  \item
  \item
  \item
  \end{itemize}
\end{itemize}

VIENNA --- On a misty Saturday morning in Vienna, on a course specially
chosen for speed, in an athletic spectacle of historic proportions,
Eliud Kipchoge of Kenya ran 26.2 miles in a once-inconceivable time of
\href{https://twitter.com/EliudKipchoge/status/1182934782684123136}{1
hour 59 minutes 40 seconds.}

In becoming the first person to cover the marathon distance in less than
two hours, Kipchoge, 34, achieved a sports milestone granted almost
mythical status in the running world, breaking through a temporal
barrier that many would have deemed untouchable only a few years ago.

Kipchoge, an eight-time major marathon winner and three-time Olympic
medalist, pounded his chest twice as he crossed the finish line in
Vienna's leafy Prater Park, where the majority of the run had unfolded
on a long straightaway of recently paved road, with roundabouts on
either end.

Cheered on by a thick crowd of spectators, he was lifted into the air by
members of his team, including the 41 professional runners who had acted
as pacesetters during the run.

For Kipchoge, the feat merely burnished his credentials as the world's
greatest marathoner.

``Together, when we run, we can make this world a beautiful world,''
Kipchoge said after finishing.

For all its magnitude, the accomplishment will be regarded largely as a
symbolic one. The eye-popping time, which was 10 seconds quicker than
the 1:59:50 time Kipchoge and his team had set out to achieve, will not
be officially recognized as a world record because it was not run under
open marathon conditions and because it featured a dense rotation of
professional pacesetters.

What the event lacked in officially sanctioned gravitas, though, it
seemed determined to make up for with theater and grandiose
proclamations.

The run, organized by the petrochemical company INEOS, featured a cycle
of hype and commercial buildup more reminiscent of a heavyweight
prizefight than a road race.

Organizers billed the two-hour mark as ``the last barrier of modern
athletics'' and tried to get a hashtag, \#nohumanislimited, trending on
social media.

Kipchoge repeatedly compared a potential sub-two-hour marathon to
humanity's first journey onto the surface of the moon.

``The pressure was very big on my shoulders,'' said Kipchoge, who
revealed he had received a call from President Uhuru Kenyatta of Kenya
the night before the run.

Whatever the scope of the achievement, it required a prodigious amount
of planning.

Seeking the most welcoming environment for Kipchoge to attempt such a
feat, the event's organizers had settled on Vienna: It was not too warm,
not too cold and not at all hilly. The altitude, 540 feet above sea
level, was just right, and it was only one time zone away from
Kipchoge's training camp in Kaptagat, Kenya, where he had worked out for
the past four months under the guidance of his longtime coach, Patrick
Sang.

He had led a monastic existence there, eating, sleeping and exercising
for the sole purpose of running fast. To his normal preparations he
added workouts focused on core strength in order to lessen the strain on
his hamstrings.

On Saturday, Kipchoge showed the subtlest signs of strain on his face in
the first half of the run and fell a couple seconds behind his desired
pace in a few portions. He ran the final stretches of the marathon with
his lips curled into a gentle smile. Afterward, he walked with a barely
perceptible limp.

``There are no guarantees in sports,'' Jim Ratcliffe, the billionaire
founder of INEOS, said to Kipchoge after the finish. ``You could have
had a bad day. But you had a really good day.''

Kipchoge had made an attempt at the two-hour barrier once before. In
2017,
\href{https://www.nytimes3xbfgragh.onion/2017/05/06/sports/eliud-kipchoge-marathon-nike-shoes.html?searchResultPosition=6}{in
a similar event organized by Nike}, he ran a 2:00:25 marathon around an
auto racetrack in Monza, Italy. It was by far the fastest marathon ever
run, but it was not officially recognized as a world record because it
was not run under normal race conditions

Since then, and in officially sanctioned major marathons, Kipchoge
produced the two fastest times in history at the time they were run,
posting
\href{https://www.nytimes3xbfgragh.onion/2018/09/16/sports/eliud-kipchoge-marathon-record.html}{a
world-record time} of 2:01:39 in Berlin in 2018 and 2:02:37 last April
in London.

``Berlin was about running a world record,'' Kipchoge said this past
week. ``Vienna is about running and breaking history, like the first man
on the moon.''

He arrived in Austria on Tuesday, but the exact start date for the
attempt was not finalized until the following day, and the precise start
time was not settled until Friday afternoon.

What materialized on Saturday was perhaps the most finely tuned,
carefully orchestrated marathon-length run in history.

Kipchoge got out of his hotel bed at 4:50 a.m. and had oatmeal for
breakfast.

At 8:15 a.m., after a three-hour wait that he called ``the hardest time
ever in my life,'' he set out from the Reichsbrücke, a picturesque
bridge spanning the Danube, and charged across a stretch of downhill
road that led him into the park. There, he ran around a 9.6-kilometer
flat circuit, more than 90 percent of which unfurled in a straight line.
Portions of the road were painted with lines to highlight the fastest
possible path.

Kipchoge --- who wore a white singlet, white sneakers (Nikes, as of yet
unreleased to the public, built around a carbon-fiber plate) and white
sleeves on his arms --- had immense support. He ran behind an electric
timing car driving 4:34 per mile (with a second car on standby) and with
his flock of rotating pacesetters (35 on the course, six on reserve) who
happened to include some of the best distance runners in the world,
including former world and Olympic gold medalists like Bernard Lagat and
Matthew Centrowitz.

Those pacemakers, wearing black jerseys and stern expressions, formed a
protective, aerodynamic pocket around Kipchoge, five of them running in
front in an open-V formation and two more in the back. They knew exactly
where to run thanks to a pattern of thick, green laser beams projected
onto the street by the timing car. At predetermined times, the seven
pacemakers would make way for another group of seven to slide in and
take over.

A team member on a bicycle periodically pedaled into the pack to deliver
Kipchoge a carbohydrate-heavy cocktail of gels and fluids.

``Looking at the 1:59:40 time, I got so emotional,'' said Lagat, a
two-time Olympic medalist.

Down the final stretch, as it was clear that the milestone was easily in
reach, the pacesetters, timing car and accompanying cyclists all peeled
away, leaving Kipchoge alone to soak in the shouts and applause of the
crowd.

After crossing the finish line, Kipchoge jumped into the arms of his
wife, Grace, and children. Through all his years of competition, all the
victories and medals and records in his career, this was the first time
his family had watched him run in person.

Advertisement

\protect\hyperlink{after-bottom}{Continue reading the main story}

\hypertarget{site-index}{%
\subsection{Site Index}\label{site-index}}

\hypertarget{site-information-navigation}{%
\subsection{Site Information
Navigation}\label{site-information-navigation}}

\begin{itemize}
\tightlist
\item
  \href{https://help.nytimes3xbfgragh.onion/hc/en-us/articles/115014792127-Copyright-notice}{©~2020~The
  New York Times Company}
\end{itemize}

\begin{itemize}
\tightlist
\item
  \href{https://www.nytco.com/}{NYTCo}
\item
  \href{https://help.nytimes3xbfgragh.onion/hc/en-us/articles/115015385887-Contact-Us}{Contact
  Us}
\item
  \href{https://www.nytco.com/careers/}{Work with us}
\item
  \href{https://nytmediakit.com/}{Advertise}
\item
  \href{http://www.tbrandstudio.com/}{T Brand Studio}
\item
  \href{https://www.nytimes3xbfgragh.onion/privacy/cookie-policy\#how-do-i-manage-trackers}{Your
  Ad Choices}
\item
  \href{https://www.nytimes3xbfgragh.onion/privacy}{Privacy}
\item
  \href{https://help.nytimes3xbfgragh.onion/hc/en-us/articles/115014893428-Terms-of-service}{Terms
  of Service}
\item
  \href{https://help.nytimes3xbfgragh.onion/hc/en-us/articles/115014893968-Terms-of-sale}{Terms
  of Sale}
\item
  \href{https://spiderbites.nytimes3xbfgragh.onion}{Site Map}
\item
  \href{https://help.nytimes3xbfgragh.onion/hc/en-us}{Help}
\item
  \href{https://www.nytimes3xbfgragh.onion/subscription?campaignId=37WXW}{Subscriptions}
\end{itemize}
