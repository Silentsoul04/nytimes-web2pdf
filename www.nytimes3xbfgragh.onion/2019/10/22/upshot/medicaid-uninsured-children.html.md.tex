Sections

SEARCH

\protect\hyperlink{site-content}{Skip to
content}\protect\hyperlink{site-index}{Skip to site index}

\href{https://myaccount.nytimes3xbfgragh.onion/auth/login?response_type=cookie\&client_id=vi}{}

\href{https://www.nytimes3xbfgragh.onion/section/todayspaper}{Today's
Paper}

\href{/section/upshot}{The Upshot}\textbar{}Medicaid Covers a Million
Fewer Children. Baby Elijah Was One of Them.

\url{https://nyti.ms/2ocfes0}

\begin{itemize}
\item
\item
\item
\item
\item
\item
\end{itemize}

Advertisement

\protect\hyperlink{after-top}{Continue reading the main story}

Upshot

Supported by

\protect\hyperlink{after-sponsor}{Continue reading the main story}

\hypertarget{medicaid-covers-a-million-fewer-children-baby-elijah-was-one-of-them}{%
\section{Medicaid Covers a Million Fewer Children. Baby Elijah Was One
of
Them.}\label{medicaid-covers-a-million-fewer-children-baby-elijah-was-one-of-them}}

Officials point to rising employment, but the uninsured rate is climbing
as families run afoul of new paperwork and as fear rises among
immigrants.

\includegraphics{https://static01.graylady3jvrrxbe.onion/images/2019/10/13/upshot/00UP-CHILDREN1/00UP-CHILDREN1-articleLarge.jpg?quality=75\&auto=webp\&disable=upscale}

\href{https://www.nytimes3xbfgragh.onion/by/abby-goodnough}{\includegraphics{https://static01.graylady3jvrrxbe.onion/images/2018/06/14/multimedia/author-abby-goodnough/author-abby-goodnough-thumbLarge-v2.png}}\href{https://www.nytimes3xbfgragh.onion/by/margot-sanger-katz}{\includegraphics{https://static01.graylady3jvrrxbe.onion/images/2019/12/13/reader-center/author-margot-sanger-katz/author-margot-sanger-katz-thumbLarge.png}}

By \href{https://www.nytimes3xbfgragh.onion/by/abby-goodnough}{Abby
Goodnough} and
\href{https://www.nytimes3xbfgragh.onion/by/margot-sanger-katz}{Margot
Sanger-Katz}

\begin{itemize}
\item
  Published Oct. 22, 2019Updated Oct. 25, 2019
\item
  \begin{itemize}
  \item
  \item
  \item
  \item
  \item
  \item
  \end{itemize}
\end{itemize}

HOUSTON --- The baby's lips were turning blue from lack of oxygen in the
blood when his mother, Kristin Johnson, rushed him to an emergency room
here last month. Only after he was admitted to intensive care with a
respiratory virus did Ms. Johnson learn that he had been dropped from
Medicaid coverage.

The 9-month-old, Elijah, had joined a growing number of children around
the country with no health insurance, a trend that new Census Bureau
data suggests is most pronounced in Texas and a handful of other states.
Two of Elijah's older siblings lost Medicaid coverage two years ago for
reasons Ms. Johnson never understood, and she got so stymied trying to
prove their eligibility that she gave up.

``I've been on this emotional roller coaster,'' Ms. Johnson, 34, said of
Elijah's loss of coverage, an error that happened apparently because she
didn't respond quickly enough to a letter asking for new proof of
income. ``It's been a very scary month.''

Nationwide, more than a million children disappeared from the rolls of
the two main state-federal health programs for lower-income children,
Medicaid and the Children's Health Insurance Program, between December
2017 and June, the most recent month with complete data.

Some state and federal officials have portrayed the drop --- 3 percent
of enrolled children --- as a success story, arguing that more Americans
are getting coverage from employers in an improving economy. But there
is growing evidence that administrative changes aimed at fighting fraud
and waste --- and rising fears of deportation in immigrant communities
--- are pushing large numbers of children out of the programs, and that
many of them are now going without coverage. The declines are
concentrated in a minority of states; in other places, public coverage
has actually increased.

An analysis of new census data by The New York Times shows the number of
children in the United States without any kind of insurance rose by more
than 400,000 in a two-year period, between 2016 and 2018, after decades
of progress toward universal coverage for children.

Some of the states that saw the largest increases in uninsured children
--- like Tennessee and Texas --- were those that created rules to check
the eligibility of families more frequently or that reset their lists
with new computer systems. In some states with large immigrant
populations like Florida, doctors and patient advocates report growing
concern among parents that signing up their children (who are citizens)
may hurt their own chances of getting a green card or increase their
risk of deportation.

When asked about the drop in Medicaid enrollment, government officials
tend to point first to the improved economy, which has undoubtedly
enabled some families to gain jobs with private insurance.

``Unemployment remains low, wage growth is up, \& we now see fewer
people relying on public assistance,'' Seema Verma, the administrator of
the Centers for Medicare and Medicaid Services, wrote on Twitter in
April. ``That's something to celebrate.''

In many states with large declines, like Tennessee and Missouri,
officials cited the stronger job market.

Kelli Weldon, a spokeswoman for the Texas Health and Human Services
Commission, cited ``record-low unemployment levels'' for its contraction
in Medicaid enrollment.

But the census analysis also shows increases in the rate of uninsured
children in states with enrollment declines, including Tennessee, Texas,
Idaho and Utah.

In Texas, the number of uninsured children rose by around 120,000
between 2016 and 2018. State officials increased paperwork requirements
in 2014 for families covered under both Medicaid and CHIP, which serves
children whose income is slightly higher than Medicaid's.

Instead of checking eligibility once a year, as many states do, Texas
enrolls children for six months and then checks databases for four
consecutive months to ensure family income is still low enough to
qualify. If the databases show the income has gone over the limit,
families are notified by mail and have 10 days to prove otherwise or
lose Medicaid.

\includegraphics{https://static01.graylady3jvrrxbe.onion/images/2019/10/13/upshot/00UP-CHILDREN2/merlin_162577137_1e0702e4-e89e-4e2e-9c18-2ec3eabea176-articleLarge.jpg?quality=75\&auto=webp\&disable=upscale}

A bipartisan bill in the state legislature this spring sought to make
income checks annual again after
\href{https://www.texastribune.org/2019/04/22/texas-takes-thousands-kids-medicaid-every-month-due-red-tape/}{data
suggested several thousand eligible children} were being dropped from
Medicaid each month, but it never got a vote.

Other states have also begun checking family incomes more often, or
removing families who may have moved if mail is returned to the state.

``The way they are doing this seems clearly designed to throw people off
this program,'' said Eliot Fishman, a senior director at the consumer
group Families USA, who was a top Medicaid official in the Obama
administration.

When Tennessee updated its enrollment computer system in 2016, it
generated thousands of errors. Medicaid and CHIP enrollment in the state
has declined by more than 55,000 children since January 2018, according
to the Georgetown Center for Children and Families.

Tennessee's Medicaid director, Gabe Roberts, said that besides the
improved economy, the decline in enrollment was a result of updating the
computer system and clearing up a backlog of old cases.

Gordon Bonnyman, co-founder of the Tennessee Justice Center, which has
been helping families struggling with lost coverage, was skeptical,
saying the state response has revealed ``a remarkable lack of curiosity
about what happened to these kids.''

The census shows that about 25,000 more children there have become
uninsured since 2016.

A\href{https://www.nytimes3xbfgragh.onion/2016/09/27/upshot/its-easy-for-obamacare-critics-to-overlook-the-merits-of-medicaid-expansion.html}{large
body of evidence} shows that Medicaid coverage for children has lasting
effects on their lives, improving
their\href{http://www-personal.umich.edu/~mille/MillerWherry_Prenatal2014.pdf}{health},\href{https://www.nber.org/papers/w20178.pdf}{educational
attainment}
and\href{https://www.nytimes3xbfgragh.onion/2015/01/13/upshot/how-medicaid-for-children-recoups-much-of-its-cost-in-the-long-run.html}{even
adult earnings}. In 2010, the Affordable Care Act made it easier for
states to check whether families qualified for Medicaid without
requiring them to fill out paperwork, a strategy proven to increase
coverage rates. The A.C.A. also made it harder for states to expel poor
families for paperwork errors.

The changes helped the uninsured rate among children reach its
\href{https://ccf.georgetown.edu/wp-content/uploads/2017/09/Uninsured-rate-for-kids-10-17.pdf}{lowest
level ever} in 2016, with fewer than 5 percent without coverage.

Trump administration officials have not explicitly tried to limit
children's Medicaid coverage. But Ms. Verma has repeatedly encouraged
state officials to safeguard ``program integrity,'' by doing more
vigorous checks of enrollees' eligibility. More recently, her office
reviewed the reductions and concluded that problems with state computer
systems may be a factor in some places.

``While the economy is the most consistent driver of enrollment that we
observed, we have found evidence that other more state-specific factors
may be driving individual state experiences,'' an agency spokesman,
Johnathan Monroe, said in an email.

Medicaid and CHIP eligibility does depend on household income, meaning
that, as wages rise, some families may be earning too much to qualify.
Yet the patterns in coverage suggest reasons beyond improved finances.
In Tennessee, for example, the biggest declines in Medicaid enrollment
have come in counties with the highest unemployment rates, a Justice
Center
\href{https://www.tnjustice.org/wp-content/uploads/2019/07/How-Tennessee-Became-an-Outlier-in-the-Rising-Number-of-Uninsured-Children-and-What-Must-Happen-to-Reverse-the-Trend-1.pdf}{analysis}
found.

Image

Maricela, a single mother, fears that re-enrolling her children in
Medicaid will hurt her chances for citizenship.Credit...Ilana
Panich-Linsman for The New York Times

History has shown that when states require more paperwork from Medicaid
beneficiaries, more
\href{https://www.nytimes3xbfgragh.onion/2018/01/18/upshot/medicaid-enrollment-obstacles-kentucky-work-requirement.html}{eligible
people fall through the cracks}. Medicaid beneficiaries tend to move
often; to have unstable hours and incomes; and to have literacy
challenges that can make it hard to submit detailed renewal packages or
verify their incomes frequently.

The specter of a pending
``\href{https://www.nytimes3xbfgragh.onion/2019/08/14/us/immigration-public-charge-welfare.html}{public
charge}'' rule --- which could penalize green card applicants who use
public benefits like Medicaid --- is causing many immigrant patients to
decline enrollment**,** according to a Kaiser Family Foundation
\href{https://www.kff.org/medicaid/press-release/many-community-health-centers-report-that-immigrant-patients-are-declining-to-enroll-in-medicaid-or-renew-their-coverage-amid-concerns-about-changes-to-public-charge-rules/}{survey}
of community health centers. This month a federal judge temporarily
\href{https://www.nytimes3xbfgragh.onion/2019/10/11/us/immigration-public-charge-injunction.html}{blocked}
that rule from taking effect.

Texas leads the nation in the number of uninsured children and adults.
In Houston, Maricela, a single mother, had carefully filled out the
paperwork to re-enroll her younger two children, both citizens, in
Medicaid every year since they were born --- until now. A permanent
resident from El Salvador who earns minimum wage as a hotel maintenance
worker, she was so worried about jeopardizing her status that she
decided to let their coverage lapse in August. Because of the
deportation risk, she agreed to share only her first name.

``My worst fear is that I could end up without my legal status and be
separated from my children,'' Maricela said this month at Epiphany
Community Health Services, a nonprofit group that helps people find
health coverage. ``That would be fatal for me.''

Her older son, 11, has asthma; at his last doctor's visit before his
coverage ended, she pleaded for extra medicine. His main treatment, a
generic version of Singulair, could cost \$150 a month without
insurance. Listening to him cough at night, she finally decided to take
the risk and re-enroll both boys in Medicaid.

``I had to do it,'' she said. ``But I'm afraid.''

Dr. Sogol Pahlavan, a Houston pediatrician, said the rate of her
patients on Medicaid dropped to 70 percent in 2018, from 75 percent a
year earlier. She is part of a practice that has 10,000 patients, and
the number of uninsured has grown commensurately, with families citing
both the impending public charge rule and administrative hurdles.

``It's definitely going to affect the community, because somebody
ultimately has to bear that cost,'' she said. ``These kids are still
here; their chronic disease isn't going away just because they're losing
health coverage.''

Image

Ms. Johnson and family at home.~Credit...Ilana Panich-Linsman for The
New York Times

\begin{center}\rule{0.5\linewidth}{\linethickness}\end{center}

For Ms. Johnson, Elijah's stay at Texas Children's Hospital led to an
appointment with an enrollment counselor who helped her try to figure
out what had happened. Trying to re-enroll her older children earlier
this year, she was asked for proof of income and missed the 10-day
window to provide it; that may be why Texas dropped Elijah from Medicaid
even though he qualified because he was a baby.

All of her children are now re-enrolled. But she has started receiving
thousands of dollars in bills from the baby's hospital stay --- bills
she is counting on Medicaid to cover retroactively. And she is haunted
by what might have happened if the hospital where she took Elijah had
considered the case nonurgent and turned them away.

``I went to the E.R. thinking he had insurance,'' she said. ``If the
receptionist had not seen him turning blue, she might have just said,
`He's not covered, so we can't see him today.' I do think about that.''

\begin{center}\rule{0.5\linewidth}{\linethickness}\end{center}

Abby Goodnough reported from Houston, and Margot Sanger-Katz from
Washington. Josh Katz contributed research from New York.

Advertisement

\protect\hyperlink{after-bottom}{Continue reading the main story}

\hypertarget{site-index}{%
\subsection{Site Index}\label{site-index}}

\hypertarget{site-information-navigation}{%
\subsection{Site Information
Navigation}\label{site-information-navigation}}

\begin{itemize}
\tightlist
\item
  \href{https://help.nytimes3xbfgragh.onion/hc/en-us/articles/115014792127-Copyright-notice}{©~2020~The
  New York Times Company}
\end{itemize}

\begin{itemize}
\tightlist
\item
  \href{https://www.nytco.com/}{NYTCo}
\item
  \href{https://help.nytimes3xbfgragh.onion/hc/en-us/articles/115015385887-Contact-Us}{Contact
  Us}
\item
  \href{https://www.nytco.com/careers/}{Work with us}
\item
  \href{https://nytmediakit.com/}{Advertise}
\item
  \href{http://www.tbrandstudio.com/}{T Brand Studio}
\item
  \href{https://www.nytimes3xbfgragh.onion/privacy/cookie-policy\#how-do-i-manage-trackers}{Your
  Ad Choices}
\item
  \href{https://www.nytimes3xbfgragh.onion/privacy}{Privacy}
\item
  \href{https://help.nytimes3xbfgragh.onion/hc/en-us/articles/115014893428-Terms-of-service}{Terms
  of Service}
\item
  \href{https://help.nytimes3xbfgragh.onion/hc/en-us/articles/115014893968-Terms-of-sale}{Terms
  of Sale}
\item
  \href{https://spiderbites.nytimes3xbfgragh.onion}{Site Map}
\item
  \href{https://help.nytimes3xbfgragh.onion/hc/en-us}{Help}
\item
  \href{https://www.nytimes3xbfgragh.onion/subscription?campaignId=37WXW}{Subscriptions}
\end{itemize}
