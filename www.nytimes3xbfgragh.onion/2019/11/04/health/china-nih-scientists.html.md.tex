Sections

SEARCH

\protect\hyperlink{site-content}{Skip to
content}\protect\hyperlink{site-index}{Skip to site index}

\href{https://www.nytimes3xbfgragh.onion/section/health}{Health}

\href{https://myaccount.nytimes3xbfgragh.onion/auth/login?response_type=cookie\&client_id=vi}{}

\href{https://www.nytimes3xbfgragh.onion/section/todayspaper}{Today's
Paper}

\href{/section/health}{Health}\textbar{}Vast Dragnet Targets Theft of
Biomedical Secrets for China

\url{https://nyti.ms/2pDupLp}

\begin{itemize}
\item
\item
\item
\item
\item
\item
\end{itemize}

Advertisement

\protect\hyperlink{after-top}{Continue reading the main story}

Supported by

\protect\hyperlink{after-sponsor}{Continue reading the main story}

\hypertarget{vast-dragnet-targets-theft-of-biomedical-secrets-for-china}{%
\section{Vast Dragnet Targets Theft of Biomedical Secrets for
China}\label{vast-dragnet-targets-theft-of-biomedical-secrets-for-china}}

Nearly 200 investigations are underway at major academic centers.
Critics fear that researchers of Chinese descent are being unfairly
targeted.

\includegraphics{https://static01.graylady3jvrrxbe.onion/images/2019/09/30/science/00CHINESE-SCIENTISTS1/merlin_30858778_8b2b56b6-135c-4378-949f-99f541521665-articleLarge.jpg?quality=75\&auto=webp\&disable=upscale}

\href{https://www.nytimes3xbfgragh.onion/by/gina-kolata}{\includegraphics{https://static01.graylady3jvrrxbe.onion/images/2018/02/16/multimedia/author-gina-kolata/author-gina-kolata-thumbLarge.jpg}}

By \href{https://www.nytimes3xbfgragh.onion/by/gina-kolata}{Gina Kolata}

\begin{itemize}
\item
  Nov. 4, 2019
\item
  \begin{itemize}
  \item
  \item
  \item
  \item
  \item
  \item
  \end{itemize}
\end{itemize}

\href{https://cn.nytimes3xbfgragh.onion/usa/20191105/china-nih-scientists/}{阅读简体中文版}\href{https://cn.nytimes3xbfgragh.onion/usa/20191105/china-nih-scientists/zh-hant/}{閱讀繁體中文版}

The scientist at M.D. Anderson Cancer Center in Houston was hardly
discreet. ``Here is the bones and meet of what you want,'' he wrote in a
misspelled email to researchers in China.

Attached was a confidential research proposal, according to
administrators at the center. The scientist had access to the document
only because he had been asked to review it for the National Institutes
of Health --- and the center had examined his email because federal
officials had asked them to investigate him.

The N.I.H. and the F.B.I. have begun a vast effort to root out
scientists who they say are stealing biomedical research for other
countries from institutions across the United States. Almost all of the
incidents they uncovered and that are under investigation involve
scientists of Chinese descent, including naturalized American citizens,
allegedly stealing for China.

Seventy-one institutions, including many of the most prestigious medical
schools in the United States, are now investigating 180 individual cases
involving potential theft of intellectual property. The cases began
after the N.I.H., prompted by information provided by the F.B.I., sent
18,000 letters last year urging administrators who oversee government
grants to be vigilant.

So far, the N.I.H. has referred 24 cases in which there may be evidence
of criminal activity to the inspector general's office of the Department
of Health and Human Services, which may turn over the cases for criminal
prosecution. ``It seems to be hitting every discipline in biomedical
research,'' said Dr. Michael Lauer, deputy director for extramural
research at the N.I.H.

The investigations have fanned fears that China is exploiting the
relative openness of the American scientific system to engage in
wholesale economic espionage. At the same time, the scale of the dragnet
has sent a tremor through the ranks of biomedical researchers, some of
whom say ethnic Chinese scientists are being unfairly targeted for
scrutiny as Washington's geopolitical competition with Beijing
intensifies.

``You could take a dart board with medical colleges with significant
research programs and, as far as I can tell, you'd have a 50-50 chance
of hitting a school with an active case,'' said Dr. Ross McKinney Jr.,
chief scientific officer of the Association of American Medical
Colleges.

The alleged theft involves not military secrets, but scientific ideas,
designs, devices, data and methods that may lead to profitable new
treatments or diagnostic tools.

Some researchers under investigation have obtained patents in China on
work funded by the United States government and owned by American
institutions, the N.I.H. said. Others are suspected of setting up labs
in China that secretly duplicated American research, according to
government officials and university administrators.

The N.I.H. has not named most of the scientists under investigation,
citing due process, and neither have most of the institutions involved.
``As with any personnel matter, we typically do not share names or
details of affected individuals,'' said Brette Peyton, a spokeswoman at
M.D. Anderson.

But roughly a dozen scientists are known to have resigned or been fired
from universities and research centers across the United States so far.
Some have declined to discuss the allegations against them; others have
denied any wrongdoing.

In several cases, scientists supported by the N.I.H. or other federal
agencies are accused of accepting funding from the Chinese government in
violation of N.I.H. rules. Some have said that they did not know the
arrangements had to be disclosed or were forbidden.

\includegraphics{https://static01.graylady3jvrrxbe.onion/images/2019/09/30/science/00CHINESE-SCIENTISTS2/merlin_156683919_a5d34f3d-d3a3-4b30-981e-06746801391a-articleLarge.jpg?quality=75\&auto=webp\&disable=upscale}

In August, Feng Tao, 48, a chemist at the University of Kansas known as
Franklin, was indicted on four counts of fraud for
allegedly\href{https://www.insidehighered.com/news/2019/08/23/kansas-professor-indicted-allegedly-failing-disclose-appointment-chinese-university}{failing
to disclose a full-time appointment at a Chinese university}while
receiving federal funds.

His lawyer, Peter R. Zeidenberg, declined to comment on Dr. Tao's case
but suggested that prosecutors were targeting academics nationwide who
had made simple mistakes.

``Professors, they get their summers off,'' he said in an interview.
``Oftentimes they will take appointments in China for the summer. They
don't believe they have to report that.''

``They next thing you know, they are being charged with wire fraud with
20-year penalties,'' he added. ``It's like, are you kidding me?''

The investigations have left Chinese and Chinese-American academics
feeling ``that they will be targeted and that they are at risk,'' said
Frank Wu, a law professor at the University of California Hastings
School of the Law and former president of the Committee of 100, an
organization of prominent Chinese-Americans.

Dr. Wu and other critics said the cases recalled the government's
five-year investigation of Wen Ho Lee, a scientist at the Los Alamos
National Laboratory who was accused in 1999 of stealing nuclear warhead
plans for China and incarcerated for months, only to be freed
\href{https://www.nytimes3xbfgragh.onion/2001/02/04/us/the-making-of-a-suspect-the-case-of-wen-ho-lee.html}{after
the government's case essentially collapsed}. He pleaded guilty to a
single felony count of mishandling secrets.

More recently, the Justice Department has been forced to drop theft
charges against at least four Chinese-American scientists since 2014:
two former Eli Lilly scientists in Indiana, a
\href{https://www.nytimes3xbfgragh.onion/2015/05/10/business/accused-of-spying-for-china-until-she-wasnt.html}{National
Weather Service hydrologist in Ohio} and a
\href{https://www.nytimes3xbfgragh.onion/2015/09/12/us/politics/us-drops-charges-that-professor-shared-technology-with-china.html?module=inline}{professor
at Temple University} in Philadelphia. The Justice Department
\href{https://www.nytimes3xbfgragh.onion/2016/04/27/us/after-missteps-us-tightens-rules-for-national-security-cases.html}{changed
its rules in 2016, giving greater oversight over these national security
cases to prosecutors in Washington}.

But Dr. Lauer and other officials said the investigations into
biomedical research have uncovered clear evidence of wrongdoing. In one
case at M.D. Anderson, a scientist who had packed a suitcase with
computer hard drives containing research data was stopped at the airport
on the way to China, Dr. Lauer and officials at the center said.

Overall, they argued, the cases paint a disturbing picture of economic
espionage in which the Chinese government has been taking advantage of a
biomedical research system in the United States built on trust and the
free exchange of ideas.

``How would you feel if you were a U.S. scientist sending your best idea
to the government in a grant application, and someone ended up doing
your project in China?'' Dr. McKinney asked.

\hypertarget{this-was-something-we-had-never-seen}{%
\subsection{`This was something we had never
seen.'}\label{this-was-something-we-had-never-seen}}

Image

The F.B.I. director Christopher Wray appearing before the Senate
Judiciary Committee on July 23.Credit...Erin Schaff/The New York Times

Concern at the N.I.H. about the theft of biomedical research stretches
back at least to June 2016, when the F.B.I. contacted N.I.H. officials
with unusual questions about the American scientific research system.

How did peer review happen? What sort of controls were in place? ``They
needed to know how our system worked as compared to, say, national
defense,'' Dr. Lauer said.

The F.B.I. declined to discuss ongoing investigations, including why it
initiated so many and how targets were selected. But Christopher Wray,
director of the F.B.I., told the Senate Judiciary Committee in July that
China is using ``nontraditional collectors'' of intelligence, and is
attempting to ``steal their way up the economic ladder at our expense.''

The F.B.I.'s field office for commercial counterespionage, in Houston,
asked administrators from Texas academic and medical centers to attend
classified meetings in the summer of 2018 to discuss evidence of
intellectual property theft. The administrators were given emergency
security clearances and told to sign nondisclosure agreements.

Then, acting on information from the F.B.I. and other sources, the
N.I.H. in late August 2018 began sending letters to medical centers
nationwide asking administrators to investigate individual scientists.

``This was something we had never seen,'' Dr. Lauer said. ``It took us a
while to grasp the seriousness of the problem.''

Some of the first inklings of trouble were discovered by administrators
at M.D. Anderson, a prominent cancer research and treatment center.
Between August 2018 and January 2019, five letters arrived at the center
from the N.I.H. asking administrators to investigate the activities of
five faculty members.

Dr. Peter Pisters, president of the cancer center, said he and his
colleagues reviewed faculty emails, and they turned up disturbing
evidence.

Among the redacted emails provided to The New York Times was one by a
scientist planning to whisk proprietary test materials to colleagues in
China. ``I should be able to bring the whole sets of primers to you (if
I can figure out how to get a dozen tubes of frozen DNA onto an
airplane),'' he wrote.

Image

Li Xiao-Jiang, right, and Li Shihua in Guangzhou, China. They were
employed at Emory University in Atlanta for more than 20
years.Credit...Lam Yik Fei for The New York Times

The redacted M.D. Anderson emails also suggest that a scientist at the
medical center sent data and research to the Chinese government in
exchange for a \$75,000 one-year ``appointment'' under the Thousand
Talents Program, which Beijing established a decade ago to recruit
scientists to Chinese universities.

Researchers are legally obligated to disclose such payments to the
N.I.H. and to their academic institutions, and the scientist had not
done so, according to an internal report on the investigation.

Still another scientist at M.D. Anderson had forwarded a confidential
research proposal to a contact in China, writing, ``Attached please find
an application about mitochondrial DNA mutation in tumor development.
Please keep it to yourself.''

Administrators at M.D. Anderson said three of the scientists had
resigned and one had retired. The fifth case involved a scientist whose
transgressions may not be serious enough to be fired.

Dr. Xifeng Wu, who left M.D. Anderson and is now dean of the School of
Public Health at Zhejiang University in China, declined to comment on
the circumstances of her resignation. ``I would like to focus on my
research,'' she said.

M.D. Anderson is not the only institution wrestling with possible
scientific misconduct.

Last month, two married scientists, Yu Zhou, 49, and Li Chen, 46, who
had worked at Nationwide Children's Hospital in Columbus, Ohio, for a
decade, were indicted on charges that they stole technology developed at
the hospital and used it to apply for Chinese patents and set up biotech
companies in China and the United States.

Dr. Zhou's lawyer, Glenn Seiden, said in an email that the couple did
not commit any crimes, and that Dr. Zhou is a ``trailblazer'' in
scientific research.

In May, two scientists at Emory University in Atlanta, Dr. Li Xiao-Jiang
and Dr. Li Shihua, were fired after administrators discovered that Dr.
Li Xiao-Jiang had received funding from China's Thousand Talents
Program.

The couple had worked there for more than two decades, researching
Huntington's disease. University administrators declined to provide
further information.

``They treated us like criminals,'' Dr. Li Xiao-Jiang said in an
interview near Jinan University in southern China, where he and his wife
now work. He disputed the suggestion that they had failed to report ties
to China.

``Our work is for humanity,'' Dr. Li Shihua added. ``You can't say if I
worked in China, I'm not loyal to the U.S.''

In July, Dr. Kang Zhang, the former chief of eye genetics at the
University of California, San Diego, resigned after local journalists
\href{https://inewsource.org/2019/07/06/thousand-talents-program-china-fbi-kang-zhang-ucsd/}{disclosed
his involvement with a biotech firm in China} that seemed to rely on
research he had performed at the university.

\href{https://inewsource.org/2019/07/06/thousand-talents-program-china-fbi-kang-zhang-ucsd/}{Dr.
Zhang, also a member of
the}\href{https://inewsource.org/2019/07/06/thousand-talents-program-china-fbi-kang-zhang-ucsd/}{Thousand
Talents Program}, did not tell the university about his role. His
lawyer, Leo Cunningham, said that Dr. Zhang's suspension was not related
to his involvement with the Chinese biotech firm or the program, but
instead to his conduct as an investigator in a clinical trial two years
earlier.

Image

Dr. Michael Lauer, deputy director for extramural research at the
National Institutes of Health in Bethesda, Md. ``We know there are
companies formed in China for which we funded the research,'' he
said.Credit...Lexey Swall for The New York Times

What is coming to light, Dr. Lauer said, is ``a tapestry of incidents.''

Start-up companies in China, federal officials say, were founded on
scientific and medical technology that the N.I.H. developed with
taxpayer money. ``We know there are companies formed in China for which
we funded the research,'' Dr. Lauer said.

Some scientists of Chinese descent also secretly received patents in
China for research conducted in the United States, according to Dr.
Lauer, and some researchers in the Thousand Talents Program signed
contracts that require them to provide the Chinese government with
confidential results obtained in the United States or other lab
discoveries.

``If the N.I.H. funded it, it should be available to U.S. taxpayers,''
said Dr. McKinney, of the Association of American Medical Colleges.
``But if a project is also funded in China, it is moving intellectual
property to China.''

\hypertarget{espionage-or-racism}{%
\subsection{Espionage or racism?}\label{espionage-or-racism}}

Image

The National Institutes of Health in Bethesda, Md. Officials claim they
are not targeting ethnically Chinese scientists.~Credit...J. Scott
Applewhite/Associated Press

Federal and academic officials stress that they are not targeting
Chinese researchers on the basis of their ethnicity. But the F.B.I.'s
silence regarding how so many investigations began has exacerbated
concern that the government's efforts to uncover economic espionage may
tar all Chinese and Chinese-American scientists --- and make it more
difficult to recruit Chinese students and scholars.

``We can't tell who is guilty or innocent, but look at the actual effect
on people of Chinese descent,'' said Mr. Wu, the law professor. ``People
are living in fear. It is a question of impact rather than intent.''

With the Trump administration taking a harder line against China,
including imposing tariffs intended to punish violations of intellectual
property rights, Mr. Wu sees a sharp reversal in attitudes about China
and the Chinese.

``I am getting calls and emails constantly now from ethnic Chinese ---
even those who are U.S. citizens --- who feel threatened,'' he said. But
few are willing to step forward with allegations of discrimination, he
added.

To Dr. Lauer, the charges of racism are unfounded. ``Not all the foreign
influence cases involve China,'' he said. ``But the vast majority do.''

The real question, he added, is how to preserve the open exchange of
scientific ideas in the face of growing security concerns. At M.D.
Anderson, administrators are tightening controls to make data less
freely available.

People can no longer use personal laptops on the wireless network. The
center has barred the use of flash drives and disabled USB ports. And
all of its employees' computers can now be monitored remotely.

The N.I.H. is clamping down, too. It recommends that reviewers of grant
applications have limited ability to download or print them. Those
traveling to certain regions should use loaner computers, it says, and
academic institutions should be alert to frequent foreign travel by
scientists, or frequent publishing with colleagues outside the United
States.

The National Science Foundation has commissioned an independent
scientific advisory group to recommend ways of balancing openness and
security, and warned employees that they are prohibited from
participating in programs like China's Thousand Talents Program.

The F.B.I. has given research institutions tools to scan emails for
keywords in Mandarin that might tip off administrators to breaches,
according to Dr. McKinney.

``The effects this will have on long-term, trusting relationships are
hard for us to face,'' he said. ``We just are not used to systematic
cheating.''

Javier Hernandez contributed reporting from Beijing.

Advertisement

\protect\hyperlink{after-bottom}{Continue reading the main story}

\hypertarget{site-index}{%
\subsection{Site Index}\label{site-index}}

\hypertarget{site-information-navigation}{%
\subsection{Site Information
Navigation}\label{site-information-navigation}}

\begin{itemize}
\tightlist
\item
  \href{https://help.nytimes3xbfgragh.onion/hc/en-us/articles/115014792127-Copyright-notice}{©~2020~The
  New York Times Company}
\end{itemize}

\begin{itemize}
\tightlist
\item
  \href{https://www.nytco.com/}{NYTCo}
\item
  \href{https://help.nytimes3xbfgragh.onion/hc/en-us/articles/115015385887-Contact-Us}{Contact
  Us}
\item
  \href{https://www.nytco.com/careers/}{Work with us}
\item
  \href{https://nytmediakit.com/}{Advertise}
\item
  \href{http://www.tbrandstudio.com/}{T Brand Studio}
\item
  \href{https://www.nytimes3xbfgragh.onion/privacy/cookie-policy\#how-do-i-manage-trackers}{Your
  Ad Choices}
\item
  \href{https://www.nytimes3xbfgragh.onion/privacy}{Privacy}
\item
  \href{https://help.nytimes3xbfgragh.onion/hc/en-us/articles/115014893428-Terms-of-service}{Terms
  of Service}
\item
  \href{https://help.nytimes3xbfgragh.onion/hc/en-us/articles/115014893968-Terms-of-sale}{Terms
  of Sale}
\item
  \href{https://spiderbites.nytimes3xbfgragh.onion}{Site Map}
\item
  \href{https://help.nytimes3xbfgragh.onion/hc/en-us}{Help}
\item
  \href{https://www.nytimes3xbfgragh.onion/subscription?campaignId=37WXW}{Subscriptions}
\end{itemize}
