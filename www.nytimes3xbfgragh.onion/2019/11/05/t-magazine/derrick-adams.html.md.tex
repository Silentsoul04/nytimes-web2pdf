Sections

SEARCH

\protect\hyperlink{site-content}{Skip to
content}\protect\hyperlink{site-index}{Skip to site index}

\href{https://myaccount.nytimes3xbfgragh.onion/auth/login?response_type=cookie\&client_id=vi}{}

\href{https://www.nytimes3xbfgragh.onion/section/todayspaper}{Today's
Paper}

Derrick Adams Can Make Art Out of Anything (Even Rubber Gloves)

\url{https://nyti.ms/33jQuNJ}

\begin{itemize}
\item
\item
\item
\item
\item
\end{itemize}

Advertisement

\protect\hyperlink{after-top}{Continue reading the main story}

Supported by

\protect\hyperlink{after-sponsor}{Continue reading the main story}

\hypertarget{derrick-adams-can-make-art-out-of-anything-even-rubber-gloves}{%
\section{Derrick Adams Can Make Art Out of Anything (Even Rubber
Gloves)}\label{derrick-adams-can-make-art-out-of-anything-even-rubber-gloves}}

The artist takes our ``Make T Something'' challenge and creates a
sculpture in under one hour using some unexpected materials.

\includegraphics{https://static01.graylady3jvrrxbe.onion/images/2019/10/16/t-magazine/16tmag-derrick-adams-02/16tmag-derrick-adams-02-videoSixteenByNineJumbo1600.png}

By Kat Herriman

\begin{itemize}
\item
  Nov. 5, 2019
\item
  \begin{itemize}
  \item
  \item
  \item
  \item
  \item
  \end{itemize}
\end{itemize}

Growing up in the '90s, as television's cultural monopoly gave way to
that of the early internet, the Baltimore-born artist Derrick Adams
processed the era's evolving visual language in his sketchbook. What he
observed --- the predominant whiteness of sitcoms and talk shows, the
exoticized depiction of people of color --- left a permanent impression,
one that still inspires him to imagine his own alternative cultural
narratives. ``My images come from things I want to see in the world,
portraits of blackness that have not yet had their moment,'' he says of
his work. ``What is the point of perpetuating, repeating, what is
already out there?''

Adams's large-scale installations use painting, sculpture, performance
and photography to tell stories about the intricacies and politics of
domestic life and how it has been portrayed throughout history. For
instance, for ``Sanctuary,'' his 2018 show at the Museum of Arts and
Design in New York, the artist erected a miniature wooden highway in a
room whose walls were hung with collages of brick walls and abstract
shapes that evoked quickly glimpsed landscapes. The installation was the
result of his intensive research into ``The Negro Motorist Green Book,''
a guide published between 1936 and 1967 that advised black travelers on
how to navigate through the Jim Crow South, and, like much of his work,
it offered new ways to understand seemingly familiar environments: the
home, the city, the screen, the South.

\emph{{[}}\href{https://www.nytimes3xbfgragh.onion/newsletters/t-list?module=inline}{\emph{Sign
up here}} \emph{for the T List newsletter, a weekly roundup of what T
Magazine editors are noticing and coveting now.{]}}

To coincide with his current exhibition,
``\href{http://www.bmoreart.com/events/derrick-adams-where-im-from}{Where
I'm From,}'' at Baltimore City Hall Gallery, we invited Adams to take
part in our ``Make T Something'' series, in which a person must create
something in under one hour using only a copy of The New York Times,
some basic craft supplies and one wild-card item of their choosing (for
Adams, a box of black disposable gloves). Presented with two different
recent editions of The Times, Adams selected the paper with a front-page
story about the aftermath of Hurricane Dorian in the Bahamas. ``Usually
when I'm working in my studio, the art comes from my drawings and
prolonged research, but for this project it was about a gut reaction to
something immediate,'' he says. The resulting mixed-media sculpture,
``If Only Pompey Were Here To Save Us,'' is a tangle of wire and
inflated black hands, surrounding an image of a tragedy, that hints at
the revisionist histories that Adams's work so often tackles.

``\href{http://www.bmoreart.com/events/derrick-adams-where-im-from}{Where
I'm From}'' is on view at Baltimore City Hall Gallery, 100 N. Holliday
Street, Baltimore, through November 22.

Advertisement

\protect\hyperlink{after-bottom}{Continue reading the main story}

\hypertarget{site-index}{%
\subsection{Site Index}\label{site-index}}

\hypertarget{site-information-navigation}{%
\subsection{Site Information
Navigation}\label{site-information-navigation}}

\begin{itemize}
\tightlist
\item
  \href{https://help.nytimes3xbfgragh.onion/hc/en-us/articles/115014792127-Copyright-notice}{©~2020~The
  New York Times Company}
\end{itemize}

\begin{itemize}
\tightlist
\item
  \href{https://www.nytco.com/}{NYTCo}
\item
  \href{https://help.nytimes3xbfgragh.onion/hc/en-us/articles/115015385887-Contact-Us}{Contact
  Us}
\item
  \href{https://www.nytco.com/careers/}{Work with us}
\item
  \href{https://nytmediakit.com/}{Advertise}
\item
  \href{http://www.tbrandstudio.com/}{T Brand Studio}
\item
  \href{https://www.nytimes3xbfgragh.onion/privacy/cookie-policy\#how-do-i-manage-trackers}{Your
  Ad Choices}
\item
  \href{https://www.nytimes3xbfgragh.onion/privacy}{Privacy}
\item
  \href{https://help.nytimes3xbfgragh.onion/hc/en-us/articles/115014893428-Terms-of-service}{Terms
  of Service}
\item
  \href{https://help.nytimes3xbfgragh.onion/hc/en-us/articles/115014893968-Terms-of-sale}{Terms
  of Sale}
\item
  \href{https://spiderbites.nytimes3xbfgragh.onion}{Site Map}
\item
  \href{https://help.nytimes3xbfgragh.onion/hc/en-us}{Help}
\item
  \href{https://www.nytimes3xbfgragh.onion/subscription?campaignId=37WXW}{Subscriptions}
\end{itemize}
