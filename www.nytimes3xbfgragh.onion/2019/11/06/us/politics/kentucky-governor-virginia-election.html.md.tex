Sections

SEARCH

\protect\hyperlink{site-content}{Skip to
content}\protect\hyperlink{site-index}{Skip to site index}

\href{https://www.nytimes3xbfgragh.onion/section/politics}{Politics}

\href{https://myaccount.nytimes3xbfgragh.onion/auth/login?response_type=cookie\&client_id=vi}{}

\href{https://www.nytimes3xbfgragh.onion/section/todayspaper}{Today's
Paper}

\href{/section/politics}{Politics}\textbar{}The G.O.P.'s Election Day
Problem in the Suburbs Is Getting Worse

\url{https://nyti.ms/2JWv55I}

\begin{itemize}
\item
\item
\item
\item
\item
\item
\end{itemize}

Advertisement

\protect\hyperlink{after-top}{Continue reading the main story}

Supported by

\protect\hyperlink{after-sponsor}{Continue reading the main story}

\hypertarget{the-gops-election-day-problem-in-the-suburbs-is-getting-worse}{%
\section{The G.O.P.'s Election Day Problem in the Suburbs Is Getting
Worse}\label{the-gops-election-day-problem-in-the-suburbs-is-getting-worse}}

In Virginia, Pennsylvania and Kentucky, many voters in historically
Republican suburbs supported Democratic candidates, in part because of
antipathy toward President Trump.

\includegraphics{https://static01.graylady3jvrrxbe.onion/images/2019/11/06/us/politics/06elections-voters/merlin_163903995_55417cf4-0931-4d64-ae23-465944ce67ba-articleLarge.jpg?quality=75\&auto=webp\&disable=upscale}

By \href{https://www.nytimes3xbfgragh.onion/by/trip-gabriel}{Trip
Gabriel},
\href{https://www.nytimes3xbfgragh.onion/by/jonathan-martin}{Jonathan
Martin} and
\href{https://www.nytimes3xbfgragh.onion/by/alexander-burns}{Alexander
Burns}

\begin{itemize}
\item
  Nov. 6, 2019
\item
  \begin{itemize}
  \item
  \item
  \item
  \item
  \item
  \item
  \end{itemize}
\end{itemize}

RICHMOND, Va. --- For the second Election Day in as many years, suburban
voters demonstrated enormous political power in electing or aiding
Democratic candidates in historically Republican areas, underscoring the
drift of many moderate voters from the G.O.P. in the era of President
Trump.

In the Virginia suburbs of Norfolk, Richmond, and Washington, D.C.,
Democratic candidates flipped six legislative seats from Republican
control on Tuesday --- crucial gains that helped Democrats take control
of both chambers of the legislature and put state government under
one-party control for the first time in a generation. Many Democrats ran
on gun control issues and other local concerns, but also excoriated Mr.
Trump's conduct in office.

In suburbs around Philadelphia, which were battlegrounds in the 2016
election and will be again in 2020, Democrats notched historic wins.
They defeated the last Republicans on the five-seat Delaware County
Council, in a suburb that kept electing Republicans to local offices
while rejecting many Republican statewide candidates, and they took
control of the board of commissioners in Bucks County for the first time
since the 1980s.

And in the Kentucky governor's race on Tuesday, Democrats carried the
home counties of Louisville and Lexington by nearly triple the margin
they ran up in the 2015 governor's race. Four years ago, too, the
Republican candidate for governor, Matt Bevin, carried the conservative
Kentucky suburbs south of Cincinnati; this time around, two of those
three counties broke for the Democratic candidate, Andy Beshear.

Mr. Bevin, who was on the ballot this week seeking a second term as
governor, is now about 5,100 votes behind Mr. Beshear, who has claimed
victory. Mr. Bevin on Wednesday formally asked state officials to
undertake a check and recanvass of the voting machines and absentee
ballots in the race, citing ``irregularities'' without providing
details.

For Mr. Trump and other Republican leaders, the ongoing political
realignment of the suburbs --- which was essential to Democrats flipping
Republican-held congressional seats in 2018 and retaking the House ---
is a disconcerting disadvantage that they have shown little ability to
reverse. Democratic officials, in turn, increasingly believe they can
press a center-left agenda with little risk of backlash because moderate
voters will remain in their grip as long as Mr. Trump is in office and
effectively make the G.O.P. a no-go zone for this demographic.

``Our coalition is growing and is more secure,'' said Gov. Gina Raimondo
of Rhode Island, the chair of the Democratic Governors Association,
before quickly adding that Democrats cannot take their new voters for
granted. ``You have to earn it,'' said Ms. Raimondo, counseling her
party's candidates to avoid ideological purity tests and instead focus
on conveying to voters ``that we're going to do a better job making
their lives easier.''

But she couldn't help noting one factor behind the party's good fortune.
``Even some Republicans are just done with President Trump,'' she said.

For his part, Mr. Trump focused on other Republican victories in
Kentucky on Tuesday night, an implicit nod to Mr. Bevin's own
unpopularity with voters. He also argued that his rally in Kentucky on
Monday night had helped Mr. Bevin gain ``at least 15 points in last
days.'' In truth, every public and private poll showed a single-digit
contest in the final weeks of the race.

In interviews on Wednesday afternoon in the suburbs of Richmond, several
voters said they had decided to cast ballots for Democrats in part
because of their frustration with Mr. Trump.

Katie Morris, a grant writer, said she didn't ``have a problem with the
conservative agenda'' and usually voted Republican. But no longer.

``In general, I am very turned off by the way our country is going,''
Ms. Morris said, the day after voters in her district west of Richmond
elected the first Muslim woman to the State Senate ---
\href{https://www.nytimes3xbfgragh.onion/2019/11/06/us/ghazala-hashmi-virginia-senate.html}{Ghazala
Hashmi}, a Democrat who will replace a Republican incumbent.

``In general these days I'm pretty liberal-leaning as a result of
Trump,'' Ms. Morris said.

But another voter, Martha Grattan, was feeling plenty ideological on
Wednesday after the Republicans she supports had fared badly. Asked what
was on her mind as she cast her ballot, she said, ``Next year.''

``I want Republicans to be put back in office next year,'' she added,
referring to Mr. Trump and G.O.P. members of Congress.

\includegraphics{https://static01.graylady3jvrrxbe.onion/images/2019/11/06/us/politics/06elections-bevin/merlin_163923858_5ab0d000-8848-4925-8b1a-294a088a8662-articleLarge.jpg?quality=75\&auto=webp\&disable=upscale}

For decades, these suburbs were cornerstones of the Republican electoral
coalition, a vital constituency for conservative candidates seeking to
overcome Democrats' popularity in densely populated cities and populist
rural precincts. But these areas have steadily shifted away from the
G.O.P. as the party has come to be defined less by its traditional
center-right agenda --- like taxes and public safety --- than by the
preoccupations of rural white conservatives, on matters like protecting
gun rights, restricting abortion and cracking down on illegal
immigration.

Mr. Trump's election in 2016 drastically accelerated that migration away
from the Republican Party, spurring an exodus of already-uneasy moderate
voters away from a party defined by Mr. Trump's caustic persona and
hard-right views on race and immigration.

Each year since Mr. Trump took office has brought new evidence of his
party's decline in these areas, rich with college-educated voters and
upwardly mobile communities of immigrants and young people. In 2009, in
the first year of Mr. Obama's presidential term, the suburbs of Northern
Virginia propelled the G.O.P. to a sweep of the statewide offices. Eight
years later, it was Democrats who rolled through counties like Fairfax
and Loudoun to seize all of the state's constitutional offices by wide
margins.

There was no sign in Kentucky, Pennsylvania, Virginia and other states
this week that Republicans are close to containing the damage Mr. Trump
has done in these kinds of areas, let alone reversing it. Democratic
Party leaders, meanwhile, emerged from Election Day with the optimism
that they are building on the gains they made in the 2018 midterms. A
year after Democrats claimed 40 House seats and a series of
governorships thanks to a surge of support from suburbanites, the
results in Virginia in particular make clear that their drift from the
G.O.P. won't be easily reversed.

Virginia's Democratic governor, Ralph Northam, made clear on Wednesday
that he plans to take advantage of the historic shift from red to blue
in his state, showing no signs of worry that Democrats might pay a price
at the ballot box in 2020 if they push an aggressive policy agenda.

In an open meeting of his cabinet, Mr. Northam said ``we have a unique
opportunity in the next two years,'' saying ``the landscape has
changed'' and that he planned to push for a major new package of gun
control policies, criminal justice reform, early childhood education,
the decriminalization of marijuana and greater access to health
services.

He particularly zeroed in on gun legislation, a divisive issue in parts
of once-red Virginia but a political priority for many suburban voters.

``I really think a large part of the results that we saw yesterday were
Virginians saying they've had enough,'' Mr. Northam said. He noted that
he called a special session of the General Assembly in July to consider
eight gun measures, following a mass shooting in Virginia Beach that
took 12 lives, but Republicans called it a stunt and quickly voted to
adjourn.

``We had less than 90 minutes of dialogue, with no results,'' Mr.
Northam said.

Ms. Hashmi, the victorious candidate for State Senate outside Richmond,
had focused much of her campaign on enacting new gun control
restrictions. Everytown for Gun Safety, former New York Mayor Michael R.
Bloomberg's gun control organization, contributed \$102,000 to her
campaign, while the National Rifle Association did not contribute to her
Republican opponent's campaign.

Everytown said it gave \$2.5 million to back Democrats in Virginia. The
N.R.A. gave \$350,269 to Republican candidates and organizations,
according to state campaign finance records.

Mr. Beshear, the Kentucky Democrat, also asserted himself politically on
Wednesday, saying he
\href{https://www.nytimes3xbfgragh.onion/2019/11/05/us/politics/ky-va-ms-elections-recap.html}{would
push ahead} with a transition to power and would soon start naming
members to his cabinet and filling other roles in his administration.

At a news conference, though, he stuck to a conciliatory message of
bridging political divides in Kentucky by focusing on issues where there
was common ground --- the kind of messaging that many suburban voters
like to hear.

``Last night, the election ended,'' Mr. Beshear told reporters, standing
with members of a local teachers union in the Muhammad Ali Center in
Louisville. He noted a looming deadline to submit a state budget in
January. ``The politics part of this is over,'' he added. ``It's time
for governance.''

Image

Gov. Ralph Northam of Virginia held a cabinet meeting in Richmond on
Wednesday.Credit...Julia Rendleman for The New York Times

Mr. Beshear declined to extrapolate any lessons for the broader
Democratic Party based on his success in Kentucky. Indeed, he strained
to avoid tying himself to national political issues and focused narrowly
on the statewide issues that had been at the heart of Mr. Bevin's
unpopularity as governor, like public education, pensions for public
employees and health care.

``I'm not worried about what national pundits or what national Democrats
are saying,'' Mr. Beshear said. ``I'm worried about our families here in
Kentucky and doing a good job for them.'' He added, ``I believe this
race is about our families wanting someone that cares about them, that
reflects their values and is focused on those issues that they are
anxious about at the end of the day.''

Image

Andy Beshear spoke about his campaign's transition plans in Louisville,
Ky., on Wednesday.Credit...Luke Sharrett for The New York Times

In Virginia, several voters made clear that they, too, wanted
politicians to focus on state and local issues, though some were also
focused on Washington. In interviews, some voters said their passions
were not so much aimed at impeaching Mr. Trump, as at wanting the
president to act more presidential, and for Congress to work together to
move the country forward.

Schuyler VanValkenburg, a Democrat re-elected to his House seat in
Henrico County, which he first won in 2017 in the first phase of a blue
wave that has reshaped the politics of the suburbs under Mr. Trump, said
suburban voters are less partisan than many people assume.

``Henrico County is not an ideological place,'' he said. ``Educational
opportunity for their children and the chance of a better life for the
next generation are what people care about.''

It's also increasingly diverse. The school district where Mr.
VanValkenburg teaches 12th-grade government includes students who
together speak more than 80 languages. Voters, he said, ``want
pragmatic, sound governance.''

``What we've seen is a Republican Party that's becoming increasingly
ideological,'' he said.

Trip Gabriel reported from Richmond and Jonathan Martin and Alexander
Burns from New York. Rick Rojas contributed reporting from Louisville
and Reid J. Epstein from Washington.

Advertisement

\protect\hyperlink{after-bottom}{Continue reading the main story}

\hypertarget{site-index}{%
\subsection{Site Index}\label{site-index}}

\hypertarget{site-information-navigation}{%
\subsection{Site Information
Navigation}\label{site-information-navigation}}

\begin{itemize}
\tightlist
\item
  \href{https://help.nytimes3xbfgragh.onion/hc/en-us/articles/115014792127-Copyright-notice}{©~2020~The
  New York Times Company}
\end{itemize}

\begin{itemize}
\tightlist
\item
  \href{https://www.nytco.com/}{NYTCo}
\item
  \href{https://help.nytimes3xbfgragh.onion/hc/en-us/articles/115015385887-Contact-Us}{Contact
  Us}
\item
  \href{https://www.nytco.com/careers/}{Work with us}
\item
  \href{https://nytmediakit.com/}{Advertise}
\item
  \href{http://www.tbrandstudio.com/}{T Brand Studio}
\item
  \href{https://www.nytimes3xbfgragh.onion/privacy/cookie-policy\#how-do-i-manage-trackers}{Your
  Ad Choices}
\item
  \href{https://www.nytimes3xbfgragh.onion/privacy}{Privacy}
\item
  \href{https://help.nytimes3xbfgragh.onion/hc/en-us/articles/115014893428-Terms-of-service}{Terms
  of Service}
\item
  \href{https://help.nytimes3xbfgragh.onion/hc/en-us/articles/115014893968-Terms-of-sale}{Terms
  of Sale}
\item
  \href{https://spiderbites.nytimes3xbfgragh.onion}{Site Map}
\item
  \href{https://help.nytimes3xbfgragh.onion/hc/en-us}{Help}
\item
  \href{https://www.nytimes3xbfgragh.onion/subscription?campaignId=37WXW}{Subscriptions}
\end{itemize}
