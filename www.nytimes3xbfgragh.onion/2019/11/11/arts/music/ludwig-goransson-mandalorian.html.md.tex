Sections

SEARCH

\protect\hyperlink{site-content}{Skip to
content}\protect\hyperlink{site-index}{Skip to site index}

\href{https://www.nytimes3xbfgragh.onion/section/arts/music}{Music}

\href{https://myaccount.nytimes3xbfgragh.onion/auth/login?response_type=cookie\&client_id=vi}{}

\href{https://www.nytimes3xbfgragh.onion/section/todayspaper}{Today's
Paper}

\href{/section/arts/music}{Music}\textbar{}How Ludwig Goransson Became
Directors' Secret Musical Weapon

\url{https://nyti.ms/2ryttbU}

\begin{itemize}
\item
\item
\item
\item
\item
\end{itemize}

Advertisement

\protect\hyperlink{after-top}{Continue reading the main story}

Supported by

\protect\hyperlink{after-sponsor}{Continue reading the main story}

\hypertarget{how-ludwig-goransson-became-directors-secret-musical-weapon}{%
\section{How Ludwig Goransson Became Directors' Secret Musical
Weapon}\label{how-ludwig-goransson-became-directors-secret-musical-weapon}}

The 35-year-old Swede has worked closely with Ryan Coogler and Donald
Glover. Up next: the score for the ``Star Wars'' series ``The
Mandalorian.''

\includegraphics{https://static01.graylady3jvrrxbe.onion/images/2019/11/12/arts/11ludwig1/merlin_163534308_76114244-eaf9-46f7-9927-b7804af4b09a-articleLarge.jpg?quality=75\&auto=webp\&disable=upscale}

By Tim Greiving

\begin{itemize}
\item
  Published Nov. 11, 2019Updated Nov. 15, 2019
\item
  \begin{itemize}
  \item
  \item
  \item
  \item
  \item
  \end{itemize}
\end{itemize}

The first few months of 2019 were huge for ``Black Panther'' and Donald
Glover. Ryan Coogler's superhero movie
\href{https://www.nytimes3xbfgragh.onion/2019/02/24/movies/oscars-academy-awards.html}{took
home three Oscars}, including the prize for best score. At the Grammys,
where it won again, Glover's musical alter ego
\href{https://www.nytimes3xbfgragh.onion/2019/02/10/arts/music/grammy-awards.html}{Childish
Gambino picked up four trophies} for his ambitious political music video
and song ``This Is America.'' Both had something in common: a
longhaired, lightly bearded Swedish musician named Ludwig Goransson.

Goransson may not be a household name, but he's a well-known face behind
the scenes. (His brief moment in the spotlight came when he accepted the
record of the year Grammy for ``This Is America'' and
\href{https://www.youtube.com/watch?v=W9yvwLOAThw}{thanked 21 Savage},
who was absent in ICE detention.) He did innovative soundtrack work for
Coogler's Rocky film ``Creed'' as well as the director's breakthrough,
``Fruitvale Station.'' Goransson is currently at work on Christopher
Nolan's ``Tenet,'' but his next major project arrives on Tuesday: He
scored all eight episodes of the Disney Plus ``Star Wars'' series, ``The
Mandalorian.''

The key to the modern-western sound of ``The Mandalorian''? A flute
theme that Goransson came up with once Jon Favreau, the show's creator,
shared his vision --- which involved a lonesome rider and a samurai
inspiration. The series exists in ``more of a dystopic part of the `Star
Wars' history,'' Favreau said, ``and technology and deconstruction are
themes that we explore.''

With that in mind, Goransson locked himself in his studio for a month
and intuitively bought a bunch of rarely heard bass woodwind recorders.
He started improvising, going into an almost meditative state, he said,
creating a sprawling four hours of score that he spent the past year
writing and recording with top Hollywood studio musicians.

Goransson, 35, said he fell into his relationships with some of
Hollywood's most exciting young talents by chance. He hit it off with
Coogler, a fellow student at the University of Southern California, in
2007 over a game of pool at a frat house when Coogler brought up his
favorite Swedish hip-hop artists.

``He was a football player, and he had really long dreads,'' Goransson
said. ``Maybe it's because we come from different backgrounds that we
just have so much to talk about.''

\includegraphics{https://static01.graylady3jvrrxbe.onion/images/2019/11/12/arts/11ludwig2/11ludwig2-articleLarge.jpg?quality=75\&auto=webp\&disable=upscale}

After graduating from
\href{https://music.usc.edu/departments/scoring/}{U.S.C.'s screen
scoring program}, Goransson got his first big gig as a composer for the
NBC series ``Community.'' Glover, one of the show's stars, came to
Goransson's studio to record vocals for an outrageous Irish-tango cover
of the song ``Somewhere Out There'' from ``An American Tail.''

``We kind of laughed --- we had a good time together,'' said Goransson.
``A couple weeks later, he emailed me and was like, `Hey man, I'm also a
rapper, so I wondered if you could take a listen to this, and maybe mix
my song?'''

What drew Glover in? ``I had the classical background and jazz
background,'' Goransson said. ``I could bring something different to the
table.''

Goransson did not grow up in the hip-hop world, but music has been his
constant since he was ``little Ludde'' from Linkoping, Sweden. His
mother, a florist from Poland, and his father, a guitar teacher at the
local music school, filled the house with songs ranging from classical
to rock to Swedish folk.

He was named after Beethoven. (``My dad wanted to name me Albert after
Albert King, the big guitar blues player,'' he said. ``But my mom said,
`No --- Ludwig Beethoven.''') And he's always had long hair: ``Everybody
thought he was this adorable little girl,'' said his sister, Jessika,
``because he had the same hair that he has now. I have a bit of hair
envy.''

Goransson started playing guitar when he was 6, but his breakthrough
came three years later, when his father was learning Metallica's ``Enter
Sandman'' at the request of his students. ``He got obsessed,'' his
father, Tomas, said in a phone interview. ``He started to practice
playing every day for three or four hours.''

Goransson spent most of his youth in his family's basement, teaching
himself how to use a drum machine and a digital eight-track recorder,
and nurturing another fascination: soundtracks. He loved the music of
John Williams and Danny Elfman, and won the chance to have his work
performed by a professional orchestra as a senior in high school. He
wrote an Elfman-inspired piece called ``Five Minutes to Christmas,'' and
when he heard ``a big classical orchestra play something that I'd
written,'' he said, ``I was like, Oh wow. This is something I want to do
for a living.''

Image

When Goransson heard ``a big classical orchestra play something that I'd
written, I was like, Oh wow. This is something I want to do for a
living.''Credit...Julian Berman for The New York Times

He went to the Royal College of Music in Stockholm to major in jazz
guitar, but soon left for U.S.C., where he scored dozens of student
films. ``They were all pretty bad,'' he said, ``but there was one of
them that was actually good'' --- Coogler's short, ``Locks,'' a
precursor to his vérité-style debut,
\href{https://www.nytimes3xbfgragh.onion/2013/07/12/movies/fruitvale-station-is-based-on-the-story-of-oscar-grant-iii.html}{``Fruitvale
Station,''} about the killing of a young black man by a white officer on
a Oakland, Calif., subway platform.

``It was always a collaboration that's to be respected to the utmost,
and that respect is kind of compounded because we were friends first,''
Coogler said in an interview. ``We've been close friends for so long
that we're like family now.''

When Goransson got the job scoring ``Black Panther,'' Marvel's 2018
juggernaut about the king of a fictional African nation and his
righteously angry cousin, he knew he had to go to Africa if he was going
to get it right. He recorded the talking-drum player Massamba Diop
playing a motif for T'Challa --- the drum literally says the character's
name --- and the flutist Amadou Ba playing a theme on his Fula flute for
Erik Killmonger. He then fashioned a symphonic score with an African
heartbeat.

Goransson tries to give every film a sonic identity inherent to its
world. For ``Fruitvale Station,'' he used recordings of an actual BART
Station. On ``Creed,'' he sampled a boxing training session at Coogler's
old gym in Oakland and converted those sounds into beats and rhythms for
the film's fight sequences.

Tessa Thompson, who starred as the singer-songwriter Bianca in the
``Creed'' movies, said she found her character by creating songs
\emph{as} Bianca in Goransson's studio.

``He has this real spirit of play, and so he can really experiment,''
she said. ``He has a way of getting really close to the artists that he
works with, so they are really deep collaborations.''

Goransson briefly dabbled in making his own music (in 2012 he recorded
an EP as Ludovin), but realized he's much more comfortable as a partner
than a star. In addition to his work with Childish Gambino, he's
produced tracks for Chance the Rapper and the band Haim. For his
soundtracks, he's produced and co-written with Future, Meek Mill and
Kendrick Lamar.

On the phone from Sweden, Goransson's mother, Maria, recalled another of
his youthful obsessions --- time. When he left for U.S.C., he made a
schedule for his career: get a job as an assistant one year after
graduation, be scoring his own projects after three years, win an
Academy Award within 12 years.

``Sometimes,'' she said, ``he's much faster than what he planned.''

Advertisement

\protect\hyperlink{after-bottom}{Continue reading the main story}

\hypertarget{site-index}{%
\subsection{Site Index}\label{site-index}}

\hypertarget{site-information-navigation}{%
\subsection{Site Information
Navigation}\label{site-information-navigation}}

\begin{itemize}
\tightlist
\item
  \href{https://help.nytimes3xbfgragh.onion/hc/en-us/articles/115014792127-Copyright-notice}{©~2020~The
  New York Times Company}
\end{itemize}

\begin{itemize}
\tightlist
\item
  \href{https://www.nytco.com/}{NYTCo}
\item
  \href{https://help.nytimes3xbfgragh.onion/hc/en-us/articles/115015385887-Contact-Us}{Contact
  Us}
\item
  \href{https://www.nytco.com/careers/}{Work with us}
\item
  \href{https://nytmediakit.com/}{Advertise}
\item
  \href{http://www.tbrandstudio.com/}{T Brand Studio}
\item
  \href{https://www.nytimes3xbfgragh.onion/privacy/cookie-policy\#how-do-i-manage-trackers}{Your
  Ad Choices}
\item
  \href{https://www.nytimes3xbfgragh.onion/privacy}{Privacy}
\item
  \href{https://help.nytimes3xbfgragh.onion/hc/en-us/articles/115014893428-Terms-of-service}{Terms
  of Service}
\item
  \href{https://help.nytimes3xbfgragh.onion/hc/en-us/articles/115014893968-Terms-of-sale}{Terms
  of Sale}
\item
  \href{https://spiderbites.nytimes3xbfgragh.onion}{Site Map}
\item
  \href{https://help.nytimes3xbfgragh.onion/hc/en-us}{Help}
\item
  \href{https://www.nytimes3xbfgragh.onion/subscription?campaignId=37WXW}{Subscriptions}
\end{itemize}
