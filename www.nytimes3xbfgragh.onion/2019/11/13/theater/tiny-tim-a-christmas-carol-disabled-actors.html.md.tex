Sections

SEARCH

\protect\hyperlink{site-content}{Skip to
content}\protect\hyperlink{site-index}{Skip to site index}

\href{https://www.nytimes3xbfgragh.onion/section/theater}{Theater}

\href{https://myaccount.nytimes3xbfgragh.onion/auth/login?response_type=cookie\&client_id=vi}{}

\href{https://www.nytimes3xbfgragh.onion/section/todayspaper}{Today's
Paper}

\href{/section/theater}{Theater}\textbar{}Rethinking Tiny Tim: Should a
Disabled Actor Play the Role?

\url{https://nyti.ms/2q97DLG}

\begin{itemize}
\item
\item
\item
\item
\item
\item
\end{itemize}

Advertisement

\protect\hyperlink{after-top}{Continue reading the main story}

Supported by

\protect\hyperlink{after-sponsor}{Continue reading the main story}

\hypertarget{rethinking-tiny-tim-should-a-disabled-actor-play-the-role}{%
\section{Rethinking Tiny Tim: Should a Disabled Actor Play the
Role?}\label{rethinking-tiny-tim-should-a-disabled-actor-play-the-role}}

The Broadway production of ``A Christmas Carol,'' following the lead
from London, answers a strong yes. Other theaters may follow suit.

\includegraphics{https://static01.graylady3jvrrxbe.onion/images/2019/11/17/arts/17tinytim1/merlin_163919703_820d3b40-be4f-40f7-bf36-8a75acedd8ec-articleLarge.jpg?quality=75\&auto=webp\&disable=upscale}

\href{https://www.nytimes3xbfgragh.onion/by/michael-paulson}{\includegraphics{https://static01.graylady3jvrrxbe.onion/images/2018/02/20/multimedia/author-michael-paulson/author-michael-paulson-thumbLarge.jpg}}

By \href{https://www.nytimes3xbfgragh.onion/by/michael-paulson}{Michael
Paulson}

\begin{itemize}
\item
  Published Nov. 13, 2019Updated Nov. 14, 2019
\item
  \begin{itemize}
  \item
  \item
  \item
  \item
  \item
  \item
  \end{itemize}
\end{itemize}

Sebastian Ortiz, 7, a first grader now making his Broadway debut in
``\href{https://achristmascarolbroadway.com/}{A Christmas Carol},''
scrunched up his face as he paused to think about what it means to be
playing Tiny Tim onstage. ``Always be brave,'' he said.

``Always look on the bright side,'' suggested Jai Ram Srinivasan, 8, a
third grader sharing the role.

The two boys share a combination of bashfulness and optimism as they
embark on an adventure unlike anything either has attempted before.

They share a dressing room, with coloring books and board games and a
superhero corner tricked out with a Marvel Comics-themed mat and
pillows.

And they share a diagnosis: both of them have cerebral palsy.

Charles Dickens introduced young Tim Cratchit to the world 176 years
ago, and ever since the character has been a symbol of ``A Christmas
Carol'' --- the sickly but sanguine child whose plight helps transform
Ebenezer Scrooge.

But now, in an era in which authenticity and representation have become
entertainment industry watchwords, the presenters of some of the many
theatrical adaptations that are staged every winter are rethinking who
gets to play this iconic role, and often concluding that it should be a
child with a disability.

In Illinois, Paris Strickland, an 11-year-old girl whose spine was
compressed by cancer during childhood, will play Tiny Tim for the third
year in a row at Chicago's Goodman Theater. (Yes, at many theaters girls
play the role.)

``Paris knows that life can be fragile, she knows what a difference
strength of character and hope and relying on loved ones can make in a
person's life,'' said the Goodman production's director, Henry
Wishcamper. ``I think that comes out clearly onstage.''

\includegraphics{https://static01.graylady3jvrrxbe.onion/images/2019/11/17/arts/17tinytim2/merlin_163920294_6e4dfb25-2bf1-4845-8dec-423d34a16b1f-articleLarge.jpg?quality=75\&auto=webp\&disable=upscale}

In Texas, Jonathan Rizzo, a 12-year-old boy with muscular dystrophy,
\href{http://planomagazine.com/ntpa-repertory-theatre-casts-a-real-life-tiny-tim/}{last
year played Tim} in a musical adaptation (``Scrooge: The Musical'') at
North Texas Performing Arts Repertory Theater.

``It brought a dimension of honesty and truth to the show,'' said the
director, Mike Mazur.

And now, Broadway. A new adaptation of ``A Christmas Carol,'' starring
Campbell Scott and opening Nov. 20, cast Sebastian and Jai to alternate
in the role.

``One of my bugbears with `Christmas Carol' is when Tiny Tim is not
played by a disabled person, because he's supposed to be a disabled
character,'' said \href{https://nyti.ms/32oH9Do}{Jack Thorne}, the Tony
Award-winning playwright (``Harry Potter and the Cursed Child'') who
wrote the new adaptation for the Old Vic Theater in London, where it has
been running each holiday season since 2017.

Thorne, who for years struggled with the debilitating effects of chronic
cholinergic urticaria, a skin ailment, has worked with
\href{https://graeae.org/}{a theater company championing the disabled,}
and insisted on casting disabled children in the role of Tiny Tim.
``When there's a shortage of parts for people with disabilities,'' he
said, ``it's really important we have disabled people play disabled
parts.''

So in London, the casting call for the role made it clear: ``Applicants
without a disability will not be considered.'' In New York, the language
was subtler: ``Performers with disabilities are encouraged to
audition.''

The show's New York casting director, Jillian Cimini, said she and her
team had found Sebastian, Jai, and other candidates by reaching out to a
variety of agents and programs who work with the disabled.

Sebastian, who lives in Manhattan, was referred by
\href{https://dancingdreams.org/}{Dancing Dreams}, a New York nonprofit
that offers dance classes for children with medical or physical
challenges. Jai, who lives in Virginia, learned about the opportunity by
word of mouth; he has modeled clothing for people with disabilities from
\href{https://mashable.com/2018/04/06/tommy-hilfiger-tommy-adaptive-disibility-friendly-clothing/}{Tommy
Hilfiger} and
\href{https://www.usatoday.com/story/money/2019/06/12/adaptive-apparel-goes-more-mainstream-kohls-adds-options-kids/1415312001/}{Kohl's}.

Their casting comes at a time of increasing visibility for disabled
performers onstage. This year, \href{https://nyti.ms/2XfQNJP}{Ali
Stroker} became the first performer who uses a wheelchair to win a Tony
Award, for her work as Ado Annie in a revival of
``\href{https://nyti.ms/2I2dF8p}{Oklahoma!}'' Last year,
``\href{https://nyti.ms/2sVkNaV}{Cost of Living},'' a play by
\href{https://nyti.ms/1opad9v}{Martyna Majok} about people with
disabilities, won
\href{https://www.pulitzer.org/winners/martyna-majok}{the Pulitzer Prize
in drama}.

Over the three seasons in London, Tiny Tim has been played by boys and
girls with dwarfism, cardiomyopathy, cerebral palsy, and ataxia, as well
as other issues affecting their limbs (twisted tibias, a missing arm)
and eyesight.

``With some of our Tims it's very visible, and with others it's not, but
I think audiences can smell authenticity, and they can feel it,'' Thorne
said.

At the 116-year-old Lyceum Theater, where the Broadway production is now
in previews, Sebastian and Jai share the dressing room closest to the
stage, and ramps have been installed into both wings and the performance
area.

Both of them say they do not see their conditions as limitations.

``I just fight through it, and try to be the best I can,'' said Jai,
whose left-side weakness means he walks with a limp and who wears an arm
brace to stabilize his wrist.

``It doesn't affect me at all,'' said Sebastian, who wears orthotics on
both legs and uses a posterior walker to get around.

For the children who have already played the role in other productions,
the significance is obvious.

``It was really cool, because he has a disease like me, and it was kind
of relatable,'' said Jonathan, who played the role in Texas.

And Paris, whose family was told she had neuroblastoma when she was just
nine days old, said she too feels a kinship with Tiny Tim, whom she
plays in Chicago. ``We both have difficulties walking, but we're still
happy all the time.''

For their parents, the opportunities stir a mix of emotions. In one of
Scrooge's visions, he sees a future in which Tiny Tim has died.
Jonathan's mother, Tiffany Rizzo, noting that her child's prognosis is
uncertain, said that scene ``was really hard to watch --- it does hit
close to home.'' But, she said, ``For Tiny Tim there's hope,'' she said,
``and we have hope for Jonathan.''

Matthew Warchus, a Tony winner (for ``God of Carnage'') who directed the
London and New York productions of Thorne's ``A Christmas Carol,'' said
he is mindful of how the young performers are seen. ``The admiration
mustn't be condescending,'' he said.

But he said the message to the audience is important: ``It opens our
eyes and our imaginations to all the different versions of human beings
that surround us, and that's a good thing.'' (Warchus is also protective
of the child actors: he refused to let them be photographed playing the
role, which they do in alternate performances, saying in a statement:
``It would be unrepresentative and unfair if an image of just one of
them were to be published.'')

Tiny Tim has long been a character of concern for disability advocates,
in part because of his high visibility. Dickens describes the child's
condition with pity: ``Alas for Tiny Tim, he bore a little crutch, and
had his limbs supported by an iron frame!'' He is best known for his
benediction, ``God bless us, every one!'' which continues to prompt
tears.

``Other characters fall by the wayside and contemporary ideas take over,
but this story is so persistent,'' said Julia Miele Rodas, a disability
studies scholar and Victorianist at Bronx Community College. She said
many advocates dislike the symbolism of disability represented by a
helpless child. But she said Tim can also be seen ``as kind of
subversive, because he's the one who empowers the change in Scrooge.''

Image

Jonathan Rizzo, a 12-year-old boy with muscular dystrophy, last year
played Tiny Tim in Texas.Credit...via North Texas Performing Arts

Scrooge, of course, starts the story as a miserly grouch, and ends it
charitably. In some stage adaptations, Tiny Tim is healed by Scrooge's
generosity --- he no longer limps or uses a crutch --- a resolution that
advocates for the disabled say is unrealistic and problematic.

``The storytelling shortcut of having a character start the play with a
disability and end without it is a nondisabled way to look at it,'' said
Talleri McRae, a co-founder of the
\href{https://www.nationaldisabilitytheatre.org/}{National Disability
Theater} and a consultant on access, inclusion and education at the
Actors Theater of Louisville, which has been staging ``A Christmas
Carol'' annually for 44 years. She cautioned against the use of Tim as
``inspiration porn.''

The Louisville theater has not yet cast a child with a disability as
Tiny Tim, but, arguing that disabled characters should be able to play
any role, has this year invited Patrick Taylor, a 30-year-old actor with
Tourette's syndrome, to play a young Scrooge.

And the Louisville theater, like others concerned about how disability
is treated in the show, has modified the ending --- in their production,
McRae said, Tim's surroundings will improve thanks to Scrooge's
generosity, but Tim ``will keep his crutch and his limp.''

At the close of the Chicago production, Tim no longer needs a crutch,
but continues to limp, and is able to join in some but not all of the
activities with other children.

``Part of the story we're able to tell is that Tim is in a much better
place, but he's not miraculously healed,'' Wishcamper said. ``It's not a
movie-of-the-week ending for him.''

In the Broadway production, the child's disability does not change; the
focus is on Scrooge's transformation and Tim's empowerment. ``Scrooge is
forgiven by a child,'' Warchus said. ``That's the arc.''

Advertisement

\protect\hyperlink{after-bottom}{Continue reading the main story}

\hypertarget{site-index}{%
\subsection{Site Index}\label{site-index}}

\hypertarget{site-information-navigation}{%
\subsection{Site Information
Navigation}\label{site-information-navigation}}

\begin{itemize}
\tightlist
\item
  \href{https://help.nytimes3xbfgragh.onion/hc/en-us/articles/115014792127-Copyright-notice}{©~2020~The
  New York Times Company}
\end{itemize}

\begin{itemize}
\tightlist
\item
  \href{https://www.nytco.com/}{NYTCo}
\item
  \href{https://help.nytimes3xbfgragh.onion/hc/en-us/articles/115015385887-Contact-Us}{Contact
  Us}
\item
  \href{https://www.nytco.com/careers/}{Work with us}
\item
  \href{https://nytmediakit.com/}{Advertise}
\item
  \href{http://www.tbrandstudio.com/}{T Brand Studio}
\item
  \href{https://www.nytimes3xbfgragh.onion/privacy/cookie-policy\#how-do-i-manage-trackers}{Your
  Ad Choices}
\item
  \href{https://www.nytimes3xbfgragh.onion/privacy}{Privacy}
\item
  \href{https://help.nytimes3xbfgragh.onion/hc/en-us/articles/115014893428-Terms-of-service}{Terms
  of Service}
\item
  \href{https://help.nytimes3xbfgragh.onion/hc/en-us/articles/115014893968-Terms-of-sale}{Terms
  of Sale}
\item
  \href{https://spiderbites.nytimes3xbfgragh.onion}{Site Map}
\item
  \href{https://help.nytimes3xbfgragh.onion/hc/en-us}{Help}
\item
  \href{https://www.nytimes3xbfgragh.onion/subscription?campaignId=37WXW}{Subscriptions}
\end{itemize}
