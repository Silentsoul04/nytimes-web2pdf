Sections

SEARCH

\protect\hyperlink{site-content}{Skip to
content}\protect\hyperlink{site-index}{Skip to site index}

\href{https://www.nytimes3xbfgragh.onion/section/world/asia}{Asia
Pacific}

\href{https://myaccount.nytimes3xbfgragh.onion/auth/login?response_type=cookie\&client_id=vi}{}

\href{https://www.nytimes3xbfgragh.onion/section/todayspaper}{Today's
Paper}

\href{/section/world/asia}{Asia Pacific}\textbar{}Hong Kong Colleges
Become Besieged Citadels as Police Close In

\url{https://nyti.ms/372a0AA}

\begin{itemize}
\item
\item
\item
\item
\item
\item
\end{itemize}

Advertisement

\protect\hyperlink{after-top}{Continue reading the main story}

Supported by

\protect\hyperlink{after-sponsor}{Continue reading the main story}

\hypertarget{hong-kong-colleges-become-besieged-citadels-as-police-close-in}{%
\section{Hong Kong Colleges Become Besieged Citadels as Police Close
In}\label{hong-kong-colleges-become-besieged-citadels-as-police-close-in}}

Police have begun raiding the edges of the biggest campuses to make
arrests, leading student activists to engage with them in pitched
battles that resemble medieval sieges.

\includegraphics{https://static01.graylady3jvrrxbe.onion/images/2019/11/13/world/13hk-campus1/merlin_164268936_6e31f40b-3dc6-4119-9405-0c9f9d379c8c-articleLarge.jpg?quality=75\&auto=webp\&disable=upscale}

By \href{https://www.nytimes3xbfgragh.onion/by/edward-wong}{Edward Wong}
and Ezra Cheung

\begin{itemize}
\item
  Nov. 13, 2019
\item
  \begin{itemize}
  \item
  \item
  \item
  \item
  \item
  \item
  \end{itemize}
\end{itemize}

HONG KONG --- Seething with anger, the black-clad
students\href{https://www.nytimes3xbfgragh.onion/2019/11/11/world/asia/hong-kong-protests-shooting.html}{hurled
gasoline bombs, threw bricks and even aimed flaming arrows at the riot
police,} who answered with tear-gas volleys and rubber bullets that
hurtled into Hong Kong's university grounds for the first time.

And with those battles on Monday and Tuesday at the territory's largest
universities, another unspoken rule in the antigovernment protests that
have been convulsing Hong Kong for six months was shattered: the
sanctity of educational campuses from the police.

The clashes turned what had been sanctuaries for the students at the
core of the movement into scenes that evoked
\href{https://www.nytimes3xbfgragh.onion/2019/11/12/world/asia/photos-hong-kong-protests.html}{medieval
citadels under siege}.

They opened a new chapter that threatens to further disrupt the Asian
financial capital, which has struggled for normalcy despite the
increasingly violent protests against the Chinese Communist authorities
in Beijing
\href{https://www.nytimes3xbfgragh.onion/2019/11/18/world/asia/hong-kong-protests.html}{who
have the last word} over Hong Kong's future.

Hong Kong has
\href{https://www.nytimes3xbfgragh.onion/2019/10/31/world/asia/hong-kong-halloween.html}{fallen
into recession} as tourists have fled and as its busy shopping areas
become backdrops for street battles between demonstrators and police
officers. The world is asking hard questions about what could befall
Hong Kong as Beijing
\href{https://www.nytimes3xbfgragh.onion/2019/09/06/business/hong-kong-fitch-downgrade.html}{further
tightens control} over a city that is supposed to operate under its own
laws.

{[}\href{https://www.nytimes3xbfgragh.onion/2019/11/13/world/asia/hong-kong-protests.html}{\emph{Why
are people protesting in Hong Kong?}}{]}

The most dramatic student-versus-police clash unfolded late Tuesday
night at a barricaded bridge leading to the campus of the Chinese
University of Hong Kong. For hours, police officers fired hundreds of
rounds of tear gas and rubber bullets and students hurled Molotov
cocktails and bricks, and practiced firing bows with flaming arrows.
More than 100 injured students were brought to a makeshift first-aid
clinic in a gym.

\includegraphics{https://static01.graylady3jvrrxbe.onion/images/2019/11/13/world/13hk-campus2/merlin_164268861_ca284d41-7ac3-42cc-a22a-ad7958fecb4b-articleLarge.jpg?quality=75\&auto=webp\&disable=upscale}

By targeting campuses, the police have breached the last refuge of the
protesters, a move that brings the violence to the heart of the
universities and invokes the pivotal and fraught role of student
activism in the global history of democracy movements.

``One thing that people have realized is that the protests, the
movement, the conflict, is unavoidable,'' Gabriel Fung, a 19-year-old
second-year student at the University of Hong Kong, said. ``It's going
to reach you wherever you are at some point.''

It is at these universities where young leaders and other students have
been organizing revolts against the Chinese Communist Party and
spreading the pro-democracy ideas that undergird the protests. And here,
too, that the students discuss the wealth inequality and cultural
homogenization that have led to visions of a bleak future among many of
their generation.

In Hong Kong, university administrators and professors now find
themselves in a difficult position, trying to preach tolerance and walk
a tightrope of furious demands from students, the police and government
officials. Two schools on Wednesday ended their semesters weeks early.

``Not a single place in Hong Kong is exempt from the rule of law, and
that includes universities,'' John Lee, the secretary for security, said
Wednesday at a news conference. ``Universities are not supposed to be
the breeding ground of violence.''

Image

Protesters at the Chinese University of Hong Kong in Sha Tin on
Wednesday.Credit...Lam Yik Fei for The New York Times

The showdown has been brewing for years, going back to the
\href{https://www.nytimes3xbfgragh.onion/2014/10/08/world/asia/hong-kong-people-looking-in-mirror-see-fading-chinese-identity.html}{pro-democracy
Umbrella Movement of 2014}. And the roots of the protests in many ways
harken back to social movements elsewhere.

On mainland China, students have led campaigns calling for sweeping
political change, notably in 1919 and 1989. In the United States in the
late 1960s and early 1970s, violence broke out on campuses during
anti-Vietnam War protests, most horrifically at
\href{https://www.nytimes3xbfgragh.onion/1970/05/05/archives/4-kent-state-students-killed-by-troops-8-hurt-as-shooting-follows.html}{Kent
State University} in 1970, when Ohio National Guard troops opened fire
on students, killing four and injuring nine.

Student activists in Hong Kong have lived by an exhausting weekly rhythm
since the movement began in early June: protest on weekends, show up on
Mondays for class, study for exams and apply for internships or jobs in
between it all. Many argue with parents who disagree with their politics
or tactics. Hundreds have been arrested in recent months and quickly
released by the police, as required by law.

It was the death of a university student this month that set off the
current round of protests and violence.
\href{https://www.nytimes3xbfgragh.onion/2019/11/07/world/asia/hong-kong-protest-student-dies.html}{Chow
Tsz-lok}, a student at the Hong Kong University of Science and
Technology, sustained a fatal injury after falling from a parking garage
near a police action on Nov. 4. Thousands attended candlelit memorial
rallies last weekend, and his photograph is on posters and makeshift
shrines all over campuses, since he is now a martyr for other students.

Image

A ceremony to pay tribute to Chow Tsz-lok, a university student whose
death made him a martyr for other students.Credit...Lam Yik Fei for The
New York Times

Roiled by the latest unrest, universities canceled classes from Monday
to Friday. That meant protesters have been able to hit the streets at
dawn on weekdays after sleeping a few hours. On campus, activists have
sprayed fresh graffiti, including phrases cursing administrators.

The fraught situation led police officers on Wednesday to organize an
evacuation of dozens of mainland Chinese students across the border to
Shenzhen, where hotels offered them free rooms.

One graduate student at the University of Hong Kong said he and others
from the mainland still felt safer on campuses than on the streets. He
said many students do not openly express pro-Beijing opinions and
sometimes avoid speaking loudly in Mandarin, the dominant language back
home. (He spoke on the condition of anonymity because of the tensions.)

Some university departments have delayed recruitment drives of mainland
and foreign students to come up with new strategies; a drop-off in
enrollment by mainland graduate students, who often pay full tuition,
would lead to budget problems.

Hong Kong's public universities, which have more than 86,000
undergraduate and nearly 11,000 graduate students, each have distinct
characters. That means the students have occupied different roles in the
movement, and the protests have played out in different ways on each
campus.

Image

A staff member escorted a group of mainland Chinese students off the
City University of Hong Kong campus.Credit...Lam Yik Fei for The New
York Times

The Chinese University of Hong Kong, with 20,000 students, is considered
the most radical campus. Most of its students are Cantonese-speaking
locals, some of whom live nearby with their parents in dense apartment
blocks. And the campus is high in the hills of Sha Tin, isolated from
the city center, which is an hour's ride away by subway.

On Monday, the police arrested five students on the campus's edge,
administrators said. The next morning, the police, still at the border,
confronted front line students, and clashes took place over 20 hours.
\href{http://www.cuhk.edu.hk/governance/officers/rocky-tuan/english/biography.html}{Rocky
S. Tuan}, the president, who has been known for
\href{https://www.hongkongfp.com/2019/10/11/i-not-one-hong-kong-student-removes-mask-accuses-police-sexual-assault/}{trying
to engage with students} during the movement, showed up during a lull in
the evening to urge the students to be calm.

``You all should know that I really want to help you. I will do
everything within my capability,'' he said. ``It is the university's
responsibility to maintain peace on campus, not the police.''

But as Mr. Tuan began walking away, the police fired tear gas. Mr. Tuan
himself was enveloped in the gas. Students set fires to keep the police
from advancing, and scores formed human chains to pass along bricks,
umbrellas and bottled water to the front lines. Students sitting on one
patch of road made gasoline bombs as if on an assembly line.

``It was a savage move and a type of police violence when they tried to
encroach on the university,'' said Timothy Chow, 23, an engineering
student who graduated in June. ``This is why we have to protect our
Chinese University of Hong Kong.''

``When I saw our compatriots and Chinese University staff being hurt by
the police, I felt particularly furious and wanted to come back to
defend our university,'' he added.

Image

A demonstration at the University of Hong Kong in September.Credit...Lam
Yik Fei for The New York Times

At the University of Hong Kong this week, front-line students also set
up barricades and, against the advice of professors, threw paving bricks
off balconies, even though it is considered the most established of the
territory's schools.

Founded in 1911, it is the territory's oldest university. Many of its
students are foreigners or Hong Kong residents who attended
international schools. English is the main language, and the university
aims to
\href{https://www.sppoweb.hku.hk/sdplan/eng/the-enabling-platform/enhancing-ou-mainland-presence.php}{open
a mainland China campus}. Among its alumni are many police commanders
and Carrie Lam, the territory's chief executive who is reviled by
protesters.

On Monday, the students were on edge in part because the police
\href{https://www.hongkongfp.com/2019/11/11/video-hong-kong-police-pepper-spray-hku-students-following-arrest-near-residential-halls/}{had
taken a student from a dormitory area}early that morning.

Two liberal law
professors,\href{https://news.rthk.hk/rthk/en/component/k2/1491306-20191111.htm}{}\href{https://news.rthk.hk/rthk/en/component/k2/1491306-20191111.htm}{Hualing
Fu}\href{https://news.rthk.hk/rthk/en/component/k2/1491306-20191111.htm}{and
Johannes Chan}, urged a group of front-line protesters in masks not to
resort to violence and to understand that the struggle for democracy was
a lifetime commitment,
\href{https://twitter.com/maryhui/status/1193726398864576512}{according
to video footage}. But one masked woman shouted they had no choice, and
asked: ``How many people are we going to sacrifice?''

``We are better, we are different,'' Mr. Fu said.

``But we shall not forgive,'' a young man shouted, ``we shall not
forget.''

Image

Protest art at the University of Hong Kong in September.Credit...Lam Yik
Fei for The New York Times

On Monday and Tuesday mornings, police officers arrived at campus
entrances to try to clear the barricades. They fired tear gas, but
retreated.

Students have called on the president,
\href{https://presidentoffice.hku.hk/}{Xiang Zhang}, to forcefully
condemn the police, but he has refrained from doing so, and, unlike Mr.
Tuan, rarely holds open forums. On occasion, professors have shown up at
the front lines to speak to students, as
\href{http://www.socsc.hku.hk/bio/Hayward.htm}{William Hayward}, dean of
social sciences, did on Tuesday.

``Obviously, as it goes on and as it gets more polarized, this becomes
increasingly a challenge,'' Mr. Hayward later said of student
engagement. ``Some of them do really open up, but at the same time, you
know, of course they're trying to figure out --- is he on our side or is
he trying to silence us?''

As night fell on Tuesday, students traded shifts at the barricades,
walking past a famous eight-meter statue of orange corpses,
\href{https://www.hongkongfp.com/2018/05/05/pillar-shame-history-hong-kongs-harrowing-tribute-tiananmen-massacre-victims/}{``The
Pillar of Shame,''} that memorializes the 1989 massacre of pro-democracy
students and workers around Tiananmen Square in Beijing by the Chinese
government.

Paul Mozur and Katherine Li contributed reporting.

Advertisement

\protect\hyperlink{after-bottom}{Continue reading the main story}

\hypertarget{site-index}{%
\subsection{Site Index}\label{site-index}}

\hypertarget{site-information-navigation}{%
\subsection{Site Information
Navigation}\label{site-information-navigation}}

\begin{itemize}
\tightlist
\item
  \href{https://help.nytimes3xbfgragh.onion/hc/en-us/articles/115014792127-Copyright-notice}{©~2020~The
  New York Times Company}
\end{itemize}

\begin{itemize}
\tightlist
\item
  \href{https://www.nytco.com/}{NYTCo}
\item
  \href{https://help.nytimes3xbfgragh.onion/hc/en-us/articles/115015385887-Contact-Us}{Contact
  Us}
\item
  \href{https://www.nytco.com/careers/}{Work with us}
\item
  \href{https://nytmediakit.com/}{Advertise}
\item
  \href{http://www.tbrandstudio.com/}{T Brand Studio}
\item
  \href{https://www.nytimes3xbfgragh.onion/privacy/cookie-policy\#how-do-i-manage-trackers}{Your
  Ad Choices}
\item
  \href{https://www.nytimes3xbfgragh.onion/privacy}{Privacy}
\item
  \href{https://help.nytimes3xbfgragh.onion/hc/en-us/articles/115014893428-Terms-of-service}{Terms
  of Service}
\item
  \href{https://help.nytimes3xbfgragh.onion/hc/en-us/articles/115014893968-Terms-of-sale}{Terms
  of Sale}
\item
  \href{https://spiderbites.nytimes3xbfgragh.onion}{Site Map}
\item
  \href{https://help.nytimes3xbfgragh.onion/hc/en-us}{Help}
\item
  \href{https://www.nytimes3xbfgragh.onion/subscription?campaignId=37WXW}{Subscriptions}
\end{itemize}
