Sections

SEARCH

\protect\hyperlink{site-content}{Skip to
content}\protect\hyperlink{site-index}{Skip to site index}

\href{https://www.nytimes3xbfgragh.onion/section/business/media}{Media}

\href{https://myaccount.nytimes3xbfgragh.onion/auth/login?response_type=cookie\&client_id=vi}{}

\href{https://www.nytimes3xbfgragh.onion/section/todayspaper}{Today's
Paper}

\href{/section/business/media}{Media}\textbar{}The Streaming Era Has
Finally Arrived. Everything Is About to Change.

\url{https://nyti.ms/2XqxVp6}

\begin{itemize}
\item
\item
\item
\item
\item
\item
\end{itemize}

Advertisement

\protect\hyperlink{after-top}{Continue reading the main story}

Supported by

\protect\hyperlink{after-sponsor}{Continue reading the main story}

\hypertarget{the-streaming-era-has-finally-arrived-everything-is-about-to-change}{%
\section{The Streaming Era Has Finally Arrived. Everything Is About to
Change.}\label{the-streaming-era-has-finally-arrived-everything-is-about-to-change}}

Once a generation, Hollywood experiences a seismic shift. It is
happening again.

\includegraphics{https://static01.graylady3jvrrxbe.onion/images/2019/11/08/business/08streaming-02/08streaming-02-master1050.jpg}

\href{https://www.nytimes3xbfgragh.onion/by/brooks-barnes}{\includegraphics{https://static01.graylady3jvrrxbe.onion/images/2018/02/16/multimedia/author-brooks-barnes/author-brooks-barnes-thumbLarge.jpg}}

By \href{https://www.nytimes3xbfgragh.onion/by/brooks-barnes}{Brooks
Barnes}

\begin{itemize}
\item
  Published Nov. 18, 2019Updated Nov. 19, 2019
\item
  \begin{itemize}
  \item
  \item
  \item
  \item
  \item
  \item
  \end{itemize}
\end{itemize}

LOS ANGELES --- Every three decades, or roughly once a generation,
Hollywood experiences a seismic shift. The transition from silent films
to talkies in the 1920s. The rise of broadcast television in the 1950s.
The raucous ``I Want My MTV'' cable boom of the 1980s.

It is happening again. The long-promised streaming revolution --- the
next great leap in how the world gets its entertainment --- is finally
here.

Streaming services, of course, have been challenging the Hollywood
status quo for years. Netflix began streaming movies and television
shows in 2007 and has grown into a giant, spending \$12 billion on
programming this year to entertain more than 158 million subscribers
worldwide. There are 271 online video services available in the United
States, according to the research firm Parks Associates, one for
seemingly every predilection --- \href{https://pongalo.com/}{Pongalo}
for telenovelas, \href{https://aerocinema.com/}{AeroCinema} for aviation
documentaries, \href{https://www.shudder.com/}{Shudder} for horror
movies, \href{https://horselifestyle.tv/en}{Horse Lifestyle} for
equine-themed content. (Offerings include a series called ``Marvin the
Tap Dancing Horse.'')

While all this was happening, however, the three biggest old-line media
companies --- Disney, NBCUniversal and WarnerMedia --- largely stayed on
the sidelines. Charging into the streaming fray would mean putting
billions of dollars in profit from existing cable networks like USA,
Disney Channel and TBS at risk. Building video platforms of the size
needed to compete with Netflix and Amazon would be frightfully
expensive. And mastering the underlying technology would require a
\href{https://www.nytimes3xbfgragh.onion/2012/10/22/business/media/disney-struggling-to-find-its-digital-footing-overhauls-disneycom.html}{sharp
learning curve}. Better to bide their time. When it became clear that
protecting their existing business model was more perilous than
embracing the future, no matter now disruptive in the near term, they
would act.

That time is now. And everything is changing.

``I get asked all the time, `Where does this stop? \emph{When does it
stop}?''' said Brett Sappington, a senior Parks Associates analyst and
researcher. ``The truth is that it is only getting started.''

Disney Plus arrived on Tuesday and costs less (\$6.99 a month) than a
single tub of popcorn at big-city movie theaters. It allows anyone with
a high-speed internet connection to instantly watch Disney, Pixar,
``Star Wars'' and Marvel movies, along with
\href{https://www.nytimes3xbfgragh.onion/2019/11/07/arts/television/disney-plus-togo-noelle-lady-and-tramp.html}{original
series and films}, 30 seasons of ``The Simpsons'' and 7,500 episodes of
old Disney-branded TV shows. ``We're all in,'' Robert A. Iger, Disney's
chief executive, said in April at an
\href{https://www.nytimes3xbfgragh.onion/2019/04/11/business/media/disney-plus-streaming.html}{event
unveiling the service}.

Disney said on Wednesday that more than 10 million people had already
signed up for the service. Analysts had been hoping for eight million by
the end of the year.

In May, WarnerMedia will introduce HBO Max (\$14.99 a month), which will
offer 10,000 hours of instant entertainment, including the entirety of
``Friends'' and ``South Park,'' hundreds of Warner Bros. movies,
everything Batman, the HBO library, 50 years' worth of ``Sesame Street''
episodes, and CNN documentaries. ``We're all in,'' John Stankey,
WarnerMedia's chief executive, said at an
\href{https://www.nytimes3xbfgragh.onion/2019/10/29/business/media/hbo-max-price.html}{HBO
Max promotional event} on Oct. 29.

\href{https://www.nytimes3xbfgragh.onion/2019/09/17/business/media/peacock-nbcuniversal-streaming.html}{Peacock},
an NBCUniversal streaming service also scheduled for a spring debut,
will offer \emph{15,000 hours of content}: complete seasons of ``The
Office'' and ``Frasier,'' Universal films like ``The Fast and the
Furious'' and ``Despicable Me,'' Telemundo shows, every episode of
``Saturday Night Live,'' a new reboot of ``Battlestar Galactica.''
Peacock, unlike Disney Plus and HBO Max, will carry advertising.
NBCUniversal is expected to disclose pricing details (and presumably
declare that it is ``all in'') at an event of its own in the coming
months.

As the Big Three entertainment companies launch their video platforms,
streaming competition is mounting from Silicon Valley. Apple rolled out
\href{https://www.nytimes3xbfgragh.onion/2019/09/11/business/media/apple-tv-plus-price.html}{Apple
TV Plus} on
\href{https://www.nytimes3xbfgragh.onion/2019/10/30/business/media/apple-tv-plus.html}{Nov.
1}. Facebook and Snapchat are determined to become bigger video forces.
And never count out YouTube, part of the Google family. Feeling the need
for more ``quick bite'' videos while standing in line at the grocery
store?
\href{https://www.nytimes3xbfgragh.onion/2019/10/07/business/media/quibi-espn-katzenberg.html}{Quibi},
a streaming start-up led by Meg Whitman and Jeffrey Katzenberg, is due
in April.

\includegraphics{https://static01.graylady3jvrrxbe.onion/images/2019/11/13/business/13streaming-revolution2/merlin_153921912_d37bc8d6-566b-4b18-a57c-5413823b4960-articleLarge.jpg?quality=75\&auto=webp\&disable=upscale}

The onslaught is upending how Hollywood does business in almost every
way.

Instead of relying exclusively on middlemen (cable system operators,
multiplex chains) to get shows and movies to viewers, traditional
entertainment companies are for the first time selling content directly
to consumers. As a result, studios are releasing fewer films in
theaters; WarnerMedia said recently that ``Superintelligence,'' a
Melissa McCarthy comedy scheduled for theatrical release in December,
would instead debut in the spring ---
\href{https://www.forbes.com/sites/scottmendelson/2019/10/17/melissa-mccarthy-superintelligence-hbo-max-warner-bros-adam-sandler-netflix-kevin-hart-andy-serkis/\#18ba41f0321c}{directly
on HBO Max}.

With more original movies bypassing big screens, the line between TV and
film is
\href{https://www.cnn.com/2019/08/26/entertainment/streaming-wars-column/index.html}{blurring},
prompting once-unthinkable operating questions. Studios, for instance,
employ separate executive teams to oversee the development and
production of movies and television series. Should that siloed approach
end?

There has even been some muttering about whether the
\href{https://www.hollywoodreporter.com/news/as-lines-between-film-tv-blur-academies-should-merge-1222139}{Emmys
and the Oscars should merge}.

So much change is suddenly happening so quickly that viewers are
becoming overwhelmed and, studies suggest, not in a good way. For some
people, the cable bundle is starting to seem downright manageable in
comparison.

``Consumers are upset about the imminent changes in the media
landscape,'' consumer behavior researchers at the Langston Company, a
Colorado consultancy,
\href{https://thelangstonco.com/langston-streaming-study}{concluded in a
September report}. ``These negative feelings are driven by fears of
fragmentation, erosion of perceived value and the friction-cost of
having multiple streaming accounts.''

Nearly 50 percent of consumers are frustrated by the growing number of
subscription services required to see the content they want to watch,
according to
\href{https://www.fluentco.com/resources/streaming-services-whitepaper/}{an
August white paper} by Fluent, a digital marketing company.

Without question, analysts say, the flood of new streaming services will
cause more people to cancel traditional cable subscriptions. Cable
television is still the entertainment industry's cash cow, but millions
of customers in the United States have already cut the cord. The annual
pace of subscriber decline hit 5.4 percent in the second quarter, a
statistic Craig Moffett, a senior analyst at MoffettNathanson, referred
to in a recent report as ``freaking ugly.''

For traditional companies like Disney and NBCUniversal, each of which
run vast cable networks, that means reduced ad sales and harder
negotiations with distributors over fees. ``All signs point to
subscriber losses continuing to accelerate,'' Richard Greenfield, a
founder of the LightShed Partners research firm, wrote in a client note.
``Virtually every ambitious, must-see TV show is headed for a
direct-to-consumer platform, with TV/basic cable taking the proverbial
leftovers.''

Image

Netflix is~spending \$12 billion on programming this year to entertain
166 million subscribers worldwide.Credit...Hunter Kerhart for The New
York Times

Big cable channels like ESPN, Fox News, Bravo and HGTV aren't going
anywhere, but channels that are already poorly rated --- BabyFirst,
Ovation,
\href{https://www.nytimes3xbfgragh.onion/2016/02/29/arts/television/viceland-a-new-cable-channel-aims-to-stand-out.html?searchResultPosition=1}{Viceland}
--- will have a harder time staying in business, analysts say. The
culling of the herd has already started, with cable outlets like Cloo,
Esquire, Pivot and Al Jazeera America calling it quits in recent years.
Glenn Beck will
\href{https://www.hollywoodreporter.com/news/glenn-becks-blaze-end-linear-tv-1252902}{pull
the plug} on his Blaze cable channel next month.

Even so, some of the biggest changes involve talent.

Netflix and other tech companies, including Apple and Amazon, have been
\href{https://www.latimes.com/projects/la-et-netflix-job-report/}{steadily
poaching writer-producers} from established studios and television
networks by offering eye-popping pay packages. Kenya Barris
(``black-ish''), Ryan Murphy (``American Horror Story''), Shonda Rhimes
(``Grey's Anatomy'') and David Benioff and D.B. Weiss (``Game of
Thrones'') have all high-tailed it to Netflix, following stars like Adam
Sandler and David Letterman. The establishment has recently been
punching back. To keep Greg Berlanti, the TV whiz behind shows like
``The Flash'' and ``Riverdale,'' Warner Bros. dug
\href{https://www.nytimes3xbfgragh.onion/2018/06/07/business/media/greg-berlanti-warner-bros.html}{deep
into its pockets}. Warner completed a
\href{https://www.nytimes3xbfgragh.onion/2019/06/17/business/media/jj-abrams-warnermedia.html}{similar
deal} with J.J. Abrams in September.

``There is money being thrown at people and ideas and scripts at a level
that has never happened before in Hollywood,'' said Mr. Sappington, the
Parks Associates analyst.

Even Netflix is starting to experience sticker shock. Ted Sarandos, the
company's chief content officer, told analysts on an October conference
call that new bidders were driving up prices for ``elite'' content. ``On
a very competitive show, there has probably been 30 percent price
escalation since last year,'' Mr. Sarandos said.

Most definitely, streaming money is sloshing through the Hollywood
economy. Producers in backwaters like children's television are
\href{https://www.nytimes3xbfgragh.onion/2019/10/11/business/media/netflix-children-movies-streaming.html}{now
in hot demand}. Midlevel publicists are driving new luxury cars.
Florists, caterers, set decorators, chauffeurs, hair stylists,
headhunters --- it's gravy train time.

But fewer Hollywood people are turning cartwheels than outsiders might
think. To keep their content assembly lines speeding
(\href{https://www.broadcastingcable.com/news/just-shy-of-500-scripted-originals-says-fx-survey}{495
scripted original series} aired in 2018, an 85 percent increase from
2011) companies are stretching some employees to a breaking point.
Because streaming services order fewer episodes and cancel series after
shorter runs, rank-and-file writers are having to switch jobs more
frequently.

There is also a
\href{https://deadline.com/2019/07/hollywood-profit-participation-tv-deals-changes-disney-streaming-services-1202641423/}{fundamental
shift with employment contracts} underway. Disney, for instance, has
adopted new terms for TV shows. Under the old model, in place for
decades, show creators were paid handsome fees from the beginning. But
the big money came in success: a slice of profits from rerun sales.
Disney, following a
\href{https://www.wsj.com/articles/the-war-for-talent-in-the-age-of-netflix-11569038435}{model
popularized by Netflix}, now offers higher upfront payments but little
or no ``back end.'' Other traditional companies are doing the same; they
say it allows for distribution flexibility inside their corporate
ecosystems (broadcast, cable, streaming).

Image

Courtney Kemp, with 50 Cent, is the creator of ``Power'' and has said
streaming companies want to ``never have to tell us the truth about the
value of our content.''Credit...Nina Westervelt for The New York Times

The shift has rankled members of the Writers Guild of America, which
represents about 13,000 screenwriters and has been whispering about a
potential strike. The W.G.A.'s contract with studios expires on May 1.
Studio contracts with two additional Hollywood unions, SAG-AFTRA
(actors) and the Directors Guild of America, expire on June 30.

Courtney Kemp, creator of the Starz drama ``Power,'' campaigned on the
topic during September elections for the writers' guild's West Coast
board. ``The companies are looking actively to `buy us out' up front, so
they don't have to share profits with us, and they don't have to pay us
for reuse --- and they will never have to tell us the truth about the
value of our content,'' Ms. Kemp wrote in her campaign statement.

``They will own your intellectual property outright and forever,'' Ms.
Kemp continued. ``As my 8-year-old daughter would say --- no backsies.
And that's an issue worth striking over.''

Revolutions are not known for their tranquillity.

Advertisement

\protect\hyperlink{after-bottom}{Continue reading the main story}

\hypertarget{site-index}{%
\subsection{Site Index}\label{site-index}}

\hypertarget{site-information-navigation}{%
\subsection{Site Information
Navigation}\label{site-information-navigation}}

\begin{itemize}
\tightlist
\item
  \href{https://help.nytimes3xbfgragh.onion/hc/en-us/articles/115014792127-Copyright-notice}{©~2020~The
  New York Times Company}
\end{itemize}

\begin{itemize}
\tightlist
\item
  \href{https://www.nytco.com/}{NYTCo}
\item
  \href{https://help.nytimes3xbfgragh.onion/hc/en-us/articles/115015385887-Contact-Us}{Contact
  Us}
\item
  \href{https://www.nytco.com/careers/}{Work with us}
\item
  \href{https://nytmediakit.com/}{Advertise}
\item
  \href{http://www.tbrandstudio.com/}{T Brand Studio}
\item
  \href{https://www.nytimes3xbfgragh.onion/privacy/cookie-policy\#how-do-i-manage-trackers}{Your
  Ad Choices}
\item
  \href{https://www.nytimes3xbfgragh.onion/privacy}{Privacy}
\item
  \href{https://help.nytimes3xbfgragh.onion/hc/en-us/articles/115014893428-Terms-of-service}{Terms
  of Service}
\item
  \href{https://help.nytimes3xbfgragh.onion/hc/en-us/articles/115014893968-Terms-of-sale}{Terms
  of Sale}
\item
  \href{https://spiderbites.nytimes3xbfgragh.onion}{Site Map}
\item
  \href{https://help.nytimes3xbfgragh.onion/hc/en-us}{Help}
\item
  \href{https://www.nytimes3xbfgragh.onion/subscription?campaignId=37WXW}{Subscriptions}
\end{itemize}
