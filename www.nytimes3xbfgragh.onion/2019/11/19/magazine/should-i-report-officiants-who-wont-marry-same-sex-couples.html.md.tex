Sections

SEARCH

\protect\hyperlink{site-content}{Skip to
content}\protect\hyperlink{site-index}{Skip to site index}

\href{https://myaccount.nytimes3xbfgragh.onion/auth/login?response_type=cookie\&client_id=vi}{}

\href{https://www.nytimes3xbfgragh.onion/section/todayspaper}{Today's
Paper}

Should I Report Officiants Who Won't Marry Same-Sex Couples?

\url{https://nyti.ms/2CXYfNG}

\begin{itemize}
\item
\item
\item
\item
\item
\item
\end{itemize}

Advertisement

\protect\hyperlink{after-top}{Continue reading the main story}

Supported by

\protect\hyperlink{after-sponsor}{Continue reading the main story}

\href{/column/the-ethicist}{The Ethicist}

\hypertarget{should-i-report-officiants-who-wont-marry-same-sex-couples}{%
\section{Should I Report Officiants Who Won't Marry Same-Sex
Couples?}\label{should-i-report-officiants-who-wont-marry-same-sex-couples}}

\includegraphics{https://static01.graylady3jvrrxbe.onion/images/2019/11/24/magazine/24mag-ethicist/24mag-ethicist-articleLarge.jpg?quality=75\&auto=webp\&disable=upscale}

By Kwame Anthony Appiah

\begin{itemize}
\item
  Nov. 19, 2019
\item
  \begin{itemize}
  \item
  \item
  \item
  \item
  \item
  \item
  \end{itemize}
\end{itemize}

\emph{I'm a professional wedding officiant and longtime L.G.B.T.Q.
rights advocate. While I recognize that some officiants and other
vendors may have ethical or religious objections to same-sex unions, I
rejoice that marriage equality is the law of the land.}

\emph{Many wedding expos, websites, registries, professional
organizations and social media groups have nondiscrimination policies
requiring all vendors to serve same-sex couples. I know several
officiants who participate in these and who will not marry L.G.B.T.Q.
couples. I'm wrestling with whether to ``out'' them to the gatekeepers.}

\emph{These other officiants aren't costing me business. If anything,
their exclusiveness may cause some couples to seek me out. But it's a
matter of principle --- those who serve the public should not be allowed
to discriminate, and same-sex couples should be spared a jolting refusal
as they plan their special day.}

\emph{Do I have a duty to report these noncompliant officiants? Should I
remind them of the nondiscrimination policies? Or should I mind my own
business and let an aggrieved couple report them?} Name Withheld

\textbf{Our law wisely} respects religious conscience within very broad
limits, and you correctly recognize that religious conscience, not just
unthinking bigotry, might guide people who object to same-sex unions.
That doesn't mean every conscientious religious decision is exempt from
moral criticism. As the English moral philosopher Elizabeth Anscombe
once put it, ``A man's conscience may tell him to do the vilest
things.''

So it's perfectly legitimate, even admirable, for private groups to
adopt policies of nondiscrimination that go beyond legal requirements.
It's perfectly legitimate, even admirable, for members or participants
to help enforce these policies. (Reminding the noncompliant of their
obligations is one way of doing so.) But are you dutybound to police the
policy?

Whether you're required to report a transgression generally depends both
upon its moral gravity and upon whether, as an observer, you're
especially well positioned to do so. Suppose you see someone committing
a minor parking violation that nobody else is in a position to have
seen. That passes the test of observer privilege but fails the test of
magnitude. Hence: no duty to report. Suppose you see a brawl in a
crowded club, but so has everyone else present. That passes the test of
magnitude but fails the test of observer privilege: again, no duty to
report.

Failing to abide by a group's L.G.B.T.Q. nondiscrimination clause is
bad, but not bodily-injury bad, and there's no reason to think that
you're uniquely well positioned to turn in these rule breakers. ``Duty''
is a high bar. You have every right to report them, but you shouldn't
beat yourself up if you leave it to others.

\emph{I'm a doctor in an urban emergency room in California, and I'm
struggling with two classes of patients who are becoming more common in
our E.R: patients experiencing homelessness, and patients with chronic
pain requiring opiate therapy.}

\emph{By law, E.R.s are required to medically screen and stabilize all
patients. What this means is that any person can come to the emergency
room with any medical complaint and be given a warm place to stay until
said medical complaint is evaluated. While this law is being used
appropriately by the vast majority of patients, a small subset of
patients (often the most vulnerable) take advantage of it. They know
that if they present to the E.R. with a medical complaint --- real or
imagined --- they will be guaranteed a bed for a few hours and a meal
(per California law). We will often see the same handful of people once
or twice a day. We know that they often have no other access to food or
shelter, and we want to be helpful. The problem is that the E.R. is not
meant for shelter and food. First, it is a very costly use of resources.
Second, these patients often divert scant resources such as ambulances
and beds from others who have acute medical needs. We often have to
weigh whether to provide the desired food, shelter or clothing or deny
those resources in hopes that the patients are helped elsewhere.}

\emph{Similarly, we have seen an uptick in chronic-pain patients
abandoned by primary-care clinics that no longer administer opiates due
to the unclear crackdown on opiate prescribing, even legitimate opiate
prescribing. Patients often come in desperate because of their ongoing
pain, or because of the withdrawal from medicines taken safely for
years. Some will even threaten to start using heroin if we don't
prescribe opiates, which we know is a real possibility. And again, while
we want to help, we cannot have the E.R. become the default place for
people to get pain medicine when others won't help.}

\emph{I struggle with these questions daily. The reality is that it is
costing the health care system \$200-\$300 to provide a patient with a
cold turkey sandwich. How do I, as a physician, proceed?} Name Withheld

\textbf{Given the situation} you describe, you have to go on doing what
you're doing. If people show up with a medical complaint, even one you
have doubts about, you have to treat them appropriately and, apparently,
under the law that means they get a meal; it also means dispensing
painkillers, including opiates, when (but only when) that's medically
indicated.

The solution isn't for you to change what you're doing in the hospital.
It's for the state of California to make sensible provisions for its
citizens, without unduly burdening particular institutions. And you can
best help with that by working to get a shelter opened near your
hospital, say, or joining together with your medical colleagues to put
pressure on elected officials, including state legislators, and drawing
public attention to your concerns. We need effective policies to deal
with the food, shelter and health care needs of our least fortunate
fellow citizens.

That's not just on you; it's on all of us. The private and nonprofit
sectors can also help, in partnership with the public --- as with the
Commonwealth Care Alliance, in Massachusetts. The Center for Medicare
and Medicaid Innovation, established by the Affordable Care Act, has
been tracking promising pilot programs that deploy social services and
primary care to keep ``high users'' out of the E.R. But at the moment,
you have good reason to complain: That we still haven't addressed these
problems adequately makes it harder for people like you to do your job.

Advertisement

\protect\hyperlink{after-bottom}{Continue reading the main story}

\hypertarget{site-index}{%
\subsection{Site Index}\label{site-index}}

\hypertarget{site-information-navigation}{%
\subsection{Site Information
Navigation}\label{site-information-navigation}}

\begin{itemize}
\tightlist
\item
  \href{https://help.nytimes3xbfgragh.onion/hc/en-us/articles/115014792127-Copyright-notice}{©~2020~The
  New York Times Company}
\end{itemize}

\begin{itemize}
\tightlist
\item
  \href{https://www.nytco.com/}{NYTCo}
\item
  \href{https://help.nytimes3xbfgragh.onion/hc/en-us/articles/115015385887-Contact-Us}{Contact
  Us}
\item
  \href{https://www.nytco.com/careers/}{Work with us}
\item
  \href{https://nytmediakit.com/}{Advertise}
\item
  \href{http://www.tbrandstudio.com/}{T Brand Studio}
\item
  \href{https://www.nytimes3xbfgragh.onion/privacy/cookie-policy\#how-do-i-manage-trackers}{Your
  Ad Choices}
\item
  \href{https://www.nytimes3xbfgragh.onion/privacy}{Privacy}
\item
  \href{https://help.nytimes3xbfgragh.onion/hc/en-us/articles/115014893428-Terms-of-service}{Terms
  of Service}
\item
  \href{https://help.nytimes3xbfgragh.onion/hc/en-us/articles/115014893968-Terms-of-sale}{Terms
  of Sale}
\item
  \href{https://spiderbites.nytimes3xbfgragh.onion}{Site Map}
\item
  \href{https://help.nytimes3xbfgragh.onion/hc/en-us}{Help}
\item
  \href{https://www.nytimes3xbfgragh.onion/subscription?campaignId=37WXW}{Subscriptions}
\end{itemize}
