Sections

SEARCH

\protect\hyperlink{site-content}{Skip to
content}\protect\hyperlink{site-index}{Skip to site index}

\href{https://myaccount.nytimes3xbfgragh.onion/auth/login?response_type=cookie\&client_id=vi}{}

\href{https://www.nytimes3xbfgragh.onion/section/todayspaper}{Today's
Paper}

How to Throw a Romantic Greek Dinner Party

\url{https://nyti.ms/2qiSRSR}

\begin{itemize}
\item
\item
\item
\item
\item
\item
\end{itemize}

Advertisement

\protect\hyperlink{after-top}{Continue reading the main story}

Supported by

\protect\hyperlink{after-sponsor}{Continue reading the main story}

Entertaining With

\hypertarget{how-to-throw-a-romantic-greek-dinner-party}{%
\section{How to Throw a Romantic Greek Dinner
Party}\label{how-to-throw-a-romantic-greek-dinner-party}}

The chef Mina Stone opened a casual Greek restaurant at MoMA PS1 in
Queens last week and celebrated the occasion at an artist's studio in
Brooklyn.

\includegraphics{https://static01.graylady3jvrrxbe.onion/images/2019/11/19/t-magazine/19tmag-mina-slide-JLUH/19tmag-mina-slide-JLUH-articleLarge.jpg?quality=75\&auto=webp\&disable=upscale}

By
\href{https://www.nytimes3xbfgragh.onion/by/thessaly-la-force}{Thessaly
La Force}

\begin{itemize}
\item
  Published Nov. 19, 2019Updated Nov. 20, 2019
\item
  \begin{itemize}
  \item
  \item
  \item
  \item
  \item
  \item
  \end{itemize}
\end{itemize}

On a recent fall evening in Red Hook, Brooklyn, the sky was shifting
colors quickly, from blue-gray to violet, and then swiftly, it was dark.
Assembled in the verdant courtyard of the artist
\href{http://cassandramacleod.com/}{Cassandra MacLeod}'s studio were the
friends of the chef
\href{https://www.nytimes3xbfgragh.onion/2015/01/30/t-magazine/mina-stone-cooking-for-artists-recipe.html}{Mina
Stone}, who last week opened her Mediterranean-inspired restaurant
\href{http://www.minas.nyc/}{Mina's} **** at MoMA PS1 with her husband,
Alex Eagleton. Open bottles of natural Greek wines from the specialty
Greek importer Eklektikon sat in an ice bath on a circular wooden table,
alongside sliced cheese, apples and dried figs. Stone had been hard at
work at MacLeod's studio all week, cooking and photographing dishes for
her second cookbook. Her first,
``\href{https://www.artbook.com/9780984721078.html}{Cooking
for}\textbf{\href{https://www.artbook.com/9780984721078.html}{}}\href{https://www.artbook.com/9780984721078.html}{Artists}''
(2015), was about her experience working with the art dealer Gavin
Brown, at whose gallery and home she cooked large dinners after
openings, as well as working with the artist Urs Fischer, for whom she
prepared studio lunches.

\includegraphics{https://static01.graylady3jvrrxbe.onion/images/2019/11/19/t-magazine/19tmag-mina-slide-E6O7/19tmag-mina-slide-E6O7-articleLarge.jpg?quality=75\&auto=webp\&disable=upscale}

Cooking was also how Stone fell in love. Eagleton, like Stone, is Greek,
and when the two first met, they agreed they could never find the kind
of Greek food they knew and loved. They got together to cook the dishes
they liked to eat with their families. It was innocent, or at least to
Eagleton. ``Let's just say that one of us was the fish, and one of us
was the fisherman,'' he jokes. ``I was the fisherman!'' Stone adds with
a laugh.

Image

Dinner was held in the studio, the tabletop decorated with dahlias in
bud vases.~Credit...Paul Quitoriano

In MacLeod's candlelit studio, decorated with small bud vases filled
with dahlias, and over several courses, Stone thanked everyone who had
gathered there that night to celebrate with her: artists such as
\href{https://www.maiaruthlee.club/}{Maia Ruth Lee},
\href{https://caseykaplangallery.com/artists/crowner/}{Sarah Crowner},
\href{http://www.elizabethjaeger.com/}{Elizabeth Jaeger} and
\href{http://chris-dorland.com/}{Chris Dorland}; the
\href{https://www.nytimes3xbfgragh.onion/2019/06/26/arts/design/kate-fowle-moma-ps1-director.html}{new
MoMA PS1 director} Kate Fowle; the writer
\href{http://www.nickmcdonell.com/}{Nick McDonell}; and other longtime
friends such as the cook and author Julia Sherman and the curator and
producer Eliza Ryan. Eagleton placed --- not so surreptitiously ---
bottles of ouzo and mezcal on the table to coax the night along, and
guests lingered on the studio's balcony until late, as MacLeod's black
cat, Pineapple, slinked around the shadows. Below, a guide to how to
throw a casual but artistic Mediterranean-inspired dinner like Stone's.

Image

Stone salts the roasted beets.~Credit...Paul Quitoriano

\hypertarget{fresh-ingredients-can-be-transporting}{%
\subsubsection{Fresh Ingredients Can Be
Transporting}\label{fresh-ingredients-can-be-transporting}}

Greece is a huge influence on Stone. She and Eagleton spend much of
their summers on Paros, an island in the Cyclades so famous for its
chickpeas that it holds a festival celebrating the humble legume every
July. Stone's mother's side of the family is from Pireas, a port city
**** on the mainland, and Eagelton was born and raised nearby in Athens.
**** The **** Greek food they cherish is deceptively simple, in part
because the ingredients often are common but of the highest quality.
Stone likes to buy her olive oil, feta cheese, olives and yogurt at the
Greek specialty market \href{https://www.titanfoods.net/}{Titan Foods}
in Queens. She prefers to use lemon juice to dress her dishes instead of
vinegar --- ``it adds a cleaner sensibility,'' she says. Same goes for
her tzatziki, which she makes out of Fage yogurt, grated cucumber and
garlic and drizzles with some fresh lemon juice (for more sensitive
types, she recommends placing a peeled garlic clove in the tzatziki to
let it absorb some of its flavor, removing the clove before serving).
Stone's vision is bound less by authenticity and more by the idea of
serving an experience that mirrors how Greek families eat together. Many
items she's serving at Mina's, like mezzethaki --- essentially a shot of
ouzo with a snack, though Stone is serving the snacks separately from
the beverage --- embrace the concept. She says: ``I think it would be
appropriate to have a spot where you can get great Greek food and it
doesn't serve just fish or gyros; it's what you would get if you went to
a home in Greece, such as lentils or chickpeas or what is traditionally
seen as a kind of impoverished cuisine.''

Image

Dinner was served family style, including braised chicken with lemons
and potatoes.~Credit...Paul Quitoriano

\hypertarget{serve-family-style-}{%
\subsubsection{Serve Family-Style }\label{serve-family-style-}}

Dinner was served in two courses: First came the freshest Greek salad I
had ever tasted (the last of summer's tomatoes, red onion, cucumbers,
mint, green pepper and oregano topped with barrel-aged feta imported
straight from Greece), a sesame-encrusted bread called koulouri, fresh
olives, melitzanosalata ** (an eggplant dip with lemon juice and
parsley) and tzatziki. Then arrived steamed broccoli rabe drizzled with
lemon juice, roasted beets with vinegar and braised chicken with lemon
and potatoes. Nearly all the ingredients were sourced from
\href{https://www.essexonlakechamplain.com/essex-farm/}{Essex Farm},
located in upstate New York. There was a simple pleasure in eating the
food in front of you, bite after bite, until it was magically gone.
Stone told me of a Greek word, \emph{papara}, which means ripping off a
piece bread and dipping it in the oil and tomato juice of the Greek
salad or mopping up the sauce from the pan for the chicken.

Image

Greek salad with fresh feta cheese from Titan Foods in
Queens.Credit...Paul Quitoriano

Image

The baker Savannah Turley made a variation of a Greek street food called
koulouri*.* Credit...Paul Quitoriano

\hypertarget{let-your-food-tell-a-story}{%
\subsubsection{Let Your Food Tell a
Story}\label{let-your-food-tell-a-story}}

Stone likes to work with a baker named Savannah Turley **** and had
instructed her to ``bake something sesame-y and seedy. If you want to
braid it or do anything beautiful, please do!'' So Turley began
researching Greek breads and decided on koulouri, a street bread
typically sold to people on the go in the morning. It is usually
drizzled with sesame seeds, but it can also be stuffed with cheese,
chocolate or tahini. It's related to the Turkish bread simit (often
called the Turkish bagel), and several versions of the O-shaped street
food **** appear across the former Ottoman Empire. ``This is one of the
many things I love about bread,'' Turley adds. ``It has such a history
and you learn so much about a culture by digging into their bread.''
Stone's own style of serving --- of preparing everything in advance as
much of possible --- reminded me of the storytelling you find in the
best kinds of cookbooks. She says: ``I was thinking the other day about
how a lot of Greek matriarchs get up at 8 in the morning, they make
lunch, and it sits around on the stove until it's lunchtime, and it's
briefly heated up and served and then paired with one other thing that
you cut up at that time, like the salad. To me, that is something I
apply to entertaining in my house. I like being able to have the thing
that can be made ahead of time, heated up, and then the thing that you
make right as you're ready to sit.''

Image

Stone's guests included artists, curators, writers and
more.~Credit...Paul Quitoriano

\hypertarget{pair-complex-company-with-low-concept-dishes}{%
\subsubsection{Pair Complex Company With Low-Concept
Dishes}\label{pair-complex-company-with-low-concept-dishes}}

I've attended several of Stone's dinners for Gavin Brown, and I remember
being struck by her comforting stews, rice, beans and salads --- in
short, the unfussiness of it all. You often grabbed a bowl and stood in
line and she ladled the food onto your plate, which you ate on your
knees, sitting in a circle with other artists, curators and collectors.
Stone's cookbook has become a kind of guide for me at home, not just as
an introduction to Greek food but also on how to cook for guests without
going over the top. Her very first Gavin Brown dinner was inspired by
the French art dealer Ambroise Vollard, who worked with artists such as
\href{https://www.nytimes3xbfgragh.onion/topic/person/paul-cezanne}{Paul
Cézanne} and
\href{https://www.nytimes3xbfgragh.onion/topic/person/pierre-auguste-renoir}{Pierre-Auguste
Renoir} at the turn of the 20th century. ``{[}His dinners{]} would be in
his basement, and it was at a time where this really wasn't
appropriate,'' **** says Stone. ``He would always make curry, and it
would be artists and art dealers having dinner side by side, and that
was the inspiration. It was a happy accident.'' Everything about Stone's
ethos is geared toward **** creating an informal yet interesting
atmosphere, one where it feels like having a home-cooked meal with
outstanding company --- just without being left to clean up the dirty
dishes.

Image

Dessert included portokalopita, a cake made with phyllo pastry and
orange syrup.Credit...Paul Quitoriano

Image

Slices of cantaloupe with mint and olive oil.Credit...Paul Quitoriano

\hypertarget{dessert-can-be-decadent-but-casual-}{%
\subsubsection{Dessert Can Be Decadent but Casual
}\label{dessert-can-be-decadent-but-casual-}}

To finish the meal, **** Stone served **** something few guests had
tasted before, a kind of bread pudding called portokalopita that
involves taking shredded phyllo pastry and baking it with yogurt, eggs,
sugar, olive oil and orange rind. After it's cooked, Stone pours an
orange syrup over it. It was like a bread pudding, but it was also sort
of like a sponge cake, and it was decadently delicious and utterly
surprising --- the zest of the orange mixed with the sticky sweetness of
the sugar. Stone let dinner slowly settle with her guests, allowing
everyone to mingle, chat or smoke cigarettes outside. She
unceremoniously plopped down slices of portokalopita ** arranged in its
own baking tin with some parchment paper along with a platter of sliced
Fuji apples drizzled with mint and Greek honey (the bees feast on thyme
nectar, resulting in a surprising kick) and let guests serve themselves.
I ate mine, quickly, with my fingers. Soon, the dessert platters were
depleted.

Advertisement

\protect\hyperlink{after-bottom}{Continue reading the main story}

\hypertarget{site-index}{%
\subsection{Site Index}\label{site-index}}

\hypertarget{site-information-navigation}{%
\subsection{Site Information
Navigation}\label{site-information-navigation}}

\begin{itemize}
\tightlist
\item
  \href{https://help.nytimes3xbfgragh.onion/hc/en-us/articles/115014792127-Copyright-notice}{©~2020~The
  New York Times Company}
\end{itemize}

\begin{itemize}
\tightlist
\item
  \href{https://www.nytco.com/}{NYTCo}
\item
  \href{https://help.nytimes3xbfgragh.onion/hc/en-us/articles/115015385887-Contact-Us}{Contact
  Us}
\item
  \href{https://www.nytco.com/careers/}{Work with us}
\item
  \href{https://nytmediakit.com/}{Advertise}
\item
  \href{http://www.tbrandstudio.com/}{T Brand Studio}
\item
  \href{https://www.nytimes3xbfgragh.onion/privacy/cookie-policy\#how-do-i-manage-trackers}{Your
  Ad Choices}
\item
  \href{https://www.nytimes3xbfgragh.onion/privacy}{Privacy}
\item
  \href{https://help.nytimes3xbfgragh.onion/hc/en-us/articles/115014893428-Terms-of-service}{Terms
  of Service}
\item
  \href{https://help.nytimes3xbfgragh.onion/hc/en-us/articles/115014893968-Terms-of-sale}{Terms
  of Sale}
\item
  \href{https://spiderbites.nytimes3xbfgragh.onion}{Site Map}
\item
  \href{https://help.nytimes3xbfgragh.onion/hc/en-us}{Help}
\item
  \href{https://www.nytimes3xbfgragh.onion/subscription?campaignId=37WXW}{Subscriptions}
\end{itemize}
