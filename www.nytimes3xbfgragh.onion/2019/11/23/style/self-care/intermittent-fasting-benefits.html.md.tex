Sections

SEARCH

\protect\hyperlink{site-content}{Skip to
content}\protect\hyperlink{site-index}{Skip to site index}

\href{https://www.nytimes3xbfgragh.onion/section/style/self-care/}{Self-Care}

\href{https://myaccount.nytimes3xbfgragh.onion/auth/login?response_type=cookie\&client_id=vi}{}

\href{https://www.nytimes3xbfgragh.onion/section/todayspaper}{Today's
Paper}

\href{/section/style/self-care/}{Self-Care}\textbar{}What Is
Intermittent Fasting and Does It Really Work?

\url{https://nyti.ms/2KNpYVM}

\begin{itemize}
\item
\item
\item
\item
\item
\item
\end{itemize}

Advertisement

\protect\hyperlink{after-top}{Continue reading the main story}

Supported by

\protect\hyperlink{after-sponsor}{Continue reading the main story}

Scam or Not

\hypertarget{what-is-intermittent-fasting-and-does-it-really-work}{%
\section{What Is Intermittent Fasting and Does It Really
Work?}\label{what-is-intermittent-fasting-and-does-it-really-work}}

Yes --- but fasting offers weight loss similar to any reduction in
calories. The best diet is the one where you are healthy, hydrated and
living your best life. If fasting works for you, go for it.

\includegraphics{https://static01.graylady3jvrrxbe.onion/images/2019/11/22/fashion/22scam-fasting-1/22scam-fasting-1-articleLarge.jpg?quality=75\&auto=webp\&disable=upscale}

\href{https://www.nytimes3xbfgragh.onion/by/crystal-martin}{\includegraphics{https://static01.graylady3jvrrxbe.onion/images/2019/03/01/multimedia/author-crystal-martin/author-crystal-martin-thumbLarge.png}}

By \href{https://www.nytimes3xbfgragh.onion/by/crystal-martin}{Crystal
Martin}

\begin{itemize}
\item
  Published Nov. 23, 2019Updated Jan. 2, 2020
\item
  \begin{itemize}
  \item
  \item
  \item
  \item
  \item
  \item
  \end{itemize}
\end{itemize}

People who choose not to eat for 12 hours a day claim fasting gives you
more sleep, energy and abs. Are these people just annoying or are they
onto something?

Generally, intermittent fasting is a diet strategy that involves
alternating periods of eating and extended fasting **** (meaning no food
at all or very low calorie consumption). ``There's quite a bit of debate
in our research community: How much of the benefits of intermittent
fasting are just due to the fact that it helps people eat less? Could
you get the same benefits by just cutting your calories by the same
amount?'' said Courtney M. Peterson, Ph.D., an assistant professor in
the Department of Nutrition Sciences at the University of Alabama at
Birmingham who studies time-restricted feeding, a form of intermittent
fasting.

We asked Dr. Peterson and a few other experts to help us sort out the
real from the scam on intermittent fasting.

\begin{center}\rule{0.5\linewidth}{\linethickness}\end{center}

\hypertarget{how-do-i-try-intermittent-fasting}{%
\subsection{How do I try intermittent
fasting?}\label{how-do-i-try-intermittent-fasting}}

There are four popular fasting approaches: periodic fasting,
time-restricted feeding, alternate-day fasting and the 5:2 diet.
Time-restricted feeding, sometimes called daily intermittent fasting, is
perhaps the easiest and most popular fasting method. Daily intermittent
fasters restrict eating to certain time periods each day, say 11 in the
morning to 7 at night. The fasting period is usually around 12 or more
hours that, helpfully, includes time spent sleeping overnight. Periodic
fasting will feel most familiar: no food or drinks with calories for
24-hour periods. Another type of fast, alternate-day fasting requires
severe calorie reduction every other day. Lastly, the 5:2 method was
popularized by author Kate Harrison's book ``The 5:2 Diet" and requires
fasting on two nonconsecutive days a week.

\begin{center}\rule{0.5\linewidth}{\linethickness}\end{center}

\hypertarget{is-this-a-scam}{%
\subsection{Is This A Scam?}\label{is-this-a-scam}}

\hypertarget{is-}{%
\subsubsection{Is ...}\label{is-}}

Celery Juice

,

Kombucha

,

Activated Charcoal

,

CBD

,

Turmeric

,

Fish Oil

,

Chlorophyll

,

Intermittent Fasting

,

The Keto Diet

,

Probiotics

,

Collagen

,

Coffee

,

\hypertarget{a-scam}{%
\subsubsection{A Scam?}\label{a-scam}}

Facts about wellness.

Will these trends change your life --- or

take your money?

\begin{center}\rule{0.5\linewidth}{\linethickness}\end{center}

\hypertarget{is-fasting-an-effective-weight-loss-method}{%
\subsection{Is fasting an effective weight-loss
method?}\label{is-fasting-an-effective-weight-loss-method}}

If you are obese or overweight, fasting is an effective weight-loss
method, if you stick to it. But it is no more effective than a diet that
restricts your daily calories. We know this because there were no
additional weight-loss or cardiovascular benefits of fasting two days
per week, over an ordinary calorie-restriction diet,
\href{https://theconversation.com/intermittent-fasting-is-no-better-than-conventional-dieting-for-weight-loss-new-study-finds-107829}{in
a study} of 150 obese adults over the course of 50 weeks.

But you should also consider how difficult the diet will be to stick to.
In a study of 100 randomized obese and overweight adults
\href{https://jamanetwork.com/journals/jamainternalmedicine/fullarticle/2623528}{published
in 2017}, the dropout rate was higher with those who were fasting, 38
percent, compared with 29 percent for calorie restrictors and 26 percent
for those who kept eating as they normally did.

``Some people really struggle with having to monitor their intake and
constantly record food in an app every day. So the takeaway of the study
was if daily calorie restriction doesn't work for you, maybe
alternate-day fasting would be a little easier,'' said Krista Varady,
Ph.D., professor of nutrition at the University of Illinois at Chicago
and the senior author of the study. ``There's nothing magical here.
We're tricking people into eating less food, in different ways,''
\href{https://www.nytimes3xbfgragh.onion/2017/05/03/well/eat/fasting-offers-no-special-weight-loss-benefits.html}{she
said in 2017.}

There is some new evidence that shows different forms of fasting are not
equal --- in part because some are easier than others, but also because
some forms of fasting better match our body's natural circadian rhythm,
thus lowering insulin levels, increasing fat-burning hormones and
decreasing appetite.

Basically, because our metabolism has evolved to digest food during the
day and rest at night, changing the timing of meals to earlier in the
day may be beneficial.

In \href{https://www.ncbi.nlm.nih.gov/pubmed/31339000}{a study}done in
Dr. Peterson's lab, 11 adults did time-restricted feeding (eating from 8
a.m. to 2 p.m.) and a control 12-hour eating period, for four days each.
On the last day of each session, researchers measured energy expenditure
and hunger hormones and found that time-restricted feeding improves the
appetite hormone ghrelin and increases fat burning. ``It's shown to
reduce the amount of fat in the liver, which is a risk factor for
diabetes and cardiovascular disease,'' said Dr. Peterson.

Bottom line: If you want to lose weight and are someone who hates
counting calories, you might consider fasting, as both methods offer
similar weight-loss benefits.

\begin{center}\rule{0.5\linewidth}{\linethickness}\end{center}

\hypertarget{should-i-try-intermittent-fasting}{%
\subsection{Should I try intermittent
fasting?}\label{should-i-try-intermittent-fasting}}

The most effective diet is the one you can stick to while still living
your best life. It's hard to know which will work best before trying,
but doctors and recent studies offer some guidance. Dr. Peterson said
that complete, zero-calorie fasts generally prove to be too difficult to
maintain. ``People stick with them maybe for the short-term, but they
get quite hungry in the long-term,'' she said.

Time-restricted feeding --- fasting overnight and into the next morning
--- is likely the easiest form of fasting to comply with. A longer than
normal fasting period each night allows you to burn through some of your
sugar stores, called glycogen. That does a couple things. It gives your
body a little bit more time to burn fat. It also may help your body get
rid of any extra salt in your diet, which would lower your blood
pressure, Dr. Peterson said.

\begin{center}\rule{0.5\linewidth}{\linethickness}\end{center}

\hypertarget{ive-made-the-decision-to-fast-so-how-long-should-i-fast-for}{%
\subsection{I've made the decision to fast. So how long should I fast
for?}\label{ive-made-the-decision-to-fast-so-how-long-should-i-fast-for}}

There aren't any studies right now that state exactly how long one
should fast. Researchers, like Dr. Peterson, are working on that. The
minimum amount of time it takes to make fasting efficacious hasn't been
proven via study, but the prevailing notion is it's somewhere between 12
and 18 hours. But it can take a few days --- sometimes weeks --- of
fasting regularly for your body to start burning fat for fuel. Brooke
Alpert, nutritionist and author of ``The Diet Detox,'' suggests starting
by moving your last meal to around 7 p.m. She said the reason for this
is our bodies are better at doing some things at certain times. ``Our
bodies are better at processing sugar in the morning than at night,''
said Dr. Varady. So eat bigger meals in the morning, for example.

And how often do you have to do daily intermittent fasting to see the
benefit? Again, there hasn't been a study that's shown exactly how many
days you need to fast, but a recent study in rodents showed they get
about the same benefits fasting five days per week as they did fasting
every day. ``The great thing is that we're learning is that this type of
fasting isn't all or nothing,'' said Dr. Peterson.

Advertisement

\protect\hyperlink{after-bottom}{Continue reading the main story}

\hypertarget{site-index}{%
\subsection{Site Index}\label{site-index}}

\hypertarget{site-information-navigation}{%
\subsection{Site Information
Navigation}\label{site-information-navigation}}

\begin{itemize}
\tightlist
\item
  \href{https://help.nytimes3xbfgragh.onion/hc/en-us/articles/115014792127-Copyright-notice}{©~2020~The
  New York Times Company}
\end{itemize}

\begin{itemize}
\tightlist
\item
  \href{https://www.nytco.com/}{NYTCo}
\item
  \href{https://help.nytimes3xbfgragh.onion/hc/en-us/articles/115015385887-Contact-Us}{Contact
  Us}
\item
  \href{https://www.nytco.com/careers/}{Work with us}
\item
  \href{https://nytmediakit.com/}{Advertise}
\item
  \href{http://www.tbrandstudio.com/}{T Brand Studio}
\item
  \href{https://www.nytimes3xbfgragh.onion/privacy/cookie-policy\#how-do-i-manage-trackers}{Your
  Ad Choices}
\item
  \href{https://www.nytimes3xbfgragh.onion/privacy}{Privacy}
\item
  \href{https://help.nytimes3xbfgragh.onion/hc/en-us/articles/115014893428-Terms-of-service}{Terms
  of Service}
\item
  \href{https://help.nytimes3xbfgragh.onion/hc/en-us/articles/115014893968-Terms-of-sale}{Terms
  of Sale}
\item
  \href{https://spiderbites.nytimes3xbfgragh.onion}{Site Map}
\item
  \href{https://help.nytimes3xbfgragh.onion/hc/en-us}{Help}
\item
  \href{https://www.nytimes3xbfgragh.onion/subscription?campaignId=37WXW}{Subscriptions}
\end{itemize}
