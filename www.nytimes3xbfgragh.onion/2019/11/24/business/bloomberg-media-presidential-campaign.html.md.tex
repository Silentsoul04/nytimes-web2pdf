Sections

SEARCH

\protect\hyperlink{site-content}{Skip to
content}\protect\hyperlink{site-index}{Skip to site index}

\href{https://www.nytimes3xbfgragh.onion/section/business}{Business}

\href{https://myaccount.nytimes3xbfgragh.onion/auth/login?response_type=cookie\&client_id=vi}{}

\href{https://www.nytimes3xbfgragh.onion/section/todayspaper}{Today's
Paper}

\href{/section/business}{Business}\textbar{}Bloomberg News Sets Out How
It Will Cover Its Owner

\url{https://nyti.ms/33cqN0B}

\begin{itemize}
\item
\item
\item
\item
\item
\end{itemize}

\begin{itemize}
\item
  \href{https://www.nytimes3xbfgragh.onion/live/2020/09/11/us/trump-vs-biden?action=click\&pgtype=Article\&state=default\&region=TOP_BANNER\&context=storylines_menu}{Election
  Updates}
\item
  \href{https://www.nytimes3xbfgragh.onion/interactive/2020/us/elections/election-states-biden-trump.html?action=click\&pgtype=Article\&state=default\&region=TOP_BANNER\&context=storylines_menu}{Paths
  to 270}
\item
  \href{https://www.nytimes3xbfgragh.onion/interactive/2019/us/elections/2020-presidential-election-calendar.html?action=click\&pgtype=Article\&state=default\&region=TOP_BANNER\&context=storylines_menu}{Key
  Dates}
\item
  \href{https://www.nytimes3xbfgragh.onion/interactive/2020/08/31/us/politics/vote-by-mail-deadlines.html?action=click\&pgtype=Article\&state=default\&region=TOP_BANNER\&context=storylines_menu}{Voting
  by Mail}
\item
  \href{https://www.nytimes3xbfgragh.onion/newsletters/politics?action=click\&pgtype=Article\&state=default\&region=TOP_BANNER\&context=storylines_menu}{Politics
  Newsletter}
\end{itemize}

Advertisement

\protect\hyperlink{after-top}{Continue reading the main story}

Supported by

\protect\hyperlink{after-sponsor}{Continue reading the main story}

\hypertarget{bloomberg-news-sets-out-how-it-will-cover-its-owner}{%
\section{Bloomberg News Sets Out How It Will Cover Its
Owner}\label{bloomberg-news-sets-out-how-it-will-cover-its-owner}}

A fraught moment for a newsroom that, its editor in chief says, will not
do in-depth investigations of Mike Bloomberg or any other Democrats
running for president.

\includegraphics{https://static01.graylady3jvrrxbe.onion/images/2019/11/24/business/24mikemedia1/merlin_164556327_0a64b28c-3087-4ef8-ade3-a4bb9a1a34fd-articleLarge.jpg?quality=75\&auto=webp\&disable=upscale}

\href{https://www.nytimes3xbfgragh.onion/by/marc-tracy}{\includegraphics{https://static01.graylady3jvrrxbe.onion/images/2018/02/20/multimedia/author-marc-tracy/author-marc-tracy-thumbLarge.jpg}}

By \href{https://www.nytimes3xbfgragh.onion/by/marc-tracy}{Marc Tracy}

\begin{itemize}
\item
  Published Nov. 24, 2019Updated Feb. 18, 2020
\item
  \begin{itemize}
  \item
  \item
  \item
  \item
  \item
  \end{itemize}
\end{itemize}

Get ready to cover the boss's presidential campaign, with some caveats.

That is the message roughly 2,700 journalists at Bloomberg L.P., the
financial data company owned in large part by
\href{https://www.nytimes3xbfgragh.onion/2020/02/18/podcasts/the-daily/michael-bloomberg-democrats.html}{Michael
Bloomberg}, received on Sunday morning after Mr. Bloomberg, the former
mayor of New York City,
\href{https://www.nytimes3xbfgragh.onion/2019/11/24/us/politics/michael-bloomberg-2020-presidency.html}{formally
announced his candidacy for president as a Democrat}.

``We will write about virtually all aspects of this presidential contest
in much the same way as we have done so far,'' John Micklethwait,
Bloomberg Editorial and Research's editor in chief, said in the memo, in
which he always referred to Mr. Bloomberg simply as ``Mike.''

``We will describe who is winning and who is losing,'' Mr. Micklethwait
added. ``We will look at policies and their consequences. We will carry
polls, we will interview candidates and we will track their campaigns,
including Mike's. We have already assigned a reporter to follow his
campaign (just as we did when Mike was in City Hall). And in the stories
we write on the presidential contest, we will make clear that our owner
is now a candidate.''

But, the memo said, Bloomberg's outlets, which also include Bloomberg
Businessweek and several industry-specific sites, will not do in-depth
investigations of Mr. Bloomberg --- or any of his Democratic rivals.

\includegraphics{https://static01.graylady3jvrrxbe.onion/images/2019/11/24/business/24mikemedia2/merlin_75492425_c485c6f2-2f79-49d2-a4ac-dbfe3cf256ed-articleLarge.jpg?quality=75\&auto=webp\&disable=upscale}

On Sunday morning, the main Bloomberg website featured
\href{https://www.bloomberg.com/news/articles/2019-11-24/michael-bloomberg-joins-crowded-2020-democratic-field}{an
article} by Mark Niquette about Mr. Bloomberg's entry into the ``crowded
2020 Democratic field.''

This unusual policy of avoiding in-depth investigations of the
Democratic field echoes the similarly unusual way that the outlet
covered Mr. Bloomberg's 12-year tenure in City Hall as well as its
practices regarding rivals of Bloomberg L.P. As an internal guide
\href{https://www.nytimes3xbfgragh.onion/2014/09/08/business/media/bloomberg-news-stands-out-with-editorial-policy-to-not-report-on-itself.html}{instructs}:
``Bloomberg News doesn't originate stories about the company'' or cover
Mr. Bloomberg's ``wealth or personal life.'' (But, the memo added,
Bloomberg would not change its coverage of President Trump so long as he
is not a direct rival of Mr. Bloomberg's.)

Mr. Micklethwait also said that several journalists in the opinion
section would take leaves of absences to join Mr. Bloomberg's campaign.
They include Timothy L. O'Brien, the executive editor of Bloomberg
Opinion, and David Shipley, the senior executive editor. Mr. Shipley was
previously an editor at The New York Times opinion section, and Mr.
O'Brien, also a former Times editor and reporter, is known for his 2005
biography of President Trump. The section's unsigned editorials will go
on hiatus, the memo said, and a
\href{https://www.bloomberg.com/opinion/articles/2019-11-24/a-note-from-bloomberg-opinion}{note}
on the Bloomberg Opinion website said it would not accept outside op-ed
articles about the campaign.

MSNBC also confirmed that Mr. O'Brien would no longer be a network
contributor while he is working on the Bloomberg campaign.

The moment is fraught for one of the most prominent global newsrooms in
the country, which now has to document the candidacy of its owner, one
of the richest men in the world --- one who has mused about selling his
holdings if he ran for president. Mr. Bloomberg
\href{https://www.thewrap.com/mike-bloomberg-says-hell-sell-his-media-company-if-he-runs-for-president/}{said}
last year: ``I don't want the reporters I'm paying to write a bad story
about me. I don't want them to be independent.''

Mr. Micklethwait acknowledged the uneasy relationship in his memo.
``There is no point in trying to claim that covering this presidential
campaign will be easy,'' he said, ``for a newsroom that has built up its
reputation for independence in part by not writing about ourselves (and
very rarely about our direct competitors).''

\includegraphics{https://static01.graylady3jvrrxbe.onion/images/2017/01/29/podcasts/the-daily-album-art/the-daily-album-art-articleInline-v2.jpg?quality=75\&auto=webp\&disable=upscale}

\hypertarget{listen-to-the-daily-michael-bloombergs-not-so-secret-weapon}{%
\subsubsection{Listen to `The Daily': Michael Bloomberg's Not-So-Secret
Weapon}\label{listen-to-the-daily-michael-bloombergs-not-so-secret-weapon}}

The media tycoon and former New York mayor has paid his way into a
position of influence in the Democratic Party. But can he buy a
presidential nomination?

transcript

Back to The Daily

bars

0:00/34:22

-34:22

transcript

\hypertarget{listen-to-the-daily-michael-bloombergs-not-so-secret-weapon-1}{%
\subsection{Listen to `The Daily': Michael Bloomberg's Not-So-Secret
Weapon}\label{listen-to-the-daily-michael-bloombergs-not-so-secret-weapon-1}}

\hypertarget{hosted-by-michael-barbaro-produced-by-luke-vander-ploeg-and-edited-by-wendy-dorr}{%
\subsubsection{Hosted by Michael Barbaro, produced by Luke Vander Ploeg,
and edited by Wendy
Dorr}\label{hosted-by-michael-barbaro-produced-by-luke-vander-ploeg-and-edited-by-wendy-dorr}}

\hypertarget{the-media-tycoon-and-former-new-york-mayor-has-paid-his-way-into-a-position-of-influence-in-the-democratic-party-but-can-he-buy-a-presidential-nomination}{%
\paragraph{The media tycoon and former New York mayor has paid his way
into a position of influence in the Democratic Party. But can he buy a
presidential
nomination?}\label{the-media-tycoon-and-former-new-york-mayor-has-paid-his-way-into-a-position-of-influence-in-the-democratic-party-but-can-he-buy-a-presidential-nomination}}

\begin{itemize}
\item
  michael barbaro\\
  From The New York Times, I'm Michael Barbaro. This is ``The Daily.''
\item
  {[}music{]}\\
  Today: Despite a late entry into the Democratic presidential race,
  Michael Bloomberg has surged in the polls and is winning key
  endorsements before he's even on the ballot. Alex Burns on the hidden
  infrastructure of influence and persuasion behind Bloomberg's campaign
  and the dilemma that it poses for Democrats.
\item
  michael barbaro\\
  It's Tuesday, February 18.

  So Alex, Happy Valentine's Day.
\item
  alexander burns\\
  Happy Valentine's Day.
\item
  michael barbaro\\
  Thank you. You know, in honor of Valentine's Day, I want you to watch
  this ad, Alex, and read what's being played on the screen.
\item
  alexander burns\\
  OK. It says, ``Roses are red. Violets are blue. Your associates are
  criminals. What does that make you? America deserves better. Defeat
  Trump.'' And it says, ``Paid for by Mike Bloomberg 2020.''
\item
  michael barbaro\\
  Alex, what are we seeing here?
\item
  alexander burns\\
  Well, we're seeing the kind of in-your-face taunting of the
  president's former associates who are in jail or about to go to jail.
  People like Roger Stone and Paul Manafort. Their faces sort of
  surrounded by hearts and flowers. That all adds up to this
  crowd-pleasing taunt directed at the president by this campaign that
  has enough money to do anything.
\item
  michael barbaro\\
  What kind of money are we talking about?
\item
  alexander burns\\
  Well, we're talking about at least \$400 million so far. That's an
  astonishing sum of money for any campaign, let alone a campaign that
  has only existed for about three months. And that \$400 million has
  gone heavily into TV advertising, a lot of digital advertising, a lot
  of polling, an enormous amount of staff that they say that they have
  over 2,000 people on staff right now. That's larger than most
  campaigns are in a general election. And this is a primary campaign.
\item
  michael barbaro\\
  And it is only February.
\item
  alexander burns\\
  Right, and our reporting is that Mike Bloomberg has essentially
  directed his campaign to spend whatever it takes.
\item
  michael barbaro\\
  And is all that spending working?
\item
  alexander burns\\
  Well, it's gotten him pretty far in not a lot of time. He has gone
  from not a candidate to somewhere in the double-digit range in the
  polls. He's either second or third in a lot of national polls,
  depending on what time period exactly it was taken in. He's caught up
  in some places to Joe Biden, who has obviously had a bit of a rough
  patch recently. And so if the theory of the Mike Bloomberg campaign
  was, spend whatever it takes in the winter so that if Joe Biden takes
  a dive, you're the guy waiting, I would say that strategy has worked
  pretty well so far.
\item
  michael barbaro\\
  Right. And I'm reminded of the fact that it has worked without Mike
  Bloomberg ever even showing up on the debate stage, which is normally
  how a candidate rises in the polls.
\item
  alexander burns\\
  In some ways, it might work because Mike Bloomberg hasn't showed up on
  the debate stage yet. And I'm not just saying that to be sort of
  snarky, that he has managed to create this idealized version of
  Michael Bloomberg ---
\item
  archived recording (michael bloomberg)\\
  Anyone hear the slogan, ``Mike will get it done?''
\end{itemize}

alexander burns

--- and broadcast it to the country ---

\begin{itemize}
\tightlist
\item
  archived recording (michael bloomberg)\\
  Let me tell you what ``it'' is.
\end{itemize}

alexander burns

--- with these hundreds of millions of dollars in television ads.

\begin{itemize}
\tightlist
\item
  archived recording (michael bloomberg)\\
  As president, I'll offer common sense plans, and I will get it done.
  So let's stay on the offensive.
\end{itemize}

michael barbaro

So this idea that Michael Bloomberg is trying to buy the Democratic
primary --- or at least buy his way into it --- it's not untrue, at this
point.

alexander burns

No. His position in this race is totally inseparable from the amount of
money that he has spent on his campaign. We had two people dive into
this race at the 11th hour. One was Mike Bloomberg, with his \$400
million campaign. The other was Deval Patrick, the two-term governor of
Massachusetts, one of the few African-American governors in American
history. Mike Bloomberg is now somewhere in the mid-teens in the polls.
Deval Patrick dropped out after the New Hampshire primary. And in fact,
Mike Bloomberg, who was not on the ballot in New Hampshire, got more
write-in votes in that state than Deval Patrick got as a candidate
listed on the ballot.

michael barbaro

Wow.

alexander burns

It's the power of television advertising in a state that overlaps
heavily with TV markets that Michael Bloomberg is in all the time.

michael barbaro

Right.

alexander burns

But if you think his spending in this campaign has been consequential,
the picture is actually much broader and deeper than that.

{[}music{]}

alexander burns

He has been spending not hundreds of millions of dollars over months,
but billions of dollars over years to build a political and
philanthropic empire for himself, and a national profile and network of
influence that we are now seeing applied in so many ways in this race.

michael barbaro

And this, Alex, is the investigation that you have been working on these
past few weeks.

alexander burns

That's right. My colleague Nick Kulish and I have spent actually the
last couple months sort of combing through all the spending that Mike
Bloomberg has done in the political arena and in philanthropy,
essentially over the course of his public career, and trying to track
where that money has gone and what kind of friends it has made for him.

Mike Bloomberg is worth estimated around \$60 billion. It's a fortune
built on a financial information and news company. And he has used that
money to advance a whole range of causes that he personally cares about,
as well as the politicians who he sees as strong leaders for those
causes, for well over a decade at this point. Those activities have
accelerated dramatically since he left office as mayor.

michael barbaro

Which was 2013.

alexander burns

Right.

michael barbaro

And what are some examples of the causes and the characters who receive
this money?

alexander burns

He gives an enormous amount of money to causes like fighting climate
change, like advancing gun control policies. And more recently, like
electing Democrats for the sake of electing Democrats.

michael barbaro

And what's complicated about that?

alexander burns

Well in the context of an election, what's complicated is that you have
somebody who can make or break your cause or organization or campaign
with his personal checkbook.

{[}music{]}

So one example. At the end of September in 2018, Emily's List, the
premiere pro-choice women's Democratic group in the country, is hosting
a major fundraising luncheon in New York. Every announced speaker is a
prominent Democratic woman except for one, who's Mike Bloomberg. Shortly
before the luncheon, I am out in Seattle as Mike Bloomberg is engaging
in all kinds of public-minded activities. And we do an interview, and I
ask him about the \#MeToo movement. He is somebody who is trying to
position himself, now nationally, as a progressive.

\begin{itemize}
\tightlist
\item
  archived recording\\
  Tonight, new questions about Michael Bloomberg and the company he
  founded that made him a billionaire.
\end{itemize}

alexander burns

But he himself has been accused of making all kinds of crude remarks to
and about women.

\begin{itemize}
\item
  archived recording\\
  You have been accused in the past of making lewd and sexist comments,
  and fostering a frat-like culture at your company that was
  uncomfortable for some female employees. ABC has actually spoken to
  several women who want to share their stories, but you won't release
  them from their N.D.A.s. As Senator Warren put it --- she was here
  last week --- she said if your company has an enviable record, what do
  you have to hide? Why not ---
\item
  archived recording (michael bloomberg)\\
  We don't have anything to hide. But we made legal agreements, which
  both sides wanted to keep certain things from coming out. They have a
  right to do that.
\end{itemize}

alexander burns

As a corporate leader, he ran a company that has faced serious
discrimination allegations.

\begin{itemize}
\item
  archived recording (michael bloomberg)\\
  Did I ever tell a bawdy joke? Yeah sure, I did. And do I regret it?
  Yes, it's embarrassing. But you know, that's the way I grew up.
\item
  archived recording\\
  What kind of a joke?
\item
  archived recording (michael bloomberg)\\
  Bawdy.
\end{itemize}

alexander burns

So when I ask him about this, his response is to say that a lot of the
things that he has heard coming out of the \#MeToo movement are
alarming, but he doesn't know how true they are. And he specifically
raises the case of Charlie Rose, the disgraced news anchor, who for
years taped his show in the offices of Mike Bloomberg's company.

michael barbaro

And is a friend of Mike Bloomberg.

alexander burns

And is a personal friend of long standing. And he said that essentially,
he wasn't going to judge him one way or another until a court
adjudicated the issue. And then he acknowledged that a court would
probably never adjudicate the issue. This is not a sentiment in keeping
with the progressive moment at all. Certainly not in keeping with the
perspective of Emily's List.

michael barbaro

In fact, you might even say it's the kind of thing that would get a
person disinvited from a speaking slot at a luncheon for Emily's List.

alexander burns

And it very nearly did. What our reporting indicates is that Emily's
List was mortified by his comments in that interview, and that there was
a very serious internal debate about whether to ask him not to appear at
the luncheon. But that ultimately, the leaders of the organization
decided that he was somebody they could not afford to alienate. And what
ended up happening was, in late September, Mike Bloomberg got up at this
Emily's List luncheon in New York. In the background, the Kavanaugh
hearings are raging in Washington. And he announces that he is going to
spend more money helping elect women to office than anyone else in
history. He ends up spending more than \$100 million supporting
Democrats in the midterm elections, including helping elect 21 new
members to the House of Representatives. Of the 21 winning members he
supported, 15 were women.

michael barbaro

Wow.

alexander burns

So if there are things about his world view and his personal conduct and
his business record that make progressive women very uncomfortable,
there is nothing uncomfortable about the way he used his checkbook in
2018.

michael barbaro

What's another example of the complexity of Mike Bloomberg's money?

alexander burns

So I spoke to a number of people, current and former employees of the
Center for American Progress --- major a liberal think tank in
Washington, D.C. --- about a report that was being prepared in February
of 2015. It was about Islamophobia in America. And we reviewed a draft
of the report. And the draft included a whole chapter of more than 4,000
words about the New York Police Department under Mike Bloomberg and its
surveillance of Muslim Americans.

michael barbaro

This was a very invasive style of policing. It involved going into
mosques, getting into communities with undercover police officers. It
was highly unusual.

alexander burns

It was. And in the draft of this report, it was held up as an example of
really the most sophisticated, institutionalized form of Islamophobia
and discrimination in law enforcement.

michael barbaro

Pretty damning.

alexander burns

Except it was never published.

michael barbaro

Mm-hmm.

alexander burns

When the draft was submitted, one senior official at the Center for
American Progress pointed out that if it were published in its current
form, it would draw a strong response from Bloomberg's world. And when
the report was published, not only was the chapter entirely removed, but
Mike Bloomberg's name was never mentioned.

michael barbaro

Not once.

alexander burns

Not once.

michael barbaro

And does Mike Bloomberg give to the Center for American Progress? Or is
this about the very real possibility that someday he will give, or he
will give to causes near and dear to this group?

alexander burns

It's both. At the time the draft was submitted, Bloomberg had already
given nearly \$1.5 million in grants to CAP. Since then, he's given an
additional \$400,000, as far as we know.

michael barbaro

What do these two stories tell you?

alexander burns

What they tell you is that Mike Bloomberg's money has become this
gravitational force unlike any other in Democratic politics, where
people are tailoring their activities and their agendas to try to align
themselves with him, even if he's not explicitly asking them to. In the
case of that CAP report, the folks on the Mike Bloomberg side said they
were never aware of this report being prepared or the controversy around
its preparation. We have no reason to believe ---

michael barbaro

It's self-censorship.

alexander burns

Right. We have no reason to believe, based on our reporting, that anyone
who works directly for Bloomberg said you need to remove that chapter.
But the chapter was removed.

michael barbaro

In other words, the Democratic world wakes up every day with some level
of fear that they might offend Mike Bloomberg and a desire to please
him.

alexander burns

And this is what you're now seeing happen on a national scale in this
campaign. It's not that you've seen some massive stampede of support
towards Mike Bloomberg within the Democratic establishment. But what you
have seen is people who may have been critical of him in the past, when
he was a Republican, not necessarily speaking up so loudly in this
campaign. You have seen people who might be offended by elements of his
record as mayor coming out and either endorsing him for president or
calling him a very plausible contender for this Democratic nomination.
And you have seen, of course, on the strength of all that television
advertising, Mike Bloomberg tell a story about who he is and what he has
done with his money that Democratic voters seem to find pretty
appealing.

{[}music{]}

michael barbaro

We'll be right back.

{[}music{]}

michael barbaro

So, Alex, just how much money are we actually talking about? What did
the investigation that you and Nick did find about the scale of this? I
covered City Hall for many years when Bloomberg was mayor. And when I
covered City Hall, we actually did an investigation into how much money
the mayor was giving away during that period. And I remember the number
being pretty staggering.

alexander burns

It was. It was in the range of \$260 million. And what Nick and I found
in this investigation is that at the time that story was written, the
true number was about 10 times that.

michael barbaro

I'm sorry?

alexander burns

Yeah, it was about 10 --- look, you did a great job with the information
that was available.

michael barbaro

{[}LAUGHTER{]}

alexander burns

In the course of reporting this story though, what became clear to us
was that there is the philanthropy that we know of --- that's disclosed
on tax forms or that's announced publicly. And then there's other kinds
of philanthropy that Bloomberg engages in that is much, much harder to
track down. In fact, impossible to track down without cooperation of
some kind from Mike Bloomberg himself. And what it all adds up to is
more than \$10 billion ---

michael barbaro

Wow.

alexander burns

--- of giving on philanthropy and spending on politics. And it's
overwhelmingly weighted towards philanthropy. So when we think of the
spending on this campaign, or on his mayoral campaigns or on the
Democratic Party, it is a fairly small fraction of the money that Mike
Bloomberg gives away as a matter of course.

michael barbaro

I think I understand the power of that philanthropic giving from the
examples you gave around Emily's List and the Center for American
Progress. Can you give us a little bit more a sense of his giving to
candidates?

alexander burns

Well, for years his political giving was really agenda-driven and
issue-driven more than it was driven by the party label associated with
a candidate. What our reporting showed is that Bloomberg has spent about
\$270 million on a pair of organizations, one that kind of grew out of
the other. The first is called Mayors Against Illegal Guns, which he
operated from within City Hall. After the Sandy Hook shooting, that
morphed into a group known as Everytown. Now, arguably, the anchor gun
control group in the country. And what that money has gone to over the
years has been a range of causes associated with the gun issue, whether
it's Democratic candidates, Republican candidates, litigation related to
gun laws, helping lawmakers in different states and cities draft gun
laws and then defend them in court, providing cities and states that are
thinking of passing gun control laws with data to justify it. And for
years, he was a sort of party-neutral donor. If you were with him on
guns, he was with you. And most of the people who were with him on guns
were Democrats. But if you were, for instance, Senator Pat Toomey ---
very conservative Republican from Pennsylvania, who agreed to introduce
background checks legislation in the Senate --- Mike Bloomberg ended up
spending almost \$12 million in the 2016 election helping Toomey get
re-elected.

michael barbaro

Wow. And is it clear to you that as with Mike Bloomberg's other
philanthropic donations, that these political donations have resulted in
the same kind of reticence to cross him, to challenge him?

alexander burns

You know, one of the most vivid anecdotes that we heard about the
gratitude the candidates feel to Mike Bloomberg came from just a couple
weeks ago, when he went to Capitol Hill to ask lawmakers there for their
support in the presidential race. And normally, when a presidential
candidate goes up to the Hill, they kind of go hat-in-hand, meeting with
members of Congress and saying, could I please have your support? In at
least one meeting that we heard about with the new Democrat coalition
--- it's a group of moderates in the Democratic caucus --- Bloomberg
sits down and the lawmakers go around the table introducing themselves
to him. And according to two people who were in the room, one after
another began by saying, thank you so much. You spent this much in my
race. Or you supported me in my last two elections. And many of those
people have not endorsed him. But most of them haven't endorsed somebody
else. And the Bloomberg advisers that we have spoken to have again
stressed that we have never promised or implied that you will get
support in the future in exchange for an endorsement. But every
Democratic member of Congress in a tough race can look at what Mike
Bloomberg did for the party in 2018 and draw their own conclusions.

michael barbaro

So how are we going to see all of this, everything that you've been
describing here, play out in the Democratic primary as Bloomberg
actually starts showing up on the ballot?

alexander burns

Well, what we've already seen in the Democratic primary is that as he's
gone around the country, he has been able to reliably find important
partners to appear with him or host him, who have often benefited from
his philanthropy or his political spending in the past. Take the San
Francisco Bay Area. You have had, just in that one metro, Mike Bloomberg
has spent on school board races, on ---

michael barbaro

School board races?

alexander burns

School board races. On ballot initiatives to tax soda and ban
e-cigarettes. That's all political spending. And from his philanthropy,
he has given out dozens of grants to museums and dance companies and
climate organizations. And the mayor of San Francisco, one of the most
prominent politicians in the state, has endorsed him for president. Now,
we can't say that Mayor London Breed endorsed Mike Bloomberg just
because he has put all this money into San Francisco. But for a mayor
who's thinking about who to support, it's just an inescapable factor to
think about. That this is a guy who has underwritten not just
politicians here, but other institutions that make up kind of the civic
and cultural backbone of the city. None of the other candidates is
really competing for this kind of support right now in the same way.
Because they have been tied up in Iowa and New Hampshire, and now Nevada
and South Carolina. And when they have been doing that, Mike Bloomberg
has been jumping across the country on a private jet, meeting with
prominent local officials, many of them prominent African-American
officials, to ask them to consider supporting his campaign. And he has
done that actually with quite a bit of success so far.

michael barbaro

And why is it important that he's cultivating black elected officials?

alexander burns

I think there are two big reasons. One of them is that his own record as
mayor is going to be troubling to a lot of African-American voters. His
policies on law enforcement, his view of invasive searches, largely
targeting black and Hispanic men.

michael barbaro

Stop-and-frisk.

alexander burns

Stop-and-frisk. So if he is going to reassure African-Americans, who are
the cornerstone of the Democratic Party, that he's an acceptable
candidate, he needs to convince leaders within the black community to
help him make that case.

\begin{itemize}
\tightlist
\item
  archived recording (michael tubbs)\\
  {[}APPLAUSE{]} Good morning. Yes, welcome to Stockton. My name is
  Michael Tubbs. I have the honor of being the mayor of the great city
  of Stockton. I'm excited ---
\end{itemize}

alexander burns

When I was in California with Bloomberg late last year, he got the
endorsement of the young black mayor of Stockton ---

\begin{itemize}
\tightlist
\item
  archived recording (michael tubbs)\\
  We have to have a candidate with the record, with the resources and
  the relationships, to not just make sure we beat Donald Trump, but
  make sure something like Donald Trump never happens again.
\end{itemize}

alexander burns

--- who is a very progressive guy on basically everything, who stood
next to Bloomberg. And when they got a question about stop-and-frisk,
said ---

\begin{itemize}
\tightlist
\item
  archived recording (michael tubbs)\\
  I think we all recognize now in 2019 that stop-and-frisk is not a good
  policy. It's terrible. Courts have decided, he's apologized and moved
  on. I think two things for me. Number one: If you look at every
  candidate in the field, there's an issue with criminal justice. You
  have folks who wrote the `94 crime bill, which created mass
  incarceration. You have folks who voted for the `94 crime bill. You
  have folks who supported Ronald Reagan. So there's not a candidate in
  2019 who has a criminal justice record that's where we are today. So I
  think the ---
\end{itemize}

alexander burns

--- nobody in this campaign has a perfect record on law enforcement. And
Mike Bloomberg has the resources and the record to defeat President
Trump.

{[}music{]}

michael barbaro

I want a pause on this. Because for a black mayor to be asked about
stop-and-frisk and to say, eh, no one's perfect. The reason that would
seem quite striking is that stop-and-frisk was a policing policy deemed
unconstitutional, as practiced in New York under the Bloomberg
administration, that disproportionately targeted black and Latino men.
And by many accounts, not just humiliated them, but kind of
generationally scarred the black community in New York.

alexander burns

Right. And it might get tougher as you see more and more video and audio
of Mike Bloomberg talking about stop-and-frisk, as we have seen in the
last few days.

\begin{itemize}
\tightlist
\item
  archived recording\\
  A scratchy recording of former New York City Mayor Michael Bloomberg
  has been circulating on social media since Tuesday, appearing to show
  the billionaire presidential candidate defending the city's
  stop-and-frisk policy in starkly racial terms. Quote, ``We put all the
  cops in minority neighborhoods. Yes, that's true. Why did we do it?
  Because that's where all the crime is.''
\end{itemize}

alexander burns

A lot of these mayors don't particularly want to be on the hook for
defending every offensive or divisive turn of phrase that Mike Bloomberg
has used over the years. But at least on the surface level, there have
been enough signals of reassurance --- he has apologized for the policy
--- that these mayors who see him as an ally on a whole lot of other
things, have been willing to get on his side in the primary.

{[}music{]}

Because the only path for a moderate candidate to the Democratic
nomination is at least with some significant support from black voters.
Right now, one of the biggest questions in the race is whether Joe Biden
is going to hang on to the support that he has had from
African-Americans all along, or whether, as he struggles in these
largely white early states, those voters are going to look elsewhere.
And what Mike Bloomberg hopes is that they will look to him.

michael barbaro

And from everything you're saying, underlying that strategy of reaching
out to African-American leaders, African-American elected officials, and
then kind of taking them from Joe Biden is this kind of philanthropic
juggernaut of Mike Bloomberg.

alexander burns

It's being able to indicate, or explicitly say, that he has been their
friend on a lot of other things. And a lot of other things can involve
gun control. It can involve public education or public health. But a
whole lot of that comes out of the activities of Mike Bloomberg's
foundation and the activities of his political vehicles.

michael barbaro

So this is what you were talking about earlier, Alex. This dilemma that
Democrats are facing. It's the same one that progressive groups have
been facing for years, groups like Center for American Progress. How
much can they overlook, in terms of Mike Bloomberg's record, in exchange
for getting this extraordinary financial support in causes that they
believe in and that he believes in?

alexander burns

Right. If you believe that the Democratic Party is a party deeply
concerned about economic inequality and about the disproportionate
concentration of wealth and power in the hands of a small number of
people, it seems really incongruous to think that Mike Bloomberg could
be a plausible candidate for the nomination, even setting aside issues
like stop-and-frisk or his conduct with women.

michael barbaro

But!

alexander burns

But for a lot of Democratic voters, in an election that they regard as a
national emergency, if the only question is who can beat Trump, people
are at least apparently open to the idea that a guy with \$60 billion,
who has shown that he knows how to use it, might be that candidate.

michael barbaro

And not just a billionaire willing to use that money. But a billionaire
who represents their agenda, right? I mean, there has for a long time
been this anxiety within the Democratic Party, within progressive
America, over concentration of wealth. Over the power of the oligarchy.
But if the oligarch is their oligarch, and spends all his money on their
causes, then that really, really complicates this, right?

alexander burns

Absolutely. He has put before Democratic voters a pretty clear
proposition, which is that on the one hand, they may be offended by his
wealth and a number of other things about him. But if they're offended
by his wealth, do they care more about that or about climate change,
which he has used his wealth to try to address? In some ways, it's not a
perfect comparison. But you could think about it as related to the
choice that Republican voters made in embracing Donald Trump. This was a
guy who, in almost every outward respect, should have been completely
offensive to socially conservative voters who ultimately embraced him.
And they embraced him because they decided, if we are going to live in
this morally degraded, libertine country where our values are under
assault, well, this guy may be a Hollywood celebrity libertine, but he
is our Hollywood celebrity libertine.

michael barbaro

Right. There's a pragmatism do it. He was going to do their bidding. And
effectively has, when it comes to things like conservative judges on the
Supreme Court.

alexander burns

Absolutely. So if you're convinced that your best option to save the
Supreme Court, to deal with climate change, to deal with gun violence,
is by embracing a guy who's just going to buy his way to victory on that
stuff, well, buying your way to victory might be a more desirable
proposition than losing.

michael barbaro

You know, it strikes me that this would become an even more complicated
issue if Bloomberg --- and this is by no means assured. In fact, it
still seems quite improbable. But, who knows? --- if he becomes the
nominee and becomes president. Because then you would have a president
who would be among the nation's premier philanthropists, political
donors. And there would always be a question hovering over why someone
supports him, why someone is endorsing his legislation, showing up at
the White House for a signing ceremony, right? And there would always be
this sense that he would probably level internal opposition within the
world of the Democrats, because they would want access to his money.

alexander burns

And potentially level opposition on the other side, too. Because we've
never had a president who was willing and able to say, if you vote for
my bill, I will help you win re-election. And if you don't, I will spend
\$100 million dollars trying to defeat you.

{[}music{]}

michael barbaro

And his message would be, yeah, I'm really rich. So you just have to
trust me. And you have to trust my moral compass.

alexander burns

It's what's in his ads that we've been talking about. He's putting it in
front of people right now, this idea that he's a self-made billionaire
who has put that money to causes that you, the Democratic primary
electorate, care about. It doesn't take some great feat of imagination
to get voters thinking about what that would mean in the general
election and what that could mean for a president.

{[}music{]}

michael barbaro

Thank you, Alex.

alexander burns

Thank you.

{[}music{]}

\begin{itemize}
\tightlist
\item
  archived recording (joe biden)\\
  \$60 billion can buy you a lot of advertising. But it can't erase your
  record. There's a lot to talk about with Michael Bloomberg. You're not
  ---
\end{itemize}

michael barbaro

Over the weekend, after largely ignoring Mike Bloomberg for weeks, the
top Democratic presidential candidates, including Joe Biden on ``Meet
the Press,'' began attacking him for his past statements and policies,
especially stop-and-frisk.

\begin{itemize}
\tightlist
\item
  archived recording (joe biden)\\
  His position on issues relating to the African-American community,
  from stop-and-frisk to the way he talked about Obama, I mean ---
\end{itemize}

michael barbaro

At a rally in Las Vegas, Bernie Sanders delivered a similar attack.

\begin{itemize}
\tightlist
\item
  archived recording (bernie sanders)\\
  Regardless of how much money a multibillionaire candidate is willing
  to spend on his election, we will not create the energy and excitement
  we need to defeat Donald Trump if that candidate pursued, advocated
  for and enacted racist policies like stop-and-frisk, which caused
  communities of color in his city to live in fear.
\end{itemize}

{[}music{]}

michael barbaro

We'll be right back.

{[}music{]}

Here's what else you need to know today. More than 1,000 former federal
prosecutors and Justice Department officials are calling on Attorney
General William Barr to resign over his intervention in the sentencing
of Roger Stone, a friend of the president's who was convicted of lying
to Congress. In a letter released over the weekend, the officials
accused Barr and Trump of quote, ``interference in the fair
administration of justice.'' Federal prosecutors had originally
recommended a sentence of seven to nine years for Stone, prompting Trump
to lash out at the Department of Justice and Barr to overrule the
prosecutors and seek a shorter sentence.

That's it for ``The Daily.'' I'm Michael Barbaro. See you tomorrow.

Michael Grynbaum contributed reporting.

\hypertarget{our-2020-election-guide}{%
\section{Our 2020 Election Guide}\label{our-2020-election-guide}}

Updated ~Sept. 11, 2020

\begin{center}\rule{0.5\linewidth}{\linethickness}\end{center}

\begin{itemize}
\item ~
  \hypertarget{the-latest}{%
  \subsection{The Latest}\label{the-latest}}

  \begin{itemize}
  \item
    Joe Biden and President Trump put
    \href{https://www.nytimes3xbfgragh.onion/2020/09/11/us/politics/shanksville-trump-biden.html?action=click\&pgtype=Article\&state=default\&region=BELOW_MAIN_CONTENT\&context=storylines_guide}{hostilities
    on hold today to travel to ground zero and then to Shanksville, Pa.,
    where they separately honored 9/11 victims}.
  \end{itemize}
\item ~
  \hypertarget{how-to-win-270}{%
  \subsection{How to Win 270}\label{how-to-win-270}}

  \begin{itemize}
  \item
    Joe Biden and Donald Trump need 270 electoral votes to reach the
    White House. Try building
    \href{https://www.nytimes3xbfgragh.onion/interactive/2020/us/elections/election-states-biden-trump.html?action=click\&pgtype=Article\&state=default\&region=BELOW_MAIN_CONTENT\&context=storylines_guide}{your
    own coalition of battleground states}~to see potential outcomes.
  \end{itemize}
\item ~
  \hypertarget{voting-by-mail}{%
  \subsection{Voting by Mail}\label{voting-by-mail}}

  \begin{itemize}
  \item
    Will you have enough time to vote by mail in your state? Yes, but
    it's risky to procrastinate.
    \href{https://www.nytimes3xbfgragh.onion/interactive/2020/08/31/us/politics/vote-by-mail-deadlines.html?action=click\&pgtype=Article\&state=default\&region=BELOW_MAIN_CONTENT\&context=storylines_guide}{Check
    your state's deadline.}
  \item
    \href{https://www.nytimes3xbfgragh.onion/interactive/2020/us/elections/joe-biden.html?action=click\&pgtype=Article\&state=default\&region=BELOW_MAIN_CONTENT\&context=storylines_guide}{}

    \hypertarget{joe-biden}{%
    \section{Joe Biden}\label{joe-biden}}

    \hypertarget{democrat}{%
    \subsection{Democrat}\label{democrat}}

    \href{https://www.nytimes3xbfgragh.onion/interactive/2020/us/elections/donald-trump.html?action=click\&pgtype=Article\&state=default\&region=BELOW_MAIN_CONTENT\&context=storylines_guide}{}

    \hypertarget{donald-trump}{%
    \section{Donald Trump}\label{donald-trump}}

    \hypertarget{republican}{%
    \subsection{Republican}\label{republican}}
  \end{itemize}
\item
  \hypertarget{keep-up-with-our-coverage}{%
  \subsection{Keep Up With Our
  Coverage}\label{keep-up-with-our-coverage}}

  \begin{itemize}
  \item
    Get an
    \href{https://www.nytimes3xbfgragh.onion/newsletters/politics?action=click\&pgtype=Article\&state=default\&region=BELOW_MAIN_CONTENT\&context=storylines_guide}{email}~recapping
    the day's news
  \item
    Download our mobile app on
    \href{https://apps.apple.com/us/app/nytimes/id284862083?ls=1\&mat_click_id=5c79ae7455014fd1bd66b5610c05b8f2-20191112-16948\&referrer=mat_click_id\%3D5c79ae7455014fd1bd66b5610c05b8f2-20191112-16948\%26link_click_id\%3D722930677036718082}{iOS}~and
    \href{http://a.localytics.com/android?id=com.nytimes.android\&referrer=utm_source\%3Dother_nyt_mobile_web\%26utm_medium\%3DWeb\%2520page\%26utm_term\%3DGeneral\%2520Mobile\%2520Page\%26utm_campaign\%3DNYT\%2520Mobile\%2520General\%2520Page}{Android}~and
    turn on Breaking News and Politics alerts
  \end{itemize}
\end{itemize}

Advertisement

\protect\hyperlink{after-bottom}{Continue reading the main story}

\hypertarget{site-index}{%
\subsection{Site Index}\label{site-index}}

\hypertarget{site-information-navigation}{%
\subsection{Site Information
Navigation}\label{site-information-navigation}}

\begin{itemize}
\tightlist
\item
  \href{https://help.nytimes3xbfgragh.onion/hc/en-us/articles/115014792127-Copyright-notice}{©~2020~The
  New York Times Company}
\end{itemize}

\begin{itemize}
\tightlist
\item
  \href{https://www.nytco.com/}{NYTCo}
\item
  \href{https://help.nytimes3xbfgragh.onion/hc/en-us/articles/115015385887-Contact-Us}{Contact
  Us}
\item
  \href{https://www.nytco.com/careers/}{Work with us}
\item
  \href{https://nytmediakit.com/}{Advertise}
\item
  \href{http://www.tbrandstudio.com/}{T Brand Studio}
\item
  \href{https://www.nytimes3xbfgragh.onion/privacy/cookie-policy\#how-do-i-manage-trackers}{Your
  Ad Choices}
\item
  \href{https://www.nytimes3xbfgragh.onion/privacy}{Privacy}
\item
  \href{https://help.nytimes3xbfgragh.onion/hc/en-us/articles/115014893428-Terms-of-service}{Terms
  of Service}
\item
  \href{https://help.nytimes3xbfgragh.onion/hc/en-us/articles/115014893968-Terms-of-sale}{Terms
  of Sale}
\item
  \href{https://spiderbites.nytimes3xbfgragh.onion}{Site Map}
\item
  \href{https://help.nytimes3xbfgragh.onion/hc/en-us}{Help}
\item
  \href{https://www.nytimes3xbfgragh.onion/subscription?campaignId=37WXW}{Subscriptions}
\end{itemize}
