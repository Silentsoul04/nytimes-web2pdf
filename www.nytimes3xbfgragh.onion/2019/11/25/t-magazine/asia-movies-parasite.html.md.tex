Sections

SEARCH

\protect\hyperlink{site-content}{Skip to
content}\protect\hyperlink{site-index}{Skip to site index}

\href{https://myaccount.nytimes3xbfgragh.onion/auth/login?response_type=cookie\&client_id=vi}{}

\href{https://www.nytimes3xbfgragh.onion/section/todayspaper}{Today's
Paper}

Why Does Rage Define `Parasite' and Other Popular East Asian Movies?

\url{https://nyti.ms/2DbXuAD}

\begin{itemize}
\item
\item
\item
\item
\item
\item
\end{itemize}

Advertisement

\protect\hyperlink{after-top}{Continue reading the main story}

Supported by

\protect\hyperlink{after-sponsor}{Continue reading the main story}

Notes on the Culture

\hypertarget{why-does-rage-define-parasite-and-other-popular-east-asian-movies}{%
\section{Why Does Rage Define `Parasite' and Other Popular East Asian
Movies?}\label{why-does-rage-define-parasite-and-other-popular-east-asian-movies}}

Many thriller and horror films from Japan, China and South Korea reveal
a complicated relationship between those societies and the ancient
tenets of Confucianism.

\includegraphics{https://static01.graylady3jvrrxbe.onion/images/2019/12/05/t-magazine/05tmag-rage/05tmag-rage-articleLarge.jpg?quality=75\&auto=webp\&disable=upscale}

By
\href{https://www.nytimes3xbfgragh.onion/by/thessaly-la-force}{Thessaly
La Force}

\begin{itemize}
\item
  Published Nov. 25, 2019Updated Nov. 26, 2019
\item
  \begin{itemize}
  \item
  \item
  \item
  \item
  \item
  \item
  \end{itemize}
\end{itemize}

\emph{Editor's note: This article contains spoilers for ``Parasite.''}

THE CENTRAL OBJECT in the director
\href{https://www.nytimes3xbfgragh.onion/2019/10/30/movies/bong-joon-ho-parasite.html}{Bong
Joon Ho}'s newest film,
``\href{https://www.nytimes3xbfgragh.onion/watching/titles/movies/890255}{Parasite},''
which won the Palme d'Or at this year's Cannes Film Festival, is a
\emph{suseok}, or an ornamental rock. Scholar's rocks, as they are also
called, represent the unity of humans and the cosmos as venerated in
Confucianism. They are formed by nature into aesthetically pleasing
shapes --- and, as we soon learn in ``Parasite,'' are harbingers of good
luck. The film opens with the Kim family, who live in a basement
apartment on a dead-end street in Seoul. They are broke and unemployed,
resorting to folding pizza boxes for a nearby restaurant to make money.
By chance, the son of the family, Ki-woo, is visited by a former
classmate, Min, who is quitting his gig as an English tutor to a wealthy
schoolgirl to study abroad and wants Ki-woo to take over. Before
leaving, Min gives the Kims a suseok that once belonged to his
grandfather. Fortune is such an abstract idea to the struggling Kims
that Ki-woo's mother, Chung-sook, wonders why Min couldn't have just
brought them something to eat instead.

And yet, whether through coincidence or the rock's ancient powers, the
moment the suseok enters the Kims' lives, their luck changes. Ki-woo
arrives at the home of the Park family, a Modernist palace set in the
upper-class Seongbuk-dong neighborhood situated high above the rest of
the city. Going by the name Kevin, he shows the sweet and gullible Mrs.
Park a doctored diploma, which she dismisses --- personal
recommendations matter more than paperwork. Her other child is a
difficult and artistic little boy, and Ki-woo slyly suggests that Mrs.
Park hire an art tutor he knows, Jessica, who is in fact his sister,
Ki-jung. Soon, through a series of subtle deceptions and maneuvers, the
Kims infiltrate every part of the Park household's staff: Their father,
Ki-taek, takes over as chauffeur after the Kims lead the Parks to
believe that their previous driver is a sexual deviant; their mother
replaces the housekeeper, Moon-gwang, after the Kims convince the Parks
(falsely, of course) that Moon-gwang has tuberculosis.

\includegraphics{https://static01.graylady3jvrrxbe.onion/images/2019/11/27/t-magazine/27tmag-parasite/merlin_162276381_cd15b563-be04-4bd5-9906-9fc0ffe8e6ad-articleLarge.jpg?quality=75\&auto=webp\&disable=upscale}

In this way, the Kims turn the Parks into a financial life raft, and
their scheme seems perfectly sound until they discover an even
lower-class leech living among them: the former housekeeper's husband,
who, hunted by loan sharks, has been stashed away by his wife in a
subbasement that even the Parks aren't aware exists. The room is heavily
symbolic --- the poorer the person is in ``Parasite,'' the farther
underground he dwells --- and yet it is also a practicality: Rooms such
as this, we are told, are a common amenity in wealthy homes, a safeguard
against nuclear attack, perhaps, or a place to hide your worst secrets.
This discovery throws the Kims' plans into disarray and, like Chekhov's
gun, the suseok returns, not as a symbol of fortune but as a weapon,
setting off an explosion of violence with a Shakespearean-level death
toll.

If a desire for wealth propels ``Parasite,'' then class differences are
the film's foundation. Mrs. Park is ``nice because she's rich,'' says
Chung-sook, observing what money actually affords people. And yet the
suseok is a metaphor for something more ancient --- the Confucian
philosophy that still influences South Korean society, a place where
fundamental beliefs about obedience and respect have been manipulated to
create a highly wealthy and functional economy, one in which women are
not considered equal to men and where there is an ever-widening divide
between rich and poor: the result of a relentless pursuit of rapid
economic growth.

Image

A still from the Korean director Kim Jee-woon's 2003 psychological
horror film, ``A Tale of Two Sisters.''Credit...Courtesy of Kino Lorber

Like South Korean cinema, the staples of East Asian and some Southeast
Asian cinema are steeped in florid personal vengeance narratives ---
from
\href{https://www.nytimes3xbfgragh.onion/1998/09/07/movies/akira-kurosawa-film-director-is-dead-at-88.html}{Akira
Kurosawa}'s 1957 ``Throne of Blood,'' an adaptation of Macbeth, to Kim
Ki-young's psychosexual 1960 thriller ``The Housemaid'' (and its equally
disturbing 2010 remake by Im Sang-soo) to the vengeful ghosts of Japan's
1998 horror film
``\href{https://www.nytimes3xbfgragh.onion/watching/titles/movies/20369}{Ringu}''
to the ultraviolence of
\href{https://www.nytimes3xbfgragh.onion/2017/10/16/t-magazine/park-chan-wook.html}{Park
Chan-wook}'s vengeance trilogy, which includes the acclaimed 2003 movie
``\href{https://www.nytimes3xbfgragh.onion/watching/recommendations/oldboy}{Oldboy}.''
These films are among the most violent and gruesome in cinematic
history: gothic spectacles of anger and obsession. They present families
and relationships that seemingly obey the tenets of a harmonious
society. But eventually, something goes wrong, harmony is disrupted and
violence ensues. All of the films contain elements of exoticism:
Submissive women are seduced; a man eats a live octopus. These details
reveal, in part, why these movies surprise and delight American
audiences. But below the surface is a deeper rupture. These movies both
reinforce certain Confucian values and simultaneously combat stereotypes
about Asians: that they are obedient, dutiful, loyal, timid and fearful.
Of course, nothing could be further from the truth.

Just as
\href{https://www.nytimes3xbfgragh.onion/topic/person/alfred-hitchcock}{Alfred
Hitchcock} invented an entirely new genre of film by channeling European
wartime anxiety, films such as ``Parasite'' challenge globalization and
its effects. The Park family displays their wealth not just in their
ability to afford a full-time staff but also in their embrace of Western
culture. Mr. Park works for a multinational company; Mrs. Park casually
drops English words into her speech. Yet what powers the story is the
profound rage that runs beneath all the characters' lives, an infection
about to erupt.

Image

Choi Min-sik in ``Oldboy'' (2003).Credit...Courtesy of the Film Society
of Lincoln Center

CONFUCIANISM ORIGINATED IN ancient China with the scholar and
philosopher Confucius, who was born in 551 B.C. After being formally
adopted as a political ideology during the Han dynasty (from 206 B.C. to
A.D. 220), a golden age of learning and law whose influence lasted for
nearly two millenniums, it traveled east, first to Korea and then Japan,
by means of its own popularity but also the dominance of the Chinese
Empire. Confucianism proposes the idea that people are fundamentally
good, that we are capable of improving ourselves through education and
self-cultivation. It emphasizes loyalty, sacrificing one's own goals and
satisfaction in order to maintain traditional hierarchies and the status
quo: A citizen is faithful to his country, the son to the father, the
wife to her husband, the younger brother to his older brother. In more
contemporary times, the philosophy has re-emerged as a political
ideology: In 2013,
\href{https://www.nytimes3xbfgragh.onion/topic/person/xi-jinping}{President
Xi Jinping} of China
\href{https://sinosphere.blogs.nytimes3xbfgragh.onion/2013/11/26/xi-pays-homage-to-confucius-a-figure-back-in-favor/}{made
a pilgrimage to Qufu}, Confucius's hometown, and promised to make ``the
past serve the present.'' But it has also occasionally been used ---
much in the way democratic ideals are employed to promote a neoliberal,
Western agenda --- to justify the larger mechanics of political
maneuvering. On the one hand, it's surprising that these East Asian
societies that so value obedience should have perfected the revenge
narrative in popular culture, though on the other, it isn't at all: When
the idea of obedience is used to justify authoritarian governments and
socially rigid hierarchies, rebellion is never far-off.

But why is cinema, in particular, such a powerful tool for telling
stories of rage and revenge? The contemporary literature of East and
Southeast Asia also touches on these topics: The 2007 South Korean novel
``\href{https://www.nytimes3xbfgragh.onion/2016/02/03/books/the-vegetarian-a-surreal-south-korean-novel.html}{The
Vegetarian},'' by
\href{https://www.nytimes3xbfgragh.onion/2016/05/18/books/han-kang-wins-man-booker-international-prize-for-fiction-with-the-vegetarian.html}{Han
Kang}, tells of a wife's revulsion to meat that upends her place in
society; the short stories of the Japanese writer
\href{https://www.nytimes3xbfgragh.onion/2018/10/30/t-magazine/japanese-stories-books.html}{Taeko
Kono}, whose violent fantasies of disemboweling toddlers can be
difficult to read, speak to a deep-seated rage of being an independent
woman in 1960s Japan. But fear is more easily manufactured with movies,
a visual medium that lends itself well to making the gruesome and
ridiculous seem possible.

Movies are also easier to export. Martial arts films of the '60s and
'70s required little in the way of dialogue --- the plot was advanced by
a well-choreographed fight. Similarly, these revenge films rely on a
lexicon of violence: Nearly every culture understands the danger of a
hidden gun, of looking into dark corners during the middle of the night.
And as disparate as these films can be, they've also created visual
tropes of their own: Eyeballs and ears are gouged with blunt objects,
people are shot point blank, people fling themselves from buildings.
Women --- thin and unsparing, tough and uninterested in sex --- often
take center stage. Sex, incidentally, is rarely a focal point, but when
it is, it is in service of character development or humor --- ``Buy me
drugs,'' Mrs. Park coos to her husband in ``Parasite'' in the middle of
the act, in a scene that is as bizarre as it is pathetic. By contrast,
in American horrors and thrillers, a woman who has been sexualized
onscreen is usually the first to die.

\includegraphics{https://static01.graylady3jvrrxbe.onion/images/2019/11/21/autossell/21tmag-touchofsin/21tmag-touchofsin-videoSixteenByNine3000.png}

THE REVENGE NARRATIVE of East Asian cinema is often rooted in the
breaking of tradition.
\href{https://sinosphere.blogs.nytimes3xbfgragh.onion/2013/10/18/q-a-jia-zhangke-on-his-new-film-a-touch-of-sin/}{Jia
Zhangke}'s 2013
``\href{https://www.nytimes3xbfgragh.onion/watching/titles/a-touch-of-sin}{A
Touch of Sin}'' examines what happens when individuals choose to
confront corruption and inequality. It tells four loosely intermingled
stories of a group of ordinary Chinese citizens; the first centers on
Dahai, a poor villager in Northern China's Shanxi Province, who is angry
that the village boss of the local coal mine hasn't fairly distributed
the profits from its sale. What follows is a classic sequence of
violence, in which Dahai, rifle in hand, enacts bloody revenge against
each person who has caused him distress --- from the coal mine owner to
the idiot farmer who savagely whips his horse. It's hard not to cheer
for Dahai, who represents a simple desire for equality, as he leaves a
path of bodies behind him --- here is someone who seems to be
broadcasting his anguish beyond his private enemies and onto society as
a whole.

Which is to say that the morality in ``A Touch of Sin,'' as in
``Parasite,'' is askew. This, too, has become one of the major emblems
in today's Asian cinema. Near the end of Chan-wook's 2005 ``Lady
Vengeance,'' a young woman named Lee Geum-ja, who has been wrongfully
imprisoned for 13 years for the death of a 5-year-old boy, finally has
the man actually responsible for the crime tied up before her. She
offers the assembled group of parents whose children were also murdered
by the man a choice: They can hand the case over to the detective (who
is also present) or they can solve the problem themselves. They choose
the latter, and the resulting scene is at once violent, cathartic,
therapeutic, restorative but also utterly grotesque and horrifying.

It's telling that most of these films, unlike most of Western cinema,
rarely incorporate an authority figure such as the police or a judge ---
if they do appear, it is often as an accessory. The fight for justice
nearly always happens on the individual level, but in the interest of a
shared goal of vengeance, which is both a repudiation of Confucianism as
well as an embrace of it. If Western films depict vigilantism as
romantic, East Asian films embrace the idea that the individual is
sometimes the best person to answer to his wrongs. Western horrors and
thrillers operate with and against Puritanical values --- evil is innate
and must be purged, purity is often defiled and can never be recovered.
But the Analects, an ancient text composed of ideas and sayings directly
attributed to Confucius, espouses the transformative power of virtue.
Nothing should be coerced, nothing forced. Confucius said: ``Not to mend
one's ways when one has erred is to err indeed.'' Justice is more
complex when one has been wronged, and when morality becomes
disconnected from a clear set of laws. In a Confucian society, where
there is no distinct sense of heaven or hell, where a deity will not
necessarily punish you for your sins and where citizens must ultimately
manage one another, these movies suggest a different course of action.
Violence is not necessarily immoral, if done for the right reasons. Just
be aware of what such actions ultimately do to one's self. As Confucius
also said: ``Recompense injury with justice, and recompense kindness
with kindness.''

Image

Im Su-jung as Su-mi in ``A Tale of Two Sisters''
(2003).Credit...Courtesy of Kino Lorber

THERE IS A Korean word, \emph{han}, that has been used to describe the
violence of Asian cinema. The word doesn't have an English equivalent
but encompasses feelings such as suffering, anger, resignation, grief,
pain, longing and revenge. The term became popular in the 1970s, as
Koreans advocated for a kind of cultural authenticity. But its origins
are from the Japanese occupation of Korea in the early 20th century,
when the Japanese art critic Yanagi Soetsu described the artworks of
Korea he admired as possessing a kind of ``beauty of sorrow.'' In the
'70s, the poet
\href{https://www.nytimes3xbfgragh.onion/1987/07/31/world/voice-of-dissent-in-south-korea-speaks-in-verse.html}{Kim
Chi Ha} likened han to a ``people eating monster,'' saying that
``accumulated han is inherited and transmitted, boiling in the blood of
the people.'' ``Han'' may be a distinctly Korean term, but it is the one
that best describes contemporary Asian cinema writ large as it attempts
to reckon the present with its past --- it stands for a collective
trauma, a larger idea of suffering that can move through generations and
settles into the bedrock of history. Today, the idea of loyalty, of
obedience and self-improvement, can seem hopelessly outdated, as can the
idea of achieving a collective harmony in the face of poverty and greed.
Rage is a destructive emotion in this equation, but within art, it is
also radical and, in rare moments, elucidating. The best of these films
understand that the outcome of pitting people against one another can be
violent, that it will invariably end badly. But they also understand
that a repressive society can transform individuals into monsters.

In ``Parasite,'' none of the families involved are responsible for the
inequality of the society that has made their situations so different,
and neither are they necessarily best equipped to answer for it. These
films appeal to a need to confront a deeply inflexible world. They're
not interested in showing the hero's journey that results in both
victory and a personal transformation. Which is why we cheer for our
doomed protagonists even when we know that tragedy is inevitable. These
films make us recognize that our desires and our impulses --- our sense
of what is wrong and right, but also what we irrationally want --- are
often rooted in a past that can be hard to see, like the edges of a
riverbed from which a beautiful limestone rock was once lifted.

Advertisement

\protect\hyperlink{after-bottom}{Continue reading the main story}

\hypertarget{site-index}{%
\subsection{Site Index}\label{site-index}}

\hypertarget{site-information-navigation}{%
\subsection{Site Information
Navigation}\label{site-information-navigation}}

\begin{itemize}
\tightlist
\item
  \href{https://help.nytimes3xbfgragh.onion/hc/en-us/articles/115014792127-Copyright-notice}{©~2020~The
  New York Times Company}
\end{itemize}

\begin{itemize}
\tightlist
\item
  \href{https://www.nytco.com/}{NYTCo}
\item
  \href{https://help.nytimes3xbfgragh.onion/hc/en-us/articles/115015385887-Contact-Us}{Contact
  Us}
\item
  \href{https://www.nytco.com/careers/}{Work with us}
\item
  \href{https://nytmediakit.com/}{Advertise}
\item
  \href{http://www.tbrandstudio.com/}{T Brand Studio}
\item
  \href{https://www.nytimes3xbfgragh.onion/privacy/cookie-policy\#how-do-i-manage-trackers}{Your
  Ad Choices}
\item
  \href{https://www.nytimes3xbfgragh.onion/privacy}{Privacy}
\item
  \href{https://help.nytimes3xbfgragh.onion/hc/en-us/articles/115014893428-Terms-of-service}{Terms
  of Service}
\item
  \href{https://help.nytimes3xbfgragh.onion/hc/en-us/articles/115014893968-Terms-of-sale}{Terms
  of Sale}
\item
  \href{https://spiderbites.nytimes3xbfgragh.onion}{Site Map}
\item
  \href{https://help.nytimes3xbfgragh.onion/hc/en-us}{Help}
\item
  \href{https://www.nytimes3xbfgragh.onion/subscription?campaignId=37WXW}{Subscriptions}
\end{itemize}
