How Spices Have Made, and Unmade, Empires

\url{https://nyti.ms/37QUbgC}

\begin{itemize}
\item
\item
\item
\item
\item
\end{itemize}

\includegraphics{https://static01.graylady3jvrrxbe.onion/images/2019/11/27/t-magazine/27tmag-spice/27tmag-spice-articleLarge.jpg?quality=75\&auto=webp\&disable=upscale}

Sections

\protect\hyperlink{site-content}{Skip to
content}\protect\hyperlink{site-index}{Skip to site index}

Food Matters

\hypertarget{how-spices-have-made-and-unmade-empires}{%
\section{How Spices Have Made, and Unmade,
Empires}\label{how-spices-have-made-and-unmade-empires}}

From turmeric in Nicaragua to cardamom in Guatemala, nonnative
ingredients are redefining trade routes and making unexpected
connections across lands.

A chayote centerpiece surrounded by turmeric root, saffron threads,
whole cloves, cinnamon sticks and cardamom pods.Credit...Photo by Kyoko
Hamada. Styled by Suzy Kim

Supported by

\protect\hyperlink{after-sponsor}{Continue reading the main story}

By Ligaya Mishan

\begin{itemize}
\item
  Published Nov. 27, 2019Updated Nov. 29, 2019
\item
  \begin{itemize}
  \item
  \item
  \item
  \item
  \item
  \end{itemize}
\end{itemize}

IN THE HILLY Boaco region of central Nicaragua, the turmeric plants on
Celia Dávila and Gonzalo González's farm stand over four feet tall ---
thriving giants, although as natives of South and Southeast Asia,
they're actually newcomers to this land. Coffee once ruled these fields,
but as its price has grown unstable, smallholder farmers like Dávila and
González, 52 and 65, respectively, have had to turn to alternative
crops, among them this strange arrival that yields knobby rhizomes of
shocking orange flesh, rarely eaten unadulterated; instead, the
underground stems are dried and pulverized into a musky powder with a
throb of bitterness, which is most widely recognized worldwide as the
earthy base note and color in many Indian dishes. Nicaraguans have no
particular use for the spice, which has yet to make inroads in the local
diet. But
\href{https://cooking.nytimes3xbfgragh.onion/tag/turmeric}{Americans
do}, having suddenly and belatedly awakened to turmeric's health
benefits, some 3,000 years after they were first set down in the Atharva
Veda, one of Hinduism's foundational sacred texts.

It's a story at once old and new, a latter-day spice route making
unexpected connections between the grandmother in India,
\href{https://www.nytimes3xbfgragh.onion/2017/01/19/magazine/a-grandmothers-secret-turmeric-prescription.html}{stirring
turmeric into warm milk} for a sniffily child; the Goop acolyte in
California, sipping an après-yoga prepackaged
\href{https://www.nytimes3xbfgragh.onion/2018/01/05/t-magazine/matilda-goad-turmeric-elixir-recipe.html}{turmeric
``elixir,''} whose makers extol the ``body harmonizing'' powers of the
spice's key chemical compound, curcumin; and Dávila wielding a pickax in
rural Nicaragua. She is not alone in her embrace of this new harvest:
Farmers in Costa Rica, Hawaii and even Minnesota are planting turmeric
with an eye on an expanding market. Nor is turmeric the only spice to
flourish far from home. The food writer Max Falkowitz
\href{https://www.saveur.com/cardamom-trade-in-guatemala/}{has
documented} the work of small-scale farmers in Guatemala, mostly poor
and of indigenous descent, who now grow more than half the world's
cardamom, a crop that belonged for millenniums to India and was brought
to the Central American cloud forests by a German immigrant in the early
20th century. Cardamom is one of the most expensive spices --- so
valuable that all of it departs Guatemala for sale elsewhere. As with
turmeric in Nicaragua, its absence is hardly registered by local cooks,
to whom the spice is an interloper.

Spices were among the first engines of globalization, not in the modern
sense of a world engulfed by ever-larger corporations but in the ways
that we began to become aware, desirous even, of cultures other than our
own. Such desire, unchecked, once led to colonialism. After Dutch
merchants nearly tripled the price of black pepper, the British
countered in 1600 by founding the East India Company, a precursor to
modern multinationals and the first step toward the Raj. In the
following decades, the Dutch sought a monopoly on cloves, which once had
grown nowhere but the tropical islands of Ternate and Tidore in what is
today Indonesia, and then in 1652 introduced the scorched-earth policy
known as extirpation, felling and burning tens of thousands of clove
trees. This was both an ecological disaster and horribly effective: For
more than a century, the Dutch kept supplies low and prices high, until
a Frenchman (surnamed, in one of history's inside jokes, Poivre, or
``pepper'') arranged a commando operation to smuggle out a few
clove-tree seedlings. Among their ultimate destinations were Zanzibar
and Pemba, off the coast of East Africa, which until the mid-20th
century dominated the world's clove market.

The craving for spices still brings the risk of exploitation, both
economically, as farmers in the developing world see only a sliver of
the profits, and in the form of cultural appropriation. In the West,
we're prone to taking what isn't ours and acting as if we discovered it,
conveniently forgetting its history and context. Or else we reduce it to
caricature, cooing over turmeric-stained golden lattes while invoking
the mystic wisdom of the East. At the same time, a world without
borrowing and learning from our neighbors would be pallid and parochial
--- a world, in effect, without spice.

SPICES ARE LUXURIES, ornamental to our lives. They provide little
nutritional value and, beyond a few medicinal applications, are entirely
unnecessary to survival. What they offer is an escape from tedium --- a
reason to take joy in food beyond the baseline requirements of
existence. Where herbs are often chosen to complement and flatter the
ingredients they adorn, spices call attention to themselves,
transforming and sometimes even usurping a dish, so it becomes a mere
vehicle and excuse for spice itself. Roast spices in a pan before
cooking with them, as is done in India, and they seize the air, the
fragrance like a liberated genie.

There's righteous bemusement in India over newly converted Americans
proselytizing on behalf of turmeric. For centuries, the West ignored it.
Other spices from the East were coveted and fetishized, launching a
thousand ships, notably
\href{https://www.nytimes3xbfgragh.onion/2015/01/29/t-magazine/cinnamon-st-lucia-lamb-curry-recipe.html}{cinnamon},
nutmeg, black pepper and cloves. But turmeric languished, overshadowed
by its cousin ginger, punchy and sweet, and coming off the worse in its
superficial kinship to lofty
\href{https://www.nytimes3xbfgragh.onion/2015/12/30/dining/saffron-iran.html}{saffron},
the world's most expensive spice, although the two share little beyond
the ability to turn whatever they touch the color of gold. (This hasn't
stopped spice sellers and cooks throughout the ages from trying to pass
off turmeric as a cheaper saffron; even its scientific name, curcuma,
comes from an Arabic word that originally meant saffron, \emph{kurkum},
a wistful reminder of its status, in Western eyes, as a dupe.)
Meanwhile, \emph{haldi}, ``turmeric'' in Hindi, manifests in over 95
percent of Indian dishes, according to the Delhi-based food writer
Marryam H. Reshii, who writes in
``\href{http://marryamhreshii.com/about-marryamhreshii/books/the-flavour-of-spice/}{The
Flavour of Spice}'' (2017) that its absence in cooking ``is often
considered blasphemous or at least idiosyncratic.''

\includegraphics{https://static01.graylady3jvrrxbe.onion/images/2019/11/27/t-magazine/27tmag-spice-02/27tmag-spice-02-articleLarge.jpg?quality=75\&auto=webp\&disable=upscale}

Then again, the West has always been late to the party, sidelined
geographically from the bounty of the East. Many of the spices used in
Western cooking come from the seeds, bark, roots, rhizomes, flowers and
fruits of plants born in Asia. Traders brought cloves north from
Southeast Asia to Han dynasty China, where courtiers were not allowed to
speak to the emperor unless their breath had been purified by cloves
(known as ``chicken-tongue spice''); and to arid Arabia, where in the
1970s cloves were excavated, still intact, from a ceramic pot in a house
dating back to 1750 B.C. in the Babylonian city of Terqa in modern
Syria.

Not until Greek and Roman antiquity did the West learn of these
treasures, as Arab traders became the intermediaries between the
hemispheres. They tried to keep the origins of spices shrouded in
mystery to prevent customers from finding or planting them on their own;
in the fifth century B.C. the Greek historian Herodotus reported tales
of cassia gathered from a lake guarded by ``winged animals, much
resembling bats, which screech horribly, and are very valiant,'' and of
cinnamon sticks knocked out of the nests of enormous birds, both in
unknown Arabian locales. To the ancient Greeks, spices were ``the
product of an exceptional union between the earth and the fire of the
sun,'' the Belgian historian Marcel Detienne writes in
``\href{https://press.princeton.edu/books/paperback/9780691001043/the-gardens-of-adonis}{The
Gardens of Adonis: Spices in Greek Mythology}'' (1972) --- a literal
embodiment of their often tropical origins. They served as emblems of
all that lay beyond the known world, be that defined in terms of
geographic distance or the more nebulous passage between life and death;
the Greeks, Detienne argues, used spices ``to mediate between the near
and the far-away and to link the above and the below,'' notably in
funeral rites and sacred devotions. In one version of the phoenix myth,
when death finally looms after a thousand years, the bird readies a nest
of cinnamon and frankincense to help ensure its resurrection. During the
Roman Empire, Nero burned a year's supply of cinnamon at the funeral of
his second wife, Poppaea, perhaps regretting that, as recorded by early
historians, he himself had murdered her. (On a more earthly note, spices
were also employed as tools of seduction --- Caesar was reportedly
beguiled by the cinnamon wafting from Cleopatra's hair --- and served
practical purposes, mitigating the salt in preserved foods and masking
bad breath and odors from poor sanitation.)

The Romans eventually figured out how to bypass the middlemen to find
the sources of those spices themselves. Their yearning for these potent
scents and flavors drove them into the monsoon winds --- an advancement
in navigation skills --- toward India and its cache of black pepper. In
the first century A.D., pepper was ``bought by weight like gold or
silver,'' as recorded by the Roman historian Pliny the Elder, who
worried that the empire would squander its wealth on such spices. At its
height, a pound of pepper cost half a month's wages; Alaric the
Visigoth, on the verge of sacking Rome in 410 A.D., exacted 3,000 pounds
of black pepper as part of the city's ransom.

Pepper's value was sustained in Europe throughout the Middle Ages, as
landlords accepted peppercorns as rent and daughters were married off
with peppercorn dowries. Only in the mid-17th century did Europeans
begin to turn away from spices, in part because they had become more
readily accessible and lost their ability to confer status on those
wealthy enough to afford them, but also, as the historian T. Sarah
Peterson has argued, because of advances in science and medicine and a
new skepticism toward spices' supposed occult capabilities. The
historian W.E. Mead, writing in 1931 in
``\href{https://www.taylorfrancis.com/books/9780429202148}{The English
Medieval Feast},'' dismissed Middle Age diners as ``coarse eaters'' with
palates dulled from overexposure to spices ``by which the most innocent
meats and fruits were doctored and disguised until the cook himself
could hardly distinguish from the taste what had entered into their
composition.'' In the meantime, in the regions of the world where spices
were native, they simply continued to be part of the landscape and
culture, subjects of neither idolatry nor condemnation --- until
Europeans brought their new, more minimalist culinary standards to the
countries they colonized, suppressing indigenous cuisines and the very
ingredients they once fought wars over.

IN NICARAGUA, Dávila and González leave their turmeric plants in the
fields for two years instead of the typical six to 12 months, and that
longer gestation --- abetted by partial shade instead of direct sunlight
--- appears to have boosted the amount of curcumin in the rhizomes as
well as deepened their orange hue. Reshii's ``The Flavour of Spice''
reports a high of 6.5 percent curcumin in turmeric from Kerala, India,
compared to an average of 3 to 3.5 percent in the crop from nearby Tamil
Nadu; Nicaraguan turmeric has registered at 7.9 percent. It's ideal for
a market primarily interested in the spice for its curative rather than
culinary properties, even as the
\href{https://www.nytimes3xbfgragh.onion/2019/10/16/style/self-care/turmeric-benefits.html}{health
benefits of curcumin} remain unproven beyond a few preliminary clinical
trials that suggest its potential as an anti-inflammatory and an
antioxidant.

Dávila and González's current crop is still in the ground, but the
harvest of fellow farmers in their cooperative is available in the
United States through
\href{https://www.nytimes3xbfgragh.onion/2017/07/10/dining/spices-burlap-and-barrel.html?module=inline}{Burlap
\& Barrel}, a spice purveyor based in New York. Ethan Frisch, one of the
company's founders, visited the couple last spring and was intrigued to
find that they had no plans to use their turmeric in the kitchen. He
asked if they might try the leaves, if not the spice itself, the way
they use banana leaves, to wrap tamales. Eyebrows were raised. A crazy
notion, to change the way things have been done for hundreds of years.

It may take time, but the spice could still win converts here. Consider
what happened to nutmeg, which once grew only on the Banda Islands of
modern Indonesia and now flourishes on Grenada in the Caribbean. In the
early 17th century, the Dutch slaughtered Banda's indigenous inhabitants
to gain control of the spice; out of 15,000 natives, barely 1,000
remained. In London, nutmeg was marked up at more than 60,000 times its
Banda price. It was the Frenchman Poivre, again, who smuggled seedlings
to the West, where the spice eventually gained a second home in Grenada,
nearly 12,000 miles away. Today, it suffuses jams, cakes, ice cream, the
batter for fried fish and a syrup for basting chicken. It even holds
pride of place on the country's flag.

Note, however, that
\href{https://well.blogs.nytimes3xbfgragh.onion/2014/11/25/a-warning-on-nutmeg/}{nutmeg
is considered an intoxicant} and is classified by some Muslim jurists as
\emph{haram}, as it's laced with myristicin, which has hallucinogenic
properties, and safrole, a chemical sometimes used in synthesizing the
psychedelic MDMA. Malcolm X wrote in his 1964 autobiography of getting
high off nutmeg while in prison in Massachusetts in the 1940s --- ``a
penny matchbox full of nutmeg had the kick of three or four reefers''
--- and the spice was reportedly banned from New Jersey state prison
kitchens at one point. So the connections multiply: from 17th-century
Dutch colonialists to the Black Panthers of 1960s America, and farmers
in balmy Grenada --- even to the frantic crush and heave of Manhattan, a
bit of swampland once called New Amsterdam, which those same Dutchmen
saw fit to pawn off on the British in 1667 in exchange for Run, one of
the nutmeg-producing Banda Islands, and so tiny, it's barely visible on
a world map. The Dutch didn't care --- Run had nutmeg, after all. They
thought they'd got the better deal.

Photo assistant: Carlos J.

Advertisement

\protect\hyperlink{after-bottom}{Continue reading the main story}

\hypertarget{site-index}{%
\subsection{Site Index}\label{site-index}}

\hypertarget{site-information-navigation}{%
\subsection{Site Information
Navigation}\label{site-information-navigation}}

\begin{itemize}
\tightlist
\item
  \href{https://help.nytimes3xbfgragh.onion/hc/en-us/articles/115014792127-Copyright-notice}{©~2020~The
  New York Times Company}
\end{itemize}

\begin{itemize}
\tightlist
\item
  \href{https://www.nytco.com/}{NYTCo}
\item
  \href{https://help.nytimes3xbfgragh.onion/hc/en-us/articles/115015385887-Contact-Us}{Contact
  Us}
\item
  \href{https://www.nytco.com/careers/}{Work with us}
\item
  \href{https://nytmediakit.com/}{Advertise}
\item
  \href{http://www.tbrandstudio.com/}{T Brand Studio}
\item
  \href{https://www.nytimes3xbfgragh.onion/privacy/cookie-policy\#how-do-i-manage-trackers}{Your
  Ad Choices}
\item
  \href{https://www.nytimes3xbfgragh.onion/privacy}{Privacy}
\item
  \href{https://help.nytimes3xbfgragh.onion/hc/en-us/articles/115014893428-Terms-of-service}{Terms
  of Service}
\item
  \href{https://help.nytimes3xbfgragh.onion/hc/en-us/articles/115014893968-Terms-of-sale}{Terms
  of Sale}
\item
  \href{https://spiderbites.nytimes3xbfgragh.onion}{Site Map}
\item
  \href{https://help.nytimes3xbfgragh.onion/hc/en-us}{Help}
\item
  \href{https://www.nytimes3xbfgragh.onion/subscription?campaignId=37WXW}{Subscriptions}
\end{itemize}
