Sections

SEARCH

\protect\hyperlink{site-content}{Skip to
content}\protect\hyperlink{site-index}{Skip to site index}

\href{https://www.nytimes3xbfgragh.onion/section/health}{Health}

\href{https://myaccount.nytimes3xbfgragh.onion/auth/login?response_type=cookie\&client_id=vi}{}

\href{https://www.nytimes3xbfgragh.onion/section/todayspaper}{Today's
Paper}

\href{/section/health}{Health}\textbar{}Dr. Janette Sherman, 89, Early
Force in Environmental Science, Dies

\url{https://nyti.ms/2suE5ZP}

\begin{itemize}
\item
\item
\item
\item
\item
\end{itemize}

Advertisement

\protect\hyperlink{after-top}{Continue reading the main story}

Supported by

\protect\hyperlink{after-sponsor}{Continue reading the main story}

\hypertarget{dr-janette-sherman-89-early-force-in-environmental-science-dies}{%
\section{Dr. Janette Sherman, 89, Early Force in Environmental Science,
Dies}\label{dr-janette-sherman-89-early-force-in-environmental-science-dies}}

In one case, discovering that autoworkers shared the same diseases, she
pinpointed the cause as chemicals in the factories --- not, as was
thought, cigarettes.

\includegraphics{https://static01.graylady3jvrrxbe.onion/images/2019/12/01/obituaries/01Sherman-obit2/00Sherman2-articleLarge.jpg?quality=75\&auto=webp\&disable=upscale}

\href{https://www.nytimes3xbfgragh.onion/by/katharine-q-seelye}{\includegraphics{https://static01.graylady3jvrrxbe.onion/images/2020/03/23/reader-center/author-katharine-q-seelye/author-katharine-q-seelye-thumbLarge.png}}

By
\href{https://www.nytimes3xbfgragh.onion/by/katharine-q-seelye}{Katharine
Q. Seelye}

\begin{itemize}
\item
  Nov. 29, 2019
\item
  \begin{itemize}
  \item
  \item
  \item
  \item
  \item
  \end{itemize}
\end{itemize}

When Dr. Janette Sherman was practicing internal medicine in suburban
Detroit in the 1970s, she noticed that several of her patients were
reporting similar symptoms, and that they all worked in automobile
factories.

She soon realized that they were all being exposed to the same hazardous
chemicals, including arsenic. She shared her findings with the consumer
activist Ralph Nader's Health Research Group, and in 1973 they issued a
report on the health of 489 Detroit autoworkers.

Their jobs, the report said, were ``associated with increased amounts of
chronic bronchitis, chronic obstructive lung disease, or other disabling
and killing diseases.''

A key finding was that
nonsmokers\href{https://www.nytimes3xbfgragh.onion/1973/09/09/archives/heartlung-ills-tied-to-job-perils-study-challenges-ideas-on-the.html?searchResultPosition=2}{had
just as much chronic illness as smokers}. The nonsmokers also had a 50
percent greater chance of developing those diseases than nonsmokers
whose jobs did not expose them to the dust, smoke, fumes, chemicals and
exhaust from forklift trucks to be found in factories. Such diseases had
previously been attributed to cigarettes.

Dr. Sherman, who died on Nov. 7 at 89 in Alexandria, Va., ``made the
connection that this was not a lifestyle issue --- this was a work
issue,'' her daughter, Connie Bigelow, said. ``People were being made
sick by their work.''

Dr. Sherman testified on behalf of thousands of these autoworkers as
they sought compensation for their illnesses while pressing for cleaner
work environments, labeling of the hazardous materials they were working
with and regular monitoring of their health.

Through the efforts of the United Auto Workers union, many of these
changes came about.

As an internist, Dr. Sherman started out by treating the autoworkers.
But she shifted her focus to researching the causes of their illnesses
and trying to prevent them, becoming a pioneer in occupational and
environmental health.

A chemist by training, she took up toxicology and helped pinpoint how
hazardous substances, toxic chemicals and nuclear radiation could lead
to cancer, birth defects and other diseases. Some of the chemicals she
identified as particularly harmful have since been banned or restricted.

Over the course of her career Dr. Sherman served as a medical-legal
expert witness in more than 5,000 workers' compensation claims. Her
medical-legal files, among the largest collections of their kind in the
United States, are preserved at the National Library of Medicine at the
National Institutes of Health in Bethesda, Md.

In addition to testifying on behalf of workers, Dr. Sherman served as an
expert witness for residents in communities affected by environmental
hazards, most famously the
\href{https://www.nytimes3xbfgragh.onion/1984/10/30/nyregion/love-canal-a-look-back.html}{Love
Canal} neighborhood of Niagara Falls, N.Y. Developed in the 1950s atop a
toxic chemical landfill, the area became the site of one of the worst
environmental disasters in American history in the late 1970s, prompting
President Jimmy Carter to declare an emergency. Dr. Sherman was among
those urging that residents be evacuated, which they were.

She also studied the continuing health effects of the world's worst
nuclear disasters, in 1986 at the
\href{https://www.nytimes3xbfgragh.onion/2019/02/06/books/review-midnight-chernobyl-adam-higginbotham.html}{Chernobyl}
power plant in Ukraine and in 2011 at
the\href{https://www.britannica.com/event/Fukushima-accident}{Fukushima}
plant in Japan.

Her work often pitted her against powerful business and political
interests.

``She definitely went up against the corporate establishment,'' Ms.
Bigelow, her daughter, said. ``She was always on the side of the
worker.''

Sometimes she was ``threatened and hassled,'' Ms. Bigelow said. ``She
talked about being in hotel rooms with attorneys, and they'd turn on the
TV and the radio and the shower because they were afraid they were being
bugged.''

Ms. Bigelow said that her mother, as a woman in what was largely a man's
field, felt she had no room for error.

``She did very careful, very detailed research,'' she said. ``I suspect
there were some doctors who shunned her because of her work, but she
didn't care. She did what she thought was right.''

\includegraphics{https://static01.graylady3jvrrxbe.onion/images/2019/12/01/obituaries/01Sherman-obit3/00Sherman1-articleLarge.jpg?quality=75\&auto=webp\&disable=upscale}

Janette Dexter Miller was born in Buffalo on July 10, 1930, to Wilma and
Frank Miller. (Miller was also her mother's maiden name.) Both parents
were pharmacists. They divorced when Janette was a toddler, and her
mother raised her in Warsaw, N.Y., just east of Buffalo.

An athletic young woman, Janette planned to major in physical education
when she went to Western Michigan College of Education in Kalamazoo, now
Western Michigan University. But while there, she took a job in a
chemistry lab to help pay for school and became interested in science.
She ended up majoring in chemistry and biology and graduated in 1952.

That same year she entered into the first of her three marriages.

Dr. Sherman went on to Michigan State University in Lansing, where, from
1956 to 1960, she studied German and mathematics part time, though she
did not obtain an advanced degree. She then enrolled in medical school
at Wayne State University in Detroit, where she was one of only a
handful of women studying for a medical degree. She had recently been
divorced and was raising two children on her own at the time. She
graduated in 1964 and later set up her own private practice just north
of Detroit, where she first encountered the autoworkers.

Dr. Sherman, who was a professor of oncology and medicine at Wayne State
from 1976 to 1988, consulted with or served on a number of advisory
boards and government agencies, including the Environmental Protection
Agency and the National Cancer Institute.

She wrote two books, ``Chemical Exposure and Disease: Diagnostic and
Investigative Techniques'' (1988) and ``Life's Delicate Balance: Causes
and Prevention of Breast Cancer'' (2000).

She also edited ``Chernobyl: Consequences of the Catastrophe for People
and the Environment'' (2007), which analyzed thousands of articles in
the scientific literature and concluded that the Chernobyl disaster had
caused an estimated 985,000 premature deaths. That number far exceeded
previous estimates, the highest of which was about 50,000, and led to
criticism of the book in the academic press.

Dr. Sherman had studied the effects of radiation early in her career and
later worked with Joseph Mangano, executive director of the nonprofit
Radiation and Public Health Project. By analyzing the baby teeth of
children who lived near nuclear reactors, they suggested in five
peer-reviewed journal articles that even small doses of radiation had
caused increases in childhood cancer. Some scientists were skeptical,
saying no direct link could be proved.

``We were alone in doing the research but not alone in our concern that
radiation from nuclear reactors is getting into people's bodies and
harming them,'' Mr. Mangano said.

Dr. Sherman's first marriage, to John Bigelow in 1952, ended in divorce
in 1960. Her marriage to Howard Sherman in 1965 also ended in divorce,
in 1972. In 1987 she married her high school sweetheart, Donald
Nevinger. He died in 2005.

Dr. Sherman, who died at an assisted living community, had a combination
of dementia and Addison's disease, Ms. Bigelow said. In addition to Ms.
Bigelow, she is survived by her son, Charles Bigelow; two stepchildren,
Kevin Nevinger and Donna Kellogg; and five grandchildren.

At 56, Dr. Sherman took up the cello. ``It was a lifelong dream, and her
goal was to be last chair in a community orchestra,'' her daughter said.
She achieved that goal, playing with the all-volunteer symphony
orchestra in McLean, Va., for several years.

Advertisement

\protect\hyperlink{after-bottom}{Continue reading the main story}

\hypertarget{site-index}{%
\subsection{Site Index}\label{site-index}}

\hypertarget{site-information-navigation}{%
\subsection{Site Information
Navigation}\label{site-information-navigation}}

\begin{itemize}
\tightlist
\item
  \href{https://help.nytimes3xbfgragh.onion/hc/en-us/articles/115014792127-Copyright-notice}{©~2020~The
  New York Times Company}
\end{itemize}

\begin{itemize}
\tightlist
\item
  \href{https://www.nytco.com/}{NYTCo}
\item
  \href{https://help.nytimes3xbfgragh.onion/hc/en-us/articles/115015385887-Contact-Us}{Contact
  Us}
\item
  \href{https://www.nytco.com/careers/}{Work with us}
\item
  \href{https://nytmediakit.com/}{Advertise}
\item
  \href{http://www.tbrandstudio.com/}{T Brand Studio}
\item
  \href{https://www.nytimes3xbfgragh.onion/privacy/cookie-policy\#how-do-i-manage-trackers}{Your
  Ad Choices}
\item
  \href{https://www.nytimes3xbfgragh.onion/privacy}{Privacy}
\item
  \href{https://help.nytimes3xbfgragh.onion/hc/en-us/articles/115014893428-Terms-of-service}{Terms
  of Service}
\item
  \href{https://help.nytimes3xbfgragh.onion/hc/en-us/articles/115014893968-Terms-of-sale}{Terms
  of Sale}
\item
  \href{https://spiderbites.nytimes3xbfgragh.onion}{Site Map}
\item
  \href{https://help.nytimes3xbfgragh.onion/hc/en-us}{Help}
\item
  \href{https://www.nytimes3xbfgragh.onion/subscription?campaignId=37WXW}{Subscriptions}
\end{itemize}
