Sections

SEARCH

\protect\hyperlink{site-content}{Skip to
content}\protect\hyperlink{site-index}{Skip to site index}

\href{https://www.nytimes3xbfgragh.onion/section/world/europe}{Europe}

\href{https://myaccount.nytimes3xbfgragh.onion/auth/login?response_type=cookie\&client_id=vi}{}

\href{https://www.nytimes3xbfgragh.onion/section/todayspaper}{Today's
Paper}

\href{/section/world/europe}{Europe}\textbar{}How an Anti-Brexit London
District Could Help Boris Johnson Triumph

\url{https://nyti.ms/33AfhMT}

\begin{itemize}
\item
\item
\item
\item
\item
\end{itemize}

Advertisement

\protect\hyperlink{after-top}{Continue reading the main story}

Supported by

\protect\hyperlink{after-sponsor}{Continue reading the main story}

\hypertarget{how-an-anti-brexit-london-district-could-help-boris-johnson-triumph}{%
\section{How an Anti-Brexit London District Could Help Boris Johnson
Triumph}\label{how-an-anti-brexit-london-district-could-help-boris-johnson-triumph}}

With a big but divided pro-European vote, one of Labour's most glamorous
election targets risks becoming a symbol of opposition campaign failure.

\includegraphics{https://static01.graylady3jvrrxbe.onion/images/2019/12/01/world/01election-twocities1/merlin_160292061_22e8e685-6493-4e84-8d8e-a9e154cfd205-articleLarge.jpg?quality=75\&auto=webp\&disable=upscale}

\href{https://www.nytimes3xbfgragh.onion/by/benjamin-mueller}{\includegraphics{https://static01.graylady3jvrrxbe.onion/images/2018/02/20/multimedia/author-benjamin-mueller/author-benjamin-mueller-thumbLarge.jpg}}

By
\href{https://www.nytimes3xbfgragh.onion/by/benjamin-mueller}{Benjamin
Mueller}

\begin{itemize}
\item
  Published Nov. 30, 2019Updated Dec. 12, 2019
\item
  \begin{itemize}
  \item
  \item
  \item
  \item
  \item
  \end{itemize}
\end{itemize}

LONDON --- The Labour Party canvassers gathered after dark outside a
tube station in Pimlico, a pocket of central London that, by all
appearances, should be fertile terrain. Nearly three-quarters of the
surrounding district voted to stay in the European Union, among the
strongest ``Remain'' votes in Britain, putting the pro-Brexit
Conservatives at risk in a seat they had held since the district lines
were drawn in 1950.

But the district, the Cities of London and Westminster, with its rows of
white stucco townhouses and crowded housing projects, may now become a
parable on the left for why Prime Minister Boris Johnson holds a
commanding position less than two weeks before the election.

Brexit has sent tremors through the British political system, shaking up
the traditional left-right, class-based divisions. While the
Conservatives have capitalized on the upheaval, building a
\href{https://www.nytimes3xbfgragh.onion/2019/11/25/world/europe/uk-election-conservatives-labour.html}{coalition
of pro-Brexit voters} across regional and class lines, the left has so
far struggled to win converts and overcome its own divisions.

Mr. Johnson is on course for a 68-seat majority in Parliament, a
\href{https://www.thetimes.co.uk/article/mrp-election-poll-boris-johnson-heads-for-big-majority-qrqsq9f7r}{major
new polling analysis showed}, with Labour hemorrhaging pro-Brexit seats
in working-class sections of middle and northern England and a fractured
left failing to win significant numbers of anti-Brexit seats in the
south that seemed ripe for the taking.

\includegraphics{https://static01.graylady3jvrrxbe.onion/images/2019/12/01/world/01election-twocities2/merlin_160548351_244fb499-e69b-4b55-b52e-8712ee687ee3-articleLarge.jpg?quality=75\&auto=webp\&disable=upscale}

With Mr. Johnson still deeply unpopular, undecided voters may yet swing
Labour's way. Recent polls suggest the Conservative lead has begun to
shrink, putting many seats with razor-thin margins potentially in play.
But Labour's leftist leader, Jeremy Corbyn, has lately dug in against
accusations of anti-Semitism in the party and criticisms that his Brexit
policy was incoherent.

Setting off from the tube station last week, the scores of Labour
canvassers were quickly confronted with a treacherous political rip
tide: Labourites turned off by Mr. Corbyn; die-hard Remainers who, fed
up with Labour's ever-evolving stance on Brexit, had decamped to the
staunchly anti-Brexit Liberal Democrats; and even former Remainers who
now resignedly conceded that democracy demanded Brexit be done.

``It's desperate times --- it's very difficult to know how to vote,''
Philip Rudge, 73, who lives in the east of the district, said a few days
earlier. ``I've been Labour all my life, but I've been dismayed to see
the infighting and back-stabbing and so on. Corbyn's not a leader.
Labour will have to win an election against the leadership.''

This London district, known informally as the Two Cities, is in many
respects a mirror image of pro-Brexit, working-class Labour strongholds
in northern England being targeted by the Conservatives. Stocked with
bankers and lawyers who once made up the Conservative base, but who want
to stay in the European Union, the Two Cities is precisely the kind of
seat that Brexit could help deliver to a left-leaning party.

But with Mr. Corbyn failing to ignite the enthusiasm he did in 2017, and
some right-wing anti-Brexit voters drifting back into the Conservative
fold, the widely prophesied new coalition of the left has not
materialized.

Image

The Labour party's leftist leader, Jeremy Corbyn, has lately dug in
against accusations of anti-Semitism in the party and criticisms that
his Brexit policy was incoherent.Credit...Henry Nicholls/Reuters

In the Two Cities, the left is also suffering from a second problem: the
anti-Brexit vote being split between Labour and the Liberal Democrats, a
smaller, more centrist party that has stormed back from obscurity by
arguing for lawmakers to summarily reverse Brexit.

Anti-Brexit activists are pleading with people to vote tactically ---
meaning to vote for whichever Remain party stands the best chance of
winning a given seat --- and polls suggest that Britons are doing so in
greater numbers than before, for good reason. While there are roughly
half a dozen parties in Britain's Brexit-battered Parliament, only one
can win any given seat: When supporters of a given cause split their
votes between several candidates, they risk letting an opponent come
through the middle.

But disagreements on the economy and foreign policy still run deep on
the left. And with the Liberal Democrats neck-and-neck with Labour in
districts like the Two Cities, that has left even the most calculating
anti-Brexit voters confused about what to do.

``I would say I'm a tactical voter normally, but it's not clear at this
stage what the tactic should be,'' said Fern Watson, 36, who is opposed
to Brexit, bracing against the cold in the Barbican, a brutalist estate
on the eastern edge of the district. ``I don't really see either Labour
or the Lib Dems as my natural political home, and I think a lot of
people of my age and education level feel the same.''

Image

A polling station for the Brexit referendum in the Barbican in the City
of London in 2016.Credit...Andrew Testa for The New York Times

She had visited three different websites purporting to tell people how
to vote in individual precincts to stop Brexit. One of them said Labour,
and the other two the Liberal Democrats.

Current polling suggests the Remain vote will split in the Two Cities,
allowing a weakened Conservative candidate to hold the seat. Across the
country, were only 120,000 more Remainers to vote tactically,
\href{https://www.bestforbritain.org/new_tactical_voting_recommendations_updated_mrp_polling}{one
analysis showed}, that would be enough to defeat Mr. Johnson on Dec. 12.

But for now, in crucial London districts, the race has become a battle
of bar charts, as both Labour and the Liberal Democrats try to prove
they are best positioned to win three-way fights for seats. Labour has
printed reams of them showing how it cut into the Conservatives' lead in
the 2017 election, capitalizing on the same shifts that have turned
American cities into progressive bulwarks.

But the Liberal Democrats, relying on more recent polling, have
distributed their own sheafs of charts with exactly the opposite
message.

Couple that with the hazy mechanics of how a left-wing coalition would
actually try to stop Brexit, and Remain voters are stuck in a confusing
predicament.

``If you are a Leave voter, the route to your destination is now really
clear and simple,'' said Rob Ford, a politics professor and the editor
of ``Sex, Lies and Politics: The Secret Influences That Drive our
Political Choices.'' ``Whereas if you're on the Remain side, what's the
route to your desired destination? It's as clear as the channel on a
foggy day right now.''

Remain voters are torn by Mr. Corbyn's cautious, some would say muddled,
Brexit policy, in which he would negotiate a new exit deal with Brussels
and then put it beside Remain in a public vote in which he himself would
stay neutral.

One voter, Philip Jeremy, 60, asked about Labour's Brexit policy, said
bluntly: ``Corbyn doesn't have one.'' So desperate is Mr. Jeremy not to
see either major party steering the country that he said he wanted the
election to deliver no clear signals at all.

``I prefer a hung Parliament, just so none of them do anything too
drastic,'' Mr. Jeremy said.

Sitting as it does at the heart of London, the Two Cities district
covers not only Buckingham Palace and Parliament but also the
well-mannered homes of many senior lawmakers, making it a trophy scalp
for the opposition.

Image

The Two Cities district includes Parliament.Credit...Andrew Testa for
The New York Times

But it also has considerable areas of poverty, where allegiances to
Labour are strong and its message should resonate: The party has focused
heavily on health care, housing, climate change and income inequality.

Those policies have drawn some pro-Brexit voters into the fold, like
Jalil Abdul, 75, who has lived for four decades in Walden House, a
public housing block in Pimlico that had been
\href{https://www.mirror.co.uk/news/uk-news/council-tenants-win-fight-billionaire-20064135}{targeted
for redevelopment by a 28-year-old billionaire}.

``This year, I like the Labour Party,'' Mr. Abdul said, ``because for
the last three years the Conservative Party has failed at doing
anything.''

But polls suggest many anti-Brexit Conservatives are sticking by Mr.
Johnson, not out of love and admiration for him as much as fear and
loathing for his opponent, Mr. Corbyn.

``We have a choice of one of two prime ministers, either Boris, or
Jeremy Corbyn,'' said Christopher Wyke, 64, a Conservative who lives and
works in the City of London, the financial district, and who himself
supports Brexit. ``If you vote for anybody but the Conservatives, you
risk getting Corbyn, so there's no choice. Even people who are
Remainers, they still don't want Corbyn. He'd be infinitely more
dangerous.''

And the Liberal Democrats have alienated some voters who might otherwise
be amenable to their centrist economic policies by taking a stark
position on Brexit: revoking it altogether, without a public vote.

Even anti-Brexit Labourites are no longer a shoo-in to vote against the
Conservatives.

Gordon Nardell, the Labour candidate, broke off from the party activists
outside the tube station last week to knock on some doors alone. The
first answer seemed to startle him: a middle-aged man who said he was a
longtime Labour supporter and backed Remain in 2016, but now wanted Mr.
Johnson to get Brexit done.

``The vote was to leave, so you know, recognize the vote,'' the man
said. ``To me, once you vote, that's it --- you either accept it, or if
you don't accept it, democracy means nothing.''

Advertisement

\protect\hyperlink{after-bottom}{Continue reading the main story}

\hypertarget{site-index}{%
\subsection{Site Index}\label{site-index}}

\hypertarget{site-information-navigation}{%
\subsection{Site Information
Navigation}\label{site-information-navigation}}

\begin{itemize}
\tightlist
\item
  \href{https://help.nytimes3xbfgragh.onion/hc/en-us/articles/115014792127-Copyright-notice}{©~2020~The
  New York Times Company}
\end{itemize}

\begin{itemize}
\tightlist
\item
  \href{https://www.nytco.com/}{NYTCo}
\item
  \href{https://help.nytimes3xbfgragh.onion/hc/en-us/articles/115015385887-Contact-Us}{Contact
  Us}
\item
  \href{https://www.nytco.com/careers/}{Work with us}
\item
  \href{https://nytmediakit.com/}{Advertise}
\item
  \href{http://www.tbrandstudio.com/}{T Brand Studio}
\item
  \href{https://www.nytimes3xbfgragh.onion/privacy/cookie-policy\#how-do-i-manage-trackers}{Your
  Ad Choices}
\item
  \href{https://www.nytimes3xbfgragh.onion/privacy}{Privacy}
\item
  \href{https://help.nytimes3xbfgragh.onion/hc/en-us/articles/115014893428-Terms-of-service}{Terms
  of Service}
\item
  \href{https://help.nytimes3xbfgragh.onion/hc/en-us/articles/115014893968-Terms-of-sale}{Terms
  of Sale}
\item
  \href{https://spiderbites.nytimes3xbfgragh.onion}{Site Map}
\item
  \href{https://help.nytimes3xbfgragh.onion/hc/en-us}{Help}
\item
  \href{https://www.nytimes3xbfgragh.onion/subscription?campaignId=37WXW}{Subscriptions}
\end{itemize}
