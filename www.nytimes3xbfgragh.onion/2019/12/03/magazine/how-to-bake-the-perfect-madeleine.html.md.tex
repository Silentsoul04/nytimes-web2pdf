Sections

SEARCH

\protect\hyperlink{site-content}{Skip to
content}\protect\hyperlink{site-index}{Skip to site index}

\href{https://myaccount.nytimes3xbfgragh.onion/auth/login?response_type=cookie\&client_id=vi}{}

\href{https://www.nytimes3xbfgragh.onion/section/todayspaper}{Today's
Paper}

How to Bake the Perfect Madeleine

\url{https://nyti.ms/33Papn4}

\begin{itemize}
\item
\item
\item
\item
\item
\item
\end{itemize}

Advertisement

\protect\hyperlink{after-top}{Continue reading the main story}

Supported by

\protect\hyperlink{after-sponsor}{Continue reading the main story}

\href{/column/magazine-eat}{Eat}

\hypertarget{how-to-bake-the-perfect-madeleine}{%
\section{How to Bake the Perfect
Madeleine}\label{how-to-bake-the-perfect-madeleine}}

\includegraphics{https://static01.graylady3jvrrxbe.onion/images/2019/12/08/magazine/08mag-eat/08mag-eat-articleLarge.jpg?quality=75\&auto=webp\&disable=upscale}

By Dorie Greenspan

\begin{itemize}
\item
  Dec. 3, 2019
\item
  \begin{itemize}
  \item
  \item
  \item
  \item
  \item
  \item
  \end{itemize}
\end{itemize}

Early in my baking education, I bought a tinned-metal plaque with a
dozen shell-shaped indentations. It was an indulgent purchase, since the
pan was designed expressly to make just one pastry, a madeleine, a sweet
I had not only never baked but one I hadn't even tasted. Over the years,
I've accumulated a stack of madeleine pans; the original plaque shares
cupboard space with nonstick, silicone and mini-madeleine pans too.
They've all seen plenty of use, but never more than recently: Since
September, when I met Cédric Grolet, I've made, I think, at least 200 of
the little teacakes. Come for dinner, and the odds are good that the
last morsel I'll serve you will be a madeleine. A perfect madeleine.

Grolet,
\href{https://www.theworlds50best.com/stories/News/cedric-grolet-worlds-best-pastry-chef.html}{named
Best Pastry Chef in the World} last year by the World's 50 Best
Restaurants group, is almost as well known for his smile, playfulness
and stylized tattoos as he is for the stunning desserts he creates for
Le Meurice hotel in Paris, where we met. The day before, I went to his
patisserie around the corner from the hotel --- he has since opened a
large shop on the nearby Avenue de l'Opéra --- and bought a few of his
specialties, many of which are finished only when you order them, as
they would be in a restaurant, so that all the elements that are meant
to be crackly or chewy or warm or cool have a fighting chance of being
just that when you settle down to eat them.

\includegraphics{https://static01.graylady3jvrrxbe.onion/images/2019/12/08/magazine/08mag-eat-02/08mag-eat-02-articleLarge.jpg?quality=75\&auto=webp\&disable=upscale}

Among my stash was a pair of Grolet's fruits, a lemon and a fig. Like
the 34-year-old chef's million and a half other
\href{https://www.instagram.com/cedricgrolet/}{followers on Instagram},
I knew that when I cut them, their \emph{trompe l'oeil} shells would
break with a crisp snap, and there'd be a layer of cream and a tumble of
fruit --- cooked and jammed --- sliced in different sizes, offering
different textures. The Fabergé-like pastries, each tasting essentially
like the fruits that inspired them, reminded me of those Russian dolls
that reveal something new as each layer is lifted.

When I asked him about that cascade of flavors, Grolet said, completely
without artifice, ``I'm a pastry chef, good is easy for me.'' And then
he added that what he wanted was for his desserts to \emph{déranger}, to
astound people. I'd experienced how his intricate pastries could
disorient, but I wondered if there were rules of \emph{dérangement} for
a home baker like me. As he talked about his work, I made mental notes
about simplicity: Grolet uses the word ``simple'' to describe even his
most complex desserts; about focusing on a single flavor --- remarkably,
each of his pastries is built on one flavor; about concentrating on
ingredients, taste and texture and letting these, rather than extraneous
decorations, make a dessert beautiful. It was when Grolet asked if I'd
tasted his madeleines that I knew he'd found a way for me to understand
the principles that guide him. And that's how I went home committed to
madeleines, simplicity and \emph{dérangement}.

Madeleines are simple in every way, but one batch can be a master class
in baking. For starters, a madeleine has a specific look. You get the
lovely ribbed-shell shape with no effort --- the pan dictates the form.
But the cake must be golden brown, and the shell side should have a
light crust --- generously buttering the pan and dusting the interior
with flour gives you the crustiness. Then there's the characteristic
bump, the counterbalance to the roly-poly shell. The bump can be a
hillock or an Everest; it depends on the recipe (this one has baking
powder and the power to bump high) and something called oven spring: get
the batter really cold; get the oven really hot; get a big bump.

In general, the flavor of madeleines is mild, and that's true of the
Earl Grey madeleines that I've been perfecting. Their seductive flavor
--- and their aroma --- depend on a quartet of complementary
ingredients: Earl Grey tea (if you use a fine-quality loose tea rather
than the powdery leaves from a bag, you'll have better flavor); citrus
zest --- Grolet uses bergamot, the fruit that gives Earl Grey tea its
distinctive flavor, but it can be hard to find, so Meyer lemons or a
lemon and a clementine can stand in; honey; and browned butter. In
French, brown butter is called \emph{beurre noisette}, hazelnut butter,
and it's a good name to remember as you melt and color the butter for
the madeleines. The butter is fundamental to the flavor, and so you must
cook it until it is unequivocally brown and until you catch the whiff of
hazelnuts. It will have dark specks --- they're meant to be there, and
they're delicious.

Madeleines are essentially spongecakes, génoises to be specific, and the
key to obtaining a sublime sponge is patience and precision. It's most
important that none of the ingredients be cold: the eggs must be room
temperature and the butter-honey mixture and the milk must be warm.
Think cozy equilibrium. The fact that the butter is stirred into the
batter at the end is unusual, but vital. Trust tradition here. Also
trust that the texture of your madeleines will be better --- and the
bump bigger --- if you give the batter a long rest in the refrigerator
and use it when it's cold. Chilling the batter is convenient, too; with
batter in the fridge, it's possible to make just a couple or an entire
batch of madeleines in minutes and to serve them warm.

And warm is key. The fragrance is most present, the texture is most
supple and the pleasures are most pronounced when the madeleines are
freshly baked. Grolet serves his teacakes within five minutes of their
coming from the oven. And I do, too. Like his, my simple, toasty, brown
madeleines are now a quiet \emph{dérangement}.

\href{https://cooking.nytimes3xbfgragh.onion/recipes/1020684-earl-grey-madeleines}{Recipe:
Earl Grey Madeleines}

Advertisement

\protect\hyperlink{after-bottom}{Continue reading the main story}

\hypertarget{site-index}{%
\subsection{Site Index}\label{site-index}}

\hypertarget{site-information-navigation}{%
\subsection{Site Information
Navigation}\label{site-information-navigation}}

\begin{itemize}
\tightlist
\item
  \href{https://help.nytimes3xbfgragh.onion/hc/en-us/articles/115014792127-Copyright-notice}{©~2020~The
  New York Times Company}
\end{itemize}

\begin{itemize}
\tightlist
\item
  \href{https://www.nytco.com/}{NYTCo}
\item
  \href{https://help.nytimes3xbfgragh.onion/hc/en-us/articles/115015385887-Contact-Us}{Contact
  Us}
\item
  \href{https://www.nytco.com/careers/}{Work with us}
\item
  \href{https://nytmediakit.com/}{Advertise}
\item
  \href{http://www.tbrandstudio.com/}{T Brand Studio}
\item
  \href{https://www.nytimes3xbfgragh.onion/privacy/cookie-policy\#how-do-i-manage-trackers}{Your
  Ad Choices}
\item
  \href{https://www.nytimes3xbfgragh.onion/privacy}{Privacy}
\item
  \href{https://help.nytimes3xbfgragh.onion/hc/en-us/articles/115014893428-Terms-of-service}{Terms
  of Service}
\item
  \href{https://help.nytimes3xbfgragh.onion/hc/en-us/articles/115014893968-Terms-of-sale}{Terms
  of Sale}
\item
  \href{https://spiderbites.nytimes3xbfgragh.onion}{Site Map}
\item
  \href{https://help.nytimes3xbfgragh.onion/hc/en-us}{Help}
\item
  \href{https://www.nytimes3xbfgragh.onion/subscription?campaignId=37WXW}{Subscriptions}
\end{itemize}
