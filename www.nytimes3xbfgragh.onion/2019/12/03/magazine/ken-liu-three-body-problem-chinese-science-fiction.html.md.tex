Sections

SEARCH

\protect\hyperlink{site-content}{Skip to
content}\protect\hyperlink{site-index}{Skip to site index}

How Chinese Sci-Fi Conquered America

\url{https://nyti.ms/2OK8Um0}

\begin{itemize}
\item
\item
\item
\item
\item
\item
\end{itemize}

\includegraphics{https://static01.graylady3jvrrxbe.onion/images/2019/12/08/magazine/08-mag-liu/08-mag-liu-articleLarge.jpg?quality=75\&auto=webp\&disable=upscale}

Feature

\hypertarget{how-chinese-sci-fi-conquered-america}{%
\section{How Chinese Sci-Fi Conquered
America}\label{how-chinese-sci-fi-conquered-america}}

The translator Ken Liu has done more than anyone to bridge the gap
between Chinese science fiction and American readers.

Ken Liu outside his home in Stoughton, Mass.Credit...Amani Willett for
The New York Times

Supported by

\protect\hyperlink{after-sponsor}{Continue reading the main story}

By
\href{https://www.nytimes3xbfgragh.onion/by/alexandra-alter}{Alexandra
Alter}

\begin{itemize}
\item
  Dec. 3, 2019
\item
  \begin{itemize}
  \item
  \item
  \item
  \item
  \item
  \item
  \end{itemize}
\end{itemize}

\href{https://cn.nytimes3xbfgragh.onion/culture/20191206/ken-liu-three-body-problem-chinese-science-fiction/}{阅读简体中文版}\href{https://cn.nytimes3xbfgragh.onion/culture/20191206/ken-liu-three-body-problem-chinese-science-fiction/zh-hant/}{閱讀繁體中文版}

In the fall of 2012, Ken Liu received an intriguing offer from a Chinese
company with a blandly bureaucratic name: China Educational Publications
Import and Export Corporation, Ltd. It was seeking an English-language
translator for a trippy science-fiction novel titled ``The Three-Body
Problem.'' Liu --- an American computer programmer turned corporate
lawyer turned science-fiction writer --- was a natural choice: fluent in
Mandarin, familiar with Chinese sci-fi tropes and culture and a rising
star in the genre. Liu had only translated short fiction at the time,
though, and capturing the novel in all its complexity seemed daunting.

``The Three-Body Problem'' was unlike anything Liu had ever read. A
mind-bending epic set in Beijing, Inner Mongolia and on a distant
planet, the novel was full of heady technical passages about quantum
theory, nanotechnology, orbital mechanics and astrophysics, intertwined
with profound moral questions about the nature of good and evil and
humanity's place in the universe.

But as he began translating, Liu was confronted by what seemed like a
more fundamental problem: The narrative structure didn't make sense. The
story careered around in time, bouncing between present-day China, as a
panic builds among scientists and government officials over a coming
alien invasion, and Beijing in 1967, near the start of the Cultural
Revolution, when an astrophysicist watches helplessly as her father, a
physics professor, is killed by members of Mao's Red Guard for being a
``reactionary academic authority.'' The astrophysicist loses faith in
humanity and uses a high-power radio transmitter to broadcast a defiant
message to aliens in a nearby solar system, an act that has dire
consequences.

Studying the novel's chaotic timeline, Liu pinpointed what he felt was
the story's natural beginning: the scenes of political violence and
oppression during the Cultural Revolution, a traumatic moment that
triggers the interstellar clash that follows. In a move that was
unusually invasive for a translator, he suggested pulling up the
historical flashback, which was buried in the middle of the narrative,
and turning it into the novel's beginning.

When Liu proposed this radical change to the author, a rising figure in
China's burgeoning science-fiction scene named Liu Cixin, he was
prepared to be overruled. Instead, the author instantly agreed. ``That
is how I wanted it originally!'' Liu recalls him saying.

As it turned out, the Cultural Revolution had torn Liu Cixin's family
apart. He was just 3 when the political upheaval began, and still
remembers hearing gunshots at night and seeing trucks full of men
wearing red armbands patrolling the city where he lived in Shanxi
province. When the situation there became too volatile, his parents, who
worked in a coal mine, sent him away to live with relatives in Henan.
The brutality of Mao Zedong's revolution was also central to the story
that Liu Cixin wanted to tell in ``The Three-Body Problem.'' But his
Chinese publisher worried that the opening scenes were too politically
charged and would never make it past government censors, so they were
placed later in the narrative, he says, to make them less conspicuous.
Liu reluctantly agreed to the change, but felt the novel was diminished.
``The Cultural Revolution appears because it's essential to the plot,''
Liu Cixin told me during a Skype interview through an interpreter. ``The
protagonist needs to have total despair in humanity.''

When the English translation of
``\href{https://www.nytimes3xbfgragh.onion/2014/11/11/books/liu-cixins-the-three-body-problem-is-published-in-us.html}{The
Three-Body Problem}'' was published in 2014, it was hailed as a
groundbreaking work of speculative fiction. President Barack Obama
\href{https://www.nytimes3xbfgragh.onion/2017/01/16/books/obamas-secret-to-surviving-the-white-house-years-books.html}{praised}
the novel, calling it
``\href{https://www.nytimes3xbfgragh.onion/2017/01/18/books/president-obamas-reading-list.html}{just
wildly imaginative}.'' Mark Zuckerberg recommended it to his tens of
millions of Facebook followers; George R.R. Martin
\href{https://grrm.livejournal.com/426205.html}{blogged about it}.
Publishers around the world chased after translation rights, which
eventually sold in 26 languages, including Turkish and Estonian. It won
the
\href{https://sinosphere.blogs.nytimes3xbfgragh.onion/2015/08/24/science-fiction-prize-is-awarded-to-chinese-writer-for-first-time}{2015
Hugo Award}, one of the genre's most prestigious honors, making Liu
Cixin the first Asian author to win the prize for best novel. It was
also the first time a novel in translation had won the prize. The book
and its two sequels went on to sell nearly nine million copies
worldwide.

Now, Liu Cixin says, he recommends that Chinese sci-fi fans who speak
English read Ken Liu's translation of ``The Three-Body Problem'' rather
than the Chinese version. ``Usually when Chinese literature gets
translated to a foreign language, it tends to lose something,'' he says.
``I don't think that happened with `The Three-Body Problem.' I think it
gained something.''

\textbf{The success of} ``The Three-Body Problem'' not only turned Liu
Cixin into a global literary star; it opened the floodgates for new
translations of Chinese science fiction. This, in turn, has made Ken Liu
a critical conduit for Chinese writers seeking Western audiences, a
literary brand as sought-after as the best-selling authors he
translates. (Among Chinese sci-fi authors and fans, he is often referred
to affectionately as Xiao Liu, Little Liu, to distinguish him from Liu
Cixin, who is known as Da Liu, Big Liu.) Liu's translations have
reshaped the global science-fiction landscape, which has long been
dominated by American and British authors. Over the past decade, he has
translated five novels and more than 50 works of short fiction by dozens
of Chinese authors, many of whom he has discovered and championed
himself.

\includegraphics{https://static01.graylady3jvrrxbe.onion/images/2019/12/08/magazine/08-mag-liu-03/08-mag-liu-03-articleLarge.jpg?quality=75\&auto=webp\&disable=upscale}

This year alone, Liu published three major new translations: ``Broken
Stars,'' an anthology of short fiction by 14 Chinese sci-fi writers; a
translation of ``The Redemption of Time,'' by Li Jun, who writes under
the pen name Baoshu, which takes place in the aftermath of an
interstellar war; and a translation of Chen Qiufan's ``Waste Tide,'' a
grim dystopian novel that unfolds on a polluted peninsula on the coast
of China, where impoverished migrant workers recycle the world's
electronic trash. Next year, Saga Press will publish Liu's 624-page
translation of Hao Jingfang's novel ``Vagabonds,'' a meandering
philosophical parable about an ideological rift between a communalistic
human colony on Mars and an increasingly capitalistic Earth.

Some of the most thought-provoking science-fiction writers in China
aren't being published through traditional channels, so Liu searches
internet forums and social-media messaging sites like Weibo, WeChat and
the self-publishing platform Douban. He has found sci-fi stories in
unusual corners of the internet, including a forum for alumni of
Tsinghua University. Chinese friends send him screenshots of stories
published on apps that are hard to access outside of China. As an
emissary for some of China's most provocative and boundary-breaking
writers, Liu has become much more than a scout and a translator. He's
now a fixer, an editor and a curator --- a savvy interpreter who has
done more than anyone to bridge the imagination gap between the world's
current, fading superpower and its ascendant one.

Liu has also grown adept at navigating political minefields, finding
ways to transmit writers' political or social critiques without being
too direct. Some of the writers Liu translates use the framework of
science fiction to explore the dystopian consequences of China's rapid
economic and technological transformation, setting a story in the
distant future or on another planet in order to tackle taboo issues like
the lack of social freedoms, the exploitation of migrant workers,
government land seizures, economic inequality and environmental
destruction. In an odd inversion, some of the stories he has translated
into English have not been officially published in China, at times
because of their politically sensitive nature. ``It's a very tricky
dance of trying to get the message that they're trying to convey out,
without painting the writers as dissidents,'' Liu told me over coffee
one day, as we sat in the kitchen of his home in Massachusetts. ``A lot
of Chinese writers are very skilled at writing something ambiguously,
such that there are multiple meanings in the text. I have to ask them,
how explicit do you want me to be in terms of making a certain point
here, because in the original it's very constrained, so how much do you
want me to tease out the implications you're making? And sometimes we
have a discussion about exactly what that means and how they want it to
be done.''

It's no surprise that sci-fi is booming in China, where the breakneck
pace of technological transformation can feel surreal. Economic growth
has lifted hundreds of millions of Chinese citizens out of poverty, and
brought extreme wealth to the upper and political class, but technology
has also become a tool of state oppression. Some Chinese factories have
outfitted workers with devices that measure brain-wave activity to
monitor their emotional fluctuations and alertness. Bird-shaped drones
have been used to surreptitiously spy on citizens, and surveillance
through facial-recognition technology is widespread. On social media and
messaging apps, posts containing certain banned words are automatically
censored. China is now also leveraging its technology to conquer the
solar system: After lagging behind in the space race for decades, the
nation recently made a historic landing on the far side of the moon,
where it has plans to build a permanent research base, and aims to have
a rover exploring Mars next year.

``In China, there's this official propaganda position that science
fiction is about imagination and this is what the future is all about,''
Liu told an audience in New York in April, when he appeared on a panel
with Chen Qiufan at the Museum of Chinese in America and spoke about the
growing popularity of Chinese science fiction. ``In reality, much of the
most interesting science fiction is much more subversive,'' he
continued. ``It is a kind of wry commentary on what is happening in
society. And because so many things are changing in China so rapidly,
science fiction feels like oftentimes the most realistic way to describe
what's happening.''

\textbf{Ken Liu was born} in 1976 in Lanzhou, an industrial city in
Gansu Province in Northwest China. His parents moved abroad when he was
4 --- his father went to study statistics in East Germany, while his
mother pursued her graduate degree in chemistry in the United States ---
and Liu remained in China with his paternal grandparents, both science
professors who were ``book hoarders,'' he says.

As a young boy, he was a promiscuous reader. He read his aunt's
Taiwanese romance novels, his grandmother's Mandarin translations of
``Sherlock Holmes'' and her copy of ``Romance of the Three Kingdoms,'' a
14th-century historical epic set during the Han dynasty. He read his
grandfather's mathematics and chemistry manuals, which he didn't
understand but tore through anyway. In elementary school, he came across
Mandarin translations of American science fiction. He read Philip K.
Dick's ``Do Androids Dream of Electric Sheep?'' and didn't realize it
was science fiction, mistaking the descriptions of a post-apocalyptic
urban hellscape where humans enslave androids for a realistic depiction
of life in America. He was particularly struck by the notion of a world
without animals, where people had robots for pets --- ``It seemed to fit
with my idea of the U.S. as a very high-tech place,'' Liu recalls. He
also picked up Mandarin editions of novelizations of movies like ``The
Empire Strikes Back'' and ``E.T.,'' which gave him a taste of American
pop culture. Liu remembers being baffled by the big suburban houses in
``E.T.,'' and by the notion of a holiday where kids dressed in costumes
and got candy from strangers. ``For me, it was a window into American
life,'' he says.

Image

Portraits of Liu's grandmother (left) and his wife's grandmother sit on
a shelf in his kitchen. Liu credits his grandmother's love of origami as
the inspiration for his first acclaimed short story, ``The Paper
Menagerie.''Credit...Amani Willett for The New York Times

Reading ``E.T.'' didn't fully prepare Liu for life in America. When he
was 11, he moved to Palo Alto, Calif., where his mother worked as a
pharmaceutical chemist and his father worked as a statistical analyst.
He didn't speak English and hadn't lived with his parents since he was a
toddler. He enrolled in a public school, where he went by Ken, a name
his mother picked because it was the closest English analog to his
Chinese name, Yukun.

Books provided a familiar refuge. He learned English in about a year,
and soon was reading novels like ``A Wrinkle in Time'' and ``The
Yearling,'' then moved on to American classics by authors like Faulkner
and Melville, and science fiction by Orson Scott Card, Margaret Atwood
and Arthur C. Clarke. He excelled in school and went to Harvard, where
he majored in English and studied computer science.

When he graduated in 1998, Liu worked as a software engineer, first at
Microsoft, and then at a start-up called Idiom Technologies, where he
met his wife, Lisa Tang Liu. The work wasn't glamorous --- he built what
he describes as ``back-office-database-type stuff'' --- but he liked it:
``It was much more fun to work at that level because you're closer to
the machines.'' Then the dot-com bubble burst, and Lisa was laid off
from her job as a project manager. Liu grew disillusioned with the tech
industry and began searching for something new. He went into programming
because he liked rules and systems, so he decided to try another
rules-based trade and went to Harvard Law School. After graduating, he
clerked for a federal judge, then worked as a corporate lawyer
specializing in international tax planning and real estate. It was
demanding, and not particularly stimulating. Liu, who at that point had
two young daughters, and had grown up apart from his own parents, didn't
want to be an absent father. He became a litigation consultant
specializing in patent infringement and technology cases --- a job that
brought him close to machines again, examining source codes and
disassembling smartphones and tablets to study the underlying mechanics.

Throughout his shape-shifting professional odyssey, Liu wrote fiction,
though he never imagined he could make a living from it. Eventually, he
published his short fiction in sci-fi magazines, and won acclaim for his
strange, surreal stories, which sometimes take place on distant planets
or intergalactic spaceships heading for habitable worlds, but often
center on strained family bonds. His 2011 story, ``The Paper
Menagerie,'' about an American boy whose mother, a Chinese immigrant,
makes him delicate origami animals that come to life, won the Hugo, the
Nebula and the World Fantasy Award, making Liu the first author to sweep
the genre's three major awards for a single work. Four years later, he
published ``The Grace of Kings,'' an epic fantasy novel that drew on
both Western mythology and epics and on historical legends about the Han
dynasty. In 2017, he quit his job as a litigation consultant to focus on
writing.

\textbf{Liu and I first} met on a freezing day in early March in
Stoughton, a small town outside Boston where he lives with his wife,
Lisa, now a photographer, and their two daughters, who are 7 and 9. Liu
--- who at 43 is wiry and energetic, with a close buzz cut, thick
eyebrows and a round, boyish face --- met me at the train station, and
had absent-mindedly left his Airpods in his ears. As we trudged through
piles of snow on unplowed sidewalks, we talked about a screen adaptation
of one of his stories, and his forthcoming translation of Hao Jingfang's
novel, which he compared to Ursula K. Le Guin's ``The Dispossessed.''
The novel unfolds in 2201, a century after a human colony on Mars
declared independence from Earth, where society has become increasingly
technocratic and capitalistic. Liu told me that he isn't sure how
American sci-fi fans will respond to the book, which is more of a
philosophical thought experiment than a plot-driven space odyssey.
``It's not the sort of thing popular American taste favors,'' Liu says.
``It will be valuable for American readers to be exposed to it.''

At his home --- a small, cheerful house that's full of his daughters'
drawings and Lego creations --- Liu showed me his office: a dark,
cavelike room on the basement floor, cluttered with classic sci-fi and
fantasy works by Le Guin, Neil Gaiman, and Tolkien, books in Mandarin by
contemporary Chinese authors, computer-programming manuals, a Classical
Chinese dictionary, copies of Chinese epics like ``Journey to the West''
and ``Romance of the Three Kingdoms,'' an annotated edition of
Confucius' Analects and shelves of Chinese sci-fi magazines and
anthologies. Near his desk, he keeps his four Hugo Awards, two for his
own short fiction and two for translations.

Liu told me that he never set out to be a translator, a profession that
doesn't pay especially well. ``Translation seemed incredibly boring and
technical,'' he says. In fact, it was a Chinese writer who first
discovered Liu, not the other way around. In 2009, Chen Qiufan read one
of Liu's short stories, ``The Algorithms for Love,'' in an online
English-language sci-fi magazine, and sent Liu an email to say how much
he liked it. They kept in touch, and a year later, Chen asked Liu for
his opinion on an English translation of one of his stories, which he
had commissioned from a translation company. Liu wasn't impressed and
offered to edit it, but ended up redoing the translation from scratch.

Image

Liu at home.Credit...Amani Willett for The New York Times

The story, ``The Fish of Lijiang,'' takes place in a future China, where
corporations manipulate their employees' sense of the passage of time in
order to boost workers' productivity. Liu's translation was published in
the sci-fi magazine Clarkesworld in 2011, and won the Science Fiction
and Fantasy Translation Award for short fiction the following year.

Liu realized there was a growing appetite for Chinese science fiction.
As he read more of it, he was stunned to discover a huge and diverse
body of literature --- works that ranged from hard sci-fi, surreal
horror and cyberpunk to dystopian alternate histories, political satires
and chuanyue time-travel tales, a popular sub-subgenre in which a
modern-day protagonist is transported back in time, often to a Chinese
dynastic period. ``I had no idea there was a vibrant science-fiction
community in China,'' Liu says.

At the time, few people outside China did. Before the success of ``The
Three-Body Problem,'' Western publishers and literary agents were
largely oblivious to the proliferation of sci-fi in China. ``Ken Liu was
basically working by himself,'' says Mingwei Song, an associate
professor at Wellesley College who specializes in modern Chinese
literature. ``Only a few people outside China saw the rise of science
fiction there. After his translations, it suddenly became visible.''

For Liu, the discovery felt more personal. He got the same giddy feeling
he had as a boy when he first read Chinese translations of American
science fiction, a sense that he had entered a portal into another
world. Reading science fiction written in his native language gave him
new insights into his former home, a place that had changed almost
beyond recognition since he left.

\textbf{Liu's approach to} translation is unorthodox --- perhaps because
he came to it somewhat late in his eclectic career. Strict fidelity to
the source material is not his chief goal, nor is producing a smooth,
Americanized version. ``It's not a sentence-by-sentence or word-by-word
recreation,'' he says. ``It's about, how do I recreate the overall
effect?''

Rather than glossing over cultural and colloquial nuances that would be
lost on most Western readers, he tries to highlight them. To the
irritation of his publishers, he sometimes resorts to footnotes to
explain unfamiliar terms or episodes from Chinese history, rather than
omitting or Anglicizing them. In his recent anthology, ``Broken Stars,''
he used footnotes to explain the principles of Chinese alchemy, to
describe how an agrarian rebellion against the Tang Dynasty in the ninth
century led to the dynasty's demise, and to unpack a rather layered
``inside joke for Chinese sci-fi fans.''

Some cultural references remain untranslatable. In his introduction to
``The First Emperor's Games,'' a satirical story by Ma Boyong that
hinges on the delightfully absurd premise that China's first emperor was
a video-game addict, Liu notes that ``much of the humor of the story
depends on knowledge of Chinese internet culture and ancient Chinese
history, so liberal use of Wikipedia may be necessary for some
readers.''

The only times Liu got evasive during our conversations were when I
asked him about the political implications of his translation work.
Dissident writers have been jailed in China, and Liu often worries for
the safety of the authors he works with.

``These writers are very creative and courageous in doing what they do,
but as somebody who is not subject to the same constraints and the same
kind of pressures that they are under, I try not to bring them trouble
with what I'm saying,'' Liu told me when we were sitting in his kitchen,
speaking quickly and somewhat urgently, but taking, as he often does,
extreme care with his words. ``As a translator, it's very easy to slip
into the role where you feel like you're explaining, or are in a
superior position to the author to say what you think they meant to say,
or to say what you think ought to be said. I think it's very dangerous.
When you're translating somebody from a different culture, who is
subject to a different political system and who is writing for a
different audience than you are, you have to be very careful about not
substituting your voice for the author's voice and not taking away the
author's prerogative to tell the story she wants to tell.''

Sometimes the writers Liu works with feel they have more freedom in an
English translation to draw pointed parallels to contemporary Chinese
society. When Ma Boyong published his 2005 short story, ``The City of
Silence'' --- which takes place in the year 2046 in a repressive country
where censorship is so extreme that citizens can only use words from a
list of approved, ``healthy'' phrases --- he set the story in an
alternative New York to avoid directly evoking China's suppression of
free speech. For the English version, Liu and Ma worked together to
restore what Ma originally wanted to convey, and New York was changed to
``the Capital of the State,'' making the similarities to China's
censorship apparatus more explicit.

\textbf{Recently, rising} political tensions with and within China have
made Liu's translation projects even more delicate. This year marked the
30th anniversary of the Tiananmen Square protests, a grim milestone that
brought fresh crackdowns on free speech, as state censors have grown
even more vigilant, all against the backdrop of a trade war with the
United States and mass protests in Hong Kong. Some writers who once felt
bold enough to tackle political and social issues, however obliquely,
have been reluctant to publish their work, or have started
self-censoring to avoid trouble.

``The political climate inside China has shifted drastically from when I
first started doing this,'' Liu says. ``It's gotten much harder for me
to talk about the work of Chinese authors without putting them in an
awkward position or causing them trouble.'' Liu usually travels to China
at least once a year to network and meet new writers, and has attended
the Chinese Nebula and Galaxy Awards, the country's most well known
science-fiction prizes. But this year he was denied a long-term visa,
without explanation, prompting him to cancel his planned trip.

In another alarming setback, when his American publisher tried to send
copies of his recent translations to writers in China, the shipments
failed to arrive. It was unclear whether the books were seized or simply
disappeared into a bureaucratic black hole. Liu finally managed to get
copies distributed through visiting Chinese friends, each of whom
carried a few copies back in their suitcases. In April, when I met Liu
at the Museum of Chinese in America, he seemed irritated by the
cumbersome workaround, which he called ``preposterous.''

But later, when I asked if he felt he was being blacklisted by the
Chinese government because of his translation work, Liu deflected and
declined to speculate. ``I don't want to magnify the problem,'' Liu told
me, as we sat in a cafe a few blocks from the museum. ``If the authors
want to say something daring, then I will honor that, but I'm not going
to impose my own politics on them. There's a lot of room to say what you
want to say if you leave things ambiguous.''

In his recent anthology, ``Broken Stars,'' Liu published his translation
of a dystopian novella by Baoshu, titled, ``What Has Passed Shall in
Kinder Light Appear.'' In the narrative, history runs backward, and
China devolves from a superpower into an impoverished, unstable country,
as the protagonist grows older and lives through pivotal events in
reverse chronological order, witnessing the 2008 Beijing Olympics, then
the Tiananmen protests, the Cultural Revolution, the years of famine and
the Japanese occupation. The story's narrator, ``a rising star of
science fiction,'' at one point makes a metafictional observation about
the risk he's taking by writing about politically taboo subjects, noting
that some critics claim ``that my work was an example of capitalist
liberalism and contained metaphors criticizing the Communist Party.''

Baoshu's novella was never printed in China. Liu's English version is
the only officially published edition. When I asked Liu about whether
publishing an English version was risky for Baoshu, he paused, weighing
his words carefully, and finally said simply, ``I'm glad I can bring
this work out.''

Advertisement

\protect\hyperlink{after-bottom}{Continue reading the main story}

\hypertarget{site-index}{%
\subsection{Site Index}\label{site-index}}

\hypertarget{site-information-navigation}{%
\subsection{Site Information
Navigation}\label{site-information-navigation}}

\begin{itemize}
\tightlist
\item
  \href{https://help.nytimes3xbfgragh.onion/hc/en-us/articles/115014792127-Copyright-notice}{©~2020~The
  New York Times Company}
\end{itemize}

\begin{itemize}
\tightlist
\item
  \href{https://www.nytco.com/}{NYTCo}
\item
  \href{https://help.nytimes3xbfgragh.onion/hc/en-us/articles/115015385887-Contact-Us}{Contact
  Us}
\item
  \href{https://www.nytco.com/careers/}{Work with us}
\item
  \href{https://nytmediakit.com/}{Advertise}
\item
  \href{http://www.tbrandstudio.com/}{T Brand Studio}
\item
  \href{https://www.nytimes3xbfgragh.onion/privacy/cookie-policy\#how-do-i-manage-trackers}{Your
  Ad Choices}
\item
  \href{https://www.nytimes3xbfgragh.onion/privacy}{Privacy}
\item
  \href{https://help.nytimes3xbfgragh.onion/hc/en-us/articles/115014893428-Terms-of-service}{Terms
  of Service}
\item
  \href{https://help.nytimes3xbfgragh.onion/hc/en-us/articles/115014893968-Terms-of-sale}{Terms
  of Sale}
\item
  \href{https://spiderbites.nytimes3xbfgragh.onion}{Site Map}
\item
  \href{https://help.nytimes3xbfgragh.onion/hc/en-us}{Help}
\item
  \href{https://www.nytimes3xbfgragh.onion/subscription?campaignId=37WXW}{Subscriptions}
\end{itemize}
