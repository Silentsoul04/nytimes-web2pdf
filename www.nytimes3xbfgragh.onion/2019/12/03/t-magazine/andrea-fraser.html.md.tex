Have We Finally Caught Up With Andrea Fraser?

\url{https://nyti.ms/2OIFdS2}

\begin{itemize}
\item
\item
\item
\item
\item
\end{itemize}

\includegraphics{https://static01.graylady3jvrrxbe.onion/images/2019/12/03/t-magazine/03tmag-fraser-slide-4N0C/03tmag-fraser-slide-4N0C-articleLarge.jpg?quality=75\&auto=webp\&disable=upscale}

Sections

\protect\hyperlink{site-content}{Skip to
content}\protect\hyperlink{site-index}{Skip to site index}

\hypertarget{have-we-finally-caught-up-with-andrea-fraser}{%
\section{Have We Finally Caught Up With Andrea
Fraser?}\label{have-we-finally-caught-up-with-andrea-fraser}}

For over 30 years, the artist has waged a conceptual battle against the
murky ethics of the art world. Now, finally, the larger culture is
taking cues from her.

Andrea Fraser, photographed in Los Angeles in September. Hair by Danni
Katz at Tracey Mattingly using Unite. Makeup by Grace Phillips at Tracey
Mattingly using Chanel Beauty.Credit...Michael Christopher Brown

Supported by

\protect\hyperlink{after-sponsor}{Continue reading the main story}

By Zoë Lescaze

\begin{itemize}
\item
  Published Dec. 3, 2019Updated Dec. 4, 2019
\item
  \begin{itemize}
  \item
  \item
  \item
  \item
  \item
  \end{itemize}
\end{itemize}

IT WAS NEARLY closing time at the museum: Long shadows had already
devoured most of downtown Los Angeles and the hangar-like space was
hushed. \href{https://www.art.ucla.edu/faculty/fraser.html}{Andrea
Fraser} stood before a small TV watching ``Official Welcome,'' one of
her landmark performances. Onscreen, she appeared in a tasteful black
dress at the opening of her 2003 exhibition in Hamburg, Germany, making
remarks from a lectern. She thanked the curators who had made her
retrospective possible; she thanked her mother, who had flown all the
way from California for the event. The crowd clapped on cue. But then
the woman went rogue, as though a computer virus had infected her
gracious-artist software. ``Thank you, Andrea,'' she boomed in a deep
male voice, morphing into the first of more than a dozen art-world
personae who take over the speech --- a piratical dealer, a fawning
patron, an artist best known for pickling a shark. Before it's over, she
has shed her dress and then her Gucci bra, thong and stilettos to
address the audience fully nude. ``It takes a lot of courage to do what
she does,'' she said, assuming the voice of an approving critic. ``She
goes far beyond where most artists have the intelligence or audacity to
operate.'' The 30-minute piece --- funny, brash and often excruciating
--- ends after the woman, back in her dress but somehow more exposed
than ever, breaks down in sobs.

Tears pooled behind Fraser's glasses as she watched her younger self.
But the moment passed quickly, and she returned to critiquing the work
on view, at the
\href{https://www.moca.org/visit/geffen-contemporary}{Geffen
Contemporary at the Museum of Contemporary Art}, with cleareyed
detachment. ``I could have given that a little more time,'' she said,
catching herself rush a beat.

``Official Welcome'' is a case study in the intellectual rigor, physical
bravura and satirical wit Fraser brings to diagnosing the collective
delusions, material excesses, fraught politics, grandiose rhetoric,
bumptious egos, ingrained biases and sundry pretenses of the art world.
For the past 30 years, she has reigned unchallenged as the doyenne of
institutional critique, a branch of conceptual art concerned with the
internal machinery of museums and other social constructs. Lately, that
machinery has been the subject of intense public scrutiny. This year
alone, activists have stormed the exhibition halls of elite museums,
targeting their private funding sources. The Louvre
\href{https://www.nytimes3xbfgragh.onion/2019/07/17/arts/design/sackler-family-louvre.html}{scrubbed
the Sackler name} from one of its wings following demonstrations
condemning the family's ties to the opioid crisis just months after the
\href{https://www.nytimes3xbfgragh.onion/2019/03/22/arts/guggenheim-sackler-family-donations.html}{Solomon
R. Guggenheim Museum} and the
\href{https://www.nytimes3xbfgragh.onion/2019/05/15/arts/design/met-museum-sackler-opioids.html}{Metropolitan
Museum of Art} in New York, as well as the
\href{https://www.nytimes3xbfgragh.onion/2019/03/19/arts/design/national-portrait-gallery-sackler-donation-goldin.html}{National
Portrait Gallery} and
\href{https://www.nytimes3xbfgragh.onion/2019/03/21/arts/design/tate-modern-sackler-britain-opioid-art.html}{Tate}
in London, embargoed Sackler donations. This summer, the businessman
Warren Kanders
\href{https://www.nytimes3xbfgragh.onion/2019/07/25/arts/whitney-warren-kanders-resigns.html}{resigned}
from the Whitney Museum of American Art's board of trustees amid outrage
over his company's production of tear gas canisters used on migrants at
the U.S.-Mexico border.

\includegraphics{https://static01.graylady3jvrrxbe.onion/images/2019/12/03/t-magazine/03tmag-fraser-slide-F22F/03tmag-fraser-slide-F22F-articleLarge.jpg?quality=75\&auto=webp\&disable=upscale}

Fraser is a fierce critic of the status quo, but her approach is
different from that of most artist-activists. For most of her career,
she has fused archival research, psychoanalysis, sexuality and humor in
diverse projects that examine the art world's subconscious --- the web
of desires that shape its economy, power dynamics and social
relationships --- and her own place within the system. In 2001, she
spoofed the museum world's fetish for glitzy new buildings by grinding
sensuously against the curvaceous limestone walls of the
\href{https://www.nytimes3xbfgragh.onion/topic/person/frank-gehry}{Frank
Gehry}-designed Guggenheim Bilbao, her acid-green dress hiked up around
her waist, driven to ecstasy by the audio guide. Two years later, she
filmed a sexual encounter with a collector to create her most notorious
work, ``Untitled,'' a piece some praise as the ultimate comment on art
as empty commerce and others dismiss as an attention-grabbing stunt.

Fraser's occasional bouts of nudity often eclipse her careful choice of
costumes, reflecting a savvy that extends to her civilian life ---
clothes are the only belongings Fraser accumulates other than books.
They are props people use to perform themselves, and Fraser's eye for
the subtle ways that they broadcast self-assurance or camouflage
fragility helps give her work its bite. In the language of her art,
clothes serve as verbs --- active and dynamic forces. Her outfit at the
Geffen that hot day in late September --- a floral jumpsuit, suede
mustard-colored mules and violet toenail polish --- delivered a jaunty
riposte to the little black dress she wore in the video. Fraser swears
at parking meters like a New Yorker (she was based in the city for 25
years) and rhapsodizes about gardening like the Los Angeles transplant
she has become (she's lived there since 2006 and currently shares a home
in the San Fernando Valley with her husband, Andy Stewart, and their toy
poodles, Winnicott and Bowlby).

All of Fraser's works combine a trenchant and unsparing intellect with a
magnetic physical presence. She stands with the alert posture of a
dancer and passes through the world with the kind of fluid, feline
awareness that makes it difficult to imagine her tripping or dropping
her keys. At 54, her face is practically unlined. Intently impassive
when she is listening to others, her features fly into motion when she
is discussing matters close to her heart, from psychoanalytic theory to
samba dancing. Listening to Fraser speak in the measured timbre of a
veteran academic, as she often does, is a bit like listening to an opera
singer softly hum a tune --- it can be suspenseful, knowing just how
much power she is capable of unleashing, how much voltage is being kept
under control.

In ``Official Welcome,'' Fraser's physical powers are on full display as
she stomps and strips and mimics the gestures of recognizable art-world
figures. (For the script, Fraser culled excerpts from actual speeches
and interviews, seamlessly melding them with her own writing.) ``You
know, it's fun to sell big artworks --- and it's profitable. In the end,
a good artist is a rich artist,'' she bellows as she swaggers across the
stage as an unmistakable heavyweight dealer. Minutes later, she morphs
again: ``Most of the work we collect is about sex or excrement,'' she
chirps in the perky tone of a prominent West Coast collector. ``We like
to think of ourselves as connoisseurs of art's subculture.''

The art world is easy to roast --- its most absurd characters are often
the most oblivious --- and it tends to skewer itself without any outside
assistance. But Fraser's work is not mere polemics or parody. She bares
her own insecurities as she examines those of museums, galleries,
viewers and patrons. ``Official Welcome'' may mock the art world's
rituals of florid praise and faux humility, but the performance also
reflects Fraser's lifelong sense of being an outsider --- a position she
consciously draws upon in her work. ``Yeah, it was fun to write,'' she
said, ``but to some extent, this piece was driven by my sense of
resentment and envy of my professional peers about whom all these great
things were said.'' To Fraser, this sort of vulnerability and
self-examination is crucial if a work is going to engage viewers. ``Art
functions through empathy. \ldots{} {[}When you{]} see someone else
who's struggling with something and grappling with something, that
creates a space for finding that in yourself,'' she said. For her, these
performances are not idle exercises but dogged attempts to change
audiences and the larger ecosystems they inhabit. And, if the events of
the past year are any indication, these provocations have worked ---
even if most audiences don't fully realize that the world has caught up
to Andrea Fraser.

\includegraphics{https://static01.graylady3jvrrxbe.onion/images/2019/12/03/t-magazine/3tmag-fraser-01/3tmag-fraser-01-threeByTwoMediumAt2X.png}

``WHAT DO YOU need to know about me to understand my work?'' Fraser
asked six of her graduate students. It was the first day of fall classes
at the University of California, Los Angeles, where Fraser is a tenured
professor in the Department of Art. They sat in a half circle in a stark
white room illuminated by the eye of a large projector. Fraser, in a
black dress and multicolor Issey Miyake scarf, was explaining early
sources of her critical approach, and the lecture had the riveting,
unpredictable atmosphere of one of her performances. ``That I was the
youngest in a family of five,'' she continued. ``It was extremely
competitive, and fairness became extremely important to me from that
position. I had to defend my little share, right? My little piece of the
pie.'' Her obsession with equity, she said, again tearing up, partly
``comes down to that, to being the runt.''

Fraser was born in Montana in 1965 and grew up on the West Coast. Her
parents married two months after they met in New York, where her father,
the son of a cattle rancher, was studying philosophy at Columbia
University and her Puerto Rican-born mother was taking painting classes
at the Art Students League. The family moved to the Bay Area in 1967.
``Pretty quickly, the context of Berkeley began to unravel the family,''
Fraser later told me, over a tray of chicken and rice at a tiny Jamaican
spot in Culver City. ``We became hippies very quickly, my mother got
involved in the women's movement, became a lesbian a bit before that. My
brothers, I think, were selling drugs when they were 10, 11? We were all
pretty precocious.''

The artist grew up memorizing
\href{https://www.nytimes3xbfgragh.onion/2012/03/29/books/adrienne-rich-feminist-poet-and-author-dies-at-82.html}{Adrienne
Rich} poems, browsing ``Our Bodies, Ourselves'' and crafting banners for
gay pride marches in her mother's kitchen. She remembers cutting class
and catching a bus into San Francisco to see
\href{https://www.nytimes3xbfgragh.onion/2018/02/07/t-magazine/judy-chicago-dinner-party.html}{Judy
Chicago}'s major feminist installation ``The Dinner Party'' at the age
of 13. Two years later, she quit going to school altogether (her mother
wrote her a note) and made her way to New York's East Village, where she
applied to the School of Visual Arts.

While she waited on her acceptance, Fraser visited the Met three or four
times a week. ``I was pretty freaked out about having dropped out of
high school and what was going to happen to me,'' she said. ``I had to
sort of redeem myself.'' Soon, Fraser knew most of the museum by heart,
from the lavish period rooms to the Greek and Roman marbles. She was
attracted to ``East Coast cultural institutions and status codes,''
despite feeling, or precisely because she felt, ``deeply illegitimate
--- as a high school dropout, as a hippie kid, as a half-Puerto Rican
kid \ldots{} I think I was able, from the very beginning, to recognize,
even if I couldn't use the words `ambivalence' or `conflicted
investments,''' she said, assuming a deep professorial register to mock
her own preferred terms, ``how much I wanted from these institutions
\ldots{} and that I could find a kind of legitimacy in that world. And,
at the same time, I did feel absolutely crushed by it.''

\hypertarget{for-fraser-her-performances-are-not-idle-exercises-but-dogged-attempts-to-change-audiences-and-the-larger-ecosystems-they-inhabit}{%
\subsection{For Fraser, her performances are not idle exercises but
dogged attempts to change audiences and the larger ecosystems they
inhabit.}\label{for-fraser-her-performances-are-not-idle-exercises-but-dogged-attempts-to-change-audiences-and-the-larger-ecosystems-they-inhabit}}

At S.V.A., Fraser found her tribe: a group of young artists, including
\href{https://www.nytimes3xbfgragh.onion/topic/person/mark-dion}{Mark
Dion},
\href{https://www.nytimes3xbfgragh.onion/2019/05/13/t-magazine/site-specific-art.html}{Tom
Burr}, Gregg Bordowitz and Collier Schorr, who gathered around Craig
Owens, the art critic and gay activist, among other postmodernist
teachers. Fraser stood out from the start. ``Andrea was scary
brilliant,'' said Bordowitz, who became Fraser's boyfriend.
``Frighteningly brilliant, very intimidating. And at the same time, very
fragile, because I think she even scared herself sometimes with what she
saw and understood about the art world and its terrible
contradictions.'' At 18, Fraser left S.V.A. for the Whitney Independent
Study Program, then a theory-intensive boot camp. There, she studied
with the artist Barbara Kruger, whose work critiques systems of power
and control and the cultures they create. (They now teach together at
U.C.L.A.) Kruger praised Fraser's ``incredibly brilliant mind,'' but
Fraser saw herself quite differently. ``At the Whitney program, my image
of myself was that I was just, like, hiding under the seminar table in
fear,'' she said.

BEFORE FRASER CAME along, institutional critique was the domain of
older, mostly male European artists who had launched the movement in the
late 1960s amid the protests sweeping the Western world. In 1969,
activists wrestled in a pool of bovine blood inside the lobby of the
Museum of Modern Art to censure two trustees, Nelson and
\href{https://www.nytimes3xbfgragh.onion/2017/03/20/business/david-rockefeller-dead-chase-manhattan-banker.html}{David
Rockefeller}, who had wartime manufacturing ties to jet fighters and
napalm. The German artist Hans Haacke made these demands for
accountability his very practice by presenting a mordant installation in
an exhibition at the museum the following year. His piece,
``\href{https://www.nytimes3xbfgragh.onion/2019/07/15/t-magazine/most-important-contemporary-art.html}{MoMA
Poll},'' asked viewers to drop ballots into boxes to indicate whether
Nelson's failure to denounce President Nixon's Indochina policy would be
grounds for them to vote against him (he was up for re-election as
governor of New York). By the end of the exhibition, nearly twice as
many participants had answered yes than no. Haacke's project reflected
the broader anti-establishment ethos of the time, but he and the other
architects of institutional critique, including the Belgian artist
Marcel Broodthaers and the French conceptualist
\href{https://www.lissongallery.com/artists/daniel-buren}{Daniel Buren},
were also reacting to a specific transformation in the history of
museums and the art world at large.

If the old museum was a mausoleum safeguarding dusty treasures for the
enjoyment of the educated few, the new museum that emerged in the 1960s
and '70s courted broad audiences with blockbuster shows, expensive
advertising campaigns, new wings, after-hours events and gift shops. The
methods of
\href{https://www.nytimes3xbfgragh.onion/topic/person/thomas-hoving}{Thomas
Hoving}, the freewheeling director of the Met between 1967 and 1977,
exemplified the tactics museums began deploying to ratchet up their
attendance and revenue. A 1969 multimedia exhibition titled ``Harlem on
My Mind,'' for instance, was unabashedly intended, Hoving wrote in his
1993 autobiography, ``to chronicle the creativity of the downtrodden
blacks and, at the same time, encourage them to come to the museum.''

This was also, and not coincidentally, the moment when the contemporary
art market exploded. As early as 1960, the dealer Peggy Guggenheim was
lamenting how ``the entire art movement has become an enormous business
venture.'' Collectors, she wrote, were spending ``unheard-of'' sums
``merely for investment, placing pictures in storage without even seeing
them, phoning their gallery every day for the latest quotation as though
they were waiting to sell stock.'' This approach became a method in
1973, the year a New York-based collector named
\href{https://www.nytimes3xbfgragh.onion/1986/01/03/obituaries/robert-scull-prominent-collector-of-pop-art.html}{Robert
Scull} flipped 50 works by American artists for \$2.3 million (about
\$12 million in today's dollars) that he had bought several years
earlier for \$150,000 (about \$860,000). These figures seem minuscule by
today's standards, but the sale nonetheless announced a new era: the age
of art collecting as investment strategy. As museums chased mainstream
audiences, they assumed a paradoxical role as the seemingly moral
counterparts to the marketplace --- temples to art untainted by dollar
signs --- but were hardly immune to the wealth reshaping the industry as
they pursued funding streams for high-profile expansions and big-ticket
exhibitions.

Image

Fraser's work is not mere polemics or parody: She bares her own
insecurities as she examines those of museums, galleries, viewers and
patrons.Credit...Michael Christopher Brown

Image

Fraser's installation ``Um Monumento às Fantasias Descartadas'' (``A
Monument to Discarded Fantasies,'' 2003) at Museum Ludwig,
Cologne.Credit...Photo: copyright Rheinisches Bildarchiv Köln, courtesy
of the artist and Galerie Nagel Draxler

Fraser wove her predecessors' critical threads into her own practice but
with two key differences: None of the early practitioners of
institutional critique had used his own body as his primary medium, or
acknowledged his own stake (emotional, economic or otherwise) in the
systems he examined. Fraser made herself the site of her art and
explored her own fragility in the process, effectively redefining the
genre. ``I think phrases like `institutional critique' can have the
whiff of academic theory,'' said
\href{https://www.nytimes3xbfgragh.onion/2016/05/03/arts/design/for-scott-rothkopf-a-swift-ascendance-in-the-whitney-hierarchy.html}{Scott
Rothkopf}, the chief curator of the Whitney, ``and one of the things
that makes her work so important is that the clarity and the depth and
the rigor of her thought is matched by tremendous emotional breadth.''
Fraser emerged from the Whitney program armed with strategies plucked
from feminist performance, postmodern theory and psychoanalysis that she
used to form a fresh, hybrid approach. Her great innovation was a
``radical empathy,'' said Connie Butler, the chief curator of the
\href{https://hammer.ucla.edu/}{Hammer Museum} in Los Angeles. Fraser
can, and is willing to, probe ``difficult, complex issues'' by fully
assuming diverse, and sometimes repellent, voices and positions. ``I
can't think of anyone else who does that,'' she said. By inhabiting the
figures and roles Fraser saw as legitimate, she also discovered a means
of negotiating her own fraught participation in the systems they
represent. ``It was an artistic strategy, but it was also a life
strategy,'' Fraser said.

Her big break came in 1989, when she was invited to give a lecture at
the \href{https://philamuseum.org/}{Philadelphia Museum of Art} and
proposed a subversive performance instead. Leading the camera through
the august galleries, Fraser (in the guise of a ladylike docent named
Jane Castelton) shifts seamlessly between lofty praise for the
masterpieces on display (``resplendently amazingly flawless'') and the
museum itself (``a place apart from the mundane demands of reality''
that provides ``a training in taste'') to grim accounts of the squalid
poorhouses that appeared in America at the same time that the country's
oldest art museums were being established (``The inmates are lodged in
rooms of about 22 feet by 45 feet \ldots{} and are classed according to
their general habits and characteristics, separating the more deserving
from the abandoned and worthless''). The video, which is roughly 30
minutes long, is a disturbing, spellbinding portrait of a country whose
long history of inequality haunts its cultural institutions. At the time
that Fraser made ``Museum Highlights: A Gallery Talk,'' that inequality
was approaching its present levels. Reagan- and Bush-era cuts to
cultural funding meant that museums increasingly had to seek corporate
sponsorship and private donors. Fraser's piece ostensibly targeted the
robber-baron philanthropy of the Gilded Age, but she was also
implicating museums' current embrace of free-market capitalism.

Soon, Fraser was participating in prestigious biennials and working with
a New York dealer,
\href{https://www.nytimes3xbfgragh.onion/2003/03/06/arts/colin-de-land-47-art-dealer-who-fostered-experimentation.html}{Colin
de Land}. (Later, in 2002, she started showing at
\href{http://www.petzel.com/}{Petzel}, a larger New York gallery.) But
participating in the gallery world sat uneasily with Fraser, who
continued to dissect its systems from the inside. She produced her
videos in unlimited editions, undermining their value as rarefied
commodities, and occasionally renounced making salable objects
altogether. ``I've been in and out of strategies of trying to manage my
conflicted feelings about the market, about selling art,'' Fraser told
me. She then shook her head, a little amused, perhaps, by the
self-inflicted agony of her struggle against a commercial sphere in
which countless artists are glad, even grateful, to exist at all.

ONE OF FRASER'S (many) problems with the art world is that relationships
are rarely genuine, and much of her work deals with questions of
authenticity. Artists, dealers and patrons often ``perform'' fictional
friendships, she said, ``but fundamentally, they're transactional.''
Fraser described collectors who believed certain artists and dealers
were their friends, only to get hurt when they hit a rough patch and
couldn't afford to keep buying. ``That's not something I want to
participate in,'' she said.

She mined her conflicted feelings about the market to create
``Untitled,'' which, to the artist's exasperation, is often seen as her
defining work. In the video, Fraser has sex with a collector, who
prepurchased a copy of the recording. Shot from a single camera mounted
near the ceiling of a New York hotel room, the unedited footage shows
Fraser greet the unidentified man and offer him a glass of wine. They
appear to talk a little (there is no audio), and then they undress,
sleep together and talk some more. The piece avoids pornographic clichés
--- there are no close-ups; it's as though the encounter had been
secretly filmed by a security camera. The whole thing takes an hour.
``You see, like, a penis!'' Fraser exclaimed, baffled by the scandalized
reactions that ``Untitled'' continues to inspire. ``You see a teeny,
tiny, little speck for about two seconds like three times. You see
boobs.''

Image

Fraser in her 2001 video ``Little Frank and His Carp'' at the Guggenheim
Bilbao.Credit...Courtesy of the artist

When the video went on display at Petzel in 2003, the work shocked the
art world. The press was harsh. There were gratuitous comments on
Fraser's appearance --- Jerry Saltz noted in Artnet Magazine that Fraser
is ``in excellent shape for a 39 year old'' and ``gives an attentive
blow job.'' ``I think some people felt that the piece was almost too
literal and didn't have the complexities of some of her other
performances, and I think she felt very misunderstood,'' said Tom Burr.
``There's an emotional, personal side to it, where you get battered and
hurt by all of these kinds of conditions that you're trying to speak
about. You still want to be liked.''

Fraser considered the backlash a compliment to the work. ``For me, one
of the clearest signs that `Untitled' is a successful piece is that it
didn't only upset people outside of the art world but a lot of people
inside the art world as well,'' she told
\href{https://brooklynrail.org/2004/10/art/andrea-fraser}{The Brooklyn
Rail} the following year. Today, she'll admit that the experience was
harrowing, mostly because the price the collector paid became an
obsession for viewers. For Fraser, the amount was symbolic, which is why
she will never disclose it; the piece was about the desires and
fantasies that drive artists, patrons and dealers to collude in a market
that reduces art to a transaction and a meditation on the experience of
selling intimate parts of one's self. ``Untitled'' can be read as a
comment on the exploitation artists suffer at the hands of profiteering
collectors and opportunistic dealers, but ironically, Fraser was worried
about the patron. ``I had a tremendous amount of power in that piece,''
she said. ``I used to joke it started out as a prostitution piece and
could become an extortion piece --- I have a videotape of a man having
sex in a hotel room with a woman who's not his wife.'' But the price,
which was widely, incorrectly, reported as \$20,000, overwhelmed any
considerations of Fraser's agency. ``So that's what was the most painful
for me,'' said Fraser, ``being exposed publicly in the art economy as
cheap.''

The experience was alienating in other ways as well. In the two years
Fraser spent formulating ``Untitled'' and exhibiting it at Petzel, she
remembers only two people asking her about the work directly. ``And this
was a period when I would hear secondhand that people were having heated
arguments about it, everybody was talking about it. Nobody talked to me
about it.'' One of the paradoxes of ``Untitled,'' she said, is that
while it is, in part, about intimacy, ``the experience of doing it was
incredibly isolating.''

\hypertarget{the-united-states-fraser-writes-has-become-a-plutocracy-and-its-museums-have-effectively-become-pay-to-play-country-clubs-for-millionaires}{%
\subsection{The United States, Fraser writes, has become a plutocracy,
and its museums have effectively become pay-to-play country clubs for
millionaires.}\label{the-united-states-fraser-writes-has-become-a-plutocracy-and-its-museums-have-effectively-become-pay-to-play-country-clubs-for-millionaires}}

``Untitled'' occupies an uneasy place in the contemporary canon. On one
hand, it followed in a long tradition of radical feminist performance
and video art that includes the late
\href{https://www.nytimes3xbfgragh.onion/2017/10/31/t-magazine/feminist-artists-judith-bernstein-betty-tompkins.html}{Carolee
Schneemann}'s ``Fuses'' of 1964-67, a montage of the artist and her
husband having sex, shot from the perspective of their cat, and
\href{https://www.nytimes3xbfgragh.onion/2019/02/14/t-magazine/martha-rosler.html}{Martha
Rosler}'s footage of a male doctor measuring and clinically reporting
the dimensions of her naked hips, limbs and breasts in
``\href{http://www.vdb.org/titles/vital-statistics-citizen-simply-obtained}{Vital
Statistics of a Citizen, Simply Obtained},'' from 1977. Women had long
used their bodies to critique social relations and hierarchies, but the
fact that so much of Fraser's work had existed within an intellectual
framework, steeped in discourse and terminology, meant that no one
really knew how to contextualize her more explicit use of her sexuality.
By 2003, critics and audiences alike had essentially decided that women
artists could use their bodies, or they could use their brains. Fraser
never felt that she had to choose between the two.

These days, Fraser warns her students against including five kinds of
content in their art --- cute animals, babies and young children,
popular music, sex and certain kinds of violence --- because they tend
to overwhelm viewers' ability to think about art in nuanced or complex
ways. She acknowledges the consequences of breaking her own rule.
``Artists are responsible, I believe, on some level for \ldots{} what
their work activates in other people,'' she said. ``And so on some
level, I'm responsible for the responses --- to `Untitled' and to other
works of mine --- that I abhor.''

What continues to unsettle people about ``Untitled'' isn't the sex,
though --- the art world likely lost that last bit of innocence in 1991,
when
\href{https://www.nytimes3xbfgragh.onion/2018/04/17/t-magazine/east-village-artist-jeff-koons-peter-halley.html}{Jeff
Koons} exhibited a series of graphic works portraying himself in
flagrante delicto with his porn-star wife --- it's that the piece calls
out art-world commerce for what it is: commerce, plain and simple.
People in almost any industry would prefer not to dwell on the
transactional basis of their relationships with others, but that's what
Fraser asked her peers to join her in doing when she took up the old
metaphor of artist as prostitute, and implicated everyone involved: the
dealer as pimp, the collector as john, the viewer as voyeur. The work
threatened the high self-opinion of the art world, which rarely
questions its own integrity. Fraser tried, however bluntly, to tear down
pretense and expose the ways in which intimacy is performed. In the end,
it was the piece that got torn apart.

Image

The artist performing ``Official Welcome'' at the San Francisco Museum
of Modern Art in 2012.Credit...Photo: Suzy Poling, courtesy of the
artist

Image

A still from ``Projection'' (2008).Credit...Two-channel video
installation, courtesy of the artist and Galerie Nagel Draxler

BY 2006, FRASER was on the brink of quitting. ``I was really fed up,''
she said. ``I was fed up with the art world, but I was also fed up with
being poor and being broke and being in debt and struggling to live in
my own apartment in New York.'' She was considering pursuing a Ph.D. in
anthropology when salvation came in the form of U.C.L.A., where she has
taught for the past 13 years. Tenure alleviated the financial strain of
operating outside the commercial art world. She cut ties with Petzel in
2011 and never signed with another American gallery. But Fraser's work
has always made more sense in museums, her contentious muses ---
especially now as these institutions reinvent themselves all over again.

Today, museums are scrambling to redefine the canon and compensate, as
much as they can, for centuries of exclusion with a surge in exhibitions
devoted to women and artists of color. Tokenism and hollow attempts at
mere correctness abound, but so do real revelations, as every exhibition
season brings overlooked artists to light and recognizes neglected
icons: Most recently, a remodeled MoMA opened its fall season with shows
by two black artists, the 64-year-old performance maverick
\href{https://www.nytimes3xbfgragh.onion/2018/03/02/t-magazine/pope-l-artist.html}{Pope.L}
and the pioneering assemblage artist
\href{https://www.nytimes3xbfgragh.onion/2019/09/04/arts/design/betye-saar.html}{Betye
Saar}, who, at age 93, was also getting her first solo show at the
museum. For these overdue efforts to continue, and for museums to ensure
that they, as institutions, remain relevant to contemporary audiences
(in more profound ways than as selfie backdrops), curatorial and staff
diversity is essential. In New York, diversity reporting has become
requisite for museums to receive city funding. Trustee diversity remains
elusive, however. A
\href{https://www.aam-us.org/2018/01/19/museum-board-leadership-2017-a-national-report/}{2017
study} commissioned by the American Alliance of Museums found that 46
percent of all American museum boards are 100 percent white. Meanwhile,
institutions' acceptance of questionable money hasn't changed much since
Haacke's day. The inherent hypocrisy of museums --- as protectors of
culture, funded by the very people compromising that culture's values
--- is increasingly unacceptable to audiences. This is a transitional
moment, and it isn't yet clear what museums might become, only that they
are changing. As galleries increasingly display work representing a
breadth of backgrounds, boardrooms are the last parts of the museum in
need of urgent reconsideration.

Recently, Fraser's analysis of museum governance has become more
explicitly political. When Steven Mnuchin, a trustee of MOCA in Los
Angeles at the time,
\href{https://www.nytimes3xbfgragh.onion/2016/05/10/business/dealbook/donald-trumps-pick-for-fund-raiser-is-rife-with-contradictions.html}{became}
the national finance chairman for the Trump campaign, Fraser began to
wonder what the politics of other museum patrons really were. The result
was a 950-page study titled
``\href{https://mitpress.mit.edu/books/2016}{2016 in Museums, Money, and
Politics}.'' The book breaks down the donations of 5,458 museum board
members to party-aligned organizations during the general election.
Fraser was horrified, she said, by the realization that people who
supported institutions professing diversity and equality could
simultaneously fund candidates with conservative positions on issues
like immigration.
(\href{https://www.nytimes3xbfgragh.onion/topic/person/steven-cohen}{Steven
Cohen}, the billionaire hedge-fund manager who sits on the board of MOCA
and MoMA, for instance, gave an estimated \$6,793,500 to Republican
causes --- including Paul Ryan's congressional campaign --- in 2016 and
in the first half of the 2018 midterm election cycle.) ``She uncovered
the false notion that museums are Democratic,'' said Cuauhtémoc Medina,
the chief curator of the
\href{https://muac.unam.mx/acerca-de-nosotros?lang=en}{University Museum
of Contemporary Art} in Mexico City, who organized Fraser's 2016
exhibition there. ``She discovered that we're in serious trouble.'' But
Fraser came away from the project convinced that the bigger issue is
that the super rich --- political affiliations aside --- run the
country. The United States, she writes, has become a plutocracy, and its
museums have effectively become pay-to-play country clubs for
millionaires.

Critics of the book tend to state that Fraser ``doesn't offer
solutions'' or ``doesn't go far enough.'' And while she doesn't propose
specific reforms, she has been busy pursuing answers to the problems
that have underscored her practice. She believes, for instance, that
museums need to democratize internally, and would benefit from artist,
staff and community councils with board representatives. She has sought
these roles herself. ``I'm on three boards and two councils, so it feels
like I've gone to seed or something,'' she said. ``But it's sort of the
part of the evolution of what I do and institutional critique ---
realizing that you also have to step up.'' She believes collection
artists should call ``not just for a protest, but for a meeting'' with
the board members and staff of the museums that hold their work, to
discuss how boards might come to include more people who are not
``defined by their wealth.''

\includegraphics{https://static01.graylady3jvrrxbe.onion/images/2019/12/03/t-magazine/3tmag/3tmag-threeByTwoMediumAt2X.png}

In the wake of recent protests --- including one in October at the new
MoMA's opening party where people picketed an entrance over the board
member Larry Fink, whose company invests in private prisons in the
country on behalf of its clients --- other museums are anticipating
their own day of reckoning. ``You can't possibly know everything about
where every cent of every donor's money is invested,'' said Butler, the
chief curator of the Hammer, where Fraser sits on the artist council. At
a moment when more boycotts seem inevitable, Fraser has become one of
the artists that museum people will occasionally call to get
off-the-record advice on difficult issues --- including where, or how,
to draw a line when it comes to patronage. ``We have to think
differently about who our supporters are and where those funding streams
come from,'' said Butler. Previously, when it came to accepting
donations, ``the line seemed to be like, unless you could prove
criminality and murder, you would take the money, you know? I think the
fact that we're having a more nuanced discussion about this has a lot to
do with the research that Andrea has done.''

NOW THAT HER study is finished, Fraser is preparing two new
performances, including a museum tour --- her first since 1991 --- for
the \href{https://www.artic.edu/}{Art Institute of Chicago}. She had
been invited to do a tour there but didn't have a hook until she began
thinking back to her earlier work. In 2016, Fraser created an
\href{https://www.nytimes3xbfgragh.onion/2016/02/28/arts/design/the-whitney-will-be-alive-with-the-sounds-of-sing-sing.html}{audio
installation} at the Whitney's recently opened building in Lower
Manhattan on the Hudson River, playing sounds she had recorded in a cell
block of Sing Sing Correctional Facility 32 miles north. That piece, as
well as her early reflections on poorhouses, was on her mind when she
found a focus: ``Prisons are the new poorhouses,'' she said. The tour
might limn the unlikely parallels between museums, which encourage
transgression, and prisons, which punish it. Since the 1970s, the number
of both institutions has tripled in the United States.

Another new performance, this one for the Hammer, may involve Fraser
assuming an array of disparate voices with the chameleonic prowess she
brought to ``Official Welcome.'' As we left the Geffen, we passed the
monitor playing that ferocious piece one last time. Fraser originally
thought she would still be performing ``Official Welcome'' at 60 and
that, as she continued to shed that Gucci thong, the aging of her body
would become part of the work --- a means of confronting the way the art
world deals, or doesn't, with older female artists. But the last time
Fraser performed the vitriolic monologue was in 2012. By then, her
position had changed, and even she had to admit she had secured a spot
in contemporary art Valhalla. ``Performing it,'' she said, ``just began
to feel kind of sour and ungracious.''

Watching it, though, is still a poignant experience, partly because
Fraser doesn't want to give up on the art world, no matter how
disagreeable it gets. ``I want to believe that it means something,'' she
said, as she opened the door to the warm California air. ``It's
something that I hold on to, testing the art world to be true to my
hopes for it.''

Advertisement

\protect\hyperlink{after-bottom}{Continue reading the main story}

\hypertarget{site-index}{%
\subsection{Site Index}\label{site-index}}

\hypertarget{site-information-navigation}{%
\subsection{Site Information
Navigation}\label{site-information-navigation}}

\begin{itemize}
\tightlist
\item
  \href{https://help.nytimes3xbfgragh.onion/hc/en-us/articles/115014792127-Copyright-notice}{©~2020~The
  New York Times Company}
\end{itemize}

\begin{itemize}
\tightlist
\item
  \href{https://www.nytco.com/}{NYTCo}
\item
  \href{https://help.nytimes3xbfgragh.onion/hc/en-us/articles/115015385887-Contact-Us}{Contact
  Us}
\item
  \href{https://www.nytco.com/careers/}{Work with us}
\item
  \href{https://nytmediakit.com/}{Advertise}
\item
  \href{http://www.tbrandstudio.com/}{T Brand Studio}
\item
  \href{https://www.nytimes3xbfgragh.onion/privacy/cookie-policy\#how-do-i-manage-trackers}{Your
  Ad Choices}
\item
  \href{https://www.nytimes3xbfgragh.onion/privacy}{Privacy}
\item
  \href{https://help.nytimes3xbfgragh.onion/hc/en-us/articles/115014893428-Terms-of-service}{Terms
  of Service}
\item
  \href{https://help.nytimes3xbfgragh.onion/hc/en-us/articles/115014893968-Terms-of-sale}{Terms
  of Sale}
\item
  \href{https://spiderbites.nytimes3xbfgragh.onion}{Site Map}
\item
  \href{https://help.nytimes3xbfgragh.onion/hc/en-us}{Help}
\item
  \href{https://www.nytimes3xbfgragh.onion/subscription?campaignId=37WXW}{Subscriptions}
\end{itemize}
