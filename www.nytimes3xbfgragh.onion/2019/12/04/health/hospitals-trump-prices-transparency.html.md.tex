Sections

SEARCH

\protect\hyperlink{site-content}{Skip to
content}\protect\hyperlink{site-index}{Skip to site index}

\href{https://www.nytimes3xbfgragh.onion/section/health}{Health}

\href{https://myaccount.nytimes3xbfgragh.onion/auth/login?response_type=cookie\&client_id=vi}{}

\href{https://www.nytimes3xbfgragh.onion/section/todayspaper}{Today's
Paper}

\href{/section/health}{Health}\textbar{}Hospitals Sue Trump to Keep
Negotiated Prices Secret

\url{https://nyti.ms/2YkRijQ}

\begin{itemize}
\item
\item
\item
\item
\item
\end{itemize}

Advertisement

\protect\hyperlink{after-top}{Continue reading the main story}

Supported by

\protect\hyperlink{after-sponsor}{Continue reading the main story}

\hypertarget{hospitals-sue-trump-to-keep-negotiated-prices-secret}{%
\section{Hospitals Sue Trump to Keep Negotiated Prices
Secret}\label{hospitals-sue-trump-to-keep-negotiated-prices-secret}}

The administration wants to require hospitals to reveal the rates they
privately negotiate with insurers for all sorts of procedures, amid the
public outcry over surprise medical bills.

\includegraphics{https://static01.graylady3jvrrxbe.onion/images/2019/12/04/science/04HOSPITALS1/04HOSPITALS1-articleLarge.jpg?quality=75\&auto=webp\&disable=upscale}

\href{https://www.nytimes3xbfgragh.onion/by/reed-abelson}{\includegraphics{https://static01.graylady3jvrrxbe.onion/images/2018/07/16/multimedia/author-reed-abelson/author-reed-abelson-thumbLarge.png}}

By \href{https://www.nytimes3xbfgragh.onion/by/reed-abelson}{Reed
Abelson}

\begin{itemize}
\item
  Published Dec. 4, 2019Updated Dec. 5, 2019
\item
  \begin{itemize}
  \item
  \item
  \item
  \item
  \item
  \end{itemize}
\end{itemize}

The nation's hospital groups sued the Trump administration on Wednesday
over a new federal rule that would require them to disclose the
discounted prices they give insurers for all sorts of procedures.

The hospitals, including the American Hospital Association,
\href{https://www.aha.org/system/files/media/file/2019/12/hospital-groups-lawsuit-over-illegal-rule-mandating-public-disclosure-individually-negotiated-rates-12-4-19.pdf\%20.pdf}{argued
in a lawsuit} filed in United States District Court in Washington that
the new rule ``is unlawful, several times over.''

They argued that the administration exceeded its legal authority in
\href{https://www.nytimes3xbfgragh.onion/2019/11/15/health/list-hospital-prices-trump.html}{issuing
the rule last month} as part of its efforts to make the health care
system much more transparent to patients. The lawsuit contends the
requirement to disclose their private negotiations with insurers
violates their First Amendment rights.

``We make the case that the burden placed on our members to come up with
this information is extensive,'' Tom Nickels, an executive vice
president with the American Hospital Association, said in an interview.

The administration wanted the disclosure rule, which would go into
effect in 2021, to allow patients to better shop for deals on a range of
services, from M.R.I.s to hip replacements.

``Hospitals should be ashamed that they aren't willing to provide
American patients the cost of a service before they purchase it,''
Caitlin Oakley, a spokeswoman for the Department of Health and Human
Services, said in an emailed statement. ``President Trump and Secretary
Azar are committed to providing patients the information they need to
make their own informed health care decisions and will continue to fight
for transparency in America's health care system.''

While the administration already requires hospitals to post some of
their list prices, the public outcry over surprise medical bills and
high out-of-pocket costs led the administration to seek even more detail
on the discounted prices that are kept secret between hospitals and
insurers.

Patients have long complained that they are completely in the dark about
what a doctor's visit or surgery will cost until after they receive the
bill. Knee surgery, for example, can cost thousands of dollars more at
one hospital than at another in the same region.

The administration clearly anticipated a legal challenge. In fact, when
he announced the hospital price disclosure rule, Alex M. Azar II, the
health and human services secretary, was adamant that the rule would
withstand a court challenge. ``We may face litigation, and we feel we
are on a very firm legal footing,'' he said last month.

``H.H.S. has been willing under this administration to test the limits
of their authority, that would subject them to more litigation,'' said
Emily J. Cook, a health care lawyer in Los Angeles. While her firm is
not representing any of the plaintiffs in the lawsuit, one of them is a
client, she said.

At the heart of the administration's efforts is an attempt to tackle
rising hospital costs, which have outpaced the increase in physician
prices,
\href{https://www.healthaffairs.org/doi/10.1377/hlthaff.2018.05424}{according
to a recent study by health economists in Health Affairs.}The economists
estimated that hospital inpatient prices increased 42 percent from 2007
to 2014.

\href{https://www.chicagotribune.com/opinion/commentary/ct-opinion-health-care-prices-20191203-mpphzha4ofhwhftwid3od4mxoi-story.html}{In
an op-ed article published in The Chicago Tribune on Tuesday}, Seema
Verma, the administrator for the Centers for Medicaid and Medicare
Services, promoted the administration's efforts to benefit patients.

``The decades-long norm of price obscurity is just fine for those who
get to set the prices with little accountability and reap the profits,
but that stale and broken status quo is bleeding patients dry,'' she
wrote. ``The price transparency delivered by these rules will put
downward pressure on prices and restore patients to their rightful place
at the center of American health care.''

The hospital groups, which also include the Association of American
Medical Colleges, the National Association of Children's Hospitals and
the Federation of American Hospitals, which represents for-profit
hospitals, argued in the lawsuit that the rule would not accomplish the
administration's aim of helping consumers avoid surprise bills. Three
individual hospitals also joined the case.

``America's hospitals and health systems remain committed to providing
patients with the information they need to make informed health care
decisions,'' the lawsuit said. It contended that the rule ``will
generate confusion about patients' financial obligations, not quell
it.'' The lawsuit was first
\href{https://www.wsj.com/articles/hospital-groups-sue-to-block-price-transparency-rule-11575460685}{reported
by The Wall Street Journal.}

The Trump administration has also proposed a rule requiring insurers to
allow patients to get advanced estimates of their out-of-pocket costs
before they see a doctor or go to the hospital. The industry's major
trade associations wrote a letter on Tuesday, requesting an additional
90 days to comment on the proposal, pushing the deadline to mid-April.

``The sheer volume of data that the government is proposing health plans
disclose is staggering --- dollar amounts for every single item or
service, for every single provider and facility, for every single
individual and employer plan,'' wrote executives from America's Health
Insurance Plans and the Blue Cross Blue Shield Association.

The administration's efforts to push the industry to make more
information public to patients faces a real legal challenge, and it has
been unsuccessful in other attempts.

When the administration earlier tried to require pharmaceutical
companies to disclose the list price of prescription drugs in television
ads, the drug companies argued that the Department of Health and Human
Services had exceeded its regulatory authority and that the requirement
also violated its First Amendment rights.
\href{https://www.nytimes3xbfgragh.onion/2019/07/08/health/drug-prices-tv-ads-trump.html?rref=collection\%2Fbyline\%2Fkatie-thomas\&action=click\&contentCollection=undefined\&region=stream\&module=inline\&version=latest\&contentPlacement=6\&pgtype=collection}{A
federal judge ruled last}summer that H.H.S. had, in fact, exceeded its
regulatory authority with the rule, which was seen as largely symbolic
because list prices are not what patients typically pay. The decision is
under appeal.

The hospitals have also been successful to date in other cases arguing
that the agency lacks the authority to issue payment rules, said Ms.
Cook, although the government could also win those cases on appeal.

The courts are displaying less deference to rule making by government
agencies, said Douglas Hallward-Driemeier, a lawyer in Washington who
works with clients to challenge administrative actions.

The hospitals could also succeed by raising First Amendment grounds, he
said. ``That argument has gained a lot of traction over last 10 years,''
he said.

\textbf{\emph{{[}}\href{http://on.fb.me/1paTQ1h}{\emph{Like the Science
Times page on Facebook.}}} ****** \emph{\textbar{} Sign up for the}
\textbf{\href{http://nyti.ms/1MbHaRU}{\emph{Science Times
newsletter.}}\emph{{]}}}

Advertisement

\protect\hyperlink{after-bottom}{Continue reading the main story}

\hypertarget{site-index}{%
\subsection{Site Index}\label{site-index}}

\hypertarget{site-information-navigation}{%
\subsection{Site Information
Navigation}\label{site-information-navigation}}

\begin{itemize}
\tightlist
\item
  \href{https://help.nytimes3xbfgragh.onion/hc/en-us/articles/115014792127-Copyright-notice}{©~2020~The
  New York Times Company}
\end{itemize}

\begin{itemize}
\tightlist
\item
  \href{https://www.nytco.com/}{NYTCo}
\item
  \href{https://help.nytimes3xbfgragh.onion/hc/en-us/articles/115015385887-Contact-Us}{Contact
  Us}
\item
  \href{https://www.nytco.com/careers/}{Work with us}
\item
  \href{https://nytmediakit.com/}{Advertise}
\item
  \href{http://www.tbrandstudio.com/}{T Brand Studio}
\item
  \href{https://www.nytimes3xbfgragh.onion/privacy/cookie-policy\#how-do-i-manage-trackers}{Your
  Ad Choices}
\item
  \href{https://www.nytimes3xbfgragh.onion/privacy}{Privacy}
\item
  \href{https://help.nytimes3xbfgragh.onion/hc/en-us/articles/115014893428-Terms-of-service}{Terms
  of Service}
\item
  \href{https://help.nytimes3xbfgragh.onion/hc/en-us/articles/115014893968-Terms-of-sale}{Terms
  of Sale}
\item
  \href{https://spiderbites.nytimes3xbfgragh.onion}{Site Map}
\item
  \href{https://help.nytimes3xbfgragh.onion/hc/en-us}{Help}
\item
  \href{https://www.nytimes3xbfgragh.onion/subscription?campaignId=37WXW}{Subscriptions}
\end{itemize}
