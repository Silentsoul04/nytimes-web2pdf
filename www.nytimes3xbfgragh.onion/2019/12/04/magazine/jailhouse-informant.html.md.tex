How This Con Man's Wild Testimony Sent Dozens to Jail, and 4 to Death
Row

\url{https://nyti.ms/33PgQ9R}

\begin{itemize}
\item
\item
\item
\item
\item
\item
\end{itemize}

\includegraphics{https://static01.graylady3jvrrxbe.onion/images/2019/12/08/magazine/08mag-cover2/08mag-cover2-articleLarge-v5.jpg?quality=75\&auto=webp\&disable=upscale}

Sections

\protect\hyperlink{site-content}{Skip to
content}\protect\hyperlink{site-index}{Skip to site index}

Feature

\hypertarget{how-this-con-mans-wild-testimony-sent-dozens-to-jail-and-4-to-death-row}{%
\section{How This Con Man's Wild Testimony Sent Dozens to Jail, and 4 to
Death
Row}\label{how-this-con-mans-wild-testimony-sent-dozens-to-jail-and-4-to-death-row}}

Paul Skalnik is a grifter and criminal. Now a man may be executed
because of his dubious testimony. Why did prosecutors rely on him as an
informant?

Some of Paul Skalnik's mug shots, spanning a decade of criminal
activity.Credit...

Supported by

\protect\hyperlink{after-sponsor}{Continue reading the main story}

By \href{https://www.nytimes3xbfgragh.onion/by/pamela-colloff}{Pamela
Colloff}

\begin{itemize}
\item
  Dec. 4, 2019
\item
  \begin{itemize}
  \item
  \item
  \item
  \item
  \item
  \item
  \end{itemize}
\end{itemize}

\emph{This article is a partnership between}
\href{https://www.propublica.org/article/hes-a-liar-a-con-artist-and-a-snitch-his-testimony-could-soon-send-a-man-to-his-death}{\emph{ProPublica}}\emph{,
where Pamela Colloff is a senior reporter, and The New York Times
Magazine, where she is a staff writer.}

\textbf{W} hen detective John Halliday paid a visit to the Pinellas
County Jail on Dec. 4, 1986, his highest-profile murder case was in
trouble. Halliday, who was 35 and investigated homicides for the local
sheriff's office, had spent more than a decade policing Pinellas County,
a peninsula edged by white-sugar-sand beaches on Florida's Gulf Coast,
west of Tampa. It is a place that
\href{http://fairpunishment.org/wp-content/uploads/2016/12/FPP-TooBroken_II.pdf}{outpaces
virtually} all other counties in the nation in the number of defendants
it has sentenced to death. Prosecutors who pursued the biggest cases
there in the 1980s relied on Halliday, who embodied the county's
law-and-order ethos. Powerfully built and 6-foot-4, with a mane of dirty
blond hair and a tan mustache, he was skilled at marshaling the facts
that prosecutors needed to win convictions.

He had worked the case for the past year and a half, ever since the body
of a
\href{https://www.newspapers.com/clip/27881593/tampa_bay_times/}{14-year-old
girl named Shelly Boggio} was found, nude, floating in an inland
waterway near the town of Indian Rocks Beach. Her murder was singular in
its violence. Her body bore 31 stab wounds, many of them to her hands,
as if she had tried to shield herself from the ferocity of the attack.
She was most likely still alive, the medical examiner determined, when
she was dragged into the water and left to drown. Her older sister
identified her by the silver ring, eagle-shaped and inset with
turquoise, that she wore on her left hand.

The crime scene yielded few clues. No murder weapon was left behind, and
no fingerprints or other forensic evidence was recovered. If Boggio was
sexually assaulted, the medical examiner found, any trace of sperm may
have been washed away during her time in the water. ``It was one of
Pinellas County's cruelest murders,'' The St. Petersburg Times observed,
``and there was little evidence.''

Halliday's investigation quickly zeroed in on two men, Jack Pearcy and
James Dailey, who lived together and were new to Pinellas County. The
facts, what few there were, pointed overwhelmingly to Pearcy, a
29-year-old construction worker with a history of arrests for violence
against women. Pearcy pursued the teenager before her death, and Pearcy
picked her up on the last afternoon of her life, when she was thumbing a
ride with her twin sister and a friend. The girls spent the afternoon
and evening with Pearcy, Dailey and other housemates, drinking wine
coolers and smoking marijuana. After the other two girls went home,
Pearcy took Boggio to a beachfront bar, where she was last seen,
barefoot and disheveled, around midnight.

Pearcy acknowledged that he drove her to the lovers' lane along the
Intracoastal Waterway where she was killed. But he tried to shift blame
to Dailey, claiming that he picked up his housemate before he and Boggio
headed down to the water. And while Pearcy admitted to the police that
he stabbed Boggio at least once, and he provided details about the crime
that were known only to investigators, he insisted that it was Dailey
who was the actual killer.

This was all that connected Dailey, a 38-year-old itinerant Vietnam
veteran, to the crime: the word of its prime suspect. No physical or
forensic evidence linked him to the murder, nor did any discernible
motive. He would later say he had been asleep in the early-morning hours
when Pearcy was out alone with Boggio, only to be awaked by Pearcy, who
said he needed to talk; Pearcy drove him to a nearby causeway, where
they drank beer and smoked a few joints at the water's edge. Pearcy's
girlfriend and a longtime friend of Pearcy's said they saw the two men
come home together that morning, hours before Boggio's body was found,
and that Dailey's jeans were wet.

The state attorney's office in Clearwater pressed forward with the most
serious charge it could bring against the men, ensuring that they would
be tried for first-degree murder --- a crime punishable by death.
Pearcy's trial came first and ended with a guilty verdict in November
1986. But at the penalty phase, the jury recommended that he be
sentenced to life in prison. It was a blow to the state attorney's
office, which would argue, in a forceful sentencing memo to the court,
that ``no evidence exists that Pearcy was not the main actor in this
child's brutal murder.'' Pearcy had dodged the electric chair after
participating in, and most likely carrying out, one of the county's most
monstrous crimes. Prosecutors had only one more chance to secure a death
sentence for Boggio's murder.

\includegraphics{https://static01.graylady3jvrrxbe.onion/images/2019/12/08/magazine/08-mag-Informants-04/08-mag-Informants-04-articleLarge.jpg?quality=75\&auto=webp\&disable=upscale}

Ten days after the conclusion of Pearcy's trial, Halliday visited the
Pinellas County Jail. At his direction, jailers began pulling inmates
who were housed near Dailey out of their cells. One by one, the men were
taken to a small, windowless room, where Halliday was waiting. He
pressed each man for information. Had Dailey ever talked about his case?
Ever admitted to anything?

Four men who were questioned that day testified at a 2018 evidentiary
hearing to the same unsettling detail in the interview room. Newspaper
articles about Boggio's murder were laid out conspicuously before them.
``I got a very uneasy feeling looking at the newspaper articles,''
Michael Sorrentino, one of the four, testified at the hearing. ``Had I
wanted to say something, or fabricate something, all the tools were
there to give them whatever they might be looking for.''

Halliday testified at the hearing that there were no newspapers in the
interview room. Either way, no one gave Halliday any useful information
that day. In a slender, lined notebook, the detective recorded what each
inmate told him. ``Nothing,'' he jotted down after one interview. After
others, he wrote:

\begin{quote}
\emph{Said Dailey denies charge}\\
\emph{doesn't know a thing}\\
\emph{Nothing}\\
\emph{Knows nothing. Didn't even know Dailey.}\\
\emph{stays to himself. Knows nothing.}\\
\emph{Refused to come to be interviewed.}\\
\emph{``Wish I could have helped you but its a little outa my league.''}
\end{quote}

Halliday's visit was a bust. But in the Pinellas County Jail, the word
was out: The Boggio case needed a snitch.

\textbf{In jail, it is} widely understood that helping prosecutors and
the police can earn extraordinary benefits, from reduced sentences to
dismissed charges. By the time Dailey's trial began the following summer
in Clearwater, in June 1987, no fewer than three inmates had come
forward claiming to have heard Dailey confess to the killing. The first
two worked in the jail's law library, where they professed to have heard
Dailey say about the murder, ``I'm the one that did it.'' They also told
the jury of ferrying several handwritten notes between Dailey and
Pearcy; in the letters shown to the jury, Dailey appeared eager to
appease his co-defendant, whom prosecutors planned to put on the stand.
But the two jailhouse informants were eclipsed by a third inmate, who
had contacted Halliday to say that he had some information. He told a
much more damning story --- one that placed Dailey at the scene of the
crime and put the knife in his hand. It was exactly what prosecutors
needed.

That witness was Paul Skalnik, a familiar figure around the Pinellas
County Courthouse. He had appeared before the court numerous times as a
jailhouse informant and was skilled at providing the sort of incendiary
details that brought a defendant's guilt into sudden, terrible focus.
Skalnik began working with Halliday in 1983, when the detective was
investigating a triple homicide, and Skalnik helped send two men to
death row, cementing his status as an invaluable resource. Because he
was a known snitch, he was held in protective custody, in a single cell
where he was shielded from inmates who might want to do him harm.
Despite this considerable impediment, Skalnik claimed --- just a few
weeks before jury selection in Dailey's trial began --- to have procured
Dailey's confession.

Assistant State Attorney Beverly Andrews called Skalnik to the stand
before resting her case on a Friday afternoon, ensuring that Skalnik's
words would be left to linger in jurors' minds over the weekend. Pearcy
--- the only person to have offered an account of the murder --- had by
then refused to testify against Dailey, leaving a gaping hole at the
center of the state's case. Skalnik, who was facing 20 years in prison
on charges of grand theft, stepped into the void. Dark-haired and
stocky, with olive skin that offset his gray-blue eyes, Skalnik had a
wide, expressive face that was malleable like an actor's, registering
emotions with almost vaudevillian embellishment. His words had a stagy
yet captivating sincerity.

Andrews began by leading him through a series of questions that were
designed to establish his trustworthiness. No, he had not been promised
anything in return for his testimony, he assured the jury solemnly. And
yes, he conceded, he had been convicted of some felonies --- ``five or
six, if I am not mistaken'' --- but he was quick to tell the jury that
he had not only assisted the state numerous times as a jailhouse
informant; he had also once been a police officer. ``I still do have law
enforcement inside of me,'' he said.

The story he told the jury was simple but arresting. He was passing by
Dailey's cell early one morning, he explained, when Dailey sought his
counsel. Dailey was under the impression that Skalnik had worked as a
private investigator, Skalnik said, and wanted his legal advice.

It was then, Skalnik testified, as they stood at the bars of Dailey's
cell, that Dailey came clean, confiding that he had stabbed the girl and
then thrown the knife away. What Dailey said ``was so hard to comprehend
and to accept,'' he told the jury earnestly. ``I had seen this gentleman
walking in the hallways, laughing and kidding with other inmates. And
all of a sudden, to see a man's eyes, and to describe how he can stab a
young girl --- and she was screaming and staring at him and would not
die. ...''

``Were those Mr. Dailey's words as best you can remember?'' Andrews
asked.

``As best I can recall,'' Skalnik said gravely. `` `She is screaming,
staring at me, and would not die.' ''

It didn't matter that Skalnik had few other details about the murder.
Any questions as to his truthfulness were put to rest when Andrews
called Halliday as her final witness. The detective vouched for Skalnik,
testifying that the inmate had supplied him with reliable information in
other cases, yielding ``extremely positive results.''

Dailey would not testify on the advice of his attorneys, but Skalnik
offered a vivid, first-person account of a confession. In a trial that
had been long on conjecture but short on hard evidence, his testimony
became the linchpin of the state's case --- so much so that Andrews
would cite him more than a dozen times in her closing argument. She
assured the jury that Skalnik was ``honest'' and ``reliable.''

It was with that imprimatur of credibility that jurors found Dailey
guilty. They also recommended, by a rare unanimous vote, that he be
executed. Judge Thomas Penick Jr. of the Circuit Court formally
sentenced Dailey to death on Aug. 7, 1987.

Five days later, Skalnik was released from jail. A Florida Parole and
Probation Commission memo stated that his release was ``due to his
cooperation with the State Attorney's Office in the first-degree murder
trial.'' It was a remarkable turn of events given that he had been
identified as a flight risk just a year earlier, after violating the
terms of his parole. ``This man has been, is and always will be a danger
to society,'' his parole officer had warned. Now he was released on his
own recognizance and did not have to post bond. Skalnik was a free man.

\textbf{By then, Skalnik} had been in and out of jail half a dozen
times. One of his earliest brushes with the law came nearly a decade
earlier, in Texas, through an unusual series of events that began with a
misbegotten Christmas present for his third wife, Rozelle Rogers. Their
whirlwind romance started in 1977, when Rogers, a divorced mother of
two, was leading a quiet life near her well-to-do parents in
Friendswood, south of Houston. ``He just came in out of nowhere and
swept her off her feet,'' said Rogers's daughter, Lisa Rogers.

Image

Skalnik in a nursing home In Corsicana, Tex., in October.Credit...Eli
Durst for The New York Times

At 28, Skalnik already had two brief marriages behind him, but Rogers,
who knew little about his past, saw the promise of a new future in the
magnetic, seemingly worldly man who lavished her with rapturous flattery
and told her he was a top executive at Southwest Airlines. He was less
conventional than someone in the corporate world --- he wore a gold
necklace and a diamond-studded watch with his three-piece suit --- but
he seemed like a catch.

After getting married in Las Vegas, Skalnik moved into Rogers's condo,
where he soon established a routine. On weekday mornings, he bounded out
the door, briefcase in hand, telling Rogers he was headed to the airport
to fly to Dallas, where Southwest is headquartered, and assuring her
that he would be back home in time for dinner. He always carried a
business-size checkbook with him, with which he was profligate, treating
her to jewelry, stays at posh big-city hotels and box seats at
University of Texas football games, where he boasted of once having been
a running back for the team. He tried to outdo himself with each
ostentatious flourish, giving Rogers not just one but two brand-new
vehicles: a baby blue Lincoln Continental and a customized Dodge van
with red velvet curtains and CB radios. Once he presented Lisa, then 14,
with a gold-nugget Seiko watch that sparkled with diamonds. When the
eighth grader asked him, point blank, if it was real, Skalnik grinned
and replied, ``Looks real to me.''

Lisa had taken note of Skalnik's idiosyncrasies: how he filched the
robes and towels from their luxury hotel rooms and squirreled away the
purple velvet bags that his Crown Royal came in after he had drained the
bottles. It didn't escape her attention, either, that he always kept his
handgun within reach, even stashing it under the driver's seat when he
was behind the wheel. In the months that followed, Skalnik's behavior
became more unpredictable, and he began heading out of town ``on
business,'' he said. Eventually, his absences stretched into weeks and
then months, until he disappeared altogether.

Unknown to Rogers --- who would ultimately have to post notices in the
newspaper announcing her intent to divorce him --- Skalnik had, by the
spring of 1978, landed in the Harris County Jail, in Houston, for
passing a dozen bad checks, one of which he had used to buy Rogers a
microwave for Christmas. Everything Skalnik had told her was a lie. He
had financed their new vehicles, and nearly everything else, on the
strength of her good credit, taking out loans he never paid back,
opening credit cards in her name and draining her checking account.
Rogers, who died in 2005, was left reeling. ``He wrecked my mom's
credit, and he wrecked her life,'' Lisa said.

His arrest was the first time he found himself in serious trouble,
though he had been grifting since at least the early 1970s, when he
worked as a police officer. He had lasted just 14 months with the Austin
Police Department, stepping down in 1973 after he wrote a string of bad
checks. After apologizing for any embarrassment he had brought upon the
department, he was never charged. He got off easy again when he was
arrested for grand larceny in Orange County, Fla., three years later,
after he posed as a furniture salesman and pocketed \$700 from an
unwitting customer. Even though the police found evidence in his car
that suggested he was running another scam --- he had a stash of checks,
IDs and stationery bearing the seal of the Texas State House of
Representatives --- he received probation. Only his arrest in Houston in
the spring of 1978, which violated the terms of his Florida probation,
brought his lucky streak to an end. Suddenly he was behind bars and
looking at jail time in another state.

It was here, at the Harris County Jail, that Skalnik's career as an
informant began. As he sat contemplating his future, Thomas Hirschi, a
defendant in a case that was all over the news, was booked into the jail
and placed in a nearby cell. Hirschi was one of the
``\href{https://www.latinousa.org/2016/10/21/remembering-moody-park-death-sparked-houston-riot/}{Moody
Park Three},'' a trio of anti-police-brutality activists whom the D.A.'s
office had charged with inciting a riot at Moody Park, in Houston, that
left 15 people hospitalized. The uprising came a year after the death of
a 23-year-old laborer named José Campos Torres at the hands of Houston
police officers. The officers who beat Torres and pushed him into a
bayou to drown --- ``Let's see if the wetback can swim,'' one famously
taunted --- received only slaps on the wrist. City leaders blamed
``outside agitators'' like Hirschi, who had called for Torres's killers
to be brought to justice, for the riot rather than acknowledging that
years of police brutality pushed the Mexican-American community to its
breaking point. To the Moody Park Three's supporters, it was a frame-up.
To prosecutors, it was a case they needed to win.

Skalnik would later place a call to the D.A.'s office, claiming to have
information on the case. Prosecutors put him on the stand when Hirschi
and his co-defendants, Travis Morales and Mara Youngdahl, were tried
together in May 1979.

Skalnik's first time testifying as a jailhouse informant was less
sure-footed than his later turns as a witness for the state, but he
hewed closely to a story line that he would use again and again in the
years to come. He told the jury that he was standing outside Hirschi's
cell when the young activist decided to unburden himself and confess
that Morales's plan all along had been ``to incite the Mexican-American
youngsters,'' Skalnik said. As Skalnik spoke, Hirschi sat at the defense
table in disbelief. ``I'd never seen the guy before,'' Hirschi told me.
``Never seen his face, didn't know his name.''

Hirschi and his co-defendants were looking at up to 20 years in prison
if convicted. ``We weren't naïve, but to actually see this unfold in
front of us --- to watch him lie when our lives were on the line --- was
pretty shocking,'' Hirschi said.

Skalnik's strategy paid off. Prosecutors prevailed in the end; the Moody
Park Three were found guilty, though the jury declined to give them any
prison time. After Skalnik spent two months in jail in Houston, he was
sentenced that November to a year in the Orange County Jail, in Florida,
for violating his probation. But four days before Christmas, a Florida
Circuit Court judge abruptly reversed course, stating that he had
``received a recommendation'' that Skalnik be moved from the jail to a
work-release program --- a privilege normally forbidden to repeat
offenders. The judge did not specify whether Texas prosecutors were
behind that recommendation. But the lesson was unmistakable: The best
way for a man behind bars to help himself was to help prosecutors.

Just two months later, in February 1980, Skalnik was not only out of
jail but also married to a woman who believed that her clean-cut,
churchgoing husband was a law student. It was this marriage --- his
fourth --- that brought him to Pinellas County. While living with her in
St. Petersburg, he got engaged to another woman in nearby Largo, telling
her he was a Dallas attorney who wanted to move his law practice to
Florida. He persuaded her to take out \$3,500 in loans to help him set
up his new law office --- money he promised to pay back but never did.
His wife learned of his engagement only when he was arrested for grand
theft. Skalnik, who faced up to five years in prison, was left in the
Pinellas County Jail to await his trial.

Skalnik had another plan. As he had done in Houston, he placed a call to
prosecutors. So began a yearslong working relationship between Skalnik
and the state attorney's office in Clearwater, which would extend
through much of the 1980s and involve at least 11 local prosecutors.
``The state attorney's office has always characterized me as an honest,
forthright witness,'' he later wrote to a judge, ``and together, we
never lost a case.''

Ten days before his grand-theft case went to trial in August 1981,
Skalnik provided the state attorney's office with information on three
different defendants who were charged with murder but whose cases had
not yet gone to trial. In return, prosecutors offered him a deal. If he
pleaded guilty, they would recommend that he spend no more than three
years in prison --- two fewer than he was facing. They also left open
the possibility that he could secure a sweeter deal if he cooperated
further. (``Probation was discussed!'' states a handwritten note in the
state attorney's files.) Skalnik took the plea, and his sentencing was
postponed while he quietly went to work as a jailhouse informant.
Prosecutors would hold off on making a sentencing recommendation until
they saw exactly how much Skalnik had to offer and how helpful he could
be to them. In the meantime, he would remain in jail, a snitch.

Skalnik made himself busy that fall and winter, and into the following
June, testifying for the state in two drug-trafficking trials and
providing a damaging deposition in a high-profile murder case. In each
case, Skalnik could truthfully say under oath that he had not been
promised anything in return for his testimony because no specific
agreements had yet been struck. The narratives he told were strikingly
similar, featuring inmates who not only freely admitted their guilt but
also did so spontaneously in the same oddly stilted language. In a
drug-trafficking case that ended in a guilty verdict, Skalnik testified
that the defendant had struck up a conversation with him --- midway
through the accused's trial, no less --- that began with the
declaration, ``We were loading the boat with 24,000 pounds of marijuana
in Colombia.''

Skalnik was rewarded on June 30, 1982, when, with the backing of the
state attorney's office, he was sentenced to probation. For someone who
had racked up five criminal charges in nearly as many years and left the
state the last time he was on probation, it was an astonishing feat.

\textbf{Buried deep} in thousands of pages of court records spread
across two states lies evidence to suggest that Skalnik was one of the
most prolific, and most effective, jailhouse informants in American
history. ``I have placed 34 individuals in prison, including four on
death row,'' he boasted in a 1984 letter to Senator Lawton Chiles of
Florida, in a request for favorable treatment --- a number that, while
inflated at the time, would ultimately prove accurate. During a single
six-year span, from 1981 to 1987, Skalnik testified or supplied
information in at least 37 cases in Pinellas County alone. Many were
cases in which people faced the most serious possible charges and the
most severe penalties. Eighteen defendants whose cases Skalnik provided
information on were under indictment for murder. A vast majority of
their cases ended in convictions or plea deals. Four were sentenced to
death.

The state attorney's office in Clearwater, in an emailed statement, said
Skalnik independently got fellow inmates to confide in him, then
contacted prosecutors or the Pinellas County sheriff's office. ``He at
no time was an `agent' of the sheriff's office or the state attorney's
office,'' it said. ``The state attorney's office never provided any
leniency to Paul Skalnik in exchange for his testimony.'' All
information provided by Skalnik, the statement said, was independently
verified, and the office has never received any information to indicate
that his testimony was ``incorrect.''

Still, Skalnik's journey through the criminal-justice system affords a
rare opportunity to see exactly how prosecutors and jailhouse informants
work together. These insights are possible because of a rare confluence
of forces, including Skalnik's extensive history of informing and
Florida's strong public-record laws, which enabled ProPublica and The
New York Times Magazine to obtain thousands of pages of police reports,
arrest records, jail logs, probation and parole records, pretrial
interviews and correspondence that document his activity in sometimes
granular detail. This reporting follows decades of litigation waged by
public defenders and pro bono attorneys representing death-row inmates
in whose cases Skalnik played a role. The full record provides a vivid
picture of how jailhouse informants are used, showing which benefits
Skalnik was afforded, which crimes he eluded punishment for and, most
clearly, how the state attorney's office put this witness, who was
dubbed ``a con man extraordinaire,'' in the words of one warrant for his
arrest, on the stand in cases where defendants' lives hung in the
balance.

In response to detailed questions about the Dailey trial, Beverly
Andrews (now Beverly Andringa), who prosecuted the case, said in an
email that she has ``very little memory'' of the more-than-30-year-old
case, but she said that she ``never willfully and intentionally provided
false evidence or testimony to a court or jury on any case.'' Robert
Heyman, a prosecutor who tried the case with her, pointed out that
Skalnik had been vetted by law enforcement and called to testify by
other prosecutors. ``If we did not believe that his testimony was
truthful, we wouldn't have had him testify,'' Heyman said.

Yet again and again, prosecutors have shown that they are willing to
rely on the testimony of witnesses like Skalnik, even in cases in which
the death penalty is in play. ``Jailhouse informants are common in
prosecutions of very serious crimes, including ones that carry life and
even death sentences,'' said Michelle Feldman, the Innocence Project's
state campaigns director, whose work focuses on legislative efforts to
regulate the use of jailhouse informants. ``Since the courts don't track
them, it's hard to say which jurisdictions use them the most or how
often they testify. But they remain an entrenched feature of criminal
prosecutions, even though they are the most unreliable kind of
witnesses.''

What makes them so unreliable, she emphasized, is the widespread
understanding in jail that prosecutors can offer substantial benefits in
exchange for cooperation --- rewards that may include not just reduced
sentences or improved jail conditions but cash payments. ``There is a
very strong incentive to lie and very little disincentive not to,''
Feldman said.

\textbf{The consequences} of snitch testimony can be catastrophic. Of
the 367 DNA exonerations in the United States to date, jailhouse
informants played a role in nearly one in five of the underlying
wrongful convictions.
\href{http://www.law.northwestern.edu/legalclinic/wrongfulconvictions/documents/SnitchSystemBooklet.pdf}{A
seminal 2004 study} conducted by Northwestern Law School's Center on
Wrongful Convictions found that testimony from jailhouse snitches and
other criminal informants was the leading cause of wrongful convictions
in capital cases. Today nearly a quarter of death-row exonerations ---
22 percent --- stem from cases in which prosecutors relied on a
jailhouse informant.

Informants often end up on the stand when other evidence is weak; a case
that is based on rigorous forensic work or witness testimony that can be
independently corroborated does not need a snitch to paper over the
gaps. The most unreliable witnesses, then, may testify in the least
sound cases --- and in cases in which the stakes are the highest.

Given how opaque and unchecked prosecutors' use of jailhouse informants
is, it is impossible to quantify how often they factor into criminal
cases. Usually the only glimpse of the government's reliance on them
comes when a scandal erupts, as it did in Los Angeles in the late 1980s,
after a serial snitch named Leslie Vernon White went public. In an
interview for ``60 Minutes,'' he demonstrated how easy it was to
manufacture a confession, procuring key details of a murder on camera in
the course of just a few phone calls. His claims led to a grand-jury
investigation into jailhouse informants that was the first of its kind.
The inquiry exposed extensive prosecutorial misconduct and the
widespread misuse of jailhouse informants, who had concocted
persuasive-sounding confessions in a variety of ingenious ways. Some
impersonated law-enforcement officers to make calls eliciting
information; others sent friends and relatives to court hearings to suss
out other defendants' cases. Many were fed information by law
enforcement, who shared arrest reports, photos and case files with
inmates, even escorting them to crime scenes so they could better shape
their testimony to fit the evidence. The grand jury identified upward of
150 cases, and perhaps as many as 250, that were affected.

In the wake of the scandal, the Los Angeles County district attorney's
office instituted reforms to provide more oversight of prosecutors who
put jailhouse informants on the stand, but beyond Los Angeles, little
changed. The Chicago Tribune raised the alarm in 1999, when it
highlighted prosecutors' overreliance on jailhouse informants in
death-penalty cases in Illinois, and found that such testimony had
helped convict or condemn four of the state's 12 death-row exonerees. In
the aftermath of the report, which identified numerous problems with the
death penalty in Illinois, Gov. George Ryan declared a statewide
moratorium on executions, but nationally, the harms of jailhouse
informants went unaddressed.

In the 1990s and 2000s,
\href{https://www.law.umich.edu/special/exoneration/Pages/Exoneration-by-Year.aspx}{the
accrual of DNA exonerations} --- made possible by the advent of a
then-new and revelatory technology --- laid bare the fact that snitch
testimony had contributed to wrongful convictions across the country.
Nevertheless, the authors of the 2004 Northwestern Law study were
fatalistic, writing, ``The reality is that neither legislatures nor
courts are about to ban snitch testimony in the prevailing
tough-on-crime political climate.''

In 2014, a quarter-century after the Los Angeles snitch scandal began,
\href{http://www.abajournal.com/magazine/article/secret_snitches_california_case_uncovers_long_standing_practice_of_planting/news/article/do_you_volunteer_on_a_regular_basis/?utm_campaign=sidebar}{another
scandal broke} in neighboring Orange County. A local public defender was
able to show that for years, sheriff's deputies had engaged in a
practice of strategically planting informants in the cells of defendants
who were awaiting trial. Inmates who produced incriminating information
--- including ``confessions'' they elicited with threats of violence ---
were rewarded with money and sentence reductions.

Orange County's top prosecutors and law-​enforcement officials were
implicated, and according to a pending A.C.L.U. lawsuit against the
county's district attorney's office and Sheriff's Department, at least
140 cases were tainted. Though no law-enforcement officials were fired
or disciplined, the scandal contributed to voters' ouster of Orange
County's longtime district attorney, Tony Rackauckas, and convictions in
dozens of cases were subsequently challenged.

Alexandra Natapoff, a law professor at the University of California,
Irvine, who is the nation's foremost legal scholar on criminal
informants, said the parallels between the two California scandals show
how little has changed in 30 years and how little we know about how
often jailhouse informants continue to be used across the country. ``In
Orange County, a sophisticated jailhouse-informant system remained under
the radar, not disclosed in court cases for decades,'' Natapoff said.
``The accident is that we know about it, not that it happened.''

The benefits jailhouse informants receive, she added, are rarely
apparent to jurors, because prosecutors often bestow them on the back
end, after a trial's conclusion. ``Many jailhouse informants can
truthfully state to the jury that they have not been promised any
benefit, even though realistically they expect to be compensated for
their testimony,'' Natapoff said. ``Ironically, jurors will often be the
only people in the courtroom who do not understand this arrangement.''

\textbf{Karen Parker was} 12 when she crossed paths with Skalnik. ``He
appeared out of nowhere,'' she told me. ``He befriended my mom and dad,
and suddenly he was in our life.'' It was July 1982, the middle of a
long and restive summer, and Parker usually passed the time at the beach
or riding her bike around her working-class neighborhood in Seminole,
south of Clearwater. Skalnik --- who had found work with her next-door
neighbor's brother, a private investigator --- was often around, holding
her father rapt with stories of his days as a police officer. He struck
Parker as impossibly cool, a sharp dresser with a certain louche charm.
``He was magnetic, out of the ordinary --- not like other people I
knew,'' she said. ``And he was very attentive to me. He'd give me that
extra look, and I had the sense that he was interested. I was drawn
in.'' She was thirsty for male approval; at home, where her father was
stern and critical, she received none. To Parker, Skalnik's attention
was exhilarating.

One day that July, she went fishing with her next-door neighbors in
Largo, and Skalnik joined them for the outing. It was dark when they
returned to Seminole, and he summoned her to come sit beside him in the
front seat of his silver Cadillac. By then, her neighbors had gone
inside. Suddenly he was kissing Parker, his hands slipping under her
T-shirt. Then his fingers were inside her. ``He took my hand and put it
on his penis,'' she said. ``He had me masturbate him until he
ejaculated.'' Parker had just finished the seventh grade. Skalnik was
32.

His voice turned cold afterward, she said, when he advised her,
obliquely, to keep quiet. ``One of these days you're going to open your
mouth too many times,'' he said, suggesting that doing so could land her
in ``J.D.C.'' --- juvenile detention. ``The only one who is going to be
in trouble is you.''

Parker spoke to only one person about what had happened, a 16-year-old
girl who lived next door and glimpsed Skalnik kissing her in the car.
But the story eventually leaked out later that year, after Skalnik was
arrested for grand theft. Upon learning of the assault, Parker's parents
took her to the sheriff's office. Skalnik was charged in December 1982
with ``lewd and lascivious conduct on a child under 14,'' a felony
punishable by up to 15 years in prison.

The case against him was a strong one. Parker's description of the
assault was bolstered by the eyewitness account from the 16-year-old
neighbor she confided in and another from the 16-year-old's boyfriend,
who, as he approached the car, saw ``movement in the suspect's lap
(suspect's hand or victim's) of a masturbatory nature,'' according to
the police report. And unlike some victims of child sexual abuse, Parker
was old enough to clearly articulate what happened to her and was
willing to testify. ``I don't think it is right that he is calling me a
liar and I am not,'' she told investigators. There were also the results
of a polygraph examination that the sheriff's office administered to
Parker. (``Have you ever heard of a 12-year-old girl having to take a
polygraph?'' she asked me, still incredulous.) Polygraphs have since
been shown to be unreliable and are
\href{https://www.justice.gov/jm/criminal-resource-manual-262-polygraphs-introduction-trial}{not
generally admissible} in court, but they were central to many
law-enforcement agencies' investigations in the 1980s; the fact that
Parker passed, and that her account of the assault was found to be
truthful, was meaningful at the time. All told, prosecutors had a case
they could take to trial. Child sexual-assault cases are routinely
prosecuted on far less.

But the state attorney's office would ultimately decline to try Skalnik.
In a plea hearing that took place on March 10, 1983, prosecutors agreed
to dismiss the molestation charge. In return, Skalnik pleaded no contest
to new charges of grand theft, for which he had been arrested the
previous November. (True to form, Skalnik had tricked a woman into
giving him nearly \$5,000 on the promise of starting a travel agency
together, and defrauded a couple out of more than \$20,000 by assuring
them that he could deliver discounted cars that were forfeited to the
state in narcotics cases --- all of which violated his probation in the
previous grand-theft case involving his fiancée.) These charges carried
much lighter punishments than child molestation. The state attorney's
notice to the court dismissing the lewd-and-lascivious conduct charge
said simply, ``There is insufficient evidence available at this time.''

Skalnik took the plea deal, for which he received concurrent five-year
sentences. But instead of being sent to state prison to serve out his
punishment, he would remain in the Pinellas County Jail, where he could
continue to work as a jailhouse informant, gathering information on
defendants who had not yet gone to trial. (Prosecutors argued that
keeping him in the county jail was for his safety, given that his
testimony had helped send men to prison.) And with the child-molestation
charge out of the way, Skalnik came across to jurors as a far more
innocuous figure than he actually was --- that is, as a former police
officer turned small-time scam artist, rather than as a child molester.

Parker belongs to a group of crime victims who remain forgotten in a
criminal-justice system that allows jailhouse informants to be released,
and to continue committing crimes, because prosecutors exchanged
leniency for their testimony. Parker never knew about prosecutors' deal
with Skalnik, only that he was never punished for what he did to her.
``No one ever said, `That's wrong,' '' she told me. ``The message I got
was that what he did was O.K. --- that it wasn't serious, it wasn't a
crime.'' In her father's eyes, she said, it was she who was to blame for
what happened. ``Everyone liked Paul, and they believed Paul, and I was
seen as the troublemaker,'' she said. After Skalnik's arrest, her
father's harsh criticism of her escalated, until it became unbearable.
Parker ran away the following year, when she was 13, and left home for
good when she was 14, taking refuge at a runaway shelter. ``I didn't
trust anyone for a long, long time,'' she said.

Image

Karen Parker in Florida. She says she was sexually abused by Skalnik the
summer after she finished seventh grade. Skalnik was 32.Credit...Eli
Durst for The New York Times

\textbf{Skalnik, meanwhile,} continued to be valuable to the state
attorney's office after his molestation charge was dismissed in 1983.
That year and the next, he testified in four high-profile murder trials,
three of which ended with death sentences. All three of the men who were
condemned to die --- Richard Cooper, Kenneth Gardner and J.D. Walton ---
had, without question, been present at the scenes of the horrendous
crimes they stood accused of. But because several people were charged in
connection to each murder, the key question at Cooper's, Gardner's and
Walton's trials was one of culpability: How much of a role did the
accused play, and were his actions egregious enough to warrant the
electric chair? Prosecutors used Skalnik to show that each man was not
just guilty but also deserved death.

At the time, in the 1980s, appearing
\href{https://www.upi.com/Archives/1982/07/04/Early-release-from-prison-a-response-to-overcrowding-Does-it-work/6141394603200/}{soft
on crime} was a surefire way to be voted out of office. ``In Florida,
prosecutors, judges, the attorney general, the governor --- everyone
wanted to prove how tough they were,'' Stephen Bright told me. Bright,
one of the nation's pre-eminent capital defense attorneys and a visiting
lecturer at Yale Law School, went to Clearwater in 1985 to challenge the
conviction and death sentence of a man who was bound for the electric
chair. Gov. Bob Graham of Florida, who earned the nickname
\href{https://www.orlandosentinel.com/news/os-xpm-2003-04-20-0304190009-story.html}{Governor
Jell-O} because he was seen as weak and ineffective, reinvented himself
by signing death warrants, increasing the number of warrants he signed
when he ran for re-election in 1982, and again when he ran for Senate in
1986. ``In Florida,'' Bright added, ``it seemed like there couldn't be
enough death sentences.''

Even so, Pinellas County stood out. For a three-year period, from 1982
to 1984, it sent more people to death row than any other county in
Florida. At the time, the state attorney's office was run by a
hard-charging prosecutor named James T. Russell, who stood just 5-foot-5
but enjoyed a fearsome reputation. A perfectionist whose moral universe
had no shades of gray, Russell pursued a law-and-order agenda that
appealed to his constituents, who were disproportionately older and
overwhelmingly white. ``Put more criminals in prison, and there will be
less crime on the streets,'' he told a local civic group in 1981,
condemning what he perceived to be a system that placed too much
emphasis on rehabilitating people who broke the law. (Russell died in
2006.) Few questioned the bare-knuckled tactics behind his office's
conviction rate, which reached 92 percent in 1990. So fierce was the
drive to rack up wins that prosecutors ``sought the death penalty in
nearly every first-degree murder case,'' according to a 1988 survey by a
local public defender --- a strategy that allowed them to leverage the
threat of the electric chair to extract guilty pleas from defendants.

The demand for convictions and long, tough sentences made Skalnik's
testimony invaluable. The confessions he recounted were lurid and
dramatic, strewn with provocative details that prosecutors used not just
to show the guilt of the defendants but also to establish that they were
diabolically evil. Skalnik told of victims' begging for their lives and
of remorseless killers who laughed after their slaughters, boasting that
they had outsmarted prosecutors and the police. Gardner, who was
convicted in a grisly stabbing death of a hardware-store owner,
supposedly bragged to Skalnik, ``I killed him, but they'll never prove
it.'' Walton, who was found guilty of carrying out the execution of
three men after a botched robbery, considered the whole thing ``a funny
joke,'' Skalnik told jurors. And Cooper, one of Walton's co-defendants,
supposedly introduced himself to Skalnik with the brash declaration,
``I'm one of the men involved in the triple-murder slayings they thought
was a Mafia gangland killing.''

Though Cooper might have earned the jury's mercy because he was a
teenager, Skalnik turned that potentially mitigating fact on its head by
sharing an offhand comment he attributed to Cooper. ``He said no jury
would ever sentence him to the death chair,'' Skalnik testified,
``because he's 19 years old and because he's got that little baby
face.'' The jury recommended that Cooper be put to death. (Gardner and
Cooper would eventually be resentenced to life in prison.)

The confessions he claimed these men volunteered to him --- and that the
state attorney's office had him repeat to juries --- were all the more
extraordinary given that he was held in protective custody and that his
reputation as a snitch was well known to other men in the jail.
``Beginning to encounter more and more inmates who recognize him,''
stated a note in his file from January 1983. Nevertheless, Skalnik was
sometimes moved closer to, or even into the same cell as, a defendant in
a newsworthy case. Cooper was assigned to a two-man cell with Skalnik;
Gardner was later assigned to a cell that adjoined Skalnik's. It was
during their brief time in proximity to him that each supposedly came
clean. A third inmate, a co-defendant of Cooper's named Terry Van Royal,
protested when Skalnik was moved into his cell. ``I told the guard I
would not be in the same cell with him,'' Van Royal later wrote in an
affidavit, ``because I knew who he was and what he did.''

If defense attorneys tried to suggest that Skalnik's preternatural
ability to extract men's most closely held secrets was too good to be
true, Skalnik would insist that he stood to gain nothing from his
testimony, as he did during the 1983 murder trial of Freddie Gaines. A
24-year-old handyman, Gaines was charged with stabbing his girlfriend's
ex-lover to death in a bar brawl --- a chance encounter, Gaines told the
jury, that turned violent. But Skalnik's testimony jettisoned any notion
that Gaines acted spontaneously; to hear him tell it, Gaines had carried
out a calculated, coldblooded murder. Skalnik said Gaines boasted of
bringing a knife to the bar and seeking out the victim, telling Skalnik
he should have been charged ``with open-heart surgery.'' Skalnik's voice
swelled with emotion as he spoke, so much that he once appeared to be on
the brink of tears.

Gaines told me that he was floored when he heard Skalnik testify and
that he leaned over and told his lawyer: ``He's sitting right there
telling a lie. Me and this man ain't never talked before.'' But
Skalnik's testimony was effective, recasting a possible crime of passion
into a premeditated execution --- a distinction that would help earn
Gaines a conviction for first-degree murder and a life sentence, rather
than a lesser charge like manslaughter, which carried a penalty of up to
15 years.

Image

Freddie Gaines in Florida's Taylor Correctional Institution in November.
Skalnik's testimony helped convict Gaines of first-degree murder in a
barroom brawl, rather than a lesser charge like
manslaughter.Credit...Eli Durst for The New York Times

Skalnik assured the jury that he had come forward with no other motive
than to preserve public safety. ``I used to be a police officer,'' he
said, adding that he became an informant after hearing other men in the
jail brag that they were going to beat their charges. Assistant State
Attorney Bruce Young bolstered the idea that Skalnik's intentions were
selfless. ``It's your understanding that nothing can be done for you as
far as eliminating or reducing your sentence?'' Young asked.

``That's correct,'' Skalnik replied.

Young continued, ``Even if your lawyer wanted to go in front of a judge,
the judge would have no jurisdiction to reduce the sentence, is that
correct?''

``Yes, sir,'' Skalnik said with a note of resignation. ``That's
correct.''

Just three months later, Young wrote to the Florida Department of
Corrections to request leniency. ``Mr. Skalnik has been a state witness
in a number of very important cases, including several first-degree
murder cases, and his testimony has been honest and truthful in all
cases,'' Young said. Expressing concern for Skalnik's safety, he asked
that his star witness not be sent to state prison but remain in the
jail; better yet, Young proposed, the State of Florida could simply
release him. ``If Mr. Skalnik is eligible for parole,'' he wrote, ``I
would urge that Mr. Skalnik be considered for parole.'' (Young did not
respond to detailed requests for comment.)

One of Skalnik's most loyal supporters would also go to bat for him.
John Halliday --- the detective who previously worked with Skalnik and
would later receive key information from him in the Dailey case ---
called the parole board that fall on his behalf. ``Mr. Halliday would
like the commission to know the subject has been of great assistance to
the sheriff's office,'' reads an interoffice memo documenting the call.
Halliday wrote directly to a parole commissioner the next month, urging
him to consider Skalnik for release. ``I have never done this for an
inmate during my 10 years in law enforcement,'' Halliday added.

(Halliday declined to review a detailed request for comment, referring
it to Keith Johnson, an investigator for the state attorney's office.
Johnson referred the request to the state attorney's office, which
declined to comment about Halliday, as did a spokeswoman for the
Pinellas County sheriff's office, who noted that the cases in question
took place long ago.)

On March 19, 1985, Skalnik was paroled. Having served about half of his
five-year sentence for grand theft, he was free, despite assessments
from the Department of Corrections that judged him to be a ``con artist
of the highest degree'' who was at ``high risk of further unlawful
behavior.'' Sure enough, after Skalnik was released, he cheated an
elderly woman out of tens of thousands of dollars for two Lincoln Town
Cars he never delivered; conned another woman out of thousands more with
a phony real estate deal; and duped a jewelry-store clerk into taking a
check from what turned out to be a defunct bank account for a \$6,100
gold Rolex. He also married and divorced his fifth wife.

By Nov. 24, 1986, he was back in the Pinellas County Jail, where he
would claim, the following summer, to have procured James Dailey's
confession. ``Mr. Skalnik's deceitful nature knows no bounds,'' an
unknown person wrote in a handwritten letter to the state attorney's
office, urging prosecutors to punish him as harshly as the law would
allow. ``How many chances will this man be given? How many more people
will he hurt and victimize?''

\textbf{As reform-minded} prosecutors have swept into office over the
past five years in cities like Chicago, St. Louis, Dallas, Philadelphia
and Boston, some district attorney's offices have begun to re-evaluate
the way they have always done business. ``People in these communities
have made very clear that a win-at-all-costs approach is not what they
want anymore and does not make them safer,'' said Miriam Krinsky, a
former federal prosecutor and executive director of Fair and Just
Prosecution, a network for progressive prosecutors. In an effort to stem
mass incarceration, reformers have focused their energy on trying to
address the big, structural problems that most directly affect people's
liberty, like changing the cash-bail system and diverting defendants to
drug treatment instead of prison.

So far, the use of jailhouse informants has received relatively little
consideration. ``It's an issue that is just starting to gain
attention,'' Krinsky said. ``There is a new dialogue about whether
prosecutors should institute safeguards that would allow them to
continue using jailhouse informants but proceed with caution, or whether
to steer clear of jailhouse informants completely.''

In some state legislatures, the idea of bringing greater scrutiny to
jailhouse informants has slowly begun to gain traction.
\href{https://www.dallasnews.com/news/politics/2017/09/20/snitch-testimony-sent-innocent-man-to-prison-for-18-years-texas-lawmakers-hope-he-s-the-last/}{In
2017, Texas lawmakers} tried to strip away the secretive nature of
snitch deals by compelling prosecutors to keep track of and disclose the
sort of rudimentary information that defendants and their lawyers are
often lacking. This includes a full accounting of the benefits that
jailhouse informants have received for their testimony, their criminal
records and the previous cases in which they testified. Last year,
\href{https://www.usnews.com/news/politics/articles/2018-11-28/overriding-rauners-veto-illinois-passes-law-reforming-use-of-jailhouse-snitches}{Illinois
passed legislation} that requires judges to hold pretrial ``reliability
hearings'' to evaluate whether informants, in light of the benefits they
have been promised and their histories as informants, should be allowed
to testify. In July, Connecticut
\href{https://apnews.com/9f8858ef3fbf4965874d314ce41ec69c}{became the
first state} to enact a statewide tracking system for jailhouse
informants that documents where and when such witnesses have previously
testified and what benefits they received in return.

Florida took action after a staggering number of its death-row inmates
were exonerated; to date, 29 condemned men have been cleared of their
convictions,
\href{https://deathpenaltyinfo.org/state-and-federal-info/state-by-state/florida}{more
than} in any other state. A commission appointed by the Florida Supreme
Court to study wrongful convictions recommended that prosecutors
disclose the deals they make with jailhouse informants, and in response,
\href{https://www.news-journalonline.com/article/LK/20140615/News/605067010/DN}{the
court changed the rules} of criminal procedure in 2014 to require the
disclosure of such deals as well as other details related to the
informant. The new requirement was intended to introduce transparency
--- but in practice, it does not address the common problem that
prosecutors may not need to make explicit promises at all, because the
potential for leniency is implicit and well understood.

Reformers hope that new legislation, though imperfect, could still deter
prosecutors from relying on jailhouse informants. ``When you put reforms
in place that require tracking and disclosing information about these
witnesses, what often comes to light is a good deal of information that
could discourage prosecutors from wanting to move forward,'' says
Rebecca Brown, director of policy for the Innocence Project. ``Once they
have a fuller understanding of all the factors that would underlie that
informant's testimony, they have to confront questions like: Is this
reliable enough to move forward with?''

But in a vast majority of states, no reforms have been passed at all.
Perjury charges for jailhouse snitches are very rare, even when their
testimony is later proved to have been demonstratively false. So, too,
are any meaningful consequences for prosecutors who fail to disclose
agreements made with a jailhouse informant at the time of trial, or who
mislead juries into thinking that an informant will not receive rewards
after testifying, or who conceal facts about a jailhouse informant's
criminal history that might undermine his credibility. No legislation
has yet addressed the outsize but largely invisible role that jailhouse
informants play in plea deals, in which prosecutors may use the mere
specter of an informant's future testimony to intimidate defendants into
not taking their cases to trial. And more radical ideas --- like an
outright ban on jailhouse informants in capital cases --- have stalled,
allowing prosecutors to continue using snitch testimony to secure the
starkest, most irrevocable punishment.

\textbf{On Aug. 7,} 1987, five weeks after James Dailey's trial ended in
a guilty verdict, the 41-year-old Vietnam veteran came to court to be
formally sentenced to death. He had remained mute throughout his trial,
but that day, he finally rose to speak. Tall and angular, with dark hair
and a long, mournful face, he began by recognizing the ``terrible kind
of pain'' Shelly Boggio's murder had caused her twin sister, and the
anguish felt by her family members, who sat in the courtroom, weeping.
``I say these things as a caring human being and as a person wrongfully
convicted of this heinous crime,'' he declared. He had been condemned by
Pinellas County's ``win at all costs'' system of justice, he said, in
which ``truth is allowed to be manipulated and paid liars are allowed to
testify.'' His trial, he added before he was led away in shackles, had
been a ``mockery of justice.''

Image

James Dailey at Florida State Prison in November. ``I never talked to
Paul Skalnik in my life,'' he said. ``We knew how many guys he had
snitched on. There wasn't any hiding the fact.''Credit...Eli Durst for
The New York Times

Skalnik, who was released five days later, was supposed to be back in
court that October for his trial on charges of grand theft. But by the
time his trial date rolled around, he had skipped town, having absconded
with a rented Lincoln Town Car shortly before he was due to marry a
woman who believed he worked undercover for the F.B.I. Prosecutors were
left in the lurch; their star witness, who was slated to testify in
three coming murder trials, was suddenly a fugitive from justice.

Skalnik, meanwhile, was hiding out in Austin, Tex., where he was busy
practicing a sort of absurdist performance art. Passing himself off as
``J. Paul Bourne,'' a high roller who was flush with oil money, he
managed to buy \$27,000 worth of jewelry with forged checks while also
running ``a new unknown type scam,'' according to documents from the
Travis County district attorney's office --- a con that involved opening
bank accounts on the promise that millions of dollars would be wired in.
He also married, and soon divorced, his sixth wife.

Following his arrest on a forgery charge in February 1988, he tried to
reprise his role as a snitch, but an assistant district attorney in
Austin saw what should have been clear to any prosecutor. Skalnik, she
warned in an interoffice memo, ``is a BIG con artist.'' Skalnik was soon
extradited back to Florida, where he was booked, once again, into the
Pinellas County Jail.

By then, his relationship with the state attorney's office had soured,
his decision to bolt to Texas having made a farce of the trust it had
put in him. But if prosecutors thought they could distance themselves
from Skalnik, they had failed to discern the game he was playing. When
they balked at his demands for a lenient plea deal in the summer of
1988, he turned on them.

With the help of his public defender, Skalnik filed a motion with the
trial court in which he claimed a history of extensive prosecutorial
misconduct. In the motion, he asserted that prosecutors had coached him
on how to testify in numerous cases so as to give jurors the false
impression that he ``had actually heard all these `confessions,' and had
no agreement with the state for a reward for his testimony.''
Prosecutors ``knew of the potential questionability of said
confessions,'' the motion charged. Skalnik provided the names of 11
prosecutors whom he accused of misconduct but provided few specifics. He
claimed to have given information or testimony in more than 50 cases and
suggested that much of that evidence was tainted.

Just as the men whom Skalnik leveled outrageous claims against over the
years had faced accusations that were maddeningly difficult to disprove,
prosecutors found themselves on the defensive, scrambling to discredit
what Skalnik claimed was the honest truth. In formal responses submitted
to the court, the state attorney's office categorically denied his
assertions, dismissing them as ``falsehoods, ranging in degree from
gross exaggeration to preposterous fabrication'' --- a richly
paradoxical about-face for an office that had asked scores of jurors to
take him at his word. Trying to preserve the integrity of the cases
Skalnik had participated in, prosecutors simultaneously argued that his
earlier testimony as a state witness ``was credible, was often
independently substantiated and withstood extensive cross-examination.''

In fact, behind the scenes, an investigator with the state attorney's
office had difficulty verifying that Skalnik had provided information
that could be independently corroborated. Of the two examples Detective
Halliday provided --- he said Skalnik's tips led law enforcement to a
ski mask worn during the committing of a murder and to a gun used in
another killing --- only the claim about the ski mask checked out; of
the other, the investigator wrote: ``This information is incorrect. The
information from Skalnik was accurate; however it came months after the
gun was retrieved.''

Skalnik brought forth his grievances, the state attorney's office told
the court, only after he failed to blackmail prosecutors into cutting
him a favorable deal. Yet in the end, Skalnik got exactly what he
wanted. After Skalnik withdrew his motion claiming that they had engaged
in misconduct, he and prosecutors arrived at what appeared to be a
mutually beneficial arrangement --- one that would both appease Skalnik
and send him far from Pinellas County. For a total of six felonies ---
four counts of grand theft and two counts of failure to appear in court
--- he would receive a five-year sentence. He entered his plea on the
condition that his sentence be served in Texas, where he had time left
on a bail-jumping charge.

Skalnik ended up evading even that relatively meager punishment. In
November 1989, after completing seven months in prison in Huntsville,
Tex., on the bail-jumping charge, the State of Texas --- which never
agreed to allow him to serve his Florida sentence there --- released
him. Ultimately Florida abandoned its efforts to extradite him. ``The
commission has received information which has caused it to conclude that
return of said person is not warranted,'' read one notably oblique 1991
Florida Parole Commission memo.

Skalnik had been let loose on the world again.

\textbf{In 1991,} Misty Anderson was living with her mother and two
younger sisters in Friendswood, Tex. --- the same town where Skalnik
passed himself off as an airline executive in the late 1970s. Her
mother, who declined to be interviewed for this article and whose name
is being withheld to protect her privacy, wed Skalnik after a short
courtship; she did not know that he was already married, much less that
she was his eighth wife. (Skalnik married his seventh wife shortly after
his release from prison.) Masquerading as a prosperous real estate
developer, Skalnik lavished Anderson's mother with gifts: big bouquets
of roses, jewelry, even a used Jaguar. He also began sowing division
between her and her eldest daughter.

In his campaign to undermine the 15-year-old, who disliked him from the
start, Skalnik accused her of stealing the engagement ring he had given
her mother --- a ring whose glittering gemstone, he said, was a
seven-carat diamond. Anderson, whose most fervent wish was for her
parents to get back together, saw Skalnik as an interloper, and a
calculating and tacky one at that. She was stunned when he accused her
of stealing the ring, which she suspected was actually set with a cubic
zirconium. ``He said I'd taken it,'' she told me. ``He set me up.''

As punishment, Skalnik grounded Anderson, insisting that she could be
reformed only through a punishing regimen that he ordered her to carry
out over her summer vacation, when temperatures soared into the 90s.
``Every day I had to dig holes in the ground along our fence line, under
the hot sun, with no water,'' Anderson told me. ``I was not allowed to
shower, not allowed to brush my teeth. I was only allowed to eat once a
day. I would get so faint that I would see stars.'' When Skalnik
permitted her to come inside, she had to stay in her room, shut off from
the rest of the world. Only after a month of isolation, when she was at
her most desperate and vulnerable, did Skalnik offer her a way out. ``He
came in my bedroom and said, `I have an idea that's going to make things
better between us,' '' she said.

Anderson did not speak a word about the sexual abuse that followed ---
``I didn't think anyone would believe me,'' she explained --- until late
that summer, when she summoned the courage to confide in a family
friend. ``She told me that she had to report the abuse, and that's how
it all started,'' Anderson said. The spell Skalnik seemed to have cast
over her mother was, in an instant, broken. ``As soon as my mother heard
what he'd been doing, she called the police,'' Anderson said. Skalnik
was arrested and charged with sexual assault of a child.

In the eyes of the law, Skalnik was a first-time sex offender. ``With a
prior conviction for sexual assault of a child, he would have been
looking at 25 years to life,'' said Margaret Hindman, the former
assistant district attorney in Galveston County who prosecuted him.
Instead, he faced two to 20 years. Still, Hindman pursued him with a
vigor that Pinellas County prosecutors had not. She was astonished when
I told her of Skalnik's long run as a state witness. ``This guy clearly
was grandiose, delusional and had narcissistic-personality disorder,''
she said. ``He boasted that he was with the F.B.I., that he was with the
C.I.A., and none of it checked out. It's hard to believe prosecutors
relied on him.''

Skalnik professed his innocence, but he pleaded no contest in exchange
for a 10-year prison sentence --- a deal that was not as harsh as
Anderson would have liked, though it spared her from having to endure a
trial. She was in college when Skalnik first came up for parole, and she
wrote to the parole board's members, urging them not to grant him early
release, and they abided by her wishes.

In 2002, he was released after serving a decade in prison. Rather than
register as a sex offender, as he was required to do by law, he simply
disappeared. He was arrested the following year in Middlesex County,
Mass., just west of Boston, for larceny and forgery after he stole
thousands of dollars from unsuspecting clients who had hired him under
the false belief that he was an attorney. He pleaded guilty and served
time in state prison, then fled the state around 2009 after repeatedly
violating the terms of his probation.

He managed to live under the radar for the next six years in East Texas,
where he went by the name E. Paul Smith. Claiming to be an attorney, an
undercover Homeland Security agent, an ex-fighter pilot who had been
shot down over Vietnam and a terminally ill cancer patient, he worked a
variety of small-time scams. ``He was writing up people's wills, and
doing legal work for them, and investing their money, though no one ever
saw any returns,'' said Shirley Saathoff, a retired U.S. Marshals senior
inspector who began investigating Skalnik in the summer of 2015 after
the daughter of one of his love interests figured out his real name and
looked up his criminal record. ``He hurt and used a lot of women,''
Saathoff added. Everything, even his wedding to a woman named Judy
Smith, who would have been his ninth wife, turned out to be a sham, down
to the phony marriage license he had her date and sign. (She is now Judy
Beaty.) ``Paul put just enough truth into a lie to make you believe
it,'' she told me.

When law enforcement finally caught up with him that October and
arrested him for failing to register as a sex offender, he had over 30
fake IDs in his possession --- as well as a framed law-school diploma, a
legal dictionary embossed with the words E. PAUL SMITH, ESQ., ATTORNEY
AT LAW and a handgun. After he was arrested, he asked to speak to law
enforcement. ``He wanted to cut a deal,'' James Ferris, an investigator
with the Panola County Sheriff's Department, told me one morning in his
tidy office in Carthage, Tex. ``He started telling me that he could be
useful inside the jail, and I told him I was not interested in speaking
with him further.'' Ferris was emphatic about why he wouldn't want to
work with Skalnik. ``I would never be able to say on the stand that I
believed the information he gave me was true and credible.''

\textbf{As James Dailey's} appeals slowly advanced through the courts,
his attorneys at the Capital Collateral Regional Counsel --- a state
agency that represents indigent death-row inmates --- argued that the
state had, by putting Skalnik on the stand, used false testimony to
convict him. To prove it, they pointed to the claims Skalnik himself
made in 1988, when he accused prosecutors of knowing that the
confessions he recounted were highly suspect and of concealing from
juries the rewards he was given for his testimony. But the courts were
indifferent. In a 2007 opinion, the Florida Supreme Court noted that
Skalnik's claims of prosecutorial misconduct had never been
substantiated. ``Skalnik disavowed the accusations,'' read the opinion,
and ``unequivocally stated that they were false.'' The court also
accepted the government's assurances that prosecutors had not engaged in
wrongdoing. ``The prosecutor in Dailey's case also testified that she
believed Skalnik's testimony to be truthful at the time of trial,'' its
justices wrote in their opinion. And with that, any hope of challenging
the veracity of Skalnik's testimony effectively came to an end.

Eight years later, in 2015, the Florida Commission on Offender Review
declined to recommend Dailey's case for a clemency hearing. By then,
Dailey and another man, J.D. Walton, were the only people Skalnik
testified against who remained on death row. Dailey's prospects looked
grim; after several rounds of appeals, the inexorable fact of his
execution loomed.

The following year, a new attorney at the C.C.R.C., Chelsea Shirley,
started digging into his case. Shirley, who was less than three years
out of law school, brought fresh eyes and indefatigable energy to the
decades-long case file and the effort to win Dailey a new trial. At 27,
she was younger than the case itself.

As Shirley read the numerous accounts that Dailey's co-defendant, Jack
Pearcy, had given about the night of the crime, she saw nothing to
suggest that her client had actually taken part in the murder. ``Through
the years, Pearcy suggested --- but never explicitly said --- that he
committed this crime by himself,'' Shirley told me. She was particularly
struck by a sworn statement he made to Dailey's attorneys in 1993. In
it, Pearcy divulged that he had been alone with Boggio in the
early-morning hours of May 6, 1985, making him the last known person to
see her alive; he did not say what happened to Boggio, only that he
returned home alone. ``I went in, got Jim up,'' Pearcy said of Dailey.
``I told him, `Come on, let's go smoke a couple joints, drink a beer or
something.' '' He and Dailey then drove to a nearby causeway, he said,
and began tossing a Frisbee around. ``He ended up going out in the
water,'' Pearcy said, ``while we was playing Frisbee. We drank beer, we
smoked a couple of joints.'' His account provided an explanation for why
Dailey's jeans were wet when the two men returned home. It was the same
story Dailey had told his attorneys before his own trial --- a story
they warned him sounded too far-fetched to repeat to a jury.

On April 20, 2017, Shirley drove to Sumter Correctional Institution in
Bushnell, Fla., an hour north of Tampa, to see Pearcy. She was still in
the early stages of her investigation; she did not yet know that she
would interview two men who had been incarcerated with Pearcy at
different times, who would tell her that Pearcy told them Dailey had
nothing to do with the murder. Shirley went to see Pearcy only with the
hope that he might be ready, after 30 years, to talk.

Pearcy, a compact, muscular man with penetrating blue eyes, did not seem
surprised that she had come to visit him, and he agreed to meet with
her. She began by reviewing several previous accounts he had given of
the hours surrounding Boggio's murder, in which he suggested that Dailey
was at home when he and the teenager headed out into the night. Pearcy
listened and nodded along. Finally he asked if he could look at a
document she had placed on the table between them; it was an affidavit
she had prepared that summarized his previous statements, but it
concluded with a declaration that went one step further. ``James Dailey
was not present when Shelly Boggio was killed,'' it read. ``I alone am
responsible for Shelly Boggio's death.''

Pearcy read the affidavit line by line, and when he finally spoke, his
voice was devoid of emotion. ``If you can give me a pen, I'll sign it,''
he told Shirley. He said that he would be willing to testify in court to
attest to the accuracy of the affidavit; he just wanted to tell his
mother first, he said, to prepare her. It was an astounding admission
--- and it was enough, Shirley hoped, to win her client a new trial.

Pearcy's affidavit helped persuade a judge to grant an evidentiary
hearing, which was held on Jan. 3, 2018. Shirley brought some additional
legal firepower. Laura Fernandez, a clinical lecturer and research
scholar at Yale Law School, had recently joined Dailey's legal team. She
--- along with her colleague Cyd Oppenheimer, also a Yale-trained lawyer
--- would become a driving force in the effort to overturn Dailey's
conviction. But when Pearcy was called to the stand, he had a change of
heart. He explained that he had spoken to someone with the state --- he
did not specify whom --- and was worried about how his testimony could
affect his chances for parole. ``I spoke with all my family, and they
told me I needed to do what I thought was right, but that I needed to
not make a rash decision, since my parole just got denied for seven
years,'' he said. His family had advised him, he said, to ``think about
what I was doing.'' When questioned about the truthfulness of his
affidavit, he invoked his Fifth Amendment right against
self-incrimination. The judge denied Dailey's bid for a new trial.

Dailey's lawyers appealed the decision to the Florida Supreme Court,
citing information they had uncovered that, they argued, warranted a new
trial. This included revelations about the other two jailhouse
informants, Pablo DeJesus and James Leitner, who testified against
Dailey in 1987. Travis Smith, who was incarcerated at the Pinellas
County Jail at the same time as the two informants, testified at the
evidentiary hearing that he heard them concoct a fictitious story about
Dailey, which they planned to take to prosecutors so they could win
reduced sentences. (Pablo DeJesus died in 2012; Leitner has never
publicly recanted his testimony.) The state attorney's office's records
reflect that DeJesus and Leitner --- who told jurors they would receive
no reward for their testimony --- haggled with prosecutors for reduced
sentences in the months leading up to Dailey's trial, a benefit they
were each granted after his conviction.

But Florida's highest criminal court was unmoved, finding that Smith's
account, and other evidence Dailey's lawyers presented, including proof
that Skalnik misrepresented his criminal record at Dailey's trial, had
come to light too late. ``Dailey neglects to explain why this
information could not have been discovered earlier,'' the court stated
in an opinion on Oct. 3 --- in essence blaming Dailey's lawyers for not
uncovering facts that prosecutors had spent years obfuscating.

It was the end of the road for Dailey. A week earlier, Gov. Ron DeSantis
had signed his death warrant. ``This was one of the most gruesome crimes
in the history of Pinellas County,'' DeSantis,
\href{https://www.miamiherald.com/latest-news/article220303490.html}{a
native of the county} who grew up just north of Clearwater, later told
reporters. ``This has been litigated over and over and over, and so at
some point you need to do justice.'' The day and time of Dailey's death
was set: His execution was to be carried out on Nov. 7 at 6 p.m.

\textbf{Earlier this fall,} I went to see Skalnik in a nursing home in
the East Texas town of Corsicana. I found him alone in a drab, cluttered
room where the blinds were drawn and a television was on low. He had
been released from prison in June, after having romanced the mother of
another inmate, persuading her to fill his commissary account each week.
He lay in bed, shirtless, his thinning gray hair uncombed. Even flat on
his back, he cut a shockingly large, Falstaffian figure. He was
bedridden and ill --- though with what, he did not say. Every so often,
nurses turned him so that he did not develop bedsores, and he sometimes
grimaced in pain as he spoke. ``I think I'm going to die,'' he
whispered.

During the afternoon I spent with him, and on a subsequent visit, it
became clear that the last person who could provide a deeper
understanding of Paul Skalnik was Skalnik himself. He was a master of
misdirection, sidestepping hard questions while portraying himself as
the unfairly maligned hero of a story that featured a supporting cast of
cunning and vindictive women who were after his riches. Of both charges
of child molestation, he insisted that he had been wrongly accused, a
victim of girls who lied to the authorities.

As we talked, I eyed his tattoos. His right shoulder was emblazoned with
United States Marines iconography, and his left shoulder bore the words
``From Texas to Vietnam.'' Skalnik told me that the scar on his right
knee was a result of being shot down over Laos when he fought in the
Vietnam War. In fact, his available military records show that he was
never on combat duty and never served overseas.

He insisted that his testimony in Dailey's trial, and in the many other
cases he played a role in, had been truthful. ``I never lied on the
stand,'' he told me. ``At least to the best of my knowledge.'' When I
told him of Dailey's impending execution, he was unmoved. But he seemed
surprised when I mentioned that Freddie Gaines --- the 24-year-old who
stabbed his girlfriend's ex-lover in a bar brawl --- was still in
prison, 36 years later. ``I think that was a crime of passion,'' Skalnik
said. ``He doesn't need life,'' he added, of Gaines's sentence. ``I'd
give him 10 and let him go home.''

When I reminded Skalnik that he was the witness whose testimony
established premeditation at Gaines's trial, he appeared shocked. ``This
was your testimony, that he had planned this,'' I reminded him. ``Does
that ring true to you? Do you think he told you that?''

``No!'' Skalnik cried. ``No.'' He shook his head resolutely. But when I
later tried to return to his apparent recantation, his tone shifted.
There is no statute of limitations on perjury in Florida in capital
felony cases, and Skalnik was reluctant to reverse himself. ``I won't
retract what I said,'' he told me. ``Whatever I testified to was fact.''

Proudly, he told me more than once, ``I never lost a case.''

A week later, I went to Florida State Prison in Raiford, west of
Jacksonville, to interview Dailey. He had recently caught an unexpected
break. On Oct. 23, a Federal District Court judge granted him
\href{https://www.tampabay.com/news/crime/2019/10/23/federal-judge-grants-james-dailey-temporary-stay-of-execution/}{a
limited stay of execution}, to provide his newly appointed federal
attorneys more time to research and present their appeals.

Florida State Prison is a monolithic, 1960s-era penitentiary hemmed in
on all sides by level farmland and coils of razor wire. It is also home
to the so-called death house, where inmates with active death warrants
are held in the weeks leading up to their executions. For our meeting,
Dailey was led into the tiny, fluorescent-lit room where final
interviews with the condemned are conducted. His hands were manacled to
a chain belt at his waist, and his feet were bound by leg irons. At 73,
he moved slowly. Behind his thick-framed, prison-issued glasses, he had
heavy circles under his eyes.

Though Dailey had been granted a stay, it was clear that what lay ahead
weighed heavily upon him. He had been convicted in the era of Old
Sparky, the straight-backed oak chair in which 240 prisoners went to
their deaths before 2000, when the Florida Legislature made lethal
injection the preferred method of execution. As Dailey passed days, and
then weeks, in the death house, he experienced another, different kind
of torment: the anticipation of waiting. His cell was just 30 feet from
the execution chamber. At the time of our interview, he had already been
measured for his state-issued burial suit.

After more than three decades in prison, Dailey seemed even-tempered,
agreeable, even acquiescent. A lieutenant at the prison would later take
me aside to tell me that Dailey's disciplinary record was almost
nonexistent --- a feat for anyone who has been incarcerated for so long.

I asked Dailey about an observation his mother made when she testified
during the penalty phase of his trial, hoping to persuade the jury to
show him mercy. ``I sent two lovely young men to the Air Force and
Marines,'' she said of Dailey and one of his brothers, ``and they came
back, and they were different boys.'' Dailey closed his eyes at the
memory. ``I was messed up,'' he said. He explained that he was sent to
\href{https://www.historynet.com/big-ears-three-and-the-battle-of-bien-hoa.htm}{Bien
Hoa}, an air base northeast of Saigon, in 1968 during the Tet Offensive;
after rocket attacks hit the base, he and other soldiers assisted the
injured --- men whose limbs had been blown off, their faces ravaged, who
drifted between life and death. ``I wasn't built for that,'' he told me.
``I started drinking real bad.'' To sleep, he had to finish as much as a
fifth of alcohol. After three tours in Vietnam, he found no relief when
he returned home. He sank into addiction and became, he told me, ``a
vagabond.''

His relationship with Pearcy, whom he met in a bar in Kansas, was
elemental. ``We smoked dope together,'' Dailey said simply. He insisted
that he had nothing to do with Boggio's murder and that he believed
Pearcy was driving Boggio home when Pearcy left the house with the
teenager, and he went to sleep.

Dailey understood that any chance of proving his innocence was lost when
Skalnik took the stand. ``I never talked to Paul Skalnik in my life,''
he told me, his voice rising. ``We all knew he was an infamous snitch
and an ex-police officer. We knew everything about him. We knew how many
guys he had snitched on. There wasn't any hiding the fact. The officers
would tell us! The officers that worked in the dang jail, they'd say,
`Don't talk to him.'

``It was impossible for him to get my confession the way he said he got
it,'' he continued. Even if Skalnik --- under protective custody, and a
stranger to Dailey, no less --- had somehow managed to strike up a
conversation, the distance between the two men would have prohibited any
sort of meaningful or intimate discussion. ``He would've had to holler
at me,'' Dailey said. ``And I would have had to speak loudly to confess
to him.''

Dailey told me something that he'd thought back to many times over the
years: He had been moved from one cell to another shortly before his
trial was slated to begin. Jail logs, in which inmates' cell assignments
are recorded, confirm that on May 1, 1987 --- just 18 days before
Dailey's scheduled trial --- Dailey was moved from the lower G wing of
the jail to the upper G wing, where Skalnik was held. ``Right away, I
told the sergeant, I said, `Get me out of here,' '' Dailey told me.
`` `This is a damn setup.' '' Skalnik claimed to have elicited Dailey's
confession just two days later. Five days after that, he was talking to
the state attorney's office. It was one of many troubling facts that the
jury in Dailey's trial never heard.

Dailey's stay of execution will remain in place through Dec. 30. After
that, Governor DeSantis can set a new execution date for as soon as
January. When that day comes, Dailey will be asked to walk from his cell
to the execution chamber, where he will lie down on the gurney. Leather
restraining straps will be fastened across his body, and an IV line will
be inserted into his arm. Finally, the signal will be given to the
executioner to begin the flow of lethal drugs. At that moment, the State
of Florida will be asking its citizens to trust that Dailey killed
Shelly Boggio that night beside the dark water, and that he received a
fair trial, and that justice has finally been served. It will be asking
them, as it has time and time again, to believe the word of Paul
Skalnik.

Advertisement

\protect\hyperlink{after-bottom}{Continue reading the main story}

\hypertarget{site-index}{%
\subsection{Site Index}\label{site-index}}

\hypertarget{site-information-navigation}{%
\subsection{Site Information
Navigation}\label{site-information-navigation}}

\begin{itemize}
\tightlist
\item
  \href{https://help.nytimes3xbfgragh.onion/hc/en-us/articles/115014792127-Copyright-notice}{©~2020~The
  New York Times Company}
\end{itemize}

\begin{itemize}
\tightlist
\item
  \href{https://www.nytco.com/}{NYTCo}
\item
  \href{https://help.nytimes3xbfgragh.onion/hc/en-us/articles/115015385887-Contact-Us}{Contact
  Us}
\item
  \href{https://www.nytco.com/careers/}{Work with us}
\item
  \href{https://nytmediakit.com/}{Advertise}
\item
  \href{http://www.tbrandstudio.com/}{T Brand Studio}
\item
  \href{https://www.nytimes3xbfgragh.onion/privacy/cookie-policy\#how-do-i-manage-trackers}{Your
  Ad Choices}
\item
  \href{https://www.nytimes3xbfgragh.onion/privacy}{Privacy}
\item
  \href{https://help.nytimes3xbfgragh.onion/hc/en-us/articles/115014893428-Terms-of-service}{Terms
  of Service}
\item
  \href{https://help.nytimes3xbfgragh.onion/hc/en-us/articles/115014893968-Terms-of-sale}{Terms
  of Sale}
\item
  \href{https://spiderbites.nytimes3xbfgragh.onion}{Site Map}
\item
  \href{https://help.nytimes3xbfgragh.onion/hc/en-us}{Help}
\item
  \href{https://www.nytimes3xbfgragh.onion/subscription?campaignId=37WXW}{Subscriptions}
\end{itemize}
