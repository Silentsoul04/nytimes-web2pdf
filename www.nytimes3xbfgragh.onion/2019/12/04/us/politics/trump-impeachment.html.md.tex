Sections

SEARCH

\protect\hyperlink{site-content}{Skip to
content}\protect\hyperlink{site-index}{Skip to site index}

\href{https://www.nytimes3xbfgragh.onion/section/politics}{Politics}

\href{https://myaccount.nytimes3xbfgragh.onion/auth/login?response_type=cookie\&client_id=vi}{}

\href{https://www.nytimes3xbfgragh.onion/section/todayspaper}{Today's
Paper}

\href{/section/politics}{Politics}\textbar{}Scholars Call Trump's
Actions on Ukraine an Impeachable Abuse of Power

\url{https://nyti.ms/2Rn3TRN}

\begin{itemize}
\item
\item
\item
\item
\item
\item
\end{itemize}

Advertisement

\protect\hyperlink{after-top}{Continue reading the main story}

Supported by

\protect\hyperlink{after-sponsor}{Continue reading the main story}

\hypertarget{scholars-call-trumps-actions-on-ukraine-an-impeachable-abuse-of-power}{%
\section{Scholars Call Trump's Actions on Ukraine an Impeachable Abuse
of
Power}\label{scholars-call-trumps-actions-on-ukraine-an-impeachable-abuse-of-power}}

Democrats and Republicans clashed over the Constitution and President
Trump's conduct as the House Judiciary Committee formally began its
impeachment proceedings.

\includegraphics{https://static01.graylady3jvrrxbe.onion/images/2019/12/06/us/politics/04dc-impeach1/04dc-impeach1-videoSixteenByNine3000.jpg}

\href{https://www.nytimes3xbfgragh.onion/by/nicholas-fandos}{\includegraphics{https://static01.graylady3jvrrxbe.onion/images/2018/11/06/multimedia/author-nicholas-fandos/author-nicholas-fandos-thumbLarge-v2.png}}\href{https://www.nytimes3xbfgragh.onion/by/michael-d-shear}{\includegraphics{https://static01.graylady3jvrrxbe.onion/images/2018/06/13/multimedia/author-michael-d-shear/author-michael-d-shear-thumbLarge-v2.png}}

By \href{https://www.nytimes3xbfgragh.onion/by/nicholas-fandos}{Nicholas
Fandos} and
\href{https://www.nytimes3xbfgragh.onion/by/michael-d-shear}{Michael D.
Shear}

\begin{itemize}
\item
  Published Dec. 4, 2019Updated Dec. 31, 2019
\item
  \begin{itemize}
  \item
  \item
  \item
  \item
  \item
  \item
  \end{itemize}
\end{itemize}

WASHINGTON --- The House of Representatives on Wednesday opened a
critical new phase of the
\href{https://www.nytimes3xbfgragh.onion/2019/12/13/us/politics/impeachment-vote.html}{impeachment}
proceedings against
\href{https://www.nytimes3xbfgragh.onion/2019/12/13/us/politics/impeachment-vote.html}{President
Trump}, featuring legal scholars vigorously debating whether his conduct
and the available evidence rose to the constitutional threshold
necessary for his removal from office.

In a daylong hearing convened by the Judiciary Committee, three
constitutional scholars invited by Democrats testified that evidence of
Mr. Trump's efforts to pressure Ukraine for political gain clearly met
the definition of an impeachable abuse of power. They said his defiance
of Congress's investigative requests was further grounds for charging
him.

A fourth scholar invited by Republicans disagreed, warning that
Democrats were barreling forward with a shoddy case for the president's
removal based on inadequate evidence, and risked damaging the integrity
of a sacred process enshrined in the Constitution.

The spirited exchange unfolded as the Judiciary Committee began
determining which
\href{https://www.nytimes3xbfgragh.onion/2019/12/13/us/politics/impeachment-vote.html}{impeachment}
charges to lodge against Mr. Trump based on an investigation by the
House Intelligence Committee. The president abused his power, sought to
subvert an American election and endangered national security when he
pressured Ukraine for political favors, Democrats said.

In an investigative report released on Tuesday, they also concluded that
Mr. Trump pressured President Volodymyr Zelensky of Ukraine to announce
investigations into former Vice President Joseph R. Biden Jr. and other
Democrats, while withholding a White House meeting and \$391 million in
vital security assistance.

Within days and despite unanimous Republican opposition, the panel could
begin drafting and debating articles of impeachment, eyeing a vote by
the full House before Christmas. Democrats signaled on Wednesday that
the charges could be based not just on the Ukraine matter but also on
earlier evidence that Mr. Trump may have obstructed justice when he
sought to thwart federal investigators scrutinizing his campaign's ties
to Russia's election interference operation.

But on Wednesday, in the wood and plasterwork-adorned chambers of the
House Ways and Means Committee, lawmakers and the scholars they invited
sparred over history and precedent as they prepared to embark on the
third impeachment of a sitting president in American history.

Invoking arguments between the framers of the Constitution and
impeachment precedents dating to monarchical England, the scholars
dissected the quality of the evidence before the House and how to define
at least one possible impeachment charge, bribery.

The three law professors invited by Democrats said that Mr. Trump's
behavior was not only an egregious abuse of his power for personal gain,
but the textbook definition of the kind of conduct that the nation's
founders sought to guard against when they drafted the impeachment
clause of the Constitution.

``If what we're talking about is not impeachable, then nothing is
impeachable,''
\href{https://www.nytimes3xbfgragh.onion/2019/12/04/us/politics/michael-gerhardt.html}{Michael
J. Gerhardt}, a professor at the University of North Carolina, told the
panel. ``This is precisely the misconduct that the framers created the
Constitution, including impeachment, to protect against.''

\includegraphics{https://static01.graylady3jvrrxbe.onion/images/2019/12/04/us/politics/04dc-impeach2-sub/merlin_165433521_5367d503-b16f-4016-a601-3fbb582d54af-articleLarge.jpg?quality=75\&auto=webp\&disable=upscale}

But a fourth witness,
\href{https://www.nytimes3xbfgragh.onion/2019/12/04/us/politics/jonathan-turley.html}{Jonathan
Turley}, a law professor at George Washington University, cautioned
House Democrats against rushing into an impeachment based on an
incomplete set of facts and overly broad standards. He conceded that the
president's conduct may have been impeachable, but said Democrats risked
tainting the validity of the Constitution's only remedy for grave
presidential misconduct outside an election.

``I am concerned about lowering impeachment standards to fit a paucity
of evidence and an abundance of anger,'' he said. ``To impeach a
president on such a record would be to expose every future president to
the same type of inchoate impeachment.''

In offering the argument, Mr. Turley, who said he had not voted for Mr.
Trump and did not condone his behavior, handed Republicans what could be
a potent counterpoint to put to a divided public.

The dispute unfolded as members of both parties braced for a historic
confrontation over Mr. Trump's impeachment.

``Are you ready?'' Speaker Nancy Pelosi, Democrat of California, asked a
roomful of Democrats meeting behind closed doors on Wednesday morning
before the Judiciary Committee's proceedings began. They were, the
lawmakers answered in unison, according to people in attendance who
discussed the session on the condition of anonymity to describe a
private gathering.

Nearby, Vice President Mike Pence delivered his own battle cry to
Republicans at their weekly conference meeting. ``Turn up the heat'' on
House Democrats, he instructed, according to an official familiar with
his comments who spoke about them on the condition of anonymity.

The panel invited lawyers for Mr. Trump to participate in Wednesday's
hearing, but they declined, citing what they called an inherently unfair
process, and instead huddled privately with Senate Republicans over
lunch to discuss a likely Senate trial. The White House counsel, Pat
Cipollone, told senators that the president was eager to present a case
for his defense in the Senate, should the House vote to impeach him.

Across the Capitol at the Judiciary Committee, the scholars debated
whether Mr. Trump's actions met the standards of impeachment laid out in
the Constitution, which says a president can be removed for ``treason,
bribery or other high crimes and misdemeanors.''

\href{https://www.nytimes3xbfgragh.onion/2019/12/04/us/politics/noah-feldman.html}{Noah
Feldman}, a professor at Harvard,
\href{https://int.graylady3jvrrxbe.onion/data/documenthelper/6550-noah-feldman-testimony/8457c4c46d96010b546e/optimized/full.pdf\#page=1}{argued
that Mr. Trump's decision} to withhold a White House meeting and
military assistance from Ukraine while he demanded political favors from
its president was a classic impeachable abuse of power.

Image

A video of testimony by Gordon D. Sondland, the American ambassador to
the European Union, played during the hearing.Credit...Erin Schaff/The
New York Times

``The essential definition of high crimes and misdemeanors is the abuse
of office,'' he said. ``The framers considered the office of the
presidency to be a public trust.''

\href{https://www.nytimes3xbfgragh.onion/2019/12/04/us/politics/pamela-karlan.html}{Pamela
S. Karlan}, a Stanford law professor, went further, arguing that Mr.
Trump's actions toward Ukraine could constitute another offense outlined
in the Constitution: bribery. She defined that offense as ``when an
official solicited, received or offered a personal favor or benefit to
influence official action.''

``If you conclude that he asked for the investigation of Vice President
Biden and his son for political reasons, that is to aid his re-election,
then, yes, you have bribery here,'' Ms. Karlan said.

But Mr. Turley argued that Democrats were tarnishing the very concept of
impeachment by sloppily applying what should be an ironclad set of
standards. He said Democrats and the other witnesses were interpreting
the concept of bribery too broadly to describe Mr. Trump's conduct.

``This isn't improvisational jazz --- close enough is not good enough,''
Mr. Turley said. ``If you're going to accuse a president of bribery, you
need to make it stick, because you're trying to remove a duly elected
president of the United States.''

Mr. Turley also disputed that Mr. Trump could be fairly charged with
obstruction of Congress. Without going to court to ask a judge to
enforce their subpoenas, he argued, Democrats had a case that lacked
important validation and could even be an abuse of the House's power.

Within the first half-hour of the hearing, it was clear that the
proceedings had entered a more cantankerous stage. The panel, stacked
with some of the House's most ideologically progressive and conservative
lawmakers, lived up to its reputation. Republicans repeatedly sought to
halt the proceedings with parliamentary demands, while the Democrats
pressed forward.

Image

Republicans repeatedly tried to grind the proceedings to a halt with
parliamentary demands.Credit...Erin Schaff/The New York Times

``President Trump welcomed foreign interference in the 2016 election,''
said Representative Jerrold Nadler, Democrat of New York and the
Judiciary Committee's chairman, trying to establish a web of misconduct
by Mr. Trump beyond Ukraine. ``He demanded it for the 2020 election.''

Mr. Nadler added: ``The president has shown us his pattern of conduct.
If we do not act to hold him in check --- now --- President Trump will
almost certainly try again to solicit interference in the election for
his personal, political benefit.''

Pressing his colleagues to ``stand behind the oath you have taken,'' Mr.
Nadler said, ``Our democracy depends on it.''

Questioning the witnesses, Democrats hinted at what articles of
impeachment they are considering. They included abuse of power and
bribery related to the Ukraine matter and obstruction of justice
stemming from Mr. Trump's attempts to impede the investigation by Robert
S. Mueller III, the special counsel who investigated the campaign's ties
to Russian election interference. They also appeared to be building a
case to charge Mr. Trump with obstruction of Congress for his refusal to
allow aides to testify in the impeachment inquiry and a blockade on
documentary evidence requested by the House.

Republicans on the panel, some of Mr. Trump's most ardent defenders,
sought to portray the case against him as a political hit job. And they
disputed forcefully that Democrats had proved that Mr. Trump directed a
pressure campaign on Ukraine.

``This is not an impeachment,'' said Representative Doug Collins,
Republican of Georgia. ``This is simply a railroad job, and today is a
waste of time.''

Mr. Trump,
\href{https://www.nytimes3xbfgragh.onion/2019/12/04/world/europe/nato-live-updates-trump-macron.html}{in
Britain for the 70th anniversary} of the North Atlantic Treaty
Organization, called impeachment ``a dirty word that should only be used
in special occasions.'' When the hearing concluded, his spokeswoman
dismissed the proceedings as a ``sham process,'' proclaimed the
president's innocence, and branded the opinions of the three legal
scholars called by Democrats as the product of ``political bias.''

Image

Pamela S. Karlan and Noah Feldman made the case that Mr. Trump abused
his power.Credit...Anna Moneymaker/The New York Times

The Judiciary Committee is expected to convene additional hearings
before it drafts and debates impeachment articles. In the coming days,
the panel will almost certainly hear a formal presentation of evidence
from Democratic and Republican lawyers for the Intelligence Committee on
the conclusions of their investigation into the Ukraine matter.

The panel could also hold another session to allow Mr. Trump's lawyers
to issue a formal defense, including by calling witnesses, if the White
House requests it. Mr. Nadler has given the president and his team until
Friday to decide whether to participate. Democrats must also decide
whether to grant Republicans a minority day of hearings they formally
requested on Wednesday.

As the session wore on, tensions flared between lawmakers and even the
witnesses.

Representative Andy Biggs, Republican of Arizona, sought to undercut the
scholars invited by the Democrats, reading aloud critical statements
they had made about Mr. Trump and accusing them of setting aside the law
in favor of ``preconceived notions and bias.'' Democrats implied Mr.
Turley, who argued in favor of Mr. Clinton's impeachment two decades
ago, was being disingenuous this time around.

\href{https://www.nytimes3xbfgragh.onion/2019/12/04/us/politics/pamela-karlan.html}{Ms.
Karlan}'s attempt at a pun to make a point about titles of nobility ---
``The president can name his son Barron; he can't make him a baron'' ---
drew howls of outrage from the Republicans including Melania Trump, the
first lady, who called the invocation of her teenager's name ``very
angry and obviously biased public pandering.''

Ms. Karlan, who tangled with Republicans several times during the
hearing, interrupted the proceedings to apologize for the remark,
saying, ``It was wrong of me to do that.'' But she added that she wished
the president would also apologize for the things that he had done.

Image

Spectators at the hearing on Wednesday.Credit...Erin Schaff/The New York
Times

Sheryl Gay Stolberg, Emily Cochrane and Catie Edmondson contributed
reporting.

Advertisement

\protect\hyperlink{after-bottom}{Continue reading the main story}

\hypertarget{site-index}{%
\subsection{Site Index}\label{site-index}}

\hypertarget{site-information-navigation}{%
\subsection{Site Information
Navigation}\label{site-information-navigation}}

\begin{itemize}
\tightlist
\item
  \href{https://help.nytimes3xbfgragh.onion/hc/en-us/articles/115014792127-Copyright-notice}{©~2020~The
  New York Times Company}
\end{itemize}

\begin{itemize}
\tightlist
\item
  \href{https://www.nytco.com/}{NYTCo}
\item
  \href{https://help.nytimes3xbfgragh.onion/hc/en-us/articles/115015385887-Contact-Us}{Contact
  Us}
\item
  \href{https://www.nytco.com/careers/}{Work with us}
\item
  \href{https://nytmediakit.com/}{Advertise}
\item
  \href{http://www.tbrandstudio.com/}{T Brand Studio}
\item
  \href{https://www.nytimes3xbfgragh.onion/privacy/cookie-policy\#how-do-i-manage-trackers}{Your
  Ad Choices}
\item
  \href{https://www.nytimes3xbfgragh.onion/privacy}{Privacy}
\item
  \href{https://help.nytimes3xbfgragh.onion/hc/en-us/articles/115014893428-Terms-of-service}{Terms
  of Service}
\item
  \href{https://help.nytimes3xbfgragh.onion/hc/en-us/articles/115014893968-Terms-of-sale}{Terms
  of Sale}
\item
  \href{https://spiderbites.nytimes3xbfgragh.onion}{Site Map}
\item
  \href{https://help.nytimes3xbfgragh.onion/hc/en-us}{Help}
\item
  \href{https://www.nytimes3xbfgragh.onion/subscription?campaignId=37WXW}{Subscriptions}
\end{itemize}
