Sections

SEARCH

\protect\hyperlink{site-content}{Skip to
content}\protect\hyperlink{site-index}{Skip to site index}

\href{https://myaccount.nytimes3xbfgragh.onion/auth/login?response_type=cookie\&client_id=vi}{}

\href{https://www.nytimes3xbfgragh.onion/section/todayspaper}{Today's
Paper}

When the Whole-Animal Ethos Includes the Dinner Plates

\url{https://nyti.ms/34ROrkx}

\begin{itemize}
\item
\item
\item
\item
\item
\end{itemize}

Advertisement

\protect\hyperlink{after-top}{Continue reading the main story}

Supported by

\protect\hyperlink{after-sponsor}{Continue reading the main story}

In Studio

\hypertarget{when-the-whole-animal-ethos-includes-the-dinner-plates}{%
\section{When the Whole-Animal Ethos Includes the Dinner
Plates}\label{when-the-whole-animal-ethos-includes-the-dinner-plates}}

The ceramist Gregg Moore is helping to ensure that absolutely nothing
goes to waste at the New York restaurant Blue Hill at Stone Barns.

\includegraphics{https://static01.graylady3jvrrxbe.onion/images/2019/12/08/t-magazine/08tmag-bonechina-slide-CULN/08tmag-bonechina-slide-CULN-articleLarge.jpg?quality=75\&auto=webp\&disable=upscale}

By Nick Marino

\begin{itemize}
\item
  Published Dec. 5, 2019Updated Dec. 6, 2019
\item
  \begin{itemize}
  \item
  \item
  \item
  \item
  \item
  \end{itemize}
\end{itemize}

As the chef and co-owner of the pioneering Hudson Valley farm-to-table
restaurant
\href{https://www.nytimes3xbfgragh.onion/2015/05/11/t-magazine/blue-hill-stone-barns-sothebys-gallery.html}{Blue
Hill at Stone Barns}, Dan Barber is besieged by artisans trying to
collaborate with him. But one day, in the spring of 2015, he heard from
a Pennsylvania ceramist named \href{http://www.greggfmoore.com/}{Gregg
Moore} and instantly knew that this call was different. ``He started
talking to me in a way that I hadn't heard anybody who designs anything
--- especially plateware --- talk to me,'' Barber says. ``Just asking
me, not in an accusatory way, `If you're thinking about food and the way
it's grown and raised and everything that goes into the creation of a
plate, why are you not thinking about the plate itself?'''

Together, the chef and potter conceived the purest distillation of the
farm's whole-animal philosophy: china made from the bones of its own
cows. Barber would raise the cows on pasture, as always, and after
slaughtering them for meat, he'd give the bones (mostly femurs, their
marrow consumed) to Moore, who would transform them into tableware. The
finished china, which comes in three pieces --- a plate, a bowl and a
water cup --- is so light and white that it seems to float. When it
became the restaurant's default table setting earlier this year, guests
couldn't keep their hands off it. ``Everyone immediately grabs the plate
and touches it, looks at it, turns it around,'' says Philippe Gouze, the
restaurant's director of operations. ``It's an instant attraction.''

Image

Moore's china cups, plates and bowls for Blue Hill at Stone Barns, made
from cow bones, seen here in various stages of processing.Credit...Sarah
Stolfa

Moore, 44, the son of a New Jersey butcher, is a visual and performing
arts professor at Arcadia University, outside Philadelphia, and a
rigorous student of his craft. His bone china recipe traces back to the
one formulated circa 1799 by Josiah Spode, the godfather of
\href{https://www.nytimes3xbfgragh.onion/1992/02/16/travel/in-the-cradle-of-english-china.html}{British
porcelain}. Beginning in the mid-18th century, European ceramists
attempted to replicate smooth porcelain from China (hence the name),
which had been invented over a thousand years prior. The English artist
Thomas Frye first experimented with bone ash --- the remnants after
water, fat and connective tissue are burned off --- before Spode
fine-tuned the process, creating a durable but delicate product. ``What
differentiates bone china from all other ceramic materials is that it's
made out of an element that was once living,'' Moore says. ``So it has
the ability to express, if we look carefully enough, the quality of life
of that living entity.'' Since Blue Hill cows roam freely, for instance,
their bones are stronger, which allows Moore to fire his wares at higher
temperatures, lending them a creamy translucency.

Moore retrieves buckets of bones from Blue Hill after Barber's team has
cleaned them by boiling them in a stockpot for several hours. At this
stage, they are a ruddy mahogany and may still carry a slight bovine
odor. Back in his studio, Moore boils them again, then fires them in a
gas kiln, which removes any remaining organic material and transforms
the living tissue into pure calcium phosphate. He mixes the ash with
water in a ball mill, which rotates to create a kind of sludge. From
there, he dries the slurry, pulverizes it into powder, combines it with
more water and two other minerals from Spode's method --- kaolin and
Cornish stone --- and then molds the pieces and flash-fires them in an
oxidized electric kiln at about 2,400 degrees Fahrenheit. While his
plates and bowls come in relatively conventional circular shapes, the
cups warp from the heat: One might look like a kidney bean, another a
Modernist bathtub.

\includegraphics{https://static01.graylady3jvrrxbe.onion/images/2019/12/08/t-magazine/08tmag-bonechina-slide-R971/08tmag-bonechina-slide-R971-articleLarge.jpg?quality=75\&auto=webp\&disable=upscale}

Given that Blue Hill dinners can last well over three hours and comprise
some 30 different dishes, those cups have taken on outsize importance.
(``If you think about the tabletop,'' Moore says, ``what's the one
object that remains the entire meal?'') And yet his other pieces speak
more directly to the creatures that begot them: The plates hold pats of
butter churned from the udder of a single cow; the bowls cradle a
sweetened custard made from their milk. ``My work, when it leaves the
studio, is not done,'' Moore says. ``It's just started. Its life is on
the table and in the kitchen.''

Advertisement

\protect\hyperlink{after-bottom}{Continue reading the main story}

\hypertarget{site-index}{%
\subsection{Site Index}\label{site-index}}

\hypertarget{site-information-navigation}{%
\subsection{Site Information
Navigation}\label{site-information-navigation}}

\begin{itemize}
\tightlist
\item
  \href{https://help.nytimes3xbfgragh.onion/hc/en-us/articles/115014792127-Copyright-notice}{©~2020~The
  New York Times Company}
\end{itemize}

\begin{itemize}
\tightlist
\item
  \href{https://www.nytco.com/}{NYTCo}
\item
  \href{https://help.nytimes3xbfgragh.onion/hc/en-us/articles/115015385887-Contact-Us}{Contact
  Us}
\item
  \href{https://www.nytco.com/careers/}{Work with us}
\item
  \href{https://nytmediakit.com/}{Advertise}
\item
  \href{http://www.tbrandstudio.com/}{T Brand Studio}
\item
  \href{https://www.nytimes3xbfgragh.onion/privacy/cookie-policy\#how-do-i-manage-trackers}{Your
  Ad Choices}
\item
  \href{https://www.nytimes3xbfgragh.onion/privacy}{Privacy}
\item
  \href{https://help.nytimes3xbfgragh.onion/hc/en-us/articles/115014893428-Terms-of-service}{Terms
  of Service}
\item
  \href{https://help.nytimes3xbfgragh.onion/hc/en-us/articles/115014893968-Terms-of-sale}{Terms
  of Sale}
\item
  \href{https://spiderbites.nytimes3xbfgragh.onion}{Site Map}
\item
  \href{https://help.nytimes3xbfgragh.onion/hc/en-us}{Help}
\item
  \href{https://www.nytimes3xbfgragh.onion/subscription?campaignId=37WXW}{Subscriptions}
\end{itemize}
