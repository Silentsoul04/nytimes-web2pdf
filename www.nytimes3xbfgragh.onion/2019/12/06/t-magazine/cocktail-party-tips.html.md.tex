Sections

SEARCH

\protect\hyperlink{site-content}{Skip to
content}\protect\hyperlink{site-index}{Skip to site index}

\href{https://myaccount.nytimes3xbfgragh.onion/auth/login?response_type=cookie\&client_id=vi}{}

\href{https://www.nytimes3xbfgragh.onion/section/todayspaper}{Today's
Paper}

How Can I Sound Smart at a Cocktail Party?

\url{https://nyti.ms/387YPad}

\begin{itemize}
\item
\item
\item
\item
\item
\end{itemize}

Advertisement

\protect\hyperlink{after-top}{Continue reading the main story}

Supported by

\protect\hyperlink{after-sponsor}{Continue reading the main story}

\hypertarget{how-can-i-sound-smart-at-a-cocktail-party}{%
\section{How Can I Sound Smart at a Cocktail
Party?}\label{how-can-i-sound-smart-at-a-cocktail-party}}

``The feeling of being an outsider, excluded because you don't have
access to certain signifiers, is a powerful one,'' says one of our
advice columnists.

\includegraphics{https://static01.graylady3jvrrxbe.onion/images/2019/12/06/t-magazine/06tmag-therapist-slide-68K5/06tmag-therapist-slide-68K5-superJumbo.jpg}

By Ligaya Mishan

\begin{itemize}
\item
  Dec. 6, 2019
\item
  \begin{itemize}
  \item
  \item
  \item
  \item
  \item
  \end{itemize}
\end{itemize}

\emph{In T's advice column}
\href{https://www.nytimes3xbfgragh.onion/column/culture-therapist?module=inline}{\emph{Culture
Therapist}}\emph{, either}
\href{https://www.nytimes3xbfgragh.onion/by/ligaya-mishan?module=inline}{\emph{Ligaya
Mishan}} \emph{or}
\href{https://www.nytimes3xbfgragh.onion/by/megan-o-grady?module=inline}{\emph{Megan
O'Grady}} \emph{solves your problems using art. Have a question? Need
some comfort? Email us at}
\href{mailto:advice@NYTimes.com}{\emph{advice@NYTimes.com}}\emph{.}

\emph{Q: Dear Culture Therapist,}

I'm a 30-year-old American who recently moved to Europe, where I have a
high-powered job in the fashion and design industry. My problem is this:
I feel I don't know anything. Cultural references go over my head; I
don't know as much about modern art, design, film and architecture as
people assume I do. What and who do you feel are the aesthetic schools
of thought, artistic developments and people that a culturally fluent
person needs to know about (at least to fake their way through a
cocktail party)? \emph{--- Name Withheld}

A: Surely none of us know all that we should. (Unless you're
\href{https://www.nytimes3xbfgragh.onion/2019/10/24/magazine/london-review-of-books-mary-kay-wilmers.html}{Mary-Kay
Wilmers}, perhaps.) When first asked to write this
\href{https://www.nytimes3xbfgragh.onion/column/culture-therapist}{column},
I questioned if my cultural knowledge was deep enough, and still do. But
how else can we approach the expanse and tumult of modern life except
with humility and an acute awareness of our limitations --- along with a
recognition that there is only so much information we can absorb in an
increasingly accelerated world?

One thing you should know is that you were hired for your job --- which
sounds like a coveted one, doubtless vied for by a number of
accomplished candidates --- because you are qualified for it, which
means you already have a sizable cache of cultural knowledge. May I
gently suggest that, like many of us, you may be experiencing a case of
\href{https://www.nytimes3xbfgragh.onion/guides/working-womans-handbook/overcome-impostor-syndrome}{impostor
syndrome}, uncertain that you've properly earned your spot in the ranks
of the culturati? Remember that the people around you don't necessarily
know more; given that you're an expatriate, removed from your natural
element, they likely possess a different (as opposed to superior) set of
reference points.

\includegraphics{https://static01.graylady3jvrrxbe.onion/images/2019/12/06/t-magazine/06tmag-therapist-slide-KACP/06tmag-therapist-slide-KACP-articleLarge.jpg?quality=75\&auto=webp\&disable=upscale}

Still, the feeling of being an outsider, excluded because you don't have
access to certain signifiers, is a powerful one. It's a theme that runs
through all of art and literature. One of the more compelling takes on
it in recent years is the intense emotional isolation of the main
character in the 2018 film
``\href{https://www.nytimes3xbfgragh.onion/watching/titles/movies/674682}{Burning},''
from the Korean director Lee Chang-dong, loosely based on a
\href{https://www.newyorker.com/magazine/1992/11/02/barn-burning}{story}
by
\href{https://www.nytimes3xbfgragh.onion/topic/person/haruki-murakami}{Haruki
Murakami} (which in turn was inspired by a
\href{http://faculty.weber.edu/jyoung/english\%206710/barn\%20burning.pdf}{story}
by
\href{https://www.nytimes3xbfgragh.onion/topic/person/william-faulkner}{William
Faulkner}). A young would-be writer, abandoned by his mother as a child
and raised on a dilapidated farm by a father prone to violence, is
reluctantly drawn into the orbit of a rich, worldly urbanite, whose
cosseted milieu the poorer man lacks the tools to navigate or
comprehend. The film's aura of malignant mystery hints at a hidden
crime, both literal and in the stark inequity of the lives it portrays.
It's not a comforting vision but an incisive one --- and worthy of
archiving in your collection of cultural allusions.

I have no secret shortcuts to gaining greater fluency beyond being open
and attentive to what's around you. On a practical level, you might set
yourself a weekly regimen: Go to art galleries, lectures and concerts.
Dip into the PBS
``\href{https://www.nytimes3xbfgragh.onion/watching/titles/civilizations}{Civilizations}''
series and take an
\href{https://www.thegreatcoursesplus.com/history-of-european-art}{online
course} in European art. Make your way through the
\href{https://www.criterion.com/}{Criterion Collection}, which since
last spring has
\href{https://www.nytimes3xbfgragh.onion/2018/11/16/arts/criterion-channel-streaming-spring-2019.html}{offered
a streaming service} of renowned international, independent and
art-house films alongside commentaries and interviews with filmmakers.
Browse sites like \href{https://www.aldaily.com/}{Arts \& Letters
Daily}, \href{https://hyperallergic.com/}{Hyperallergic}, Artforum,
\href{https://www.itsnicethat.com/}{It's Nice That} and --- because
music should also be on your list ---
\href{https://www.therestisnoise.com/}{The Rest Is Noise}. (While you're
at it, listen to this year's
\href{https://www.nytimes3xbfgragh.onion/2019/04/15/business/media/pulitzer-prize-winners.html}{winner
of the Pulitzer Prize in music}, the Tennessee-born composer Ellen
Reid's opera
``\href{https://ellenreidmusic.com/work/p-r-i-s-m/}{Prism},'' which
juxtaposes whispers and shimmery pointillism, static and extremities of
sound to represent the fragmentary aftermath of a sexual assault. It
speaks to both a particular musical and larger cultural and political
moment.)

\includegraphics{https://static01.graylady3jvrrxbe.onion/images/2019/12/06/t-magazine/06tmag-haco/06tmag-haco-superJumbo.png}

As you wander, let yourself fall down rabbit holes, and trust in what
intrigues you. For in the end, what will illuminate this newly acquired
knowledge --- and make cocktail-party conversation fun, rather than a
numbing act of recitation --- is the unique bend of your mind. Consider
how a critical framework can transform the most ordinary objects, from
the Q-tips and Bubble Wrap celebrated in the Italian curator
\href{https://www.nytimes3xbfgragh.onion/2014/07/13/business/surrounded-by-great-design-at-moma-and-not-afraid-to-use-it.html}{Paola
Antonelli}'s 2005 survey,
``\href{https://www.moma.org/calendar/exhibitions/124}{Humble
Masterpieces: Everyday Marvels of Design},'' to the trimming of
fingernails in a \href{https://www.youtube.com/watch?v=6I1gfOlNNo4}{2009
performance} of the American composer
\href{https://www.nytimes3xbfgragh.onion/topic/person/john-cage}{John
Cage}'s piece
``\href{https://www.johncage.org/pp/John-Cage-Work-Detail.cfm?work_ID=18}{0'00,''}''
whose score consists of a single instruction: ``In a situation provided
with maximum amplification (no feedback), perform a disciplined
action.'' Dialed all the way up through the speakers, each metallic snap
of the clippers is a shock.

Keep in mind, too, as you make the rounds of all those cocktail parties,
that often the people we think of as the best conversationalists are in
fact the best listeners. In this age when so much seems to revolve
around declarations of self and peacocking on social media, listening
can be a radical act, a pause in the compulsive narration of our own
lives to enter into the consciousness of another. The American artist
and AIDS activist
\href{https://www.nytimes3xbfgragh.onion/2018/09/07/magazine/the-rage-and-tenderness-of-david-wojnarowiczs-art.html}{David
Wojnarowicz}, who died in 1992, gave voice to those on the fringes in
his 1982 chapbook ``Sounds in the Distance,'' later expanded and
posthumously published as
``\href{https://groveatlantic.com/book/the-waterfront-journals/}{The
Waterfront Journals}'': an archive of transcribed monologues
commemorating --- and insisting on the value and importance of --- the
desires expressed by the drifters and hustlers, junkies and drag queens
he met. In the British writer
\href{https://www.nytimes3xbfgragh.onion/by/rachel-cusk}{Rachel Cusk}'s
recently completed
``\href{https://us.macmillan.com/series/outlinetrilogy/}{Outline}''
trilogy, the narrator is less a character than a conduit, passing on the
stories shared by the strangers around her and in doing so creating a
portrait of an anxious, uneasy world (and, slyly, almost incidentally,
of her deepest self).

Image

Credit...Courtesy of Picador

A final thought: Don't be afraid to say, ``I don't know.'' That opens a
door. In 2009, the Australian choreographer
\href{https://www.nytimes3xbfgragh.onion/2018/10/09/arts/lucy-guerin-tere-oconnor.html}{Lucy
Guerin} \href{https://vimeo.com/88420801}{paired} trained and untrained
dancers to perform a sequence of technically complex movements --- a
sort of irreverent, postmodern ``Dancing With the Stars,'' honoring the
unpredictable and teasing out the contours of each dancer's relationship
to the body (``Friend or foe?'' Guerin asks). Although laughter is
invoked, there is something to be learned from the rawness of the
movements. Guerin has said that she wishes she could ``untrain''
herself, shedding the memory of motions embedded by other
choreographers. Her novices have no such burden, and no time for
calculation --- they just leap out into space, without thinking ahead to
where they'll land. In the gap between knowing and unknowing is the
pleasure.

Advertisement

\protect\hyperlink{after-bottom}{Continue reading the main story}

\hypertarget{site-index}{%
\subsection{Site Index}\label{site-index}}

\hypertarget{site-information-navigation}{%
\subsection{Site Information
Navigation}\label{site-information-navigation}}

\begin{itemize}
\tightlist
\item
  \href{https://help.nytimes3xbfgragh.onion/hc/en-us/articles/115014792127-Copyright-notice}{©~2020~The
  New York Times Company}
\end{itemize}

\begin{itemize}
\tightlist
\item
  \href{https://www.nytco.com/}{NYTCo}
\item
  \href{https://help.nytimes3xbfgragh.onion/hc/en-us/articles/115015385887-Contact-Us}{Contact
  Us}
\item
  \href{https://www.nytco.com/careers/}{Work with us}
\item
  \href{https://nytmediakit.com/}{Advertise}
\item
  \href{http://www.tbrandstudio.com/}{T Brand Studio}
\item
  \href{https://www.nytimes3xbfgragh.onion/privacy/cookie-policy\#how-do-i-manage-trackers}{Your
  Ad Choices}
\item
  \href{https://www.nytimes3xbfgragh.onion/privacy}{Privacy}
\item
  \href{https://help.nytimes3xbfgragh.onion/hc/en-us/articles/115014893428-Terms-of-service}{Terms
  of Service}
\item
  \href{https://help.nytimes3xbfgragh.onion/hc/en-us/articles/115014893968-Terms-of-sale}{Terms
  of Sale}
\item
  \href{https://spiderbites.nytimes3xbfgragh.onion}{Site Map}
\item
  \href{https://help.nytimes3xbfgragh.onion/hc/en-us}{Help}
\item
  \href{https://www.nytimes3xbfgragh.onion/subscription?campaignId=37WXW}{Subscriptions}
\end{itemize}
