Sections

SEARCH

\protect\hyperlink{site-content}{Skip to
content}\protect\hyperlink{site-index}{Skip to site index}

\href{https://myaccount.nytimes3xbfgragh.onion/auth/login?response_type=cookie\&client_id=vi}{}

\href{https://www.nytimes3xbfgragh.onion/section/todayspaper}{Today's
Paper}

On the Rooftops of Paris, a New Kind of Urban Garden

\url{https://nyti.ms/2PjQIyE}

\begin{itemize}
\item
\item
\item
\item
\item
\end{itemize}

Advertisement

\protect\hyperlink{after-top}{Continue reading the main story}

Supported by

\protect\hyperlink{after-sponsor}{Continue reading the main story}

By Design

\hypertarget{on-the-rooftops-of-paris-a-new-kind-of-urban-garden}{%
\section{On the Rooftops of Paris, a New Kind of Urban
Garden}\label{on-the-rooftops-of-paris-a-new-kind-of-urban-garden}}

The landscape architect Arnaud Casaus is creating green spaces wilder
and warmer than those found at street level.

\includegraphics{https://static01.graylady3jvrrxbe.onion/images/2019/12/06/t-magazine/06tmag-casaus-slide-RB8H/06tmag-casaus-slide-RB8H-articleLarge.jpg?quality=75\&auto=webp\&disable=upscale}

By \href{https://www.nytimes3xbfgragh.onion/by/kurt-soller}{Kurt Soller}

\begin{itemize}
\item
  Published Dec. 6, 2019Updated Dec. 7, 2019
\item
  \begin{itemize}
  \item
  \item
  \item
  \item
  \item
  \end{itemize}
\end{itemize}

STARING OUT OVER the banister from the rooftop terrace of an
eighth-floor penthouse on the Marais's Rue Vieille du Temple, it's
immediately clear you're in Paris: Across the park below, past the
mansard roofs of the low Haussmannian buildings that have fronted these
streets since the late 19th century, the Eiffel Tower and the Place de
la Bastille's column pierce the gray clouds in the middle distance and,
looking west, the cyan blue and cherry-red tubes of the Centre Pompidou
dominate the skyline, their chromatic hues clashing against the beige
city.

These are pleasant, familiar views --- but if you turn back toward the
private terrace's adjoined apartment (which is owned by a television
producer), you'll spot the roof's true focal point: a dense thicket of
plants, grounded in weathered terra-cotta pots, layered with such
variety and quantity as to completely shroud the walls and corners of
this 754-square-foot deck. The verdant, V-shaped tableau, as absorbing
as it is disorienting in this metropolitan context, evokes the fantasy
of being a parched desert traveler stumbling across a fecund oasis. This
sense of sudden displacement is further echoed by the plants themselves,
nearly none of which technically belong in Paris: Among the dozens of
varieties, there's Agave x nigra, a hardy desert succulent; Phillyrea
angustifolia, a silvery-leafed bush native to the Mediterranean region;
and Aristaloe aristata, squat and spiky, which hails from South Africa.

Image

Casaus amid the gaura and Verbena bonariensis at the Rue Vieille du
Temple garden.Credit...Marion Berrin

Image

His own terrace, planted with Guernsey lily, society garlic,
Hardenbergia violacea, star jasmine, red yucca, ghost plant, garden
thyme, common sage and more.Credit...Marion Berrin

Such exotic species were sourced from nurseries throughout the city and
online --- from places such as
\href{http://www.panglobalplants.com/}{Pan-Global Plants}, a mail-order
business in Gloucestershire, England --- by the Parisian gardener
\href{http://www.arnaudcasaus.fr/}{Arnaud Casaus}, who in recent years
has challenged the conventions of French formal gardens, with their
symmetrical boxwood hedges, polite rows of pastel tulips and spherical
topiaries. The 45-year-old Casaus has reanimated several terraces,
balconies, patios and other small-scale plots throughout Paris with his
wild and naturalistic style, which is informed as much by his eye for
rare international plants as by his own sense of chaos and spontaneity.
``It's like cooking,'' he told me earlier this year, as we toured
several of his private residential projects. ``You have a recipe in your
hand, then you go to the market and find something that you never
thought about before. So maybe you still have the same recipe, but you
change it a little bit --- for me, gardening is like that.'' Much in the
way that contemporary chefs focus on mixing international influences,
supporting small-batch growers, heralding hyper-seasonality and
colliding several historical and regional references at once, Casaus is
among a group of landscape architects --- including
\href{https://www.danielnolandesign.com/}{Daniel Nolan} in San
Francisco, \href{http://www.gianmatteomalchiodi.com/}{Gianmatteo
Malchiodi} in Parma, Italy, and Rick Eckersley in Melbourne, Australia
--- who are redefining their craft largely by ignoring its traditions,
choosing instead to create bountiful juxtapositions in unexpected
settings. His work dovetails with a
\href{https://www.nytimes3xbfgragh.onion/2019/10/05/world/europe/paris-anne-hildago-green-city-climate-change.html}{larger
green movement underway in Paris}, where, since 2014, the city has been
installing dozens of tiny, idiosyncratic public gardens; in 2015, Mayor
Anne Hidalgo announced an initiative,
\href{https://www.paris.fr/pages/un-permis-pour-vegetaliser-paris-2689}{\emph{permis
de végétaliser}} (``license to vegetate''), that provides permits and
tools to help residents (or their landscapers) develop their own urban
plots, with a goal of adding 247 acres of vertical and roof gardens
throughout Paris by next year. For Casaus, this often involves stacking
visually distinct levels of, say, prickly cactuses and wispy flowering
bushes, or branchy ornamental trees and soft grasses, against a
balustrade or facade. He prefers to work in tight quarters not only
because those are what tend to be available in the city but also because
it allows him to distill and compound his contrast-driven vision.

\includegraphics{https://static01.graylady3jvrrxbe.onion/images/2019/12/06/t-magazine/06tmag-casaus-slide-0IM0/06tmag-casaus-slide-0IM0-articleLarge.jpg?quality=75\&auto=webp\&disable=upscale}

CASAUS'S PROFILE MAY have risen with the city's green wave, but his
peripatetic style remains indebted to his itinerant past: Born a few
hours southeast of Paris, in the Burgundy countryside of Dijon, his
earliest memories are of tilling his grandfather's enormous vegetable
garden and eating its tomatoes. It was then, at the age of 6, that he
knew he ``would always work within nature,'' he recalls, and when it
came time to go to college, he chose instead to enroll in landscape
school in the South of France. He never completed his studies, opting to
spend two and a half years in the late '90s in Lebanon, where he learned
to cultivate and propagate local plants. In 2000, he headed to Morocco
to establish his own nursery with a former classmate; a year later, he
returned to France, where he met his boyfriend, Jerome, a lawyer, and
decided to stay in Paris.

A few years later, he befriended
\href{https://www.nytimes3xbfgragh.onion/2015/09/23/t-magazine/minimalism-design-studio-ko.html}{Karl
Fournier and Olivier Marty} of the then-rising architecture firm
\href{http://www.studioko.fr/\#en-intro}{Studio KO}, who shared the same
cactus-and-desert-infused North African aesthetic that Casaus had spent
his 20s perfecting. As the duo won acclaim for their minimalist villas,
hotels and museums, Casaus served as something like their in-house
landscape architect: In 2015, for Fournier and Marty's own home in
Corsica, he planted an Egyptian palm tree in a field overgrown with
wildflowers and created a split-reed pergola studded with jasmine and
wisteria; in 2017, at a Berber-style lodge that Studio KO helped build
in Marrakesh, Casaus set a pair of Agave scabra incongruously into a
lawn of wild, waist-high grass, their pointed leaves poking through like
errant rabbits' ears; and last year, at the hillside
\href{https://www.nytimes3xbfgragh.onion/2018/09/14/t-magazine/los-angeles-dream-house-flamingo-estate.html}{Los
Angeles estate} of the creative director Richard Christiansen, he
installed a classical tiered garden of agave, plumeria and a dozen
different types of basil.

Image

The glass doors that lead from Casaus's apartment into his outdoor
space.Credit...Marion Berrin

When he began collaborating with Fournier and Marty, Casaus would often
seek out a regional nursery that could educate him on the local climate,
soil and vegetation. He still prefers to do that when he takes on a
project in a new locale. But the more he traveled, the more he realized
that his options were more flexible than he once assumed: What thrives
in Kyoto, say, or Hong Kong or Tangier or Oaxaca or even Southern
California, might flourish just as well on one of his Parisian roof
terraces, which tend to be exposed to high sun and yet partially shaded
and protected from the wind, and thus hospitable to equatorial,
Mediterranean and Eastern flora. That revelation changed his thinking
about landscape architecture, a discipline that's influenced by
increasing globalization --- but also by the unignorable realities of
climate science: ``The weather is changing,'' he says. ``I can see
that.'' Over the last two decades in Paris, the winter temperatures have
risen, allowing him to cultivate, say, Mexican feather grass sooner than
he could have in the past.

So long as he has this opportunity, Casaus believes it's both his
mission and right to reimagine what a rooftop garden might be.
(Greenery, after all, is one salve against global warming.) This is
perhaps most evident in one of his ongoing Parisian projects: his own
terrace, just 108 square feet, accessible through a glass door in the
kitchen of his fifth-floor apartment on Rue d'Aboukir in the Second
Arrondissement. Out here, the city beyond is all but invisible, blocked
by an unceasing canopy comprising a veritable United Nations of
horticulture: Akebia quinata --- or chocolate vine --- native to Japan,
China and Korea; red yucca, from the Chihuahuan Desert in West Texas;
Tulbaghia violacea (so-called society garlic) imported from South
Africa; and common sage, which grows throughout the Mediterranean,
nestled amid myriad other flowering plants that tangle together to form
the purest representation of his democratic ethos. On hot summer nights
--- of which there are increasingly many in Paris --- Casaus and his
boyfriend pull their futon from its living-room frame and drag it
outside to the wood decking near a cafe table, the balcony's only
permanent furniture. And then they drift asleep to the sound of the
streets below and the bees buzzing overhead, hidden in a forest all
their own.

Advertisement

\protect\hyperlink{after-bottom}{Continue reading the main story}

\hypertarget{site-index}{%
\subsection{Site Index}\label{site-index}}

\hypertarget{site-information-navigation}{%
\subsection{Site Information
Navigation}\label{site-information-navigation}}

\begin{itemize}
\tightlist
\item
  \href{https://help.nytimes3xbfgragh.onion/hc/en-us/articles/115014792127-Copyright-notice}{©~2020~The
  New York Times Company}
\end{itemize}

\begin{itemize}
\tightlist
\item
  \href{https://www.nytco.com/}{NYTCo}
\item
  \href{https://help.nytimes3xbfgragh.onion/hc/en-us/articles/115015385887-Contact-Us}{Contact
  Us}
\item
  \href{https://www.nytco.com/careers/}{Work with us}
\item
  \href{https://nytmediakit.com/}{Advertise}
\item
  \href{http://www.tbrandstudio.com/}{T Brand Studio}
\item
  \href{https://www.nytimes3xbfgragh.onion/privacy/cookie-policy\#how-do-i-manage-trackers}{Your
  Ad Choices}
\item
  \href{https://www.nytimes3xbfgragh.onion/privacy}{Privacy}
\item
  \href{https://help.nytimes3xbfgragh.onion/hc/en-us/articles/115014893428-Terms-of-service}{Terms
  of Service}
\item
  \href{https://help.nytimes3xbfgragh.onion/hc/en-us/articles/115014893968-Terms-of-sale}{Terms
  of Sale}
\item
  \href{https://spiderbites.nytimes3xbfgragh.onion}{Site Map}
\item
  \href{https://help.nytimes3xbfgragh.onion/hc/en-us}{Help}
\item
  \href{https://www.nytimes3xbfgragh.onion/subscription?campaignId=37WXW}{Subscriptions}
\end{itemize}
