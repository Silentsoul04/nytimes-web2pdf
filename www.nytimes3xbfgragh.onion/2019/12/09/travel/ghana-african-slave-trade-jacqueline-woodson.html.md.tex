Sections

SEARCH

\protect\hyperlink{site-content}{Skip to
content}\protect\hyperlink{site-index}{Skip to site index}

\href{/section/travel}{Travel}\textbar{}Jacqueline Woodson on Africa,
America and Slavery's Fierce Undertow

\url{https://nyti.ms/2RyuziJ}

\begin{itemize}
\item
\item
\item
\item
\item
\item
\end{itemize}

\includegraphics{https://static01.graylady3jvrrxbe.onion/images/2019/12/09/travel/09Ghana-7/09Ghana-7-articleLarge.jpg?quality=75\&auto=webp\&disable=upscale}

\hypertarget{jacqueline-woodson-on-africa-america-and-slaverys-fierce-undertow}{%
\section{Jacqueline Woodson on Africa, America and Slavery's Fierce
Undertow}\label{jacqueline-woodson-on-africa-america-and-slaverys-fierce-undertow}}

The African-American novelist journeys to Ghana, once a hub of the
trans-Atlantic slave trade, as the nation invites descendants of
enslaved Africans to call it their ``home.'' But can you go home again?

Children play on the shores of the Atlantic ocean against a backdrop of
Elmina Castle, where captured Africans were packed into dungeons before
their enslavement in the New World.Credit...Francis Kokoroko for The New
York Times

Supported by

\protect\hyperlink{after-sponsor}{Continue reading the main story}

By Jacqueline Woodson

\begin{itemize}
\item
  Dec. 9, 2019
\item
  \begin{itemize}
  \item
  \item
  \item
  \item
  \item
  \item
  \end{itemize}
\end{itemize}

\textbf{IN THE COASTAL TOWN} of Elmina, Ghana, the Atlantic Ocean
crashes against the rocks with such a ferocity, I make our kids move
back away from the gray-blue water. Four hundred years have passed since
captured Africans were forced across these waves on their way to bondage
in the New World and now, standing at the edge of this violent water,
startled by my own anxiety, I feel something deep and old and
terrifying. Call it hydrophobia. Call it genetic memory. I've always had
a fear of the ocean, the fierce pull of its undercurrent, the crest of
its powerful waves and most of all, its seeming infinity --- the way it
moves to a place where the skyline caresses it. Then drops off into
nothing at all.

As the water lashes near where I stand in this West African nation, from
whose ports millions of Africans passed through on their way to the
United States, Latin America and the Caribbean, it is impossible to not
viscerally \emph{feel} this memory everywhere in my body. Not far from
here, captured Africans walked onto slave ships.

I call again to my children. Tell them to be careful. What I want to do
right now is pull them close, hug them hard. I think of the people
chained and trembling and I know that by the luck of history and by the
grace of time, I am standing here, unshackled.

Now, Ghana has invited the descendants of the enslaved to a place it
wants us to call ``home.'' While Ghana is not the first sub-Saharan
country from which Africans were forced onto ships, many black bodies,
including my own ancestors, were sold and traded from here. In its
efforts to bring the African diaspora together, Ghana's leaders are also
hoping to make amends for the complicity of Africans in selling their
own people into what would become the trans-Atlantic slave trade.

\textbf{GHANA'S INVITATION} is wrapped up in a massive marketing
campaign called ``\href{https://www.yearofreturn.com/}{Year of
Return}.'' The commemoration is described as a ``landmark spiritual and
birth-right journey,'' urging black folks to seek out our roots, invest
in this country, and to educate our children on African soil. This year,
there have been festivals, concerts, workshops and other events to mark
400 years since the arrival of the first captured Africans in the
English colony of Virginia. But even as I stand at this shore, at once
terrified and moved, I still feel some kind of way about this
invitation. Curious. And cautious.

``African-American dollars should be reinvested in Africa'' reverberates
through the Year of Return narrative. It makes sense that so many of us
would much rather support black-owned businesses with our hard-earned
money. But more than this, as I examine my unwillingness to tolerate
America's insidious racism, its violence against black and brown bodies,
and the daily micro aggressions so many people of color have long
experienced, I find myself drawn to Ghana's offer.

\includegraphics{https://static01.graylady3jvrrxbe.onion/images/2019/12/09/travel/09Ghana-8/merlin_165289212_b95007b9-5316-4443-8081-7b6a283a8904-articleLarge.jpg?quality=75\&auto=webp\&disable=upscale}

But was Ghana drawn to
\href{https://www.jacquelinewoodson.com/all-about-me/my-biography/}{\emph{me}}\emph{?}
And to my queer family? My white partner, Juliet, and our biracial
children who, at 11 and 17, know half their DNA traces them to western
Africa.

Ghana's Year of Return website celebrates ``the cumulative resilience of
all the victims of the trans-Atlantic slave trade.'' It promised
everything from a welcoming World Music Festival to a Natural Hair Expo
to a First Bath Of Return and Naming Ceremony in which participants, as
is custom for African babies, are bathed, given African names and
presented to their extended African family.

While these events sounded interesting and somewhat moving, it was not
the way I wanted to see Ghana for the first time. I wanted the
circumstance rather than the pomp. I wanted truth.

\textbf{I HAD NEVER BEEN TO AFRICA.} But stepping out of that airport
the first morning, it felt as though I had always known Ghana. The deep
heat of the early morning so much like the South Carolina of my
childhood. The dark bodies that seemed to fill every space easily
absorbed my own dark body. And the smell --- of petrol and cooking oil,
of nuts roasting and plantains frying, of sweat and sewage --- quickly
swept me up out of jet-lag into the \emph{right now} of Ghana's capital,
Accra.

We had planned to begin our journey here with a visit to the slave
castles and forts on the coast. But like a curtain all of us were a bit
afraid to open, we knew the trip would reveal more of what we already
knew --- that Africans, including children, were sometimes kept for
months in dungeons, until enough were gathered to pack the hold of a
slave ship. That some of the captured Africans died in confinement while
others died during the Middle Passage, the longest leg of the triangular
journey between Europe, Africa and the Americas.

As we ate a breakfast of eggs and plantains in the cramped kitchen of an
apartment we'd rented in Accra's Tesano district, we tried to ready
ourselves. Thick metal bars blocked the light coming in from the lush
garden behind the house in the upper middle class area, and the shower,
often cold, sometimes didn't come on at all. Again and again, the small
daily inconveniences reminded us that we were no longer in America.

Behind our home, in what's known as the boys' quarters, a ``houseboy''
named Ali, who didn't look more than 15, squatted over a cookpot, making
rice over a coal fire. Ali lived alone in a darkened cement structure
with a single window. The ``housegirl,'' a woman who looked to be in her
60s, lived in an adjacent structure. Neither have electricity or running
water.

Ali was quiet and shy. When we left, we had to call him to unlock the
padlock to let us out of the gated building. In the night, when we
returned, we had to call him to let us back in. No matter the hour.
Bringing ``houseboys'' and ``girls,'' often distant relatives, to work
in the homes of better-off families is not unusual among the upper
classes. They are often paid very little and sometimes offered technical
training in lieu of an education. Our host's home was separated from
ours by a lovely screened-in porch adorned with cushioned wicker
furniture, decorative bird cages and palm fronds.

As I watched Ali crouched over his pot of rice, shyly smiling up at me
then quickly lowering his eyes, it was hard not to wonder about a
continuum here --- a people among whom had lived Africans willing to
help sell their own into slavery; a people among whom lived those
willing to employ ``a houseboy'' and a ``housegirl'' in her 60s, who
lived in darkened rooms, while a family lived comfortably a few feet
away.

And while I am completely sober in my knowledge that it was the evil of
white folks that kept my ancestors in bondage for centuries, I found
myself struggling to come to terms with those who worked with white
traders to move black bodies into chattel slavery.

``\href{https://www.nytimes3xbfgragh.onion/2010/04/23/opinion/23gates.html?searchResultPosition=7}{The
sad truth} is that without complex business partnerships between African
elites and European traders and commercial agents, the slave trade to
the New World would have been impossible, at least on the scale it
occurred,'' the Harvard scholar Henry Louis Gates Jr. wrote in 2010 in
The New York Times.

We decided we needed a few days before facing the brutality of
enslavement and headed instead to
\href{https://shai-hills-resource-reserve.business.site/\#gallery}{Shai
Hills Resource Reserve}, a wildlife and hiking retreat.

Image

A view from the Mogo Hills, home to baboons in the Shai Hills Resource
Reserve. The retreat provided a respite before the author visited the
slave fortress, Elmina Castle.Credit...Francis Kokoroko for The New York
Times

More than an hour north of Accra, the vast reserve has 31 types of
mammals, 13 different reptiles and 175 species of birds. But we'd come
for the baboons, which we'd been told were engaging and friendly. For
the next few hours, as my son sat in commune with baboon babies, adults
and elders, we took in the sites of the hilly retreat.

Its many caves still showed signs of the Shai people of the Ga-Adangbe
ethnic group, some of Ghana's earliest inhabitants. Thrones carved from
rock were nearly hidden high up inside the caves. As we hiked the steep
hills, jumped across gaps that plummeted into black valleys, reached the
top to gape at the undulating land below us, I had no deep feeling of
belonging to this place. Not here. Not yet.

\textbf{AS A BLACK CHILD} of the 70s, the Africa I learned about in
school, books and via television, felt irrelevant to me. Like many of my
friends, I would turn on the Saturday morning cartoons to find some
unfortunate character trussed inside caldrons of boiling water as
drooling cannibals --- supposedly African, supposedly savages ---
circled them. The Africa I saw, from National Geographic pictures of
bare-chested African women to TV ads featuring starving children, was as
unfamiliar as the Middle Passage itself.

I was born in Ohio and raised in South Carolina and Brooklyn. For a long
time, all that mattered about who I was began and ended in America ---
home to my grandparents, great-grandparents and greats beyond that. But
I am older now. And a mother who, thanks to DNA testing, understood how
deeply my roots connect to West Africa.

Yet, like the push and pull of the water here, I remained both drawn to
and dubious of Ghana's beckoning.

A few weeks before our trip, I called Professor Gates and asked what he
thought of Ghana's Year of Return. He said he believed it to be a
positive gesture. ``Any program increasing the knowledge of Americans in
general and African-Americans in particular about African history and
the history of the slave trade is a good thing.''

Days after that call, I logged onto the Ghanaian website for information
on visas and was startled by
\href{https://www.ghanaconsulatenewyork.org/gcn/FamilyClientStart.aspx}{red
letters} across the top of the page:

``Families are restricted to a nuclear family which is a family group
consisting of a father, mother and their children.''

Eye roll. We are not that. At the Ghana Consulate the following Monday,
I waited with my family's stack of visa applications, ready to challenge
this notion of ``family.'' But moments later, when the woman behind the
counter asked where our children's father was and I replied, ``They have
two mothers,'' without a pause, she responded, ``Two mothers, huh?
O.K.,'' and cleared our applications. And there it was --- the push, the
pull. Where only minutes before I was ready to fight, now I wanted to
hug her.

Image

Elmina Castle, overlooking the Atlantic Ocean, was built in 1482 by the
Portuguese and captured by the Dutch in 1637.Credit...Francis Kokoroko
for The New York Times

A week into our trip, we were finally ready to see ``the door of no
return'' at Ghana's oldest European structure, Elmina Castle. The drive
from Ghana's capital to the town of Elmina took about three hours.

The massive stone castle, built in 1482 by the Portuguese and captured
by the Dutch in 1637, is among dozens of colonial slave forts that
remain along the Ghanaian coast. While many forts have been turned into
prisons, government offices, guesthouses, or simply left abandoned,
Elmina Castle and another infamous fortress, Cape Coast Castle, see
thousands of tourists annually.

500 miles

NIGERIA

GHANA

Atlantic

Ocean

TOGO

IVORY

COAST

Shai Hills

resource

Reserve

30 miles

ghana

N1

Gulf of

Guinea

Elmina

Cape Coast Castle

Elmina

Castle

nima

1 mile

independence ave.

Nima

Market

ghana

ring rd. E.

dr. busia hwy.

Accra

oblogo rd.

Kempinski Hotel

Makola Market

Jamestown Cafe

Gulf of

Guinea

jamestown

Street data from OpenStreetMap

nima

500 miles

1 mile

independence ave.

Nima

Market

ghana

GHANA

Atlantic

Ocean

ring rd. E.

dr. busia hwy.

Accra

IVORY

COAST

TOGO

oblogo rd.

Kempinski Hotel

Shai Hills

resource

Reserve

Makola Market

30 miles

ghana

Jamestown Cafe

N1

jamestown

Gulf of

Guinea

Gulf of

Guinea

Elmina

Cape Coast Castle

Elmina Castle

Street data from OpenStreetMap

By The New York Times

As we stood in an airless dungeon with its stone walls, dirt floor,
barred windows and doorways, we listened somberly while our guide broke
down our ancestors' harrowing journey --- the hundreds of captured
people packed into these small, still fetid spaces. The shackled women
brought out into the courtyards so the Dutch governor from his balcony
could choose the woman he wanted to rape that day. The chosen woman
getting washed and brought to him. We heard the stories of resisters
being shackled in the courtyard for days beneath a brutal sun, sent to
solitary confinement or killed. Above them in the opulent quarters of
the castle, which housed a sanctuary for white missionaries and a
church, Dutch officers enjoyed lavish meals and ocean views.

\textbf{WE TRIED NOT TO IMAGINE} what it was like. But, painful as it
was, we imagined it anyway.

Image

The courtyard of Elmina Castle, where shackled Africans were beaten and
chosen for rape.Credit...Francis Kokoroko for The New York Times

Image

Visitors touring Elmina Castle. Captured Africans were sometimes
confined for months in its dungeons.Credit...Francis Kokoroko for The
New York Times

Then, from the darkness of the haunting space, our guide stepped out
into the bright cool air. With us still inside, he gently closed the
gate. ``This is the last place enslaved Africans ever saw in Ghana,'' he
said. With the click of the latch, small sobs escaped from the strangers
around me. ``This is the point of no return,'' he said.

But we had returned. Different. Still alive.

Now, as the kids take selfies along the shore by the castle, the current
has me again calling them away from the water. \emph{But the light,
Mommy,} my daughter calls back, reminding me, that at 17, in the age of
Instagram, even this far from our home in Brooklyn, it is still a
moment's light that matters most.

Hours have passed since our tour of the castle and as my partner and I
nervously watch the children near the roiling waves, we stand under the
canopy of a restaurant at the luxury beach resort,
\href{https://coconutgrovehotelsghana.com/}{Coconut Grove}. Behind us,
the outdoor dining room is nearly at capacity with huge groups of
African-Americans, mostly women, who arrived earlier in charter buses.
Ranging from their early 20s to well into their 80s, with elaborately
styled hair, beautifully manicured nails and T-shirts representing their
sororities or affiliations with historically black colleges, all of them
are somewhere on the lovely spectrum.

Others in these groups lounge around the pool in brightly printed
batakari shirts and flowing pata pata dresses. Just one week ago, the
words to describe these African garments were as foreign as the land I
stood on. But that easily, that quickly, all of it has become a part of
my own language. As though I've always spoken it. I am too shy to
approach these beautiful people --- who have traveled here in groups and
seem to know each other as they move from table to table chatting,
giving hugs, drinking and laughing.

Almost two weeks into my time in Ghana, my heart remains in my throat.
There is nowhere in this country where the eye can land and the body not
feel, at once, both a deep pain and an immense joy.

Image

In Elmina, Ghana, schoolchildren play on colonial lion statues on their
walk home after school.Credit...Francis Kokoroko for The New York Times

As I watch my children, I think of our days in Accra, where smiling
children ran past straggling goats toward slow-moving cars on bustling
streets, their closed hands tapping their mouths then opening out toward
passengers. And where, on a nearly deserted corner in the city, a lone
woman fried marinated sweet plantains over an open fire then sprinkled
them with fresh peanuts and with a gaptoothed smile, handed the kelewele
over to me. ``You are my sister, I see it in your eyes,'' she said,
dropping the few Ghanaian cedis I've paid into her apron pocket. As I
lifted the salty-sweet snack to my mouth, I wondered if it was only
black folks who felt as I do --- as though everything around me is at
once, heartbreakingly familiar and foreign.

Now, far from Accra, as my children climb onto rocks separating the
restaurant at Coconut Grove from the ocean, their arms stretched out for
balance, their faces a deep brown from the many days of sun, I think
about what I've asked myself since landing on African soil. Can I belong
here? Has this country truly called me home?

\textbf{AS TRUE AS AN UNDERTOW}, I feel the water pulling me back across
it, taking me to where it took my ancestors centuries ago. To a land as
foreign to them as the African chiefs who offered brown bodies over for
\href{https://www.theguardian.com/commentisfree/2007/mar/31/epiloguetothedebateonslav}{weapons},
\href{http://www.discoveringbristol.org.uk/slavery/routes/bristol-to-africa/trade-goods/slave-trade-goods/}{brass
and cotton}. As foreign as the white captors who raped, brutalized and
enslaved those same bodies. I feel the pull of the history that brought
me here. And the history that took me away. A feeling as old as my body
itself overcomes me --- that I have never felt whole in one place. In
Africa, like America, I am only halfway home. My body belonging to both
and neither place.

Returning to Accra two days later, I dragged my reluctant children, now
wearied by the many teachable moments of this journey, to Nima, one of
the city's most impoverished districts. Its lively outdoor Nima Market
has everything from used refrigerators and fresh fish floating in
buckets to antique Kente cloth. Nima is also home to a long-running
program for young people, Spread Out Initiative.

Located in a cramped structure only steps from a four-foot drop into a
trash-and-sewage-filled gutter, young Muslim children, mostly hijabed
girls, gathered on this Saturday to recite their spoken word, read from
their developing novels and engage in conversations around literature.
As my family sat in circled community with the young people, I was
struck by the vitality and joy on their faces, the assuredness and pride
with which they shared their words. All of it was a far cry from the
famished African children on the TV screens of my childhood. All of it,
in my mind, was the best kind of celebration.

Later that day, we went to the Art Center, an outdoor market in Accra.
In tented stalls, African craftspeople aggressively hawked their wares,
everything from African drums and T-shirts to lizardhead shoes and
purses. I sent my kids off on their own to wander the market and watched
as a dozen merchants trailed them, their bounty held out, their lilted
``\emph{I'll give you a nice price''} echoing. As the merchants came at
them from all sides, my partner and I laughed as our bewildered kids
picked up their pace, unsure how to deflect the kindness in the
merchants' voices, their sweet smiles, their tenacity.

Image

A street scene in central Accra. Mural by Ghana graffiti And Accra
metropolitan assemblyCredit...Francis Kokoroko for The New York Times

And then, for the first time, it hit me. As the circle of brown men
followed my children, I realized how physically safe I felt here in
Accra. Like the Deep South of my childhood, where the many brown bodies
of family protected me from an evil I knew was out there, here too,
African men, women and children around me felt old and safe, comforting
and good. And as with family, Ghana felt like complicated love.

That evening, we checked into Accra's five-star
\href{https://www.kempinski.com/en/accra/hotel-gold-coast-city/?utm_source=google\&utm_medium=cpc\&source=S46992213\&\&utm_campaign=KIACC1\%20-\%20Kempinksi+Gold+Coast+City+Accra+-+Brand\&utm_content=kempinski+hotel+ghana\&gclid=Cj0KCQjw6KrtBRDLARIsAKzvQIFwU_T4N4YMRqr--nPyS3qjB6IsipfvgDQo1QskMKvWl8cLF_aX4OIaAk3mEALw_wcB\&gclsrc=aw.ds}{Kempinski
Hotel} because it's fancy, has a huge pool surrounded by grassy fields
and because, for our last few days here, we wanted the water to actually
come out of the spigot when we turned it on. In the lobby, I spotted an
elegantly dressed African-American woman. When she smiled, my shyness
dropped away and I asked if I could sit with her for a while.

Ayesha Hakeem owns African Connections, which operates tours and
conferences and helps people relocating to Africa. She first came to
West Africa in the 1970s, when she was a student at Northwestern
University and returned for good after practicing law for 20 years.

``What was so phenomenal was learning how much African-American culture
reflects and is a part of the African culture,'' she said.

As we talked, we realized we're both products of the Great Migration ---
her family settled in Chicago, mine in New York City. And we spent
childhoods listening to our mothers and grandmothers make offhand
remarks like ``don't let anyone sweep your feet or you'll never marry.''

Ayesha recounts hearing these same words from African women when she
arrived in Ghana.

\textbf{THE NEXT MORNING,} as my family climbs out of our car in
Jamestown, one of Accra's oldest districts, we are met by the stench of
sewage, rotting fish, the briny smell of ocean and a seemingly endless
number of brightly painted fishing boats.

Scrawny dogs scatter as we approach, as quickly as children run toward
us, taking our own children's hands and pulling them to join in their
exploration of the dilapidated boats and sealife at the water's edge.

Later, as we eat our lunch --- bowls of ``red red,'' a Ghanaian dish of
palm oil and black-eyed peas --- at the
\href{https://www.lonelyplanet.com/ghana/accra/nightlife/jamestown-cafe/a/poi-dri/1556679/355309}{JamesTown
Café}, where artists, intellectuals and musicians gather in the evening,
I taste the history of my own family's black-eyed peas over rice in each
spicy bite.

Image

Market day at Makola Market in Accra, one of Ghana's busiest shopping
district.Credit...Francis Kokoroko for The New York Times

After a long hang at the cafe, we head to the
\href{https://visitghana.com/shopping_location/makola-market/}{Makola
Market}in the center of Accra. There, young girls called Kayayei from
the Hausa ethnic group in the northern parts of West Africa wait to be
hired. ``Kaya'' means, among other things, ``load'' in the Hausa
language, and ``Yei'' in the Ga language means ``women.''

These girls move through the market carrying people's large shopping
bundles in metal bowls propped on their heads with the hope of raising
money for their own dowries. My longtime friend, Catherine McKinley, an
African-American writer and curator who was in Ghana working on an art
installation, waved over a small girl. ``Let's be Auntie to this
child,'' she said, laughing. ``Get her dowry paid off in one job.''

We are two of the few foreigners at this market, but Catherine, having
spent many years in Ghana, navigates it easily. We hire the puzzled
child to follow behind us carrying nothing. Her back and neck straight,
she eyes us suspiciously. I see parts of my own young self in her
questioning, the 10-year-old side-eye waiting for the truth.

Minutes later, when we pay her what we hope is the equivalent of two
dowries, we see, for the first time, the way her face expands into a
disbelieving smile, her dark eyes now shining. As crowds move around us,
bartering for nuts, fabrics, plastic rugs, toothbrushes, wigs and even
here, for the toil of bodies, I watch the girl disappear into the chaos,
her silver bowl becoming one of many silver bowls. Her brown body, one
of many brown bodies.

Image

A view of Elmina Castle at night.Credit...Francis Kokoroko for The New
York Times

The streets grow quieter as the day ends, and we walk them one more time
before we leave the next morning. On our final night in Ghana, we are
walking to remember it. An old woman in a faded pata pata dress,
crouched and sweeping with a handle-less broom, uprights herself as we
pass. ``Akwaaba!'' she said. ``Welcome.'' ``Medaase,'' I respond.
``Thank you.''

``Could you guys live here,'' I asked our children, as they walked ahead
of us, as comfortable as if they were walking down our neighborhood
streets in Brooklyn.

``Yeah, I guess,'' my son said. ``If we had to.''

``I couldn't,'' I tell them. ``Not for always. Some of the time maybe.
We can go and come back. Keep going and coming back.''

``I guess that's what makes us African,'' my son said. ``And American.''

\begin{center}\rule{0.5\linewidth}{\linethickness}\end{center}

Jacqueline Woodson is the author of the National Book Award winner
``Brown Girl Dreaming.'' She serves as the Library of Congress's
national ambassador for young people's literature. Her latest novel is
``Red at the Bone.''

\begin{center}\rule{0.5\linewidth}{\linethickness}\end{center}

\emph{\textbf{Follow}}
\textbf{\href{https://twitter.com/nytimestravel}{\emph{NY Times Travel
on Twitter}}\emph{,}}
\textbf{\href{https://www.instagram.com/nytimestravel/}{\emph{Instagram}}}
\emph{\textbf{and}}
\textbf{\href{https://www.facebookcorewwwi.onion/nytimestravel/}{\emph{Facebook}}\emph{.}}
\href{https://www.nytimes3xbfgragh.onion/newsletters/traveldispatch}{\emph{Get
weekly updates from our Travel Dispatch newsletter, with tips on
traveling smarter, destination coverage and photos from all over the
world.}}

Advertisement

\protect\hyperlink{after-bottom}{Continue reading the main story}

\hypertarget{site-index}{%
\subsection{Site Index}\label{site-index}}

\hypertarget{site-information-navigation}{%
\subsection{Site Information
Navigation}\label{site-information-navigation}}

\begin{itemize}
\tightlist
\item
  \href{https://help.nytimes3xbfgragh.onion/hc/en-us/articles/115014792127-Copyright-notice}{©~2020~The
  New York Times Company}
\end{itemize}

\begin{itemize}
\tightlist
\item
  \href{https://www.nytco.com/}{NYTCo}
\item
  \href{https://help.nytimes3xbfgragh.onion/hc/en-us/articles/115015385887-Contact-Us}{Contact
  Us}
\item
  \href{https://www.nytco.com/careers/}{Work with us}
\item
  \href{https://nytmediakit.com/}{Advertise}
\item
  \href{http://www.tbrandstudio.com/}{T Brand Studio}
\item
  \href{https://www.nytimes3xbfgragh.onion/privacy/cookie-policy\#how-do-i-manage-trackers}{Your
  Ad Choices}
\item
  \href{https://www.nytimes3xbfgragh.onion/privacy}{Privacy}
\item
  \href{https://help.nytimes3xbfgragh.onion/hc/en-us/articles/115014893428-Terms-of-service}{Terms
  of Service}
\item
  \href{https://help.nytimes3xbfgragh.onion/hc/en-us/articles/115014893968-Terms-of-sale}{Terms
  of Sale}
\item
  \href{https://spiderbites.nytimes3xbfgragh.onion}{Site Map}
\item
  \href{https://help.nytimes3xbfgragh.onion/hc/en-us}{Help}
\item
  \href{https://www.nytimes3xbfgragh.onion/subscription?campaignId=37WXW}{Subscriptions}
\end{itemize}
