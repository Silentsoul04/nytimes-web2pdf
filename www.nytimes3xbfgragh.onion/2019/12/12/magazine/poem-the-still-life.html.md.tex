Sections

SEARCH

\protect\hyperlink{site-content}{Skip to
content}\protect\hyperlink{site-index}{Skip to site index}

\href{https://myaccount.nytimes3xbfgragh.onion/auth/login?response_type=cookie\&client_id=vi}{}

\href{https://www.nytimes3xbfgragh.onion/section/todayspaper}{Today's
Paper}

Poem: The Still Life

\url{https://nyti.ms/2PABzsV}

\begin{itemize}
\item
\item
\item
\item
\item
\end{itemize}

Advertisement

\protect\hyperlink{after-top}{Continue reading the main story}

Supported by

\protect\hyperlink{after-sponsor}{Continue reading the main story}

\hypertarget{poem-the-still-life}{%
\section{Poem: The Still Life}\label{poem-the-still-life}}

By Mark Sanders

Selected by Naomi Shihab Nye

\begin{itemize}
\item
  Published Dec. 12, 2019Updated Jan. 2, 2020
\item
  \begin{itemize}
  \item
  \item
  \item
  \item
  \item
  \end{itemize}
\end{itemize}

To engage the concept of a ``still life'' painting, this tender poem by
Mark Sanders pauses and holds. Memory permeates awareness again and
again, illuminating the absent, suggesting not that we keep it alive but
that it may keep us alive. Sometimes the plums or ceramic pitchers in a
painting feel more vivid than the ones on our tables right in front of
us. It may take, as the poet William Stafford used to say, a certain
tilt of the head to perceive it. Here in this poem, no one is gone.
\emph{Selected by Naomi Shihab Nye}

Image

\hypertarget{the-still-life}{%
\subsection{The Still Life}\label{the-still-life}}

\emph{By Mark Sanders}

Now --- just at that silent place,\\
between sadness and gratitude,\\
wind-worn balances we all weather ---\\
a cardinal leaps from a bare trim limb,\\
its red bloom lingering. The sun down\\
in deepening darkness\\
where night clouds consume it,\\
evanescence of orange and purple.

How moment passes, how memory\\
holds. The heart must break\\
if it has ever felt joy. The heart must\\
break because diminished things matter,\\
and having mattered hold, still.\\
You were here. For us. Then break, heart.\\
Your fingers lie upon the pulse of our days.

\begin{center}\rule{0.5\linewidth}{\linethickness}\end{center}

\textbf{Naomi Shihab Nye} is the 2019-21 Young People's Poet Laureate of
the Poetry Foundation, Chicago. \textbf{Mark Sanders} is the associate
dean of the College of Liberal and Applied Arts at Stephen F. Austin
State University in Nacogdoches, Tex. His most recent book is ``In a
Good Time,'' published by WSC Press.

Illustration by R.O. Blechman

Advertisement

\protect\hyperlink{after-bottom}{Continue reading the main story}

\hypertarget{site-index}{%
\subsection{Site Index}\label{site-index}}

\hypertarget{site-information-navigation}{%
\subsection{Site Information
Navigation}\label{site-information-navigation}}

\begin{itemize}
\tightlist
\item
  \href{https://help.nytimes3xbfgragh.onion/hc/en-us/articles/115014792127-Copyright-notice}{©~2020~The
  New York Times Company}
\end{itemize}

\begin{itemize}
\tightlist
\item
  \href{https://www.nytco.com/}{NYTCo}
\item
  \href{https://help.nytimes3xbfgragh.onion/hc/en-us/articles/115015385887-Contact-Us}{Contact
  Us}
\item
  \href{https://www.nytco.com/careers/}{Work with us}
\item
  \href{https://nytmediakit.com/}{Advertise}
\item
  \href{http://www.tbrandstudio.com/}{T Brand Studio}
\item
  \href{https://www.nytimes3xbfgragh.onion/privacy/cookie-policy\#how-do-i-manage-trackers}{Your
  Ad Choices}
\item
  \href{https://www.nytimes3xbfgragh.onion/privacy}{Privacy}
\item
  \href{https://help.nytimes3xbfgragh.onion/hc/en-us/articles/115014893428-Terms-of-service}{Terms
  of Service}
\item
  \href{https://help.nytimes3xbfgragh.onion/hc/en-us/articles/115014893968-Terms-of-sale}{Terms
  of Sale}
\item
  \href{https://spiderbites.nytimes3xbfgragh.onion}{Site Map}
\item
  \href{https://help.nytimes3xbfgragh.onion/hc/en-us}{Help}
\item
  \href{https://www.nytimes3xbfgragh.onion/subscription?campaignId=37WXW}{Subscriptions}
\end{itemize}
