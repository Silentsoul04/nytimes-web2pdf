Sections

SEARCH

\protect\hyperlink{site-content}{Skip to
content}\protect\hyperlink{site-index}{Skip to site index}

\href{/section/us}{U.S.}\textbar{}`A Cesspool of a Dungeon': The Surging
Population in Rural Jails

\url{https://nyti.ms/35dxJfP}

\begin{itemize}
\item
\item
\item
\item
\item
\item
\end{itemize}

\includegraphics{https://static01.graylady3jvrrxbe.onion/images/2019/12/13/us/13JAILS-courtroom/merlin_165820452_0473d282-b5ac-4d9a-8f76-e972a2a87911-articleLarge.jpg?quality=75\&auto=webp\&disable=upscale}

\hypertarget{a-cesspool-of-a-dungeon-the-surging-population-in-rural-jails}{%
\section{`A Cesspool of a Dungeon': The Surging Population in Rural
Jails}\label{a-cesspool-of-a-dungeon-the-surging-population-in-rural-jails}}

As urban areas embrace alternatives to incarceration, the addiction
crisis in rural American communities and a different approach to
criminal justice have overwhelmed small-town jails.

The courtroom in Hamblen County sees a steady stream of people accused
of crimes driven by drug addiction. Many are repeat offenders.Credit...

Supported by

\protect\hyperlink{after-sponsor}{Continue reading the main story}

By
\href{https://www.nytimes3xbfgragh.onion/by/richard-a-oppel-jr}{Richard
A. Oppel Jr.}

Photographs by Kristine Potter

\begin{itemize}
\item
  Dec. 13, 2019
\item
  \begin{itemize}
  \item
  \item
  \item
  \item
  \item
  \item
  \end{itemize}
\end{itemize}

MORRISTOWN, Tenn. --- The Hamblen County Jail has been described as a
dangerously overcrowded ``cesspool of a dungeon,'' with inmates sleeping
on mats in the hallways, lawyers forced to meet their clients in a
supply closet and the people inside subjected to ``horrible conditions''
every day.

And that's the county sheriff talking.

Jail populations used to be concentrated in big cities. But since 2013,
the number of people locked up in rural, conservative counties such as
Hamblen has skyrocketed, driven by the nation's drug crisis.

Like a lot of Appalachia, Morristown, Tenn., about an hour east of
Knoxville, has been devastated by methamphetamine and opioid use.
Residents who commit crimes to support their addiction pack the 255-bed
jail, which had 439 inmates at the end of October, according to
\href{https://www.tn.gov/content/dam/tn/correction/documents/JailOctober2019.pdf}{the
latest state data}.

Many cities have invested in treatment options and diversion programs to
help drug users. But those alternatives aren't available in a lot of
small towns.

``In the big city, you get a ticket and a trip to the clinic,'' said
Jacob Kang-Brown, a senior research associate at the Vera Institute of
Justice, which released a report on Friday analyzing jail populations.
``But in a smaller area, you might get three months in jail.''

\includegraphics{https://static01.graylady3jvrrxbe.onion/images/2019/12/13/us/13JAILS-jarnigan/merlin_165819609_ad697c96-91a7-4904-9e79-5debf7c192d8-articleLarge.jpg?quality=75\&auto=webp\&disable=upscale}

The disparity has meant that while jail populations have dropped 18
percent in urban areas since 2013, they have climbed 27 percent in rural
areas during that same period, according to
\href{https://www.vera.org/publications/people-in-jail-in-2019}{estimates
in the report from Vera,} a nonprofit group that works to improve
justice systems. The estimates are drawn from a sample of data from
about 850 counties across the country.

There are now about 167,000 inmates in urban jails and 184,000 in rural
ones, Vera said. Suburban jail populations have remained about the same
since 2013, while small and midsize cities saw a 7 percent increase.

Rural jails now lock up people at a rate more than double that of urban
areas. And increasingly, those inmates are women. Hamblen County
officials said the number of female inmates in their jail has doubled in
the past decade.

Image

The Hamblen County Jail held 439 inmates at the end of October,
according to the latest state data. With only 225 beds, that means many
inmates sleep on mats on the floor.

Image

The men's jail in Hamblen County is in a basement under the courthouse
and the sheriff's office.

Drug use isn't the only reason that some rural jails are packed. State
prisons sometimes pay counties with extra bed space to house inmates,
and so does the federal government. The number of Immigration and
Customs Enforcement detainees held in jails rose by about 4,300 from
2013 to 2017, Vera estimates.

Small towns also lag cities in efforts to reduce incarceration, such as
releasing nonviolent offenders without requiring them to
\href{https://www.nytimes3xbfgragh.onion/2018/03/31/us/bail-bonds-extortion.html}{pay
hefty bail amounts} while awaiting their day in court.

The rural jail boom runs counter to a nationwide,
\href{https://www.nytimes3xbfgragh.onion/2018/12/18/us/politics/senate-criminal-justice-bill.html}{largely
bipartisan push} toward reducing incarceration, which has been embraced
by everyone from the American Civil Liberties Union to
\href{https://www.nytimes3xbfgragh.onion/2018/12/14/us/politics/jared-kushner-criminal-justice-bill.html}{President
Trump's son-in-law, Jared Kushner.} Sentencing law revisions have led
state and federal prison populations to drop since 2009, following
\href{https://www.sentencingproject.org/criminal-justice-facts/}{a
four-decade boom}.

Many cities have seen the number of people in jails, which hold people
convicted of minor crimes or awaiting trial, plummet in similar fashion.
In Nashville, 200 miles west of Morristown, the inmate population has
fallen 28 percent since 2013, according to Vera. Other cities with big
declines include Buffalo, Chicago, New York City, Oakland and
Philadelphia.

But in places like Hamblen County, with a population of 65,000, the
system works differently. People get arrested on charges like
possession, shoplifting to pay for their addiction or failing a drug
test while on probation, and bail is set too high for many to afford.

Almost everyone in the county jail is there because of charges related
to addiction, said the sheriff, Esco Jarnagin.

Image

Emma Partin, a corrections officer in Hamblen County, pointed out the
overcrowded cell pods she oversees from the security room.

Image

Inside, many lose jobs and are further cut off from family and friends.
The odds of getting back on track on the outside dwindle, and the cycle
repeats.

Few know more about this cycle than Kim Coffey, who has worked in many
aspects of Hamblen County's criminal justice system --- for a defense
lawyer, as a bail bondswoman and as a juvenile drug addiction counselor.

Now she is one of the sheriff's jailers. At work, she often sees her
daughter, who has spent much of the last decade in and out of jail after
getting hooked on pain pills following an injury.

``They were giving her hydros like they were Tic Tacs,'' Ms. Coffey
said, referring to hydrocodone, a powerful opioid that her 29-year-old
daughter took for more than a year. ``Then they cut her off cold turkey.
Her back was still in pain, and she did what she felt she had to do. You
go to the next available thing. Now, it's meth.''

Given the lack of options, Ms. Coffey said jail was sometimes the best
place for her daughter. ``At least when she's here, I know she's alive.
I know she's not in a ditch somewhere.''

Yet she wishes the county had more treatment options and job-training
programs that could help inmates like her daughter. ``If we could help
people to finds jobs, then they wouldn't go back to drugs, because
otherwise you go back to what you know to make a living.''

Image

Kim Coffey is a corrections officer at the jail, where she often sees
her daughter, who has spent much of the last decade behind bars.

As she spoke, another guard shouted, ``Hey, we got a fight!'' Ms. Coffey
rushed off to help break up a brawl between female inmates, who now
account for a third of the jail's population.

Fights in the jail are a common occurrence, said a former sheriff's
deputy who is now serving time himself; he said he stole a commercial
lawn mower after getting hooked on pain pills following shoulder
surgery.

``Tensions run high when you got 60 people in a 20-man pod,'' said the
former deputy, who asked that his name not be used because he feared he
could face retaliation.

One recent six-month stretch had more than 150 inmate-on-inmate
assaults, according to a judge's findings in an ongoing federal lawsuit,
which also said the jail suffered from ``overcrowding, insufficient
security checks, inadequate staffing, difficulty with properly
classifying inmates, failure to provide information about reporting
sexual assault to inmates, and many incidents of inmate-on-inmate
assault.''

Hamblen County officials have proposed a new justice center that would
include a jail twice the size of the current one, as well as new
courtrooms. But the \$73 million price tag has drawn protests from some
taxpayers.

Over the past six years, the county's annual jail expenses have risen to
\$4.4 million from \$2.6 million, said Bill Brittain, the county's top
administrator.

Image

Overcrowding at the Hamblen County Jail means sick inmates are kept on a
floor in the hallway.

Image

With so little space, tempers flare: One recent six-month stretch saw
more than 150 inmate-on-inmate assaults.

Image

Hamblen County has proposed replacing its current criminal justice
center with a new one that would double the size of the jail at a cost
of \$73 million.

Defense lawyers have proposed other options to address the crisis,
including a pilot program for pretrial supervision similar to one in
nearby Knoxville and other cities. It would have allowed some low-risk
defendants to avoid having to post bail, and to avoid jail even if
convicted.

But judges rejected the proposal because of fears that defendants would
flee, said Willie Santana, a former prosecutor in Knoxville who is now
one of four lawyers in the Hamblen County public defender's office.
``The whole system is geared toward generating pleas and putting people
in jail,'' he said.

For many inmates, that means the jail has been a revolving door. More
than three-quarters of the 850 new cases that Mr. Santana handled in the
past year involved a client who had previously been incarcerated for
something drug-related, he said.

Many small cities and rural areas haven't embraced efforts to make it
easier for nonviolent offenders to get on with their lives after scrapes
with the law. And even in rural areas that might favor more treatment
over incarceration,
\href{https://www.nytimes3xbfgragh.onion/2018/07/17/us/hospital-closing-missouri-pregnant.html}{hospitals
have shut down}, limiting their choices.

``You don't have any treatment options, or at least it seems to them
that they don't, so many judges and prosecutors feel that they have no
choice but to lock people up,'' said Pamela Metzger, director of the
Deason Criminal Justice Reform Center at SMU Dedman School of Law.

Image

Willie Santana, an assistant public defender in Hamblen County, advising
his clients before their court hearings.

Image

Willie Santana, an assistant public defender, meets inmates in the
supply closet at the Hamblen County Jail.

Image

Despite the overcrowding in Hamblen County, the sheriff and some other
officials are skeptical that big-city solutions could work here. Sheriff
Jarnagin said he favors education and prevention over treatment.

``We can't cure them once they get on some of these drugs,'' he said.
``It's jail, or the graveyard.''

Pretrial diversion, he added, would reduce jail numbers, but would also
mean criminals running loose. ``They're going to commit a crime and be
right back in here on something else.''

One ray of hope has been a jail-to-work program for female inmates,
administered by a local treatment facility. It takes just eight women at
a time, but of its 45 graduates over the past two years, only seven have
committed new crimes, Mr. Brittain said.

Buoyed by the program's success, he wants to start a similar one for
men. Treatment would be a better option for most inmates at the jail, he
said.

``East Tennessee is a very conservative area, and folks believe that
people who commit the crime need to do the time, but that's costing the
local government tremendous money to do that,'' Mr. Brittain said. ``We
can't build our way out of jail overcrowding. We've got to change some
of the ways we detain and punish people. We've got to do something
different.''

Advertisement

\protect\hyperlink{after-bottom}{Continue reading the main story}

\hypertarget{site-index}{%
\subsection{Site Index}\label{site-index}}

\hypertarget{site-information-navigation}{%
\subsection{Site Information
Navigation}\label{site-information-navigation}}

\begin{itemize}
\tightlist
\item
  \href{https://help.nytimes3xbfgragh.onion/hc/en-us/articles/115014792127-Copyright-notice}{©~2020~The
  New York Times Company}
\end{itemize}

\begin{itemize}
\tightlist
\item
  \href{https://www.nytco.com/}{NYTCo}
\item
  \href{https://help.nytimes3xbfgragh.onion/hc/en-us/articles/115015385887-Contact-Us}{Contact
  Us}
\item
  \href{https://www.nytco.com/careers/}{Work with us}
\item
  \href{https://nytmediakit.com/}{Advertise}
\item
  \href{http://www.tbrandstudio.com/}{T Brand Studio}
\item
  \href{https://www.nytimes3xbfgragh.onion/privacy/cookie-policy\#how-do-i-manage-trackers}{Your
  Ad Choices}
\item
  \href{https://www.nytimes3xbfgragh.onion/privacy}{Privacy}
\item
  \href{https://help.nytimes3xbfgragh.onion/hc/en-us/articles/115014893428-Terms-of-service}{Terms
  of Service}
\item
  \href{https://help.nytimes3xbfgragh.onion/hc/en-us/articles/115014893968-Terms-of-sale}{Terms
  of Sale}
\item
  \href{https://spiderbites.nytimes3xbfgragh.onion}{Site Map}
\item
  \href{https://help.nytimes3xbfgragh.onion/hc/en-us}{Help}
\item
  \href{https://www.nytimes3xbfgragh.onion/subscription?campaignId=37WXW}{Subscriptions}
\end{itemize}
