Sections

SEARCH

\protect\hyperlink{site-content}{Skip to
content}\protect\hyperlink{site-index}{Skip to site index}

\href{https://www.nytimes3xbfgragh.onion/section/politics}{Politics}

\href{https://myaccount.nytimes3xbfgragh.onion/auth/login?response_type=cookie\&client_id=vi}{}

\href{https://www.nytimes3xbfgragh.onion/section/todayspaper}{Today's
Paper}

\href{/section/politics}{Politics}\textbar{}Trump's Inaccurate
Statements About the Conflict With Iran

\url{https://nyti.ms/2RdrnaH}

\begin{itemize}
\item
\item
\item
\item
\item
\end{itemize}

Advertisement

\protect\hyperlink{after-top}{Continue reading the main story}

Supported by

\protect\hyperlink{after-sponsor}{Continue reading the main story}

Fact Check

\hypertarget{trumps-inaccurate-statements-about-the-conflict-with-iran}{%
\section{Trump's Inaccurate Statements About the Conflict With
Iran}\label{trumps-inaccurate-statements-about-the-conflict-with-iran}}

In his address from the White House, the president was misleading in
describing the nuclear deal negotiated by the Obama administration.

\includegraphics{https://static01.graylady3jvrrxbe.onion/images/2020/01/08/us/politics/08dc-factcheck/merlin_166837941_89327e15-2682-4fc6-961b-cd907e28066e-articleLarge.jpg?quality=75\&auto=webp\&disable=upscale}

\href{https://www.nytimes3xbfgragh.onion/by/linda-qiu}{\includegraphics{https://static01.graylady3jvrrxbe.onion/images/2018/06/12/multimedia/author-linda-qiu/author-linda-qiu-thumbLarge.png}}

By \href{https://www.nytimes3xbfgragh.onion/by/linda-qiu}{Linda Qiu}

\begin{itemize}
\item
  Published Jan. 8, 2020Updated Jan. 14, 2020
\item
  \begin{itemize}
  \item
  \item
  \item
  \item
  \item
  \end{itemize}
\end{itemize}

President Trump,
\href{https://www.nytimes3xbfgragh.onion/2020/01/08/world/middleeast/trump-speech.html}{responding
during a White House address} on Wednesday to the
\href{https://www.nytimes3xbfgragh.onion/2020/01/07/world/middleeast/iran-fires-missiles-us.html}{missile
strikes by Iran}, assailed the nuclear agreement reached by his
predecessor and praised American military might. The
\href{https://www.nytimes3xbfgragh.onion/2020/01/08/us/politics/trump-address.html}{10-minute
address} contained numerous inaccuracies and claims that lacked
evidence. Here's a fact check.

What Mr. Trump Said

\emph{``Iran's hostilities substantially increased after the foolish
Iran nuclear deal was signed in 2013, and they were given \$150 billion,
not to mention \$1.8 billion in cash.''}

\textbf{This is misleading.} The agreement reached by Iran, the United
States and a number of other nations to constrain Tehran's nuclear
program did not directly provide American money to Iran, but it did
\href{https://www.nytimes3xbfgragh.onion/2016/01/17/world/middleeast/iran-sanctions-lifted-nuclear-deal.html}{release
about \$100 billion} in previously frozen Iranian assets. Much of the
amount was tied up by debt obligations, for example,
\href{https://www.washingtonpost.com/world/national-security/hearing-on-iran-nuclear-deal-opens-with-sharp-gop-criticism/2015/07/23/2003439e-309d-11e5-8f36-18d1d501920d_story.html?utm_term=.a19fd79013dc}{\$20
billion to China} for financing projects in Iran. Estimates for the
actual amount available to Iran range from \$35 billion to \$65 billion.

A separate
\href{https://www.nytimes3xbfgragh.onion/2016/08/19/world/middleeast/iran-us-cash-payment-prisoners.html}{\$1.7
billion transfer of cash to Iran} was to settle a decades-long dispute
and was agreed to in negotiations that happened parallel to the nuclear
deal. Before the 1979 revolution, Iran's shah had paid \$400 million for
American military goods but, after he was overthrown, the equipment was
never delivered. The clerics who seized control demanded the money back,
but the United States refused. The additional \$1.3 billion is interest
accumulated over 35 years.

Iran and other parties to the nuclear accord signed an
\href{https://www.nytimes3xbfgragh.onion/2013/11/24/world/middleeast/talks-with-iran-on-nuclear-deal-hang-in-balance.html}{interim
agreement in 2013}, but the formal agreement
\href{https://www.nytimes3xbfgragh.onion/2015/07/15/world/middleeast/iran-nuclear-deal-is-reached-after-long-negotiations.html}{was
not reached until 2015}. The White House did not respond when asked for
evidence of increased Iranian ``hostilities.''

It is worth noting that before Mr. Trump withdrew the United States from
the nuclear agreement in 2018, his administration
\href{https://www.nytimes3xbfgragh.onion/2017/07/17/us/politics/trump-iran-nuclear-deal-recertify.html}{repeatedly
certified} that Iran was in compliance.

Afterward, as his so-called maximum-pressure campaign on Iran continued,
tensions between the United States and Iran ``escalated significantly,''
\href{https://www.everycrsreport.com/files/20200106_R45795_9689491d73fa893072e210735ae913435277c51e.pdf}{according
to a recent Congressional Research Service report}. Mr. Trump's claim
blaming the nuclear accord for Iranian aggression rather than his
withdrawal from it is ``almost an inverted reality,'' said Jim Walsh, a
research associate at M.I.T.'s Security Studies Program and an expert on
nuclear issues and the Middle East.

He said that attacks by the four groups supported by Iran and designated
by some governments as terrorist organizations --- Hezbollah, Hamas,
Palestinian Islamic Jihad and the Popular Front for the Liberation of
Palestine-General Command --- actually declined after the nuclear deal.

Attacks carried out by these groups decreased from more than 80 in 2014
to six in 2017, before increasing to more than 40 in 2018, according to
the
\href{https://www.start.umd.edu/gtd/search/Results.aspx?start_yearonly=\&end_yearonly=\&start_year=\&start_month=\&start_day=\&end_year=\&end_month=\&end_day=\&asmSelect0=\&perpetrator=399\&perpetrator=407\&perpetrator=1930\&perpetrator=2108\&perpetrator=838\&perpetrator=4413\&dtp2=all\&success=yes\&casualties_type=b\&casualties_max=}{Global
Terrorism Database} maintained by the University of Maryland's National
Consortium for the Study of Terrorism and Responses to Terrorism. And
while Iran has been a violent and destabilizing force across the region,
Mr. Trump's assertion that Tehran had ``created hell'' lacked context in
some cases.

Iranian aid to President Bashar al-Assad of Syria in that country's
civil war and Tehran's backing of Houthi rebels in Yemen
\href{https://www.theguardian.com/world/2011/may/08/iran-helping-syrian-regime-protesters}{both}
\href{https://www.yahoo.com/news/iran-arming-yemens-huthi-rebels-since-2009-un-202936346.html?guccounter=1\&guce_referrer=aHR0cHM6Ly9lbi53aWtpcGVkaWEub3JnLw\&guce_referrer_sig=AQAAANfAtm7ElL-V-ArfyRoFMf8i9Pks3YI97lUAXHnGAVRYkSI9JPbe7SldOsNA6zSEASgtNaubY_nmARO39x4afp-T3lhYgJAxKRGj61dtVm5iNHjno5e6viM214GWRHmAVkkfIVLk2SmICJwIHaxhRDQR8p5SO0mPkgHFD-1pngp9}{predate}
the signing of the nuclear agreement, formally known as the Joint
Comprehensive Plan of Action.

``There's nothing that Iran was doing after J.C.P.O.A. that it wasn't
doing before,'' said Vali R. Nasr, a professor of Middle East studies at
Johns Hopkins University and a State Department official in the Obama
administration.

Calling Iran's backing of the Houthi rebels against the Saudi
Arabia-aligned government in Yemen terrorism is ``devaluing the word to
the point where it's meaningless,'' said Anthony Cordesman, an expert on
military affairs and the Middle East at the Center for Strategic and
International Studies.

As for Iran's activities in Afghanistan and Iraq, Mr. Cordesman said,
``they were more aggressive there because they were working to attack
ISIS --- as we were.''

What Mr. Trump said

\emph{``The missiles fired last night at us and our allies were paid for
with the funds made available by the last administration.''}

\textbf{This lacks evidence.} The White House did not respond when asked
to substantiate this claim, and experts noted there was no proof that
Iranian assets unfrozen by the deal paid for the missiles.

``There's a certain fungibility here,'' Mr. Walsh said. If the Iranian
foreign minister, Mohammad Javad Zarif, ``took a dollar on the street,
did that fund the missile attack?'' he added. ``That's not very useful
from an analytical perspective. Nor is the case that giving them money
caused them to attack the U.S.''

``We have no indication,'' Mr. Cordesman said, ``whether these missiles
are funded by the money from the J.C.P.O.A.''

The director of national intelligence's annual report on worldwide
threats
\href{https://www.dni.gov/files/ODNI/documents/2019-ATA-SFR---SSCI.pdf\#page=30}{in
2019} did note that Iran continued to develop and improve military
capabilities including ballistic missiles, but it did not tie those
efforts to the nuclear deal. Furthermore, the annual reports warned of
the same efforts in
\href{https://www.dni.gov/files/documents/Unclassified_2015_ATA_SFR_-_SASC_FINAL.pdf\#page=18}{2015},
\href{https://www.dni.gov/files/documents/Intelligence\%20Reports/2014\%20WWTA\%20\%20SFR_SSCI_29_Jan.pdf\#page=9}{2014},
\href{https://www.dni.gov/files/documents/Intelligence\%20Reports/UNCLASS_2013\%20ATA\%20SFR\%20FINAL\%20for\%20SASC\%2018\%20Apr\%202013.pdf\#page=13}{2013},
\href{https://www.dni.gov/files/documents/Newsroom/Testimonies/20120131_testimony_ata.pdf\#page=7}{2012}
and
\href{https://www.dni.gov/files/documents/Newsroom/Testimonies/20110310_testimony_clapper.pdf\#page=5}{before}.

Critics of the Iran deal, including Mr. Trump, have long argued that it
was inadequate because it did not address Iran's ability to develop
ballistic missiles. Those restrictions
\href{https://www.armscontrol.org/2015-08/appendix-e-iran\%E2\%80\%99s-ballistic-missiles-nuclear-deal}{have
instead been established} by the United Nations Security Council
resolutions.

The diplomatic accord was an arms deal with a very narrow aim of curbing
Iran's nuclear ambitions, ``not a nonaggression pact, not a form of a
friendship treaty,'' Mr. Nasr of Johns Hopkins said. ``Whether there
could have been more in the deal, of course. But piling in expectations
is disingenuous.''

What Mr. Trump Said

\emph{``The very defective J.C.P.O.A. expires shortly anyway and gives
Iran a clear and quick path to nuclear breakout.''}

\textbf{This is exaggerated.} The major provisions limiting Iran's
nuclear capabilities last a decade or longer. And the agreement
increased the ``breakout'' period --- the time it would take Iran to
produce enough fuel for one weapon --- to at least a year from
\href{https://fas.org/sgp/crs/nuke/R43333.pdf}{an estimated two to three
months}. If the deal had been left in place and fully adhered to, Iran
would not have been able to achieve nuclear breakout until 2030.

The agreement also prohibits Iran from pursuing nuclear weapons
permanently. ``Iran reaffirms that under no circumstances will Iran ever
seek, develop or acquire any nuclear weapons,'' the first paragraph of
the deal reads.

The American Israel Public Affairs Committee, a vocal critic of the
deal, said it
``\href{https://www.aipac.org/-/media/publications/policy-and-politics/fact-sheets/other/the-iran-nuclear-deal-expiration-dates-and-consequences.pdf}{largely
expires after only 15 years}.''

Under the deal's terms, Iran agreed not to use more than 5,060
centrifuges to enrich uranium --- and to abide by limits on the types of
centrifuges allowed in its research and development program --- for 10
years. Limits on enrichment levels, facilities and stockpiles last for
15 years, according to \href{https://fas.org/sgp/crs/nuke/R43333.pdf}{a
report} from the Congressional Research Service.

Under the terms of the accord, Iran also agreed to convert a deep
underground enrichment facility into a ``technology center'' that cannot
contain nuclear material and where the number of centrifuges is limited
for 15 years. Several provisions on plutonium, including forbidding the
construction of new heavy water reactors,
\href{https://www.belfercenter.org/sites/default/files/legacy/files/IranDealDefinitiveGuide.pdf\#page=13}{last
for 15 years}.

Inspectors are to monitor centrifuges and related infrastructure for 15
years, verify inventory for 20 years and monitor uranium mines for 25
years.

What Mr. Trump Said

\emph{``We are now the No. 1 producer of oil and natural gas anywhere in
the world. We are independent, and we do not need Middle East oil.''}

\textbf{This is misleading.} The United States has been the largest
producer of oil and gas in the world
\href{https://www.eia.gov/todayinenergy/detail.php?id=13251}{since
2013}, a trend that began under the Obama administration thanks in large
part to advances in shale drilling techniques.

The Energy Information Administration
\href{https://www.eia.gov/todayinenergy/detail.php?id=38152}{projected
in January 2019} that the United States will produce more energy than it
imports this year, the first time since 1950. But that is not the same
thing as not importing oil from the Middle East at all. In 2018,
\href{https://www.eia.gov/totalenergy/data/monthly/pdf/mer.pdf\#page=71}{the
United States imported} more than 1.5 million barrels a day from the
Persian Gulf.

What Was Said

\emph{``The American military has been completely rebuilt under my
administration at a cost of \$2.5 trillion.''}

\textbf{This is exaggerated.} The \$2.5 trillion figure refers to the
\href{https://comptroller.defense.gov/Portals/45/Documents/defbudget/fy2020/FY20_Green_Book.pdf\#page=29}{total
defense budgets} of the past four fiscal years: \$606 billion the 2017
fiscal year (which began before Mr. Trump took office), \$671 billion in
2018, \$685 billion in 2019 and \$718 billion in 2020. But the amount
spent on procurement --- buying and upgrading equipment --- was about
\$562 billion over that period.

Mr. Trump's use of the phrase ``completely rebuilt'' is somewhat
subjective. Though the Trump administration has invested in operational
readiness over the past few years, there are signs that the military
continues to face substantial challenges in addressing an array of
threats from around the world.

For example, the military earned a middling grade of
``\href{https://www.heritage.org/military-strength/assessment-us-military-power/conclusion-us-military-power}{marginal}''
from the conservative Heritage Foundation's annual index of strength,
based on factors like shortages in personnel and aging equipment. The
think tank noted that American forces are probably capable of meeting
the demands of a single major regional conflict but ``would be very
hard-pressed to do more and certainly would be ill-equipped to handle
two nearly simultaneous major regional contingencies.''

What Was Said

\emph{``Three months ago, after destroying 100 percent of ISIS and its
territorial caliphate, we killed the savage leader of ISIS, al-Baghdadi,
who was responsible for so much death.''}

\textbf{This is exaggerated.} The Islamic State lost its final
territories
\href{https://www.stateoig.gov/system/files/q3fy2019_leadig_oir_report.pdf}{in
March 2019}, ending the physical ``caliphate,'' but the terrorist group
has not been destroyed. The recent confrontation with Iran has
\href{https://www.nytimes3xbfgragh.onion/2020/01/05/us/politics/us-isis-iran.html}{halted
the United States' campaign} against ISIS.

Just this week, Defense Secretary Mark T. Esper and Gen. Mark A. Milley,
the chairman of the Joint Chiefs of Staff,
\href{https://www.defense.gov/Newsroom/Transcripts/Transcript/Article/2051321/press-gaggle-with-secretary-of-defense-dr-mark-t-esper-and-chairman-of-the-join/}{said
that the fight} against the group was continuing.

Mr. Trump alluded to the organization's endurance in his speech when he
said: ``ISIS is a natural enemy of Iran. The destruction of ISIS is good
for Iran. And we should work together on this and other shared
priorities.''

Curious about the accuracy of a claim? Email
\href{mailto:factcheck@NYTimes.com}{\nolinkurl{factcheck@NYTimes.com}}.

Advertisement

\protect\hyperlink{after-bottom}{Continue reading the main story}

\hypertarget{site-index}{%
\subsection{Site Index}\label{site-index}}

\hypertarget{site-information-navigation}{%
\subsection{Site Information
Navigation}\label{site-information-navigation}}

\begin{itemize}
\tightlist
\item
  \href{https://help.nytimes3xbfgragh.onion/hc/en-us/articles/115014792127-Copyright-notice}{©~2020~The
  New York Times Company}
\end{itemize}

\begin{itemize}
\tightlist
\item
  \href{https://www.nytco.com/}{NYTCo}
\item
  \href{https://help.nytimes3xbfgragh.onion/hc/en-us/articles/115015385887-Contact-Us}{Contact
  Us}
\item
  \href{https://www.nytco.com/careers/}{Work with us}
\item
  \href{https://nytmediakit.com/}{Advertise}
\item
  \href{http://www.tbrandstudio.com/}{T Brand Studio}
\item
  \href{https://www.nytimes3xbfgragh.onion/privacy/cookie-policy\#how-do-i-manage-trackers}{Your
  Ad Choices}
\item
  \href{https://www.nytimes3xbfgragh.onion/privacy}{Privacy}
\item
  \href{https://help.nytimes3xbfgragh.onion/hc/en-us/articles/115014893428-Terms-of-service}{Terms
  of Service}
\item
  \href{https://help.nytimes3xbfgragh.onion/hc/en-us/articles/115014893968-Terms-of-sale}{Terms
  of Sale}
\item
  \href{https://spiderbites.nytimes3xbfgragh.onion}{Site Map}
\item
  \href{https://help.nytimes3xbfgragh.onion/hc/en-us}{Help}
\item
  \href{https://www.nytimes3xbfgragh.onion/subscription?campaignId=37WXW}{Subscriptions}
\end{itemize}
