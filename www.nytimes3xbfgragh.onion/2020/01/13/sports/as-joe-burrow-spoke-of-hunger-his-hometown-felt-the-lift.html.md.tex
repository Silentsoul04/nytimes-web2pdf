Sections

SEARCH

\protect\hyperlink{site-content}{Skip to
content}\protect\hyperlink{site-index}{Skip to site index}

\href{https://www.nytimes3xbfgragh.onion/section/sports}{Sports}

\href{https://myaccount.nytimes3xbfgragh.onion/auth/login?response_type=cookie\&client_id=vi}{}

\href{https://www.nytimes3xbfgragh.onion/section/todayspaper}{Today's
Paper}

\href{/section/sports}{Sports}\textbar{}As Joe Burrow Spoke of Hunger,
His Hometown Felt the Lift

\url{https://nyti.ms/3a7bhrq}

\begin{itemize}
\item
\item
\item
\item
\item
\item
\end{itemize}

Advertisement

\protect\hyperlink{after-top}{Continue reading the main story}

Supported by

\protect\hyperlink{after-sponsor}{Continue reading the main story}

\hypertarget{as-joe-burrow-spoke-of-hunger-his-hometown-felt-the-lift}{%
\section{As Joe Burrow Spoke of Hunger, His Hometown Felt the
Lift}\label{as-joe-burrow-spoke-of-hunger-his-hometown-felt-the-lift}}

The L.S.U. quarterback's comments after winning the Heisman Trophy
resonated with a region that struggles with poverty.

\includegraphics{https://static01.graylady3jvrrxbe.onion/images/2020/01/13/multimedia/13cfp-burrow2-print/13cfp-burrow2-print-articleLarge.jpg?quality=75\&auto=webp\&disable=upscale}

\href{https://www.nytimes3xbfgragh.onion/by/billy-witz}{\includegraphics{https://static01.graylady3jvrrxbe.onion/images/2018/02/16/multimedia/author-billy-witz/author-billy-witz-thumbLarge.jpg}}

By \href{https://www.nytimes3xbfgragh.onion/by/billy-witz}{Billy Witz}

\begin{itemize}
\item
  Jan. 13, 2020
\item
  \begin{itemize}
  \item
  \item
  \item
  \item
  \item
  \item
  \end{itemize}
\end{itemize}

THE PLAINS, Ohio --- Athens High School is perched atop a hill. As the
hill slopes toward flat lands to the west, a terraced parking lot and a
tidy football stadium have been carved out. At the bottom are trailer
homes stacked side by side.

To the students whose parents work at nearby Ohio University, or who
might otherwise enjoy the fruits of a comfortable existence in a
tight-knit community, those homes at the bottom of the hill are a
persistent reminder of the cycle of poverty, the scourge of drug
addiction and the fading light of hope that has long enveloped
Appalachia.

Joe Burrow has been gone from Athens High School for five years now, off
to climb other mountains. But when he jotted down a few bullet points on
a hotel notepad and walked up six steps of a Midtown Manhattan stage
last month
\href{https://www.nytimes3xbfgragh.onion/2019/12/14/sports/heisman-trophy-joe-burrow.html?action=click\&module=RelatedLinks\&pgtype=Article}{to
accept the Heisman Trophy}, he had not forgotten what it was like at the
bottom of the hill.

``I'm up here for all those kids in Athens and Athens County that go
home to not a lot of food on the table, hungry after school,'' Burrow
said that night, pausing at times for effect. ``You guys can be up here,
too.''

Burrow spoke for six minutes, thanking his family, his teammates and his
coaches at Louisiana State University --- which he, as the team's
quarterback, has led to Monday night's national championship game,
against Clemson. He also thanked the coaches at Ohio State, where he
began his college career. Several times, he stopped to wipe away tears.

But it was those 30 seconds that he spoke, with clarity and authority,
about the troubles of his hometown, where he arrived in the third grade
as a son of a football coach and stayed put, that carried the greatest
resonance.

\includegraphics{https://static01.graylady3jvrrxbe.onion/images/2020/01/13/multimedia/13cfp-burrow2/merlin_166899579_6c78e7c8-3050-4b00-ae1e-f1afbce7ac4f-articleLarge.jpg?quality=75\&auto=webp\&disable=upscale}

Ask any of the people from this rural region, from Athens up to Buchtel
and down to Tuppers Plains, what they thought of Burrow's speech, and
chances are they will admit --- not at all grudgingly --- that it
brought tears to their eyes. One of them was Will Drabold, who graduated
from Athens High School three years ahead of Burrow.

A communications consultant who worked in Washington after graduating
from Ohio University, he is the rare young person migrating back to
Athens. Burrow's speech ``was like being struck by lightning,'' he said.
The next morning, Drabold was determined to do something: He put up a
Facebook page asking for donations to the Athens County Food Pantry. The
goal was \$1,000, which he started with a \$50 pledge.

Within 24 hours, the drive had raised \$80,000. By Sunday, nearly a
month later, it had raised more than \$503,000 --- more than five times
the all-volunteer organization's annual budget. Similarly, a food pantry
in Baton Rouge, La., has raised more than \$60,000. Other charitable
groups in southern Ohio have received a modest bump.

Karin Bright, the board president of the Athens County Food Pantry, said
that the board would be deliberate in how it used the windfall, but that
a primary objective would be expanding the reach of the organization,
which provides food for about 400 families per month. That could mean
adding commercial freezer space so that more meat, as well as frozen
fruit and vegetables, can be distributed. Also on the table: further
connecting with the area's social workers.

``The financial impact is going to be enormous,'' Bright said. ``We want
to make sure this money is used wisely.''

Image

Kristin Ashcraft, left, and Korey Wachenschwanz went over food options
with Kathy Hartman, center, at the Athens County Food
Pantry.~Credit...Andrew Spear for The New York Times

On Thursday, dozens of paper grocery bags stocked with beef stew,
chicken, tuna, canned fruit and vegetables, rice, pasta, sauce and bread
stood ready on a broad table at the Athens County Job and Family
Services offices, where volunteers from the pantry logged arrivals,
asked families how much they needed and distributed a corresponding
amount of food.

One man came for a household of 10. A young woman with scabbed skin
toting a young boy arrived. Another man came for an older neighbor who
was ill. By midafternoon, 42 families had been served.

``There's a lot of research, and you hear `food insecurity,' but you
don't know it until you live it,'' said Nicolette Dioguardi, a retired
lawyer who volunteers. ``Until you've eaten chicken back soup and
popcorn for dinner, you don't know what food insecurity is.''

Cheryl, a neatly dressed woman who did not want to give her surname,
never expected to be stopping by. It is one of three places where she
receives food each month. She said that she retired from the county
health department after 15 years and that her husband, a diabetic,
retired from a supermarket chain with plans to spend winters in Florida.
But a mudslide badly damaged their home and wiped out their savings.
Their pension checks leave them \$200 a month for food and gas.

``I'm embarrassed to be here,'' she said. ``It's a lifestyle I never
planned on.''

There are few better places in southeast Ohio to get a window into
poverty and hunger than at its schools, many of which draw from large,
sparsely populated districts set among the wooded hills and valleys.
Teachers are attuned to spot backpacks with a broken strap, shoes with a
flapping sole. At Meigs High School in Pomeroy, Ohio, teachers stocked a
closet with winter jackets, mittens and socks for any student in need.

``We're trying to help them survive,'' said Courtney Irvin, a teacher at
the school, which is in Meigs County, one of the state's poorest.

Image

Joe Burrow accepting the Heisman Trophy in New York last
month.~Credit...Todd Van Emst/Heisman Trust, via USA Today Sports, via
Reuters

Thus, the conditions for canceling school in Meigs County for cold
weather are extreme: below 20 degrees for multiple hours during bus
times. The reality is that for some children, they will be safer at
school, where they can be assured of being warm and getting one
substantial meal.

That concern is so pervasive that many teachers keep food supplies in
their desk. So, too, does Robin Burrow, the principal at Eastern
Elementary School in Meigs County. She is also Joe's mother.

Her office is bright and cheery, a welcoming place for ``kiddos,'' as
she calls them, from kindergarten through fourth grade. The office is
dotted with photos of her husband, Jimmy, and Joe; there is a bookcase
filled with stuffed animal tigers and teddy bears, bracelets and
candles; and the accent colors are purple and gold.

Below her desk is a box of macaroni-and-cheese dinners.

How often does she give them out?

``Every day,'' she said.

The poverty rate at the school --- or those eligible for free or reduced
lunch --- is 36 percent. Every other Friday, bags of food are sent home
with 100 children, about 20 percent of the school's enrollment. One of
Robin Burrow's biggest concerns is what happens during the two weeks
that schools are closed over winter break.

``Honestly, some kiddos we could go do home visits and electricity is an
extension cord down the street to run a refrigerator,'' she said. ``I
guess my umbrella statement would be that when our kiddos are in our
building, they know 100 percent that they are taken care of, that we'll
do everything for them to be safe, happy and healthy. Until a child's
basic needs are met, they can't even begin to be educated.''

Image

Amanda Cochran would sit in the stands at the football games when Joe
Burrow was in high school, cheering for the team.Credit...Andrew Spear
for The New York Times

The Burrows have lived a comfortable life on Jimmy's salary as an
assistant coach at Ohio University, where he retired last year, and
Robin's as an educator. They tried to cushion Joe from the poverty in
the area, but sheltering him from it would never have been possible the
way it is in areas where private schools and exclusive communities can
build moats between the haves and the have-nots.

``I understood it was a poor area when I was young because you're
driving through it and you see these low-income homes that I hadn't
really seen before,'' said Joe Burrow, who was born in Ames, Iowa, and
lived in Fargo, N.D., before arriving in The Plains when he was 8. ``I'd
lived in upper-middle-class neighborhoods before we moved to Athens and
The Plains. You understand, but you don't really understand the
magnitude until you get older.''

Joe's father was thinking of selling their house last year and moving to
Baton Rouge, but Joe Burrow didn't want to lose his connection to the
place.

He said he mentioned poverty and hunger in his speech not because he
hoped for an outpouring of support, but because he wanted to acknowledge
where he was from and how growing up in southeast Ohio had shaped him.
``I just mentioned it because that was in my heart at the time,'' he
said on Saturday.

Nathan White, who is Athens High School's football coach and was the
offensive coordinator when Burrow played there, traveled to New York
last month. He watched the speech at a hotel before joining the Burrows'
party.

``That's the first moment I didn't feel like his former football
coach,'' White said. ``I just felt like a guy from Athens.''

Joseph D. Kittle Jr., back home at a bar in Athens, was another who
watched the speech as if Burrow were speaking only to him. Kittle grew
up dirt poor in Trimble, Ohio, at a time when the brick plants, iron
works and coal mines were flickering out, the hills were stripped bare
of timber and the rivers were dying from chemicals. He graduated from
Ohio University, went to Harvard for graduate school and was gone until
later in life, returning to care for his parents and then marrying a
childhood friend, Beverly Drake.

``We haven't been in charge of our destiny for a long time,'' Kittle
said. ``We weren't really taught to brag a lot, and in fact we were
taught not to draw attention to ourselves. Here's someone who has every
reason in the world to brag, and he's not doing that.''

Image

Larry Hashman loaded food from the Athens County Food Pantry into his
car.Credit...Andrew Spear for The New York Times

Kittle noted how Burrow, after being announced as the Heisman winner,
went over to hug two of his former coaches from Ohio State, who had two
of their own players as finalists.

``If you think about it, nothing was about him,'' Kittle added. ``The
speech itself had this flow to it and a cadence --- the way it was
delivered was so powerful, and then what he had to say was very simple
and very elegant. It was really written in a style that reflects how
people think here. He wasn't trying to have an impact on the food bank,
but the humility spoke for itself and it tore at people's
heartstrings.''

When school let out on Thursday, Athens High looked like any other
campus --- students scrambled down to the parking lot, eager to jump in
their cars and get to wherever they were going. It was easy to imagine
young Joey Burrow being among them --- and how on some days he might
take note of the trailers at the bottom of the hill.

In one of them now lives one of his old classmates, home with her three
children. A block away, Amanda Cochran lives with her young child in a
trailer, trying to make ends meet as a home health care worker. She
would sit in the stands at the football games when Burrow was in high
school, cheering for the team. It was nice, she said, that he remained
just as she remembered him, down to earth.

``You know, we're a pretty poor county,'' she said. ``For him to come
from this community and to show it, you can really tell where his heart
is.''

Advertisement

\protect\hyperlink{after-bottom}{Continue reading the main story}

\hypertarget{site-index}{%
\subsection{Site Index}\label{site-index}}

\hypertarget{site-information-navigation}{%
\subsection{Site Information
Navigation}\label{site-information-navigation}}

\begin{itemize}
\tightlist
\item
  \href{https://help.nytimes3xbfgragh.onion/hc/en-us/articles/115014792127-Copyright-notice}{©~2020~The
  New York Times Company}
\end{itemize}

\begin{itemize}
\tightlist
\item
  \href{https://www.nytco.com/}{NYTCo}
\item
  \href{https://help.nytimes3xbfgragh.onion/hc/en-us/articles/115015385887-Contact-Us}{Contact
  Us}
\item
  \href{https://www.nytco.com/careers/}{Work with us}
\item
  \href{https://nytmediakit.com/}{Advertise}
\item
  \href{http://www.tbrandstudio.com/}{T Brand Studio}
\item
  \href{https://www.nytimes3xbfgragh.onion/privacy/cookie-policy\#how-do-i-manage-trackers}{Your
  Ad Choices}
\item
  \href{https://www.nytimes3xbfgragh.onion/privacy}{Privacy}
\item
  \href{https://help.nytimes3xbfgragh.onion/hc/en-us/articles/115014893428-Terms-of-service}{Terms
  of Service}
\item
  \href{https://help.nytimes3xbfgragh.onion/hc/en-us/articles/115014893968-Terms-of-sale}{Terms
  of Sale}
\item
  \href{https://spiderbites.nytimes3xbfgragh.onion}{Site Map}
\item
  \href{https://help.nytimes3xbfgragh.onion/hc/en-us}{Help}
\item
  \href{https://www.nytimes3xbfgragh.onion/subscription?campaignId=37WXW}{Subscriptions}
\end{itemize}
