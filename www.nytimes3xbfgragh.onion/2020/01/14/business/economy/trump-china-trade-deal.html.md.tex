Sections

SEARCH

\protect\hyperlink{site-content}{Skip to
content}\protect\hyperlink{site-index}{Skip to site index}

\href{https://www.nytimes3xbfgragh.onion/section/business/economy}{Economy}

\href{https://myaccount.nytimes3xbfgragh.onion/auth/login?response_type=cookie\&client_id=vi}{}

\href{https://www.nytimes3xbfgragh.onion/section/todayspaper}{Today's
Paper}

\href{/section/business/economy}{Economy}\textbar{}China Trade Deal
Details Protections for American Firms

\url{https://nyti.ms/2u3pGnW}

\begin{itemize}
\item
\item
\item
\item
\item
\item
\end{itemize}

Advertisement

\protect\hyperlink{after-top}{Continue reading the main story}

Supported by

\protect\hyperlink{after-sponsor}{Continue reading the main story}

\hypertarget{china-trade-deal-details-protections-for-american-firms}{%
\section{China Trade Deal Details Protections for American
Firms}\label{china-trade-deal-details-protections-for-american-firms}}

The agreement is expected to include significant concessions to protect
U.S. technology and trade secrets, but its success hinges on whether
China will follow through on its commitments.

\includegraphics{https://static01.graylady3jvrrxbe.onion/images/2020/01/14/business/14dc-chinatrade-01/14dc-chinatrade-01-articleLarge.jpg?quality=75\&auto=webp\&disable=upscale}

\href{https://www.nytimes3xbfgragh.onion/by/ana-swanson}{\includegraphics{https://static01.graylady3jvrrxbe.onion/images/2018/12/10/multimedia/author-ana-swanson/author-ana-swanson-thumbLarge.png}}

By \href{https://www.nytimes3xbfgragh.onion/by/ana-swanson}{Ana Swanson}

\begin{itemize}
\item
  Jan. 14, 2020
\item
  \begin{itemize}
  \item
  \item
  \item
  \item
  \item
  \item
  \end{itemize}
\end{itemize}

\href{https://cn.nytimes3xbfgragh.onion/business/20200115/trump-china-trade-deal/}{阅读简体中文版}\href{https://cn.nytimes3xbfgragh.onion/business/20200115/trump-china-trade-deal/zh-hant/}{閱讀繁體中文版}

WASHINGTON --- The
\href{https://www.nytimes3xbfgragh.onion/2019/12/12/business/economy/trump-china-trade-deal.html}{trade
deal} that President Trump will sign on Wednesday includes commitments
by China to curtail practices that American firms complain put them at a
disadvantage and force them to hand over valuable intellectual property
to Chinese firms, according to several people with knowledge of the
deal.

Those concessions, along with China's agreement to buy \$200 billion
worth of American goods and to allow greater access to its markets, are
expected to be announced at a White House ceremony for the signing of
the
\href{https://www.nytimes3xbfgragh.onion/2019/12/31/us/politics/trump-china-trade-dea.html}{long-awaited
trade deal}.

As part of the agreement, China has promised to punish Chinese firms
that
\href{https://www.nytimes3xbfgragh.onion/2018/06/22/technology/china-micron-chips-theft.html}{infringe
on or steal corporate trade secrets}, satisfying a concern of American
businesses. China will also refrain from directing Chinese companies to
obtain delicate foreign technologies through acquisitions, including
halting purchases by state-owned enterprises that ``harm'' American
interests. American officials say Beijing has used the practice to leap
to the forefront of advanced industries, like semiconductors.

Another primary concern of American companies --- a requirement that
they turn over technology as a condition of doing business in the
country --- is also addressed in the deal. China has agreed not to force
companies to transfer technology, which it has done by requiring joint
ventures with Chinese firms and forcing companies to license their
intellectual property at low prices.

Trump administration officials say the deal to be signed on Wednesday is
only the first step in talks that are expected to help cool tensions
between the world's two largest economies and start to stabilize
relations after more than a year of escalating threats from both sides.
Mr. Trump has said the second phase of the agreement would be negotiated
``\href{https://www.nytimes3xbfgragh.onion/2019/12/31/us/politics/trump-china-trade-dea.html}{at
a later date}.''

To prevent China from violating the agreement, the administration will
\href{https://www.nytimes3xbfgragh.onion/2020/01/06/business/economy/trade-war-tariffs.html}{continue
to have tariffs} on \$360 billion worth of goods, along with the threat
of future tariffs if China reneges on its promises. The deal does not
include any agreement for future tariff reductions, according to a
spokesman for the Office of the United States Trade Representative.

The success of the deal hinges on whether China will follow through on
its commitments on paper --- something Trump administration officials
and China hawks say it has failed to do in the past. Some critics say
China's promises appear both broad and vague and overlap with other
changes it has been pursuing anyway.

Still, the concessions may go at least part of the way toward resolving
some of the business community's concerns about China's treatment of
foreign firms and the kind of unfair trade practices that Mr. Trump said
his administration would end.

The agreement was ``more positive'' than expected, Myron Brilliant, the
executive vice president of the U.S. Chamber of Commerce, said at a news
conference in Beijing on Monday. He added that striking an agreement had
calmed tensions in a long-running trade war.

``We are pleased from what we've heard,'' Mr. Brilliant said.

\includegraphics{https://static01.graylady3jvrrxbe.onion/images/2020/01/14/business/14DC-TRADE-02/merlin_160540425_17d3afb7-26c5-4b09-8598-8ce2076069fc-articleLarge.jpg?quality=75\&auto=webp\&disable=upscale}

Administration officials say the tariff threat gives the deal more teeth
than previous pacts with China. But it also raises the possibility that
both countries could wind up back in the same type of tit-for-tat trade
war that has inflicted
\href{https://www.nytimes3xbfgragh.onion/2020/01/03/business/manufacturing-trump-trade-war.html}{economic
damage} across the globe.

Text of the trade deal has not been made public in either English or
Chinese. It appears to include significant concessions, but it remains
to be seen how the pact's legal language will translate into action.

For instance, China has yet to admit that it ever forced foreign
companies to transfer technology to Chinese firms, said Derek Scissors,
a resident scholar at the American Enterprise Institute. Reading the
agreement from the Chinese perspective, he said, they have committed to
continue doing the same thing they have always been doing.

``We've had the Chinese agree in this public fashion to things we think
were important before, and it hasn't made a difference,'' Mr. Scissors
said.

Clete Willems, a partner at Akin Gump who helped to advise on trade
policy until he left the administration last year, said the deal would
fulfill three of the four major conditions laid out in the
administration's
\href{https://ustr.gov/sites/default/files/Section\%20301\%20FINAL.PDF}{initial
report} that justified tariffs on Chinese goods. That included a
requirement that China not direct its companies to acquire sensitive
foreign technology.

Mr. Willems said the deal also contained new language protecting trade
secrets, including a promise to set up judicial proceedings and criminal
penalties for Chinese entities that steal confidential business
information. It would also provide greater patent protection for the
pharmaceutical sector.

The one major concern outlined in the administration's report that was
not addressed in the trade deal is cybertheft, Mr. Willems said. China
had rebuffed American demands to include promises to refrain from
hacking American firms in the text, insisting it was not a trade issue.

``We didn't fix every single problem with China in this agreement, there
is no question about that,'' Mr. Willems said. ``But what was done is
really significant.''

Some analysts have expressed skepticism that a broad threat of tariffs
on the overall Chinese economy would really deter Chinese companies bent
on gaining a technological edge by stealing trade secrets.

Senator Chuck Schumer, the New York Democrat, sent a letter to Mr. Trump
on Tuesday expressing ``serious concern'' about the potential for
entering into a weak trade deal.

``China pledging to make short-term purchases of American goods will not
address the fundamental problems that undermine long-term U.S. economic
opportunity, prosperity, and security,'' he said.

The Trump administration itself has cited China's failure to live up to
its agreements. In March 2018, the Office of the United States Trade
Representative detailed a pattern of failed promises by the Chinese
government to no longer force foreign companies to transfer technology
to Chinese firms. China had failed to live up to that commitment ``on at
least eight occasions since 2010,'' the trade office said.

The deal also includes large purchasing agreements that Mr. Trump has
said will raise exports and shrink the American trade deficit with
China, but that experts say might be hard to achieve.

As part of the agreement, China has committed to purchasing an
additional \$200 billion of goods over the next two years. That total
includes \$50 billion of new oil and gas exports, \$32 billion of new
agriculture, \$78 billion of additional manufactured goods and \$38
billion of new services, according to three people briefed on the deal.

Some trade experts have said the agricultural export commitments, which
would translate to \$16 billion in new shipments a year, would be
difficult to meet without rerouting shipments to other countries.

But the targets for manufacturing and services, which include tourism
and education, may be even harder. The number of Chinese students coming
to the United States has been trending downward. And exports of
manufactured goods, which will include Boeing airplanes, medical
devices, automobiles and auto parts and factory equipment, are set far
above current levels.

The agreement also includes substantial changes to Chinese regulations
surrounding food, which Robert Lighthizer, Mr. Trump's chief negotiator,
discussed in a briefing with reporters in December. The changes will
reduce barriers for products including meat, poultry, pet food, seafood,
animal feed, baby formula, dairy and biotech, likely increasing American
exports to China in those categories.

Image

Robert Lighthizer, the United States trade representative.Credit...Anna
Moneymaker/The New York Times

The first-phase agreement does not address some of the
\href{https://www.nytimes3xbfgragh.onion/2019/05/12/business/china-trump-trade-subsidies.html}{administration's
bigger concerns} about
\href{https://www.nytimes3xbfgragh.onion/interactive/2018/11/25/world/asia/china-economy-strategy.html}{China's
economic practices}, including its use of subsidies and state plans to
build domestic industries that flood the global market with low-priced
products, often driving American competitors out of business. Critics
say the practice has undermined American industries like steel and solar
panels, and could prove detrimental to high-tech manufacturers of
electric vehicles, computer chips and robots.

The Trump administration, which had hoped to curtail state subsidies as
part of a trade deal, tried to head off criticism on Tuesday morning by
announcing progress on a multilateral effort to address these practices.

Mr. Lighthizer met with ministers from Japan and the European Union in
Washington, and resolved to press for changes at the World Trade
Organization that would ban many of the subsidies that China provides to
its industries.

He said the three would work together to restrict a variety of unfair
subsidies and funds provided through state-owned enterprises, which the
W.T.O. had previously ruled were not subject to its subsidy rules. Both
are practices China has relied on.

Jennifer Hillman, a trade expert at the Council on Foreign Relations who
has worked at both the trade office and the W.T.O., said the statement
represented ``great promise to correct one of the major problems with
the W.T.O. rules: its inability to discipline subsidies.''

``What remains to be seen is whether these good ideas can be brought
into a formal agreement that is binding on China and others,'' she said.

Keith Bradsher contributed reporting from Beijing.

Advertisement

\protect\hyperlink{after-bottom}{Continue reading the main story}

\hypertarget{site-index}{%
\subsection{Site Index}\label{site-index}}

\hypertarget{site-information-navigation}{%
\subsection{Site Information
Navigation}\label{site-information-navigation}}

\begin{itemize}
\tightlist
\item
  \href{https://help.nytimes3xbfgragh.onion/hc/en-us/articles/115014792127-Copyright-notice}{©~2020~The
  New York Times Company}
\end{itemize}

\begin{itemize}
\tightlist
\item
  \href{https://www.nytco.com/}{NYTCo}
\item
  \href{https://help.nytimes3xbfgragh.onion/hc/en-us/articles/115015385887-Contact-Us}{Contact
  Us}
\item
  \href{https://www.nytco.com/careers/}{Work with us}
\item
  \href{https://nytmediakit.com/}{Advertise}
\item
  \href{http://www.tbrandstudio.com/}{T Brand Studio}
\item
  \href{https://www.nytimes3xbfgragh.onion/privacy/cookie-policy\#how-do-i-manage-trackers}{Your
  Ad Choices}
\item
  \href{https://www.nytimes3xbfgragh.onion/privacy}{Privacy}
\item
  \href{https://help.nytimes3xbfgragh.onion/hc/en-us/articles/115014893428-Terms-of-service}{Terms
  of Service}
\item
  \href{https://help.nytimes3xbfgragh.onion/hc/en-us/articles/115014893968-Terms-of-sale}{Terms
  of Sale}
\item
  \href{https://spiderbites.nytimes3xbfgragh.onion}{Site Map}
\item
  \href{https://help.nytimes3xbfgragh.onion/hc/en-us}{Help}
\item
  \href{https://www.nytimes3xbfgragh.onion/subscription?campaignId=37WXW}{Subscriptions}
\end{itemize}
