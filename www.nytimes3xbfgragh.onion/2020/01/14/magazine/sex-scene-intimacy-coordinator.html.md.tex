The Sex Scene Evolves for the \#MeToo Era

\url{https://nyti.ms/35NyaNv}

\begin{itemize}
\item
\item
\item
\item
\item
\item
\end{itemize}

\includegraphics{https://static01.graylady3jvrrxbe.onion/images/2020/01/19/magazine/19-mag-Intimacy/19-mag-Intimacy-articleLarge-v3.jpg?quality=75\&auto=webp\&disable=upscale}

Sections

\protect\hyperlink{site-content}{Skip to
content}\protect\hyperlink{site-index}{Skip to site index}

Feature

\hypertarget{the-sex-scene-evolves-for-the-metoo-era}{%
\section{The Sex Scene Evolves for the \#MeToo
Era}\label{the-sex-scene-evolves-for-the-metoo-era}}

Studios and theaters are hiring intimacy coordinators to help actors.
The twist: They're also making the scenes sexier.

The intimacy coordinator Chelsea Pace working with Ryan Meyer (lying on
his back) and Josh Fulton during a rehearsal for the short film ``1781''
in Ossining, N.Y.Credit...Jackie Molloy for The New York Times

Supported by

\protect\hyperlink{after-sponsor}{Continue reading the main story}

By Lizzie Feidelson

\begin{itemize}
\item
  Published Jan. 14, 2020Updated Feb. 4, 2020
\item
  \begin{itemize}
  \item
  \item
  \item
  \item
  \item
  \item
  \end{itemize}
\end{itemize}

\textbf{O}n a Saturday afternoon in November, Chelsea Pace marched
through a thickly wooded area in a park in Ossining, N.Y., toward two
actors in Revolutionary War-era dress sitting in a pile of leaves. They
were shooting a short film by a young director named Ethan Fuirst about
soldiers who meet by chance and have a tryst in the forest. Pace is an
intimacy coordinator: a professional facilitator of simulated sex and
nudity in theater and film. She and her colleagues --- there are at
least 45 now working across the United States, Canada and Britain ---
often explain what they do by calling it ``fight choreography for sex
scenes.''

Earlier, in the parking lot, she laid one palm on top of the other,
interlaced her fingers and rehearsed the scene's choreog­raphy to
herself with her hands, flipping her palms this way and that and
grinding the heels of her hands against each other like miniature
gyrating pelvises. ``What we want to avoid is what we call G.P.M.s, or
`grunts per move,' '' she told me. Earnestly, and rather accurately, she
performed a trio of sotto-voce moans: ``When you just get, `unh, unh,
unh.' ''

Until recently, intimacy coordination was virtually nonexistent. But in
the space of about two years, alongside the evolution of the \#MeToo
movement, the profession has become highly sought across parts of the
entertainment industry newly made answerable for the vulnerability of
their actors and the exploitability of their hierarchies. In the fall of
2018, HBO began requiring the presence of an intimacy coordinator on any
set that called for nudity or intimacy, and this past summer, intimacy
directors (in theater, the profession is ``director,'' rather than
``coordinator'') began peppering crews on Broadway. The role includes
aspects of both counselor and choreographer, introducing consent
practices to the artistic process and a new area of technical expertise
to theater and film.

While the director of photography focused her lens, the actors lay down
to practice. Pace had held a rehearsal earlier that week with Fuirst and
two actors via video call. Now one snaked his hand around the other's
thigh; the other turned. Their eyes met. They embraced. A few gyrations
in one position, a gruff rolling maneuver, then a few more gyrations in
another. Pace counted out loud to time the length of each position.

Improvised or lightly choreographed intimacy used to be the norm in film
and television, where the expectation was that actors would experiment
together to see what worked and take new directions in the moment, as
director and crew worked together to solve the inevitable technical
weirdness that accompanies two people pretending to have sex. ``I took
photos of every possible angle you could have sex in,'' the director
\href{https://www.nytimes3xbfgragh.onion/2015/03/01/movies/shooting-film-and-tv-sex-scenes-what-really-goes-on.html}{Judd
Apatow told The New York Times} in 2015. ``On the day, all of it goes
out the window.'' ``It's not the most comfortable thing in the world for
sure,'' the actor Kathryn Hahn --- who recently worked with an intimacy
coordinator on the set of the HBO show ``Mrs. Fletcher,'' in which
\href{https://www.nytimes3xbfgragh.onion/interactive/2019/10/09/magazine/kathryn-hahn-mrs-fletcher.html}{she
plays an empty-nester who begins expanding her sexual horizons} --- told
me of traditional sex scenes. ``You kind of want to just get through
it.''

Pace bent down encouragingly, hands on her knees. Like most intimacy
coordinators, she wore the kind of dark, functional clothing that made
her blend in with the rest of a film crew, but her physical presence
seemed designed to welcome curiosity, feet planted and eye contact
direct. One actor positioned himself gingerly on top of the other, his
face in the crook of his partner's neck. ``You're going to want to drop
your back knee further behind you,'' she told him. ``Further. Even
further. It'll look good from where the camera is behind you.'' As the
actor extended his leg, his body elongated, and his pelvis lowered;
although it wasn't mashed directly on top of his partner's, it began to
look much more as if it were.

``We have to get it,'' a production assistant called out, eyeing the
sunset. Pace reminded the actors to breathe as if they were doing
kundalini breathing in a yoga class. Her voice was low, modulated to be
heard by only the two people she was speaking to and then, when she felt
like it, by everyone. ``Accelerate more at the top of each circle, so
there's an emphasis,'' she said, rotating a finger. (Pace uses
``circles'' to describe what one partner might do with his pelvis while
astride the other.) ``Right now it's smooth.''

The director of photography approached. Could Pace rotate the entire
setup by 90 degrees? Fuirst expressed mild panic, but Pace had already
choreographed the scene as a fluid series of gestures instead of
breaking it into different poses for different shots. The whole sequence
shifted quickly on an axis.

Pace and Fuirst hunched together over a monitor, viewing the first of
several takes. The leaves covering the ground, impossibly crunchy and
loud, required everyone watching to be utterly motionless while the
camera rolled. Once Fuirst thought he'd gotten it, the crew dissolved
like a held breath, with congratulations and stomping. Pace handed out
hand warmers. ``That was fascinating,'' one actor said to her. The other
grinned and picked a few leaves out of his hair.

\textbf{The practice of} relying on an expert to help stage sex scenes
is so new to the entertainment industry that even fairly young actors
speak with authority about how different it used to be. The 27-year-old
actor Humberly González, who recently finished working with an intimacy
coordinator for the first time on the set of the Netflix show
``Jupiter's Legacy,'' described a different, fairly typical gig for
television to me. She and her scene partner, whom she met earlier that
day, were going to be filmed kissing from outside a camping tent,
outlined in silhouette. There was no rehearsal and no specific
choreography. When it was time to shoot, the two actors clambered inside
the tent and were instructed to ``just go for it,'' González recalled,
while the director watched from outside, shouting evaluations. The
actors improvised ``full on'' as the crew struggled to best instruct
them to position themselves so the shadows would look right. It took
several hours to painstakingly establish the simple series of movements
by trial and error. Based on what she described, it didn't seem totally
clear that the actors even needed to kiss --- I thought of 2000s
production footage of ``Hair'' that I had recently seen on YouTube, in
which the actors created shadow play of sexual appendages using their
arms. In González's situation, she and her scene partner were touching
the whole time, and he became unintentionally aroused. ``It was so
awkward,'' González said.

As with other intimate scenes she had done, the potential repercussions
for speaking up trumped any feelings of discomfort González might have
had. She was a willing participant in the shoot, and she wondered in
retrospect if she might have been perceived as a problem, or even ``lose
the job,'' she said, if she had asked to do the scene differently.
Addressing an actor's emotional needs, she knew, could be harder than
finding a replacement for a complainer. As an actor, ``There's always
this very scary feeling of: If I share my true feelings, am I going to
be hired again?'' Besides, challenging artistic environments were part
of what had made her want to be a performer. She seemed uncertain
whether to name what she had experienced as a problem, exactly. At the
time, she explained, she opted for commiseration --- her scene partner
was mortified, too. ``I was like, `Our job is crazy!' ''

Despite the inherent awkwardness of shooting such scenes, ``I never
thought it was anything where there was anything else needed,'' Hahn,
who has mostly worked with female directors, said. She perceived the
lack of mediation between herself and her director to be as much a part
of their ``sacred connection'' as of the professional culture of
moviemaking. Actors often say, as Hahn did, that sex scenes don't feel
as if they are ``about'' sex anyway --- or even, occasionally, that they
don't exactly feel like acting. Sexual intimacy for stage or screen is
at once unnatural to perform and physically literal in a way that other
scenes aren't. Actors and audiences tend to have a different ``physical
and psychological barrier'' with other kinds of physical contact ---
especially with violence --- than they do with depictions of sex, the
intimacy director Siobhan Richardson told me recently. Everyone knows
violence is fake: ``I'm never actually choking somebody,'' she said. The
fact that penetration is simulated in sex scenes is almost beside the
point: Intimate proximity between actors almost never is. ``Every kiss
in every film,'' the film theorist Linda Williams writes in her book
``Screening Sex,'' ``is already a kind of documentary.''

An unchoreographed sex scene is not always an irresponsible one, and
some directors do take pains to choreograph simulated sex. But sex
scenes are ripe both for failures of communication and for deliberate
abuses of power, which, in the absence of a clear protocol, can be hard
to tell apart. In December,
\href{https://www.hollywoodreporter.com/features/ruth-wilson-left-affair-hostile-environment-nudity-issues-1263553}{The
Hollywood Reporter reported} that Ruth Wilson left Showtime's ``The
Affair'' in part because of the way the show's sex scenes were handled;
sources claimed that she felt pressured past her comfort level on set
and ignored when she took objection with the frequency of nudity on the
show. For the show's final season, this year, the network hired an
intimacy coordinator. After the release of the 2013 Abdellatif Kechiche
film ``Blue Is the Warmest Color,'' the film's lead actors, Léa Seydoux
and Adèle Exarchopoulos, spoke about feeling ``horrible'' and
``manipulated'' during the exhaustive shoot, which, in a manner most
chalked up to Kechiche's directing style, included days devoted to
extracting satisfactory performances of simulated sex. Further along the
spectrum, decades after the filming of Bernardo Bertolucci's 1972 film
``Last Tango in Paris,'' and nearly seven years after
\href{https://www.dailymail.co.uk/tvshowbiz/article-469646/I-felt-raped-Brando.html}{Maria
Schneider told The Daily Mail} she had ``felt a little raped'' on set,
Bertolucci admitted that he and Marlon Brando conspired to have Brando
surprise the much younger and less famous Schneider with unscripted
sexual contact --- in order,
\href{https://www.youtube.com/watch?v=RMl4xCGcdfA}{Bertolucci told a
2013 interviewer}, to give her performance a quality of ``real rage''
that he presumably did not think her capable of producing otherwise.
Similar issues have been reported in the theater world: In 2016, the
respected nonunion Chicago theater company Profiles closed after
\href{https://www.chicagoreader.com/chicago/profiles-theatre-theater-abuse-investigation/Content?oid=22415861}{The
Chicago Reader revealed} a widely known, 20-year pattern of abuse by its
artistic director that included sexual harassment of actors during
performances of intimate scenes.

These are well publicized incidents, but practically no actor lacks ``a
story of something that happened to them or that they witnessed'' going
wrong in an intimate scene, the intimacy director Claire Warden told me;
it can seem hard to avoid not just because of abusive behavior but
because a lackadaisical attitude about physical boundaries is baked into
those scenes' traditional physical mechanics. Several years ago, while
on ``The Howard Stern Show,'' Sarah Silverman briefly mentioned a
``brutal sex scene'' she experienced with an extra on the set of a
comedy, during which she said she was ``looking around going, `Is there
going to be a board or something between us?' ''

In January 2018, while preparing to shoot the second season of ``The
Deuce,'' David Simon's series about the blossoming porn industry in
1970s New York, the actor Emily Meade, who played a sex worker and porn
actress named Lori, decided to ask HBO to hire someone expressly to
facilitate sex scenes. Other actresses' revelations about harassment
that fall had begun to make her realize how much she had ``begun to
develop armor'' and was ``sort of just completely detaching'' from her
body when she was doing simulated sex and nudity, she told me. ``I
didn't want to have to shut down and dissociate in order to do my job.''

HBO hired Alicia Rodis, a founder of Intimacy Directors International,
now the leading organization for intimacy coordination, to work on the
show. After the season wrapped, Rodis moved in-house, and a host of
other networks, including Netflix, Showtime, FX and TNT, began hiring
coordinators for the first time.

Meade's relationship with Rodis ``completely transformed'' her
experience as an actor. She and Rodis are currently consulting with
SAG-AFTRA on the union's first set of guidelines for intimacy
coordination. Before working with an intimacy coordinator, ``the only
power I ever had in my career was just saying no'' to a sex scene, she
said. ``There's always been a very complicated relationship with how you
communicate, and what power you have, once you say yes.'' Having a
professional to discuss her concerns about each scene before shooting
``brought a sort of consciousness and awareness and purposefulness to
these scenes that I had never had before,'' she said. Meade had
approached HBO worried about the toll sex scenes were taking on her
personally; she now realized that the toll had also been an artistic
one.

Typically, an actor is supposed to ignore any distractions of her body
--- the fact that she is hungry, cold, physically intimidated or
coursing with adrenaline --- although a director measures an artist, at
least in part, precisely by how deftly she manages to show these falling
away when she performs. One definition of artistry in performance is the
sophistication with which an actor marks the space between art and the
everyday. Yet it is still fairly novel to argue that the material
conditions of an artist's physical existence are of fundamental
importance to the question of how and whether she can do her best work.
Working with an intimacy coordinator, ``I've found so many different
layers within'' the performance of sex, Meade said. She sounded both
strident and buoyant about the discovery, the way people do when they've
just solved the source of a long-bothersome bug infestation or allergic
reaction.

To her surprise, Hahn said, she, too, had found intimacy coordinators
``indispensable.'' Usually, after getting to set for a sex scene, Hahn
recalled, she and her partner would ``kind of fake it a little bit until
the camera person told you what was working and what wasn't.'' With the
coordinator right beside her during shooting, telling her exactly how
the shot was going to be constructed, and how it looked, sex scenes were
much quicker to do and contained less ``shine'' extraneous to artmaking.
``You actually got to that meat, all the actual underneath, so much
faster,'' Hahn said. On first hearing the term ``intimacy coordinator,''
she joked that it sounded silly; it reminded her of lingerie. ``I'm sure
I'll do scenes in the future without them,'' she said. ``But hopefully
not.''

\includegraphics{https://static01.graylady3jvrrxbe.onion/images/2020/01/19/magazine/19-mag-Intimacy-02/19-mag-Intimacy-02-articleLarge.jpg?quality=75\&auto=webp\&disable=upscale}

\textbf{In September,} a few weeks before the Broadway previews for
\href{https://www.nytimes3xbfgragh.onion/2019/10/06/theater/slave-play-review-broadway.html}{Jeremy
O. Harris's ``Slave Play''} began, Claire Warden arrived to finesse an
intimate scene at a Midtown rehearsal studio. The actor Ato
Blankson-Wood, who plays a character the audience first meets as an
enslaved man, Gary, hopped up on a prop wagon and stood cockily on its
lip, arms akimbo, pelvis pointed square to the front of the room.

Although stage directions, especially in ambitious plays, often barely
describe what will take place onstage, Warden, who was one of Intimacy
Directors International's first members, told me she wondered, How are
we going to do this? when she first read ``Slave Play.'' This scene
called for an increasingly erotic ``meleed pas-de-deux'' between two
characters, in which ``bloodied faces begin colliding, open mouths
swallowing chins and noses and ears and missing mouths completely.'' The
scufflers are an ``off-white'' man named Dustin and ``blue-black'' Gary,
one of three interracial couples we meet on an antebellum-era
plantation. In the first act of the play, which is only the second
Broadway show with an intimacy director on staff (the first was last
summer's revival of Terrence McNally's
\href{https://www.nytimes3xbfgragh.onion/2019/05/30/theater/frankie-and-johnny-review-audra-mcdonald.html}{``Frankie
and Johnny in the Clair de Lune,'}' which Warden also worked on),
revolves around taboo sexual encounters between each couple. Later --- a
spoiler --- we learn that the antebellum scenes are role plays practiced
by contemporary couples undergoing an outlandish experimental therapy.
The second half of the play is spent dissecting the racial and sexual
dynamics displayed in the first.

Blankson-Wood was in athleisure for rehearsal, but during the show he
would be wearing shiny black boots and a pair of Calvin Klein briefs,
his chest covered in sweat. From atop the wagon, he tried a line. ``Get
on your knees and open your mouth, close your eyes and let your tongue
dance around,'' he ordered in a loud voice. As he spoke, he dropped into
a low, suggestive crouch.

He paused, straightened. ``Is that stupid?'' he asked.

``I like the exposure here,'' Warden said thoughtfully, squaring her own
pelvis. ``But as soon as you start, if your pelvis is present, then we
are going to start assuming oral sex. Which I don't want them to.''

The scene would culminate in erotic bootlicking, with Gary standing
astride the wagon in a position of domination over Dustin: It was the
play's most ``sexually odd'' scene, as Harris put it to me later. He
described it as an attempt to further ``queer'' the audience's
expectations of sexual power dynamics by showing intimacy between
cisgender homosexual men in which penetration wasn't the central sexual
act. He also wanted it to be sexy: ``The audience's mirror neurons
should be going off.''

Blankson-Wood shifted. The actor James Cusati-Moyer, playing Dustin, got
on his knees and crawled forward, eyes closed. ``Can you guide him
slightly?'' Warden asked. Blankson-Wood improvised a ``tisk-tisk'' sound
against the back of his teeth, the way you'd call a cat. It was
delicately, compellingly dirty, and the production staff sitting around
the edges of the room made sounds of instinctual approval. ``Ooh! Yes!''
Warden said.

``Someone in London actually did that to me on the street,''
Blankson-Wood said. The room groaned.

Warden confirmed with Cusati-Moyer that he was O.K. being summoned by
that particular noise.

Several minutes later, the group got stuck on a tiny piece of blocking.
Before he and Blankson-Wood became entangled on the wagon, there were a
few lines of sparring dialogue, and Cusati-Moyer thought it would feel
natural to counter Blankson-Wood with a single step in his direction
during one of his lines. Warden thought he should stand his ground. The
decision would help set up the rest of the scene, which told a story
about how racial hierarchies might dissolve, recongeal and then rear up
monstrously again over the course of a single sexual encounter. ``Any
specific turn or subtle movement affects everything,'' Cusati-Moyer said
later.

In Susan Sontag's 1967 essay ``The Pornographic Imagination,'' about
Restoration-era literary pornography, she writes that obscene literature
is not all merely obscene: Instead, explicit scenes ``carry with them
something that touches upon the reader's whole experience of his
humanity --- and his limits as a personality and a body.'' As the scene
between Gary and Dustin crescendos, Dustin licks Gary's boot while
making throaty, encouraging ``urm-hrm''s of approval. The sound at once
conveys the maker's own pleasure and meanwhile is an energetic
performance of enjoyment for a partner. That specific tone of voice, its
texture of both private and performed pleasure, is one I don't think
I've ever seen deployed onstage before. It is a big part of what makes
you feel, as an audience member, that Dustin loves Gary, wants him to
feel enjoyed.

Filmed sex scenes can be subtler than theatrical ones, because stage
actors have to exaggerate gestures to register before a crowd, but
Cusati-Moyer also told me he feels more empowered onstage. In theater,
you're ``in control of the moment,'' he said, not worrying about how
it'll be seen when replicated. Onstage, however, the repeatability of a
moment is key; when the dynamic between actors feels too sketched out or
porous, the reactions of the audience can threaten to destabilize their
connection. ``I'm a person who needs repetition,'' Blankson-Wood said.
``I need to know that this is where my hand is going every time, so that
when I do it in the moment it's coming from emotion and not technique.''

Robert O'Hara, the production's director, is known for his deftness
helming bold, sexually provocative material, and before working with an
intimacy director, he made a habit of having ``a conversation about our
bodies and also about the respect that we must have'' with his casts,
and issuing disclaimers during auditions warning actors who weren't
comfortable taking risks to reconsider auditioning. But he now viewed
those efforts as incomplete; Warden's field seemed as distinct from his
own as that of a costume or lighting designer. ``I don't know what to
say when it comes to certain things --- or how to design a shirt, or how
to focus a light,'' he said. During a second ``Slave Play'' rehearsal in
September, Warden and Cusati-Moyer had a lengthy sidebar about the wound
cover they would place on top of Gary's boot leather so that
Cusati-Moyer wouldn't actually have to be licking its hide. As someone
with a different relationship to the power hierarchy of the production,
O'Hara said, he had come to realize actors ``feel like they can say
certain things to her that they may not want to say in front of me.''

\textbf{Warden began training} to be an intimacy coordinator in 2017
after reading about a fight choreographer named Tonia Sina who had spent
the last 14 years developing a set of protocols for what she termed
``intimacy for the stage.'' Sina began her thinking while completing
graduate studies in movement coaching for theater, and she honed it in
subsequent years working with students at university programs and as a
consultant for professional theater companies. She, Siobhan Richardson
and Alicia Rodis, the inti­macy coordinator now at HBO, founded Intimacy
Directors International in 2015.

Sina told me she had spent a decade trying to bring a consent-based
approach to theater education, which she knew firsthand had widespread
problems with staging intimate scenes. Her organization's core ``five
pillars'' of intimacy, which she wrote with Rodis, Richardson and other
colleagues, are now available as a shareable sheet on their website for
anyone to use. She and Rodis built on their own experiences by
interviewing trauma counselors and people involved in BDSM, who were
helpful, Rodis told me, because they ``work, and live, and love within
guidelines that have to be very explicit.'' They began offering weeklong
intensives for prospective coordinators in 2017, but training for the
discipline primarily consists of a lengthy shadowing process. ``It's a
very individual discipline,'' Warden said. ``The way I work, and my
aesthetic, is different than my colleagues'. Tonia's work is very
restrained, elegant, `the tilt of the wrist.' My aesthetic, I refer to
it as epic realism. It's often dark and powerful and a little right at
the edge of humanity.''

On set, intimacy coordinators reserve the power to stop a simulated sex
scene after too many takes or to veto a particular move if it hasn't
been properly vetted and discussed with the performer. An intimacy
coordinator is the one who makes sure a set is closed before a scene
begins, ordering crew out so they aren't milling around within eyeshot,
which used to be common. Before now, Rodis told me, costume and makeup
artists often served as de facto intimacy coordinators --- they are
still the ones who glue on pubic wigs and hold out robes for actors ---
and she says people in those roles often approach her to confide their
gratitude that a set of duties they've long performed on the fly, like
pulling actors aside when they are clearly uncomfortable, or even, in
the case of one costume designer turned intimacy coordinator, speaking
up to directors themselves, have finally been formalized and given a
name.

The form's technical aspects are most similar to those of fight
choreography, which also revolves around deconstructing movement and
engineering a look of passion and spontaneity between two bodies. Rather
than choreographing contact so it looks painful but isn't, inti­macy
coordination can involve creating physical arrangements that aren't as
intimate as they look --- often using specially developed foam or
lamb's-wool barriers, which are more sophisticated than modesty
garments. (Those include a flimsy nylon thing called a Shibue, which is
marketed as lingerie and is known for peeling off.)

Workshops also teach specific technical skills related to realistic
portrayals of affectionate touch: A recent trainee named Mitchell McCoy,
one of only a handful of male intimacy coordinators in the field, which
is primarily dominated by women, most of them white, said one memorable
lesson was about how emotion tends to manifest in the hands. If you
``don't know how to handle someone in a physical capacity,'' he said,
touch tends to be constrained to finger pads. He demonstrated a
dispassionate, there-there pat on the edge of the table. If motivated
``more from the heart,'' touch tends to incorporate the palm. If
sexually motivated, touch might include pressure through the hand's
heel.

The specifics of every scene tend to differ so much from project to
project, however, that it's more common to employ a more flexible
problem-solving approach than to import isolated moves whole-cloth. An
intimacy coordinator's most specialized skill is less a technical trick
you could teach in an hour and more a way of talking and describing
movement to elicit a particular quality of response. This is extremely
difficult to do well --- a bit like being an advanced teacher of dance
or yoga combined with being an astute and savvy coach. What do you say
to an athlete midgame, in a moment of tension or uncertainty, to get his
or her body to move differently, better, with more nuance?

Pace uses simple vocabulary terms when instructing the actors on how to
perform --- like ``circles,'' for what someone might do with his pelvis,
or ``visible power shift,'' for switches in position. The body moves
more realistically when it is responding to descriptive language, it
turns out, than it does while being cued with the words for sex
positions. If this sounds counterintuitive, picture a director telling
an actor to ``make it a little sexier'' --- which is about what the
director sees, not what the actor feels --- versus an intimacy
coordinator instructing her to walk more slowly and deliberately, or to
close the distance between her body and her scene partner's.

When it comes to depiction itself, intimacy coordination doesn't lead to
markedly different visual treatments of sex: A director's vision, and a
network's restrictions, are still paramount, and intimacy coordination
is designed to help fulfill whatever a director's visual and dramatic
goals for a scene might be. On HBO's
\href{https://www.nytimes3xbfgragh.onion/2019/06/14/arts/television/euphoria-review-hbo.html}{``Euphoria,''}
for example, a series about teenagers grappling with various forms of
addiction and abuse, this season intimacy coordinators helped facilitate
unflinchingly graphic depictions of sex performed by very young actors;
on FX's
\href{https://www.nytimes3xbfgragh.onion/2018/06/01/arts/television/pose-review-fx-ryan-murphy.html}{``Pose,''}
about the gay and transgender ballroom community, they facilitated a
love scene between two H.I.V.-positive characters that was
groundbreaking in its subject matter but shot in the tastefully banal
way of most cinematic intercourse, in a quick-cutting, nipple-grazing
montage. I sometimes wondered, though, whether having a sex-positive,
consent-educated, inspiringly unafraid person toting nipple covers in
many skin shades might eventually have some palpable effect on how
intimacy looked by the time it reached a viewer. Intimacy coordination
creates choreography specific not only to a director's vision but also
to a role, taking into account how both the character and the actor
playing her might feel. If Hollywood stories are indeed moving away from
those told largely by cisgender white men, that shift seems as if it
would inevitably be hastened by the presence of a professional whose job
it is to suggest subtler revelations about the physical and
psychological experiences of sex.

Katja Blichfeld, a creator of
\href{https://www.nytimes3xbfgragh.onion/2016/09/16/arts/television/review-high-maintenance-hbo.html}{``High
Maintenance,''} HBO's gently transgressive episodic comedy about an
unnamed weed dealer and the New Yorkers who make up his clientele, told
me that this season, she and her co-writer Isaac Oliver initially wrote
a scene with stage instructions that read, ``so-and-so eats so-and-so's
ass on the terrace,'' she recalled, laughing. Her background in casting
makes her primarily attuned to the way actors define a scene, and as a
result, she says, she approaches writing loosely: ``My first thought is
like, what's fun, what's funny?'' Before shooting, Rodis, newly hired to
coordinate the show, reviewed the script. Her questions, posed to help
make the scene look more realistic --- ``How is this touch or physical
act deepening the narrative?'' --- made Blichfeld question why she
needed that sex act. What did it reveal about the characters?
Ultimately, she said, she realized she didn't need it: It was extraneous
to the story, and she came up with something else --- a close-cropped
make-out scene --- before the actors even got to set. Some intimacy
coordinators even hope that they might participate in script writing in
the future; Pace offers consultations on postproduction.

\textbf{Late last September,} the intimacy coordinator Mitchell McCoy
arrived at the theater of the School of Drama at the New School in
Manhattan to work a scene between two sophomores in the B.F.A. program
who were just starting rehearsals on an adaptation of ``Fuenteovejuna,''
a 17th-century Lope de Vega play about Spanish villagers who overthrow a
villainous military commander. Programs like the Yale School of Drama,
Juilliard and the New School have been integrating intimacy coordination
more regularly into their curriculums; both actors in this rehearsal had
been introduced to the discipline during a workshop their freshman year.
Neither had worked with an intimacy director yet, though. As faculty
members watched from folding chairs, McCoy and the actors sat
cross-legged on the floor. That day, he explained, he would be breaking
their first kiss into what he called a ``rough skeletal structure.''

Mouth corners twitching, the two actors, Chace Chester and Malaika
Wilson, who are 20, stood to assume their embrace. Chester, who
previously studied acting at the prestigious Baltimore School for the
Arts, later said he had never actually kissed anyone onstage before.
When he first heard about the discipline, he was ``confused,'' he told
me. ``I thought, you know, you just kiss the kiss.'' After being cast in
the production, ``I thought I would be comfortable with it,'' he said.
It wasn't until he was standing there in front of Wilson that he
realized he had no idea what to do with his hands.

``You should have both hands come around the bottom rib height,'' McCoy
told him. To Wilson, he said, ``If you're both O.K. with it, slide your
right hand downstage over the left trapezius, basically coming up to the
back of his neck.'' McCoy asked them to try holding the pose for 10
seconds. ``So basically coming into sort of like the awkward high school
dance hug.'' They stood primly, then broke away. ``Awesome!'' he said.

Chester looked thoughtful. ``How long is 10 seconds?'' he wondered
aloud. McCoy nodded. ``We have to sort of feel it out --- that's totally
natural if right now it varies.''

Next they tried murmuring kissing sounds, their heads wiggling back and
forth like a pair of dashboard bobble heads. They laughed. ``Great!''
McCoy said. ``That was really special.''

For half an hour they kept practicing, rocking back and forth into a
tableau of a kissing position and out again. The process was comically
incremental, but the inching progress had a soothing effect. Music
wafted from down the hall, and the atmosphere lightened. The mood
between the actors seemed at once calm and energized, the way real
intimacy can be. ``If you're like, `This is weird, humans don't do this
quite often!' that's totally O.K.,'' McCoy said.

``It's weird,'' Chester admitted, shrugging, ``but I \emph{am} more
comfortable.''

Advertisement

\protect\hyperlink{after-bottom}{Continue reading the main story}

\hypertarget{site-index}{%
\subsection{Site Index}\label{site-index}}

\hypertarget{site-information-navigation}{%
\subsection{Site Information
Navigation}\label{site-information-navigation}}

\begin{itemize}
\tightlist
\item
  \href{https://help.nytimes3xbfgragh.onion/hc/en-us/articles/115014792127-Copyright-notice}{©~2020~The
  New York Times Company}
\end{itemize}

\begin{itemize}
\tightlist
\item
  \href{https://www.nytco.com/}{NYTCo}
\item
  \href{https://help.nytimes3xbfgragh.onion/hc/en-us/articles/115015385887-Contact-Us}{Contact
  Us}
\item
  \href{https://www.nytco.com/careers/}{Work with us}
\item
  \href{https://nytmediakit.com/}{Advertise}
\item
  \href{http://www.tbrandstudio.com/}{T Brand Studio}
\item
  \href{https://www.nytimes3xbfgragh.onion/privacy/cookie-policy\#how-do-i-manage-trackers}{Your
  Ad Choices}
\item
  \href{https://www.nytimes3xbfgragh.onion/privacy}{Privacy}
\item
  \href{https://help.nytimes3xbfgragh.onion/hc/en-us/articles/115014893428-Terms-of-service}{Terms
  of Service}
\item
  \href{https://help.nytimes3xbfgragh.onion/hc/en-us/articles/115014893968-Terms-of-sale}{Terms
  of Sale}
\item
  \href{https://spiderbites.nytimes3xbfgragh.onion}{Site Map}
\item
  \href{https://help.nytimes3xbfgragh.onion/hc/en-us}{Help}
\item
  \href{https://www.nytimes3xbfgragh.onion/subscription?campaignId=37WXW}{Subscriptions}
\end{itemize}
