Sections

SEARCH

\protect\hyperlink{site-content}{Skip to
content}\protect\hyperlink{site-index}{Skip to site index}

\href{https://myaccount.nytimes3xbfgragh.onion/auth/login?response_type=cookie\&client_id=vi}{}

\href{https://www.nytimes3xbfgragh.onion/section/todayspaper}{Today's
Paper}

Letter of Recommendation: `The Happy Song' by Imogen Heap

\url{https://nyti.ms/2TjsxE9}

\begin{itemize}
\item
\item
\item
\item
\item
\end{itemize}

Advertisement

\protect\hyperlink{after-top}{Continue reading the main story}

Supported by

\protect\hyperlink{after-sponsor}{Continue reading the main story}

\href{/column/letter-of-recommendation}{Letter of Recommendation}

\hypertarget{letter-of-recommendation-the-happy-song-by-imogen-heap}{%
\section{Letter of Recommendation: `The Happy Song' by Imogen
Heap}\label{letter-of-recommendation-the-happy-song-by-imogen-heap}}

\includegraphics{https://static01.graylady3jvrrxbe.onion/images/2020/01/19/magazine/19Mag-LOR-1/19Mag-LOR-1-articleLarge-v4.jpg?quality=75\&auto=webp\&disable=upscale}

By Mark O'Connell

\begin{itemize}
\item
  Jan. 14, 2020
\item
  \begin{itemize}
  \item
  \item
  \item
  \item
  \item
  \end{itemize}
\end{itemize}

The song I listened to most this past year was ``The Happy Song,'' by
the English singer-songwriter Imogen Heap. I didn't play any other songs
by Imogen Heap last year, or any other year. I don't think I've ever
even heard any other songs by her. But this song, ``The Happy Song,'' I
played every single day, often eight or 10 times in succession. As
music, I could take it or leave it. It's not great. But from a strictly
utilitarian point of view? From the point of view of sheer
effectiveness? Surely the greatest song ever recorded.

The reason I play it so much is for the effect it has on my 18-month-old
daughter. She'll be screaming her head off about something ---
tiredness, crankiness, being left alone for 10 minutes with a father who
is incapable of breastfeeding her and is therefore of limited practical
use --- and at some point I'll remember that ``The Happy Song'' exists,
and immediately put it on the stereo. The song opens with the sound of a
baby's gurgling laughter, an introduction that never fails to capture my
daughter's attention. The correct description for what is going on here
is that \emph{it gives her pause}.

Credit...CreditVideo by Imogen Heap

By the time the music proper kicks in, with its jouncing 4/4 strings and
its sprightly whistled melody, she has invariably stopped crying
altogether. She turns toward the speakers, tears still rolling down her
cheeks, and a smile starts to spread across her face, and she nods her
head emphatically, fixing me all the while with a look of intensifying
joy that I understand to mean: ``Let us take a moment to appreciate what
an absolute banger this song is.'' By the time Imogen Heap starts
singing very plummily about choo-choo trains and aeroplanes and rockets
to the stars, the song has worked its affective alchemy on my daughter.
She raises a single plump arm above her head, swinging her little hips
to some loose approximation of the beat. Once it's over, she will,
without fail, give me a quizzical look and say, ``App-EE?'' --- which I
take to mean: ``Is there anything to be said for giving the old `Happy
Song' another spin?''

The slightly unnerving fact about this song is that it was designed with
this precise effect in mind. The London ad agency BETC, working on
behalf of the baby-food behemoth Cow \& Gate, wanted to engineer a piece
of music to delight children between the ages of 6 months and 2 years.
There's a video on the agency's website that
\href{http://www.betc.co.uk/the-sound-of-happiness}{documents the
creation of ``the world's first song scientifically proven to make
babies happy.''} During a monthslong testing period, the team --- which
included both a developmental and a musical psychologist --- asked
British parents to tell them which sounds made their infants happiest.
They then gathered recordings of the most popular of these sounds, which
they tested on actual babies, measuring heart rates and facial
expressions and vocalizations. The video includes footage of babies
wired up to heart monitors, as scientists pore over complicated-looking
data-modeling software. The findings of all this research were
eventually handed over to Imogen Heap, whose resulting song incorporates
many of the sounds --- beeping horns, ringing bells, meowing cats ---
determined to be the most captivating to the most babies.

What we are talking about here is, in some unavoidably literal sense,
mind control. And the song is such an effective dopamine-delivery
mechanism that I sometimes wonder, as I cue it up for the ninth time in
a row, whether I am unwittingly laying down the precise neural pathways
in my daughter's tender little brain that will ensure a lifetime of
addictive behavior. There is something creepy, too, about the way the
song attempts to achieve its ends, leveraging the emotions of babies to
increase parents' awareness of a baby-food brand. And you wouldn't have
to lean too hard into this interpretation to start seeing the song ---
which was conceived as a corporate-branding exercise, germinated in a
mulch of data and audience testing, optimized for maximal engagement and
delivered via algorithmic targeting --- as a troubling intensification
of existing trends in the production of culture under capitalism. When I
think about it like this, there's a sense in which ``The Happy Song''
flies in the face of my arguably quixotic parenting ethos, much of which
boils down to: ``Keep capitalism as far as possible from the children
for as long as possible.''

But these are also somewhat abstract considerations, given that since
``The Happy Song'' came into our lives, the total number of Cow \& Gate
products purchased by either myself or my wife remains zero. Based on
this admittedly small sample, the song is far more effective at making
babies happy than it is at making adults buy stuff. And that's what is
so joyous about the song: the fact that it works. She's unhappy, and
then the song comes on, and then she's happy. In its simplicity, it
feels like a kind of magic.

The world is a complex and, in many ways, unthinkably dark place, and I
am well aware that the window of time in which it is possible to
transform my daughter's unhappiness into joy by playing a jaunty little
song is already closing. If the ad agency's research is accurate, my
daughter remains within its target demographic for less than four more
months. And it's the knowledge of this ephemerality that makes the song,
and its effect on her, so precious. It won't work forever, because she
won't forever be so small and innocent. But right now it works. Right
now it's the greatest song ever written.

Advertisement

\protect\hyperlink{after-bottom}{Continue reading the main story}

\hypertarget{site-index}{%
\subsection{Site Index}\label{site-index}}

\hypertarget{site-information-navigation}{%
\subsection{Site Information
Navigation}\label{site-information-navigation}}

\begin{itemize}
\tightlist
\item
  \href{https://help.nytimes3xbfgragh.onion/hc/en-us/articles/115014792127-Copyright-notice}{©~2020~The
  New York Times Company}
\end{itemize}

\begin{itemize}
\tightlist
\item
  \href{https://www.nytco.com/}{NYTCo}
\item
  \href{https://help.nytimes3xbfgragh.onion/hc/en-us/articles/115015385887-Contact-Us}{Contact
  Us}
\item
  \href{https://www.nytco.com/careers/}{Work with us}
\item
  \href{https://nytmediakit.com/}{Advertise}
\item
  \href{http://www.tbrandstudio.com/}{T Brand Studio}
\item
  \href{https://www.nytimes3xbfgragh.onion/privacy/cookie-policy\#how-do-i-manage-trackers}{Your
  Ad Choices}
\item
  \href{https://www.nytimes3xbfgragh.onion/privacy}{Privacy}
\item
  \href{https://help.nytimes3xbfgragh.onion/hc/en-us/articles/115014893428-Terms-of-service}{Terms
  of Service}
\item
  \href{https://help.nytimes3xbfgragh.onion/hc/en-us/articles/115014893968-Terms-of-sale}{Terms
  of Sale}
\item
  \href{https://spiderbites.nytimes3xbfgragh.onion}{Site Map}
\item
  \href{https://help.nytimes3xbfgragh.onion/hc/en-us}{Help}
\item
  \href{https://www.nytimes3xbfgragh.onion/subscription?campaignId=37WXW}{Subscriptions}
\end{itemize}
