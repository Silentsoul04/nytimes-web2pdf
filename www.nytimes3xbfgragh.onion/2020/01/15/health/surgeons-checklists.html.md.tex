Sections

SEARCH

\protect\hyperlink{site-content}{Skip to
content}\protect\hyperlink{site-index}{Skip to site index}

\href{https://www.nytimes3xbfgragh.onion/section/health}{Health}

\href{https://myaccount.nytimes3xbfgragh.onion/auth/login?response_type=cookie\&client_id=vi}{}

\href{https://www.nytimes3xbfgragh.onion/section/todayspaper}{Today's
Paper}

\href{/section/health}{Health}\textbar{}Where Surgeons Don't Bother With
Checklists

\url{https://nyti.ms/2QW5qOo}

\begin{itemize}
\item
\item
\item
\item
\item
\end{itemize}

Advertisement

\protect\hyperlink{after-top}{Continue reading the main story}

Supported by

\protect\hyperlink{after-sponsor}{Continue reading the main story}

Global health

\hypertarget{where-surgeons-dont-bother-with-checklists}{%
\section{Where Surgeons Don't Bother With
Checklists}\label{where-surgeons-dont-bother-with-checklists}}

In many poor countries, older surgeons resist being questioned, and
operations are more often emergencies, which leaves less time to review
checklists.

\includegraphics{https://static01.graylady3jvrrxbe.onion/images/2020/01/15/science/15GLOBALSURGERY3/merlin_167124678_fb091847-7e8a-4598-9237-f259ddf1c491-articleLarge.jpg?quality=75\&auto=webp\&disable=upscale}

\href{https://www.nytimes3xbfgragh.onion/by/donald-g-mcneil-jr}{\includegraphics{https://static01.graylady3jvrrxbe.onion/images/2018/06/13/multimedia/author-donald-g-mcneil-jr/author-donald-g-mcneil-jr-thumbLarge-v4.png}}

By
\href{https://www.nytimes3xbfgragh.onion/by/donald-g-mcneil-jr}{Donald
G. McNeil Jr.}

\begin{itemize}
\item
  Jan. 15, 2020
\item
  \begin{itemize}
  \item
  \item
  \item
  \item
  \item
  \end{itemize}
\end{itemize}

Ten years ago, checklists for surgeons were all the rage. Inspired by
the preflight routines of airline pilots, surgical checklists were shown
to prevent tragic errors, reduce infections and save lives.

Dr. Atul Gawande, a Harvard-trained surgeon, championed them in The New
Yorker and wrote a book about them, ``The Checklist Manifesto: How to
Get Things Right.''

A two-minute, 19-item checklist was eventually endorsed by the World
Health Organization, which advocated its use by every hospital in the
world. The checklist is
\href{https://unitar.org/about/news-stories/news/unitar-and-partners-launch-new-mobile-application-who-surgical-safety-checklist}{even
available as a cellphone app} from the United Nations Institute for
Training and Research.

It includes many simple steps for surgeons who are preparing to operate,
some as basic as ascertaining that the right patient is on the table and
the incision site correctly marked, and that anesthetics, oxygen and
transfusion blood are on hand.

Now that a decade has passed since the W.H.O.'s recommendation, a
nonprofit founded by Dr. Gawande has surveyed almost 1,500 hospitals in
94 countries to see how often a surgical checklist is used.

Adoption has been spotty, according to the report released on Wednesday
by the nonprofit, \href{https://www.lifebox.org/}{Lifebox}, and by
\href{https://www.ariadnelabs.org/areas-of-work/safe-surgery-checklist/}{Ariadne
Labs}, a joint venture of the Harvard T.H. Chan School of Public Health
and Brigham and Women's Hospital, both in Boston. (Both institutions
seek to improve surgery and anesthesia in poor and middle-income
countries.)

In wealthy countries, a list is used in 90 percent of surgeries, the
report found. But in poor countries, a checklist
\href{https://www.lifebox.org/checkinginonthechecklist/}{is used only
about a third of the time}.

The study blamed many factors: surgeons who resent the implication that
they may make dangerous mistakes, lax enforcement by hospital
administrators and the powerlessness of nurses in some cultures.

In poor countries, most surgeries are emergencies, such as cesarean
sections, appendectomies or trauma repair after a car crash. Rushed
surgical teams are less likely to use a checklist.

\includegraphics{https://static01.graylady3jvrrxbe.onion/images/2020/01/15/science/15GLOBALSURGERY2/15GLOBALSURGERY2-articleLarge.jpg?quality=75\&auto=webp\&disable=upscale}

In poor countries, there are often failures in support systems intended
to ensure the availability of oxygen, blood transfusions and
postoperative antibiotics, and sometimes even just clean operating
rooms.

When checklists are strictly adhered to, missing even one of those
elements can prevent the operation from beginning. Instead, some
hospitals just avoid the checklist.

There is enormous room for improvement, the study found.

For example, up to 95 percent of Africa's population does not have
access to safe and affordable surgery, and African patients are twice as
likely to die after an operation, compared with the global average, said
Dr. Bruce M. Biccard, an anesthesiologist at the University of Cape Town
and a leader of the African Surgical Outcomes Study, which produced some
of the data used in the new report.

In countries where doctors and nurses do not speak one of the six
official languages of the United Nations --- Arabic, Chinese, English,
French, Russian or Spanish --- a checklist is less likely to be used.
Also, cultural barriers have hindered its adoption.

The checklist includes requirements that doctors and nurses introduce
themselves and confirm that all have the same understanding of how the
operation will ideally proceed. But teamwork can be difficult to
introduce, both in traditional cultures based on hierarchy and obedience
and in intensely competitive environments like those in American medical
schools.

Even hospitals that faithfully used a checklist often adapted it to
local circumstances, the report found. Some translated it into Tagalog
and Amharic, for example. A West African surgical team added a
requirement that the hospital's generator be working. A Guatemalan team
added pain-control medication to the list of requirements.

A decade ago, as a young surgeon in Ethiopia, Dr. Abebe Bekele, the dean
of the University of Global Health Equity in Rwanda, tried to introduce
the checklist to the hospital where he worked in Addis Ababa, after
seeing it when he trained in Seattle.

``No one else was on board,'' Dr. Bekele, who was a consultant on the
Lifebox report, said in a telephone interview. His own team used it when
he insisted, but not when working with other surgeons, who were usually
much older than their nursing staffs.

``I had the momentum but not the wisdom on how to implement it,'' he
said.

Image

In a training course in Siuna, Nicaragua, participants reviewing a
surgical checklist.Credit...Jarred Forrester/Lifebox

Two years later, however, after studies showed the checklist reduced
deaths by 24 percent and major complications by 60 percent, he had more
success ``by bringing everyone into a single room to discuss and write
down the checklist.''

Since then, younger surgeons have become more receptive, Dr. Bekele
said, and now the checklist is taught in the
\href{https://www.pri.org/stories/2012-12-20/ethiopias-crowded-medical-schools}{rapidly
expanding medical schools} in Ethiopia.

\textbf{\emph{{[}}\href{http://on.fb.me/1paTQ1h}{\emph{Like the Science
Times page on Facebook.}}} ****** \emph{\textbar{} Sign up for the}
\textbf{\href{http://nyti.ms/1MbHaRU}{\emph{Science Times
newsletter.}}\emph{{]}}}

Outside influences can be powerful, the report noted. It singled out the
charity \href{https://www.mercyships.org/}{Mercy Ships} as instrumental
in introducing the checklist to Africa. Mercy Ships has a hospital ship
that docks for months in various port cities in Africa, performing
thousands of surgeries.

At each port of call, it also trains dozens of local medical
practitioners.

When the Africa Mercy was docked in Cotonou, Benin, from 2016 to 2017,
for example, it
\href{https://www.mercyships.org.uk/who-checklist-success-in-benin/}{sent
teams to 36 hospitals} to lead workshops in using a surgical checklist.

Before the training, only about 30 percent of surgeries performed
incorporated checklists. Afterward, nearly 90 percent did --- and
compliance was still at 86 percent during follow-up visits a year later.

Advertisement

\protect\hyperlink{after-bottom}{Continue reading the main story}

\hypertarget{site-index}{%
\subsection{Site Index}\label{site-index}}

\hypertarget{site-information-navigation}{%
\subsection{Site Information
Navigation}\label{site-information-navigation}}

\begin{itemize}
\tightlist
\item
  \href{https://help.nytimes3xbfgragh.onion/hc/en-us/articles/115014792127-Copyright-notice}{©~2020~The
  New York Times Company}
\end{itemize}

\begin{itemize}
\tightlist
\item
  \href{https://www.nytco.com/}{NYTCo}
\item
  \href{https://help.nytimes3xbfgragh.onion/hc/en-us/articles/115015385887-Contact-Us}{Contact
  Us}
\item
  \href{https://www.nytco.com/careers/}{Work with us}
\item
  \href{https://nytmediakit.com/}{Advertise}
\item
  \href{http://www.tbrandstudio.com/}{T Brand Studio}
\item
  \href{https://www.nytimes3xbfgragh.onion/privacy/cookie-policy\#how-do-i-manage-trackers}{Your
  Ad Choices}
\item
  \href{https://www.nytimes3xbfgragh.onion/privacy}{Privacy}
\item
  \href{https://help.nytimes3xbfgragh.onion/hc/en-us/articles/115014893428-Terms-of-service}{Terms
  of Service}
\item
  \href{https://help.nytimes3xbfgragh.onion/hc/en-us/articles/115014893968-Terms-of-sale}{Terms
  of Sale}
\item
  \href{https://spiderbites.nytimes3xbfgragh.onion}{Site Map}
\item
  \href{https://help.nytimes3xbfgragh.onion/hc/en-us}{Help}
\item
  \href{https://www.nytimes3xbfgragh.onion/subscription?campaignId=37WXW}{Subscriptions}
\end{itemize}
