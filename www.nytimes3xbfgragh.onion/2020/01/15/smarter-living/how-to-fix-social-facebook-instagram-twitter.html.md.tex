Sections

SEARCH

\protect\hyperlink{site-content}{Skip to
content}\protect\hyperlink{site-index}{Skip to site index}

\href{https://www.nytimes3xbfgragh.onion/section/smarter-living}{Smarter
Living}

\href{https://myaccount.nytimes3xbfgragh.onion/auth/login?response_type=cookie\&client_id=vi}{}

\href{https://www.nytimes3xbfgragh.onion/section/todayspaper}{Today's
Paper}

\href{/section/smarter-living}{Smarter Living}\textbar{}How to Turn
Depressing Social Media Into a Positive Influence

\url{https://nyti.ms/2srdM72}

\begin{itemize}
\item
\item
\item
\item
\item
\end{itemize}

Advertisement

\protect\hyperlink{after-top}{Continue reading the main story}

Supported by

\protect\hyperlink{after-sponsor}{Continue reading the main story}

\hypertarget{how-to-turn-depressing-social-media-into-a-positive-influence}{%
\section{How to Turn Depressing Social Media Into a Positive
Influence}\label{how-to-turn-depressing-social-media-into-a-positive-influence}}

Don't let Facebook, Instagram or Twitter become negative aspects of your
life. Here's how to fix them.

\includegraphics{https://static01.graylady3jvrrxbe.onion/images/2020/01/20/smarter-living/00sl_socialfix/00sl_socialfix-articleLarge.jpg?quality=75\&auto=webp\&disable=upscale}

By Geoffrey Morrison

\begin{itemize}
\item
  Published Jan. 15, 2020Updated June 2, 2020
\item
  \begin{itemize}
  \item
  \item
  \item
  \item
  \item
  \end{itemize}
\end{itemize}

A friend of mine digitally disappeared recently. Dropped all social
media. Several others mentioned they might do the same. No doubt you've
seen similar posts from people you know (and people you don't) about
needing a break from their digital lives.

On one hand, I can sympathize. The current state of the modern world is
a billion voices screaming for your attention, and it's easy to let the
most negative ones filter through and bring you down. It can be
exhausting, and if your real life is already a struggle, adding digital
gloom can be overwhelming.

It doesn't have to be that way. Through conscious and careful effort
I've forced the three major social media platforms to become positive
aspects of my life, or at worst, neutral. While some of that is a boast
about how awesome my friends are, it's really about knowing how to make
these services work for you in a beneficial way, not drag you down with
their endless streams of detrimental detritus. Sure, using apps and
browser plug-ins, and even just turning off notifications, can help, but
the steps here are all within each platform's own apps and are meant to
help a lot.

Here's how.

\hypertarget{facebook}{%
\subsection{Facebook}\label{facebook}}

Facebook is a behemoth. With
\href{https://www.statista.com/statistics/346167/facebook-global-dau/}{over
1.5 billion daily active users}, it's a safe bet that you have an
account, or at least you did at some point. What started as an innocuous
way to share photos with friends has become something else entirely.

And because Facebook has become such an integral part of modern
existence, its quirks become part of the social mores. Jokes about
``unfriending'' someone, and the political and social ramifications that
go with that, are everywhere. Blocking people is nearly as severe,
preventing others from seeing or commenting on your posts. In some
situations, those paths can be useful. The easiest way to make Facebook
livable, though, is far stealthier:
\href{https://www.facebookcorewwwi.onion/help/190078864497547}{Unfollow}.

For example, take Jonathan Burbridge, 43 years old. He recently came
close to dropping Facebook entirely. His reason was, in my opinion,
perfectly logical: ``Facebook insists on showing me every argument one
of my friends has with their racist relatives that I've never met.''
While Facebook
has\href{https://newsroom.fb.com/news/2019/05/more-personalized-experiences/}{updated
its algorithm} a lot recently, it still
\href{https://about.fb.com/news/2019/04/people-publishers-the-community/}{thinks
you'd be interested} in what your friends are talking about. If your
friends get in arguments all the time with strangers, you're going to
see that.

His solution, besides the nuclear option of dropping FB altogether, was
to unfollow everyone. If you unfollow someone, you don't see their posts
or their conversations. You're still friends and they can still see your
posts, but you won't see theirs or their comments on stranger's feeds.
Your own news feed is also \ldots{} rather empty. However, you can
choose to click to see what your friends have posted. You'll also still
see posts for any groups you're in, presuming you still follow them.
Jonathan re-followed a select few people, ones he knows who do not argue
online.

You certainly don't have to be as severe as Jonathan, but a healthy
unfollowing regimen can work wonders. Have a friend who only posts
negative things? Or one who only writes rude comments? Go to their page,
and click unfollow. Your feed will become that much quieter and that
much more pleasant.

\hypertarget{instagram}{%
\subsection{Instagram}\label{instagram}}

I love Instagram. My Instagram feed consists of 60 percent dog photos,
30 percent travel photos and 10 percent photos posted by friends. It is
always, and entirely, a positive part of my digital life. I follow one
account that is, I kid you
not,\href{https://www.instagram.com/boopmynose}{just close-ups of dog
noses}. How can you not smile when something like that shows up in your
feed?

Instagram is significantly easier to make pleasant compared to Facebook
and Twitter: Just don't follow accounts or hashtags you don't like. If
someone only posts pictures of their amazing travels (and of
that,\href{https://www.instagram.com/inveterate_adventurer/}{I am very
guilty}), and that brings you down \ldots{} don't follow them. You can
also
\href{https://help.instagram.com/196883487377501/?helpref=hc_fnav\&bc\%5B0\%5D=Instagram\%20Help\&bc\%5B1\%5D=Managing\%20Your\%20Account}{limit
who sees your posts} to cut down on people commenting, and even mute
other people's
\href{https://instagram-press.com/blog/2018/05/22/introducing-mute/}{Feeds}
and \href{https://help.instagram.com/290238234687437}{Stories} if you
want, similar to unfollowing a profile on Facebook.

The exception to this simplicity is the Search tab, which shows
seemingly random photos from random accounts. I've heard people complain
that Instagram only highlights influencers and bad beauty tips on the
Search page. You know what mine shows? Sixty percent dog photos, 30
percent travel photos and 10 percent of other photos. The search page is
absolutely \emph{not} random.
Instagram\href{https://blog.hootsuite.com/instagram-algorithm/}{knows
what you look at}. Not just what you search for but what you tap on.
Keep tapping on something and Instagram will assume that's what you want
to see. Here's how specific it is: Insta doesn't just show me ``dogs,''
it shows me mostly corgis and shiba inus, which are my two favorite
\emph{types} of dogs. And you know what, that's exactly what I keep
tapping on so of course it continues to show me that.

So, quite simply, if Instagram is showing you things you don't like,
don't click on it. Click on something else. It won't take long to adjust
and show you that. If you want to get rid of what you don't like faster,
you can even \href{https://help.instagram.com/1105548539497125}{have the
app show it less}. Or just skip the Search page altogether and only look
at the accounts you follow, assuming you've culled that list down to
just the ones
that\href{https://www.nytimes3xbfgragh.onion/2019/01/02/style/marie-kondo-netflix.html}{spark
joy}.

\hypertarget{twitter}{%
\subsection{Twitter}\label{twitter}}

While Instagram is easy to make positive, and Facebook generally so,
Twitter is not. It's far more difficult to avoid what you don't want to
see. You might follow a writer you like who posts about elves 99 percent
of the time, then all of a sudden they post about orcs. \emph{Day
ruined}.

But on the other hand, there are a lot of funny accounts. There's one
account that just posts \href{https://twitter.com/WholesomeMeme}{random,
delightfully wholesome things} and it's just adorable. So it is possible
to make positive, it just takes more effort than the other two
platforms.

To start, the unfollow rule from Facebook and Insta applies here as
well. Muting and blocking accounts is a good place to start. This gets
rid of the most terribly toxic tweeters. Semi-usefully you can also mute
words. This is time consuming, but it lets you slightly filter whatever
regularly brings you down. This is a one-at-a-time manual process, but
it's something. Annoyingly, it's not perfect. You can certainly mute the
word ``orc'' but if someone tweets ``Look at this article'' and the
article title is ``Orcs are better than elves'' you're still going to
see it. If there's a specific account that always retweets content you
don't like, but you like their original tweets, you can also
\href{https://help.twitter.com/en/using-twitter/retweet-faqs}{turn off
retweets on a per-account basis}.

You can also
\href{https://help.twitter.com/en/safety-and-security/control-your-twitter-experience}{enable
the Quality filter and other advanced filters} to cut down on replies
from accounts with only a few followers (i.e. likely spam or bots), new
accounts and so on. There's no option to
mute\href{https://knowyourmeme.com/memes/anime-profile-pictures}{men
with anime profile photos} but there should be.

To make Twitter as pleasant as possible, unfollow with extreme
prejudice. Only follow accounts that only post content you're sure to
enjoy. It's also worth considering many of the best accounts also have
Instagram and Facebook.

\hypertarget{the-physical-world}{%
\subsection{The physical world}\label{the-physical-world}}

The tips above are how to make these services show you what you want to
see. But what do you show the world? This is certainly a very different
question. Amy Liu, 22, felt she was putting on a fake persona in her
digital life, and that was one of the reasons she dropped social media.
``I felt I was portraying this fake version of me that wasn't real,
especially on Instagram where your life essentially can't have flaws and
missteps, because it's all reduced to a pretty picture.''

I can certainly understand this. I travel for half the year, and I post
a lot of stories on my Instagram about where I am and what I'm doing.
It's public, but it's also for my friends. I rarely, even on my private
Facebook account that's just for friends and family, post about negative
experiences. When I've done this in the past, I've gotten comments along
the lines of ``you're traveling, everything is perfect, stop
complaining.'' That's unfair, and untrue, yet when what you choose to
present to the world seems perfect, people think everything \emph{is}
perfect. I've stopped making posts like that, which is a challenge when
things aren't great, but I've learned that social media isn't usually
the place to get support.

Here's the thing: You don't have to post anything you don't want to.
\emph{You don't have to post at all.} There is a pressure to share, of
course, but what you share and how much is entirely up to you. It's
unlikely anyone will notice if you don't post in a week or more,
especially if you remain active and comment on other's posts (if you
want).

So what happened with my digitally disappeared friend? Well, she's still
gone. For selfish reasons that makes me sad, as I don't get to see what
she's up to with any regularity. But we still talk every few weeks to
catch up. For her, though, it has been an entirely positive step. She
doesn't know if she'll be back, but knows that's an option if she wants.

That fact might make the decision easier for you, too: You could always
reactivate your accounts. Or even just start a new one, with a very
select group of friends. Perhaps after some time off, you'll have a
better idea of what you miss and what you want out of these services,
and fine tune them to only show you that. It's a tough world out there.
I recommend puppies. Lots of puppies.

Advertisement

\protect\hyperlink{after-bottom}{Continue reading the main story}

\hypertarget{site-index}{%
\subsection{Site Index}\label{site-index}}

\hypertarget{site-information-navigation}{%
\subsection{Site Information
Navigation}\label{site-information-navigation}}

\begin{itemize}
\tightlist
\item
  \href{https://help.nytimes3xbfgragh.onion/hc/en-us/articles/115014792127-Copyright-notice}{©~2020~The
  New York Times Company}
\end{itemize}

\begin{itemize}
\tightlist
\item
  \href{https://www.nytco.com/}{NYTCo}
\item
  \href{https://help.nytimes3xbfgragh.onion/hc/en-us/articles/115015385887-Contact-Us}{Contact
  Us}
\item
  \href{https://www.nytco.com/careers/}{Work with us}
\item
  \href{https://nytmediakit.com/}{Advertise}
\item
  \href{http://www.tbrandstudio.com/}{T Brand Studio}
\item
  \href{https://www.nytimes3xbfgragh.onion/privacy/cookie-policy\#how-do-i-manage-trackers}{Your
  Ad Choices}
\item
  \href{https://www.nytimes3xbfgragh.onion/privacy}{Privacy}
\item
  \href{https://help.nytimes3xbfgragh.onion/hc/en-us/articles/115014893428-Terms-of-service}{Terms
  of Service}
\item
  \href{https://help.nytimes3xbfgragh.onion/hc/en-us/articles/115014893968-Terms-of-sale}{Terms
  of Sale}
\item
  \href{https://spiderbites.nytimes3xbfgragh.onion}{Site Map}
\item
  \href{https://help.nytimes3xbfgragh.onion/hc/en-us}{Help}
\item
  \href{https://www.nytimes3xbfgragh.onion/subscription?campaignId=37WXW}{Subscriptions}
\end{itemize}
