Sections

SEARCH

\protect\hyperlink{site-content}{Skip to
content}\protect\hyperlink{site-index}{Skip to site index}

\href{https://www.nytimes3xbfgragh.onion/section/nyregion}{New York}

\href{https://myaccount.nytimes3xbfgragh.onion/auth/login?response_type=cookie\&client_id=vi}{}

\href{https://www.nytimes3xbfgragh.onion/section/todayspaper}{Today's
Paper}

\href{/section/nyregion}{New York}\textbar{}Mayor de Blasio Offers
`Minimalist' \$95.3 Billion Budget, Warning of Cuts

\url{https://nyti.ms/38f60wu}

\begin{itemize}
\item
\item
\item
\item
\item
\end{itemize}

Advertisement

\protect\hyperlink{after-top}{Continue reading the main story}

Supported by

\protect\hyperlink{after-sponsor}{Continue reading the main story}

\hypertarget{mayor-de-blasio-offers-minimalist-953-billion-budget-warning-of-cuts}{%
\section{Mayor de Blasio Offers `Minimalist' \$95.3 Billion Budget,
Warning of
Cuts}\label{mayor-de-blasio-offers-minimalist-953-billion-budget-warning-of-cuts}}

The mayor proposed a city budget with the smallest percentage increase
in his six years in office, fearing a state deficit.

\includegraphics{https://static01.graylady3jvrrxbe.onion/images/2020/01/17/nyregion/17nycbudget-print/merlin_167243481_e618e3f3-8f07-4130-a26f-cbacdee80bb0-articleLarge.jpg?quality=75\&auto=webp\&disable=upscale}

\href{https://www.nytimes3xbfgragh.onion/by/emma-g-fitzsimmons}{\includegraphics{https://static01.graylady3jvrrxbe.onion/images/2018/07/18/multimedia/author-emma-g-fitzsimmons/author-emma-g-fitzsimmons-thumbLarge.png}}\href{https://www.nytimes3xbfgragh.onion/by/jeffery-c-mays}{\includegraphics{https://static01.graylady3jvrrxbe.onion/images/2018/07/18/multimedia/author-jeffery-c-mays/author-jeffery-c-mays-thumbLarge.png}}

By \href{https://www.nytimes3xbfgragh.onion/by/emma-g-fitzsimmons}{Emma
G. Fitzsimmons} and
\href{https://www.nytimes3xbfgragh.onion/by/jeffery-c-mays}{Jeffery C.
Mays}

\begin{itemize}
\item
  Jan. 16, 2020
\item
  \begin{itemize}
  \item
  \item
  \item
  \item
  \item
  \end{itemize}
\end{itemize}

The divide between Mayor Bill de Blasio and Gov. Andrew M. Cuomo has
colored
\href{https://www.nytimes3xbfgragh.onion/2018/04/22/nyregion/cuomo-deblasio-feud-nyc.html}{many
city vs. state squabbles}, seeping into disputes over subway repairs,
school funding and solving the homelessness crisis.

Now it appears to be threatening the city's budget.

On Thursday, Mr. de Blasio released a \$95.3 billion budget proposal
that called for a 2.7 percent increase --- the smallest percentage
increase in his six years in office.

Even though city revenues are expected to be strong, the mayor said that
the state's projected
\href{https://www.nytimes3xbfgragh.onion/2019/12/20/nyregion/democrats-progressive-ny-budget.html}{\$6
billion deficit} could loom large over New York City, and that the state
could cut its funding to the city, or ask the city to increase its share
of payments for things like the subway.

``We've never seen this kind of state deficit, and we've never seen this
kind of threat to our Medicaid recipients,'' Mr. de Blasio said at a
news conference, pledging to protect New Yorkers from any state cuts.

Mr. de Blasio, a Democrat in his second term, has
\href{https://www.nytimes3xbfgragh.onion/2018/02/01/nyregion/nyc-de-blasio-budget-spending.html}{overseen
a period of booming growth in tax revenue} and economic prosperity,
allowing him to spend more than past mayors: Mr. de Blasio, for example,
has expanded the city's work force past 350,000 people.

But this year, officials in New York City and other municipalities are
worried that state leaders might try to balance the budget on the backs
of local governments. The heart of the state budget deficit stems from
the state's overspending on Medicaid.

Dani Lever, the governor's communications director, pushed back against
the mayor, teasing him for his
\href{https://www.nytimes3xbfgragh.onion/2020/01/15/nyregion/bagel-ny-toasted-de-blasio.html}{preferred
bagel}: a toasted whole wheat with extra cream cheese, a pick that was
widely derided a day earlier.

``We have heard of smoke and mirrors and political straw men,'' Ms.
Lever said in a statement. ``How the mayor can claim he is reacting to
cuts from the state, before the state has even proposed a budget, is
spreading the political cream cheese too thick even for a toasted
bagel.''

Other potential sinkholes for the city exist.

Mr. de Blasio faces pressure to address the crisis that has devastated
the city's taxi industry. A panel appointed by Mr. de Blasio and the
City Council is expected to propose a
\href{https://www.nytimes3xbfgragh.onion/2020/01/15/nyregion/nyc-taxi-medallion-bailout.html}{bailout
of thousands of taxi medallion owners} that could cost as much as \$500
million.

Mr. de Blasio said he liked the plan because it relies largely on
private financing as part of a public-private partnership.

``It's the best idea I've heard so far,'' the mayor told reporters,
though he said he was reluctant to commit significant direct funding
from the city.

On Thursday, Mr. de Blasio announced his nominee to be the next leader
of the city agency that oversees the taxi industry: Aloysee Heredia
Jarmoszuk, who currently serves as chief of staff to the deputy mayor
for operations. The City Council failed to approve the previous official
he nominated for the job.

Mr. de Blasio could be pushed to spend elsewhere. Leaders at the
Metropolitan Transportation Authority, which oversees the city's subway
and buses, have also called on the city to pay \$3 billion to fix the
transit system in the coming years. Funding for a taxi bailout or new
subway spending was not included in the mayor's preliminary budget.

The
\href{https://www.nytimes3xbfgragh.onion/2019/06/15/nyregion/nyc-budget-funding.html}{city's
last budget was \$92.8 billion} and included spending increases on
things like providing social workers at city schools and a program to
boost participation in the federal census. The previous budget included
funding for discounted subway and bus fares for some of the poorest New
Yorkers.

City officials attributed \$1.6 billion of the growth in the preliminary
budget to the settling of outstanding labor union contracts and said it
would cost \$175 million to implement changes related to bail and
discovery reform.

Scott M. Stringer, the city comptroller and a mayoral candidate in 2021,
said that Mr. de Blasio must protect the city from leaders in Albany who
frequently target New York City's budget when the state budget is in
peril.

``You can't play checkers with Albany, you have to play chess,'' said
Mr. Stringer, who characterized the mayor's plan as a ``minimalist
budget.'' ``There has to be a strategy to go to Albany and thinking
about how we keep them at bay.''

Mr. de Blasio admitted that his budget plan was missing the ``juicy''
parts that he has typically highlighted in previous years, especially
initiatives to help low-income New Yorkers, but he said he wanted to
exercise caution until he sees how things play out with the state.

One area where he wants to add funding: \$98 million in capital funds to
improve street safety on Fourth Avenue in Brooklyn. The upgrades would
be part of his Vision Zero safety plan to eliminate traffic deaths --- a
program that some worry is faltering after traffic deaths rose in the
city last year.

Maria Doulis, vice president of the Citizens Budget Commission, warned
that it was still early in the budget process and that the City Council
must weigh in.

``What about the City Council's priorities?'' she said. ``There are a
lot of factors that may provide pressure on the city going forward.''

The Council speaker, Corey Johnson, suggested that the mayor needed to
prioritize some of the city's most intractable problems such as
affordable housing and homelessness.

``I can't control what the mayor does,'' said Mr. Johnson, also a
candidate for mayor in 2021. ``But on the Council, we are going to
continue to lead on the big ideas.''

Shortly before announcing his budget, Mr. de Blasio was joined at City
Hall by Richard Carranza, the city schools chancellor, to celebrate a
bright spot for his administration: The four-year high school graduation
rate rose to 77.3 percent last year, up from 68.4 percent in 2014.

Advertisement

\protect\hyperlink{after-bottom}{Continue reading the main story}

\hypertarget{site-index}{%
\subsection{Site Index}\label{site-index}}

\hypertarget{site-information-navigation}{%
\subsection{Site Information
Navigation}\label{site-information-navigation}}

\begin{itemize}
\tightlist
\item
  \href{https://help.nytimes3xbfgragh.onion/hc/en-us/articles/115014792127-Copyright-notice}{©~2020~The
  New York Times Company}
\end{itemize}

\begin{itemize}
\tightlist
\item
  \href{https://www.nytco.com/}{NYTCo}
\item
  \href{https://help.nytimes3xbfgragh.onion/hc/en-us/articles/115015385887-Contact-Us}{Contact
  Us}
\item
  \href{https://www.nytco.com/careers/}{Work with us}
\item
  \href{https://nytmediakit.com/}{Advertise}
\item
  \href{http://www.tbrandstudio.com/}{T Brand Studio}
\item
  \href{https://www.nytimes3xbfgragh.onion/privacy/cookie-policy\#how-do-i-manage-trackers}{Your
  Ad Choices}
\item
  \href{https://www.nytimes3xbfgragh.onion/privacy}{Privacy}
\item
  \href{https://help.nytimes3xbfgragh.onion/hc/en-us/articles/115014893428-Terms-of-service}{Terms
  of Service}
\item
  \href{https://help.nytimes3xbfgragh.onion/hc/en-us/articles/115014893968-Terms-of-sale}{Terms
  of Sale}
\item
  \href{https://spiderbites.nytimes3xbfgragh.onion}{Site Map}
\item
  \href{https://help.nytimes3xbfgragh.onion/hc/en-us}{Help}
\item
  \href{https://www.nytimes3xbfgragh.onion/subscription?campaignId=37WXW}{Subscriptions}
\end{itemize}
