Sections

SEARCH

\protect\hyperlink{site-content}{Skip to
content}\protect\hyperlink{site-index}{Skip to site index}

\href{https://myaccount.nytimes3xbfgragh.onion/auth/login?response_type=cookie\&client_id=vi}{}

\href{https://www.nytimes3xbfgragh.onion/section/todayspaper}{Today's
Paper}

Whipping Up Chocolate Mousse Is Stressful. A Blender Makes It Easy.

\url{https://nyti.ms/2t7YOTX}

\begin{itemize}
\item
\item
\item
\item
\item
\end{itemize}

Advertisement

\protect\hyperlink{after-top}{Continue reading the main story}

Supported by

\protect\hyperlink{after-sponsor}{Continue reading the main story}

\href{/column/magazine-eat}{Eat}

\hypertarget{whipping-up-chocolate-mousse-is-stressful-a-blender-makes-it-easy}{%
\section{Whipping Up Chocolate Mousse Is Stressful. A Blender Makes It
Easy.}\label{whipping-up-chocolate-mousse-is-stressful-a-blender-makes-it-easy}}

\includegraphics{https://static01.graylady3jvrrxbe.onion/images/2020/02/02/magazine/02mag-eat/02mag-eat-articleLarge.jpg?quality=75\&auto=webp\&disable=upscale}

By \href{https://www.nytimes3xbfgragh.onion/by/tejal-rao}{Tejal Rao}

\begin{itemize}
\item
  Jan. 29, 2020
\item
  \begin{itemize}
  \item
  \item
  \item
  \item
  \item
  \end{itemize}
\end{itemize}

I was impressed by Danou, a family friend who lived next door to me in
rural France, not just because she was significantly older (retired) and
smoked Gauloises (one pack a day) and had two dogs (two!), but because
her go-to dessert was chocolate mousse. She had a life outside the
kitchen, but so much of the time I spent with her was on the weekends,
hanging around her house as she got ready for Sunday lunch, helping her
make that mousse. It was a process, the same each time, and my
responsibilities were fairly small. I cracked and separated the eggs and
checked the bowls for egg shell. I broke the chocolate along segment
lines and checked the bowl for foil bits. I set a pot of water on to
boil, to melt the chocolate. Then I climbed onto the counter to get the
Bowl.

Danou didn't mess around with martini glasses or tiny coupes, which is
another part of what was so impressive about her. She whipped and folded
and scraped the dark, airy batter into a giant ceramic bowl and shoved
it into the fridge, which was already jammed full. Later, after the
batter was set, after dinner, she dared me to turn the bowl upside down
over my head --- it was so heavy, it hurt my wrists. She served mousse
family-style, the bowl passed around with an enormous metal spoon,
everyone helping themselves. I know it now as a restaurant dessert --- a
chilled ramekin pulled from the fridge --- but this is the spirit of
chocolate mousse. A big bowl going around, again and again, the portions
never uniform or predetermined.

I don't know anyone who makes chocolate mousse at home now, let alone
once a week. Maybe because the traditional method that Danou employed,
for which you melt the chocolate over gently simmering water, then beat
hot cream into the melted chocolate, and then split your eggs into
whites and yolks, and then mix the yolks, at room temperature, into the
chocolate, and then whip the whites into a stiff meringue --- but not
too stiff! --- and then fold the meringue into the chocolate until there
are absolutely no streaks, but also not for too long, because the mousse
starts to deflate, is so stressful. And because an imperfect mousse will
announce itself immediately with tiny lumps of some kind or an
unappealing density.

But chocolate mousse is a home cook's dessert --- fast and unfussy. Not
Danou's version, but blender chocolate mousse, which is infinitely more
reliable, and which would be an affront to traditional chocolate mousse,
if it didn't work so well. The recipe I'm now devoted to came to me via
the pastry chef Natasha Pickowicz, who runs the pastry kitchens at Café
Altro Paradiso and Flora Bar, in Manhattan. It has somewhat murky
origins in a Junior League cookbook from the 1980s, and like all great
shortcut recipes, it was shared and shared. The mousse passed through
kitchens in Florida and New Jersey, getting tweaked along the way, most
recently traveling with Monica Stolbach, a pastry cook, to Pickowicz's
kitchen in New York, and ending up on more than one dessert menu because
it's so delicious and straightforward, especially for cooks who don't
typically make desserts.

Pickowicz, who does typically make desserts, ups the egg yolks and adds
butter to make the mousse richer and includes some instant coffee for
the depth that comes only with bitterness, but as she put it, ``The base
recipe is pretty untouchable.'' So I don't touch it.

Crack eggs into the blender, add chopped dark chocolate, then pour hot
sugar syrup in with the motor running. This cooks the eggs and melts the
chocolate at the same time --- if you don't have a fancy blender, just
keep yours going, and it will, eventually, do the job. Fold this cooled
mixture into softly whipped cream, and you're done. There's wiggle room
with the additions, but I stick with a bit of very strong coffee, a
splash of brandy or rum --- the original recipe called for Kahlúa ---
and some vanilla essence. However you tweak it, the mousse sets within a
couple of hours, airy and firm, smooth and creamy, with no chance of
bits and no pocked surface of fallen, broken bubbles. The blender does
all the work.

If you're feeling flashy, you could put the mousse straight from the
blender into ramekins or glasses or some other cute little bowls and
then spoon over a little extra whipped cream and grated chocolate when
it comes time to serve. But I like it best in a massive bowl, passed
around just as it is --- Danou-style --- the dark, glossy surface daring
you to tip it upside down.

\textbf{Recipe:}
\href{https://cooking.nytimes3xbfgragh.onion/recipes/1020831-blender-chocolate-mousse}{Blender
Chocolate Mousse}

Advertisement

\protect\hyperlink{after-bottom}{Continue reading the main story}

\hypertarget{site-index}{%
\subsection{Site Index}\label{site-index}}

\hypertarget{site-information-navigation}{%
\subsection{Site Information
Navigation}\label{site-information-navigation}}

\begin{itemize}
\tightlist
\item
  \href{https://help.nytimes3xbfgragh.onion/hc/en-us/articles/115014792127-Copyright-notice}{©~2020~The
  New York Times Company}
\end{itemize}

\begin{itemize}
\tightlist
\item
  \href{https://www.nytco.com/}{NYTCo}
\item
  \href{https://help.nytimes3xbfgragh.onion/hc/en-us/articles/115015385887-Contact-Us}{Contact
  Us}
\item
  \href{https://www.nytco.com/careers/}{Work with us}
\item
  \href{https://nytmediakit.com/}{Advertise}
\item
  \href{http://www.tbrandstudio.com/}{T Brand Studio}
\item
  \href{https://www.nytimes3xbfgragh.onion/privacy/cookie-policy\#how-do-i-manage-trackers}{Your
  Ad Choices}
\item
  \href{https://www.nytimes3xbfgragh.onion/privacy}{Privacy}
\item
  \href{https://help.nytimes3xbfgragh.onion/hc/en-us/articles/115014893428-Terms-of-service}{Terms
  of Service}
\item
  \href{https://help.nytimes3xbfgragh.onion/hc/en-us/articles/115014893968-Terms-of-sale}{Terms
  of Sale}
\item
  \href{https://spiderbites.nytimes3xbfgragh.onion}{Site Map}
\item
  \href{https://help.nytimes3xbfgragh.onion/hc/en-us}{Help}
\item
  \href{https://www.nytimes3xbfgragh.onion/subscription?campaignId=37WXW}{Subscriptions}
\end{itemize}
