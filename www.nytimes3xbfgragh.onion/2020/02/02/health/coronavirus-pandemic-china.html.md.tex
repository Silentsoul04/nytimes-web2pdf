Sections

SEARCH

\protect\hyperlink{site-content}{Skip to
content}\protect\hyperlink{site-index}{Skip to site index}

\href{https://www.nytimes3xbfgragh.onion/section/health}{Health}

\href{https://myaccount.nytimes3xbfgragh.onion/auth/login?response_type=cookie\&client_id=vi}{}

\href{https://www.nytimes3xbfgragh.onion/section/todayspaper}{Today's
Paper}

\href{/section/health}{Health}\textbar{}Wuhan Coronavirus Looks
Increasingly Like a Pandemic, Experts Say

\url{https://nyti.ms/2GOL83q}

\begin{itemize}
\item
\item
\item
\item
\item
\item
\end{itemize}

\hypertarget{the-coronavirus-outbreak}{%
\subsubsection{\texorpdfstring{\href{https://www.nytimes3xbfgragh.onion/news-event/coronavirus?name=styln-coronavirus-national\&region=TOP_BANNER\&block=storyline_menu_recirc\&action=click\&pgtype=Article\&impression_id=4eeb4450-f29d-11ea-b48a-7b43fab1412c\&variant=undefined}{The
Coronavirus
Outbreak}}{The Coronavirus Outbreak}}\label{the-coronavirus-outbreak}}

\begin{itemize}
\tightlist
\item
  live\href{https://www.nytimes3xbfgragh.onion/2020/09/09/world/covid-19-coronavirus.html?name=styln-coronavirus-national\&region=TOP_BANNER\&block=storyline_menu_recirc\&action=click\&pgtype=Article\&impression_id=4eeb4451-f29d-11ea-b48a-7b43fab1412c\&variant=undefined}{Latest
  Updates}
\item
  \href{https://www.nytimes3xbfgragh.onion/interactive/2020/us/coronavirus-us-cases.html?name=styln-coronavirus-national\&region=TOP_BANNER\&block=storyline_menu_recirc\&action=click\&pgtype=Article\&impression_id=4eeb6b60-f29d-11ea-b48a-7b43fab1412c\&variant=undefined}{Maps
  and Cases}
\item
  \href{https://www.nytimes3xbfgragh.onion/interactive/2020/science/coronavirus-vaccine-tracker.html?name=styln-coronavirus-national\&region=TOP_BANNER\&block=storyline_menu_recirc\&action=click\&pgtype=Article\&impression_id=4eeb6b61-f29d-11ea-b48a-7b43fab1412c\&variant=undefined}{Vaccine
  Tracker}
\item
  \href{https://www.nytimes3xbfgragh.onion/2020/09/02/your-money/eviction-moratorium-covid.html?name=styln-coronavirus-national\&region=TOP_BANNER\&block=storyline_menu_recirc\&action=click\&pgtype=Article\&impression_id=4eeb6b62-f29d-11ea-b48a-7b43fab1412c\&variant=undefined}{Eviction
  Moratorium}
\item
  \href{https://www.nytimes3xbfgragh.onion/interactive/2020/09/02/magazine/food-insecurity-hunger-us.html?name=styln-coronavirus-national\&region=TOP_BANNER\&block=storyline_menu_recirc\&action=click\&pgtype=Article\&impression_id=4eeb6b63-f29d-11ea-b48a-7b43fab1412c\&variant=undefined}{American
  Hunger}
\end{itemize}

Advertisement

\protect\hyperlink{after-top}{Continue reading the main story}

Supported by

\protect\hyperlink{after-sponsor}{Continue reading the main story}

Global health

\hypertarget{wuhan-coronavirus-looks-increasingly-like-a-pandemic-experts-say}{%
\section{Wuhan Coronavirus Looks Increasingly Like a Pandemic, Experts
Say}\label{wuhan-coronavirus-looks-increasingly-like-a-pandemic-experts-say}}

Rapidly rising caseloads alarm researchers, who fear the virus may make
its way across the globe. But scientists cannot yet predict how many
deaths may result.

\includegraphics{https://static01.graylady3jvrrxbe.onion/images/2020/02/02/world/02virus-pandemic-sub/02virus-pandemic-sub-articleLarge.jpg?quality=75\&auto=webp\&disable=upscale}

\href{https://www.nytimes3xbfgragh.onion/by/donald-g-mcneil-jr}{\includegraphics{https://static01.graylady3jvrrxbe.onion/images/2018/06/13/multimedia/author-donald-g-mcneil-jr/author-donald-g-mcneil-jr-thumbLarge-v4.png}}

By
\href{https://www.nytimes3xbfgragh.onion/by/donald-g-mcneil-jr}{Donald
G. McNeil Jr.}

\begin{itemize}
\item
  Published Feb. 2, 2020Updated Feb. 20, 2020
\item
  \begin{itemize}
  \item
  \item
  \item
  \item
  \item
  \item
  \end{itemize}
\end{itemize}

\href{https://cn.nytimes3xbfgragh.onion/china/20200203/coronavirus-pandemic-china/}{阅读简体中文版}\href{https://cn.nytimes3xbfgragh.onion/china/20200203/coronavirus-pandemic-china/zh-hant/}{閱讀繁體中文版}

The
\href{https://www.nytimes3xbfgragh.onion/2020/02/20/world/asia/japan-coronavirus-clusters.html}{Wuhan
coronavirus} spreading from China is now likely to become a pandemic
that circles the globe, according to many of the world's leading
infectious disease experts.

The prospect is daunting. A pandemic --- an ongoing epidemic on two or
more continents --- may well have global consequences, despite the
extraordinary travel restrictions and quarantines now imposed by
\href{https://www.nytimes3xbfgragh.onion/2020/02/20/world/asia/japan-coronavirus-clusters.html}{China}
and other countries, including the United States.

Scientists do not yet know how lethal the new coronavirus is, however,
so there is uncertainty about how much damage a pandemic might cause.
But there is growing consensus that the pathogen is readily transmitted
between humans.

The Wuhan coronavirus is spreading more like influenza, which is highly
transmissible, than like its slow-moving viral cousins, SARS and MERS,
scientists have found.

``It's very, very transmissible, and it almost certainly is going to be
a pandemic,'' said Dr. Anthony S. Fauci, director of the National
Institute of Allergy and Infectious Disease.

``But will it be catastrophic? I don't know.''

In the last three weeks, the number of lab-confirmed cases has soared
from about 50 in China to more than
\href{https://www.nytimes3xbfgragh.onion/2020/02/02/world/asia/china-coronavirus.html}{17,000
in at least 23 countries; there have been more than 360 deaths}.

But various epidemiological models estimate that the real number of
cases is 100,000 or even more. While that expansion is not as rapid as
that of flu or measles, it is an enormous leap beyond what virologists
saw when SARS and MERS emerged.

When SARS was vanquished in July 2003 after spreading for nine months,
only 8,098 cases had been confirmed. MERS has been circulating since
2012, but there have been only about 2,500 known cases.

The biggest uncertainty now, experts said, is how many people around the
world will die. SARS killed about 10 percent of those who got it, and
MERS now kills about one of three.

The \href{https://wwwnc.cdc.gov/eid/article/12/1/05-0979_article}{1918
``Spanish flu''} killed only about 2.5 percent of its victims --- but
because it infected so many people and medical care was much cruder
then, an estimated 50 million died, perhaps even more.

By contrast, the highly transmissible H1N1 ``swine flu'' pandemic of
2009
killed\href{https://www.thelancet.com/journals/laninf/article/PIIS1473-3099(12)70121-4/fulltext}{about
285,000}, fewer than seasonal flu normally does, and had a relatively
low fatality rate, estimated at .02 percent.

The mortality rate for known cases of the Wuhan coronavirus has been
running about 2 percent, although that is likely to drop as more tests
are done and more mild cases are found.

\includegraphics{https://static01.graylady3jvrrxbe.onion/images/2020/02/01/science/31virus-pandemic/31virus-pandemic-articleLarge.jpg?quality=75\&auto=webp\&disable=upscale}

It is ``increasingly unlikely that the virus can be contained,'' said
Dr. Thomas R. Frieden, a former director of the Centers for Disease
Control and Prevention who now runs Resolve to Save Lives, a nonprofit
devoted to fighting epidemics.

\hypertarget{latest-updates-the-coronavirus-outbreak}{%
\section{\texorpdfstring{\href{https://www.nytimes3xbfgragh.onion/2020/09/09/world/covid-19-coronavirus.html?action=click\&pgtype=Article\&state=default\&region=MAIN_CONTENT_1\&context=storylines_live_updates}{Latest
Updates: The Coronavirus
Outbreak}}{Latest Updates: The Coronavirus Outbreak}}\label{latest-updates-the-coronavirus-outbreak}}

Updated 2020-09-09T12:55:42.422Z

\begin{itemize}
\tightlist
\item
  \href{https://www.nytimes3xbfgragh.onion/2020/09/09/world/covid-19-coronavirus.html?action=click\&pgtype=Article\&state=default\&region=MAIN_CONTENT_1\&context=storylines_live_updates\#link-70cea8bb}{As
  drugmakers pledge to thoroughly vet a vaccine, one company pauses its
  trials for a safety review.}
\item
  \href{https://www.nytimes3xbfgragh.onion/2020/09/09/world/covid-19-coronavirus.html?action=click\&pgtype=Article\&state=default\&region=MAIN_CONTENT_1\&context=storylines_live_updates\#link-780eaa2f}{Britain
  is expected to ban gatherings of more than six people.}
\item
  \href{https://www.nytimes3xbfgragh.onion/2020/09/09/world/covid-19-coronavirus.html?action=click\&pgtype=Article\&state=default\&region=MAIN_CONTENT_1\&context=storylines_live_updates\#link-11cec4c0}{Quarantine
  breakdowns at colleges in the U.S. are leaving some at risk.}
\end{itemize}

\href{https://www.nytimes3xbfgragh.onion/2020/09/09/world/covid-19-coronavirus.html?action=click\&pgtype=Article\&state=default\&region=MAIN_CONTENT_1\&context=storylines_live_updates}{See
more updates}

More live coverage:
\href{https://www.nytimes3xbfgragh.onion/live/2020/09/09/business/stock-market-today-coronavirus?action=click\&pgtype=Article\&state=default\&region=MAIN_CONTENT_1\&context=storylines_live_updates}{Markets}

``It is therefore likely that it will spread, as flu and other organisms
do, but we still don't know how far, wide or deadly it will be.''

In the early days of the 2009 flu pandemic, ``they were talking about
Armageddon in Mexico,'' Dr. Fauci said. (That virus first emerged in
pig-farming areas in Mexico's Veracruz State.) ``But it turned out to
not be that severe.''

An accurate estimate of the virus's lethality will not be possible until
certain kinds of studies can be done: blood tests to see how many people
have antibodies, household studies to learn how often it infects family
members, and genetic sequencing to determine whether some strains are
more dangerous than others.

Closing borders to highly infectious pathogens never succeeds
completely, experts said, because all frontiers are somewhat porous.
Nonetheless, closings and rigorous screening may slow the spread, which
will buy time for the development of drug treatments and vaccines.

Other important unknowns include who is most at risk, whether coughing
or contaminated surfaces are more likely to transmit the virus, how fast
the virus can mutate and whether it will fade out when the weather
warms.

\href{https://www.nytimes3xbfgragh.onion/interactive/2020/world/coronavirus-maps.html}{}

\includegraphics{https://static01.graylady3jvrrxbe.onion/images/2020/09/09/us/china-wuhan-coronavirus-maps-promo-1599625589362/china-wuhan-coronavirus-maps-promo-1599625589362-articleLarge.png}

\hypertarget{coronavirus-map-tracking-the-global-outbreak}{%
\subsection{Coronavirus Map: Tracking the Global
Outbreak}\label{coronavirus-map-tracking-the-global-outbreak}}

The virus has infected more than 27,593,700 people and has been detected
in nearly every country.

The effects of a pandemic would probably be harsher in some countries
than in others. While the United States and other wealthy countries may
be able to detect and quarantine the first carriers, countries with
fragile health care systems will not. The virus has already reached
\href{https://www.phnompenhpost.com/national/first-case-coronavirus-reported-kingdom}{Cambodia},
\href{https://qz.com/india/1793841/indias-first-confirmed-case-of-coronavirus-reported-in-kerala/}{India},
\href{https://www.thestar.com.my/news/nation/2020/01/30/coronavirus-eighth-positive-case-in-m039sia-confirmed-thursday-jan-30}{Malaysia},
\href{https://www.reuters.com/article/us-china-health-nepal/nepal-confirms-first-case-of-new-coronavirus-idUSKBN1ZN1S2}{Nepal},
\href{https://www.reuters.com/article/us-china-health-philippines/philippines-confirms-first-case-of-new-coronavirus-health-minister-idUSKBN1ZT0S0}{the
Philippines} and rural
\href{https://www.themoscowtimes.com/2020/01/31/russia-reports-first-coronavirus-cases-a69123}{Russia}.

``This looks far more like H1N1's spread than SARS, and I am
increasingly alarmed,'' said Dr. Peter Piot, director of the London
School of Hygiene and Tropical Medicine. ``Even 1 percent mortality
would mean 10,000 deaths in each million people.''

Other experts were more cautious.

Dr. Michael Ryan, head of emergency responses for the World Health
Organization, said in
\href{https://www.statnews.com/2020/02/01/top-who-official-says-not-too-late-to-stop-coronavirus-outbreak/}{an
interview with STAT News} on Saturday that there was ``evidence to
suggest this virus can still be contained'' and that the world needed to
``keep trying.''

Dr. W. Ian Lipkin, a virus-hunter at the Columbia University Mailman
School of Public Health who is in China advising its Center for Disease
Control and Prevention, said that although the virus is clearly being
transmitted through casual contact, labs are still behind in processing
samples.

Image

In Hyderabad, India, doctors left an isolation ward for people kept
under observation after returning from China.Credit...Mahesh
Kumar/Associated Press

But life in China has radically changed in the last two weeks. Streets
are deserted, public events are canceled, and citizens are wearing masks
and washing their hands, Dr. Lipkin said. All of that may have slowed
down what lab testing indicated was exponential growth in the infection.

It's unclear exactly how accurate tests done in overwhelmed Chinese
laboratories are. On the one hand, Chinese state media have reported
test kit shortages and processing bottlenecks, which could produce an
undercount.

But Dr. Lipkin said he knew of one lab running 5,000 samples a day,
which might produce some false-positive results, inflating the count.
``You can't possibly do quality control at that rate,'' he said.

Anecdotal reports from China, and
\href{https://www.scientificamerican.com/article/study-reports-first-case-of-coronavirus-spread-by-asymptomatic-person/}{one
published study from Germany}, indicate that some people infected with
the Wuhan coronavirus can pass it on before they show symptoms. That may
make border-screening much harder, scientists said.

Epidemiological modeling released Friday by the
\href{https://www.nytimes3xbfgragh.onion/2020/02/10/world/europe/coronavirus-europe.html}{European
Center for Disease Prevention} and Control estimated that 75 percent of
infected people reaching
\href{https://www.nytimes3xbfgragh.onion/2020/02/10/world/europe/coronavirus-europe.html}{Europe}
from China would still be in the incubation periods upon arrival, and
therefore not detected by airport screening, which looks for fevers,
coughs and breathing difficulties.

But if thermal cameras miss victims who are beyond incubation and
actively infecting others, the real number of missed carriers may be
higher than 75 percent.

\href{https://www.nytimes3xbfgragh.onion/news-event/coronavirus?action=click\&pgtype=Article\&state=default\&region=MAIN_CONTENT_3\&context=storylines_faq}{}

\hypertarget{the-coronavirus-outbreak-}{%
\subsubsection{The Coronavirus Outbreak
›}\label{the-coronavirus-outbreak-}}

\hypertarget{frequently-asked-questions}{%
\paragraph{Frequently Asked
Questions}\label{frequently-asked-questions}}

Updated September 4, 2020

\begin{itemize}
\item ~
  \hypertarget{what-are-the-symptoms-of-coronavirus}{%
  \paragraph{What are the symptoms of
  coronavirus?}\label{what-are-the-symptoms-of-coronavirus}}

  \begin{itemize}
  \tightlist
  \item
    In the beginning, the coronavirus
    \href{https://www.nytimes3xbfgragh.onion/article/coronavirus-facts-history.html?action=click\&pgtype=Article\&state=default\&region=MAIN_CONTENT_3\&context=storylines_faq\#link-6817bab5}{seemed
    like it was primarily a respiratory illness}~--- many patients had
    fever and chills, were weak and tired, and coughed a lot, though
    some people don't show many symptoms at all. Those who seemed
    sickest had pneumonia or acute respiratory distress syndrome and
    received supplemental oxygen. By now, doctors have identified many
    more symptoms and syndromes. In April,
    \href{https://www.nytimes3xbfgragh.onion/2020/04/27/health/coronavirus-symptoms-cdc.html?action=click\&pgtype=Article\&state=default\&region=MAIN_CONTENT_3\&context=storylines_faq}{the
    C.D.C. added to the list of early signs}~sore throat, fever, chills
    and muscle aches. Gastrointestinal upset, such as diarrhea and
    nausea, has also been observed. Another telltale sign of infection
    may be a sudden, profound diminution of one's
    \href{https://www.nytimes3xbfgragh.onion/2020/03/22/health/coronavirus-symptoms-smell-taste.html?action=click\&pgtype=Article\&state=default\&region=MAIN_CONTENT_3\&context=storylines_faq}{sense
    of smell and taste.}~Teenagers and young adults in some cases have
    developed painful red and purple lesions on their fingers and toes
    --- nicknamed ``Covid toe'' --- but few other serious symptoms.
  \end{itemize}
\item ~
  \hypertarget{why-is-it-safer-to-spend-time-together-outside}{%
  \paragraph{Why is it safer to spend time together
  outside?}\label{why-is-it-safer-to-spend-time-together-outside}}

  \begin{itemize}
  \tightlist
  \item
    \href{https://www.nytimes3xbfgragh.onion/2020/05/15/us/coronavirus-what-to-do-outside.html?action=click\&pgtype=Article\&state=default\&region=MAIN_CONTENT_3\&context=storylines_faq}{Outdoor
    gatherings}~lower risk because wind disperses viral droplets, and
    sunlight can kill some of the virus. Open spaces prevent the virus
    from building up in concentrated amounts and being inhaled, which
    can happen when infected people exhale in a confined space for long
    stretches of time, said Dr. Julian W. Tang, a virologist at the
    University of Leicester.
  \end{itemize}
\item ~
  \hypertarget{why-does-standing-six-feet-away-from-others-help}{%
  \paragraph{Why does standing six feet away from others
  help?}\label{why-does-standing-six-feet-away-from-others-help}}

  \begin{itemize}
  \tightlist
  \item
    The coronavirus spreads primarily through droplets from your mouth
    and nose, especially when you cough or sneeze. The C.D.C., one of
    the organizations using that measure,
    \href{https://www.nytimes3xbfgragh.onion/2020/04/14/health/coronavirus-six-feet.html?action=click\&pgtype=Article\&state=default\&region=MAIN_CONTENT_3\&context=storylines_faq}{bases
    its recommendation of six feet}~on the idea that most large droplets
    that people expel when they cough or sneeze will fall to the ground
    within six feet. But six feet has never been a magic number that
    guarantees complete protection. Sneezes, for instance, can launch
    droplets a lot farther than six feet,
    \href{https://jamanetwork.com/journals/jama/fullarticle/2763852}{according
    to a recent study}. It's a rule of thumb: You should be safest
    standing six feet apart outside, especially when it's windy. But
    keep a mask on at all times, even when you think you're far enough
    apart.
  \end{itemize}
\item ~
  \hypertarget{i-have-antibodies-am-i-now-immune}{%
  \paragraph{I have antibodies. Am I now
  immune?}\label{i-have-antibodies-am-i-now-immune}}

  \begin{itemize}
  \tightlist
  \item
    As of right
    now,\href{https://www.nytimes3xbfgragh.onion/2020/07/22/health/covid-antibodies-herd-immunity.html?action=click\&pgtype=Article\&state=default\&region=MAIN_CONTENT_3\&context=storylines_faq}{~that
    seems likely, for at least several months.}~There have been
    frightening accounts of people suffering what seems to be a second
    bout of Covid-19. But experts say these patients may have a
    drawn-out course of infection, with the virus taking a slow toll
    weeks to months after initial exposure.~People infected with the
    coronavirus typically
    \href{https://www.nature.com/articles/s41586-020-2456-9}{produce}~immune
    molecules called antibodies, which are
    \href{https://www.nytimes3xbfgragh.onion/2020/05/07/health/coronavirus-antibody-prevalence.html?action=click\&pgtype=Article\&state=default\&region=MAIN_CONTENT_3\&context=storylines_faq}{protective
    proteins made in response to an
    infection}\href{https://www.nytimes3xbfgragh.onion/2020/05/07/health/coronavirus-antibody-prevalence.html?action=click\&pgtype=Article\&state=default\&region=MAIN_CONTENT_3\&context=storylines_faq}{.
    These antibodies may}~last in the body
    \href{https://www.nature.com/articles/s41591-020-0965-6}{only two to
    three months}, which may seem worrisome, but that's~perfectly normal
    after an acute infection subsides, said Dr. Michael Mina, an
    immunologist at Harvard University. It may be possible to get the
    coronavirus again, but it's highly unlikely that it would be
    possible in a short window of time from initial infection or make
    people sicker the second time.
  \end{itemize}
\item ~
  \hypertarget{what-are-my-rights-if-i-am-worried-about-going-back-to-work}{%
  \paragraph{What are my rights if I am worried about going back to
  work?}\label{what-are-my-rights-if-i-am-worried-about-going-back-to-work}}

  \begin{itemize}
  \tightlist
  \item
    Employers have to provide
    \href{https://www.osha.gov/SLTC/covid-19/standards.html}{a safe
    workplace}~with policies that protect everyone equally.
    \href{https://www.nytimes3xbfgragh.onion/article/coronavirus-money-unemployment.html?action=click\&pgtype=Article\&state=default\&region=MAIN_CONTENT_3\&context=storylines_faq}{And
    if one of your co-workers tests positive for the coronavirus, the
    C.D.C.}~has said that
    \href{https://www.cdc.gov/coronavirus/2019-ncov/community/guidance-business-response.html}{employers
    should tell their employees}~-\/- without giving you the sick
    employee's name -\/- that they may have been exposed to the virus.
  \end{itemize}
\end{itemize}

Still, asymptomatic carriers ``are not normally major drivers of
epidemics,'' Dr. Fauci said. Most people get ill from someone they know
to be sick --- a family member, a co-worker or a patient, for example.

The virus's most vulnerable target is Africa, many experts said. More
than 1 million expatriate Chinese work there, mostly on mining, drilling
or engineering projects. Also, many Africans work and study in China and
other countries where the virus has been found.

If anyone on the continent has the virus now, ``I'm not sure the
diagnostic systems are in place to detect it,'' said Dr. Daniel Bausch,
head of scientific programs for the American Society of Tropical
Medicine and Hygiene, who is consulting with the W.H.O. on the outbreak.

South Africa and Senegal could probably diagnose it, he said. Nigeria
and some other countries have asked the W.H.O. for the genetic materials
and training they need to perform diagnostic tests, but that will take
time.

At least four African countries have suspect cases quarantined,
according to
\href{https://www.scmp.com/news/china/article/3048310/china-coronavirus-african-nations-quarantine-symptomatic-passengers}{an
article published Friday in The South China Morning Post}. They have
sent samples to France, Germany, India and South Africa for testing.

\textbf{\emph{{[}}\href{http://on.fb.me/1paTQ1h}{\emph{Like the Science
Times page on Facebook.}}} ****** \emph{\textbar{} Sign up for the}
\textbf{\href{http://nyti.ms/1MbHaRU}{\emph{Science Times
newsletter.}}\emph{{]}}}

At the moment, it seems unlikely that the virus will spread widely in
countries with vigorous, alert public health systems, said Dr. William
Schaffner, a preventive medicine specialist at Vanderbilt University
Medical Center.

``Every doctor in the U.S. has this top of mind,'' he said. ``Any
patient with fever or respiratory problems will get two questions. `Have
you been to China? Have you had contact with anyone who has?' If the
answer is yes, they'll be put in isolation right away.''

Assuming the virus spreads globally, tourism to and trade with countries
besides China may be affected --- and the urgency to find ways to halt
the virus and prevent deaths will grow.

Image

Men in protective suits greeted a plane carrying 32 Mongolian citizens
evacuated from Wuhan, China, as it arrived in
Ulaanbaatar.Credit...Byambasuren Byamba-Ochir/Agence France-Presse ---
Getty Images

It is possible that the Wuhan coronavirus will fade out as weather
warms. Many viruses, like flu, measles and norovirus, thrive in cold,
dry air. The SARS outbreak began in winter, and MERS transmission also
peaks then, though that may be related to transmission in newborn
camels.

Four mild coronaviruses cause about a quarter of the nation's common
colds, which also peak in winter.

But even if an outbreak fades in June, there could be a second wave in
the fall, as has occurred in every major flu pandemic, including those
that began in 1918 and 2009.

By that time, some remedies might be on hand, although they will need
rigorous testing and perhaps political pressure to make them available
and affordable.

In China, several
\href{https://www.sciencemag.org/news/2020/01/can-anti-hiv-combination-or-other-existing-drugs-outwit-new-coronavirus}{antiviral
drugs are being prescribed}. A common combination is pills containing
lopinavir and ritonavir with infusions of interferon, a signaling
protein that wakes up the immune system.

In the United States, the combination is sold as Kaletra by AbbVie for
H.I.V. therapy, and it is relatively expensive. In India, a dozen
generic makers produce the drugs at rock-bottom prices for use against
H.I.V. in Africa, and their products are W.H.O.-approved.

Another option may be
\href{https://www.gilead.com/news-and-press/company-statements/gilead-sciences-statement-on-the-company-ongoing-response-to-the-2019-new-coronavirus}{an
experimental drug, remdesivir}, on which the patent is held by Gilead.
The drug has not yet been approved for use against any disease.
Nonetheless, there is some evidence that it works against coronaviruses,
and Gilead has donated doses to China.

Several American companies are
\href{https://www.nytimes3xbfgragh.onion/2020/01/28/health/coronavirus-vaccine.html}{working
on a vaccine}, using various combinations of their own funds, taxpayer
money and foundation grants.

Although modern gene-chemistry techniques have made it possible to build
vaccine candidates within just days, medical ethics require that they
then be carefully tested on animals and small numbers of healthy humans
for safety and effectiveness.

That aspect of the process cannot be sped up, because dangerous side
effects may take time to appear and because human immune systems need
time to produce the antibodies that show whether a vaccine is working.

Whether or not what is being tried in China will be acceptable elsewhere
will depend on how rigorously Chinese doctors run their clinical trials.

``In God we trust,'' Dr. Schaffner said. ``All others must provide
data.''

Advertisement

\protect\hyperlink{after-bottom}{Continue reading the main story}

\hypertarget{site-index}{%
\subsection{Site Index}\label{site-index}}

\hypertarget{site-information-navigation}{%
\subsection{Site Information
Navigation}\label{site-information-navigation}}

\begin{itemize}
\tightlist
\item
  \href{https://help.nytimes3xbfgragh.onion/hc/en-us/articles/115014792127-Copyright-notice}{©~2020~The
  New York Times Company}
\end{itemize}

\begin{itemize}
\tightlist
\item
  \href{https://www.nytco.com/}{NYTCo}
\item
  \href{https://help.nytimes3xbfgragh.onion/hc/en-us/articles/115015385887-Contact-Us}{Contact
  Us}
\item
  \href{https://www.nytco.com/careers/}{Work with us}
\item
  \href{https://nytmediakit.com/}{Advertise}
\item
  \href{http://www.tbrandstudio.com/}{T Brand Studio}
\item
  \href{https://www.nytimes3xbfgragh.onion/privacy/cookie-policy\#how-do-i-manage-trackers}{Your
  Ad Choices}
\item
  \href{https://www.nytimes3xbfgragh.onion/privacy}{Privacy}
\item
  \href{https://help.nytimes3xbfgragh.onion/hc/en-us/articles/115014893428-Terms-of-service}{Terms
  of Service}
\item
  \href{https://help.nytimes3xbfgragh.onion/hc/en-us/articles/115014893968-Terms-of-sale}{Terms
  of Sale}
\item
  \href{https://spiderbites.nytimes3xbfgragh.onion}{Site Map}
\item
  \href{https://help.nytimes3xbfgragh.onion/hc/en-us}{Help}
\item
  \href{https://www.nytimes3xbfgragh.onion/subscription?campaignId=37WXW}{Subscriptions}
\end{itemize}
