Sections

SEARCH

\protect\hyperlink{site-content}{Skip to
content}\protect\hyperlink{site-index}{Skip to site index}

\href{https://www.nytimes3xbfgragh.onion/section/health}{Health}

\href{https://myaccount.nytimes3xbfgragh.onion/auth/login?response_type=cookie\&client_id=vi}{}

\href{https://www.nytimes3xbfgragh.onion/section/todayspaper}{Today's
Paper}

\href{/section/health}{Health}\textbar{}Why the New Coronavirus (Mostly)
Spares Children

\url{https://nyti.ms/2Sol5VV}

\begin{itemize}
\item
\item
\item
\item
\item
\end{itemize}

\hypertarget{the-coronavirus-outbreak}{%
\subsubsection{\texorpdfstring{\href{https://www.nytimes3xbfgragh.onion/news-event/coronavirus?name=styln-coronavirus-national\&region=TOP_BANNER\&block=storyline_menu_recirc\&action=click\&pgtype=Article\&impression_id=d9441360-f1ea-11ea-b089-5de7c98ed547\&variant=undefined}{The
Coronavirus
Outbreak}}{The Coronavirus Outbreak}}\label{the-coronavirus-outbreak}}

\begin{itemize}
\tightlist
\item
  live\href{https://www.nytimes3xbfgragh.onion/2020/09/08/world/covid-19-coronavirus.html?name=styln-coronavirus-national\&region=TOP_BANNER\&block=storyline_menu_recirc\&action=click\&pgtype=Article\&impression_id=d9441361-f1ea-11ea-b089-5de7c98ed547\&variant=undefined}{Latest
  Updates}
\item
  \href{https://www.nytimes3xbfgragh.onion/interactive/2020/us/coronavirus-us-cases.html?name=styln-coronavirus-national\&region=TOP_BANNER\&block=storyline_menu_recirc\&action=click\&pgtype=Article\&impression_id=d9441362-f1ea-11ea-b089-5de7c98ed547\&variant=undefined}{Maps
  and Cases}
\item
  \href{https://www.nytimes3xbfgragh.onion/interactive/2020/science/coronavirus-vaccine-tracker.html?name=styln-coronavirus-national\&region=TOP_BANNER\&block=storyline_menu_recirc\&action=click\&pgtype=Article\&impression_id=d9443a70-f1ea-11ea-b089-5de7c98ed547\&variant=undefined}{Vaccine
  Tracker}
\item
  \href{https://www.nytimes3xbfgragh.onion/2020/09/02/your-money/eviction-moratorium-covid.html?name=styln-coronavirus-national\&region=TOP_BANNER\&block=storyline_menu_recirc\&action=click\&pgtype=Article\&impression_id=d9443a71-f1ea-11ea-b089-5de7c98ed547\&variant=undefined}{Eviction
  Moratorium}
\item
  \href{https://www.nytimes3xbfgragh.onion/interactive/2020/09/02/magazine/food-insecurity-hunger-us.html?name=styln-coronavirus-national\&region=TOP_BANNER\&block=storyline_menu_recirc\&action=click\&pgtype=Article\&impression_id=d9443a72-f1ea-11ea-b089-5de7c98ed547\&variant=undefined}{American
  Hunger}
\end{itemize}

Advertisement

\protect\hyperlink{after-top}{Continue reading the main story}

Supported by

\protect\hyperlink{after-sponsor}{Continue reading the main story}

\hypertarget{why-the-new-coronavirus-mostly-spares-children}{%
\section{Why the New Coronavirus (Mostly) Spares
Children}\label{why-the-new-coronavirus-mostly-spares-children}}

So far, very few young children seem to be falling ill. The pattern was
seen in outbreaks of SARS and MERS, too.

\includegraphics{https://static01.graylady3jvrrxbe.onion/images/2020/02/05/science/05VIRUS-CHILDREN/05VIRUS-CHILDREN-articleLarge.jpg?quality=75\&auto=webp\&disable=upscale}

By Apoorva Mandavilli

\begin{itemize}
\item
  Published Feb. 5, 2020Updated May 11, 2020
\item
  \begin{itemize}
  \item
  \item
  \item
  \item
  \item
  \end{itemize}
\end{itemize}

\href{https://cn.nytimes3xbfgragh.onion/health/20200206/coronavirus-children/}{阅读简体中文版}\href{https://cn.nytimes3xbfgragh.onion/health/20200206/coronavirus-children/zh-hant/}{閱讀繁體中文版}

The new coronavirus has infected nearly 90,000 people, and more than
3,000 have died. But relatively few
\href{https://www.nytimes3xbfgragh.onion/2020/05/11/health/coronavirus-children-icu.html}{children}
appear to have developed severe symptoms so far, according to available
data.

``Disease in children appears to be relatively rare and mild,'' with
those under 19 years making up only 2.4 percent of the total cases,
according to a
\href{https://www.who.int/docs/default-source/coronaviruse/who-china-joint-mission-on-covid-19-final-report.pdf}{report}
published Feb. 28 by the World Health Organization.

So why aren't more children getting sick?

``My strong, educated guess is that younger people are getting infected,
but they get the relatively milder disease,'' Dr. Malik Peiris, chief of
virology at the University of Hong Kong, said last month. Dr. Peiris has
developed a diagnostic test for the new coronavirus.

The numbers so far support that theory: According to the W.H.O., only
2.5 percent of those under 19 have developed severe disease and only 0.2
percent had critical disease. There have been no deaths recorded in
children under 9.

Scientists may not be seeing more infected children because ``we don't
have data on the milder cases,'' Dr. Peiris said.

Without more information, it is also unclear whether
\href{https://www.nytimes3xbfgragh.onion/2020/05/11/health/coronavirus-children-icu.html}{children}
can transmit the disease to others. The W.H.O. report said that its team
sent to China, the epicenter of the outbreak, ``could not recall
episodes in which transmission occurred from a child to an adult.''

Still, children who are detected as infected must be shedding some virus
or they wouldn't be detected, noted Dr. Marc Lipsitch, an epidemiologist
at the Harvard T.H. Chan School of Public Health. But whether their
infectiousness is high is as yet unknown. ``It's a very high priority to
do studies to find it out,'' he said.

\hypertarget{latest-updates-the-coronavirus-outbreak}{%
\section{\texorpdfstring{\href{https://www.nytimes3xbfgragh.onion/2020/09/08/world/covid-19-coronavirus.html?action=click\&pgtype=Article\&state=default\&region=MAIN_CONTENT_1\&context=storylines_live_updates}{Latest
Updates: The Coronavirus
Outbreak}}{Latest Updates: The Coronavirus Outbreak}}\label{latest-updates-the-coronavirus-outbreak}}

Updated 2020-09-08T15:29:57.612Z

\begin{itemize}
\tightlist
\item
  \href{https://www.nytimes3xbfgragh.onion/2020/09/08/world/covid-19-coronavirus.html?action=click\&pgtype=Article\&state=default\&region=MAIN_CONTENT_1\&context=storylines_live_updates\#link-547feae1}{Senate
  Republicans plan to move forward with a scaled-back stimulus package.}
\item
  \href{https://www.nytimes3xbfgragh.onion/2020/09/08/world/covid-19-coronavirus.html?action=click\&pgtype=Article\&state=default\&region=MAIN_CONTENT_1\&context=storylines_live_updates\#link-679303d7}{Nine
  drugmakers pledge to thoroughly vet any coronavirus vaccine.}
\item
  \href{https://www.nytimes3xbfgragh.onion/2020/09/08/world/covid-19-coronavirus.html?action=click\&pgtype=Article\&state=default\&region=MAIN_CONTENT_1\&context=storylines_live_updates\#link-1c973131}{`The
  lockdown killed my father': Farmer suicides add to India's virus
  misery.}
\end{itemize}

\href{https://www.nytimes3xbfgragh.onion/2020/09/08/world/covid-19-coronavirus.html?action=click\&pgtype=Article\&state=default\&region=MAIN_CONTENT_1\&context=storylines_live_updates}{See
more updates}

More live coverage:
\href{https://www.nytimes3xbfgragh.onion/live/2020/09/08/business/stock-market-today-coronavirus?action=click\&pgtype=Article\&state=default\&region=MAIN_CONTENT_1\&context=storylines_live_updates}{Markets}

One way to find out, he said, is to look at outbreaks such as the one at
the church in South Korea. ``If there were children among those
people,'' he said, ``that would be a goldmine of data.''

The other approach is to conduct household studies, where multiple
members of a family might be infected.

In one such published case study of a family, a 10-year-old traveled to
Wuhan, China, with his family. Upon returning to Shenzhen, the other
infected family members, ranging in age from 36 to 66,
\href{https://www.thelancet.com/journals/lancet/article/PIIS0140-6736(20)30154-9/fulltext}{developed
fever, sore throat, diarrhea and pneumonia}.

The child, too, had signs of viral pneumonia in the lungs, doctors found
--- but no outward symptoms. Some scientists suspect that this is
typical of coronavirus infection in children.

``It's certainly true that children can be either asymptomatically
infected or have very mild infection,'' Dr. Raina MacIntyre said last
month. Dr. MacIntyre is an epidemiologist at the University of New South
Wales in Sydney, Australia, who has been studying the spread of the new
coronavirus.

In many ways, this pattern parallels that seen during outbreaks of SARS
and MERS, also coronaviruses. The MERS epidemics in Saudi Arabia in 2012
and in South Korea in 2015 together claimed more than 800 lives. Most
children who were infected never developed symptoms.

No children died during the SARS epidemic in 2003, and the majority of
the 800 deaths in the outbreak were in people over age 45, with men more
at risk.

Among the more than 8,000 cases of SARS, researchers at the Centers for
Disease Control and Prevention were able to identify 135 infected
children in published reports.

Children under age 12 were much less likely to be admitted to a hospital
or to need oxygen or other treatment, the researchers found. Children
over age 12 had symptoms much like those of adults.

``We don't fully understand the reason for this age-related increase of
severity,'' Dr. Peiris said. ``But we see that now --- and with SARS,
you could see that much more clearly.''

It's not unusual for viruses to trigger only mild infections in children
and much more severe illnesses in adults. Chickenpox, for example, can
be largely inconsequential in children, yet catastrophic in adults.

\href{https://www.nytimes3xbfgragh.onion/news-event/coronavirus?action=click\&pgtype=Article\&state=default\&region=MAIN_CONTENT_3\&context=storylines_faq}{}

\hypertarget{the-coronavirus-outbreak-}{%
\subsubsection{The Coronavirus Outbreak
›}\label{the-coronavirus-outbreak-}}

\hypertarget{frequently-asked-questions}{%
\paragraph{Frequently Asked
Questions}\label{frequently-asked-questions}}

Updated September 4, 2020

\begin{itemize}
\item ~
  \hypertarget{what-are-the-symptoms-of-coronavirus}{%
  \paragraph{What are the symptoms of
  coronavirus?}\label{what-are-the-symptoms-of-coronavirus}}

  \begin{itemize}
  \tightlist
  \item
    In the beginning, the coronavirus
    \href{https://www.nytimes3xbfgragh.onion/article/coronavirus-facts-history.html?action=click\&pgtype=Article\&state=default\&region=MAIN_CONTENT_3\&context=storylines_faq\#link-6817bab5}{seemed
    like it was primarily a respiratory illness}~--- many patients had
    fever and chills, were weak and tired, and coughed a lot, though
    some people don't show many symptoms at all. Those who seemed
    sickest had pneumonia or acute respiratory distress syndrome and
    received supplemental oxygen. By now, doctors have identified many
    more symptoms and syndromes. In April,
    \href{https://www.nytimes3xbfgragh.onion/2020/04/27/health/coronavirus-symptoms-cdc.html?action=click\&pgtype=Article\&state=default\&region=MAIN_CONTENT_3\&context=storylines_faq}{the
    C.D.C. added to the list of early signs}~sore throat, fever, chills
    and muscle aches. Gastrointestinal upset, such as diarrhea and
    nausea, has also been observed. Another telltale sign of infection
    may be a sudden, profound diminution of one's
    \href{https://www.nytimes3xbfgragh.onion/2020/03/22/health/coronavirus-symptoms-smell-taste.html?action=click\&pgtype=Article\&state=default\&region=MAIN_CONTENT_3\&context=storylines_faq}{sense
    of smell and taste.}~Teenagers and young adults in some cases have
    developed painful red and purple lesions on their fingers and toes
    --- nicknamed ``Covid toe'' --- but few other serious symptoms.
  \end{itemize}
\item ~
  \hypertarget{why-is-it-safer-to-spend-time-together-outside}{%
  \paragraph{Why is it safer to spend time together
  outside?}\label{why-is-it-safer-to-spend-time-together-outside}}

  \begin{itemize}
  \tightlist
  \item
    \href{https://www.nytimes3xbfgragh.onion/2020/05/15/us/coronavirus-what-to-do-outside.html?action=click\&pgtype=Article\&state=default\&region=MAIN_CONTENT_3\&context=storylines_faq}{Outdoor
    gatherings}~lower risk because wind disperses viral droplets, and
    sunlight can kill some of the virus. Open spaces prevent the virus
    from building up in concentrated amounts and being inhaled, which
    can happen when infected people exhale in a confined space for long
    stretches of time, said Dr. Julian W. Tang, a virologist at the
    University of Leicester.
  \end{itemize}
\item ~
  \hypertarget{why-does-standing-six-feet-away-from-others-help}{%
  \paragraph{Why does standing six feet away from others
  help?}\label{why-does-standing-six-feet-away-from-others-help}}

  \begin{itemize}
  \tightlist
  \item
    The coronavirus spreads primarily through droplets from your mouth
    and nose, especially when you cough or sneeze. The C.D.C., one of
    the organizations using that measure,
    \href{https://www.nytimes3xbfgragh.onion/2020/04/14/health/coronavirus-six-feet.html?action=click\&pgtype=Article\&state=default\&region=MAIN_CONTENT_3\&context=storylines_faq}{bases
    its recommendation of six feet}~on the idea that most large droplets
    that people expel when they cough or sneeze will fall to the ground
    within six feet. But six feet has never been a magic number that
    guarantees complete protection. Sneezes, for instance, can launch
    droplets a lot farther than six feet,
    \href{https://jamanetwork.com/journals/jama/fullarticle/2763852}{according
    to a recent study}. It's a rule of thumb: You should be safest
    standing six feet apart outside, especially when it's windy. But
    keep a mask on at all times, even when you think you're far enough
    apart.
  \end{itemize}
\item ~
  \hypertarget{i-have-antibodies-am-i-now-immune}{%
  \paragraph{I have antibodies. Am I now
  immune?}\label{i-have-antibodies-am-i-now-immune}}

  \begin{itemize}
  \tightlist
  \item
    As of right
    now,\href{https://www.nytimes3xbfgragh.onion/2020/07/22/health/covid-antibodies-herd-immunity.html?action=click\&pgtype=Article\&state=default\&region=MAIN_CONTENT_3\&context=storylines_faq}{~that
    seems likely, for at least several months.}~There have been
    frightening accounts of people suffering what seems to be a second
    bout of Covid-19. But experts say these patients may have a
    drawn-out course of infection, with the virus taking a slow toll
    weeks to months after initial exposure.~People infected with the
    coronavirus typically
    \href{https://www.nature.com/articles/s41586-020-2456-9}{produce}~immune
    molecules called antibodies, which are
    \href{https://www.nytimes3xbfgragh.onion/2020/05/07/health/coronavirus-antibody-prevalence.html?action=click\&pgtype=Article\&state=default\&region=MAIN_CONTENT_3\&context=storylines_faq}{protective
    proteins made in response to an
    infection}\href{https://www.nytimes3xbfgragh.onion/2020/05/07/health/coronavirus-antibody-prevalence.html?action=click\&pgtype=Article\&state=default\&region=MAIN_CONTENT_3\&context=storylines_faq}{.
    These antibodies may}~last in the body
    \href{https://www.nature.com/articles/s41591-020-0965-6}{only two to
    three months}, which may seem worrisome, but that's~perfectly normal
    after an acute infection subsides, said Dr. Michael Mina, an
    immunologist at Harvard University. It may be possible to get the
    coronavirus again, but it's highly unlikely that it would be
    possible in a short window of time from initial infection or make
    people sicker the second time.
  \end{itemize}
\item ~
  \hypertarget{what-are-my-rights-if-i-am-worried-about-going-back-to-work}{%
  \paragraph{What are my rights if I am worried about going back to
  work?}\label{what-are-my-rights-if-i-am-worried-about-going-back-to-work}}

  \begin{itemize}
  \tightlist
  \item
    Employers have to provide
    \href{https://www.osha.gov/SLTC/covid-19/standards.html}{a safe
    workplace}~with policies that protect everyone equally.
    \href{https://www.nytimes3xbfgragh.onion/article/coronavirus-money-unemployment.html?action=click\&pgtype=Article\&state=default\&region=MAIN_CONTENT_3\&context=storylines_faq}{And
    if one of your co-workers tests positive for the coronavirus, the
    C.D.C.}~has said that
    \href{https://www.cdc.gov/coronavirus/2019-ncov/community/guidance-business-response.html}{employers
    should tell their employees}~-\/- without giving you the sick
    employee's name -\/- that they may have been exposed to the virus.
  \end{itemize}
\end{itemize}

Influenza is unusual in that it has evolved with humans over thousands
of years and infects millions worldwide each year. Still, even though
thousands of young children end up in the hospital each year with
influenza, just a small percentage of them die, said Dr. Krys Johnson,
an epidemiologist at Temple University in Philadelphia.

This trend is generally true of respiratory illnesses because children
tend to eat well, and to get plenty of exercise and rest --- none of
which may be true of adults. ``The younger, most healthy segment of the
population are able to fight it off,'' she said. Adults may also be more
susceptible because they are more likely to have other diseases, such as
diabetes, high blood pressure or heart disease, that weaken their
ability to stave off infections.

The body's innate immunity, which is critical for fighting viruses, also
deteriorates with age, and particularly after middle age.

``Something happens at age 50,'' Dr. MacIntyre said. ``It declines, and
it declines exponentially, which is why for most infections we see the
highest incidence in the elderly.''

A key question about the new coronavirus is whether children who are
infected and asymptomatic are able to pass the virus to others.

``We know that young people in general --- not just kids, but young
adults and teenagers --- have the most intense contact in society,'' Dr.
MacIntyre said. Young people who don't realize they are sick may
contribute to the epidemic's momentum, she said.

To understand the epidemic fully, she and other scientists said they
need detailed data: when people were first exposed to the virus, when
they first began to show symptoms, how many and which people have mild
symptoms versus more severe disease.

With detailed data, some observations, such as the higher risk in men,
may change. Still, Dr. Mark Denison, a pediatric infectious diseases
specialist at Vanderbilt University in Nashville, said last month that
he does not expect to see a sudden uptick in infected children.

``It's hard for me to imagine that there's such a degree of
underreporting of clinical illness in children that we're only hearing
about two or three cases,'' he said.

``I think it means that there are many, many less children'' who are
infected in China, he said, ``and that they're not as much at risk.''

Advertisement

\protect\hyperlink{after-bottom}{Continue reading the main story}

\hypertarget{site-index}{%
\subsection{Site Index}\label{site-index}}

\hypertarget{site-information-navigation}{%
\subsection{Site Information
Navigation}\label{site-information-navigation}}

\begin{itemize}
\tightlist
\item
  \href{https://help.nytimes3xbfgragh.onion/hc/en-us/articles/115014792127-Copyright-notice}{©~2020~The
  New York Times Company}
\end{itemize}

\begin{itemize}
\tightlist
\item
  \href{https://www.nytco.com/}{NYTCo}
\item
  \href{https://help.nytimes3xbfgragh.onion/hc/en-us/articles/115015385887-Contact-Us}{Contact
  Us}
\item
  \href{https://www.nytco.com/careers/}{Work with us}
\item
  \href{https://nytmediakit.com/}{Advertise}
\item
  \href{http://www.tbrandstudio.com/}{T Brand Studio}
\item
  \href{https://www.nytimes3xbfgragh.onion/privacy/cookie-policy\#how-do-i-manage-trackers}{Your
  Ad Choices}
\item
  \href{https://www.nytimes3xbfgragh.onion/privacy}{Privacy}
\item
  \href{https://help.nytimes3xbfgragh.onion/hc/en-us/articles/115014893428-Terms-of-service}{Terms
  of Service}
\item
  \href{https://help.nytimes3xbfgragh.onion/hc/en-us/articles/115014893968-Terms-of-sale}{Terms
  of Sale}
\item
  \href{https://spiderbites.nytimes3xbfgragh.onion}{Site Map}
\item
  \href{https://help.nytimes3xbfgragh.onion/hc/en-us}{Help}
\item
  \href{https://www.nytimes3xbfgragh.onion/subscription?campaignId=37WXW}{Subscriptions}
\end{itemize}
