Sections

SEARCH

\protect\hyperlink{site-content}{Skip to
content}\protect\hyperlink{site-index}{Skip to site index}

\href{https://www.nytimes3xbfgragh.onion/section/world/americas}{Americas}

\href{https://myaccount.nytimes3xbfgragh.onion/auth/login?response_type=cookie\&client_id=vi}{}

\href{https://www.nytimes3xbfgragh.onion/section/todayspaper}{Today's
Paper}

\href{/section/world/americas}{Americas}\textbar{}Will an Ex-Missionary
Shield Brazil's Tribes From Outsiders?

\url{https://nyti.ms/2GYGShQ}

\begin{itemize}
\item
\item
\item
\item
\item
\end{itemize}

Advertisement

\protect\hyperlink{after-top}{Continue reading the main story}

Supported by

\protect\hyperlink{after-sponsor}{Continue reading the main story}

\hypertarget{will-an-ex-missionary-shield-brazils-tribes-from-outsiders}{%
\section{Will an Ex-Missionary Shield Brazil's Tribes From
Outsiders?}\label{will-an-ex-missionary-shield-brazils-tribes-from-outsiders}}

Critics fear uncontacted tribes may suffer ``irreparable damage'' after
President Jair Bolsonaro appointed an anthropologist who is also an
evangelical preacher to lead the National Indian Foundation.

\includegraphics{https://static01.graylady3jvrrxbe.onion/images/2020/02/05/world/05brazil/merlin_154268910_d1905598-1b8c-4684-a937-784f7369c76e-articleLarge.jpg?quality=75\&auto=webp\&disable=upscale}

By \href{https://www.nytimes3xbfgragh.onion/by/ernesto-londono}{Ernesto
Londoño} and Letícia Casado

\begin{itemize}
\item
  Feb. 5, 2020
\item
  \begin{itemize}
  \item
  \item
  \item
  \item
  \item
  \end{itemize}
\end{itemize}

RIO DE JANEIRO --- Brazil's government on Wednesday tapped a former
Christian missionary to oversee the protection of isolated Indigenous
tribes in Brazil, prompting an outcry among anthropologists and experts
within the government.

\href{https://indigenistasassociados.org.br/2020/02/03/risco-iminente-coordenacao-de-indios-isolados-deve-ter-experiencia-na-area/}{In
a rare letter of protest}, the association that represents career
employees at Brazil's Indigenous affairs agency called the appointment
of Ricardo Lopes Dias, an anthropologist and evangelical preacher, a
dangerous move that could cause ``irreparable damage'' to vulnerable
groups that have chosen to live in isolation.

Since the late 1980s, Brazil's government has largely refrained from
making contact with the dozens of tribes living in voluntary isolation,
most of them in the Amazon.

The National Indian Foundation, the federal agency created to protect
Indigenous communities, has argued that uncontacted tribes deserve to be
protected from outsiders. Such contacts are often devastating for
isolated communities, which can easily be ravaged by common diseases.

But some evangelical missionaries have long been eager to seek converts
among Indigenous peoples in the Amazon.

Mr. Dias worked for several years for an American missionary group that
sought to establish a Christian church in every Indigenous community in
Brazil.

``The initiative to establish a church in each community is at odds with
the recognition of the diversity of the communities and their
cultures,'' which is protected by the Constitution, the indigenous
agency employee association said in its statement.

Mr. Dias said in an interview Wednesday afternoon that he had ``no
interest'' in using the post to evangelize. He said he has yet to
receive guidance from senior government officials regarding the
no-contact policy that has been in force since the 1980s, adding that it
was too soon to say whether it needs to be reconsidered.

Mr. Dias defended his work as a missionary and said he was qualified for
the job.

``I understand there is a lot of apprehension regarding what the work of
missionaries entails,'' he said. ``I don't see this as a mission or an
opportunity to find new converts. I have no interest in going there with
a Bible in hand.''

The appointment comes amid
\href{https://www.nytimes3xbfgragh.onion/2019/01/02/world/americas/brazil-bolsonaro-president-indigenous-lands.html}{broader
concerns about the future of Brazil's Indigenous communities.}

President Jair Bolsonaro has long been critical of the policy of setting
aside vast territories for Indigenous groups, calling it an impediment
to economic growth. His administration is seeking to create a legal
framework that would allow mining ventures in some of those lands.

He has also compared indigenous communities living in remote areas to
\href{https://oglobo.globo.com/sociedade/bolsonaro-compara-indios-em-reservas-animais-em-zoologicos-23272902}{animals
in a zoo}.

The most vulnerable of Brazil's Indigenous communities are the groups
--- which by some estimates number more than 100 --- that have had
little or no contact with the outside world. The National Indian
Foundation has confirmed sightings of approximately 28 such communities,
and provides health care and guidance to about 11 of them that have
recently chosen to emerge from total isolation.

The missionary group Mr. Dias worked for from 1997 to 2007, which was
called New Tribes Mission at the time and is now known as Ethnos360,
argued that there is an urgent need to convert all tribes that have not
been exposed to ``the Gospel of Christ'' in order to save them from
``unrelenting spiritual darkness.''

``I've been in many of these tribes and at times you can feel this
incredible and tense darkness,'' Larry Brown, the group's chief
executive, said \href{https://ethnos360.org/about}{in a video posted on
its website}. ``But you know what I found: No darkness is too dark for
God.''

Leila Sílvia Burger Sotto-Maior, an anthropologist who retired from the
National Indian Foundation in 2018, said there is deep alarm among her
former colleagues at the agency about the fate of uncontacted and newly
contacted tribes under the current government.

The agency, known as FUNAI, has been hit with budget cuts that have
sharply limited its ability to monitor indigenous territories that have
been
\href{https://www.nytimes3xbfgragh.onion/2018/11/10/world/americas/brazil-indigenous-mining-bolsonaro.html}{invaded
by wildcat miners}, farmers and loggers.

``There is a sense that a policy that was built over so many years, and
was working, is now being dismantled,'' she said. ``There's a sense of
hopelessness.''

Márcio Santilli, a prominent Indigenous rights activist in Brazil and a
former head of FUNAI, said he worries that a government that was elected
with strong support from evangelicals will enable the kind of missionary
work that has in recent decades been officially discouraged, including
efforts to suppress the cultural traditions and spiritual practices of
Indigenous people.

``The Constitution is very clear in its protection of the cultural
identity of Indigenous people, including those in isolation,'' he said.
``There's a risk that the state would become a vehicle for religious
groups that don't want to protect the cultural identity of those
communities.''

Ernesto Londoño reported from Rio de Janeiro and Letícia Casado from
Brasília.

Advertisement

\protect\hyperlink{after-bottom}{Continue reading the main story}

\hypertarget{site-index}{%
\subsection{Site Index}\label{site-index}}

\hypertarget{site-information-navigation}{%
\subsection{Site Information
Navigation}\label{site-information-navigation}}

\begin{itemize}
\tightlist
\item
  \href{https://help.nytimes3xbfgragh.onion/hc/en-us/articles/115014792127-Copyright-notice}{©~2020~The
  New York Times Company}
\end{itemize}

\begin{itemize}
\tightlist
\item
  \href{https://www.nytco.com/}{NYTCo}
\item
  \href{https://help.nytimes3xbfgragh.onion/hc/en-us/articles/115015385887-Contact-Us}{Contact
  Us}
\item
  \href{https://www.nytco.com/careers/}{Work with us}
\item
  \href{https://nytmediakit.com/}{Advertise}
\item
  \href{http://www.tbrandstudio.com/}{T Brand Studio}
\item
  \href{https://www.nytimes3xbfgragh.onion/privacy/cookie-policy\#how-do-i-manage-trackers}{Your
  Ad Choices}
\item
  \href{https://www.nytimes3xbfgragh.onion/privacy}{Privacy}
\item
  \href{https://help.nytimes3xbfgragh.onion/hc/en-us/articles/115014893428-Terms-of-service}{Terms
  of Service}
\item
  \href{https://help.nytimes3xbfgragh.onion/hc/en-us/articles/115014893968-Terms-of-sale}{Terms
  of Sale}
\item
  \href{https://spiderbites.nytimes3xbfgragh.onion}{Site Map}
\item
  \href{https://help.nytimes3xbfgragh.onion/hc/en-us}{Help}
\item
  \href{https://www.nytimes3xbfgragh.onion/subscription?campaignId=37WXW}{Subscriptions}
\end{itemize}
