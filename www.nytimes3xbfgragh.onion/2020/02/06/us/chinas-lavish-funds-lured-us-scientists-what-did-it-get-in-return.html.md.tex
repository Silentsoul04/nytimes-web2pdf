Sections

SEARCH

\protect\hyperlink{site-content}{Skip to
content}\protect\hyperlink{site-index}{Skip to site index}

\href{https://www.nytimes3xbfgragh.onion/section/us}{U.S.}

\href{https://myaccount.nytimes3xbfgragh.onion/auth/login?response_type=cookie\&client_id=vi}{}

\href{https://www.nytimes3xbfgragh.onion/section/todayspaper}{Today's
Paper}

\href{/section/us}{U.S.}\textbar{}China's Lavish Funds Lured U.S.
Scientists. What Did It Get in Return?

\url{https://nyti.ms/31HF5Y8}

\begin{itemize}
\item
\item
\item
\item
\item
\item
\end{itemize}

Advertisement

\protect\hyperlink{after-top}{Continue reading the main story}

Supported by

\protect\hyperlink{after-sponsor}{Continue reading the main story}

\hypertarget{chinas-lavish-funds-lured-us-scientists-what-did-it-get-in-return}{%
\section{China's Lavish Funds Lured U.S. Scientists. What Did It Get in
Return?}\label{chinas-lavish-funds-lured-us-scientists-what-did-it-get-in-return}}

For years, China's Thousand Talents recruitment plan attracted U.S.
scientists with its grants. Investigators now say China used the program
to steal sensitive technology.

\includegraphics{https://static01.graylady3jvrrxbe.onion/images/2020/02/08/us/07CHINA-RECRUIT-lieber/merlin_168117990_890f1c3d-9b95-4836-8a8f-cb41e396085b-articleLarge.jpg?quality=75\&auto=webp\&disable=upscale}

\href{https://www.nytimes3xbfgragh.onion/by/ellen-barry}{\includegraphics{https://static01.graylady3jvrrxbe.onion/images/2018/10/08/multimedia/author-ellen-barry/author-ellen-barry-thumbLarge.png}}\href{https://www.nytimes3xbfgragh.onion/by/gina-kolata}{\includegraphics{https://static01.graylady3jvrrxbe.onion/images/2018/02/16/multimedia/author-gina-kolata/author-gina-kolata-thumbLarge.jpg}}

By \href{https://www.nytimes3xbfgragh.onion/by/ellen-barry}{Ellen Barry}
and \href{https://www.nytimes3xbfgragh.onion/by/gina-kolata}{Gina
Kolata}

\begin{itemize}
\item
  Published Feb. 6, 2020Updated Feb. 7, 2020
\item
  \begin{itemize}
  \item
  \item
  \item
  \item
  \item
  \item
  \end{itemize}
\end{itemize}

\href{https://cn.nytimes3xbfgragh.onion/usa/20200207/chinas-lavish-scientific-funds-fall-into-prosecutors-spotlight/}{阅读简体中文版}\href{https://cn.nytimes3xbfgragh.onion/usa/20200207/chinas-lavish-scientific-funds-fall-into-prosecutors-spotlight/zh-hant/}{閱讀繁體中文版}

More than a decade into his career as an organic chemist, Jon Antilla
found a solution to the grinding task of fund-raising that,
increasingly, was squeezing out his time in the laboratory.

Leaving a tenured position at the University of South Florida, he
relocated to Tianjin University in China, where he was awarded a grant
through a Chinese recruitment program, Thousand Talents.

He wasn't alone: Colleagues in Tianjin's chemistry department had given
up tenured positions at the University of California, San Diego, and
Texas A\&M, among other prestigious institutions, attracted by China's
readily available funding.

``We have time to think here,'' Dr. Antilla said. ``You can think about
your research.''

As Dr. Antilla proceeded with his academic career, United States
officials changed their view of China's recruitment programs, which they
say have been used to steal sensitive technology from American
laboratories.

In 2019, the Department of Energy
\href{https://physicstoday.scitation.org/do/10.1063/PT.6.2.20190620a/full/}{barred
its personnel from participating in recruitment programs} from a handful
of countries, including China. A few months later, a Senate committee
declared China's recruitment programs
\href{https://www.hsgac.senate.gov/imo/media/doc/2019-11-18\%20PSI\%20Staff\%20Report\%20-\%20China's\%20Talent\%20Recruitment\%20Plans.pdf}{a
threat to American interests}.

Thousand Talents grantees have become a focus for law enforcement
authorities in the United States, tasked by the Justice Department
\href{https://www.nytimes3xbfgragh.onion/2019/11/04/health/china-nih-scientists.html}{with
rooting out scientists who are stealing research} from American
laboratories. Dr. Antilla, like the vast majority of grantees, is not
under suspicion.

Last week,
\href{https://www.nytimes3xbfgragh.onion/2020/01/28/us/charles-lieber-harvard.html}{federal
prosecutors charged Charles M. Lieber}, an acclaimed Harvard chemist
viewed by many as a future Nobel laureate, with lying to federal
authorities about his affiliation with Thousand Talents.

Andrew E. Lelling, the United States attorney for the District of
Massachusetts, described the program as ``a very carefully designed
effort by the Chinese government to fill what it views as its own
strategic gaps,'' including nanotechnology, Dr. Lieber's specialty.

When Dr. Lieber entered into cooperation with Chinese partners, he ``was
by definition conveying sensitive information to the Chinese,'' Mr.
Lelling said. ``The moment he works at Wuhan University of Technology
and conveys it to his Chinese counterparts, that research and expertise
is now at the disposal of the Chinese government, because that's how it
works in China.''

Dr. Lieber was charged with one felony count of lying to federal
investigators. He has not entered a plea or responded publicly to the
charge. Peter K. Levitt, Dr. Lieber's attorney, declined to comment for
this article.

Security analysts are now scrutinizing a range of Chinese talent
programs and the foreign scientists who have applied to them.

``One question would be, is this a bug, or a feature of these programs,
to have a link to espionage?'' said Elsa B. Kania, an
\href{https://www.cnas.org/people/elsa-b-kania}{adjunct senior fellow}
in the Technology and National Security Program at the Center for a New
American Security. She said she hoped the response by the United States
would be ``surgical.''

``It is important to course-correct where some of these activities and
behaviors are problematic, or even egregious, without causing collateral
damage to this critical landscape of global research and innovation,''
she said.

A spokeswoman for China's embassy in Washington, Fang Hong, said the
Thousand Talents program was similar to the recruitment programs of
other countries, intended to promote international cooperation in
science.

``The Chinese government firmly opposes any breach of scientific
integrity and ethics,'' Ms. Fang said. Violations uncovered by the
United States government, she said, reflected the actions of individual
scientists, not the Chinese government.

``It is extremely irresponsible and ill intentioned to link individual
behaviors to China's talent plan,'' she said.

A more emotional response came from Rao Yi, a Chinese neurobiologist
who, after 22 years in the United States, returned to China and said he
was one of the scholars who proposed the Thousand Talents program.

``After decades of brain drains of scientist from China, Chinese economy
can now afford to recruit scientists and support them to carry out basic
science which will be of benefit to mankind,'' Dr. Rao, the director of
the Chinese Institute for Brain Research, said in a written response to
questions.

Allegations that the program was used to steal intellectual property
were ``a fat lie,'' he said.

\includegraphics{https://static01.graylady3jvrrxbe.onion/images/2020/02/08/us/00CHINA-RECRUIT-Rao/merlin_168341136_51826def-2b06-4ac8-b1cc-7d59deaf454c-articleLarge.jpg?quality=75\&auto=webp\&disable=upscale}

When the Thousand Talents recruitment program began in 2008, aiming to
entice Chinese scientists overseas to bring their research back to
China, it hardly raised an eyebrow.

Many scientists were recruited to the talent programs, enticed by
starting salaries that can be as much as three or four times their
existing salaries. More than 10,000 joined, according to William Hannas,
who was a member of the Senior Intelligence Service at the Central
Intelligence Agency.

Dr. Hannas, now lead analyst at the Center for Security and Emerging
Technology at Georgetown University, is the co-author of a forthcoming
book on what he described as China's ``informal'' technology transfers.

Not all talent program scientists were at universities. About 300 were
government scientists and about 600 worked for United States
corporations. A quarter were with biotech firms, according to James
Mulvenon, director for intelligence integration at SOS International, a
private defense contractor. He is a Chinese linguist, and the co-author,
with Dr. Hannas, of the book on China's technology transfers.

Until recently, much about the Thousand Talents program was public.

Universities mostly steered clear of investigating researchers, worried
about being accused of racial profiling and threatening academic
freedom, Dr. Mulvenon said.

Even stipulations in some talent program contracts, like ones requiring
that China own intellectual property, did not raise alarms.

In 2018, Jeff Sessions, then the attorney general, announced
\href{https://www.justice.gov/opa/speech/file/1107256/download}{a
``China initiative''} after several cases came to light of scientists
illicitly providing China with technology and research findings paid for
by federal agencies. Its goal was to increase the investigation and
prosecution of Chinese economic espionage-related crimes in the United
States. China soon removed the list of Thousand Talents members from the
internet. It recently changed the program's name to National High-end
Foreign Experts Recruitment Plan.

There was a complication for the United States investigations, though:
In academia, with its tradition of exchange and openness, researchers
from China were integrated into American labs. Scientists are allowed to
collaborate and give seminars in other countries. Most universities have
loose regulations about outside employment and income. Until very
recently, most federal agencies had no regulations prohibiting employees
from belonging to Chinese talent groups.

Suppose, Dr. Hannas said, a scientist in a university lab or a high-tech
company learned valuable techniques. And suppose that scientist then
signed a contract to work in a Chinese lab on similar technology.

``Is it illegal? Probably not,'' he said. ``Is it unethical? Hell,
yes.''

Image

The Department of Chemistry and Chemical Biology at Harvard
University.~Credit...Katherine Taylor/Reuters

So far, Dr. Mulvenon said, prosecutors have focused on
\href{https://www.nytimes3xbfgragh.onion/2019/11/04/health/china-nih-scientists.html?searchResultPosition=1}{discrete
violations} instead of arguing that the programs can be a form of
espionage. They have accused researchers of transgressions such as not
revealing large payments and research funds from China when federal
grants require them to disclose outside funding.

Among the cases brought by federal prosecutors last year was that of You
Xiaorong, a researcher who left a job in Atlanta in which she researched
BPA-free coatings for beverage cans used by the Coca-Cola Company. The
indictment
\href{https://www.justice.gov/opa/pr/one-american-and-one-chinese-national-indicted-tennessee-conspiracy-commit-theft-trade}{asserts
that Dr. You}, who is known as Shannon, was offered a Thousand Talents
award ``based on the secrets she stole.'' She is alleged to have
\href{https://www.usnews.com/news/best-states/tennessee/articles/2019-12-23/feds-chemical-engineer-at-center-of-industrial-espionage}{transferred
trade secrets worth \$120 million}, uploading files to her Google drive
and taking photographs of industrial laboratory equipment. **** Ms. You
has pleaded not guilty.

Many investigations of academics are still in progress, but some
accusations have been made public. Officials at M.D. Anderson Cancer
Center in Houston and at the National Institutes of Health, for example,
found emails and documents that they say revealed
\href{https://www.nytimes3xbfgragh.onion/2019/11/04/health/china-nih-scientists.html?searchResultPosition=2}{flagrant
violations} by some academic scientists, such as grant reviewers who
sent confidential grant proposals detailing research plans to colleagues
in China. In other cases, the N.I.H. said, researchers were getting
patents
\href{https://www.justice.gov/opa/pr/couple-who-worked-local-research-institute-10-years-charged-stealing-trade-secrets-wire-fraud}{or
starting companies in China} based on research carried out at a United
States university with support from the federal government.

Dr. Mulvenon said explicit transgressions are thought to be only a part
of the problem of keeping valuable United States technology safe.

It is, he said, a gray area. ``It's not a case of them stealing
technology by infiltrating a computer,'' he said ``What we are seeing is
them relieving the rest of the world of technology by means that are not
necessarily illegal.''

He added, ``We can't say, `We don't want to do research with the
Chinese.'''

Peter Zeidenberg, a lawyer who is representing two dozen Chinese and
Chinese-American scientists who are under investigation, noted that
prosecutors in most of the cases have not alleged any technology
transfers, and were focused instead on the scientists' failure to
disclose grants.

``They're taking an unbelievably heavy-handed approach to this,'' said
Mr. Zeidenberg, a partner in the Washington, D.C., firm Arent Fox.
``There is no compliance training on these forms. It's just a form you
get every year. Until very recently, nobody paid any attention this
stuff. Now they're cracking the whip and they're treating these people
like felons.''

Scores of Western scientists have applied for Thousand Talents grants
over the years. In interviews, several described their decision as
straightforward: China had money available.

``When you get above a postdoc level, everybody applies to every country
in the world,'' said Tim Byrnes, an
\href{https://shanghai.nyu.edu/academics/faculty/directory/tim-byrnes}{assistant
professor of physics} at N.Y.U.'s campus in Shanghai, who received a
Thousand Talents grant in 2016. ``People are crossing borders all the
time, and eventually congregate where there is the most money. China is
now pouring huge amounts of money for research.''

Like the other grantees who were interviewed, Dr. Byrnes said he had
never had to submit any reports on his research to the Chinese
government, and that all his research is published in academic journals.
He said Thousand Talents was similar to other international grant
programs.

``It's only because it's from China that everybody has this kind of
opinion of it,'' he said.

Already, he said, any affiliation with the program is damaging the
prospects of scientists seeking funding from the United States
government. ``If I write this information down, immediately the
probability of the success of this or that grant sort of plummets,'' he
said.

Dr. Antilla, the organic chemist who relocated to Tianjin, said there
were indeed ethical minefields for recipients holding positions in China
and the United States, and that he eventually decided to move his work
entirely to China to avoid them.

Particularly thorny, he said, is the practice of maintaining
laboratories in both China and the United States.

``There are questions about intellectual property --- how do you share
data, if you share it at all,'' said Mr. Antilla, now a professor of
chemistry at Zhejiang Sci-Tech University. ``It could be tricky. What
are the rules?''

But he said he was always scrupulous about reporting the money he
received to his employers in the United States. It worried him, he said,
that Thousand Talents was ``getting a bad name.''

``Basically, I think my science is for the world,'' he said. ``There's
nothing that China is getting from my science that they're keeping from
the world. I publish everything I get.''

Advertisement

\protect\hyperlink{after-bottom}{Continue reading the main story}

\hypertarget{site-index}{%
\subsection{Site Index}\label{site-index}}

\hypertarget{site-information-navigation}{%
\subsection{Site Information
Navigation}\label{site-information-navigation}}

\begin{itemize}
\tightlist
\item
  \href{https://help.nytimes3xbfgragh.onion/hc/en-us/articles/115014792127-Copyright-notice}{©~2020~The
  New York Times Company}
\end{itemize}

\begin{itemize}
\tightlist
\item
  \href{https://www.nytco.com/}{NYTCo}
\item
  \href{https://help.nytimes3xbfgragh.onion/hc/en-us/articles/115015385887-Contact-Us}{Contact
  Us}
\item
  \href{https://www.nytco.com/careers/}{Work with us}
\item
  \href{https://nytmediakit.com/}{Advertise}
\item
  \href{http://www.tbrandstudio.com/}{T Brand Studio}
\item
  \href{https://www.nytimes3xbfgragh.onion/privacy/cookie-policy\#how-do-i-manage-trackers}{Your
  Ad Choices}
\item
  \href{https://www.nytimes3xbfgragh.onion/privacy}{Privacy}
\item
  \href{https://help.nytimes3xbfgragh.onion/hc/en-us/articles/115014893428-Terms-of-service}{Terms
  of Service}
\item
  \href{https://help.nytimes3xbfgragh.onion/hc/en-us/articles/115014893968-Terms-of-sale}{Terms
  of Sale}
\item
  \href{https://spiderbites.nytimes3xbfgragh.onion}{Site Map}
\item
  \href{https://help.nytimes3xbfgragh.onion/hc/en-us}{Help}
\item
  \href{https://www.nytimes3xbfgragh.onion/subscription?campaignId=37WXW}{Subscriptions}
\end{itemize}
