Sections

SEARCH

\protect\hyperlink{site-content}{Skip to
content}\protect\hyperlink{site-index}{Skip to site index}

\href{https://www.nytimes3xbfgragh.onion/section/business/economy}{Economy}

\href{https://myaccount.nytimes3xbfgragh.onion/auth/login?response_type=cookie\&client_id=vi}{}

\href{https://www.nytimes3xbfgragh.onion/section/todayspaper}{Today's
Paper}

\href{/section/business/economy}{Economy}\textbar{}Trump's Budget Math
Grapples With Economic Reality

\url{https://nyti.ms/31Nk5z5}

\begin{itemize}
\item
\item
\item
\item
\item
\item
\end{itemize}

\begin{itemize}
\item
  \href{https://www.nytimes3xbfgragh.onion/interactive/2020/09/08/us/elections/results-new-hampshire-primary-elections.html?action=click\&pgtype=Article\&state=default\&region=TOP_BANNER\&context=storylines_menu}{New
  Hampshire Results}
\item
  \href{https://www.nytimes3xbfgragh.onion/live/2020/09/08/us/trump-vs-biden?action=click\&pgtype=Article\&state=default\&region=TOP_BANNER\&context=storylines_menu}{Election
  Updates}
\item
  \href{https://www.nytimes3xbfgragh.onion/interactive/2020/us/elections/election-states-biden-trump.html?action=click\&pgtype=Article\&state=default\&region=TOP_BANNER\&context=storylines_menu}{Paths
  to 270}
\item
  \href{https://www.nytimes3xbfgragh.onion/interactive/2020/08/31/us/politics/vote-by-mail-deadlines.html?action=click\&pgtype=Article\&state=default\&region=TOP_BANNER\&context=storylines_menu}{Voting
  by Mail}
\item
  \href{https://www.nytimes3xbfgragh.onion/interactive/2019/us/elections/2020-presidential-election-calendar.html?action=click\&pgtype=Article\&state=default\&region=TOP_BANNER\&context=storylines_menu}{Key
  Dates}
\item
  \href{https://www.nytimes3xbfgragh.onion/newsletters/politics?action=click\&pgtype=Article\&state=default\&region=TOP_BANNER\&context=storylines_menu}{Politics
  Newsletter}
\end{itemize}

Advertisement

\protect\hyperlink{after-top}{Continue reading the main story}

Supported by

\protect\hyperlink{after-sponsor}{Continue reading the main story}

\hypertarget{trumps-budget-math-grapples-with-economic-reality}{%
\section{Trump's Budget Math Grapples With Economic
Reality}\label{trumps-budget-math-grapples-with-economic-reality}}

The budget predicts the economy will grow significantly faster than most
economists anticipate.

\includegraphics{https://static01.graylady3jvrrxbe.onion/images/2020/02/10/us/politics/10dc-budgetmath/merlin_168720684_d16708d2-5039-4a6b-ad57-c5c4120b3837-articleLarge.jpg?quality=75\&auto=webp\&disable=upscale}

\href{https://www.nytimes3xbfgragh.onion/by/jim-tankersley}{\includegraphics{https://static01.graylady3jvrrxbe.onion/images/2018/10/19/multimedia/author-jim-tankersley/author-jim-tankersley-thumbLarge.png}}

By \href{https://www.nytimes3xbfgragh.onion/by/jim-tankersley}{Jim
Tankersley}

\begin{itemize}
\item
  Feb. 10, 2020
\item
  \begin{itemize}
  \item
  \item
  \item
  \item
  \item
  \item
  \end{itemize}
\end{itemize}

WASHINGTON --- President Trump's budget proposals have been defined by a
belief that the economy will grow significantly faster than most
economists anticipate. The
\href{https://www.nytimes3xbfgragh.onion/2020/02/09/us/politics/trump-border-wall-budget.html}{latest
version}, released on Monday, is a brief departure: It concedes, for the
first time, that the administration's past projections were too
optimistic.

Then it goes right back to forecasting 3 percent growth, for the better
part of a decade.

Mr. Trump's \$4.8 trillion budget proposal is
\href{https://www.nytimes3xbfgragh.onion/2019/03/11/us/politics/trump-budget.html}{slightly
larger than last year's} \$4.75 trillion request and calls for increased
spending on the military, the border wall, infrastructure and other
priorities, including extending the president's 2017 tax cuts. It also
includes trillions of dollars of cuts to safety-net programs like
Medicaid and discretionary spending programs outside of the military,
like education and the environment.

The White House makes the case that this is affordable and that the
deficit will start to fall, dropping below \$1 trillion in the 2021
fiscal year, and that the budget will be balanced by 2035. That
projection relies on rosy assumptions about growth and the accumulation
of new federal debt --- both areas where the administration's past
predictions have proved to be overconfident.

The new budget forecasts a growth rate for the United States economy of
2.8 percent this year --- or, by the metric the administration prefers
to cite, a 3.1 percent rate. That is more than a half percentage point
higher than forecasters at the Federal Reserve and the Congressional
Budget Office predict.

It then predicts growth above 3 percent annually for the next several
years if the administration's economic policies are enacted. The Fed,
the budget office and others all see growth falling below 2 percent
annually in that time. By 2030, the administration predicts the economy
will be more than 15 percent larger than forecasters at the budget
office do.

Past administrations have also dressed up their budget forecasts with
economic projections that proved far too good to be true. In its fiscal
year
\href{https://www.govinfo.gov/content/pkg/BUDGET-2011-BUD/pdf/BUDGET-2011-BUD.pdf}{2011
budget}, for example, the Obama administration predicted several years
of growth topping 4 percent in the aftermath of the 2008 financial
crisis --- a number it never came close to reaching even once.

Trump officials had considered their projections to be a break from that
trend,
\href{https://www.whitehouse.gov/articles/administration-growth-forecasts-historical-perspective/}{writing
last year} that they were the first administration on record ``to have
experienced economic growth that meets or exceeds its own forecasts in
each of its first two years in office.'' That turned out to be wrong: In
the middle of last year, the Commerce Department
\href{https://www.nytimes3xbfgragh.onion/2019/08/21/business/economy/jobs-growth-revision.html}{revised
its accounting of the 2018 growth rate downward}, to well below the rate
Trump officials had forecast. Their predictions were similarly off in
2019.

Robust economic growth rates are not the only area where the
administration's renewed optimism appears in its latest budget. It has
also revised down its estimate of the interest the federal government
would pay to borrow money over the next decade, based largely on the
assumption that the Fed, which began cutting rates in 2019, would raise
them only modestly again over the next 10 years. The changes in rate
assumptions reduce budget deficits by \$1.5 trillion over the course of
the decade, according to the administration's projections.

Essentially, administration officials are contending that rising levels
of debt in the United States will not drive up borrowing costs, as many
conservative economists have long warned, at least for the next several
years. They also believe, a rarity among economists, that a sustained
stretch of 3 percent growth would not push the Fed to raise interest
rates. Administration officials do not directly control Fed policy, but
in a companion document to the budget, the officials wrote that federal
borrowing costs would stay low ``as U.S. debt continues to be in high
demand because it is a safe haven for savings amidst global turmoil.''

As a result, the administration sees federal debt held by the public ---
the national debt, essentially --- declining from 79 percent of the
overall economy this year to 66 percent in 2030. The budget office
\href{https://www.cbo.gov/publication/56073}{sees it rising, to 98
percent}, a level not reached since 1946.

In order to justify that optimism, administration officials are
contending that their overly optimistic growth forecasts of the past
were a fluke of circumstance.

Mr. Trump's first budget,
\href{https://www.govinfo.gov/content/pkg/BUDGET-2018-BUD/pdf/BUDGET-2018-BUD.pdf}{in
the spring of 2017}, predicted growth of 2.3 percent that year using the
administration's preferred measure --- the change in the size of the
economy from the fourth quarter of the preceding year. It was a mild
undershoot; growth actually hit 2.5 percent.

The next two budgets
\href{https://www.whitehouse.gov/wp-content/uploads/2018/02/budget-fy2019.pdf}{predicted
3.1 percent growth} for 2018 and
\href{https://www.whitehouse.gov/wp-content/uploads/2019/03/budget-fy2020.pdf}{3.2
percent} for 2019. Both were off, badly. Growth was 2.5 percent in 2018,
from fourth quarter to fourth quarter, and 2.3 percent in 2019,
\href{https://www.bea.gov/news/2020/gross-domestic-product-fourth-quarter-and-year-2019-advance-estimate}{according
to the Commerce Department}.

Officials on Sunday attributed a half-point of the missed forecast last
year to the
\href{https://www.nytimes3xbfgragh.onion/2020/01/06/business/economy/trade-war-tariffs.html}{effects
of American trade policy} --- specifically, uncertainty over resolution
of trade talks with China and congressional approval of a new trade
agreement with Canada and Mexico. They said that those uncertainties
were now resolved and that growth would accelerate accordingly.

The senior administration official also said a General Motors strike,
aerospace giant Boeing's struggles with its 737 Max aircraft and
flooding in the Midwest had reduced growth by an additional three-tenths
of a percentage point last year.

Mr. Trump has long asserted that his push to negotiate with the Chinese
and reopen North American trade talks were helping the economy. In the
2016 campaign, his advisers said tariffs on Chinese imports --- even
more aggressive levies than what Mr. Trump ultimately imposed on Beijing
--- would increase growth, by pushing multinational companies to invest
in the United States instead of China.

Such an investment wave never materialized. Capital spending growth
turned negative for the last three quarters of 2019. Many forecasters
believe that decline was trade related; the budget office, among others,
is predicting a bounce-back in investment growth this year. But those
forecasters also see growth slowing, over all, as the stimulus fades
from Mr. Trump's deficit-swelling tax cuts in 2017 and spending
increases he has signed each year in office.

Partly as a result of those measures, and the administration's inability
to interest Congress in any of its most aggressive proposals for cuts,
the federal budget deficit was nearly twice as large last year as the
administration projected in its first budget: It topped \$1 trillion.
The Congressional Budget Office predicts it will continue to grow,
hitting \$1.3 trillion in 2025 as growth slows to 1.5 percent.

For that same year, the new Trump budget predicts that the deficit will
be less than half the size --- and that growth will be just under 3
percent.

A companion document to the budget includes an alternate forecast for
the full decade --- with growth much closer to, but still higher than,
independent economists' projections --- in the event that the
administration's proposed extensions of tax cuts, continued
deregulation, reductions in certain spending programs and other policy
changes are not enacted.

It notes one reason, in particular, that those efforts might not
succeeded this year: ``2020 is an election year,'' the officials write,
``and there is the risk that this will distract from implementation of
the necessary policies required for continued increases in prosperity.''

\hypertarget{our-2020-election-guide}{%
\section{Our 2020 Election Guide}\label{our-2020-election-guide}}

Updated ~Sept. 8, 2020

\begin{center}\rule{0.5\linewidth}{\linethickness}\end{center}

\begin{itemize}
\item ~
  \hypertarget{the-latest}{%
  \subsection{The Latest}\label{the-latest}}

  \begin{itemize}
  \item
    President Trump and his party are using a playbook that aims to
    alarm people about crime in their backyards. It didn't work in 2018,
    but
    \href{https://www.nytimes3xbfgragh.onion/2020/09/08/us/politics/trump-republicans-fear-strategy.html?action=click\&pgtype=Article\&state=default\&region=BELOW_MAIN_CONTENT\&context=storylines_guide}{both
    parties think it could resonate more this year}.
  \end{itemize}
\item ~
  \hypertarget{how-to-win-270}{%
  \subsection{How to Win 270}\label{how-to-win-270}}

  \begin{itemize}
  \item
    Joe Biden and Donald Trump need 270 electoral votes to reach the
    White House. Try building
    \href{https://www.nytimes3xbfgragh.onion/interactive/2020/us/elections/election-states-biden-trump.html?action=click\&pgtype=Article\&state=default\&region=BELOW_MAIN_CONTENT\&context=storylines_guide}{your
    own coalition of battleground states}~to see potential outcomes.
  \end{itemize}
\item ~
  \hypertarget{voting-by-mail}{%
  \subsection{Voting by Mail}\label{voting-by-mail}}

  \begin{itemize}
  \item
    Will you have enough time to vote by mail in your state? Yes, but
    it's risky to procrastinate.
    \href{https://www.nytimes3xbfgragh.onion/interactive/2020/08/31/us/politics/vote-by-mail-deadlines.html?action=click\&pgtype=Article\&state=default\&region=BELOW_MAIN_CONTENT\&context=storylines_guide}{Check
    your state's deadline.}
  \item
    \href{https://www.nytimes3xbfgragh.onion/interactive/2020/us/elections/joe-biden.html?action=click\&pgtype=Article\&state=default\&region=BELOW_MAIN_CONTENT\&context=storylines_guide}{}

    \hypertarget{joe-biden}{%
    \section{Joe Biden}\label{joe-biden}}

    \hypertarget{democrat}{%
    \subsection{Democrat}\label{democrat}}

    \href{https://www.nytimes3xbfgragh.onion/interactive/2020/us/elections/donald-trump.html?action=click\&pgtype=Article\&state=default\&region=BELOW_MAIN_CONTENT\&context=storylines_guide}{}

    \hypertarget{donald-trump}{%
    \section{Donald Trump}\label{donald-trump}}

    \hypertarget{republican}{%
    \subsection{Republican}\label{republican}}
  \end{itemize}
\item
  \hypertarget{keep-up-with-our-coverage}{%
  \subsection{Keep Up With Our
  Coverage}\label{keep-up-with-our-coverage}}

  \begin{itemize}
  \item
    Get an
    \href{https://www.nytimes3xbfgragh.onion/newsletters/politics?action=click\&pgtype=Article\&state=default\&region=BELOW_MAIN_CONTENT\&context=storylines_guide}{email}~recapping
    the day's news
  \item
    Download our mobile app on
    \href{https://apps.apple.com/us/app/nytimes/id284862083?ls=1\&mat_click_id=5c79ae7455014fd1bd66b5610c05b8f2-20191112-16948\&referrer=mat_click_id\%3D5c79ae7455014fd1bd66b5610c05b8f2-20191112-16948\%26link_click_id\%3D722930677036718082}{iOS}~and
    \href{http://a.localytics.com/android?id=com.nytimes.android\&referrer=utm_source\%3Dother_nyt_mobile_web\%26utm_medium\%3DWeb\%2520page\%26utm_term\%3DGeneral\%2520Mobile\%2520Page\%26utm_campaign\%3DNYT\%2520Mobile\%2520General\%2520Page}{Android}~and
    turn on Breaking News and Politics alerts
  \end{itemize}
\end{itemize}

Advertisement

\protect\hyperlink{after-bottom}{Continue reading the main story}

\hypertarget{site-index}{%
\subsection{Site Index}\label{site-index}}

\hypertarget{site-information-navigation}{%
\subsection{Site Information
Navigation}\label{site-information-navigation}}

\begin{itemize}
\tightlist
\item
  \href{https://help.nytimes3xbfgragh.onion/hc/en-us/articles/115014792127-Copyright-notice}{©~2020~The
  New York Times Company}
\end{itemize}

\begin{itemize}
\tightlist
\item
  \href{https://www.nytco.com/}{NYTCo}
\item
  \href{https://help.nytimes3xbfgragh.onion/hc/en-us/articles/115015385887-Contact-Us}{Contact
  Us}
\item
  \href{https://www.nytco.com/careers/}{Work with us}
\item
  \href{https://nytmediakit.com/}{Advertise}
\item
  \href{http://www.tbrandstudio.com/}{T Brand Studio}
\item
  \href{https://www.nytimes3xbfgragh.onion/privacy/cookie-policy\#how-do-i-manage-trackers}{Your
  Ad Choices}
\item
  \href{https://www.nytimes3xbfgragh.onion/privacy}{Privacy}
\item
  \href{https://help.nytimes3xbfgragh.onion/hc/en-us/articles/115014893428-Terms-of-service}{Terms
  of Service}
\item
  \href{https://help.nytimes3xbfgragh.onion/hc/en-us/articles/115014893968-Terms-of-sale}{Terms
  of Sale}
\item
  \href{https://spiderbites.nytimes3xbfgragh.onion}{Site Map}
\item
  \href{https://help.nytimes3xbfgragh.onion/hc/en-us}{Help}
\item
  \href{https://www.nytimes3xbfgragh.onion/subscription?campaignId=37WXW}{Subscriptions}
\end{itemize}
