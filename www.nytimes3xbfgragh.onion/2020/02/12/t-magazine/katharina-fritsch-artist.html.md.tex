A Sculptor of the Female Gaze

\url{https://nyti.ms/2HmsIqM}

\begin{itemize}
\item
\item
\item
\item
\item
\end{itemize}

\includegraphics{https://static01.graylady3jvrrxbe.onion/images/2020/02/11/t-magazine/11tmag-katharinafritsch-slide-499T/11tmag-katharinafritsch-slide-499T-articleLarge.jpg?quality=75\&auto=webp\&disable=upscale}

Sections

\protect\hyperlink{site-content}{Skip to
content}\protect\hyperlink{site-index}{Skip to site index}

\hypertarget{a-sculptor-of-the-female-gaze}{%
\section{A Sculptor of the Female
Gaze}\label{a-sculptor-of-the-female-gaze}}

Katharina Fritsch shows familiar objects as they might appear in a
dream, bringing the subliminal to light.

Katharina Fritsch, photographed last fall in her Düsseldorf, Germany,
studio with an in-progress sculpture.Credit...Bernhard Fuchs

Supported by

\protect\hyperlink{after-sponsor}{Continue reading the main story}

By \href{https://www.nytimes3xbfgragh.onion/by/megan-o-grady}{Megan
O'Grady}

\begin{itemize}
\item
  Published Feb. 12, 2020Updated Feb. 13, 2020
\item
  \begin{itemize}
  \item
  \item
  \item
  \item
  \item
  \end{itemize}
\end{itemize}

CONSIDER THE ROOSTER. The cockerel --- vigilant herald of sunrise,
barnyard strutter --- has a long iconographic history, appearing on
things like weather vanes and churches (as an emblem of St. Peter) and
French soccer jerseys (as \emph{le coq gaulois}, the unofficial national
mascot). In the Chinese zodiac, the rooster symbolizes honesty, fidelity
and protection. In art history's vast bestiary, the rooster appears most
famously in
\href{https://www.nytimes3xbfgragh.onion/2019/06/11/t-magazine/francoise-gilot-picasso.html}{Pablo
Picasso}'s 1938 ``Le Coq,'' its rainbow-colored strokes of pastel
expressing the chicken's movements, its irascibility and (fittingly, for
the artist) its virility.

\href{https://www.matthewmarks.com/new-york/artists/katharina-fritsch/}{Katharina
Fritsch}'s rooster is above all that. Over 14 feet high, with luxuriant
plumage a shade of ultramarine blue Yves Klein might have envied, the
polyester-and-fiberglass sculpture
\href{https://artsbeat.blogs.nytimes3xbfgragh.onion/2014/02/07/coming-to-trafalgar-square-an-unusual-horse-and-a-giant-thumb/}{could
be found} in London's Trafalgar Square, perched high on the square's
fourth plinth for the nearly two years it was there (it is now at the
National Gallery of Art in Washington, D.C.), piquant company for the
traditional statues of self-serious heroes of history --- King George
IV, Maj. Gen. Sir Henry Havelock and Gen. Sir Charles James Napier, who
occupy the other three. (A second rooster is
\href{https://www.nytimes3xbfgragh.onion/2016/01/22/arts/design/the-new-cherries-on-top-of-the-minneapolis-sculpture-garden.html}{in
the sculpture garden of the Walker Art Center} in Minneapolis; a third
will be shown this month at Matthew Marks Gallery in Los Angeles,
accompanied by two other sculptures.) When the Trafalgar Square rooster
was unveiled in 2013, then mayor Boris Johnson noted the irony that an
unofficial emblem of France had taken roost in a place commemorating a
British victory in the Napoleonic Wars. Fritsch's cock, however, knows
no nation. ``The French think it's their rooster; the Minnesotans think
it's their rooster. It's everyone's rooster,'' she says with equanimity.
Detached from its expected scale, context or hue --- here is a chicken,
it is blue --- the animal seems to have flown in through a rift in the
cosmic fabric, evidence of a sprightlier, less pedantic universe.

The dream life of things --- animals of all kinds, but also lanterns and
shells, strawberries and umbrellas, figures of saints and the Madonna
--- are what preoccupy Fritsch, a German sculptor famous for her eerily
smooth, outsize polyester-and-fiberglass sculptures in bright, matte,
addictive colors. All of us bring a set of private associations to our
surroundings, and Fritsch's work operates upon and expands this
relationship, revising reality just enough to unsettle us and make the
subliminal feel real and graspable and even weirdly covetable. The
initial visual startle of her work quickly becomes subcutaneous in
feeling: the realm of fantasy and superstition. Much of her work plays
with recognizable imagery --- especially that of Catholicism and the
Brothers Grimm --- but presents it as if pulled from some
half-remembered illusion. Some of her early work is more overtly about
subconscious fear, such as her 1993 sculpture ``Rattenkönig
(Rat‑King),'' a circle of 16 rats over nine feet tall with a knot of
entangled tails, which enlarges a spooky motif to its symbolic
proportions.

\includegraphics{https://static01.graylady3jvrrxbe.onion/images/2020/02/11/t-magazine/11tmag-katharinafritsch-slide-LK6D/11tmag-katharinafritsch-slide-LK6D-articleLarge.jpg?quality=75\&auto=webp\&disable=upscale}

What does it mean to see our fears and dreams take up physical space?
Fritsch's major 1988 work, ``Tischgesellschaft (Company at Table),''
features 32 blankly impassive, seated men, a nightmare vision of
``identity dissolving in an infinite space,'' as the artist described it
in 2001, or what it might look like if all of my exes were invited to
the same dinner party. As one draws closer, the men turn out to be all
the same man: her boyfriend at the time, Frank Fenstermacher, of the
German new wave band \href{http://www.bureau-b.com/plan.php}{Der Plan}.
Since then, Fritsch's oeuvre has expanded to increasingly ambiguous
tableaus. In the Museum of Modern Art sculpture garden in 2011, she
placed a
\href{https://www.moma.org/explore/inside_out/2011/06/16/katharina-fritsch-in-moma-s-garden/}{set
of stylized figures}, including a 5-foot-7 cadmium yellow Madonna; a
trio of saints in cobalt violet, green and black; and a giant gray
primeval man with a club. A black snake slithers in front of them. The
piece is indicative of Fritsch's larger role as an artist: This is
sculpture not just as allegory but as performance, almost a kind of
postmodern stand-up --- and a potent exercise in what
\href{https://www.nytimes3xbfgragh.onion/topic/person/susan-sontag}{Susan
Sontag} called ``radical juxtaposition,'' surrounded, as it is, by works
from the more famous men of sculpture, such as
\href{https://www.nytimes3xbfgragh.onion/2012/06/26/arts/26iht-moore26.html}{Henry
Moore},
\href{https://www.nytimes3xbfgragh.onion/topic/person/auguste-rodin}{Auguste
Rodin} and Picasso.

It can be difficult to locate what it is Fritsch's sculptures are trying
to say, exactly --- but this isn't a criticism. They seem familiar ---
the rats and succulent-looking fruit plucked from a long-lost fairy
tale, the fluorescent Madonna and skulls pulled from an obscure passage
of the New Testament --- and yet the pieces refuse to supply an
identifiable critique of or statement about the tropes we are so used to
seeing contemporary art address: consumerism, gender and racial
identity. (They certainly spark a certain desire to have them or to be
near them that seems intentional --- the weird, product-like quality a
strawberry might attain when enlarged, cushily recumbent and colored
blue.) But in their unknowability, in making us search for answers again
and again to no avail, Fritsch has created a remarkable and unique body
of work. It provokes sensations of nameless dread or desire rather than
a clear reaction, a kind of working lexicon not of the things that haunt
us but rather of what it is like to feel haunted.

IN THE DAYS before I met Fritsch in her studio in Düsseldorf, Germany,
last fall, as the artist was preparing for her show at Matthew Marks
Gallery, one of her animals in particular troubled me: the poodle.
Popularized in part by \href{https://www.albrecht-durer.org/}{Albrecht
Dürer} and
\href{https://www.nytimes3xbfgragh.onion/topic/person/francisco-de-goya}{Francisco
Goya}, who featured them in their paintings, the breed became the dog of
choice for early 19th-century French prostitutes and later a fad among
teenage girls of the 1950s, who put poodle appliqués on their circle
skirts. When I lived in Berlin in 2010, standard poodles had become
ironic pets among a certain arty crowd, disturbing in the way that only
a living creature employed as a fashion accessory can be. In 1996,
Fritsch completed
``\href{https://www.sfmoma.org/artwork/96.490.a-rrrrrrrrr/}{Kind Mit
Pudeln (Child With Poodles)},'' in which four concentric circles of dogs
surround a Christlike infant. The absurdity of the animal itself, with
its kitschy pompoms, contrasts neatly with their menacing arrangement,
which calls to mind the orgy scene from
``\href{https://www.imdb.com/title/tt0120663/}{Eyes Wide Shut},'' with a
hint of the final moment in
``\href{https://www.imdb.com/title/tt0063522/}{Rosemary's Baby},'' when
the coven converges on the cradle.

``I hate poodles, I must say,'' Fritsch says over breakfast at her
studio, a vast skylit space not far from a large park that was once home
to Düsseldorf's zoo, which was bombed in 1943. Her upstairs atelier
overlooks the rail yards. Fritsch is 64 but looks a decade younger; she
has a wonderfully mordant, expressive face and a brainy gameness, and is
wearing a beautiful shirt of creamy chamois yellow corduroy that once
belonged to her father, an architect. Two assistants, young men, say
hello; when I turn to greet a third, bent over a worktable, he turns out
to be a sculpture. ``Ideas emerge from my subconscious all the time,''
she explains, sometimes when she's in transit, in a car or on a train;
others originate in her sleep. ``I think everything can be a sculpture
for me. From the beginning, I wanted to create a kind of middle world
that took you behind the object again by yourself, a world that really
surprises people like they haven't seen the object before.''

Image

Fritsch's ``6. Stilleben (6th Still Life)'' (2011), which includes a
variety of Christian symbolism.Credit...Bronze, copper, epoxy and paint
© Katharina Fritsch/VG Bild-Kunst, Bonn/Courtesy of Matthew Marks
Gallery. Photo by Ivo Faber

Achieving this effect depends entirely on perfection of form. In the
two-and-a-half-year-long process of creating the rooster, Fritsch moved
the tail three times; the chest was especially difficult to get right,
as she didn't want it to resemble the proud chest of Germany's imperial
eagles, nor did she want ``a weak chicken.'' Since 2006, Fritsch has
used a computer at different stages in the development of her prototypes
--- scanning an object, making a plaster cast she then painstakingly
reshapes and remodels, then rescanning and reworking several times to
get the shape and detailing precise. To rely simply on a scan, she says,
results in work that is ``completely flat. I don't want to be
sentimental about this, but to me it has an effect. You lose this third
dimension and the sensuality of the materials, the smell and everything.
You need that.'' When I ask her how casting in polyester works, she
opens a can of the viscous stuff and shows it to me, inhaling. ``The
smell is amazing,'' she says.

In trying to pin her down on the various sources of her iconography, I
soon feel uncomfortably like a Jungian analyst. One of my favorite of
Fritsch's sculptures, ``Oktopus (Octopus)'' (2010), which features a
small deep-sea diver clutched in one of the creature's long orange arms,
has its origins in childhood fever dreams and Jules Verne, she tells me.
When Fritsch was a child, her father liked to tease her by whipping open
an antique encyclopedia to the page with a terrifyingly detailed octopus
illustration, but now she greatly admires and even identifies with the
intelligent animal. ``They are like artists, because they can change
their skin within seconds to reflect their environment. I think this is
so incredible,'' she says, explaining that when she embarks on an animal
sculpture, she first learns everything she can about it from books and
documentaries and even natural history experts. But creating an octopus
prototype proved to be a major design challenge. ``First, I tried to
make a scan of a real one --- we bought it from the fish shop --- but
you can't scan flesh because it's always moving. And so I had to be the
octopus. I \emph{was} the octopus. I was really feeling the movement,
and I knew it had to be like this,'' and here she imitates the ungainly
cephalopod's sideways slump, the extended arm, and all at once, I catch
a glimpse of how Fritsch transmits an abstract idea into form.

Fritsch mixes her own pigments; downstairs, there's an entire room for
spray-painting. She's secretive about exactly how she creates her
colors, which are brought to a paint factory to make an industrial
lacquer, but the color selection process is entirely intuitive --- ``I
visualize it immediately,'' she says. For decades now, she has worked
within a recognizable palette, one that might feel ironic in the hands
of another artist but here, applied to her identifiable yet enigmatic
imagery, feels more sinister: In addition to her iconic celestial blue
and a black so dense it seems to suck color from its surroundings, she
often uses cobalt violet, calamine pink, cadmium yellow and a particular
unearthly blue-green --- a color scheme reminiscent of Prada ads from
the mid-aughts. How completely a simple change of hue shifts our
perception, I realize as we flip through one of her catalogs together.

Part of Fritsch's genius is how her work seems to beg for
interpretation. Is her octopus a self-portrait, an earnest re-creation
of her girlhood nightmares or an attempt at taming those fears by making
the creature tenderly comic? The sculpture is sensual enough that I
can't help but identify with it; at the same time, I begin to imagine
what it might feel like to have one of those chubby arms hold me in its
grasp. This kind of ambivalence, the search for deeper meaning and its
almost inevitable unraveling through the sheer literalness of Fritsch's
creations --- her ``Rattenkönig'' really is just 16 rats in a circle ---
is part of the experience of viewing her work, which is confounding,
frustrating, funny and ultimately moving because of the search itself,
the matte porelessness that resists, refuses, interpretation. And yet
they are far too fine in their detail --- and too affecting --- to be
anything close to kitsch.

Her sculpture of a pale pink cowrie shell, for instance, over nine feet
tall and sweetly creepy, resembles a colossal \emph{vagina dentata}, I
unoriginally point out. ``You can see it like this. I see it as a
shell,'' she replies.

Image

Fritsch's ``Hahn (Cock)'' (2013), in London's Trafalgar
Square.Credit...Fiberglass, polyester resin, paint and stainless steel ©
Katharina Fritsch/VG Bild-Kunst, Bonn/Courtesy of Matthew Marks Gallery.
Photo by Ivo Faber

Image

One of Fritsch's most famous works, ``Tischgesellschaft (Company at
Table)'' (1988), which features 32 seated men.Credit...Polyester, wood,
cotton and paint © Katharina Fritsch/VG Bild-Kunst, Bonn/Courtesy of
Matthew Marks Gallery. Photo by Nic Tenwiggenhorn

``AT 5, IT WAS clear to me that I would be an artist,'' Fritsch tells me
over lunch at an Italian place in Oberkassel, a bourgeois neighborhood
on the other side of the Rhine where the experimental artist
\href{https://www.nytimes3xbfgragh.onion/1986/01/25/obituaries/joseph-beuys-sculptor-is-dead-at-64.html}{Joseph
Beuys} lived before his death in 1986. Fritsch's maternal grandfather
was a salesperson for Faber‑Castell, and his garage was filled,
tantalizingly, with art supplies. ``It was a paradise,'' she recalls.
``I was always fascinated by the pencils with all the colors.'' Growing
up in Langenberg in the 1950s and in Münster in the '60s, both near
working-class Essen, in the heart of the Ruhr valley, Germany's heavy
industry heartland, art wasn't an obvious career path. ``Maybe my
parents were secretly afraid of my never making any money, but they
really encouraged me to do that, to paint and to draw,'' she says. ``My
childhood was very sensual. It was a very artistic atmosphere.'' And a
little gothic: Fritsch kept her religious maternal grandmother company
on her many tours of German churches, including the famous 13th-century
crypts at Bamberg cathedral. ``It's very impressive when you go as a
child into the Catholic churches and you see these figures, and there's
something that's very cruel about what you see, and I was completely
attracted by that,'' she says. ``Bodies dangling from crosses and
skeletons in glass tombs?'' I ask. ``Yes,'' she laughs. ``You have
nightmares, but it's so impressive, so strong.'' At the same time,
American culture, its music and tacky consumer products, was conquering
West Germany. ``I was a big fan of Mickey Mouse and Barbie,'' she says.
``Some parents would never allow their children to have that, but my
parents or my grandparents, they were not so afraid of things like that.
We --- my friends and I --- all wanted to be more American.'' After her
application to the Münster Academy of Art was rejected, she instead
studied history and art history at the University of Münster. ``Art
history was terrible for me. It was dusty and lifeless. Art should be
alive,'' she says. The people at the Kunstakademie Düsseldorf, the
famous art school whose students had included Beuys, as well as
\href{https://www.nytimes3xbfgragh.onion/2011/10/29/arts/29iht-melikian29.html}{Gerhard
Richter} and
\href{https://tmagazine.blogs.nytimes3xbfgragh.onion/2013/05/03/artifacts-the-dinosaur-in-anselm-kiefer/}{Anselm
Kiefer}, ``seemed to be much cooler,'' she adds.

One night in 1978, Fritsch went to Düsseldorf to see a performance by
Beuys and the video art pioneer
\href{https://www.nytimes3xbfgragh.onion/2006/01/31/arts/design/nam-june-paik-73-dies-pioneer-of-video-art-whose-work-broke.html}{Nam
June Paik}, who, like Beuys, was teaching at the Kunstakademie at the
time. The occasion was a memorial tribute to
\href{https://www.nytimes3xbfgragh.onion/1978/05/11/archives/george-maciunas-artist-and-designer-organized-fluxus-to-develop.html}{George
Maciunas}, a leading figure of Fluxus, the multidisciplinary art
movement that fostered experimentation --- initially in the form of
radical performance --- while also stressing the value of art's role in
everyday life. ``It was something,'' she recalls. ``We went there in a
little car with six people and the area around the Kunstakademie was
pretty crowded. It was this new wave and punk thing that was going on
there.'' Carmen Knoebel, who was married to the artist
\href{http://www.artnet.com/artists/imi-knoebel/}{Imi Knoebel}, ran
Stone im Ratinger Hof, a music venue that, much like New York's Mudd
Club of the same era, attracted the art crowd; there, the likes of
\href{https://www.nytimes3xbfgragh.onion/topic/person/sigmar-polke}{Sigmar
Polke} and Beuys listened to Krautrock bands like Neu! and
\href{https://www.nytimes3xbfgragh.onion/2012/04/16/arts/music/talking-to-ralf-hutter-of-kraftwerk.html}{Kraftwerk}.
Fritsch applied to the city's Kunstakademie, Germany's best art school,
and got in.

Thanks in part to Beuys's legacy, Düsseldorf in the '60s and '70s
represented a place of radical liberation, becoming an essential force
in contemporary art. (Beuys was dismissed from teaching in 1972 after he
admitted 50 students to his class who had been rejected by the academy.)
His influence lived on at the school in its notable painters, like
Kiefer and Richter, but also touched Fritsch's generation of students,
among them the photographers
\href{https://www.nytimes3xbfgragh.onion/2015/05/29/magazine/candida-hofers-wide-open-spaces.html}{Candida
Höfer} and
\href{https://www.nytimes3xbfgragh.onion/2017/10/05/arts/design/thomas-ruff-whitechapel-gallery.html}{Thomas
Ruff}, the latter a good friend and frequent collaborator of Fritsch's.
Beuys believed that everyone not only could be but already was an
artist. But this everything-goes attitude was as much about the tumult
of postwar West Germany as it was a reflection of Beuys's own
philosophy. This was a generation of artists born into a chastened,
broken Germany in the aftermath of World War II, yet who came of age
during the Wirtschaftswunder, or economic miracle, in which the
industrial Ruhr area played a central role. The country's re-emergence
as a modern industrial superpower with an uneasy relationship to its
recent past defines the art of this period, which didn't so much address
this identity crisis as simply embody it, resulting in one of the most
thrillingly innovative periods in contemporary art. As Beuys, whose most
famous work includes planting 7,000 oak trees around the industrial West
German city of Kassel in 1982, once wrote: ``Only art is capable of
dismantling the repressive effects of a senile social system that
continues to totter along the deathline {[}sic{]}.'' All German artists
of Fritsch's generation, in one form or another, have long been
preoccupied with the question of what art should be and who gets to
decide, and their work reflects profound ambivalence about the
human-made world and consumer culture.

\hypertarget{i-think-everything-can-be-a-sculpture-for-me-from-the-beginning-i-wanted-to-create-a-kind-of-middle-world-a-world-that-really-surprises-people-like-they-havent-seen-the-object-before}{%
\subsection{`I think everything can be a sculpture for me. From the
beginning, I wanted to create a kind of middle world, a world that
really surprises people like they haven't seen the object
before.'}\label{i-think-everything-can-be-a-sculpture-for-me-from-the-beginning-i-wanted-to-create-a-kind-of-middle-world-a-world-that-really-surprises-people-like-they-havent-seen-the-object-before}}

In straddling a line between the symbolic and literal, living things and
objects, Fritsch's art is itself an ambivalent comment about the
elevation of the everyday to a higher realm and the fruitless search for
identity and truth in a rapidly changing world. But her very particular
aesthetic has always felt larger in scope than the postwar milieu that
fostered her, and her work seems to suggest references of all kinds,
from
\href{https://www.nytimes3xbfgragh.onion/topic/person/rene-magritte}{René
Magritte} to
\href{https://www.nytimes3xbfgragh.onion/2016/05/26/arts/international/celebrating-kazimir-malevich-a-pioneer-in-abstract-art.html}{Kazimir
Malevich} --- and, of course, a certain essentially punk desire to
provoke. When she first entered the Kunstakademie in the late 1970s,
painting still dominated, and Fritsch found freedom in the sculpture
department, as well as a mentor in the artist Fritz Schwegler (who had
been a colleague of Beuys's) and many friends whom she credits as
inspiration, including the Minsk, Belarus-born sculptor Alexej
Koschkarow, with whom she's exhibited work on several occasions. She
attributes her initial interest in multiples and industrial processes to
her grandfather back in Langenberg, not
\href{https://www.nytimes3xbfgragh.onion/2018/05/02/t-magazine/andy-warhol-photo-portraits.html}{Andy
Warhol}. At first, she experimented with ready-mades, spray-painting
flowers and toy cars with automobile paint. It was in 1987 that she made
her breakthrough work, the life-size cadmium yellow Madonna, which
became one of her first public works when the Catholic city of Münster
installed it in a town square that year (the sculpture subsequently had
its nose broken and body graffitied a few times). ``When I first painted
the Madonna yellow, it was really something,'' she says. ``Now everyone
is doing things like that, but at the time, it was really a kind of
invention.'' Fritsch, who recently retired as a professor of sculpture
at the academy, where she taught for nine years, laments the loss of
that kind of low-stakes improvisation and openness to new ideas, new
forms and new names. The Germany she lives in now more or less stands
alone as the leader of a fraying democratic Europe, which only enhances
some of the mysterious drama of Fritsch's sculptures. What does a
Christian symbol mean at a time when much of the developed world is
turning away refugees and imprisoning asylum seekers? What is a fairy
tale if not a desperate search for home? Fritsch's art raises these
questions but refuses to answer them. In the same way that her work
defies interpretation, the artist herself doesn't read too much into her
formative years, which she sums up as lean and filled with exhilarating,
if toxic and rash-inducing, material experiments. ``Back then, everybody
lived in very bad circumstances and the market wasn't so strong,'' she
says. ``We didn't care so much; nobody had any money. It was an innocent
time. We were innocent creatures.''

DEPENDING ON ONE'S mood, the odd sense of dislocation that Fritsch's
work evokes might strike you as irreverent, cleverly transgressive or
something more insidious. But the longer I'm in its presence, the more I
sense a kind of moral intelligence in her objects, which distance us
from our well-worn perceptions and feelings. Then there's the implicit
feminism in a female sculptor looking at men --- still, oddly, something
of a rarity in contemporary art. Fritsch's men --- which have included,
over the years, a monk, a doctor and a be-toqued chef --- call to mind,
respectively, Caspar David Friedrich, Faust and an employee of a
Bavarian beer hall. They are not in any way erotic. She uses friends as
models, men with a certain kind of vanity, she says; the newest work
she's preparing for the upcoming show includes two male figures holding
mobile phones. The models were the art historian Robert Fleck and the
artist \href{http://www.artnet.com/artists/matthias-lahme/}{Matthias
Lahme}, and the piece is a reflection of Fritsch's increasing concern
about the disconnections and false promises of a digital age --- our
total absorption into unreal realms and the particular seductiveness of
this form of consumption. We peruse the internet for things that we
probably shouldn't: homes, partners, employment, an unnamed and
impossible fulfillment. The oblivious blue men clutching their phones
are unsettling not because they look so different from us but because
they are exactly the people who surround us, who perhaps \emph{are} us.

``I must say that this generation of mine, we were the power women of
the 1980s, and we wanted to be strong and straightforward. But then the
generation afterward wanted to be feminine, to look nice and to have
children, and they also wanted to have a big career. It's such a
pressure,'' she says, referring to the ongoing debate about gender roles
in Germany, where women occupy powerful positions in politics but are
far less prominent in art and business. While Fritsch is single and has
neither children nor poodles --- she spends much of her days happily
occupied with running her large studio --- she's surrounded by a circle
of artist friends and is very close with her mother and sister.
Sculpture, in particular at this kind of scale, demands very hard
physical labor, and casting her molds also involves contracting with
industrial workshops staffed exclusively by men: ``You get more and more
conscious of that, how they treat you and how they often don't listen to
you.'' The fabricators, she explains, will often speak to her male
assistants instead of to her. ``And then I say, `Look, please, at me and
talk to me. I'm giving the order, I'm paying you.' Only then, you are in
the stupid position --- then you are the old bitch.''

Image

``Rattenkönig (Rat-King)'' (1991-93), with 16 rats over nine feet tall
in a circle.Credit...Polyester and paint © Katharina Fritsch/VG
Bild-Kunst, Bonn/Courtesy of Matthew Marks Gallery.~Photo by~Nic
Tenwiggenhorn

In the market, her work does not sell in the same league as
\href{https://www.nytimes3xbfgragh.onion/2018/04/17/t-magazine/east-village-artist-jeff-koons-peter-halley.html}{Jeff
Koons} or Damien Hirst, whose careers have, at times, seemed to parallel
hers: Fritsch made the Madonna in 1987; that same year, also in Münster,
Koons installed a statue of the traditional German figure of the
Kiepenkerl, a traveling merchant; she completed ``Tischgesellschaft,''
the large-scale work featuring Fenstermacher, in 1988; Koons debuted his
series of sculptures and paintings featuring himself with his lover, the
porn star Cicciolina, ``Made in Heaven,'' in 1989. Fritsch weathered the
art world's rapaciousness in the 1980s, refusing to churn out work too
fast or under pressure: As such, she never turned cynical. She rarely
speaks to the press. But she is understandably disappointed that she
isn't spoken of in the same breath as some of her male counterparts nor
widely credited for her influence on turn-of-the-century sculpture. At
the same time, her unwillingness to please has, she believes, protected
her from a factory mentality she sees in male celebrity artists, from a
``heaviness'' that isn't just about literal weight.

With this in mind, I ask her if she thinks her work has shifted in
meaning through the years, as the art world has changed, not to mention
the larger world around her, drowning, as we are, in images of things,
from memes and emojis to styles that quickly disseminate and dissipate.
It hasn't, she tells me. ``The first picture I have in my mind is still
the one that is important.''

I think of this a week later, back home in Chicago, touring future
kindergartens for my 4-year-old, when I observe a classroom of young
children Magic Markering identical photocopies of a rooster. As they
carefully fill in the cartoonishly thick black outline of its body ---
this is the kind of school at which staying in the lines is encouraged
--- I wonder if this will become the prototypical notion of ``rooster''
that sticks, the picture that springs to mind when they hear its name.
(Few of these urban preschoolers are likely to have spent much time
around live chickens.) What could this picture possibly mean to them?
The coloring-book rooster is merely an echo of an echo, a signifier
absurdly distant from the hectic, strident reality of the animal itself,
so incidental, in this context, to its own representation. Once upon a
time, our forebears gathered around a fire to tell stories; they painted
the bison that sustained them, lining cave walls with animals and
hunting scenes filled with tenderness and meaning. In doing this, they
created what was, for them, a resonant collective iconography; now, of
course, these prehistoric paintings are touching in a different way.
This, I think, is why Fritsch's work continues to unsettle: Its distance
from reality feels unnervingly reflective of the way we live today,
increasingly remote from our own animal instincts, our original fears,
hungers and joys --- the sacral coding that helped remind us, before we
made art or commerce of identity, of who we were.

Advertisement

\protect\hyperlink{after-bottom}{Continue reading the main story}

\hypertarget{site-index}{%
\subsection{Site Index}\label{site-index}}

\hypertarget{site-information-navigation}{%
\subsection{Site Information
Navigation}\label{site-information-navigation}}

\begin{itemize}
\tightlist
\item
  \href{https://help.nytimes3xbfgragh.onion/hc/en-us/articles/115014792127-Copyright-notice}{©~2020~The
  New York Times Company}
\end{itemize}

\begin{itemize}
\tightlist
\item
  \href{https://www.nytco.com/}{NYTCo}
\item
  \href{https://help.nytimes3xbfgragh.onion/hc/en-us/articles/115015385887-Contact-Us}{Contact
  Us}
\item
  \href{https://www.nytco.com/careers/}{Work with us}
\item
  \href{https://nytmediakit.com/}{Advertise}
\item
  \href{http://www.tbrandstudio.com/}{T Brand Studio}
\item
  \href{https://www.nytimes3xbfgragh.onion/privacy/cookie-policy\#how-do-i-manage-trackers}{Your
  Ad Choices}
\item
  \href{https://www.nytimes3xbfgragh.onion/privacy}{Privacy}
\item
  \href{https://help.nytimes3xbfgragh.onion/hc/en-us/articles/115014893428-Terms-of-service}{Terms
  of Service}
\item
  \href{https://help.nytimes3xbfgragh.onion/hc/en-us/articles/115014893968-Terms-of-sale}{Terms
  of Sale}
\item
  \href{https://spiderbites.nytimes3xbfgragh.onion}{Site Map}
\item
  \href{https://help.nytimes3xbfgragh.onion/hc/en-us}{Help}
\item
  \href{https://www.nytimes3xbfgragh.onion/subscription?campaignId=37WXW}{Subscriptions}
\end{itemize}
