Sections

SEARCH

\protect\hyperlink{site-content}{Skip to
content}\protect\hyperlink{site-index}{Skip to site index}

\href{https://www.nytimes3xbfgragh.onion/section/us}{U.S.}

\href{https://myaccount.nytimes3xbfgragh.onion/auth/login?response_type=cookie\&client_id=vi}{}

\href{https://www.nytimes3xbfgragh.onion/section/todayspaper}{Today's
Paper}

\href{/section/us}{U.S.}\textbar{}`It Didn't Work:' States That Ended
Parole for Violent Crimes Are Thinking Again

\url{https://nyti.ms/37koDhr}

\begin{itemize}
\item
\item
\item
\item
\item
\item
\end{itemize}

Advertisement

\protect\hyperlink{after-top}{Continue reading the main story}

Supported by

\protect\hyperlink{after-sponsor}{Continue reading the main story}

\hypertarget{it-didnt-work-states-that-ended-parole-for-violent-crimes-are-thinking-again}{%
\section{`It Didn't Work:' States That Ended Parole for Violent Crimes
Are Thinking
Again}\label{it-didnt-work-states-that-ended-parole-for-violent-crimes-are-thinking-again}}

Virginia, newly dominated by Democrats, may broaden parole for the first
time in a generation. Others states are watching.

\includegraphics{https://static01.graylady3jvrrxbe.onion/images/2020/02/08/us/00PAROLE-prison/merlin_167250789_ccfc5a41-e543-474d-bbc4-fa1f67fbb4cf-articleLarge.jpg?quality=75\&auto=webp\&disable=upscale}

By \href{https://www.nytimes3xbfgragh.onion/by/timothy-williams}{Timothy
Williams}

\begin{itemize}
\item
  Published Feb. 13, 2020Updated Feb. 14, 2020
\item
  \begin{itemize}
  \item
  \item
  \item
  \item
  \item
  \item
  \end{itemize}
\end{itemize}

WAVERLY, Va. --- After Zenas Barnes was convicted of three robberies in
the 1990s, he accepted a plea deal that stunned even veteran lawyers for
its severity: 150 years in state prison.

Mr. Barnes, **** who was 21 at the time, said that he had not realized
when he took the deal that the Virginia Legislature had, only months
before, abolished the most common type of parole, meaning that there was
a good chance he would die in prison.

Twenty-five years later, the State Legislature, newly dominated by
Democrats, is poised to
\href{http://lis.virginia.gov/cgi-bin/legp604.exe?201+doc+S0310203}{broaden
parole} for the first time in a generation. The move would give Mr.
Barnes and thousands of other prisoners convicted of violent crimes a
chance for parole, which allows inmates to be released early.

Watching closely are lawmakers across the nation, including in
\href{https://www-cdn.law.stanford.edu/wp-content/uploads/2015/07/Final-FSR-Published-Article-1.pdf}{California},
\href{https://www.nysenate.gov/legislation/bills/2019/s497}{New York},
\href{http://www.illinoissenatedemocrats.com/sen-harmon-home/7182-new-harmon-law-creates-parole-option-for-youthful-offenders}{Illinois}
and
\href{https://www.inquirer.com/news/probation-parole-philadelphia-prison-mass-incarceration-violations-corrections-secretary-john-wetzel-20190618.html}{Pennsylvania}.
Like Virginia those states decades ago virtually eliminated
\href{https://www-cdn.law.stanford.edu/wp-content/uploads/2016/01/Rhine-Petersilia-Reitz-Improving-Parole-Release-in-America.pdf}{discretionary
parole}, granted by appointed boards on a conditional basis, during an
era of surging violent crime and the imposition of progressively harsher
punishments.

``We thought we were fighting crime, and it didn't work,'' said David
Marsden, a Democratic state senator in Virginia, who has previously
introduced bills to restore parole but was blocked by Republican
majorities. ``But more recently, we've stopped trying to teach lessons
and started trying to solve problems. People are now more likely to
believe that people deserve a second chance.''

\includegraphics{https://static01.graylady3jvrrxbe.onion/images/2020/02/07/us/00PAROLE-marsden/merlin_167250798_db4c1213-d39b-476b-8ffd-ecab803658be-articleLarge.jpg?quality=75\&auto=webp\&disable=upscale}

After watching the nation's prison population grow by
\href{https://sentencingproject.org/wp-content/uploads/2016/01/Trends-in-US-Corrections.pdf}{500
percent}since the 1980s, lawmakers have aggressively sought to reduce
prison rolls as part of a growing consensus that the criminal justice
system has
\href{https://sentencingproject.org/wp-content/uploads/2016/01/Trends-in-US-Corrections.pdf}{incarcerated}
too many Americans.

Louisiana, for instance, the state with the nation's
\href{http://worldpopulationreview.com/states/prison-population-by-state/}{highest
incarceration rate}, has in recent years cut its prison population to a
level not seen since the 1990s. Last year, New York approved a law
\href{https://www.brennancenter.org/our-work/analysis-opinion/new-yorks-upcoming-bail-reform-changes-explained}{ending
cash bail} for most people charged with misdemeanors or nonviolent
felony crimes, a move aimed at preventing people from being held for
longer periods only because they could not afford to pay bail.

On the federal level, President Trump
\href{https://www.whitehouse.gov/briefings-statements/president-donald-j-trump-committed-building-successes-first-step-act/}{signed
legislation} in December 2018 that shortened prison terms for some
federal inmates.

Still,
\href{https://www.nytimes3xbfgragh.onion/2017/08/08/opinion/violent-offender-parole-sentencing-reform.html}{analysts}
say recent attempts to restore parole in California, Pennsylvania and
elsewhere were beaten back amid political pressure on lawmakers over
concerns that someone released on parole could commit a
\href{https://people.com/crime/man-out-on-parole-allegedly-strangled-honors-student-in-parking-garage-when-she-ignored-catcalls/}{serious
violent crime}.

Even after Virginia lawmakers abolished many forms of parole in the
1990s, some types of it remained available, including the possibility of
parole for prisoners older than 61 and for inmates arrested before Jan.
1, 1995.

Gone, though, for new inmates was the sort of parole most inmates had
previously been released on. On parole, offenders are generally allowed
to serve the remainders of their sentences outside of prison with
stipulations that they meet regularly with a parole officer, stay
employed, get counseling, and pass drug and alcohol tests. If they fail
to adhere to the rules, they will often go back to prison.

Separate from parole, Virginia has retained a probation system for jail
inmates, who have been convicted of misdemeanor crimes and who are also
monitored regularly by the authorities. The state also allows prison
inmates to be released early for good behavior, although prisoners are
required to serve at least 85 percent of their sentences.

Even in Virginia, where Democrats won majorities in both chambers of the
Legislature in November, and which also has a Democratic governor, Ralph
Northam, the question of expanding parole remains politically perilous.
This month, Democrats
\href{http://lis.virginia.gov/cgi-bin/legp604.exe?201+sum+SB91}{shelved
a bill} that would have restored the possibility of parole for nearly
17,000 inmates --- more than half the state's prison population.
Instead, Democrats have focused on more modest efforts to restore parole
to older inmates.

``The prevailing attitude of policymakers is we've come to the limit
because they don't want to release violent offenders,'' said Marc Mauer,
executive director of the
\href{https://www.sentencingproject.org/issues/racial-disparity/}{Sentencing
Project}, a nonprofit that advocates shorter sentences and other policy
changes to the criminal justice system.

There is no significant difference in violent crime rates between states
that allow parole and those that do not, according to
\href{https://www.prisonpolicy.org/reports/correctionalcontrol2018.html\#}{federal
data}.

But Mr. Mauer said many people associate parolees with recidivism and
violence, and their crimes often garner significant
\href{https://people.com/crime/man-out-on-parole-allegedly-strangled-honors-student-in-parking-garage-when-she-ignored-catcalls/}{public
attention}.

Republican lawmakers have warned that
\href{http://lis.virginia.gov/cgi-bin/legp604.exe?201+ful+SB91}{restoring
parole} would make Virginia --- which has the
\href{https://ucr.fbi.gov/crime-in-the-u.s/2018/crime-in-the-u.s.-2018/topic-pages/tables/table-4}{fourth
lowest}violent crime rate of any state --- more dangerous.

``When parole is granted, it will result in violent criminals being
released into our communities,'' said Robert Bell, a Republican member
of the House of Delegates. Mr. Bell added that parole ``will force
victims of violent crimes and their families to relive the worst day of
their lives over and over again.''

The Virginia Victim Assistance Network, a crime victims' advocacy group,
also opposes widening the number of people who can get parole.

``Violent crime offenders should be held accountable for the crimes they
have committed against victims and their families,'' a statement from
the group said.

Kate McCord, associate director of the
\href{http://www.vsdvalliance.org/}{Virginia Sexual and Domestic
Violence Action Alliance,}a nonprofit advocacy group, said the
organization had not taken an official position on parole. If it is
restored, she said, parole boards should have broad discretion.

``It is in the best interest of survivors to have a parole board look at
the record of the person incarcerated, the crimes they committed, and
whether they had made an effort to rehabilitate themselves,'' she said.
``And those being released should have adequate support in terms of
finding employment and having access to stable housing.''

Both chambers of the Virginia Legislature have already approved a bill
that would make hundreds of prison inmates eligible for parole because
they were convicted by juries that were not informed by courts that
defendants were no longer eligible for parole after the practice was
\href{http://www.ncrp.info/StateFactSheets.aspx?state=VA}{abolished in
1995}.

Governor Northam has said he will support it.

Mr. Northam has also said he supports a bill that would grant parole
eligibility to prisoners who are older than 50, a group that may number
in the thousands. He has not yet said whether he would sign a measure
that would restore the possibility of parole to thousands of inmates who
have served 20 years or more of their sentences. Both bills are expected
to be passed by both chambers of the Legislature.

The governor has not taken a position on the shelved bill that would
have restored the possibility of parole for more than half the state's
prison population.

During a quarter century in prison, Mr. Barnes said he had watched as
inmates tried and convicted of murder came and went, serving sentences
shorter than the ones that were common during the 1990s at the height of
the
\href{https://www.nytimes3xbfgragh.onion/2019/08/20/us/politics/criminal-justice-reform-sanders-warren.html}{tough-on-crime
era}.

Inside the Sussex II State Prison, about an hour's drive south of
Richmond, Mr. Barnes, 46, wore a prison-issue blue shirt and bluejeans.
Guards did not remove his handcuffs. His salt-and-pepper beard was
neatly trimmed, his head had been freshly shaved.

``I've owned my guilt from the beginning,'' he said of his robberies,
the last of which ended with him being shot, eight times, by the police.
He spent nearly a month in the hospital recovering from the shooting.
``I'm just asking for a little bit of mercy. I didn't deserve 150 years
for the crime, but we're past that. I'm rehabilitated. I've overhauled
my thinking.''

Image

In her home in Houston, Isabelle Barnes held a photo of her youngest son
Zenas, bottom left in a graduation gown.Credit...William Chambers for
The New York Times

In 1995, when Mr. Barnes was 20, he and a friend pretended to be
carrying concealed guns and robbed three fast food restaurants in
Norfolk in the span of a few weeks.

Mr. Barnes said that his public defender told him he was facing 600
years in prison, but that if he accepted a plea deal for 150 years he
could be released on parole in as little as five years.

``I was too green to know better,'' Mr. Barnes said. His robbery partner
had already been convicted and sentenced to more than 40 years.

David Hargett, a lawyer who would later represent Mr. Barnes on an
appeal, said that taking a plea bargain with a 150-year sentence in a
non-death-penalty case had been ``inadvisable, unreasonable, and
contrary to all conceivable reason.''

Mr. Barnes's original lawyer, who was his public defender,
\href{https://www.manta.com/d/mm8c7w9/duncan-r-st-clair-iii-associates-p-c}{Duncan
R. St. Clair III}, did not respond to a letter or multiple phone
messages left at his office.

But during a hearing to consider overturning the sentence in 2001, Mr.
St. Clair denied misleading Mr. Barnes. ``It was the best plea agreement
that I could get,'' Mr. St. Clair told the judge. ``It wasn't a good
plea agreement. It wasn't a happy day, but it was the best I could do.''

Records show that Mr. St. Clair's law license was
\href{https://www.vsb.org/docs/StClair-041715.pdf}{revoked} by the
Virginia State Bar in 2015 for misconduct, including lying to clients.

For Mr. Barnes, long stretches of idle prison time had helped him better
understand his own shortcomings, he said.

``I was able to use this time to learn about myself --- that it's about
progress, not perfection,'' he said. ``I'm proud of the man I grew
into.''

Advertisement

\protect\hyperlink{after-bottom}{Continue reading the main story}

\hypertarget{site-index}{%
\subsection{Site Index}\label{site-index}}

\hypertarget{site-information-navigation}{%
\subsection{Site Information
Navigation}\label{site-information-navigation}}

\begin{itemize}
\tightlist
\item
  \href{https://help.nytimes3xbfgragh.onion/hc/en-us/articles/115014792127-Copyright-notice}{©~2020~The
  New York Times Company}
\end{itemize}

\begin{itemize}
\tightlist
\item
  \href{https://www.nytco.com/}{NYTCo}
\item
  \href{https://help.nytimes3xbfgragh.onion/hc/en-us/articles/115015385887-Contact-Us}{Contact
  Us}
\item
  \href{https://www.nytco.com/careers/}{Work with us}
\item
  \href{https://nytmediakit.com/}{Advertise}
\item
  \href{http://www.tbrandstudio.com/}{T Brand Studio}
\item
  \href{https://www.nytimes3xbfgragh.onion/privacy/cookie-policy\#how-do-i-manage-trackers}{Your
  Ad Choices}
\item
  \href{https://www.nytimes3xbfgragh.onion/privacy}{Privacy}
\item
  \href{https://help.nytimes3xbfgragh.onion/hc/en-us/articles/115014893428-Terms-of-service}{Terms
  of Service}
\item
  \href{https://help.nytimes3xbfgragh.onion/hc/en-us/articles/115014893968-Terms-of-sale}{Terms
  of Sale}
\item
  \href{https://spiderbites.nytimes3xbfgragh.onion}{Site Map}
\item
  \href{https://help.nytimes3xbfgragh.onion/hc/en-us}{Help}
\item
  \href{https://www.nytimes3xbfgragh.onion/subscription?campaignId=37WXW}{Subscriptions}
\end{itemize}
