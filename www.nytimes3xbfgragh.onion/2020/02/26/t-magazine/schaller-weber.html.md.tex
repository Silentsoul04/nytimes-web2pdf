Sections

SEARCH

\protect\hyperlink{site-content}{Skip to
content}\protect\hyperlink{site-index}{Skip to site index}

\href{https://myaccount.nytimes3xbfgragh.onion/auth/login?response_type=cookie\&client_id=vi}{}

\href{https://www.nytimes3xbfgragh.onion/section/todayspaper}{Today's
Paper}

The Butcher Shop Keeping Old World Delicacies Alive

\url{https://nyti.ms/2ThPNk3}

\begin{itemize}
\item
\item
\item
\item
\item
\end{itemize}

Advertisement

\protect\hyperlink{after-top}{Continue reading the main story}

Supported by

\protect\hyperlink{after-sponsor}{Continue reading the main story}

The 212

\hypertarget{the-butcher-shop-keeping-old-world-delicacies-alive}{%
\section{The Butcher Shop Keeping Old World Delicacies
Alive}\label{the-butcher-shop-keeping-old-world-delicacies-alive}}

Schaller \& Weber, in New York's once predominantly German Yorkville
neighborhood, has supplied the city with knackwurst and sauerkraut for
three generations.

\includegraphics{https://static01.graylady3jvrrxbe.onion/images/2020/02/26/t-magazine/26tmag-shaller-slide-NTRZ/26tmag-shaller-slide-NTRZ-articleLarge.jpg?quality=75\&auto=webp\&disable=upscale}

By Reggie Nadelson

\begin{itemize}
\item
  Published Feb. 26, 2020Updated Feb. 28, 2020
\item
  \begin{itemize}
  \item
  \item
  \item
  \item
  \item
  \end{itemize}
\end{itemize}

\emph{In}
\href{https://www.nytimes3xbfgragh.onion/column/the-212?module=inline}{\emph{this
series}} \emph{for T, the author Reggie Nadelson revisits New York
institutions that have defined cool for decades, from time-honored
restaurants to unsung dives.}

As soon as I walk into \href{https://schallerweber.com/}{Schaller \&
Weber}, the little German grocery shop on East 86th Street, on a
blustery day in January, I sense that something has changed. Until about
four years ago, I regularly visited a friend who lived nearby, and I'd
always stop at Schaller for the Black Forest ham, the cucumber salad,
the dark heavy pumpernickel bread that's the shape and weight of a small
nuclear weapon and, of course, the sausages. This is one of the very few
\href{https://www.nytimes3xbfgragh.onion/2019/07/29/nyregion/yorkville-tall-buildings-nyc.html}{remaining
outposts} of Yorkville, New York's old German neighborhood, and over the
years, Schaller had started to feel a little dowdy. Nostalgia seemed to
have infected its customers, who would examine products as if they were
looking for some distant past in the Christmas stollen and marzipan
figurines. ``Remember this,'' I once heard an elderly woman say to her
friend, looking at a jar of lingonberry preserve.

Now, there's a buzz. The shop seems revitalized, peppy as a polka. On
this winter morning, a steady stream of customers passes by to
investigate the large selection of
\href{https://www.nytimes3xbfgragh.onion/topic/subject/austrian-wines}{Austrian
wines} and
\href{https://www.nytimes3xbfgragh.onion/2016/05/15/world/europe/beer-purity-law-a-german-tradition-and-marketing-tool-turns-500.html}{German
beers}. A group of Japanese tourists surveys **** the gorgeous pink
hams, salami, cold cuts and shelves of housemade Düsseldorf-style
mustard with horseradish; they look bemused by a bottle of currywurst
ketchup. A lady in a large black fur hat gazes knowingly and with
reverence at the display case full of sausages: Cheddar brats,
knackwurst, blutwurst, gelbwurst and a score more.

\includegraphics{https://static01.graylady3jvrrxbe.onion/images/2020/02/26/t-magazine/26tmag-shaller-slide-255Z/26tmag-shaller-slide-255Z-articleLarge.jpg?quality=75\&auto=webp\&disable=upscale}

Image

In addition to a wide array of German meats, the butcher's counter at
Schaller \& Weber sells other European delicacies including pancetta,
salami and guanciale.Credit...Paul Quitoriano

Image

The store's selection of German sausages --- from blutwurst to ringwurst
--- is unrivaled by any other in the city.Credit...Paul Quitoriano

My friend Alfred, who visits New York each year from his home in Berlin,
gives a similar assessment of the store: **** ``When I discovered
Schaller \& Weber, I was so glad that I no longer had to smuggle in
Nürnberger bratwurst or weisswurst for my friends here,'' he says.
``Their product is so authentic, I can't get better in Berlin. If you
like cured and smoked meat and sauerkraut, it's the only address in the
city.''

In the shop, there is the brisk sound of hungry shoppers. The butcher's
counter is stocked with sumptuous steaks and pork chops; there are fresh
baguettes and vegetables in large baskets. Schaller also makes and sells
a marvelous fried chicken. ``The secret,'' says Jeremy Schaller, the
store's owner, ``is that we put it in the fryers where the pork was
cooked.''

Image

Ferdinand Schaller with his team in 1957.Credit...Courtesy of Schaller
and Weber

Schaller, 41, enthusiastic and very charming, is the third generation of
his family to run the business, and he has turned it into a full-fledged
gourmet store while maintaining its original role as a purveyor of
German meats. His grandfather Ferdinand, who trained as a butcher, left
Germany in the 1920s. ``The economy was terrible,'' says Jeremy. ``And
he had a chance to travel the world as a cook on a boat.'' Ferdinand
eventually landed in New York and went into business with Tony Weber;
they opened the store in 1937, and Schaller bought Weber out in 1956.
After Ferdinand came his sons Ralph and Frank, Jeremy's uncle and
father. Later, Jeremy himself, after college and a career in the fashion
business, found he missed the store where he had played as a kid and
worked as a teenager. ``Come back,'' said his uncle. In 2015, he went.

As with so many of the best food shops in New York ---
\href{https://www.nytimes3xbfgragh.onion/2019/04/17/t-magazine/di-palos-nyc-store.html}{Di
Palo's in Little Italy},
\href{https://www.nytimes3xbfgragh.onion/2018/12/03/t-magazine/russ-and-daughters-new-york-history.html}{Russ
\& Daughters on the Lower East Side} --- at Schaller \& Weber, the
generational turnover has provided an enormous transfusion of ambition,
interest and passion, as well as better and more varied products. And
the store has not forgotten its essence: that Germanness shared by
everyone who speaks the mother tongue and loves a great sausage.

By the end of the 19th century, New York had the third largest
German-speaking population after Vienna and Berlin, mostly in Little
Germany, a neighborhood that is now part of the East Village. By the
start of the 20th century, the majority of the city's Germans had moved
uptown to Yorkville. The area --- bordered by 79th and 96th Streets,
Third Avenue and the East River --- became known as Germantown, and 86th
Street was often referred to as Sauerkraut Boulevard. By 1938, the
Staats-Zeitung, New York's biggest German-language newspaper, sold
80,000 or more copies a day. There were Hungarians, Austrians, Czechs
and Poles, too, united by a shared culture of \emph{kaffee mit schlag},
schnitzel and strudel; there were German dance halls, cafes, movie
theaters and beer halls. There were also the Nazis.

Image

A platter of Schaller \& Weber's cold cuts --- Black Forest ham,
prosciutto, salami, air-dried beef --- with pickles and housemade
Düsseldorf mustard.Credit...Paul Quitoriano

Image

A grilled bratwurst on a pretzel bun with Düsseldorf mustard, sauerkraut
and a side of potato salad.**\\
**Credit...Paul Quitoriano

During the 1930s, Yorkville was the New York base of the German American
Bund, the German-American pro-Nazi organization that held marches and
rallies across the country;
\href{https://www.nytimes3xbfgragh.onion/1939/02/21/archives/22000-nazis-hold-rally-in-garden-police-check-foes-scenes-as.html}{one
infamous gathering} at Madison Square in 1939 drew more than 20,000
people. Not long after America entered World War II in 1941, Franklin
Roosevelt, under the Alien Enemies Act, ordered the internment of more
than 11,000 innocent ethnic Germans in camps, and among them was
Ferdinand Schaller. When he was released, **** he returned to his shop
and ushered it into the postwar era.

``Jeremy gets tradition,'' says Chris Cunningham, who has worked at ****
Schaller for 25 years, first as a delivery boy, later as a sausage maker
and salesman and finally as the general manager. ``He once painted a
wall black and I thought he was out of his mind. But when he brought the
company in to the 21st century, he got it right.''

Last year, Jeremy added the
\href{https://www.nytimes3xbfgragh.onion/2015/11/04/dining/hungry-city-schallers-stube-yorkville.html}{Schaller
Stube}; \emph{stube}, in German, means ``room'' or ``bar.'' ****** In
this case, it is a counter next door to the shop where you can grab a
grilled bratwurst on a pretzel bun (there's also an outpost in the Essex
Market downtown). The stand occupies an area that was once the store's
main entrance, and you can still see the huge black steel wheel and
enormous gears used to slide sides of meat from trucks at the curbside
into the back of the shop where butchers turned them into sausage. These
days, everything --- the sausages, the charcuterie --- is made at the
Schaller production facility in Pennsylvania, but the quality remains as
good as ever. ``I think the sausage is better at Schaller than most
places in Germany even,'' says my friend Alfred.

Image

Jeremy Schaller, who has owned and run Schaller \& Weber since
2015.Credit...Paul Quitoriano

Image

Customers order a snack at the Schaller Stube next to the main store on
Second Avenue and 86th Street.Credit...Paul Quitoriano

Down a flight of stairs from the main store is a small, private room
where, on a low table, Jeremy has laid out lunch for the two of us:
salami, ham, liverwurst, bologna, pâtés and cheese, along with slices of
rye and pumpernickel. I've always loved liverwurst; when I was little,
we ate it on rye from Ruben's deli on 10th Street. There is also
landjäger --- a dried, jerky-like sausage traditionally stashed in
lederhosen pockets on Alpine hikes --- and a delicious cold Pilsner
that's always on tap upstairs. ``A good German breakfast, whether in New
York or Berlin, is not complete without landjäger or teawurst,'' Alfred
tells me later over the phone. ``When I'm in New York, I always head for
Schaller \& Weber first thing to prevent an inevitable bout of
\emph{Heimweh''} --- or ``homesickness.''

Advertisement

\protect\hyperlink{after-bottom}{Continue reading the main story}

\hypertarget{site-index}{%
\subsection{Site Index}\label{site-index}}

\hypertarget{site-information-navigation}{%
\subsection{Site Information
Navigation}\label{site-information-navigation}}

\begin{itemize}
\tightlist
\item
  \href{https://help.nytimes3xbfgragh.onion/hc/en-us/articles/115014792127-Copyright-notice}{©~2020~The
  New York Times Company}
\end{itemize}

\begin{itemize}
\tightlist
\item
  \href{https://www.nytco.com/}{NYTCo}
\item
  \href{https://help.nytimes3xbfgragh.onion/hc/en-us/articles/115015385887-Contact-Us}{Contact
  Us}
\item
  \href{https://www.nytco.com/careers/}{Work with us}
\item
  \href{https://nytmediakit.com/}{Advertise}
\item
  \href{http://www.tbrandstudio.com/}{T Brand Studio}
\item
  \href{https://www.nytimes3xbfgragh.onion/privacy/cookie-policy\#how-do-i-manage-trackers}{Your
  Ad Choices}
\item
  \href{https://www.nytimes3xbfgragh.onion/privacy}{Privacy}
\item
  \href{https://help.nytimes3xbfgragh.onion/hc/en-us/articles/115014893428-Terms-of-service}{Terms
  of Service}
\item
  \href{https://help.nytimes3xbfgragh.onion/hc/en-us/articles/115014893968-Terms-of-sale}{Terms
  of Sale}
\item
  \href{https://spiderbites.nytimes3xbfgragh.onion}{Site Map}
\item
  \href{https://help.nytimes3xbfgragh.onion/hc/en-us}{Help}
\item
  \href{https://www.nytimes3xbfgragh.onion/subscription?campaignId=37WXW}{Subscriptions}
\end{itemize}
