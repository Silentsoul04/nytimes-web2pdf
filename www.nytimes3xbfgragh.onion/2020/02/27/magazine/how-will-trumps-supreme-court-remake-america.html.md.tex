How Will Trump's Supreme Court Remake America?

\url{https://nyti.ms/2wUtN7D}

\begin{itemize}
\item
\item
\item
\item
\item
\item
\end{itemize}

\includegraphics{https://static01.graylady3jvrrxbe.onion/images/2020/03/01/magazine/01mag-originalists-topper/01mag-originalists-topper-mediumSquareAt3X.png}

Sections

\protect\hyperlink{site-content}{Skip to
content}\protect\hyperlink{site-index}{Skip to site index}

feature

\hypertarget{how-will-trumps-supreme-court-remake-america}{%
\section{How Will Trump's Supreme Court Remake
America?}\label{how-will-trumps-supreme-court-remake-america}}

On abortion, gun rights and more, the future could be determined by how
fully the court's new conservative majority embraces a rigid
understanding of the Constitution.

Credit...

Supported by

\protect\hyperlink{after-sponsor}{Continue reading the main story}

By \href{https://www.nytimes3xbfgragh.onion/by/emily-bazelon}{Emily
Bazelon}

\begin{itemize}
\item
  Published Feb. 27, 2020Updated Feb. 28, 2020
\item
  \begin{itemize}
  \item
  \item
  \item
  \item
  \item
  \item
  \end{itemize}
\end{itemize}

In July 2013, Aimee Stephens wrote a letter to her co-workers and her
employer at a funeral home in the Detroit area, where she had worked for
six years. ``What I must tell you is very difficult for me and is taking
all the courage I can muster,'' she told them. ``With the support of my
loving wife, I have decided to become the person that my mind already
is.'' After four years of counseling, Stephens explained that she was
transitioning from being a man to being a woman, and so, at the end of
an upcoming vacation, she would come back to work as her ``true self,''
wearing women's business attire. Stephens's boss told her that her
self-presentation would harm his clients and business, and he fired her.

In October,
\href{https://www.supremecourt.gov/oral_arguments/argument_transcripts/2019/18-107_c18e.pdf}{the
Supreme Court heard a lawsuit} from Stephens challenging her termination
based on Title VII of the 1964 Civil Rights Act, which prohibits
employers from discriminating on the basis of ``sex.'' When members of
Congress passed the initial law, and when they later amended it, they
didn't say they were protecting gay or transgender people. The question
in front of the court was whether the plain meaning of the word they
chose --- ``sex'' --- did so anyway. The court also heard the claims of
two gay men who were fired from their jobs.

Stephens's lawyer, David Cole, argued that she was fired because she
didn't conform to her employer's ideas about gender. Stephens didn't
fulfill an ``expectation that applies only to people assigned male sex
at birth,'' he said, ``namely, that they live and identify as a man for
their entire lives. That is disparate treatment on the basis of sex.''

Justice Neil Gorsuch, who was appointed to the
\href{https://www.nytimes3xbfgragh.onion/2020/04/20/us/politics/supreme-court-unanimous-verdicts.html}{Supreme
Court} by President Trump in 2017, asked Cole, who is the national legal
director for the A.C.L.U., how judges should now interpret an ``old''
law, written in a different era. This question is of particular
importance to Gorsuch, who says he uses a method called textualism for
deciding cases that involve a statute like Title VII. He believes that
judges should focus only on the plain meaning of the text. When the
court interprets the Constitution, Gorsuch subscribes to a similar
(though not identical) theory, originalism, in which judges adhere to
the meaning of the Constitution as people understood it when it was
ratified. Because they are simply looking at the words before them,
which they believe have a single, fixed meaning, judges like Gorsuch say
their method allows their decision-making to be ``value neutral'' --- in
contrast to judges who consider a law's purpose or consequences.

Cole responded to Gorsuch: ``We are not asking you to apply any meaning
of `sex' other than the one that everybody agrees on as of 1964, which
is sex assigned at birth, or as --- as they put it, biological sex.'' He
added, ``We're not asking you to rewrite it.''

``I agree with that,'' Gorsuch said. ``Assume for the moment I'm with
you on the textual evidence.'' If all that mattered to Gorsuch was the
text, Stephens and the other plaintiffs might have the fifth vote they
needed --- along with those of the four liberal justices, Ruth Bader
Ginsburg, Elena Kagan, Sonia Sotomayor and Stephen Breyer --- to win a
huge victory for gay and transgender rights when the justices decide the
case by the end of the court's term this June.

In one sense, that result would be a huge surprise. For the first time
in generations, the court has a majority of five staunchly conservative
justices --- Gorsuch, who filled Antonin Scalia's seat; Trump's second
appointee, Brett Kavanaugh, who replaced Anthony Kennedy in 2018; Samuel
Alito; Clarence Thomas; and Chief Justice John Roberts. Expanding the
rights of gay and transgender people would not appear to be on the menu.
But if Gorsuch meant what he said about faithfully following the text
and agreed with Cole about its meaning, it was hard to see how he could
vote against Stephens.

But then Gorsuch pivoted, with a startling question for a strict
textualist. ``At the end of the day,'' he asked, should a judge ``take
into consideration the massive social upheaval that would be entailed in
such a decision?''

Cole answered that no evidence suggests an upheaval.
\href{https://www.prri.org/press-release/new-poll-majority-americans-oppose-laws-requiring-transgender-individuals-use-bathrooms-corresponding-sex-birth-rather-gender-identity/}{A
2016 poll shows} that 80 percent of Americans think it's already illegal
to fire or refuse to hire someone for being gay, and
\href{https://transequality.org/federal-case-law-on-transgender-people-and-discrimination}{some
lower courts have} treated discrimination against transgender people as
a violation of Title VII for 20 years.

It was a telling exchange for assessing Gorsuch's commitment to his
method of deciding cases, which he has said judges should follow in
every instance. Critics
\href{https://newrepublic.com/article/81480/republicans-constitution-originalism-popular}{have}
\href{https://scholarship.law.upenn.edu/faculty_scholarship/1478/}{long}
\href{https://www.cambridge.org/core/books/originalism-as-faith/24BD81CE5C34480BEDB02E3C004137DE}{argued}
\href{https://www.nytimes3xbfgragh.onion/2017/03/22/opinion/the-problems-with-originalism.html}{that}
\href{https://www.law.du.edu/documents/denver-university-law-review/v88-3/Greene.pdf}{originalism}
\href{https://www.law.upenn.edu/live/files/71-somin158upalrev2352010pdf}{and
textualism} are riddled with inconsistencies and can be used to provide
a fig leaf for results-oriented judging. In that moment with Cole,
Gorsuch seemed caught between the plain meaning of ``sex'' and a
worldview he shares --- in other words, between principles and politics.

The line between law and politics has always been blurry, and judges
have often professed to sharpen it. Claims of unblinking fidelity to the
text have increasingly become the crowning orthodoxy on the right in
recent decades. Now Gorsuch and his conservative colleagues have a
chance to harness that energy to transform the law. ``The Trump vision
of the judiciary can be summed up in two words: `originalism' and
`textualism,'' Donald F. McGahn II, the former White House counsel, who
was instrumental in Gorsuch's and Kavanaugh's appointments,
\href{https://www.c-span.org/video/?437462-8/2017-national-lawyers-convention-white-house-counsel-mcgahn}{said
in 2017 at an event} for the Federalist Society, a group that has
\href{https://www.washingtonpost.com/graphics/2019/investigations/leonard-leo-federalists-society-courts/?itid=lk_inline_manual_1}{been
a juggernaut} for
\href{https://www.nytimes3xbfgragh.onion/2018/08/22/magazine/trump-remaking-courts-judiciary.html}{propelling
the courts to the right}. Placing judges on the courts is ``the most
important thing we've done for the country,'' Senator Mitch McConnell,
the majority leader, said last spring. He earlier promised that Trump
judges (192 and counting) will ``interpret the plain meaning of our laws
and our Constitution according to how they are written.''

Since the 1960s, conservatives have often derided liberal judges as
``activists'' who bend the law to make big changes. And until his
departure in 2018, Justice Kennedy held the Supreme Court's swing vote
and (like Sandra Day O'Connor before him) restrained his fellow
conservatives by forging a kind of national compromise on abortion
rights, marriage equality, gun laws, the regulatory powers of federal
agencies and the scope of the death penalty.

But now Gorsuch, along with Thomas and Alito, has become ``the leading
edge of a second generation of conservatives who are not afraid of
exercising judicial authority'' --- in other words, making decisions
that can significantly change the law --- said John Yoo, a law professor
at the University of California, Berkeley, and a former Thomas clerk.
(Yoo helped write memos in George W. Bush's Justice Department that
provided justification for the torture of suspects after Sept. 11.)
These three justices have shown a ``predisposition to swing for the
fences,'' Donald Verrilli, who served as a solicitor general for
President Barack Obama, told me.

The more that conservatives on the court want to overturn precedents and
strike down laws, the more useful it is for them to claim a coherent
philosophy that seems to merely follow the dictates of the Constitution
or a statute. Gorsuch is positioning himself to push his colleagues in
that direction as the public voice and salesman of originalism. Thomas,
a fellow strict originalist, rarely speaks from the bench. Roberts
\href{https://abcnews.go.com/Archives/video/sept-12-2005-john-roberts-baseball-analogy-10628259}{said
at his 2005 confirmation hearings} that judges should ``call balls and
strikes,'' but he didn't explain how. Alito
\href{https://www.yalelawjournal.com/forum/the-unitary-executive-and-the-scope-of-executive-power}{calls
himself} a ``practical originalist,''
\href{https://www.yalelawjournal.com//sam-alito-the-courts-most-consistent-conservative}{picking
and choosing} when to apply the theory. Kavanaugh, whose record on the
court is too short to reveal much,
\href{https://scholarship.law.nd.edu/cgi/viewcontent.cgi?article=4733\&context=ndlr}{suggested
in a 2017 lecture} that he didn't have one methodology by saying that
``history and tradition'' competed for ``primacy of place'' with factors
like liberty and deference to the Legislature. Questions and battles
over originalism and textualism will run through almost every major case
the justices hear this year and beyond, and they are the key to
understanding the Roberts Court.

\includegraphics{https://static01.graylady3jvrrxbe.onion/images/2020/03/01/magazine/01mag_supremecourt7/01mag_supremecourt7-articleLarge.jpg?quality=75\&auto=webp\&disable=upscale}

\textbf{Originalism may sound} like an old concept, but it's actually a
modern creation, one born of political exigencies. It dates to the
aftermath of the Supreme Court's 1973 decision
in\href{https://www.oyez.org/cases/1971/70-18}{Roe v. Wade}, which
recognized a constitutional right to abortion. That ruling was not
partisan: Roe was decided by a vote of 7 to 2, with five justices in the
majority appointed by Republican presidents and one in dissent appointed
by a Democrat. By 1980, however, the Republican Party had become more
uniformly conservative, and its leaders determined that opposing
abortion was a crucial way to win votes from evangelicals and Roman
Catholics. The party
\href{https://www.presidency.ucsb.edu/documents/republican-party-platform-1980}{promised
in its platform that year} to appoint judges who would protect
``traditional family values and the sanctity of innocent human life.''

But the pledge made it sound as if Republican-appointed judges would
pursue a political agenda. Some conservative legal thinkers were
uncomfortable with that overt mixing of politics and the law. In 1982,
law students started the Federalist Society. The students who founded it
\href{https://www.lawliberty.org/liberty-forum/is-legal-conservatism-as-accomplished-as-it-thinks-it-is/}{instructed
other students} not to ``use the adjective `conservative.'' Their
purpose, they said, was the nonideological and nonpartisan promotion of
limited government.

To maintain their distance from politics, they needed another way to
talk about abortion. Robert Bork, an early adviser to the Federalist
Society and an appeals court judge, had an answer: a return to the
framers' original conception of the Constitution.
\href{https://www.nytimes3xbfgragh.onion/1987/09/13/us/robert-bork-s-views-on-a-wide-range-of-legal-issues-in-his-own-words.html}{Bork
said} Roe erred not because abortion was wrong but because it created a
right to privacy that could not ``be found in the Constitution by any
standard method of interpretation.'' In
\href{https://www.nytimes3xbfgragh.onion/1985/07/10/us/meese-in-bar-groug-speech-criticizes-high-court.html}{a
1985 speech to the American Bar Association}, Attorney General Edwin
Meese III gave Bork's idea a wide audience, calling for judges to follow
the ``original intention'' of the Constitution's framers.

Every judge begins with the text when interpreting a law, and in some
ways, originalism seems like common sense. But used in isolation, it
doesn't reflect how the court has done its work for most of its history.
At key moments since the country's early days, the court has weighed the
purpose and consequences of a ruling as much as or more than text.

In a sense, the Constitution invited some license. The document gave
specific instructions (``The House of Representatives shall be composed
of members chosen every second year'') for the country to follow, but it
also provided open-ended principles (``freedom of speech,''
``property,'' ``liberty, ``due process'') and left questions unanswered.
The Constitution itself provided no method for interpreting it --- and
at the Supreme Court, constitutional cases were rare, in any event.
Almost all of the justices' work has
\href{https://chicagounbound.uchicago.edu/cgi/viewcontent.cgi?article=2999\&context=journal_articles}{involved
creating legal doctrines}, case by case. Their opinions have been full
of discussion of their own previous decisions much more than the
Constitution. ``The precedents shape the text, rather than the other way
around,'' the University of Chicago law professor David Strauss
\href{https://harvardlawreview.org/2015/11/does-the-constitution-mean-what-it-says/}{wrote
in The Harvard Law Review in 2015.}

In \href{https://www.oyez.org/cases/1789-1850/5us137}{the 1803 case
Marbury v. Madison}, the justices filled in a gap about which branch of
government should be the final arbiter of the Constitution's meaning by
declaring that it was the courts' job ``to say what the law is.'' In
\href{https://www.oyez.org/cases/1789-1850/17us316}{1819, in McCulloch
v. Maryland}, the court had to decide whether Congress could charter a
bank of the United States even though the Constitution did not
explicitly say so. The justices took into account the pressing concern
of the day --- the need for a national army and a bank with branches
across the states and territories to pay soldiers --- and voted
unanimously to broadly interpret Congress's power to make ``necessary''
laws to allow for the bank. ``Such is the character of human language,
that no word conveys to the mind, in all situations, one single definite
idea,'' Chief Justice John Marshall wrote.

Over the years, the justices continued to consider the demands of the
moment and their own beliefs about the policies that served the
country's interests. They
\href{https://digitalcommons.law.yale.edu/cgi/viewcontent.cgi?article=1208\&context=fss_papers}{didn't
follow a single approach} or subscribe to a particular theory, as
originalists claim to do today. Early on, the court recognized a
principle called \emph{stare decisis}, meaning ``to stand by the things
decided,'' which allowed it to maintain the law's stability. If one
ruling proved a mistake, later justices occasionally reversed it, or
they more commonly stepped around a decision they didn't like and
gradually rerouted.

In the 20th century, the justices continued to weigh the impact their
decisions would have, increasingly
\href{https://louisville.edu/law/library/special-collections/the-louis-d.-brandeis-collection/the-brandeis-brief-in-its-entirety}{taking
into account science and social science}. In 1954, the court made an
unusual request for a different kind of briefing: The justices wanted to
know about the original intentions of the framers of the 14th Amendment.
\href{https://www.oyez.org/cases/1940-1955/347us483}{In Brown v. Board
of Education}, the landmark challenge to school segregation, the court
asked whether Congress and the state legislatures that enacted the
amendment in 1868 contemplated an end to segregating public schools when
they promised that the states would guarantee ``equal protection of the
law,'' as well as ``due process'' and ``liberty.'' The N.A.A.C.P., which
was litigating the case, sent out an emergency telegram to its
supporters asking for help responding to the court's request. But in the
end, the plaintiffs and their experts could not supply the
original-meaning support for desegregation the court was looking for. At
the time the 14th Amendment was ratified,
\href{https://global.oup.com/academic/product/from-jim-crow-to-civil-rights-9780195310184?cc=us\&lang=en\&}{its
backers}\href{https://global.oup.com/academic/product/from-jim-crow-to-civil-rights-9780195310184?cc=us\&lang=en\&}{\emph{denied}}
it would lead to desegregated schools.

After an exhaustive discussion of the history at oral argument, Chief
Justice Earl Warren did not pretend otherwise. In the end, his opinion,
for a unanimous court, turned on present-day evidence about harm to
black children. ``We must look to the effect of segregation itself on
public education,'' Warren wrote. ``In approaching this problem, we
cannot turn the clock back to 1868.''

Over the next decade, one justice who joined the unanimous majority in
Brown, \href{https://digitalcommons.law.yale.edu/fss_papers/856/}{Hugo
Black, spoke up} for adhering to text and the Constitution's original
meaning. The way to change the Constitution, Black insisted, was to
amend it, though in practice he continued to vote with the majority in
many Warren Court decisions that expanded the concepts of equal
protection and due process. Consistency, even for an early form of
textual fidelity, proved difficult.

In 1967, Thurgood Marshall, who led the N.A.A.C.P.'s team of lawyers in
Brown, was asked during his confirmation hearings for the Supreme Court
whether the court's role should be ``simply to ascertain and give effect
to the intent of the framers.'' Marshall said yes, ``with the
understanding that the Constitution was meant to be a living document.''

The idea of an evolving Constitution, built from the language of the
framers but not limited to their understanding of it, became a concept
associated with liberals. Yet sometimes it has been conservatives who
leave the text and the framers behind. In the 1970s, the conservatives
on the court began to rule that the First Amendment protected commercial
speech, like advertisements, even though it had never been understood
that way before. The doctrine remains a tenet of the right.

The repeated lesson,
\href{https://harvardlawreview.org/2015/11/does-the-constitution-mean-what-it-says/}{as
Strauss argues}, is that the Constitution isn't just or even mainly its
text. It's the edifice the court has hammered together over the words,
adding and renovating over the centuries. The court still spends most of
its time and energy on its own precedents. In many important areas ---
free speech, civil rights, establishment of religion, criminal procedure
and punishment --- the doctrines the court has developed stray so far
from an originalist reading of the text that to return to it would
render American law unrecognizable.

Image

Antonin Scalia offered a more politically palatable version of
originalism at his confirmation in 1986, admitting the difficulty of
applying the theory in every case and saying it could be superseded by
the court's precedents.Credit...Bettmann/Getty Images

\textbf{In 1987, President Ronald Reagan} nominated Bork to the Supreme
Court, and his confirmation hearing proved to be the first test of
originalism's public acceptability. **** Bork argued that the
Constitution provided not only no basis for the right to privacy in Roe
but also no basis for banning literacy tests or poll taxes or for the
standard of ``one person, one vote.'' Bork's statements helped persuade
Democratic senators to oppose him, sinking his nomination.

A year after Bork's defeat, Justice Antonin Scalia, from his safe perch
on the court, offered a more politically palatable version of
originalism in a lecture at the University of Cincinnati called
\href{https://alumni.northeastern.edu/wp-content/uploads/2019/04/Antonin-Scalia-Originalism-The-Lesser-Evil.pdf}{``Originalism:
the Lesser Evil.''} The shadow of Bork hung over the conservative legal
movement, and Scalia began by admitting that originalism was ``not
without its warts.'' Its greatest defect, he said, was that it was so
difficult to apply correctly. This required ``immersing oneself in the
political and intellectual atmosphere of the time.'' To write an
originalist opinion would take significant research ``sometimes better
suited to the historian than the lawyer.''

Scalia was candid about the difficulty of applying originalism in every
case. ``In its undiluted form, at least,'' he wrote, ``it is medicine
that seems too strong to swallow.'' Almost every originalist, Scalia
said, recognized that the theory could be superseded by the court's
general rule of respect for its own past decisions, or precedents.
Scalia also acknowledged that there were things the Constitution
permitted at the founding that he couldn't imagine allowing today ---
for example, a law allowing for whipping or branding as a criminal
punishment. ``In a crunch,'' he admitted, ``I may prove a fainthearted
originalist.'' How would originalism stop judges from imposing their own
values if they could be selective about applying it? Scalia simply
promised such cases would rarely arise.

Scalia had already altered the definition of originalism. He said in a
speech two years earlier that originalism should not focus on the intent
of the framers, who disagreed among themselves. Instead, the theory
should rest on the Constitution's original public meaning: the
understanding of the text by ordinary citizens at the time, revealed
through research into founding-era sources. In practice, originalism has
slid between these definitions ever since.

Despite Scalia's efforts, originalism remained politically tainted by
the memory of Bork. Clarence Thomas steered clear of the subject at his
Supreme Court confirmation hearings in 1991. ``Our notions of what
liberty means evolves with the country,'' he said in a discussion of the
14th Amendment. ``I don't think that they could have determined in 1866
what the term in its totality would mean for the future.''

Image

Clarence Thomas downplayed his originalism to the Senate at his
confirmation hearing in 1991, saying that ``Our notions of what liberty
means evolves with the country.'' On the court, he has used originalism
as a rationale for upending whole areas of American law.Credit...John
Duricka/Associated Press

It appears in retrospect that Thomas was obscuring his views in order to
win Senate votes. On the court, he became the justice most determined to
use originalism to rip up whole fields of American law, especially to
reduce the scope of federal regulation. ``When faced with a demonstrably
erroneous precedent, my rule is simple,'' he wrote last June in a solo
concurrence --- a separate opinion agreeing with a judgment ---
\href{https://www.oyez.org/cases/2018/17-646}{in Gamble v. United
States.} ``We should not follow it.'' He has written solo opinions at a
higher rate than any other sitting justice. When Scalia was alive, he
painted Thomas as an extremist. Comparing himself with Thomas at a talk
at a synagogue 15 years ago, according to the New Yorker writer Jeffrey
Toobin, Scalia cracked: ``I am an originalist, but I am not a nut.''

But over time, positions Thomas once floated from the margins of
conservative thought have moved into its mainstream. ``Justice Thomas
has been throwing out revolutionary concepts for a long time now,'' Yoo,
his former clerk, said. ``He was interested in being proved right by
history, or by the court 20 or 40 years into the future. Now you could
say his influence is reaching its height.'' Trump officials have
expressed their appreciation for Thomas, with
\href{https://www.washingtonexaminer.com/opinion/reshaping-the-judiciary}{McGahn
calling his recent opinions} a ``driving intellectual force.'' The
administration has successfully nominated more than 10 of his former
clerks, the highest total for any justice, to the federal bench.

In 2012, what was once Thomas's radical originalist rationale for
curtailing Congress's powers to pass laws based on the Constitution's
Commerce Clause almost became the basis for striking down the Affordable
Care Act. He first argued this position in the 1995 case
\href{https://www.oyez.org/cases/1994/93-1260}{United States v. Lopez},
saying that when the Constitution was written, ``commerce'' referred
only to ``selling, buying and bartering, as well as transporting for
these purposes.'' This led to the extraordinary suggestion that the
Supreme Court had been wrong to uphold the entire social safety net of
the New Deal, because it involved ``substantial effects'' on commerce
among the states. Seventeen years later, in the Affordable Care Act
challenge, all five conservative justices embraced this thinking,
finding that Congress had indeed exceeded its commerce powers. Only
Roberts's defection from the conservative majority, in concluding that
the individual mandate was permitted by Congress's power to tax, saved
the health care law.

Thomas, who declined to talk to me, moves back and forth between
different forms of originalism, sometimes focusing on the intention of
the framers and sometimes on the 18th-century meaning of the words,
according to Ralph Rossum, a government professor at Claremont McKenna
College, in his otherwise sympathetic book, ``Understanding Clarence
Thomas.'' Sometimes, Thomas ignores originalism altogether. For example,
he provided no evidence that the First Amendment's original meaning
supported his position \href{https://www.oyez.org/cases/1995/95-489}{in
a 1996 concurrence} in which he argued that limiting the political
donations of corporations violated their free-speech rights. The
conservative majority embraced that argument 14 years later in
\href{https://www.oyez.org/cases/2008/08-205}{Citizens United v. F.E.C.}
to strike down limits on corporate campaign donations.

In his 2019 book, ``The Enigma of Clarence Thomas,'' the political
scientist Corey Robin traces Thomas's version of originalism to his code
of self-reliance. Thomas called his memoirs ``My Grandfather's Son,''
writing with reverence about his grandfather's achievement of lifting
the family out of poverty by starting a fuel-delivery service in Georgia
despite the barriers of Jim Crow. In college, Thomas was a black
nationalist who followed Malcolm X, signing his letters ``Power to the
People.'' But after law school, he became a free-market conservative.
Criticizing President Franklin D. Roosevelt and other New Deal liberals
in a 1987 speech for the Pacific Research Institute, a think tank,
Thomas said: ``These critics of `the rich' really do mean to destroy
people like my grandfather.'' His opinions often align with his belief
that the unfettered market, not government efforts to redistribute
wealth or ameliorate discrimination, is ``the guarantor of precisely the
kind of freedom upon which the black community depended,'' as Robin
writes.

Thomas's main innovation has been to deploy originalism to loose the
government's reins over the market. One advantage of originalism is that
it allows conservative judges to justify sweeping away American legal
traditions, like the broad power of Congress to regulate. ``You have to
claim to be going back to first principles,'' David Strauss says.
``Otherwise, it's just that you don't like the legal order we have.''

Image

John Roberts said judges should ``call balls and strikes'' --- without
explaining how --- when asked about his judicial philosophy at his 2005
confirmation hearing.Credit...Jason Reed/Reuters

\textbf{In December, the Supreme Court} heard a challenge to a New York
gun-control law based on the Second Amendment. It was the latest step in
an originalist quest that Thomas helped start in the 1990s to use the
Constitution to strike down gun laws.

At the time, the Supreme Court's last word on the Second Amendment dated
from 1939, when the justices found unanimously that the right to bear
arms applied only to weapons with a reasonable relationship to a
militia. In the 1980s, originalists like Bork agreed that the Second
Amendment didn't give individuals a right to bear arms. But over the
next decade, gun rights became a newly invigorating issue for
Republicans, fueled by the National Rifle Association --- and like
abortion before it, that position, too, benefited from an originalist
justification. After a few law professors --- some of them liberals ---
began to argue that the Second Amendment had been misunderstood, Thomas
referred to their work in a footnote in the unrelated 1997 case Printz
v. United States. ``A growing body of scholarly commentary indicates
that the `right to keep and bear arms' is, as the amendment's text
suggests, a personal right,'' Thomas wrote, though he acknowledged that
there was significant scholarship on the other side of the debate.

Thomas's footnote served as an invitation for lawsuits challenging local
gun-control laws, and in 2007, the justices agreed to hear a Second
Amendment challenge to a District of Columbia regulation that
effectively barred the personal ownership of handguns. The plaintiff was
Dick Heller, a police officer who couldn't obtain a license from the
District of Columbia to keep a weapon at home.

With Heller on the court's docket, historians saw a rare chance to
influence one of the biggest cases of the decade. A group of
founding-era scholars led by Jack Rakove, a Stanford historian,
concluded that the law professors Thomas cited were wrong: The Second
Amendment was not about self-defense. ``The right to keep and bear arms
became an issue in 1787-88 only because the Constitution proposed
significant changes in the governance of the militia,'' they wrote in an
amicus brief. The Federalists wanted Congress to determine exactly what
kind of militia the nation should have. A few Anti-Federalists did refer
to the personal use of guns. But the debate always focused on the role
of the militia, not a personal right of self-defense. The Federalists
won the debate and wrote the Second Amendment.

When the Supreme Court issued its decision in Heller in 2008, for the
first time in its history, the conservative majority, including Thomas,
ruled that the Second Amendment protected an ``inherent right of
self-defense.'' Scalia wrote the opinion, relying heavily on the 18th-
and early-19th-century dictionary definitions of ``keep,'' ``bear'' and
``arms,'' which could refer to the personal use of ordinary weapons.
Scalia also picked out a few Anti-Federalist quotes that supported his
position. For the most part, he bypassed the Federalist sources that
Rakove and his colleagues believed held the key.

To Rakove, Scalia's analysis was indefensibly incomplete. The founding
was ``a period of great political creativity,'' Rakove told me. As
concepts shifted, words took on new shades of meaning. The context
matters for accurately understanding the language. ``Even if you have
the best dictionaries from 1720 to 1790, you still want to think about
what the specific nature of the revolutionary-era controversy and
experience added.''

Even some conservative scholars found Scalia's treatment of the
historical sources wanting. Steven Calabresi, a law professor at
Northwestern who clerked for both Scalia and Bork and helped found the
Federalist Society, looked at all the early state constitutions and
found that Scalia had cited the ones that included a personal right to
bear arms without acknowledging that a majority of the constitutions did
not. ``Scalia was better at arguing that people should do originalist
history than actually doing it,'' Calabresi says.

In Heller, Scalia also dropped his originalist analysis in the crucial
passage of the opinion that explained how the court's decision would
affect modern gun laws. Almost surely to win the vote of Justice
Kennedy, which he needed for a majority, Scalia wrote that the court's
ruling did not ``cast doubt'' on prohibitions on the possession of
firearms by felons and the mentally ill, laws that forbid carrying
firearms in ``sensitive places'' and laws ``imposing conditions and
qualifications'' on gun buyers. Many of these laws were modern-day;
Scalia gave no historical support for letting them stand.

Image

Samuel Alito has called himself a ``practical originalist,'' picking and
choosing when to apply the theory.Credit...Chris Maddalon/Roll Call, via
Getty Images

After striking down one more handgun ban in Chicago in 2010, the court
stopped taking Second Amendment cases. As long as Kennedy remained on
the court, legislatures could respond to public outcry over gun violence
with increasing restrictions on firearms. It was a compromise that
Thomas rejected with mounting frustration, accusing his colleagues, in a
dissent in 2015, of ``relegating the Second Amendment to a second-class
right.''

The dynamic changed with Gorsuch's arrival after Trump's election. He
moved into chambers near Thomas, who invited him over for barbecue and
regularly pops into his office to talk, people who know them told me.
Gorsuch soon joined Thomas in scolding the rest of the court for
rejecting a challenge to California's ban on carrying concealed weapons.
Last year, with Kavanaugh installed in place of Kennedy, the court
finally accepted a case about a New York City ban on transporting
licensed handguns anywhere except to approved gun ranges.

Before the justices heard the case, New York lifted the ban and asked
the court to dismiss it. To the liberal justices, the grounds for a
dismissal were clear. Roberts also seemed open to dismissing the case.
But at the argument in December, Gorsuch took up the cause of trying to
keep the case alive. The court will decide whether to rule on the merits
in the next few months --- and whether to make this case its vehicle for
expanding gun rights.

Whatever the court does with the New York case, it has surfaced a new
challenge for Scalia's originalist claims about the Second Amendment.
Two years ago, Brigham Young University introduced a database of more
than 120,000 texts from the late 18th century. Previously, originalists
in search of the meaning of words during the founding era looked through
newspaper archives and other old records. The B.Y.U. database made it
possible to comprehensively assess how people at the time used the words
``bear arms.''

For originalists, the new tool is ``a paradigm-shifting technology,''
two members of the Federalist Society, the law professor Josh Blackman
and the Stanford law fellow James C. Phillips, wrote in The Harvard Law
Review's blog in August 2018. It also means that cherry-picking the
historical record to establish a dubious ``original'' meaning would be
harder to conceal. ``We can do empirics,'' says Alison LaCroix, a
historian and law professor at the University of Chicago. ``There's a
data set.''

Blackman and Phillips conducted a review of the database and found that
the dominant use of ``bear arms'' at the time of the country's founding
related to the militia. (Even so,
\href{https://www.theatlantic.com/ideas/archive/2020/02/big-data-second-amendment/607186/}{they
didn't conclude that Scalia got Heller wrong}.) LaCroix and three
linguists submitted a brief to the court last fall, in the New York
case, with studies they had each done. One found that references in the
database ``to hunting or personal self-defense'' for the phrase ``bear
arms'' were ``not just rare, they are almost nonexistent.'' The phrase
``keep arms,'' the brief stated, was also used ``almost exclusively in a
military context.''

The findings confirm what Rakove and his fellow historians showed about
the era's political history. But this time, the analysis played by the
rules of the game as Scalia defined them, by looking narrowly at the
original public meaning of the text. ``I don't care how big a fan of
Justice Scalia you are,'' Phillips told me. ``At some point, you run up
against the data.''

\textbf{In previous decades,} it was Scalia who sold originalism to the
public with brash confidence. ``You would have to be an idiot,'' he
said, to conceive of the Constitution as a ``living organism.'' Scalia
died in 2016, and now Gorsuch is remaking the role in his own image. In
a best-selling book published in September, ``A Republic, If You Can
Keep It,'' Gorsuch lays out his judicial philosophy. He says judges
should not ``interpret legal texts to produce the best outcome for
society,'' because that's the job of legislatures.

On an evening that month, a few hundred people gathered at the Richard
Nixon Presidential Library in Yorba Linda, Calif., to hear him speak
about his ideas. Gorsuch was on a book tour that included an hourlong
special for Fox News, interviews with print reporters (though he
declined my request to speak to him) and a later appearance on ``Fox \&
Friends.''

At the library, the crowd, dressed in pastels, filed past elderly
docents and into a replica of the East Room of the White House. The
audience members settled into their seats and then burst into applause
when the silver-haired justice strode into the room. Sitting near a
portrait of George Washington, he warmed up the crowd. ``It's really
nice being west of the Mississippi,'' he said with a grin, winning a
roar of anti-Washington approval. He told a story about his milkman
making a delivery to his home outside Boulder, Colo., where he and his
wife and two daughters lived before he joined the Supreme Court. In his
book, Gorsuch describes it as ``our home on the prairie'' and includes
pictures of horses, dogs and chicks.

Image

Neil Gorsuch considers himself a strict originalist, saying judges
should apply the theory in every case, and should not consider a law's
underlying purpose or the consequences of a ruling.Credit...Mark
Peterson/Redux

In fact, his house in Colorado was a gated estate that was sold for
\$1.5 million in 2017. When he lived there, he drove a gold Mercedes
convertible to work at the federal courthouse in Denver. In the early
1990s, Gorsuch met his wife, Louise, at Oxford, where she was a champion
equestrian and he studied legal philosophy after graduating from Harvard
Law School.

A former colleague says that Gorsuch urged his clerks to make money in
the private sector before they went on the bench, the path he took
himself as a corporate lawyer. In the 2000s, Gorsuch represented Philip
Anschutz, the oil-and-gas mogul, who has invested in a vast array of
businesses and conservative publications, including The Washington
Examiner. Anschutz played a role in elevating Gorsuch's legal career. In
2006, after George W. Bush was re-elected president, Anschutz lobbied
for Gorsuch's appointment to the U.S. Court of Appeals for the 10th
Circuit. He then gave Gorsuch a speaking spot at an annual dove-hunting
retreat he ran for prominent conservatives.

At the Nixon library, Gorsuch advertised his support for diversity.
Singling out three of his law clerks, Gorsuch described them as a
descendant of Mexican immigrants and Holocaust survivors, a
first-generation Chinese-American and the first Native American to ever
clerk on the Supreme Court. He praised his appeals court, the 10th
Circuit, for being ``as diverse a court on any metric you wish to
consider as any court in the country.'' In fact, judges on the 10th
Circuit are overwhelmingly white and male. Gorsuch went on to ask his
audience if they had heard people say originalism ``leads to
conservative results.'' The crowd murmured, and Gorsuch jutted his chin.
``Rubbish,'' he said.

In his book, Gorsuch asks rhetorically if there's any reason to ``only
sometimes adhere'' to the original meaning of the Constitution, and he
answers: ``For my part, I can think of none.'' This is a significant
shift. In contrast to Scalia's confession of fainthearted originalism
(which Scalia himself repudiated in 2013), Gorsuch professes to be
absolutist on the matter. He argues that to make an exception would be
to fall into a trap: ``The more leeway a judge is given, the more likely
the judge will engage, consciously or not, in motivated reasoning or
bias in reaching a result.''

The challenge, then, is to stick with the theory, even if it leads to a
result you don't like. But rather than facing up to archaic and
politically inconvenient results that originalism can dictate, Gorsuch
tends to wave them away. In his book, he addresses the charge that an
originalist reading of the Constitution could prevent a woman from
becoming president. Article II of the Constitution, after all, calls the
chief executive ``he.'' But Gorsuch says it's ``nonsense'' to think the
plain meaning of the text restricts the presidency to men, because
```he' served as a standard pronoun of indefinite gender'' when the
Constitution was written and ratified.

Some scholars are skeptical of Gorsuch's reading of Article II. In the
revolutionary era and after, the plain meaning of ``he,'' in context,
was understood to refer only to men. At the time, the use of ``he''
would have given states, if they wanted it, a basis for blocking women
from appearing on the presidential ballot, Akhil Amar, a professor at
Yale Law School, told me. ``The framers' Constitution allowed states to
bar women (and many men) from voting and holding office --- and
originalism ties its meaning now to \emph{that} world,'' Reva Siegel,
also a Yale law professor, says.

Last June, Gorsuch issued his most significant originalist opinion to
date, in Gundy v. United States, a case dealing with Congress's power to
broadly delegate policymaking authority to federal agencies. In a
dissent, Gorsuch picked up on a solo concurrence Thomas wrote in 2015
and argued that the interpretation of the Constitution that has allowed
Congress to do this --- in regulating everything from air and water
quality to banking and food safety --- is ``at war with its text and
history.''

Gorsuch said the problem mostly came from a line of cases in the 1940s,
following the New Deal expansion of government. He presented his view,
which is known as the ``nondelegation doctrine,'' as the proper original
understanding of the constitutional separation of powers between the
legislative and executive branches.

But a body of scholarship discussed in an amicus brief in Gundy belies
Gorsuch's interpretation. For example, a 2017 review of every relevant
court challenge before 1940 showed that Congress has delegated
policymaking authority to the executive branch since the founding era.
One of the review's authors, the Princeton politics professor Keith
Whittington, is a member of the Federalist Society. He and Jason
Iuliano, a law professor at Villanova University, concluded that ``the
nondelegation doctrine never actually constrained expansive delegations
of power.''

Gorsuch ignored that research, citing only a minority of scholars who
agree with him. ``I admire Justice Gorsuch's writing,'' Cass Sunstein, a
Harvard law professor and former Obama-administration official, told me.
``But his discussion in Gundy isn't close to historical standards.
There's a ton of terrific work on the nondelegation doctrine, and he
cites none of it. Then there is some not-terrific material, which he
does cite.''

In February, Nicholas Bagley and Julian Mortenson, law professors at the
University of Michigan, released a new review based on thousands of
pages of documents from the early Congresses. ``There was no
free-standing non­delegation doctrine at the founding,'' they concluded,
``and the question isn't close.'' Nonetheless, the issue will probably
arise again. The court was short a justice in Gundy, because Kavanaugh
hadn't been confirmed, and Gorsuch didn't win a majority. But last
November, Kavanaugh praised Gorsuch's Gundy opinion, sending a signal to
lawyers to bring a new case.

\textbf{Perhaps the most} significant case on the court's docket this
year is about the subject that gave rise to originalism in the first
place: abortion. On March 4, the court will hear June Medical Services
v. Russo, a challenge to a Louisiana law requiring abortion providers to
obtain admitting privileges to local hospitals. There are only three
clinics left in the state, and if the law takes effect, two of them say
they will close, because no local hospital will grant them admitting
privileges. That would leave only one provider in a state with nearly
one million women of reproductive age.

In 2016, Kennedy and the court's four liberals struck down an identical
Texas provision, based on a scientific consensus that the requirement
isn't medically necessary and ultimately harms women by preventing them
from accessing a safe procedure. The only thing that has changed in the
four years since the Texas decision is the court's composition. The new
case could be a means for the conservatives to begin dismantling the
constitutional protections for abortion that the court has built, brick
by contested brick, over decades of decisions that began with Roe v.
Wade.

Roe was rooted in a 1965 precedent, Griswold v. Connecticut. In
Griswold, the court derived a right to privacy for marital relations
from what it confusingly called ``penumbras, formed by emanations'' in
the Bill of Rights and the 14th Amendment, striking down a law that
banned the use of birth control, including for married couples. Justice
Hugo Black, the stickler for the text of the era, dissented. ``I get
nowhere in this case by talk about a constitutional `right of privacy'
as an emanation from one or more constitutional provisions,'' Black
wrote.

The critique sank in. When Justice Harry Blackmun wrote the majority
opinion in Roe, he refashioned a right to privacy ``founded in the 14th
Amendment's concept of personal liberty and restrictions upon state
action'' that was ``broad enough to encompass a woman's decision whether
or not to terminate her pregnancy.''

Later, the Supreme Court established other underpinnings for the right
to access abortion. In the 1992 case Planned Parenthood v. Casey, three
justices appointed by Republican presidents --- Kennedy, Sandra Day
O'Connor and David Souter --- devised a compromise that allowed the
states to regulate Roe, but only if they did not impose an ``undue
burden'' on women seeking abortions. Returning to the 14th Amendment,
the justices wrote: ``The controlling word in the cases before us is
`liberty.''' The court invoked gender equality, saying that the right to
decide whether and when to have a child is essential to a woman's
ability ``to participate equally in the economic and social life of the
nation.'' As Linda Greenhouse and Reva Siegel wrote in the 2019 book
``Reproductive Rights and Justice Stories,'' ``Respect for the equal
citizenship of women appears centrally in the opinion.'' It took 20
years, and perhaps a female justice, but the court saw a direct
connection between reproductive freedom and equality. The current
conservative majority, however, may undo it.

Image

Brett Kavanaugh suggested in a 2017 speech that he didn't have one
methodology and that ``history and tradition'' competed for ``primacy of
place'' with factors like liberty and deference to the
legislature.Credit...Melina Mara/The Washington Post, via Getty Images

\textbf{Some liberals have tried} to find common ground with
conservatives by blurring the boundaries between originalism and an
evolving understanding of the Constitution's open-ended principles.
Justice Elena Kagan promoted this approach to constitutional
interpretation at her Senate confirmation hearings in 2010. ``Sometimes
they laid down very specific rules,'' she said of the framers.
``Sometimes they laid down broad principles. Either way, we apply what
they say, what they meant to do.'' Kagan ended with a line that drained
originalism of its standard meaning: ``In that sense, we are all
originalists.''

Perhaps Kagan sought to disarm her partisan critics in the Senate.
Gorsuch and Kavanaugh may have seen the same benefit when they followed
her lead at their own confirmation hearings. ``I am with Justice Kagan
on this,'' Gorsuch said at his 2017 hearing, when asked for his views on
originalism. Kavanaugh repeated the refrain when it was his turn to
testify: ``As Justice Kagan said, we're all originalists now.''

But now that Gorsuch and Kavanaugh are on the court and the
conservatives are firmly in control, it's hard to see why they would go
along with a liberal effort to co-opt originalism. Kennedy, hardly an
originalist, floated a version of this in his 2015 majority opinion
recognizing a right to marriage equality for gay couples in the case
Obergefell v. Hodges. The framers ``entrusted to future generations a
charter protecting the right of all persons to enjoy liberty as we learn
its meaning,'' he wrote. The conservatives on the court at the time
recoiled, arguing in dissent that the court's decision had ``nothing to
do with'' the Constitution. Outside the court, social conservatives
warned that an ideologically neutral originalism would be useless. ``One
might say that originalism has become a Unitarian Church for the legal
profession,'' Michael Greve, a professor at the Antonin Scalia Law
School at George Mason University, wrote in an essay last July.
``Anybody is welcome, provided you believe there is one Constitution.''

The left also has something to lose if it makes support for originalism,
and textualism, sound like a single widely shared view, when in fact the
conservatives' conception of these theories remains very different from
theirs. More than a decade ago, Justice Stephen Breyer debated Scalia in
a hotel ballroom in Washington. He spoke on behalf of an approach to
judging that went back to Marbury and McCulloch: reading laws and the
Constitution in context to weigh their underlying purpose and the
consequences of interpreting them one way or another. Breyer continues
to argue urgently for this position. Last term, he wrote two solo
dissents --- not his usual practice --- to warn against what he sees as
a words-on-the-page capitulation. He is concerned that rather than
challenging a method that produces constitutional law that few people
would want, liberals are uncritically helping to normalize it and teach
it to the next generation of the legal profession. ``I don't want
textualism to take over the law schools, and I fear it is,'' he told me
this fall. ``The purpose of the law is to work, to work for the
people.''

\begin{center}\rule{0.5\linewidth}{\linethickness}\end{center}

Emily Bazelon is a staff writer for the magazine and the author of
``Charged,'' which is a 2020 finalist for The Los Angeles Times Book
Prize in the current-interest category and the Helen Bernstein Book
Award for Excellence in Journalism from the New York Public Library.

Sources for photographs in top video: Thomas, Roberts and Gorsuch from
Mandel Ngan/Agence France-Presse, via Getty Images; Alito and Kavanaugh
from Chip Somodevilla/Getty Images

Advertisement

\protect\hyperlink{after-bottom}{Continue reading the main story}

\hypertarget{site-index}{%
\subsection{Site Index}\label{site-index}}

\hypertarget{site-information-navigation}{%
\subsection{Site Information
Navigation}\label{site-information-navigation}}

\begin{itemize}
\tightlist
\item
  \href{https://help.nytimes3xbfgragh.onion/hc/en-us/articles/115014792127-Copyright-notice}{©~2020~The
  New York Times Company}
\end{itemize}

\begin{itemize}
\tightlist
\item
  \href{https://www.nytco.com/}{NYTCo}
\item
  \href{https://help.nytimes3xbfgragh.onion/hc/en-us/articles/115015385887-Contact-Us}{Contact
  Us}
\item
  \href{https://www.nytco.com/careers/}{Work with us}
\item
  \href{https://nytmediakit.com/}{Advertise}
\item
  \href{http://www.tbrandstudio.com/}{T Brand Studio}
\item
  \href{https://www.nytimes3xbfgragh.onion/privacy/cookie-policy\#how-do-i-manage-trackers}{Your
  Ad Choices}
\item
  \href{https://www.nytimes3xbfgragh.onion/privacy}{Privacy}
\item
  \href{https://help.nytimes3xbfgragh.onion/hc/en-us/articles/115014893428-Terms-of-service}{Terms
  of Service}
\item
  \href{https://help.nytimes3xbfgragh.onion/hc/en-us/articles/115014893968-Terms-of-sale}{Terms
  of Sale}
\item
  \href{https://spiderbites.nytimes3xbfgragh.onion}{Site Map}
\item
  \href{https://help.nytimes3xbfgragh.onion/hc/en-us}{Help}
\item
  \href{https://www.nytimes3xbfgragh.onion/subscription?campaignId=37WXW}{Subscriptions}
\end{itemize}
