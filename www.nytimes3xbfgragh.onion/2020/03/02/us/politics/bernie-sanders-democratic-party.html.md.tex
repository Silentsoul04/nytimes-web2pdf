Sections

SEARCH

\protect\hyperlink{site-content}{Skip to
content}\protect\hyperlink{site-index}{Skip to site index}

\href{https://www.nytimes3xbfgragh.onion/section/politics}{Politics}

\href{https://myaccount.nytimes3xbfgragh.onion/auth/login?response_type=cookie\&client_id=vi}{}

\href{https://www.nytimes3xbfgragh.onion/section/todayspaper}{Today's
Paper}

\href{/section/politics}{Politics}\textbar{}How the Democratic
Establishment Stumbled as Sanders Surged

\url{https://nyti.ms/2ThuPDq}

\begin{itemize}
\item
\item
\item
\item
\item
\end{itemize}

\begin{itemize}
\item
  \href{https://www.nytimes3xbfgragh.onion/live/2020/09/08/us/trump-vs-biden?action=click\&pgtype=Article\&state=default\&region=TOP_BANNER\&context=storylines_menu}{Election
  Updates}
\item
  \href{https://www.nytimes3xbfgragh.onion/interactive/2020/us/elections/election-states-biden-trump.html?action=click\&pgtype=Article\&state=default\&region=TOP_BANNER\&context=storylines_menu}{Paths
  to 270}
\item
  \href{https://www.nytimes3xbfgragh.onion/interactive/2020/08/31/us/politics/vote-by-mail-deadlines.html?action=click\&pgtype=Article\&state=default\&region=TOP_BANNER\&context=storylines_menu}{Voting
  by Mail}
\item
  \href{https://www.nytimes3xbfgragh.onion/interactive/2019/us/elections/2020-presidential-election-calendar.html?action=click\&pgtype=Article\&state=default\&region=TOP_BANNER\&context=storylines_menu}{Key
  Dates}
\item
  \href{https://www.nytimes3xbfgragh.onion/newsletters/politics?action=click\&pgtype=Article\&state=default\&region=TOP_BANNER\&context=storylines_menu}{Politics
  Newsletter}
\end{itemize}

Advertisement

\protect\hyperlink{after-top}{Continue reading the main story}

Supported by

\protect\hyperlink{after-sponsor}{Continue reading the main story}

\hypertarget{how-the-democratic-establishment-stumbled-as-sanders-surged}{%
\section{How the Democratic Establishment Stumbled as Sanders
Surged}\label{how-the-democratic-establishment-stumbled-as-sanders-surged}}

Mr. Sanders is making the case that his urgent message is the best match
for a trying political moment, and moderates are scrambling ahead of
Super Tuesday to stop his momentum.

\includegraphics{https://static01.graylady3jvrrxbe.onion/images/2020/03/03/us/politics/03dems-reconstruct5/merlin_169875378_490f1071-a827-4ea3-8695-e426034d8258-articleLarge.jpg?quality=75\&auto=webp\&disable=upscale}

\href{https://www.nytimes3xbfgragh.onion/by/jonathan-martin}{\includegraphics{https://static01.graylady3jvrrxbe.onion/images/2018/11/06/multimedia/author-jonathan-martin/author-jonathan-martin-thumbLarge.png}}\href{https://www.nytimes3xbfgragh.onion/by/alexander-burns}{\includegraphics{https://static01.graylady3jvrrxbe.onion/images/2018/09/25/multimedia/author-alexander-burns/author-alexander-burns-thumbLarge-v2.png}}

By \href{https://www.nytimes3xbfgragh.onion/by/jonathan-martin}{Jonathan
Martin} and
\href{https://www.nytimes3xbfgragh.onion/by/alexander-burns}{Alexander
Burns}

\begin{itemize}
\item
  March 2, 2020
\item
  \begin{itemize}
  \item
  \item
  \item
  \item
  \item
  \end{itemize}
\end{itemize}

Late last year a group of first-term House Democrats, anxious over the
party's fractious presidential race, convened a series of discussions
intended to spur unity. Led by Representatives Colin Allred of Texas and
Haley Stevens of Michigan, they considered issuing a collective
endorsement of one moderate candidate.

The group held phone calls with
\href{https://www.nytimes3xbfgragh.onion/2020/03/03/podcasts/the-daily/the-stakes-on-super-tuesday.html}{Joseph
R. Biden Jr.}, Amy Klobuchar and Pete Buttigieg. But the lawmakers could
not agree: Some were torn between the options, and others worried about
alienating voters at home who backed other contenders, like Senators
\href{https://www.nytimes3xbfgragh.onion/interactive/2020/us/elections/bernie-sanders.html}{Bernie
Sanders} and
\href{https://www.nytimes3xbfgragh.onion/interactive/2020/us/elections/elizabeth-warren.html}{Elizabeth
Warren}. A few issued solo endorsements of Mr. Biden, but the grander
plan disintegrated.

``There was not time to reach consensus over one candidate,'' said Ms.
Stevens, who eventually endorsed
\href{https://www.nytimes3xbfgragh.onion/interactive/2020/us/elections/michael-bloomberg.html}{Michael
R. Bloomberg}, recalling the ``fast-moving'' blur of the lead-up to
Iowa.

That effort was just one in a series of abandoned or ineffective plans
to rally the moderate wing of the Democratic Party, and the leaders and
institutions of the political establishment, behind a single formidable
contender who could stop the ascent of Mr. Sanders, a democratic
socialist promising a revolution in government.

A gambit to have the powerful Democratic machine in Nevada back Mr.
Biden, for instance, fizzled when the former vice president finished
fifth in New Hampshire. Two outside spending groups formed to target Mr.
Sanders were unable to inflict damage.

Now, on the eve of
\href{https://www.nytimes3xbfgragh.onion/live/2020/super-tuesday-03-03}{Super
Tuesday}, when Democrats across 15 states and territories will hand out
more than a third of the delegates required to claim the nomination, Mr.
Sanders is within reach of a clear national lead and Mr. Biden is racing
to catch up.

In the last few days, moderate Democrats acting with a new sense of
urgency have begun a large-scale effort to coalesce around Mr. Biden,
with
\href{https://www.nytimes3xbfgragh.onion/2020/03/01/us/politics/pete-buttigieg-drops-out.html}{Mr.
Buttigieg leaving the race on Sunday} and
\href{https://www.nytimes3xbfgragh.onion/2020/03/02/us/politics/amy-klobuchar-drops-out.html}{Ms.
Klobuchar abandoning her own campaign Monday}. In a sign of the new, if
frantic, spirit of unity, the two bitter foes both endorsed the former
vice president at campaign events on Monday night in Dallas.

But Mr. Sanders has become a formidable front-runner. The Vermont
senator has put forth a sweeping set of policy prescriptions --- far
more ambitious than anything Mr. Biden and other moderates have proposed
--- to address defining concerns for many Democrats, like economic
inequality and the soaring cost of health care and higher education.

His agenda has galvanized liberal voters yearning for change. Mr.
Sanders finished atop the first three voting states before trailing Mr.
Biden in South Carolina, and polls show him ahead in the delegate-rich
state of California on Tuesday.

He holds significant financial and organizational advantages that he was
able to accumulate over a long season of disarray among traditional
Democrats. And he has begun to make an increasingly direct case to the
rest of the party that his urgent message is the best match for a trying
political moment, and that he is the candidate most prepared to do
battle with President Trump.

Interviews with more than 100 Democratic elected officials, campaign
strategists, union leaders and donors revealed a party establishment
**** that spent many months distressed about the implications of
nominating Mr. Sanders but frozen in a state of anxiety over who would
be the best alternative.

Many of the most influential officials, organizations and donors in the
party remain torn between their concern about Mr. Sanders's chances in
the general election, their fear of antagonizing his supporters, and
their belated and often-rueful recognition that he has assembled a
political movement with appeal well beyond the youthful, left-wing base
he built in 2016.

Top Democrats now believe that there are only two realistic paths
forward in the presidential race: a dominant victory on Tuesday by Mr.
Sanders that **** gives him a wide lead in the delegate count, or a
battle for delegates over months of primary elections, that might allow
Mr. Biden to pull ahead or force the nomination to be decided at the
Milwaukee convention in July.

\includegraphics{https://static01.graylady3jvrrxbe.onion/images/2017/01/29/podcasts/the-daily-album-art/the-daily-album-art-articleInline-v2.jpg?quality=75\&auto=webp\&disable=upscale}

\hypertarget{listen-to-the-daily-inside-the-mind-of-a-super-tuesday-voter}{%
\subsubsection{Listen to `The Daily': Inside the Mind of a Super Tuesday
Voter}\label{listen-to-the-daily-inside-the-mind-of-a-super-tuesday-voter}}

One suburban voter --- who has a personal history with Joe Biden ---
asks himself which moderate candidate has the best chance of beating
President Trump.

transcript

Back to The Daily

bars

0:00/35:10

-35:10

transcript

\hypertarget{listen-to-the-daily-inside-the-mind-of-a-super-tuesday-voter-1}{%
\subsection{Listen to `The Daily': Inside the Mind of a Super Tuesday
Voter}\label{listen-to-the-daily-inside-the-mind-of-a-super-tuesday-voter-1}}

\hypertarget{hosted-by-michael-barbaro-produced-by-jessica-cheung-clare-toeniskoetter-and-rachel-quester-with-help-from-robert-jimison-and-edited-by-lisa-tobin}{%
\subsubsection{Hosted by Michael Barbaro; produced by Jessica Cheung,
Clare Toeniskoetter and Rachel Quester; with help from Robert Jimison;
and edited by Lisa
Tobin}\label{hosted-by-michael-barbaro-produced-by-jessica-cheung-clare-toeniskoetter-and-rachel-quester-with-help-from-robert-jimison-and-edited-by-lisa-tobin}}

\hypertarget{one-suburban-voter--who-has-a-personal-history-with-joe-biden--asks-himself-which-moderate-candidate-has-the-best-chance-of-beating-president-trump}{%
\paragraph{One suburban voter --- who has a personal history with Joe
Biden --- asks himself which moderate candidate has the best chance of
beating President
Trump.}\label{one-suburban-voter--who-has-a-personal-history-with-joe-biden--asks-himself-which-moderate-candidate-has-the-best-chance-of-beating-president-trump}}

\begin{itemize}
\item
  michael barbaro\\
  From The New York Times, I'm Michael Barbaro. This is ``The Daily.''
  Today:
\item
  archived recording (crowd)\\
  Bernie, Bernie, Bernie, Bernie!
\end{itemize}

michael barbaro

In the first weeks of the Democratic primaries, Bernie Sanders was the
only candidate to win big. And to win repeatedly.

\begin{itemize}
\tightlist
\item
  archived recording (bernie sanders)\\
  We won the popular vote in Iowa, we won the New Hampshire primary, and
  according to three networks and the A.P., we have now won the Nevada
  caucus! {[}CHEERING{]}
\end{itemize}

michael barbaro

In Nevada, he did it through sheer popularity by proving he'd built a
coalition, one made up of union workers, people with college degrees and
Hispanic voters. In Iowa and New Hampshire, he did it by narrowly
beating out his moderate rivals who split the majority of the vote, with
the leading moderate candidate, Joe Biden, losing support and stumbling
badly.

\begin{itemize}
\tightlist
\item
  archived recording\\
  The only gas that powers a candidate to the nomination is these
  delegates, and Sanders now has 43 of them. That's more than his
  closest rival Pete Buttigieg at 26. And now you can see Sanders also
  has more than triple the number of delegates of most other people in
  this race.
\end{itemize}

michael barbaro

All of this left Sanders the front-runner ---

\begin{itemize}
\tightlist
\item
  archived recording\\
  Down-ballot Democrats are very worried about Bernie ending up at the
  top of the ballot, but there's speculation that the rest of the folks
  --- the moderates and others --- can't get it together, and they can't
  stop Bernie.
\end{itemize}

michael barbaro

--- and the only one who could walk away from Super Tuesday, today, as
the likely Democratic nominee.

Then came South Carolina.

\begin{itemize}
\tightlist
\item
  archived recording (joe biden)\\
  Just days ago, the press and the pundits had declared this candidacy
  dead. Now, thanks to all of you, the heart of the Democratic Party, we
  just won and we've won big because of you.
\end{itemize}

michael barbaro

And for the first time ---

\begin{itemize}
\tightlist
\item
  archived recording (crowd)\\
  Let's go Joe! Let's go Joe! Let's go Joe!
\end{itemize}

michael barbaro

--- Joe Biden offered a vision of a different coalition, this one made
up of black voters, older voters and suburbanites.

\begin{itemize}
\tightlist
\item
  archived recording\\
  Joe Biden lapped the field with 48 percent of the vote, winning every
  single county in the state, crushing Bernie Sanders by more than a 2-1
  margin, with everyone else trailing badly. Biden's big win helped him
  close the delegate gap considerably with Sanders. It's very close.
\end{itemize}

michael barbaro

But South Carolina was just three days ago.

\begin{itemize}
\tightlist
\item
  archived recording\\
  Despite Biden's big night, the confetti could stop for him on Super
  Tuesday, as Sanders is much better organized and positioned to take a
  potentially commanding delegate lead.
\end{itemize}

michael barbaro

And so the question now is, will Sanders sweep? Or, will the moderate
wing of the party coalesce around Biden, and will the voters who had
left him come back in time for Super Tuesday? It's Tuesday, March 3.

\begin{itemize}
\item
  michael barbaro\\
  We are on a residential street in Arlington, Virginia. It's a quite
  lovely neighborhood of colonial houses. I think it's safe to say
  well-off. An affluent suburban community.
\item
  clare toeniskoetter\\
  Can I do one of my favorite things?
\item
  michael barbaro\\
  Which is what?
\item
  clare toeniskoetter\\
  To look up how much these homes sell for?
\item
  michael barbaro\\
  You would not be the first New Yorker to do that.
\item
  clare toeniskoetter\\
  So this house right behind us --- four beds, three baths, 2,600 square
  feet. Estimated price: \$1.3 million.

  This house here, \$1.3 million.

  No one's ever going to invite me to their home after hearing this.
  {[}LAUGHS{]} They've still got a Christmas tree.
\item
  knocking on door
\item
  interposing voices\\
  Oh! Cute dog. Hi.
\end{itemize}

michael barbaro

You must be Brian.

\begin{itemize}
\item
  brian keane\\
  I'm Brian.
\item
  michael barbaro\\
  I'm Michael.
\item
  brian keane\\
  Hey, Michael. How are you?
\item
  michael barbaro\\
  Jess, Clare.
\item
  brian keane\\
  Jess, how are you? Nice to meet you.
\item
  jessica cheung\\
  Hi! Nice to meet you.
\item
  brian keane\\
  Clare, Brian KEANE. How are you? Nice to meet you.
\item
  michael barbaro\\
  May we come in?
\item
  brian keane\\
  Of course, of course. Nice to meet you.
\end{itemize}

michael barbaro

So on Sunday night I went with my colleagues Jessica Cheung and Clare
Toeniskoetter to meet Brian Keane at his house in Arlington, a suburb
about 15 minutes outside Washington, D.C.

\begin{itemize}
\tightlist
\item
  jessica cheung\\
  So cute in here. They've even got a fire going on.
\end{itemize}

michael barbaro

We get there ---

\begin{itemize}
\tightlist
\item
  kate sawyer keane\\
  I'm Kate. Nice to meet you.
\end{itemize}

michael barbaro

--- and he introduces us to his kids.

\begin{itemize}
\item
  brian keane\\
  So you met Tommy. And this is Julia, and this is Maddie, and this is
  Jack.
\item
  jack keane\\
  Hello.
\item
  brian keane\\
  And you met {[}INAUDIBLE{]} right?
\item
  kate sawyer keane\\
  I just wanted to ---
\item
  michael barbaro\\
  There are so many of you!
\end{itemize}

michael barbaro

All five of them.

And his dog, a yellow lab named Annie. And then we sit down in his
living room.

\begin{itemize}
\item
  michael barbaro\\
  So, with your permission, we'll start.
\item
  brian keane\\
  Yes.
\item
  michael barbaro\\
  Tell me who you are and what you do.
\item
  brian keane\\
  So my name is Brian Keane, and I'm 52 years old. We're here in
  Arlington, Virginia.
\end{itemize}

michael barbaro

So Brian, like a lot of people in Arlington, used to work on Capitol
Hill. His wife is a partner at a law firm that represents, among many
clients, the Democratic National Committee. He now runs a nonprofit
organization and a software company, both of them focused on renewable
energy.

\begin{itemize}
\item
  michael barbaro\\
  So tell me about the politics of this community --- Arlington,
  Virginia.
\item
  brian keane\\
  Sure. It's a wealthy community. And it's a very well-read community,
  if you will. And the reason a lot of people kind of live here is
  because of the politics. And so they move to Washington --- certainly
  that's why I came here, because I love politics.
\item
  michael barbaro\\
  What's the kind of character of people's political identification, if
  you could summarize it?
\item
  brian keane\\
  It's pretty easy to assume that they're left of center.
\item
  michael barbaro\\
  And you said left of center. My sense of this community is that that
  ``center'' word is important. That it's left of center, but it's not
  liberal ``liberal.'' I went back and looked at the 2016 primary for
  Arlington County and saw that Hillary Clinton won this county by a
  2-to-1 ratio over Bernie Sanders. It wasn't really even close. He got
  32 percent. She got twice as much, 66 percent.
\item
  brian keane\\
  We have a sense that if we're going to be voting, we understand that
  you want to vote for somebody who can be elected, and so you don't
  want to kind of just say like, oh, let's just go crazy and vote for
  Mr. Potato Head because I think he's great, and see what happens.
\end{itemize}

michael barbaro

But in the most recent polling, which was conducted before South
Carolina, Sanders is actually leading the state of Virginia with 28
percent of the vote. Joe Biden has 19 percent, and Michael Bloomberg has
17 percent, which, as you can probably gather, is not welcome news to
Brian.

\begin{itemize}
\tightlist
\item
  archived recording (bernie sanders)\\
  Well, look. Let's be clear. The establishment is freaking out. The
  establishment is worried. Wall Street is worried. The drug companies
  are worried. Insurance companies are worried.
\end{itemize}

michael barbaro

When he and his friends hear Sanders going after the establishment ---

\begin{itemize}
\tightlist
\item
  brian keane\\
  When you say Washington's the problem, he's probably talking about us,
  and that's not really well received.
\end{itemize}

michael barbaro

--- they hear that as him going after them.

\begin{itemize}
\item
  brian keane\\
  Yeah, like any industry town would be, right? So if you're going to
  say, the car manufacturers stink, I think Detroit would take umbrage
  to that.

  He's just a little mean. I don't think I'd like him as my neighbor.
\item
  michael barbaro\\
  What about as your Democratic nominee? I mean, what do you ---
\item
  brian keane\\
  I think it would be really, really bad for the party.
\item
  michael barbaro\\
  Why?
\item
  brian keane\\
  I don't think, for every --- and I really, really think Trump is bad
  for America --- a socialist revolution is not what we're asking we
  need. And if we have a nominee to stand up against Trump that says we
  need a socialist revolution, we're going to lose. And we will lose our
  Senate. We'll lose any strong candidates we have in the Senate, and we
  have a lot of strong candidates.
\item
  michael barbaro\\
  You mean it will trickle down over the whole party.
\item
  brian keane\\
  Oh yeah. Badly. Badly.
\end{itemize}

michael barbaro

This is how you hear a lot of establishment Democrats talking right now,
that a Sanders nomination would trigger a meltdown inside the party.

\begin{itemize}
\item
  michael barbaro\\
  Given what you just laid out, who do most of your neighbors tend to
  support in this primary?
\item
  brian keane\\
  So it's, I've been all over the board, though. Like it just moves so
  fast. And my personal thing is so interesting, though, because I've
  always --- my heart has always been like, I'm a Joe Biden guy at
  heart.
\item
  michael barbaro\\
  What do you mean?
\item
  brian keane\\
  I love the guy, but can you win? And then in that case, it's like, so
  I need to go with a guy that's a person who's going to win. And so now
  I'm, like, ping-ponging back and forth of who can win, who can win,
  and I have to give up the guy I love, you know?
\item
  michael barbaro\\
  You say ``I'm a Joe guy'' with so much conviction that I kind of want
  to go back to understand what you mean by that.
\item
  brian keane\\
  Well, I came to Washington, D.C. to go to college here at American
  University. And this is a long story. Can I tell my long story?
\item
  michael barbaro\\
  Please tell a long story. We didn't come here for a short story.
\item
  brian keane\\
  {[}LAUGHS{]} You asked for it. But at American University ---
\end{itemize}

michael barbaro

But, to summarize: Brian went to college at American University, where
he was the head of an undergraduate group that brought speakers to
campus, and Joe Biden was one of them. This was in 1988, and Biden had
just dropped out of his first presidential run after being caught
plagiarizing a speech. And then he had a brain aneurysm that he barely
survived.

\begin{itemize}
\item
  brian keane\\
  And so the speech he gave for the Kennedy Political Union at A.U. was
  his first speech after having been driven out of the race and his
  aneurysm. So he shows up ---
\item
  michael barbaro\\
  You're there, obviously.
\item
  brian keane\\
  Yes, right. So I'm there. I'm just a senior in college, I'm a kid. And
  he shows up. He gets out of this car, and he's like my best friend.
  He's like, Brian, it's so great to see you! You have no idea how glad
  I am to see you. And I'm like, Senator Biden, so nice to meet you. No,
  you have no idea how glad I am to meet you --- uh, to see you. Not
  meet you. You know, he's very good. It's like, in case he has met me
  before, he's like, so glad to see you! And it was like {[}GASPS{]},
  we're best friends. This is amazing.
\item
  archived recording (brian keane)\\
  {[}APPLAUSE{]} Thank you. Good evening, and welcome to tonight's
  program. My name is Brian F. KEANE, and I am the director of the
  Kennedy Political Union.
\item
  brian keane\\
  Then we go to the lecture hall.
\item
  archived recording (brian keane)\\
  Senator Joseph R. Biden, Jr. first entered public service in 1970 when
  he was elected to the Council of Newcastle County --- excuse me --- in
  New Jer--- in Delaware. {[}LAUGHTER{]} Sorry.
\item
  brian keane\\
  And ---
\item
  archived recording (brian keane)\\
  Ladies and gentlemen, Senator Joseph Biden.
\item
  brian keane\\
  --- he gives this unbelievable speech.
\item
  archived recording (joe biden)\\
  Most of we public officials, when we're about to speak somewhere, we
  start off and we say, I'm really happy to be here. Well folks, I'm
  happy to be here. I'm happy to be anywhere in light of {[}LAUGHS{]}
  the last 8 or 10 months. {[}APPLAUSE{]}
\item
  brian keane\\
  And he kept going.
\item
  archived recording (joe biden)\\
  I don't see no leader right now of the Democratic Party who's able to
  unite everyone. The Southern conservatives, the Northern liberals ---
\item
  brian keane\\
  He took every single question that was there.
\item
  archived recording (joe biden)\\
  Well, the answer is yes.
\item
  archived recording (crowd)\\
  {[}LAUGHTER{]}
\item
  archived recording (joe biden)\\
  I'm serious. I'm serious. I'm serious. Don't make me laugh. The answer
  is yes, I do see the possibility of uniting us so that we can win.
\item
  brian keane\\
  They actually had to shut it down. They were like, we got to close the
  room.
\item
  michael barbaro\\
  Senator, thank you very much ---
\item
  brian keane\\
  Exactly. Yeah, we're done.
\item
  archived recording (joe biden)\\
  {[}APPLAUSE{]} Thank you.
\item
  brian keane\\
  And so, they shut the room down. So it's, like, 11 o'clock at night
  now. And he says, hey, is there like a bar around here? And so I said,
  well, actually, there's a tavern on campus. He's like, oh, you got
  time? Let's go get a beer.
\item
  michael barbaro\\
  You're having the proverbial beer with a politician in real life.
\item
  brian keane\\
  And the guy was just great. Like it's just me and Joe. And then he
  just says to the bartender, hey, how late are you guys open till? Last
  call, maybe 2 o'clock. Oh, great, great. Can I borrow your phone?
  Yeah. So he gives him his phone, and Joe dials and he goes, hey Beau,
  it's Dad. I'm here at American University with Brian KEANE. Come on
  over. Get Hunter. Come on over, and let's have a beer with Brian KEANE
  at American University.
\end{itemize}

michael barbaro

{[}LAUGHS{]}

\begin{itemize}
\tightlist
\item
  brian keane\\
  So the boys came over. And really, he was calling for a ride.
  {[}LAUGHS{]} And I was like, that was unbelievable. It was, like, the
  best night of my life.
\end{itemize}

michael barbaro

This is actually not the end of the story.

\begin{itemize}
\item
  brian keane\\
  To make this long story longer, that was 1989. So fast forward to
  2007. There's rumors that he's going to run for president. So I'm
  taking the Amtrak, and I get to 30th Street Station and I go up the
  escalator. And I'm like, I think, just to myself, I'm like, I think
  that's Joe Biden over there. And there's a guy just holding the
  newspaper. He's holding it up. And I said, excuse me, Mr. President.
  And he folds the newspaper down, and he looks at me. He looks me in
  the eye, and he goes, ``American University.''
\item
  michael barbaro\\
  Wow.
\item
  brian keane\\
  And I go, oh my God! How do you remember that? That was 1989. He goes,
  that was a great night, wasn't it? How are you man? How's it going?
  How are you? And I said, well, are you running for president? And he's
  like, well, why not? Like, why not? And so we got coffee, and then we
  waited in line for the Amtrak. And it was unbelievable.
\item
  michael barbaro\\
  What impression did he leave you with?
\item
  brian keane\\
  That he is the best politician. He loves people. And, actually, that
  is what the job is. And if you love people, you're going to do well by
  people.
\item
  michael barbaro\\
  So this time around, 2020, did you know right away? Did it feel like
  he was going to be your candidate?
\item
  brian keane\\
  Well, I said to my wife, well, I got to go with Joe. We're like best
  friends, you know? And she's like, you don't talk to the guy.
\item
  laughter
\item
  brian keane\\
  But I think it's amazing that there's a person that can make you feel
  like that.
\item
  michael barbaro\\
  Right, in the race.
\item
  brian keane\\
  Yes. So Joe enters 2020, and I said, well I got to be with Joe.
\item
  michael barbaro\\
  Partly because of the relationship you had, partly because of his
  politics, all of it?
\item
  brian keane\\
  Yeah, yeah. And that he is a Democrat's Democrat to me.
\end{itemize}

{[}music{]}

michael barbaro

We'll be right back.

\begin{itemize}
\item
  michael barbaro\\
  Tell me about what's happened since then.
\item
  brian keane\\
  So I love Joe. Joe's my guy. But he has not been a strong candidate. I
  think he doesn't look good. I think he's not sounding good. And Joe
  Biden, who is a great guy, is playing by a rulebook that was written
  in 1980 maybe, 1972. And that's really concerning to me because Donald
  Trump's going to fillet him very quickly. I mean, we're seeing these
  debates. Joe Biden's version of kind of hand-to-hand combat is to just
  say it louder, is to yell.
\item
  archived recording (elizabeth warren)\\
  --- things that touch people's lives.
\item
  archived recording (anchor)\\
  Mayor, mayor ---
\item
  archived recording (joe biden)\\
  I agree. Let me she referenced me. I agreed with the great job she
  did, and I went on the floor and got you votes. I got votes for that
  bill. I convinced people to vote for it, so lets get those things
  straight too.
\item
  brian keane\\
  I kind of feel like I'm walking through a nursing home and seeing an
  old friend, and he's just yelling at me. And it's like, you're an old
  friend, why are you yelling? He just says it louder, you know, and
  it's like, stop yelling.
\item
  archived recording (bernie sanders)\\
  --- separate health care plan.
\item
  archived recording\\
  Senator Klobuchar, I'm going to come to you.
\item
  archived recording (joe biden)\\
  My name was mentioned.
\item
  archived recording\\
  I'm going to come to you for 45 seconds. 45 seconds for Vice President
  Biden.
\item
  archived recording (joe biden)\\
  I'm the only guy who's not interrupted here, all right? And I'm going
  to interrupt now. It costs \$30 trillion. Let's get that straight.
\item
  brian keane\\
  It's like he doesn't know how to take on his opponents without just
  yelling it. And he was stumbling on these words.
\item
  archived recording (joe biden)\\
  I agree that everybody, once they, in fact --- anyway, my time is up.
  I'm sorry.
\item
  archived recording\\
  Thank you, Vice President.
\item
  michael barbaro\\
  You just felt he was faltering.
\item
  brian keane\\
  Yeah.
\item
  michael barbaro\\
  So as you're confronting what you see as his weaknesses, you're
  watching these debate performances, it sounds like you're starting to
  flirt with other candidates around this time. And where is your eye
  starting to drift?
\item
  brian keane\\
  To everybody, actually. And then along comes Bloomberg. And there's a
  guy who really does understand social media, and he has the money to
  understand social media. And that's really kind of cool. And on an
  electoral level, he really can go toe-to-toe with Trump on that and
  put his money behind the whole thing to win this thing.
\item
  michael barbaro\\
  So when it comes to Bloomberg, for you he has this combination of
  values and resources. That's the appeal.
\item
  brian keane\\
  Yeah. And as such, he could win.
\item
  michael barbaro\\
  And what about his candidacy so far, if anything, has given you any
  reservations, given you pause?
\item
  brian keane\\
  That he so quickly got off message and kind of got sidetracked.
\item
  michael barbaro\\
  In the debates.
\item
  brian keane\\
  Yeah, in the debates. And to me, the problem there is, oh, so now
  you're going to go one-on-one with Trump? Maybe you're not the guy.
  This is a problem. This all a game-day decision for me, right? So it's
  --- election's Tuesday.

  So, you know, can Trump destroy Bloomberg? Can he destroy Biden? Yes
  and yes, perhaps.

  Who has kind of the better heart, and I think I have that answer,
  because Joe is the guy. I love Joe. Who has the money to be able to do
  this? Bloomberg does. But Biden's victory in South Carolina, game
  changer for me.
\item
  michael barbaro\\
  Really?
\item
  brian keane\\
  Yeah.
\item
  michael barbaro\\
  Why?
\item
  brian keane\\
  I didn't believe he was going to be able to do that. I didn't believe
  him when he said, it's a firewall. And the numbers are unbelievable.
  It's like, wow. Joe did it. So then you come back to Bloomberg, and
  it's like, well, I don't know.
\item
  michael barbaro\\
  Sounds like the thing that matters most you was just victory. Is
  seeing Joe Biden, this person who was faltering onstage, you didn't
  like the way he was presenting, suddenly seeming electable because he
  had won an election?
\item
  brian keane\\
  No.
\item
  michael barbaro\\
  No?
\item
  brian keane\\
  The winning was really important, but that speech ---
\item
  archived recording (joe biden)\\
  Thank you, thank you, thank you, South Carolina!
\item
  brian keane\\
  --- really grabbed me.
\item
  archived recording (joe biden)\\
  Let me talk directly to Democrats across America, especially those who
  will be voting on Super Tuesday.
\item
  brian keane\\
  And it was hopeful, and it laid out an agenda.
\item
  archived recording (joe biden)\\
  If Democrats want to nominate someone who will build on Obamacare, not
  scrap it; take on the N.R.A. and gun manufacturers, not protect them;
  stand up and give the poor a fighting chance and the middle class get
  restored, not raise their taxes and keep the promises we make, then
  join us. And if the Democrats want a nominee who's a Democrat
  {[}CHEERING{]}, a lifelong Democrat, a proud Democrat, an Obama-Biden
  Democrat, join us.
\item
  brian keane\\
  And so now, he's back.
\item
  archived recording (joe biden)\\
  So here's the deal. Let's get back up. We're decent. We're brave.
  We're resilient people. We can believe again. We're better than this
  moment. We're better than this president. So get up, take back our
  country. This the United States of America. There's nothing beyond our
  capacity if we do it together. God bless you all, and may God protect
  our troops. Thank you, thank you, thank you.
\item
  michael barbaro\\
  So at the moment --- this is Sunday night, almost 9 o'clock --- you
  are leaning toward Joe Biden. OK. What are the odds that you change
  your mind between now and Super Tuesday?
\item
  brian keane\\
  It's really amazing because I'm just so back and forth and back and
  forth every day. And every hour of every day.
\item
  michael barbaro\\
  I want to talk about what I'm hearing you say, because I think this is
  a larger question happening in Virginia on Super Tuesday and part of
  the reason why we're here, which is that it seems like aside from
  defeating Trump, which is your number one priority, a big priority for
  you is not having Bernie Sanders be the nominee. Is that fair to say?
\item
  brian keane\\
  A equals B, and B equals C. Because if my number one issue is to beat
  Trump, then we need the candidate who can do that. So that's really
  the calculus that's happening here.
\item
  michael barbaro\\
  But I'm hearing you waffling over these two moderate candidates, Biden
  and Bloomberg. And the latest polling has Sanders with close to what
  he had in 2016 in Arlington. Biden and Bloomberg are around 20 percent
  too, because they're sharing a lot of voters. So if on Super Tuesday,
  Sanders sweeps a bunch of these states, and he wins in places like
  Virginia, it will likely have been because Biden and Bloomberg split
  the vote. And I wonder if you're thinking about that at all.
\item
  brian keane\\
  I am thinking about that. I think every other election has been like,
  oh, I don't watch the polls. I just vote. I vote my heart. I vote the
  issues, you know? But now we're all watching the polls. We're all kind
  of ---
\item
  michael barbaro\\
  Gaming.
\item
  brian keane\\
  It's unbelievable, right? We all think we're on CNN and it's like,
  this is crazy. But we feel like we have to because the end game is
  which of these candidates can beat Trump? And that's why I'm going to
  be looking at the polls.

  And I may actually delay my actual vote on Tuesday, unlike like ---
  Usually, I vote before I go to work. But I may vote after work,
  depending on afternoon polling.

  Because if I want to vote for the candidate --- if it looks like
  Virginia can --- let's see if I can say this right.

  If it looks like Bloomberg and Biden, if one of them is going to knock
  off Sanders in Virginia, I will vote for who can do that.
\end{itemize}

michael barbaro

So to be clear, Brian is saying that on election day, he is going to
watch the early results come in before he goes to place his own vote.
And his logic is that he's going to try to limit the chance that a
moderate split would lead to a Sanders win in Virginia.

\begin{itemize}
\tightlist
\item
  brian keane\\
  I think I'm being part of the solution, because I'm putting so much
  frigging effort into this thing. {[}LAUGHS{]} It's unbelievable, you
  know? And it is. It's amazing. And I think, actually, so many people
  are in my neighborhood and my friends, it's really amazing. And people
  are saying, so what should I do? They're really thinking this through.
\end{itemize}

michael barbaro

As Brian is talking, what he's describing about the efforts that he and
his friends and his neighbors are making is reminding me of what
happened to the Republicans back in 2016. During the same time that
year, the establishment was paralyzed by Donald Trump's success with
voters in the primaries. They were terrified of it, but they didn't
really know what to do about it and kind of did nothing, really. And
then Trump basically ran away with Super Tuesday, and after that, by the
time party leaders got organized and tried to stop him, it was too late.
And people like Brian are a bit like the Democratic version of the Never
Trump movement. They see Sanders as the Trump of the left, and they're
trying to do everything within their power to avoid that from happening
on their side.

\begin{itemize}
\item
  michael barbaro\\
  I keep thinking about the way you've talked about Sanders and the risk
  he poses to the Democratic Party. If Sanders becomes the nominee, if
  it came down to it in a general election, Sanders versus Trump, what
  would you do?
\item
  brian keane\\
  So for me, that's really going to depend on his number two.
\item
  michael barbaro\\
  There's not an automatic ``I'm going to be voting for the Democrat''
  coming out of your mouth right now.
\item
  brian keane\\
  Correct. I might take some solace in a youthful vice president.

  Elizabeth Warren could get me for sure, for sure.
\item
  michael barbaro\\
  Did you ever think you'd be in the position of imagining the
  front-runner in the Democratic presidential primary being a person
  that you can't instantly say, I'd vote for?
\item
  brian keane\\
  I could never imagine that the Democrats would nominate someone who's
  not a Democrat.

  So that's really the answer. And I don't think we'll nominate someone
  who is not a Democrat.
\item
  archived recording (joe biden)\\
  Do you think that it's inevitably hard for someone who lives in a
  community of expensive houses in suburban Virginia to ever really
  understand why people might want to break the system?
\item
  brian keane\\
  No. I totally understand why people would want to break the system,
  and I really know the system doesn't work for so many people. Millions
  and millions of people. It's not a perfect union. I mean, really
  isn't. But we are trying to create a perfect union. That's kind of the
  beauty of the country. To use the language of revolution, it's not
  responsible, it's not helpful.

  It's not really what we should be doing, especially against Trump.
\item
  michael barbaro\\
  Well, thank you very much for your time.
\item
  brian keane\\
  Thank you.
\item
  michael barbaro\\
  I really appreciate it.
\item
  brian keane\\
  Thank you so much.
\end{itemize}

{[}music{]}

\begin{itemize}
\item
  archived recording (amy klobuchar)\\
  Today, I am ending my campaign and endorsing Joe Biden for president.
\item
  archived recording (pete buttigieg)\\
  I am delighted to endorse and support Joe Biden for president.
\item
  archived recording (beto o'rourke)\\
  Tomorrow, March 3, 2020, I will be casting my ballot for Joe Biden.
\end{itemize}

michael barbaro

On Monday night, three former Democratic candidates for president
endorsed Joe Biden simultaneously in a show of force by the party's
moderate wing, as they seek to block Bernie Sanders from becoming the
nominee.

\begin{itemize}
\tightlist
\item
  archived recording (amy klobuchar)\\
  I believe, and it's the reason I'm up here, that we are never going to
  out-divide the divider-in-chief because if we spend the next four
  months dividing our party and going at each other, we will spend the
  next four years watching Donald Trump tear apart this country.
\end{itemize}

michael barbaro

At a rally in Texas, one of the 14 states to hold its primary today, Amy
Klobuchar, who had dropped out of the race hours before, was joined by
Pete Buttigieg, who dropped out a day earlier, and Beto O'Rourke, who
left the race in November.

\begin{itemize}
\tightlist
\item
  archived recording (beto o'rourke)\\
  We need somebody who can beat Donald Trump, and in Joe Biden, we have
  that man. We have someone ---
\end{itemize}

michael barbaro

Sanders, appearing on CNN, dismissed the endorsements and suggested that
they would have little impact.

\begin{itemize}
\tightlist
\item
  archived recording (bernie sanders)\\
  From day one, we have been taking on the establishment. And let me be
  very clear, it is no surprise they do not want me to become president.
  Because our administration will transform this country to create an
  economy and a government that works for all of the people ---
\end{itemize}

{[}music{]}

michael barbaro

We'll be right back.

Here's what else you need to know today.

\begin{itemize}
\tightlist
\item
  archived recording (dr. michael ryan)\\
  There is a point in any epidemic where you believe you can no longer
  contain the virus, like if it was influenza, and you have to shift
  your resources to saving lives.
\end{itemize}

michael barbaro

The coronavirus has now infected people on every continent except
Antarctica. But global health officials said it was still possible for
individual countries to contain the epidemic.

\begin{itemize}
\tightlist
\item
  archived recording (dr. michael ryan)\\
  But in doing that, you're accepting that you can no longer affect the
  course of the disease. You can no longer change the shape of the
  epidemic, and you're purely moving in that sense to save as many lives
  as you can. Now, W.H.O. does not believe that we're there yet.
\end{itemize}

michael barbaro

During a news conference, the leaders of the World Health Organization
said that the time for such containment was running out and urged
countries to take every possible precaution.

\begin{itemize}
\tightlist
\item
  archived recording (dr. michael ryan)\\
  And if we're lucky, and if we do the job really well, we may get the
  opportunity, we just might get the opportunity to interrupt
  transmission, but at the very ---
\end{itemize}

michael barbaro

In Europe, the E.U. raised its alert level from moderate to high as the
number of infections surged, reaching 18 of the E.U.`s 27 nations. In
South Korea, infections have doubled since Friday to more than 4,300.
And in the U.S., as of Monday night, authorities had reported 100 cases
and, in Washington State, six deaths from the virus.

{[}music{]}

That's it for ``The Daily.'' I'm Michael Barbaro. See you tomorrow.

\includegraphics{https://static01.graylady3jvrrxbe.onion/images/2020/03/02/us/politics/02dems-reconstruct2/merlin_169872849_8351fb68-1032-4801-b727-9f0e9156f19f-articleLarge.jpg?quality=75\&auto=webp\&disable=upscale}

The mood of emergency among party leaders has been simmering for months
but it only translated into a surge of support for Mr. Biden at the very
last minute.

``If we go through March 3 and don't show that we have a nominee who can
appeal to a broad swath of our party, we're going to be in serious
trouble,'' said Terry McAuliffe, the former Virginia governor and
Democratic National Committee chairman, who endorsed Mr. Biden over the
weekend.

Even now, authority figures like House Speaker Nancy Pelosi, Senate
Minority Leader Chuck Schumer and former President Barack Obama have
stayed on the sidelines, convinced their intervention would only fuel
Mr. Sanders's claims of an insider plot against him. Influential labor
groups and advocacy organizations have either stayed quiet or supported
multiple candidates.

The candidates themselves have made fitful or flawed efforts to garner
mainstream Democratic support: A number of lawmakers open to endorsing
Mr. Biden said he did little to reach out to them for most of the
campaign, letting weeks or months go by without any contact from him or
his aides.

A flashy late entrant, Mr. Bloomberg, the billionaire former New York
City mayor, initially impressed elected officials with an energetic
charm offensive and a war-machine campaign operation. But many Democrats
developed serious reservations about him after a damaging debate-stage
clash with Ms. Warren last month.

Underdog moderates persisted in pleading with party leaders for help,
promising to continue their campaigns even at the risk of playing a
spoiler's role for Mr. Biden or Mr. Bloomberg. Last week, only three
days before he dropped out of the race, Mr. Buttigieg visited the
Capitol to seek endorsements and vowed in a private meeting with
centrist Democrats that he would not withdraw. It was only after South
Carolina, when internal campaign projections were clear that they faced
a rout on Super Tuesday, that Mr. Buttigieg and Ms. Klobuchar dropped
out.

And while Mr. Sanders has made private overtures toward centrists, his
harpoons toward ``corporate Democrats'' and the scorched-earth tactics
of his supporters have continued to unnerve Democratic leaders.

Christine Pelosi, one of the speaker's daughters, last month contacted
Mr. Sanders's advisers to convey her unhappiness that the actress Susan
Sarandon, who is backing the Vermont senator, had called for Speaker
Pelosi's ouster, according to a Democrat familiar with the
conversations. (Mr. Sanders's campaign manager, Faiz Shakir, indicated
in a message to the younger Ms. Pelosi that Ms. Sarandon wasn't speaking
for the Sanders campaign, according to a Democratic official familiar
with their exchange).

Last Wednesday, the speaker herself seemed to have Mr. Sanders on her
mind. That day, she declared in a news conference that Democrats would
``wholeheartedly support'' whomever their voters nominated.

But in a closed-door session in the Capitol with her leadership team,
Ms. Pelosi's irritation was unmistakable, according to two Democratic
officials familiar with the conversation.

With Mr. Sanders largely ignoring the roster of bills her majority has
passed and boasting that only a true progressive can defeat Mr. Trump,
the speaker said House Democrats proved in 2018 how to win elections;
she cited their legislative accomplishments and dismissed the
``geniuses'' on television who claim the party is shifting sharply to
the left.

Ms. Pelosi also invoked her own liberal credentials: In her basement,
she noted, she still had old campaign signs supporting progressive
causes.

Image

Mr. Biden failed to win powerful Democratic leaders in Nevada to his
side after his poor showings in earlier states. He finished in second
place.Credit...Erin Schaff/The New York Times

\hypertarget{tug-of-war-within-the-establishment}{%
\subsection{Tug of war within the
establishment}\label{tug-of-war-within-the-establishment}}

It was not only the national Democratic establishment that declined to
unite around Mr. Biden or anyone else. Efforts in the states crumbled or
proved ineffective: After the muddled results in Iowa and New Hampshire,
Mr. Biden's campaign hoped to make a strong stand in Nevada, the first
diverse state to weigh in on the nomination.

Mr. Biden's advisers privately projected optimism that they could win
endorsements from the most important forces in Nevada politics: Gov.
Steve Sisolak, a popular moderate; the casino workers union; and Harry
Reid, the former Democratic leader in the Senate. But spooked by Mr.
Biden's debilitating loss in New Hampshire, none of them backed him
before the caucuses. Mr. Sanders won the state by an overwhelming
margin.

As in other states, Democratic leaders said there had been limited
outreach by Mr. Biden. In an interview before the caucuses, Mr. Reid
said he had not heard from him or his campaign chairman, Steve
Ricchetti, in ``several weeks.''

\href{https://www.nytimes3xbfgragh.onion/news-event/2020-election}{Election
2020 ›}

\hypertarget{live-updates}{%
\subsection{\texorpdfstring{\href{https://www.nytimes3xbfgragh.onion/live/2020/09/08/us/trump-vs-biden}{Live
Updates}}{Live Updates}}\label{live-updates}}

\href{https://www.nytimes3xbfgragh.onion/live/2020/09/08/us/trump-vs-biden\#trump-returns-to-a-familiar-theme-denouncing-coronavirus-restrictions}{}

Sept. 8, 2020, 1:18 p.m. ET

\href{https://www.nytimes3xbfgragh.onion/live/2020/09/08/us/trump-vs-biden\#trump-returns-to-a-familiar-theme-denouncing-coronavirus-restrictions}{Trump
returns to a familiar theme: denouncing coronavirus
restrictions.}\href{https://www.nytimes3xbfgragh.onion/live/2020/09/08/us/trump-vs-biden\#amid-reports-of-a-campaign-cash-crunch-trump-says-he-may-fund-the-race-with-his-own-money}{}

Sept. 8, 2020, 12:53 p.m. ET

\href{https://www.nytimes3xbfgragh.onion/live/2020/09/08/us/trump-vs-biden\#amid-reports-of-a-campaign-cash-crunch-trump-says-he-may-fund-the-race-with-his-own-money}{Amid
reports of a campaign cash crunch, Trump says he may fund the race with
his own
money.}\href{https://www.nytimes3xbfgragh.onion/live/2020/09/08/us/trump-vs-biden\#after-trump-suggests-trying-to-vote-by-mail-and-in-person-georgia-threatens-to-prosecute-voters-who-acted-similarly}{}

Sept. 8, 2020, 12:48 p.m. ET

\href{https://www.nytimes3xbfgragh.onion/live/2020/09/08/us/trump-vs-biden\#after-trump-suggests-trying-to-vote-by-mail-and-in-person-georgia-threatens-to-prosecute-voters-who-acted-similarly}{After
Trump suggests trying to vote by mail and in person, Georgia threatens
to prosecute voters who acted similarly.}

Another candidate was more proactive, pursuing Mr. Reid and many other
Democratic leaders with solicitousness: Mr. Bloomberg.

Mr. Bloomberg visited Capitol Hill and dispatched aides around the
country to seek endorsements and outline his plans for the campaign
against Mr. Trump. His advisers have paraded Democrats through their
Times Square headquarters. And even though he was not competing in
Nevada, Mr. Bloomberg visited Mr. Reid and his wife, Landra, at their
home outside Las Vegas.

By early February, Mr. Bloomberg had effectively frozen the support of
other moderates, overwhelming voters with advertising and tantalizing
Democratic power brokers with his vision for the general election. In an
illustration of Mr. Bloomberg's allure even to maverick Democrats, his
advisers reached out to Andrew Yang after he withdrew from the
presidential race to offer counsel about a possible campaign for mayor
of New York City next year.

Yet Mr. Bloomberg has not always lived up to the image of a can-do
executive he presents in his \$500 million advertising campaign.

At a private meeting with Democratic governors in Washington, right
after the Iowa caucuses, Mr. Bloomberg delivered a casual performance
that alternately charmed and puzzled several attendees. He spoke vaguely
about how he would unite the party and win over progressives, suggesting
he could heal wounds by being kind to his critics in public and sending
flowers to their wives, according to multiple people present.

In an awkward exchange, Mr. Bloomberg said he was uncertain why Ms.
Warren, a Massachusetts senator, was faring poorly in the race; when
Gov. Kate Brown of Oregon suggested gender could be a factor, Mr.
Bloomberg said he disagreed. He emerged with no new supporters.

Mr. Bloomberg still secured a stream of endorsements from prominent
Democrats. But his campaign lost a number of defectors because of his
disastrous first debate, which dented his polling numbers and emboldened
his critics.

``You can't sell Bloomberg to the country,'' said Representative Cedric
Richmond of Louisiana, a Biden co-chair. ``How's he going to build a
base with the stuff he's said?''

Image

Michael R. Bloomberg, the former New York mayor,~initially impressed
elected officials with an energetic charm offensive and a war-machine
campaign operation.Credit...Brittainy Newman/The New York Times

Even as the race moved on from their early-state successes, Ms.
Klobuchar and Mr. Buttigieg clung to hope in confounding ways.

Mr. Buttigieg insisted to donors that if he raised enough money, he
could pull off an upset at the convention. Ms. Klobuchar was seen as a
potential contender after her third-place finish in New Hampshire but,
always attuned to her press coverage, she spent time in recent days
sending personal messages to media figures complaining about their
reporting and boasting of her energetic campaign schedule.

Despite their wariness of Mr. Sanders, party leaders have hesitated to
mount an all-out effort against him. A pair of outside-spending efforts
targeting him have struggled to get off the ground: Ad campaigns in Iowa
and Nevada by a pro-Israel group failed to halt Mr. Sanders's advance,
and a second group, known as the Big Tent Project, has managed to cobble
together only a few million dollars.

At a February luncheon for vulnerable lawmakers, Representative Cheri
Bustos of Illinois, the chairwoman of the Democratic Congressional
Campaign Committee, pressed Democrats not to rule out candidates whom
they could feel pressure to support later. Her entreaty came after one
vulnerable Democrat, Representative Anthony Brindisi of New York, told
his hometown paper he would not support Mr. Sanders or Ms. Warren.

But Ms. Bustos is said to be worried about Mr. Sanders, and her
committee plans to conduct polling in swing districts to see how
nominating him might affect House races.

At the same time, Mr. Bloomberg has continued stoking that mood of alarm
as he bids for, if not the heart, then at least the more calculating
head of the Democratic establishment.

In a private briefing in Charleston, S.C., last week, one of Mr.
Bloomberg's senior advisers, Dan Kanninen, made an extensive
presentation detailing the risks of nominating Mr. Sanders. ``Florida is
off the map,'' he said, citing Mr. Sanders's past praise of some of
Fidel Castro's programs, according to one person who shared a detailed
account of the meeting.

In the same session, Mayor Muriel Bowser of Washington, D.C., warned
that Mr. Sanders had a ``cultlike following,'' and claimed Mr. Biden was
no longer a viable option: ``This is a choice between Bernie, Mike and
Trump,'' she said.

Days later, Mr. Biden won South Carolina by nearly 30 points.

Image

Mr. Sanders's loyal coalition reached an enthusiastic consensus about
him long ago.Credit...Erin Schaff/The New York Times

\hypertarget{sanderss-race-to-lose}{%
\subsection{Sanders's race to lose}\label{sanderss-race-to-lose}}

On the eve of Super Tuesday, Mr. Sanders still appears to hold an upper
hand, with a lead in national polls backed by his muscular fund-raising
machine and grass-roots following. On Sunday, he announced that he had
raised \$46.5 million in February alone.

Given Mr. Sanders's history of acrimony with the Democratic
establishment, some in the party are skeptical that he can be induced to
work cooperatively with party leadership. Some of his advisers are eager
to signal that he can, though they calculated they would have a better
chance after what they are hoping will be a breakthrough on Super
Tuesday.

Mr. Sanders has made efforts toward comity in the past: He met several
times last year with more moderate Democrats in Washington, pitching
himself as a nominee they could at least tolerate. Meeting with the New
Democrat Coalition last year, Mr. Sanders conceded that he had no hope
of being their first choice, but said that he hoped they could accept
him eventually, according to a lawmaker who heard his pitch.

Last summer, Mr. Sanders dined with another group of House members to
make the case that his economic message and outsider's approach would
have wide traction in the general election. ``Bernie's argument is,
`Don't pigeonhole me as just a progressive candidate,''' said
Representative Dan Kildee of Michigan, who attended the dinner.

But Mr. Sanders has done little over the last month to reassure skeptics
within his party, striking a posture of defiance that may have helped
accelerate the late movement toward Mr. Biden. Most senior Democrats are
hoping that Mr. Sanders's forward march can be slowed, and some have
urged Mr. Obama to intervene.

Mr. Obama has shown no inclination to do so, reasoning that he must hang
back to preserve his ability to help unify the party at the end of a
messy nominating process.

Mr. Biden's aides see that argument as so much malarkey, and they have
conveyed as much to Mr. Obama's inner circle, arguing that the former
president's leverage would evaporate once Mr. Sanders accrued millions
of votes. They have pleaded with Mr. Obama's camp to distance him from
Mr. Bloomberg, who has run saturation-level advertising showing images
of himself with the former president. When
\href{https://morningconsult.com/2020/02/27/obama-endorsement-primary-poll/}{a
poll came out last week} showing that many Democratic voters believe Mr.
Obama is supporting Mr. Bloomberg, the Biden campaign shared it with Mr.
Obama's aides.

Party leaders are already preparing for the possibility of a contested
convention, and
\href{https://www.nytimes3xbfgragh.onion/2020/02/27/us/politics/democratic-superdelegates.html?searchResultPosition=2}{many
superdelegates have indicated} they are open to selecting a nominee
besides Mr. Sanders if he collects the most delegates but falls short of
a majority.

Tom Perez, the D.N.C. chairman, has been clear in conversations with
allies that he will abide by the current rules and would not resist an
effort by superdelegates to determine the nomination on a second ballot.
Referring to the number of delegates needed for a majority, Mr. Perez
told one Democrat, ``1991 delegates are required,'' and noted that Mr.
Sanders is familiar with the guidelines.

With the identity of the Democratic standard-bearer unknown, Mr. Schumer
has begun to ponder ways of uniting the party behind whoever that person
might be: The Senate leader is particularly focused on the idea of
nominating an African-American woman for vice president, mulling names
like Senator Kamala Harris of California, Stacey Abrams of Georgia and
Representative Val Demings of Florida, according to people who have
spoken with Mr. Schumer.

As long as a contested convention remains possible, that could make it
even harder for any single candidate to put together a decisive primary
coalition.

Mr. Reid of Nevada said he was hopeful the race would not last until
Milwaukee. But if no candidate achieved a delegate majority by the end
of primary voting in June, Mr. Reid said last month that he was prepared
to step in to arrange a deal before the convention in mid-July. In the
run-up to his own state's caucuses, Mr. Reid reached for a conciliatory
role and cautioned Democrats not to put pressure on Mr. Sanders ``to
tone down anything.''

Mr. Trump, he said then, could be defeated by any of the top Democrats.

But nine days after Nevada Democrats voted, and weeks after his
intervention could have resurrected Mr. Biden and slowed Mr. Sanders in
the state, Mr. Reid had evidently reconsidered: On Monday, he issued a
belated endorsement of Mr. Biden, calling him the best candidate ``to
defeat Trump and lead our country following the trauma of Trump's
presidency.''

\hypertarget{our-2020-election-guide}{%
\section{Our 2020 Election Guide}\label{our-2020-election-guide}}

Updated ~Sept. 8, 2020

\begin{center}\rule{0.5\linewidth}{\linethickness}\end{center}

\begin{itemize}
\item ~
  \hypertarget{the-latest}{%
  \subsection{The Latest}\label{the-latest}}

  \begin{itemize}
  \item
    The campaign
    \href{https://www.nytimes3xbfgragh.onion/live/2020/09/08/us/trump-vs-biden?action=click\&pgtype=Article\&state=default\&region=BELOW_MAIN_CONTENT\&context=storylines_guide}{shifts
    to a higher gear this week}, with President Trump set to visit
    Florida and North Carolina today and Joseph R. Biden heading to
    Michigan tomorrow.
  \end{itemize}
\item ~
  \hypertarget{how-to-win-270}{%
  \subsection{How to Win 270}\label{how-to-win-270}}

  \begin{itemize}
  \item
    Joe Biden and Donald Trump need 270 electoral votes to reach the
    White House. Try building
    \href{https://www.nytimes3xbfgragh.onion/interactive/2020/us/elections/election-states-biden-trump.html?action=click\&pgtype=Article\&state=default\&region=BELOW_MAIN_CONTENT\&context=storylines_guide}{your
    own coalition of battleground states}~to see potential outcomes.
  \end{itemize}
\item ~
  \hypertarget{voting-by-mail}{%
  \subsection{Voting by Mail}\label{voting-by-mail}}

  \begin{itemize}
  \item
    Will you have enough time to vote by mail in your state? Yes, but
    it's risky to procrastinate.
    \href{https://www.nytimes3xbfgragh.onion/interactive/2020/08/31/us/politics/vote-by-mail-deadlines.html?action=click\&pgtype=Article\&state=default\&region=BELOW_MAIN_CONTENT\&context=storylines_guide}{Check
    your state's deadline.}
  \item
    \href{https://www.nytimes3xbfgragh.onion/interactive/2020/us/elections/joe-biden.html?action=click\&pgtype=Article\&state=default\&region=BELOW_MAIN_CONTENT\&context=storylines_guide}{}

    \hypertarget{joe-biden}{%
    \section{Joe Biden}\label{joe-biden}}

    \hypertarget{democrat}{%
    \subsection{Democrat}\label{democrat}}

    \href{https://www.nytimes3xbfgragh.onion/interactive/2020/us/elections/donald-trump.html?action=click\&pgtype=Article\&state=default\&region=BELOW_MAIN_CONTENT\&context=storylines_guide}{}

    \hypertarget{donald-trump}{%
    \section{Donald Trump}\label{donald-trump}}

    \hypertarget{republican}{%
    \subsection{Republican}\label{republican}}
  \end{itemize}
\item
  \hypertarget{keep-up-with-our-coverage}{%
  \subsection{Keep Up With Our
  Coverage}\label{keep-up-with-our-coverage}}

  \begin{itemize}
  \item
    Get an
    \href{https://www.nytimes3xbfgragh.onion/newsletters/politics?action=click\&pgtype=Article\&state=default\&region=BELOW_MAIN_CONTENT\&context=storylines_guide}{email}~recapping
    the day's news
  \item
    Download our mobile app on
    \href{https://apps.apple.com/us/app/nytimes/id284862083?ls=1\&mat_click_id=5c79ae7455014fd1bd66b5610c05b8f2-20191112-16948\&referrer=mat_click_id\%3D5c79ae7455014fd1bd66b5610c05b8f2-20191112-16948\%26link_click_id\%3D722930677036718082}{iOS}~and
    \href{http://a.localytics.com/android?id=com.nytimes.android\&referrer=utm_source\%3Dother_nyt_mobile_web\%26utm_medium\%3DWeb\%2520page\%26utm_term\%3DGeneral\%2520Mobile\%2520Page\%26utm_campaign\%3DNYT\%2520Mobile\%2520General\%2520Page}{Android}~and
    turn on Breaking News and Politics alerts
  \end{itemize}
\end{itemize}

Advertisement

\protect\hyperlink{after-bottom}{Continue reading the main story}

\hypertarget{site-index}{%
\subsection{Site Index}\label{site-index}}

\hypertarget{site-information-navigation}{%
\subsection{Site Information
Navigation}\label{site-information-navigation}}

\begin{itemize}
\tightlist
\item
  \href{https://help.nytimes3xbfgragh.onion/hc/en-us/articles/115014792127-Copyright-notice}{©~2020~The
  New York Times Company}
\end{itemize}

\begin{itemize}
\tightlist
\item
  \href{https://www.nytco.com/}{NYTCo}
\item
  \href{https://help.nytimes3xbfgragh.onion/hc/en-us/articles/115015385887-Contact-Us}{Contact
  Us}
\item
  \href{https://www.nytco.com/careers/}{Work with us}
\item
  \href{https://nytmediakit.com/}{Advertise}
\item
  \href{http://www.tbrandstudio.com/}{T Brand Studio}
\item
  \href{https://www.nytimes3xbfgragh.onion/privacy/cookie-policy\#how-do-i-manage-trackers}{Your
  Ad Choices}
\item
  \href{https://www.nytimes3xbfgragh.onion/privacy}{Privacy}
\item
  \href{https://help.nytimes3xbfgragh.onion/hc/en-us/articles/115014893428-Terms-of-service}{Terms
  of Service}
\item
  \href{https://help.nytimes3xbfgragh.onion/hc/en-us/articles/115014893968-Terms-of-sale}{Terms
  of Sale}
\item
  \href{https://spiderbites.nytimes3xbfgragh.onion}{Site Map}
\item
  \href{https://help.nytimes3xbfgragh.onion/hc/en-us}{Help}
\item
  \href{https://www.nytimes3xbfgragh.onion/subscription?campaignId=37WXW}{Subscriptions}
\end{itemize}
