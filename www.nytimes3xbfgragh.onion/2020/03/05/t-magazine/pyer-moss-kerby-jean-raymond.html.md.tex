Sections

SEARCH

\protect\hyperlink{site-content}{Skip to
content}\protect\hyperlink{site-index}{Skip to site index}

\href{https://myaccount.nytimes3xbfgragh.onion/auth/login?response_type=cookie\&client_id=vi}{}

\href{https://www.nytimes3xbfgragh.onion/section/todayspaper}{Today's
Paper}

The Designer Changing How We Think About Fashion and Race in America

\url{https://nyti.ms/38qkpFR}

\begin{itemize}
\item
\item
\item
\item
\item
\item
\end{itemize}

Advertisement

\protect\hyperlink{after-top}{Continue reading the main story}

Supported by

\protect\hyperlink{after-sponsor}{Continue reading the main story}

\hypertarget{the-designer-changing-how-we-think-about-fashion-and-race-in-america}{%
\section{The Designer Changing How We Think About Fashion and Race in
America}\label{the-designer-changing-how-we-think-about-fashion-and-race-in-america}}

Kerby Jean-Raymond's political, narratively rich designs for Pyer Moss
presaged today's gestures at activism on the runway. He still has much
more to say.

\includegraphics{https://static01.graylady3jvrrxbe.onion/images/2020/02/27/t-magazine/27tmag-pyermoss-slide-2M1H/27tmag-pyermoss-slide-2M1H-articleLarge.jpg?quality=75\&auto=webp\&disable=upscale}

By \href{https://www.nytimes3xbfgragh.onion/by/m-h-miller}{M.H. Miller}

\begin{itemize}
\item
  March 5, 2020
\item
  \begin{itemize}
  \item
  \item
  \item
  \item
  \item
  \item
  \end{itemize}
\end{itemize}

THE DESIGNER
\href{https://www.nytimes3xbfgragh.onion/2018/11/06/t-magazine/derrick-adams-kerby-jean-raymond-pyer-moss.html}{KERBY
Jean-Raymond}'s studio is on the Manhattan campus of the Fashion
Institute of Technology on West 27th Street, across from the school's
student center. He's been there for about three years but is in the
process of looking for a larger office in Brooklyn, where he grew up.
Jean-Raymond has outgrown the current space, and not just in a physical
sense; in the time he's been working here, his renown, as the head of
his seven-year-old label,
\href{https://www.nytimes3xbfgragh.onion/2018/11/06/fashion/kerby-jean-raymond-pyer-moss-cfda-vogue-fashion-fund.html}{Pyer
Moss}, but also as a representative of the fashion industry's future,
has grown immensely. What started as a fairly humble streetwear line has
expanded into an ambitious conceptual project. Jean-Raymond is the child
of Haitian immigrants, and his designs --- especially the way he
presents them publicly --- collectively offer a strikingly personal and
singular narrative about his own life as a black designer in America.
He's become successful as a result of this, receiving praise from
critics and counting among his clients and collaborators people like
\href{https://www.nytimes3xbfgragh.onion/2016/02/13/t-magazine/fashion/erykah-badu-styling-debut-pyer-moss-fashion-week.html}{Erykah
Badu}, \href{https://www.usherworld.com/}{Usher},
\href{https://www.nytimes3xbfgragh.onion/interactive/2019/05/20/t-magazine/rihanna-fenty-louis-vuitton.html}{Rihanna}
and
\href{https://www.nytimes3xbfgragh.onion/2016/10/17/t-magazine/michelle-obama-chimamanda-ngozi-adichie-gloria-steinem-letter.html}{Michelle
Obama}.

Throughout his career --- he founded his first fashion label, Mary's
Jungle, when he was 15 --- he's built collections referencing themes
that together read like a checklist of certain generational touchstones:
the beginning of the Iraq War (which inspired a T-shirt line, printed
with slogans like ``We won't fight another rich man's war''), the 2008
financial crisis and its effect on American politics (Pyer Moss's Bernie
vs. Bernie collection,
\href{https://www.vogue.com/fashion-shows/spring-2017-ready-to-wear/pyer-moss}{spring
2017}), the widely documented and institutionally sanctioned murder of
innocent people of color by police (Ota, Meet Saartjie,
\href{https://www.nytimes3xbfgragh.onion/interactive/2016/02/08/fashion/Designers-On-the-Rise-at-New-York-Fashion-Week.html}{spring
2016}), depression and the various pharmaceutical and chemical responses
to it (Double Bind,
\href{https://www.vogue.com/fashion-shows/fall-2016-menswear/pyer-moss}{fall
2016}). But these broad experiences are anchored by an almost novelistic
attention to detail. His clothes, which appear simple upon first glance,
reveal themselves as surprisingly nuanced when examined up close: In his
\href{https://www.nytimes3xbfgragh.onion/2019/09/07/t-magazine/new-york-fashion-week-pictures.html}{spring
2020} collection, which was inspired by the black blues and gospel
singer and guitarist
\href{https://www.nytimes3xbfgragh.onion/1973/10/10/archives/sister-rosetta-tharpe-is-dead-top-gospel-singer-since-1930.html}{Sister
Rosetta Tharpe}, a recurring print of what seems to be an abstract swirl
of black and white is actually a pattern of varying configurations of
the headstock of a Gibson SG Custom, Tharpe's preferred guitar. The
runway show for the spring 2019 collection was held in the Brooklyn
neighborhood of Weeksville, one of the first communities in New York
with black homeowners, and presented an alternate fashion history in
which black people were free to dress how they liked without the threat
of racist judgment or violence. It included colorful clothing printed
with paintings featuring happy domestic scenes by the artist
\href{https://www.nytimes3xbfgragh.onion/2019/11/05/t-magazine/derrick-adams.html}{Derrick
Adams}: a backyard barbecue, a wedding and, on a hand-beaded purple
shift dress, an image of a black father holding his baby. (Jean-Raymond
himself is not married and does not have children.) Occasionally, his
clothes take on the feel of protest signs, like a simple white T-shirt
commanding ``Stop calling 911 on the culture,'' or, as he wore to a
meeting with Vogue's editor in chief, Anna Wintour, in 2018, a black
long-sleeved shirt with the words ``If you are just learning about Pyer
Moss we forgive you.'' He thinks of his clothing as individual works
that people collect, like pieces of art (they range from \$175 for a
T-shirt to about \$5,850 for a pleated wedding gown), and of each
collection as an installation that presents a thesis. But the best word
to describe him might be ``literary'': He values story and context as
much as cut or fit. Even the label's name has a deeper narrative. It is
a lightly modified combination of his mother Vania's maiden name (Moss)
and the name to which she changed it in order to expedite a green card
(Pierre, the same as a cousin's, who was already in America).

Though he makes clothes for men and women, Jean-Raymond's most powerful
statements often concern his relationship to masculinity. He designed a
collection inspired by his father for the
\href{https://www.vogue.com/fashion-shows/fall-2017-ready-to-wear/pyer-moss}{fall
2017} season, the presentation of which was more an act of Proustian
reclamation than it was a typical runway show --- his father's
green-card photo, printed on numerous designs, was Jean-Raymond's
madeleine. Models walked in clothing that reimagined Jean-Raymond's
father as a young man. He had been a pre-med student in Haiti before
immigrating to New York in 1979, and always maintained an effortless
sense of style. The clothes in the collection were deliberately
imperfect, with sleeves that were too long or too short and slacks that
looked nearly unfinished, but in their sheer sense of character, they
were irresistible: wool overcoats with enormous lapels; a maroon
sweatsuit printed with what looked like a college logo but in fact said
``Miragwan,'' the Creole name for the commune in Western Haiti where his
father grew up; a hoodie on which a drawing of the elder Jean-Raymond
appeared, rendered as a slightly deific icon in a portrayal that was
somehow simultaneously ironic and loving. Instead of the common
fashion-show soundtrack, he deployed a 16-person live choir --- named
the Pyer Moss Tabernacle Drip Choir Drenched in the Blood --- that
performed a combination of traditional gospel and pop hits, now a
much-copied fashion-show fixture. Jean-Raymond's clothing can look like
simple sportswear, but it contains complexities that mine the depths of
his memories and attempt to revise them. His self-awareness is so
heightened that he even acknowledges his position as an unreliable
narrator: The show about his father was called
``\href{https://www.nytimes3xbfgragh.onion/2017/02/08/fashion/kerby-jean-raymond-pyer-moss-designer-immigrant-father-new-york-fashion-week.html}{My
Father as I Remember From 1980-1999},'' the first date being six years
before the designer was born.

\includegraphics{https://static01.graylady3jvrrxbe.onion/images/2020/02/27/t-magazine/27tmag-pyermoss-slide-TIE2/27tmag-pyermoss-slide-TIE2-articleLarge.jpg?quality=75\&auto=webp\&disable=upscale}

ON A GRAY afternoon in November, I visited Jean-Raymond at the studio,
where he was conducting a meeting with his 15-person staff. The space
was cramped --- metal racks held clothing from past collections ---
though it also gave off a feeling of utilitarian order. Aside from a
banana plant in the center of the room, few personal touches were
visible --- it was as if he'd already used them up in his work.
Jean-Raymond was wearing a purple
\href{https://www.nytimes3xbfgragh.onion/2018/10/15/t-magazine/alessandro-michele-gucci-interview.html}{Gucci}
sweater, a pair of black Pyer Moss pants and pristine red-and-blue
Reebok sneakers. (Since 2018, Jean-Raymond has had
\href{https://www.reebok.com/us/pyer_moss_collection}{his own line} with
the brand, which includes a popular collection of sneakers.) As he spoke
to four members of his young team, he stood before a series of boards
for his next collection, each tacked with printouts of computer-made
renderings, which he drew and wrote on.

What I noticed immediately was his calm: He wasn't lecturing or forcing
any decisions, but asking questions and listening. If he said he didn't
like a print or a silhouette, one of the staffers would ask him why, and
he would patiently explain himself: He repeated the phrase ``This just
doesn't feel new enough'' like an incantation.

Inspecting a sketch of the Pyer Moss take on basketball shorts, he said,
``I like this, but as a set, with a silk camisole --- like a silk
basketball jersey.'' He produced a notepad printed with the logo for the
Council of Fashion Designers of America --- he is, as of 2019, a member
of the organization's board --- and drew how he wanted it to look:
``It's like a spaghetti strap that becomes basketball.''

``This is too old-fashioned,'' he said, turning now to a design for a
wrap dress. ``I think we need our version of the little black dress.
What does someone wear to some of these {[}expletive{]} galas like the
ones I went to last week?'' (Early November is gala season in New York.)
He thought for a moment and answered his own question: ``You need
something backless. Think about
\href{https://www.nytimes3xbfgragh.onion/2019/12/06/t-magazine/michelle-pfeiffer-perfume-bottles.html}{Michelle
Pfeiffer} coming down the stairs in `Scarface.'''

Seeing the smallness of Jean-Raymond's operation made me consider the
intimacy of his clothing, and how all of it is impossible to separate
from its creator. At 33, he's become something of an elder statesman.
Not so long ago, few students on the F.I.T. campus recognized him; now,
many of them do. As a result, his encounters often involve dispensing
advice to young designers, especially designers of color. This has
become a major part of his morning commute, the reason he's late to
work. Whether they stop him in the street or message him on social media
or find him at a show or a talk, he tells them the same thing: Your
personal story --- your sadness and confusion and losses and victories
--- is your superpower, the one thing that can't be taken from you. It's
your fingerprint, and nothing is worth smearing it. In a business guided
by rote tradition, Jean-Raymond has carved out a unique path as an
artist. At so many popular brands, he'll tell me, the creative director
could leave and be swapped out with someone else and nothing would
really change. But to do work that is authentic, that gets at the core
of who you are, \emph{that} is where true strength lies. There's no Pyer
Moss without Kerby Jean-Raymond; the two are interchangeable.

Image

From left: \textbf{Pyer Moss} robe, \$1,095, and pants. \textbf{Pyer
Moss in Collaboration With Aurora James} boots. \textbf{Pyer Moss} top,
\$225, and pants. \textbf{Pyer Moss in Collaboration With Aurora James}
boots. \textbf{Pyer Moss} robe, price on request, and pants.
\textbf{Pyer Moss in Collaboration With Aurora James}
boots.Credit...Photo by Michelle Sank. Styled by Jason Rider

Later, we talked more about this idea at a vegan restaurant in Chelsea,
where Jean-Raymond is a regular. He has a standing appointment here
every couple of weeks with a group of childhood friends from Flatbush, a
Brooklyn neighborhood that is now arguably more threatened by
gentrification than violence, though that wasn't the case in the '90s,
when Jean-Raymond was growing up there. He told me that of his childhood
circle of friends, only about five, including himself, have not been
stunted by tragedy: Two are dead, three are in prison, and two of those
three won't get out.

Jean-Raymond is as skilled a self-mythologizer in conversation as he is
in his clothes. I sometimes felt that he knew better than I did how the
things he was saying, as he was saying them, would fit into the scope of
a magazine profile. It didn't take him long to begin discussing how, in
every creative industry, old business models --- retail, print magazines
--- are dying, and concepts like inclusion and representation have been
used to offer them a faux sheen of relevance. Progressiveness and
diversity themselves have become commodified as yet another fad that the
web of global media can exploit while patting itself on the back. Yet
few people are as altruistic as they'd like to think, Jean-Raymond said.
He then quoted
\href{https://www.nytimes3xbfgragh.onion/interactive/2017/11/29/t-magazine/jay-z-dean-baquet-interview.html}{Jay-Z}:
``You run a check up but they never give you leverage.'' Meaning, some
enormous conglomerate might give Jean-Raymond a major platform, or lend
him money to do an ambitious project, but they won't let him own his
ideas, and someday soon they'll be back to collect. ``They'll give you
the money,'' he said, ``they'll hire a black face for their company now,
they'll put black models on the runway, they might even put a head of
diversity in charge who's black, but are they giving you leverage to
make decisions for the company?'' He didn't have to answer this
question, and instead interrupted himself with another observation.
``Let me tell you what equality is,'' he said. ``Equality is when we
both'' --- he as a black person, me as a white one --- ``have the right
to be mediocre.''

Jean-Raymond has never had this luxury. His mother died on a trip to
Haiti to visit her mother when he was 7. At first, his father didn't
even tell him she was dead. Jean-Raymond's family didn't think he could
handle it, so every once in a while, they'd put him on the phone with an
aunt who pretended to be his mother. ``I still have no closure as it
pertains to my mom,'' he said. ``So I try to do a lot of things with the
brand that I feel like, from what I can remember, she would be proud
of.'' He recalls his mother cutting squares of cardboard from discarded
boxes and using them to teach her son how to fold his shirts neatly. In
various Pyer Moss designs, one can see a square of fabric stitched to
the back of a piece of clothing, a kind of phantom homage to this
memory.

His father was driving a cab in New York when he won a small amount of
money from a lotto ticket, which he used to become a licensed
electrician. For a while, he worked for General Electric but soon
started his own home-repair business, installing radio systems and
illegal cable boxes for the people in Flatbush who had money. (``Mostly
drug dealers,'' Jean-Raymond said.) At 12, Jean-Raymond's love of Nike
Air Worms, a
\href{https://www.nytimes3xbfgragh.onion/2003/06/01/magazine/no-rebound.html}{Dennis
Rodman} design for the brand, made him resolve to become a designer. He
got his first job at a sneaker store in the neighborhood called Ragga
Muffin, which granted him a certain amount of respect among his peers
--- he was still a child, but he became the go-to for anyone who wanted
a new pair of shoes.

Image

Left: \textbf{Pyer Moss} shirt, price on request, and pants, \$450.
\textbf{Pyer Moss in Collaboration With Aurora James} boots. Right:
\textbf{Pyer Moss} shirt, price on request, and pants. \textbf{Pyer Moss
in Collaboration With Aurora James} boots.Credit...Photo by Michelle
Sank. Styled by Jason Rider

He looked through the directories for public New York high schools until
he found the one that matched his prospective vocation: the High School
of Fashion Industries, located around the corner from where we were now
talking. He was nearly expelled in his first semester for throwing a
container of Wite-Out at a classmate. Instead, he explained himself to
his teacher, telling her that he felt bored and frustrated and wanted
more to do. She helped set him up with a job, and this was how
Jean-Raymond began his career in mainstream fashion, becoming a
14-year-old unpaid intern at the New York City-based
\href{https://www.kayunger.com/}{Kay Unger}. Unger had been a pioneer of
American fashion, creating her own line in 1972, which became known for
its sophisticated but obtainable designs for professional women. She was
a female creative director who had grown a small company into a
\$100-million business, an outlier who would provide a model for
Jean-Raymond of how a supposed outsider could succeed in an unforgiving
industry.

His first day was Sept. 10, 2001.

AFTER THE TERRORIST attacks of 9/11, it took three weeks for
Jean-Raymond to get back to work for his second day. He was so shy,
Unger said, he barely talked, but he also did everything she asked, and
did it well. Soon, she had him start coming in on Saturdays and paying
him for his time. She'd take him to a diner downstairs from her office
on 38th Street, and the more food he'd order, the more he'd open up. He
had a preternatural gift for drawing, and now he'd learn how things were
cut, and what to do with the scraps. Unger sent him all over town:
picking up fabrics, visiting factories.

When I asked him if his age ever posed a problem for him, he asked me,
``What do you think was worse, my age or my race?'' Every time he went
out on an errand, whether it was to a warehouse or a fashion studio, the
lobby security would send him around to the messenger door --- he got so
used to being forced into a side entrance that he stopped checking in at
the front desk altogether. ``I know all the freight entrances in New
York from 35th Street to 41st Street,'' he said. ``Every single
building, from Ninth Avenue all the way over to Sixth.''

The experience was formative. Unger would give Jean-Raymond seed money
for his first line (\$150, which seemed like a lot to someone still not
old enough to drive), and the two would talk about the business,
realizing the extent to which they occupied different industries. There
was the fashion world of Unger and her contemporaries, like
\href{https://www.nicolemiller.com/}{Nicole Miller},
\href{https://www.nytimes3xbfgragh.onion/2018/04/09/t-magazine/michael-kors-80s-photos-inspiration.html}{Michael
Kors},
\href{https://www.nytimes3xbfgragh.onion/2016/08/18/t-magazine/fashion/ralph-lauren-womens-watch.html}{Ralph
Lauren} and
\href{https://www.nytimes3xbfgragh.onion/2018/05/10/fashion/marchesa-harvey-weinstein.html}{Georgina
Chapman}, whose Marchesa line was produced by Unger and which
Jean-Raymond, on the cusp of 17, became an integral part of, and then
there was the fashion world of the sneaker store that Jean-Raymond
worked in back in Brooklyn. Despite their physical proximity to each
other, the two did not overlap. The one wouldn't even let the other
through the front door.

Image

Left: \textbf{Pyer Moss} shirt, \$850, and pants, \$850. \textbf{Pyer
Moss in Collaboration with Aurora James} boots, \$895. Right:
\textbf{Pyer Moss} shirt, \$600, cummerbund and pants, price on request.
\textbf{Pyer Moss in collaboration with Aurora James}
boots.Credit...Photo by Michelle Sank. Styled by Jason Rider

These experiences would be incorporated into Pyer Moss, not just in the
designs themselves and their message but in the way the company is run.
There are certain long-held traditions that an American fashion designer
has had to follow in order to claim success: The designer had to get his
clothes into the department stores Barneys or Bergdorf Goodman; he had
to present collections in New York each season and go to Paris to meet
with European buyers; he had to advertise in Vogue. Black American
designers achieved success via this traditional route (and notably not
in Europe, where there remain almost no black designers helming the top
French and Italian houses) only in exceptional circumstances. Being one
of the most famous people in the world was typically a prerequisite.
Consider some of Jean-Raymond's and Pyer Moss's forebears and
contemporaries:
\href{https://www.nytimes3xbfgragh.onion/topic/person/sean-combs}{Sean
Combs}'s Sean John,
\href{https://www.nytimes3xbfgragh.onion/2015/04/10/t-magazine/kanye-west-adidas-yeezy-fashion-interview.html}{Kanye
West}'s
\href{https://www.nytimes3xbfgragh.onion/2016/02/11/t-magazine/fashion/kanye-west-yeezy-season-3-new-york-fashion-week-fall-2016.html}{Yeezy}
or
\href{https://www.nytimes3xbfgragh.onion/2015/10/12/t-magazine/rihanna-miranda-july-interview.html}{Rihanna}'s
\href{https://www.nytimes3xbfgragh.onion/interactive/2019/05/20/t-magazine/rihanna-fenty-louis-vuitton.html}{Fenty}.
And still, distinctively black brands like
\href{https://www.nytimes3xbfgragh.onion/1999/03/14/style/trying-to-stay-true-to-the-street.html}{FUBU}
or
\href{https://www.nytimes3xbfgragh.onion/1994/03/03/business/company-news-a-new-life-for-cross-colours.html}{Cross
Colours}, which became behemoths
\href{https://www.nytimes3xbfgragh.onion/2015/06/25/t-magazine/nas-fresh-dressed-style.html}{in
the '90s}, were kept on the fringes --- no Vogue spreads, and good luck
finding FUBU denim jackets on the racks at Barneys. These were the
brands that inspired Jean-Raymond when he was growing up --- in recent
years, he has collaborated with FUBU, Cross Colours and Sean John on
capsule collections, and one can see their influence on the designer in
his use of bright primary colors and denim --- though their absence from
the high-end fashion world at the height of their commercial power only
seemed to confirm to Jean-Raymond that he didn't belong in that world,
either.

``Talk about an industry that never advertised for us, that never hired
us at the top, that never acknowledged the fact that we were always
their biggest influencers and ambassadors,'' Jean-Raymond said. (Of the
19 designers on the board of the CFDA, only four identify as black,
including Jean-Raymond.) In the early years of his company, he attempted
to follow the rules, but at a certain point, he realized they would
never work for him. When he held his show about police brutality in
2015, the very fact that a designer was planning on using a runway to
address a controversial topic made headlines before the show's debut,
and his original venue canceled on him, as did almost all of his buyers.
As
\href{https://www.nytimes3xbfgragh.onion/interactive/2019/03/21/t-magazine/fashion-future-history.html}{gestures
at political activism} have become commonplace on runways since the
election of
\href{https://www.nytimes3xbfgragh.onion/topic/person/donald-trump}{Donald
Trump}, the idea of Jean-Raymond inspiring this much contention now
seems absurd. But at the time, it was enough to make him conclude that
success would have to mean something different to him than it did to
everyone else. In the front row of that show, he sat numerous Black
Lives Matter activists, including Emerald Garner, whose father,
\href{https://www.nytimes3xbfgragh.onion/2019/07/16/nyregion/eric-garner-case-death-daniel-pantaleo.html}{Eric},
was filmed pleading ``I can't breathe'' as a Staten Island police
officer choked him to death for selling cigarettes on a street corner in
2014. This meant the usual power brokers were moved to the second or
third row; some of them, upon discovering they'd been downgraded from
their positions of prominence, sent word that they weren't coming unless
they sat up front. Jean-Raymond's response, as he reiterated in a video
interview at the time with Al Jazeera, was: ``Well, {[}expletive{]} you
then.''

JEAN-RAYMOND HAS been trying to do things his own way ever since. In
2018, he became the full owner of his company. He went broke doing this,
but it reset Pyer Moss in a way that's allowed him to operate on his own
terms. Now, he'll regularly skip a season because he wants to work
longer on a project (as he did during New York Fashion Week in February,
which he forwent), and he chooses to focus on his home base of New York
over other markets; he can make more money selling T-shirts at a merch
stand at a show in Brooklyn than he can showing a collection in Paris.
If he doesn't like something, he'll speak up about it, no matter which
powerful figure he might risk alienating.

This happened most recently last fall, at a gala held by the influential
industry publication The Business of Fashion, which Jean-Raymond says
had planned to do a cover story on him and then, after extensive
conversations with him and his team about his business, rescinded the
offer. Then, greeting attendees at the front entrance of the event,
called the BoF 500 and held in Paris in September during the city's
Fashion Week, was a gospel choir that included many black members.
``Homage without empathy and representation is appropriation,''
Jean-Raymond
\href{https://medium.com/@kerbyjeanraymond/peace-3d94209412fb}{wrote in
a statement} following a series of posts on Instagram that
\href{https://fashionista.com/2019/10/kerby-jean-raymond-bof-500-gala-controversy}{went
viral} while the gala was still going on. ``Instead, explore your own
culture, religion and origins. By replicating ours and excluding us ---
you prove to us that you see us as a trend. Like, we gonna die black,
are you?'' This was the exact thing Jean-Raymond spent his career
combating: the fashion industry's flagrant exploitation of those it had
deemed outsiders, monetizing them for profit, making their stories tepid
and meaningless. Imran Amed, BoF's editor in chief, responded in an
800-word
\href{https://www.businessoffashion.com/articles/editors-letter/why-im-listening-to-kerby-jean-raymond}{post
on his site} titled ``Why I'm Listening to Kerby Jean-Raymond,'' in
which he apologized to the designer (while also disagreeing with his
interpretation of the event) and cited his own experience as the gay
child of Muslim immigrants as proof that inclusivity was not a fleeting
concern. That an apology happened at all was, on one level, a testament
to Jean-Raymond's rising stature. But he was also concerned that people
wouldn't understand the deeper implications of this exchange --- that he
would be dismissed as ``just an angry black man, mad about a choir.''

Image

From left: looks from Pyer Moss fall 2016; spring 2016 (2); and spring
2017.Credit...From left: Firstview; Imaxtree (2); Firstview

Image

From left: looks from Pyer Moss fall 2017; spring 2019 (2); and spring
2020.Credit...Firstview (4)

``Let me put it this way,'' he said to me. ``Let's say we're gonna start
a Caribbean food festival. We go to Flatbush and we go to all the top
Caribbean restaurants. And we say to them, `We're gonna have you
headline this food festival, and it's gonna change the trajectory of
your business.' And they're like, `Absolutely, let's do it.' But the
condition is that, for the next four months, we need to know your
recipes, we need to know the makeup of your consumer, we need to know
what you're going to be working on next, because we want to make sure we
have the full story. Then, after all that, they say, `You know, we're
going to go a different route.' Then you show up to the food festival
and it's now sponsored by Whole Foods, Wolfgang Puck, McDonald's. It's
exactly that, and that's what I needed to explain: This was what real
appropriation looks like.'' (``I was really quite devastated and
saddened that he left our interactions feeling the way he did, and I
remain extremely sorry that he would feel that way,'' Amed said. ``If
anything good came out of that situation it just underscored how much
work we can do to address this issue.'')

Perhaps surprisingly, Jean-Raymond has also found himself outlasting
Barneys, which, after a steady decline in business that left the brand
struggling to cover rent on its properties in New York and elsewhere,
will close this year after filing for bankruptcy last summer. (``Good
riddance,'' Jean-Raymond said.) This is the clearest indication that
American fashion is changing and, in some ways, trying to catch up with
Pyer Moss. Jean-Raymond represents a new mold of designer altogether,
and a new moment in fashion in which clothes are arguably no longer the
sole purpose of a brand; a brand also needs to have a message.
Jean-Raymond's is that he never saw himself, or anything resembling
himself, in a majority of contemporary fashion, and this is why his
clothes are so personal to him, why he was so upset about a choir at a
fashion party. His designs are a means of announcing himself, and this
is his main goal --- not money or ownership or praise, but in making
clothes that make him feel seen, and in turn make other people feel seen
as well. The simple gesture of putting his immigrant father on a T-shirt
would not have fit into anyone's idea of ``luxury'' just five years ago,
but the very meaning of the word has changed. Jean-Raymond has succeeded
through talent and resolve, but he also has been fortunate to work at a
time in which the culture has changed enough to reward him for being
true to himself, something that his predecessors were largely denied.

The culmination of Jean-Raymond's career so far came last fall, at his
spring 2020 show, Sister, held during New York Fashion Week at East
Flatbush's
\href{https://www.nytimes3xbfgragh.onion/2014/12/31/realestate/commercial/kings-theater-in-flatbush-set-to-reopen-and-lift-a-neighborhood.html}{Kings
Theater}, an ornate 1920s-era movie palace. Jean-Raymond had returned to
the neighborhood where he grew up. His trademark choir had swelled to 90
members (the size of the choir rises and falls proportionally with how
much money Jean-Raymond has at a given time). The clothes had the
trademarks that are now expected of him: retro lapels, wide-shouldered
silhouettes, silk shirts cut like basketball jerseys, vivid jewel tones
and primary hues. But there was something utopian about this event in a
way that the exclusionary fashion world rarely aspires to. It was a
glimpse at a designer operating with total freedom and confidence ---
not because someone in power had let him but because he himself wanted
to.

Image

Left: \textbf{Pyer Moss} jacket and pants, price on request, and shoes,
\$595. Right: \textbf{Pyer Moss} poncho, \$675, pants, \$450, and shoes,
\$595.Credit...Photo by Michelle Sank. Styled by Jason Rider

IN DECEMBER, I ASKED Jean-Raymond to meet me at the Museum of Modern Art
in New York. Like the American fashion industry, MoMA is a legacy
institution that is reconsidering its priorities in an age of identity
politics. The museum had recently done an extensive rehang of its
collection in a renovated building, offering more space to the artists
--- many of them women and people of color --- who had for so long been
categorically ignored by American museums.

We met in the lobby. He was wearing a striped button-down shirt under a
black jacket with the Pyer Moss label stitched over the heart. He had on
thick glasses and a gold chain, and he looked good. We started in a
gallery devoted to the work of
\href{https://www.nytimes3xbfgragh.onion/2019/09/04/arts/design/betye-saar.html}{Betye
Saar}, a black assemblage artist who began her career in the '60s but,
at age 93, was only now getting attention from museums. We paused at
``\href{https://www.moma.org/collection/works/167631}{Black Girl's
Window},'' a 1969 work painted on a window that had been removed from
its hinges, showing a black face, rendered as a kind of onyx silhouette,
except for a pair of eyes that stare out from the painting eerily. ``I
like these depictions of blackness as hollow outlines,'' Jean-Raymond
said. ``It makes you have to dig deeper, to what's underneath the
blackness.'' This is a concept, he said, he's toying with in his next
show, though he wouldn't say much more than that, beyond the fact that
he has a song in mind that is serving as inspiration:
\href{https://www.nytimes3xbfgragh.onion/topic/person/prince}{Prince}'s
``Insatiable.''

As we walked through the museum, he told me about the house he just
bought near the Columbia Waterfront, in the Carroll Gardens neighborhood
of Brooklyn. While doing a cursory Google search, he discovered that his
neighbors, a white couple, living in the duplicate townhouse next door
with the exact same layout and dimensions, had paid about \$350,000 less
than Jean-Raymond when they purchased it a few weeks earlier. He could
see no reason other than his race that this would have happened, and
when he confronted his broker about it, he denied the accusation but
couldn't come up with a better reason either. He was wondering what to
do with this information, which had now tainted the place, the first
house he's ever owned. He hadn't moved in yet, but now had reservations
about doing so at all.

``You know, I did everything right,'' he told me. ``I rose to the top of
my profession, and there's this feeling that it's still not enough.''

We had arrived at a suite of 1940-41 paintings by
\href{https://www.nytimes3xbfgragh.onion/2000/06/10/arts/jacob-lawrence-dead-82-vivid-painter-who-chronicled-odyssey-black-americans.html}{Jacob
Lawrence}, whom Jean-Raymond said was his favorite artist. Called
``\href{https://www.moma.org/calendar/exhibitions/444}{The Migration
Series},'' they depicted the experience of the wide movement of
African-Americans from the rural South to Northern cities in the years
before World War II. There are only a handful of images in the series of
60 works in which a subject's eyes are open and clearly visible: One of
them is a painting of a woman slicing an enormous slab of fatback bacon,
while a child, his chin propped on the corner of the table, looks on.
It's the child's eyes that are visible, and they don't look excited but
weary, full of ugly knowledge. We lingered here for a time before we
decided to move on.

Models: M'Baye Ndiaye at Muse, Javion Robinson at Timothy Rosado,
Hasheem W at Soul and Nisaa Pouncey at Next Models. Hair by Sabrina
Szinay at M\&A Group. Makeup by Janessa Paré at The Together Company.
Set design by David De Quevedo. Casting by Ricky Michiels. Producer:
Lauren Stocker. Tailoring: Carol Ai. Photo assistant: Tracie Williams.
Stylist's assistant: Noah Delfiner. Makeup assistant: Maggie Mondanile.
Set assistant: Haley Burke.

Advertisement

\protect\hyperlink{after-bottom}{Continue reading the main story}

\hypertarget{site-index}{%
\subsection{Site Index}\label{site-index}}

\hypertarget{site-information-navigation}{%
\subsection{Site Information
Navigation}\label{site-information-navigation}}

\begin{itemize}
\tightlist
\item
  \href{https://help.nytimes3xbfgragh.onion/hc/en-us/articles/115014792127-Copyright-notice}{©~2020~The
  New York Times Company}
\end{itemize}

\begin{itemize}
\tightlist
\item
  \href{https://www.nytco.com/}{NYTCo}
\item
  \href{https://help.nytimes3xbfgragh.onion/hc/en-us/articles/115015385887-Contact-Us}{Contact
  Us}
\item
  \href{https://www.nytco.com/careers/}{Work with us}
\item
  \href{https://nytmediakit.com/}{Advertise}
\item
  \href{http://www.tbrandstudio.com/}{T Brand Studio}
\item
  \href{https://www.nytimes3xbfgragh.onion/privacy/cookie-policy\#how-do-i-manage-trackers}{Your
  Ad Choices}
\item
  \href{https://www.nytimes3xbfgragh.onion/privacy}{Privacy}
\item
  \href{https://help.nytimes3xbfgragh.onion/hc/en-us/articles/115014893428-Terms-of-service}{Terms
  of Service}
\item
  \href{https://help.nytimes3xbfgragh.onion/hc/en-us/articles/115014893968-Terms-of-sale}{Terms
  of Sale}
\item
  \href{https://spiderbites.nytimes3xbfgragh.onion}{Site Map}
\item
  \href{https://help.nytimes3xbfgragh.onion/hc/en-us}{Help}
\item
  \href{https://www.nytimes3xbfgragh.onion/subscription?campaignId=37WXW}{Subscriptions}
\end{itemize}
