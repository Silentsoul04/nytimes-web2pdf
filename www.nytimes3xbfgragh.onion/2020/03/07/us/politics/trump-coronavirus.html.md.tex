Sections

SEARCH

\protect\hyperlink{site-content}{Skip to
content}\protect\hyperlink{site-index}{Skip to site index}

\href{https://www.nytimes3xbfgragh.onion/section/politics}{Politics}

\href{https://myaccount.nytimes3xbfgragh.onion/auth/login?response_type=cookie\&client_id=vi}{}

\href{https://www.nytimes3xbfgragh.onion/section/todayspaper}{Today's
Paper}

\href{/section/politics}{Politics}\textbar{}Inside Trump Administration,
Debate Raged Over What to Tell Public

\url{https://nyti.ms/2PWFalT}

\begin{itemize}
\item
\item
\item
\item
\item
\item
\end{itemize}

\hypertarget{the-coronavirus-outbreak}{%
\subsubsection{\texorpdfstring{\href{https://www.nytimes3xbfgragh.onion/news-event/coronavirus?name=styln-coronavirus-national\&region=TOP_BANNER\&block=storyline_menu_recirc\&action=click\&pgtype=Article\&impression_id=40616690-f297-11ea-96c9-f92c07a4cad0\&variant=undefined}{The
Coronavirus
Outbreak}}{The Coronavirus Outbreak}}\label{the-coronavirus-outbreak}}

\begin{itemize}
\tightlist
\item
  live\href{https://www.nytimes3xbfgragh.onion/2020/09/09/world/covid-19-coronavirus.html?name=styln-coronavirus-national\&region=TOP_BANNER\&block=storyline_menu_recirc\&action=click\&pgtype=Article\&impression_id=40616691-f297-11ea-96c9-f92c07a4cad0\&variant=undefined}{Latest
  Updates}
\item
  \href{https://www.nytimes3xbfgragh.onion/interactive/2020/us/coronavirus-us-cases.html?name=styln-coronavirus-national\&region=TOP_BANNER\&block=storyline_menu_recirc\&action=click\&pgtype=Article\&impression_id=40616692-f297-11ea-96c9-f92c07a4cad0\&variant=undefined}{Maps
  and Cases}
\item
  \href{https://www.nytimes3xbfgragh.onion/interactive/2020/science/coronavirus-vaccine-tracker.html?name=styln-coronavirus-national\&region=TOP_BANNER\&block=storyline_menu_recirc\&action=click\&pgtype=Article\&impression_id=40618da0-f297-11ea-96c9-f92c07a4cad0\&variant=undefined}{Vaccine
  Tracker}
\item
  \href{https://www.nytimes3xbfgragh.onion/2020/09/02/your-money/eviction-moratorium-covid.html?name=styln-coronavirus-national\&region=TOP_BANNER\&block=storyline_menu_recirc\&action=click\&pgtype=Article\&impression_id=40618da1-f297-11ea-96c9-f92c07a4cad0\&variant=undefined}{Eviction
  Moratorium}
\item
  \href{https://www.nytimes3xbfgragh.onion/interactive/2020/09/02/magazine/food-insecurity-hunger-us.html?name=styln-coronavirus-national\&region=TOP_BANNER\&block=storyline_menu_recirc\&action=click\&pgtype=Article\&impression_id=40618da2-f297-11ea-96c9-f92c07a4cad0\&variant=undefined}{American
  Hunger}
\end{itemize}

Advertisement

\protect\hyperlink{after-top}{Continue reading the main story}

Supported by

\protect\hyperlink{after-sponsor}{Continue reading the main story}

\hypertarget{inside-trump-administration-debate-raged-over-what-to-tell-public}{%
\section{Inside Trump Administration, Debate Raged Over What to Tell
Public}\label{inside-trump-administration-debate-raged-over-what-to-tell-public}}

The administration's response to the coronavirus has repeatedly matched
public health experts against a hesitant White House, where worry of
panic dominates.

\includegraphics{https://static01.graylady3jvrrxbe.onion/images/2020/03/08/us/politics/08dc-virusresponse-jp1/merlin_170022120_97321499-691b-4808-8a73-6c4688fe2b0d-articleLarge.jpg?quality=75\&auto=webp\&disable=upscale}

\href{https://www.nytimes3xbfgragh.onion/by/michael-d-shear}{\includegraphics{https://static01.graylady3jvrrxbe.onion/images/2018/06/13/multimedia/author-michael-d-shear/author-michael-d-shear-thumbLarge-v2.png}}\href{https://www.nytimes3xbfgragh.onion/by/sheri-fink}{\includegraphics{https://static01.graylady3jvrrxbe.onion/images/2018/08/24/multimedia/author-sheri-fink/author-sheri-fink-thumbLarge.png}}\href{https://www.nytimes3xbfgragh.onion/by/noah-weiland}{\includegraphics{https://static01.graylady3jvrrxbe.onion/images/2019/07/23/reader-center/author-noah-weiland/author-noah-weiland-thumbLarge.png}}

By \href{https://www.nytimes3xbfgragh.onion/by/michael-d-shear}{Michael
D. Shear}, \href{https://www.nytimes3xbfgragh.onion/by/sheri-fink}{Sheri
Fink} and \href{https://www.nytimes3xbfgragh.onion/by/noah-weiland}{Noah
Weiland}

\begin{itemize}
\item
  Published March 7, 2020Updated March 9, 2020
\item
  \begin{itemize}
  \item
  \item
  \item
  \item
  \item
  \item
  \end{itemize}
\end{itemize}

WASHINGTON --- After weeks of conflicting signals from the Trump
administration about the coronavirus, the government's top health
officials decided late last month that when President Trump returned
from a trip to India, they would tell him they had to be more blunt
about the dangers of the outbreak.

If he approved, they would level with the public.

But Dr. Nancy Messonnier, the director of the National Center for
Immunization and Respiratory Diseases at the Centers for Disease Control
and Prevention, got a day ahead of the plan. At noon on Feb. 25, just as
Mr. Trump was boarding Air Force One in New Delhi for his flight home,
she told reporters on a conference call that life in the United States
was about to change.

``The disruption to everyday life might be severe,'' she said. Schools
might have to close, conferences could be canceled, businesses might
make employees work from home. She had told her own children, she said,
to prepare for ``significant disruption to our lives.''

The stock market plummeted, cable news blared apocalyptic headlines and
by the time Mr. Trump landed at Joint Base Andrews early the next
morning, his critics were accusing him of sowing confusion on an issue
of life or death.

The president immediately got on the phone with Alex M. Azar II, his
secretary of health and human services. That call scared people, he
shouted, referring to Dr. Messonnier's warnings. Are we at the point
that we will have to start closing schools? the president added,
alarmed, according to an official who heard about the call.

To health officials, the message needed to change with the outbreak.
``The epicenter was shifting'' as the number of new cases outside China
surpassed those inside, said Dr. Anne Schuchat, the principal deputy
director of the C.D.C. ``The issue of what this might mean to us became
more important.''

From the beginning, the Trump administration's attempts to forestall an
\href{https://www.nytimes3xbfgragh.onion/2020/03/07/us/coronavirus-cpac.html}{outbreak
of a virus} now spreading rapidly across the globe was marked by a
raging internal debate about how far to go in telling Americans the
truth. Even as the government's scientists and leading health experts
raised the alarm early and pushed for aggressive action, they faced
resistance and doubt at the White House --- especially from the
president --- about
\href{https://www.nytimes3xbfgragh.onion/2020/03/01/business/economy/trump-democrats-coronavirus-economy.html}{spooking
financial markets} and inciting panic.

``It's going to all work out,'' Mr. Trump said as recently as Thursday
night. ``Everybody has to be calm. It's going to work out.''

Health experts say that telling people to remain calm is an effective
message in an epidemic, and it is appropriate that it come from the
president. Clear, honest communication is also crucial, and the United
States has at times criticized China and other governments for being
less than transparent.

\hypertarget{latest-updates-the-coronavirus-outbreak}{%
\section{\texorpdfstring{\href{https://www.nytimes3xbfgragh.onion/2020/09/09/world/covid-19-coronavirus.html?action=click\&pgtype=Article\&state=default\&region=MAIN_CONTENT_1\&context=storylines_live_updates}{Latest
Updates: The Coronavirus
Outbreak}}{Latest Updates: The Coronavirus Outbreak}}\label{latest-updates-the-coronavirus-outbreak}}

Updated 2020-09-09T12:19:05.013Z

\begin{itemize}
\tightlist
\item
  \href{https://www.nytimes3xbfgragh.onion/2020/09/09/world/covid-19-coronavirus.html?action=click\&pgtype=Article\&state=default\&region=MAIN_CONTENT_1\&context=storylines_live_updates\#link-70cea8bb}{As
  drugmakers pledge to thoroughly vet a vaccine, one company pauses its
  trials for a safety review.}
\item
  \href{https://www.nytimes3xbfgragh.onion/2020/09/09/world/covid-19-coronavirus.html?action=click\&pgtype=Article\&state=default\&region=MAIN_CONTENT_1\&context=storylines_live_updates\#link-780eaa2f}{Britain
  is expected to ban gatherings of more than six people.}
\item
  \href{https://www.nytimes3xbfgragh.onion/2020/09/09/world/covid-19-coronavirus.html?action=click\&pgtype=Article\&state=default\&region=MAIN_CONTENT_1\&context=storylines_live_updates\#link-11cec4c0}{Quarantine
  breakdowns at colleges in the U.S. are leaving some at risk.}
\end{itemize}

\href{https://www.nytimes3xbfgragh.onion/2020/09/09/world/covid-19-coronavirus.html?action=click\&pgtype=Article\&state=default\&region=MAIN_CONTENT_1\&context=storylines_live_updates}{See
more updates}

More live coverage:
\href{https://www.nytimes3xbfgragh.onion/live/2020/09/09/business/stock-market-today-coronavirus?action=click\&pgtype=Article\&state=default\&region=MAIN_CONTENT_1\&context=storylines_live_updates}{Markets}

But from Mr. Trump's first comments on the virus in January to
\href{https://www.nytimes3xbfgragh.onion/2020/03/06/us/politics/trump-coronavirus-cdc.html}{rambling
remarks at the C.D.C.} on Friday, health experts say the administration
has struggled to strike an effective balance between encouraging calm,
providing key information and leading an assertive response. The
confused signals from the Trump administration, they say, left Americans
unprepared for a public health crisis and delayed their understanding of
\href{https://www.nytimes3xbfgragh.onion/interactive/2020/us/coronavirus-us-cases.html}{a
virus that has reached} at least 28 states, infected more than 300
people and killed at least 17.

\hypertarget{a-very-big-deal}{%
\subsection{A Very Big Deal}\label{a-very-big-deal}}

\includegraphics{https://static01.graylady3jvrrxbe.onion/images/2020/03/06/us/politics/00dc-virusresponse2/merlin_169516761_ad2cac7a-5ae2-4f42-bfca-04fb76a21c1a-articleLarge.jpg?quality=75\&auto=webp\&disable=upscale}

Mr. Azar was at his home in suburban Washington, on Friday, Jan. 3, when
Dr. Robert R. Redfield, the C.D.C.'s director, called to tell him China
had potentially discovered
\href{https://www.nytimes3xbfgragh.onion/2020/01/08/health/china-pneumonia-outbreak-virus.html}{a
new coronavirus}. Mr. Azar, a former pharmaceutical executive who helped
manage the response to earlier SARS and anthrax outbreaks, told his
chief of staff to make sure that the National Security Council was
aware.

This is a very big deal, Mr. Azar told him.

The Trump administration had eliminated the global health unit that had
been part of the National Security Council, but within days, a team was
meeting daily in the basement of the West Wing, pleading with Chinese
officials to allow doctors from the C.D.C. into their country.

For weeks, the Chinese refused offers of public health cooperation.
``China nice-talked it for a month,'' said Kenneth T. Cuccinelli, a top
official at the Department of Homeland Security who was working on the
coronavirus effort. ```Oh, well, thank you for the offer. Blah, blah.'''

On Saturday, Jan. 18, a day after the C.D.C.
\href{https://www.nytimes3xbfgragh.onion/2020/01/17/health/china-coronavirus-airport-screening.html}{dispatched
100 people to three American airports} to screen travelers coming from
Wuhan, China, Mr. Azar made his first call to Mr. Trump about the virus,
dialing him directly at Mar-a-Lago, his Florida estate. The president
insisted on talking about e-cigarettes first, but Mr. Azar steered him
to the virus.

Four days later, during a two-day trip to the World Economic Forum in
Switzerland, the president chose to focus on the positive.

``We have it under control,'' he said. ``It's going to be just fine.''

Image

An airplane carrying evacuees from Wuhan, China, arrived at March Air
Reserve Base in Riverside, Calif. in January.Credit...Mike Blake/Reuters

On the evening of Jan. 28, a new kind of crisis broke out in the skies.

The State Department had ordered
\href{https://www.nytimes3xbfgragh.onion/2020/01/29/world/asia/wuhan-china-coronavirus-evacuations.html}{the
evacuation of the American Consulate} in Wuhan and a 747 was in the air.
But as it headed for the United States with hundreds of passengers who
possibly carried the virus, administration officials in Washington were
in a frantic scramble about where it should land.

Dr. Robert Kadlec, the assistant health secretary for preparedness and
response, tried to secure some kind of military base in California, but
was struggling to cut through Pentagon red tape. In a panic, his staff
started booking hundreds of rooms at three hotels in the Los Angeles
area, asking for full floors so they could separate potentially infected
evacuees from other guests.

One idea was to land the plane at the Ontario airport outside Los
Angeles, and officials went so far as to schedule, then cancel, a
briefing for some members of the California congressional delegation.
After hours of wrangling, and with the plane still in the air, Mark T.
Esper, the defense secretary, said the plane could land at March Air
Reserve Base in Riverside County, which had space to house all of the
passengers.

Inside the White House, a debate broke out, centered on concerns that
had become ever-present since the virus first emerged: How would the
government's actions be perceived by the public? And what would the
president think?

At issue was whether to impose a federal quarantine order on the
evacuees to prevent them from leaving for 14 days. Such authority had
not been used since a smallpox outbreak in 1969. But officials had to
find some way to make sure the passengers did not leave the base until
it was clear they were not infected.

Mr. Azar pushed for the order but others were wary, concerned it could
cause panic. They decided to ask the passengers to voluntarily stay at
the military base. One woman balked, so California officials, who use
quarantine authority more often, stepped in and forced the passengers to
stay.

\hypertarget{time-to-provoke-china}{%
\subsection{Time to Provoke China?}\label{time-to-provoke-china}}

Image

Patients infected with the coronavirus waiting to be transferred to a
newly built hospital in Wuhan.Credit...Agence France-Presse --- Getty
Images

By the end of January, the virus was veering out of control in China,
the source of 23,000 visitors to the United States each day. Any one of
them could be the trigger for a new and undetected American outbreak.

Over four days in the White House Situation Room, the nation's top
public health and national security officials engaged in a fierce debate
over whether to take the extraordinary step of banning travel from
China.

Public health officials were initially wary. Experts have long
recommended against restricting travel during outbreaks, arguing that it
is often ineffective and can stymie the response by limiting the
movements of doctors and other health professionals trying to contain
the disease. A ban would anger China, they worried, ending any hope of
cooperation with American medical teams.

Officials at the National Security Council and Department of Homeland
Security argued that China had already proved unwilling to cooperate. A
third group inside the White House was worried that the move would
incite panic and could roil the financial markets.

\href{https://www.nytimes3xbfgragh.onion/news-event/coronavirus?action=click\&pgtype=Article\&state=default\&region=MAIN_CONTENT_3\&context=storylines_faq}{}

\hypertarget{the-coronavirus-outbreak-}{%
\subsubsection{The Coronavirus Outbreak
›}\label{the-coronavirus-outbreak-}}

\hypertarget{frequently-asked-questions}{%
\paragraph{Frequently Asked
Questions}\label{frequently-asked-questions}}

Updated September 4, 2020

\begin{itemize}
\item ~
  \hypertarget{what-are-the-symptoms-of-coronavirus}{%
  \paragraph{What are the symptoms of
  coronavirus?}\label{what-are-the-symptoms-of-coronavirus}}

  \begin{itemize}
  \tightlist
  \item
    In the beginning, the coronavirus
    \href{https://www.nytimes3xbfgragh.onion/article/coronavirus-facts-history.html?action=click\&pgtype=Article\&state=default\&region=MAIN_CONTENT_3\&context=storylines_faq\#link-6817bab5}{seemed
    like it was primarily a respiratory illness}~--- many patients had
    fever and chills, were weak and tired, and coughed a lot, though
    some people don't show many symptoms at all. Those who seemed
    sickest had pneumonia or acute respiratory distress syndrome and
    received supplemental oxygen. By now, doctors have identified many
    more symptoms and syndromes. In April,
    \href{https://www.nytimes3xbfgragh.onion/2020/04/27/health/coronavirus-symptoms-cdc.html?action=click\&pgtype=Article\&state=default\&region=MAIN_CONTENT_3\&context=storylines_faq}{the
    C.D.C. added to the list of early signs}~sore throat, fever, chills
    and muscle aches. Gastrointestinal upset, such as diarrhea and
    nausea, has also been observed. Another telltale sign of infection
    may be a sudden, profound diminution of one's
    \href{https://www.nytimes3xbfgragh.onion/2020/03/22/health/coronavirus-symptoms-smell-taste.html?action=click\&pgtype=Article\&state=default\&region=MAIN_CONTENT_3\&context=storylines_faq}{sense
    of smell and taste.}~Teenagers and young adults in some cases have
    developed painful red and purple lesions on their fingers and toes
    --- nicknamed ``Covid toe'' --- but few other serious symptoms.
  \end{itemize}
\item ~
  \hypertarget{why-is-it-safer-to-spend-time-together-outside}{%
  \paragraph{Why is it safer to spend time together
  outside?}\label{why-is-it-safer-to-spend-time-together-outside}}

  \begin{itemize}
  \tightlist
  \item
    \href{https://www.nytimes3xbfgragh.onion/2020/05/15/us/coronavirus-what-to-do-outside.html?action=click\&pgtype=Article\&state=default\&region=MAIN_CONTENT_3\&context=storylines_faq}{Outdoor
    gatherings}~lower risk because wind disperses viral droplets, and
    sunlight can kill some of the virus. Open spaces prevent the virus
    from building up in concentrated amounts and being inhaled, which
    can happen when infected people exhale in a confined space for long
    stretches of time, said Dr. Julian W. Tang, a virologist at the
    University of Leicester.
  \end{itemize}
\item ~
  \hypertarget{why-does-standing-six-feet-away-from-others-help}{%
  \paragraph{Why does standing six feet away from others
  help?}\label{why-does-standing-six-feet-away-from-others-help}}

  \begin{itemize}
  \tightlist
  \item
    The coronavirus spreads primarily through droplets from your mouth
    and nose, especially when you cough or sneeze. The C.D.C., one of
    the organizations using that measure,
    \href{https://www.nytimes3xbfgragh.onion/2020/04/14/health/coronavirus-six-feet.html?action=click\&pgtype=Article\&state=default\&region=MAIN_CONTENT_3\&context=storylines_faq}{bases
    its recommendation of six feet}~on the idea that most large droplets
    that people expel when they cough or sneeze will fall to the ground
    within six feet. But six feet has never been a magic number that
    guarantees complete protection. Sneezes, for instance, can launch
    droplets a lot farther than six feet,
    \href{https://jamanetwork.com/journals/jama/fullarticle/2763852}{according
    to a recent study}. It's a rule of thumb: You should be safest
    standing six feet apart outside, especially when it's windy. But
    keep a mask on at all times, even when you think you're far enough
    apart.
  \end{itemize}
\item ~
  \hypertarget{i-have-antibodies-am-i-now-immune}{%
  \paragraph{I have antibodies. Am I now
  immune?}\label{i-have-antibodies-am-i-now-immune}}

  \begin{itemize}
  \tightlist
  \item
    As of right
    now,\href{https://www.nytimes3xbfgragh.onion/2020/07/22/health/covid-antibodies-herd-immunity.html?action=click\&pgtype=Article\&state=default\&region=MAIN_CONTENT_3\&context=storylines_faq}{~that
    seems likely, for at least several months.}~There have been
    frightening accounts of people suffering what seems to be a second
    bout of Covid-19. But experts say these patients may have a
    drawn-out course of infection, with the virus taking a slow toll
    weeks to months after initial exposure.~People infected with the
    coronavirus typically
    \href{https://www.nature.com/articles/s41586-020-2456-9}{produce}~immune
    molecules called antibodies, which are
    \href{https://www.nytimes3xbfgragh.onion/2020/05/07/health/coronavirus-antibody-prevalence.html?action=click\&pgtype=Article\&state=default\&region=MAIN_CONTENT_3\&context=storylines_faq}{protective
    proteins made in response to an
    infection}\href{https://www.nytimes3xbfgragh.onion/2020/05/07/health/coronavirus-antibody-prevalence.html?action=click\&pgtype=Article\&state=default\&region=MAIN_CONTENT_3\&context=storylines_faq}{.
    These antibodies may}~last in the body
    \href{https://www.nature.com/articles/s41591-020-0965-6}{only two to
    three months}, which may seem worrisome, but that's~perfectly normal
    after an acute infection subsides, said Dr. Michael Mina, an
    immunologist at Harvard University. It may be possible to get the
    coronavirus again, but it's highly unlikely that it would be
    possible in a short window of time from initial infection or make
    people sicker the second time.
  \end{itemize}
\item ~
  \hypertarget{what-are-my-rights-if-i-am-worried-about-going-back-to-work}{%
  \paragraph{What are my rights if I am worried about going back to
  work?}\label{what-are-my-rights-if-i-am-worried-about-going-back-to-work}}

  \begin{itemize}
  \tightlist
  \item
    Employers have to provide
    \href{https://www.osha.gov/SLTC/covid-19/standards.html}{a safe
    workplace}~with policies that protect everyone equally.
    \href{https://www.nytimes3xbfgragh.onion/article/coronavirus-money-unemployment.html?action=click\&pgtype=Article\&state=default\&region=MAIN_CONTENT_3\&context=storylines_faq}{And
    if one of your co-workers tests positive for the coronavirus, the
    C.D.C.}~has said that
    \href{https://www.cdc.gov/coronavirus/2019-ncov/community/guidance-business-response.html}{employers
    should tell their employees}~-\/- without giving you the sick
    employee's name -\/- that they may have been exposed to the virus.
  \end{itemize}
\end{itemize}

By Thursday, Jan. 30, the public health officials had come around. Mr.
Azar, Dr. Redfield and
\href{https://www.nytimes3xbfgragh.onion/2020/03/11/us/politics/anthony-fauci-coronavirus.html}{Dr.
Anthony S. Fauci}, the director of the National Institute of Allergy and
Infectious Diseases, agreed that a ban on travel from the epidemic's
center could buy some time to put into place prevention and testing
measures. ``There was so much we didn't know about this virus,'' Dr.
Redfield said in an interview. ``We were rapidly understanding it was
much more transmissible, that it had a great ability to go global.''

The debate moved that afternoon to the Oval Office, where Mr. Azar and
others urged the president to approve the ban. ``The situation has
changed radically,'' Mr. Azar told Mr. Trump.

Others in the room urged being more cautious, arguing that a ban could
have unforeseen consequences. ``This is unprecedented,'' warned
Kellyanne Conway, the president's counselor. Mr. Trump was skeptical,
though he would later claim that everyone around him had been against
the idea. The two countries were in delicate trade negotiations. Was
this the time to provoke China? he asked. And what about the
consequences on the economy?

The president sided with his more aggressive aides, and
\href{https://www.nytimes3xbfgragh.onion/2020/01/31/business/china-travel-coronavirus.html}{announced
the ban next day}.

Still, Mr. Trump was publicly upbeat about the effects of the virus. At
a campaign rally in New Hampshire in early February, as the World Health
Organization was announcing new cases by the tens of thousands,
\href{https://www.cnn.com/2020/02/11/politics/trump-new-hampshire-rally/index.html}{he
said of the coronavirus}, ``By April, you know, in theory, when it gets
a little warmer, it miraculously goes away.''

In fact, the fight against the virus was already beginning to stumble.

A system used to track travelers returning from China went offline just
as state officials were told to begin monitoring them. Mr. Azar said at
a congressional hearing that he needed at least 300 million respirator
masks for health care workers, but the national emergency stockpile, the
government's reserve of disaster supplies, held only 12 million, and
many of those had expired.

And a C.D.C. coronavirus test distributed to state labs had a flawed
component that led to sometimes inconclusive results, crippling the
nation's testing capacity for weeks, despite assurances by the
administration that it was quickly being resolved.

Americans
\href{https://www.nytimes3xbfgragh.onion/2020/02/22/world/asia/coronavirus-japan-cruise-ship.html}{stranded
in Japan on a cruise ship}, the Diamond Princess, were finally returned
home Feb. 17, but the president
\href{https://www.nytimes3xbfgragh.onion/2020/02/22/us/politics/trump-coronavirus-cruise-ship.html}{became
enraged} when he learned that 14 of the passengers had tested positive
for the virus in the process of being transferred to government planes.

He later said that he was worried that bringing back people who tested
positive for the virus would increase the public tally of people
infected in the United States.

The month ended with
\href{https://www.nytimes3xbfgragh.onion/2020/02/27/us/politics/coronavirus-us-whistleblower.html}{a
whistle-blower's claim} that workers from the Department of Health and
Human Services had been sent to greet returning Americans from China at
two military bases in California without the personal protective gear
that is required for anyone coming into contact with potentially exposed
patients. None of the workers tested positive for the virus, but the
allegation shook Congress.

\hypertarget{i-like-the-numbers-being-where-they-are}{%
\subsection{`I Like the Numbers Being Where They
Are.'}\label{i-like-the-numbers-being-where-they-are}}

Image

Mr. Trump visiting the C.D.C. in Atlanta on Friday.Credit...T.J.
Kirkpatrick for The New York Times

The president's motorcade pulled onto the main C.D.C. campus in Atlanta
just before 4:30 p.m. on Friday, passing protesters holding signs that
said ``Have faith in science'' and ``We need a vaccine against Trump.''

Ten weeks after the virus first emerged in China, the total number of
confirmed cases in the world surged past 100,000 and public health
experts warned darkly that the outbreak was far from over. The United
States, they said, faces weeks, if not months, of uncertainty and
continued disruptions in education, businesses, commerce, medicine,
government and daily life.

``Time matters,'' Dr. Redfield said in an interview on Friday.

Last week, Vice President Mike Pence
\href{https://www.nytimes3xbfgragh.onion/2020/02/27/us/politics/us-coronavirus-pence.html}{was
given control of the public messaging}, and although Mr. Pence has
\href{https://www.nytimes3xbfgragh.onion/2020/03/06/us/politics/pence-trump-coronavirus.html}{had
some mixed messages} of his own --- he promised more tests before they
were available --- the White House has since displayed more discipline.
Mr. Pence holds twice daily conference calls with officials from across
the country, and a virus task force he leads issues daily talking
points, with comment from the health professionals, to make sure the
message is consistent.

But the president still has his bullhorn. During his visit to the
C.D.C., Mr. Trump told reporters that he was not inclined to let 21
people who tested positive for the virus on a cruise ship off the coast
of California onto American soil.

``They would like to have the people come off,'' he said. ``I would like
to have the people stay.'' The president said he would allow health
experts to make the final decision, but he made clear again where he
stood.

His concern? It would increase the tally for the number of people
infected in the United States. ``Because I like the numbers being where
they are,'' the president said.

Michael D. Shear and Noah Weiland reported from Washington, and Sheri
Fink from New York. Reporting was contributed by Mike Baker from
Seattle; Nicholas Bogel-Burroughs and Emma Fitzsimmons from New York;
Katie Thomas from Chicago; and Emily Cochrane, Zolan Kanno-Youngs, Lara
Jakes and Abby Goodnough from Washington.

Advertisement

\protect\hyperlink{after-bottom}{Continue reading the main story}

\hypertarget{site-index}{%
\subsection{Site Index}\label{site-index}}

\hypertarget{site-information-navigation}{%
\subsection{Site Information
Navigation}\label{site-information-navigation}}

\begin{itemize}
\tightlist
\item
  \href{https://help.nytimes3xbfgragh.onion/hc/en-us/articles/115014792127-Copyright-notice}{©~2020~The
  New York Times Company}
\end{itemize}

\begin{itemize}
\tightlist
\item
  \href{https://www.nytco.com/}{NYTCo}
\item
  \href{https://help.nytimes3xbfgragh.onion/hc/en-us/articles/115015385887-Contact-Us}{Contact
  Us}
\item
  \href{https://www.nytco.com/careers/}{Work with us}
\item
  \href{https://nytmediakit.com/}{Advertise}
\item
  \href{http://www.tbrandstudio.com/}{T Brand Studio}
\item
  \href{https://www.nytimes3xbfgragh.onion/privacy/cookie-policy\#how-do-i-manage-trackers}{Your
  Ad Choices}
\item
  \href{https://www.nytimes3xbfgragh.onion/privacy}{Privacy}
\item
  \href{https://help.nytimes3xbfgragh.onion/hc/en-us/articles/115014893428-Terms-of-service}{Terms
  of Service}
\item
  \href{https://help.nytimes3xbfgragh.onion/hc/en-us/articles/115014893968-Terms-of-sale}{Terms
  of Sale}
\item
  \href{https://spiderbites.nytimes3xbfgragh.onion}{Site Map}
\item
  \href{https://help.nytimes3xbfgragh.onion/hc/en-us}{Help}
\item
  \href{https://www.nytimes3xbfgragh.onion/subscription?campaignId=37WXW}{Subscriptions}
\end{itemize}
