Sections

SEARCH

\protect\hyperlink{site-content}{Skip to
content}\protect\hyperlink{site-index}{Skip to site index}

\href{https://www.nytimes3xbfgragh.onion/section/health}{Health}

\href{https://myaccount.nytimes3xbfgragh.onion/auth/login?response_type=cookie\&client_id=vi}{}

\href{https://www.nytimes3xbfgragh.onion/section/todayspaper}{Today's
Paper}

\href{/section/health}{Health}\textbar{}Not His First Epidemic: Dr.
Anthony Fauci Sticks to the Facts

\url{https://nyti.ms/2wEC13z}

\begin{itemize}
\item
\item
\item
\item
\item
\end{itemize}

\hypertarget{the-coronavirus-outbreak}{%
\subsubsection{\texorpdfstring{\href{https://www.nytimes3xbfgragh.onion/news-event/coronavirus?name=styln-coronavirus-national\&region=TOP_BANNER\&block=storyline_menu_recirc\&action=click\&pgtype=Article\&impression_id=ccd65240-f52e-11ea-b984-0101ae12a377\&variant=undefined}{The
Coronavirus
Outbreak}}{The Coronavirus Outbreak}}\label{the-coronavirus-outbreak}}

\begin{itemize}
\tightlist
\item
  live\href{https://www.nytimes3xbfgragh.onion/2020/09/12/world/covid-19-coronavirus.html?name=styln-coronavirus-national\&region=TOP_BANNER\&block=storyline_menu_recirc\&action=click\&pgtype=Article\&impression_id=ccd65241-f52e-11ea-b984-0101ae12a377\&variant=undefined}{Latest
  Updates}
\item
  \href{https://www.nytimes3xbfgragh.onion/interactive/2020/us/coronavirus-us-cases.html?name=styln-coronavirus-national\&region=TOP_BANNER\&block=storyline_menu_recirc\&action=click\&pgtype=Article\&impression_id=ccd65242-f52e-11ea-b984-0101ae12a377\&variant=undefined}{Maps
  and Cases}
\item
  \href{https://www.nytimes3xbfgragh.onion/interactive/2020/science/coronavirus-vaccine-tracker.html?name=styln-coronavirus-national\&region=TOP_BANNER\&block=storyline_menu_recirc\&action=click\&pgtype=Article\&impression_id=ccd65243-f52e-11ea-b984-0101ae12a377\&variant=undefined}{Vaccine
  Tracker}
\item
  \href{https://www.nytimes3xbfgragh.onion/2020/09/10/us/politics/fda-coronavirus-vaccine.html?name=styln-coronavirus-national\&region=TOP_BANNER\&block=storyline_menu_recirc\&action=click\&pgtype=Article\&impression_id=ccd65244-f52e-11ea-b984-0101ae12a377\&variant=undefined}{F.D.A.
  Regulators' Self-Defense}
\item
  \href{https://www.nytimes3xbfgragh.onion/2020/09/09/upshot/coronavirus-surprise-test-fees.html?name=styln-coronavirus-national\&region=TOP_BANNER\&block=storyline_menu_recirc\&action=click\&pgtype=Article\&impression_id=ccd67950-f52e-11ea-b984-0101ae12a377\&variant=undefined}{Surprise
  Test Fees}
\end{itemize}

Advertisement

\protect\hyperlink{after-top}{Continue reading the main story}

Supported by

\protect\hyperlink{after-sponsor}{Continue reading the main story}

\hypertarget{not-his-first-epidemic-dr-anthony-fauci-sticks-to-the-facts}{%
\section{Not His First Epidemic: Dr. Anthony Fauci Sticks to the
Facts}\label{not-his-first-epidemic-dr-anthony-fauci-sticks-to-the-facts}}

Where politicians fumble and other government health officials step
back, he steps up to explain.

\includegraphics{https://static01.graylady3jvrrxbe.onion/images/2020/03/04/science/00VIRUS-FAUCI1/00VIRUS-FAUCI1-articleLarge.jpg?quality=75\&auto=webp\&disable=upscale}

By \href{https://www.nytimes3xbfgragh.onion/by/denise-grady}{Denise
Grady}

\begin{itemize}
\item
  Published March 8, 2020Updated March 11, 2020
\item
  \begin{itemize}
  \item
  \item
  \item
  \item
  \item
  \end{itemize}
\end{itemize}

\href{https://www.nytimes3xbfgragh.onion/2020/03/11/us/politics/anthony-fauci-coronavirus.html}{Dr.
Anthony Fauci}, the nation's leading expert on infectious diseases, is
widely respected for his ability to explain science without talking down
to his audience --- and lately, for managing to correct the president's
pronouncements without saying he is wrong.

\href{https://www.nytimes3xbfgragh.onion/2020/03/11/us/politics/anthony-fauci-coronavirus.html}{President
Trump} said that drug companies would make a
\href{https://www.nytimes3xbfgragh.onion/2020/03/11/us/politics/anthony-fauci-coronavirus.html}{coronavirus}
vaccine ready ``soon.'' Dr. Fauci has repeatedly stepped up after the
president to the lectern during televised briefings or at White House
round tables to amend that timetable, giving a more accurate estimate of
at least a year or 18 months.

Mr. Trump said a ``cure'' might be possible. Dr. Fauci explained that
antiviral drugs were being studied to see if they might make the illness
less severe. The president also said the disease would go away in the
spring. Dr. Fauci said maybe so, but because it was caused by a new
virus, there was no way to tell.

Experts like Dr. Fauci should be the ones who speak to the public during
epidemics, said Representative Donna E. Shalala of Florida, who was his
boss during the Clinton administration, when she led the Department of
Health and Human Services.

``I think Tony is playing the same exact role that he has in the past
--- to make sure the science is accurate and clear,'' Ms. Shalala said.
``During a health emergency, it's the scientists and physicians that are
the credible people to the American public, not politicians.''

``I used to make him wear his white coat when talking to the public,''
she added. ``I used to make them all wear their white coats so people
were reassured that they are real doctors.''

On Sunday, Dr. Fauci appeared on at least two television news shows,
warning that as the virus spread --- there were more than 450 cases in
at least 33 states --- some stricter measures to isolate the infected
might be considered. He also offered advice for older adults and those
with underlying health conditions who are most at risk, saying they
should avoid cruises, flights and large gatherings of people.

At a recent Congressional subcommittee hearing, one representative asked
Dr. Fauci if he had a twin, because he seemed to be everywhere lately.
Another offered medical advice: lemon, honey and bourbon for his voice,
which has been hoarse for weeks from endless rounds of talking to
lawmakers, scientists, health officials and reporters.

If Dr. Fauci has become the explainer-in-chief of the coronavirus
epidemic, it is in part because other government scientists have left a
vacuum, avoiding the news media spotlight or being reined in by the
Trump administration and accused of exaggerating the threat from the
virus. When reporters call Dr. Fauci, he calls them back.

Dr. Robert Redfield, director of the Centers for Disease Control and
Prevention, is famous for being unknown: He has given few interviews or
public talks outside those required by administration news conferences.
Dr. Nancy Messonnier, director of the C.D.C.'s National Center for
Immunization and Respiratory Diseases, who warned on Feb. 25 that
\href{https://www.nytimes3xbfgragh.onion/2020/02/25/health/coronavirus-us.html}{the
coronavirus was likely to spread} in the United States, and that ``this
could be bad,'' came under fire from the administration and has toned
down her advisories.

``Nancy Messonnier told it like it is,'' said Dr. Thomas Frieden, the
former director of the C.D.C. ``And she was 100 percent right, and they
silenced the messenger.''

The C.D.C. has made few of its other scientists available for interviews
about the coronavirus. And it is now facing sharp criticism for long
delays in providing states with kits to test for the virus, and for
setting such tight limits on the tests performed at its headquarters in
Atlanta that some diagnoses were delayed or missed, possibly leaving
infected people to spread the virus to others.

\hypertarget{latest-updates-the-coronavirus-outbreak}{%
\section{\texorpdfstring{\href{https://www.nytimes3xbfgragh.onion/2020/09/11/world/covid-19-coronavirus.html?action=click\&pgtype=Article\&state=default\&region=MAIN_CONTENT_1\&context=storylines_live_updates}{Latest
Updates: The Coronavirus
Outbreak}}{Latest Updates: The Coronavirus Outbreak}}\label{latest-updates-the-coronavirus-outbreak}}

Updated 2020-09-12T12:04:20.515Z

\begin{itemize}
\tightlist
\item
  \href{https://www.nytimes3xbfgragh.onion/2020/09/11/world/covid-19-coronavirus.html?action=click\&pgtype=Article\&state=default\&region=MAIN_CONTENT_1\&context=storylines_live_updates\#link-dfb8a16}{Fauci
  cautions the virus could disrupt life in the U.S. until `maybe even
  towards the end of 2021.'}
\item
  \href{https://www.nytimes3xbfgragh.onion/2020/09/11/world/covid-19-coronavirus.html?action=click\&pgtype=Article\&state=default\&region=MAIN_CONTENT_1\&context=storylines_live_updates\#link-7104d154}{From
  Asia to Africa, China promotes its vaccine candidates to win friends.}
\item
  \href{https://www.nytimes3xbfgragh.onion/2020/09/11/world/covid-19-coronavirus.html?action=click\&pgtype=Article\&state=default\&region=MAIN_CONTENT_1\&context=storylines_live_updates\#link-393ad215}{The
  other way the virus will kill: hunger.}
\end{itemize}

\href{https://www.nytimes3xbfgragh.onion/2020/09/11/world/covid-19-coronavirus.html?action=click\&pgtype=Article\&state=default\&region=MAIN_CONTENT_1\&context=storylines_live_updates}{See
more updates}

More live coverage:
\href{https://www.nytimes3xbfgragh.onion/live/2020/09/11/business/stock-market-today-coronavirus?action=click\&pgtype=Article\&state=default\&region=MAIN_CONTENT_1\&context=storylines_live_updates}{Markets}

Dr. Fauci, 79, does not seem to be in the cross hairs. Although he was
recently told he must clear his many TV, radio and print interviews
through the White House, he has said that he has not been muzzled.

``You don't want to go to war with a president,''
\href{https://www.politico.com/news/2020/03/03/anthony-fauci-trump-coronavirus-crisis-118961}{Dr.
Fauci told Politico.} But he thinks the public needs solid,
understandable medical information, especially during crises. He makes a
point of not trying to ``razzle-dazzle'' audiences with his knowledge,
he told an
\href{https://www.c-span.org/video/?323680-1/qa-dr-anthony-fauci}{interviewer
in 2015}, adding that his education at a Jesuit high school and college
taught him intellectual rigor and the value of public service. As
director of the National Institute of Allergy and Infectious Diseases
since 1984, Dr. Fauci has lasted through six presidents, and has
declined multiple requests to lead his agency's parent organization, the
National Institutes of Health. He has led federal efforts to combat
diseases caused by emerging viruses, including H.I.V., SARS, the 2009
swine pandemic, MERS, Ebola and now the new coronavirus.

He was an architect of the President's Emergency Plan for AIDS Relief,
started in 2003 by President George W. Bush, to fight H.I.V. globally.

In 2001, after anthrax was mailed to Senator Tom Daschle of South
Dakota, the Bush administration tried initially to minimize the threat,
but
\href{https://www.nytimes3xbfgragh.onion/2001/10/25/us/nation-challenged-response-stung-criticism-aides-gather-coordinate-efforts.html}{Dr.
Fauci was characteristically blun}t.

``This is material that is quite formidable, that is infecting people
with inhalation anthrax, infecting them in the absence of direct
contact,'' he said. ``You can call it whatever you want to call it with
regard to grade and size or weaponized or not weaponized. The fact is,
it is acting like a highly efficient bioterrorist agent.''

Previous disease outbreaks have kept him on the lookout for new
pathogens, and in January, reports of unexplained pneumonia cases in
Wuhan, China, set off alarms in his head.

``Even before we knew it was a coronavirus, I said it certainly sounds
like a coronavirus-SARS type thing,'' he said. ``As soon as it was
identified, I called a meeting of top-level people and said, `Let's
start working on a vaccine right now.'''

He is among the most highly paid federal employees, earning about
\$400,000 a year, more than the vice president or the chief justice of
the United States. He oversees a budget of \$5.89 billion for the 2020
fiscal year. The Trump administration proposed a \$769 million cut, but
Congress rejected it.

Unflappable though Dr. Fauci may seem, the new coronavirus worries him,
especially its spread in the Seattle area and the toll it is taking in a
nursing home there. He said he thought it was still possible to stop the
disease, but health authorities would have to know where it was to do
that, and he said the lack of testing capacity was a huge obstacle.

Test kits sent by the C.D.C. to states were flawed, and it has taken
weeks to begin replacing them. And strict regulations from the Food and
Drug Administration hampered outside labs' efforts to begin to do their
own tests.

But Dr. Fauci declined to join the chorus of blame against the C.D.C.

``It's so easy to play Monday morning quarterback,'' he said. ``Every
element of the federal government was working full-speed to try to get
this out and get things going.''

``What I really want to concentrate on is the idea of getting enough
tests out there to get an evidence-based foundation of where we are in
this country,'' he said. ``Seattle is of concern. We need to know the
extent of the community spread.''

At times in the past, Dr. Frieden said, Dr. Fauci may have over-promised
a bit in his enthusiasm for projects of his that have yet to pan out ---
an H.I.V. vaccine, and a universal flu vaccine that would provide
long-lasting immunity to multiple strains so that people would not need
another shot every year.

``It's the role you have to play to get money from the Hill, and the
role of a researcher to be eternally optimistic about a cure being
around the corner,'' Dr. Frieden said. ``He's very effective at asking
for money from Congress.''

He is also a fair-minded colleague, Dr. Frieden added. During West
Africa's Ebola epidemic in 2014, after two nurses treating a patient who
had traveled to the United States became infected,
\href{https://www.nytimes3xbfgragh.onion/2014/10/15/us/cdc-says-it-should-have-responded-more-quickly-to-dallas-ebola-case.html}{the
C.D.C. was harshly criticized} for not having helped hospitals to
prepare better. Dr. Fauci did not pile on.

``Not everything we did was perfect,'' said Dr. Frieden, who then headed
the C.D.C. ``There are many people in similar roles as Tony who might
have taken advantage of that for the interest of their institution, and
Tony was always honorable.''

``If we did something wrong he was frank about that,'' he continued.
``When we were getting attacked unfairly, he defended us. That's very
unusual behavior in the bureaucratic situation you see in Washington
sometimes. I think that's why Tony is so respected.''

During the early years of the AIDS epidemic, the U.S. government and its
health agencies were a frequent target of criticism and protests, blamed
for inaction on potential treatments and inadequate funding for research
and medical care.

When AIDS activists demonstrated outside his institute in 1988,
demanding a chance to try experimental drugs, Dr. Fauci surprised the
leaders by inviting them into his office. He wound up working with them
to develop new ways to expand access without compromising the clinical
trials that were urgently needed to determine which drugs worked --- an
approach that was later applied to research in other diseases as well.

``There are a lot of world class scientists, but Tony has a special set
of skills,'' Ms. Shalala said. ``An ability to communicate, high
integrity and an understanding of politics --- and to know to stay out
of politics in order to protect scientists.''

Dr. Fauci acknowledged that his age puts him in one of the groups at
highest risk for becoming severely ill from a coronavirus infection. But
it has not slowed him down. He sleeps only about five hours a night,
works most of his waking hours and jumps on the Metro to get around the
Washington, D.C., area, from his office to the Capitol and the White
House. Most days, he runs or power walks 3.5 miles.

``I'm not worried about myself,'' he said. ``I'm worried about the job I
have to do.''

Andrew Jacobs contributed reporting. Alain Delaqueriere contributed
research.

Advertisement

\protect\hyperlink{after-bottom}{Continue reading the main story}

\hypertarget{site-index}{%
\subsection{Site Index}\label{site-index}}

\hypertarget{site-information-navigation}{%
\subsection{Site Information
Navigation}\label{site-information-navigation}}

\begin{itemize}
\tightlist
\item
  \href{https://help.nytimes3xbfgragh.onion/hc/en-us/articles/115014792127-Copyright-notice}{©~2020~The
  New York Times Company}
\end{itemize}

\begin{itemize}
\tightlist
\item
  \href{https://www.nytco.com/}{NYTCo}
\item
  \href{https://help.nytimes3xbfgragh.onion/hc/en-us/articles/115015385887-Contact-Us}{Contact
  Us}
\item
  \href{https://www.nytco.com/careers/}{Work with us}
\item
  \href{https://nytmediakit.com/}{Advertise}
\item
  \href{http://www.tbrandstudio.com/}{T Brand Studio}
\item
  \href{https://www.nytimes3xbfgragh.onion/privacy/cookie-policy\#how-do-i-manage-trackers}{Your
  Ad Choices}
\item
  \href{https://www.nytimes3xbfgragh.onion/privacy}{Privacy}
\item
  \href{https://help.nytimes3xbfgragh.onion/hc/en-us/articles/115014893428-Terms-of-service}{Terms
  of Service}
\item
  \href{https://help.nytimes3xbfgragh.onion/hc/en-us/articles/115014893968-Terms-of-sale}{Terms
  of Sale}
\item
  \href{https://spiderbites.nytimes3xbfgragh.onion}{Site Map}
\item
  \href{https://help.nytimes3xbfgragh.onion/hc/en-us}{Help}
\item
  \href{https://www.nytimes3xbfgragh.onion/subscription?campaignId=37WXW}{Subscriptions}
\end{itemize}
