Sections

SEARCH

\protect\hyperlink{site-content}{Skip to
content}\protect\hyperlink{site-index}{Skip to site index}

\href{https://www.nytimes3xbfgragh.onion/section/world/asia}{Asia
Pacific}

\href{https://myaccount.nytimes3xbfgragh.onion/auth/login?response_type=cookie\&client_id=vi}{}

\href{https://www.nytimes3xbfgragh.onion/section/todayspaper}{Today's
Paper}

\href{/section/world/asia}{Asia Pacific}\textbar{}`Proselytizing
Robots': Inside South Korean Church at Outbreak's Center

\url{https://nyti.ms/2IvU2DQ}

\begin{itemize}
\item
\item
\item
\item
\item
\item
\end{itemize}

\hypertarget{the-coronavirus-outbreak}{%
\subsubsection{\texorpdfstring{\href{https://www.nytimes3xbfgragh.onion/news-event/coronavirus?name=styln-coronavirus-national\&region=TOP_BANNER\&block=storyline_menu_recirc\&action=click\&pgtype=Article\&impression_id=214bd4e0-f52e-11ea-aca2-5fcae7f3caf4\&variant=undefined}{The
Coronavirus
Outbreak}}{The Coronavirus Outbreak}}\label{the-coronavirus-outbreak}}

\begin{itemize}
\tightlist
\item
  live\href{https://www.nytimes3xbfgragh.onion/2020/09/12/world/covid-19-coronavirus.html?name=styln-coronavirus-national\&region=TOP_BANNER\&block=storyline_menu_recirc\&action=click\&pgtype=Article\&impression_id=214bd4e1-f52e-11ea-aca2-5fcae7f3caf4\&variant=undefined}{Latest
  Updates}
\item
  \href{https://www.nytimes3xbfgragh.onion/interactive/2020/us/coronavirus-us-cases.html?name=styln-coronavirus-national\&region=TOP_BANNER\&block=storyline_menu_recirc\&action=click\&pgtype=Article\&impression_id=214bd4e2-f52e-11ea-aca2-5fcae7f3caf4\&variant=undefined}{Maps
  and Cases}
\item
  \href{https://www.nytimes3xbfgragh.onion/interactive/2020/science/coronavirus-vaccine-tracker.html?name=styln-coronavirus-national\&region=TOP_BANNER\&block=storyline_menu_recirc\&action=click\&pgtype=Article\&impression_id=214bd4e3-f52e-11ea-aca2-5fcae7f3caf4\&variant=undefined}{Vaccine
  Tracker}
\item
  \href{https://www.nytimes3xbfgragh.onion/2020/09/10/us/politics/fda-coronavirus-vaccine.html?name=styln-coronavirus-national\&region=TOP_BANNER\&block=storyline_menu_recirc\&action=click\&pgtype=Article\&impression_id=214bd4e4-f52e-11ea-aca2-5fcae7f3caf4\&variant=undefined}{F.D.A.
  Regulators' Self-Defense}
\item
  \href{https://www.nytimes3xbfgragh.onion/2020/09/09/upshot/coronavirus-surprise-test-fees.html?name=styln-coronavirus-national\&region=TOP_BANNER\&block=storyline_menu_recirc\&action=click\&pgtype=Article\&impression_id=214bfbf0-f52e-11ea-aca2-5fcae7f3caf4\&variant=undefined}{Surprise
  Test Fees}
\end{itemize}

Advertisement

\protect\hyperlink{after-top}{Continue reading the main story}

Supported by

\protect\hyperlink{after-sponsor}{Continue reading the main story}

\hypertarget{proselytizing-robots-inside-south-korean-church-at-outbreaks-center}{%
\section{`Proselytizing Robots': Inside South Korean Church at
Outbreak's
Center}\label{proselytizing-robots-inside-south-korean-church-at-outbreaks-center}}

Even before the coronavirus scourge, the Shincheonji Church of Jesus was
viewed suspiciously. Now it has become the most vilified church in South
Korea.

\includegraphics{https://static01.graylady3jvrrxbe.onion/images/2020/03/09/world/09skorea-church-1sub/merlin_170187330_14d578aa-bd8e-4cd4-8f28-3baaf980dcfd-articleLarge.jpg?quality=75\&auto=webp\&disable=upscale}

\href{https://www.nytimes3xbfgragh.onion/by/choe-sang-hun}{\includegraphics{https://static01.graylady3jvrrxbe.onion/images/2018/07/18/multimedia/author-choe-sang-hun/author-choe-sang-hun-thumbLarge.png}}

By \href{https://www.nytimes3xbfgragh.onion/by/choe-sang-hun}{Choe
Sang-Hun}

\begin{itemize}
\item
  March 10, 2020
\item
  \begin{itemize}
  \item
  \item
  \item
  \item
  \item
  \item
  \end{itemize}
\end{itemize}

SEOUL, South Korea --- More than 1.2 million citizens have called for
the secretive church to be disbanded. One province asked the public to
report church members to a hotline for coronavirus testing. Smartphone
apps help identify the church's 1,100 once-obscure facilities in South
Korea, most already plastered with ``off-limits'' signs by
disease-control officials.

Even before the coronavirus scourge, South Korea's
\href{https://www.nytimes3xbfgragh.onion/2020/02/21/world/asia/south-korea-coronavirus-shincheonji.html}{Shincheonji
Church of Jesus} had faced increased suspicion over its tactics to
attract tens of thousands of recruits. But in the month since the church
was identified as the epicenter of infections in the country, it has
become the target of scorn, vilification and open hatred.

The founder, Lee Man-hee, 88, who has promised its 240,000 members entry
to the ``new heaven and new earth,'' is now the potential subject of a
prosecutor investigation into possible murder charges.

Parents of recruits accuse him of ``brainwashed slavery.'' Former
members describe him as another in a long-line of spiritual snake-oil
salesmen in South Korea, a fertile ground for untraditional religious
sects.

A large majority of the country's more than 7,500
\href{http://www.nytimes3xbfgragh.onion/2020/03/10/world/coronavirus-news.html}{coronavirus}patients
are Shincheonji members in Daegu, a city in the southeast, or people who
had come into contact with them. An additional cluster of cases has
emerged in Cheongdo, a county near Daegu that is Mr. Lee's birthplace
and a regular pilgrimage destination for his followers.

The church has protested what it called ``scapegoating'' by South
Koreans eager to discredit what had been the fastest-growing religious
sect in the country, as other big churches worry about declining
membership.

``The entire society has gone berserk against our church since the virus
outbreak,'' said Lee Young-Soo, 54, a Shincheonji member whose sister, a
fellow church member, died after having fallen from her seventh-floor
apartment in the southern city of Ulsan last month. Ms. Lee said her
sister had confided that her husband's long-running abuse over her
church had intensified after the virus outbreak.

Another Shincheonji member who church officials said had suffered
spousal abuse, a 42-year-old mother of two, died after having fallen
from her 11th floor apartment on Monday night. The police are
investigating both cases.

``The society is so wrong, and I am so saddened,'' Ms. Lee said.

Still, the church is inextricably linked to the spread of the affliction
in South Korea, one of the largest outbreaks outside China.

The church's clandestine nature is part of what made it a focal point
for the country's anger and fear. Officials have struggled to locate and
screen church members for the virus.

Kwon Jun-Wook, a senior disease-control official, said last month that
when officials had tried to reach church members, they found many
incommunicado. Daegu's mayor said Tuesday that dozens of Shincheonji
members must be immediately tested for the coronavirus or face fines.
The city of Seoul has accused Mr. Lee and his disciples of failing to
provide full membership lists.

``Lee Man-hee is a psychopath who has lied and lied until he believed
his own lie that he was the true messiah,'' said Jeong Ji-su, a former
disciple who left last July.

\includegraphics{https://static01.graylady3jvrrxbe.onion/images/2020/03/09/world/09skorea-church-2sub/merlin_169881213_2cee3279-040d-420c-b26a-d00907ba3b3b-articleLarge.jpg?quality=75\&auto=webp\&disable=upscale}

\hypertarget{an-era-of-messiahs}{%
\subsection{An Era of Messiahs}\label{an-era-of-messiahs}}

Mr. Lee is far from the first person claiming to be a messiah in South
Korea.

\href{https://www.nytimes3xbfgragh.onion/2007/07/07/world/asia/07korea.html}{Shamanism}
--- worshiping a multitude of deities including dead parents, ancient
warriors and mountain spirits --- has infused society for millenniums,
interacting with new arrivals like Christianity and making some Koreans
amenable to embracing new belief systems, said Koo Se-woong, a scholar
who has researched Korean religions.

\hypertarget{latest-updates-the-coronavirus-outbreak}{%
\section{\texorpdfstring{\href{https://www.nytimes3xbfgragh.onion/2020/09/11/world/covid-19-coronavirus.html?action=click\&pgtype=Article\&state=default\&region=MAIN_CONTENT_1\&context=storylines_live_updates}{Latest
Updates: The Coronavirus
Outbreak}}{Latest Updates: The Coronavirus Outbreak}}\label{latest-updates-the-coronavirus-outbreak}}

Updated 2020-09-12T12:04:20.515Z

\begin{itemize}
\tightlist
\item
  \href{https://www.nytimes3xbfgragh.onion/2020/09/11/world/covid-19-coronavirus.html?action=click\&pgtype=Article\&state=default\&region=MAIN_CONTENT_1\&context=storylines_live_updates\#link-dfb8a16}{Fauci
  cautions the virus could disrupt life in the U.S. until `maybe even
  towards the end of 2021.'}
\item
  \href{https://www.nytimes3xbfgragh.onion/2020/09/11/world/covid-19-coronavirus.html?action=click\&pgtype=Article\&state=default\&region=MAIN_CONTENT_1\&context=storylines_live_updates\#link-7104d154}{From
  Asia to Africa, China promotes its vaccine candidates to win friends.}
\item
  \href{https://www.nytimes3xbfgragh.onion/2020/09/11/world/covid-19-coronavirus.html?action=click\&pgtype=Article\&state=default\&region=MAIN_CONTENT_1\&context=storylines_live_updates\#link-393ad215}{The
  other way the virus will kill: hunger.}
\end{itemize}

\href{https://www.nytimes3xbfgragh.onion/2020/09/11/world/covid-19-coronavirus.html?action=click\&pgtype=Article\&state=default\&region=MAIN_CONTENT_1\&context=storylines_live_updates}{See
more updates}

More live coverage:
\href{https://www.nytimes3xbfgragh.onion/live/2020/09/11/business/stock-market-today-coronavirus?action=click\&pgtype=Article\&state=default\&region=MAIN_CONTENT_1\&context=storylines_live_updates}{Markets}

``Promising a new world and a `New Jerusalem' in Korea is nothing new
among Christian churches in South Korea,'' Dr. Koo said.

As the country has suffered war and deprivation in the past century, 120
self-styled messiahs promising a new world of peace have emerged, 70
commanding sizable followings. The best known is the Rev. Sun Myung
Moon, founder of the Unification Church,
\href{https://www.nytimes3xbfgragh.onion/2012/09/03/world/asia/rev-sun-myung-moon-founder-of-unification-church-dies-at-92.html}{who
died in 2012}.

Some ended up in jail on fraud
or\href{https://www.nytimes3xbfgragh.onion/2018/11/22/world/asia/south-korea-pastor-rape.html}{rape
charges} or lived in disgrace after the rapture they had promised never
came. But their apostles split and spread, rebranding themselves into
new sects. Mr. Lee was one of them.

On the day she first encountered the church in September 2016, Ms.
Jeong, 25, recalled in an interview, she had been hurrying through a
Seoul subway station. Two friendly looking women and a young man about
her age asked if she could spare a few minutes to give feedback on a
movie script.

They entered a fast-food restaurant, where her initiation began. She
became a full member in July 2017. Ms. Jeong said she had been
brainwashed and spent the next two years recruiting fellow young people
just as she had been.

Shincheonji recruiters seek converts who might be vulnerable, first
uncovering personal problems, like low self-esteem in Ms. Jeong's case.
They offer counseling, building friendship and persuading recruits that
Bible studies can help, former members said.

``The entire church was a con job,'' Ms. Jeong said. ``When they
targeted you for proselytizing, everyone who approached you in the cloak
of chance encounter was a member of Shincheonji --- only that you didn't
know it.''

Other former members report proselytizers approached with free tarot
card readings, personality tests and foreign-language classes.

``Young people are attracted to Shincheonji at times like this when
opportunities dwindle because the church raises hopes of jobs as pastors
and preachers and promises a spiritually fulfilling life, even promising
eternal life and high priestship when the new world comes,'' said Kim
Seong-Ja, 58, a former member.

Young converts often lived together in cheap crowded rooms, former
members said. Dating, or any distraction from the goal of winning
converts, was discouraged.

``We were nothing but proselytizing robots,'' said Lee Ho-yeon, 24, a
former member. ``I spent as little as possible on food and used what
little money I left in my pocket to buy Starbucks coffee for people I
wanted to convert.''

Because of Shincheonji's suspect image, proselytizers would delay
revealing their affiliation until confident that their recruits were
ready, former members said.

Once they ​passed
\href{https://www.youtube.com/watch?v=CUgv5BZ3FQo}{written tests} after
months of Bible studies, converts ​became participants in a
\href{https://www.youtube.com/watch?v=wHHijQhjseY}{spectacular
commencement ceremony}. Life as a church member included meetings,
proselytizing missions on the streets and daily progress reports on how
many people they had tried to recruit and how their recruits were doing
in Bible studies, former members said.

Worshipers were told to keep their membership secret from relatives. If
parents became suspicious, former members said, handlers had told them
to lie.

When parents tried to stop worshipers from attending the church, many
left home. Ms. Lee said she ate as little as possible at home in case
her parents mixed her food with sleeping pills to get her away from the
church.

``They said it was OK to lie to our parents, as it was OK for moms to
call bitter medicine a chocolate when feeding it to a sick baby,'' said
Stella Seo, 29, who was in Shincheonji for seven years until late 2018.

Image

Lee Ho-yeon, center, left Shincheonji in 2015.~Credit...Woohae Cho for
The New York Times

\hypertarget{we-were-taught-not-to-be-afraid-of-illness}{%
\subsection{`We Were Taught Not to Be Afraid of
Illness'}\label{we-were-taught-not-to-be-afraid-of-illness}}

The practice of disavowing membership carried over to how at least some
church members responded to the coronavirus outbreak. South Koreans were
outraged when it was revealed last month that church members had
received a message telling them to deny their affiliation with
Shincheonji and to keep proselytizing even after the outbreak was
reported among its congregation.

Shincheonji said that the instruction was not church policy and that it
had punished the official who issued it.

While Shincheonji's recruitment methods have drawn recent condemnation
for helping spread the virus, its approach has long galled more
mainstream churches, which have accused it of sending undercover
proselytizers, known as harvesters, into their congregations and
stealing members.

Sometimes, the harvesters were accused of sowing internal discord in a
church and taking it over, a feat celebrated as ``moving a mountain''
among Shincheonji members.

Shincheonji said that disgruntled former members and traditional
churches alarmed over their shrinking congregations had spread false
rumors to discredit the church.

But in video footage of internal lecturing viewed by The New York Times,
a proselytizing instructor said that the old way of founding a church
and then building a congregation ``is too expensive, takes too much
manpower and is too time-consuming.''

``It's better to swallow existing churches,'' she said to a chorus of
amens. ``But you must keep this strategy to yourself.''

Mr. Lee has defended the church's response to the outbreak, and
Shincheonji has issued statements through a spokesman repeating that the
church was cooperating with the government and demanding an end to
``scapegoating.''

Eo Kwang-il, 38, a Shincheonji member, said that because of overwhelming
bias against their church, members hid their affiliation and used ruses
to win converts. But he said the church never forced members to abandon
school or jobs for the sake of proselytizing.

How the church conducts gatherings, however, has drawn scrutiny as a
spreader of the disease. Worshipers sit packed tightly on the floor and
attend even when sick, former members say.

``We were taught not to be afraid of illness,'' said Lee Ho-yeon, who
left the church in 2015. A church leader boasted to followers on Feb. 9
that although hundreds of people had died in Wuhan, China, where the
outbreak began, no Shincheonji worshipers there became sick, according
to the audio file of the sermon released by Yoon Jae-Deok, an expert on
religious groups like Shincheonji.

The crowded conditions of Shincheonji churches did make them more
vulnerable to contagious diseases, said Hwang Gui-hag, the editor in
chief of the Seoul-based Law Times, which specializes in church news.

But he said that in his view, the central and regional governments,
caught off-guard by the virus, had found a convenient point of blame in
the church. He said mainstream churches had a vested interest in
disparaging Shincheonji, as did the ``cult hunters'' who demonize the
church so that families hire them to remove relatives.

Shincheonji's followers call ​Mr. ​Lee, the founder, by biblical
references like ``the Promised Pastor​.'' He once gathered his 12
deputies and
\href{https://blog.naver.com/downwave/150137689704}{re-created the
scene} from the Last Supper.

``It's embarrassing now to admit this,'' said Ms. Seo. ``But when I
first saw him, I was so overwhelmed with emotions that I wept.''

Ms. Seo said she had felt proud when she and thousands of other church
members gathered under a scorching summer sun to practice for the
``​\href{https://www.youtube.com/watch?v=2HqOEyd44wQ}{world ​​​peace
festivals}'' that Mr. Lee and
\href{https://www.youtube.com/watch?v=8Q5xUWfG3bo}{Kim Nam-hee, a former
deputy,} organized. Tens of thousands were mobilized to dance, sing and
perform.

Ms. Kim, the former deputy, split with Mr. Lee in 2017 and has since
called him ``no messiah but just a plain old religious con man.'' The
church called her ``an apostate.''

Image

Public officials in Gyeonggi Province sealing one of the 10 Shincheonji
facilities in the city of Guri on Sunday.Credit...Woohae Cho for The New
York Times

\hypertarget{attempts-at-reconversion}{%
\subsection{Attempts at `Reconversion'}\label{attempts-at-reconversion}}

Distressed parents have taken children to ``cult hunters'' --- activist
pastors affiliated with mainstream churches who offer ``reconversion''
services. In some cases, Shincheonji representatives file a missing
person's report with the police and visit the homes or work sites of
relatives to protest.

Both of Ms. Seo's parents and her brother quit their jobs to stay with
her as she violently resisted a reconversion program. Ms. Seo said she
had refused to eat for eight days and slashed her wrist and thigh with a
razor blade.

``As far as I was concerned, it was a fight against Satan,'' she said.

After two months in the program, Ms. Seo rejected the church.

Other families said their efforts to persuade a relative to leave the
church had failed.

Doo Song-ja, 64, took her 33-year-old daughter, Hong Eun-hwa, twice to a
reconversion program, but she refused to leave the church.

``The only thing left for me is to erase her from my memory,'' Ms. Doo
said. ``But how can I?''

Shincheonji has said that social bias against it has intensified lately,
with members reporting taunting at work and death threats.

``Many church members were afraid to come out and reveal their church
membership, given the overwhelming blaming coming from politicians and
news media that called Shincheonji the originator of the virus
outbreak,'' Kim Si-mon, a church spokesman, said in a statement.

The government has repeatedly warned that the battle against the
coronavirus depends on how quickly infected church members can be
isolated. Mr. Lee has urged them to ``follow the government's
instructions,'' avoid gatherings and proselytize only online.

Some parents, like Choi Mi-sook, 56, mother of Shincheonji member Kim
Yoo-jeong, 25, remain distraught.

``I don't even know where she lives, whether she is tested for the
coronavirus, whether she is well or even alive,'' the mother said.

Image

Photos of Ms. Doo's daughter, Hong Eun-hwa.Credit...Woohae Cho for The
New York Times

Advertisement

\protect\hyperlink{after-bottom}{Continue reading the main story}

\hypertarget{site-index}{%
\subsection{Site Index}\label{site-index}}

\hypertarget{site-information-navigation}{%
\subsection{Site Information
Navigation}\label{site-information-navigation}}

\begin{itemize}
\tightlist
\item
  \href{https://help.nytimes3xbfgragh.onion/hc/en-us/articles/115014792127-Copyright-notice}{©~2020~The
  New York Times Company}
\end{itemize}

\begin{itemize}
\tightlist
\item
  \href{https://www.nytco.com/}{NYTCo}
\item
  \href{https://help.nytimes3xbfgragh.onion/hc/en-us/articles/115015385887-Contact-Us}{Contact
  Us}
\item
  \href{https://www.nytco.com/careers/}{Work with us}
\item
  \href{https://nytmediakit.com/}{Advertise}
\item
  \href{http://www.tbrandstudio.com/}{T Brand Studio}
\item
  \href{https://www.nytimes3xbfgragh.onion/privacy/cookie-policy\#how-do-i-manage-trackers}{Your
  Ad Choices}
\item
  \href{https://www.nytimes3xbfgragh.onion/privacy}{Privacy}
\item
  \href{https://help.nytimes3xbfgragh.onion/hc/en-us/articles/115014893428-Terms-of-service}{Terms
  of Service}
\item
  \href{https://help.nytimes3xbfgragh.onion/hc/en-us/articles/115014893968-Terms-of-sale}{Terms
  of Sale}
\item
  \href{https://spiderbites.nytimes3xbfgragh.onion}{Site Map}
\item
  \href{https://help.nytimes3xbfgragh.onion/hc/en-us}{Help}
\item
  \href{https://www.nytimes3xbfgragh.onion/subscription?campaignId=37WXW}{Subscriptions}
\end{itemize}
