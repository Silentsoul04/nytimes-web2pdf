Sections

SEARCH

\protect\hyperlink{site-content}{Skip to
content}\protect\hyperlink{site-index}{Skip to site index}

\href{https://www.nytimes3xbfgragh.onion/section/us}{U.S.}

\href{https://myaccount.nytimes3xbfgragh.onion/auth/login?response_type=cookie\&client_id=vi}{}

\href{https://www.nytimes3xbfgragh.onion/section/todayspaper}{Today's
Paper}

\href{/section/us}{U.S.}\textbar{}Coronavirus Is Shutting Schools. Is
America Ready for Virtual Learning?

\url{https://nyti.ms/2U4gfxX}

\begin{itemize}
\item
\item
\item
\item
\item
\end{itemize}

\hypertarget{school-reopenings}{%
\subsubsection{\texorpdfstring{\href{https://www.nytimes3xbfgragh.onion/spotlight/schools-reopening?name=styln-coronavirus-schools-reopening\&region=TOP_BANNER\&block=storyline_menu_recirc\&action=click\&pgtype=Article\&impression_id=b7994850-f4b7-11ea-b2ea-e95a284183ee\&variant=undefined}{School
Reopenings}}{School Reopenings}}\label{school-reopenings}}

\begin{itemize}
\tightlist
\item
  \href{https://www.nytimes3xbfgragh.onion/2020/09/08/us/school-districts-cyberattacks-glitches.html?name=styln-coronavirus-schools-reopening\&region=TOP_BANNER\&block=storyline_menu_recirc\&action=click\&pgtype=Article\&impression_id=b7994851-f4b7-11ea-b2ea-e95a284183ee\&variant=undefined}{Remote
  Learning Glitches}
\item
  \href{https://www.nytimes3xbfgragh.onion/2020/09/08/upshot/children-testing-shortfalls-virus.html?name=styln-coronavirus-schools-reopening\&region=TOP_BANNER\&block=storyline_menu_recirc\&action=click\&pgtype=Article\&impression_id=b7994852-f4b7-11ea-b2ea-e95a284183ee\&variant=undefined}{Limited
  Testing for Children}
\item
  \href{https://www.nytimes3xbfgragh.onion/2020/09/10/us/des-moines-school-opening-coronavirus.html?name=styln-coronavirus-schools-reopening\&region=TOP_BANNER\&block=storyline_menu_recirc\&action=click\&pgtype=Article\&impression_id=b7996f60-f4b7-11ea-b2ea-e95a284183ee\&variant=undefined}{District
  Defies Reopening Order}
\item
  \href{https://www.nytimes3xbfgragh.onion/interactive/2020/us/covid-college-cases-tracker.html?name=styln-coronavirus-schools-reopening\&region=TOP_BANNER\&block=storyline_menu_recirc\&action=click\&pgtype=Article\&impression_id=b7996f61-f4b7-11ea-b2ea-e95a284183ee\&variant=undefined}{Tracking
  College Cases}
\end{itemize}

Advertisement

\protect\hyperlink{after-top}{Continue reading the main story}

Supported by

\protect\hyperlink{after-sponsor}{Continue reading the main story}

\hypertarget{coronavirus-is-shutting-schools-is-america-ready-for-virtual-learning}{%
\section{Coronavirus Is Shutting Schools. Is America Ready for Virtual
Learning?}\label{coronavirus-is-shutting-schools-is-america-ready-for-virtual-learning}}

Educators experienced with remote learning warn that closures can affect
children's academic progress, safety and social lives.

\includegraphics{https://static01.graylady3jvrrxbe.onion/images/2020/03/11/us/11virus-onlineschools5/11virus-onlineschools5-articleLarge.png?quality=75\&auto=webp\&disable=upscale}

\href{https://www.nytimes3xbfgragh.onion/by/dana-goldstein}{\includegraphics{https://static01.graylady3jvrrxbe.onion/images/2018/06/12/multimedia/author-dana-goldstein/author-dana-goldstein-thumbLarge.png}}

By \href{https://www.nytimes3xbfgragh.onion/by/dana-goldstein}{Dana
Goldstein}

\begin{itemize}
\item
  Published March 13, 2020Updated March 17, 2020
\item
  \begin{itemize}
  \item
  \item
  \item
  \item
  \item
  \end{itemize}
\end{itemize}

More than 30,000
\href{https://www.edweek.org/ew/section/multimedia/map-coronavirus-and-school-closures.html}{K-12
schools in the United States are being shuttered} because of worries
about spreading the coronavirus, affecting at least 20 million students,
most of whom will be asked to shift to online learning. Though
\href{https://www.nytimes3xbfgragh.onion/2020/03/10/opinion/coronavirus-school-closing.html}{health
experts disagree} to what extent school closures will help, entire
states, including Ohio, Illinois and Maryland, and some of the nation's
largest cities, including Los Angeles and Houston, announced closings in
recent days.

Educators experienced with remote learning warn that closures are a
serious threat to children's academic progress, safety and social lives.
They say that running a classroom digitally is much harder than bringing
an adult workplace online, and that it can disproportionately affect
low-income students and those with special needs.

Here are some of the warnings and tips that teachers well-versed in
remote learning have for schools planning to move online.

\hypertarget{not-every-home-has-computers-or-high-speed-internet}{%
\subsection{Not every home has computers or high-speed
internet.}\label{not-every-home-has-computers-or-high-speed-internet}}

The
\href{https://www.census.gov/content/dam/Census/library/publications/2018/acs/ACS-39.pdf}{vast
majority} of households with children have broadband internet, but there
are still big disparities by income, race and the education level of
parents.

Low-income families are more likely to rely on smartphones for internet
access, and children in those households may not be able to use more
sophisticated learning software that requires a tablet or computer. It
is not unusual, educators say, for siblings to try to complete their
schoolwork on a single cellphone.

Nate Ridgway, a social studies teacher at Beech Grove High School near
Indianapolis, regularly creates video lessons for his students. His
school also provides Chromebooks that students can take home. But when
his district had two ``e-learning'' days this year because of snow, he
noticed that some disadvantaged students fell behind academically.

``We as educators have to be so, so careful about this expectation that
we go completely online,'' he said. ``Anywhere from 10 to 25 percent of
my students may not have internet access at home.''

\includegraphics{https://static01.graylady3jvrrxbe.onion/images/2020/03/11/us/00VIRUS-ONLINESCHOOLING/00VIRUS-ONLINESCHOOLING-articleLarge.png?quality=75\&auto=webp\&disable=upscale}

\hypertarget{younger-children-require-lots-of-adult-supervision}{%
\subsection{Younger children require lots of adult
supervision.}\label{younger-children-require-lots-of-adult-supervision}}

Younger students need help to learn online --- lots of help. Parents may
need to
\href{https://www.nytimes3xbfgragh.onion/interactive/2020/03/10/us/covid-19-seattle-washington-home-schooling-remote.html}{assist
their child} with turning on a device, logging into an app, reading
instructions, clicking in the right place, typing answers and staying on
task.

Even the tech-savviest adult will find this difficult while working from
home at the same time --- a more common scenario as the coronavirus
spreads. Parents who continue to work outside the home when schools are
closed will need to arrange child care, where technical help could be
scarce.

In Salinas, Calif., Ben Cogswell prides himself on his tech-savvy
kindergarten classroom. He records videos of himself reading beloved
children's books and reviewing words via flashcards. Many of his
students speak Spanish at home, and he hopes to strengthen their English
skills outside school hours.

But access to these resources can be a challenge, he said. Even though
the district provides all students with a Chromebook, not all parents
--- even those with home WiFi --- know how to connect that device to the
internet.

One of the best things schools can do to prepare for closures, Mr.
Cogswell said, is to make sure parents can text message teachers and
connect to the apps and web resources their children will need.

Is the broader American education system ready?

``I would say no,'' he warned.

Image

Sarah Giddings, a high school teacher in Ypsilanti, Mich., during
a~collaborative meeting with a student about her progress in
geometry/art class.Credit...Screenshot

\hypertarget{even-great-teachers-lack-expertise-in-creating-online-lessons}{%
\subsection{Even great teachers lack expertise in creating online
lessons.}\label{even-great-teachers-lack-expertise-in-creating-online-lessons}}

While there are lots of exceptional teachers, not all of them are ready
to move their instruction online.

Online lessons need to have more clearly written-out themes and
directions for students, said Sarah Giddings, a teacher at WAVE, a high
school in Ypsilanti, Mich., that blends online and in-person learning.

``You can be a fantastic teacher, but writing curriculum is hard,'' she
said.

Education technology firms have aggressively promoted their products as
school closures become more widespread. But educators who have pioneered
online learning say some of the best tools --- like Google Hangouts or
Flipgrid, an interactive video platform --- cost nothing.

\href{https://www.nytimes3xbfgragh.onion/spotlight/schools-reopening?action=click\&pgtype=Article\&state=default\&region=MAIN_CONTENT_3\&context=storylines_keepup}{}

\hypertarget{school-reopenings-}{%
\subsubsection{School Reopenings ›}\label{school-reopenings-}}

\hypertarget{back-to-school}{%
\paragraph{Back to School}\label{back-to-school}}

Updated Sept. 11, 2020

The latest on how schools are reopening amid the pandemic.

\begin{itemize}
\item
  \begin{itemize}
  \tightlist
  \item
    School officials in Des Moines are refusing to hold in-person
    classes,
    \href{https://www.nytimes3xbfgragh.onion/2020/09/10/us/des-moines-school-opening-coronavirus.html?action=click\&pgtype=Article\&state=default\&region=MAIN_CONTENT_3\&context=storylines_keepup}{despite
    an order from Iowa's governor and a judge's ruling}, risking school
    funding and their jobs because they think it's unsafe.
  \item
    The University of Illinois at Urbana-Champaign had one of the most
    comprehensive plans by a major college to keep the virus under
    control. But it
    \href{https://www.nytimes3xbfgragh.onion/2020/09/10/health/university-illinois-covid.html?action=click\&pgtype=Article\&state=default\&region=MAIN_CONTENT_3\&context=storylines_keepup}{failed
    to account for students partying}.
  \item
    College students are
    \href{https://www.nytimes3xbfgragh.onion/2020/09/10/technology/coronavirus-quarantines-college.html?action=click\&pgtype=Article\&state=default\&region=MAIN_CONTENT_3\&context=storylines_keepup}{using
    apps to shame their schools}~into better coronavirus plans.
  \item
    For some families, the pandemic
    \href{https://www.nytimes3xbfgragh.onion/2020/09/10/parenting/family-second-language-coronavirus.html?action=click\&pgtype=Article\&state=default\&region=MAIN_CONTENT_3\&context=storylines_keepup}{has
    meant a return to their native languages}.
  \end{itemize}
\end{itemize}

``A teacher's favorite price is free,'' Mr. Ridgway said. He and other
educators warn against using learning tools that make it difficult to
log in, are inaccessible via mobile devices or require downloading
special software. Remote learning, they say, should simply require
signing into a website.

Image

Christopher Bakk, a social studies teacher in Racine, Wis., in an
introduction video he sent to online students this
year.Credit...Screenshot

\hypertarget{students-with-special-needs-can-be-the-hardest-to-teach-virtually}{%
\subsection{Students with special needs can be the hardest to teach
virtually.}\label{students-with-special-needs-can-be-the-hardest-to-teach-virtually}}

Christopher W. Bakk, a social studies teacher at Turning Point Academy
in Racine, Wis., has taught special education students both in-person
and remotely, via the Wisconsin eSchool Network.

Some of those students have behavioral issues and thrive online because
there are fewer social distractions, he said. But others find it
difficult to have less direct access to teachers and peers. ``The
self-discipline is a struggle,'' Mr. Bakk said.

In addition, many students in this generation are nervous about speaking
over the phone, which Mr. Bakk said can be a crucial tool for teachers
to check on students who are learning at home. He approaches those
discussions with humor to help self-conscious teenagers relax.

``You have to ease their anxiety,'' he said.

\hypertarget{schools-provide-more-than-academic-skills}{%
\subsection{Schools provide more than academic
skills.}\label{schools-provide-more-than-academic-skills}}

Even when the devices, WiFi, software, lesson plans and adult
supervision are all in place, a lot is lost when schools transition
students to remote learning. Many children rely on schools for free or
affordable meals, for counseling and for after-school activities while
parents work.

When schools are closed, children lose a crucial social outlet. And
families, especially those who work in the service sector and cannot
easily adjust their schedules, can struggle to find appropriate child
care.

``If you think about it, the school is a city we provide to kids,'' said
Mr. Ridgway. When that city shuts down, he said, no online learning
platform can replace all the structure and vibrancy that is lost.

Kate Taylor contributed reporting.

Advertisement

\protect\hyperlink{after-bottom}{Continue reading the main story}

\hypertarget{site-index}{%
\subsection{Site Index}\label{site-index}}

\hypertarget{site-information-navigation}{%
\subsection{Site Information
Navigation}\label{site-information-navigation}}

\begin{itemize}
\tightlist
\item
  \href{https://help.nytimes3xbfgragh.onion/hc/en-us/articles/115014792127-Copyright-notice}{©~2020~The
  New York Times Company}
\end{itemize}

\begin{itemize}
\tightlist
\item
  \href{https://www.nytco.com/}{NYTCo}
\item
  \href{https://help.nytimes3xbfgragh.onion/hc/en-us/articles/115015385887-Contact-Us}{Contact
  Us}
\item
  \href{https://www.nytco.com/careers/}{Work with us}
\item
  \href{https://nytmediakit.com/}{Advertise}
\item
  \href{http://www.tbrandstudio.com/}{T Brand Studio}
\item
  \href{https://www.nytimes3xbfgragh.onion/privacy/cookie-policy\#how-do-i-manage-trackers}{Your
  Ad Choices}
\item
  \href{https://www.nytimes3xbfgragh.onion/privacy}{Privacy}
\item
  \href{https://help.nytimes3xbfgragh.onion/hc/en-us/articles/115014893428-Terms-of-service}{Terms
  of Service}
\item
  \href{https://help.nytimes3xbfgragh.onion/hc/en-us/articles/115014893968-Terms-of-sale}{Terms
  of Sale}
\item
  \href{https://spiderbites.nytimes3xbfgragh.onion}{Site Map}
\item
  \href{https://help.nytimes3xbfgragh.onion/hc/en-us}{Help}
\item
  \href{https://www.nytimes3xbfgragh.onion/subscription?campaignId=37WXW}{Subscriptions}
\end{itemize}
