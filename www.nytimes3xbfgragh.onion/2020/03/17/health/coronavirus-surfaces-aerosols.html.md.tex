Sections

SEARCH

\protect\hyperlink{site-content}{Skip to
content}\protect\hyperlink{site-index}{Skip to site index}

\href{https://www.nytimes3xbfgragh.onion/section/health}{Health}

\href{https://myaccount.nytimes3xbfgragh.onion/auth/login?response_type=cookie\&client_id=vi}{}

\href{https://www.nytimes3xbfgragh.onion/section/todayspaper}{Today's
Paper}

\href{/section/health}{Health}\textbar{}How Long Will Coronavirus Live
on Surfaces or in the Air Around You?

\url{https://nyti.ms/3daUa9J}

\begin{itemize}
\item
\item
\item
\item
\item
\end{itemize}

\hypertarget{the-coronavirus-outbreak}{%
\subsubsection{\texorpdfstring{\href{https://www.nytimes3xbfgragh.onion/news-event/coronavirus?name=styln-coronavirus-national\&region=TOP_BANNER\&block=storyline_menu_recirc\&action=click\&pgtype=Article\&impression_id=f9067c50-f1c3-11ea-8fac-d5bd7dfae44a\&variant=undefined}{The
Coronavirus
Outbreak}}{The Coronavirus Outbreak}}\label{the-coronavirus-outbreak}}

\begin{itemize}
\tightlist
\item
  live\href{https://www.nytimes3xbfgragh.onion/2020/09/08/world/covid-19-coronavirus.html?name=styln-coronavirus-national\&region=TOP_BANNER\&block=storyline_menu_recirc\&action=click\&pgtype=Article\&impression_id=f906a360-f1c3-11ea-8fac-d5bd7dfae44a\&variant=undefined}{Latest
  Updates}
\item
  \href{https://www.nytimes3xbfgragh.onion/interactive/2020/us/coronavirus-us-cases.html?name=styln-coronavirus-national\&region=TOP_BANNER\&block=storyline_menu_recirc\&action=click\&pgtype=Article\&impression_id=f906a361-f1c3-11ea-8fac-d5bd7dfae44a\&variant=undefined}{Maps
  and Cases}
\item
  \href{https://www.nytimes3xbfgragh.onion/interactive/2020/science/coronavirus-vaccine-tracker.html?name=styln-coronavirus-national\&region=TOP_BANNER\&block=storyline_menu_recirc\&action=click\&pgtype=Article\&impression_id=f906a362-f1c3-11ea-8fac-d5bd7dfae44a\&variant=undefined}{Vaccine
  Tracker}
\item
  \href{https://www.nytimes3xbfgragh.onion/2020/09/02/your-money/eviction-moratorium-covid.html?name=styln-coronavirus-national\&region=TOP_BANNER\&block=storyline_menu_recirc\&action=click\&pgtype=Article\&impression_id=f906a363-f1c3-11ea-8fac-d5bd7dfae44a\&variant=undefined}{Eviction
  Moratorium}
\item
  \href{https://www.nytimes3xbfgragh.onion/interactive/2020/09/02/magazine/food-insecurity-hunger-us.html?name=styln-coronavirus-national\&region=TOP_BANNER\&block=storyline_menu_recirc\&action=click\&pgtype=Article\&impression_id=f906a364-f1c3-11ea-8fac-d5bd7dfae44a\&variant=undefined}{American
  Hunger}
\end{itemize}

Advertisement

\protect\hyperlink{after-top}{Continue reading the main story}

Supported by

\protect\hyperlink{after-sponsor}{Continue reading the main story}

\hypertarget{how-long-will-coronavirus-live-on-surfaces-or-in-the-air-around-you}{%
\section{How Long Will Coronavirus Live on Surfaces or in the Air Around
You?}\label{how-long-will-coronavirus-live-on-surfaces-or-in-the-air-around-you}}

A new study could have implications for how the general public and
health care workers try to avoid transmission of the virus.

\includegraphics{https://static01.graylady3jvrrxbe.onion/images/2020/03/19/science/17VIRUS-GERMS1/17VIRUS-GERMS1-articleLarge.jpg?quality=75\&auto=webp\&disable=upscale}

By
\href{https://www.nytimes3xbfgragh.onion/by/apoorva-mandavilli}{Apoorva
Mandavilli}

\begin{itemize}
\item
  Published March 17, 2020Updated July 6, 2020
\item
  \begin{itemize}
  \item
  \item
  \item
  \item
  \item
  \end{itemize}
\end{itemize}

\href{https://www.nytimes3xbfgragh.onion/es/2020/03/25/espanol/coronavirus-aire-como-se-transmite.html}{Leer
en español}

The coronavirus can live for three days on some surfaces, like plastic
and steel, new research suggests. Experts say the risk of consumers
getting infected from touching those materials is still low, although
they offered additional warnings about how long the virus survives in
air, which may have important implications for medical workers.

The new study,
\href{https://www.nejm.org/doi/full/10.1056/NEJMc2004973}{published
Tuesday in The New England Journal of Medicine}, also suggests that the
virus disintegrates over the course of a day on cardboard, lessening the
worry among consumers that deliveries will spread the virus during this
period of staying and working from home.

When the virus becomes suspended in droplets smaller than five
micrometers --- known as aerosols --- it can stay suspended for about a
half-hour, researchers said, before drifting down and settling on
surfaces where it can linger for hours. (In the study's experimental
setup, the virus stayed suspended for three hours, but it would drift
down much sooner under most conditions.) The finding on aerosol in
particular is inconsistent with the World Health Organization's position
that the virus is not transported by air.

The virus lives longest on plastic and steel, surviving for up to 72
hours. But the amount of viable virus decreases sharply over this time.
It also does poorly on copper, surviving four hours. On cardboard, it
survives up to 24 hours, which suggests packages that arrive **** in the
mail should have only low levels of the virus **** --- unless the
delivery person has coughed or sneezed on it or has handled it with
contaminated hands.

That's true in general. Unless the people who handle any of these
materials are sick, the actual risk of getting infected from any of
these materials is low, experts said.

``Everything at the grocery store and restaurant takeout containers and
bags could in theory have infectious virus on them,'' said Dr. Linsey
Marr, who was not a member of the research team but is an expert in the
transmission of viruses by aerosol at Virginia Tech in Blacksburg. ``We
could go crazy discussing these `what ifs' because everyone is a
potential source, so we have to focus on the biggest risks.''

\hypertarget{latest-updates-the-coronavirus-outbreak}{%
\section{\texorpdfstring{\href{https://www.nytimes3xbfgragh.onion/2020/09/08/world/covid-19-coronavirus.html?action=click\&pgtype=Article\&state=default\&region=MAIN_CONTENT_1\&context=storylines_live_updates}{Latest
Updates: The Coronavirus
Outbreak}}{Latest Updates: The Coronavirus Outbreak}}\label{latest-updates-the-coronavirus-outbreak}}

Updated 2020-09-08T11:04:36.368Z

\begin{itemize}
\tightlist
\item
  \href{https://www.nytimes3xbfgragh.onion/2020/09/08/world/covid-19-coronavirus.html?action=click\&pgtype=Article\&state=default\&region=MAIN_CONTENT_1\&context=storylines_live_updates\#link-4a77847f}{As
  senators return to Washington, an impasse over a virus relief package
  looms.}
\item
  \href{https://www.nytimes3xbfgragh.onion/2020/09/08/world/covid-19-coronavirus.html?action=click\&pgtype=Article\&state=default\&region=MAIN_CONTENT_1\&context=storylines_live_updates\#link-679303d7}{Nine
  drugmakers pledge to thoroughly vet any coronavirus vaccine.}
\item
  \href{https://www.nytimes3xbfgragh.onion/2020/09/08/world/covid-19-coronavirus.html?action=click\&pgtype=Article\&state=default\&region=MAIN_CONTENT_1\&context=storylines_live_updates\#link-1c973131}{`The
  lockdown killed my father': Farmer suicides add to India's virus
  misery.}
\end{itemize}

\href{https://www.nytimes3xbfgragh.onion/2020/09/08/world/covid-19-coronavirus.html?action=click\&pgtype=Article\&state=default\&region=MAIN_CONTENT_1\&context=storylines_live_updates}{See
more updates}

More live coverage:
\href{https://www.nytimes3xbfgragh.onion/live/2020/09/08/business/stock-market-today-coronavirus?action=click\&pgtype=Article\&state=default\&region=MAIN_CONTENT_1\&context=storylines_live_updates}{Markets}

If people are concerned about the risk, they could wipe down packages
with disinfectant wipes and wash their hands, she said.

It is unclear why cardboard should be a less hospitable environment for
the virus than plastic or steel, but it may be explained by the
absorbency or fibrous quality of the packaging compared with the other
surfaces.

That the virus can survive and stay infectious in aerosols is also
important for health care workers.

For weeks experts have maintained that the virus is not airborne. But in
fact, it can travel through the air and stay suspended for that period
of about a half-hour.

The virus does not linger in the air at high enough levels to be a risk
to most people who are not physically near an infected person. But the
procedures health care workers use to care for infected patients are
likely to generate aerosols.

``Once you get a patient in with severe pneumonia, the patients need to
be intubated,'' said Dr. Vincent Munster, a virologist at the National
Institute of Allergy and Infectious Diseases who led the study. ``All
these handlings might generate aerosols and droplets.''

Health care workers might also collect those tiny droplets and larger
ones on their protective gear when working with infected patients. They
might resuspend these big and small droplets into the air when they take
off this protective gear and become exposed to the virus then, Dr. Marr
cautioned.

A study that is being reviewed by experts
\href{https://www.biorxiv.org/content/10.1101/2020.03.08.982637v1}{bears
out this fear}. And another study, published March 4 in JAMA, also
indicates that the virus is transported by air. That study, based in
Singapore, found the
\href{https://jamanetwork.com/journals/jama/fullarticle/2762692}{virus
on a vent} in the hospital room of an infected patient, where it could
only have reached via the air.

Dr. Marr said the World Health Organization had so far referred to the
virus as not airborne, but that health care workers should wear gear,
including respirator masks, assuming that it is.

``Based on aerosol science and recent findings on flu virus,'' she said,
``surgical masks are probably insufficient.''

Dr. Marr said based on physics, an aerosol released at a height of about
six feet would fall to the ground after 34 minutes. The findings should
not cause the general public to panic, however, because the virus
disperses quickly in the air.

\href{https://www.nytimes3xbfgragh.onion/news-event/coronavirus?action=click\&pgtype=Article\&state=default\&region=MAIN_CONTENT_3\&context=storylines_faq}{}

\hypertarget{the-coronavirus-outbreak-}{%
\subsubsection{The Coronavirus Outbreak
›}\label{the-coronavirus-outbreak-}}

\hypertarget{frequently-asked-questions}{%
\paragraph{Frequently Asked
Questions}\label{frequently-asked-questions}}

Updated September 4, 2020

\begin{itemize}
\item ~
  \hypertarget{what-are-the-symptoms-of-coronavirus}{%
  \paragraph{What are the symptoms of
  coronavirus?}\label{what-are-the-symptoms-of-coronavirus}}

  \begin{itemize}
  \tightlist
  \item
    In the beginning, the coronavirus
    \href{https://www.nytimes3xbfgragh.onion/article/coronavirus-facts-history.html?action=click\&pgtype=Article\&state=default\&region=MAIN_CONTENT_3\&context=storylines_faq\#link-6817bab5}{seemed
    like it was primarily a respiratory illness}~--- many patients had
    fever and chills, were weak and tired, and coughed a lot, though
    some people don't show many symptoms at all. Those who seemed
    sickest had pneumonia or acute respiratory distress syndrome and
    received supplemental oxygen. By now, doctors have identified many
    more symptoms and syndromes. In April,
    \href{https://www.nytimes3xbfgragh.onion/2020/04/27/health/coronavirus-symptoms-cdc.html?action=click\&pgtype=Article\&state=default\&region=MAIN_CONTENT_3\&context=storylines_faq}{the
    C.D.C. added to the list of early signs}~sore throat, fever, chills
    and muscle aches. Gastrointestinal upset, such as diarrhea and
    nausea, has also been observed. Another telltale sign of infection
    may be a sudden, profound diminution of one's
    \href{https://www.nytimes3xbfgragh.onion/2020/03/22/health/coronavirus-symptoms-smell-taste.html?action=click\&pgtype=Article\&state=default\&region=MAIN_CONTENT_3\&context=storylines_faq}{sense
    of smell and taste.}~Teenagers and young adults in some cases have
    developed painful red and purple lesions on their fingers and toes
    --- nicknamed ``Covid toe'' --- but few other serious symptoms.
  \end{itemize}
\item ~
  \hypertarget{why-is-it-safer-to-spend-time-together-outside}{%
  \paragraph{Why is it safer to spend time together
  outside?}\label{why-is-it-safer-to-spend-time-together-outside}}

  \begin{itemize}
  \tightlist
  \item
    \href{https://www.nytimes3xbfgragh.onion/2020/05/15/us/coronavirus-what-to-do-outside.html?action=click\&pgtype=Article\&state=default\&region=MAIN_CONTENT_3\&context=storylines_faq}{Outdoor
    gatherings}~lower risk because wind disperses viral droplets, and
    sunlight can kill some of the virus. Open spaces prevent the virus
    from building up in concentrated amounts and being inhaled, which
    can happen when infected people exhale in a confined space for long
    stretches of time, said Dr. Julian W. Tang, a virologist at the
    University of Leicester.
  \end{itemize}
\item ~
  \hypertarget{why-does-standing-six-feet-away-from-others-help}{%
  \paragraph{Why does standing six feet away from others
  help?}\label{why-does-standing-six-feet-away-from-others-help}}

  \begin{itemize}
  \tightlist
  \item
    The coronavirus spreads primarily through droplets from your mouth
    and nose, especially when you cough or sneeze. The C.D.C., one of
    the organizations using that measure,
    \href{https://www.nytimes3xbfgragh.onion/2020/04/14/health/coronavirus-six-feet.html?action=click\&pgtype=Article\&state=default\&region=MAIN_CONTENT_3\&context=storylines_faq}{bases
    its recommendation of six feet}~on the idea that most large droplets
    that people expel when they cough or sneeze will fall to the ground
    within six feet. But six feet has never been a magic number that
    guarantees complete protection. Sneezes, for instance, can launch
    droplets a lot farther than six feet,
    \href{https://jamanetwork.com/journals/jama/fullarticle/2763852}{according
    to a recent study}. It's a rule of thumb: You should be safest
    standing six feet apart outside, especially when it's windy. But
    keep a mask on at all times, even when you think you're far enough
    apart.
  \end{itemize}
\item ~
  \hypertarget{i-have-antibodies-am-i-now-immune}{%
  \paragraph{I have antibodies. Am I now
  immune?}\label{i-have-antibodies-am-i-now-immune}}

  \begin{itemize}
  \tightlist
  \item
    As of right
    now,\href{https://www.nytimes3xbfgragh.onion/2020/07/22/health/covid-antibodies-herd-immunity.html?action=click\&pgtype=Article\&state=default\&region=MAIN_CONTENT_3\&context=storylines_faq}{~that
    seems likely, for at least several months.}~There have been
    frightening accounts of people suffering what seems to be a second
    bout of Covid-19. But experts say these patients may have a
    drawn-out course of infection, with the virus taking a slow toll
    weeks to months after initial exposure.~People infected with the
    coronavirus typically
    \href{https://www.nature.com/articles/s41586-020-2456-9}{produce}~immune
    molecules called antibodies, which are
    \href{https://www.nytimes3xbfgragh.onion/2020/05/07/health/coronavirus-antibody-prevalence.html?action=click\&pgtype=Article\&state=default\&region=MAIN_CONTENT_3\&context=storylines_faq}{protective
    proteins made in response to an
    infection}\href{https://www.nytimes3xbfgragh.onion/2020/05/07/health/coronavirus-antibody-prevalence.html?action=click\&pgtype=Article\&state=default\&region=MAIN_CONTENT_3\&context=storylines_faq}{.
    These antibodies may}~last in the body
    \href{https://www.nature.com/articles/s41591-020-0965-6}{only two to
    three months}, which may seem worrisome, but that's~perfectly normal
    after an acute infection subsides, said Dr. Michael Mina, an
    immunologist at Harvard University. It may be possible to get the
    coronavirus again, but it's highly unlikely that it would be
    possible in a short window of time from initial infection or make
    people sicker the second time.
  \end{itemize}
\item ~
  \hypertarget{what-are-my-rights-if-i-am-worried-about-going-back-to-work}{%
  \paragraph{What are my rights if I am worried about going back to
  work?}\label{what-are-my-rights-if-i-am-worried-about-going-back-to-work}}

  \begin{itemize}
  \tightlist
  \item
    Employers have to provide
    \href{https://www.osha.gov/SLTC/covid-19/standards.html}{a safe
    workplace}~with policies that protect everyone equally.
    \href{https://www.nytimes3xbfgragh.onion/article/coronavirus-money-unemployment.html?action=click\&pgtype=Article\&state=default\&region=MAIN_CONTENT_3\&context=storylines_faq}{And
    if one of your co-workers tests positive for the coronavirus, the
    C.D.C.}~has said that
    \href{https://www.cdc.gov/coronavirus/2019-ncov/community/guidance-business-response.html}{employers
    should tell their employees}~-\/- without giving you the sick
    employee's name -\/- that they may have been exposed to the virus.
  \end{itemize}
\end{itemize}

``It sounds scary,'' she said, ``but unless you're close to someone, the
amount you've been exposed to is very low.''

Dr. Marr compared this to cigarette smoke or a foggy breath on a frosty
day. The closer and sooner another person is to the exhaled smoke or
breath, the more of a whiff they might catch; for anyone farther than a
few feet away, there is too little of the virus in the air to be any
danger.

To assess the ability of the virus to survive in the air, the
researchers created what Dr. Munster described as ``bizarre experiments
done under very ideal controllable experimental conditions.'' They used
a rotating drum to suspend the aerosols, and provided temperature and
humidity levels that closely mimic hospital conditions.

In this setup, the virus survived and stayed infectious for up to three
hours, but its ability to infect drops sharply over this time, he said.

He said the aerosols might stay aloft only for about 10 minutes, but Dr.
Marr disagreed with that assessment, and said they could stay in the air
for three times longer. She also said that the experimental setup might
be less comfortable for the virus than a real-life setting.

For example, she said, the researchers used a relative humidity of 65
percent. ``Many, but not all viruses, have shown that they survive worst
at this level of humidity,'' she said. They do best at lower or much
higher humidity. The humidity in a heated house is less than 40 percent,
``at which the virus might survive even longer,'' she said.

Mucus and respiratory fluids might also allow the virus to survive
longer than the laboratory fluids the researchers used for their
experiments.

Other experts said the paper's findings illustrate the urgent need for
more information about the virus's ability to survive in aerosols, and
under different conditions.

``We need more experiments like this, in particular, extending the
experimental sampling time for aerosolized virus beyond three hours and
testing survival under different temperature and humidity conditions,''
said Dr. Jeffrey Shaman, an environmental health sciences expert at
Columbia University.

Dr. Munster noted that, over all, the new coronavirus seems no more
capable of surviving for long periods than its close cousins SARS and
MERS, which caused previous epidemics. That suggests there are other
reasons, such as transmission by people who don't have symptoms, for its
ability to cause a pandemic.

Advertisement

\protect\hyperlink{after-bottom}{Continue reading the main story}

\hypertarget{site-index}{%
\subsection{Site Index}\label{site-index}}

\hypertarget{site-information-navigation}{%
\subsection{Site Information
Navigation}\label{site-information-navigation}}

\begin{itemize}
\tightlist
\item
  \href{https://help.nytimes3xbfgragh.onion/hc/en-us/articles/115014792127-Copyright-notice}{©~2020~The
  New York Times Company}
\end{itemize}

\begin{itemize}
\tightlist
\item
  \href{https://www.nytco.com/}{NYTCo}
\item
  \href{https://help.nytimes3xbfgragh.onion/hc/en-us/articles/115015385887-Contact-Us}{Contact
  Us}
\item
  \href{https://www.nytco.com/careers/}{Work with us}
\item
  \href{https://nytmediakit.com/}{Advertise}
\item
  \href{http://www.tbrandstudio.com/}{T Brand Studio}
\item
  \href{https://www.nytimes3xbfgragh.onion/privacy/cookie-policy\#how-do-i-manage-trackers}{Your
  Ad Choices}
\item
  \href{https://www.nytimes3xbfgragh.onion/privacy}{Privacy}
\item
  \href{https://help.nytimes3xbfgragh.onion/hc/en-us/articles/115014893428-Terms-of-service}{Terms
  of Service}
\item
  \href{https://help.nytimes3xbfgragh.onion/hc/en-us/articles/115014893968-Terms-of-sale}{Terms
  of Sale}
\item
  \href{https://spiderbites.nytimes3xbfgragh.onion}{Site Map}
\item
  \href{https://help.nytimes3xbfgragh.onion/hc/en-us}{Help}
\item
  \href{https://www.nytimes3xbfgragh.onion/subscription?campaignId=37WXW}{Subscriptions}
\end{itemize}
