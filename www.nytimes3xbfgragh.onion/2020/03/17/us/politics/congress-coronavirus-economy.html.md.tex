Sections

SEARCH

\protect\hyperlink{site-content}{Skip to
content}\protect\hyperlink{site-index}{Skip to site index}

\href{https://www.nytimes3xbfgragh.onion/section/politics}{Politics}

\href{https://myaccount.nytimes3xbfgragh.onion/auth/login?response_type=cookie\&client_id=vi}{}

\href{https://www.nytimes3xbfgragh.onion/section/todayspaper}{Today's
Paper}

\href{/section/politics}{Politics}\textbar{}Congress Is Knitting a
Coronavirus Safety Net. It Already Has Big Holes.

\url{https://nyti.ms/2QoRL1I}

\begin{itemize}
\item
\item
\item
\item
\item
\end{itemize}

\hypertarget{the-coronavirus-outbreak}{%
\subsubsection{\texorpdfstring{\href{https://www.nytimes3xbfgragh.onion/news-event/coronavirus?name=styln-coronavirus-national\&region=TOP_BANNER\&block=storyline_menu_recirc\&action=click\&pgtype=Article\&impression_id=6acc1730-f27f-11ea-bc60-4f9e0d7af645\&variant=undefined}{The
Coronavirus
Outbreak}}{The Coronavirus Outbreak}}\label{the-coronavirus-outbreak}}

\begin{itemize}
\tightlist
\item
  live\href{https://www.nytimes3xbfgragh.onion/2020/09/08/world/covid-19-coronavirus.html?name=styln-coronavirus-national\&region=TOP_BANNER\&block=storyline_menu_recirc\&action=click\&pgtype=Article\&impression_id=6acc1731-f27f-11ea-bc60-4f9e0d7af645\&variant=undefined}{Latest
  Updates}
\item
  \href{https://www.nytimes3xbfgragh.onion/interactive/2020/us/coronavirus-us-cases.html?name=styln-coronavirus-national\&region=TOP_BANNER\&block=storyline_menu_recirc\&action=click\&pgtype=Article\&impression_id=6acc1732-f27f-11ea-bc60-4f9e0d7af645\&variant=undefined}{Maps
  and Cases}
\item
  \href{https://www.nytimes3xbfgragh.onion/interactive/2020/science/coronavirus-vaccine-tracker.html?name=styln-coronavirus-national\&region=TOP_BANNER\&block=storyline_menu_recirc\&action=click\&pgtype=Article\&impression_id=6acc3e40-f27f-11ea-bc60-4f9e0d7af645\&variant=undefined}{Vaccine
  Tracker}
\item
  \href{https://www.nytimes3xbfgragh.onion/2020/09/02/your-money/eviction-moratorium-covid.html?name=styln-coronavirus-national\&region=TOP_BANNER\&block=storyline_menu_recirc\&action=click\&pgtype=Article\&impression_id=6acc3e41-f27f-11ea-bc60-4f9e0d7af645\&variant=undefined}{Eviction
  Moratorium}
\item
  \href{https://www.nytimes3xbfgragh.onion/interactive/2020/09/02/magazine/food-insecurity-hunger-us.html?name=styln-coronavirus-national\&region=TOP_BANNER\&block=storyline_menu_recirc\&action=click\&pgtype=Article\&impression_id=6acc3e42-f27f-11ea-bc60-4f9e0d7af645\&variant=undefined}{American
  Hunger}
\end{itemize}

Advertisement

\protect\hyperlink{after-top}{Continue reading the main story}

Supported by

\protect\hyperlink{after-sponsor}{Continue reading the main story}

\hypertarget{congress-is-knitting-a-coronavirus-safety-net-it-already-has-big-holes}{%
\section{Congress Is Knitting a Coronavirus Safety Net. It Already Has
Big
Holes.}\label{congress-is-knitting-a-coronavirus-safety-net-it-already-has-big-holes}}

Lawmakers and policy experts say a paid-leave program that passed the
House would still discourage some virus-stricken workers from staying
home.

\includegraphics{https://static01.graylady3jvrrxbe.onion/images/2020/03/17/us/politics/17dc-virus-paidleave/merlin_170651118_8a653aab-2824-428c-ad04-18f672e97c69-articleLarge.jpg?quality=75\&auto=webp\&disable=upscale}

\href{https://www.nytimes3xbfgragh.onion/by/jim-tankersley}{\includegraphics{https://static01.graylady3jvrrxbe.onion/images/2018/10/19/multimedia/author-jim-tankersley/author-jim-tankersley-thumbLarge.png}}\href{https://www.nytimes3xbfgragh.onion/by/emily-cochrane}{\includegraphics{https://static01.graylady3jvrrxbe.onion/images/2018/11/28/multimedia/author-emily-cochrane/author-emily-cochrane-thumbLarge-v3.png}}

By \href{https://www.nytimes3xbfgragh.onion/by/jim-tankersley}{Jim
Tankersley} and
\href{https://www.nytimes3xbfgragh.onion/by/emily-cochrane}{Emily
Cochrane}

\begin{itemize}
\item
  March 17, 2020
\item
  \begin{itemize}
  \item
  \item
  \item
  \item
  \item
  \end{itemize}
\end{itemize}

WASHINGTON --- In an effort to slow the spread of coronavirus and limit
its damage to the economy, congressional leaders and President Trump are
stitching together a new safety net for workers who contract the
coronavirus or lose their jobs because of it.

But compromises in the legislation set to pass the Senate this week have
ripped holes in that net, exposing millions of workers to financial
risks that could push them to continue reporting to work --- even on the
front lines of the pandemic --- and accelerate the infection rate
nationwide.

Democrats in Congress now say they will attempt to patch those holes by
expanding a government-funded paid-leave benefit and making unemployment
payments more generous for workers laid off as the economy rapidly
shutters under government orders seeking to curb the virus. But that
push will almost certainly have to wait for the next phase of the
congressional response, an increasingly expensive fight that could meet
resistance from business groups worried that the program could drive
small companies out of existence by allowing critical workers to stay at
home.

After a month of debate over the severity of the crisis, Mr. Trump and
lawmakers have in recent days largely coalesced around a strategy for
fighting the pandemic. It is based on the idea that the virus will not
spread as quickly if Americans largely stay home from work.

Economists say that executing that idea includes telling Americans to
work from home if they can and paying them not to go to the office if
they are sick or caring for children whose school is canceled. Ideally,
experts say, all but the most essential workers would stay home --- and
so would everyone who is ill. The government should then compensate the
workers and businesses hurt by the ensuing economic slowdown.

``We need people to socially isolate and socially distance, and we need
people to stay home if they're sick,'' said Heather Boushey, the
president of the Washington Center for Equitable Growth, a left-leaning
think tank. ``Our solutions need to be calibrated to the problem at
hand.''

So far, Ms. Boushey and others across the political spectrum say,
Congress's efforts have fallen short.

\hypertarget{latest-updates-the-coronavirus-outbreak}{%
\section{\texorpdfstring{\href{https://www.nytimes3xbfgragh.onion/2020/09/08/world/covid-19-coronavirus.html?action=click\&pgtype=Article\&state=default\&region=MAIN_CONTENT_1\&context=storylines_live_updates}{Latest
Updates: The Coronavirus
Outbreak}}{Latest Updates: The Coronavirus Outbreak}}\label{latest-updates-the-coronavirus-outbreak}}

Updated 2020-09-09T09:27:30.328Z

\begin{itemize}
\tightlist
\item
  \href{https://www.nytimes3xbfgragh.onion/2020/09/08/world/covid-19-coronavirus.html?action=click\&pgtype=Article\&state=default\&region=MAIN_CONTENT_1\&context=storylines_live_updates\#link-313b443d}{AstraZeneca
  halts a vaccine trial to investigate a participant's illness.}
\item
  \href{https://www.nytimes3xbfgragh.onion/2020/09/08/world/covid-19-coronavirus.html?action=click\&pgtype=Article\&state=default\&region=MAIN_CONTENT_1\&context=storylines_live_updates\#link-4438dd7}{Facing
  a surge in cases, Britain plans to limit most gatherings to six
  people.}
\item
  \href{https://www.nytimes3xbfgragh.onion/2020/09/08/world/covid-19-coronavirus.html?action=click\&pgtype=Article\&state=default\&region=MAIN_CONTENT_1\&context=storylines_live_updates\#link-679303d7}{Nine
  drugmakers pledge to thoroughly vet any coronavirus vaccine.}
\end{itemize}

\href{https://www.nytimes3xbfgragh.onion/2020/09/08/world/covid-19-coronavirus.html?action=click\&pgtype=Article\&state=default\&region=MAIN_CONTENT_1\&context=storylines_live_updates}{See
more updates}

More live coverage:
\href{https://www.nytimes3xbfgragh.onion/live/2020/09/08/business/stock-market-today-coronavirus?action=click\&pgtype=Article\&state=default\&region=MAIN_CONTENT_1\&context=storylines_live_updates}{Markets}

The House
\href{https://www.nytimes3xbfgragh.onion/2020/03/13/us/politics/trump-coronavirus-relief-congress.html}{passed
a sweeping virus response bill} early Saturday, but Democrats and the
Trump administration continued to negotiate through Monday over
so-called technical corrections to a \$100 billion program in the
legislation. That program is meant to offer paid leave to workers
infected with the virus or otherwise hurt by it. The negotiations
narrowed the program, in part over the administration's concerns about
burdening small businesses, despite objections from some lawmakers and
economists.

The changes passed the House late Monday evening, with few lawmakers
present or even able to read the legislation before it was approved. A
procedural maneuver allows legislation to be approved without the entire
chamber present so long as a single lawmaker does not object.

Under those changes, workers affected by the pandemic --- as a result of
quarantine, caring for a family member, or closed schools and lack of
child care --- receive two weeks of sick leave. Any paid leave provided
after that time is limited to workers with children whose school or
child care has been closed. The original legislation offered 10
additional weeks of paid leave at two-thirds pay for all workers
affected by the pandemic.

The bill allows exemptions for workers in companies with more than 500
employees or fewer than 50. Under the amended version, health care
providers or emergency responders could be declared exempt from
receiving the additional paid leave by the Labor secretary, who is now
given the discretion to make that decision.

Ms. Boushey said she was alarmed by those changes. ``I don't want my
first responder to be sick,'' she said, adding that the scope of
benefits in the bill for workers who lose their jobs amid the crisis was
also inadequate. She and other economists, like
\href{https://www.forbes.com/sites/andrewbiggs/2020/03/17/boost-unemployment-benefits-to-fight-a-coronavirus-recession/\#60e522d043ee}{Andrew
Biggs of the conservative American Enterprise Institute}, say the
government should effectively make up for the full lost wages of workers
who become unemployed, rather than partially doing so. Ms. Boushey said
lawmakers should also fully compensate workers whose hours are reduced
during the pandemic.

Senate Democrats, including Chuck Schumer of New York, the Democratic
leader, say they will push for expansions of paid leave and unemployment
benefits in a new bill, in its early stages on Capitol Hill, that is
likely to include \$1 trillion or more of economic stimulus, like direct
payments to workers.

Senator Patty Murray, Democrat of Washington, introduced another
paid-leave bill on Tuesday along with other Democratic lawmakers, as
part of an effort to expand upon those provisions in a third coronavirus
relief package still under discussion.

``How do we make sure that people will stay home and not spread the
virus?'' said Ms. Murray, whose state has been a center of the outbreak
in the United States. ``Either we give people the means to stay home or
this will continue to spread.''

\href{https://www.nytimes3xbfgragh.onion/news-event/coronavirus?action=click\&pgtype=Article\&state=default\&region=MAIN_CONTENT_3\&context=storylines_faq}{}

\hypertarget{the-coronavirus-outbreak-}{%
\subsubsection{The Coronavirus Outbreak
›}\label{the-coronavirus-outbreak-}}

\hypertarget{frequently-asked-questions}{%
\paragraph{Frequently Asked
Questions}\label{frequently-asked-questions}}

Updated September 4, 2020

\begin{itemize}
\item ~
  \hypertarget{what-are-the-symptoms-of-coronavirus}{%
  \paragraph{What are the symptoms of
  coronavirus?}\label{what-are-the-symptoms-of-coronavirus}}

  \begin{itemize}
  \tightlist
  \item
    In the beginning, the coronavirus
    \href{https://www.nytimes3xbfgragh.onion/article/coronavirus-facts-history.html?action=click\&pgtype=Article\&state=default\&region=MAIN_CONTENT_3\&context=storylines_faq\#link-6817bab5}{seemed
    like it was primarily a respiratory illness}~--- many patients had
    fever and chills, were weak and tired, and coughed a lot, though
    some people don't show many symptoms at all. Those who seemed
    sickest had pneumonia or acute respiratory distress syndrome and
    received supplemental oxygen. By now, doctors have identified many
    more symptoms and syndromes. In April,
    \href{https://www.nytimes3xbfgragh.onion/2020/04/27/health/coronavirus-symptoms-cdc.html?action=click\&pgtype=Article\&state=default\&region=MAIN_CONTENT_3\&context=storylines_faq}{the
    C.D.C. added to the list of early signs}~sore throat, fever, chills
    and muscle aches. Gastrointestinal upset, such as diarrhea and
    nausea, has also been observed. Another telltale sign of infection
    may be a sudden, profound diminution of one's
    \href{https://www.nytimes3xbfgragh.onion/2020/03/22/health/coronavirus-symptoms-smell-taste.html?action=click\&pgtype=Article\&state=default\&region=MAIN_CONTENT_3\&context=storylines_faq}{sense
    of smell and taste.}~Teenagers and young adults in some cases have
    developed painful red and purple lesions on their fingers and toes
    --- nicknamed ``Covid toe'' --- but few other serious symptoms.
  \end{itemize}
\item ~
  \hypertarget{why-is-it-safer-to-spend-time-together-outside}{%
  \paragraph{Why is it safer to spend time together
  outside?}\label{why-is-it-safer-to-spend-time-together-outside}}

  \begin{itemize}
  \tightlist
  \item
    \href{https://www.nytimes3xbfgragh.onion/2020/05/15/us/coronavirus-what-to-do-outside.html?action=click\&pgtype=Article\&state=default\&region=MAIN_CONTENT_3\&context=storylines_faq}{Outdoor
    gatherings}~lower risk because wind disperses viral droplets, and
    sunlight can kill some of the virus. Open spaces prevent the virus
    from building up in concentrated amounts and being inhaled, which
    can happen when infected people exhale in a confined space for long
    stretches of time, said Dr. Julian W. Tang, a virologist at the
    University of Leicester.
  \end{itemize}
\item ~
  \hypertarget{why-does-standing-six-feet-away-from-others-help}{%
  \paragraph{Why does standing six feet away from others
  help?}\label{why-does-standing-six-feet-away-from-others-help}}

  \begin{itemize}
  \tightlist
  \item
    The coronavirus spreads primarily through droplets from your mouth
    and nose, especially when you cough or sneeze. The C.D.C., one of
    the organizations using that measure,
    \href{https://www.nytimes3xbfgragh.onion/2020/04/14/health/coronavirus-six-feet.html?action=click\&pgtype=Article\&state=default\&region=MAIN_CONTENT_3\&context=storylines_faq}{bases
    its recommendation of six feet}~on the idea that most large droplets
    that people expel when they cough or sneeze will fall to the ground
    within six feet. But six feet has never been a magic number that
    guarantees complete protection. Sneezes, for instance, can launch
    droplets a lot farther than six feet,
    \href{https://jamanetwork.com/journals/jama/fullarticle/2763852}{according
    to a recent study}. It's a rule of thumb: You should be safest
    standing six feet apart outside, especially when it's windy. But
    keep a mask on at all times, even when you think you're far enough
    apart.
  \end{itemize}
\item ~
  \hypertarget{i-have-antibodies-am-i-now-immune}{%
  \paragraph{I have antibodies. Am I now
  immune?}\label{i-have-antibodies-am-i-now-immune}}

  \begin{itemize}
  \tightlist
  \item
    As of right
    now,\href{https://www.nytimes3xbfgragh.onion/2020/07/22/health/covid-antibodies-herd-immunity.html?action=click\&pgtype=Article\&state=default\&region=MAIN_CONTENT_3\&context=storylines_faq}{~that
    seems likely, for at least several months.}~There have been
    frightening accounts of people suffering what seems to be a second
    bout of Covid-19. But experts say these patients may have a
    drawn-out course of infection, with the virus taking a slow toll
    weeks to months after initial exposure.~People infected with the
    coronavirus typically
    \href{https://www.nature.com/articles/s41586-020-2456-9}{produce}~immune
    molecules called antibodies, which are
    \href{https://www.nytimes3xbfgragh.onion/2020/05/07/health/coronavirus-antibody-prevalence.html?action=click\&pgtype=Article\&state=default\&region=MAIN_CONTENT_3\&context=storylines_faq}{protective
    proteins made in response to an
    infection}\href{https://www.nytimes3xbfgragh.onion/2020/05/07/health/coronavirus-antibody-prevalence.html?action=click\&pgtype=Article\&state=default\&region=MAIN_CONTENT_3\&context=storylines_faq}{.
    These antibodies may}~last in the body
    \href{https://www.nature.com/articles/s41591-020-0965-6}{only two to
    three months}, which may seem worrisome, but that's~perfectly normal
    after an acute infection subsides, said Dr. Michael Mina, an
    immunologist at Harvard University. It may be possible to get the
    coronavirus again, but it's highly unlikely that it would be
    possible in a short window of time from initial infection or make
    people sicker the second time.
  \end{itemize}
\item ~
  \hypertarget{what-are-my-rights-if-i-am-worried-about-going-back-to-work}{%
  \paragraph{What are my rights if I am worried about going back to
  work?}\label{what-are-my-rights-if-i-am-worried-about-going-back-to-work}}

  \begin{itemize}
  \tightlist
  \item
    Employers have to provide
    \href{https://www.osha.gov/SLTC/covid-19/standards.html}{a safe
    workplace}~with policies that protect everyone equally.
    \href{https://www.nytimes3xbfgragh.onion/article/coronavirus-money-unemployment.html?action=click\&pgtype=Article\&state=default\&region=MAIN_CONTENT_3\&context=storylines_faq}{And
    if one of your co-workers tests positive for the coronavirus, the
    C.D.C.}~has said that
    \href{https://www.cdc.gov/coronavirus/2019-ncov/community/guidance-business-response.html}{employers
    should tell their employees}~-\/- without giving you the sick
    employee's name -\/- that they may have been exposed to the virus.
  \end{itemize}
\end{itemize}

In a sign that the debate defies traditional partisan lines, the
conservative group Heritage Action for America said on Tuesday that the
House's paid-leave plan should be expanded to more workers.

But business groups and their advocates continue to raise concerns about
the paid-leave efforts.

Some small business groups have opposed the plan. ``The main thing I've
heard, and it's almost been in unanimity among small business owners,''
said Senator Mike Braun, Republican of Indiana, ``is that they don't
like the structure, putting the onus out of the gate onto small business
owners.''

Senator Shelley Moore Capito, Republican of West Virginia, who supports
the package put forward by the House, said that ``there's concerns
about, say, doctor's offices and health care workers.''

``If the requirements are for paid sick leave, how are you going to move
that in with essential workers? I think that's one thing,'' she said.
``I'm sure there's some questions on the mandate aspect of it, and how
it will actually work.''

``I think there was an admission in the room --- and probably in the
whole building --- that the bureaucracy of trying to figure the quickest
and easiest and less painful way to direct dollars to people always
takes longer than what people would think,'' she added, speaking after a
closed lunch with Senate Republicans. ``So I think that's a source of
concern for everybody.''

Others raised concerns that workers would leave their jobs if the
program was too generous, jeopardizing business health.

``I don't like the family leave setup,'' said Senator Lindsey Graham,
Republican of South Carolina. ``I think we should have gone through the
unemployment insurance route. We should have said to any company out
there, any worker, that if you can't work because of the coronavirus you
are going to get your check.''

``My focus is not giving people a check from the government,'' he told
reporters on Tuesday. ``My focus is to make sure you get your paycheck
from your employer.

Nicholas Fandos contributed reporting.

Advertisement

\protect\hyperlink{after-bottom}{Continue reading the main story}

\hypertarget{site-index}{%
\subsection{Site Index}\label{site-index}}

\hypertarget{site-information-navigation}{%
\subsection{Site Information
Navigation}\label{site-information-navigation}}

\begin{itemize}
\tightlist
\item
  \href{https://help.nytimes3xbfgragh.onion/hc/en-us/articles/115014792127-Copyright-notice}{©~2020~The
  New York Times Company}
\end{itemize}

\begin{itemize}
\tightlist
\item
  \href{https://www.nytco.com/}{NYTCo}
\item
  \href{https://help.nytimes3xbfgragh.onion/hc/en-us/articles/115015385887-Contact-Us}{Contact
  Us}
\item
  \href{https://www.nytco.com/careers/}{Work with us}
\item
  \href{https://nytmediakit.com/}{Advertise}
\item
  \href{http://www.tbrandstudio.com/}{T Brand Studio}
\item
  \href{https://www.nytimes3xbfgragh.onion/privacy/cookie-policy\#how-do-i-manage-trackers}{Your
  Ad Choices}
\item
  \href{https://www.nytimes3xbfgragh.onion/privacy}{Privacy}
\item
  \href{https://help.nytimes3xbfgragh.onion/hc/en-us/articles/115014893428-Terms-of-service}{Terms
  of Service}
\item
  \href{https://help.nytimes3xbfgragh.onion/hc/en-us/articles/115014893968-Terms-of-sale}{Terms
  of Sale}
\item
  \href{https://spiderbites.nytimes3xbfgragh.onion}{Site Map}
\item
  \href{https://help.nytimes3xbfgragh.onion/hc/en-us}{Help}
\item
  \href{https://www.nytimes3xbfgragh.onion/subscription?campaignId=37WXW}{Subscriptions}
\end{itemize}
