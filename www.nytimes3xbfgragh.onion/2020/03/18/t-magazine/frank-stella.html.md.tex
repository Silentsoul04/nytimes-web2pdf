Sections

SEARCH

\protect\hyperlink{site-content}{Skip to
content}\protect\hyperlink{site-index}{Skip to site index}

\href{https://myaccount.nytimes3xbfgragh.onion/auth/login?response_type=cookie\&client_id=vi}{}

\href{https://www.nytimes3xbfgragh.onion/section/todayspaper}{Today's
Paper}

The Constellation of Frank Stella

\url{https://nyti.ms/2Wldw6d}

\begin{itemize}
\item
\item
\item
\item
\item
\item
\end{itemize}

Advertisement

\protect\hyperlink{after-top}{Continue reading the main story}

Supported by

\protect\hyperlink{after-sponsor}{Continue reading the main story}

The Great Read

Notes on the Culture

\hypertarget{the-constellation-of-frank-stella}{%
\section{The Constellation of Frank
Stella}\label{the-constellation-of-frank-stella}}

The artist's Minimalist abstractions helped change the direction of
painting at the start of his career. Now at the end of it, the
83-year-old artist looks back to his beginnings.

\includegraphics{https://static01.graylady3jvrrxbe.onion/images/2020/03/18/t-magazine/18tmag-stella-slide-NPH2/18tmag-stella-slide-NPH2-articleLarge.jpg?quality=75\&auto=webp\&disable=upscale}

By \href{https://www.nytimes3xbfgragh.onion/by/megan-o-grady}{Megan
O'Grady}

\begin{itemize}
\item
  March 18, 2020
\item
  \begin{itemize}
  \item
  \item
  \item
  \item
  \item
  \item
  \end{itemize}
\end{itemize}

STARS ---~THE KIND that appear in the cosmos --- have coordinates, not
addresses, and the same is true for certain earthbound luminaries, too.
One gloomy November morning, I follow my GPS to an anonymous set of
buildings in the Hudson Valley. The rain buckets down forebodingly, but
I know I'm on the right track when I make out a set of immense
cast-aluminum and stainless-steel sculptures by the side of the road, a
few of them distinctly stellar in shape. For good measure, the name
``Stella'' is spray-painted on a piece of wood indicating the entrance.

This hangar-like structure, about a 90-minute drive north of Manhattan,
has been
\href{https://www.nytimes3xbfgragh.onion/2019/02/17/arts/design/frank-stella-black-paintings.html}{Frank
Stella}'s studio for the past two decades. The vast space, more easily
traversed by golf cart than on foot, is divided into rooms for both
fabrication and display. Here, I find more star variations: The grandest
has 12 points and is made of glossy black carbon fiber. At over 20 by 20
feet, it's puffily imposing and gently comic. Its neighbors are a pair
of cleverly interlocking wooden stars, one in teak, another in birch,
the humble quality of the carpentry a counterpoint to their complexity
of form, reminiscent of
\href{https://www.nytimes3xbfgragh.onion/2017/11/16/t-magazine/christies-da-vinci-auction-who-was-there.html}{da
Vinci}'s illustrations of the Platonic solids. More futuristic are two
slightly smaller ones made from polished stainless steel; they're what
might have resulted if
\href{https://www.nytimes3xbfgragh.onion/2013/12/17/t-magazine/gallery-paying-tribute-to-buckminster-fuller.html}{Buckminster
Fuller} had created cat toys for giants. When I look closer, I notice
that some of them have built-in bases on their bottommost points that
resemble little shoes: These stars have their feet planted on the
ground.

As does the man himself. Stella, dressed in khakis and a blue fleece
zip-up that has ``Team Stella'' stitched on it in white, is now 83, but
he's retained the scrappy, unpretentious persona he's famous for, as
well as the curly hair and glasses. This is the man who, nearly six
decades ago, gave Minimalism its great tagline by proclaiming: ``What
you see is what you see,'' his words a rallying cry for what art could
be, and, equally, could do without. A fixed light in American art's
galaxy since the 1960s, he has arguably influenced visual representation
as powerfully as
\href{https://www.nytimes3xbfgragh.onion/2018/05/02/t-magazine/andy-warhol-photo-portraits.html}{Andy
Warhol}.

\includegraphics{https://static01.graylady3jvrrxbe.onion/images/2020/03/18/t-magazine/18tmag-stella-slide-D52D/18tmag-stella-slide-D52D-articleLarge.jpg?quality=75\&auto=webp\&disable=upscale}

Unlike many mid-20th-century artists who rose fast only to seemingly
collapse under the pressure of their own reputations, Stella kept
pushing himself by using new forms, materials and technologies. When he
felt he'd reached the limits of the flat canvas, he built out from it in
reliefs inspired by ``Moby-Dick'' and Polish villages. In the 1980s and
'90s, he made metal sculptures that looked like race cars or jet engines
turned inside out, as well as unwieldy canvases covered in Pop-colored
riots of form --- operatic assemblages of cones, pillars and
graffiti-like brushwork, like something
\href{https://www.nytimes3xbfgragh.onion/topic/person/charlie-sheen}{Charlie
Sheen}'s character might have had in his home in the 1987 film ``Wall
Street.'' That the godfather of Minimalist painting turned into a
progenitor of the contemporary baroque has always flummoxed critics.

Perhaps the secret to his longevity, his decade-upon-decade habit of
creating, is again a matter of balanced forces, the measures he's taken
to temper his bright-burning ambition. When we meet, the artist has just
celebrated the arrival of his fifth grandchild, Sophie. (Stella, who has
five children, has been married to Harriet McGurk, a pediatrician, since
1978, and they live in the same house in Greenwich Village he's owned
since the 1960s; his first wife was the art critic
\href{https://www.thecut.com/2019/04/i-was-married-four-times-once-to-a-famous-artist.html}{Barbara
Rose}.) He seems to lack any real self-destructive impulse; he never
succumbed to matters of lifestyle. When I ask him if he has any vices,
he dodges. ``You have to ask my wife,'' he says dryly.

He has (at least) two, it turns out: cigars and fast cars, both of which
have informed his work in various ways, from sculptures based on
three-dimensional representations of his own smoke rings to his use of
technological innovations derived from the auto industry, like
carbon-fiber skin over steel or aluminum frames. In 1982, he was caught
driving his silver Ferrari 105 miles per hour in a 55-mile-per-hour zone
on the Taconic State Parkway, and in lieu of jail time, he delivered
public lectures on his painting. His racing days are now long over, and
he can no longer do much of the physical labor involved in art-making.
And so it might seem he's come full circle, returning to the deceptively
simple geometries he was making six decades ago, only now expanded into
three dimensions.

\includegraphics{https://static01.graylady3jvrrxbe.onion/images/2020/03/18/t-magazine/tmag-frank-stella/tmag-frank-stella-videoSixteenByNine3000.png}

Tentatively scheduled to open in May, a new show at the
\href{http://aldrichart.org/}{Aldrich Contemporary Art Museum}, in
Ridgefield, Conn., ``Frank Stella's Stars, A Survey,'' will focus on
Stella's use of the form at both ends of his career. Many artists have
become fixated on the creation of a particular shape or motif throughout
their lives:
\href{https://www.nytimes3xbfgragh.onion/2019/02/18/t-magazine/jasper-johns.html}{Jasper
Johns} and flags;
\href{https://www.nytimes3xbfgragh.onion/2019/06/11/t-magazine/francoise-gilot-picasso.html}{Pablo
Picasso} and guitars;
\href{https://www.nytimes3xbfgragh.onion/2016/10/03/t-magazine/art/louise-bourgeois-turning-inwards.html}{Louise
Bourgeois} and spiders. The ceaseless exploration of one form helps
create an artist's aesthetic universe, and Stella is part of this
tradition. \emph{Stella} means ``star'' in Italian, but the artist's
interest in the shape --- in geometry, the star polygon is recognized in
both two and three dimensions with varying numbers of points --- is
spatial, not narcissistic. He initially made star drawings in the 1960s
(a set of lithographs from 1967 titled
``\href{https://www.artsy.net/artwork/frank-stella-star-of-persia-i-and-ii}{Star
of Persia I; and II},'' first exhibited at the Aldrich in 1969, will be
included in the show), though the majority of the exhibition will
showcase the more recent sculptural work I observed in his studio, a
study of the potential of the star in different materials, scales and
formal variations, never repeated in the same way. ``Even with something
as stable and as knowable as the star, Stella is able to reinvent it
every time he approaches it and make you look at it in a different
way,'' says Richard Klein, the Aldrich's director of exhibitions.

Star polygons have long been bound up with all sorts of human
metaphysical projection, used as religious symbols and in ranking
systems. As motifs associated with honor and glory and jobs well done,
they decorate everything from national flags and sheriff's badges to
toilet-training charts. But most of all, they symbolize the limits of
human understanding, their geometric representation inseparable from
their existence as celestial objects, luminous spheres of gas held
together by their own gravity. Their lyricism aside, stars are our most
archaic form of navigation as well as our best clues to the dimensions
of the universe. Because light travels at a finite speed, the glow of a
distant star is perceived by our earthbound eyes long after it has
ceased to exist. Similarly finite, perhaps, is the rate of human
understanding: In art history, we're continually revising the past based
on our relative position to it; the importance of an artist or an entire
movement might become visible only in retrospect. So what, one wonders,
is left to say about a man who has been famous now since the 1950s, and
all the more so at a time in which figuration and portraiture have made
comebacks, and when we're all questioning art's relevance in a scary new
decade?

Image

More works in Stella's studio, including ``Hercules and Achelous''
(2017, center).Credit...Photo by Douglas DuBois. Frank Stella,
``Hercules and Achelous,'' 2017, aluminum © 2020 Frank Stella/Artists
Rights Society (ARS), New York

STELLA NEVER WENT to art school, but from an early age, he had a
no-nonsense relationship with a paintbrush: His father, a gynecologist,
paid his way through medical school by painting houses, with Stella as
his young assistant. ``My father would make me sand the floor; we had to
do the sanding and scraping before you could hold the brush and then
paint on the wall. So it was that kind of apprenticeship and
familiarity,'' he says. While repainting the porch of their fishing
cabin in New Hampshire --- Stella grew up in the Boston suburb of Malden
--- his mother, a fashion illustrator and homemaker, decided to make a
\href{https://www.nytimes3xbfgragh.onion/topic/person/jackson-pollock}{Jackson
Pollock} on the floor, dripping the paint in swirls. ``And my father had
to explain to her that maybe it was good in art, but it wasn't going to
work as a floor covering because we didn't have any sealer.''

A story in one of his mother's Vogue magazines, featuring models posed
in front of a painterly
\href{https://www.artsy.net/artist/franz-kline}{Franz Kline}-esque
Abstract Expressionist backdrop, provided him with an early clue that
art wasn't only about figuration. At Phillips Academy in Andover, Mass.,
in the early '50s, when European abstraction was a prevailing force in
studio art, Stella was especially influenced by the work of
\href{https://www.nytimes3xbfgragh.onion/2015/05/14/nyregion/a-connecticut-exhibit-highlights-the-murals-of-hans-hofmann.html}{Hans
Hofmann}, a kind of proto-Abstract Expressionist from the '40s, and the
Bauhaus color theorist
\href{https://www.nytimes3xbfgragh.onion/1976/03/26/archives/josef-albers-artist-and-teacher-dies.html}{Josef
Albers}. ``I had no mimetic ability,'' Stella tells me, ``but I was
never interested in finding one, or cultivating one. No, I worked
directly with the materials, actually. The big deal in postwar American
painting was `its materiality,' and so that was heaven for me.''

Image

Stella's ``Ambergris'' (1993).Credit...Frank Stella, ``Ambergris,''
1993, from the ``Moby-Dick Deckle Edges'' series, lithograph, etching,
aquatint, relief, engraving and screen print on TGL handmade paper,
courtesy of the National Gallery of Art © 2020 Frank Stella/Artists
Rights Society (ARS), New York

He started painting more seriously at Princeton, where he played
lacrosse and wrestled, majored in history and studied art with
\href{https://www.nytimes3xbfgragh.onion/1974/10/28/archives/william-c-seitz-art-scholar-dies-excurator-at-the-modern-taught-at.html}{William
Seitz}, who would become a curator at the
\href{https://www.nytimes3xbfgragh.onion/topic/organization/museum-of-modern-art}{Museum
of Modern Art}, and with the painter
\href{https://www.nytimes3xbfgragh.onion/1999/11/29/arts/stephen-greene-82-painter-with-distinctive-abstract-style.html}{Stephen
Greene}. After graduating in 1958, Stella moved to New York. ``When I
left school,'' he says, ``I wanted to see what it was like to paint all
the time. And at that time, it was between the Korean War and Vietnam,
and we still had selective service. My induction exam was in September,
so I thought, `I'll go to New York {[}in the meantime{]}, get a place,
and I'll just paint and work and do odd jobs, and see what it's like to
do nothing but paint for three or four months.' And then, unfortunately
--- or rather, fortunately --- I failed my induction exams. And when I
called up my father, I said, `I'm sorry, I have to go back to New York,
I failed the exam.' He said, `Too bad, it would have made a man of you.'
The most important thing for them was that I shouldn't be a burden on
society.'' He pauses. ``And we know what they meant by `society.'''

Stella was only 23 when his work was included in a group show, ``Sixteen
Americans,'' at MoMA in 1959. His ``Black Paintings'' --- bands of matte
enamel (he used house painter's brushes and house paint) separated by
pinstripes of exposed canvas --- startled critics for their extremity of
reduction, their intentionally flat affect, their refusal to appease.
Cool, clever, and somehow less angstily reverential in feel than the
Abstract Expressionist era that it helped supplant, Stella's work is now
widely seen as a crucial evolutionary link in modern art, and a catalyst
for the Minimalist movement to come. His emphasis on two-dimensional
surfaces was a clear rejection of the idea of painting as a window into
a three-dimensional space.

His participation in the MoMA show, alongside Jasper Johns,
\href{https://www.nytimes3xbfgragh.onion/2015/06/03/t-magazine/robert-rauschenberg-endless-combinations.html}{Robert
Rauschenberg},
\href{https://www.nytimes3xbfgragh.onion/2018/02/08/t-magazine/ellsworth-kelly-austin-last-work.html}{Ellsworth
Kelly} and
\href{https://www.nytimes3xbfgragh.onion/1988/04/18/obituaries/louise-nevelson-artist-renowned-for-wall-sculptures-is-dead-at-88.html}{Louise
Nevelson}, launched his career --- four of his paintings were included
in the exhibition --- but his first gallery show, with New York's
\href{https://www.castelligallery.com/}{Leo Castelli} a year later,
resulted in few sales. Stella eked out a living painting houses, renting
cold-water flats and sharing studio space with
\href{https://www.nytimes3xbfgragh.onion/2011/07/17/arts/design/carl-andres-work-is-the-subject-of-a-new-book-and-show.html}{Carl
Andre} and
\href{https://www.nytimes3xbfgragh.onion/1984/04/03/obituaries/hollis-frampton-dead-at-48-film-maker-photographer.html}{Hollis
Frampton}, his friends from Phillips Academy, but listening to him, it's
impossible not to feel nostalgia for a time in which you could arrive in
Manhattan, these days largely a gated community for the wealthy, and
simply go about making your art.

Image

Stella's ``Firuzabad'' (1970).Credit...Frank Stella, ``Firuzabad,''
1970, synthetic polymer paint on canvas, digital image © The Museum of
Modern Art/Licensed by SCALA/Art Resource, N.Y. © 2020 Frank
Stella/Artists Rights Society (ARS), New York

``THERE'S AN ELEMENT of luck and things like that to it, but the fact of
the matter is that the system was pretty supportive,'' says Stella when
I remark on how he seemed to be exactly the right artist at exactly the
right time. In New York, he was granted a sense of license to do
whatever he wanted with paint, inspired by the artists he revered, among
them
\href{https://www.nytimes3xbfgragh.onion/topic/person/willem-de-kooning}{Willem
de Kooning}, Barnett Newman and Pollock. Stella found his own canvases
growing larger --- large enough to have to be placed on the floor.
``They were no longer easel paintings,'' he says. ``Basically, I was
standing up in front of a painting that was a little bit bigger than I
was, and that was the working on it, like the way you would paint a wall
in a house. And that was the kind of thing that I felt comfortable
with.'' He singles out the abstract painter
\href{https://www.nytimes3xbfgragh.onion/topic/person/helen-frankenthaler}{Helen
Frankenthaler}, who studied under Hofmann, as the artist he believes was
one of the most undervalued in her lifetime. ``They were always
interesting, always good, and very, very difficult paintings she made,
and she was lucky if she could sell any of them,'' he recalls. Early in
his career, she
\href{https://www.nytimes3xbfgragh.onion/2019/10/08/t-magazine/artist-trades.html}{proposed
a trade} with him, but he was too intimidated to take her up on it.

When I ask him if he's in touch with anyone from that time, he shakes
his head. ``No. The problem now is everybody's dead or dying. I'm in the
category of `Is he still alive?' artists. Yeah, you laugh, but I can
show you a letter --- a guy was asking if I was still alive because he
liked my work so much.''

By the end of the 1960s, Stella had lost interest in flat surfaces. He
started making constructions of felt, paper and wood that protruded from
the surface of a stretched canvas in a relief. He named these works,
like 1971's ``Chodorow II,'' after synagogues destroyed by the Nazis. In
a way, the work could be seen as a kind of inverse of the type of
painting that had dominated Western art since the Renaissance, which
drew viewers into the canvas. ``The idea behind it all was to build a
painting rather than paint a painting,'' says Stella. ``If I built it
first, it was all mine, and then I could paint on that --- and that's
all.'' The simple story would be that the Minimalist turned Maximalist
when the former wore out its usefulness.

Image

Stella's ``Fez (2)'' (1964).Credit...Frank Stella, ``Fez (2),'' 1964,
fluorescent alkyd on canvas, gift of Lita Hornick, digital image © The
Museum of Modern Art/Licensed by SCALA/Art Resource, N.Y. © 2020 Frank
Stella/Artists Rights Society (ARS), New York

On many occasions, material experimentation offered a pathway forward:
``That's a kind of necessity, because you get bogged down, you get
worried. You're always looking for something, as they say, a way out of
the darkness. And it's inevitable that you look to things. You look to
what other people are doing, and you look to what's available, and you
can't help looking for things. Mostly you look within the art world, but
that seems like a limited vision, so you have to look outside. You have
to get with the real world eventually.''

In at least one such moment, Stella found himself compelled to look back
in order to move forward. He used his 1982-83 residency at the American
Academy in Rome to delve into the legacy of
\href{https://www.nytimes3xbfgragh.onion/topic/person/caravaggio}{Caravaggio}
and Rubens. That research led eventually to ``Working Space,'' his 1986
book derived from a series of lectures that he delivered at Harvard in
the early '80s, in which he framed his new work as an answer to a crisis
in abstract painting. Stella's ``Moby-Dick'' series, which he began that
year and continued until 1997, considered abstraction's ability to
illustrate narratives, with silhouettes alluding to waves and ships. The
'90s and early aughts were critically tough for Stella's hectic forms,
and yet many works from this time --- his mural-size
``Moby-Dick''-inspired 1992 print, ``The Fountain,'' for example, or his
underrated work in rugged painted metal, especially 2004's ``Ngebat,'' a
twisted construction of stainless steel and carbon fiber --- now seem
freshly exhilarating. You could argue that every artist working in
Europe and America today has, in some fashion, been unconsciously
influenced by Stella, and there are those who more explicitly credit him
as an influence, such as the assemblage artist
\href{https://www.marianneboeskygallery.com/artists/jessica-jackson-hutchins/bio}{Jessica
Jackson Hutchins} and the abstract painter
\href{https://www.petzel.com/artists/sarah-morris}{Sarah Morris}.

Before I leave, Stella takes me on a tour of recent work, leading me
behind the curtain hanging in the back of the studio. ``So, now you're
going into the space where no women are allowed,'' he jokes. And lo,
there I behold Stella's industrial sander, his spray-painter, as well as
a glimpse of new work being fabricated for a private collector.

Image

Stella's ``Grajau I'' (1975).Credit...Frank Stella, ``Grajau I,'' 1975,
mixed media, aluminum, courtesy of The Glass House, A Site of the
National Trust for Historic Preservation, photo by Andy Romer © 2020
Frank Stella/Artists Rights Society (ARS), New York

If entropy is the natural direction of all things --- the laws of
physics, anyway, as well as contemporary art --- some things in our
universe do, in fact, remain constant: Stella's star, at least, built on
the principles of space, light, speed and seemingly infinite expansion,
is unlikely to dim from art history anytime soon. ``Basically,
everything is about being an artist,'' he says as we part ways. He pulls
out a cigar as I thank him and gather my coat and umbrella. ``You're
welcome,'' he smiles. ``And don't say anything about the smoking.''

It's an open question just how well Stella's ethos has fared over time.
Once so thrillingly radical, Minimalist painting has inevitably lost
some of its charge over the years; at a time in which art is often
wrapped up in social and political questions, shunning pictorial
representation and symbolic meaning for the essentials of color, shape
and composition can feel oddly safe, something everyone can get behind:
colorful geometries that could be printed on an Ikea duvet. And yet the
sheer scale and panache of Stella's early work are undeniable. At the
Art Institute of Chicago's Modern Wing, I often observe tourists
stopping dead in their tracks in front of ``Hatra I,'' one of the first
``Protractor'' paintings Stella made beginning in 1967, which consist of
sweeping, intersecting arcs, the shape of the canvas echoing that of the
paint. Glowing with bright acrylic and measuring 20 by 10 feet, it still
imparts a contact high. Sitting in Stella's presence and revisiting his
work with him, I think what a misunderstanding it is to consider
Minimalism as soulless or academic, a mere visual palate cleanser. On
the contrary, it seeks feelings less easily named, an almost somatic
response, a full-body awareness. What you see is what you see, but what
you feel has always been important, too.

Advertisement

\protect\hyperlink{after-bottom}{Continue reading the main story}

\hypertarget{site-index}{%
\subsection{Site Index}\label{site-index}}

\hypertarget{site-information-navigation}{%
\subsection{Site Information
Navigation}\label{site-information-navigation}}

\begin{itemize}
\tightlist
\item
  \href{https://help.nytimes3xbfgragh.onion/hc/en-us/articles/115014792127-Copyright-notice}{©~2020~The
  New York Times Company}
\end{itemize}

\begin{itemize}
\tightlist
\item
  \href{https://www.nytco.com/}{NYTCo}
\item
  \href{https://help.nytimes3xbfgragh.onion/hc/en-us/articles/115015385887-Contact-Us}{Contact
  Us}
\item
  \href{https://www.nytco.com/careers/}{Work with us}
\item
  \href{https://nytmediakit.com/}{Advertise}
\item
  \href{http://www.tbrandstudio.com/}{T Brand Studio}
\item
  \href{https://www.nytimes3xbfgragh.onion/privacy/cookie-policy\#how-do-i-manage-trackers}{Your
  Ad Choices}
\item
  \href{https://www.nytimes3xbfgragh.onion/privacy}{Privacy}
\item
  \href{https://help.nytimes3xbfgragh.onion/hc/en-us/articles/115014893428-Terms-of-service}{Terms
  of Service}
\item
  \href{https://help.nytimes3xbfgragh.onion/hc/en-us/articles/115014893968-Terms-of-sale}{Terms
  of Sale}
\item
  \href{https://spiderbites.nytimes3xbfgragh.onion}{Site Map}
\item
  \href{https://help.nytimes3xbfgragh.onion/hc/en-us}{Help}
\item
  \href{https://www.nytimes3xbfgragh.onion/subscription?campaignId=37WXW}{Subscriptions}
\end{itemize}
