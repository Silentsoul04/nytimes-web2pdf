Sections

SEARCH

\protect\hyperlink{site-content}{Skip to
content}\protect\hyperlink{site-index}{Skip to site index}

\href{https://www.nytimes3xbfgragh.onion/section/politics}{Politics}

\href{https://myaccount.nytimes3xbfgragh.onion/auth/login?response_type=cookie\&client_id=vi}{}

\href{https://www.nytimes3xbfgragh.onion/section/todayspaper}{Today's
Paper}

\href{/section/politics}{Politics}\textbar{}Andrew Yang's
\$1,000-a-Month Idea May Have Seemed Absurd Before. Not Now.

\url{https://nyti.ms/2x6UuWJ}

\begin{itemize}
\item
\item
\item
\item
\item
\end{itemize}

\begin{itemize}
\item
  \href{https://www.nytimes3xbfgragh.onion/interactive/2020/09/08/us/elections/results-new-hampshire-primary-elections.html?action=click\&pgtype=Article\&state=default\&region=TOP_BANNER\&context=storylines_menu}{New
  Hampshire Results}
\item
  \href{https://www.nytimes3xbfgragh.onion/live/2020/09/09/us/trump-vs-biden?action=click\&pgtype=Article\&state=default\&region=TOP_BANNER\&context=storylines_menu}{Election
  Updates}
\item
  \href{https://www.nytimes3xbfgragh.onion/interactive/2020/us/elections/election-states-biden-trump.html?action=click\&pgtype=Article\&state=default\&region=TOP_BANNER\&context=storylines_menu}{Paths
  to 270}
\item
  \href{https://www.nytimes3xbfgragh.onion/interactive/2020/08/31/us/politics/vote-by-mail-deadlines.html?action=click\&pgtype=Article\&state=default\&region=TOP_BANNER\&context=storylines_menu}{Voting
  by Mail}
\item
  \href{https://www.nytimes3xbfgragh.onion/interactive/2019/us/elections/2020-presidential-election-calendar.html?action=click\&pgtype=Article\&state=default\&region=TOP_BANNER\&context=storylines_menu}{Key
  Dates}
\item
  \href{https://www.nytimes3xbfgragh.onion/newsletters/politics?action=click\&pgtype=Article\&state=default\&region=TOP_BANNER\&context=storylines_menu}{Politics
  Newsletter}
\end{itemize}

Advertisement

\protect\hyperlink{after-top}{Continue reading the main story}

Supported by

\protect\hyperlink{after-sponsor}{Continue reading the main story}

\hypertarget{andrew-yangs-1000-a-month-idea-may-have-seemed-absurd-before-not-now}{%
\section{Andrew Yang's \$1,000-a-Month Idea May Have Seemed Absurd
Before. Not
Now.}\label{andrew-yangs-1000-a-month-idea-may-have-seemed-absurd-before-not-now}}

Mr. Yang championed putting money in the pockets of every American adult
during his presidential campaign. Now, amid the coronavirus crisis,
``This thing is going to pass,'' he says.

\includegraphics{https://static01.graylady3jvrrxbe.onion/images/2020/03/18/us/politics/18yang-ubi/merlin_154851534_1d56cc6d-8e26-42c8-a3df-404b751701a7-articleLarge.jpg?quality=75\&auto=webp\&disable=upscale}

\href{https://www.nytimes3xbfgragh.onion/by/matt-stevens}{\includegraphics{https://static01.graylady3jvrrxbe.onion/images/2019/04/03/multimedia/author-matt-stevens/author-matt-stevens-thumbLarge.png}}\href{https://www.nytimes3xbfgragh.onion/by/isabella-grullon-paz}{\includegraphics{https://static01.graylady3jvrrxbe.onion/images/2020/02/21/reader-center/author-isabella-grullon-paz/author-isabella-grullon-paz-thumbLarge.png}}

By \href{https://www.nytimes3xbfgragh.onion/by/matt-stevens}{Matt
Stevens} and
\href{https://www.nytimes3xbfgragh.onion/by/isabella-grullon-paz}{Isabella
Grullón Paz}

\begin{itemize}
\item
  March 18, 2020
\item
  \begin{itemize}
  \item
  \item
  \item
  \item
  \item
  \end{itemize}
\end{itemize}

For more than two years,
\href{https://www.nytimes3xbfgragh.onion/2020/04/29/nyregion/andrew-yang-mayor-nyc.html}{Andrew
Yang} traveled the country as a presidential candidate trying to
convince voters that a crisis was coming. The economy was going to
evolve, he warned, jobs would be automated away in droves and many
Americans were going to find themselves at home without a paycheck.

``And now,'' Mr. Yang observed from inside his family's weekend home in
upstate New York, ``we've all been sent home at once.''

To be sure, fear of an impending global pandemic resulting from a novel
coronavirus was not the reason Mr. Yang spent months insisting the
federal government provide American adults with a universal basic income
of \$1,000 per month. But a global pandemic has arrived. And the fallout
from the outbreak has plunged the country into a grim and uncertain
reality.

So only now --- with millions of Americans facing the prospect of no
work and wondering how they will pay the bills --- have proposals
similar to Mr. Yang's signature policy prescription gained wide,
bipartisan approval. In the sort of political turnabout that may only be
possible when society faces dire need, giving free money to Americans
suddenly appears not only rational but critically necessary to many
Democrats and key Republicans.

In a telephone interview Tuesday night, Mr. Yang expressed confidence
that legislation would soon put some form of basic income or stimulus
checks into Americans' bank accounts. ``This thing is going to pass,''
he said. ``And it's going to pass for a very obvious reason: Money in
our hands is vital to prevent our economy from collapsing.''

Over the past week, proposals to dole out direct payments to workers and
their families while schools and businesses shut down have come from
progressives like Senator Bernie Sanders of Vermont, moderate Democrats
like Senator Michael Bennet of Colorado and Republicans like Senator
Mitt Romney of Utah.

Those calls for relief were
\href{https://www.nytimes3xbfgragh.onion/2020/03/17/us/politics/stimulus-package.html?action=click\&module=Spotlight\&pgtype=Homepage}{bolstered
by the White House}, which got behind the idea of rapidly sending money
to Americans. An outline of a Treasury Department proposal
\href{https://www.nytimes3xbfgragh.onion/2020/03/18/us/politics/donald-trump-coronavirus-trump-stimulus.html}{obtained
Wednesday by The New York Times} called for sending two rounds of checks
directly to taxpayers on April 6 and May 18. Payments would be fixed and
their sizes dependent on income and family size, according to the
outline, which put the cost of each payment at \$250 billion, part of a
\$1 trillion economic stabilization package.

\href{https://www.c-span.org/video/?470459-1/president-trump-invokes-defense-production-act-response-coronavirus\&start=1536}{Speaking
to reporters at a White House briefing on Wednesday}, President Trump
said the exact amount of the checks was still ``to be determined,''
while lauding the talks surrounding the payments as ``very bipartisan.''

Mr. Yang said he would recommend payments of \$1,000 per adult and \$500
per child in the near term with the hope that they could continue in
perpetuity.

``My big concern is that we should make it consistent, and once a month
so that if this crisis continues, people don't see their savings
evaporate,'' he said. ``If you were to have a lump sum in March, that
money's not going to last until May or June.''

``I hope this does become fixed policy,'' he added. ``But we're in a
crisis right now and the important thing is to just get money into
Americans' hands.''

Mr. Yang said he had harbored doubts about whether the federal
government would consider giving money directly to citizens. But he
repeatedly expressed ``thrill'' that ideas similar to his own were being
discussed at the highest level of government.

``No one would wish this as a circumstance we'd ever face as a
country,'' he said. ``But I do feel some degree of pride in that I
believe that my campaign --- with the help of hundreds of thousands of
supporters around the country --- helped advance a set of solutions that
it turns out the country needed in a time of crisis.''

Throughout his presidential campaign, Mr. Yang called for a monthly
basic income that would be provided to American adults from the time
they turned 18 until their death. Though several lawmakers have
suggested somewhat similar measures to address the coronavirus crisis,
many of the plans call for sending a limited number of checks or
providing other mechanisms for relief, some of which have been deployed
during past downturns.

\href{https://www.nytimes3xbfgragh.onion/news-event/2020-election}{Election
2020 ›}

\hypertarget{live-updates}{%
\subsection{\texorpdfstring{\href{https://www.nytimes3xbfgragh.onion/live/2020/09/09/us/trump-vs-biden}{Live
Updates}}{Live Updates}}\label{live-updates}}

\href{https://www.nytimes3xbfgragh.onion/live/2020/09/09/us/trump-vs-biden\#democrats-worry-about-a-partisan-slant-at-the-postal-service-where-trump-allies-dominate-the-board}{}

Sept. 9, 2020, 8:51 a.m. ET

\href{https://www.nytimes3xbfgragh.onion/live/2020/09/09/us/trump-vs-biden\#democrats-worry-about-a-partisan-slant-at-the-postal-service-where-trump-allies-dominate-the-board}{Democrats
worry about a partisan slant at the Postal Service, where Trump allies
dominate the
board.}\href{https://www.nytimes3xbfgragh.onion/live/2020/09/09/us/trump-vs-biden\#the-nras-new-candidate-grades-show-a-continuing-decline-in-support}{}

Sept. 9, 2020, 8:19 a.m. ET

\href{https://www.nytimes3xbfgragh.onion/live/2020/09/09/us/trump-vs-biden\#the-nras-new-candidate-grades-show-a-continuing-decline-in-support}{The
N.R.A.'s new candidate grades show a continuing decline in
support.}\href{https://www.nytimes3xbfgragh.onion/live/2020/09/09/us/trump-vs-biden\#biden-announces-a-plan-to-push-companies-to-keep-jobs-in-the-us}{}

Sept. 9, 2020, 8:15 a.m. ET

\href{https://www.nytimes3xbfgragh.onion/live/2020/09/09/us/trump-vs-biden\#biden-announces-a-plan-to-push-companies-to-keep-jobs-in-the-us}{Biden
announces a plan to push companies to keep jobs in the U.S.}

In other words, one person's temporary universal basic income is another
person's stimulus check or tax rebate.

Mr. Romney, for instance, has suggested
\href{https://www.romney.senate.gov/romney-calls-urgent-action-additional-coronavirus-response-measures}{a
one-time \$1,000 check} --- not universal basic income --- as a starting
point; Representative Tulsi Gabbard of Hawaii has
\href{https://gabbard.house.gov/news/press-releases/rep-tulsi-gabbard-introduces-resolution-calling-emergency-universal-basic}{proposed}
\$1,000 payments until the emergency has ended; and Representatives Tim
Ryan of Ohio and Ro Khanna of California have
\href{https://timryan.house.gov/media/press-releases/congressmen-tim-ryan-ro-khanna-propose-cash-infusion-working-americans-during}{proposed
an emergency earned-income tax credit} that they said would provide a
check to Americans who earned less than \$65,000 last year.

Mr. Sanders on Tuesday came out in favor of
``\href{https://berniesanders.com/issues/emergency-response-coronavirus-pandemic/}{\$2,000
cash payments} to every person in America every month for the duration
of the crisis.'' And a group of Democratic senators --- led by Mr.
Bennet, Cory Booker of New Jersey and Sherrod Brown of Ohio ---
\href{https://www.bennet.senate.gov/public/_cache/files/9/2/925b76ae-271c-4e77-be5f-eb0d7f5c5ac6/B772110F661116948B00571DD63BB0E7.cash-payments-letter.pdf}{proposed
sending as much as \$4,500 to nearly every adult and child} in
installments this year.

In an interview, Mr. Khanna, a co-chair of Mr. Sanders's presidential
campaign, said he would be willing to work with those who wanted to send
checks quickly, but was concerned that under some plans, financial help
would not be targeted to those who need it most.

``This can't just be a one-time thing,'' he said. ``It's got to be more
sustained, and when you're sustaining it, it would become far better to
give more dramatic relief to the working families and those who are
unemployed.''

Indeed, the
\href{https://www.nytimes3xbfgragh.onion/2020/03/15/business/economy/coronavirus-economy-impact.html}{economic
ripple effect} of the virus is already being felt most acutely by
service workers and hourly workers whose jobs have evaporated as
Americans have been urged to practice ``social distancing.'' With so
many Americans forced to stay home, some experts said that sending them
money was the best way to stimulate the economy.

``We're in a situation now where we're almost surely going to have a
recession. And unlike most recessions, we don't want people going to
work --- we want a lot of people to stay at home and not be interacting
socially,'' said N. Gregory Mankiw, a professor of economics at Harvard
and a former adviser to President George W. Bush. Direct payments, he
said, would allow people to avoid putting their health at risk for a
paycheck.

But Jason Furman, a Harvard economics professor who advised President
Barack Obama, noted that while universal basic income had ``a lot of
attractive features,'' it would be expensive to administer in the long
run.

``Cash is very flexible and lets people make the choices that are best
for them,'' Professor Furman said. ``But as a large-scale permanent
program it would be very difficult to finance.''

Though Mr. Yang favors making payments permanent, he stressed the need
to ``take care of our people and then figure out what we're going to
do'' after the crisis recedes.

``I'm thrilled that common sense is prevailing,'' he said, before
acknowledging that his idea may have been slightly ahead of its time.

He
\href{https://www.nytimes3xbfgragh.onion/2020/02/11/us/politics/andrew-yang-drops-out.html}{suspended
his campaign in February}, he noted, ``and then we adopt universal basic
income in March.''

Reporting was contributed by Emily Cochrane, Nicholas Fandos, Alan
Rappeport and Jim Tankersley.

\hypertarget{our-2020-election-guide}{%
\section{Our 2020 Election Guide}\label{our-2020-election-guide}}

Updated ~Sept. 9, 2020

\begin{center}\rule{0.5\linewidth}{\linethickness}\end{center}

\begin{itemize}
\item ~
  \hypertarget{the-latest}{%
  \subsection{The Latest}\label{the-latest}}

  \begin{itemize}
  \item
    Joe Biden heads today to Michigan, a battleground state where
    President Trump has resumed advertising ahead of a visit there on
    Thursday.
    \href{https://www.nytimes3xbfgragh.onion/live/2020/09/09/us/trump-vs-biden?action=click\&pgtype=Article\&state=default\&region=BELOW_MAIN_CONTENT\&context=storylines_guide}{Read
    live updates}.
  \end{itemize}
\item ~
  \hypertarget{how-to-win-270}{%
  \subsection{How to Win 270}\label{how-to-win-270}}

  \begin{itemize}
  \item
    Joe Biden and Donald Trump need 270 electoral votes to reach the
    White House. Try building
    \href{https://www.nytimes3xbfgragh.onion/interactive/2020/us/elections/election-states-biden-trump.html?action=click\&pgtype=Article\&state=default\&region=BELOW_MAIN_CONTENT\&context=storylines_guide}{your
    own coalition of battleground states}~to see potential outcomes.
  \end{itemize}
\item ~
  \hypertarget{voting-by-mail}{%
  \subsection{Voting by Mail}\label{voting-by-mail}}

  \begin{itemize}
  \item
    Will you have enough time to vote by mail in your state? Yes, but
    it's risky to procrastinate.
    \href{https://www.nytimes3xbfgragh.onion/interactive/2020/08/31/us/politics/vote-by-mail-deadlines.html?action=click\&pgtype=Article\&state=default\&region=BELOW_MAIN_CONTENT\&context=storylines_guide}{Check
    your state's deadline.}
  \item
    \href{https://www.nytimes3xbfgragh.onion/interactive/2020/us/elections/joe-biden.html?action=click\&pgtype=Article\&state=default\&region=BELOW_MAIN_CONTENT\&context=storylines_guide}{}

    \hypertarget{joe-biden}{%
    \section{Joe Biden}\label{joe-biden}}

    \hypertarget{democrat}{%
    \subsection{Democrat}\label{democrat}}

    \href{https://www.nytimes3xbfgragh.onion/interactive/2020/us/elections/donald-trump.html?action=click\&pgtype=Article\&state=default\&region=BELOW_MAIN_CONTENT\&context=storylines_guide}{}

    \hypertarget{donald-trump}{%
    \section{Donald Trump}\label{donald-trump}}

    \hypertarget{republican}{%
    \subsection{Republican}\label{republican}}
  \end{itemize}
\item
  \hypertarget{keep-up-with-our-coverage}{%
  \subsection{Keep Up With Our
  Coverage}\label{keep-up-with-our-coverage}}

  \begin{itemize}
  \item
    Get an
    \href{https://www.nytimes3xbfgragh.onion/newsletters/politics?action=click\&pgtype=Article\&state=default\&region=BELOW_MAIN_CONTENT\&context=storylines_guide}{email}~recapping
    the day's news
  \item
    Download our mobile app on
    \href{https://apps.apple.com/us/app/nytimes/id284862083?ls=1\&mat_click_id=5c79ae7455014fd1bd66b5610c05b8f2-20191112-16948\&referrer=mat_click_id\%3D5c79ae7455014fd1bd66b5610c05b8f2-20191112-16948\%26link_click_id\%3D722930677036718082}{iOS}~and
    \href{http://a.localytics.com/android?id=com.nytimes.android\&referrer=utm_source\%3Dother_nyt_mobile_web\%26utm_medium\%3DWeb\%2520page\%26utm_term\%3DGeneral\%2520Mobile\%2520Page\%26utm_campaign\%3DNYT\%2520Mobile\%2520General\%2520Page}{Android}~and
    turn on Breaking News and Politics alerts
  \end{itemize}
\end{itemize}

Advertisement

\protect\hyperlink{after-bottom}{Continue reading the main story}

\hypertarget{site-index}{%
\subsection{Site Index}\label{site-index}}

\hypertarget{site-information-navigation}{%
\subsection{Site Information
Navigation}\label{site-information-navigation}}

\begin{itemize}
\tightlist
\item
  \href{https://help.nytimes3xbfgragh.onion/hc/en-us/articles/115014792127-Copyright-notice}{©~2020~The
  New York Times Company}
\end{itemize}

\begin{itemize}
\tightlist
\item
  \href{https://www.nytco.com/}{NYTCo}
\item
  \href{https://help.nytimes3xbfgragh.onion/hc/en-us/articles/115015385887-Contact-Us}{Contact
  Us}
\item
  \href{https://www.nytco.com/careers/}{Work with us}
\item
  \href{https://nytmediakit.com/}{Advertise}
\item
  \href{http://www.tbrandstudio.com/}{T Brand Studio}
\item
  \href{https://www.nytimes3xbfgragh.onion/privacy/cookie-policy\#how-do-i-manage-trackers}{Your
  Ad Choices}
\item
  \href{https://www.nytimes3xbfgragh.onion/privacy}{Privacy}
\item
  \href{https://help.nytimes3xbfgragh.onion/hc/en-us/articles/115014893428-Terms-of-service}{Terms
  of Service}
\item
  \href{https://help.nytimes3xbfgragh.onion/hc/en-us/articles/115014893968-Terms-of-sale}{Terms
  of Sale}
\item
  \href{https://spiderbites.nytimes3xbfgragh.onion}{Site Map}
\item
  \href{https://help.nytimes3xbfgragh.onion/hc/en-us}{Help}
\item
  \href{https://www.nytimes3xbfgragh.onion/subscription?campaignId=37WXW}{Subscriptions}
\end{itemize}
