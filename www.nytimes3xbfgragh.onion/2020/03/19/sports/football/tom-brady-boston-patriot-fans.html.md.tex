Sections

SEARCH

\protect\hyperlink{site-content}{Skip to
content}\protect\hyperlink{site-index}{Skip to site index}

\href{https://www.nytimes3xbfgragh.onion/section/sports/football}{Pro
Football}

\href{https://myaccount.nytimes3xbfgragh.onion/auth/login?response_type=cookie\&client_id=vi}{}

\href{https://www.nytimes3xbfgragh.onion/section/todayspaper}{Today's
Paper}

\href{/section/sports/football}{Pro Football}\textbar{}Boston Offers
Bouquets to Tom Brady and Disdain to the Patriots

\url{https://nyti.ms/33xBjl6}

\begin{itemize}
\item
\item
\item
\item
\item
\item
\end{itemize}

Advertisement

\protect\hyperlink{after-top}{Continue reading the main story}

Supported by

\protect\hyperlink{after-sponsor}{Continue reading the main story}

\hypertarget{boston-offers-bouquets-to-tom-brady-and-disdain-to-the-patriots}{%
\section{Boston Offers Bouquets to Tom Brady and Disdain to the
Patriots}\label{boston-offers-bouquets-to-tom-brady-and-disdain-to-the-patriots}}

Brady's time as a Patriot defined an era of sports success in New
England. His departure on Tuesday anguished fans already besieged by bad
news.

\includegraphics{https://static01.graylady3jvrrxbe.onion/images/2020/03/20/sports/19bradyboston-print/merlin_141891342_42284dc6-cc95-4e9e-950e-1e3b63f339d4-articleLarge.jpg?quality=75\&auto=webp\&disable=upscale}

\href{https://www.nytimes3xbfgragh.onion/by/david-waldstein}{\includegraphics{https://static01.graylady3jvrrxbe.onion/images/2018/02/20/multimedia/author-david-waldstein/author-david-waldstein-thumbLarge.jpg}}

By \href{https://www.nytimes3xbfgragh.onion/by/david-waldstein}{David
Waldstein}

\begin{itemize}
\item
  March 19, 2020
\item
  \begin{itemize}
  \item
  \item
  \item
  \item
  \item
  \item
  \end{itemize}
\end{itemize}

Rick Field, the founder of Rick's Picks pickles and a lifelong
\href{https://www.nytimes3xbfgragh.onion/2020/04/22/sports/football/patriots-nfl-draft-bill-belichick.html}{New
England Patriots} fan, vividly remembers the moment that changed his
favorite team's destiny.

On Sept. 23, 2001, he gathered with friends in front of a TV to watch a
Patriots home game against the Jets.

Drew Bledsoe, the Patriots' beloved quarterback at the time,
was\href{http://www.nfl.com/videos/nfl-films-presents/09000d5d8227f09b/Bledsoe-to-Brady-The-hit-that-changed-history}{knocked
out of the game} by an injury and when a relatively unknown backup
jogged out to replace Bledsoe, Field joked to his worried friends, ``Let
the Tom Brady era begin.''

The remark was brushed off as another example of Field's eccentric
humor. After all, few New England fans even knew who Tom Brady was at
the time. But never were more prophetic words uttered in the world of
sports.

\href{https://www.nytimes3xbfgragh.onion/2020/03/17/sports/football/tom-brady-2007-season.html}{The
Tom Brady era}was far more successful, and lasted far longer, than Field
could ever have imagined: six Super Bowl titles for a franchise that had
never won one before, nine A.F.C. championships, 17 A.F.C. East crowns
and three Most Valuable Player Awards.

``The Brady era was nothing more than the universe realigning after 40
years of some of the worst football ever played by the New England
Patriots,'' Field said on Wednesday. ``So we're back to even now, as far
as I'm concerned.''

The era ended unceremoniously with a combination of sorrow and anger on
Tuesday morning when Brady, 42, announced through social media that he
was
\href{https://www.nytimes3xbfgragh.onion/2020/03/17/sports/football/tom-brady-patriots.html}{leaving
the Patriots}after 20 incomparable years.

\includegraphics{https://static01.graylady3jvrrxbe.onion/images/2020/03/19/sports/19brady-boston/19brady-boston-articleLarge.jpg?quality=75\&auto=webp\&disable=upscale}

It was later reported that he would sign as a free agent
\href{https://www.nytimes3xbfgragh.onion/2020/03/18/sports/football/tom-brady-tampa-bay-buccaneers.html}{with
the Tampa Bay Buccaneers}. **** One of the great sports treasures in
Boston history was leaving for a backwater outpost. But before that,
Brady saluted New England
\href{https://twitter.com/TomBrady/status/1239895697207549952}{on his
Twitter account}.

``My children were born and raised here,'' he wrote, ``and you always
embraced this California kid as your own.''

The region was already grappling with the loss of Mookie Betts, the Red
Sox outfielder who was
\href{https://www.nytimes3xbfgragh.onion/2020/02/05/sports/baseball/mookie-betts-trade-red-sox.html}{traded
to the Los Angeles Dodgers} last month. Betts, 27, was not only the best
player on the Red Sox, but arguably the most popular as well. Last week,
race organizers
\href{https://www.nytimes3xbfgragh.onion/2020/03/13/sports/boston-marathon-london-marathon-postponed-coronavirus.html}{postponed
the Boston Marathon}, an annual rite of spring in the city, until
September because of mounting concerns about the coronavirus pandemic.

Then came the announcement from a cherished athlete whose success in the
area is surpassed only by Bill Russell of the Celtics.

``It's really different and far more significant than Mookie leaving,''
said Marc Bertrand, the co-host of a popular Boston radio show **** on
\href{https://985thesportshub.com/}{98.5 The Sports Hub}. **** ``Brady
leaving is about the emotional attachment for everything that has
happened. Mookie leaving was about everything that could happen in the
future.''

All week Bertrand, who grew up in Quincy, Mass., just outside Boston,
served --- along with his co-host,
\href{https://www.pro-football-reference.com/players/Z/ZolaSc00.htm}{Scott
Zolak}, a former Patriots quarterback --- as a kind of therapeutic
sounding board for masses of distraught Patriots and Brady fans.

He estimated that over the last two days, about 75 percent of callers to
his show sided with Brady over the Patriots. One of them said it would
take two years for him to rediscover his devotion to the team.

Another angry caller to The Sports Hub said she hoped the Patriots go
winless next year, and before long someone had placed flowers at the
front door of Brady's TB12 Performance \& Recovery Center on Boylston
Street in Boston. The man told the city's Channel 25 that he lives in
the area and was ``pretty torn up,'' about Brady leaving and ``wanted to
pay my respects.''

Image

In the span of a little over a month, two of Boston's biggest sports
heroes --- Brady and Red Sox outfielder Mookie Betts --- departed the
city.Credit...Adam Glanzman/Getty Images

Normally March 17 is a special day of celebration for Boston, its Irish
history highlighted by the annual St. Patrick's Day parade and
surrounding festivities. But the parade was canceled and bars and
restaurants were closed because of the coronavirus outbreak, leaving the
Brady talk to dominate the airwaves and social media.

``It was a total downer,'' Bertrand said. ``But in a strange way, it
took people's minds off of everything else that is going on. A lot of
people forgot that yesterday was even St. Patrick's Day, and obviously
it's a huge holiday here.''

Like cities across the globe, all of Boston's promising sports teams
have been shuttered because of the outbreak. The Bruins had the most
wins in the National Hockey League, and fans had been optimistic about a
second consecutive trip to the Stanley Cup finals. The Boston Celtics
had the third-best record in the N.B.A.'s Eastern Conference when that
league
\href{https://www.nytimes3xbfgragh.onion/2020/03/11/sports/basketball/nba-season-suspended-coronavirus.html}{halted
operations} last week in response to the outbreak.

In an eerie coincidence, Brady's Patriots career began in the throes of
one national crisis (he took over on the first N.F.L. Sunday after the
terrorist attacks of Sept. 11) and ended in the grips of another.

His\href{https://www.cnn.com/2019/01/24/sport/rams-patriots-2002-super-bowl-xxxvi/index.html}{first
Super Bowl win}, on Feb. 3, 2002, ended a 16-year championship drought
in the region and ushered in an era during which Boston teams have won
12 titles and became known as the city of champions.

Now, everyone is left to wonder what the next era will look like, while
fans like Field, who now lives in Brooklyn, savor the two decades of
memories since that California kid jogged on to the field.

``People say, `You guys in Boston have it pretty good,''' Field
explained. ``But I was born in 1963. I watched the Patriots lose to the
Miami Dolphins, 51-0, for 25 years before anything good ever happened.''

Advertisement

\protect\hyperlink{after-bottom}{Continue reading the main story}

\hypertarget{site-index}{%
\subsection{Site Index}\label{site-index}}

\hypertarget{site-information-navigation}{%
\subsection{Site Information
Navigation}\label{site-information-navigation}}

\begin{itemize}
\tightlist
\item
  \href{https://help.nytimes3xbfgragh.onion/hc/en-us/articles/115014792127-Copyright-notice}{©~2020~The
  New York Times Company}
\end{itemize}

\begin{itemize}
\tightlist
\item
  \href{https://www.nytco.com/}{NYTCo}
\item
  \href{https://help.nytimes3xbfgragh.onion/hc/en-us/articles/115015385887-Contact-Us}{Contact
  Us}
\item
  \href{https://www.nytco.com/careers/}{Work with us}
\item
  \href{https://nytmediakit.com/}{Advertise}
\item
  \href{http://www.tbrandstudio.com/}{T Brand Studio}
\item
  \href{https://www.nytimes3xbfgragh.onion/privacy/cookie-policy\#how-do-i-manage-trackers}{Your
  Ad Choices}
\item
  \href{https://www.nytimes3xbfgragh.onion/privacy}{Privacy}
\item
  \href{https://help.nytimes3xbfgragh.onion/hc/en-us/articles/115014893428-Terms-of-service}{Terms
  of Service}
\item
  \href{https://help.nytimes3xbfgragh.onion/hc/en-us/articles/115014893968-Terms-of-sale}{Terms
  of Sale}
\item
  \href{https://spiderbites.nytimes3xbfgragh.onion}{Site Map}
\item
  \href{https://help.nytimes3xbfgragh.onion/hc/en-us}{Help}
\item
  \href{https://www.nytimes3xbfgragh.onion/subscription?campaignId=37WXW}{Subscriptions}
\end{itemize}
