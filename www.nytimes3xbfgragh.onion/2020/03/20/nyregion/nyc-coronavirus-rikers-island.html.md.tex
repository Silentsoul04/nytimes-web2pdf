Sections

SEARCH

\protect\hyperlink{site-content}{Skip to
content}\protect\hyperlink{site-index}{Skip to site index}

\href{https://www.nytimes3xbfgragh.onion/section/nyregion}{New York}

\href{https://myaccount.nytimes3xbfgragh.onion/auth/login?response_type=cookie\&client_id=vi}{}

\href{https://www.nytimes3xbfgragh.onion/section/todayspaper}{Today's
Paper}

\href{/section/nyregion}{New York}\textbar{}`A Storm Is Coming': Fears
of an Inmate Epidemic as the Virus Spreads in the Jails

\url{https://nyti.ms/2QxuPxh}

\begin{itemize}
\item
\item
\item
\item
\item
\end{itemize}

\hypertarget{the-coronavirus-outbreak}{%
\subsubsection{\texorpdfstring{\href{https://www.nytimes3xbfgragh.onion/news-event/coronavirus?name=styln-coronavirus-national\&region=TOP_BANNER\&block=storyline_menu_recirc\&action=click\&pgtype=Article\&impression_id=f12a3130-f4ba-11ea-b616-c3ae477f9ed6\&variant=undefined}{The
Coronavirus
Outbreak}}{The Coronavirus Outbreak}}\label{the-coronavirus-outbreak}}

\begin{itemize}
\tightlist
\item
  live\href{https://www.nytimes3xbfgragh.onion/2020/09/11/world/covid-19-coronavirus.html?name=styln-coronavirus-national\&region=TOP_BANNER\&block=storyline_menu_recirc\&action=click\&pgtype=Article\&impression_id=f12a3131-f4ba-11ea-b616-c3ae477f9ed6\&variant=undefined}{Latest
  Updates}
\item
  \href{https://www.nytimes3xbfgragh.onion/interactive/2020/us/coronavirus-us-cases.html?name=styln-coronavirus-national\&region=TOP_BANNER\&block=storyline_menu_recirc\&action=click\&pgtype=Article\&impression_id=f12a3132-f4ba-11ea-b616-c3ae477f9ed6\&variant=undefined}{Maps
  and Cases}
\item
  \href{https://www.nytimes3xbfgragh.onion/interactive/2020/science/coronavirus-vaccine-tracker.html?name=styln-coronavirus-national\&region=TOP_BANNER\&block=storyline_menu_recirc\&action=click\&pgtype=Article\&impression_id=f12a3133-f4ba-11ea-b616-c3ae477f9ed6\&variant=undefined}{Vaccine
  Tracker}
\item
  \href{https://www.nytimes3xbfgragh.onion/2020/09/10/us/politics/fda-coronavirus-vaccine.html?name=styln-coronavirus-national\&region=TOP_BANNER\&block=storyline_menu_recirc\&action=click\&pgtype=Article\&impression_id=f12a3134-f4ba-11ea-b616-c3ae477f9ed6\&variant=undefined}{F.D.A.
  Regulators' Self-Defense}
\item
  \href{https://www.nytimes3xbfgragh.onion/2020/09/09/upshot/coronavirus-surprise-test-fees.html?name=styln-coronavirus-national\&region=TOP_BANNER\&block=storyline_menu_recirc\&action=click\&pgtype=Article\&impression_id=f12a3135-f4ba-11ea-b616-c3ae477f9ed6\&variant=undefined}{Surprise
  Test Fees}
\end{itemize}

Advertisement

\protect\hyperlink{after-top}{Continue reading the main story}

Supported by

\protect\hyperlink{after-sponsor}{Continue reading the main story}

\hypertarget{a-storm-is-coming-fears-of-an-inmate-epidemic-as-the-virus-spreads-in-the-jails}{%
\section{`A Storm Is Coming': Fears of an Inmate Epidemic as the Virus
Spreads in the
Jails}\label{a-storm-is-coming-fears-of-an-inmate-epidemic-as-the-virus-spreads-in-the-jails}}

A growing chorus of officials and public defenders in New York City are
calling for the release of people who are especially vulnerable to the
coronavirus. The alternative, they say, may be a public health
catastrophe.

\includegraphics{https://static01.graylady3jvrrxbe.onion/images/2020/03/20/nyregion/20nyvirus-jails/merlin_170787702_0b48f10f-a4e9-4d64-b0d2-779ba201fbca-articleLarge.jpg?quality=75\&auto=webp\&disable=upscale}

By \href{https://www.nytimes3xbfgragh.onion/by/jan-ransom}{Jan Ransom}
and \href{https://www.nytimes3xbfgragh.onion/by/alan-feuer}{Alan Feuer}

\begin{itemize}
\item
  Published March 20, 2020Updated April 23, 2020
\item
  \begin{itemize}
  \item
  \item
  \item
  \item
  \item
  \end{itemize}
\end{itemize}

It started with a jails investigator in an office three miles from
Rikers Island. Then, a correction officer at a security checkpoint near
the entrance to the jail complex got it. Hours later, it was an inmate
in a crowded housing unit.

Within days, the investigator had died and three more correction
officers and two other staff members had tested positive for
\href{https://www.nytimes3xbfgragh.onion/2020/04/23/podcasts/the-daily/jails-inmates-coronavirus.html}{the
coronavirus}, confirming fears that the highly contagious disease had
arrived in the nation's second-largest jail system, endangering 5,300
inmates and twice as many guards.On Thursday, the jail system's chief
physician, Ross MacDonald, took to Twitter with a warning: ``A storm is
coming.''

\begin{quote}
A storm is coming and I know what I'll be doing when it claims my first
patient. What will you be doing? What will you have done? We have told
you who is at risk. Please let as many out as you possibly can. end.

--- Ross MacDonald (@RossMacDonaldMD)
\href{https://twitter.com/RossMacDonaldMD/status/1240455802193883137?ref_src=twsrc\%5Etfw}{March
19, 2020}
\end{quote}

He was part of a growing chorus of public defenders and officials in New
York City, led by Mayor Bill de Blasio, who have been pushing for the
state courts and the city's district attorneys to release from city
jails people who are especially vulnerable to the virus.

The alternative, they have said, may be a public health catastrophe.

Similar scenarios are playing out in jails and prisons throughout the
state and across the country as correction staff members and inmates
have tested positive for the virus. Two correction officers in upstate
New York prisons, one correction officer in Westchester and an inmate in
a Nassau County jail have been found to have the disease, as have two
inmates in a federal prison in California.

Mayor de Blasio said his administration was working with prosecutors to
free elderly and infirm inmates. On Friday, the district attorneys in
Manhattan, Brooklyn, Queens and the Bronx said they had consented to the
release of dozens of inmates, though the final decision will be up to
the courts.

``These are unprecedented times,'' the Queens district attorney, Melinda
Katz, said. ``We are doing this in a truncated period of time.''

Officials in major cities in California, Florida and Pennsylvania had
already taken similar steps to slow the spread of the virus.

But public defenders and New York City officials said the process of
setting people free had been hampered by uncertainties over who could
authorize their release, concerns over public safety and worries about
where to send people once they were out.

\hypertarget{latest-updates-the-coronavirus-outbreak}{%
\section{\texorpdfstring{\href{https://www.nytimes3xbfgragh.onion/2020/09/11/world/covid-19-coronavirus.html?action=click\&pgtype=Article\&state=default\&region=MAIN_CONTENT_1\&context=storylines_live_updates}{Latest
Updates: The Coronavirus
Outbreak}}{Latest Updates: The Coronavirus Outbreak}}\label{latest-updates-the-coronavirus-outbreak}}

Updated 2020-09-12T05:29:13.829Z

\begin{itemize}
\tightlist
\item
  \href{https://www.nytimes3xbfgragh.onion/2020/09/11/world/covid-19-coronavirus.html?action=click\&pgtype=Article\&state=default\&region=MAIN_CONTENT_1\&context=storylines_live_updates\#link-dfb8a16}{Fauci
  cautions the virus could disrupt life in the U.S. until `maybe even
  towards the end of 2021.'}
\item
  \href{https://www.nytimes3xbfgragh.onion/2020/09/11/world/covid-19-coronavirus.html?action=click\&pgtype=Article\&state=default\&region=MAIN_CONTENT_1\&context=storylines_live_updates\#link-7104d154}{From
  Asia to Africa, China promotes its vaccine candidates to win friends.}
\item
  \href{https://www.nytimes3xbfgragh.onion/2020/09/11/world/covid-19-coronavirus.html?action=click\&pgtype=Article\&state=default\&region=MAIN_CONTENT_1\&context=storylines_live_updates\#link-393ad215}{The
  other way the virus will kill: hunger.}
\end{itemize}

\href{https://www.nytimes3xbfgragh.onion/2020/09/11/world/covid-19-coronavirus.html?action=click\&pgtype=Article\&state=default\&region=MAIN_CONTENT_1\&context=storylines_live_updates}{See
more updates}

More live coverage:
\href{https://www.nytimes3xbfgragh.onion/live/2020/09/11/business/stock-market-today-coronavirus?action=click\&pgtype=Article\&state=default\&region=MAIN_CONTENT_1\&context=storylines_live_updates}{Markets}

``For everyone's safety, this decision cannot be rushed,'' Freddi
Goldstein, Mr. de Blasio's spokeswoman, said on Thursday. ``We need to
determine both public health risk and public safety risk.''

City officials would like to release several hundred people sent to
Rikers Island for minor parole violations, near the end of their
sentences or detained on low bail.

\includegraphics{https://static01.graylady3jvrrxbe.onion/images/2020/03/20/nyregion/20nyvirus-jails2/merlin_170789997_831afca3-ccb6-4b6b-b8e0-4236d7ac8a9f-articleLarge.jpg?quality=75\&auto=webp\&disable=upscale}

\includegraphics{https://static01.graylady3jvrrxbe.onion/images/2017/01/29/podcasts/the-daily-album-art/the-daily-album-art-articleInline-v2.jpg?quality=75\&auto=webp\&disable=upscale}

\hypertarget{listen-to-the-daily-getting-off-rikers-island}{%
\subsubsection{Listen to `The Daily': Getting Off Rikers
Island}\label{listen-to-the-daily-getting-off-rikers-island}}

We spoke to a Rikers Island resident who has the coronavirus. ``They are
killing us,'' he said. ``What are we supposed to do?''

transcript

Back to The Daily

bars

0:00/20:12

-20:12

transcript

\hypertarget{listen-to-the-daily-getting-off-rikers-island-1}{%
\subsection{Listen to `The Daily': Getting Off Rikers
Island}\label{listen-to-the-daily-getting-off-rikers-island-1}}

\hypertarget{hosted-by-michael-barbaro-and-megan-twohey-produced-by-daniel-guillemette-adizah-eghan-sydney-harper-annie-brown-and-alexandra-leigh-young-and-edited-by-lisa-chow-and-liz-o-baylen}{%
\subsubsection{Hosted by Michael Barbaro and Megan Twohey, produced by
Daniel Guillemette, Adizah Eghan, Sydney Harper, Annie Brown and
Alexandra Leigh Young, and edited by Lisa Chow and Liz O.
Baylen}\label{hosted-by-michael-barbaro-and-megan-twohey-produced-by-daniel-guillemette-adizah-eghan-sydney-harper-annie-brown-and-alexandra-leigh-young-and-edited-by-lisa-chow-and-liz-o-baylen}}

\hypertarget{we-spoke-to-a-rikers-island-resident-who-has-the-coronavirus-they-are-killing-us-he-said-what-are-we-supposed-to-do}{%
\paragraph{We spoke to a Rikers Island resident who has the coronavirus.
``They are killing us,'' he said. ``What are we supposed to
do?''}\label{we-spoke-to-a-rikers-island-resident-who-has-the-coronavirus-they-are-killing-us-he-said-what-are-we-supposed-to-do}}

\begin{itemize}
\item
  michael barbaro\\
  From The New York Times, I'm Michael Barbaro. This is ``The Daily.''
\item
  {[}music{]}
\item
  archived recording\\
  Well, some call it the most dangerous place on the planet when it
  comes to facing a coronavirus outbreak --- a packed county jail could
  be a disaster waiting to happen.
\end{itemize}

michael barbaro

Across the U.S. ---

\begin{itemize}
\item
  archived recording 1\\
  The Los Angeles County Sheriff's Department said ---
\item
  archived recording 2\\
  In Chicago, the Cook County Jail has ---
\item
  archived recording 3\\
  New York calls for quick action now growing louder after ---
\end{itemize}

michael barbaro

Jails and prisons, with their cramped quarters and communal living, have
become hotbeds for the spread of the coronavirus ---

\begin{itemize}
\item
  archived recording 1\\
  Prisoners are sounding the alarm on the ballooning outbreak, writing
  messages on windows reading, ``Help. We matter.'' And ``We're dying.''
\item
  archived recording 2\\
  Everybody is losing it. I mean, it's not just the detainees but also
  people that are working here.
\item
  archived recording 3\\
  We cannot change the fundamental nature of jail. We cannot socially
  distance dozens of elderly men living in a dorm sharing a bathroom.
  Think of a cruise ship recklessly boarding more passengers each day.
\end{itemize}

michael barbaro

--- prompting local governments to take the unprecedented step of
releasing thousands of inmates ---

\begin{itemize}
\tightlist
\item
  archived recording\\
  People are confined to their homes, but this pandemic is actually
  bringing freedom to some New York City inmates. Mayor de Blasio said
  ---
\end{itemize}

michael barbaro

--- and raising the fraught question of who is let out and who remains
in custody. Today, Megan Twohey speaks with our colleague Alan Feuer
about the story of one inmate trying to get out of the second largest
jail in the country, Rikers Island in New York.

It's Thursday, April 23.

megan twohey

So, Alan, tell me about Mitch Pomerance.

alan feuer

So I met Mitch Pomerance through his lawyer, Laura Eraso. And Mitch is
54. And before the virus even landed at Rikers, Mitch was already in bad
health. In fact, his health was so bad at one point that he had to be
transported off the Island to a nearby hospital, where he underwent
surgery to drain fluid from his lungs. And so he's been working with
Laura, his lawyer, trying to build a case to get off of Rikers Island.

\begin{itemize}
\item
  laura eraso\\
  Hey, Mitch. Can you hear me?
\item
  mitch pomerance\\
  Yes.
\end{itemize}

alan feuer

And we recorded three of their conversations over the course of a week,
earlier this month.

\begin{itemize}
\tightlist
\item
  laura eraso\\
  So you know, like I had told you a little bit about earlier regarding
  today, we were in line all day for the writ to be heard. But
  unfortunately, the court only goes to 4:30, and they weren't able to
  squeeze it in on the calendar.
\end{itemize}

alan feuer

The court system, don't forget, has more or less shut down because of
the pandemic.

megan twohey

Right.

alan feuer

And on the day that we recorded our first call with Mitch, there was
indeed a backlog in the court system, and the judge didn't have time to
hear Mitch's case.

megan twohey

And while Mitch is waiting for the judge to hear his case, what is the
situation like for him inside the jail?

\begin{itemize}
\tightlist
\item
  laura eraso\\
  I know your sister told me that somebody else in your dorm had been
  taken out?
\end{itemize}

alan feuer

So Mitch is housed in a dorm at Rikers. And that means that he lives
with a dozen or so inmates in one open room. And he tells his lawyer
that one person in that dorm has tested positive for Coivd-19.

\begin{itemize}
\tightlist
\item
  mitch pomerance\\
  Yes, we were tested again the other day, and it turns out that at
  least one is a carrier. So this guy actually, although he has no
  symptoms, he actually has the disease, they say, and he can pass it on
  to the rest of us. And so they took him out today, and it's real scary
  because no one was told.
\end{itemize}

alan feuer

Mitch is also telling his lawyer here that the staff at Rikers hasn't
cleaned the dorms since that person tested positive.

\begin{itemize}
\item
  laura eraso\\
  So what did they --- did they just test you and leave? Or did they do
  any other --- did they clean or sanitize or anything?
\item
  mitch pomerance\\
  {[}LAUGHS{]} They didn't clean. They didn't even empty the garbage
  yet. We have garbage overflowing all the garbage pails, and they
  didn't even clean. They gave us a new test two days ago, three days
  ago, and he's the only one --- the only they told us about so far. So
  I'm sure --- I'm positive there's more. I'm praying it's not me. I'm
  praying it's not me.
\item
  laura eraso\\
  I mean, was he wearing a mask or anything?
\item
  mitch pomerance\\
  No, never. They gave us one mask for the whole week yesterday --- last
  night. My mask fell apart like four days ago. So I had no mask. I've
  had no mask for four days.
\end{itemize}

megan twohey

And how does that description of the conditions, at least in his
particular dorm, how does that square with what you've learned in your
reporting?

alan feuer

Well, from the start of the crisis, the Department of Corrections has
taken several measures to slow the spread of the virus. They've asked
inmates to sleep head to toe at night. You know, they wanted to get one
mouth and nose as far away from another mouth and nose as they could.
They had the cleaning staff clean the common areas and the housing areas
as best as possible. But the fact is, conditions at Rikers remain very
unsanitary. You know, the inmates can't get hand sanitizer because it's
an alcohol-based product, and they're not allowed to have alcohol.
Oftentimes, the only way for them to get soap for their own personal
hygiene is to buy it in the commissary. So these measures that were put
in place to stop the spread of the virus haven't always worked.

\begin{itemize}
\item
  laura eraso\\
  I know you guys sleep in pretty close proximity. How close do you
  think you were to him?
\item
  mitch pomerance\\
  About 5 feet, 6 feet from where he sleeps --- he slept --- over the
  next aisle over, one person over.
\item
  laura eraso\\
  Wow.
\item
  mitch pomerance\\
  So that was about 6 feet from me, yeah.
\item
  laura eraso\\
  But how do you feel? How do your lungs feel? I know you're going
  through that too.
\item
  mitch pomerance\\
  It's constantly --- where they operate, it hurts a lot.
\end{itemize}

megan twohey

So it sounds like Mitch's risk of getting Covid-19 is really high.

alan feuer

Yes. But, in fact, Mitch's lawyer, Laura, expects they could get a
decision from the judge very soon about the question of his release.

\begin{itemize}
\item
  laura eraso\\
  And hopefully we get a result tomorrow, but we can talk more about
  that, OK?
\item
  mitch pomerance\\
  OK, thank you so much for everything. I appreciate your help. I'll be
  talking to you tomorrow.
\item
  laura eraso\\
  All right, stay well.
\item
  mitch pomerance\\
  Good luck. Thank you.
\end{itemize}

alan feuer

But when they talk the next day ---

\begin{itemize}
\item
  attorney\\
  Hey, can you hear me?
\item
  mitch pomerance\\
  Yeah.
\end{itemize}

alan feuer

--- she's got some bad news. She and her team went in front of the
judge. They argued the case. And the judge has said no, Mitch can't get
out.

megan twohey

Hmm.

\begin{itemize}
\item
  laura eraso\\
  I'm trying to do it as quickly as possible. I realize that someone
  else who was sleeping next to you had tested positive.
\item
  mitch pomerance\\
  Someone else? Not someone else --- four people.
\item
  laura eraso\\
  Wow. So they came back today ---
\item
  mitch pomerance\\
  Four people, four people.
\item
  laura eraso\\
  --- with the test?
\item
  mitch pomerance\\
  Yeah, four people --- today. This is crazy. This is actually crazy.
  They're killing us. They are killing us. What are we supposed to do?
  What do we do? We need to file a writ. I need to get out of here. I
  need to get out of here. They're killing me. If I catch this, I'm
  dead. I don't know what to do. We need somebody to step in. We need
  somebody important to step in --- a congressman, a senator, somebody,
  a court, a judge. Somebody's gotta do something for us --- somebody.
\end{itemize}

megan twohey

So by this point, Rikers has already released hundreds of inmates.

alan feuer

Correct.

megan twohey

So how are they deciding who stays and who goes?

alan feuer

Well, so far they've released 650 people. And of those 650 people,
you're generally talking about three different categories of inmates.
First, there are those who have been accused of non-violent, low-level
offenses. There are also people who are at Rikers serving what's called
a city year, a sentence that is short, less than a year, and so they're
about to get out anyhow. And then there's a third category of people who
are at the Island because they've committed a technical violation of
their parole, meaning they were out on parole for a previous crime, and
they got caught doing something minor like smoking a joint or drinking a
beer on their sidewalk.

megan twohey

So how does Mitch fit into this picture?

alan feuer

So the challenge for Mitch is that his case is just a lot more
complicated.

Mitch has served a combined 22 years in prison for selling drugs and
committing multiple robberies. The state considers him a violent
offender. He got out of prison in 2018. But then, last summer, he was
rearrested for an attempted robbery charge while he was on parole. So
Mitch just doesn't really fit neatly into any of those categories for
people that were getting off Rikers Island. And yet, he's medically
vulnerable, which is another factor that judges are weighing in
releasing people from Rikers and that the city itself is prioritizing
people like that --- those people, who, if they catch Covid-19, they're
more likely to die.

\begin{itemize}
\item
  laura eraso\\
  We're trying to put as much pressure on the governor, on New York
  state docs, on Commissioner Annucci to release everyone and release
  them safely. I mean, as I've said before ---
\item
  mitch pomerance\\
  I don't care about everyone. I want to get out. I'm trying to get out.
\item
  laura eraso\\
  No, I know.
\item
  mitch pomerance\\
  I don't care about anybody else. Everyone else is not sick like I am.
  I'm going to die if I have this. If I get it, I'm going to die. Simple
  as that. Simple as that. Let's not play games. I'm going to die. I
  don't care about anybody else right now. I'm being selfish about this.
\end{itemize}

alan feuer

So Mitch represents this really difficult but interesting tension that a
lot of courts are facing across the country right now. Does he pose too
much of a risk to public safety to be let out, or is he medically
vulnerable enough to be let out? Should judges be prioritizing the
safety of the public, or should they be thinking about the health of the
individual inmates?

{[}music{]}

megan twohey

So, Alan, it sounds like the judge, in denying Mitch's request to get
out of Rikers, is putting considerations of public safety above his
health.

alan feuer

Yeah, absolutely.

megan twohey

So what does Mitch say about that?

\begin{itemize}
\item
  alan feuer\\
  Hey, Mitch, Alan Feuer from The New York Times. How are you?
\item
  mitch pomerance\\
  I'm OK, I guess. Things are horrible here.
\end{itemize}

alan feuer

Well, I asked him. I just put it to him if he was a public safety threat
given his rap sheet.

\begin{itemize}
\item
  alan feuer\\
  Well, let me ask you this.
\item
  mitch pomerance\\
  Yes, sir.
\item
  alan feuer\\
  Is there any way you can understand the judge's decision? You know,
  it's an attempted robbery charge. I'm not saying, you know, you
  haven't been proven guilty yet.
\item
  mitch pomerance\\
  Right.
\item
  alan feuer\\
  But can you understand how a judge might make the decision that's been
  made here?
\item
  mitch pomerance\\
  Can I understand? Yes, I can understand. I can, without a doubt, that
  the judge doesn't want to jump out the window and grant anything for
  fear of me going out and doing another crime, committing another
  crime. So, again, I do understand that. I absolutely do. But I can't
  say anything more than look at my proof.
\item
  alan feuer\\
  Do you see yourself as a threat to public safety?
\item
  mitch pomerance\\
  I don't. I don't.
\item
  alan feuer\\
  Explain that to me.
\item
  mitch pomerance\\
  I can't walk, first off. I'm in a wheelchair. If I can get up and walk
  two steps, it would be a miracle. I can't walk. I mean, just look at
  the proof of what I have. At least with a clear conscience, look at it
  with open eyes, and then make a decision.
\end{itemize}

megan twohey

So what happens next?

\begin{itemize}
\item
  mitch pomerance\\
  Hi, Laura.
\item
  laura eraso\\
  Hey, Mitch, can you hear me?
\item
  mitch pomerance\\
  Yes, I can.
\end{itemize}

alan feuer

Mitch and Laura talked again the day after the judge denied his request
to leave Rikers.

\begin{itemize}
\item
  laura eraso\\
  All right, so what did they tell you when they gave you your test?
\item
  mitch pomerance\\
  So just five minutes ago the R.N. came, and he came around the dorm,
  and of the 12 people we have left in a dorm, everyone's positive but
  three people. So now we're positive, and we don't know what's going to
  happen now. There's so many people that are positive ---
\end{itemize}

alan feuer

Mitch tests positive for Covid-19.

megan twohey

Wow.

\begin{itemize}
\item
  mitch pomerance\\
  Yeah, there's so many people that are positive on Rikers Island, it's
  out of control. It's out of control. I don't feel good at all. I'm
  having a problem breathing already. I told them I'm having a problem
  breathing this morning. This is ---
\item
  laura eraso\\
  OK, well, I mean, we're going to move fast on this. I just was
  actually on the phone ---
\end{itemize}

megan twohey

So is that it? Is Mitch out of options?

alan feuer

Well, what this does --- the change that this makes in Mitch's case from
a legal perspective is that it allows Laura, his lawyer, to make a
totally different argument in front of the judge. Now, instead of just
saying that Mitch is potentially at risk of contracting the disease, she
can argue that because he already has it, and he's got this terrible
preexisting condition in his lungs, that he's not going to be able to
get the proper medical care he needs at Rikers Island.

\begin{itemize}
\item
  laura eraso\\
  I'm going to try to --- I'm going to get these papers filed today, and
  I'm not going to wait for the other affirmation. So we're going to
  handle this, and this definitely needs to be reargued in light of this
  fact, OK?
\item
  mitch pomerance\\
  Yeah.
\item
  laura eraso\\
  So just concentrate right now on trying to take care of yourself ---
  your mental health, your physical health. So I'm going to be in touch.
  Call me at the end of the day, and I can confirm with you that I was
  able to finish that up and get it filed, OK?
\item
  mitch pomerance\\
  All right, thanks.
\item
  laura eraso\\
  OK, bye. Take care.
\item
  mitch pomerance\\
  Goodbye.
\end{itemize}

alan feuer

These jails, like Navy ships or meat processing plants, they are petri
dishes for infection. But unlike service members or essential workers,
inmates in jails, they aren't necessarily the most sympathetic
population in the world. Still, should part of their punishment be to
potentially contract a disease like Covid-19? I mean, is that supposed
to be included in the price of going to jail?

Mitch is still there at Rikers, waiting for an answer.

\begin{itemize}
\item
  alan feuer\\
  So help me understand right now what an ordinary, average day of yours
  looks like today?
\item
  mitch pomerance\\
  Hmm, well, I need help. I need help showering. So there's this guy,
  Eddie, who helps me shower. He helps me get in and out of the shower
  in the morning time when he gets up. I take a shower, have breakfast,
  and go back to bed for a couple hours and take my medication. I'm just
  doing a lot of reading. That's it. I try to stay in contact with my
  family on the phone.
\item
  alan feuer\\
  And how are you feeling these days?
\item
  mitch pomerance\\
  You know, I'm a little agitated right now. Any time I move around,
  it's hard for me to breathe. So right now, I'm trying to draw breath.
  So I have a real bad headache right now.
\item
  automated speaker\\
  You have one minute left.
\item
  mitch pomerance\\
  I'm just trying to get out of here. I just want to be --- I just want
  to live again. That's it.
\end{itemize}

{[}music{]}

megan twohey

Well, Alan, thank you so much for taking the time to share this with us.

alan feuer

Well, thanks for having me, Megan.

michael barbaro

We'll be right back.

{[}music{]}

michael barbaro

Here's what else you need to know today. In a major discovery, public
health officials in California now say that the coronavirus killed a
resident there on Feb. 6. That discovery changes the timeline of the
virus in the U.S. by revealing that infections began much earlier than
previously thought. Until now, the first U.S. death was believed to have
occurred in Washington State on Feb. 26, about three weeks later. The
Feb. 6 death, in the town of Santa Clara, is believed to be the result
of community spread, suggesting that the virus was circulating on the
West Coast well before public health officials had realized. And The
Times reports that Chinese government operatives were involved in
spreading false warnings to Americans about an impending national
lockdown in March. The warnings, which arrived as text messages and
social media posts, alarmed millions of Americans. U.S. officials said
that the tactics resembled past attempts by Russia to widen social
divisions within the U.S. That's it for ``The Daily.'' I'm Michael
Barbaro. See you tomorrow.

Public defenders and advocates for inmates have called for sending home
all inmates with pre-existing medical conditions, those over 50 and
anyone jailed for a parole violation.

``It is a ticking time bomb,'' said Justine Olderman, executive director
of the Bronx Defenders. ``We're looking for bold action and
leadership.''

On Saturday, the Board of Correction, the city agency that serves as a
watchdog over the jails, said that the number of confirmed coronavirus
cases at Rikers had jumped from eight to 38 --- 21 detainees, 12 jail
employees and five correctional health workers.

Board officials said there were also 58 inmates being monitored in the
contagious disease unit up from 27 people on Tuesday.

A person familiar with the matter said a previously closed jails
facility had been reopened to accommodate the growing number of inmates
being placed into quarantine.

Dr. Robert Cohen, a member of the Board of Correction, said, ``The most
important thing we can do right now is discharge all of the people who
are old and have serious medical issues --- those people are likely to
die from a coronavirus infection.''

Seventeen percent of the city's jail population is over 50, and a
majority of that group has an underlying health condition, according to
data provided by the city's Department of Correction.

City corrections officials said they had begun screening all personnel
entering the jail for fevers and doing medical checks of inmates going
to and from court. Visiting inmates has been suspended. Arts and
education programs have been cut back.

Detainees have been instructed to sleep head-to-toe, to maintain three
feet of distance between them, and to not sit on each other's beds.

Even so, said Dr. Rachael Bedard, a geriatrician who works at the jail,
it has been --- and most likely will be --- difficult to stem the spread
of the virus in a place where people live in cramped and often
unsanitary conditions.

``The only meaningful public health intervention here is to depopulate
the jails dramatically,'' she said.

Rikers has an 88-bed contagious disease unit with air-controlled cells;
the infected inmate is housed there now. But the unit does not have
ventilators, so inmates who become severely ill will be sent to Bellevue
Hospital Center.

Corrections officials said that they had stepped up cleaning and that
inmates and staff members were given sanitation wipes and general
disinfectant. Guards have also been supplied with gloves and respiratory
masks.

But inmates, union officials and Rikers staff members say conditions in
the jail complex remain unsanitary.

Inmates have complained to their lawyers in recent days that they did
not have soap or cleaning products. One told his lawyer in a letter that
his housing unit had not been cleaned in several days.

He said when he arrived at the jail, he was held in a pen with dozens of
others, some of whom were coughing, and described the area as
``extremely dirty,'' according to the Legal Aid Society, the city's
largest public defender group.

Rayshad Jackson, who was released from Rikers Island on Friday, said
jail officials had not informed inmates about the viral outbreak
wreaking havoc globally.

He learned about the virus three days ago from news reports. The news
caused a small riot in the jail, he said.

``No one knew what this was,'' said Mr. Jackson, who had been detained
on a parole violation and has chronic asthma and sleep apnea.

One staff member, who spoke anonymously for fear of retribution from the
city, said that many correction officers did not have access to hand
sanitizer, masks or gloves. Many of the facilities, the staff member
said, were poorly ventilated and, despite the department's public
statements, some spaces remained uncleaned for days.

Elias Husamudeen, the president of the Correction Officers' Benevolent
Association, said his 11,000 officers had been given only 3,000 masks.

Mr. Husamudeen said the department needed to segregate new inmates
coming into the jail and provide more supplies. If not, he said, ``the
crisis will grow worse with each passing day.''

State prisons face a similar problem. So far two correction officers,
including one at the Sing Sing Correctional Facility and another at the
Shawangunk Correctional Facility, and a civilian employee in Albany have
tested positive for the virus, state prison officials said.

Gov. Andrew M. Cuomo and state prison officials have declined to share
details about their plans for addressing an outbreak, citing security
concerns.

Civilian employees in the prison system were ordered to remain home for
two weeks to limit the number of people entering the prisons, and visits
by friends and family have been suspended.

Foster Thompson, an inmate serving time for murder at Sing Sing, said in
a telephone interview this week that an inmate in a nearby cell was
sneezing, hacking and complaining loudly of body aches. The next day, he
said, about 40 inmates in his housing unit went to a clinic for medical
attention, but were turned away.

Mr. Thompson said prison officials had recently canceled social and
educational programs to keep inmates apart. But crowds are everywhere,
he said.

With little else to do, hundreds of prisoners have been gathering in the
yard. Eighty men at a time pack into the bathhouse showers, and with
visits suspended, there are long lines for the phones.

``There's no way be away from people,'' Mr. Thompson said. ``Everybody's
basically right on top of each other.''

William K. Rashbaum and Jesse McKinley contributed reporting.

Advertisement

\protect\hyperlink{after-bottom}{Continue reading the main story}

\hypertarget{site-index}{%
\subsection{Site Index}\label{site-index}}

\hypertarget{site-information-navigation}{%
\subsection{Site Information
Navigation}\label{site-information-navigation}}

\begin{itemize}
\tightlist
\item
  \href{https://help.nytimes3xbfgragh.onion/hc/en-us/articles/115014792127-Copyright-notice}{©~2020~The
  New York Times Company}
\end{itemize}

\begin{itemize}
\tightlist
\item
  \href{https://www.nytco.com/}{NYTCo}
\item
  \href{https://help.nytimes3xbfgragh.onion/hc/en-us/articles/115015385887-Contact-Us}{Contact
  Us}
\item
  \href{https://www.nytco.com/careers/}{Work with us}
\item
  \href{https://nytmediakit.com/}{Advertise}
\item
  \href{http://www.tbrandstudio.com/}{T Brand Studio}
\item
  \href{https://www.nytimes3xbfgragh.onion/privacy/cookie-policy\#how-do-i-manage-trackers}{Your
  Ad Choices}
\item
  \href{https://www.nytimes3xbfgragh.onion/privacy}{Privacy}
\item
  \href{https://help.nytimes3xbfgragh.onion/hc/en-us/articles/115014893428-Terms-of-service}{Terms
  of Service}
\item
  \href{https://help.nytimes3xbfgragh.onion/hc/en-us/articles/115014893968-Terms-of-sale}{Terms
  of Sale}
\item
  \href{https://spiderbites.nytimes3xbfgragh.onion}{Site Map}
\item
  \href{https://help.nytimes3xbfgragh.onion/hc/en-us}{Help}
\item
  \href{https://www.nytimes3xbfgragh.onion/subscription?campaignId=37WXW}{Subscriptions}
\end{itemize}
