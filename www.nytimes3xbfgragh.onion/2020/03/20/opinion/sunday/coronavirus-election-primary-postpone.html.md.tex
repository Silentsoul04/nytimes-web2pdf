Sections

SEARCH

\protect\hyperlink{site-content}{Skip to
content}\protect\hyperlink{site-index}{Skip to site index}

\href{https://www.nytimes3xbfgragh.onion/section/opinion/sunday}{Sunday
Review}

\href{https://myaccount.nytimes3xbfgragh.onion/auth/login?response_type=cookie\&client_id=vi}{}

\href{https://www.nytimes3xbfgragh.onion/section/todayspaper}{Today's
Paper}

\href{/section/opinion/sunday}{Sunday Review}\textbar{}We Can't Let
Coronavirus Postpone Elections

\url{https://nyti.ms/2U9tP4H}

\begin{itemize}
\item
\item
\item
\item
\item
\end{itemize}

Advertisement

\protect\hyperlink{after-top}{Continue reading the main story}

\href{/section/opinion}{Opinion}

Supported by

\protect\hyperlink{after-sponsor}{Continue reading the main story}

\hypertarget{we-cant-let-coronavirus-postpone-elections}{%
\section{We Can't Let Coronavirus Postpone
Elections}\label{we-cant-let-coronavirus-postpone-elections}}

Even in war, America has kept up its democratic traditions. We can't
stop now.

By Jon Meacham

Mr. Meacham is the author, most recently, of ``The Hope of Glory:
Reflections on the Last Words of Jesus from the Cross.''

\begin{itemize}
\item
  March 20, 2020
\item
  \begin{itemize}
  \item
  \item
  \item
  \item
  \item
  \end{itemize}
\end{itemize}

\includegraphics{https://static01.graylady3jvrrxbe.onion/images/2020/03/22/opinion/20Meacham1/merlin_170643612_bc521903-8003-4f2c-b474-6f242139caa1-articleLarge.jpg?quality=75\&auto=webp\&disable=upscale}

Darkness reigned. It was 1864, and the nation was split into two warring
camps. Casualties rose steadily --- previously unimaginable numbers,
ultimately reaching about 750,000 dead --- and fighting continued
throughout the year. Gen. George McClellan, the Democratic nominee,
posed a genuine threat to a second term for Abraham Lincoln. McClellan
promised a quick, negotiated end to the war; a Lincoln defeat would have
led to a permanently divided nation and the preservation of slavery in
the Southern states.

The fate of the war, the future of the Republic, the nature of the
American experiment: Everything hung in the balance. And to preserve
that experiment, Lincoln insisted that the presidential election go
forward.

The president was fully prepared to lose the election and, according to
due constitutional form, to surrender power the following March. In
August 1864, in a private note, he wrote, ``This morning, as for some
days past, it seems exceedingly probable that this Administration will
not be re-elected.'' But he would accept the verdict of the voters. Here
was an incumbent president, the commander in chief of a nation facing a
sustained armed rebellion, unilaterally subsuming his own ambitions and
his own priorities to the very constitutional order then under siege.

He was unwilling to sacrifice democracy to save it --- a lesson we need
to bear in mind as the coronavirus pandemic threatens primaries in many
states and could, in the Age of Trump, conceivably affect the general
election in November.

Connecticut, Georgia, Kentucky, Louisiana and Ohio have postponed their
state primaries because of concerns over the transmission of the
coronavirus. Delaware, New York, Pennsylvania and Rhode Island are
scheduled for April 28 --- about the time the curve we are hoping to
flatten may or may not be trending in the best of directions.

It will be tempting for leaders in those states and others to postpone
their primaries. Instead, in this hour of crisis, state officials ---
understandably scrambling to secure their people --- should do all they
can to hold their elections as soon as possible. The legitimacy of the
eventual Democratic nominee could depend on it.

History is on the side of proceeding in times of uncertainty. There's
something in the American character that has long insisted on pressing
ahead with democracy's fundamental task: the casting of ballots and the
choosing of leaders. In addition to the Lincoln example, historians know
that James Madison was re-elected amid the War of 1812; the midterm
elections of 1814 took place not long after the British had invaded
Washington; the 1918 balloting occurred despite the ravages of the
Spanish flu; the 1932 election went forward in the face of the Great
Depression; and Franklin Roosevelt was re-elected in 1944, during World
War II. Even 9/11 delayed the New York City mayoral election only by a
matter of weeks.

We have world enough and time --- and, in several states, the experience
--- to make the voting in November safe and secure. Colorado offers us
perhaps the most promising model. A ``vote at home'' state (Hawaii,
Oregon and Washington have forms of this, too), Colorado mails ballots
to all registered voters well in advance of Election Day. Voters can
either mail them back or drop them off at central locations at any point
in the weeks-long window of time. Most people have chosen this option;
think of it as curbside democracy.

There are security issues, of course: ballots could be intercepted and
illegally cast by people with access to a person's mail. There are,
however, signature-checking safeguards in place. No system --- including
the current one --- is perfect. But we can't let the perfect be the
enemy of the good. This coming Monday, Senators Amy Klobuchar and Ron
Wyden are introducing legislation to make mail-in ballots available to
every voter in America.

We need to have these kinds of conversations about the election
honestly, rationally, and now. The sooner the better, for chaos could
lead to a nightmare scenario: the possibility that President Trump might
take advantage of the unfolding health crisis to delay the November
election.

Alarmist? Not for anyone who's paid even glancing attention to the
president's will to power and contempt for constitutional convention.
Though he's recently signaled that postponement isn't an issue, in the
past he's also joked --- at least we think he was joking --- about
blowing past the two-term limit imposed on presidents by the 22nd
Amendment. He retweeted Jerry Falwell Jr.'s suggestion that Mr. Trump be
given two additional years in office to make up for time lost to the
Mueller inquiry. And he has long trafficked in conspiracy theories about
unproven voter fraud in 2016.

The good news is that the Constitution so ably defended by Lincoln gives
the executive virtually no control over the timing of elections. Anxious
about monarchal absolutism, the founders invested Congress, not the
president, with the power to schedule the selection of presidential
electors. In a 1934 decision, the Supreme Court held that Congress has
every ``power essential to preserve the department and institutions of
the general government from impairment or destruction, whether
threatened by force or corruption.'' By statute, Congress has set the
date --- the first Tuesday after the first Monday in November, every
four years --- and the power to alter that date lies not with the branch
established by Article II, the executive, but with the lawmakers whose
authority is rooted in Article I.

Scholars do not believe that even action by the president under
national-emergency powers could postpone the quadrennial election unless
Congress agreed --- which means the Democratic House may be the only
bulwark against constitutional chaos come fall.

These are early days in this crisis. We could do worse, though, than to
look to Lincoln. When he wrote his letter about possible defeat in the
summer of 1864, he said he would stay at his post until the last hour,
true to his oath, making the best of things with President-elect
McClellan to ``save the Union between the election and the inauguration;
as he will have secured his election on such ground that he cannot
possibly save it afterwards.'' In the gloom of disunion, a bit of light.
May we find our way forward as Lincoln --- and all of us --- did in
other days of darkness.

Jon Meacham is the author, most recently, of ``The Hope of Glory:
Reflections on the Last Words of Jesus from the Cross.''

\emph{The Times is committed to publishing}
\href{https://www.nytimes3xbfgragh.onion/2019/01/31/opinion/letters/letters-to-editor-new-york-times-women.html}{\emph{a
diversity of letters}} \emph{to the editor. We'd like to hear what you
think about this or any of our articles. Here are some}
\href{https://help.nytimes3xbfgragh.onion/hc/en-us/articles/115014925288-How-to-submit-a-letter-to-the-editor}{\emph{tips}}\emph{.
And here's our email:}
\href{mailto:letters@NYTimes.com}{\emph{letters@NYTimes.com}}\emph{.}

\emph{Follow The New York Times Opinion section on}
\href{https://www.facebookcorewwwi.onion/nytopinion}{\emph{Facebook}}\emph{,}
\href{http://twitter.com/NYTOpinion}{\emph{Twitter (@NYTopinion)}}
\emph{and}
\href{https://www.instagram.com/nytopinion/}{\emph{Instagram}}\emph{.}

Advertisement

\protect\hyperlink{after-bottom}{Continue reading the main story}

\hypertarget{site-index}{%
\subsection{Site Index}\label{site-index}}

\hypertarget{site-information-navigation}{%
\subsection{Site Information
Navigation}\label{site-information-navigation}}

\begin{itemize}
\tightlist
\item
  \href{https://help.nytimes3xbfgragh.onion/hc/en-us/articles/115014792127-Copyright-notice}{©~2020~The
  New York Times Company}
\end{itemize}

\begin{itemize}
\tightlist
\item
  \href{https://www.nytco.com/}{NYTCo}
\item
  \href{https://help.nytimes3xbfgragh.onion/hc/en-us/articles/115015385887-Contact-Us}{Contact
  Us}
\item
  \href{https://www.nytco.com/careers/}{Work with us}
\item
  \href{https://nytmediakit.com/}{Advertise}
\item
  \href{http://www.tbrandstudio.com/}{T Brand Studio}
\item
  \href{https://www.nytimes3xbfgragh.onion/privacy/cookie-policy\#how-do-i-manage-trackers}{Your
  Ad Choices}
\item
  \href{https://www.nytimes3xbfgragh.onion/privacy}{Privacy}
\item
  \href{https://help.nytimes3xbfgragh.onion/hc/en-us/articles/115014893428-Terms-of-service}{Terms
  of Service}
\item
  \href{https://help.nytimes3xbfgragh.onion/hc/en-us/articles/115014893968-Terms-of-sale}{Terms
  of Sale}
\item
  \href{https://spiderbites.nytimes3xbfgragh.onion}{Site Map}
\item
  \href{https://help.nytimes3xbfgragh.onion/hc/en-us}{Help}
\item
  \href{https://www.nytimes3xbfgragh.onion/subscription?campaignId=37WXW}{Subscriptions}
\end{itemize}
