Sections

SEARCH

\protect\hyperlink{site-content}{Skip to
content}\protect\hyperlink{site-index}{Skip to site index}

\href{https://www.nytimes3xbfgragh.onion/section/politics}{Politics}

\href{https://myaccount.nytimes3xbfgragh.onion/auth/login?response_type=cookie\&client_id=vi}{}

\href{https://www.nytimes3xbfgragh.onion/section/todayspaper}{Today's
Paper}

\href{/section/politics}{Politics}\textbar{}The Lessons of the Elections
of 1918

\url{https://nyti.ms/3a9Kwm0}

\begin{itemize}
\item
\item
\item
\item
\item
\end{itemize}

\begin{itemize}
\item
  \href{https://www.nytimes3xbfgragh.onion/interactive/2020/09/08/us/elections/results-new-hampshire-primary-elections.html?action=click\&pgtype=Article\&state=default\&region=TOP_BANNER\&context=storylines_menu}{New
  Hampshire Results}
\item
  \href{https://www.nytimes3xbfgragh.onion/live/2020/09/08/us/trump-vs-biden?action=click\&pgtype=Article\&state=default\&region=TOP_BANNER\&context=storylines_menu}{Election
  Updates}
\item
  \href{https://www.nytimes3xbfgragh.onion/interactive/2020/us/elections/election-states-biden-trump.html?action=click\&pgtype=Article\&state=default\&region=TOP_BANNER\&context=storylines_menu}{Paths
  to 270}
\item
  \href{https://www.nytimes3xbfgragh.onion/interactive/2020/08/31/us/politics/vote-by-mail-deadlines.html?action=click\&pgtype=Article\&state=default\&region=TOP_BANNER\&context=storylines_menu}{Voting
  by Mail}
\item
  \href{https://www.nytimes3xbfgragh.onion/interactive/2019/us/elections/2020-presidential-election-calendar.html?action=click\&pgtype=Article\&state=default\&region=TOP_BANNER\&context=storylines_menu}{Key
  Dates}
\item
  \href{https://www.nytimes3xbfgragh.onion/newsletters/politics?action=click\&pgtype=Article\&state=default\&region=TOP_BANNER\&context=storylines_menu}{Politics
  Newsletter}
\end{itemize}

Advertisement

\protect\hyperlink{after-top}{Continue reading the main story}

Supported by

\protect\hyperlink{after-sponsor}{Continue reading the main story}

\hypertarget{the-lessons-of-the-elections-of-1918}{%
\section{The Lessons of the Elections of
1918}\label{the-lessons-of-the-elections-of-1918}}

A nation ravaged by the Spanish flu figured out how to vote back then.
Not without incident, but with democracy intact.

\includegraphics{https://static01.graylady3jvrrxbe.onion/images/2020/03/20/us/politics/20virus-election-spanishflu1/20virus-election-spanishflu1-articleLarge.jpg?quality=75\&auto=webp\&disable=upscale}

\href{https://www.nytimes3xbfgragh.onion/by/dionne-searcey}{\includegraphics{https://static01.graylady3jvrrxbe.onion/images/2018/10/15/multimedia/author-dionne-searcey/author-dionne-searcey-thumbLarge-v2.png}}

By \href{https://www.nytimes3xbfgragh.onion/by/dionne-searcey}{Dionne
Searcey}

\begin{itemize}
\item
  March 21, 2020
\item
  \begin{itemize}
  \item
  \item
  \item
  \item
  \item
  \end{itemize}
\end{itemize}

Across the country, citizens were ordered to hunker in their homes to
avoid catching a deadly virus even as some people thought it was nothing
worse than a seasonal cold. In the midst of fear and sickness,
politicians had to decide how to hold scheduled elections, and the
global pandemic was subject to political spin.

The year was 1918 when a deadly flu outbreak gripped the nation,
infecting about a third of the world's population and killing 675,000
people in the United States alone.

That crisis, which was known as the Spanish flu, took place in a
completely different time technologically and politically. But the
reaction then, where local governments took charge and made decisions on
how to proceed with voting, offer some guidance for the situation today
as the pandemic arrives in a federal election year.

In the 1918 election --- midterm contests, where President Woodrow
Wilson's Democratic Party was fighting to keep control of Congress ---
keeping polling places open was a patchwork of decisions by local
officials.

``Everything became this kind of wheeler-dealer hustle,'' said Kristin
Watkins, an expert in pandemics and director of grants at Pikes Peak
Community College in Colorado Springs whose studies involved reviewing
1918 elections.

Throughout the nation's history, wars, natural disasters and even
terrorist attacks have disrupted campaigns. This crisis seems different.
The enemy is invisible and comes as the country is politically divided,
with splits that are starting to seep into government's --- and
individuals' --- responses.

Congress has come together to support some bipartisan relief measures,
and Senate Republicans have proposed direct cash payments to some
Americans, an idea championed most recently by
\href{https://www.nytimes3xbfgragh.onion/2020/03/18/us/politics/universal-basic-income-andrew-yang.html}{former
Democratic presidential candidate Andrew Yang}. But President Trump's
labeling of the virus as the ``Wuhan virus'' or the ``Chinese virus''
has prompted accusations from Democrats that he is trying to pin the
blame for the outbreak on a rival power he has tangled with on trade and
other issues, in addition to cries that the label is racist.

Scholars and pollsters
\href{https://iai.tv/articles/republicans-assert-party-loyalty-by-ignoring-medical-advice-on-covid-19-auid-1380}{are
talking about} whether Democrats are washing their hands more than
Republicans, reflecting perhaps a partisan response to whether extra
hygiene is necessary or overkill. And right now in the U.S., actions as
simple and potentially poignant as whether to stay home sometimes are
being viewed as a partisan response.

``We're in a little bit of an unprecedented place,'' said Nancy
Martorano Miller, associate professor of political science at the
University of Dayton in Ohio, where the
\href{https://www.nytimes3xbfgragh.onion/2020/03/16/us/politics/virus-primary-2020-ohio.html}{Democratic
primary was postponed} at the last minute this week. ``The situation is
also moving very fast.''

On Monday, Ohio's governor and top state health officials ignored a
court ruling issued just hours earlier and postponed the state's primary
by declaring a public health emergency. Ohio's health director, Dr. Amy
Acton, had issued the order based on concerns that the coronavirus
outbreak put both voters and poll workers in potential danger.

\includegraphics{https://static01.graylady3jvrrxbe.onion/images/2020/03/22/us/politics/22virus-election-spanishflu2/merlin_170592924_1022fb4f-cb8c-4b9c-a032-d83e328abb45-articleLarge.jpg?quality=75\&auto=webp\&disable=upscale}

At least five states have delayed primaries, and Wisconsin, Pennsylvania
and others are in heated discussions about whether to do so. Many
state's primaries are scheduled for later this spring, around the time
when experts think novel coronavirus cases could peak in the United
States.

There's also the matter of political conventions, the events that bring
together thousands of party members for days of unity, rallying and
carousing to be capped off with iconic images of balloons dropping on
giddy delegates. Party officials are scrambling to come up with backup
plans in case the nominating conventions can't go on as normal.

In recent days, dozens of political scientists from universities across
the nation
\href{https://docs.google.com/forms/d/e/1FAIpQLSceYvQht71xf4Gb364HkqFXNKxD9R_KU0sTYerBIwK2gkfgsA/viewform}{signed
a letter} imploring government officials to use the next eight months to
ensure that polling in November goes smoothly by doing things like
expanding early voting and offering a universal vote-by-mail option.

``We must make sure that the election takes place this coming November,
and that it is a free, fair, and democratic election in which all
citizens have the chance to participate,'' read the group's statement.

Americans have faced major challenges during elections in the past.
Voting has taken place during wartime and in the aftermath of
hurricanes. This year a tornado struck parts of Tennessee on the morning
of its primary. Polls were allowed to stay open longer than normal.
Sept. 11, 2001, the day of the World Trade Center and Pentagon attacks,
was also Primary Day in New York. Voting was postponed two weeks.

Candidates also have made major changes to their campaigns in response
to unfolding events that affect the masses.

Former President Barack Obama and former Senator John McCain suspended
their campaigns and returned to Washington for bailout talks during the
2008 financial crisis. Last year, former Representative Beto O'Rourke
briefly suspended his campaign to return to El Paso after a deadly
shooting at Walmart there.

``This is decision-making in flux,'' Professor Martorano Miller said.

In 1918, midterm elections were playing out during a flu pandemic ---
and during World War I, adding extra heft to decisions that voters would
make at the polls. Some incumbents were criticized for leaving
Washington to campaign when important decisions were being made, so they
communicated with voters remotely, by writing letters and issuing news
releases.

One candidate campaigned by car, stopping the vehicle and having an aide
play a cornet to draw a crowd, until public gatherings were banned. At
the polls, workers in some places wore masks and voters spaced
themselves as they queued up.

Quarantines were in place in many areas, but the levels of social
distancing varied among communities. Trades were made between campaigns
and local government officials who opened polling places in exchange
for, say, allowing a play to be performed in front of a crowd, said Dr.
Watkins, the public health historian who studied pandemics.

Dr. Watkins said she is struck by similarities between the 1918 outbreak
and the current one. The shutdowns of businesses and gatherings. And the
way some government officials have warned people not to underestimate
the power of the virus. In 1918, they produced ads that featured Uncle
Sam, saying, ``Coughs and sneezes spread diseases, as dangerous as
poison gas shells.''

In
\href{https://digitalcommons.unmc.edu/cgi/viewcontent.cgi?referer=https://www.google.com/\&httpsredir=1\&article=1042\&context=etd}{her
research}, Dr. Watkins pored over old newspaper stories to study how
various communities dealt with the pandemic during 1918 midterms in
Nebraska, where worked at the University of Nebraska Medical Center,
which received some of the first Coronavirus cases from a cruise ship
and also treated Ebola patients after a West African outbreak.

In Wayne, Neb., a small community with an opera house and a teachers'
college back then, local newspapers were filled with obituaries. A sick
ward was set up at the school to handle 63 flu patients and students and
kitchen staff pitched in to help. Unfounded cures involving repeated
deep breathing circulated. Doctors and nurses reported being overworked.
Movie houses closed their doors and the state prohibited public
gatherings.

Dr. Watkins has written and starred in one-woman plays about Typhoid
Mary and the stigmatization of people placed in quarantine, performing
them for public health workers to help them understand ``how we judge
and how we point fingers,'' she said.

Image

A suffragist gave out bonbons to men at a polling place in New York on
Election Day in November 1918. Across the nation, women did not yet have
the right to vote.Credit...FPG/Archive Photos, via Getty Images

In early November 1918, the statewide ban on public gatherings was
lifted and politicians were allowed to campaign for five days before
polls were opened. Men --- women did not yet have the right to vote ---
filed in to cast ballots for a Senate seat, which the incumbent
Republican senator was able to hold on to.

Afterward, infections and deaths climbed, said Dr. Watkins.

``The disease appeared to be reaching a significant amount of the
population, greater than ever before; and the timing coincides with the
lifting of the quarantine,'' Dr. Watkins
\href{https://digitalcommons.unmc.edu/cgi/viewcontent.cgi?referer=https://www.google.com/\&httpsredir=1\&article=1042\&context=etd}{wrote
in her dissertation}, noting that ``the political machine disregarded
the health and safety of its citizens.''

That year, turnout across the nation was very low for the midterms, said
Julian E. Zelizer, a presidential historian at Princeton University.

That result --- low turnout, voters getting fatally ill --- is the worst
outcome for any election. To avoid such an outcome this year, many
political scientists and researchers are calling for more early and
absentee voting as well as the loosening of restrictions on showing
identification in person.

``Our main concern needs to be doing everything possible to increase
voting participation and eliminating barriers, especially given the
heath situation,'' said Professor Zelizer.

But when he thinks about 1918, the fact that elections were held at all,
he said, should offer optimism for the future.

``There have been moments like this but overall it's not as if the
system is suspended,'' he said. ``We have a pretty strong commitment to
moving through.''

\hypertarget{our-2020-election-guide}{%
\section{Our 2020 Election Guide}\label{our-2020-election-guide}}

Updated ~Sept. 8, 2020

\begin{center}\rule{0.5\linewidth}{\linethickness}\end{center}

\begin{itemize}
\item ~
  \hypertarget{the-latest}{%
  \subsection{The Latest}\label{the-latest}}

  \begin{itemize}
  \item
    President Trump and his party are using a playbook that aims to
    alarm people about crime in their backyards. It didn't work in 2018,
    but
    \href{https://www.nytimes3xbfgragh.onion/2020/09/08/us/politics/trump-republicans-fear-strategy.html?action=click\&pgtype=Article\&state=default\&region=BELOW_MAIN_CONTENT\&context=storylines_guide}{both
    parties think it could resonate more this year}.
  \end{itemize}
\item ~
  \hypertarget{how-to-win-270}{%
  \subsection{How to Win 270}\label{how-to-win-270}}

  \begin{itemize}
  \item
    Joe Biden and Donald Trump need 270 electoral votes to reach the
    White House. Try building
    \href{https://www.nytimes3xbfgragh.onion/interactive/2020/us/elections/election-states-biden-trump.html?action=click\&pgtype=Article\&state=default\&region=BELOW_MAIN_CONTENT\&context=storylines_guide}{your
    own coalition of battleground states}~to see potential outcomes.
  \end{itemize}
\item ~
  \hypertarget{voting-by-mail}{%
  \subsection{Voting by Mail}\label{voting-by-mail}}

  \begin{itemize}
  \item
    Will you have enough time to vote by mail in your state? Yes, but
    it's risky to procrastinate.
    \href{https://www.nytimes3xbfgragh.onion/interactive/2020/08/31/us/politics/vote-by-mail-deadlines.html?action=click\&pgtype=Article\&state=default\&region=BELOW_MAIN_CONTENT\&context=storylines_guide}{Check
    your state's deadline.}
  \item
    \href{https://www.nytimes3xbfgragh.onion/interactive/2020/us/elections/joe-biden.html?action=click\&pgtype=Article\&state=default\&region=BELOW_MAIN_CONTENT\&context=storylines_guide}{}

    \hypertarget{joe-biden}{%
    \section{Joe Biden}\label{joe-biden}}

    \hypertarget{democrat}{%
    \subsection{Democrat}\label{democrat}}

    \href{https://www.nytimes3xbfgragh.onion/interactive/2020/us/elections/donald-trump.html?action=click\&pgtype=Article\&state=default\&region=BELOW_MAIN_CONTENT\&context=storylines_guide}{}

    \hypertarget{donald-trump}{%
    \section{Donald Trump}\label{donald-trump}}

    \hypertarget{republican}{%
    \subsection{Republican}\label{republican}}
  \end{itemize}
\item
  \hypertarget{keep-up-with-our-coverage}{%
  \subsection{Keep Up With Our
  Coverage}\label{keep-up-with-our-coverage}}

  \begin{itemize}
  \item
    Get an
    \href{https://www.nytimes3xbfgragh.onion/newsletters/politics?action=click\&pgtype=Article\&state=default\&region=BELOW_MAIN_CONTENT\&context=storylines_guide}{email}~recapping
    the day's news
  \item
    Download our mobile app on
    \href{https://apps.apple.com/us/app/nytimes/id284862083?ls=1\&mat_click_id=5c79ae7455014fd1bd66b5610c05b8f2-20191112-16948\&referrer=mat_click_id\%3D5c79ae7455014fd1bd66b5610c05b8f2-20191112-16948\%26link_click_id\%3D722930677036718082}{iOS}~and
    \href{http://a.localytics.com/android?id=com.nytimes.android\&referrer=utm_source\%3Dother_nyt_mobile_web\%26utm_medium\%3DWeb\%2520page\%26utm_term\%3DGeneral\%2520Mobile\%2520Page\%26utm_campaign\%3DNYT\%2520Mobile\%2520General\%2520Page}{Android}~and
    turn on Breaking News and Politics alerts
  \end{itemize}
\end{itemize}

Advertisement

\protect\hyperlink{after-bottom}{Continue reading the main story}

\hypertarget{site-index}{%
\subsection{Site Index}\label{site-index}}

\hypertarget{site-information-navigation}{%
\subsection{Site Information
Navigation}\label{site-information-navigation}}

\begin{itemize}
\tightlist
\item
  \href{https://help.nytimes3xbfgragh.onion/hc/en-us/articles/115014792127-Copyright-notice}{©~2020~The
  New York Times Company}
\end{itemize}

\begin{itemize}
\tightlist
\item
  \href{https://www.nytco.com/}{NYTCo}
\item
  \href{https://help.nytimes3xbfgragh.onion/hc/en-us/articles/115015385887-Contact-Us}{Contact
  Us}
\item
  \href{https://www.nytco.com/careers/}{Work with us}
\item
  \href{https://nytmediakit.com/}{Advertise}
\item
  \href{http://www.tbrandstudio.com/}{T Brand Studio}
\item
  \href{https://www.nytimes3xbfgragh.onion/privacy/cookie-policy\#how-do-i-manage-trackers}{Your
  Ad Choices}
\item
  \href{https://www.nytimes3xbfgragh.onion/privacy}{Privacy}
\item
  \href{https://help.nytimes3xbfgragh.onion/hc/en-us/articles/115014893428-Terms-of-service}{Terms
  of Service}
\item
  \href{https://help.nytimes3xbfgragh.onion/hc/en-us/articles/115014893968-Terms-of-sale}{Terms
  of Sale}
\item
  \href{https://spiderbites.nytimes3xbfgragh.onion}{Site Map}
\item
  \href{https://help.nytimes3xbfgragh.onion/hc/en-us}{Help}
\item
  \href{https://www.nytimes3xbfgragh.onion/subscription?campaignId=37WXW}{Subscriptions}
\end{itemize}
