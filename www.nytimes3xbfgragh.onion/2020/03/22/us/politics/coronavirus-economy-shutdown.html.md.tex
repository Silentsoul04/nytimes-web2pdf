Sections

SEARCH

\protect\hyperlink{site-content}{Skip to
content}\protect\hyperlink{site-index}{Skip to site index}

\href{https://www.nytimes3xbfgragh.onion/section/politics}{Politics}

\href{https://myaccount.nytimes3xbfgragh.onion/auth/login?response_type=cookie\&client_id=vi}{}

\href{https://www.nytimes3xbfgragh.onion/section/todayspaper}{Today's
Paper}

\href{/section/politics}{Politics}\textbar{}The U.S. Shut Down Its
Economy. Here's What Needs to Happen in Order to Restart.

\url{https://nyti.ms/3bbYofw}

\begin{itemize}
\item
\item
\item
\item
\item
\end{itemize}

\hypertarget{the-coronavirus-outbreak}{%
\subsubsection{\texorpdfstring{\href{https://www.nytimes3xbfgragh.onion/news-event/coronavirus?name=styln-coronavirus-national\&region=TOP_BANNER\&block=storyline_menu_recirc\&action=click\&pgtype=Article\&impression_id=59f65b60-f27e-11ea-879c-3d80aa080e10\&variant=undefined}{The
Coronavirus
Outbreak}}{The Coronavirus Outbreak}}\label{the-coronavirus-outbreak}}

\begin{itemize}
\tightlist
\item
  live\href{https://www.nytimes3xbfgragh.onion/2020/09/08/world/covid-19-coronavirus.html?name=styln-coronavirus-national\&region=TOP_BANNER\&block=storyline_menu_recirc\&action=click\&pgtype=Article\&impression_id=59f68270-f27e-11ea-879c-3d80aa080e10\&variant=undefined}{Latest
  Updates}
\item
  \href{https://www.nytimes3xbfgragh.onion/interactive/2020/us/coronavirus-us-cases.html?name=styln-coronavirus-national\&region=TOP_BANNER\&block=storyline_menu_recirc\&action=click\&pgtype=Article\&impression_id=59f68271-f27e-11ea-879c-3d80aa080e10\&variant=undefined}{Maps
  and Cases}
\item
  \href{https://www.nytimes3xbfgragh.onion/interactive/2020/science/coronavirus-vaccine-tracker.html?name=styln-coronavirus-national\&region=TOP_BANNER\&block=storyline_menu_recirc\&action=click\&pgtype=Article\&impression_id=59f68272-f27e-11ea-879c-3d80aa080e10\&variant=undefined}{Vaccine
  Tracker}
\item
  \href{https://www.nytimes3xbfgragh.onion/2020/09/02/your-money/eviction-moratorium-covid.html?name=styln-coronavirus-national\&region=TOP_BANNER\&block=storyline_menu_recirc\&action=click\&pgtype=Article\&impression_id=59f68273-f27e-11ea-879c-3d80aa080e10\&variant=undefined}{Eviction
  Moratorium}
\item
  \href{https://www.nytimes3xbfgragh.onion/interactive/2020/09/02/magazine/food-insecurity-hunger-us.html?name=styln-coronavirus-national\&region=TOP_BANNER\&block=storyline_menu_recirc\&action=click\&pgtype=Article\&impression_id=59f6a980-f27e-11ea-879c-3d80aa080e10\&variant=undefined}{American
  Hunger}
\end{itemize}

Advertisement

\protect\hyperlink{after-top}{Continue reading the main story}

Supported by

\protect\hyperlink{after-sponsor}{Continue reading the main story}

\hypertarget{the-us-shut-down-its-economy-heres-what-needs-to-happen-in-order-to-restart}{%
\section{The U.S. Shut Down Its Economy. Here's What Needs to Happen in
Order to
Restart.}\label{the-us-shut-down-its-economy-heres-what-needs-to-happen-in-order-to-restart}}

Whole sectors of the United States economy have gone dark to slow the
spread of the coronavirus. Here's what comes next.

\includegraphics{https://static01.graylady3jvrrxbe.onion/images/2020/03/20/business/20DC-VIRUS-ECON-01/merlin_170747688_65eb91ab-bd9f-4f1e-b0c3-549b48a029e8-articleLarge.jpg?quality=75\&auto=webp\&disable=upscale}

\href{https://www.nytimes3xbfgragh.onion/by/jim-tankersley}{\includegraphics{https://static01.graylady3jvrrxbe.onion/images/2018/10/19/multimedia/author-jim-tankersley/author-jim-tankersley-thumbLarge.png}}

By \href{https://www.nytimes3xbfgragh.onion/by/jim-tankersley}{Jim
Tankersley}

\begin{itemize}
\item
  March 22, 2020
\item
  \begin{itemize}
  \item
  \item
  \item
  \item
  \item
  \end{itemize}
\end{itemize}

WASHINGTON --- The American economy has stopped working.

We're going to try turning it off and back on again.

With confirmed
\href{https://www.nytimes3xbfgragh.onion/interactive/2020/world/coronavirus-maps.html}{cases
of the coronavirus} escalating rapidly, government officials have almost
overnight switched off activity in large sectors of the United States.
They want as few people as possible in close contact with one another in
order to slow
\href{https://www.nytimes3xbfgragh.onion/news-event/coronavirus}{the
pandemic}, which may be even more widespread than official statistics
suggest.

The federal government has discouraged gatherings of 10 or more people.
\href{https://www.nytimes3xbfgragh.onion/2020/03/19/us/California-stay-at-home-order-virus.html}{California
told 40 million residents} to leave the house only for absolute
necessities. Bars, shopping malls, dine-in restaurants and a host of
other businesses are closing across the country. Millions of people have
been
\href{https://www.nytimes3xbfgragh.onion/2020/03/19/business/economy/coronavirus-employers-unemployment.html}{laid
off}, or are about to be.

Just as there is a public health strategy driving the government orders
closing businesses and limiting daily activity outside the home, there
is also an economic strategy for putting large parts of the economy on
ice. It requires aggressive action by the federal government, funded by
what would be the most expansive borrowing the country has seen since
World War II.

Whether the United States looks back at those job cuts as a quick blip
of prevention or a devastating spiral into an economic depression
depends a lot on what Congress and President Trump do in the next few
days.

Here's what economists say needs to happen.

\hypertarget{aim-to-put-the-v-in-recovery}{%
\subsection{Aim to put the `v' in
`recovery.'}\label{aim-to-put-the-v-in-recovery}}

The United States is already falling into a
\href{https://www.nytimes3xbfgragh.onion/2020/03/21/business/economy/coronavirus-recession.html}{sharp
economic contraction}: It is producing far fewer goods and services now
than it did a month or a quarter ago. That contraction will persist as
long as
\href{https://www.nytimes3xbfgragh.onion/2020/03/15/business/economy/coronavirus-economy-impact.html}{businesses
are unable to open and people are not able to work}. This is not
happening because of any choices those workers or businesses made; it's
a mandate from the government that has frozen a lot of economic
activity.

At some point --- possibly when a vaccine for the virus comes to market,
or possibly as soon as the rate of infection starts declining and
widespread testing allows for more confidence that another surge is not
imminent --- governments will lift their restrictions and activity will
start to thaw.

Ideally, it would thaw quickly, with shops and restaurants reopening,
workers rehired, factory production lines restarted and people spending
money on things they didn't need or couldn't buy during the freeze. In
that situation, the economy would grow much faster for a while than it
normally does, as consumers unleash their pent-up demand.

Economists call that a ``V-shaped'' recovery, because growth plunges and
then shoots up. It's what they're aiming for now, but it could be hard
to pull off.

``What a recession from something like this should look like is a sudden
stop and recovery,'' said R. Glenn Hubbard, a Columbia University
economist who was a top White House economist for President George W.
Bush. ``What could happen, though, is a doom loop.''

\includegraphics{https://static01.graylady3jvrrxbe.onion/images/2020/03/20/business/20DC-VIRUS-ECON-02/merlin_170656824_7ae71916-c7f9-4581-a0b7-ba4eb5ed84c2-articleLarge.jpg?quality=75\&auto=webp\&disable=upscale}

\hypertarget{extend-companies-a-very-large-lifeline}{%
\subsection{Extend companies a very large
lifeline.}\label{extend-companies-a-very-large-lifeline}}

The ``doom loop'' that Mr. Hubbard and many other economists fear
describes a situation in which an even moderately protracted shutdown of
economic activity permanently kills waves of small businesses --- and
possibly entire industries, like airlines --- that cannot survive very
long without customers.

A typical small business in the United States does not have enough cash
on hand to cover even a month of expenses if its revenues are completely
disrupted, according to
\href{https://institute.jpmorganchase.com/content/dam/jpmc/jpmorgan-chase-and-co/institute/pdf/institute-growth-vitality-cash-flows.pdf}{research
by the JPMorgan Chase Institute}. In minority communities, where profit
margins are often narrower, the typical cash reserve is even smaller.

\hypertarget{latest-updates-the-coronavirus-outbreak}{%
\section{\texorpdfstring{\href{https://www.nytimes3xbfgragh.onion/2020/09/08/world/covid-19-coronavirus.html?action=click\&pgtype=Article\&state=default\&region=MAIN_CONTENT_1\&context=storylines_live_updates}{Latest
Updates: The Coronavirus
Outbreak}}{Latest Updates: The Coronavirus Outbreak}}\label{latest-updates-the-coronavirus-outbreak}}

Updated 2020-09-09T08:22:37.235Z

\begin{itemize}
\tightlist
\item
  \href{https://www.nytimes3xbfgragh.onion/2020/09/08/world/covid-19-coronavirus.html?action=click\&pgtype=Article\&state=default\&region=MAIN_CONTENT_1\&context=storylines_live_updates\#link-313b443d}{AstraZeneca
  halts a vaccine trial to investigate a participant's illness.}
\item
  \href{https://www.nytimes3xbfgragh.onion/2020/09/08/world/covid-19-coronavirus.html?action=click\&pgtype=Article\&state=default\&region=MAIN_CONTENT_1\&context=storylines_live_updates\#link-4438dd7}{Facing
  a surge in cases, Britain plans to limit most gatherings to six
  people.}
\item
  \href{https://www.nytimes3xbfgragh.onion/2020/09/08/world/covid-19-coronavirus.html?action=click\&pgtype=Article\&state=default\&region=MAIN_CONTENT_1\&context=storylines_live_updates\#link-679303d7}{Nine
  drugmakers pledge to thoroughly vet any coronavirus vaccine.}
\end{itemize}

\href{https://www.nytimes3xbfgragh.onion/2020/09/08/world/covid-19-coronavirus.html?action=click\&pgtype=Article\&state=default\&region=MAIN_CONTENT_1\&context=storylines_live_updates}{See
more updates}

More live coverage:
\href{https://www.nytimes3xbfgragh.onion/live/2020/09/08/business/stock-market-today-coronavirus?action=click\&pgtype=Article\&state=default\&region=MAIN_CONTENT_1\&context=storylines_live_updates}{Markets}

Economists say that means Congress needs to act boldly, and fast, to
keep money flowing to business owners to ensure they can reopen when the
crisis abates.

There are several possible ways to try to do that.

Steven Hamilton, an economist at George Washington University who has
been one of the loudest public voices calling for aggressive assistance
to small businesses, and Stan Veuger of the American Enterprise
Institute,
\href{https://thedispatch.com/p/any-stimulus-plan-must-help-small}{want
banks to offer loans} to cover lost revenues for small businesses ---
and for the federal government to forgive the loans if the companies
don't lay off workers. Mr. Hubbard and Michael R. Strain of the American
Enterprise Institute have a similar proposal.

Adam Ozimek, the chief economist at Upwork, and John Lettieri, the
president of the Economic Innovation Group in Washington,
\href{https://eig.org/news/main-street-rescue-and-resiliency-program}{want
the government to guarantee loans} with little or no interest that small
businesses would pay back over a long period, regardless of whether they
lay off workers. Mr. Ozimek said it would be wrong not to help companies
that have already been forced into layoffs by government decisions and
delays in a federal response.

``When the government is this late to the party,'' he said, ``they
shouldn't punish small businesses who acted fast.''

Economists stress that a successful program would be expensive: \$1
trillion or more. Mr. Hubbard said a \$300 billion loan program,
\href{https://www.nytimes3xbfgragh.onion/2020/03/19/us/politics/1000-checks-coronavirus-stimulus.html}{as
Senate Republicans proposed} on Thursday, would be ``woefully
inadequate.''

Mr. Hamilton said this week that he worried members of Congress had
``not come to terms with the scale'' of what was needed. ``Any fiscal
package less than \$1.5 trillion will be inadequate,'' he said, ``and
frankly lead to a Great Depression-level economic collapse.''

\hypertarget{provide-big-relief-for-workers}{%
\subsection{Provide big relief for
workers.}\label{provide-big-relief-for-workers}}

Companies are only half the equation. For the shutdown/restart strategy
to work, economists say, lawmakers must also keep money flowing to
workers affected by the economic chill so they can continue to buy
groceries, pay mortgage or rent and seek medical care if they are
injured or sick.

One way to do that is by helping businesses --- and hopefully keeping as
many people as possible on payrolls, even if they are not working. But
workers who lose jobs or hours will need more direct help.

\href{https://www.nytimes3xbfgragh.onion/news-event/coronavirus?action=click\&pgtype=Article\&state=default\&region=MAIN_CONTENT_3\&context=storylines_faq}{}

\hypertarget{the-coronavirus-outbreak-}{%
\subsubsection{The Coronavirus Outbreak
›}\label{the-coronavirus-outbreak-}}

\hypertarget{frequently-asked-questions}{%
\paragraph{Frequently Asked
Questions}\label{frequently-asked-questions}}

Updated September 4, 2020

\begin{itemize}
\item ~
  \hypertarget{what-are-the-symptoms-of-coronavirus}{%
  \paragraph{What are the symptoms of
  coronavirus?}\label{what-are-the-symptoms-of-coronavirus}}

  \begin{itemize}
  \tightlist
  \item
    In the beginning, the coronavirus
    \href{https://www.nytimes3xbfgragh.onion/article/coronavirus-facts-history.html?action=click\&pgtype=Article\&state=default\&region=MAIN_CONTENT_3\&context=storylines_faq\#link-6817bab5}{seemed
    like it was primarily a respiratory illness}~--- many patients had
    fever and chills, were weak and tired, and coughed a lot, though
    some people don't show many symptoms at all. Those who seemed
    sickest had pneumonia or acute respiratory distress syndrome and
    received supplemental oxygen. By now, doctors have identified many
    more symptoms and syndromes. In April,
    \href{https://www.nytimes3xbfgragh.onion/2020/04/27/health/coronavirus-symptoms-cdc.html?action=click\&pgtype=Article\&state=default\&region=MAIN_CONTENT_3\&context=storylines_faq}{the
    C.D.C. added to the list of early signs}~sore throat, fever, chills
    and muscle aches. Gastrointestinal upset, such as diarrhea and
    nausea, has also been observed. Another telltale sign of infection
    may be a sudden, profound diminution of one's
    \href{https://www.nytimes3xbfgragh.onion/2020/03/22/health/coronavirus-symptoms-smell-taste.html?action=click\&pgtype=Article\&state=default\&region=MAIN_CONTENT_3\&context=storylines_faq}{sense
    of smell and taste.}~Teenagers and young adults in some cases have
    developed painful red and purple lesions on their fingers and toes
    --- nicknamed ``Covid toe'' --- but few other serious symptoms.
  \end{itemize}
\item ~
  \hypertarget{why-is-it-safer-to-spend-time-together-outside}{%
  \paragraph{Why is it safer to spend time together
  outside?}\label{why-is-it-safer-to-spend-time-together-outside}}

  \begin{itemize}
  \tightlist
  \item
    \href{https://www.nytimes3xbfgragh.onion/2020/05/15/us/coronavirus-what-to-do-outside.html?action=click\&pgtype=Article\&state=default\&region=MAIN_CONTENT_3\&context=storylines_faq}{Outdoor
    gatherings}~lower risk because wind disperses viral droplets, and
    sunlight can kill some of the virus. Open spaces prevent the virus
    from building up in concentrated amounts and being inhaled, which
    can happen when infected people exhale in a confined space for long
    stretches of time, said Dr. Julian W. Tang, a virologist at the
    University of Leicester.
  \end{itemize}
\item ~
  \hypertarget{why-does-standing-six-feet-away-from-others-help}{%
  \paragraph{Why does standing six feet away from others
  help?}\label{why-does-standing-six-feet-away-from-others-help}}

  \begin{itemize}
  \tightlist
  \item
    The coronavirus spreads primarily through droplets from your mouth
    and nose, especially when you cough or sneeze. The C.D.C., one of
    the organizations using that measure,
    \href{https://www.nytimes3xbfgragh.onion/2020/04/14/health/coronavirus-six-feet.html?action=click\&pgtype=Article\&state=default\&region=MAIN_CONTENT_3\&context=storylines_faq}{bases
    its recommendation of six feet}~on the idea that most large droplets
    that people expel when they cough or sneeze will fall to the ground
    within six feet. But six feet has never been a magic number that
    guarantees complete protection. Sneezes, for instance, can launch
    droplets a lot farther than six feet,
    \href{https://jamanetwork.com/journals/jama/fullarticle/2763852}{according
    to a recent study}. It's a rule of thumb: You should be safest
    standing six feet apart outside, especially when it's windy. But
    keep a mask on at all times, even when you think you're far enough
    apart.
  \end{itemize}
\item ~
  \hypertarget{i-have-antibodies-am-i-now-immune}{%
  \paragraph{I have antibodies. Am I now
  immune?}\label{i-have-antibodies-am-i-now-immune}}

  \begin{itemize}
  \tightlist
  \item
    As of right
    now,\href{https://www.nytimes3xbfgragh.onion/2020/07/22/health/covid-antibodies-herd-immunity.html?action=click\&pgtype=Article\&state=default\&region=MAIN_CONTENT_3\&context=storylines_faq}{~that
    seems likely, for at least several months.}~There have been
    frightening accounts of people suffering what seems to be a second
    bout of Covid-19. But experts say these patients may have a
    drawn-out course of infection, with the virus taking a slow toll
    weeks to months after initial exposure.~People infected with the
    coronavirus typically
    \href{https://www.nature.com/articles/s41586-020-2456-9}{produce}~immune
    molecules called antibodies, which are
    \href{https://www.nytimes3xbfgragh.onion/2020/05/07/health/coronavirus-antibody-prevalence.html?action=click\&pgtype=Article\&state=default\&region=MAIN_CONTENT_3\&context=storylines_faq}{protective
    proteins made in response to an
    infection}\href{https://www.nytimes3xbfgragh.onion/2020/05/07/health/coronavirus-antibody-prevalence.html?action=click\&pgtype=Article\&state=default\&region=MAIN_CONTENT_3\&context=storylines_faq}{.
    These antibodies may}~last in the body
    \href{https://www.nature.com/articles/s41591-020-0965-6}{only two to
    three months}, which may seem worrisome, but that's~perfectly normal
    after an acute infection subsides, said Dr. Michael Mina, an
    immunologist at Harvard University. It may be possible to get the
    coronavirus again, but it's highly unlikely that it would be
    possible in a short window of time from initial infection or make
    people sicker the second time.
  \end{itemize}
\item ~
  \hypertarget{what-are-my-rights-if-i-am-worried-about-going-back-to-work}{%
  \paragraph{What are my rights if I am worried about going back to
  work?}\label{what-are-my-rights-if-i-am-worried-about-going-back-to-work}}

  \begin{itemize}
  \tightlist
  \item
    Employers have to provide
    \href{https://www.osha.gov/SLTC/covid-19/standards.html}{a safe
    workplace}~with policies that protect everyone equally.
    \href{https://www.nytimes3xbfgragh.onion/article/coronavirus-money-unemployment.html?action=click\&pgtype=Article\&state=default\&region=MAIN_CONTENT_3\&context=storylines_faq}{And
    if one of your co-workers tests positive for the coronavirus, the
    C.D.C.}~has said that
    \href{https://www.cdc.gov/coronavirus/2019-ncov/community/guidance-business-response.html}{employers
    should tell their employees}~-\/- without giving you the sick
    employee's name -\/- that they may have been exposed to the virus.
  \end{itemize}
\end{itemize}

Many economists, including Claudia Sahm of the Washington Center for
Equitable Growth and N. Gregory Mankiw and Jason Furman of Harvard
University, have called on lawmakers to send checks of \$1,000 or more
to all Americans as quickly as possible. Both Mr. Trump and Steven
Mnuchin, the Treasury secretary, have voiced support for such payments.
At least a scaled-back version of that plan is likely to be included in
the stimulus bill being negotiated in Congress, with payments headed to
low- and middle-income families.

But those payments will not be sufficient to cover costs of necessities
for people who have suddenly seen their incomes shrink or vanish.

For that, economists say, Congress needs
\href{https://www.nytimes3xbfgragh.onion/2020/03/17/us/politics/congress-coronavirus-economy.html}{an
emergency safety net} that would sustain income for all workers who are
laid off or otherwise hurt by the effects of the virus. Ideally, the
economists say, that safety net would aid the public health strategy
embedded in the economic shutdowns by paying as many people as possible
to stay home from work and practice social distancing.

That could mean increased unemployment benefits and more generous paid
sick leave fully funded by the government. It could also mean something
like what the British government announced Friday: a plan to encourage
businesses to keep paying workers by assuming up to 80 percent of their
wage costs.

``You need an unemployment system that in this moment can pivot to 100
percent reimbursement to 100 percent of people who are not at work,''
said Heather Boushey, the president of the Washington Center for
Equitable Growth, a think tank focused on inequality and growth. ``And
wage replacement for people who have hours cut back.''

\hypertarget{dont-think-of-this-as-a-normal-rescue}{%
\subsection{Don't think of this as a `normal'
rescue.}\label{dont-think-of-this-as-a-normal-rescue}}

Americans also need to start thinking of this crisis as different from
almost any economic shock before it.

Concerns that have guided economists in the past, like whether policies
discourage people from working, do not apply in the same way now: It is
hard to discourage work in sectors that the government has ordered to
shut down. The same may hold for restrictions that some lawmakers want
to place on spending any government aid to business, like limiting
grants to businesses that keep all their workers on salary, Mr. Lettieri
and others say.

Joseph S. Vavra, an economist at the University of Chicago Booth School
of Business, said that policymakers typically try to stimulate consumer
demand during a recession and start recovery as quickly as possible.
Right now, the goal is almost the opposite.

``I don't think what we're trying to do is to get people to go out and
shop,'' he said. ``What we're trying to do is provide some assistance to
households so they can sit at home and don't have to go out and shop.''

The better parallel here might be World War II. The government is trying
to win a fight against a pandemic. That will mean spending big on the
battlefront --- public health efforts, like ventilators and masks ---
and asking personal and economic sacrifices from individuals to defeat
the virus and get life back to normal again.

Ben Casselman and Neil Irwin contributed reporting.

Advertisement

\protect\hyperlink{after-bottom}{Continue reading the main story}

\hypertarget{site-index}{%
\subsection{Site Index}\label{site-index}}

\hypertarget{site-information-navigation}{%
\subsection{Site Information
Navigation}\label{site-information-navigation}}

\begin{itemize}
\tightlist
\item
  \href{https://help.nytimes3xbfgragh.onion/hc/en-us/articles/115014792127-Copyright-notice}{©~2020~The
  New York Times Company}
\end{itemize}

\begin{itemize}
\tightlist
\item
  \href{https://www.nytco.com/}{NYTCo}
\item
  \href{https://help.nytimes3xbfgragh.onion/hc/en-us/articles/115015385887-Contact-Us}{Contact
  Us}
\item
  \href{https://www.nytco.com/careers/}{Work with us}
\item
  \href{https://nytmediakit.com/}{Advertise}
\item
  \href{http://www.tbrandstudio.com/}{T Brand Studio}
\item
  \href{https://www.nytimes3xbfgragh.onion/privacy/cookie-policy\#how-do-i-manage-trackers}{Your
  Ad Choices}
\item
  \href{https://www.nytimes3xbfgragh.onion/privacy}{Privacy}
\item
  \href{https://help.nytimes3xbfgragh.onion/hc/en-us/articles/115014893428-Terms-of-service}{Terms
  of Service}
\item
  \href{https://help.nytimes3xbfgragh.onion/hc/en-us/articles/115014893968-Terms-of-sale}{Terms
  of Sale}
\item
  \href{https://spiderbites.nytimes3xbfgragh.onion}{Site Map}
\item
  \href{https://help.nytimes3xbfgragh.onion/hc/en-us}{Help}
\item
  \href{https://www.nytimes3xbfgragh.onion/subscription?campaignId=37WXW}{Subscriptions}
\end{itemize}
