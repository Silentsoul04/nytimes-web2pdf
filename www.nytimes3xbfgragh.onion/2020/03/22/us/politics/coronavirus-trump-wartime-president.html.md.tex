Sections

SEARCH

\protect\hyperlink{site-content}{Skip to
content}\protect\hyperlink{site-index}{Skip to site index}

\href{https://www.nytimes3xbfgragh.onion/section/politics}{Politics}

\href{https://myaccount.nytimes3xbfgragh.onion/auth/login?response_type=cookie\&client_id=vi}{}

\href{https://www.nytimes3xbfgragh.onion/section/todayspaper}{Today's
Paper}

\href{/section/politics}{Politics}\textbar{}`Wartime President'? Trump
Rewrites History in an Election Year

\url{https://nyti.ms/3dfdUJo}

\begin{itemize}
\item
\item
\item
\item
\item
\item
\end{itemize}

\begin{itemize}
\item
  \href{https://www.nytimes3xbfgragh.onion/live/2020/09/09/us/trump-vs-biden?action=click\&pgtype=Article\&state=default\&region=TOP_BANNER\&context=storylines_menu}{Election
  Updates}
\item
  \href{https://www.nytimes3xbfgragh.onion/interactive/2020/09/08/us/elections/results-new-hampshire-primary-elections.html?action=click\&pgtype=Article\&state=default\&region=TOP_BANNER\&context=storylines_menu}{New
  Hampshire Results}
\item
  \href{https://www.nytimes3xbfgragh.onion/interactive/2020/us/elections/election-states-biden-trump.html?action=click\&pgtype=Article\&state=default\&region=TOP_BANNER\&context=storylines_menu}{Paths
  to 270}
\item
  \href{https://www.nytimes3xbfgragh.onion/interactive/2020/08/31/us/politics/vote-by-mail-deadlines.html?action=click\&pgtype=Article\&state=default\&region=TOP_BANNER\&context=storylines_menu}{Voting
  by Mail}
\item
  \href{https://www.nytimes3xbfgragh.onion/interactive/2019/us/elections/2020-presidential-election-calendar.html?action=click\&pgtype=Article\&state=default\&region=TOP_BANNER\&context=storylines_menu}{Key
  Dates}
\item
  \href{https://www.nytimes3xbfgragh.onion/newsletters/politics?action=click\&pgtype=Article\&state=default\&region=TOP_BANNER\&context=storylines_menu}{Politics
  Newsletter}
\end{itemize}

Advertisement

\protect\hyperlink{after-top}{Continue reading the main story}

Supported by

\protect\hyperlink{after-sponsor}{Continue reading the main story}

News Analysis

\hypertarget{wartime-president-trump-rewrites-history-in-an-election-year}{%
\section{`Wartime President'? Trump Rewrites History in an Election
Year}\label{wartime-president-trump-rewrites-history-in-an-election-year}}

The president is brazenly grabbing his only clear option to bolster his
re-election hopes, portraying himself as a take-charge leader the
country can't afford to lose.

\includegraphics{https://static01.graylady3jvrrxbe.onion/images/2020/03/22/us/politics/22trump-wartime-03/22trump-wartime-03-articleLarge-v3.jpg?quality=75\&auto=webp\&disable=upscale}

\href{https://www.nytimes3xbfgragh.onion/by/annie-karni}{\includegraphics{https://static01.graylady3jvrrxbe.onion/images/2019/02/05/multimedia/author-annie-karni/author-annie-karni-thumbLarge.png}}\href{https://www.nytimes3xbfgragh.onion/by/maggie-haberman}{\includegraphics{https://static01.graylady3jvrrxbe.onion/images/2018/07/12/multimedia/author-maggie-haberman/author-maggie-haberman-thumbLarge.png}}\href{https://www.nytimes3xbfgragh.onion/by/reid-j-epstein}{\includegraphics{https://static01.graylady3jvrrxbe.onion/images/2019/06/25/reader-center/author-reid-epstein/9e877853d8234217b58e5762253aa771-thumbLarge.png}}

By \href{https://www.nytimes3xbfgragh.onion/by/annie-karni}{Annie
Karni},
\href{https://www.nytimes3xbfgragh.onion/by/maggie-haberman}{Maggie
Haberman} and
\href{https://www.nytimes3xbfgragh.onion/by/reid-j-epstein}{Reid J.
Epstein}

\begin{itemize}
\item
  Published March 22, 2020Updated March 25, 2020
\item
  \begin{itemize}
  \item
  \item
  \item
  \item
  \item
  \item
  \end{itemize}
\end{itemize}

WASHINGTON --- With the economy faltering and the political landscape
unsettled as the
\href{https://www.nytimes3xbfgragh.onion/2020/03/25/podcasts/the-daily/trump-coronavirus.html?action=click\&module=Briefings\&pgtype=Homepage}{coronavirus
death toll} climbs, a stark and unavoidable question now confronts
President Trump and his advisers: Can he save his campaign for
re-election when so much is suddenly going so wrong?

After three years of Republicans' championing signs of financial
prosperity that were to be Mr. Trump's chief re-election argument, the
president has never needed a new message to voters as he does now, not
to mention luck. At this point, the president has one clear option for
how to proceed politically, and is hoping that an array of factors will
break his way.

The option, which he has brazenly pushed in recent days, is to cast
himself as a ``wartime president'' who looks in charge of a nation under
siege while his likely Democratic opponent, former Vice President
\href{https://www.nytimes3xbfgragh.onion/interactive/2020/us/elections/joe-biden.html}{Joseph
R. Biden Jr.}, is largely out of sight hunkered down in Delaware. This
gambit, however, requires a rewriting of history --- Mr. Trump's muted
approach to the virus early on --- and it's far from clear if many
voters will accept the idea of him as a wartime leader.

Then there are other variables that he and his allies hope will fall in
their favor: that the outbreak of the virus will slow and,
\href{https://www.nytimes3xbfgragh.onion/2020/03/22/health/warm-weather-coronavirus.html}{in
the warmer months}, dissipate; that the states will get it under
control; that the federal government's steps taken so far will
\href{https://www.nytimes3xbfgragh.onion/interactive/2020/03/19/world/coronavirus-flatten-the-curve-countries.html}{flatten
the curve}; that Mr. Biden and the Democrats will look impotent and
inconsequential by comparison; and that enough voters will move past his
initial efforts to
\href{https://www.nytimes3xbfgragh.onion/interactive/2020/03/18/us/trump-coronavirus-statements-timeline.html}{play
down the virus's dangers}.

The great unknown, of course --- and the tremendous risk to Mr. Trump's
political fate, no matter what he says or does --- is that the human
cost, the economic toll, and the longevity and course of the pandemic
are all X factors that will most likely play out for months and could be
strongly salient if not severe by the time of the November general
election.

In perhaps the best-case scenario for Mr. Trump, the patina of a
``wartime president'' could prove to be influential with casual voters
who don't dig into the details of his belated response to the
coronavirus, which included dismissing the criticism of his handling of
the threat as a Democratic ``hoax'' and contributing to a slow start in
testing for the virus.

``He is counting on people being so traumatized on a day-to-day basis
that they will forget his inaction,'' said Douglas Brinkley, a professor
of history at Rice University. ``It's better for him to be a wartime
commander in chief than someone who, when the big crisis hit, misread it
completely.''

Politically, Mr. Brinkley said, the new posture made sense. ``He can
claim credit for the curve flattening at some point,'' he said, ``and
hope people will be afraid to push a leader out of office if the crisis
pushes into the fall.''

The president's course correction showed some quick results that were
seized on by his political advisers. An
\href{https://www.ipsos.com/en-us/news-polls/abc-news-coronavirus-poll}{ABC
News poll released last week} showed that 55 percent of Americans
approved of Mr. Trump's response to the pandemic, up from 43 percent the
week before.

Rarely have incumbent presidents seen their arguments for re-election
evaporate so quickly. Mr. Trump and his advisers had planned to argue
that the strong economy warranted a second term and that supporters and
detractors alike wanted their 401(k)'s in the Trump-era stock market;
that has now become an impossible sell. And as the administration
negotiates an enormous bailout package with Congress for multiple
industries, his strategy of caricaturing Democrats as ``socialists'' is
not tenable either.

So Mr. Trump is trying to mount a new version of his old argument: that
``I alone can fix it,'' as he said at the 2016 Republican convention
about the nation's problems. He is counting on enough voters believing
they have to stick by a leader in the midst of an immense global crisis.

Addressing fearful Americans, Mr. Trump said on Sunday evening: ``You
have a leader that will always fight for you and I will not stop until
we win. This is going to be a victory.'' He added at another point, ``No
American is alone as long as we're united.''

But shortly after reading his new script on Sunday night, Mr. Trump
mocked Senator Mitt Romney of Utah for self-quarantining. ``Romney's in
isolation? Gee, that's too bad,'' he said sarcastically at the briefing
room podium.

Mr. Trump is on uncertain political ground. His poll numbers in critical
swing states like Pennsylvania and Michigan have been wavering, with
most surveys showing him behind Mr. Biden and Senator
\href{https://www.nytimes3xbfgragh.onion/interactive/2020/us/elections/bernie-sanders.html}{Bernie
Sanders} of Vermont.

Republican officials are banking on voters seeing him as take-charge in
contrast to Mr. Biden and Mr. Sanders, who have been following the
government's guidance about staying indoors but have not yet found
memorable ways to show how they would lead in this crisis.

Mr. Biden and Mr. Sanders have held news conferences from remote
locations, and Mr. Biden has participated in ``virtual'' fund-raisers.
While Mr. Biden has
\href{https://www.nytimes3xbfgragh.onion/interactive/2020/us/elections/delegate-count-primary-results.html}{an
all but insurmountable lead} in the race for the Democratic nomination,
he has no real ability to steer events because he is not officially the
presumptive nominee, and therefore is not the head of the party either.

As a result, Mr. Biden finds himself with far fewer options than Mr.
Trump or
\href{https://www.nytimes3xbfgragh.onion/2020/03/17/us/politics/governors-coronavirus-trump.html}{Democratic
governors like Andrew M. Cuomo} of New York, who have real authority and
are holding news conferences often broadcast live.

In turn, Republican officials and Sanders allies are pushing out attacks
on Mr. Biden's low profile, asking ``Where's Joe?'' in emails to
supporters and the news media. (A spokesman, T.J. Ducklo, said Sunday
that Mr. Biden had not been tested for the virus because he had shown no
symptoms.)

\includegraphics{https://static01.graylady3jvrrxbe.onion/images/2017/01/29/podcasts/the-daily-album-art/the-daily-album-art-articleInline-v2.jpg?quality=75\&auto=webp\&disable=upscale}

\hypertarget{listen-to-the-daily-raring-to-go-by-easter}{%
\subsubsection{Listen to `The Daily': `Raring to Go by
Easter'}\label{listen-to-the-daily-raring-to-go-by-easter}}

President Trump said he would like to allow shuttered businesses around
the United States to reopen within weeks, defying the advice of top
health officials.

transcript

Back to The Daily

bars

0:00/30:14

-30:14

transcript

\hypertarget{listen-to-the-daily-raring-to-go-by-easter-1}{%
\subsection{Listen to `The Daily': `Raring to Go by
Easter'}\label{listen-to-the-daily-raring-to-go-by-easter-1}}

\hypertarget{hosted-by-michael-barbaro-produced-by-michael-simon-johnson-and-jessica-cheung-with-help-from-austin-mitchell-and-sydney-harper-and-edited-by-lisa-tobin-and-mike-benoist}{%
\subsubsection{Hosted by Michael Barbaro; produced by Michael Simon
Johnson and Jessica Cheung; with help from Austin Mitchell and Sydney
Harper; and edited by Lisa Tobin and Mike
Benoist}\label{hosted-by-michael-barbaro-produced-by-michael-simon-johnson-and-jessica-cheung-with-help-from-austin-mitchell-and-sydney-harper-and-edited-by-lisa-tobin-and-mike-benoist}}

\hypertarget{president-trump-said-he-would-like-to-allow-shuttered-businesses-around-the-united-states-to-reopen-within-weeks-defying-the-advice-of-top-health-officials}{%
\paragraph{President Trump said he would like to allow shuttered
businesses around the United States to reopen within weeks, defying the
advice of top health
officials.}\label{president-trump-said-he-would-like-to-allow-shuttered-businesses-around-the-united-states-to-reopen-within-weeks-defying-the-advice-of-top-health-officials}}

\begin{itemize}
\item
  michael barbaro\\
  From The New York Times, I'm Michael Barbaro. This is ``The Daily.''
\item
  {[}music{]}\\
  Last week, President Trump sounded newly serious about combating the
  coronavirus, calling himself a quote, ``wartime president.'' Maggie
  Haberman on why days later, and with the situation only worsening, the
  president is abandoning that message.

  It's Wednesday, March 25.
\item
  maggie haberman\\
  Hello, guys.
\item
  michael barbaro\\
  Hi.
\item
  maggie haberman\\
  Hi.
\item
  michael barbaro\\
  Maggie, it is Tuesday afternoon. Can you tell us about what just
  happened on the Fox News Channel?
\item
  archived recording (bill hemmer)\\
  Over the next two hours, the president, the vice president, and the
  officials tasked with leading our nation's response on the virus
  pandemic will join us to answer your questions all across America.
\end{itemize}

michael barbaro

We just saw Vice President Mike Pence and President Trump sit for two
hours at a town hall meeting --- virtual town hall meeting --- with Fox
News from the White House, where they took questions by remote.

\begin{itemize}
\tightlist
\item
  archived recording\\
  I think a lot of us right now are just wondering, what is the
  potential for a national stay-at-home order? Is this something that
  America could be seeing in our near future?
\end{itemize}

maggie haberman

Pence answered a bunch of questions first.

\begin{itemize}
\tightlist
\item
  archived recording (mike pence)\\
  Carly, I can tell you that at no point has the White House coronavirus
  task force discussed what some people call a nationwide lockdown.
\end{itemize}

maggie haberman

Then, President Trump came on for the second hour.

\begin{itemize}
\tightlist
\item
  archived recording (donald trump)\\
  Our people are full of vim and vigor and energy. They don't want to be
  locked into a house or an apartment or some space. It's not for our
  country. We're not --- we're not built that way.
\end{itemize}

maggie haberman

And his message was even louder of a message that he's been delivering
for the last day or so, which is that while we have to take the
coronavirus seriously ---

\begin{itemize}
\tightlist
\item
  archived recording (donald trump)\\
  You know, I don't want the cure to be worse than the problem itself.
\end{itemize}

maggie haberman

--- in his words, the cure can't be worse than the disease.

\begin{itemize}
\tightlist
\item
  archived recording (donald trump)\\
  --- the problem being, obviously, the problem. And you know, you can
  destroy a country this way by closing it down.
\end{itemize}

maggie haberman

And by that, he means that the hits to the economy are becoming
unsustainable. That it can't go on forever.

\begin{itemize}
\tightlist
\item
  archived recording (donald trump)\\
  You're going to lose people. You're going to have suicides by the
  thousands. You're going to have all sorts of things happen. You're
  going to have instability. You can't just come in and say, let's close
  up the United States of America, the biggest, the most successful
  country in the world by far.
\end{itemize}

maggie haberman

And then he broke some news.

\begin{itemize}
\item
  archived recording (donald trump)\\
  I'd love to have it open by Easter. OK? I would love to have it open
  by Easter.
\item
  archived recording\\
  Oh wow, OK.
\end{itemize}

maggie haberman

And that news was that he believes that by April 12, which is Easter,
that that could be when the country and its economy are reopened.

\begin{itemize}
\tightlist
\item
  archived recording (donald trump)\\
  It's such an important day for other reasons, but I'll make it an
  important day for this, too. I would love to have the country opened
  up and just raring to go by Easter.
\end{itemize}

michael barbaro

Maggie, this seems very much at odds with the messaging coming from more
local leaders and health officials in the areas of the U.S. that have
been most directly hit by this pandemic so far. I'm thinking, for
example, of the governor of New York, Andrew Cuomo, who was warning New
York residents ---

\begin{itemize}
\tightlist
\item
  archived recording (andrew cuomo)\\
  Look, this can go on for several months, OK?
\end{itemize}

michael barbaro

--- that they should be preparing for four, six ---

\begin{itemize}
\tightlist
\item
  archived recording (andrew cuomo)\\
  --- eight months, nine months.
\end{itemize}

michael barbaro

Nine months of life under isolation and shutdown to fight the
coronavirus.

maggie haberman

Michael, about half an hour or so before Mike Pence started this town
hall ---

\begin{itemize}
\tightlist
\item
  archived recording (andrew cuomo)\\
  You have 20,000 ventilators in the stockpile. Release the ventilators
  to New York.
\end{itemize}

maggie haberman

--- Andrew Cuomo was pleading with the federal government to send more
resources, especially ventilators, because the number of cases that are
severe in New York is growing and keeps getting bigger and bigger. And
it is outpacing the number of materials that they have for doctors to
treat them.

\begin{itemize}
\tightlist
\item
  archived recording (andrew cuomo)\\
  I need the ventilators in 14 days. Only the federal government has
  that power.
\end{itemize}

maggie haberman

You have health officials in New York, health officials in California,
health officials in President Trump's own government saying we are not
just a mere two or three weeks away from things going back to normal.
And they're basing that not just on idle projections, but watching what
those curves have looked like in terms of the spread of the virus.

michael barbaro

Also, Maggie, help us understand how we got here and why this is the
message from the president at this critical moment --- when how we
respond, what measures we take, and how long we take those measures
really matters. And so I wonder where you think that starts. Where do we
begin to understand that?

maggie haberman

Michael, you need to go back to January 22 when the president was in
Davos for the World Economic Forum.

\begin{itemize}
\item
  archived recording (joe kernen)\\
  It's great to see you. Thank you for joining us again in Davos. We've
  done this before.
\item
  archived recording (donald trump)\\
  That's right.
\end{itemize}

maggie haberman

And he did an interview with CNBC. And at that point, the virus was
already in the U.S.

\begin{itemize}
\tightlist
\item
  archived recording (joe kernen)\\
  The C.D.C. has identified a case of coronavirus in Washington state.
\end{itemize}

maggie haberman

And he was asked by the interviewer if he was concerned that this could
become a pandemic.

\begin{itemize}
\item
  archived recording (joe kernen)\\
  Have you been briefed by the C.D.C.?
\item
  archived recording (donald trump)\\
  I have.
\item
  archived recording (joe kernen)\\
  Are there worries about a pandemic at this point?
\item
  archived recording (donald trump)\\
  No, we're not at all.
\end{itemize}

maggie haberman

And the president's response was, no, not at all.

\begin{itemize}
\item
  archived recording (donald trump)\\
  It's one person coming in from China, and we have it under control.
  It's going to be just fine.
\item
  archived recording (joe kernen)\\
  OK.
\end{itemize}

maggie haberman

He didn't want to talk about it publicly at the time.

michael barbaro

And why do you think that was?

maggie haberman

Well, according to a number of people who were in contact with him, it
was because he didn't want to rattle the financial markets. That he was
hoping that it was going to stay under control, and the stock markets
are his political weathervane, and he thinks they need to stay up in
order for him to win re-election. And he didn't want to do anything to
disturb that. And he didn't want to create a panic.

michael barbaro

OK. So what happened next?

maggie haberman

So after that, a couple of days later, as there were more cases and it
was clear that it was spreading out of China --- where it originated ---
the president took this move that he was widely criticized for by
Democrats and even some Republicans at the time. Which was he halted a
number of flights from China into the U.S.

\begin{itemize}
\item
  archived recording (sean hannity)\\
  Disney's closed.
\item
  archived recording (donald trump)\\
  Yeah.
\item
  archived recording (sean hannity)\\
  Movie theaters are closed. Hospitals being built. I think we're now up
  to our eighth case in the United States. How concerned are you?
\item
  archived recording (donald trump)\\
  Well, we pretty much shut it down coming in from China.
\end{itemize}

maggie haberman

The idea was to halt the spread of the disease, keep transmissions to a
minimum. He was accused of xenophobia. He was accused of making a racist
move. At the end of the day, it was probably effective, because it did
actually take a pretty aggressive measure against the spread of the
virus. The problem is, it was one of the last things that he did for
several weeks.

michael barbaro

Hmm. So the right decision in retrospect, but not accompanied by similar
actions that might have contained transmission.

maggie haberman

That's exactly right. In the same way that George W. Bush was criticized
for his ``Mission Accomplished'' banner about Iraq, the president
treated that moment as if it was his mission accomplished moment. He did
not do anything after that in terms of alerting the public, or telling
people to be safe, or telling people to take precautions. And it
basically squandered several weeks within the U.S.

michael barbaro

Right. Looking back at the timeline, we can now see that on the same day
that the president stopped those flights from China, the coronavirus was
already being reported by the W.H.O. in Japan and South Korea, and those
countries are still sending their citizens to the United States on
flights that have not been stopped. So the horse is out of the barn.

maggie haberman

Exactly. It was not anything close to a ``whole-of-government
approach.'' And at that point, there was a task force that was formed,
and it was being led by the health and human services secretary, Alex
Azar. But it was outside of the White House, and it was rife with all
kinds of turf battles. And the president, meanwhile, was still trying
not to talk about it.

michael barbaro

And succeeding in that, for the most part.

maggie haberman

And succeeding in that, for the most part. It was not something that
came up in interviews that he did, which were mostly with friendly
interviewers who weren't going to ask him things that he didn't want to
talk about. And look, it's not as if it wasn't getting news coverage.
The New York Times had it on the front page almost every day from the
end of January. It was very clear that this was a global crisis, but it
was not being treated as an American crisis. And I think a lot of that
is because the president just was not talking about it.

michael barbaro

And do we know what information the president is receiving during this
time? Is he getting a bunch of briefings --- one would think he would be
--- that are conveying the seriousness of the approaching situation?

maggie haberman

There's conflicting information, Michael, about exactly how specific and
how alarmed the briefing materials the president was receiving were at
this time. We understand that a lot of folks in the National Security
Council were taking it very seriously, and that information had been
passed to him. We understand that Alex Azar, the health and human
services secretary, took it very seriously, but it's not clear that he
was sharing all of that with the president or that he was being allowed
to tell it to the president. There were some people in the White House
who viewed Alex Azar in particular as quote-unquote ``alarmist'' ---
thought that he was overstating the threat. And when the president
doesn't want to take something particularly seriously, he'll often poll
test advisers until he finds the one who agrees with him that he
shouldn't take it seriously. And I have every reason to believe that he
was looking for people to affirm his sense that this didn't really need
to be addressed. And one of the places that he would go to hear his own
thoughts affirmed or for solace was Fox News.

\begin{itemize}
\tightlist
\item
  archived recording (jeanine pirro)\\
  If you've ever had a question whether the mainstream media distorts,
  whips up, throw things out of focus or has an agenda, especially when
  it comes to the Trump administration, look no further than
  coronavirus.
\end{itemize}

maggie haberman

They were very much echoing what he believed and wanted to believe,
which was that the criticisms about inactivity that he wasn't doing
enough was all part of an effort to harm him.

\begin{itemize}
\tightlist
\item
  archived recording (trish regan)\\
  This is yet another attempt to impeach the president. And sadly, it
  seems they care very little for any of the destruction they are
  leaving in their wake --- losses in the stock market. All this,
  unfortunately, just part of the political casualties for them.
\end{itemize}

{[}music{]}

maggie haberman

At this point, the president is in India being fitted by Prime Minister
Modi, and public health officials start basically taking matters into
their own hands. They start giving public warnings. One top health
expert gave a press briefing where she said that ---

\begin{itemize}
\tightlist
\item
  archived recording (dr. nancy messonnier)\\
  Now, it's not so much a question of if this will happen anymore, but
  rather more a question of exactly when this will happen.
\end{itemize}

maggie haberman

It was no longer a question of if the virus would spread in the United
States, but when.

\begin{itemize}
\tightlist
\item
  archived recording (dr. nancy messonnier)\\
  And how many people in this country will become infected, and how many
  of those will develop severe or more complicated disease.
\end{itemize}

maggie haberman

And that hospitals and businesses and schools should start making
preparations accordingly.

michael barbaro

Right.

maggie haberman

This was as the president was now on his way back from India, and the
stock market reacted terribly to these warnings. And the president was
furious. He called the health and human services secretary, saying that
the remarks had rattled people. He called one of his top economic
advisers, Larry Kudlow, wondering what could be done to stop the slide.
But at this point, as angry as the president was, it was clear to him
and to his advisers that this was no longer something he could ignore.

{[}music{]}

michael barbaro

We'll be right back.

So Maggie, what happens once the president recognizes that this is
something that has to be addressed?

maggie haberman

The president put Vice President Mike Pence in charge of the task force.

\begin{itemize}
\tightlist
\item
  archived recording (mike pence)\\
  Good afternoon. We just completed today's lengthy meeting of the White
  House coronavirus task force.
\end{itemize}

maggie haberman

And that was a big moment, because this task force that had been kind of
diffuse and fighting with itself was for the first time being run from
the White House. And it was meant to signal that the president was
taking this seriously. So Mike Pence had control of this thing for
basically two weeks. And during that time ---

\begin{itemize}
\tightlist
\item
  archived recording (mike pence)\\
  We're continuing to lean into this effort in full partnership with
  state and local health authorities around the country ---
\end{itemize}

maggie haberman

--- he was trying to communicate that they were working on a plan ---

\begin{itemize}
\tightlist
\item
  archived recording (mike pence)\\
  --- to ensure that we do everything to prevent the spread of the
  disease.
\end{itemize}

maggie haberman

To address the spread, that they were working on guidelines, that they
were aware of problems with testing for this virus that have plagued
this administration for weeks.

\begin{itemize}
\tightlist
\item
  archived recording (mike pence)\\
  To mitigate its expansion and to provide necessary treatment to
  Americans that have been impacted.
\end{itemize}

maggie haberman

And after two weeks, Vice President Mike Pence was getting a lot of
praise for his demeanor in these briefings.

\begin{itemize}
\item
  archived recording (mike pence)\\
  If I may, we'll be back here every day. Get used to seeing us. We're
  going to bring the experts in. We're going to make sure and give you
  the best and most high quality, real time information from the best
  people in the world. So thank you all for being here.
\item
  archived recording\\
  Thank you. You're welcome back any time.
\end{itemize}

maggie haberman

And that became a point of concern for some of President Trump's own
advisers, who didn't want to see Mike Pence get all of the attention.

michael barbaro

Huh. So the people around the president didn't want the vice resident,
rather than the president, to be the one seen as quarterbacking this
major national crisis?

maggie haberman

That's right. There was concern among the president's top advisers that
it would look as if Vice President Pence was basically doing the job the
president should be doing --- calming a nation, giving out accurate
information, sounding as if he's in charge. And that led to this idea
that the president should give an Oval Office address. So on March 11,
the same day the W.H.O. declared the coronavirus a global pandemic,
President Trump sat in the Oval Office behind the Resolute Desk at 9
p.m., and the klieg lights came on and the teleprompter started rolling.
And he gave an address to the nation.

\begin{itemize}
\tightlist
\item
  archived recording (donald trump)\\
  My fellow Americans, tonight I want to speak with you about our
  nation's unprecedented response to the coronavirus outbreak that
  started in China and is now spreading throughout the world.
\end{itemize}

maggie haberman

For the first time, he acknowledged that this could impact older people.

\begin{itemize}
\tightlist
\item
  archived recording (donald trump)\\
  The highest risk is for elderly population with underlying health
  conditions. The elderly population must be very, very careful.
\end{itemize}

maggie haberman

He talked about a ban on most travel from Europe.

\begin{itemize}
\tightlist
\item
  archived recording (donald trump)\\
  To keep new cases from entering our shores, we will be suspending all
  travel from Europe to the United States for the next 30 days. The new
  rules will go into effect Friday at midnight.
\end{itemize}

maggie haberman

But the address, which was brief, as Oval Office addresses usually are,
was seen as a disaster.

michael barbaro

Why?

maggie haberman

The President looked uncomfortable. He stumbled over the teleprompter,
which he never does well with.

\begin{itemize}
\tightlist
\item
  archived recording (donald trump)\\
  I am confident that by counting and continuing to take these tough
  measures, we will ---
\end{itemize}

maggie haberman

And it was riddled with errors, including about the travel ban.

\begin{itemize}
\tightlist
\item
  archived recording (donald trump)\\
  --- and these prohibitions will not only apply to the tremendous
  amount of trade in cargo, but various other things as we get approval.
\end{itemize}

maggie haberman

He suggested that it would apply to cargo and trade. It didn't, and
those mistakes sent the stock markets plummeting.

michael barbaro

Right. And my assumption was that the President hoped the speech would
do the exact opposite, which is it would give confidence to the stock
market and send it back up.

maggie haberman

That's right. The markets continued to tank over the next couple of
days. And aides started realizing that there had to be a major course
correction or the presidency could be threatened. And the President
realized this too. So on March 17 ---

\begin{itemize}
\tightlist
\item
  archived recording (donald trump)\\
  I would like to begin by announcing some important developments in our
  war against the Chinese virus.
\end{itemize}

maggie haberman

--- we saw a pretty different tone from President Trump as he talked
about this virus.

\begin{itemize}
\tightlist
\item
  archived recording (donald trump)\\
  We'll be invoking the Defense Production Act just in case we need it.
  Last week, I signed an emergency declaration under the Stafford Act.
\end{itemize}

maggie haberman

He described it soberly. He suddenly seemed willing to answer questions
without being combative.

\begin{itemize}
\item
  archived recording (journalist)\\
  Do you consider America to be on a wartime footing in terms of
  fighting this virus?
\item
  archived recording (donald trump)\\
  I do. I actually do. I'm looking at it that way, because you know ---
\end{itemize}

maggie haberman

He described himself as a wartime president. He seemed to be taking this
seriously in all of the ways that a nation usually looks for a leader to
take such a crisis seriously.

\begin{itemize}
\tightlist
\item
  archived recording (donald trump)\\
  It's a very tough situation, here. You have to do things. You have to
  close parts of an economy that six weeks ago were the best they've
  ever been. We had the best economy we've ever had. And then one day
  you have to close it down in order to defeat this enemy. But we're
  doing it, and we're doing it well. And I'd say the American people
  have been incredible.
\end{itemize}

maggie haberman

This was dramatically different from what we had heard just a few days
earlier.

michael barbaro

Well, so, Maggie, how then do we get to today, where one week later, the
situation with this pandemic has only gotten dramatically worse? The
virus is exploding in places like New York. The number of infections and
deaths are rising across the country. And yet, the president's message
has now kind of reverted back to where it was weeks and weeks ago. His
language has changed. His overall comportment and the words he's using
--- they've all kind of returned to a period where he was not taking
this as seriously.

maggie haberman

There have been people in the President's circle who, this entire time,
even as the President changed his tone, still thought that some of the
moves that the government was making were too aggressive. And those
aides started, late last week, talking about the fact that they might
want to revisit some of these guidelines and ease up on some of them for
targeted groups after this initial 15-day period had ended, which is
going to be March 30th. The president started getting the message in
earnest on Sunday night that this is something that he needed to worry
about. That there might not be an economy to return to once the country
was fully back to normal. And so he tweeted on Sunday night that the
cure couldn't be worse than the disease. And that was the beginning of a
massive shift of the federal government, which had moved toward
aggressive measures to mitigate this virus and its spread, to suddenly
suggesting that they could see the end nearing.

{[}music{]}

\begin{itemize}
\item
  archived recording (donald trump)\\
  We can't lose a Boeing. And we can't lose some of these companies. And
  companies --- frankly, Bill --- that were solid as, like, AAA
  companies. Because of what's happened over the last couple of weeks,
  they go from AAA to being, like, they could use a hand.
\item
  archived recording\\
  Tough time.
\item
  archived recording (donald trump)\\
  We can't --- you're right. We can't lose those companies. If we lose
  those companies, were talking about hundreds of thousands of jobs,
  millions of jobs. The faster we go back, the better it's going to be.
  We have a pent-up energy that's going to be unbelievable.
\end{itemize}

michael barbaro

Is there a meaningful contingency of conservative leaders, thinkers, and
politicians, economists even, people in business who feel this way, who
feel like there's been an overreaction to this virus in the form of
shutting down the American economy?

maggie haberman

There are. Some of them are people who have been the president's
advisers on and off for a while, like Stephen Moore of FreedomWorks,
who's an economist and who advised the president at various points. He
wrote an op-ed for The Wall Street Journal with Art Laffer, who the
president gave the Medal of Freedom to not that long ago. And in that
op-ed, they said, essentially, that the government can't sustain this.
That the economy can't sustain this. And that there needs to be less
draconian moves made to keep people safe but still allow the country to
run.

michael barbaro

Mm-hmm.

maggie haberman

There is no public health expert in the government telling the president
that these moves are too severe. There is no public health expert in the
government telling the president that the curve is about to let up on
the spread of the virus. Everything the president has done about this
virus has been a reaction to something, and right now he is reacting to
pressure to reopen portions of the economy, as the job losses are facing
potential millions by April.

michael barbaro

You know, I don't know whether the President would frame it this way,
but I wonder if he's forcing all of us to reckon with what is the
ultimate moral dilemma of this pandemic, which is what economic and
social cost we're willing to pay to save some uncertain number of lives.
And he seems to be saying, in effect, I'm willing to take the risk that
a certain number of Americans will get sick and will die for the greater
economic good and health of the United States.

\begin{itemize}
\item
  archived recording (donald trump)\\
  I mean, think of it. We average 36,000 people --- death, death. I'm
  not talking about cases, I'm talking about death. 36,000 deaths a
  year. People die, 36 --- from the flu. But we've never closed down the
  country for the flu. So you say to yourself, what is this all about?
  Now ---
\item
  archived recording\\
  How did you ---
\item
  archived recording (donald trump)\\
  It's never been done.
\item
  archived recording\\
  How did you process that?
\item
  archived recording (donald trump)\\
  Not good. I wasn't happy about it.
\end{itemize}

maggie haberman

Michael, I think that's very much what he's saying. And in fairness to
him, Governor Cuomo has openly voiced the same moral dilemma that he is
wrestling with. It's just that Governor Cuomo came down on the other
side of it, which was that there is no cost that can be put on human
life.

\begin{itemize}
\tightlist
\item
  archived recording (andrew cuomo)\\
  Yeah, my mother is not expendable. And your mother is not expendable.
  And our brothers and sisters they're not expendable. And we're not
  going to accept a premise that human life is disposable. And we're not
  going to put a dollar figure on human life. The first order of
  business is save lives, period. Whatever it costs.
\end{itemize}

maggie haberman

New York has been, as we know, much harder hit than most of the rest of
the country. Most of the rest of the country has not had to go through
what New York is going through right now with surges in hospital stays
and a number of sick people. The president says most of the country
agrees with him, and maybe that's why.

{[}music{]}

But the president is taking a really large gamble, and going with his
gut, that the greater good will be served for the rest of the country by
trying to preserve the economic health of the country more quickly than
his health experts would like him to.

michael barbaro

Maggie, thank you very much.

maggie haberman

Michael, thank you.

michael barbaro

On Tuesday night, Dr. Anthony Fauci, an infectious disease specialist
and an influential member of the president's coronavirus task force, was
asked about the president's plan to reopen the U.S. economy by Easter.

\begin{itemize}
\item
  archived recording (journalist)\\
  Where are you now with this timeline of 19 days from now?
\item
  archived recording (dr. anthony fauci)\\
  So that's really very flexible. We just had a conversation with the
  president in the Oval Office talking about, you know, you can look at
  a date, but you've got to be very flexible. On a literally day-by-day
  and week-by-week basis, you need to evaluate the feasibility of what
  you're trying to do.
\end{itemize}

michael barbaro

With the president standing beside him, Fauci said it would be foolhardy
to ease restrictions if major parts of the country were still in the
throes of the pandemic.

\begin{itemize}
\tightlist
\item
  archived recording (dr. anthony fauci)\\
  Obviously, no one is going to want to tone down things when you see
  what's going on in a place like New York City. I mean, that's just,
  you know, good public health practice and common sense.
\end{itemize}

{[}music{]}

michael barbaro

We'll be right back.

Here's what else you need to know today.

\begin{itemize}
\tightlist
\item
  archived recording (narendra modi)\\
  {[}SPEAKING HINDI{]}
\end{itemize}

michael barbaro

Nationwide lockdowns over the virus continued on Tuesday with India
becoming the latest and largest country to require citizens to remain
indoors, in India's case, for the next 21 days.

\begin{itemize}
\tightlist
\item
  archived recording (narendra modi)\\
  {[}SPEAKING HINDI{]}
\end{itemize}

michael barbaro

In a televised speech, Prime Minister Narendra Modi told Indians, quote,
``If you can't handle these 21 days, this country will go back 21
years.'' And in the United States, Senate leaders said they were nearing
a deal on a historic \$2 trillion stimulus bill after days of objections
from Democrats over who would monitor billions of dollars in loans to
American businesses.

\begin{itemize}
\tightlist
\item
  archived recording (chuck schumer)\\
  We've been fighting very hard that any bailout fund --- money to
  industries that have trouble --- have real oversight and transparency.
  That's vitally important.
\end{itemize}

michael barbaro

On Tuesday, Democrats said they had persuaded Republicans and the Trump
administration to allow an independent inspector and a congressional
oversight board to scrutinize the loans, and were almost ready to
support the bill.

\begin{itemize}
\tightlist
\item
  archived recording (chuck schumer)\\
  I hope, I pray, that we can come together very quickly and pass in
  large numbers a bipartisan bill that will help the American people who
  so badly, badly, badly need our help.
\end{itemize}

{[}music{]}

michael barbaro

That's it for ``The Daily.'' I'm Michael Barbaro. See you tomorrow.

\includegraphics{https://static01.graylady3jvrrxbe.onion/images/2020/03/22/us/politics/22trump-wartime/merlin_170773923_0acde43a-d9e9-48ae-bfc2-03fa50c785f1-articleLarge.jpg?quality=75\&auto=webp\&disable=upscale}

But the ``wartime'' strategy also presents risks to Mr. Trump. In his
new posture, he is trying to rewrite recent history, erasing his
comments from as recently as a week ago when
\href{https://www.nytimes3xbfgragh.onion/2020/03/15/us/politics/trump-flynn-pardon.html}{he
told Americans that they needed to ``just relax''} because ``it all will
pass.'' It is also undercut by his resistance to calls for additional
federal action from governors in hard-hit states.

Mr. Trump's temperament is also dissimilar to ``wartime presidents''
with whom he is seeking to compare himself. Over the course of his
presidency, Mr. Trump has struggled to stick to any bipartisan message,
or speak emotionally to the pain and fear of Americans during crisis
points like natural disasters or mass shootings.

\href{https://www.nytimes3xbfgragh.onion/news-event/2020-election}{Election
2020 ›}

\hypertarget{live-updates}{%
\subsection{\texorpdfstring{\href{https://www.nytimes3xbfgragh.onion/live/2020/09/09/us/trump-vs-biden}{Live
Updates}}{Live Updates}}\label{live-updates}}

\href{https://www.nytimes3xbfgragh.onion/live/2020/09/09/us/trump-vs-biden\#trumps-decision-to-go-maskless-at-a-north-carolina-rally-concerns-dr-fauci}{}

Sept. 9, 2020, 11:40 a.m. ET

\href{https://www.nytimes3xbfgragh.onion/live/2020/09/09/us/trump-vs-biden\#trumps-decision-to-go-maskless-at-a-north-carolina-rally-concerns-dr-fauci}{Trump's
decision to go maskless at a North Carolina rally concerns Dr.
Fauci.}\href{https://www.nytimes3xbfgragh.onion/live/2020/09/09/us/trump-vs-biden\#the-go-to-republican-election-lawyer-called-trumps-claims-of-voting-fraud-unsustainable}{}

Sept. 9, 2020, 11:33 a.m. ET

\href{https://www.nytimes3xbfgragh.onion/live/2020/09/09/us/trump-vs-biden\#the-go-to-republican-election-lawyer-called-trumps-claims-of-voting-fraud-unsustainable}{The
go-to Republican election lawyer called Trump's claims of voting fraud
`unsustainable.'}\href{https://www.nytimes3xbfgragh.onion/live/2020/09/09/us/trump-vs-biden\#a-marist-poll-shows-biden-widening-his-lead-in-pennsylvania}{}

Sept. 9, 2020, 10:37 a.m. ET

\href{https://www.nytimes3xbfgragh.onion/live/2020/09/09/us/trump-vs-biden\#a-marist-poll-shows-biden-widening-his-lead-in-pennsylvania}{A
Marist poll shows Biden widening his lead in Pennsylvania.}

``That's why it's so hard to be a wartime president,'' said Michael
Beschloss, the historian and author of ``Presidents of War.'' ``Not only
are you coming up with a strategy and tactics, but at the same time you
have to let Americans know that you know how hard this is for them.''

Mr. Trump, so far, has
\href{https://www.nytimes3xbfgragh.onion/2020/03/21/us/politics/trump-coronavirus-leadership.html}{struggled
to feel anyone's pain}, unlike past presidents, while continuing to play
out the fights with the news media that enliven his base. Last week, he
lashed out at a journalist who had prompted him to explain what his
message was to the millions of Americans watching him from home, who
felt scared.

The president has also continued to credit his own administration's
response. But Mr. Beschloss added that ``part of being a wartime
president is being willing to give people bad news,'' a job Mr. Trump
has mostly left to others.

At the same time, he has been timid of using wartime powers to fight
what he has called an ``invisible enemy.'' Last week, for instance, he
\href{https://www.nytimes3xbfgragh.onion/2020/03/20/us/politics/trump-coronavirus-supplies.html}{resisted
invoking the Defense Production Act}, a federal law that grants
presidents extraordinary powers to force American industries to ensure
the availability of critical equipment.

Mr. Trump's allies are aware that his re-election now hangs almost
entirely on how he handles the crisis. And the question is whether he
will be seen as President George W. Bush was in the immediate aftermath
of the attacks of Sept. 11, 2001, when he was widely viewed as bringing
the nation together, or if he will be compared to Mr. Bush amid the
fallout from Hurricane Katrina, when he tried to minimize a crisis that
eventually became too big for him to ignore, and during which Mr. Bush
praised cabinet officials even as the federal government bungled the
response.

Aides said that how Mr. Trump ends up being perceived would also depend
on Mr. Trump's own disposition during the crisis. It is not clear to
them whether he will be able to maintain his focus on the crisis for
months, especially as the economic situation worsens. Over the weekend,
some Trump advisers weren't ready to accept the likelihood of jobless
claims climbing into the millions by next month.

The White House press secretary, Stephanie Grisham, defended Mr. Trump's
response as apolitical. ``While it seems many in the media continue to
use every opportunity to destroy this president, the fact remains that
he has risen to fight this crisis by taking aggressive historic action
to protect the health, wealth and well-being of the American people,''
Ms. Grisham said in an email.

Some Democrats, meanwhile, said the mistakes made at the beginning of
the response had already colored how Mr. Trump would be remembered both
in the history books and at the ballot box in November. ``At the end of
the day, this will be Katrina with the waters at a much higher level,
and lasting a longer time,'' said Geoff Garin, a Democratic pollster.

But there is an emerging split in Democratic circles about whether to
attack Mr. Trump's response to the virus in real time, or whether the
gravity of the moment calls for a pause in negative political
advertising and attacks.

Some of Mr. Trump's highest-profile political adversaries leading states
that have become epicenters of the virus have been striking conciliatory
notes as they seek federal assistance. Mr. Cuomo said the president was
``fully engaged'' on the crisis. Gov. Gavin Newsom, Democrat of
California, described a recent phone conversation with Mr. Trump as ``a
privilege.''

Other antagonists have continued to criticize him. Mayor Bill de Blasio
of New York tapped into one of Mr. Trump's greatest fears when he
referred to him on CNN as
the\href{https://www.fox5ny.com/news/de-blasio-to-trump-lack-of-military-mobilization-is-immoral}{``Herbert
Hoover of his generation,''} comparing him to a president who failed to
recognize or take bold actions to stave off the stock market crash of
1929 and the subsequent Great Depression.

The debate about whether to embrace or attack Mr. Trump in a national
emergency played out most succinctly on Twitter
\href{https://twitter.com/davidplouffe/status/1240306292788301824}{between
David Axelrod and David Plouffe}, the two men who led Barack Obama's
presidential campaigns.

Mr. Axelrod said that voters would have plenty of time to judge Mr.
Trump's handling of the coronavirus crisis, ``but now doesn't seem the
moment for negative ads.'' Mr. Plouffe responded that time was of the
essence, and that Democrats couldn't afford to ``disarm'' and let Mr.
Trump create his own reality.

Image

Mr. Trump at Saturday's coronavirus news briefing. Last week, asked by a
journalist about his message for scared Americans, he said that ``you're
a terrible reporter.''Credit...Al Drago for The New York Times

Veterans of John Kerry's 2004 campaign said Mr. Biden was in a stronger
position against Mr. Trump than they were when they faced an election
against Mr. Bush. Back then, Mr. Bush still basked in good will from his
performance in the aftermath of the Sept. 11 attacks.

Some Democratic Party officials said a comparison between Mr. Biden and
Mr. Trump at a moment of crisis only helped Mr. Biden.

``You can see the contrast between the steady, assured, informed and
strong leadership that Vice President Biden has shown and the bungling,
chaotic and dishonest start-stop approach that Mr. Trump has shown us
since the beginning of this crisis,'' said Gilberto Hinojosa, chairman
of the Texas Democratic Party.

Thomas Kaplan contributed reporting.

\hypertarget{our-2020-election-guide}{%
\section{Our 2020 Election Guide}\label{our-2020-election-guide}}

Updated ~Sept. 9, 2020

\begin{center}\rule{0.5\linewidth}{\linethickness}\end{center}

\begin{itemize}
\item ~
  \hypertarget{the-latest}{%
  \subsection{The Latest}\label{the-latest}}

  \begin{itemize}
  \item
    Joe Biden heads today to Michigan, a battleground state where
    President Trump has resumed advertising ahead of a visit there on
    Thursday.
    \href{https://www.nytimes3xbfgragh.onion/live/2020/09/09/us/trump-vs-biden?action=click\&pgtype=Article\&state=default\&region=BELOW_MAIN_CONTENT\&context=storylines_guide}{Read
    live updates}.
  \end{itemize}
\item ~
  \hypertarget{how-to-win-270}{%
  \subsection{How to Win 270}\label{how-to-win-270}}

  \begin{itemize}
  \item
    Joe Biden and Donald Trump need 270 electoral votes to reach the
    White House. Try building
    \href{https://www.nytimes3xbfgragh.onion/interactive/2020/us/elections/election-states-biden-trump.html?action=click\&pgtype=Article\&state=default\&region=BELOW_MAIN_CONTENT\&context=storylines_guide}{your
    own coalition of battleground states}~to see potential outcomes.
  \end{itemize}
\item ~
  \hypertarget{voting-by-mail}{%
  \subsection{Voting by Mail}\label{voting-by-mail}}

  \begin{itemize}
  \item
    Will you have enough time to vote by mail in your state? Yes, but
    it's risky to procrastinate.
    \href{https://www.nytimes3xbfgragh.onion/interactive/2020/08/31/us/politics/vote-by-mail-deadlines.html?action=click\&pgtype=Article\&state=default\&region=BELOW_MAIN_CONTENT\&context=storylines_guide}{Check
    your state's deadline.}
  \item
    \href{https://www.nytimes3xbfgragh.onion/interactive/2020/us/elections/joe-biden.html?action=click\&pgtype=Article\&state=default\&region=BELOW_MAIN_CONTENT\&context=storylines_guide}{}

    \hypertarget{joe-biden}{%
    \section{Joe Biden}\label{joe-biden}}

    \hypertarget{democrat}{%
    \subsection{Democrat}\label{democrat}}

    \href{https://www.nytimes3xbfgragh.onion/interactive/2020/us/elections/donald-trump.html?action=click\&pgtype=Article\&state=default\&region=BELOW_MAIN_CONTENT\&context=storylines_guide}{}

    \hypertarget{donald-trump}{%
    \section{Donald Trump}\label{donald-trump}}

    \hypertarget{republican}{%
    \subsection{Republican}\label{republican}}
  \end{itemize}
\item
  \hypertarget{keep-up-with-our-coverage}{%
  \subsection{Keep Up With Our
  Coverage}\label{keep-up-with-our-coverage}}

  \begin{itemize}
  \item
    Get an
    \href{https://www.nytimes3xbfgragh.onion/newsletters/politics?action=click\&pgtype=Article\&state=default\&region=BELOW_MAIN_CONTENT\&context=storylines_guide}{email}~recapping
    the day's news
  \item
    Download our mobile app on
    \href{https://apps.apple.com/us/app/nytimes/id284862083?ls=1\&mat_click_id=5c79ae7455014fd1bd66b5610c05b8f2-20191112-16948\&referrer=mat_click_id\%3D5c79ae7455014fd1bd66b5610c05b8f2-20191112-16948\%26link_click_id\%3D722930677036718082}{iOS}~and
    \href{http://a.localytics.com/android?id=com.nytimes.android\&referrer=utm_source\%3Dother_nyt_mobile_web\%26utm_medium\%3DWeb\%2520page\%26utm_term\%3DGeneral\%2520Mobile\%2520Page\%26utm_campaign\%3DNYT\%2520Mobile\%2520General\%2520Page}{Android}~and
    turn on Breaking News and Politics alerts
  \end{itemize}
\end{itemize}

Advertisement

\protect\hyperlink{after-bottom}{Continue reading the main story}

\hypertarget{site-index}{%
\subsection{Site Index}\label{site-index}}

\hypertarget{site-information-navigation}{%
\subsection{Site Information
Navigation}\label{site-information-navigation}}

\begin{itemize}
\tightlist
\item
  \href{https://help.nytimes3xbfgragh.onion/hc/en-us/articles/115014792127-Copyright-notice}{©~2020~The
  New York Times Company}
\end{itemize}

\begin{itemize}
\tightlist
\item
  \href{https://www.nytco.com/}{NYTCo}
\item
  \href{https://help.nytimes3xbfgragh.onion/hc/en-us/articles/115015385887-Contact-Us}{Contact
  Us}
\item
  \href{https://www.nytco.com/careers/}{Work with us}
\item
  \href{https://nytmediakit.com/}{Advertise}
\item
  \href{http://www.tbrandstudio.com/}{T Brand Studio}
\item
  \href{https://www.nytimes3xbfgragh.onion/privacy/cookie-policy\#how-do-i-manage-trackers}{Your
  Ad Choices}
\item
  \href{https://www.nytimes3xbfgragh.onion/privacy}{Privacy}
\item
  \href{https://help.nytimes3xbfgragh.onion/hc/en-us/articles/115014893428-Terms-of-service}{Terms
  of Service}
\item
  \href{https://help.nytimes3xbfgragh.onion/hc/en-us/articles/115014893968-Terms-of-sale}{Terms
  of Sale}
\item
  \href{https://spiderbites.nytimes3xbfgragh.onion}{Site Map}
\item
  \href{https://help.nytimes3xbfgragh.onion/hc/en-us}{Help}
\item
  \href{https://www.nytimes3xbfgragh.onion/subscription?campaignId=37WXW}{Subscriptions}
\end{itemize}
