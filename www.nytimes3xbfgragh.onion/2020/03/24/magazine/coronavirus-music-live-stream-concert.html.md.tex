Sections

SEARCH

\protect\hyperlink{site-content}{Skip to
content}\protect\hyperlink{site-index}{Skip to site index}

\href{https://myaccount.nytimes3xbfgragh.onion/auth/login?response_type=cookie\&client_id=vi}{}

\href{https://www.nytimes3xbfgragh.onion/section/todayspaper}{Today's
Paper}

Livestreaming the Seattle Symphony Became a Source of Connection in Dark
Times

\url{https://nyti.ms/33Ga1ZN}

\begin{itemize}
\item
\item
\item
\item
\item
\item
\end{itemize}

\hypertarget{the-coronavirus-outbreak}{%
\subsubsection{\texorpdfstring{\href{https://www.nytimes3xbfgragh.onion/news-event/coronavirus?name=styln-coronavirus-national\&region=TOP_BANNER\&block=storyline_menu_recirc\&action=click\&pgtype=Article\&impression_id=f2d49420-f52e-11ea-b9b1-3dadc7cfc0e9\&variant=undefined}{The
Coronavirus
Outbreak}}{The Coronavirus Outbreak}}\label{the-coronavirus-outbreak}}

\begin{itemize}
\tightlist
\item
  live\href{https://www.nytimes3xbfgragh.onion/2020/09/12/world/covid-19-coronavirus.html?name=styln-coronavirus-national\&region=TOP_BANNER\&block=storyline_menu_recirc\&action=click\&pgtype=Article\&impression_id=f2d49421-f52e-11ea-b9b1-3dadc7cfc0e9\&variant=undefined}{Latest
  Updates}
\item
  \href{https://www.nytimes3xbfgragh.onion/interactive/2020/us/coronavirus-us-cases.html?name=styln-coronavirus-national\&region=TOP_BANNER\&block=storyline_menu_recirc\&action=click\&pgtype=Article\&impression_id=f2d4bb30-f52e-11ea-b9b1-3dadc7cfc0e9\&variant=undefined}{Maps
  and Cases}
\item
  \href{https://www.nytimes3xbfgragh.onion/interactive/2020/science/coronavirus-vaccine-tracker.html?name=styln-coronavirus-national\&region=TOP_BANNER\&block=storyline_menu_recirc\&action=click\&pgtype=Article\&impression_id=f2d4bb31-f52e-11ea-b9b1-3dadc7cfc0e9\&variant=undefined}{Vaccine
  Tracker}
\item
  \href{https://www.nytimes3xbfgragh.onion/2020/09/10/us/politics/fda-coronavirus-vaccine.html?name=styln-coronavirus-national\&region=TOP_BANNER\&block=storyline_menu_recirc\&action=click\&pgtype=Article\&impression_id=f2d4bb32-f52e-11ea-b9b1-3dadc7cfc0e9\&variant=undefined}{F.D.A.
  Regulators' Self-Defense}
\item
  \href{https://www.nytimes3xbfgragh.onion/2020/09/09/upshot/coronavirus-surprise-test-fees.html?name=styln-coronavirus-national\&region=TOP_BANNER\&block=storyline_menu_recirc\&action=click\&pgtype=Article\&impression_id=f2d4bb33-f52e-11ea-b9b1-3dadc7cfc0e9\&variant=undefined}{Surprise
  Test Fees}
\end{itemize}

Advertisement

\protect\hyperlink{after-top}{Continue reading the main story}

Supported by

\protect\hyperlink{after-sponsor}{Continue reading the main story}

\href{/column/screenland}{Screenland}

\hypertarget{livestreaming-the-seattle-symphony-became-a-source-of-connection-in-dark-times}{%
\section{Livestreaming the Seattle Symphony Became a Source of
Connection in Dark
Times}\label{livestreaming-the-seattle-symphony-became-a-source-of-connection-in-dark-times}}

\includegraphics{https://static01.graylady3jvrrxbe.onion/images/2020/03/29/magazine/29mag-screenland-1/29mag-screenland-1-articleLarge.jpg?quality=75\&auto=webp\&disable=upscale}

By Brooke Jarvis

\begin{itemize}
\item
  March 24, 2020
\item
  \begin{itemize}
  \item
  \item
  \item
  \item
  \item
  \item
  \end{itemize}
\end{itemize}

There was a bit of a cacophony at first, as conflicting melodies emerged
and subsided. This was the chaos of preparation, the warm-up, but after
a few moments the orchestra went quiet. It felt like a long, deep
breath, and so I took one, too. The graphic advertising the concert as
``Live From Benaroya Hall'' in Seattle disappeared, revealing the
conductor, as he took his place onstage, all eyes on his upheld baton.
The music started softly and then grew stronger, filling my living room:
floating flutes, a charge of clarinets, the friendly vibrato of
bassoons. The strings played a single note, across seven octaves.

\includegraphics{https://static01.graylady3jvrrxbe.onion/images/2020/03/29/magazine/29mag-screenland-vid-promo/29mag-screenland-vid-promo-videoSixteenByNineJumbo1600.png}

It had been a scary week in Seattle, then the center of the
\href{https://www.nytimes3xbfgragh.onion/news-event/coronavirus}{coronavirus
outbreak in the United States}. The virus was spreading in nursing
homes, and there were more deaths being reported every day. The governor
of Washington had just banned all large gatherings and closed all
schools in the Seattle area. There were runs on food and supplies. There
were already layoffs and sure to be a lot more. Though it was clear that
we were at the very beginning of what would be a long and spiraling
crisis, the region's hospitals were even now running low on supplies,
personnel and beds for critically ill patients. Every small decision ---
to go to the store, to see friends, to eat at restaurants, to visit the
elderly --- was suddenly taking on a new moral weight. Within just a few
days, as it sank in what truly caring for one another needed to look
like, even those choices would be gone.

When I heard that the Seattle Symphony, which had been ordered to close
like everything else, would be livestreaming free concerts during the
crisis, I almost cried. I had never actually been to one of its
performances before, even though I live less than two miles from
Benaroya Hall. But now, seeing my city shut down around me, I couldn't
wait to watch. The performance felt symbolic: a declaration that
connection and solidarity and collective beauty would continue, that we
could still gather together even as we stayed apart. I thought
immediately of the tiny poem Bertolt Brecht wrote in the midst of World
War II: ``In the dark times/Will there be singing?/There will be
singing./Of the dark times.''

I pictured the musicians, dressed in their black suits and dresses,
playing to the emptiness of a grand theater, while the rest of us
gathered around our laptops --- like the families, hungry for
reassurance, who listened to F.D.R.'s fireside chats during the Great
Depression. Another viewer thought of a different historical analogue:
the musicians on the Titanic who kept playing music for the ship's
passengers even as it sank. ``Getting big `Gentlemen, it has been a
privilege' vibes,'' he wrote in the chat box that accompanied the video.
``Thanks for this.''

In fact, the performance wasn't to an empty theater, or technically live
at all --- it was a livestream of a concert filmed the previous
September. Alexander White, the symphony's associate principal trumpet
and chairman of the musicians' labor organization, told me that the idea
of continuing performances without audiences, which was under
consideration just two days earlier, evaporated the day before the
livestream. The symphony had been rehearsing Tchaikovsky's Symphony No.
5 for its upcoming shows when the governor's news conference announcing
regulations on group gatherings began. ``We realized the orchestra
couldn't actually safely be together,'' White said. As a brass player,
he was particularly aware of all the breath and moisture that regularly
moves through a crowd of musicians. For the first time White could
remember, everyone stopped playing mid-rehearsal, packed up and left.

In the chat box for the concert, viewers seemed puzzled. New arrivals
kept asking why the video showed a live audience in a shuttered city. A
commenter named David explained, ``Not live, but not dead either.''
Someone else wrote: ``Yeah, it's confusing. But hey, music.''

Over the next few days, as I stayed home and spent too much time reading
the news, it began to seem that the more people were separated and
confused and scared, the more there was music. Yo-Yo Ma started posting
performances with the hashtag \#SongsOfComfort, and more than three
million people
\href{https://twitter.com/YoYo_Ma/status/1238572657278431234}{watched
him play Antonin Dvorak's ``Going Home.''} The Metropolitan Opera in New
York announced it would be
\href{https://www.metopera.org/user-information/nightly-met-opera-streams/}{streaming
previously filmed performances every night} free; hundreds of thousands
watched. High school students who wouldn't get to perform the spring
musicals they'd been practicing started singing for Twitter instead. A
Seattle musician named Marina Albero, who suddenly found all her gigs
canceled and the schools where she teaches closed, started organizing
what she called ``The Quarantine Sessions,'' streamed performances that
would allow musicians to still play and audiences to still support them.
(When I called her, she stressed that the money, while welcome, wasn't
the main point. ``It's about being together and making something
beautiful,'' she said. ``Nobody is anything alone. That's what this
situation is demonstrating.'')

And from Italy, where a cascade of deaths in overwhelmed hospitals
presaged what we feared our own crisis would become, video after video
emerged of people in lockdown, standing on their balconies or leaning
out their windows, uniting the music of their violins and tambourines
and accordions and saxophones. They played patriotic tunes and folk
songs. They played ``Smoke on the Water'' and ``Tequila.'' Elderly women
stuck inside stepped onto their balconies and danced.

It took about an hour for the Seattle Symphony to perform Gustav
Mahler's Symphony No. 1 in D major. The symphony is a glorious jumble,
rejected by its first audiences as too modern: It incorporates klezmer
accents, folk-dance melodies, a funeral march and victorious horn
crescendos. I kept waiting for the performance to feel solemn and
historic, to get goosebumps of the kind I have when I read Brecht's poem
or think about people singing ``There'll Always Be an England'' during
the Blitz. But instead it felt like life, strange and confusing and
funny and scary and beautiful, and still going on. In the chat box,
people leaned into the surreality of the situation, making jokes about
the rude noise of one another's candy wrappers, about being tall and
blocking other people's views of the stage, about whether ``clap''
emojis are acceptable between movements, when real clapping, per
symphony etiquette, is not. ``Mahler is an absolute unit of a
composer,'' someone wrote; sex bots invaded the chat. People celebrated
the music, told one another where they were watching from and wished one
another health and luck and safety in a changed and scary world. White,
the trumpet player, watched the chat from his own computer. ``It was
endearing and heartening,'' he said. ``But it was also reality.''

By the time it was over, nearly 90,000 people from Seattle and around
the world had tuned in. By comparison, 4,835 people bought tickets for
the original three-day run of the symphony, back in the other world that
was last September. The symphony made plans for more shows: experimental
solos filmed in homes or the empty hall; group pieces merged together
electronically; more livestreams of past performances. I knew I would
want to watch them. I wanted the deep breath, the feeling of connection,
even the jokes about sex bots. I wanted the woodwinds making the soft
sounds of nature and the brass section trumpeting victory, whatever that
might mean now.

Advertisement

\protect\hyperlink{after-bottom}{Continue reading the main story}

\hypertarget{site-index}{%
\subsection{Site Index}\label{site-index}}

\hypertarget{site-information-navigation}{%
\subsection{Site Information
Navigation}\label{site-information-navigation}}

\begin{itemize}
\tightlist
\item
  \href{https://help.nytimes3xbfgragh.onion/hc/en-us/articles/115014792127-Copyright-notice}{©~2020~The
  New York Times Company}
\end{itemize}

\begin{itemize}
\tightlist
\item
  \href{https://www.nytco.com/}{NYTCo}
\item
  \href{https://help.nytimes3xbfgragh.onion/hc/en-us/articles/115015385887-Contact-Us}{Contact
  Us}
\item
  \href{https://www.nytco.com/careers/}{Work with us}
\item
  \href{https://nytmediakit.com/}{Advertise}
\item
  \href{http://www.tbrandstudio.com/}{T Brand Studio}
\item
  \href{https://www.nytimes3xbfgragh.onion/privacy/cookie-policy\#how-do-i-manage-trackers}{Your
  Ad Choices}
\item
  \href{https://www.nytimes3xbfgragh.onion/privacy}{Privacy}
\item
  \href{https://help.nytimes3xbfgragh.onion/hc/en-us/articles/115014893428-Terms-of-service}{Terms
  of Service}
\item
  \href{https://help.nytimes3xbfgragh.onion/hc/en-us/articles/115014893968-Terms-of-sale}{Terms
  of Sale}
\item
  \href{https://spiderbites.nytimes3xbfgragh.onion}{Site Map}
\item
  \href{https://help.nytimes3xbfgragh.onion/hc/en-us}{Help}
\item
  \href{https://www.nytimes3xbfgragh.onion/subscription?campaignId=37WXW}{Subscriptions}
\end{itemize}
