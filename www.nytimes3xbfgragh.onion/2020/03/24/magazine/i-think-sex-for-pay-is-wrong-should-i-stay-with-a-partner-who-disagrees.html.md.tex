Sections

SEARCH

\protect\hyperlink{site-content}{Skip to
content}\protect\hyperlink{site-index}{Skip to site index}

\href{https://myaccount.nytimes3xbfgragh.onion/auth/login?response_type=cookie\&client_id=vi}{}

\href{https://www.nytimes3xbfgragh.onion/section/todayspaper}{Today's
Paper}

I Think Sex for Pay Is Wrong. Should I Stay With a Partner Who
Disagrees?

\url{https://nyti.ms/3bpH9aW}

\begin{itemize}
\item
\item
\item
\item
\item
\item
\end{itemize}

Advertisement

\protect\hyperlink{after-top}{Continue reading the main story}

Supported by

\protect\hyperlink{after-sponsor}{Continue reading the main story}

\href{/column/the-ethicist}{The Ethicist}

\hypertarget{i-think-sex-for-pay-is-wrong-should-i-stay-with-a-partner-who-disagrees}{%
\section{I Think Sex for Pay Is Wrong. Should I Stay With a Partner Who
Disagrees?}\label{i-think-sex-for-pay-is-wrong-should-i-stay-with-a-partner-who-disagrees}}

\includegraphics{https://static01.graylady3jvrrxbe.onion/images/2020/03/29/magazine/29Ethicist/29Ethicist-articleLarge.jpg?quality=75\&auto=webp\&disable=upscale}

By Kwame Anthony Appiah

\begin{itemize}
\item
  March 24, 2020
\item
  \begin{itemize}
  \item
  \item
  \item
  \item
  \item
  \item
  \end{itemize}
\end{itemize}

\emph{I am a man in my 20s, currently in a relationship with a
50-some-year-old man. We have been dating monogamously for about six
months, and it's been a rewarding relationship for both of us. It
recently came up that before we were together, he paid people for sex in
two instances. He described both those instances as involving ``good''
intentions between the two parties.}

\emph{I strongly believe that sex for pay, in any circumstance, is
morally wrong, because I don't believe sex should be a paid service and
because sex for pay reinforces a very deep problem within our society,
even with ``good'' intentions. This is a fundamental issue for me and
for people in my social circle.}

\emph{My partner and I had a long discussion about this. He disagreed
with some of my viewpoints; he would still be OK with paying for sex if
our relationship did not exist. I expressed that this is not OK with me,
even though our monogamous relationship would prevent him from engaging
in this behavior. (I believe that he would not cheat on me, given that
at the beginning of our relationship, he strongly expressed his desire
for a monogamous relationship when I suggested polyamory.) He views this
as a small issue in the past, which I shouldn't worry about.}

\emph{I could see us going far together, but the issue of paying for sex
is very fundamental to me, and something I would like my friends and
partner(s) to have similar viewpoints about. Would it be wrong for me to
stay with him, given how I feel about his view on paying for sex, even
if he never did it again? Or would it be wrong to get out of this
relationship? I realize couples should work through their differences,
but my viewpoint on paying for sex is not one that is going to change
because of a relationship}. Name Withheld

\textbf{Whether or not} sex for pay can be morally acceptable is a
question on which reasonable people differ. Purely theoretical disputes
about contestable issues like this are generally less important in the
ethical shaping of relationships than concrete disagreements about what
to do, and you both agree that he won't have sex outside your
relationship, paid or not. Still, a theoretical disagreement may be
supercharged by disagreements that have practical effects, in the larger
world if not in a particular relationship. And the notion of what's
contestable bears some weight. Most decent people wouldn't want to be
married to someone committed to white supremacy, say, even if the
commitment remained in the realm of theory. For you, it appears,
countenancing sex for pay represents a similar enormity: The issue, you
say, is ``fundamental'' to you.

You don't really explain why, however. It would seem that the
disagreement you have with your partner reflects a deeper disagreement
about the meaning of sex and sexuality. Perhaps you think that sex is
the sort of thing that can properly occur only in the context of a
loving relationship. You may believe that, even in the most
benign-seeming circumstances --- circumstances that don't involve
financial hardship, gross inequality between parties or other potential
indicators of exploitation --- the sex-for-pay transaction involves the
instrumentalizing of another's body. Perhaps you think that there is
something special about sex, and you wouldn't be terribly disturbed if
your partner thought that it was fine to pay for a dinner escort. Or
perhaps you don't like instrumentalizing people in any way, and so you
would find a paid dinner companion unacceptable too.

I'd encourage you to work out precisely what it is about sex for pay
that makes it, in your estimation, inherently repugnant. Then you can
see if your partner agrees with you about the more fundamental values at
stake, whether or not he thinks sex work is OK. Simply declaring that
you take the wrongness of sex work to be axiomatic isn't a respectful
way of trying to negotiate a disagreement. But then the discordance of
your views on this issue must already have undermined your respect for
your partner --- otherwise you wouldn't be thinking of leaving him over
it. As you mull over your relationship, bear in mind that your lack of
respect equally gives him reason to leave you.

\emph{I am a 40-something-year-old woman living in the Bay Area. I moved
here from the East Coast in my early 20s, established a career and found
a group of friends, and I am currently in a relationship with a
wonderful man who is firmly rooted here. Unless some significant change
occurs, I foresee living here indefinitely. My parents, along with most
of the rest of my family, still live on the East Coast. We have always
been close, and despite the distance, I see them several times a year.
It may be that they're getting older, or that I'm getting older, but the
distance between us has seemed more significant in the last few years.
My youthful reasons for moving so far away, which were partly a whim and
partly an attempt to escape ``ending up'' in my hometown, now seem
frivolous and shortsighted. I feel a sense of guilt that after all my
parents have done for me, I won't be there for them as they get older.
Am I living up to my responsibilities as a daughter?} Name Withheld

\textbf{If children have} a duty to move back home and abandon the life
they have made elsewhere in order to be near their aging parents, the
world is full of delinquent children. Nothing you say suggests that your
parents need you nearby, however much they might enjoy your company.
They have not planned for an old age that requires your return, it would
seem. The issue here, then, isn't best seen as being about
responsibilities. (But if it were, for what it's worth, I think any such
responsibilities would be those of a child and not particularly of a
daughter.)

What you could reasonably wonder is whether the life you have made
elsewhere is the one you want, especially if you would now value
spending time closer to your parents and being more present in their
lives. You should have no expectation, however, that your partner and
your friends (who, after all, have different homes and different
parents) will be of like mind. Many professional people in modern
societies find that the moral center of their lives shifts from the
family into which they were born to the family they make through love
and friendship as they mature. That is only one possible way of doing
things, though, and you may value an older model in which the family of
your birth is the center of your life.

Ethics, in its original, Aristotelian sense, is concerned with what it
is for a life to go well. So you're raising paradigmatic ethical
questions. But answering them is for you alone, even though it may be
wise to seek the counsel of others. You'll have to think about
everything you value and cherish --- including all your relationships
--- if you're to come to a satisfactory answer.

\emph{I'm a nurse practitioner working at a primary-care clinic for
low-income patients. One of my patients is a 16-year-old who told me
that she did not come back for a birth-control refill because she wants
to have a baby. She has been having unprotected sex with her partner for
several months and was worried something was wrong because she did not
become pregnant. She is not in school but is looking for full-time work;
her partner has a steady job. I don't know how old her partner is, but
she said that he wants a baby, too.}

\emph{I normally try not to let my personal views impact my patient-care
decisions, but I have concerns that this patient is making a choice that
is not in her own best interests. Would it be ethical for me to steer
her away from trying to get pregnant? While she is of the age of consent
to have sex, she is not yet an adult. Or, as her health care provider,
do I have an ethical duty to try to help her conceive?} Name Withheld

\textbf{You're her health care} provider. You should certainly tell her
about the medical consequences of pregnancy. But the social and economic
consequences don't fall within your professional competence. An
intervention about her life choices may seem moralizing and intrusive to
her, and it could drive her away; and then she'd be losing your guidance
on the things you are trained to help her with.

Advertisement

\protect\hyperlink{after-bottom}{Continue reading the main story}

\hypertarget{site-index}{%
\subsection{Site Index}\label{site-index}}

\hypertarget{site-information-navigation}{%
\subsection{Site Information
Navigation}\label{site-information-navigation}}

\begin{itemize}
\tightlist
\item
  \href{https://help.nytimes3xbfgragh.onion/hc/en-us/articles/115014792127-Copyright-notice}{©~2020~The
  New York Times Company}
\end{itemize}

\begin{itemize}
\tightlist
\item
  \href{https://www.nytco.com/}{NYTCo}
\item
  \href{https://help.nytimes3xbfgragh.onion/hc/en-us/articles/115015385887-Contact-Us}{Contact
  Us}
\item
  \href{https://www.nytco.com/careers/}{Work with us}
\item
  \href{https://nytmediakit.com/}{Advertise}
\item
  \href{http://www.tbrandstudio.com/}{T Brand Studio}
\item
  \href{https://www.nytimes3xbfgragh.onion/privacy/cookie-policy\#how-do-i-manage-trackers}{Your
  Ad Choices}
\item
  \href{https://www.nytimes3xbfgragh.onion/privacy}{Privacy}
\item
  \href{https://help.nytimes3xbfgragh.onion/hc/en-us/articles/115014893428-Terms-of-service}{Terms
  of Service}
\item
  \href{https://help.nytimes3xbfgragh.onion/hc/en-us/articles/115014893968-Terms-of-sale}{Terms
  of Sale}
\item
  \href{https://spiderbites.nytimes3xbfgragh.onion}{Site Map}
\item
  \href{https://help.nytimes3xbfgragh.onion/hc/en-us}{Help}
\item
  \href{https://www.nytimes3xbfgragh.onion/subscription?campaignId=37WXW}{Subscriptions}
\end{itemize}
