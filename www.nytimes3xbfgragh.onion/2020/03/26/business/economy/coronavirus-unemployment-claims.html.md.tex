Sections

SEARCH

\protect\hyperlink{site-content}{Skip to
content}\protect\hyperlink{site-index}{Skip to site index}

\href{https://www.nytimes3xbfgragh.onion/section/business/economy}{Economy}

\href{https://myaccount.nytimes3xbfgragh.onion/auth/login?response_type=cookie\&client_id=vi}{}

\href{https://www.nytimes3xbfgragh.onion/section/todayspaper}{Today's
Paper}

\href{/section/business/economy}{Economy}\textbar{}`It's a Wreck': 3.3
Million File Unemployment Claims as Economy Comes Apart

\url{https://nyti.ms/3bvPHgr}

\begin{itemize}
\item
\item
\item
\item
\item
\end{itemize}

\hypertarget{the-coronavirus-outbreak}{%
\subsubsection{\texorpdfstring{\href{https://www.nytimes3xbfgragh.onion/news-event/coronavirus?name=styln-coronavirus-markets\&region=TOP_BANNER\&block=storyline_menu_recirc\&action=click\&pgtype=Article\&impression_id=9c5aab50-f1d4-11ea-9feb-8792ec8b6371\&variant=undefined}{The
Coronavirus
Outbreak}}{The Coronavirus Outbreak}}\label{the-coronavirus-outbreak}}

\begin{itemize}
\tightlist
\item
  live\href{https://www.nytimes3xbfgragh.onion/2020/09/08/world/covid-19-coronavirus.html?name=styln-coronavirus-markets\&region=TOP_BANNER\&block=storyline_menu_recirc\&action=click\&pgtype=Article\&impression_id=9c5aab51-f1d4-11ea-9feb-8792ec8b6371\&variant=undefined}{Latest
  Updates}
\item
  \href{https://www.nytimes3xbfgragh.onion/interactive/2020/us/coronavirus-us-cases.html?name=styln-coronavirus-markets\&region=TOP_BANNER\&block=storyline_menu_recirc\&action=click\&pgtype=Article\&impression_id=9c5aab52-f1d4-11ea-9feb-8792ec8b6371\&variant=undefined}{Maps
  and Cases}
\item
  \href{https://www.nytimes3xbfgragh.onion/interactive/2020/science/coronavirus-vaccine-tracker.html?name=styln-coronavirus-markets\&region=TOP_BANNER\&block=storyline_menu_recirc\&action=click\&pgtype=Article\&impression_id=9c5aab53-f1d4-11ea-9feb-8792ec8b6371\&variant=undefined}{Vaccine
  Tracker}
\item
  \href{https://www.nytimes3xbfgragh.onion/2020/09/02/your-money/eviction-moratorium-covid.html?name=styln-coronavirus-markets\&region=TOP_BANNER\&block=storyline_menu_recirc\&action=click\&pgtype=Article\&impression_id=9c5ad260-f1d4-11ea-9feb-8792ec8b6371\&variant=undefined}{Eviction
  Moratorium}
\item
  \href{https://www.nytimes3xbfgragh.onion/interactive/2020/09/02/magazine/food-insecurity-hunger-us.html?name=styln-coronavirus-markets\&region=TOP_BANNER\&block=storyline_menu_recirc\&action=click\&pgtype=Article\&impression_id=9c5ad261-f1d4-11ea-9feb-8792ec8b6371\&variant=undefined}{American
  Hunger}
\end{itemize}

Advertisement

\protect\hyperlink{after-top}{Continue reading the main story}

Supported by

\protect\hyperlink{after-sponsor}{Continue reading the main story}

\hypertarget{its-a-wreck-33-million-file-unemployment-claims-as-economy-comes-apart}{%
\section{`It's a Wreck': 3.3 Million File Unemployment Claims as Economy
Comes
Apart}\label{its-a-wreck-33-million-file-unemployment-claims-as-economy-comes-apart}}

The weekly figure is among the first data on the economic toll of the
vast disruption of normal life and commerce caused by the coronavirus
pandemic.

\includegraphics{https://static01.graylady3jvrrxbe.onion/images/2020/03/26/business/26virus-jobs1a/merlin_170735211_f804388b-9379-4dbf-98e2-9bb4ae7b2980-articleLarge.jpg?quality=75\&auto=webp\&disable=upscale}

\href{https://www.nytimes3xbfgragh.onion/by/ben-casselman}{\includegraphics{https://static01.graylady3jvrrxbe.onion/images/2018/11/09/multimedia/author-ben-casselman/author-ben-casselman-thumbLarge.png}}\href{https://www.nytimes3xbfgragh.onion/by/patricia-cohen}{\includegraphics{https://static01.graylady3jvrrxbe.onion/images/2018/02/16/multimedia/author-patricia-cohen/author-patricia-cohen-thumbLarge.jpg}}\href{https://www.nytimes3xbfgragh.onion/by/tiffany-hsu}{\includegraphics{https://static01.graylady3jvrrxbe.onion/images/2018/12/06/multimedia/author-tiffany-hsu/author-tiffany-hsu-thumbLarge.png}}

By \href{https://www.nytimes3xbfgragh.onion/by/ben-casselman}{Ben
Casselman},
\href{https://www.nytimes3xbfgragh.onion/by/patricia-cohen}{Patricia
Cohen} and
\href{https://www.nytimes3xbfgragh.onion/by/tiffany-hsu}{Tiffany Hsu}

\begin{itemize}
\item
  Published March 26, 2020Updated April 3, 2020
\item
  \begin{itemize}
  \item
  \item
  \item
  \item
  \item
  \end{itemize}
\end{itemize}

More than three million people filed for
\href{https://www.nytimes3xbfgragh.onion/2020/04/03/upshot/coronavirus-jobless-rate-great-depression.html}{unemployment
benefits} last week, sending a collective shudder throughout the economy
that is unlike anything Americans have experienced.

The alarming numbers, in a report released by the Labor Department on
Thursday, provide some of the first hard data on the economic toll of
the
\href{https://www.nytimes3xbfgragh.onion/2020/04/03/upshot/coronavirus-jobless-rate-great-depression.html}{coronavirus}
pandemic, which has shut down whole swaths of American life faster than
government statistics can keep track.

Just three weeks ago, barely 200,000 people applied for jobless
benefits, a historically low number. In the half-century that the
government has tracked applications, the worst week ever, with 695,000
so-called initial claims, had been in 1982.

Thursday's figure of nearly 3.3 million set a grim record. ``A large
part of the economy just collapsed,'' said Ben Herzon, executive
director of IHS Markit, a business data and analytics firm.

Image

An entrance to the Bellagio in Las Vegas has been boarded
up.Credit...Joe Buglewicz for The New York Times

Image

An AT\&T store in Chicago is closed under a state order tightly limiting
business operations.Credit...Taylor Glascock for The New York Times

The numbers provided only the first hint of the economic cataclysm in
progress. Even comparatively optimistic forecasters expect millions more
lost jobs, and with them foreclosures, evictions and bankruptcies.
Thousands of businesses have closed in response to the pandemic, and
many will never reopen. Some economists say the decline in gross
domestic product this year could rival the worst years of the Great
Depression.

And there was fresh evidence on Thursday of the relentless course of the
virus itself. Cases in the United States now exceed 80,000, the most of
any nation, even China and Italy,
\href{https://www.nytimes3xbfgragh.onion/interactive/2020/us/coronavirus-us-cases.html}{according
to a New York Times database}, and more than 1,000 deaths across the
country have been linked to the virus.

At least 160 million people nationwide have been ordered to stay home.
Many hospitals are overwhelmed, while essential protective gear is in
short supply. ``We are the new global epicenter of the disease,'' said
Dr. Sara Keller, an infectious-disease specialist at Johns Hopkins
Medicine. ``Now all we can do is to slow the transmission as much as
possible.''

The situation in the New Orleans area is
\href{https://www.nytimes3xbfgragh.onion/2020/03/26/us/coronavirus-louisiana-new-orleans.html}{particularly
acute}, with the city reporting more than 800 cases, a higher total than
most states.

In New York, the state hardest hit, Gov. Andrew M. Cuomo reported a 40
percent increase in hospitalized patients in one day, to well over
5,000. The surge dashed hopes that had been raised a day before, when
Mr. Cuomo said the state's social-distancing measures seemed to be
slowing the growth in hospitalizations.

\hypertarget{latest-updates-the-coronavirus-outbreak-and-the-economy}{%
\section{\texorpdfstring{\href{https://www.nytimes3xbfgragh.onion/live/2020/09/08/business/stock-market-today-coronavirus?action=click\&pgtype=Article\&state=default\&region=MAIN_CONTENT_1\&context=storylines_live_updates}{Latest
Updates: The Coronavirus Outbreak and the
Economy}}{Latest Updates: The Coronavirus Outbreak and the Economy}}\label{latest-updates-the-coronavirus-outbreak-and-the-economy}}

\href{https://www.nytimes3xbfgragh.onion/live/2020/09/08/business/stock-market-today-coronavirus?action=click\&pgtype=Article\&state=default\&region=MAIN_CONTENT_1\&context=storylines_live_updates\#want-to-buy-a-used-car-so-does-everyone-else}{7m
ago}

\href{https://www.nytimes3xbfgragh.onion/live/2020/09/08/business/stock-market-today-coronavirus?action=click\&pgtype=Article\&state=default\&region=MAIN_CONTENT_1\&context=storylines_live_updates\#want-to-buy-a-used-car-so-does-everyone-else}{Want
to buy a used car? So does everyone else.}

\href{https://www.nytimes3xbfgragh.onion/live/2020/09/08/business/stock-market-today-coronavirus?action=click\&pgtype=Article\&state=default\&region=MAIN_CONTENT_1\&context=storylines_live_updates\#general-motors-takes-a-2-billion-stake-in-nikola}{31m
ago}

\href{https://www.nytimes3xbfgragh.onion/live/2020/09/08/business/stock-market-today-coronavirus?action=click\&pgtype=Article\&state=default\&region=MAIN_CONTENT_1\&context=storylines_live_updates\#general-motors-takes-a-2-billion-stake-in-nikola}{General
Motors takes a \$2 billion stake in Nikola.}

\href{https://www.nytimes3xbfgragh.onion/live/2020/09/08/business/stock-market-today-coronavirus?action=click\&pgtype=Article\&state=default\&region=MAIN_CONTENT_1\&context=storylines_live_updates\#states-are-slashing-funding-as-congress-stalls-on-aid}{2h
ago}

\href{https://www.nytimes3xbfgragh.onion/live/2020/09/08/business/stock-market-today-coronavirus?action=click\&pgtype=Article\&state=default\&region=MAIN_CONTENT_1\&context=storylines_live_updates\#states-are-slashing-funding-as-congress-stalls-on-aid}{States
are slashing funding as Congress stalls on aid.}

\href{https://www.nytimes3xbfgragh.onion/live/2020/09/08/business/stock-market-today-coronavirus?action=click\&pgtype=Article\&state=default\&region=MAIN_CONTENT_1\&context=storylines_live_updates}{See
more updates}

More live coverage:
\href{https://www.nytimes3xbfgragh.onion/2020/09/08/world/covid-19-coronavirus.html?action=click\&pgtype=Article\&state=default\&region=MAIN_CONTENT_1\&context=storylines_live_updates}{Global}

President Trump said the federal government planned to designate areas
as being at high, medium or low risk for spreading the virus to guide
local decisions on imposing or relaxing restrictions on movement and
commerce.

The terrifying speed of the U.S. economic collapse from the pandemic has
spurred lawmakers to action. Late Wednesday night, senators agreed on a
\$2 trillion aid package that would provide cash payments to nearly all
Americans and would expand the unemployment system, among other changes.
Final congressional approval is expected on Friday.

The legislative action has helped buoy financial markets. A three-day
rally has lifted stocks in the S\&P 500 index more than 17 percent,
including a rise of 6.2 percent on Thursday, though prices remain far
lower than they were a month ago.

As staggering as the figures are for jobless claims, they almost
certainly understate the problem. Some part-time and low-wage workers
don't qualify for unemployment benefits. Nor do gig workers, independent
contractors and the self-employed, although the emergency aid package
passed by the Senate would broaden eligibility to include many of them.
Others who do qualify may not know it. And the sudden rush of layoffs
\href{https://www.nytimes3xbfgragh.onion/2020/03/19/business/coronavirus-unemployment-states.html}{led
to jammed phone lines and overwhelmed computer servers} at unemployment
offices across the country, leaving many people unable to file claims.

The evening that Elise Quivey, 25, heard she was being furloughed from
her job in Chicago as a web designer for a cruise ship company, she
immediately clicked on the state's unemployment benefits website. The
pages wouldn't load. The next morning, as she tried to fill out the
online form, error messages kept flashing.

\includegraphics{https://static01.graylady3jvrrxbe.onion/images/2020/03/26/business/26virus-jobs4/26virus-jobs4-articleLarge.jpg?quality=75\&auto=webp\&disable=upscale}

Days of calling have resulted in nagging busy signals. She is hoping
that her claim made it through, and that she will receive aid within a
few weeks, but she is not optimistic.

``There's so many things up in the air right now, and it's so
stressful,'' she said. ``It's a wreck.''

Despite the glitches, Thursday's figures suggest the scale of the
problem. In a single week, the pandemic wiped out a year and a half of
job gains. The past two weeks' claims alone would be enough to push the
unemployment rate up to 5.7 percent from 3.5 percent in February --- a
half-century low that now seems like ancient history.

The worst could be yet to come. Mr. Herzon of IHS Markit said he
expected a similarly large number next Thursday, when the Labor
Department releases its report on new claims filed this week.

Some forecasters think the unemployment rate could hit 10 percent this
summer, which would equal the highest level from the last recession more
than a decade ago. Back then, it took nearly two years for the jobless
rate to reach that height.

``What is really hard to fathom is just how fast these numbers are going
to escalate,'' said Carl Tannenbaum, chief economist at Northern Trust.

Still, while there is little doubt that the numbers will get worse in
the short term, some economists remain optimistic that the pain will be
relatively short-lived. The congressional relief package is intended to,
in effect, press ``pause'' on the economy, allowing idled workers and
shuttered businesses to keep paying their bills so that they can spring
back quickly once the health crisis eases. If it works, the recovery
could be relatively swift; if it doesn't, the cascade of layoffs and
business failures could stretch on far longer.

Quintina Moore-Caraway, a ramp agent at George Bush Intercontinental
Airport in Houston, was at work on March 13 when her supervisor called
her over. She was being furloughed, without pay, at the end of her
shift.

``They said I could finish out my day on Friday, don't come in on
Saturday, and I haven't been back since, with no pay,'' she said.

Image

Quintina Moore-Caraway was furloughed from her job almost two weeks ago,
leaving her scrambling to find new employment.~Credit...Michael
Starghill Jr. for The New York Times

Ms. Moore-Caraway, 46, was barely getting by on the \$10 an hour she
earned at the airport. She has no savings, and no idea how she will pay
her \$688 rent bill on April 1. She hasn't been in the job long enough
to qualify for unemployment, and the few places still hiring during the
pandemic aren't near bus routes.

``Through all the hurricanes, floods, I've never seen anything like
this,'' she said. ``On the movies I have, not in real life.''

Low-wage workers --- many of them black, like Ms. Moore-Caraway, or
Hispanic --- have been hit especially hard by the sudden economic
reversal. Many work in the industries most affected by the outbreak,
such as restaurants and travel, and few can work from home. They are
also less likely to have sick leave or other paid time off, and they
have less money saved to help overcome a missed paycheck.

Black and Hispanic workers ``always bear the brunt'' of economic
slowdowns, said Alix Gould-Werth, a researcher at the Washington Center
for Equitable Growth, a left-leaning think tank. ``Now they're bearing
the brunt of these twin crises, the health crisis and the economic
crisis.''

Image

A New York State Department of Labor office in Brooklyn. States'
unemployment offices have been swamped.Credit...Hiroko Masuike/The New
York Times

Some help may be on the way for workers like Ms. Moore-Caraway. Under
the congressional
\href{https://www.nytimes3xbfgragh.onion/2020/03/25/us/politics/whats-in-coronavirus-stimulus-bill.html?action=click\&module=Spotlight\&pgtype=Homepage}{aid
package},
\href{https://www.nytimes3xbfgragh.onion/article/coronavirus-stimulus-package-questions-answers.html}{most
families} would receive \$1,200 per adult and \$500 per child in direct
payments. The bill would also increase unemployment benefits by \$600 a
week and extend how long laid-off workers could receive benefits. And it
would waive some requirements for receiving jobless benefits, like the
requirement that recipients look for work.

It would not, however, expand the food-assistance program formerly known
as food stamps.

An earlier relief bill, passed by Congress last week, provided \$1
billion to help state unemployment systems that are breaking under the
stress of record call volumes. Departments across the country reported
huge spikes in call volumes and online applications.

The surge in applications was a particular challenge because departments
were staffed --- and funded --- for a labor market that had until
recently been setting records for its strength.

Colorado's Department of Labor and Employment had 70 people working on
unemployment claims before the coronavirus outbreak. It added 90 people
to help respond to calls and process claims on Monday, pulling them off
other jobs in the department. Roughly 80 percent of workers are at home,
while the rest are in a call center in downtown Denver that during the
last recession had hundreds of workers dealing with claims.

Image

Riley Travis at the Colorado Department of Labor and Employment, which
has added dozens of people to help respond to a deluge of
calls.Credit...Daniel Brenner for The New York Times

``Where we maybe had eight months to prepare heading into the recession,
we had five days to respond to coronavirus,'' said Cher Haavind, the
department's deputy executive director.

On Monday, nearly 100,000 call attempts were made by 10 a.m., when the
call center normally receives 6,000 calls in an entire week. The
department put in place a new process under which those filing claims
submit their forms at specified times based on their last names. Even
so, Ms. Haavind said, the crush of applications has strained not just
the department's systems but also its employees.

``They are talking to stressed-out people, and they are also stressed
out,'' she said.

For laid-off workers, the anxiety is racking.

Mere weeks ago, Bill Copperfield had steady work installing drywall in
commercial buildings in Hawaii. Then he caught a cold and, in the
suddenly cautious world of coronavirus, was told not to come to work,
meaning he wasn't paid. By the time he was healthy again, the job had
shut down and the state government was telling nonessential workers to
stay home. He has tried repeatedly to file an unemployment claim, but
hasn't managed to get through.

``So right now I am three weeks without income and I've got my rent
coming up, I've got food I've got to buy,'' Mr. Copperfield said.
``Definitely I won't be paying bills this month.''

Mr. Copperfield, 45, has been laid off in the past, including during the
2008-9 housing crisis, in which he ended up losing his home to
foreclosure. But even then, he said, he was able to go out and find work
as a handyman or even sell fish he caught.

``At least then I could go out and hustle work, even if it wasn't in my
field,'' he said. ``Nobody can work right now, we're on like lockdown.
And even if I could find side work, I'd be putting my family at risk.''

Advertisement

\protect\hyperlink{after-bottom}{Continue reading the main story}

\hypertarget{site-index}{%
\subsection{Site Index}\label{site-index}}

\hypertarget{site-information-navigation}{%
\subsection{Site Information
Navigation}\label{site-information-navigation}}

\begin{itemize}
\tightlist
\item
  \href{https://help.nytimes3xbfgragh.onion/hc/en-us/articles/115014792127-Copyright-notice}{©~2020~The
  New York Times Company}
\end{itemize}

\begin{itemize}
\tightlist
\item
  \href{https://www.nytco.com/}{NYTCo}
\item
  \href{https://help.nytimes3xbfgragh.onion/hc/en-us/articles/115015385887-Contact-Us}{Contact
  Us}
\item
  \href{https://www.nytco.com/careers/}{Work with us}
\item
  \href{https://nytmediakit.com/}{Advertise}
\item
  \href{http://www.tbrandstudio.com/}{T Brand Studio}
\item
  \href{https://www.nytimes3xbfgragh.onion/privacy/cookie-policy\#how-do-i-manage-trackers}{Your
  Ad Choices}
\item
  \href{https://www.nytimes3xbfgragh.onion/privacy}{Privacy}
\item
  \href{https://help.nytimes3xbfgragh.onion/hc/en-us/articles/115014893428-Terms-of-service}{Terms
  of Service}
\item
  \href{https://help.nytimes3xbfgragh.onion/hc/en-us/articles/115014893968-Terms-of-sale}{Terms
  of Sale}
\item
  \href{https://spiderbites.nytimes3xbfgragh.onion}{Site Map}
\item
  \href{https://help.nytimes3xbfgragh.onion/hc/en-us}{Help}
\item
  \href{https://www.nytimes3xbfgragh.onion/subscription?campaignId=37WXW}{Subscriptions}
\end{itemize}
