Sections

SEARCH

\protect\hyperlink{site-content}{Skip to
content}\protect\hyperlink{site-index}{Skip to site index}

\href{/section/us}{U.S.}\textbar{}The Lost Month: How a Failure to Test
Blinded the U.S. to Covid-19

\url{https://nyti.ms/2JkLno5}

\begin{itemize}
\item
\item
\item
\item
\item
\item
\end{itemize}

\hypertarget{the-coronavirus-outbreak}{%
\subsubsection{\texorpdfstring{\href{https://www.nytimes3xbfgragh.onion/news-event/coronavirus?name=styln-coronavirus-national\&region=TOP_BANNER\&block=storyline_menu_recirc\&action=click\&pgtype=Article\&impression_id=328af5c0-f4bf-11ea-ad38-030638cf0afc\&variant=undefined}{The
Coronavirus
Outbreak}}{The Coronavirus Outbreak}}\label{the-coronavirus-outbreak}}

\begin{itemize}
\tightlist
\item
  live\href{https://www.nytimes3xbfgragh.onion/2020/09/11/world/covid-19-coronavirus.html?name=styln-coronavirus-national\&region=TOP_BANNER\&block=storyline_menu_recirc\&action=click\&pgtype=Article\&impression_id=328af5c1-f4bf-11ea-ad38-030638cf0afc\&variant=undefined}{Latest
  Updates}
\item
  \href{https://www.nytimes3xbfgragh.onion/interactive/2020/us/coronavirus-us-cases.html?name=styln-coronavirus-national\&region=TOP_BANNER\&block=storyline_menu_recirc\&action=click\&pgtype=Article\&impression_id=328af5c2-f4bf-11ea-ad38-030638cf0afc\&variant=undefined}{Maps
  and Cases}
\item
  \href{https://www.nytimes3xbfgragh.onion/interactive/2020/science/coronavirus-vaccine-tracker.html?name=styln-coronavirus-national\&region=TOP_BANNER\&block=storyline_menu_recirc\&action=click\&pgtype=Article\&impression_id=328b1cd0-f4bf-11ea-ad38-030638cf0afc\&variant=undefined}{Vaccine
  Tracker}
\item
  \href{https://www.nytimes3xbfgragh.onion/2020/09/10/us/politics/fda-coronavirus-vaccine.html?name=styln-coronavirus-national\&region=TOP_BANNER\&block=storyline_menu_recirc\&action=click\&pgtype=Article\&impression_id=328b1cd1-f4bf-11ea-ad38-030638cf0afc\&variant=undefined}{F.D.A.
  Regulators' Self-Defense}
\item
  \href{https://www.nytimes3xbfgragh.onion/2020/09/09/upshot/coronavirus-surprise-test-fees.html?name=styln-coronavirus-national\&region=TOP_BANNER\&block=storyline_menu_recirc\&action=click\&pgtype=Article\&impression_id=328b1cd2-f4bf-11ea-ad38-030638cf0afc\&variant=undefined}{Surprise
  Test Fees}
\end{itemize}

\includegraphics{https://static01.graylady3jvrrxbe.onion/images/2020/03/28/multimedia/28virus-testing-1/28virus-testing-1-articleLarge-v2.jpg?quality=75\&auto=webp\&disable=upscale}

\hypertarget{the-lost-month-how-a-failure-to-test-blinded-the-us-to-covid-19}{%
\section{The Lost Month: How a Failure to Test Blinded the U.S. to
Covid-19}\label{the-lost-month-how-a-failure-to-test-blinded-the-us-to-covid-19}}

Aggressive screening might have helped contain the coronavirus in the
United States. But technical flaws, regulatory hurdles and lapses in
leadership let it spread undetected for weeks.

Credit...Grant Hindsley for The New York Times

Supported by

\protect\hyperlink{after-sponsor}{Continue reading the main story}

By \href{https://www.nytimes3xbfgragh.onion/by/michael-d-shear}{Michael
D. Shear},
\href{https://www.nytimes3xbfgragh.onion/by/abby-goodnough}{Abby
Goodnough},
\href{https://www.nytimes3xbfgragh.onion/by/sheila-kaplan}{Sheila
Kaplan}, \href{https://www.nytimes3xbfgragh.onion/by/sheri-fink}{Sheri
Fink}, \href{https://www.nytimes3xbfgragh.onion/by/katie-thomas}{Katie
Thomas} and
\href{https://www.nytimes3xbfgragh.onion/by/noah-weiland}{Noah Weiland}

\begin{itemize}
\item
  Published March 28, 2020Updated April 1, 2020
\item
  \begin{itemize}
  \item
  \item
  \item
  \item
  \item
  \item
  \end{itemize}
\end{itemize}

WASHINGTON --- Early on, the dozen federal officials charged with
defending America against the
\href{https://www.nytimes3xbfgragh.onion/2020/04/01/us/politics/coronavirus-trump.html}{coronavirus}
gathered day after day in the White House Situation Room, consumed by
crises. They grappled with how to evacuate the United States consulate
in Wuhan, China, ban Chinese travelers and extract Americans from the
Diamond Princess and other cruise ships.

The members of the coronavirus task force typically devoted only five or
10 minutes, often at the end of contentious meetings, to talk about
testing, several participants recalled. The Centers for Disease Control
and Prevention, its leaders assured the others, had developed a
diagnostic model that would be rolled out quickly as a first step.

But as the deadly virus spread from China with ferocity across the
United States between late January and early March, large-scale testing
of people who might have been infected did not happen --- because of
technical flaws, regulatory hurdles, business-as-usual bureaucracies and
lack of leadership at multiple levels, according to interviews with more
than 50 current and former public health officials, administration
officials, senior scientists and company **** executives.

The result was a lost month, when the world's richest country --- armed
with some of the most highly trained scientists and infectious disease
specialists --- squandered its best chance of containing the virus's
spread. Instead, Americans were left largely blind to the scale of a
looming public health catastrophe.

The absence of robust screening until it was ``far too late'' revealed
failures across the government, said Dr. Thomas Frieden, the former
C.D.C. director. Jennifer Nuzzo, an epidemiologist at Johns Hopkins,
said the
\href{https://www.nytimes3xbfgragh.onion/2020/04/01/us/politics/coronavirus-trump.html}{Trump
administration} had ``incredibly limited'' views of the pathogen's
potential impact. Dr. Margaret Hamburg, the former commissioner of the
Food and Drug Administration, said the lapse enabled ``exponential
growth of cases.''

And Dr. Anthony S. Fauci, a top government scientist involved in the
fight against the virus, told members of Congress that the early
inability to test was ``a failing'' of the administration's response to
a deadly, global pandemic. ``Why,'' he asked later in
\href{https://www.sciencemag.org/news/2020/03/i-m-going-keep-pushing-anthony-fauci-tries-make-white-house-listen-facts-pandemic}{a
magazine interview}, ``were we not able to mobilize on a broader
scale?''

Across the government, they said, three agencies responsible for
detecting and combating threats like the coronavirus failed to ****
prepare quickly enough. Even as scientists looked at China and sounded
alarms, none of the agencies' directors conveyed the urgency required to
spur a no-holds-barred defense.

Dr. Robert R. Redfield, 68, a former military doctor and prominent AIDS
researcher who directs the C.D.C., trusted his veteran scientists to
create the world's most precise test for the coronavirus and share it
with state laboratories. When flaws in the test became apparent in
February, he promised a quick fix, though it took weeks to settle on a
solution.

\includegraphics{https://static01.graylady3jvrrxbe.onion/images/2020/03/28/multimedia/28virus-testing-3/merlin_169661985_f0a9537e-3c7e-4bfa-83bd-af48456415be-articleLarge.jpg?quality=75\&auto=webp\&disable=upscale}

The C.D.C. also tightly restricted who could get tested and was slow to
conduct ``community-based surveillance,'' a standard screening practice
to detect the virus's reach. Had the United States been able to track
its earliest movements and identify hidden hot spots, local quarantines
might have confined the disease.

Dr. Stephen Hahn, 60, the commissioner of the Food and Drug
Administration, enforced regulations that paradoxically made it tougher
for hospitals, private clinics and companies to deploy diagnostic tests
in an emergency. **** Other countries that had mobilized businesses were
performing tens of thousands of tests daily, compared with fewer than
100 on average **** in the United States, frustrating local health
officials, lawmakers and desperate Americans.

Image

Regulations at the F.D.A., led by Dr. Stephen Hahn, made it difficult
for hospitals to test patients at the same rate as in other
countries.Credit...Anna Moneymaker/The New York Times

Alex M. Azar II, who led the Department of Health and Human Services,
oversaw the two other agencies and coordinated the government's public
health response to the pandemic. While he grew frustrated as public
criticism over the testing issues intensified, he was unable to push
either agency to speed up or change course.

Mr. Azar, 52, who chaired the coronavirus task force until late
February, when Vice President Mike Pence took charge, had been at odds
for months with the White House over other issues. The task force's
chief liaison to the president was Mick Mulvaney, the acting White House
chief of staff, who was being forced out by Mr. Trump. Without
high-level interest --- or demands for action --- the testing issue
festered.

Image

Under Alex M. Azar II, the health secretary, the C.D.C. and F.D.A.
failed to break out of their business-as-usual habits.Credit...Anna
Moneymaker/The New York Times

At the start of that crucial lost month, when his government could have
rallied, the president was distracted by impeachment and dismissive of
the threat to the public's health or the nation's economy. By the end of
the month, Mr. Trump claimed the virus was about to dissipate in the
United States, saying: ``It's going to disappear. One day --- it's like
a miracle --- it will disappear.''

By early March, after federal officials finally announced changes to
expand testing, it was too late. With the early lapses, containment was
no longer an option. The tool kit of epidemiology would shift ---
lockdowns, social disruption, intensive medical treatment --- in hopes
of mitigating the harm.

Now, the United States has more than 100,000 coronavirus cases, the most
of any country in the world. Deaths are rising, cities are shuttered,
the economy is sputtering and everyday life is upended. And still, many
Americans sickened by the virus cannot get tested.

In a statement, Judd Deere, a White House spokesman, said that ``any
suggestion that President Trump did not take the threat of Covid-19
seriously or that the United States was not prepared is false.'' He
added that at Mr. Trump's direction, the administration had ``expanded
testing capacities.''

Dr. Bruce Aylward, a senior adviser at the World Health Organization,
led an expert team to China last month to research the mysterious new
virus. Testing, he said, was ``absolutely vital'' for understanding how
to defeat a disease --- what distinguishes it from others, the spectrum
of illness and, most important, its path through populations.

``You want to know whether or not you have it,'' Dr. Aylward said. ``You
want to know whether the people around you have it. Because you know
what? Then you could stop it.''

\hypertarget{latest-updates-the-coronavirus-outbreak}{%
\section{\texorpdfstring{\href{https://www.nytimes3xbfgragh.onion/2020/09/11/world/covid-19-coronavirus.html?action=click\&pgtype=Article\&state=default\&region=MAIN_CONTENT_1\&context=storylines_live_updates}{Latest
Updates: The Coronavirus
Outbreak}}{Latest Updates: The Coronavirus Outbreak}}\label{latest-updates-the-coronavirus-outbreak}}

Updated 2020-09-12T06:09:34.374Z

\begin{itemize}
\tightlist
\item
  \href{https://www.nytimes3xbfgragh.onion/2020/09/11/world/covid-19-coronavirus.html?action=click\&pgtype=Article\&state=default\&region=MAIN_CONTENT_1\&context=storylines_live_updates\#link-dfb8a16}{Fauci
  cautions the virus could disrupt life in the U.S. until `maybe even
  towards the end of 2021.'}
\item
  \href{https://www.nytimes3xbfgragh.onion/2020/09/11/world/covid-19-coronavirus.html?action=click\&pgtype=Article\&state=default\&region=MAIN_CONTENT_1\&context=storylines_live_updates\#link-7104d154}{From
  Asia to Africa, China promotes its vaccine candidates to win friends.}
\item
  \href{https://www.nytimes3xbfgragh.onion/2020/09/11/world/covid-19-coronavirus.html?action=click\&pgtype=Article\&state=default\&region=MAIN_CONTENT_1\&context=storylines_live_updates\#link-393ad215}{The
  other way the virus will kill: hunger.}
\end{itemize}

\href{https://www.nytimes3xbfgragh.onion/2020/09/11/world/covid-19-coronavirus.html?action=click\&pgtype=Article\&state=default\&region=MAIN_CONTENT_1\&context=storylines_live_updates}{See
more updates}

More live coverage:
\href{https://www.nytimes3xbfgragh.onion/live/2020/09/11/business/stock-market-today-coronavirus?action=click\&pgtype=Article\&state=default\&region=MAIN_CONTENT_1\&context=storylines_live_updates}{Markets}

``You can't stop it,'' he warned, ``if you can't see it.''

Image

Medical workers with a patient in Wuhan, China, in January. The U.S.
health secretary would declare a public health emergency at the end of
the month.Credit...EPA, via Shutterstock

\hypertarget{a-startling-setback}{%
\subsection{A Startling Setback}\label{a-startling-setback}}

The first time Dr. Robert Redfield heard about the severity of the virus
from his Chinese counterparts was around New Year's Day, when he was on
vacation with his family. He spent so much time on the phone that they
barely saw him. And what he heard rattled him; in one grim conversation
about the virus days later, George F. Gao, the director of the Chinese
Center for Disease Control and Prevention, burst into tears.

Dr. Redfield, a longtime AIDS researcher, had
\href{https://www.nytimes3xbfgragh.onion/2018/03/21/health/cdc-robert-redfield-aids.html}{never
run a government agency} before his appointment to lead the C.D.C. in
2018. Until then, his biggest priorities had been fighting the opioid
epidemic and the spread of H.I.V. Suddenly, a man who preferred treating
patients in Haiti or Africa to being in the public glare was facing a
new pandemic threat.

At first, Dr. Redfield's agency moved quickly.

On Jan. 7, the C.D.C. created an
``\href{https://www.who.int/docs/default-source/coronaviruse/situation-reports/20200123-sitrep-3-2019-ncov.pdf}{incident
management system}'' for the coronavirus and advised travelers to Wuhan
to take precautions. By Jan. 20, just two weeks after Chinese scientists
shared the genetic sequence of the virus, the C.D.C. had developed its
own test, as usual, and deployed it to detect the country's first
coronavirus case.

``That's our prime mission,'' Dr. Redfield said later in an interview,
``to get eyes on this thing.''

Assessing the virus would prove challenging. It was so new that
scientists had little information to work with. China provided limited
data, and rebuffed an early attempt by Mr. Azar and Dr. Redfield to send
C.D.C. experts there to learn more. That the virus could cause no
symptoms and still spread --- something not initially known --- made it
all the more difficult to understand.

To identify the virus, the C.D.C. test used three small genetic
sequences to match up with portions of a virus's genome extracted from a
swab. A German-developed test that the W.H.O. was distributing to other
countries used just two, potentially making it less precise.

But soon after the F.D.A. cleared the C.D.C. to share its test kits with
state health department labs, some discovered a problem. The third
sequence, or ``probe,'' gave inconclusive results. While the C.D.C.
explored the cause --- contamination or a design issue --- it told those
state labs to stop testing.

The startling setback stalled the C.D.C.'s efforts to track the virus
when it mattered most. By mid-February, the nation was testing only
about 100 samples per day, according to
\href{https://www.cdc.gov/coronavirus/2019-ncov/cases-updates/testing-in-us.html}{the
C.D.C.'s website}.

Dr. Redfield played down the problem in task force meetings and
conversations with Mr. Azar, assuring him it would be fixed quickly,
several administration officials said.

With capacity so limited, the C.D.C.'s criteria for who was tested
remained extremely narrow for weeks to come: only people who had
recently traveled to China or had been in contact with someone who had
the virus.

The lack of tests in the states also meant local public health officials
could not use another essential epidemiological tool: surveillance
testing. To see where the virus might be hiding, nasal swab samples from
people screened for the common flu would also be checked for the
coronavirus.

The C.D.C. announced a plan on Feb. 14 to perform the screening in five
high-risk cities: New York, Chicago, Los Angeles, San Francisco and
Seattle. An agency official
\href{https://www.cdc.gov/media/releases/2020/t0214-covid-19-update.html.html}{said
it could provide} ``an early warning signal to trigger a change in our
response strategy.'' But most of the cities could not carry it out.

``Had we had done more testing from the very beginning and caught cases
earlier,'' said Dr. Nuzzo, of Johns Hopkins, ``we would be in a far
different place.''

The consequences became clear by the end of February. For the first
time, someone with no known exposure to the virus or history of travel
tested positive, in the Seattle area, where the U.S.'s first case had
been detected more than a month earlier. The virus had probably
\href{https://www.nytimes3xbfgragh.onion/2020/03/10/us/coronavirus-testing-delays.html}{been
spreading}there and elsewhere for weeks, researchers later concluded.
Without a more complete picture of who had been infected, public health
workers could not do ``contact tracing'' --- finding all those with whom
any contagious people had interacted and then quarantining them to stop
further transmission.

The C.D.C. gave little thought to adopting the test being used by the
W.H.O. The C.D.C.'s test was working in its own lab --- still processing
samples from states --- which gave agency officials confidence. Dr. Anne
Schuchat, the agency's principal deputy director, would later say that
the C.D.C. did not think ``we needed somebody else's test.''

And the German-designed W.H.O. test had not been through the American
regulatory approval process, which would take time.

Throughout February, Dr. Redfield shuttled between Atlanta, where the
C.D.C. is based, and Washington, holding multiple calls every day with
Mr. Azar and participating in the coronavirus task force.

Mr. Azar's take-charge style contrasted with the more deliberative
manner of Dr. Redfield, who lacked the kind of commanding television
presence that impressed Mr. Trump. He was ``a consensus person,'' as one
colleague described him, who sought to avoid conflict. He relied heavily
on some of the C.D.C.'s career scientists, like Dr. Schuchat and Dr.
Nancy Messonnier, the director of the agency's National Center for
Immunization and Respiratory Diseases.

Image

Dr. Nancy Messonnier, director of the National Center for Immunization
and Respiratory Diseases, has taken a public role in the
crisis.Credit...Amanda Voisard/Reuters

Under scrutiny from Congress, Dr. Redfield offered reassurances.
Responding on Feb. 24 to a
\href{https://kilmer.house.gov/imo/media/doc/Letter\%20to\%20CDC\%20Director\%20Redfield\%202.3.2020.pdf}{letter
from 49 members of Congress} about the need for testing in the states,
he wrote, ``CDC's aggressive response enables us to identify potential
cases early and make sure that they are properly handled.''

Days later, his agency provided a workaround, telling state and local
health department labs that they could finally begin testing. Rather
than awaiting replacements, they should use their C.D.C. test kits and
leave out the problematic third probe.

Meanwhile, the agency's epidemiologists were growing more concerned as
the virus spread in South Korea and Italy. On Feb. 25, Dr. Messonnier
gave a briefing with a
\href{https://www.nytimes3xbfgragh.onion/2020/03/07/us/politics/trump-coronavirus.html}{much
blunter warning} than usual. ``Disruption to everyday life might be
severe,'' she said.

Mr. Trump, returning from a trip to India, was furious, according to
senior administration officials. Later that day, Mr. Azar seemed to be
\href{https://www.statnews.com/2020/02/25/cdc-expects-community-spread-of-coronavirus-as-top-official-warns-disruptions-could-be-severe/}{tamping
down the level of concern}. All Dr. Messonnier had meant, he said at a
news conference, was that people should ``start thinking about, in their
own lives, what that might involve.''

``Might,'' Mr. Azar repeated emphatically. ``Might involve.''

Image

The C.D.C.'s flawed testing kit for the new coronavirus.Credit...Centers
for Disease Control and Prevention

\hypertarget{barriers-to-testing}{%
\subsection{Barriers to Testing}\label{barriers-to-testing}}

Dr. Stephen Hahn's first day as F.D.A. commissioner came just six weeks
before Mr. Azar declared a public health emergency on Jan. 31. A
\href{https://www.nytimes3xbfgragh.onion/2019/11/01/health/fda-commissioner-hahn.html}{radiation
oncologist and researcher} who helped turn around MD Anderson in
Houston, one of the nation's leading cancer centers, Dr. Hahn had come
to Washington to oversee a sprawling federal agency that regulates
everything from lifesaving therapies to dog food.

But overnight, his mission --- to manage 15,000 employees in a culture
defined by precision and caution --- was upended. A pathogen that Mr.
Trump would later call the ``invisible enemy'' was hurtling toward the
United States. It would fall to the newly arrived Dr. Hahn to help build
a huge national capacity for testing by academic and private labs.

Instead, under his leadership, the F.D.A. became a significant
roadblock, according to current and former officials as well as
researchers and doctors at laboratories around the country.

Private-sector tests were supposed to be the next tier after the C.D.C.
fulfilled its obligation to jump-start screening at public labs. In
other countries hit hard by the coronavirus, governments acted quickly
to speed tests to their populations. In South Korea, for example,
regulators in early February
\href{https://www.nytimes3xbfgragh.onion/reuters/2020/03/18/world/asia/18reuters-health-coronavirus-testing-specialreport.html}{summoned
executives from 20 medical manufacturers}, easing rules as they demanded
tests.

But Dr. Hahn took a cautious approach. He was not proactive in reaching
out to manufacturers, and instead deferred to his scientists, following
the F.D.A.'s often cumbersome methods for approving medical screening.

Even the nation's public health labs were looking for the F.D.A.'s help.
``We are now many weeks into the response with still no diagnostic or
surveillance test available outside of C.D.C. for the vast majority of
our member laboratories,'' Scott Becker, chief executive of the
Association of Public Health Laboratories, wrote to Mr. Hahn in late
February. ``We believe a more expeditious route is needed at this
time.''

Ironically, it was Mr. Azar's emergency declaration that established the
rules Dr. Hahn insisted on following. Designed to make it easier for
drugmakers to pursue vaccines and other therapies during a crisis, such
a declaration lets the F.D.A. speed approvals that could otherwise take
a year or more.

But the emergency announcement created a new barrier for hospitals and
laboratories that wanted to create their own tests to diagnose the
coronavirus. Usually, they faced minimal federal regulation. But once
Mr. Azar took action, they were subject to an F.D.A. process called an
``emergency use authorization.''

Even though researchers around the country quickly began creating tests
that could diagnose Covid-19, many said they were hindered by the
F.D.A.'s approval process. The new tests sat unused at labs around the
country.

Stanford was one of them. Researchers at the world-renowned university
had a working test by February, based on protocols published by the
W.H.O. The organization had
\href{https://www.who.int/dg/speeches/detail/who-director-general-s-opening-remarks-at-the-media-briefing-on-2019-novel-coronavirus}{already
delivered} more than 250,000 of the German-designed tests to 70
laboratories around the world, and doctors at the Stanford lab wanted to
be prepared for a pandemic.

``Even if it didn't come, it would be better to be ready than not to be
ready,'' said Dr. Benjamin Pinsky, the lab's medical director.

But in the face of what he called ``relatively tight'' rules at the
F.D.A., Dr. Pinsky and his colleagues decided against even trying to win
permission. The Stanford clinical lab would not begin testing
coronavirus samples until early March, when Dr. Hahn finally relaxed the
rules.

Executives at bioMérieux, a French diagnostics company, had a similar
experience. The company makes a countertop testing system, BioFire, that
is routinely used to check for the flu and other respiratory illnesses
in 1,700 hospitals around the country. It can provide results in about
45 minutes.

``A lot of us said, you know, your typical E.U.A. is just much too
demanding,'' said Dr. Mark Miller, the company's chief medical officer,
referring to the emergency approval. ``It's going to take much too much
time. And can't you do something to shorten that?''

Officials at the F.D.A. tried to be responsive, Dr. Miller said. But
rather than throw out the rules, the agency only modified the regulatory
requirements, still requiring weeks of discussions and negotiations.

After conversations with the F.D.A. in mid-February, the company
\href{https://www.biomerieux.com/en/biomerieux-receives-emergency-use-authorization-biofirer-covid-19-test}{received
emergency approval} for its BioFire test on March 24. (The company also
began talking to the F.D.A. in January about another type of test, but
decided not to pursue it in the United States for now.) Dr. Miller said
that while he was ultimately satisfied with the F.D.A.'s actions, the
overall response by the government was too slow, especially when it came
to logistical questions like getting enough testing supplies to those
who needed them.

``You've got other countries --- and I'm sorry, unfortunately, the U.S.
is one of those --- where they've been slow, disorganized,'' he said.
``There are still not enough tests available there to test everybody who
needs it.''

In an emailed statement, Dr. Hahn maintained that his agency had moved
as quickly as it safely could to ensure that tests would be accurate.
``Since the early days of this pandemic,'' he said, ``the F.D.A.'s doors
have always been and still remain open to test developers.''

Image

Mr. Azar speaking about the public health response in January after five
cases were confirmed in the United States.Credit...Samuel Corum/Getty
Images

\hypertarget{a-lack-of-trust}{%
\subsection{A Lack of Trust}\label{a-lack-of-trust}}

Alex Azar had sounded confident at the end of January. At a news
conference in the hulking H.H.S. headquarters in Washington, he said he
had the government's response to the new coronavirus under control,
pointing out high-ranking jobs he had held in the department during the
2003 SARS outbreak and other infectious threats.

``I know this playbook well,'' he
\href{https://www.hhs.gov/about/leadership/secretary/speeches/2020-speeches/remarks-at-coronavirus-press-briefing.html}{told
reporters}.

A Yale-trained lawyer who once served as the top attorney at the health
department, Mr. Azar
\href{https://www.nytimes3xbfgragh.onion/2017/11/26/us/politics/alex-azar-senate-confirmation-hearing-hhs.html}{had
spent a decade as a top executive at Eli Lilly}, one of the world's
largest drug companies. But he caught Mr. Trump's attention in part
because of other credentials: After law school, Mr. Azar was a clerk for
some of the nation's most conservative judges, including
Justice\href{https://en.wikipedia.org/wiki/Antonin_Scalia}{Antonin
Scalia} of the
\href{https://en.wikipedia.org/wiki/United_States_Supreme_Court}{Supreme
Court}. And for two years, he worked as Ken Starr's deputy on the
Clinton Whitewater investigation.

As Mr. Trump's second health secretary, confirmed at the beginning of
2018, Mr. Azar has been quick to compliment the president and focus on
the issues he cares about: lowering drug prices and fighting opioid
addiction. On Feb. 6 --- even as the
\href{https://www.marketwatch.com/story/coronavirus-update-565-deaths-more-than-28000-cases-worldwide-yum-china-reports-significant-interruption-2020-02-06}{W.H.O.
announced} that there were more than 28,000 coronavirus cases around the
globe --- Mr. Azar was in the second row in the White House's East Room,
demonstrating his loyalty to the president as Mr. Trump
\href{https://www.nytimes3xbfgragh.onion/2020/02/06/us/politics/trump-impeachment.html}{claimed
vindication} from his impeachment acquittal the day before and lashed
out at ``evil'' lawmakers and the F.B.I.'s ``top scum.''

As public attention on the virus threat intensified in January and
February, Mr. Azar grew increasingly frustrated about the harsh
spotlight on his department and the leaders of agencies who reported to
him, according to people familiar with the response to the virus inside
the agencies.

Described as a prickly boss by some administration officials, Mr. Azar
has had a longstanding ****
\href{https://www.nytimes3xbfgragh.onion/2019/12/10/us/politics/trump-seema-verma-azar.html}{feud
with Seema Verma}, the Medicare and Medicaid chief, who recently became
a regular presence at Mr. Trump's televised briefings on the pandemic.
Mr. Azar did not include Dr. Hahn on the virus task force he led, though
some of the F.D.A. commissioner's aides participated in H.H.S. meetings
on the subject.

And tensions grew between the secretary and Dr. Redfield as the testing
issue persisted. Mr. Azar and Dr. Redfield have been on the phone as
often as a half-dozen times a day. But throughout February, as the
C.D.C. test faltered, Mr. Azar became convinced that Dr. Redfield's
agency was providing him with inaccurate information about testing that
the secretary repeated publicly, according to several administration
officials.

In one instance, Mr. Azar appeared on
\href{https://www.cbsnews.com/news/transcript-alex-azar-on-face-the-nation-march-1-2020/?utm_source=dlvr.it\&utm_medium=twitter}{Sunday
morning news programs} and said that more than 3,600 people had been
tested for the virus. In fact, the real number was much smaller because
many patients were tested multiple times, an error the C.D.C. had to
correct in congressional testimony that week. One health department
official said Mr. Azar was repeatedly assured that the C.D.C.'s test
would be widely available within a week or 10 days, only to be given the
same promise a week later.

Asked about criticism of his agency's response to the pandemic, Dr.
Redfield said: ``I'm personally not focused on whether they're pointing
fingers here or there. We're focused on doing all we can to get through
this outbreak as quickly as possible and keep America safe.''

For all Mr. Azar's complaints, however, he continued to defer to the
scientists at the two agencies, according to several administration
officials. Mr. Azar's allies said he was told by Dr. Redfield and Dr.
Fauci that the C.D.C. had the resources it needed, that there was no
reason to believe the virus was spreading through the country from
person to person and that it was important to test only people who met
certain criteria.

But even in the face of a crescendo of complaints from doctors and
health care researchers around the country, Mr. Azar failed to push
those under him to do the one thing that could have helped: broader
testing.

In a statement, Caitlin Oakley, Mr. Azar's spokeswoman, said that the
secretary had ``empowered and followed the guidance of world-renowned
U.S. scientists'' on the testing issue. ``Any insinuation that Secretary
Azar did not respond with needed urgency to the response or testing
efforts,'' she said, ``are just plain wrong and disproven by the
facts.''

By Feb. 26, Dr. Fauci was concerned that the stalled testing had become
an urgent issue that needed to be addressed. He called Brian Harrison,
Mr. Azar's chief of staff, and asked him to gather the group of
officials overseeing screening efforts.

Around noon on Feb. 27, Dr. Hahn, Dr. Redfield and top aides from the
F.D.A. and H.H.S. dialed in to a conference call. Mr. Harrison began
with an ultimatum: No one leaves until we resolve the lag in testing. We
don't have answers and we need them, one senior administration official
recalled him saying. Get it done.

By the end of the day, the group agreed that the F.D.A. should loosen
regulations so that hospitals and independent labs could move forward
quickly with their own tests.

But the evening before, Mr. Azar had been effectively removed as the
leader of the task force when Mr. Trump abruptly put Mr. Pence in
charge, a decision so last-minute that even the top health officials in
the White House learned of it while watching the announcement.

Image

President Trump announcing the rollout of additional tests at the C.D.C.
in early March.Credit...T.J. Kirkpatrick for The New York Times

\hypertarget{a-tacit-acknowledgment}{%
\subsection{A Tacit Acknowledgment}\label{a-tacit-acknowledgment}}

Previous presidents have moved quickly to confront disease threats from
inside the White House by installing a ``czar'' to manage the effort.

During an outbreak of the Ebola virus in 2014, President Barack Obama
tapped Ron Klain, his vice president's former chief of staff, to direct
the response from the West Wing. Mr. Obama later created an office of
global health security inside the National Security Council to
coordinate future crises.

``If you look historically in the United States when it is challenged
with something like this --- whether it's H.I.V. crises, whether it's
pandemic, whether it's whatever --- man, they pull out all the stops
across the system and they make it work,'' said Dr. Aylward, the W.H.O.
epidemiologist.

But faced with the coronavirus, Mr. Trump chose not to have the White
House lead the planning until nearly two months after it began. Mr.
Obama's global health office had been disbanded a year earlier. And
until Mr. Pence took charge, the task force lacked a single White House
official with the power to compel action.

Since then, testing has ramped up quickly, with nearly 100 labs at
hospitals and elsewhere performing it. On Friday, the health care giant
Abbott
\href{https://abbott.mediaroom.com/2020-03-27-Abbott-Launches-Molecular-Point-of-Care-Test-to-Detect-Novel-Coronavirus-in-as-Little-as-Five-Minutes}{said
it had received emergency approval} for a portable test that could
detect the virus in five minutes.

The president boasted on Tuesday that the United States had ``created a
new system that now we are doing unbelievably big numbers'' of tests for
the virus. The U.S., he said, had done more testing for the coronavirus
in the last eight days than South Korea had done in eight weeks.

Image

Vice President Mike Pence discussing 15-day federal guidelines this
month to ``slow the spread'' of the pandemic.Credit...Doug Mills/The New
York Times

Yet hospitals and clinics across the country still must deny tests to
those with milder symptoms, trying to save them for the most serious
cases, and they often wait a week for results. In tacit acknowledgment
of the shortage, Mr. Trump asked South Korea's president on Monday to
send as many test kits as possible from the 100,000 produced there
daily, more than the country needs.

Public health experts reacted positively to the increased capacity. But
having the ability to diagnose the disease three months after it was
first disclosed by China does little to address why the United States
was unable to do so sooner, when it might have helped reduce the toll of
the pandemic.

``Testing is the crack that split apart the rest of the response, when
it should have tied everything together,'' said Dr. Nahid Bhadelia, ​the
medical director of the Special Pathogens Unit at Boston University
School of Medicine.

``It seeps into every other aspect of our response, touches all of us,''
she said. ``The delay of the testing has impacted the response across
the board.''

Eric Lipton contributed reporting from Washington and Choe Sang-Hun from
Seoul, South Korea.

Advertisement

\protect\hyperlink{after-bottom}{Continue reading the main story}

\hypertarget{site-index}{%
\subsection{Site Index}\label{site-index}}

\hypertarget{site-information-navigation}{%
\subsection{Site Information
Navigation}\label{site-information-navigation}}

\begin{itemize}
\tightlist
\item
  \href{https://help.nytimes3xbfgragh.onion/hc/en-us/articles/115014792127-Copyright-notice}{©~2020~The
  New York Times Company}
\end{itemize}

\begin{itemize}
\tightlist
\item
  \href{https://www.nytco.com/}{NYTCo}
\item
  \href{https://help.nytimes3xbfgragh.onion/hc/en-us/articles/115015385887-Contact-Us}{Contact
  Us}
\item
  \href{https://www.nytco.com/careers/}{Work with us}
\item
  \href{https://nytmediakit.com/}{Advertise}
\item
  \href{http://www.tbrandstudio.com/}{T Brand Studio}
\item
  \href{https://www.nytimes3xbfgragh.onion/privacy/cookie-policy\#how-do-i-manage-trackers}{Your
  Ad Choices}
\item
  \href{https://www.nytimes3xbfgragh.onion/privacy}{Privacy}
\item
  \href{https://help.nytimes3xbfgragh.onion/hc/en-us/articles/115014893428-Terms-of-service}{Terms
  of Service}
\item
  \href{https://help.nytimes3xbfgragh.onion/hc/en-us/articles/115014893968-Terms-of-sale}{Terms
  of Sale}
\item
  \href{https://spiderbites.nytimes3xbfgragh.onion}{Site Map}
\item
  \href{https://help.nytimes3xbfgragh.onion/hc/en-us}{Help}
\item
  \href{https://www.nytimes3xbfgragh.onion/subscription?campaignId=37WXW}{Subscriptions}
\end{itemize}
