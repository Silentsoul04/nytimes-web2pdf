Sections

SEARCH

\protect\hyperlink{site-content}{Skip to
content}\protect\hyperlink{site-index}{Skip to site index}

\href{https://www.nytimes3xbfgragh.onion/section/style}{Style}

\href{https://myaccount.nytimes3xbfgragh.onion/auth/login?response_type=cookie\&client_id=vi}{}

\href{https://www.nytimes3xbfgragh.onion/section/todayspaper}{Today's
Paper}

\href{/section/style}{Style}\textbar{}Stress Baking More Than Usual?

\url{https://nyti.ms/2WUJNkX}

\begin{itemize}
\item
\item
\item
\item
\item
\item
\end{itemize}

\hypertarget{the-coronavirus-outbreak}{%
\subsubsection{\texorpdfstring{\href{https://www.nytimes3xbfgragh.onion/news-event/coronavirus?name=styln-coronavirus-national\&region=TOP_BANNER\&block=storyline_menu_recirc\&action=click\&pgtype=Article\&impression_id=808c19c0-f267-11ea-aa98-cfc271ad1787\&variant=undefined}{The
Coronavirus
Outbreak}}{The Coronavirus Outbreak}}\label{the-coronavirus-outbreak}}

\begin{itemize}
\tightlist
\item
  live\href{https://www.nytimes3xbfgragh.onion/2020/09/08/world/covid-19-coronavirus.html?name=styln-coronavirus-national\&region=TOP_BANNER\&block=storyline_menu_recirc\&action=click\&pgtype=Article\&impression_id=808c40d0-f267-11ea-aa98-cfc271ad1787\&variant=undefined}{Latest
  Updates}
\item
  \href{https://www.nytimes3xbfgragh.onion/interactive/2020/us/coronavirus-us-cases.html?name=styln-coronavirus-national\&region=TOP_BANNER\&block=storyline_menu_recirc\&action=click\&pgtype=Article\&impression_id=808c40d1-f267-11ea-aa98-cfc271ad1787\&variant=undefined}{Maps
  and Cases}
\item
  \href{https://www.nytimes3xbfgragh.onion/interactive/2020/science/coronavirus-vaccine-tracker.html?name=styln-coronavirus-national\&region=TOP_BANNER\&block=storyline_menu_recirc\&action=click\&pgtype=Article\&impression_id=808c40d2-f267-11ea-aa98-cfc271ad1787\&variant=undefined}{Vaccine
  Tracker}
\item
  \href{https://www.nytimes3xbfgragh.onion/2020/09/02/your-money/eviction-moratorium-covid.html?name=styln-coronavirus-national\&region=TOP_BANNER\&block=storyline_menu_recirc\&action=click\&pgtype=Article\&impression_id=808c40d3-f267-11ea-aa98-cfc271ad1787\&variant=undefined}{Eviction
  Moratorium}
\item
  \href{https://www.nytimes3xbfgragh.onion/interactive/2020/09/02/magazine/food-insecurity-hunger-us.html?name=styln-coronavirus-national\&region=TOP_BANNER\&block=storyline_menu_recirc\&action=click\&pgtype=Article\&impression_id=808c40d4-f267-11ea-aa98-cfc271ad1787\&variant=undefined}{American
  Hunger}
\end{itemize}

Advertisement

\protect\hyperlink{after-top}{Continue reading the main story}

Supported by

\protect\hyperlink{after-sponsor}{Continue reading the main story}

\hypertarget{stress-baking-more-than-usual}{%
\section{Stress Baking More Than
Usual?}\label{stress-baking-more-than-usual}}

Confined to their homes, Americans are kneading dough.

\includegraphics{https://static01.graylady3jvrrxbe.onion/images/2020/04/02/fashion/30VIRUS-STRESSBREAD-knead/30VIRUS-STRESSBREAD-knead-articleLarge-v2.jpg?quality=75\&auto=webp\&disable=upscale}

By Alexandra Marvar

\begin{itemize}
\item
  March 30, 2020
\item
  \begin{itemize}
  \item
  \item
  \item
  \item
  \item
  \item
  \end{itemize}
\end{itemize}

As the country
\href{https://www.nytimes3xbfgragh.onion/interactive/2020/us/coronavirus-stay-at-home-order.html}{shuts
down city by city} and families are being confined to their homes, one
thing's for sure: Americans are baking.

``I came home from work last week one night, I had just anchored `All
Things Considered,' and I was feeling anxious and like everything is
spinning out of control,'' said Mary Louise Kelly, 49, the NPR anchor,
over the phone from her home in Washington, D.C. ``It's always a joy,
when you do something abstract like broadcasting, to do something
physical. Also, it just seemed so damned wholesome --- a loaf of banana
bread. It felt like what we needed, all warm and golden, like this must
be a force for good in uncertain times.''

Ms. Kelly adapted a pumpkin bread recipe, throwing in chopped-up
caramels, using brown sugar instead of white, all-purpose flour instead
of wheat, and in place of pumpkin, three ``extremely ripe'' bananas.

Then she
\href{https://twitter.com/NPRKelly/status/1240452816117866496}{posted}
to Twitter: ``Anyone else in their kitchen sipping red wine and
aggressively baking banana bread at 9:40 p.m.? No? Just me?
\#coronavirusbaking.''

Quickly, thousands of likes and comments --- and more pictures of fresh
baked goods --- poured in. ``It was a moment where absolutely everybody
felt the same way, where Twitter hits a nerve, and you think `Oh my God,
I'm not alone, there's someone else out there with a glass of wine,
baking at 9 o'clock at night,'' Ms. Kelly said.

In the past few weeks, social media has been flooded with the photos of
``\href{https://www.instagram.com/henrycavill/?utm_source=ig_embed}{isolation
loaves}'' and
``\href{https://twitter.com/hashtag/quarantinecookies?src=hash}{quarantine
cookies}.'' Baking necessities like
\href{https://www.washingtonpost.com/news/voraciously/wp/2020/03/24/people-are-baking-bread-like-crazy-and-now-were-running-out-of-flour-and-yeast/}{flour
and yeast are in short supply} and best-selling bread makers that were
\href{https://mashable.com/shopping/march-17-bread-makers-on-sale-amazon-best-buy/}{in
stock} as of mid-March
\href{https://people.com/food/bread-makers-viral-amazon/}{are now sold
out} across the internet.

But for those who can still get the ingredients, baking provides a
combination of distraction, comfort and --- especially with bread
recipes, which can take days to complete --- something to look forward
to.

\hypertarget{latest-updates-the-coronavirus-outbreak}{%
\section{\texorpdfstring{\href{https://www.nytimes3xbfgragh.onion/2020/09/08/world/covid-19-coronavirus.html?action=click\&pgtype=Article\&state=default\&region=MAIN_CONTENT_1\&context=storylines_live_updates}{Latest
Updates: The Coronavirus
Outbreak}}{Latest Updates: The Coronavirus Outbreak}}\label{latest-updates-the-coronavirus-outbreak}}

Updated 2020-09-09T06:37:11.014Z

\begin{itemize}
\tightlist
\item
  \href{https://www.nytimes3xbfgragh.onion/2020/09/08/world/covid-19-coronavirus.html?action=click\&pgtype=Article\&state=default\&region=MAIN_CONTENT_1\&context=storylines_live_updates\#link-313b443d}{AstraZeneca
  halts a vaccine trial to investigate a participant's illness.}
\item
  \href{https://www.nytimes3xbfgragh.onion/2020/09/08/world/covid-19-coronavirus.html?action=click\&pgtype=Article\&state=default\&region=MAIN_CONTENT_1\&context=storylines_live_updates\#link-4438dd7}{Facing
  a surge in cases, Britain plans to limit most gatherings to six
  people.}
\item
  \href{https://www.nytimes3xbfgragh.onion/2020/09/08/world/covid-19-coronavirus.html?action=click\&pgtype=Article\&state=default\&region=MAIN_CONTENT_1\&context=storylines_live_updates\#link-679303d7}{Nine
  drugmakers pledge to thoroughly vet any coronavirus vaccine.}
\end{itemize}

\href{https://www.nytimes3xbfgragh.onion/2020/09/08/world/covid-19-coronavirus.html?action=click\&pgtype=Article\&state=default\&region=MAIN_CONTENT_1\&context=storylines_live_updates}{See
more updates}

More live coverage:
\href{https://www.nytimes3xbfgragh.onion/live/2020/09/08/business/stock-market-today-coronavirus?action=click\&pgtype=Article\&state=default\&region=MAIN_CONTENT_1\&context=storylines_live_updates}{Markets}

This is especially true of sourdough, which doesn't require yeast but
does require patience over several days of fermentation to make a
starter, and hours to autolyse and proof the dough and let it rise. It's
the bread loaf of choice for Craig Spencer, an avid baker and emergency
room doctor in New York City, who said he saw this pandemic coming in
time to stock up on flour.

Dr. Spencer's long shifts at NewYork-Presbyterian Hospital are
unimaginably trying, especially as
\href{https://www.nytimes3xbfgragh.onion/2020/03/26/nyregion/coronavirus-brooklyn-hospital.html}{facilities
are running out of ventilators}, masks and other essential supplies.
When he is home with his family, Dr. Spencer, 38, allocates time to
bake.

``The thing I think about when I'm making my bread is only my bread,''
he said over the phone, while folding his levain in his Manhattan
apartment. ``I'm not thinking about coronavirus or anything else.''

This is a coping mechanism for plenty of others on or near the front
line of the pandemic response.

Jeremy Konyndyk, 42, a pandemic preparedness expert and senior policy
fellow at the Center for Global Development, took up baking five years
ago while he was director of the United States Agency for International
Development Office of U.S. Foreign Disaster Assistance under the Obama
Administration and was helping to lead the U.S. response to Ebola. He
\href{https://www.nytimes3xbfgragh.onion/2020/03/07/opinion/trump-coronavirus-us.html}{continues
to researc}h disease response and policy.

While Mr. Konyndyk no longer bears the burden of responsibility for
leading the national charge against a pandemic, these are still anxious
times for him.

When his thoughts are spiraling, Mr. Konyndyk resets his focus by
baking. (He is also a sourdough enthusiast.)

``The nature of my job was: You're trying to make some very, very bad
things somewhat less bad, constantly trying to stave off harm and
damage. What I began to appreciate about baking was it was just very
different from that,'' Mr. Konyndyk said. ``It's all about creating, and
you have a tangible product at the end.''

``In disaster relief --- famine, war, epidemics --- the crisis goes on
and on and on. There's no point where you're finished,'' he said.
``Baking lets you take a project from beginning to end.''

For similar reasons, Lily Adams, a fellow at the Institute of Politics
and Public Service at Georgetown University, and a former staffer for
the presidential campaigns of Hillary Clinton and then Kamala Harris,
spent a recent day of social isolation making croissants.

She used a many-step recipe with requirements she'd seen on ``The Great
British Baking Show*.''*

``I thought, it kind of looks impossible, so I might as well try it,
because who's going to know if it's a complete failure?'' Ms. Adams, 33,
said. ``It's not like I'm having a dinner party.''

\includegraphics{https://static01.graylady3jvrrxbe.onion/images/2020/04/02/fashion/30VIRUS-STRESSBREAD-1/30VIRUS-STRESSBREAD-1-articleLarge.jpg?quality=75\&auto=webp\&disable=upscale}

``In politics, if something goes wrong, there are a million possible
explanations for why it didn't go as planned,'' she said. ``But baking
is a science. It's all about chemical reactions. So, if something
doesn't turn out like you'd hoped, there's usually a very quick
explanation. It's nice certainty in a life or a career path that doesn't
have that.''

Recipes don't need to be complicated to offer a respite from chaos.
Desiree Stennett, 31, a business reporter in Memphis, said she decided
to spend part of a recent weekend on a no-knead loaf after a grueling
week of on-the-ground reporting. ``I picked this recipe because the
YouTube videos promised it would not go wrong, and I don't have the
mental capacity to fail at bread making right now,'' she said.

Dr. Spencer's current routine is to proof his loaves overnight, bake
them in the morning, and while they're in the oven, tend to media
requests that have been rolling in since Barack Obama retweeted (and
praised) Dr.
\href{https://twitter.com/Craig_A_Spencer/status/1242302400762908685}{Spencer's
account} of a day in the E.R. Then he will enjoy some wonderful bread
with his family before heading back to the hospital for the next shift.

``We have a 16-month old, my wife is working more than a full-time job,
I have three jobs including the clinical one --- like everyone else,
we're just trying to figure out how to do this, to map efficiency and
time to get the most work done,'' Dr. Spencer said. ``We haven't really
figured it out yet, but at least we have bread.''

Advertisement

\protect\hyperlink{after-bottom}{Continue reading the main story}

\hypertarget{site-index}{%
\subsection{Site Index}\label{site-index}}

\hypertarget{site-information-navigation}{%
\subsection{Site Information
Navigation}\label{site-information-navigation}}

\begin{itemize}
\tightlist
\item
  \href{https://help.nytimes3xbfgragh.onion/hc/en-us/articles/115014792127-Copyright-notice}{©~2020~The
  New York Times Company}
\end{itemize}

\begin{itemize}
\tightlist
\item
  \href{https://www.nytco.com/}{NYTCo}
\item
  \href{https://help.nytimes3xbfgragh.onion/hc/en-us/articles/115015385887-Contact-Us}{Contact
  Us}
\item
  \href{https://www.nytco.com/careers/}{Work with us}
\item
  \href{https://nytmediakit.com/}{Advertise}
\item
  \href{http://www.tbrandstudio.com/}{T Brand Studio}
\item
  \href{https://www.nytimes3xbfgragh.onion/privacy/cookie-policy\#how-do-i-manage-trackers}{Your
  Ad Choices}
\item
  \href{https://www.nytimes3xbfgragh.onion/privacy}{Privacy}
\item
  \href{https://help.nytimes3xbfgragh.onion/hc/en-us/articles/115014893428-Terms-of-service}{Terms
  of Service}
\item
  \href{https://help.nytimes3xbfgragh.onion/hc/en-us/articles/115014893968-Terms-of-sale}{Terms
  of Sale}
\item
  \href{https://spiderbites.nytimes3xbfgragh.onion}{Site Map}
\item
  \href{https://help.nytimes3xbfgragh.onion/hc/en-us}{Help}
\item
  \href{https://www.nytimes3xbfgragh.onion/subscription?campaignId=37WXW}{Subscriptions}
\end{itemize}
