Sections

SEARCH

\protect\hyperlink{site-content}{Skip to
content}\protect\hyperlink{site-index}{Skip to site index}

\href{https://www.nytimes3xbfgragh.onion/section/politics}{Politics}

\href{https://myaccount.nytimes3xbfgragh.onion/auth/login?response_type=cookie\&client_id=vi}{}

\href{https://www.nytimes3xbfgragh.onion/section/todayspaper}{Today's
Paper}

\href{/section/politics}{Politics}\textbar{}Problems in F.B.I. Wiretap
Applications Go Beyond Trump Aide Surveillance, Review Finds

\url{https://nyti.ms/2w2HlxH}

\begin{itemize}
\item
\item
\item
\item
\item
\item
\end{itemize}

Advertisement

\protect\hyperlink{after-top}{Continue reading the main story}

Supported by

\protect\hyperlink{after-sponsor}{Continue reading the main story}

\hypertarget{problems-in-fbi-wiretap-applications-go-beyond-trump-aide-surveillance-review-finds}{%
\section{Problems in F.B.I. Wiretap Applications Go Beyond Trump Aide
Surveillance, Review
Finds}\label{problems-in-fbi-wiretap-applications-go-beyond-trump-aide-surveillance-review-finds}}

The bureau has routinely botched work on surveillance applications for
its national security investigations, an inspector general said.

\includegraphics{https://static01.graylady3jvrrxbe.onion/images/2020/03/31/us/politics/31dc-fbi/merlin_166133349_5df53972-c201-41ce-9ce7-236a26ed5b90-articleLarge.jpg?quality=75\&auto=webp\&disable=upscale}

\href{https://www.nytimes3xbfgragh.onion/by/charlie-savage}{\includegraphics{https://static01.graylady3jvrrxbe.onion/images/2018/06/12/multimedia/author-charlie-savage/author-charlie-savage-thumbLarge-v2.png}}

By \href{https://www.nytimes3xbfgragh.onion/by/charlie-savage}{Charlie
Savage}

\begin{itemize}
\item
  March 31, 2020
\item
  \begin{itemize}
  \item
  \item
  \item
  \item
  \item
  \item
  \end{itemize}
\end{itemize}

WASHINGTON --- An inspector general uncovered pervasive problems in the
F.B.I.'s preparation of wiretap applications, according to a memo
released Tuesday about an audit that grew out of a
\href{https://www.nytimes3xbfgragh.onion/2019/12/11/us/politics/fisa-surveillance-fbi.html?action=click\&module=RelatedLinks\&pgtype=Article}{damning
report} last year about errors and omissions in applications to target a
former Trump campaign adviser during the Russia investigation.

The \href{https://oig.justice.gov/reports/2020/a20047.pdf}{follow-up
audit} of unrelated cases by the Justice Department's independent
watchdog, Michael E. Horowitz, revealed a broader pattern of sloppiness
by the F.B.I. in seeking permission to use powerful tools to eavesdrop
on American soil in national security cases. It comes at a time when
Congress is debating new limits on the Foreign Intelligence Surveillance
Act, or FISA.

The finding of systemic incompetence is devastating for the F.B.I. But
in the Trump era, the discovery is leavened by an unusual side benefit
for the bureau: It undercuts the narrative fostered by President Trump
and his supporters that the botching of applications to surveil his
campaign adviser Carter Page is evidence that the F.B.I. engaged in a
politically biased conspiracy.

Mr. Horowitz's investigators reviewed so-called Woods files, where the
F.B.I. is supposed to catalog supporting documentation for factual
claims in a FISA application, in a random sample of 29 requests to
wiretap someone as part of a terrorism or espionage investigation under
FISA. They found problems with all 29.

``We do not have confidence that the F.B.I. has executed its Woods
Procedures in compliance with F.B.I. policy, or that the process is
working as it was intended to help achieve the `scrupulously accurate'
standard for FISA applications,'' the inspector general report said.

Testing the FISA applications against their underlying evidence
``identified apparent errors or inadequately supported facts in all of
the 25 applications we reviewed,'' the report said.

The other four could not be scrutinized at all because the F.B.I. could
not even locate the required Woods file. In the 25 applications
reviewed, there was an average of about 20 problems each. One alone had
65 issues.

In a statement appended to the report, the F.B.I. said that it accepted
the findings but also that it believed it was already addressing the
source of the problems through corrective steps it put in place after
the earlier report about Mr. Page. Among those steps are greater
training and new checklists that officials preparing documents for FISA
applications must follow.

Some of the problems the review found matched issues the inspector
general previously uncovered with the Page applications. For example,
F.B.I. policy requires that when investigators submit to a court a
factual assertion by a confidential source they must internally document
that the source's handler was consulted about those facts and the
source's credibility and background.

The Page applications had relied in part on information provided by
Christopher Steele, a former British intelligence agent who compiled a
dossier of claims about purported Trump-Russia links. The inspector
general previously found that the applications overstated the value of
Mr. Steele's prior assistance --- representations his handling agent had
not approved.

The follow-up audit released on Tuesday found a pattern: About half of
the applications contained information from confidential sources, and
``many of them'' lacked the required documentation of any check-in with
their handling agents.

Another problem the inspector general uncovered with the Page
applications was that when the F.B.I. sought renewed permission from the
court to continue wiretapping him, it failed to reverify facts repeated
from earlier applications.

For example, after the first renewal of the Page surveillance, the
F.B.I. talked to a person Mr. Steele had relied upon and who expressed
disagreement with how the dossier portrayed his information. But law
enforcement officials continued to cite that information in the second
and third renewal applications without noting the source's objections,
which most likely would have given the judge reason to view it with
greater skepticism.

A failure to consistently reverify recycled statements of fact during
renewals also appears to be common, the new report said.

In one case, F.B.I. agents even repeated errors in applications
submitted for wiretap renewals. In others, they only corroborated new
facts, not earlier assertions they had verified when originally seeking
the wiretaps and were relying on again.

``This practice directly contradicts F.B.I. policy,'' the report said.

The F.B.I.'s systematic sloppiness in preparing FISA applications could
be even worse than the new audit indicates because Mr. Horowitz's office
did not look through the voluminous raw case files in search of any
mitigating facts the applications had omitted, which was among the
problems his office's closer scrutiny of the Page applications
identified.

The system
\href{https://www.nytimes3xbfgragh.onion/2020/02/23/us/politics/fisa-surveillance-fbi.html}{is
inherently vulnerable to the risk that lower-level agents will
cherry-pick evidence} when compiling factual summaries, leaving out
evidence that weakens their case when they seek permission to conduct
surveillance either deliberately or through confirmation bias, current
and former national security officials have said.

Audits to catch omissions are particularly difficult because they
require the time and resources to achieve a deep-dive understanding of
extensive investigative files, like the inspector general's office spent
over a year doing with the Russia case and Mr. Page.

As part of its changes since the Page report, the F.B.I. has created new
checklists to remind agents to consider whether there is any mitigating
evidence. The FISA court has
\href{https://www.nytimes3xbfgragh.onion/2020/03/04/us/politics/fisa-court-fbi-surveillance.html}{ordered}
officials to swear in future cases that applications to the court
contain ``all information that might reasonably call into question the
accuracy of the information or the reasonableness of any F.B.I.
assessment in the application, or otherwise raise doubts about the
requested findings.''

The new report took no position on the scale of the errors in the
application materials --- whether they were minor or could have changed
law enforcement officials' decisions to seek wiretap orders or a judge's
decision to approve them.

The F.B.I. and the Justice Department's National Security Division also
occasionally audit FISA applications for accuracy. The inspector general
report said it examined 34 such accuracy review reports covering 42 FISA
applications at eight field offices between 2014 and 2019.

Those reviews found a total of 390 issues in 39 of the 42 applications,
``including unverified, inaccurate, or inadequately supported facts, as
well as typographical errors,'' it said.

The findings of systemic problems with its work seeking FISA
applications comes at a delicate time for the F.B.I., which has
expressed chagrin at the errors and omissions in the Page applications
while promoting the importance of preserving its ability to use national
security wiretaps when investigating suspected spies and terrorists.

In March, the House
\href{https://www.nytimes3xbfgragh.onion/2020/03/11/us/politics/house-passes-fbi-surveillance-bill.html}{passed
new curbs on FISA as part of a bill} to extend three F.B.I. tools for
national security investigations that expired on March 15. For example,
the bill would push the FISA court to appoint an outsider to critique
the government's arguments when a wiretap application raises serious
issues about First Amendment activity, which could include political
campaigns.

Some libertarian-leaning senators of both parties have argued that the
House bill falls short and that more new restrictions are necessary. One
of them, Senator Ron Wyden, Democrat of Oregon, said on Tuesday that the
new findings underscored the need for greater safeguards.

``The inspector general's findings of widespread abuses indicate that
Carter Page was not singled out,'' Mr. Wyden said. ``Congress should
write reforms into black-letter law to ensure that every American's
rights are protected, not just friends of the president.''

Marc Raimondi, a spokesman for the Justice Department's national
security division, pointed to the changes the department and the F.B.I.
have already made. He also noted that Attorney General William P. Barr
has
\href{https://www.nytimes3xbfgragh.onion/2020/02/05/us/politics/barr-2020-investigations.html}{further
required} the bureau to get higher-level approval to open politically
sensitive investigations and supports changes in the House's bill.

``The department is committed to putting the inspector general's
recommendations into practice and to implementing reforms that will
ensure that all FISA applications are complete and accurate,'' Mr.
Raimondi said.

Advertisement

\protect\hyperlink{after-bottom}{Continue reading the main story}

\hypertarget{site-index}{%
\subsection{Site Index}\label{site-index}}

\hypertarget{site-information-navigation}{%
\subsection{Site Information
Navigation}\label{site-information-navigation}}

\begin{itemize}
\tightlist
\item
  \href{https://help.nytimes3xbfgragh.onion/hc/en-us/articles/115014792127-Copyright-notice}{©~2020~The
  New York Times Company}
\end{itemize}

\begin{itemize}
\tightlist
\item
  \href{https://www.nytco.com/}{NYTCo}
\item
  \href{https://help.nytimes3xbfgragh.onion/hc/en-us/articles/115015385887-Contact-Us}{Contact
  Us}
\item
  \href{https://www.nytco.com/careers/}{Work with us}
\item
  \href{https://nytmediakit.com/}{Advertise}
\item
  \href{http://www.tbrandstudio.com/}{T Brand Studio}
\item
  \href{https://www.nytimes3xbfgragh.onion/privacy/cookie-policy\#how-do-i-manage-trackers}{Your
  Ad Choices}
\item
  \href{https://www.nytimes3xbfgragh.onion/privacy}{Privacy}
\item
  \href{https://help.nytimes3xbfgragh.onion/hc/en-us/articles/115014893428-Terms-of-service}{Terms
  of Service}
\item
  \href{https://help.nytimes3xbfgragh.onion/hc/en-us/articles/115014893968-Terms-of-sale}{Terms
  of Sale}
\item
  \href{https://spiderbites.nytimes3xbfgragh.onion}{Site Map}
\item
  \href{https://help.nytimes3xbfgragh.onion/hc/en-us}{Help}
\item
  \href{https://www.nytimes3xbfgragh.onion/subscription?campaignId=37WXW}{Subscriptions}
\end{itemize}
