Sections

SEARCH

\protect\hyperlink{site-content}{Skip to
content}\protect\hyperlink{site-index}{Skip to site index}

\href{https://www.nytimes3xbfgragh.onion/section/health}{Health}

\href{https://myaccount.nytimes3xbfgragh.onion/auth/login?response_type=cookie\&client_id=vi}{}

\href{https://www.nytimes3xbfgragh.onion/section/todayspaper}{Today's
Paper}

\href{/section/health}{Health}\textbar{}The Coronavirus Patients
Betrayed by Their Own Immune Systems

\url{https://nyti.ms/3bOln0J}

\begin{itemize}
\item
\item
\item
\item
\item
\item
\end{itemize}

\hypertarget{the-coronavirus-outbreak}{%
\subsubsection{\texorpdfstring{\href{https://www.nytimes3xbfgragh.onion/news-event/coronavirus?name=styln-coronavirus-national\&region=TOP_BANNER\&block=storyline_menu_recirc\&action=click\&pgtype=Article\&impression_id=7bc8d6a0-f2ba-11ea-bb9c-a332a2ba7fb3\&variant=undefined}{The
Coronavirus
Outbreak}}{The Coronavirus Outbreak}}\label{the-coronavirus-outbreak}}

\begin{itemize}
\tightlist
\item
  live\href{https://www.nytimes3xbfgragh.onion/2020/09/09/world/covid-19-coronavirus.html?name=styln-coronavirus-national\&region=TOP_BANNER\&block=storyline_menu_recirc\&action=click\&pgtype=Article\&impression_id=7bc8d6a1-f2ba-11ea-bb9c-a332a2ba7fb3\&variant=undefined}{Latest
  Updates}
\item
  \href{https://www.nytimes3xbfgragh.onion/interactive/2020/us/coronavirus-us-cases.html?name=styln-coronavirus-national\&region=TOP_BANNER\&block=storyline_menu_recirc\&action=click\&pgtype=Article\&impression_id=7bc8d6a2-f2ba-11ea-bb9c-a332a2ba7fb3\&variant=undefined}{Maps
  and Cases}
\item
  \href{https://www.nytimes3xbfgragh.onion/interactive/2020/science/coronavirus-vaccine-tracker.html?name=styln-coronavirus-national\&region=TOP_BANNER\&block=storyline_menu_recirc\&action=click\&pgtype=Article\&impression_id=7bc8fdb0-f2ba-11ea-bb9c-a332a2ba7fb3\&variant=undefined}{Vaccine
  Tracker}
\item
  \href{https://www.nytimes3xbfgragh.onion/2020/09/02/your-money/eviction-moratorium-covid.html?name=styln-coronavirus-national\&region=TOP_BANNER\&block=storyline_menu_recirc\&action=click\&pgtype=Article\&impression_id=7bc8fdb1-f2ba-11ea-bb9c-a332a2ba7fb3\&variant=undefined}{Eviction
  Moratorium}
\item
  \href{https://www.nytimes3xbfgragh.onion/2020/09/09/upshot/coronavirus-surprise-test-fees.html?name=styln-coronavirus-national\&region=TOP_BANNER\&block=storyline_menu_recirc\&action=click\&pgtype=Article\&impression_id=7bc8fdb2-f2ba-11ea-bb9c-a332a2ba7fb3\&variant=undefined}{Surprise
  Test Fees}
\end{itemize}

Advertisement

\protect\hyperlink{after-top}{Continue reading the main story}

Supported by

\protect\hyperlink{after-sponsor}{Continue reading the main story}

\hypertarget{the-coronavirus-patients-betrayed-by-their-own-immune-systems}{%
\section{The Coronavirus Patients Betrayed by Their Own Immune
Systems}\label{the-coronavirus-patients-betrayed-by-their-own-immune-systems}}

A ``cytokine storm'' becomes an all-too-frequent phenomenon,
particularly among the young. But treatments are being tested.

\includegraphics{https://static01.graylady3jvrrxbe.onion/images/2020/04/01/science/01VIRUS-CYTOKINESTORM1/01VIRUS-CYTOKINESTORM1-articleLarge.jpg?quality=75\&auto=webp\&disable=upscale}

By Apoorva Mandavilli

\begin{itemize}
\item
  April 1, 2020
\item
  \begin{itemize}
  \item
  \item
  \item
  \item
  \item
  \item
  \end{itemize}
\end{itemize}

The 42-year-old man arrived at a hospital in Paris on March 17 with a
fever, cough and the ``ground glass opacities'' in both lungs that are a
trademark of infection with the new coronavirus.

Two days later, his condition suddenly worsened and his oxygen levels
dropped. His body, doctors suspected, was in the grip of a cytokine
storm, a dangerous overreaction of the immune system. The phenomenon has
become all too common in the coronavirus pandemic, but it is also
pointing to potentially helpful drug treatments.

When the body first encounters a virus or a bacterium, the immune system
ramps up and begins to fight the invader. The foot soldiers in this
fight are molecules called cytokines that set off a cascade of signals
to cells to marshal a response. Usually, the stronger this immune
response, the stronger the chance of vanquishing the infection, which is
partly why children and younger people are
\href{https://www.nytimes3xbfgragh.onion/2020/02/05/health/coronavirus-children.html}{less
vulnerable over all} to coronavirus. And once the enemy is defeated, the
immune system is hard-wired to shut itself off.

``For most people and most infections, that's what happens,'' said Dr.
Randy Cron, an expert on cytokine storms at the University of Alabama at
Birmingham.

But in some cases --- as much as 15 percent of people battling any
serious infection, according to Dr. Cron's team --- the immune system
keeps raging long after the virus is no longer a threat. It continues to
release cytokines that keep the body on an exhausting full alert. In
their misguided bid to keep the body safe, these cytokines attack
multiple organs including the lungs and liver, and may eventually lead
to death.

In these people, it's their body's response, rather than the virus, that
ultimately causes harm.

Cytokine storms can overtake people of any age, but some scientists
believe that they may explain why healthy young people died during the
1918 pandemic and more recently during the SARS, MERS and H1N1
epidemics. They are also a complication of various autoimmune diseases
like lupus and Still's disease, a form of arthritis. And they may offer
clues as to why otherwise healthy young people with coronavirus
infection are succumbing to acute respiratory distress syndrome, a
common consequence of a cytokine storm.

Reports from China and Italy have described young patients with clinical
outcomes that seem consistent with this phenomenon. It's very likely
that some of these patients developed a cytokine storm, Dr. Cron said.

\hypertarget{latest-updates-the-coronavirus-outbreak}{%
\section{\texorpdfstring{\href{https://www.nytimes3xbfgragh.onion/2020/09/09/world/covid-19-coronavirus.html?action=click\&pgtype=Article\&state=default\&region=MAIN_CONTENT_1\&context=storylines_live_updates}{Latest
Updates: The Coronavirus
Outbreak}}{Latest Updates: The Coronavirus Outbreak}}\label{latest-updates-the-coronavirus-outbreak}}

Updated 2020-09-09T16:25:40.452Z

\begin{itemize}
\tightlist
\item
  \href{https://www.nytimes3xbfgragh.onion/2020/09/09/world/covid-19-coronavirus.html?action=click\&pgtype=Article\&state=default\&region=MAIN_CONTENT_1\&context=storylines_live_updates\#link-279e24e2}{Top
  U.S. health officials update Congress on vaccine development and
  distribution plans.}
\item
  \href{https://www.nytimes3xbfgragh.onion/2020/09/09/world/covid-19-coronavirus.html?action=click\&pgtype=Article\&state=default\&region=MAIN_CONTENT_1\&context=storylines_live_updates\#link-5b0bf0d1}{As
  drugmakers pledge to thoroughly vet vaccines, one company pauses its
  trials for a safety review.}
\item
  \href{https://www.nytimes3xbfgragh.onion/2020/09/09/world/covid-19-coronavirus.html?action=click\&pgtype=Article\&state=default\&region=MAIN_CONTENT_1\&context=storylines_live_updates\#link-58edc4cb}{Britain
  bans gatherings of more than six people.}
\end{itemize}

\href{https://www.nytimes3xbfgragh.onion/2020/09/09/world/covid-19-coronavirus.html?action=click\&pgtype=Article\&state=default\&region=MAIN_CONTENT_1\&context=storylines_live_updates}{See
more updates}

More live coverage:
\href{https://www.nytimes3xbfgragh.onion/live/2020/09/09/business/stock-market-today-coronavirus?action=click\&pgtype=Article\&state=default\&region=MAIN_CONTENT_1\&context=storylines_live_updates}{Markets}

In the case of the 42-year-old patient, the suspected cytokine storm led
his doctors to eventually try tocilizumab, a drug they have sometimes
used to soothe an immune system in distress.

After just two doses of the drug, spaced eight hours apart, the
patient's fever rapidly disappeared, his oxygen levels rose and a chest
scan showed his lungs clearing. The case report, described in an
upcoming paper in Annals of Oncology, joins dozens of accounts from
Italy and China, all indicating that tocilizumab might be an effective
antidote to the coronavirus in some people.

On March 5, China approved the drug to treat serious cases of Covid-19,
the disease caused by the coronavirus, and authorized clinical trials.
On March 23, the U.S. Food and Drug Administration granted approval to
the pharmaceutical company Roche to test the drug in hundreds of people
with coronavirus infection.

Tocilizumab is approved to quieten the chatter of immune molecules in
rheumatoid arthritis and in some types of cancer. It mutes the activity
of a specific cytokine called interleukin-6 that is associated with an
over-exuberant immune response.

``That's the rationale for using the drug,'' said Dr. Laurence Albiges,
who cared for the patient at the Gustave Roussy Cancer Center in Paris.

Even as researchers look for treatments, they are trying to learn more
about why some people's immune systems go into this dangerous overdrive.
Genetic factors explain the risk, at least in some kinds of cytokine
storms.

There are many variations on the phenomenon, and they go by many names:
systemic inflammatory response syndrome, cytokine release syndrome,
macrophage activation syndrome, hemophagocytic lymphohistiocytosis.

Broadly speaking, they are all marked by an unbridled surge in immune
molecules, and may all result in the fatal shutdown of multiple organs.

But many doctors are unfamiliar with this niche concept or how to treat
it, experts said.

``Everyone's talking about cytokine storm as if it were a
well-recognized phenomenon, but you could have asked medics two weeks
ago and they wouldn't have heard of it,'' said Dr. Jessica Manson, an
immunologist at University College London Hospital.

A patient battling a cytokine storm may have an abnormally fast heart
rate, fever and a drop in blood pressure. Apart from a surge in
interleukin-6, the body may also show high swirling levels of molecules
called interleukin-1, interferon-gamma, C-reactive protein and tumor
necrosis factor-alpha.

This storm, if it develops, becomes obvious a few days into the
infection. But the sooner doctors catch on to it and treat it, the more
likely the patient is to survive. Too late, and the storm may be beyond
control, or may already have caused too much damage.

There is a relatively simple, rapid and easily available test that can
detect whether a patient's body has been taken over by a cytokine storm.
It looks for high levels of a protein called ferritin.

But if the test does suggest a cytokine storm is underway, what then?

The seemingly obvious solution is to quell the storm, Dr. Cron said:
``If it's the body's response to the infection that's killing you, you
need to treat that.''

The reality is trickier, especially given the lack of reliable data for
Covid-19. But noting that drugs like tocilizumab are taken regularly by
people with arthritis, Dr. Cron said the benefit would probably outweigh
potential harm if someone is facing death.

``We need evidence-based data, but in a pandemic, where we're flying by
the seat of our pants, we always have to treat the patient in front of
us,'' he said.

Other drugs might also be useful against cytokine storms. For example, a
drug called anakinra mutes interleukin-1, another of the wayward
proteins. Clinical trials of anakinra for Covid-19 are also underway. A
report published this week suggested that
\href{https://www.nytimes3xbfgragh.onion/2020/04/01/health/hydroxychloroquine-coronavirus-malaria.html}{hydroxychloroquine},
a much-spotlighted malaria drug that also calms an overactive immune
response, might also be effective as a treatment for those who are
mildly ill from coronavirus.

Doctors could also turn to corticosteroids, which broadly turn down the
entire immune response. That poses its own danger, by exposing the
patient to other opportunistic infections, especially in a hospital.
``It's about getting the balance right between suppression of the
over-exuberant immune response and still allowing the immune response to
fight the virus,'' Dr. Manson said.

A group of experts convened two weeks ago to discuss the best ways to
collect more data and to treat patients who appear to have cytokine
storm. It's already clear that the complexities of the immune system and
the course of coronavirus mean there is no single best treatment.

At the Gustave Roussy Cancer Center, doctors treated another coronavirus
patient with tocilizumab. That individual did not show any improvement
with the drug.

``The response to the pathogen, the virus, is totally different in
different individuals,'' said Dr. Fabrice André, an oncologist at the
center. ``The trials will determine in which patients it works.''

\textbf{\emph{{[}}\href{http://on.fb.me/1paTQ1h}{\emph{Like the Science
Times page on Facebook.}}} ****** \emph{\textbar{} Sign up for the}
\textbf{\href{http://nyti.ms/1MbHaRU}{\emph{Science Times
newsletter.}}\emph{{]}}}

Advertisement

\protect\hyperlink{after-bottom}{Continue reading the main story}

\hypertarget{site-index}{%
\subsection{Site Index}\label{site-index}}

\hypertarget{site-information-navigation}{%
\subsection{Site Information
Navigation}\label{site-information-navigation}}

\begin{itemize}
\tightlist
\item
  \href{https://help.nytimes3xbfgragh.onion/hc/en-us/articles/115014792127-Copyright-notice}{©~2020~The
  New York Times Company}
\end{itemize}

\begin{itemize}
\tightlist
\item
  \href{https://www.nytco.com/}{NYTCo}
\item
  \href{https://help.nytimes3xbfgragh.onion/hc/en-us/articles/115015385887-Contact-Us}{Contact
  Us}
\item
  \href{https://www.nytco.com/careers/}{Work with us}
\item
  \href{https://nytmediakit.com/}{Advertise}
\item
  \href{http://www.tbrandstudio.com/}{T Brand Studio}
\item
  \href{https://www.nytimes3xbfgragh.onion/privacy/cookie-policy\#how-do-i-manage-trackers}{Your
  Ad Choices}
\item
  \href{https://www.nytimes3xbfgragh.onion/privacy}{Privacy}
\item
  \href{https://help.nytimes3xbfgragh.onion/hc/en-us/articles/115014893428-Terms-of-service}{Terms
  of Service}
\item
  \href{https://help.nytimes3xbfgragh.onion/hc/en-us/articles/115014893968-Terms-of-sale}{Terms
  of Sale}
\item
  \href{https://spiderbites.nytimes3xbfgragh.onion}{Site Map}
\item
  \href{https://help.nytimes3xbfgragh.onion/hc/en-us}{Help}
\item
  \href{https://www.nytimes3xbfgragh.onion/subscription?campaignId=37WXW}{Subscriptions}
\end{itemize}
