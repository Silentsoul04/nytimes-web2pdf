Sections

SEARCH

\protect\hyperlink{site-content}{Skip to
content}\protect\hyperlink{site-index}{Skip to site index}

Cannabis Scientists Are Chasing the Perfect High

\url{https://nyti.ms/3dNAusP}

\begin{itemize}
\item
\item
\item
\item
\item
\item
\end{itemize}

\includegraphics{https://static01.graylady3jvrrxbe.onion/images/2020/03/27/magazine/5mag-cannabis/5mag-cannabis-articleLarge.jpg?quality=75\&auto=webp\&disable=upscale}

The Great ReadFeature

\hypertarget{cannabis-scientists-are-chasing-the-perfect-high}{%
\section{Cannabis Scientists Are Chasing the Perfect
High}\label{cannabis-scientists-are-chasing-the-perfect-high}}

Chemists at some of the biggest legal-weed companies are after an
elusive prize: a predictable, reliable product.

Credit...Illustration by Erik Carter

Supported by

\protect\hyperlink{after-sponsor}{Continue reading the main story}

By Gary Greenberg

\begin{itemize}
\item
  Published April 1, 2020Updated April 13, 2020
\item
  \begin{itemize}
  \item
  \item
  \item
  \item
  \item
  \item
  \end{itemize}
\end{itemize}

\hypertarget{listen-to-this-article}{%
\subsubsection{Listen to This Article}\label{listen-to-this-article}}

Audio Recording by Audm

\emph{To hear more audio stories from publishers, like The New York
Times, download}
\href{https://www.audm.com/?utm_source=nytmag\&utm_medium=embed\&utm_campaign=decoding-cannabis}{\emph{Audm
for iPhone or Android}}\emph{.}

The retail showroom of INSA, a farm-to-bong cannabis company in western
Massachusetts, is a clean industrial space on the first floor of a
four-story brick building in the old mill town Easthampton. When I
visited recently, before the coronavirus shut down recreational sales
and forbade crowds, the crew of eight behind the glass display cases
looked a lot like the staff you'd see dispensing lattes at Starbucks or
troubleshooting iPads at the Genius Bar: young, racially diverse,
smiling. They were all wearing black T-shirts with the INSA motto,
``Uncommon Cannabis.'' Standing in line with me were a white-haired
couple leaning on canes; a 40-something woman in a black pantsuit, who
complained that the wait would be longer than her lunch break; a bald
man in a tweed jacket; and a pair of women in perms and polyester
discussing the virtues of a strain called Green Crack. We were all
waiting at a discreet distance from the counter, as you would at the
bank, for the next available ``budtender.''

I got Ben, who described for me the wares that fill the cases like rings
and watches in a jewelry store: waxes and dabs and oils and buds and
edibles, most of them, he said, processed in a lab and kitchen on the
other side of the wall behind him, using weed grown on the upper three
floors. He sounded a little apologetic when he told me that while he
knew why the bud I was pointing to was called Peyote Critical --- ``It
speaks a little bit to its parentage, Peyote Purple and Critical Kush''
--- he hadn't tried it, so he wasn't entirely sure how it would affect
me.

Ben took me around a corner to another glass case, this one displaying
vaporizers in different shapes and sizes. He pulled a box off a shelf
behind him. It was a \$35, 350-milligram disposable vape pen loaded with
Jack Herer, a strain named for a legendary cannabis activist. If I
bought this, he said, I should ``resist the temptation to take big rips
--- four seconds at the max, then pull that pen away and inhale to get a
nice full set of lungs.'' Ben felt more certain about the effects of
Jack Herer than Peyote Critical, especially after he took a look at the
label. ``The primary terpene in here is limonene,'' he said, which
should make me ``energetic and uplifted.'' But there were more terpenes
at work, Ben said. ``You've got pinene coming in at 2.83 percent, good
for memory retention and alertness, and then myrcene, which should help
balance out some of the raciness from the limonene. Myrcene is good for
your brain's absorption of metabolizing THC but also has relaxing,
sedating qualities.''

Terpenes are the compounds that give the different strains of cannabis
their distinctive aromas. According to Ben, they are also what
``modulates the high''; each variety of weed has its own terpene
profile, which helps account for one of the riddles of cannabis: that
even if two strains share the same psychoactive molecule --- 9-delta
tetrahydrocannabinol, or THC --- one can make its users collapse in
gales of laughter while the other produces paranoia and yet another
seems (at least for a moment) to reveal the secrets of the universe.
INSA is confident enough that it has figured out which kind of high each
of its products will deliver that it offers a customer-satisfaction
guarantee if the experience it advertises doesn't match what the user
gets.

The company, which was founded in 2013, was named for the two main
varieties of cannabis: indica and sativa. Before budtenders and
black-market dealers knew anything about terpenes, they told their
customers that indica strains were sedating and sativa strains euphoric.
But scientists have shown that while indica plants tend to be short and
bushy and quicker to mature and sativas tall and spiky and slower, those
differences do not correspond to any kind of consistent differences in
the chemical profiles of the plants. ``We've been trying to re-educate
our consumers,'' Peter Gallagher, chief executive of INSA, told me.

His firm isn't the only one changing the way cannabis is marketed.
Wherever weed is legal, companies are claiming that they have figured
out how to produce a bespoke high. The promises are specific --- one
California company, MedMen, offers its customers ``a surefire explosive
orgasm'' --- and backed by scientific-sounding terminology like
``terpene profile'' and ``cannabinoid breakdown.'' Some of the research
these companies cite to support what they are advertising has been
published and peer-reviewed, but much of the recent work on the effects
of cannabis has been conducted privately, and the companies are guarding
their results as trade secrets. MedMen canceled my interview with its
chief executive when it learned that I wanted to talk about the science
behind its claims.

The cannabis business, then, has arrived at a critical moment. Now that
pot has become something like a regular consumer product, customers are
increasingly seeking the same ``proven consistency'' they expect from
potato chips and soap. The financial stakes are clear: Despite lingering
prohibitions in 17 states, legal cannabis is already an \$8 billion
industry in the United States. Domestic sales of alcohol, humankind's
other favorite intoxicant, topped \$200 billion last year. But to make
cannabis as popular as booze requires solving that original problem:
It's hard to imagine millions of people becoming new recreational users
without being able to promise them that the product they're spending
money on --- the average purchase at INSA is around \$90 --- will give
them the effect they want.

Companies like MedMen and INSA may have decided that they've already
cracked the code, but it remains to be seen whether that's even possible
with a plant as complex as cannabis. What those companies know for
certain, however, is that the billion-dollar race to find out has
already begun.

\includegraphics{https://static01.graylady3jvrrxbe.onion/images/2020/03/27/magazine/5mag-cannabis-03/5mag-cannabis-03-articleLarge.jpg?quality=75\&auto=webp\&disable=upscale}

\textbf{Cannabis has been} consumed in one form or another for thousands
of years, but it wasn't until 1964 that a team led by the Israeli
researcher Raphael Mechoulam identified THC as the molecule that got
users high. By then, cannabis prohibition had been widespread for more
than 25 years, creating formidable bureaucratic obstacles to researchers
who wished to work with the plant. But Mechoulam kept at it, isolating
another cannabinoid, cannabigerol (CBG), and mapping the structure of
cannabidiol (CBD). All these chemicals, it turned out, had a role to
play in the body's response to cannabis. In 1998, Mechoulam coined the
term the ``entourage effect'' to describe the complicated interplay
between cannabinoids and the body's own neurotransmitters in determining
the drug's effects.

While Mechoulam was still conducting his research, an American
neuro­logist named Ethan Russo was zeroing in on terpenes as a major
source of the variability in the effects of different strains of weed
--- or ``chemovars,'' as he prefers to call them. Not long ago, I went
to visit him at his home on an island in Puget Sound, where he walked me
through his past two decades of trying to conduct conventional research
on this unconventional subject.

Russo told me that while he had been interested in botanical treatments
since reading Euell Gibbons's ``Stalking the Healthful Herbs'' while
still in his teens, he knew little about cannabis as a medicine. That
began to change when patients in his private practice in Montana began
to report success with plant remedies, including cannabis, for chronic
conditions like migraines. Intrigued by the results they reported, he
began to study herbal medicine in earnest. Eventually, he wrote a
textbook on the subject, which included a chapter on cannabis.

In 1996, while writing the book, Russo was introduced to aromatherapy
with essential oils. ``I realized how evocative they were,'' he told me.
He also knew that the same molecules that gave essential oils their
punch --- the terpenes --- were present in cannabis, and that
aficionados often said that ``the nose knows,'' meaning that a strain
that smelled good to a user was likely to yield felicitous results. He
began to suspect that the terpenes ``were having a major modulatory
effect on THC'' and thus held a key to understanding the wildly variable
effects of the drug.

He received Food and Drug Administration approval to run a clinical
trial of cannabis as a treatment for migraines, but the National
Institute on Drug Abuse, which must approve such research with illegal
drugs, refused to sign off. The bureaucratic resistance ``really
stimulated my sense of adolescent rebellion,'' he told me, and he
decided to investigate terpenes' role in the entourage effect on his own
--- legally, but using a method that, while once the backbone of medical
research, had fallen into disrepute: self-experimentation.

He ordered terpenes from chemical-supply houses and, along with a few
friends, began to blind-test them by transferring small amounts of them
from coded bottles into a vaporizer designed to minimize odors and
keeping track of their effects. In 2004, he went to Amsterdam, where he
was able to obtain pure THC legally, pair it with different combinations
of terpenes and record the effects they had on a group of volunteers.

Russo's research was not without its problems. Scents could not be
totally eliminated, the effects of THC couldn't be successfully blinded
and the prodigious daily cannabis intake of at least one participant
made him a poor judge of the effects of individual THC/terpene
combinations. Still, Russo found consistent correlations. THC alone, he
found, lowered mood and distorted perception, and proved over all to be
``really hard to function on.'' He recalled one session in which, as it
turned out, he had inhaled pure THC. ``It was my turn to make dinner
that night, and it was like: `Oh, God, I'm not sure I can do this.
Where's the knife? What do I need to do next?' Everything was so hard.''
But throw in pinene, the terpene that gives a pine woods its scent, and
``all of a sudden that's gone. You're clear. You have no problem
remembering anything.'' Limonene, one source of citrus's distinctive
odor, also cured the THC blues, ``making this unpleasant thing vibrant
and alive and electric.'' On the other hand, some terpenes just made
things worse --- like myrcene, an oil that smells a little like cloves
and is present in high concentrations in hops, on which, Russo recalled,
``I can't function, I can't think, I can't move.''

In 2010, at a conference honoring Mechoulam, Russo presented a paper
called
\href{https://www.ncbi.nlm.nih.gov/pmc/articles/PMC3165946/?_escaped_fragment_=po=31.2808}{``Taming
THC,''} which compiled more than 400 studies that strengthened the case
for the role terpenes played in the variable effects of pot. It did not
directly mention Russo's D.I.Y. research, but a careful reader could
find observations about the effects of specific combinations on memory,
cognition and mood --- that myrcene-heavy strains may produce
``couchlock,'' that pinene might be an ``antidote'' to the negative
effects of THC --- that were at least as indebted to Russo's experiments
in Amsterdam as to anything in the scientific literature. The paper was
published the following year in the prestigious, widely read British
Journal of Pharmacology.

Russo was not the only cannabis researcher studying terpenes, and
``Taming THC'' was not the first scientific article to speculate about
their role in cannabis intoxication. It was also meant to be the
starting point for more rigorous research into terpenes, not the final
word on their effects. But the article, with its concise charts of
correlations between terpenes and drug effects, came along at a crucial
moment in the history of pot: By 2011, 15 states had approved medical
marijuana, and Colorado and Washington were on the verge of making the
drug legal for recreational use.

A new industry was ready to burst into being, and here, in the
legitimate academic press, was a paper providing a map to what Russo
called a ``pharmacological treasure trove.'' If the paper's promises
held up, a company could even take aim at the most tempting prize of
all: the vast number of Americans who had never tried weed before, and
others who had aged out of it but might be brought back on board. For
that market, it wasn't enough for cannabis to be legal; the drug had to
be as predictable as a pre-dinner martini.

\textbf{``Taming THC'' laid} out an ambitious scientific agenda for
anyone seeking to further test the paper's claims: ``high throughput
pharmacological screening,'' animal experiments to specify mechanisms of
action, molecular studies to establish just how terpenes and
cannabinoids interact, animal-behavior studies, brain-imaging research
and human clinical trials. Nearly a decade later, this agenda, which is
modeled on pharmaceutical drug development, remains unfulfilled.

Recently, however, a few companies in the United States and Canada have
begun an aggressive investigation into the entourage effect, though they
are forgoing many protocols of the pharmaceutical industry. Last year, I
met Jon Cooper, the founder of a company called Ebbu, at a co-working
space in Denver. Cooper had been toying with the idea of a cannabis
start-up ever since Colorado legalized the drug, but he was deterred by
his own history with pot. ``I'd had some awesome experiences that I
wished I could have all the time,'' he told me. But he'd also had ``some
completely horrific experiences that I never ever wanted again.'' Cooper
says he couldn't sell something he didn't believe in; but what if he
could figure out how to ``capture in a bottle the awesome experience, so
every store I walk into, I could get that same experience. Wouldn't that
be amazing?''

A year after Cooper started Ebbu in 2013, he approached Brian Reid, who
was running a lab at the University of Colorado's school of pharmacy,
hoping they could collaborate. Reid's specialty was exactly the ``high
throughput'' screening Russo had called for, in which algorithms are
used to quickly determine which potential drugs would interact with
which cellular targets. The university's lawyers, worried about a
possible loss of federal funding, nixed the deal, but Reid eventually
decided to go to work for Ebbu directly; in 2016, he became its chief
science officer.

Reid began to, in Cooper's words, ``crank data.'' As long as he did not
ask for government money, he could do high-grade pharmaceutical research
using human subjects without the usual regulatory scrutiny. Ebbu went
straight into human trials of the most likely drivers of the entourage
effect. That's not as reckless as it might appear. Reid points out that
no one is known to have ever died from an overdose of cannabis.

Colorado law forbids cannabis companies to give away products, so Ebbu
offered samples for \$1 to people who agreed to fill out an online
questionnaire about their experiences. Their responses were correlated
with the chemical profiles of the extracts in order to gather evidence
about which combinations produced which effects. In June 2016, the
company announced that it had identified eight terpenes and three
cannabinoids that modify the effect of THC in a predictable way. The
company said this was a ``significant milestone'' in its quest to
develop formulations that would ``enable consumers to choose a desired
experience.''

In late 2016, Ebbu started distributing a formulation called Genesis to
dispensaries around Colorado. It yielded a ``really happy, focused
high,'' Cooper told me, one ``that was extremely blissful.'' It did not
cause anxiety or make its users stupid or sleepy or goofy: ``They could
always communicate, they could talk to their kids, they were able to do
all these things and function while still having this amazing
experience.'' Cooper loved it. His wife and his brother-in-law and his
friends loved it. The consumers who bought Genesis loved it so much,
according to Cooper, that it gained a ``cult following'' --- people who
``stopped consuming all other products and started consuming only
Genesis.''

Still, Cooper kept funding more research. ``Every single dollar that was
coming in the door,'' he said, ``I was investing in science and
intellectual property.'' His approach seemed vindicated in the fall of
2018, when Canopy Growth, the largest cannabis conglomerate in Canada,
bought Ebbu for a reported \$330 million in cash and stock. According to
Bruce Linton, the founder and former chief executive of Canopy, it
wasn't Genesis or any other product that made Ebbu such an attractive
buy. It was the data the company had cranked out about breeding,
extracting, formulating and using cannabis, about what consumers wanted
and about which combinations of chemicals could be counted upon to give
it to them. The value, according to Linton, was obvious: Whether
customers want to go to a comedy club, or listen to music, or just
``diminish the anxiety of the workweek,'' they're all ``buying
outcomes,'' which Canopy, using the scientific knowledge it bought from
Ebbu, could ``containerize.'' (Linton left Canopy last year as part of a
shake-up at Constellation Brands, the multinational corporation with a
controlling stake in Canopy.)

Russo's dream of open scientific exchange, however, has remained
elusive. The biggest companies in the cannabis world are keeping their
research under wraps, and without the necessity of answering to
regulators, it's likely to stay that way. Ebbu did release some findings
from its consumer research at the 2018 meeting of the International
Cannabis Research Society, where it presented a poster called
``Cannabinoid and Terpene Formulations Elicit Distinct Mood Effects.''
The poster shows correlations between different mixtures and consumer
experience, using terms like ``active'' and ``chill'' or ``conflicted''
and ``Zen.'' But the poster, which like all posters was not peer
reviewed, did not explain how those categories were defined or which
terpenes or cannabinoids were in the products, nor did it disclose the
data underlying the findings. Several participants thought the poster
was contrary to the spirit of the conference, which was about open
scientific inquiry. To them, it looked like nothing more than an
advertisement.

Image

Credit...Illustration by Erik Carter

\textbf{As more of} the compounds in cannabis are isolated, a few
companies are looking at ways to eliminate one stubborn source of
variability: the plants themselves. Ebbu's intellectual property
includes a patent for using an inkjet printer to spit out cannabinoids
and terpenes in precisely measured ratios determined by the user.
Brought in from the black-market wilderness by deep-pocketed,
consumer-savvy companies, cannabis may become just another designer
drug.

At INSA, the Jack Herer vape oil may be named after a known strain, but
it is not made by extracting or distilling a Jack Herer plant. Rather,
it's formulated in INSA's lab to emulate the chemical profile of that
variety. The company can obtain its THC and other cannabinoids from any
cannabis plant, and it buys its terpenes from outside suppliers. Peter
Gallagher says INSA does not hide the fact that its vape oils are
manufactured products that, like pharmaceutical drugs, are created by
isolating and combining compounds. Indeed, he envisions an exciting
future when ``you could come into the store and build your own blend of
certain proportions of cannabinoids and terpenes.''

Recent research has shown that it's possible to grow cannabinoids from
yeast, cutting out the need for any horticulture at all --- a prospect
that has already attracted industry attention. After all, greenhouses
take up more space than laboratories, molecules are easier to patent
than plants and once you figure out how to do it all in a petri dish,
you don't have to worry about weather or insects. Ethan Russo, however,
thinks producers should be cautious in taking this approach. ``The idea
that you're going to bottle this up and eliminate cannabis'' is a bad
one, he told me. He doesn't doubt that a few of the more than 500
chemicals in the plant can be identified as critical to its effects,
but, he says, ``that doesn't tell the whole story'' any more than the
flute and violin lines alone can convey the entire impact of a symphony.
What's missing is the way the entourage works together not only to
create the effects of the plant but also to provide a counterpoint to
its potential dangers. ``It's vastly preferable to take the effort, time
and money to develop a specific chemovar of cannabis that's going to do
the same thing and do it better and demonstrably more safely,'' he says.

At least one researcher is making that effort, with the help of some
willing human volunteers. For the past three years, a neuroscientist
named Adie Rae has designed the survey used to judge the winners of the
Cultivation Classic, an annual competition sponsored by an alt-weekly in
Portland, Ore. Cannabis competitions are common, but the Cultivation
Classic may be the only one that requires its judges to spend an
afternoon listening to a scientist talk about predictive algorithms and
blinded studies.

Last year, I joined Rae as she addressed a crowd of 160 people in a
conference room in downtown Portland. The participants had been handed
black zipper bags that contained a dozen tiny glass jars, each labeled
with a number and housing a single bud of locally sourced, organically
grown weed. Their mission, Rae explained, was to take a 48-hour
``tolerance break'' and then, over the course of the next month, sample
each flower, paying careful attention to their psychological state
before, during and afterward. A four-digit PIN enclosed in their kits
would give them access to a website on which they were to rate the
extent to which the sample gave them the experience they wanted, whether
it made them sleepy or stimulated, sociable or introspective,
cognitively impaired or creative, if it gave them side effects like
redeye or anxiety, and if its aroma and appearance and taste were to
their liking.

Rae makes information on the chemistry of the winning plants and the
effects that users reported available on the Cultivation Classic
website. ``We wouldn't take these data and hand them over to Monsanto or
some other corporate juggernaut,'' she says. ``We want the folks who
have social-justice components in their workplaces, who are mindful of
the resources they use.''

Her most recent findings might disappoint any cannabis company, large or
small. After she crunched the Cultivation Classic numbers, Rae could not
find strong correlations between any single terpene and the high that
resulted. ``If we just look at the individual terpenes and try to
correlate them with any of our measures about the experience, we have a
total scatterplot,'' Rae told me. ``It's like a shotgun. The points are
everywhere.'' She could reach only one conclusion: ``There is just no
association with any singular terpene for any question we have.''

Rae was neither surprised nor disheartened. After all, she explained,
the entourage effect relies on the interactions among hundreds of
chemicals, and the attempt to parse it is still in its infancy. ``This
is a fishing expedition,'' Rae says. ``Are there any meaningful
conclusions to be drawn at all? Is it all about what a person expects
from their experience? Is it all about their own endogenous cannabinoid
system? Is there any pattern whatsoever besides that THC is
intoxicating? We don't really know.''

Not that she is about to abandon her research: A deeper analysis of the
interactions of terpenes and cannabinoids, which she plans to perform,
may yet yield correlations. Rae likens it to music, suggesting that
``terpenes might be the timbre of the experience, while THC is the
volume.'' She's also aware that the biggest confounding factor in the
attempt to parse the entourage is the person who is hosting it. Someone
who smokes a bowl at the end of a long day of physical labor may well
have a different experience if she smokes the same pot with friends at a
party or before she does yoga. Someone who buys weed with a particular
expectation may well have an outcome shaped by that expectation. And
perhaps most important, the particulars of an individual's
neurochemistry can change the way a plant's chemicals affect the brain.

The best hope for someone seeking a predictable high may come directly
from users, who can inform one another about the virtues of particular
varieties --- just as they did in the old days, only this time with help
from big data. Rae foresees an app that can tell a person what cultivars
are liked by people who liked what she liked, and for which purposes. It
wouldn't promise quite the ``surefire'' experience marketed by some of
the bigger cannabis companies, but it would be reasonably reliable, as
far as psychoactive substances go.

Of course, that would complicate the prospects of an industry that is
going all in on the idea that the entourage effect, whatever its
constituents and dynamics, can be unpacked and put to use in the market.
That doesn't bother Rae. In the end, she says, there's not much point to
taking drugs if the outcome is written on the label and if the drug is
not taken mindfully. ``We have to understand that there is always going
to be some level of exploration,'' she told me. The ongoing elusiveness
of the entourage effect ``is what's exciting, because it's not
necessarily telling us more about the plant; it's telling us more about
ourselves.''

Rae's approach to cannabis is similar to that envisioned by Russo when
he put down his copy of ``Stalking the Healthful Herbs'' and started his
study of terpenes. But the future of the drug probably doesn't belong to
them --- it belongs to companies like MedMen and Canopy and maybe, one
day, even Monsanto. Consumer capitalism is endlessly resourceful at
transforming almost any human desire into a standardized product on the
shelf. The smart money has already placed its bets, and the humble
cannabis plant can put up a fight for only so long.

Advertisement

\protect\hyperlink{after-bottom}{Continue reading the main story}

\hypertarget{site-index}{%
\subsection{Site Index}\label{site-index}}

\hypertarget{site-information-navigation}{%
\subsection{Site Information
Navigation}\label{site-information-navigation}}

\begin{itemize}
\tightlist
\item
  \href{https://help.nytimes3xbfgragh.onion/hc/en-us/articles/115014792127-Copyright-notice}{©~2020~The
  New York Times Company}
\end{itemize}

\begin{itemize}
\tightlist
\item
  \href{https://www.nytco.com/}{NYTCo}
\item
  \href{https://help.nytimes3xbfgragh.onion/hc/en-us/articles/115015385887-Contact-Us}{Contact
  Us}
\item
  \href{https://www.nytco.com/careers/}{Work with us}
\item
  \href{https://nytmediakit.com/}{Advertise}
\item
  \href{http://www.tbrandstudio.com/}{T Brand Studio}
\item
  \href{https://www.nytimes3xbfgragh.onion/privacy/cookie-policy\#how-do-i-manage-trackers}{Your
  Ad Choices}
\item
  \href{https://www.nytimes3xbfgragh.onion/privacy}{Privacy}
\item
  \href{https://help.nytimes3xbfgragh.onion/hc/en-us/articles/115014893428-Terms-of-service}{Terms
  of Service}
\item
  \href{https://help.nytimes3xbfgragh.onion/hc/en-us/articles/115014893968-Terms-of-sale}{Terms
  of Sale}
\item
  \href{https://spiderbites.nytimes3xbfgragh.onion}{Site Map}
\item
  \href{https://help.nytimes3xbfgragh.onion/hc/en-us}{Help}
\item
  \href{https://www.nytimes3xbfgragh.onion/subscription?campaignId=37WXW}{Subscriptions}
\end{itemize}
