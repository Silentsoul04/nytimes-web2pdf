Sections

SEARCH

\protect\hyperlink{site-content}{Skip to
content}\protect\hyperlink{site-index}{Skip to site index}

\href{https://www.nytimes3xbfgragh.onion/section/world/europe}{Europe}

\href{https://myaccount.nytimes3xbfgragh.onion/auth/login?response_type=cookie\&client_id=vi}{}

\href{https://www.nytimes3xbfgragh.onion/section/todayspaper}{Today's
Paper}

\href{/section/world/europe}{Europe}\textbar{}Covid-19 Changed How the
World Does Science, Together

\url{https://nyti.ms/2Uy7HRD}

\begin{itemize}
\item
\item
\item
\item
\item
\item
\end{itemize}

\hypertarget{the-coronavirus-outbreak}{%
\subsubsection{\texorpdfstring{\href{https://www.nytimes3xbfgragh.onion/news-event/coronavirus?name=styln-coronavirus-national\&region=TOP_BANNER\&block=storyline_menu_recirc\&action=click\&pgtype=Article\&impression_id=931eace0-f2cd-11ea-bbb3-1fcabe959d46\&variant=undefined}{The
Coronavirus
Outbreak}}{The Coronavirus Outbreak}}\label{the-coronavirus-outbreak}}

\begin{itemize}
\tightlist
\item
  live\href{https://www.nytimes3xbfgragh.onion/2020/09/09/world/covid-19-coronavirus.html?name=styln-coronavirus-national\&region=TOP_BANNER\&block=storyline_menu_recirc\&action=click\&pgtype=Article\&impression_id=931ed3f0-f2cd-11ea-bbb3-1fcabe959d46\&variant=undefined}{Latest
  Updates}
\item
  \href{https://www.nytimes3xbfgragh.onion/interactive/2020/us/coronavirus-us-cases.html?name=styln-coronavirus-national\&region=TOP_BANNER\&block=storyline_menu_recirc\&action=click\&pgtype=Article\&impression_id=931ed3f1-f2cd-11ea-bbb3-1fcabe959d46\&variant=undefined}{Maps
  and Cases}
\item
  \href{https://www.nytimes3xbfgragh.onion/interactive/2020/science/coronavirus-vaccine-tracker.html?name=styln-coronavirus-national\&region=TOP_BANNER\&block=storyline_menu_recirc\&action=click\&pgtype=Article\&impression_id=931ed3f2-f2cd-11ea-bbb3-1fcabe959d46\&variant=undefined}{Vaccine
  Tracker}
\item
  \href{https://www.nytimes3xbfgragh.onion/2020/09/02/your-money/eviction-moratorium-covid.html?name=styln-coronavirus-national\&region=TOP_BANNER\&block=storyline_menu_recirc\&action=click\&pgtype=Article\&impression_id=931ed3f3-f2cd-11ea-bbb3-1fcabe959d46\&variant=undefined}{Eviction
  Moratorium}
\item
  \href{https://www.nytimes3xbfgragh.onion/2020/09/09/upshot/coronavirus-surprise-test-fees.html?name=styln-coronavirus-national\&region=TOP_BANNER\&block=storyline_menu_recirc\&action=click\&pgtype=Article\&impression_id=931ed3f4-f2cd-11ea-bbb3-1fcabe959d46\&variant=undefined}{Surprise
  Test Fees}
\end{itemize}

Advertisement

\protect\hyperlink{after-top}{Continue reading the main story}

Supported by

\protect\hyperlink{after-sponsor}{Continue reading the main story}

\hypertarget{covid-19-changed-how-the-world-does-science-together}{%
\section{Covid-19 Changed How the World Does Science,
Together}\label{covid-19-changed-how-the-world-does-science-together}}

Never before, scientists say, have so many of the world's researchers
focused so urgently on a single topic. Nearly all other research has
ground to a halt.

\includegraphics{https://static01.graylady3jvrrxbe.onion/images/2020/03/31/world/00virus-scientists1/00virus-scientists1-articleLarge-v2.jpg?quality=75\&auto=webp\&disable=upscale}

\href{https://www.nytimes3xbfgragh.onion/by/matt-apuzzo}{\includegraphics{https://static01.graylady3jvrrxbe.onion/images/2018/06/14/multimedia/author-matt-apuzzo/author-matt-apuzzo-thumbLarge.png}}\href{https://www.nytimes3xbfgragh.onion/by/david-d-kirkpatrick}{\includegraphics{https://static01.graylady3jvrrxbe.onion/images/2018/10/15/multimedia/author-david-d-kirkpatrick/author-david-d-kirkpatrick-thumbLarge-v2.png}}

By \href{https://www.nytimes3xbfgragh.onion/by/matt-apuzzo}{Matt Apuzzo}
and
\href{https://www.nytimes3xbfgragh.onion/by/david-d-kirkpatrick}{David
D. Kirkpatrick}

\begin{itemize}
\item
  Published April 1, 2020Updated April 14, 2020
\item
  \begin{itemize}
  \item
  \item
  \item
  \item
  \item
  \item
  \end{itemize}
\end{itemize}

\href{https://cn.nytimes3xbfgragh.onion/world/20200402/coronavirus-science-research-cooperation/}{阅读简体中文版}\href{https://cn.nytimes3xbfgragh.onion/world/20200402/coronavirus-science-research-cooperation/zh-hant/}{閱讀繁體中文版}

Using
\href{https://twitter.com/realDonaldTrump/status/1244029041432244224}{flag-draped
memes} and
\href{https://www.scmp.com/news/china/military/article/3075396/how-chinas-military-took-frontline-role-coronavirus-crisis}{military}
\href{https://www.nytimes3xbfgragh.onion/2020/03/22/us/politics/coronavirus-trump-wartime-president.html}{terminolog}y,
the Trump administration and its Chinese counterparts have cast
coronavirus research as
\href{https://twitter.com/realdonaldtrump/status/1237813728558678026}{national
imperatives},
\href{https://www.abacusnews.com/tech/how-china-and-us-are-racing-develop-coronavirus-vaccines/article/3075760?utm_content=article\&utm_medium=Social\&utm_source=Facebook\#Echobox=1584522387}{sparking
talk} of a
\href{https://www.nytimes3xbfgragh.onion/2020/03/19/us/politics/coronavirus-vaccine-competition.html}{biotech
arms race}.

The world's
\href{https://www.nytimes3xbfgragh.onion/2020/04/14/science/coronavirus-preprint-servers.html}{scientists},
for the most part, have responded with a collective eye roll.

``Absolutely ridiculous,'' said Jonathan Heeney, a Cambridge University
researcher working on a
\href{https://www.nytimes3xbfgragh.onion/2020/04/14/science/coronavirus-preprint-servers.html}{coronavirus}
vaccine.

``That isn't how things happen,'' said Adrian Hill, the head of the
Jenner Institute at Oxford, one of the largest vaccine research centers
at an academic institution.

While political leaders have locked their borders, scientists have been
shattering theirs, creating a global collaboration unlike any in
history. Never before, researchers say, have so many experts in so many
countries focused simultaneously on a single topic and with such
urgency. Nearly all other research has ground to a halt.

Normal imperatives like academic credit have been set aside. Online
repositories make studies available months ahead of journals.
Researchers have \href{https://www.gisaid.org/}{identified and shared}
hundreds of viral genome sequences. More than 200 clinical trials have
been launched, bringing together hospitals and laboratories around the
globe.

``I never hear scientists --- true scientists, good quality scientists
--- speak in terms of nationality,'' said Dr. Francesco Perrone, who is
leading a coronavirus clinical trial in Italy. ``My nation, your nation.
My language, your language. My geographic location, your geographic
location. This is something that is really distant from true top-level
scientists.''

\includegraphics{https://static01.graylady3jvrrxbe.onion/images/2020/03/31/world/00virus-scientists2/merlin_171120969_5f1bb448-d9e2-46ae-bc7b-a8b9e113e609-articleLarge.jpg?quality=75\&auto=webp\&disable=upscale}

On a recent morning, for example, scientists at the University of
Pittsburgh discovered that a ferret exposed to Covid-19 particles had
developed a high fever --- a potential advance toward animal vaccine
testing. Under ordinary circumstances, they would have started work on
an academic journal article.

``But you know what? There is going to be plenty of time to get papers
published,'' said Paul Duprex, a virologist leading the university's
vaccine research. Within two hours, he said, he had shared the findings
with scientists around the world on a World Health Organization
conference call. ``It is pretty cool, right? You cut the crap, for lack
of a better word, and you get to be part of a global enterprise.''

For Mr. Trump, the unabashedly ``America First'' president, Dr. Duprex
and other American scientists represent the world's best hope for a
vaccine. ``America will get it done!'' the president
\href{https://twitter.com/realdonaldtrump/status/1237813728558678026}{declared}.

But trying to sew a ``Made in the USA'' label onto scientific research
gets complicated.

Dr. Duprex's lab in Pittsburgh is collaborating with the Pasteur
Institute in Paris and the Austrian drug company Themis Bioscience. The
consortium has received funding from the Coalition for Epidemic
Preparedness Innovation, a Norway-based organization financed by the
Bill and Melinda Gates Foundation and a group of governments, and is in
talks with the Serum Institute of India, one of the largest vaccine
manufacturers in the world.

Vaccine researchers at Oxford recently made use of animal-testing
results shared by the National Institutes of Health's Rocky Mountain
Laboratory in Montana.

Image

An ICU ward converted for coronavirus patients at the Papa Giovanni
XXIII hospital in Bergamo, Italy.Credit...Fabio Bucciarelli for The New
York Times

Separately, the French public-health research center Inserm is
sponsoring clinical trials on four drugs that may help treat Covid-19
patients. The trials are underway in France, with plans to expand
quickly to other nations.

In some ways, the coronavirus response reflects a medical community that
has long been international in scope. At Massachusetts General Hospital,
a team of Harvard doctors is testing the effectiveness of inhaled nitric
oxide on coronavirus patients. The research is being carried out in
conjunction with Xijing Hospital in China and a pair of hospitals in
northern Italy. Doctors in those centers have been collaborating for
years.

But the coronavirus has ignited the scientific community in ways that no
other outbreak or medical mystery has before. That reflects the scope of
the pandemic and the fact that, for many researchers, the hot zone is no
longer an impoverished village in the developing world. It is their
hometowns.

``This is playing at home,'' said Professor Hill, of Oxford. He has
worked on vaccines for Ebola, malaria and tuberculosis, diseases that
have been most prevalent in Africa. ``But for Covid, it is happening
right here.''

Several scientists said the closest comparison to this moment might be
the height of the AIDS epidemic in the 1990s, when scientists and
doctors locked arms to combat the disease. But today's technology and
\href{https://www.sciencemag.org/news/2020/02/completely-new-culture-doing-research-coronavirus-outbreak-changes-how-scientists\#}{the
pace of information-sharing} dwarfs what was possible three decades ago.

As a practical matter, medical scientists today have little choice but
to study the coronavirus if they want to work at all. Most other
laboratory research has been put on hold because of social distancing,
lockdowns or work-from-home restrictions.

Image

At Massachusetts General Hospital, a team of Harvard doctors is testing
the effectiveness of inhaled nitric oxide on coronavirus
patients.Credit...Michael Dwyer/Associated Press

The pandemic is also eroding the secrecy that pervades academic medical
research, said Dr. Ryan Carroll, a Harvard Medical professor who is
involved in the coronavirus trial there. Big, exclusive research can
lead to grants, promotions and tenure, so scientists often work in
secret, suspiciously hoarding data from potential competitors, he said.

``The ability to work collaboratively, setting aside your personal
academic progress, is occurring right now because it's a matter of
survival,'' he said.

One small measure of openness can be found on the servers of medRxiv and
bioRxiv, two online archives that share academic research before it has
been reviewed and published in journals. The archives have been deluged
with coronavirus research from across the globe. Despite the
nationalistic tone set by the Chinese president, Xi Jinping, Chinese
researchers have contributed a significant portion of the coronavirus
research available in the archive.

Though Chinese officials
\href{https://www.nytimes3xbfgragh.onion/2020/02/01/world/asia/china-coronavirus.html}{initially
covered up the outbreak} and have since
\href{https://www.nytimes3xbfgragh.onion/2020/02/28/world/asia/china-coronavirus-response-propaganda.html}{used
it for propaganda purposes}, Chinese scientists have in many ways led
the world's coronavirus research. A Chinese laboratory made public the
initial viral genome in January, a disclosure that formed the basis for
coronavirus tests worldwide. And some of today's most promising clinical
trials can trace their origins to early Chinese research on the disease.

Few areas of the world have been spared. Last year, Jamal Ahmadzadeh, an
epidemiologist at Urmia University in Iran, warned that the world needed
a rapid-alert system in response to MERS, another coronavirus. No
country was immune to the risk, he wrote. In an email last week, as Iran
grappled with one of the world's worst coronavirus outbreaks, he wrote
that defeating the virus required information-sharing across
laboratories and across borders.

Even scientists working in fields beyond infectious diseases have been
drawn into the effort. Dr. Perrone, who is supervising an Italian
clinical trial of the immunosuppressive drug tocilizumab, is a cancer
specialist. He is involved because of his experience running clinical
trials for the National Cancer Institute in Naples.

Dr. Perrone said the coronavirus pandemic may make medical science more
nimble long after the emergency has passed. Ten days after researchers
conceived of the trial, the normally laborious government approval
process was complete and doctors began enrolling patients, he said.
``This should be a lesson for the future,'' he said.

Image

Volunteers disinfecting a shopping complex in Wuhan, China, on
Tuesday.Credit...Aly Song/Reuters

While Mr. Trump has touted American pharmaceutical prowess, and big drug
companies like Pfizer and Johnson \& Johnson have announced that they
are bankrolling coronavirus vaccine research, the biggest drug companies
focus on drugs they can sell year after year in affluent countries, not
during short-lived crises centered in the developing world. Vaccine
research has been seen as insufficiently profitable.

When Ebola captured the world's attention in 2014, for example, the drug
giants that chased a vaccine all took major losses on their investments.
The first vaccine, originally devised by a Canadian government lab and
now sold by Merck, was approved for sale last year, long after the
epidemic faded.

``Of course there are people in competition. This is the human
condition,'' said Dr. Yazdan Yazdanpanah, the director of infectious
disease at Inserm in France. ``What is important is to come up with a
solution for everyone. The way to achieve that is to collaborate.''

Advertisement

\protect\hyperlink{after-bottom}{Continue reading the main story}

\hypertarget{site-index}{%
\subsection{Site Index}\label{site-index}}

\hypertarget{site-information-navigation}{%
\subsection{Site Information
Navigation}\label{site-information-navigation}}

\begin{itemize}
\tightlist
\item
  \href{https://help.nytimes3xbfgragh.onion/hc/en-us/articles/115014792127-Copyright-notice}{©~2020~The
  New York Times Company}
\end{itemize}

\begin{itemize}
\tightlist
\item
  \href{https://www.nytco.com/}{NYTCo}
\item
  \href{https://help.nytimes3xbfgragh.onion/hc/en-us/articles/115015385887-Contact-Us}{Contact
  Us}
\item
  \href{https://www.nytco.com/careers/}{Work with us}
\item
  \href{https://nytmediakit.com/}{Advertise}
\item
  \href{http://www.tbrandstudio.com/}{T Brand Studio}
\item
  \href{https://www.nytimes3xbfgragh.onion/privacy/cookie-policy\#how-do-i-manage-trackers}{Your
  Ad Choices}
\item
  \href{https://www.nytimes3xbfgragh.onion/privacy}{Privacy}
\item
  \href{https://help.nytimes3xbfgragh.onion/hc/en-us/articles/115014893428-Terms-of-service}{Terms
  of Service}
\item
  \href{https://help.nytimes3xbfgragh.onion/hc/en-us/articles/115014893968-Terms-of-sale}{Terms
  of Sale}
\item
  \href{https://spiderbites.nytimes3xbfgragh.onion}{Site Map}
\item
  \href{https://help.nytimes3xbfgragh.onion/hc/en-us}{Help}
\item
  \href{https://www.nytimes3xbfgragh.onion/subscription?campaignId=37WXW}{Subscriptions}
\end{itemize}
