Sections

SEARCH

\protect\hyperlink{site-content}{Skip to
content}\protect\hyperlink{site-index}{Skip to site index}

\href{https://www.nytimes3xbfgragh.onion/section/travel}{Travel}

\href{https://myaccount.nytimes3xbfgragh.onion/auth/login?response_type=cookie\&client_id=vi}{}

\href{https://www.nytimes3xbfgragh.onion/section/todayspaper}{Today's
Paper}

\href{/section/travel}{Travel}\textbar{}Why Is Getting a Refund From an
Online Travel Agency So Hard?

\url{https://nyti.ms/2XgX6MP}

\begin{itemize}
\item
\item
\item
\item
\item
\item
\end{itemize}

\href{https://www.nytimes3xbfgragh.onion/spotlight/at-home?action=click\&pgtype=Article\&state=default\&region=TOP_BANNER\&context=at_home_menu}{At
Home}

\begin{itemize}
\tightlist
\item
  \href{https://www.nytimes3xbfgragh.onion/2020/09/07/travel/route-66.html?action=click\&pgtype=Article\&state=default\&region=TOP_BANNER\&context=at_home_menu}{Cruise
  Along: Route 66}
\item
  \href{https://www.nytimes3xbfgragh.onion/2020/09/04/dining/sheet-pan-chicken.html?action=click\&pgtype=Article\&state=default\&region=TOP_BANNER\&context=at_home_menu}{Roast:
  Chicken With Plums}
\item
  \href{https://www.nytimes3xbfgragh.onion/2020/09/04/arts/television/dark-shadows-stream.html?action=click\&pgtype=Article\&state=default\&region=TOP_BANNER\&context=at_home_menu}{Watch:
  Dark Shadows}
\item
  \href{https://www.nytimes3xbfgragh.onion/interactive/2020/at-home/even-more-reporters-editors-diaries-lists-recommendations.html?action=click\&pgtype=Article\&state=default\&region=TOP_BANNER\&context=at_home_menu}{Explore:
  Reporters' Google Docs}
\end{itemize}

Advertisement

\protect\hyperlink{after-top}{Continue reading the main story}

Supported by

\protect\hyperlink{after-sponsor}{Continue reading the main story}

\hypertarget{why-is-getting-a-refund-from-an-online-travel-agency-so-hard}{%
\section{Why Is Getting a Refund From an Online Travel Agency So
Hard?}\label{why-is-getting-a-refund-from-an-online-travel-agency-so-hard}}

Booking sites are scrambling to handle an unprecedented number of
cancellation requests, and many customers say they are falling short.

\includegraphics{https://static01.graylady3jvrrxbe.onion/images/2020/04/12/travel/12OTA-rev/12OTA-rev-articleLarge.jpg?quality=75\&auto=webp\&disable=upscale}

\href{https://www.nytimes3xbfgragh.onion/by/tariro-mzezewa}{\includegraphics{https://static01.graylady3jvrrxbe.onion/images/2018/08/24/opinion/tariro-headshot/tariro-headshot-thumbLarge-v2.png}}

By \href{https://www.nytimes3xbfgragh.onion/by/tariro-mzezewa}{Tariro
Mzezewa}

\begin{itemize}
\item
  Published April 3, 2020Updated April 7, 2020
\item
  \begin{itemize}
  \item
  \item
  \item
  \item
  \item
  \item
  \end{itemize}
\end{itemize}

\emph{Travel and travel planning are being disrupted by the worldwide
spread of the coronavirus. For the latest updates,
read}\href{https://www.nytimes3xbfgragh.onion/news-event/coronavirus?action=click\&module=RelatedLinks\&pgtype=Article}{\emph{The
New York Times's Covid-19 coverage here}}\emph{.}

For many travelers looking to save money, booking through a third party
site like Expedia, Priceline or Orbitz has become second nature,
especially for those looking for last-minute travel or package deals.

Then came the coronavirus pandemic. Many people attempting to cancel
trips and get refunds have learned that including a middleman when
booking a trip can make things complicated.

Luisa Ciaffa paid \$1,545 for two flights to Florence, Italy, on
Kiwi.com, a Czech online travel tech company that's popular among
students, in particular, for its low fares. On March 12, with Italy on
lockdown and the World Health Organization having declared Covid-19 a
pandemic, she tried to cancel and get a refund from Kiwi.com. The
process, she said, ``made absolutely no sense.''

She called and spent a long time on hold. When she got through to
someone, she was offered a refund of 10 euros. On social media, other
travelers who booked trips costing hundreds or thousands of dollars on
Kiwi.com lamented that they, too, were offered 10 euros and nothing
more. She was eventually given a \$111 credit on Kiwi.com.

``Someone in what was clearly a busy call center said that our flights
hadn't been canceled yet, so it was my choice to cancel and not travel,
so I couldn't get a refund,'' Ms. Ciaffa said.

When asked about the policy, Raymond Vrijenhoek, vice president for
brand and strategic communications at Kiwi.com, said in an email that
refunds are dependent on the policies set by airlines.

\begin{center}\rule{0.5\linewidth}{\linethickness}\end{center}

\hypertarget{6-things-to-keep-in-mind-about-otas-and-refunds}{%
\subsubsection{6 things to keep in mind about O.T.A.s and
refunds}\label{6-things-to-keep-in-mind-about-otas-and-refunds}}

\begin{itemize}
\item
  Online travel agencies (O.T.A.s) were overwhelmed by calls during the
  recent spike in travelers looking to cancel. A number of them,
  including Expedia and Priceline, have introduced online cancellation
  \href{https://www.priceline.com/help-page/faq/237898/advisories-\&-covid-19-information-}{tools}
  to make the process easier. Check their sites before calling.
\item
  Many airlines are now letting you cancel flights directly with them,
  even if you booked through an O.T. A.; try your airline first.
\item
  Was your fare or room nonrefundable? If you didn't read the fine print
  when you booked, check it now. Many providers are relaxing their
  rules, but not all.
\item
  Online travel agencies can't preemptively issue a refund to someone
  without someone at the hotel or airline signing off on that refund.
\item
  The big home rental sites have different policies. Airbnb is letting
  guests cancel and get a full
  refund\href{https://news.airbnb.com/a-letter-to-hosts/}{through May
  31}. VRBO leaves it up to hosts.
\item
  The new rules are still evolving. If you have a future reservation,
  wait until the date is approaching to cancel.
\end{itemize}

\begin{center}\rule{0.5\linewidth}{\linethickness}\end{center}

Kiwi.com is one of the smaller companies known as online travel agencies
(O.T.A.s), which include big names like Expedia.com, Priceline.com,
Kayak and numerous smaller outfits. While some specialize in a
particular segment, like SnapTravel, which helps people book hotels,
they often serve as one-stop shops for trips that can involve flights,
rental cars and hotels.

Under normal circumstances, getting a refund from an O.T.A. is a pretty
straightforward process. A traveler contacts the third party booking
site and requests a refund, and the third party contacts the airline or
hotel to process the refund. That's if the traveler has booked a
refundable fare, which many people who have wanted to cancel their plans
did not.

On social media and in emails to The Times, many travelers complained
that getting help from their O.T.A. has been particularly difficult
during the coronavirus pandemic. One site,
\href{https://bookit.com/covid-19/}{Bookit.com}, completely suspended
its operations, and told customers to contact their credit card
companies for assistance.

Here's what has happened.

\hypertarget{they-were-overwhelmed-by-panic-day}{%
\subsection{They were overwhelmed by `panic
day'}\label{they-were-overwhelmed-by-panic-day}}

The agencies say they were overwhelmed by the immediate spike in
travelers looking to cancel.

``The Wednesday Tom Hanks said he had coronavirus, and the N.B.A. season
got shut down was panic day,'' Hussein Fazal, the chief executive of
SnapTravel, said. ``Everyone in the U.S. started panicking, we saw a
spike in volume and then the travel bans came.''

``We're getting hundreds of thousands of more calls on any given day,''
said Sarah Waffle Gavin, vice president for global communications and
corporate brand at Expedia, which includes sites like Orbitz and
HomeAway under its umbrella. ``If that was the only problem, we could
totally solve it.''

Expedia's call volumes have been five to seven times higher than
average, amounting to thousands more calls than it would normally
receive, even during its busiest times. VRBO, the home booking site, saw
its call volume increase more than 300 percent.

SnapTravel's Mr. Fazal said call volume has been five times higher than
usual and its chat volume has been three times higher than usual; its
average call wait time before Covid-19 was 45 seconds and now it is 7.5
minutes, he said (though travelers have complained of spending hours on
hold with various agencies).

\hypertarget{they-were-prepared-for-the-wrong-thing}{%
\subsection{They were prepared for the wrong
thing}\label{they-were-prepared-for-the-wrong-thing}}

The spike in refund requests occurred while companies were
simultaneously trying to equip their own teams to work remotely, said
Olivier Pailhès, co-founder and chief executive of Aircall, a
cloud-based phone system that provides its technology to companies. In
mid-to-late March, Aircall had a spike in calls from its customers in
the travel industry.

``We had a 100 to 400 percent increase in call volume from our clients
who were trying to figure out how to help people changing plans or
canceling trips,'' Mr. Pailhes said, adding that his company had its
best week of business in March. ``A lot of companies in travel have to
go remote and our company is perfect for that.''

Expedia said it had to act quickly to get employees prepared to work
from home to respect social distancing and shelter in place policies
that were enacted around the world. The company has backup plans, but as
with many industries, they were based on the idea that any emergency
would be localized.

``We have a resiliency plan, so if there's an earthquake somewhere or a
coup somewhere else, we roll the calls from that call center over to
another call center elsewhere in the world,'' Ms. Waffle Gavin said.
``But this isn't an earthquake or a coup. This wasn't an isolated
incident. This was happening to everybody all at once.''

Some employees lacked Wi-Fi or laptops at home, which meant the company
had to figure out if people could still come into offices and how to
ensure they could socially distance in that space.

``The VRBO customer service team pulled off a herculean task --- adding
250 agents to take calls by shifting people from other parts of the
business and accelerating new hire training,'' wrote Melanie Fish, a
spokeswoman for VRBO, in an email.

\hypertarget{nonrefundable-really}{%
\subsection{`Nonrefundable.' Really?}\label{nonrefundable-really}}

Many of the fares booked through the O.T.A.'s were cheaper nonrefundable
fares. By design, these rooms and flights are more affordable because
they bank on people's certainty and willingness to take a risk. One
analyst estimated that half or more of trips booked on Booking.com and
Expedia are nonrefundable.

But when faced with a worldwide pandemic, travelers have felt they were
being forced into an unfair position --- told by authorities not to
travel because they could risk their health and the health of others,
and getting no relief from travel companies who they felt were holding
their money hostage.

Mr. Fazal of SnapTravel said his company is currently mostly getting
requests for refunds from people who booked nonrefundable rooms and are
trying to get their money back anyway.

``If a booking is refundable, it's easy,'' he said. ``But when it's
nonrefundable, it's harder. Every O.T.A. is built like this --- the
system is built to make it hard to cancel, so we're having to go and
make exceptions.''

When Bookit.com shut down, Coty Johnson was left in a lurch, unsure of
how to get the \$3,600 he'd spent on nonrefundable flights and a hotel
for his honeymoon in Jamaica back. No one at Bookit has responded to his
emails and calls. His hotel won't issue a refund, and American Airlines
offered only a voucher.

``When we called the resort and airlines they informed us that the trip
was never paid for and there was no refund they could process to us
either, so we are still out \$3,600 and a honeymoon,'' he said.

\hypertarget{then-airlines-and-hotels-got-more-flexible}{%
\subsection{Then airlines and hotels got more
flexible}\label{then-airlines-and-hotels-got-more-flexible}}

In the face of the rush to cancel, many of the airlines moved to a more
flexible policy in which they waived change fees and are letting even
those who booked nonrefundable fares get a credit for a future trip. The
big hotel chains made the same decision.

Major airlines, including
\href{https://www.aa.com/i18n/travel-info/coronavirus-updates.jsp}{American},
\href{https://www.delta.com/mytrips/}{Delta} and
\href{https://www.united.com/en/us/manageres/mytrips}{United}, also
started letting people who had booked their flights through an online
travel agency like Priceline or Travelocity get a credit or refund from
the airline rather than sending them back to the online travel agency,
as used to be the policy.

Giving you a credit for a future flight lets the airlines hang onto your
money and also encourages you to travel with them once the pandemic is
over.

\hypertarget{otas-say-they-couldnt-follow-suit}{%
\subsection{O.T.A.s say they couldn't follow
suit}\label{otas-say-they-couldnt-follow-suit}}

But an O.T.A. doesn't have your money. When you book with a third-party
site, they take your payment and parcel it out to the various suppliers
of your vacation services.

Online travel agencies are dependent on the decisions of their hotel and
airline suppliers, so they can't preemptively issue a refund to someone
without someone at the hotel or airline signing off on that refund.

Mr. Vrijenhoek, of Kiwi.com, said that is the case at his O.T.A. ``We
are not holding any refunds from customers. It just takes some time to
claim and receive the money back from the airlines,'' he said, adding
that the workload at the company's customer care centers was
``unprecedented,'' leading to delays.

``Most travelers will never know the name of the bus that took them from
the airport to the hotel, or that it was paid for 60 days ago and their
hotel was paid for a while ago,'' said Jeff Ment, a travel industry
lawyer. ``When you've paid for a trip, that money goes to all those
parts of your trip, and for that O.T.A. to get it back is very
difficult.''

Chris Anderson, a professor at the Cornell School of Hotel
Administration echoed this. ``A lot more interaction has to happen
between staff at the online travel agent and staff at the hotel before a
refund or credit can be offered, so it's no longer a simple online
transaction,'' Mr. Anderson said.

Instead of reaching out to a hotel after a customer requests a refund,
Mr. Fazal said, SnapTravel representatives have been asked by hotels
that are also overwhelmed by people hoping to cancel, to gather requests
for refunds and send them in bulk, rather than as they occur. A refund
process that usually would take two to three days to get necessary
approvals may now take several weeks, he said.

\hypertarget{and-just-who-is-responsible-can-be-fuzzy}{%
\subsection{And just who is responsible can be
fuzzy}\label{and-just-who-is-responsible-can-be-fuzzy}}

Many travelers say they have been caught in limbo between the online
agency and the actual provider of the service. Brad Tinnin is one of
them.

Mr. Tinnin has spent the last several weeks trying to get a refund from
VRBO for a house he booked for a trip to Palm Springs, Calif. The home
rental, which totaled \$2,600, was managed by a company called Oranj
Palm Vacation and was near a rental his neighbors booked on Airbnb for
the same trip.

As cases of the coronavirus increased in the United States, he and his
wife decided that it wouldn't be wise to travel from their home in St.
Louis, Mo. He canceled the flights which he had booked directly through
American Airlines and was given a flight credit for the full cost of the
flight. That process, he said, was seamless.

His neighbors received a full refund from Airbnb, but ``getting a refund
from Vrbo or the management company or whoever is supposed to give the
actual refund has been impossible,'' Mr. Tinnin said.

Mr. Tinnin reached out to Oranj Palm Vacation, which told him to contact
his insurance company and VRBO. VRBO told him to contact Oranj Palm
Vacation. Oranj Palm said it could cancel his reservation, relist the
home and if someone else booked it, he would receive a refund, but Oranj
would keep 10 percent. Then he was told by the company that his
insurance company would be better equipped to help. The insurance
company isn't covering coronavirus cancellations.

Last week, VRBO said that it reached out to the property manager, and
Mr. Tinnin will receive a 50 percent refund and the other 50 percent
will be applied as a credit to a future stay.

Mr. Tinnin is not eager to book with VRBO again. ``I'll never use them
again after this ordeal,'' he said, noting that the Palm Springs trip
would have been his ninth booking with the company in four years, and a
tenth trip for later this year was also booked.

Sarah Firshein contributed reporting.

\emph{\textbf{Follow New York Times Travel}}
\emph{on}\href{https://www.instagram.com/nytimestravel/}{\emph{Instagram}}\emph{,}\href{https://twitter.com/nytimestravel}{\emph{Twitter}}
\emph{and}\href{https://www.facebookcorewwwi.onion/nytimestravel/}{\emph{Facebook}}\emph{.
And}\href{https://www.nytimes3xbfgragh.onion/newsletters/traveldispatch}{\emph{sign
up for our weekly Travel Dispatch newsletter}} \emph{to receive expert
tips on traveling smarter and inspiration for your next vacation.}

Advertisement

\protect\hyperlink{after-bottom}{Continue reading the main story}

\hypertarget{site-index}{%
\subsection{Site Index}\label{site-index}}

\hypertarget{site-information-navigation}{%
\subsection{Site Information
Navigation}\label{site-information-navigation}}

\begin{itemize}
\tightlist
\item
  \href{https://help.nytimes3xbfgragh.onion/hc/en-us/articles/115014792127-Copyright-notice}{©~2020~The
  New York Times Company}
\end{itemize}

\begin{itemize}
\tightlist
\item
  \href{https://www.nytco.com/}{NYTCo}
\item
  \href{https://help.nytimes3xbfgragh.onion/hc/en-us/articles/115015385887-Contact-Us}{Contact
  Us}
\item
  \href{https://www.nytco.com/careers/}{Work with us}
\item
  \href{https://nytmediakit.com/}{Advertise}
\item
  \href{http://www.tbrandstudio.com/}{T Brand Studio}
\item
  \href{https://www.nytimes3xbfgragh.onion/privacy/cookie-policy\#how-do-i-manage-trackers}{Your
  Ad Choices}
\item
  \href{https://www.nytimes3xbfgragh.onion/privacy}{Privacy}
\item
  \href{https://help.nytimes3xbfgragh.onion/hc/en-us/articles/115014893428-Terms-of-service}{Terms
  of Service}
\item
  \href{https://help.nytimes3xbfgragh.onion/hc/en-us/articles/115014893968-Terms-of-sale}{Terms
  of Sale}
\item
  \href{https://spiderbites.nytimes3xbfgragh.onion}{Site Map}
\item
  \href{https://help.nytimes3xbfgragh.onion/hc/en-us}{Help}
\item
  \href{https://www.nytimes3xbfgragh.onion/subscription?campaignId=37WXW}{Subscriptions}
\end{itemize}
