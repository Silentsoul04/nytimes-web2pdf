Sections

SEARCH

\protect\hyperlink{site-content}{Skip to
content}\protect\hyperlink{site-index}{Skip to site index}

\href{https://myaccount.nytimes3xbfgragh.onion/auth/login?response_type=cookie\&client_id=vi}{}

\href{https://www.nytimes3xbfgragh.onion/section/todayspaper}{Today's
Paper}

If My Classmates Are Going to Cheat on an Online Exam, Why Can't I?

\url{https://nyti.ms/34hBbWZ}

\begin{itemize}
\item
\item
\item
\item
\item
\item
\end{itemize}

\hypertarget{school-reopenings}{%
\subsubsection{\texorpdfstring{\href{https://www.nytimes3xbfgragh.onion/spotlight/schools-reopening?name=styln-coronavirus-schools-reopening\&region=TOP_BANNER\&block=storyline_menu_recirc\&action=click\&pgtype=Article\&impression_id=769bf890-f1d6-11ea-940c-23fcdfc65706\&variant=undefined}{School
Reopenings}}{School Reopenings}}\label{school-reopenings}}

\begin{itemize}
\tightlist
\item
  \href{https://www.nytimes3xbfgragh.onion/2020/09/04/us/bar-exam-coronavirus.html?name=styln-coronavirus-schools-reopening\&region=TOP_BANNER\&block=storyline_menu_recirc\&action=click\&pgtype=Article\&impression_id=769bf891-f1d6-11ea-940c-23fcdfc65706\&variant=undefined}{Delayed
  Licensing Exams}
\item
  \href{https://www.nytimes3xbfgragh.onion/interactive/2020/08/31/us/coronavirus-cases-children.html?name=styln-coronavirus-schools-reopening\&region=TOP_BANNER\&block=storyline_menu_recirc\&action=click\&pgtype=Article\&impression_id=769bf892-f1d6-11ea-940c-23fcdfc65706\&variant=undefined}{Rising
  Cases Among Children}
\item
  \href{https://www.nytimes3xbfgragh.onion/2020/09/01/world/schools-reopen-globe-students.html?name=styln-coronavirus-schools-reopening\&region=TOP_BANNER\&block=storyline_menu_recirc\&action=click\&pgtype=Article\&impression_id=769bf893-f1d6-11ea-940c-23fcdfc65706\&variant=undefined}{School
  Around the World}
\item
  \href{https://www.nytimes3xbfgragh.onion/interactive/2020/us/covid-college-cases-tracker.html?name=styln-coronavirus-schools-reopening\&region=TOP_BANNER\&block=storyline_menu_recirc\&action=click\&pgtype=Article\&impression_id=769c1fa0-f1d6-11ea-940c-23fcdfc65706\&variant=undefined}{Tracking
  College Cases}
\end{itemize}

Advertisement

\protect\hyperlink{after-top}{Continue reading the main story}

Supported by

\protect\hyperlink{after-sponsor}{Continue reading the main story}

\href{/column/the-ethicist}{The Ethicist}

\hypertarget{if-my-classmates-are-going-to-cheat-on-an-online-exam-why-cant-i}{%
\section{If My Classmates Are Going to Cheat on an Online Exam, Why
Can't
I?}\label{if-my-classmates-are-going-to-cheat-on-an-online-exam-why-cant-i}}

\includegraphics{https://static01.graylady3jvrrxbe.onion/images/2020/04/12/magazine/12Ethicist/12Ethicist-articleLarge.jpg?quality=75\&auto=webp\&disable=upscale}

By Kwame Anthony Appiah

\begin{itemize}
\item
  April 7, 2020
\item
  \begin{itemize}
  \item
  \item
  \item
  \item
  \item
  \item
  \end{itemize}
\end{itemize}

\emph{Because of efforts to slow the spread of Covid-19, the large
university I attend has, like many others, transitioned to online
instruction for the foreseeable future. In-person classes are not
prohibited, but the administration has strongly recommended against
them. Because we are on the quarter system, our final exams are
scheduled for next week, immediately before our spring break. This means
that professors must choose from a number of less-than-ideal options for
administering exams.}

\emph{Some have given students the choice of an optional final, or
canceled them altogether, basing their grade entirely on past work.
Others have chosen to use an online service that monitors students while
they take their tests in order to ensure that they do not cheat. The
downside of this service is that it requires access to a computer with a
webcam, a reliable internet connection and access to a quiet, empty
room. Those requirements pose a challenge for many college students,
particularly those with fewer resources --- and more roommates.}

\emph{In consideration of this, another popular option is to require an
online exam with a request for academic honesty as the only safeguard
against cheating. Based on conversations I've had with and heard among
classmates, I think it is fair to assume that the vast majority of
students will take advantage of the resources now available to them
(i.e., notes, friends, the internet) in order to succeed. This will
result in a much higher average performance than an in-person exam
would, putting anyone who does not cheat at a disadvantage as any
grading on a curve would hurt him or her.}

\emph{While I know that it is dishonest to cheat and I value my
integrity, I also want to maintain a high G.P.A., and it seems that
those goals are in conflict with each other. I plan to take my exam for
an uncurved class in accordance with academic-honesty policies, but
would it be entirely unacceptable to consult my friends or notes
minimally during a curved class's final?} Delaney

\textbf{It's a familiar} protest: ``But everyone else is doing it!'' You
won't be surprised that the Ethicist takes a dim view of this argument.
Cheating, being a form of dishonesty, is wrong even when rampant.

But beyond the poor choices your classmates are making, I'm concerned
about the poor choices your professors are making. A setup that
encourages cheating and penalizes honesty is a badly designed one.
Students should not be led unnecessarily into temptation.

Why not tell your instructors, in both the curved and uncurved classes,
that it makes sense to have an open-book exam? Doing this might require
changing the test. But given the circumstances you describe, it may be
the only responsible option.

If a professor insists on ignoring these realities, however, you should
still do the honest thing. Ethics is always, in part, about what kind of
person you ought to be. Even though your integrity could cost you on the
curve, it has distinct advantages when it comes to looking yourself in
the eye.

\emph{I'm the executive director of a small nonprofit whose employees
can easily work remotely. In implementing our remote-work plan, I shared
with the staff that the office isn't 100 percent off limits: I will stop
by a couple of times a week to get the mail; another person will be by
from time to time to water his plants; we may need to pick up a document
here and there. All of that is fine and doesn't contradict our attempt
to do the social distancing public-health officials are recommending.}

\emph{But for me, working at home is very unappealing. I live in a very
small house with a spouse who will also most likely be home, no private
space to set up for my work and a dog that is a big barker, which will
interrupt the many phone meetings I will need to conduct.}

\emph{From an ethical perspective, may I work from the office alone,
while everyone else works remotely? If I do so, do I need to be explicit
with our team and offer it to others?}

\emph{I believe that most people see remote work as an attractive option
--- you don't need to dress up, can be running laundry while working,
can have lunch with your spouse. Not me. May I go to the office, even
though we have instituted a remote-work policy?} Name Withheld

\href{https://www.nytimes3xbfgragh.onion/spotlight/schools-reopening?action=click\&pgtype=Article\&state=default\&region=MAIN_CONTENT_3\&context=storylines_keepup}{}

\hypertarget{school-reopenings-}{%
\subsubsection{School Reopenings ›}\label{school-reopenings-}}

\hypertarget{back-to-school}{%
\paragraph{Back to School}\label{back-to-school}}

Updated Sept. 4, 2020

The latest on how schools are reopening amid the pandemic.

\begin{itemize}
\item
  \begin{itemize}
  \tightlist
  \item
    There have been at least
    \href{https://www.nytimes3xbfgragh.onion/interactive/2020/us/covid-college-cases-tracker.html?name=styln-coronavirus-schools-reopening\&action=click\&pgtype=Article\&state=default\&region=MAIN_CONTENT_3\&context=storylines_keepup\&region=TOP_BANNER█=storyline_menu_recirc\&action=click\&pgtype=Article\&impression_id=149dfe80-eea3-11ea-aea8-57f827c5e458\&variant=1_Show}{51,000
    coronavirus cases}~at more than 1,000 American college campuses
    since the pandemic began, the latest New York Times's survey shows.
  \item
    \href{https://www.nytimes3xbfgragh.onion/2020/09/03/nyregion/new-york-suny-oneonta-coronavirus.html?action=click\&pgtype=Article\&state=default\&region=MAIN_CONTENT_3\&context=storylines_keepup}{SUNY
    Oneonta}~canceled in-person classes and sent students home because
    of a coronavirus outbreak.
  \item
    \href{https://www.nytimes3xbfgragh.onion/2020/09/04/world/americas/latin-america-education.html?\&action=click\&pgtype=Article\&state=default\&region=MAIN_CONTENT_3\&context=storylines_keepup}{Millions
    of college students}~in Latin America are leaving their studies
    because of the pandemic.
  \item
    Professional licensing exams have been severely disrupted by the
    coronavirus, making it difficult for
    \href{https://www.nytimes3xbfgragh.onion/2020/09/04/us/bar-exam-coronavirus.html?action=click\&pgtype=Article\&state=default\&region=MAIN_CONTENT_3\&context=storylines_keepup}{newly
    trained lawyers, doctors}~and others to start their careers.
  \end{itemize}
\end{itemize}

\textbf{Your worry is,} in essence, that you would be taking advantage
of your position of authority to grant yourself a privilege. So suppose
you were just another staff member. Would a fair-minded boss grant you
permission to do this?

To make that determination, this boss would need to survey staff members
and confirm your hunch that most people prefer to work remotely. If you
were the only person who wanted to work in the office, there'd be no
worries either about unfairness or about social exposure. Even if a
couple of you wanted to work in the office --- assuming that your state
hasn't ordered nonessential workers to stay at home and that you'd be
getting to work in a way (by car, say) that risked no further exposure
--- you each might be able to maintain distancing on the occasions when
you were both in the building. Under those circumstances, what you have
in mind could be just fine: There would be no harm in doing what a
reasonable boss would agree to your doing. You'll need to be
transparent, then, and make sure that your assumptions are warranted.
But as long as your use of the office is something you can defend to
your staff in this sort of way, they shouldn't regard it as unfair.

\emph{My partner and I are both employed and have two children in a day
care center. Our oldest attends four days a week, and our youngest
attends two days a week. My partner works four days a week as a
registered nurse at a clinic, and I am employed full time at a
higher-education institution as a director.}

\emph{Recently, in response to the coronavirus, my employer required all
nonessential employees to work remotely until further notice. In
addition to a number of other directives, the governor of our state
encouraged day care centers to stay open.}

\emph{We are in a fortunate position to suffer no significant financial
consequence keeping our children at home while I work remotely.
Admittedly, I would not be as productive in this situation as I could
otherwise be, but my employer is understanding. However, even if we
choose to keep our children at home, we will continue to be charged
weekly tuition from the day care center so long as it remains open.}

\emph{My partner and I have debated continuing to take our children to
day care. Is it selfish to send your children to day care if you can
easily keep them at home? Are we selfish to focus on ``getting our
money's worth'' for each week? Are families with the means to stay at
home with children obligated to do so for the public good, even if it
means forfeiting fees for something they're not using?} James E.,
Minneapolis

\textbf{Your partner is} a nurse and so should be in a position to
assess whether the kids' going to day care poses risks to them and to
others, including you. Public-health experts, I will note, tend to be
more concerned with the prospect of children's spreading infection than
falling ill with it. If getting sick isn't a big problem for most kids,
though, having sick parents definitely is. So do your best to maintain
social distancing for their sake as well as yours. And that probably
means keeping them home. You think that having kids underfoot will
reduce your productivity. You might consider how your productivity will
be affected if you come down with Covid-19.

As for your sunk costs? You should forget about getting value for money,
given that you're clearly in a position to put that aside. Even if you
don't send your kids to the center, you'll be supporting a place that is
relied upon by parents who don't have the luxury of deciding to keep
their kids at home. In this respect, you're making another social
contribution, beyond the one you'd make by social distancing. More
we-thinking and less me-thinking is one thing that's needed from all of
us right now.

Advertisement

\protect\hyperlink{after-bottom}{Continue reading the main story}

\hypertarget{site-index}{%
\subsection{Site Index}\label{site-index}}

\hypertarget{site-information-navigation}{%
\subsection{Site Information
Navigation}\label{site-information-navigation}}

\begin{itemize}
\tightlist
\item
  \href{https://help.nytimes3xbfgragh.onion/hc/en-us/articles/115014792127-Copyright-notice}{©~2020~The
  New York Times Company}
\end{itemize}

\begin{itemize}
\tightlist
\item
  \href{https://www.nytco.com/}{NYTCo}
\item
  \href{https://help.nytimes3xbfgragh.onion/hc/en-us/articles/115015385887-Contact-Us}{Contact
  Us}
\item
  \href{https://www.nytco.com/careers/}{Work with us}
\item
  \href{https://nytmediakit.com/}{Advertise}
\item
  \href{http://www.tbrandstudio.com/}{T Brand Studio}
\item
  \href{https://www.nytimes3xbfgragh.onion/privacy/cookie-policy\#how-do-i-manage-trackers}{Your
  Ad Choices}
\item
  \href{https://www.nytimes3xbfgragh.onion/privacy}{Privacy}
\item
  \href{https://help.nytimes3xbfgragh.onion/hc/en-us/articles/115014893428-Terms-of-service}{Terms
  of Service}
\item
  \href{https://help.nytimes3xbfgragh.onion/hc/en-us/articles/115014893968-Terms-of-sale}{Terms
  of Sale}
\item
  \href{https://spiderbites.nytimes3xbfgragh.onion}{Site Map}
\item
  \href{https://help.nytimes3xbfgragh.onion/hc/en-us}{Help}
\item
  \href{https://www.nytimes3xbfgragh.onion/subscription?campaignId=37WXW}{Subscriptions}
\end{itemize}
