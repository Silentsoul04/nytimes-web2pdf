Sections

SEARCH

\protect\hyperlink{site-content}{Skip to
content}\protect\hyperlink{site-index}{Skip to site index}

\href{https://www.nytimes3xbfgragh.onion/section/business}{Business}

\href{https://myaccount.nytimes3xbfgragh.onion/auth/login?response_type=cookie\&client_id=vi}{}

\href{https://www.nytimes3xbfgragh.onion/section/todayspaper}{Today's
Paper}

\href{/section/business}{Business}\textbar{}Black-Owned Businesses Could
Face Hurdles in Federal Aid Program

\url{https://nyti.ms/2wtwAob}

\begin{itemize}
\item
\item
\item
\item
\item
\end{itemize}

\hypertarget{the-coronavirus-outbreak}{%
\subsubsection{\texorpdfstring{\href{https://www.nytimes3xbfgragh.onion/news-event/coronavirus?name=styln-coronavirus-markets\&region=TOP_BANNER\&block=storyline_menu_recirc\&action=click\&pgtype=Article\&impression_id=49b204f0-f28f-11ea-87c3-11143d5ed90a\&variant=undefined}{The
Coronavirus
Outbreak}}{The Coronavirus Outbreak}}\label{the-coronavirus-outbreak}}

\begin{itemize}
\tightlist
\item
  live\href{https://www.nytimes3xbfgragh.onion/2020/09/09/world/covid-19-coronavirus.html?name=styln-coronavirus-markets\&region=TOP_BANNER\&block=storyline_menu_recirc\&action=click\&pgtype=Article\&impression_id=49b204f1-f28f-11ea-87c3-11143d5ed90a\&variant=undefined}{Latest
  Updates}
\item
  \href{https://www.nytimes3xbfgragh.onion/interactive/2020/us/coronavirus-us-cases.html?name=styln-coronavirus-markets\&region=TOP_BANNER\&block=storyline_menu_recirc\&action=click\&pgtype=Article\&impression_id=49b22c00-f28f-11ea-87c3-11143d5ed90a\&variant=undefined}{Maps
  and Cases}
\item
  \href{https://www.nytimes3xbfgragh.onion/interactive/2020/science/coronavirus-vaccine-tracker.html?name=styln-coronavirus-markets\&region=TOP_BANNER\&block=storyline_menu_recirc\&action=click\&pgtype=Article\&impression_id=49b22c01-f28f-11ea-87c3-11143d5ed90a\&variant=undefined}{Vaccine
  Tracker}
\item
  \href{https://www.nytimes3xbfgragh.onion/2020/09/02/your-money/eviction-moratorium-covid.html?name=styln-coronavirus-markets\&region=TOP_BANNER\&block=storyline_menu_recirc\&action=click\&pgtype=Article\&impression_id=49b22c02-f28f-11ea-87c3-11143d5ed90a\&variant=undefined}{Eviction
  Moratorium}
\item
  \href{https://www.nytimes3xbfgragh.onion/interactive/2020/09/02/magazine/food-insecurity-hunger-us.html?name=styln-coronavirus-markets\&region=TOP_BANNER\&block=storyline_menu_recirc\&action=click\&pgtype=Article\&impression_id=49b22c03-f28f-11ea-87c3-11143d5ed90a\&variant=undefined}{American
  Hunger}
\end{itemize}

Advertisement

\protect\hyperlink{after-top}{Continue reading the main story}

Supported by

\protect\hyperlink{after-sponsor}{Continue reading the main story}

\hypertarget{black-owned-businesses-could-face-hurdles-in-federal-aid-program}{%
\section{Black-Owned Businesses Could Face Hurdles in Federal Aid
Program}\label{black-owned-businesses-could-face-hurdles-in-federal-aid-program}}

Minority business owners have always struggled to secure bank loans.
Now, many banks want to deal only with existing customers when making
loans through the government's \$349 billion aid package.

\includegraphics{https://static01.graylady3jvrrxbe.onion/images/2020/04/10/business/10virus-minoritybiz1/10virus-minoritybiz1-articleLarge-v2.jpg?quality=75\&auto=webp\&disable=upscale}

\href{https://www.nytimes3xbfgragh.onion/by/emily-flitter}{\includegraphics{https://static01.graylady3jvrrxbe.onion/images/2019/06/19/reader-center/author-emily-flitter/author-emily-flitter-thumbLarge.png}}

By \href{https://www.nytimes3xbfgragh.onion/by/emily-flitter}{Emily
Flitter}

\begin{itemize}
\item
  Published April 10, 2020Updated June 4, 2020
\item
  \begin{itemize}
  \item
  \item
  \item
  \item
  \item
  \end{itemize}
\end{itemize}

Four years ago, Yasmine Young walked down the street from Diaspora
Salon, the business she owns in Baltimore, to the Bank of America branch
where she has a business checking account. She was thinking of getting a
credit card, but when a banker there described the requirements she
would have to meet to qualify for one, Ms. Young says she left feeling
too discouraged to apply.

She eventually got a card from Capital One, never thinking that the
decision would one day bring her to the brink of disaster.

After she was forced to temporarily close Diaspora because of the
coronavirus pandemic, Ms. Young tried to get an emergency loan under the
federal government's \$349 billion relief program for small businesses.
But Bank of America, one of the biggest banks participating in the
program, refused to consider her application. Because Ms. Young had a
credit card from Capital One, Bank of America said, Capital One was her
primary bank. And Capital One was not yet accepting emergency relief
loan applications. As of Friday, a week after the program was started,
Capital One's website still advised borrowers to keep checking back for
updates and said its online application would be available ``shortly.''

Ms. Young is among the thousands of small-business owners at risk of
being shut out of the government effort, known as the Paycheck
Protection Program, because of limits set by lenders grappling with
overwhelming demand. These loans, which do not have to be repaid if the
money is used for payroll, rent or mortgage expenses, could be a
lifeline for struggling businesses --- if they can get them.

And for small-business owners like Ms. Young, who is black, the hurdles
could be much higher. That's because minority-owned businesses often
have weaker banking relationships than their white-owned counterparts
--- one legacy of the practice of redlining, or refusing to lend to
people in communities of color. Research shows that black and Latino
business owners are denied loans at higher rates.

Anticipating that minority business owners could struggle to tap federal
aid, some lawmakers are proposing ways to earmark additional funds
specifically for minority-owned businesses. And on Wednesday, a group of
prominent black investors, including John W. Rogers Jr., the billionaire
co-chief executive of Ariel Investments, a mutual fund manager, sent a
letter to lawmakers expressing concern that the emergency loan program
was already leaving black borrowers behind.

A 2016 study by economists at the Stanford Institute for Economic Policy
Research found that only 1 percent of black business owners get a bank
loan during their first year of business compared with 7 percent of
white owners. Twice as many white business owners --- 30 percent of the
total --- use business credit cards during their inaugural year,
compared with black owners, among whom only 15 percent rely on a credit
card. Black businesses also start out with far less capital --- whether
from investments or bank loans --- than white businesses, the study
found.

\hypertarget{latest-updates-the-coronavirus-outbreak-and-the-economy}{%
\section{\texorpdfstring{\href{https://www.nytimes3xbfgragh.onion/live/2020/09/08/business/stock-market-today-coronavirus?action=click\&pgtype=Article\&state=default\&region=MAIN_CONTENT_1\&context=storylines_live_updates}{Latest
Updates: The Coronavirus Outbreak and the
Economy}}{Latest Updates: The Coronavirus Outbreak and the Economy}}\label{latest-updates-the-coronavirus-outbreak-and-the-economy}}

\href{https://www.nytimes3xbfgragh.onion/live/2020/09/08/business/stock-market-today-coronavirus?action=click\&pgtype=Article\&state=default\&region=MAIN_CONTENT_1\&context=storylines_live_updates\#the-latest-under-armour-announces-layoffs-and-lubys-will-liquidate}{14h
ago}

\href{https://www.nytimes3xbfgragh.onion/live/2020/09/08/business/stock-market-today-coronavirus?action=click\&pgtype=Article\&state=default\&region=MAIN_CONTENT_1\&context=storylines_live_updates\#the-latest-under-armour-announces-layoffs-and-lubys-will-liquidate}{The
latest: Under Armour announces layoffs, and Luby's will liquidate.}

\href{https://www.nytimes3xbfgragh.onion/live/2020/09/08/business/stock-market-today-coronavirus?action=click\&pgtype=Article\&state=default\&region=MAIN_CONTENT_1\&context=storylines_live_updates\#lululemon-reports-a-quarterly-profit-as-consumers-flock-to-yoga-pants}{14h
ago}

\href{https://www.nytimes3xbfgragh.onion/live/2020/09/08/business/stock-market-today-coronavirus?action=click\&pgtype=Article\&state=default\&region=MAIN_CONTENT_1\&context=storylines_live_updates\#lululemon-reports-a-quarterly-profit-as-consumers-flock-to-yoga-pants}{Lululemon
reports a quarterly profit as consumers flock to yoga pants.}

\href{https://www.nytimes3xbfgragh.onion/live/2020/09/08/business/stock-market-today-coronavirus?action=click\&pgtype=Article\&state=default\&region=MAIN_CONTENT_1\&context=storylines_live_updates\#the-work-from-home-challenge-for-employees-of-color}{16h
ago}

\href{https://www.nytimes3xbfgragh.onion/live/2020/09/08/business/stock-market-today-coronavirus?action=click\&pgtype=Article\&state=default\&region=MAIN_CONTENT_1\&context=storylines_live_updates\#the-work-from-home-challenge-for-employees-of-color}{The
work-from-home challenge for employees of color.}

\href{https://www.nytimes3xbfgragh.onion/live/2020/09/08/business/stock-market-today-coronavirus?action=click\&pgtype=Article\&state=default\&region=MAIN_CONTENT_1\&context=storylines_live_updates}{See
more updates}

More live coverage:
\href{https://www.nytimes3xbfgragh.onion/2020/09/09/world/covid-19-coronavirus.html?action=click\&pgtype=Article\&state=default\&region=MAIN_CONTENT_1\&context=storylines_live_updates}{Global}

``Black-owned businesses continue to rely on family loans to a greater
degree than white-owned firms in the three years following the firm's
founding,'' the researchers found. ``This suggests that access to formal
debt channels remains limited for minorities.''

The institutions that frequently lend to minority-owned businesses,
especially those in low-income neighborhoods, are nonprofit
organizations called
\href{https://www.nytimes3xbfgragh.onion/2020/06/04/business/minority-businesses-damage-lenders.html}{Community
Development Financial Institutions}. They rely on government funding and
charitable donations to make loans, and grew out of earlier efforts to
help African-Americans build wealth in the wake of slavery and
segregation.

However, only 78 of 950 such organizations are participating in the
government program, according to a Treasury spokesman. The rest do not
have authorization to participate because they have not previously been
approved by the Small Business Administration to make loans backed by
the agency. Their representatives say they do not yet have clarity from
the government on how to get that approval quickly enough to participate
in the emergency program.

On Monday, the leaders of the Opportunity Finance Network, a group that
represents the community development organizations, met with Treasury
officials to suggest ways that the government could quickly approve the
lenders for the program. But later,
\href{https://ofn.org/articles/paycheck-protection-program-update-cdfi-eligibility}{in
a statement} posted on its website, the group said that Treasury
officials were ``noncommittal'' about what, if any, actions they would
take in response to the group's recommendations. A Treasury spokesman
did not provide a timeline for when more community organizations might
be approved to participate.

In their letter, the group of black investors, including Ariel's
co-chief executive Mellody Hobson, who serves on the board of JPMorgan
Chase, proposed that a quarter of the \$250 billion in additional money
for small businesses that Congress is considering adding to the program
be set aside for black businesses.

``This roughly \$68 billion will only begin to address the disparities
within capitalism brought into relief by coronavirus,'' they wrote. ``By
prioritizing clients that already have existing lines of credit, black
businesses and nonprofits find themselves yet again excluded from
live-saving relief.''

Some federal lawmakers have proposed that part of the second stimulus
could be directed to local business development organizations that focus
specifically on helping minority businesses so that they can hire
lawyers and offer advice to minority-owned businesses about how to get
help during the pandemic. The \$2 trillion CARES Act allotted \$10
million for minority chambers, but some lawmakers say that is far from
enough.

During a call on Wednesday with Vice President Mike Pence, members of
the Congressional Black Caucus expressed concern that funds from the
program were not reaching black business owners, according to
Representative Emanuel Cleaver II, Democrat of Missouri and a caucus
member who was briefed on the call after it took place. Mr. Pence said
that the program had gotten off to a shaky start in general, Mr. Cleaver
said.

``He did admit that they had had some difficulty,'' Mr. Cleaver said.
``The fear now is that if we don't act quickly, by the time we get the
help for minority businesses by way of having advisers through minority
chambers, the fear is the money will be exhausted.''

A spokeswoman for Mr. Pence did not return a call seeking comment.

Racial discrimination in banking is outlawed on paper, but it continues
in practice --- often in subtle forms. In 2018, for instance, the
National Community Reinvestment Coalition, a nonprofit organization that
works with banks to increase the flow of private capital into poor and
underserved communities, sent ``mystery shoppers'' to 32 different banks
in Los Angeles. It found that potential borrowers with identical
financial profiles were treated differently by bankers based on their
race. Black and Latino borrowers were asked for more detailed financial
documents and were given less information about many banks' available
products than white borrowers.

Some minority business owners have avoided dealing with banks entirely.
When Carlos Swepson, a chef in New York, wanted to start a restaurant,
he borrowed \$240,000 from his parents, who mortgaged their house to
lend him the money. Later, when he wanted to expand, Mr. Swepson raised
\$400,000 from friends, one of whom also took out a mortgage. Another
borrowed against a retirement account to fund him.

\includegraphics{https://static01.graylady3jvrrxbe.onion/images/2020/04/10/business/10virus-minoritybiz2c/merlin_171443931_82222a25-18f9-4aed-9e5c-7a2994a5e104-articleLarge.jpg?quality=75\&auto=webp\&disable=upscale}

``I never had a loan from a bank,'' Mr. Swepson said, adding that he
does not feel comfortable dealing with banks. ``The big ones, I was
getting lost with them,'' Mr. Swepson said. His business checking
account is at Infinity Federal Credit Union, a small credit union based
in New Jersey from which he is also seeking an emergency loan under the
government program. The paperwork, while routine for banking customers,
is complicated and unlike anything Mr. Swepson is accustomed to filling
out. He has sought help from a patron of his restaurant who happens to
be a lawyer.

Ms. Young, too, started Diaspora without any kind of bank loan, going so
far as to learn plumbing techniques from her uncle so that she could
install her own shampoo bowl. Her mother helped her paint and decorate
the space. She built her own website. When she had to hire an assistant,
she researched employment law. Money she earned doing customers' hair
went right back into the salon to pay for things like another styling
station, a janitor and a contracted bookkeeper.

``I just built myself up,'' she said. ``I kept building, building,
building, saving, saving, saving.''

Two years after she gave up trying to get a credit card through Bank of
America in 2016, a friend told her about the Capital One Spark Card,
explaining what terms like ``cash back'' and ``rewards'' meant, and
helped her apply.

(Ms. Young, now 35, had one credit card for three years during college
but had canceled it when her financial situation worsened. Because she
had not taken out any kind of loan since, Ms. Young said she had no
recent credit history.)

``We have received and are processing more than 250,000 applications
from small businesses around the country,'' said Bill Halldin, a Bank of
America spokesman. He said the bank's position was that Ms. Young had a
lending relationship elsewhere and that it was ``unfortunate'' that the
lender had not started taking applications under the program when Bank
of America turned her away.

This week, Ms. Young joined a lawsuit against Bank of America filed by
borrowers whom the bank turned away because they had credit cards from
other banks. By joining the lawsuit, which is seeking class-action
status, she met another Baltimore-based business owner, Amy Elias, who
helped her apply for a government-backed loan through Fundera, online
loan marketplace. She is awaiting the outcome of the application.

Advertisement

\protect\hyperlink{after-bottom}{Continue reading the main story}

\hypertarget{site-index}{%
\subsection{Site Index}\label{site-index}}

\hypertarget{site-information-navigation}{%
\subsection{Site Information
Navigation}\label{site-information-navigation}}

\begin{itemize}
\tightlist
\item
  \href{https://help.nytimes3xbfgragh.onion/hc/en-us/articles/115014792127-Copyright-notice}{©~2020~The
  New York Times Company}
\end{itemize}

\begin{itemize}
\tightlist
\item
  \href{https://www.nytco.com/}{NYTCo}
\item
  \href{https://help.nytimes3xbfgragh.onion/hc/en-us/articles/115015385887-Contact-Us}{Contact
  Us}
\item
  \href{https://www.nytco.com/careers/}{Work with us}
\item
  \href{https://nytmediakit.com/}{Advertise}
\item
  \href{http://www.tbrandstudio.com/}{T Brand Studio}
\item
  \href{https://www.nytimes3xbfgragh.onion/privacy/cookie-policy\#how-do-i-manage-trackers}{Your
  Ad Choices}
\item
  \href{https://www.nytimes3xbfgragh.onion/privacy}{Privacy}
\item
  \href{https://help.nytimes3xbfgragh.onion/hc/en-us/articles/115014893428-Terms-of-service}{Terms
  of Service}
\item
  \href{https://help.nytimes3xbfgragh.onion/hc/en-us/articles/115014893968-Terms-of-sale}{Terms
  of Sale}
\item
  \href{https://spiderbites.nytimes3xbfgragh.onion}{Site Map}
\item
  \href{https://help.nytimes3xbfgragh.onion/hc/en-us}{Help}
\item
  \href{https://www.nytimes3xbfgragh.onion/subscription?campaignId=37WXW}{Subscriptions}
\end{itemize}
