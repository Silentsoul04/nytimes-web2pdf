Sections

SEARCH

\protect\hyperlink{site-content}{Skip to
content}\protect\hyperlink{site-index}{Skip to site index}

\href{https://www.nytimes3xbfgragh.onion/section/politics}{Politics}

\href{https://myaccount.nytimes3xbfgragh.onion/auth/login?response_type=cookie\&client_id=vi}{}

\href{https://www.nytimes3xbfgragh.onion/section/todayspaper}{Today's
Paper}

\href{/section/politics}{Politics}\textbar{}Trump Says States Can Start
Reopening While Acknowledging the Decision Is Theirs

\url{https://nyti.ms/2yhardn}

\begin{itemize}
\item
\item
\item
\item
\item
\end{itemize}

\hypertarget{the-coronavirus-outbreak}{%
\subsubsection{\texorpdfstring{\href{https://www.nytimes3xbfgragh.onion/news-event/coronavirus?name=styln-coronavirus-national\&region=TOP_BANNER\&block=storyline_menu_recirc\&action=click\&pgtype=Article\&impression_id=ef06d1d0-f292-11ea-b230-4985fadaf784\&variant=undefined}{The
Coronavirus
Outbreak}}{The Coronavirus Outbreak}}\label{the-coronavirus-outbreak}}

\begin{itemize}
\tightlist
\item
  live\href{https://www.nytimes3xbfgragh.onion/2020/09/09/world/covid-19-coronavirus.html?name=styln-coronavirus-national\&region=TOP_BANNER\&block=storyline_menu_recirc\&action=click\&pgtype=Article\&impression_id=ef06f8e0-f292-11ea-b230-4985fadaf784\&variant=undefined}{Latest
  Updates}
\item
  \href{https://www.nytimes3xbfgragh.onion/interactive/2020/us/coronavirus-us-cases.html?name=styln-coronavirus-national\&region=TOP_BANNER\&block=storyline_menu_recirc\&action=click\&pgtype=Article\&impression_id=ef06f8e1-f292-11ea-b230-4985fadaf784\&variant=undefined}{Maps
  and Cases}
\item
  \href{https://www.nytimes3xbfgragh.onion/interactive/2020/science/coronavirus-vaccine-tracker.html?name=styln-coronavirus-national\&region=TOP_BANNER\&block=storyline_menu_recirc\&action=click\&pgtype=Article\&impression_id=ef06f8e2-f292-11ea-b230-4985fadaf784\&variant=undefined}{Vaccine
  Tracker}
\item
  \href{https://www.nytimes3xbfgragh.onion/2020/09/02/your-money/eviction-moratorium-covid.html?name=styln-coronavirus-national\&region=TOP_BANNER\&block=storyline_menu_recirc\&action=click\&pgtype=Article\&impression_id=ef06f8e3-f292-11ea-b230-4985fadaf784\&variant=undefined}{Eviction
  Moratorium}
\item
  \href{https://www.nytimes3xbfgragh.onion/interactive/2020/09/02/magazine/food-insecurity-hunger-us.html?name=styln-coronavirus-national\&region=TOP_BANNER\&block=storyline_menu_recirc\&action=click\&pgtype=Article\&impression_id=ef06f8e4-f292-11ea-b230-4985fadaf784\&variant=undefined}{American
  Hunger}
\end{itemize}

Advertisement

\protect\hyperlink{after-top}{Continue reading the main story}

Supported by

\protect\hyperlink{after-sponsor}{Continue reading the main story}

\hypertarget{trump-says-states-can-start-reopening-while-acknowledging-the-decision-is-theirs}{%
\section{Trump Says States Can Start Reopening While Acknowledging the
Decision Is
Theirs}\label{trump-says-states-can-start-reopening-while-acknowledging-the-decision-is-theirs}}

The guidelines released by the president effectively mean that any
restoration of American society will take place on a patchwork basis.

\includegraphics{https://static01.graylady3jvrrxbe.onion/images/2020/04/16/us/politics/16dc-trump-virus/merlin_171650388_e7ca5235-bbee-48d2-b65b-bdc9f5439861-articleLarge.jpg?quality=75\&auto=webp\&disable=upscale}

\href{https://www.nytimes3xbfgragh.onion/by/peter-baker}{\includegraphics{https://static01.graylady3jvrrxbe.onion/images/2018/06/13/multimedia/peter-baker/peter-baker-thumbLarge-v2.png}}\href{https://www.nytimes3xbfgragh.onion/by/michael-d-shear}{\includegraphics{https://static01.graylady3jvrrxbe.onion/images/2018/06/13/multimedia/author-michael-d-shear/author-michael-d-shear-thumbLarge-v2.png}}

By \href{https://www.nytimes3xbfgragh.onion/by/peter-baker}{Peter Baker}
and \href{https://www.nytimes3xbfgragh.onion/by/michael-d-shear}{Michael
D. Shear}

\begin{itemize}
\item
  April 16, 2020
\item
  \begin{itemize}
  \item
  \item
  \item
  \item
  \item
  \end{itemize}
\end{itemize}

WASHINGTON --- President Trump told the nation's governors on Thursday
that they could begin reopening businesses, restaurants and other
elements of daily life by May 1 or earlier if they wanted to, but
abandoned his threat to use what he had claimed was his absolute
authority to impose his will on them.

On a day when the nation's
\href{https://www.nytimes3xbfgragh.onion/interactive/2020/world/coronavirus-maps.html}{death
toll from the coronavirus increased} by more than 2,000 for a total over
30,000, the president released a set of nonbinding guidelines that
envisioned a slow return to work and school over weeks or months. Based
on each state's conditions, the guidelines in effect guarantee that any
restoration of American society will take place on a patchwork basis
rather than on a one-size-fits-all prescription from Washington that
some of the governors had feared in recent days.

``We are not opening all at once, but one careful step at a time,'' Mr.
Trump told reporters during a briefing at the White House.

Mr. Trump essentially gave cover to mainly Republican governors of
states in the South and West that have not been as hard hit by the
pandemic to begin reopening sooner. The president, who has previously
said that as many as 29 states could reopen soon, told governors on a
conference call before his announcement that some of them were ``in
very, very good shape'' and could move further and faster to resuming
economic and social activities.

If they follow the guidelines, New York and other states in the
Northeast, as well as states in the Midwest and West, that have seen
large outbreaks would remain shuttered for weeks until new cases of the
virus and death tolls fall and hospital capacity is restored.

The guidelines envision proceeding without the comprehensive testing
program that many public health experts have sought and opened the
president to criticism that in his eagerness to start rebuilding a
cratered economy, he may have encouraged some states to move too quickly
and leave themselves exposed to a second wave of the coronavirus.

Speaker Nancy Pelosi dismissed the guidelines even as she pushed for
more testing. ``The White House's vague and inconsistent document does
nothing to make up for the president's failure to listen to the
scientists and produce and distribute national rapid testing,'' she said
in a statement.

The 18-page document released by the White House provided mostly general
guidance and did not confront some difficult questions, including how to
finance the billions of dollars necessary for expanded testing; whether
travel should be restricted between states; when the ban on
international travel from Europe and elsewhere would be lifted; and how
the states should deal with future shortages of protective equipment if
the virus resurged in the fall.

The guidelines assume the ability to quickly contain future outbreaks by
quarantining sick people and their contacts, but they provide no
specifics about how strained public health systems around the country
will achieve that goal.

``You're going to call your own shots,'' Mr. Trump told the governors,
according to an audio recording provided to The New York Times. ``You're
going to be calling the shots. We'll be standing right alongside of you,
and we're going to get our country open and get it working. People want
to get working.''

\hypertarget{latest-updates-the-coronavirus-outbreak}{%
\section{\texorpdfstring{\href{https://www.nytimes3xbfgragh.onion/2020/09/09/world/covid-19-coronavirus.html?action=click\&pgtype=Article\&state=default\&region=MAIN_CONTENT_1\&context=storylines_live_updates}{Latest
Updates: The Coronavirus
Outbreak}}{Latest Updates: The Coronavirus Outbreak}}\label{latest-updates-the-coronavirus-outbreak}}

Updated 2020-09-09T11:49:12.203Z

\begin{itemize}
\tightlist
\item
  \href{https://www.nytimes3xbfgragh.onion/2020/09/09/world/covid-19-coronavirus.html?action=click\&pgtype=Article\&state=default\&region=MAIN_CONTENT_1\&context=storylines_live_updates\#link-70cea8bb}{As
  drugmakers pledge to thoroughly vet a vaccine, one company pauses its
  trials for a safety review.}
\item
  \href{https://www.nytimes3xbfgragh.onion/2020/09/09/world/covid-19-coronavirus.html?action=click\&pgtype=Article\&state=default\&region=MAIN_CONTENT_1\&context=storylines_live_updates\#link-780eaa2f}{Britain
  is expected to ban gatherings of more than six people.}
\item
  \href{https://www.nytimes3xbfgragh.onion/2020/09/09/world/covid-19-coronavirus.html?action=click\&pgtype=Article\&state=default\&region=MAIN_CONTENT_1\&context=storylines_live_updates\#link-11cec4c0}{Quarantine
  breakdowns at colleges in the U.S. are leaving some at risk.}
\end{itemize}

\href{https://www.nytimes3xbfgragh.onion/2020/09/09/world/covid-19-coronavirus.html?action=click\&pgtype=Article\&state=default\&region=MAIN_CONTENT_1\&context=storylines_live_updates}{See
more updates}

More live coverage:
\href{https://www.nytimes3xbfgragh.onion/live/2020/09/09/business/stock-market-today-coronavirus?action=click\&pgtype=Article\&state=default\&region=MAIN_CONTENT_1\&context=storylines_live_updates}{Markets}

At the evening briefing, the president conceded that the choice of how
and when to reopen the country would not be his. ``If they need to
remain closed,'' he said, ``we will allow them to do that.''

Mr. Trump's choice of words amounted to a significant reversal only
three days after he insisted that ``the president of the United States
calls the shots'' and that he had the ``total'' authority to decide how
and when the country would end widespread lockdowns. Several governors
rebelled at the notion, defying Mr. Trump's assertion of unilateral
power and declaring that they would come to their own conclusions.

The president said a little more than three weeks ago that he wanted to
reopen the country by Easter, April 12, then changed the date to May 1
before declaring that when to do it would be ``the biggest decision I've
ever had to make.''

He has repeatedly lurched from one position to another as his
administration has struggled to confront what he calls an ``invisible
enemy.''

For weeks, he played down the threat from the coronavirus, predicting it
would ``miraculously'' disappear in warm weather. As the number of cases
overwhelmed some hospitals, Mr. Trump blamed governors for failing to
prepare, even as he claimed credit for federal help that was slow to
arrive.

The federal guidelines, which recommend phased reopenings depending on
case levels and hospital capacity, came as governors were already
setting their own courses.

Gov. Andrew M. Cuomo of New York announced that the state's sweeping
shutdown would last until at least May 15, while Gov. Mike DeWine of
Ohio said he planned to begin lifting restrictions on public activities
starting May 1. Gov. Tony Evers of Wisconsin said residents must stay at
home until May 26, and in Missouri, Kansas City and St. Louis County
both extended similar orders.

A bipartisan group of governors from the Midwest that included Mr.
DeWine and Mr. Evers announced the formation of a regional coalition to
weigh next steps, which the governors said would be ``fact-based'' and
``data-driven.'' Other coalition members include Gov. J.B. Pritzker of
Illinois, Gov. Gretchen Whitmer of Michigan, Gov. Tim Walz of Minnesota,
Gov. Eric Holcomb of Indiana and Gov. Andy Beshear of Kentucky.

States elsewhere in the country with fewer cases and smaller, more rural
and more distant populations may take their cue from Mr. Trump and begin
moving to lift restrictions.

The fitful movement toward reopening came as another 5.2 million
Americans filed for unemployment benefits, bringing the total number of
people put out of work in the past four weeks to
\href{https://www.nytimes3xbfgragh.onion/2020/04/16/business/economy/unemployment-numbers-coronavirus.html}{a
staggering 22 million}. Facing the worst economic crisis since the Great
Depression only six months before an election, Mr. Trump has felt
enormous pressure to get business restarted and put Americans back to
work.

A federal loan program intended to help small businesses keep workers on
their payrolls has proved woefully insufficient. The administration said
Thursday that the Paycheck Protection Program had
\href{https://www.nytimes3xbfgragh.onion/2020/04/15/us/politics/coronavirus-small-business-program.html}{run
out of money}, leaving millions of businesses unable to apply for the
loans while Congress struggled to reach a deal to replenish the funds.

The guidelines released by the president --- titled, ``Opening Up
America Again'' --- urge states not to lift stay-at-home or travel
restrictions until they reach a 14-day period in which the number of
coronavirus cases is steadily declining, hospitals are not overwhelmed
and robust testing is in place for health care workers and others.

``The dominating drive of this was to make sure this is done in the
safest way possible,'' said Dr. Anthony S. Fauci, the director of the
National Institute of Allergy and Infectious Diseases and a member of
the president's coronavirus task force, who spoke at the White House
briefing alongside Mr. Trump.

In states judged to be doing well enough to enter the first phase,
schools would remain closed and people would still be urged to avoid
socializing in groups of more than 10. But some large public places ---
including restaurants, movie theaters, sporting venues and places of
worship --- would be allowed to operate under strict physical distancing
protocols. Elective surgeries could resume and gyms could reopen as long
as they maintained physical distancing. Bars would remain closed.

In the second phase, which could begin after another two-week decline in
the number of coronavirus cases, schools could reopen and people would
be advised to avoid social gatherings of more than 50.

By the third phase, states with no evidence of a resurgence of
infections would be able to resume unrestricted staffing of work sites,
visits to hospitals and nursing homes, and the operation of large venues
under limited social distancing protocols. Bars could reopen with
increased standing room.

In addition to the guidelines, the Centers for Disease Control and
Prevention are expected to soon announce that the agency will hire
hundreds of people to
\href{https://www.nytimes3xbfgragh.onion/2020/04/16/us/coronavirus-massachusetts-contact-tracing.html}{perform
contact tracing} as part of the push to allow the country to go back to
work and school, according to a federal official.

\href{https://www.nytimes3xbfgragh.onion/news-event/coronavirus?action=click\&pgtype=Article\&state=default\&region=MAIN_CONTENT_3\&context=storylines_faq}{}

\hypertarget{the-coronavirus-outbreak-}{%
\subsubsection{The Coronavirus Outbreak
›}\label{the-coronavirus-outbreak-}}

\hypertarget{frequently-asked-questions}{%
\paragraph{Frequently Asked
Questions}\label{frequently-asked-questions}}

Updated September 4, 2020

\begin{itemize}
\item ~
  \hypertarget{what-are-the-symptoms-of-coronavirus}{%
  \paragraph{What are the symptoms of
  coronavirus?}\label{what-are-the-symptoms-of-coronavirus}}

  \begin{itemize}
  \tightlist
  \item
    In the beginning, the coronavirus
    \href{https://www.nytimes3xbfgragh.onion/article/coronavirus-facts-history.html?action=click\&pgtype=Article\&state=default\&region=MAIN_CONTENT_3\&context=storylines_faq\#link-6817bab5}{seemed
    like it was primarily a respiratory illness}~--- many patients had
    fever and chills, were weak and tired, and coughed a lot, though
    some people don't show many symptoms at all. Those who seemed
    sickest had pneumonia or acute respiratory distress syndrome and
    received supplemental oxygen. By now, doctors have identified many
    more symptoms and syndromes. In April,
    \href{https://www.nytimes3xbfgragh.onion/2020/04/27/health/coronavirus-symptoms-cdc.html?action=click\&pgtype=Article\&state=default\&region=MAIN_CONTENT_3\&context=storylines_faq}{the
    C.D.C. added to the list of early signs}~sore throat, fever, chills
    and muscle aches. Gastrointestinal upset, such as diarrhea and
    nausea, has also been observed. Another telltale sign of infection
    may be a sudden, profound diminution of one's
    \href{https://www.nytimes3xbfgragh.onion/2020/03/22/health/coronavirus-symptoms-smell-taste.html?action=click\&pgtype=Article\&state=default\&region=MAIN_CONTENT_3\&context=storylines_faq}{sense
    of smell and taste.}~Teenagers and young adults in some cases have
    developed painful red and purple lesions on their fingers and toes
    --- nicknamed ``Covid toe'' --- but few other serious symptoms.
  \end{itemize}
\item ~
  \hypertarget{why-is-it-safer-to-spend-time-together-outside}{%
  \paragraph{Why is it safer to spend time together
  outside?}\label{why-is-it-safer-to-spend-time-together-outside}}

  \begin{itemize}
  \tightlist
  \item
    \href{https://www.nytimes3xbfgragh.onion/2020/05/15/us/coronavirus-what-to-do-outside.html?action=click\&pgtype=Article\&state=default\&region=MAIN_CONTENT_3\&context=storylines_faq}{Outdoor
    gatherings}~lower risk because wind disperses viral droplets, and
    sunlight can kill some of the virus. Open spaces prevent the virus
    from building up in concentrated amounts and being inhaled, which
    can happen when infected people exhale in a confined space for long
    stretches of time, said Dr. Julian W. Tang, a virologist at the
    University of Leicester.
  \end{itemize}
\item ~
  \hypertarget{why-does-standing-six-feet-away-from-others-help}{%
  \paragraph{Why does standing six feet away from others
  help?}\label{why-does-standing-six-feet-away-from-others-help}}

  \begin{itemize}
  \tightlist
  \item
    The coronavirus spreads primarily through droplets from your mouth
    and nose, especially when you cough or sneeze. The C.D.C., one of
    the organizations using that measure,
    \href{https://www.nytimes3xbfgragh.onion/2020/04/14/health/coronavirus-six-feet.html?action=click\&pgtype=Article\&state=default\&region=MAIN_CONTENT_3\&context=storylines_faq}{bases
    its recommendation of six feet}~on the idea that most large droplets
    that people expel when they cough or sneeze will fall to the ground
    within six feet. But six feet has never been a magic number that
    guarantees complete protection. Sneezes, for instance, can launch
    droplets a lot farther than six feet,
    \href{https://jamanetwork.com/journals/jama/fullarticle/2763852}{according
    to a recent study}. It's a rule of thumb: You should be safest
    standing six feet apart outside, especially when it's windy. But
    keep a mask on at all times, even when you think you're far enough
    apart.
  \end{itemize}
\item ~
  \hypertarget{i-have-antibodies-am-i-now-immune}{%
  \paragraph{I have antibodies. Am I now
  immune?}\label{i-have-antibodies-am-i-now-immune}}

  \begin{itemize}
  \tightlist
  \item
    As of right
    now,\href{https://www.nytimes3xbfgragh.onion/2020/07/22/health/covid-antibodies-herd-immunity.html?action=click\&pgtype=Article\&state=default\&region=MAIN_CONTENT_3\&context=storylines_faq}{~that
    seems likely, for at least several months.}~There have been
    frightening accounts of people suffering what seems to be a second
    bout of Covid-19. But experts say these patients may have a
    drawn-out course of infection, with the virus taking a slow toll
    weeks to months after initial exposure.~People infected with the
    coronavirus typically
    \href{https://www.nature.com/articles/s41586-020-2456-9}{produce}~immune
    molecules called antibodies, which are
    \href{https://www.nytimes3xbfgragh.onion/2020/05/07/health/coronavirus-antibody-prevalence.html?action=click\&pgtype=Article\&state=default\&region=MAIN_CONTENT_3\&context=storylines_faq}{protective
    proteins made in response to an
    infection}\href{https://www.nytimes3xbfgragh.onion/2020/05/07/health/coronavirus-antibody-prevalence.html?action=click\&pgtype=Article\&state=default\&region=MAIN_CONTENT_3\&context=storylines_faq}{.
    These antibodies may}~last in the body
    \href{https://www.nature.com/articles/s41591-020-0965-6}{only two to
    three months}, which may seem worrisome, but that's~perfectly normal
    after an acute infection subsides, said Dr. Michael Mina, an
    immunologist at Harvard University. It may be possible to get the
    coronavirus again, but it's highly unlikely that it would be
    possible in a short window of time from initial infection or make
    people sicker the second time.
  \end{itemize}
\item ~
  \hypertarget{what-are-my-rights-if-i-am-worried-about-going-back-to-work}{%
  \paragraph{What are my rights if I am worried about going back to
  work?}\label{what-are-my-rights-if-i-am-worried-about-going-back-to-work}}

  \begin{itemize}
  \tightlist
  \item
    Employers have to provide
    \href{https://www.osha.gov/SLTC/covid-19/standards.html}{a safe
    workplace}~with policies that protect everyone equally.
    \href{https://www.nytimes3xbfgragh.onion/article/coronavirus-money-unemployment.html?action=click\&pgtype=Article\&state=default\&region=MAIN_CONTENT_3\&context=storylines_faq}{And
    if one of your co-workers tests positive for the coronavirus, the
    C.D.C.}~has said that
    \href{https://www.cdc.gov/coronavirus/2019-ncov/community/guidance-business-response.html}{employers
    should tell their employees}~-\/- without giving you the sick
    employee's name -\/- that they may have been exposed to the virus.
  \end{itemize}
\end{itemize}

Under the plan, the official said that the federal government would also
help states pay for more medical personnel to help track the spread of
the coronavirus by contacting people who test positive to see who they
had contact with three or four days before they started showing
symptoms. ``If we see a hot spot developing, we've learned a lot,'' Mr.
Trump said. ``We'll be able to suppress it, whack it.''

Many public health experts have cautioned that hiring several hundred
people for the entire country will be nowhere near enough to keep track
of the virus as it spreads. Dr. Thomas R. Frieden, a former C.D.C.
director, said there were estimates that the country would need to hire
as many as 300,000 such workers.

The federal guidelines outline much the same strategy that a number of
local and state governments have already adopted in anticipation of the
day when social restrictions are eased, interviews with health officials
in a half-dozen states show.

Several governors had expressed concern that Mr. Trump would try to
pressure the states to reopen too quickly and had made clear that they
were not going to bend to the president's will if he continued to insist
he could order them to end restrictions.

Gov. Jay Inslee of Washington State, which was the first hit hard by the
virus, said on Thursday that while his state has been one of the most
successful in flattening the curve, the number of cases has plateaued
without going down, meaning the danger is not over.

Although the president cannot impose his will through fiat, Mr. Inslee
said before the call that Mr. Trump's public comments could be dangerous
if he encouraged the public to think the crisis would end prematurely.
``If he repeats the error he made at the beginning of this, it could be
equally fatal,'' Mr. Inslee said in an interview. ``We lost a month
because of his failure to recognize the seriousness of this.''

Other governors said they would continue with their own plans to reopen
their states in gradual phases, often in coordination with other states
in their regions.

``We have a plan to start opening Ohio back up, ''
\href{https://twitter.com/GovMikeDeWine/status/1250861666578960389?s=20}{Mr.
DeWine said on Twitter}. ``It's going to be gradual- one thing after
another. We want to do this in a thoughtful way that engenders
confidence and ensures customers and employees are safe.''

Gov. Ron DeSantis of Florida, who had said on Wednesday that he would
create a task force to make recommendations on how to reopen
restaurants, events, businesses and schools, added that he might also
issue specific guidelines for South Florida, the hardest-hit part of the
state.

``I think that we're going to be able to come up with a thoughtful
approach'' to restart the economy, said Mr. DeSantis, a Republican.

Gov. Gavin Newsom of California praised Mr. Trump for recognizing
different circumstances across the country.

``I do want to extend a broad-strokes appreciation for what I heard from
the president as it relates to recognizing the differentiation as it
exists and persists in terms of conditions in counties, not just in
states, across this nation,'' Mr. Newsom said.

Mr. Trump sounded upbeat during his meeting with the governors and
acknowledged no troubles, despite continuing problems experienced by
many of those on the call. ``We're in excellent shape on testing,'' the
president said, hailing the newer version of a test for the coronavirus
that involves a saliva test and complaining in graphic detail about the
earlier version that was conducted on him to rule out any infection.

``I was a victim of the first test, meaning I had to go through it, and
I didn't like what was happening,'' he said. ``They tell you that it
goes up your nose and then they hang a right and it goes under your eye,
and I said you got to be kidding. I called it an operation not a test.''

After days of toggling back and forth between conflict and conciliation
with the governors, he went out of his way on Thursday's call to heap
praise on them.

``You're very capable people,'' he said. ``I think in all cases very
capable people.''

Reporting was contributed by Nicholas Fandos, Jonathan Martin and Sharon
LaFraniere from Washington, Sheila Kaplan from New York, Patricia Mazzei
from Miami, Thomas Fuller from San Francisco and Julie Bosman from
Chicago.

Advertisement

\protect\hyperlink{after-bottom}{Continue reading the main story}

\hypertarget{site-index}{%
\subsection{Site Index}\label{site-index}}

\hypertarget{site-information-navigation}{%
\subsection{Site Information
Navigation}\label{site-information-navigation}}

\begin{itemize}
\tightlist
\item
  \href{https://help.nytimes3xbfgragh.onion/hc/en-us/articles/115014792127-Copyright-notice}{©~2020~The
  New York Times Company}
\end{itemize}

\begin{itemize}
\tightlist
\item
  \href{https://www.nytco.com/}{NYTCo}
\item
  \href{https://help.nytimes3xbfgragh.onion/hc/en-us/articles/115015385887-Contact-Us}{Contact
  Us}
\item
  \href{https://www.nytco.com/careers/}{Work with us}
\item
  \href{https://nytmediakit.com/}{Advertise}
\item
  \href{http://www.tbrandstudio.com/}{T Brand Studio}
\item
  \href{https://www.nytimes3xbfgragh.onion/privacy/cookie-policy\#how-do-i-manage-trackers}{Your
  Ad Choices}
\item
  \href{https://www.nytimes3xbfgragh.onion/privacy}{Privacy}
\item
  \href{https://help.nytimes3xbfgragh.onion/hc/en-us/articles/115014893428-Terms-of-service}{Terms
  of Service}
\item
  \href{https://help.nytimes3xbfgragh.onion/hc/en-us/articles/115014893968-Terms-of-sale}{Terms
  of Sale}
\item
  \href{https://spiderbites.nytimes3xbfgragh.onion}{Site Map}
\item
  \href{https://help.nytimes3xbfgragh.onion/hc/en-us}{Help}
\item
  \href{https://www.nytimes3xbfgragh.onion/subscription?campaignId=37WXW}{Subscriptions}
\end{itemize}
