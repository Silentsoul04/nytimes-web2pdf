Sections

SEARCH

\protect\hyperlink{site-content}{Skip to
content}\protect\hyperlink{site-index}{Skip to site index}

\href{https://www.nytimes3xbfgragh.onion/section/business}{Business}

\href{https://myaccount.nytimes3xbfgragh.onion/auth/login?response_type=cookie\&client_id=vi}{}

\href{https://www.nytimes3xbfgragh.onion/section/todayspaper}{Today's
Paper}

\href{/section/business}{Business}\textbar{}Land O'Lakes Removes Native
American Woman From Its Products

\url{https://nyti.ms/2yrcJGJ}

\begin{itemize}
\item
\item
\item
\item
\item
\end{itemize}

Advertisement

\protect\hyperlink{after-top}{Continue reading the main story}

Supported by

\protect\hyperlink{after-sponsor}{Continue reading the main story}

\hypertarget{land-olakes-removes-native-american-woman-from-its-products}{%
\section{Land O'Lakes Removes Native American Woman From Its
Products}\label{land-olakes-removes-native-american-woman-from-its-products}}

The company, a farmer-owned cooperative formed in 1921, said it would
replace the decades-old illustration with photos of its members on some
products.

\includegraphics{https://static01.graylady3jvrrxbe.onion/images/2020/04/18/us/17ALTxp-landolakes-print/17xp-landolake-image3-articleLarge.jpg?quality=75\&auto=webp\&disable=upscale}

\href{https://www.nytimes3xbfgragh.onion/by/christine-hauser}{\includegraphics{https://static01.graylady3jvrrxbe.onion/images/2018/02/16/multimedia/author-christine-hauser/author-christine-hauser-thumbLarge.jpg}}

By
\href{https://www.nytimes3xbfgragh.onion/by/christine-hauser}{Christine
Hauser}

\begin{itemize}
\item
  Published April 17, 2020Updated April 20, 2020
\item
  \begin{itemize}
  \item
  \item
  \item
  \item
  \item
  \end{itemize}
\end{itemize}

For nearly a century, an illustration of a Native American woman with a
feather in her hair has adorned the packaging of Land O'Lakes cheese and
butter products. But not for much longer.

The company, founded in 1921 by a group of Minnesota dairy farmers, is
phasing in a new design ahead of its 100th anniversary. Instead of the
depiction of the woman, some products will be labeled ``Farmer-Owned''
and feature an illustration of a field and lake, or photographs of its
farmers, the company announced.

The new design, which started appearing on tubs of butter spread, food
service products and deli cheese in February, is now being used on
packages of stick butter and will be ``fully rolled out'' by the end of
the year, the company has said.

This week, officials and Native American representatives applauded the
change, which is similar to steps that other U.S. companies, sports
teams and universities have undertaken to address or phase out the use
of Native American imagery in logos and mascots.

\includegraphics{https://static01.graylady3jvrrxbe.onion/images/2020/04/17/us/17xp-landolakes-image/17xp-landolakes-image-articleLarge.jpg?quality=75\&auto=webp\&disable=upscale}

Kevin Allis, the chief executive of the National Congress of American
Indians, a public education and advocacy group, said the organization
saw it as a ``positive sign,'' adding, ``We encourage all companies that
peddle products displaying stereotypical Native `themed' imagery to
follow suit.''

``Americans need to learn the truth about the beauty and diversity of
tribal nations, peoples and cultures today,'' he said, ``and discarding
antiquated symbols like this are a step in the right direction.''

Lt. Gov. Peggy Flanagan of Minnesota, who is a citizen of the
\href{https://whiteearth.com/home}{White Earth Nation} of Ojibwe,
thanked the company for the ``important and needed change.''

``Native people are not mascots or logos,'' Ms. Flanagan
\href{https://twitter.com/LtGovFlanagan/status/1250515283313414145}{said
on Twitter} on Wednesday, linking to a
\href{https://minnesotareformer.com/2020/04/15/land-olakes-quietly-gets-rid-of-iconic-indian-maiden/}{Minnesota
Reformer report} about the new packaging. ``We are very much still
here.''

It was not immediately clear why Land O'Lakes, which is based in Arden
Hills, Minn., decided this year to remove an image that has adorned its
products for nearly 100 years, nor did the company make any reference in
a
\href{https://www.landolakesinc.com/Press/News/new-butter-and-dairy-packaging}{February
statement announcing the change} to the implications of such depictions
of Native Americans. A company spokeswoman did not reply to emailed
questions on Friday.

Beth Ford, the Land O'Lakes chief executive, said in the statement that
as the company looked ahead to its centennial, it recognized the need
for ``packaging that reflects the foundation and heart of our company
culture.''

``Nothing does that better than our farmer-owners whose milk is used to
produce Land O'Lakes' dairy products,'' she said.

The American Psychological Association has
\href{https://www.apa.org/about/policy/mascots.pdf}{recommended} the
immediate retirement of Native American mascots and symbols, in part
because they appear ``to have a negative impact on the self-esteem of
American Indian children.''

Some said it was demeaning cultural appropriation to use the image of
the woman, who has been depicted as kneeling in the original design and
has been used as the butt of vulgar social media jokes.

The imagery of the Land O'Lakes woman has been reflected in Native
American art. David Bradley, a sculptor and a Minnesota Chippewa, named
a piece ``Land O'Fakes'' in a 2005 show that confronted fraud in the
market and ``the commodification of Indian culture --- the packaging of
it in an attractive way to make money,''
\href{https://www.nytimes3xbfgragh.onion/2005/08/20/arts/design/a-new-dawn-for-museums-of-native-american-art.html}{as
the artist put it}.

The original logo of the company's ``butter maiden'' first appeared on
Land O'Lakes packaging in 1928, created by Arthur C. Hanson, an artist
who worked for a local advertising firm.

It was redone about 30 years later by Patrick DesJarlait, a
\href{https://www.nytimes3xbfgragh.onion/1978/02/26/archives/long-island-weekly-art-tradition-presented-with-action-and.html}{Chippewa
artist who died in 1972}. As a Native American, he was a rarity in the
illustration business at the time, his son Robert DesJarlait said in an
interview on Friday.

``He redid her features and the dress she is wearing, a Plains-style
dress with beaded panels,'' said Mr. DesJarlait, a member of the Red
Lake Ojibwe tribe in Minnesota. ``He added floral designs for the
Chippewa culture. It was basically a redesign. He gave her a clearer
image. So he was modernizing her a bit.''

Though some groups have held up the illustration as an example of
stereotyping, Mr. DesJarlait said, he didn't see it that way. ``She was
never created as a stereotype,'' he said.

Mr. DesJarlait said he believed that the company did not discard the
imagery to get rid of a stereotype, but out of discomfort with
representations of Native Americans of any kind.

``She just disappeared,'' he said.

Advertisement

\protect\hyperlink{after-bottom}{Continue reading the main story}

\hypertarget{site-index}{%
\subsection{Site Index}\label{site-index}}

\hypertarget{site-information-navigation}{%
\subsection{Site Information
Navigation}\label{site-information-navigation}}

\begin{itemize}
\tightlist
\item
  \href{https://help.nytimes3xbfgragh.onion/hc/en-us/articles/115014792127-Copyright-notice}{©~2020~The
  New York Times Company}
\end{itemize}

\begin{itemize}
\tightlist
\item
  \href{https://www.nytco.com/}{NYTCo}
\item
  \href{https://help.nytimes3xbfgragh.onion/hc/en-us/articles/115015385887-Contact-Us}{Contact
  Us}
\item
  \href{https://www.nytco.com/careers/}{Work with us}
\item
  \href{https://nytmediakit.com/}{Advertise}
\item
  \href{http://www.tbrandstudio.com/}{T Brand Studio}
\item
  \href{https://www.nytimes3xbfgragh.onion/privacy/cookie-policy\#how-do-i-manage-trackers}{Your
  Ad Choices}
\item
  \href{https://www.nytimes3xbfgragh.onion/privacy}{Privacy}
\item
  \href{https://help.nytimes3xbfgragh.onion/hc/en-us/articles/115014893428-Terms-of-service}{Terms
  of Service}
\item
  \href{https://help.nytimes3xbfgragh.onion/hc/en-us/articles/115014893968-Terms-of-sale}{Terms
  of Sale}
\item
  \href{https://spiderbites.nytimes3xbfgragh.onion}{Site Map}
\item
  \href{https://help.nytimes3xbfgragh.onion/hc/en-us}{Help}
\item
  \href{https://www.nytimes3xbfgragh.onion/subscription?campaignId=37WXW}{Subscriptions}
\end{itemize}
