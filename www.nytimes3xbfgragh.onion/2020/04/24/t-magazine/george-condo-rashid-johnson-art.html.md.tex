Sections

SEARCH

\protect\hyperlink{site-content}{Skip to
content}\protect\hyperlink{site-index}{Skip to site index}

\href{https://myaccount.nytimes3xbfgragh.onion/auth/login?response_type=cookie\&client_id=vi}{}

\href{https://www.nytimes3xbfgragh.onion/section/todayspaper}{Today's
Paper}

Two Exhibitions Respond to Art in the Age of Anxiety and Distance

\url{https://nyti.ms/2VxXlBJ}

\begin{itemize}
\item
\item
\item
\item
\item
\end{itemize}

Advertisement

\protect\hyperlink{after-top}{Continue reading the main story}

Supported by

\protect\hyperlink{after-sponsor}{Continue reading the main story}

\hypertarget{two-exhibitions-respond-to-art-in-the-age-of-anxiety-and-distance}{%
\section{Two Exhibitions Respond to Art in the Age of Anxiety and
Distance}\label{two-exhibitions-respond-to-art-in-the-age-of-anxiety-and-distance}}

George Condo and Rashid Johnson talk about their new simultaneous
digital shows, politics and how the role of artists has changed.

\includegraphics{https://static01.graylady3jvrrxbe.onion/images/2020/04/24/t-magazine/21tmag-condojohnson-slide-4VDP/21tmag-condojohnson-slide-4VDP-articleLarge.jpg?quality=75\&auto=webp\&disable=upscale}

By \href{https://www.nytimes3xbfgragh.onion/by/m-h-miller}{M.H. Miller}

\begin{itemize}
\item
  Published April 24, 2020Updated April 28, 2020
\item
  \begin{itemize}
  \item
  \item
  \item
  \item
  \item
  \end{itemize}
\end{itemize}

The artists
\href{https://www.nytimes3xbfgragh.onion/2016/11/18/t-magazine/art/george-condo-picasso-matisse-museum-berggruen.html}{George
Condo} and
\href{https://www.nytimes3xbfgragh.onion/2016/09/08/t-magazine/art/rashid-johnson-hauser-wirth-antoine-baldwin.html}{Rashid
Johnson} are old friends who both debuted new work in simultaneous
online exhibitions for Hauser \& Wirth gallery that opened last week.
Condo, 62, first emerged in the early 1980s as a figurative painter
whose portraits walked a line between realism and sardonic
conceptualism. His new show,
``\href{https://www.vip-hauserwirth.com/george-condo-drawings-for-distanced-figures/}{Drawings
for Distanced Figures},'' explores how changes in perspective distort
his subjects. Johnson, 42, first gained attention in the early 2000s for
photographs, sculptures and paintings that explored identity in
21st-century America. His show,
``\href{https://www.vip-hauserwirth.com/rashid-johnson-untitled-anxious-red-drawings/}{Untitled
Anxious Red Drawings},'' revisits his ``Anxious Men'' figures ---
square-jawed faces with wide eyes and chattering teeth --- which have
become a staple of his work. (A portion of proceeds from both shows will
benefit the
\href{https://www.who.int/emergencies/diseases/novel-coronavirus-2019/donate\#:~:text=The\%20Covid\%2D19\%20Solidarity,needs\%20are\%20the\%20greatest.}{World
Health Organization's Covid-19 Solidarity Response Fund}.) Like
everyone, both artists have had to adapt to a new set of circumstances
in the last few months that has made them see their art --- not to
mention their respective creative processes --- differently. This
interview, which has been condensed and edited for clarity, took place
via Google Hangouts (Condo and Johnson were both at ``undisclosed
locations on Long Island,'' as Condo put it). The artists discussed
producing art while sheltering in place, the virtues of online
exhibitions and how they think the Covid-19 pandemic will change their
profession.

Image

George CondoCredit...Antonio Santos for The New York Times

\textbf{Can you tell me about this new work you've both been making?}

\textbf{Rashid Johnson:} These are characters I started making around
five years ago, the
``\href{https://www.nytimes3xbfgragh.onion/interactive/2015/11/01/arts/design/01rashid-johnson-anxious-men.html}{Anxious
Men}.'' The conversation around those works then was that they must be
about police violence and brutality in communities of color. Even though
they were quite abstracted and intended as merely a representation of
the human condition and human concerns --- the idea of mental anguish
and anxiety and stress --- the assumption, because of who the author of
the work was, was that it was really determined and focused on the
position of a certain community, of the black community and what was
happening there. And I always kind of resisted that position. I always
said that I never gendered or made race the prominent concern of those
characters. But the assumption was that was how they were. As those
works evolved, and
\href{https://www.nytimes3xbfgragh.onion/interactive/2018/upshot/election-2016-voting-precinct-maps.html}{as
Trump was elected}, I feel like more people felt as if they were
graduated into this group of anxious characters. It became Democrats,
liberals, progressives, all seeing themselves as part of this
collective, this group that I had conjured. And now, with
\href{https://www.nytimes3xbfgragh.onion/news-event/coronavirus}{coronavirus}
and with the transition of those characters from where they began, as
stark black characters against sharp white backgrounds, into these red
characters, these kind of urgent characters, I think people are starting
to understand that the intention of this work was to encapsulate
everybody. We are at a time in the world where I think we can actually
say we are seeing everyone struggle. Everyone is subject to a certain
condition. Obviously, there are levels to this. Some people are subject
to it and are put in incredibly difficult and problematic positions
\href{https://www.nytimes3xbfgragh.onion/2020/03/15/world/europe/coronavirus-inequality.html}{because
of} financial and social agency, because of the actual effect of the
virus on them and their families.

\includegraphics{https://static01.graylady3jvrrxbe.onion/images/2020/04/24/t-magazine/21tmag-condojohnson-slide-QUSN/21tmag-condojohnson-slide-QUSN-articleLarge.jpg?quality=75\&auto=webp\&disable=upscale}

\textbf{George Condo:} Right before the end of the summer, I started
working on these black paintings. One of them was called ``Random
Consequence of Varying and Opposing Perspectives.'' It was basically
about the idea of marginalization --- of people being pushed to the edge
of the canvas to the point where they were almost dematerialized by the
consequences of random perspectives. And the random perspectives were,
you know, between Fox News and CNN and MSNBC and what is real and what
is not real. I created the term ``artificial realism'' back in the late
'80s. That idea about representing reality, but reality being a
construct of man-made appearances. I felt like artificial realism took
over when Trump was elected, not as an artistic discussion but a
political discussion. It was sort of like creating a formula for
disaster. Then the
\href{https://www.nytimes3xbfgragh.onion/2017/12/12/world/europe/trump-fake-news-dictators.html}{fake-news
concept} came about and everything was about fakes, and my whole thing
in the early '80s was, ``Oh I paint \emph{fake} masterpieces,'' so I
didn't have to think about whether they were or they weren't. I could
sort of objectify everything. Coming through that lens into these black
paintings, which were about my figurative works being pushed to the
extreme, to the point where the dehumanization factor was taking place
throughout a spectrum --- of racial injustices and social
discriminations at varying levels. Whether it's
\href{https://www.nytimes3xbfgragh.onion/topic/person/mitch-mcconnell}{Mitch
McConnell} or
\href{https://www.nytimes3xbfgragh.onion/topic/person/lindsey-graham}{Lindsey
Graham} or
\href{https://www.nytimes3xbfgragh.onion/2018/09/26/us/politics/brett-kavanaugh-accusers-women.html}{Brett
Kavanaugh}, you know all the things that we lived through, nothing could
compare to what was gonna come, which is this microbiological warfare
that's been created somehow --- in my opinion, to get Donald Trump
re-elected. Vote for me or you die. That's the message I'm getting from
coronavirus.

As to what I'm doing as an artist, I'm just exploring the psychological
impact of those kinds of thoughts in my work right now, and how fear,
anxiety, panic --- how do you put that into some kind of poetic language
that maintains your identity and integrity as an artist?

Image

Rashid JohnsonCredit...Gioncarlo Valentine for The New York Times

\textbf{RJ:} When I saw George's drawings, what made me so excited is
that great artists pivot. They don't invent. You can't work to a new
scene out of the blue. This is in the spirit of what George has done
over the last 30-plus years. And he was able to just pivot the work to
consume where we are. I think that's what a great artist is capable of
doing. The work is like Pac-Man. It just kind of eats everything that
comes in its way and says, ``Hey, this is part of it now, too,'' rather
than just responding and being reactionary.

\textbf{GC:} I think what we're saying is that when we started these
works, it seemed as though they started from one perspective, but now
we've grown \emph{into} that perspective, and it's become us. Once you
create something solid, you can break it apart, smash it into pieces and
rebuild. The idea of deconstruction was a 20th-century idea, but this is
a time of reconstruction as opposed to deconstruction. There's gonna be
a lot of reconstruction after this Covid-19 situation, and where this
seismic shift is going to take culture will be very interesting.
Pre-Covid, post-Covid, and I really wonder the way it's gonna be
perceived.

\textbf{RJ:} It's a total fault-line situation.

Image

Condo's ``Symbiotic Fear'' (2020).Credit...© George Condo, courtesy of
the artist and Hauser \& Wirth

\textbf{Have either of you thought at all about how what you do as
artists will be different after we emerge from isolation?}

\textbf{RJ:} I've got a question for George, actually, on that front.
George and I are friends, and one of the best walk-throughs I did of one
of my shows was George coming in and saying, ``I love the work, but this
one should have been in the front room.'' And he was totally right.
Artists always have to think about the architecture of space in terms of
how a work is consumed over the course of an exhibition. So my question
is:
\href{https://www.nytimes3xbfgragh.onion/2020/03/16/arts/design/art-galleries-online-viewing-coronavirus.html}{How
the hell do you do an online exhibition}? What is the architecture, and
what does this mean for us now?

\textbf{GC:} What I did with Hauser \& Wirth, I had help from my buddy
Peter, who's been out here staying with me. We've been out here for five
weeks. We haven't been able to see our kids or anything. I'm too
high-risk to see my kids. They're young, they could be asymptomatic. He
basically walked around the room with a camera while I was in the act of
making my drawings. To have somebody see you in the act of making
something, and then to see that flat on the screen, you can say that
thing was made by hand. I didn't have to shut down my factory for it to
be made --- I don't have one. I can remember several years ago going
over to
\href{https://www.nytimes3xbfgragh.onion/interactive/2017/11/29/t-magazine/jay-z-dean-baquet-interview.html}{Jay-Z}and
\href{https://www.nytimes3xbfgragh.onion/2014/06/03/t-magazine/beyonce-the-woman-on-top-of-the-world.html}{Beyoncé}'s
place in Bridgehampton, and she was saying, ``All I want is a guitar and
a good song. I don't need all that studio \emph{{[}expletive{]}}.'' The
idea of a singer like Beyoncé just picking up a guitar and strumming
something, well you can do that. You can't have the studio right now.
You can't have your mixing board. It's almost like we're back to reel to
reel, where someone just pushes a button and says what's happening
today, like a folk singer. I think the idea of things being homemade,
like the way
\href{https://www.nytimes3xbfgragh.onion/article/easy-recipes-coronavirus.html}{we
have to cook every night}, the way life has become about breakfast,
lunch and dinner and how to make yourself happy with your own two hands,
I think that's where we're headed. I'm hoping that, post-Covid, people
don't forget that. We all work so hard to be able to get to a point
where we could travel and take our family on vacation and go to a nice
hotel, but now you can't. So you have to make all that happiness in a
small space, the way people used to. During the Depression, a loaf of
bread would be the greatest thing that could happen to a family. I hope
people don't just go back to being money-grabbing and horrible. What do
you think is gonna happen?

Image

Johnson's ``Untitled Anxious Red Drawing'' (2020).Credit...© Rashid
Johnson, courtesy of the artist and Hauser \& Wirth

Image

Johnson's ``Untitled Anxious Red Drawing'' (2020).Credit...© Rashid
Johnson, courtesy of the artist and Hauser \& Wirth

\textbf{RJ:} I think there are two themes. One is simplicity. And I
think you captured that: All you really need is this and this, simple
ingredients. The other thing is being humbled. I think people have been
really humbled by this. Because it doesn't matter how much money you
have or how much agency you are accustomed to having, you're still home,
too. Some people's homes are nicer than others, and we know there are
people who are able to stay home whereas other people aren't. But we're
all restricted in some way. And that's humbling, for people who believed
that they could do whatever they wanted, that's just not the case
anymore. Because you are responsible to other humans. I think taking
that feeling of being humbled and applying it to the way that you behave
in the world after this is gonna be ideally the most significant
outcome. People realized that we're not masters of the universe. Nature
can come into your space at any time and it can sit you on your ass and
change what you are capable of doing. I think it will lead to a
re-examination of how and why we do things. As far as art is concerned,
people are going to be asking the existential questions: Does the work
establish a position in which we see an honest investigation of the
human condition?

\textbf{GC:} There are always layers of madness. Like post-9/11, the
entire world was cheering for these American heroes who were giving
their lives to dig people out of the rubble and it didn't matter what
race, what color you were, everyone was in it together. And
\href{https://www.nytimes3xbfgragh.onion/topic/person/george-w-bush}{George
W. Bush}'s way of thanking everybody was to start World War Nine, you
know? After this is over, I just hope that our optimistic concept of how
the world is going to be more together, more simplified, more humbled,
more compassionate is not going to be destroyed by that same kind of
retaliation.

\textbf{RJ:} Yeah, let's hope not. We definitely see people who are in
more disadvantaged positions suffering more. This has pointed a real
spotlight on wealth disparity, and on
\href{https://www.nytimes3xbfgragh.onion/interactive/2020/04/03/us/coronavirus-stay-home-rich-poor.html}{who
gets to be safe}. It's just nuts. People need to recognize that --- just
look out your window and see who's out there right now, see who's trying
to save us --- that there are people who are taking more risks and are
more exposed.

Image

Condo's ``Molecular Figures'' (2020)Credit...© George Condo, courtesy of
the artist and Hauser \& Wirth

\textbf{What do you think a more humble version of the art world looks
like? I can't think of a single moment in which I've thought that the
art world is humble.}

\textbf{RJ:} When I say ``humble,'' I mean in some ways a sense of
self-exploration, where you imagine yourself in the narrative and you
make art with the idea of the self in mind, as present in the work.
Maybe it means making less work \emph{toward} other people and more work
about substantive concerns. Showing a real investment in your relation
with and engagement to the world. Where the artist says, ``OK, I am a
part of the world, and the artwork is me, and it is all-consuming,'' as
opposed to being able to make art as if it's business as usual, where
you have a strategy and you in effect go to the office and produce a
thing with a team of people. That is probably not going to be the answer
that we need, right? In my thinking, that's not what we need from art
right now.

\textbf{GC:} You can never predict what's going to happen in the future,
but you can look back at what happened in the past sort of innocently.
Like how did
\href{https://www.nytimes3xbfgragh.onion/topic/person/pablo-picasso}{Picasso}
react to the Second World War? When the Nazis walked into his studio in
Paris and saw a skull on the table and asked him, ``Did you make that?''
he said, ``No, you did.''

\textbf{How does the experience of opening an online show compare to
your usual experience of opening a show?}

\textbf{RJ:} How we imagine what space is has changed. Just from this
exhibition I did with Hauser, thinking about digital space, like what is
that? I'm not even ready and equipped to navigate that yet. I don't know
what the hell is possible in there.

\textbf{GC:} I think it's so much better. You don't have to talk to
people and shake hands and get in selfies with people hanging around.
All that stuff was great for me in my 20s. But for artists right now, an
online exhibition saves us the whole hassle of walking into the opening
and everyone talking to you before you even have a chance to look at
your own installation. I actually kind of hate that stuff. I don't know
if it's just me.

\textbf{RJ:} Same. I hate openings, man. I've always hated openings. I
do miss the spaces. I miss installing the show and looking at the show
during the install, but I don't miss the opening at all. I went to my
opening in my pajamas. An exhibition challenges sight lines and how you
see things, but these online exhibitions are more like viewing rooms.
The viewing room is this back room at a gallery where they hang up,
like, one work, and you come into the room and it's clean and small and
you get to just sit there with the work. And that's one of my favorite
spaces at a gallery, and it's one where the public is not generally
welcome. But it's this really autonomous singular moment with a lone
artwork. A one-on-one experience.

\textbf{GC:} Artists like
\href{https://www.nytimes3xbfgragh.onion/2020/02/26/t-magazine/haegue-yang.html}{isolation}.

Advertisement

\protect\hyperlink{after-bottom}{Continue reading the main story}

\hypertarget{site-index}{%
\subsection{Site Index}\label{site-index}}

\hypertarget{site-information-navigation}{%
\subsection{Site Information
Navigation}\label{site-information-navigation}}

\begin{itemize}
\tightlist
\item
  \href{https://help.nytimes3xbfgragh.onion/hc/en-us/articles/115014792127-Copyright-notice}{©~2020~The
  New York Times Company}
\end{itemize}

\begin{itemize}
\tightlist
\item
  \href{https://www.nytco.com/}{NYTCo}
\item
  \href{https://help.nytimes3xbfgragh.onion/hc/en-us/articles/115015385887-Contact-Us}{Contact
  Us}
\item
  \href{https://www.nytco.com/careers/}{Work with us}
\item
  \href{https://nytmediakit.com/}{Advertise}
\item
  \href{http://www.tbrandstudio.com/}{T Brand Studio}
\item
  \href{https://www.nytimes3xbfgragh.onion/privacy/cookie-policy\#how-do-i-manage-trackers}{Your
  Ad Choices}
\item
  \href{https://www.nytimes3xbfgragh.onion/privacy}{Privacy}
\item
  \href{https://help.nytimes3xbfgragh.onion/hc/en-us/articles/115014893428-Terms-of-service}{Terms
  of Service}
\item
  \href{https://help.nytimes3xbfgragh.onion/hc/en-us/articles/115014893968-Terms-of-sale}{Terms
  of Sale}
\item
  \href{https://spiderbites.nytimes3xbfgragh.onion}{Site Map}
\item
  \href{https://help.nytimes3xbfgragh.onion/hc/en-us}{Help}
\item
  \href{https://www.nytimes3xbfgragh.onion/subscription?campaignId=37WXW}{Subscriptions}
\end{itemize}
