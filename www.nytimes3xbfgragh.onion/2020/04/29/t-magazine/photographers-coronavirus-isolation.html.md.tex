Sections

SEARCH

\protect\hyperlink{site-content}{Skip to
content}\protect\hyperlink{site-index}{Skip to site index}

\href{https://myaccount.nytimes3xbfgragh.onion/auth/login?response_type=cookie\&client_id=vi}{}

\href{https://www.nytimes3xbfgragh.onion/section/todayspaper}{Today's
Paper}

Eight Photographers' Pictures From Isolation

\url{https://nyti.ms/3aI8oww}

\begin{itemize}
\item
\item
\item
\item
\item
\end{itemize}

\hypertarget{the-coronavirus-outbreak}{%
\subsubsection{\texorpdfstring{\href{https://www.nytimes3xbfgragh.onion/news-event/coronavirus?name=styln-coronavirus-national\&region=TOP_BANNER\&block=storyline_menu_recirc\&action=click\&pgtype=Article\&impression_id=f7604420-f278-11ea-9a9f-018759a6103d\&variant=undefined}{The
Coronavirus
Outbreak}}{The Coronavirus Outbreak}}\label{the-coronavirus-outbreak}}

\begin{itemize}
\tightlist
\item
  live\href{https://www.nytimes3xbfgragh.onion/2020/09/08/world/covid-19-coronavirus.html?name=styln-coronavirus-national\&region=TOP_BANNER\&block=storyline_menu_recirc\&action=click\&pgtype=Article\&impression_id=f7604421-f278-11ea-9a9f-018759a6103d\&variant=undefined}{Latest
  Updates}
\item
  \href{https://www.nytimes3xbfgragh.onion/interactive/2020/us/coronavirus-us-cases.html?name=styln-coronavirus-national\&region=TOP_BANNER\&block=storyline_menu_recirc\&action=click\&pgtype=Article\&impression_id=f7604422-f278-11ea-9a9f-018759a6103d\&variant=undefined}{Maps
  and Cases}
\item
  \href{https://www.nytimes3xbfgragh.onion/interactive/2020/science/coronavirus-vaccine-tracker.html?name=styln-coronavirus-national\&region=TOP_BANNER\&block=storyline_menu_recirc\&action=click\&pgtype=Article\&impression_id=f7604423-f278-11ea-9a9f-018759a6103d\&variant=undefined}{Vaccine
  Tracker}
\item
  \href{https://www.nytimes3xbfgragh.onion/2020/09/02/your-money/eviction-moratorium-covid.html?name=styln-coronavirus-national\&region=TOP_BANNER\&block=storyline_menu_recirc\&action=click\&pgtype=Article\&impression_id=f7604424-f278-11ea-9a9f-018759a6103d\&variant=undefined}{Eviction
  Moratorium}
\item
  \href{https://www.nytimes3xbfgragh.onion/interactive/2020/09/02/magazine/food-insecurity-hunger-us.html?name=styln-coronavirus-national\&region=TOP_BANNER\&block=storyline_menu_recirc\&action=click\&pgtype=Article\&impression_id=f7606b30-f278-11ea-9a9f-018759a6103d\&variant=undefined}{American
  Hunger}
\end{itemize}

Advertisement

\protect\hyperlink{after-top}{Continue reading the main story}

Supported by

\protect\hyperlink{after-sponsor}{Continue reading the main story}

\hypertarget{eight-photographers-pictures-from-isolation}{%
\section{Eight Photographers' Pictures From
Isolation}\label{eight-photographers-pictures-from-isolation}}

Joel Meyerowitz, Renée Cox, Asako Narahashi and more share visual
diaries of the present moment.

\includegraphics{https://static01.graylady3jvrrxbe.onion/images/2020/04/17/t-magazine/art/17tmag-photog-project-slide-JNQA/17tmag-photog-project-slide-JNQA-articleLarge.jpg?quality=75\&auto=webp\&disable=upscale}

By Meara Sharma

\begin{itemize}
\item
  April 29, 2020
\item
  \begin{itemize}
  \item
  \item
  \item
  \item
  \item
  \end{itemize}
\end{itemize}

``Like a high-strung racehorse who needs extra weight in her saddle pad,
I like a handicap and relish the aesthetic challenge posed by the
limitations of the ordinary,'' writes the photographer Sally Mann in her
memoir,
``\href{https://www.hachettebookgroup.com/titles/sally-mann/hold-still/9780316247757/}{Hold
Still}'' (2015). In our stilled, stalled time, her words ring especially
true. Here we all are, burdened by untold fears, forced to make do, to
essentialize, to improvise. And also, within all of this, to open our
eyes and attend to new possibilities.

Of course, attention is the linchpin of image-making, and so T asked a
number of photographers, many of whom typically derive inspiration from
the wider world, how they are approaching this newfound intimacy with
the ordinary, and to share what they have invented within it. Some
relayed mystical encounters with nature and the animal world: Domingo
Milella discovered ancient symbols on the rugged outskirts of Bari,
Italy;
\href{http://www.richardmosse.com/projects/incoming\#home}{Richard
Mosse} communed with the craggy topography of the Burren landscape in
Ireland; \href{http://03fotos.com/asako-narahashi/}{Asako Narahashi}, in
Japan, found solace alongside a rescued cat. On the Caribbean island of
St. Kitts, \href{http://waynelawrenceonline.com/}{Wayne Lawrence}
embraced proximity to family and the lush surroundings, while in wintry
Minnesota, \href{https://alecsoth.com/photography/}{Alec Soth} gave in
to distance by chronicling his neighborhood through a pair of
binoculars, capturing the feeling of being at once near and far,
sheltered and susceptible.

Others have found ongoing projects imbued with fresh relevance. For his
series ``Chance and Necessity,'' Hitoshi Fugo captured the drama and
beauty of everyday mishaps in his Tokyo kitchen. On the beaches of Long
Island, \href{https://www.reneecox.org/}{Renée Cox} considered the
multiplicity of the self in this moment of collective inwardness. So,
too, did \href{https://www.joelmeyerowitz.com/}{Joel Meyerowitz}, who
began a daily ritual of self-portraiture at the start of the year, and
for whom the act of facing oneself honestly is a kind of celebration.

Indeed, the wide-ranging images here acknowledge but don't limit
themselves to melancholy; rather, they hold intrigue, affirmation and
even delight, reminding us that, as Meyerowitz says, photography is a
hopeful art form, an act of ``saying yes,'' of staying awake to the
world --- which, as the pandemic continues to push us into retreat, is
as vital a task as ever.

\emph{Quotes have been edited and condensed.}

\begin{center}\rule{0.5\linewidth}{\linethickness}\end{center}

\hypertarget{joel-meyerowitz}{%
\subsubsection{Joel Meyerowitz}\label{joel-meyerowitz}}

Pictures taken in London

\includegraphics{https://static01.graylady3jvrrxbe.onion/images/2020/04/17/t-magazine/art/17tmag-photog-project-slide-XKNK/17tmag-photog-project-slide-XKNK-articleLarge.jpg?quality=75\&auto=webp\&disable=upscale}

On New Year's Day, I thought, ``What could I do this year that would be
a challenge?'' I'd never really done self-portraits, so I decided to
take one a day, every day, for a year. Most of the time, if you do a
self-portrait, you take one or two or 10. And very often people try to
make themselves look as attractive as can be. But I want to leave my ego
at the door and see myself at the age that I am, looking the way that I
look --- I realize I'm not the dashing street photographer I once was.
And I'm trying to add risky components so that the work remains chancey
and even provocative. I want the picture to be somewhat out of control
and clumsy or rude or careless.

Image

``January 14, 2020.''Credit...© Joel Meyerowitz, courtesy of Howard
Greenberg Gallery

The Leicas I use have a 12-second timer, so you can set up the frame and
then live in it doing what you're doing: chopping vegetables, dining
with someone, polishing your shoes, who knows. It's as if I have a
live-in personal photographer doing a story on my life, but here the
camera is capturing all the moments that someone else couldn't, like
when I'm in the shower, when I'm taking a bath with my wife. I find a
lot of the pictures I make are of me looking. Because that's what my
life has been. In one photograph, I was lying on the floor doing my
stretches, and I looked up and saw a plaster circle on the ceiling. And
I thought, ``Suppose the camera was where I am now, looking at me
looking up at that circle.''

Image

``March 3, 2020.''Credit...© Joel Meyerowitz, courtesy of Howard
Greenberg Gallery

When I started the project, I could work in crowds. But now, my cast of
characters is me, myself, and I \ldots{} and my wife, Maggie. So that's
four of us. And the question is how to keep it interesting, especially
in a small apartment. One of the pictures I made on the street while we
were taking a walk. I passed someone's house and a flower was on the
little pedestal at the entranceway. I walked over and my shadow fell
over it, and the flower was where my heart is.

Image

``January 21, 2020.''Credit...© Joel Meyerowitz, courtesy of Howard
Greenberg Gallery

I'm 82 years old, and I'm in that vulnerable group. Plus, I have a
compromised respiratory system because after 9/11, I spent nine months
taking pictures at ground zero. Many painters start painting the four
seasons, or decaying fruit and flowers, in their later years. I'm
looking at themes of life, aging and death here --- \emph{this is what
you look like now, this is what you're stuck with} --- but trying to
have fun within the limitations. I'm willing to be the fool, in a sense,
in my own story line. Whatever comes to mind, I'll try it. It's actually
very optimistic. I've always felt that photography is a positive art
form. Every time I press a button on the camera, I'm saying yes to
something I saw, something that woke me up. Even in the face of
disasters and plagues, one has to look for the positive qualities that
give us the energy to continue.

Image

``March 31, 2020.''Credit...© Joel Meyerowitz, courtesy of Howard
Greenberg Gallery

Image

``April 3, 2020.''Credit...© Joel Meyerowitz, courtesy of Howard
Greenberg Gallery

\begin{center}\rule{0.5\linewidth}{\linethickness}\end{center}

\hypertarget{alec-soth}{%
\subsubsection{\texorpdfstring{\textbf{Alec
Soth}}{Alec Soth}}\label{alec-soth}}

\textbf{Pictures taken in Minneapolis, Minn.}

Image

``Untitled.''Credit...Alec Soth

I'm at home with two kids and my wife and many animals: two dogs, three
cats, an iguana and a hamster. I couldn't pick up a real camera to take
pictures, because that felt too much like being a real photographer. I
didn't want to give it that sense of authority. I was just kind of
overwhelmed, and I'm not a photographer who runs toward crisis. But I
had this memory of using binoculars on a safari a few years ago. I found
that looking through them renders space really beautifully --- it makes
faraway things close but in a peculiar way. On a whim, I put my iPhone
up to the binoculars and started taking pictures. It's clumsy and really
hard to do --- the power of binoculars is not as strong as most
telephoto lenses on a camera --- but I kind of enjoyed the game of it.

Image

``Untitled.''Credit...Alec Soth

So I pulled out those same binoculars and drove around in the bubble of
my minivan looking for signs of life. Nowadays, it feels like everything
is seen through panes of glass. Binoculars have multiple layers of
glass, and I shot these pictures through the added layer on the iPhone,
as well as through the car windows. So, distance, distance, distance.

Image

``Untitled.''Credit...Alec Soth

Image

``Untitled.''Credit...Alec Soth

The picture of the house with the arched window has this lens
aberration. I like how the bubble's colors are similar to those of the
fabric hanging in the window. It's one of those beautiful accidents. The
very stalker-ish photo of the guy in the window was hard to do
technically, because of the binoculars and the low light and the need
for the guy to stay still.

Image

``Untitled.''Credit...Alec Soth

My photography has always been about social distance, in a way. Social
awkwardness, social distance, all of those things. I've always thought
about this in terms of the lens, that this piece of glass is separating
me and protecting me in some ways from the world. The thing I'm trying
to process now relates to the larger ethical meaning of being a
photographer. I'm always conflicted about using people as fodder for my
artistic pursuits. And this idea of traveling great distances, driving
all over, using gas, flying places, and spreading things --- is that
really the best way to be in the world? That's partly why I admire
photographers who make work at home and teach us how to be observant of
our own lives. What will it mean for me to be an ethical photographer in
whatever new world comes out of this?

\hypertarget{latest-updates-the-coronavirus-outbreak}{%
\section{\texorpdfstring{\href{https://www.nytimes3xbfgragh.onion/2020/09/08/world/covid-19-coronavirus.html?action=click\&pgtype=Article\&state=default\&region=MAIN_CONTENT_1\&context=storylines_live_updates}{Latest
Updates: The Coronavirus
Outbreak}}{Latest Updates: The Coronavirus Outbreak}}\label{latest-updates-the-coronavirus-outbreak}}

Updated 2020-09-09T08:22:37.235Z

\begin{itemize}
\tightlist
\item
  \href{https://www.nytimes3xbfgragh.onion/2020/09/08/world/covid-19-coronavirus.html?action=click\&pgtype=Article\&state=default\&region=MAIN_CONTENT_1\&context=storylines_live_updates\#link-313b443d}{AstraZeneca
  halts a vaccine trial to investigate a participant's illness.}
\item
  \href{https://www.nytimes3xbfgragh.onion/2020/09/08/world/covid-19-coronavirus.html?action=click\&pgtype=Article\&state=default\&region=MAIN_CONTENT_1\&context=storylines_live_updates\#link-4438dd7}{Facing
  a surge in cases, Britain plans to limit most gatherings to six
  people.}
\item
  \href{https://www.nytimes3xbfgragh.onion/2020/09/08/world/covid-19-coronavirus.html?action=click\&pgtype=Article\&state=default\&region=MAIN_CONTENT_1\&context=storylines_live_updates\#link-679303d7}{Nine
  drugmakers pledge to thoroughly vet any coronavirus vaccine.}
\end{itemize}

\href{https://www.nytimes3xbfgragh.onion/2020/09/08/world/covid-19-coronavirus.html?action=click\&pgtype=Article\&state=default\&region=MAIN_CONTENT_1\&context=storylines_live_updates}{See
more updates}

More live coverage:
\href{https://www.nytimes3xbfgragh.onion/live/2020/09/08/business/stock-market-today-coronavirus?action=click\&pgtype=Article\&state=default\&region=MAIN_CONTENT_1\&context=storylines_live_updates}{Markets}

\begin{center}\rule{0.5\linewidth}{\linethickness}\end{center}

\hypertarget{renuxe9e-cox}{%
\subsubsection{Renée Cox}\label{renuxe9e-cox}}

Pictures taken in and around Amagansett, N.Y.

Image

Day 10: ``Distorted Self-Portrait.''Credit...Renée Cox

My house on Long Island is in the woods, and I'm seeing vitality: giant
wild turkeys in abundance, deer in herds who just don't care. Nature
takes back very quickly.

Image

Day 1: ''Me Myself \& I.''Credit...Renée Cox

With these photos, I was following a self-portraiture assignment I gave
to my students. I think this is a time when you can really explore
yourself, because you're not going to be setting up shoots; you're not
going to be working with other people. And that's what I've been doing.
Being an artist has been my solace; it provides me with a certain degree
of therapy. I'm in my house with my family, but I can still get in my
car and drive to the lighthouse, take photographs on the beach. I might
see one other person.

Image

Day 5: ``Shadmoor Tall Grass.''Credit...Renée Cox

Image

Day 6: ``Lines in Bath.''Credit...Renée Cox

``Self-Portrait Rocks'' was me in Montauk alone. It's a composite, a
meditation. I was feeling almost buried, asking myself, ``OK, what's
next? ** How long are we here?'' ``Me, Myself \& I'' was initially shot
in the city, and then I manipulated it after coming out here. In my
studio, I've placed two mirrors at a 70-degree angle --- an old photo
carnival trick. The image is a séance of sorts that draws on spirit
photography from the late 1800s. I was thinking about the idea of
solitude but also the question: Are we ever really alone? I hear people
saying, ``I'm bored, I'm bored,'' and I think, ``It's because you're
bored with yourself.'' You've got to learn to love yourself, to have a
conversation with yourself.

Image

Day 11: ``Self-Portrait Rocks.''Credit...Renée Cox

Image

Day 25: ``Covid-19 Neon Self-Portrait.''Credit...Renée Cox

``Covid-19 Neon Self-Portrait'' is another composite. Once you start
going down this path of looking within, then hopefully some light gets
shed. I'm trying to be grounded in the present moment. As we get further
into the pandemic, that becomes a part of the message: Deal with this
now, as opposed to thinking about when it's going to be over, or when
you can get back to what you were doing before. For all the control
freaks of the world, it must be pure hell, but I think the lesson to be
learned is: Wait a second, maybe give up that control. And what kind of
control are you seeking anyway?

\begin{center}\rule{0.5\linewidth}{\linethickness}\end{center}

\hypertarget{domingo-milella}{%
\subsubsection{\texorpdfstring{\textbf{Domingo
Milella}}{Domingo Milella}}\label{domingo-milella}}

\textbf{Pictures taken in Bari, Italy}

Image

``Untitled.''Credit...Domingo Milella

This crisis feels like an archetypal call from the forest, the animals,
the place we inhabit. It is a very deeply ecological crisis, and I don't
think we can separate the virus from what we've been doing in the
industrial age. If we keep talking about stock markets, I think we will
have failed to understand. The words ``economic,'' ``ecological'' and
``ecosystemic'' all lead back to the Greek word \emph{oikonomos}, and
the idea of home, the rules of the house. Perhaps, while confined, we
can rethink how the earth is also our home. And it's not a private home
but a collective one.

Image

``Untitled.''Credit...Domingo Milella

The virus, too, is that rare thing that puts us all in front of a
similar event. I think this is very powerful; there is something
religious about it, even. There are days I'm desperate and want to cry,
and others when I see the possibilities for harmony and for beauty and
art. I usually work with my big black box camera and camera obscura and
do not share images taken on my phone, as these were. But the moment of
crisis made me feel as though I had permission to focus on quotidian
elements. I've been in this suburban tower, which was built by my uncle
and is where I grew up, for 40 days already. My whole family is here, on
different levels --- I'm on the 13th floor, my parents are on the 12th,
my cousin is on the 11th. Down below is a Pasolini-looking countryside,
with views of project housing, the Adriatic Sea, villas from the 1700s,
bunkers from World War II, olive trees, a railway and some junk from
previous buildings.

Image

``Untitled.''Credit...Domingo Milella

The pictures I made are about being confined but also about being
resourceful in the way of a child, about looking at things with purity
and listening to the wind. Now, we have to re-challenge our imagination.
I took some pictures of a fig tree and some leaves. One day, on my way
to the tree, I met a very big black snake. It so happens that the night
before, I had taken out a skin of a different snake that I had in a
little box in my apartment. I put the skin next to a little puff of
white feather that had fallen off one of the peregrine falcons flying
by. I told my friend about all of this, and they sent me a picture from
the Middle Ages of Adam, Eve and a fig tree with a big snake wrapped
around it. Sometimes, when you stop and take time to meet a snake or
follow the flight of a falcon, something magical starts to happen.

Image

``Untitled.''Credit...Domingo Milella

Image

``Untitled.''Credit...Domingo Milella

\begin{center}\rule{0.5\linewidth}{\linethickness}\end{center}

\hypertarget{hitoshi-fugo}{%
\subsubsection{\texorpdfstring{\textbf{Hitoshi
Fugo}}{Hitoshi Fugo}}\label{hitoshi-fugo}}

\textbf{Pictures taken in Tokyo}

Image

``Chance and Necessity 09.''Credit...© Hitoshi Fugo, courtesy of Miyako
Yoshinaga gallery

Image

``Chance and Necessity 10.''Credit...© Hitoshi Fugo, courtesy of Miyako
Yoshinaga gallery

This series, ``Chance and Necessity,'' came out of two incidents I had
in the kitchen of my home in Tokyo about a year ago. I attempted to
toast leftover dumpling dough wrappers to transform them into croutons
for salad. However, I kept them in my toaster too long and ended up
burning them thoroughly. Then, I caught my breath at the radiant, black
charred wrappers. Another time, I dropped a bunch of very thin dry somen
noodles, and they scattered on the kitchen floor. They created a
stunning, unexpected image of overlapping straight lines. These moments
led me to start a new photographic series, which includes these pictures
I made in the last couple of weeks, during the quarantine.

Image

``Chance and Necessity 11.''Credit...© Hitoshi Fugo, courtesy of Miyako
Yoshinaga gallery

Image

``Chance and Necessity 13.''Credit...© Hitoshi Fugo, courtesy of Miyako
Yoshinaga gallery

This theme of chance and necessity is something I have long been
interested in, but the timeliness now is miraculous. It's about using
what you have. What I'm creating is a consequence of the collaboration
between chance and the photographer. The artwork is not the black
charred wrappers or scattered noodles themselves; it is the photographic
images that emerge.

Image

``Chance and Necessity 14.''Credit...© Hitoshi Fugo, courtesy of Miyako
Yoshinaga gallery

Image

``Chance and Necessity 12.''Credit...© Hitoshi Fugo, courtesy of Miyako
Yoshinaga gallery

Now, we are experiencing fear of something not visible. We last had that
during the Fukushima nuclear disaster nine years ago. We didn't know
what it would be like the next day, or the next week, and we were also
scared by something we could not see. People tend to think that
photographers take pictures of things that you can see, but I'm now
realizing that it might be more important to express the opposite, what
is \emph{not} seeable.

\href{https://www.nytimes3xbfgragh.onion/news-event/coronavirus?action=click\&pgtype=Article\&state=default\&region=MAIN_CONTENT_3\&context=storylines_faq}{}

\hypertarget{the-coronavirus-outbreak-}{%
\subsubsection{The Coronavirus Outbreak
›}\label{the-coronavirus-outbreak-}}

\hypertarget{frequently-asked-questions}{%
\paragraph{Frequently Asked
Questions}\label{frequently-asked-questions}}

Updated September 4, 2020

\begin{itemize}
\item ~
  \hypertarget{what-are-the-symptoms-of-coronavirus}{%
  \paragraph{What are the symptoms of
  coronavirus?}\label{what-are-the-symptoms-of-coronavirus}}

  \begin{itemize}
  \tightlist
  \item
    In the beginning, the coronavirus
    \href{https://www.nytimes3xbfgragh.onion/article/coronavirus-facts-history.html?action=click\&pgtype=Article\&state=default\&region=MAIN_CONTENT_3\&context=storylines_faq\#link-6817bab5}{seemed
    like it was primarily a respiratory illness}~--- many patients had
    fever and chills, were weak and tired, and coughed a lot, though
    some people don't show many symptoms at all. Those who seemed
    sickest had pneumonia or acute respiratory distress syndrome and
    received supplemental oxygen. By now, doctors have identified many
    more symptoms and syndromes. In April,
    \href{https://www.nytimes3xbfgragh.onion/2020/04/27/health/coronavirus-symptoms-cdc.html?action=click\&pgtype=Article\&state=default\&region=MAIN_CONTENT_3\&context=storylines_faq}{the
    C.D.C. added to the list of early signs}~sore throat, fever, chills
    and muscle aches. Gastrointestinal upset, such as diarrhea and
    nausea, has also been observed. Another telltale sign of infection
    may be a sudden, profound diminution of one's
    \href{https://www.nytimes3xbfgragh.onion/2020/03/22/health/coronavirus-symptoms-smell-taste.html?action=click\&pgtype=Article\&state=default\&region=MAIN_CONTENT_3\&context=storylines_faq}{sense
    of smell and taste.}~Teenagers and young adults in some cases have
    developed painful red and purple lesions on their fingers and toes
    --- nicknamed ``Covid toe'' --- but few other serious symptoms.
  \end{itemize}
\item ~
  \hypertarget{why-is-it-safer-to-spend-time-together-outside}{%
  \paragraph{Why is it safer to spend time together
  outside?}\label{why-is-it-safer-to-spend-time-together-outside}}

  \begin{itemize}
  \tightlist
  \item
    \href{https://www.nytimes3xbfgragh.onion/2020/05/15/us/coronavirus-what-to-do-outside.html?action=click\&pgtype=Article\&state=default\&region=MAIN_CONTENT_3\&context=storylines_faq}{Outdoor
    gatherings}~lower risk because wind disperses viral droplets, and
    sunlight can kill some of the virus. Open spaces prevent the virus
    from building up in concentrated amounts and being inhaled, which
    can happen when infected people exhale in a confined space for long
    stretches of time, said Dr. Julian W. Tang, a virologist at the
    University of Leicester.
  \end{itemize}
\item ~
  \hypertarget{why-does-standing-six-feet-away-from-others-help}{%
  \paragraph{Why does standing six feet away from others
  help?}\label{why-does-standing-six-feet-away-from-others-help}}

  \begin{itemize}
  \tightlist
  \item
    The coronavirus spreads primarily through droplets from your mouth
    and nose, especially when you cough or sneeze. The C.D.C., one of
    the organizations using that measure,
    \href{https://www.nytimes3xbfgragh.onion/2020/04/14/health/coronavirus-six-feet.html?action=click\&pgtype=Article\&state=default\&region=MAIN_CONTENT_3\&context=storylines_faq}{bases
    its recommendation of six feet}~on the idea that most large droplets
    that people expel when they cough or sneeze will fall to the ground
    within six feet. But six feet has never been a magic number that
    guarantees complete protection. Sneezes, for instance, can launch
    droplets a lot farther than six feet,
    \href{https://jamanetwork.com/journals/jama/fullarticle/2763852}{according
    to a recent study}. It's a rule of thumb: You should be safest
    standing six feet apart outside, especially when it's windy. But
    keep a mask on at all times, even when you think you're far enough
    apart.
  \end{itemize}
\item ~
  \hypertarget{i-have-antibodies-am-i-now-immune}{%
  \paragraph{I have antibodies. Am I now
  immune?}\label{i-have-antibodies-am-i-now-immune}}

  \begin{itemize}
  \tightlist
  \item
    As of right
    now,\href{https://www.nytimes3xbfgragh.onion/2020/07/22/health/covid-antibodies-herd-immunity.html?action=click\&pgtype=Article\&state=default\&region=MAIN_CONTENT_3\&context=storylines_faq}{~that
    seems likely, for at least several months.}~There have been
    frightening accounts of people suffering what seems to be a second
    bout of Covid-19. But experts say these patients may have a
    drawn-out course of infection, with the virus taking a slow toll
    weeks to months after initial exposure.~People infected with the
    coronavirus typically
    \href{https://www.nature.com/articles/s41586-020-2456-9}{produce}~immune
    molecules called antibodies, which are
    \href{https://www.nytimes3xbfgragh.onion/2020/05/07/health/coronavirus-antibody-prevalence.html?action=click\&pgtype=Article\&state=default\&region=MAIN_CONTENT_3\&context=storylines_faq}{protective
    proteins made in response to an
    infection}\href{https://www.nytimes3xbfgragh.onion/2020/05/07/health/coronavirus-antibody-prevalence.html?action=click\&pgtype=Article\&state=default\&region=MAIN_CONTENT_3\&context=storylines_faq}{.
    These antibodies may}~last in the body
    \href{https://www.nature.com/articles/s41591-020-0965-6}{only two to
    three months}, which may seem worrisome, but that's~perfectly normal
    after an acute infection subsides, said Dr. Michael Mina, an
    immunologist at Harvard University. It may be possible to get the
    coronavirus again, but it's highly unlikely that it would be
    possible in a short window of time from initial infection or make
    people sicker the second time.
  \end{itemize}
\item ~
  \hypertarget{what-are-my-rights-if-i-am-worried-about-going-back-to-work}{%
  \paragraph{What are my rights if I am worried about going back to
  work?}\label{what-are-my-rights-if-i-am-worried-about-going-back-to-work}}

  \begin{itemize}
  \tightlist
  \item
    Employers have to provide
    \href{https://www.osha.gov/SLTC/covid-19/standards.html}{a safe
    workplace}~with policies that protect everyone equally.
    \href{https://www.nytimes3xbfgragh.onion/article/coronavirus-money-unemployment.html?action=click\&pgtype=Article\&state=default\&region=MAIN_CONTENT_3\&context=storylines_faq}{And
    if one of your co-workers tests positive for the coronavirus, the
    C.D.C.}~has said that
    \href{https://www.cdc.gov/coronavirus/2019-ncov/community/guidance-business-response.html}{employers
    should tell their employees}~-\/- without giving you the sick
    employee's name -\/- that they may have been exposed to the virus.
  \end{itemize}
\end{itemize}

\begin{center}\rule{0.5\linewidth}{\linethickness}\end{center}

\hypertarget{asako-narahashi}{%
\subsubsection{\texorpdfstring{\textbf{Asako
Narahashi}}{Asako Narahashi}}\label{asako-narahashi}}

\textbf{Pictures taken in Tokyo}

Image

``Untitled.''Credit...© Asako Narahashi

Faced with such a nightmarish world, every morning I wake up and I feel
shocked that it isn't a dream. At the same time, I feel angry about the
Japanese government's slow and almost cunning response to the situation,
as well as about the insufficient information. I don't want to die under
the current administration. So, I'm trying to stay alive and live my
daily life attentively and respectfully.

Image

``Untitled.''Credit...© Asako Narahashi

My outdoor shooting activities have been suspended. But the
self-quarantine situation has given me an opportunity to look at things
that I hadn't shot much in the past, apart from the context of my usual
work. I don't think everything will ever go back to the way it was, so I
think my work will also change naturally.

Image

``Untitled.''Credit...© Asako Narahashi

I got this cat from a shelter for rescues last summer, and I'm keeping
him indoors. I've been shooting him since I took him in --- the pictures
here are part of the series ``Yabusaka'' (2020), Yabusaka being the
cat's name. He has nowhere to go but our cramped living room and the
sunroom next to it. I still can't touch him because he seems to have
been severely abused during his time as a stray. The social distance
between us is too far.

Image

``Untitled.''Credit...© Asako Narahashi

However, over the last two months, I feel we are gradually sharing the
same space and time more than ever before. Looking at him, I'm able to
forget about everything, if only for a short while. Thanks to this cat,
I feel like I'm being saved.

Image

``Untitled.''Credit...© Asako Narahashi

\begin{center}\rule{0.5\linewidth}{\linethickness}\end{center}

\hypertarget{richard-mosse}{%
\subsubsection{Richard Mosse}\label{richard-mosse}}

Pictures taken in the Burren National Park, Ireland

Image

``Mullaghmore I.''Credit...Richard Mosse

I have always thought that wandering through the hills and the fractured
limestone strata of the Burren landscape feels something like mapping
the striations of one's own mind. This is a land of texture, and it
often takes some concentration on the ground in front of you not to trip
up or fall into any of the ``clints'' or ``grikes,'' the furrowed
delineations created by millions of years of rain erosion. One must
remain focused on each step and absorbed in the present moment. This
helps distill the mind. As a photographer and as a walker, I see this
landscape inwardly, as an expression of layers of thought that become
especially evident after prolonged periods of isolation. I tried to
capture that in this mini-series, as it has been important to me.
Isolation, I've found, can be centering.

Image

``Mullaghmore II.''Credit...Richard Mosse

One of the photos shows a rag tree, which is an ancient practice in
Ireland that descends from pagan times. It is a kind of shamanic site
where people come to be healed. Those with illness and ailments will
make a pilgrimage to the site, bringing some old rag or memento that
represents their sickness and tie it to the rag tree. Doing so is said
to heal the malady, if not physically then in some spiritual way. When
Christianity arrived in Ireland in the Dark Ages, the church
appropriated this practice, and so these sites have survived and are
still popular. The spring bubbling from the rocks beneath this tree is
considered a source of holy water --- it's known locally as a holy well
--- and there are some glass mugs hanging from nails for believers to
use to drink from the purifying stream. I have visited this rag tree for
many years but have never seen it so heavily strewn with rags and other
tokens.

Image

``Mullaghmore III.''Credit...Richard Mosse

I think this moment may be the death of analog photography. And of
course, the art world was always very interpersonal, relational. It was
about showing up to talks, openings, visiting museums, experiencing the
work in person. All that seems like a memory now, replaced by the
digital. This truly has locked us, at least for now, into viewing
photography on social media and online. It will take a lot to return to
the emphasis there was, until recently, on showing up in person, on
giving the work the space to breathe. One could argue that this has the
potential to democratize photography, but remember that each time you
upload an image to social media, you're giving away the rights to a
massive corporation. It's incredibly important for us, as humans, to
show up and be present in order to create society. That's dangerous to
do now, and also currently illegal for many people, so I feel nervous
about what we stand to lose, particularly in regard to human rights and
liberal democracy.

Image

``Mullaghmore IV.''Credit...Richard Mosse

Image

``Rag Tree.''Credit...Richard Mosse

\begin{center}\rule{0.5\linewidth}{\linethickness}\end{center}

\hypertarget{wayne-lawrence}{%
\subsubsection{\texorpdfstring{\textbf{Wayne
Lawrence}}{Wayne Lawrence}}\label{wayne-lawrence}}

\textbf{Pictures taken in St. Kitts, St. Kitts and Nevis}

Image

``Taj, Bird Rock.''Credit...Wayne Lawrence

Image

``Michele With Quail Eggs, Bird Rock.''Credit...Wayne Lawrence

I'm in St. Kitts, where I'm from, in a studio on my father's property.
This was my first time spending the winter in St. Kitts since I left 26
years ago. It's always been a goal of mine; I've never gotten used to
the cold in Brooklyn, so thankfully the timing worked out in that sense.

I don't feel any pressure to go out and make pictures. I'm fairly
content waiting for this to pass. My day is pretty much this: I wake up
at about 4 o'clock in the morning, stretch and meditate for a couple
hours, eat, exercise, shower, rest for a couple hours, do some reading,
check in with my social media, cook \ldots{} I've enjoyed being in the
kitchen. It's a forced hiatus.

Image

``Lovebirds, Bird Rock."Credit...Wayne Lawrence

The beauty of living on the islands is we have a yard, so we can be
outside enjoying nature. And it's a blessing to be here with half of my
family. The pictures I sent are of my siblings, a niece, my dad in his
studio working on a sculpture and my stepmother's birds. My stepmother
raises birds and those are lovebirds. They're caged and not practicing
social distancing.

Image

``Chantelle and Mikhalia, Bird Rock.''Credit...Wayne Lawrence

Image

``Poppa Dennis in Studio, Bird Rock.''Credit...Wayne Lawrence

Whatever happens after this, we have to adapt. We have to keep living,
and we have to keep creating. Now, I'm just living day by day, thankful
for every day I can wake up and smell some fresh air and pick some fresh
fruit off my tree. I've always been an introvert, so this type of living
is not new to me. I just stepped onto my balcony, and the breeze outside
is amazing. It's the first time we've had this kind of breeze in about
two weeks, and it's just what I needed.

Advertisement

\protect\hyperlink{after-bottom}{Continue reading the main story}

\hypertarget{site-index}{%
\subsection{Site Index}\label{site-index}}

\hypertarget{site-information-navigation}{%
\subsection{Site Information
Navigation}\label{site-information-navigation}}

\begin{itemize}
\tightlist
\item
  \href{https://help.nytimes3xbfgragh.onion/hc/en-us/articles/115014792127-Copyright-notice}{©~2020~The
  New York Times Company}
\end{itemize}

\begin{itemize}
\tightlist
\item
  \href{https://www.nytco.com/}{NYTCo}
\item
  \href{https://help.nytimes3xbfgragh.onion/hc/en-us/articles/115015385887-Contact-Us}{Contact
  Us}
\item
  \href{https://www.nytco.com/careers/}{Work with us}
\item
  \href{https://nytmediakit.com/}{Advertise}
\item
  \href{http://www.tbrandstudio.com/}{T Brand Studio}
\item
  \href{https://www.nytimes3xbfgragh.onion/privacy/cookie-policy\#how-do-i-manage-trackers}{Your
  Ad Choices}
\item
  \href{https://www.nytimes3xbfgragh.onion/privacy}{Privacy}
\item
  \href{https://help.nytimes3xbfgragh.onion/hc/en-us/articles/115014893428-Terms-of-service}{Terms
  of Service}
\item
  \href{https://help.nytimes3xbfgragh.onion/hc/en-us/articles/115014893968-Terms-of-sale}{Terms
  of Sale}
\item
  \href{https://spiderbites.nytimes3xbfgragh.onion}{Site Map}
\item
  \href{https://help.nytimes3xbfgragh.onion/hc/en-us}{Help}
\item
  \href{https://www.nytimes3xbfgragh.onion/subscription?campaignId=37WXW}{Subscriptions}
\end{itemize}
