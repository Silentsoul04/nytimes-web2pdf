Sections

SEARCH

\protect\hyperlink{site-content}{Skip to
content}\protect\hyperlink{site-index}{Skip to site index}

\href{https://www.nytimes3xbfgragh.onion/section/health}{Health}

\href{https://myaccount.nytimes3xbfgragh.onion/auth/login?response_type=cookie\&client_id=vi}{}

\href{https://www.nytimes3xbfgragh.onion/section/todayspaper}{Today's
Paper}

\href{/section/health}{Health}\textbar{}Find a Vaccine. Next: Produce
300 Million Vials of It.

\url{https://nyti.ms/2yj6Snf}

\begin{itemize}
\item
\item
\item
\item
\item
\item
\end{itemize}

\hypertarget{the-coronavirus-outbreak}{%
\subsubsection{\texorpdfstring{\href{https://www.nytimes3xbfgragh.onion/news-event/coronavirus?name=styln-coronavirus-national\&region=TOP_BANNER\&block=storyline_menu_recirc\&action=click\&pgtype=Article\&impression_id=7aa06790-f28b-11ea-b918-ff3c9aed24f9\&variant=undefined}{The
Coronavirus
Outbreak}}{The Coronavirus Outbreak}}\label{the-coronavirus-outbreak}}

\begin{itemize}
\tightlist
\item
  live\href{https://www.nytimes3xbfgragh.onion/2020/09/09/world/covid-19-coronavirus.html?name=styln-coronavirus-national\&region=TOP_BANNER\&block=storyline_menu_recirc\&action=click\&pgtype=Article\&impression_id=7aa06791-f28b-11ea-b918-ff3c9aed24f9\&variant=undefined}{Latest
  Updates}
\item
  \href{https://www.nytimes3xbfgragh.onion/interactive/2020/us/coronavirus-us-cases.html?name=styln-coronavirus-national\&region=TOP_BANNER\&block=storyline_menu_recirc\&action=click\&pgtype=Article\&impression_id=7aa06792-f28b-11ea-b918-ff3c9aed24f9\&variant=undefined}{Maps
  and Cases}
\item
  \href{https://www.nytimes3xbfgragh.onion/interactive/2020/science/coronavirus-vaccine-tracker.html?name=styln-coronavirus-national\&region=TOP_BANNER\&block=storyline_menu_recirc\&action=click\&pgtype=Article\&impression_id=7aa06793-f28b-11ea-b918-ff3c9aed24f9\&variant=undefined}{Vaccine
  Tracker}
\item
  \href{https://www.nytimes3xbfgragh.onion/2020/09/02/your-money/eviction-moratorium-covid.html?name=styln-coronavirus-national\&region=TOP_BANNER\&block=storyline_menu_recirc\&action=click\&pgtype=Article\&impression_id=7aa103d0-f28b-11ea-b918-ff3c9aed24f9\&variant=undefined}{Eviction
  Moratorium}
\item
  \href{https://www.nytimes3xbfgragh.onion/interactive/2020/09/02/magazine/food-insecurity-hunger-us.html?name=styln-coronavirus-national\&region=TOP_BANNER\&block=storyline_menu_recirc\&action=click\&pgtype=Article\&impression_id=7aa103d1-f28b-11ea-b918-ff3c9aed24f9\&variant=undefined}{American
  Hunger}
\end{itemize}

Advertisement

\protect\hyperlink{after-top}{Continue reading the main story}

Supported by

\protect\hyperlink{after-sponsor}{Continue reading the main story}

\hypertarget{find-a-vaccine-next-produce-300-million-vials-of-it}{%
\section{Find a Vaccine. Next: Produce 300 Million Vials of
It.}\label{find-a-vaccine-next-produce-300-million-vials-of-it}}

Scaling up the manufacturing of syringes and other medical products
required to deliver a vaccine to millions of Americans will be just as
important as the vaccine itself.

\includegraphics{https://static01.graylady3jvrrxbe.onion/images/2020/04/30/science/30VIRUS-VAXPREP1/merlin_172039386_6e7d1fc3-7ca8-46a5-8da2-0a4a76aa7b0b-articleLarge.jpg?quality=75\&auto=webp\&disable=upscale}

\href{https://www.nytimes3xbfgragh.onion/by/knvul-sheikh}{\includegraphics{https://static01.graylady3jvrrxbe.onion/images/2020/01/03/reader-center/author-knvul-sheikh/author-knvul-sheikh-thumbLarge.png}}

By \href{https://www.nytimes3xbfgragh.onion/by/knvul-sheikh}{Knvul
Sheikh}

\begin{itemize}
\item
  Published May 1, 2020Updated May 4, 2020
\item
  \begin{itemize}
  \item
  \item
  \item
  \item
  \item
  \item
  \end{itemize}
\end{itemize}

In the midst of national shortages of testing swabs and protective gear,
some medical suppliers and health policy experts are looking ahead to
another extraordinary demand on manufacturing: Delivering a vaccine that
could potentially end the pandemic.

Making a vaccine is not easy. More than two dozen companies have
announced programs to develop a
\href{https://www.nytimes3xbfgragh.onion/2020/01/28/health/coronavirus-vaccine.html}{vaccine
against the coronavirus}, but it may still take a year or more before
one passes federal safety and efficacy tests in humans and becomes
available to the public.

Here in the United States, more than 300 million people may need to be
inoculated. That means at least as many vials and syringes --- or double
that amount, if two shots are required. To meet that demand, companies
will have to ramp up manufacturing; products that doctors give little
thought to now could easily become obstacles to vaccine delivery in the
future.

``We're thinking about the vaccine, but what if the vials it is stored
in, or rubber stoppers in the vial or the plungers in the syringes
become the constraint?'' said Prashant Yadav, who studies health care
supply chains at the Center for Global Development in Washington.

Timing the orders of medical products like syringes and all the raw
materials required to make them will be essential. Medical device
manufacturers could increase inventory or find alternative supply chains
for products that are running low, but everything will need to be
systematically planned. Adding the capacity to make millions more
syringes could take a manufacturer as long as 18 months to complete such
a large order, for example.

``The Covid-19 pandemic is creating industrywide challenges, including
expected delays in inventory replenishment for certain products,'' said
Lucy Bradlow, a spokeswoman for Cardinal Health, which makes vials and
syringes as well as other medical supplies.

Several manufacturers worry that the Trump administration may be waiting
too long before ordering an ample supply of medical equipment needed to
deliver a vaccine. One manufacturer said it had recently received an
order for syringes, but was concerned that the Biomedical Advanced
Research and Development Authority, a branch of the Health and Human
Services Department built to help with pandemic preparedness, was still
soliciting too few supplies for nationwide vaccine delivery.

\hypertarget{latest-updates-the-coronavirus-outbreak}{%
\section{\texorpdfstring{\href{https://www.nytimes3xbfgragh.onion/2020/09/09/world/covid-19-coronavirus.html?action=click\&pgtype=Article\&state=default\&region=MAIN_CONTENT_1\&context=storylines_live_updates}{Latest
Updates: The Coronavirus
Outbreak}}{Latest Updates: The Coronavirus Outbreak}}\label{latest-updates-the-coronavirus-outbreak}}

Updated 2020-09-09T10:51:08.325Z

\begin{itemize}
\tightlist
\item
  \href{https://www.nytimes3xbfgragh.onion/2020/09/09/world/covid-19-coronavirus.html?action=click\&pgtype=Article\&state=default\&region=MAIN_CONTENT_1\&context=storylines_live_updates\#link-70cea8bb}{As
  drugmakers pledge to thoroughly vet a vaccine, one company pauses its
  trials for a safety review.}
\item
  \href{https://www.nytimes3xbfgragh.onion/2020/09/09/world/covid-19-coronavirus.html?action=click\&pgtype=Article\&state=default\&region=MAIN_CONTENT_1\&context=storylines_live_updates\#link-780eaa2f}{Britain
  is expected to ban gatherings of more than six people.}
\item
  \href{https://www.nytimes3xbfgragh.onion/2020/09/09/world/covid-19-coronavirus.html?action=click\&pgtype=Article\&state=default\&region=MAIN_CONTENT_1\&context=storylines_live_updates\#link-11cec4c0}{Quarantine
  breakdowns at colleges in the U.S. are leaving some at risk.}
\end{itemize}

\href{https://www.nytimes3xbfgragh.onion/2020/09/09/world/covid-19-coronavirus.html?action=click\&pgtype=Article\&state=default\&region=MAIN_CONTENT_1\&context=storylines_live_updates}{See
more updates}

More live coverage:
\href{https://www.nytimes3xbfgragh.onion/live/2020/09/08/business/stock-market-today-coronavirus?action=click\&pgtype=Article\&state=default\&region=MAIN_CONTENT_1\&context=storylines_live_updates}{Markets}

Elleen Kane, a spokeswoman for the H.H.S., said the department had been
``working daily with our manufacturers to secure those supplies and
assist them with any anticipated obstacles in their supply chains.''

In March, the H.H.S. set up a
\href{https://www.hhs.gov/about/news/2020/03/18/hhs-announces-new-public-private-partnership-to-develop-us-based-high-speed-emergency-drug-packaging-solutions.html}{public-private
partnership} to find drug packaging solutions based in the United
States. It could be adapted for future therapeutic or vaccine delivery
for the Strategic National Stockpile, a federal cache of supplies and
medicines held in case of emergencies.

\includegraphics{https://static01.graylady3jvrrxbe.onion/images/2020/04/30/science/30VIRUS-VAXPREP2/merlin_171924087_34d2cc5d-5cae-4d01-a198-f150bf79bf7e-articleLarge.jpg?quality=75\&auto=webp\&disable=upscale}

The White House is also developing a plan, called
\href{https://www.nytimes3xbfgragh.onion/2020/04/29/us/politics/trump-coronavirus-vaccine-operation-warp-speed.html}{Operation
Warp Speed}, to accelerate vaccine production and try to get
manufacturing capacity set up in advance of a vaccine approval. But some
experts say that it is unclear whether the plan would apply to vaccine
delivery devices like syringes, and details are still scarce about which
federal agency would be responsible.

In April, New Hampshire's senators, Jeanne Shaheen and Maggie Hassan,
sent
\href{https://www.shaheen.senate.gov/imo/media/doc/2020-04-21\%20Letter\%20to\%20VP\%20Pence\%20on\%20Vaccine\%20Materials\%20-\%20Shaheen\%20\&\%20Hassan.pdf}{a
letter to Vice President Mike Pence}, urging him ``to ensure that the
federal government obtains the materials to meet the demand for a
vaccine when it becomes available.''

Of course, a lot will depend on the type of vaccine and when it is
approved. A variety of
\href{https://www.nytimes3xbfgragh.onion/2020/03/16/health/coronavirus-vaccine.html}{RNA-
and DNA-based vaccines} are currently undergoing clinical trials, as
well as more traditional types, which are made by placing instructions
for coronavirus spike proteins inside a different dead or harmless
virus.

RNA or DNA vaccines might have different storage and refrigeration
requirements because the technology has never been used for an approved
vaccine before. The final vaccine might be packaged in ready-to-use
glass syringes, which are commonly used in flu campaigns in Europe, or
in a single-dose or multidose vials that would be administered with
disposable plastic syringes, which are standard for many vaccines in the
United States.

The amount of vaccine manufactured by a company could also affect the
number of delivery systems needed, said Michael Gusmano, a health policy
expert at the Hastings Center and Rutgers School of Public Health. It is
unlikely that a pharmaceutical company will be able to match demand
immediately --- nationally or internationally.

``The good news is we have time,'' Dr. Gusmano said. Medical device
manufacturers could slowly scale up vials and syringes as a vaccine
becomes more widely available.

Early estimates of the coronavirus's infectiousness suggest that at
least 70 percent of the population would need to be immunized to reach
what experts call
\href{https://www.nytimes3xbfgragh.onion/2020/04/10/health/coronavirus-antibody-test.html}{herd
immunity}, when enough people are immune to a disease that they can also
indirectly protect others who are not immune.

\href{https://www.nytimes3xbfgragh.onion/news-event/coronavirus?action=click\&pgtype=Article\&state=default\&region=MAIN_CONTENT_3\&context=storylines_faq}{}

\hypertarget{the-coronavirus-outbreak-}{%
\subsubsection{The Coronavirus Outbreak
›}\label{the-coronavirus-outbreak-}}

\hypertarget{frequently-asked-questions}{%
\paragraph{Frequently Asked
Questions}\label{frequently-asked-questions}}

Updated September 4, 2020

\begin{itemize}
\item ~
  \hypertarget{what-are-the-symptoms-of-coronavirus}{%
  \paragraph{What are the symptoms of
  coronavirus?}\label{what-are-the-symptoms-of-coronavirus}}

  \begin{itemize}
  \tightlist
  \item
    In the beginning, the coronavirus
    \href{https://www.nytimes3xbfgragh.onion/article/coronavirus-facts-history.html?action=click\&pgtype=Article\&state=default\&region=MAIN_CONTENT_3\&context=storylines_faq\#link-6817bab5}{seemed
    like it was primarily a respiratory illness}~--- many patients had
    fever and chills, were weak and tired, and coughed a lot, though
    some people don't show many symptoms at all. Those who seemed
    sickest had pneumonia or acute respiratory distress syndrome and
    received supplemental oxygen. By now, doctors have identified many
    more symptoms and syndromes. In April,
    \href{https://www.nytimes3xbfgragh.onion/2020/04/27/health/coronavirus-symptoms-cdc.html?action=click\&pgtype=Article\&state=default\&region=MAIN_CONTENT_3\&context=storylines_faq}{the
    C.D.C. added to the list of early signs}~sore throat, fever, chills
    and muscle aches. Gastrointestinal upset, such as diarrhea and
    nausea, has also been observed. Another telltale sign of infection
    may be a sudden, profound diminution of one's
    \href{https://www.nytimes3xbfgragh.onion/2020/03/22/health/coronavirus-symptoms-smell-taste.html?action=click\&pgtype=Article\&state=default\&region=MAIN_CONTENT_3\&context=storylines_faq}{sense
    of smell and taste.}~Teenagers and young adults in some cases have
    developed painful red and purple lesions on their fingers and toes
    --- nicknamed ``Covid toe'' --- but few other serious symptoms.
  \end{itemize}
\item ~
  \hypertarget{why-is-it-safer-to-spend-time-together-outside}{%
  \paragraph{Why is it safer to spend time together
  outside?}\label{why-is-it-safer-to-spend-time-together-outside}}

  \begin{itemize}
  \tightlist
  \item
    \href{https://www.nytimes3xbfgragh.onion/2020/05/15/us/coronavirus-what-to-do-outside.html?action=click\&pgtype=Article\&state=default\&region=MAIN_CONTENT_3\&context=storylines_faq}{Outdoor
    gatherings}~lower risk because wind disperses viral droplets, and
    sunlight can kill some of the virus. Open spaces prevent the virus
    from building up in concentrated amounts and being inhaled, which
    can happen when infected people exhale in a confined space for long
    stretches of time, said Dr. Julian W. Tang, a virologist at the
    University of Leicester.
  \end{itemize}
\item ~
  \hypertarget{why-does-standing-six-feet-away-from-others-help}{%
  \paragraph{Why does standing six feet away from others
  help?}\label{why-does-standing-six-feet-away-from-others-help}}

  \begin{itemize}
  \tightlist
  \item
    The coronavirus spreads primarily through droplets from your mouth
    and nose, especially when you cough or sneeze. The C.D.C., one of
    the organizations using that measure,
    \href{https://www.nytimes3xbfgragh.onion/2020/04/14/health/coronavirus-six-feet.html?action=click\&pgtype=Article\&state=default\&region=MAIN_CONTENT_3\&context=storylines_faq}{bases
    its recommendation of six feet}~on the idea that most large droplets
    that people expel when they cough or sneeze will fall to the ground
    within six feet. But six feet has never been a magic number that
    guarantees complete protection. Sneezes, for instance, can launch
    droplets a lot farther than six feet,
    \href{https://jamanetwork.com/journals/jama/fullarticle/2763852}{according
    to a recent study}. It's a rule of thumb: You should be safest
    standing six feet apart outside, especially when it's windy. But
    keep a mask on at all times, even when you think you're far enough
    apart.
  \end{itemize}
\item ~
  \hypertarget{i-have-antibodies-am-i-now-immune}{%
  \paragraph{I have antibodies. Am I now
  immune?}\label{i-have-antibodies-am-i-now-immune}}

  \begin{itemize}
  \tightlist
  \item
    As of right
    now,\href{https://www.nytimes3xbfgragh.onion/2020/07/22/health/covid-antibodies-herd-immunity.html?action=click\&pgtype=Article\&state=default\&region=MAIN_CONTENT_3\&context=storylines_faq}{~that
    seems likely, for at least several months.}~There have been
    frightening accounts of people suffering what seems to be a second
    bout of Covid-19. But experts say these patients may have a
    drawn-out course of infection, with the virus taking a slow toll
    weeks to months after initial exposure.~People infected with the
    coronavirus typically
    \href{https://www.nature.com/articles/s41586-020-2456-9}{produce}~immune
    molecules called antibodies, which are
    \href{https://www.nytimes3xbfgragh.onion/2020/05/07/health/coronavirus-antibody-prevalence.html?action=click\&pgtype=Article\&state=default\&region=MAIN_CONTENT_3\&context=storylines_faq}{protective
    proteins made in response to an
    infection}\href{https://www.nytimes3xbfgragh.onion/2020/05/07/health/coronavirus-antibody-prevalence.html?action=click\&pgtype=Article\&state=default\&region=MAIN_CONTENT_3\&context=storylines_faq}{.
    These antibodies may}~last in the body
    \href{https://www.nature.com/articles/s41591-020-0965-6}{only two to
    three months}, which may seem worrisome, but that's~perfectly normal
    after an acute infection subsides, said Dr. Michael Mina, an
    immunologist at Harvard University. It may be possible to get the
    coronavirus again, but it's highly unlikely that it would be
    possible in a short window of time from initial infection or make
    people sicker the second time.
  \end{itemize}
\item ~
  \hypertarget{what-are-my-rights-if-i-am-worried-about-going-back-to-work}{%
  \paragraph{What are my rights if I am worried about going back to
  work?}\label{what-are-my-rights-if-i-am-worried-about-going-back-to-work}}

  \begin{itemize}
  \tightlist
  \item
    Employers have to provide
    \href{https://www.osha.gov/SLTC/covid-19/standards.html}{a safe
    workplace}~with policies that protect everyone equally.
    \href{https://www.nytimes3xbfgragh.onion/article/coronavirus-money-unemployment.html?action=click\&pgtype=Article\&state=default\&region=MAIN_CONTENT_3\&context=storylines_faq}{And
    if one of your co-workers tests positive for the coronavirus, the
    C.D.C.}~has said that
    \href{https://www.cdc.gov/coronavirus/2019-ncov/community/guidance-business-response.html}{employers
    should tell their employees}~-\/- without giving you the sick
    employee's name -\/- that they may have been exposed to the virus.
  \end{itemize}
\end{itemize}

``That's a remarkably high number, and I don't think we're anywhere
close to that just with people who have been exposed to the virus and
developed antibodies,'' Dr. Gusmano said. ``So you're talking about a
fairly massive vaccination campaign.''

Preliminary surveys in
\href{https://www.nytimes3xbfgragh.onion/2020/04/21/health/coronavirus-antibodies-california.html}{California}
and
\href{https://www.nytimes3xbfgragh.onion/2020/04/23/nyregion/coronavirus-antibodies-test-ny.html}{New
York} suggest that between 4 to 21 percent of people have developed
antibodies to the coronavirus. But the accuracy of many
\href{https://www.nytimes3xbfgragh.onion/2020/04/19/us/coronavirus-antibody-tests.html}{antibody
tests has been called into question}. And it is still unclear whether
having some of these antibodies provides effective and
\href{https://www.nytimes3xbfgragh.onion/reuters/2020/04/25/world/americas/25reuters-health-coronavirus-who.html}{long-lasting
immunity} against the coronavirus. Plus, most vaccination campaigns aim
to immunize a high proportion of the population --- around 90 percent
--- to successfully prevent transmission of disease.

To produce the number of vials and syringes needed for a coronavirus
vaccine, medical suppliers will need to increase manufacturing shifts
and overtime for their employees, as well as collaborate with American
and foreign trade authorities to expedite shipments and shorten lead
times.

A handful of manufacturers are based in the United States, but many
still have to import the glass tubing for vials, polypropylene for
syringes and rubber or silicone for small parts like the stoppers and
plungers in these devices. Becton Dickinson \& Company, one of the
world's largest manufacturers of needles and syringes, said it made
nearly all components of its needles and syringes in-house in the United
States. Other companies may source from their factories and partners
located largely in China and India, where lockdowns and export bans have
already decreased production and exports.

Although syringe manufacturing is
\href{https://www.youtube.com/watch?v=EFQApDeslS0\&feature=youtu.be}{mostly
automated}, with parts like the barrel and plunger made from a mold and
put together on an assembly line, Dr. Yadav said manufacturers in India
had told him fewer employees were able to work than needed for full
capacity.

At least 69 countries have also banned or restricted the export of
medical devices, medicines and protective equipment, according to the
Global Trade Alert project at the University of St. Gallen in
Switzerland, because of their own needs during the pandemic.
Manufacturers of medical syringes may have to find new supply channels,
including partnerships with glass and plastic manufacturers that operate
outside of the health care industry.

Some lawmakers are concerned that without more federal coordination,
individual companies will not have the capacity to match vaccine
delivery to the overwhelming demand.

``The Trump administration needs to prepare our domestic supply chain
now for the delivery of an eventual vaccine that will need to be
delivered to the entire country,'' Senator Shaheen said. ``It's vital
that federal agencies exercise better foresight so that we don't see
supply shortages like we continue to experience for testing and
protective equipment.''

Advertisement

\protect\hyperlink{after-bottom}{Continue reading the main story}

\hypertarget{site-index}{%
\subsection{Site Index}\label{site-index}}

\hypertarget{site-information-navigation}{%
\subsection{Site Information
Navigation}\label{site-information-navigation}}

\begin{itemize}
\tightlist
\item
  \href{https://help.nytimes3xbfgragh.onion/hc/en-us/articles/115014792127-Copyright-notice}{©~2020~The
  New York Times Company}
\end{itemize}

\begin{itemize}
\tightlist
\item
  \href{https://www.nytco.com/}{NYTCo}
\item
  \href{https://help.nytimes3xbfgragh.onion/hc/en-us/articles/115015385887-Contact-Us}{Contact
  Us}
\item
  \href{https://www.nytco.com/careers/}{Work with us}
\item
  \href{https://nytmediakit.com/}{Advertise}
\item
  \href{http://www.tbrandstudio.com/}{T Brand Studio}
\item
  \href{https://www.nytimes3xbfgragh.onion/privacy/cookie-policy\#how-do-i-manage-trackers}{Your
  Ad Choices}
\item
  \href{https://www.nytimes3xbfgragh.onion/privacy}{Privacy}
\item
  \href{https://help.nytimes3xbfgragh.onion/hc/en-us/articles/115014893428-Terms-of-service}{Terms
  of Service}
\item
  \href{https://help.nytimes3xbfgragh.onion/hc/en-us/articles/115014893968-Terms-of-sale}{Terms
  of Sale}
\item
  \href{https://spiderbites.nytimes3xbfgragh.onion}{Site Map}
\item
  \href{https://help.nytimes3xbfgragh.onion/hc/en-us}{Help}
\item
  \href{https://www.nytimes3xbfgragh.onion/subscription?campaignId=37WXW}{Subscriptions}
\end{itemize}
