Sections

SEARCH

\protect\hyperlink{site-content}{Skip to
content}\protect\hyperlink{site-index}{Skip to site index}

\href{https://www.nytimes3xbfgragh.onion/section/politics}{Politics}

\href{https://myaccount.nytimes3xbfgragh.onion/auth/login?response_type=cookie\&client_id=vi}{}

\href{https://www.nytimes3xbfgragh.onion/section/todayspaper}{Today's
Paper}

\href{/section/politics}{Politics}\textbar{}Profits and Pride at Stake,
the Race for a Vaccine Intensifies

\url{https://nyti.ms/2KUHyqw}

\begin{itemize}
\item
\item
\item
\item
\item
\item
\end{itemize}

\hypertarget{the-coronavirus-outbreak}{%
\subsubsection{\texorpdfstring{\href{https://www.nytimes3xbfgragh.onion/news-event/coronavirus?name=styln-coronavirus-national\&region=TOP_BANNER\&block=storyline_menu_recirc\&action=click\&pgtype=Article\&impression_id=cd9471c0-f273-11ea-85e4-7f6370b9a399\&variant=undefined}{The
Coronavirus
Outbreak}}{The Coronavirus Outbreak}}\label{the-coronavirus-outbreak}}

\begin{itemize}
\tightlist
\item
  live\href{https://www.nytimes3xbfgragh.onion/2020/09/08/world/covid-19-coronavirus.html?name=styln-coronavirus-national\&region=TOP_BANNER\&block=storyline_menu_recirc\&action=click\&pgtype=Article\&impression_id=cd9498d0-f273-11ea-85e4-7f6370b9a399\&variant=undefined}{Latest
  Updates}
\item
  \href{https://www.nytimes3xbfgragh.onion/interactive/2020/us/coronavirus-us-cases.html?name=styln-coronavirus-national\&region=TOP_BANNER\&block=storyline_menu_recirc\&action=click\&pgtype=Article\&impression_id=cd9498d1-f273-11ea-85e4-7f6370b9a399\&variant=undefined}{Maps
  and Cases}
\item
  \href{https://www.nytimes3xbfgragh.onion/interactive/2020/science/coronavirus-vaccine-tracker.html?name=styln-coronavirus-national\&region=TOP_BANNER\&block=storyline_menu_recirc\&action=click\&pgtype=Article\&impression_id=cd9498d2-f273-11ea-85e4-7f6370b9a399\&variant=undefined}{Vaccine
  Tracker}
\item
  \href{https://www.nytimes3xbfgragh.onion/2020/09/02/your-money/eviction-moratorium-covid.html?name=styln-coronavirus-national\&region=TOP_BANNER\&block=storyline_menu_recirc\&action=click\&pgtype=Article\&impression_id=cd9498d3-f273-11ea-85e4-7f6370b9a399\&variant=undefined}{Eviction
  Moratorium}
\item
  \href{https://www.nytimes3xbfgragh.onion/interactive/2020/09/02/magazine/food-insecurity-hunger-us.html?name=styln-coronavirus-national\&region=TOP_BANNER\&block=storyline_menu_recirc\&action=click\&pgtype=Article\&impression_id=cd9498d4-f273-11ea-85e4-7f6370b9a399\&variant=undefined}{American
  Hunger}
\end{itemize}

Advertisement

\protect\hyperlink{after-top}{Continue reading the main story}

Supported by

\protect\hyperlink{after-sponsor}{Continue reading the main story}

\hypertarget{profits-and-pride-at-stake-the-race-for-a-vaccine-intensifies}{%
\section{Profits and Pride at Stake, the Race for a Vaccine
Intensifies}\label{profits-and-pride-at-stake-the-race-for-a-vaccine-intensifies}}

Governments, companies and academic labs are accelerating their efforts
amid geopolitical crosscurrents, questions about safety and the
challenges of producing enough doses for billions of people.

\includegraphics{https://static01.graylady3jvrrxbe.onion/images/2020/05/03/us/politics/02dc-virus-vaccine/02dc-virus-vaccine-articleLarge.jpg?quality=75\&auto=webp\&disable=upscale}

By \href{https://www.nytimes3xbfgragh.onion/by/david-e-sanger}{David E.
Sanger},
\href{https://www.nytimes3xbfgragh.onion/by/david-d-kirkpatrick}{David
D. Kirkpatrick},
\href{https://www.nytimes3xbfgragh.onion/by/carl-zimmer}{Carl Zimmer},
\href{https://www.nytimes3xbfgragh.onion/by/katie-thomas}{Katie Thomas}
and \href{https://www.nytimes3xbfgragh.onion/by/sui-lee-wee}{Sui-Lee
Wee}

\begin{itemize}
\item
  Published May 2, 2020Updated May 20, 2020
\item
  \begin{itemize}
  \item
  \item
  \item
  \item
  \item
  \item
  \end{itemize}
\end{itemize}

WASHINGTON --- Four months after a mysterious new virus began its deadly
march around the globe, the search for a
\href{https://www.nytimes3xbfgragh.onion/2020/05/20/health/coronavirus-vaccines.html}{vaccine}
has taken on an intensity never before seen in medical research, with
huge implications for public health, the world economy and politics.

Seven of the roughly 90 projects being pursued by governments,
pharmaceutical makers, biotech innovators and academic laboratories have
reached the stage of clinical trials. With political leaders --- not
least President Trump --- increasingly pressing for progress, and with
big potential profits at stake for the industry, drug makers and
researchers have signaled that they are moving ahead at unheard-of
speeds.

But the whole enterprise remains dogged by uncertainty about whether any
\href{https://www.nytimes3xbfgragh.onion/2020/05/20/health/coronavirus-vaccines.html}{coronavirus
vaccine} will prove effective, how fast it could be made available to
millions or billions of people and whether the rush --- compressing a
process that can take 10 years into 10 months --- will sacrifice safety.

Some experts say the more immediately promising field might be the
development of treatments to speed recovery from Covid-19, an approach
that has generated some optimism in the last week through initially
encouraging research results on
\href{https://www.nytimes3xbfgragh.onion/2020/04/29/health/gilead-remdesivir-coronavirus.html}{remdesivir},
an antiviral drug previously tried in fighting Ebola.

In an era of intense nationalism, the
\href{https://www.nytimes3xbfgragh.onion/2020/03/19/us/politics/coronavirus-vaccine-competition.html}{geopolitics}
of the vaccine race are growing as complex as the medicine. The months
of
\href{https://www.nytimes3xbfgragh.onion/2020/05/01/us/politics/coronavirus-china-trump.html}{mutual
vilification} between the United States and China over the origins of
the virus have poisoned most efforts at cooperation between them. The
U.S. government is already warning that American innovations must be
protected from theft --- chiefly from Beijing.

``Biomedical research has long been a focus of theft, especially by the
Chinese government, and
\href{https://www.nytimes3xbfgragh.onion/2020/05/15/us/politics/coronavirus-vaccine-timeline.html}{vaccines}
and treatments for the coronavirus are today's holy grail,'' John C.
Demers, the assistant attorney general for national security, said on
Friday. ``Putting aside the commercial value, there would be great
geopolitical significance to being the first to develop a treatment or
vaccine. We will use all the tools we have to safeguard American
research.''

The intensity of the global research effort is such that governments and
companies are building production lines before they have anything to
produce.

``We are going to start ramping up production with the companies
involved,'' Dr. Anthony S. Fauci, the director of the National Institute
of Allergy and Infectious Diseases and the federal government's top
expert on infectious diseases, said on NBC this week. ``You don't wait
until you get an answer before you start manufacturing.''

Two of the leading entrants in the United States,
\href{https://www.nytimes3xbfgragh.onion/2020/05/19/business/johnson-baby-powder-sales-stopped.html}{Johnson
\& Johnson} and
\href{https://www.nytimes3xbfgragh.onion/2020/05/18/health/coronavirus-vaccine-moderna.html}{Moderna},
have announced partnerships with manufacturing firms, with Johnson \&
Johnson promising a billion doses of an as-yet-undeveloped vaccine by
the end of next year.

Not to be left behind, the Britain-based pharmaceutical giant
AstraZeneca said this week that it was working with a vaccine
development project at the University of Oxford to manufacture tens of
millions of doses by the end of this year.

\includegraphics{https://static01.graylady3jvrrxbe.onion/images/2020/05/03/us/politics/02dc-virus-vaccine-detail/merlin_171009114_c949d66b-5544-4b61-945c-28993760a7ba-articleLarge.jpg?quality=75\&auto=webp\&disable=upscale}

With the demand for a vaccine so intense, there are escalating calls for
``human-challenge trials'' to speed the process: tests in which
volunteers are injected with a potential vaccine and then deliberately
exposed to the coronavirus.

Because the approach involves exposing participants to a potentially
deadly disease, challenge trials are ethically fraught. But they could
be faster than simply inoculating human subjects and waiting for them to
be exposed along with everyone else, especially as the pandemic is
brought under control in big countries.

Even when promising solutions are found, there are big challenges to
scaling up production and distribution. Bill Gates, the Microsoft
founder, whose foundation is spending \$250 million to help spur vaccine
development, has warned about a critical shortage of a mundane but vital
component: medical glass.

Without sufficient supplies of the glass, there will be
\href{https://www.nytimes3xbfgragh.onion/2020/05/01/health/coronavirus-vaccine-supplies.html}{too
few vials}to transport the billions of doses that will ultimately be
needed.

The scale of the problem and the demand for a quick solution are bound
to create tensions between the profit motives of the pharmaceutical
industry, which typically fights hard to wring the most out of their
investments in new drugs, and the public's need for quick action to get
any effective vaccines to as many people as possible.

So far, much of the research and development has been supported by
governments and foundations. And much remains to be worked out when it
comes to patents and what national governments will claim in return for
their support and pledges of quick regulatory approval.

Given the stakes, it is no surprise that while scientists and doctors
talk about finding a ``global vaccine,'' national leaders emphasize
immunizing their own populations first. Mr. Trump said he was personally
in charge of
\href{https://www.nytimes3xbfgragh.onion/2020/04/29/us/politics/trump-coronavirus-vaccine-operation-warp-speed.html}{``Operation
Warp Speed''} to get 300 million doses into American arms by January.

Already, the administration has identified 14 vaccine projects it
intends to focus on, a senior administration official said, with the
idea of further narrowing the group to a handful that could go on, with
government financial help and accelerated regulatory review, to meet Mr.
Trump's goal. The winnowing of the projects to 14 was
\href{https://www.nbcnews.com/politics/white-house/health-officials-eyeing-least-one-14-potential-coronavirus-vaccines-fast-n1198326}{reported
Friday} by NBC News.

But other countries are also signaling their
\href{https://www.nytimes3xbfgragh.onion/2020/04/10/business/coronavirus-vaccine-nationalism.html}{intention
to nationalize their approaches}. The most promising clinical trial in
China is financed by the government. And in India, the chief executive
of the Serum Institute of India --- the world's largest producer of
vaccine doses --- said that most of its vaccine ``would have to go to
our countrymen before it goes abroad.''

George Q. Daley, the dean of Harvard Medical School, said thinking in
country-by-country rather than global terms would be foolhardy since it
``would involve squandering the early doses of vaccine on a large number
of individuals at low risk, rather than covering as many high-risk
individuals globally'' --- health care workers and older adults --- ``to
stop the spread'' around the world.

\hypertarget{latest-updates-the-coronavirus-outbreak}{%
\section{\texorpdfstring{\href{https://www.nytimes3xbfgragh.onion/2020/09/08/world/covid-19-coronavirus.html?action=click\&pgtype=Article\&state=default\&region=MAIN_CONTENT_1\&context=storylines_live_updates}{Latest
Updates: The Coronavirus
Outbreak}}{Latest Updates: The Coronavirus Outbreak}}\label{latest-updates-the-coronavirus-outbreak}}

Updated 2020-09-09T07:57:07.770Z

\begin{itemize}
\tightlist
\item
  \href{https://www.nytimes3xbfgragh.onion/2020/09/08/world/covid-19-coronavirus.html?action=click\&pgtype=Article\&state=default\&region=MAIN_CONTENT_1\&context=storylines_live_updates\#link-313b443d}{AstraZeneca
  halts a vaccine trial to investigate a participant's illness.}
\item
  \href{https://www.nytimes3xbfgragh.onion/2020/09/08/world/covid-19-coronavirus.html?action=click\&pgtype=Article\&state=default\&region=MAIN_CONTENT_1\&context=storylines_live_updates\#link-4438dd7}{Facing
  a surge in cases, Britain plans to limit most gatherings to six
  people.}
\item
  \href{https://www.nytimes3xbfgragh.onion/2020/09/08/world/covid-19-coronavirus.html?action=click\&pgtype=Article\&state=default\&region=MAIN_CONTENT_1\&context=storylines_live_updates\#link-679303d7}{Nine
  drugmakers pledge to thoroughly vet any coronavirus vaccine.}
\end{itemize}

\href{https://www.nytimes3xbfgragh.onion/2020/09/08/world/covid-19-coronavirus.html?action=click\&pgtype=Article\&state=default\&region=MAIN_CONTENT_1\&context=storylines_live_updates}{See
more updates}

More live coverage:
\href{https://www.nytimes3xbfgragh.onion/live/2020/09/08/business/stock-market-today-coronavirus?action=click\&pgtype=Article\&state=default\&region=MAIN_CONTENT_1\&context=storylines_live_updates}{Markets}

Given the proliferation of vaccine projects, the best outcome may be
none of them emerging as a clear winner.

``Let's say we get one vaccine quickly but we can only get two million
doses of it at the end of next year,'' said Anita Zaidi, who directs the
Bill and Melinda Gates Foundation's vaccine development program. ``And
another vaccine, just as effective, comes three months later but we can
make a billion doses. Who won that race?''

The answer, she said, ``is we will need many different vaccines to cross
the finish line.''

\hypertarget{speed-versus-safety}{%
\subsection{Speed Versus Safety}\label{speed-versus-safety}}

Image

Dr. Maurice Hilleman holds the record for the quickest delivery of a
vaccine from the lab to the clinic: four years.Credit...Associated Press

At 1 a.m. on March 21, 1963, a 5-year-old girl named Jeryl Lynn Hilleman
woke up her father. She had come down with the mumps, which had made her
miserable with a swollen jaw.

It just so happened that her father, Maurice, was a vaccine designer. So
he told Jeryl Lynn to go back to bed, drove to his lab at Merck to pick
up some equipment, and returned to swab her throat. Dr. Hilleman
refrigerated her sample back at his lab and soon got to work weakening
her viruses until they could serve as a mumps vaccine. In 1967, it was
approved by the F.D.A.

To vaccine makers, this story is the stuff of legend. Dr. Hilleman still
holds the record for the quickest delivery of a vaccine from the lab to
the clinic. Vaccines typically take ten to fifteen years of research and
testing. And only six percent of the projects that scientists launch
reach the finish line.

For a world in the grips of Covid-19, on the other hand, this story is
the stuff of nightmares. No one wants to wait four years for a vaccine,
while millions die and economies remain paralyzed.

Some of the leading contenders for a coronavirus vaccine are now
promising to have the first batches ready in record time, by the start
of next year. They have accelerated their schedules by collapsing the
standard vaccine timeline.

They are combining trials that used to be carried out one after the
other. They are pushing their formulations into production, despite the
risk that the trials will fail, leaving them with millions of useless
doses.

But some experts want to do even more to speed up the conveyor belt.
Writing last month in the journal Vaccines, the vaccine developer Dr.
Stanley A. Plotkin and Dr. Arthur L. Caplan, a bioethicist at NYU
Langone Medical Center, proposed infecting vaccinated volunteers with
the coronavirus --- the method known as challenge trials. The procedure
might cut months or years off the development but would put test
subjects at risk.

Challenge trials were used in the early days of vaccine research but now
are
\href{https://www.nytimes3xbfgragh.onion/2020/04/30/opinion/coronavirus-vaccine-covid.html}{carried
out under strict conditions}and only for illnesses, like flu and
malaria, that have established treatments.

In
\href{https://dash.harvard.edu/bitstream/handle/1/42639016/jiaa152.pdf?sequence=4\&isAllowed=y}{an
article} in March in The Journal of Infectious Diseases, a team of
researchers wrote, ``Such an approach is not without risks, but every
week that vaccine rollout is delayed will be accompanied by many
thousands of deaths globally.''

Dr. Caplan said that limiting the trials to healthy young adults could
reduce the risk, since they were less likely to suffer serious
complications from Covid-19. ``I think we can let people make the choice
and I have no doubt many would,'' he said.

Image

The manufacturing workshop at the Wuhan Institute of Biological Products
in China. The U.S. and China have clashed over the origins of the
coronavirus, dampening cooperation in developing a
vaccine.Credit...China Stringer Network/Reuters

In Congress, Representative Bill Foster, Democrat of Illinois and a
physicist, and Representative Donna E. Shalala, Democrat of Florida and
the former secretary of the Department of Health and Human Services,
organized a bipartisan group of 35 lawmakers to sign a letter asking
regulators to approve such trials.

The organizers of a website set up to promote the idea,
\href{https://1daysooner.org/}{1daysooner.org}, say they have signed up
more than 9,100 potential volunteers from 52 countries.

Some scientists caution that truly informed consent, even by willing
volunteers, may not be possible. Even medical experts do not yet know
all the effects of the virus. Those who have appeared to recover might
still face future problems.

Even without challenge trials, accelerated testing may run the risk of
missing potential side effects. A vaccine for dengue fever, and one for
SARS that never reached the market, were abandoned after making some
people more susceptible to severe forms of the diseases, not less.

``It will be extremely important to determine that does not happen,''
said Michel De Wilde, a former senior vice president of research and
development at Sanofi Pasteur, a vaccine maker in France.

When it comes to the risks from flawed vaccines, China's history is
instructive.

The Wuhan Institute of Biological Products was involved in a 2018
scandal in which ineffective vaccines for diphtheria, tetanus, whooping
cough and other conditions were injected into hundreds of thousands of
babies.

The government confiscated the Wuhan institute's ``illegal income,''
fined the company, and punished nine executives. But the company was
allowed to continue to operate. It is now running a coronavirus vaccine
project, and along with two other Chinese groups has been allowed to
combine its safety and efficacy trials.

Several Chinese scientists questioned the decision, arguing that the
vaccine should be shown to be safe before testing how well it works.

\hypertarget{nationalism-versus-globalism}{%
\subsection{Nationalism Versus
Globalism}\label{nationalism-versus-globalism}}

Image

Elderly women waiting to see health workers in Mumbai. A powerful
vaccine manufacturer in India has made it clear that any vaccine it
produces would have to first go to India's 1.3 billion people, at least
initially.Credit...Atul Loke for The New York Times

In the early days of the crisis, Harvard was approached by the Chinese
billionaire Hui Ka Yan. He arranged to give roughly \$115 million to be
split between Harvard Medical School and its affiliated hospitals and
the Guangzhou Institute of Respiratory Diseases for a collaborative
effort that would include developing coronavirus vaccines.

``We are not racing against each other, we are racing the virus,'' said
Dr. Dan Barouch, the director of the Center for Virology and Vaccine
Research at Beth Israel Deaconess Medical Center and a professor at
Harvard Medical School who is also working with Johnson \& Johnson.
``What we need is a global vaccine --- because an outbreak in one part
of the world puts the rest of the world at risk.''

That all-for-one sentiment has become a mantra among many researchers,
but it is hardly universally shared.

In India, the Serum Institute --- the heavyweight champion of vaccine
manufacturing, producing 1.5 billion doses a year --- has signed
agreements in recent weeks with the developers of four promising
potential vaccines. But in
\href{https://www.nytimes3xbfgragh.onion/reuters/2020/04/28/world/europe/28reuters-health-coronavirus-india-vaccine.html}{an
interview with Reuters}, Adar Poonawalla, the company's billionaire
chief executive, made it clear that ``at least initially'' any vaccine
the company produces would have to go to India's 1.3 billion people.

The tension between those who believe a vaccine should go where it is
needed most and those dealing with pressures to supply their own country
first is one of the defining features of the global response.

The Trump administration, which in March put out feelers to a German
biotech company to acquire its vaccine research or move it to American
shores, has awarded grants of nearly half a billion dollars each to two
U.S.-based companies, Johnson \& Johnson and Moderna.

Johnson \& Johnson, though based in New Jersey, conducts its research in
the Netherlands.

Paul Stoffels, the company's vice chairman and chief scientific officer,
said in an interview that the Department of Health and Human Services
understood ``we can't pick up our research and move it'' to the United
States. But it made sure that the company joined a partnership with
Emergent BioSolutions --- a Maryland biological production firm --- to
produce the first big batches of any approved vaccine for the United
States.

``The political reality is that it would be very, very hard for any
government to allow a vaccine made in their own country to be exported
while there was a major problem at home,'' said Sandy Douglas, a
researcher at the University of Oxford. ``The only solution is to make a
hell of a lot of vaccine in a lot of different places.''

The\href{https://www.nytimes3xbfgragh.onion/2020/04/27/world/europe/coronavirus-vaccine-update-oxford.html}{Oxford
vaccine team} has already begun scaling up plans for manufacturing by
half a dozen companies across the world, including China and India, plus
two British manufacturers and the British-based multinational
AstraZeneca.

In China, the government's instinct is to showcase the country's growth
into a technological power capable of beating the United States. There
are nine Chinese Covid-19 vaccines in development, involving 1,000
scientists and the Chinese military.

China's Center for Disease Control and Prevention predicted that one of
the vaccines could be in ``emergency use'' by September, meaning that in
the midst of the presidential election in the United States, Mr. Trump
might see television footage of Chinese citizens lining up for
injections.

``It's a scenario we have thought about,'' one member of Mr. Trump's
coronavirus task force said. ``No one wants to be around that day.''

\hypertarget{traditional-versus-new-methods}{%
\subsection{Traditional Versus New
Methods}\label{traditional-versus-new-methods}}

Image

Engineers working with monkey kidney cells at a Sinovac laboratory in
Beijing. The company announced that its Covid-19 vaccine protected
monkeys.Credit...Nicolas Asfouri/Agence France-Presse --- Getty Images

The more than 90 different vaccines under development work in radically
different ways. Some are based on designs used for generations. Others
use genetic-based strategies that are so new they have yet to lead to an
approved vaccine.

``I think in this case it's very wise to have different platforms being
tried out,'' Dr. De Wilde said.

The traditional approach is to make vaccines from viruses.

When our bodies encounter a new virus, they start learning how to make
effective antibodies against it. But they are in a race against the
virus as it multiplies. Sometimes they produce effective antibodies
quickly enough to wipe out an infection. But sometimes the virus wins.

Vaccines give the immune system a head start. They teach it to make
antibodies in advance of an infection.

\href{https://www.nytimes3xbfgragh.onion/news-event/coronavirus?action=click\&pgtype=Article\&state=default\&region=MAIN_CONTENT_3\&context=storylines_faq}{}

\hypertarget{the-coronavirus-outbreak-}{%
\subsubsection{The Coronavirus Outbreak
›}\label{the-coronavirus-outbreak-}}

\hypertarget{frequently-asked-questions}{%
\paragraph{Frequently Asked
Questions}\label{frequently-asked-questions}}

Updated September 4, 2020

\begin{itemize}
\item ~
  \hypertarget{what-are-the-symptoms-of-coronavirus}{%
  \paragraph{What are the symptoms of
  coronavirus?}\label{what-are-the-symptoms-of-coronavirus}}

  \begin{itemize}
  \tightlist
  \item
    In the beginning, the coronavirus
    \href{https://www.nytimes3xbfgragh.onion/article/coronavirus-facts-history.html?action=click\&pgtype=Article\&state=default\&region=MAIN_CONTENT_3\&context=storylines_faq\#link-6817bab5}{seemed
    like it was primarily a respiratory illness}~--- many patients had
    fever and chills, were weak and tired, and coughed a lot, though
    some people don't show many symptoms at all. Those who seemed
    sickest had pneumonia or acute respiratory distress syndrome and
    received supplemental oxygen. By now, doctors have identified many
    more symptoms and syndromes. In April,
    \href{https://www.nytimes3xbfgragh.onion/2020/04/27/health/coronavirus-symptoms-cdc.html?action=click\&pgtype=Article\&state=default\&region=MAIN_CONTENT_3\&context=storylines_faq}{the
    C.D.C. added to the list of early signs}~sore throat, fever, chills
    and muscle aches. Gastrointestinal upset, such as diarrhea and
    nausea, has also been observed. Another telltale sign of infection
    may be a sudden, profound diminution of one's
    \href{https://www.nytimes3xbfgragh.onion/2020/03/22/health/coronavirus-symptoms-smell-taste.html?action=click\&pgtype=Article\&state=default\&region=MAIN_CONTENT_3\&context=storylines_faq}{sense
    of smell and taste.}~Teenagers and young adults in some cases have
    developed painful red and purple lesions on their fingers and toes
    --- nicknamed ``Covid toe'' --- but few other serious symptoms.
  \end{itemize}
\item ~
  \hypertarget{why-is-it-safer-to-spend-time-together-outside}{%
  \paragraph{Why is it safer to spend time together
  outside?}\label{why-is-it-safer-to-spend-time-together-outside}}

  \begin{itemize}
  \tightlist
  \item
    \href{https://www.nytimes3xbfgragh.onion/2020/05/15/us/coronavirus-what-to-do-outside.html?action=click\&pgtype=Article\&state=default\&region=MAIN_CONTENT_3\&context=storylines_faq}{Outdoor
    gatherings}~lower risk because wind disperses viral droplets, and
    sunlight can kill some of the virus. Open spaces prevent the virus
    from building up in concentrated amounts and being inhaled, which
    can happen when infected people exhale in a confined space for long
    stretches of time, said Dr. Julian W. Tang, a virologist at the
    University of Leicester.
  \end{itemize}
\item ~
  \hypertarget{why-does-standing-six-feet-away-from-others-help}{%
  \paragraph{Why does standing six feet away from others
  help?}\label{why-does-standing-six-feet-away-from-others-help}}

  \begin{itemize}
  \tightlist
  \item
    The coronavirus spreads primarily through droplets from your mouth
    and nose, especially when you cough or sneeze. The C.D.C., one of
    the organizations using that measure,
    \href{https://www.nytimes3xbfgragh.onion/2020/04/14/health/coronavirus-six-feet.html?action=click\&pgtype=Article\&state=default\&region=MAIN_CONTENT_3\&context=storylines_faq}{bases
    its recommendation of six feet}~on the idea that most large droplets
    that people expel when they cough or sneeze will fall to the ground
    within six feet. But six feet has never been a magic number that
    guarantees complete protection. Sneezes, for instance, can launch
    droplets a lot farther than six feet,
    \href{https://jamanetwork.com/journals/jama/fullarticle/2763852}{according
    to a recent study}. It's a rule of thumb: You should be safest
    standing six feet apart outside, especially when it's windy. But
    keep a mask on at all times, even when you think you're far enough
    apart.
  \end{itemize}
\item ~
  \hypertarget{i-have-antibodies-am-i-now-immune}{%
  \paragraph{I have antibodies. Am I now
  immune?}\label{i-have-antibodies-am-i-now-immune}}

  \begin{itemize}
  \tightlist
  \item
    As of right
    now,\href{https://www.nytimes3xbfgragh.onion/2020/07/22/health/covid-antibodies-herd-immunity.html?action=click\&pgtype=Article\&state=default\&region=MAIN_CONTENT_3\&context=storylines_faq}{~that
    seems likely, for at least several months.}~There have been
    frightening accounts of people suffering what seems to be a second
    bout of Covid-19. But experts say these patients may have a
    drawn-out course of infection, with the virus taking a slow toll
    weeks to months after initial exposure.~People infected with the
    coronavirus typically
    \href{https://www.nature.com/articles/s41586-020-2456-9}{produce}~immune
    molecules called antibodies, which are
    \href{https://www.nytimes3xbfgragh.onion/2020/05/07/health/coronavirus-antibody-prevalence.html?action=click\&pgtype=Article\&state=default\&region=MAIN_CONTENT_3\&context=storylines_faq}{protective
    proteins made in response to an
    infection}\href{https://www.nytimes3xbfgragh.onion/2020/05/07/health/coronavirus-antibody-prevalence.html?action=click\&pgtype=Article\&state=default\&region=MAIN_CONTENT_3\&context=storylines_faq}{.
    These antibodies may}~last in the body
    \href{https://www.nature.com/articles/s41591-020-0965-6}{only two to
    three months}, which may seem worrisome, but that's~perfectly normal
    after an acute infection subsides, said Dr. Michael Mina, an
    immunologist at Harvard University. It may be possible to get the
    coronavirus again, but it's highly unlikely that it would be
    possible in a short window of time from initial infection or make
    people sicker the second time.
  \end{itemize}
\item ~
  \hypertarget{what-are-my-rights-if-i-am-worried-about-going-back-to-work}{%
  \paragraph{What are my rights if I am worried about going back to
  work?}\label{what-are-my-rights-if-i-am-worried-about-going-back-to-work}}

  \begin{itemize}
  \tightlist
  \item
    Employers have to provide
    \href{https://www.osha.gov/SLTC/covid-19/standards.html}{a safe
    workplace}~with policies that protect everyone equally.
    \href{https://www.nytimes3xbfgragh.onion/article/coronavirus-money-unemployment.html?action=click\&pgtype=Article\&state=default\&region=MAIN_CONTENT_3\&context=storylines_faq}{And
    if one of your co-workers tests positive for the coronavirus, the
    C.D.C.}~has said that
    \href{https://www.cdc.gov/coronavirus/2019-ncov/community/guidance-business-response.html}{employers
    should tell their employees}~-\/- without giving you the sick
    employee's name -\/- that they may have been exposed to the virus.
  \end{itemize}
\end{itemize}

The first vaccines, against diseases like rabies, were made from
viruses. Scientists weakened the viruses so that they could no longer
make people sick.

A number of groups are weakening the coronavirus to produce a vaccine
against Covid-19. In April, the Chinese company Sinovac announced that
their inactivated vaccine protected monkeys.

Another approach is based on the fact that our immune system makes
antibodies that lock precisely onto viruses. As scientists came to
understand this, it occurred to them that they didn't have to inject a
whole virus into someone to trigger immunity. All they needed was to
deliver the fragment of a viral protein that was the precise target.

Today these so-called subunit viral vaccines are used against hepatitis
B and shingles. Many Covid-19 subunit vaccines are now in testing.

In the 1990s, researchers began working on vaccines that enlisted our
own cells to help train the immune system. The foundation of these
vaccines is typically a virus called an adenovirus. The adenovirus can
infect our cells, but is altered so that it doesn't make us sick.

Scientists can add a gene to the adenovirus from the virus they want to
fight, creating what's known as a viral vector. Some viral vectors then
invade our cells, stimulating the immune system to make antibodies.

Researchers at the University of Oxford and the Chinese company CanSino
Biologics have created a viral vector vaccine for Covid-19, and they've
started safety trials on volunteers. Others including Johnson \& Johnson
are going to launch trials of their own in the coming months.

Some groups, including the American company Inovio Pharmaceuticals, are
taking a totally different approach. Instead of injecting viruses or
protein fragments, they're injecting pure DNA, which is read by the
cell's machinery, making a copy as an RNA molecule. The RNA is then read
by the cell's protein-building factories, making a viral protein. The
protein in turn comes out of the cell, where immune cells bump into it
and make an antibody to it.

Other teams are creating RNA molecules rather than DNA. Moderna and a
group at Imperial College London have launched safety trials for RNA
vaccines. While experimental, these genetic vaccines can be quickly
designed and tested.

\hypertarget{designing-versus-manufacturing}{%
\subsection{Designing Versus
Manufacturing}\label{designing-versus-manufacturing}}

It is one thing to design a vaccine in record time. It is an entirely
different challenge to manufacture and distribute one on a scale never
before attempted --- billions of doses, specially packaged and
transported at below-zero temperatures, to nearly every corner of the
world.

``If you want to give a vaccine to a billion people, it better be very
safe and very effective,'' said Dr. Stoffels of Johnson \& Johnson.
``But you also need to know how to make it in amounts we've never really
seen before.''

So the race is on to get ahead of the enormous logistical issues, from
basic manufacturing capacity to the shortages of medical glass and
stoppers that Mr. Gates and others have warned of.

Researchers at Johnson \& Johnson are trying to make a five-dose vial to
save precious glass, which might work if a smaller dose is enough for
inoculation.

Each potential vaccine will require its own customized production
process in special ``clean'' facilities for drug making. Building from
scratch might cost tens of millions of dollars per plant. Equipping one
existing facility could easily cost from \$5 million to \$20 million.
Ordering and installing the necessary equipment can take months.

Governments as well as organizations like the Gates Foundation and the
nonprofit Coalition for Epidemic Preparedness Innovations are putting up
money for production facilities well before any particular vaccine is
proven effective.

What's more, some vaccines --- including those being tested by the
American companies Moderna and Inovio --- rely on technology that has
never before yielded a drug that was licensed for use or mass-produced.

But even traditional processes face challenges.

Because of staff illnesses and social distancing, the pandemic this
spring slashed productivity by 20 percent at the MilliporeSigma facility
in Danvers, Mass., that supplies many drug makers with the equipment
used for brewing vaccines.

Then, about three weeks ago, the first clinical trials for new proposed
vaccines started. Urgent calls poured from customers around the world.
Even before the first phase of the first trials, manufacturers were
scrambling.

``Demand went through the roof, and everybody wanted it yesterday,''
said Udit Batra, MilliporeSigma's chief executive, who has expanded
production and asked other customers to accept delays to avoid becoming
a bottleneck.

\hypertarget{treatments-versus-vaccines}{%
\subsection{Treatments Versus
Vaccines}\label{treatments-versus-vaccines}}

Image

Doctors treating a patient infected with Covid-19 in the intensive care
unit of the Brooklyn Hospital Center. Some experts are more optimistic
about new treatments for sick patients than potential
vaccines.Credit...Victor J. Blue for The New York Times

Even as the world waits for a vaccine, a potential treatment for
coronavirus is already here --- and more could be on the way.

On Friday,
\href{https://www.nytimes3xbfgragh.onion/2020/05/01/health/coronavirus-remdesivir.html}{the
Food and Drug Administration granted emergency authorization} for the
use of remdesivir as a treatment of severely ill patients.

Remdesivir showed modest success in a federally funded clinical trial,
slowing the progression of the disease, but without significantly
reducing fatality rates.

The F.D.A.'s decision to allow its use comes as hundreds of other drugs
--- mainly existing medicines that are being used for other conditions
--- are being tested around the world to see if they hold promise. The
F.D.A. said there are currently
\href{https://www.fda.gov/drugs/coronavirus-covid-19-drugs/coronavirus-treatment-acceleration-program-ctap}{72
therapies} in trial.

Studies of drugs tend to move more quickly than vaccine trials. Vaccines
are given to millions of people who are not yet ill, so they must be
extremely safe. But in sicker people, that calculus changes, and side
effects might be an acceptable risk.

As a result, clinical trials can be conducted with fewer people. And
because drugs are tested in people who are already sick, results can be
seen more quickly than in vaccine trials, where researchers must wait to
see who gets infected.

Public health experts have cautioned there will likely be no magic pill.
Rather, they are hoping for incremental advances that make Covid-19 less
deadly.

``Almost nothing is 100 percent, especially when you are dealing with a
virus that really creates a lot of havoc in the body,'' said Dr. Luciana
Borio, a former director of medical and biodefense preparedness for the
National Security Council under President Trump.

Antiviral drugs like remdesivir battle the virus itself, slowing its
replication in the body.

The malaria drug hydroxychloroquine --- which has been
\href{https://www.nytimes3xbfgragh.onion/2020/04/06/us/politics/coronavirus-trump-malaria-drug.html}{enthusiastically
promoted} by Mr. Trump and also received emergency authorization to be
used in coronavirus patients --- showed early promise in the laboratory.
However, small, limited studies in humans
\href{https://www.nytimes3xbfgragh.onion/2020/04/24/health/fda-hydroxychloroquine-coronavirus.html}{have
so far been disappointing}.

So have some H.I.V. treatments, including a two-drug cocktail sold as
Kaletra,
\href{https://www.nytimes3xbfgragh.onion/2020/03/18/health/coronavirus-antiviral-drugs-fail.html}{which
failed in a Chinese trial}.

Image

Daniel O'Day, the chief executive of Gilead Sciences, the manufacturer
of remdesivir, in the Oval Office on Friday.Credit...Erin Schaff/The New
York Times

Other researchers have focused on identifying immunosuppressant drugs
that address the most severe form of Covid-19,
\href{https://www.nytimes3xbfgragh.onion/2020/04/01/health/coronavirus-cytokine-storm-immune-system.html}{when
the body's immune system goes into overdrive}, attacking the lungs and
other organs.

Many in the medical community are closely watching the development of
antibody drugs that could act to neutralize the virus, either once
someone is already sick or as a way of blocking the infection in the
first place.

Several hospitals
\href{https://slack-redir.net/link?url=https\%3A\%2F\%2Fwww.nytimes3xbfgragh.onion\%2F2020\%2F03\%2F26\%2Fhealth\%2Fplasma-coronavirus-treatment.html}{are
also administering plasma from recovered patients} to people who are
sick with Covid-19, in the hopes that the antibodies of survivors will
give the patients a boost.

Dr. Scott Gottlieb, a former F.D.A. commissioner, and others said that
by the fall, the treatment picture for Covid-19 could look more hopeful.

If proven effective in further trials, remdesivir may become more widely
used. One or two antibody treatments may also become available,
providing limited protection to health care workers.

Even without a vaccine, Dr. Borio said, a handful of early treatments
could make a difference. ``If you can protect people that are vulnerable
and you can treat people that come down with the disease effectively,''
she said, ``then I think it will change the trajectory of this
pandemic.''

David E. Sanger reported from Washington, David D. Kirkpatrick from
London, Carl Zimmer and Katie Thomas from New York and Sui-Lee Wee from
Singapore. Denise Grady and Maggie Haberman contributed reporting.

Advertisement

\protect\hyperlink{after-bottom}{Continue reading the main story}

\hypertarget{site-index}{%
\subsection{Site Index}\label{site-index}}

\hypertarget{site-information-navigation}{%
\subsection{Site Information
Navigation}\label{site-information-navigation}}

\begin{itemize}
\tightlist
\item
  \href{https://help.nytimes3xbfgragh.onion/hc/en-us/articles/115014792127-Copyright-notice}{©~2020~The
  New York Times Company}
\end{itemize}

\begin{itemize}
\tightlist
\item
  \href{https://www.nytco.com/}{NYTCo}
\item
  \href{https://help.nytimes3xbfgragh.onion/hc/en-us/articles/115015385887-Contact-Us}{Contact
  Us}
\item
  \href{https://www.nytco.com/careers/}{Work with us}
\item
  \href{https://nytmediakit.com/}{Advertise}
\item
  \href{http://www.tbrandstudio.com/}{T Brand Studio}
\item
  \href{https://www.nytimes3xbfgragh.onion/privacy/cookie-policy\#how-do-i-manage-trackers}{Your
  Ad Choices}
\item
  \href{https://www.nytimes3xbfgragh.onion/privacy}{Privacy}
\item
  \href{https://help.nytimes3xbfgragh.onion/hc/en-us/articles/115014893428-Terms-of-service}{Terms
  of Service}
\item
  \href{https://help.nytimes3xbfgragh.onion/hc/en-us/articles/115014893968-Terms-of-sale}{Terms
  of Sale}
\item
  \href{https://spiderbites.nytimes3xbfgragh.onion}{Site Map}
\item
  \href{https://help.nytimes3xbfgragh.onion/hc/en-us}{Help}
\item
  \href{https://www.nytimes3xbfgragh.onion/subscription?campaignId=37WXW}{Subscriptions}
\end{itemize}
