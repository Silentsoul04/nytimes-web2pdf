Will Americans Lose Their Right to Vote in the Pandemic?

\url{https://nyti.ms/2KZYqft}

\begin{itemize}
\item
\item
\item
\item
\item
\item
\end{itemize}

\begin{itemize}
\item
  \href{https://www.nytimes3xbfgragh.onion/2020/09/12/us/politics/biden-trump-poll-wisconsin-minnesota.html?action=click\&pgtype=Article\&state=default\&region=TOP_BANNER\&context=storylines_menu}{New
  York Times Poll}
\item
  \href{https://www.nytimes3xbfgragh.onion/interactive/2020/us/elections/election-states-biden-trump.html?action=click\&pgtype=Article\&state=default\&region=TOP_BANNER\&context=storylines_menu}{Paths
  to 270}
\item
  \href{https://www.nytimes3xbfgragh.onion/interactive/2019/us/elections/2020-presidential-election-calendar.html?action=click\&pgtype=Article\&state=default\&region=TOP_BANNER\&context=storylines_menu}{Voting
  Deadlines}
\item
  \href{https://www.nytimes3xbfgragh.onion/interactive/2020/08/31/us/politics/vote-by-mail-deadlines.html?action=click\&pgtype=Article\&state=default\&region=TOP_BANNER\&context=storylines_menu}{Voting
  by Mail}
\item
  \href{https://www.nytimes3xbfgragh.onion/newsletters/politics?action=click\&pgtype=Article\&state=default\&region=TOP_BANNER\&context=storylines_menu}{Politics
  Newsletter}
\end{itemize}

\includegraphics{https://static01.graylady3jvrrxbe.onion/images/2020/05/10/magazine/10mag-Voting-image2/10mag-Voting-image2-articleLarge.jpg?quality=75\&auto=webp\&disable=upscale}

Sections

\protect\hyperlink{site-content}{Skip to
content}\protect\hyperlink{site-index}{Skip to site index}

Feature

\hypertarget{will-americans-lose-their-right-to-vote-in-the-pandemic}{%
\section{Will Americans Lose Their Right to Vote in the
Pandemic?}\label{will-americans-lose-their-right-to-vote-in-the-pandemic}}

The safest way to cast a ballot will very likely be by mail. But with
opposition from the president, limited funding and time running out,
will that option be available?

Credit...Illustration by Pablo Delcan and Lisa Sheehan

Supported by

\protect\hyperlink{after-sponsor}{Continue reading the main story}

By \href{https://www.nytimes3xbfgragh.onion/by/emily-bazelon}{Emily
Bazelon}

\begin{itemize}
\item
  Published May 5, 2020Updated July 21, 2020
\item
  \begin{itemize}
  \item
  \item
  \item
  \item
  \item
  \item
  \end{itemize}
\end{itemize}

In March, as a wave of states began delaying their spring primaries
because of the coronavirus, Wisconsin's election, scheduled for April 7,
loomed. The ballot for that day included the presidential
\href{https://www.nytimes3xbfgragh.onion/2020/06/09/us/politics/atlanta-voting-georgia-primary.html}{primary},
thousands of local offices and four judgeships, including a key seat on
the Wisconsin Supreme Court. On March 17, the day after Ohio postponed
its spring election,
\href{https://www.nytimes3xbfgragh.onion/2020/07/21/us/john-lewis-voting-rights-act.html}{voting
rights} groups asked Wisconsin's Democratic governor, Tony Evers, to do
the same. ``No one wanted the election to happen more than us, but it
felt like this wave was about to hit our communities,'' Angela Lang, the
founder and executive director of the Milwaukee group Black Leaders
Organizing for Community, a nonprofit organization, told me.

While Evers weighed the idea of postponement, BLOC encouraged residents
to apply for absentee ballots, which any registered Wisconsin voter can
do by requesting one online. But some voters were struggling to figure
out how to upload their identification from their phones to the state's
MyVote website. City officials reported that they couldn't keep up with
the overwhelming demand for absentee ballots; applications in Milwaukee
rose from a typical daily count of 100 or so to between 7,000 and 8,000.
``People were waiting on their ballots and asking where they were,''
Lang said. ``We needed a plan. But we knew the governor was in a tough
position with the Legislature.''

The Wisconsin Assembly and Senate are firmly in the hands of
Republicans,
\href{https://www.nytimes3xbfgragh.onion/2017/08/29/magazine/the-new-front-in-the-gerrymandering-wars-democracy-vs-math.html}{who
drew a gerrymandered map a decade ago} that has allowed them to retain a
majority in the State Assembly even though they won only 47 percent of
the vote in 2012 and less than 45 percent in 2018. Lang, who is 30, grew
up in the city and started BLOC to increase political engagement --- and
power --- in Milwaukee's mostly black and low-income neighborhoods. And
Evers won in 2018 (defeating Scott Walker, a Republican seeking a third
term)
\href{https://www.theatlantic.com/politics/archive/2018/11/black-and-latino-turnout-helped-defeat-scott-walker/575818/}{thanks
in part to larger-than-usual turnout by black and Latino voters.}

It wasn't clear whether the governor had the legal authority to suspend
the election, and at the end of March, rather than calling for a
postponement, Evers asked the Legislature to send
\href{https://www.nytimes3xbfgragh.onion/2020/06/10/us/politics/voting-by-mail-georgia.html}{mail-in
ballots} to every registered voter, regardless of whether they had
applied for one. The Senate majority leader, Scott Fitzgerald, ridiculed
the idea as a ``complete fantasy.''

On March 26, BLOC and several other groups joined a lawsuit that argued
for postponing the election because local officials would find it
``functionally impossible'' to conduct it properly. The suit was one of
three election-related cases in Wisconsin that were consolidated before
U.S. District Judge William Conley. On April 2,
\href{https://madison.com/wsj/news/local/govt-and-politics/ill-advised-election-to-go-on-amid-covid-19-pandemic-judge-says-but-some-absentee/article_8dd80672-af4b-5bc6-b25b-f29b92a78382.html}{Conley
ruled} that while he recognized that an election on April 7 would create
``unprecedented burdens'' for voters, poll workers and the state, the
court could not change the date in lieu of the governor and the
Legislature. Instead, Judge Conley extended the deadline for voters to
return their absentee ballots to April 13, citing the testimony of local
officials that otherwise there would be no way for all the voters asking
to vote by mail to receive and return their ballots in time.

The State Legislature, the state Republican Party and the Republican
National Committee immediately appealed Conley's ruling. The next day,
April 3, Evers called the Legislature into special session. The governor
said he didn't have the power to postpone the election on his own,
demanding instead that lawmakers cancel in-person voting and extend the
mail-in deadline to late May. The governor's political opponents
rejected his request.

As the days ticked by, Milwaukee announced that it could open only five
of its 180 polling places, as poll workers --- many of whom were over
the age of 60 and at heightened risk from the virus --- pulled out of
staffing them. Green Bay said it could open two of its 31 polling sites.
Election officials rushed out absentee ballots with instructions about
the new April 13 deadline set by Judge Conley, and BLOC reached out to
voters by phone and text, explaining that they would have six extra days
to turn in their ballots.

On April 6, the day before the election, Evers issued an executive order
postponing it for two months, despite his earlier statement that he
lacked this authority. That day, the Wisconsin Supreme Court
\href{https://www.wicourts.gov/news/docs/2020AP608_2.pdf}{blocked the
governor's order by a 4-to-2 vote.} (The seventh justice, whose seat was
up for election, recused himself.) The conservative majority said that
the governor's authority by law to issue orders ``he or she deems
necessary for the security of persons and property'' didn't mean he
could override other valid laws, including those governing elections.

Later that evening, the U.S. Supreme Court voted 5 to 4 along
ideological lines and reversed Judge Conley's decision to extend the
deadline to return mail-in ballots, changing the date back to April 7.
The court's
\href{https://www.supremecourt.gov/opinions/19pdf/19a1016_o759.pdf}{unsigned
majority opinion} made no provision for the extraordinary circumstances
of the coronavirus. It didn't mention the people who hadn't yet received
their ballots, or those who had received instructions with the April 13
return date. That meant voters still awaiting ballots on April 7 ---
more than 12,000 statewide, according to preliminary data --- had to
choose between braving their polling places or sitting out the election.

\includegraphics{https://static01.graylady3jvrrxbe.onion/images/2020/05/10/magazine/10Voting-mag-02/10Voting-mag-02-articleLarge.jpg?quality=75\&auto=webp\&disable=upscale}

On Election Day, people stood in lines that wrapped around the block,
trying to keep their distance from one another. Robin Vos, the
Republican leader of the State
Assembly,\href{https://www.facebookcorewwwi.onion/watch/?v=519791292264353}{went
on Facebook Live} while wearing a mask, gloves and full-body protective
gear and assured voters that it was ``incredibly safe'' to go to the
polls. One voter tweeted about her sister, a cancer survivor who was
afraid to go out and expose herself to the virus but whose absentee
ballot hadn't arrived. ``The hardest was hearing from people who said
they marched in the civil rights era and now they couldn't vote,'' Lang
said. For days after the election, Milwaukee residents continued to take
their ballots to library drop-off sites, following the instructions they
received that extended the deadline to April 13. They would not be
counted.

In the end, the liberal candidates won in the three judicial races on
the ballot in which BLOC took a position. Lang didn't feel like
celebrating, though --- she was worried that people who went to the
polls would wind up getting sick. In the weeks after the election,
Milwaukee health officials traced
\href{https://www.wuwm.com/post/40-coronavirus-cases-milwaukee-county-linked-wisconsin-election-health-official-says\#stream/0}{at
least 40 cases} of the virus to in-person voting.

The election in Wisconsin shows that the coronavirus can block access to
the ballot just as it has closed stores and schools and so much other
civic activity. ``Ultimately there were no provisions, no accommodations
in state law for the pandemic when it came to our administration of this
election,'' says Neil Albrecht, executive director of the Milwaukee
Election Commission. If states and the federal government don't do more
to help voters in November --- starting now, with urgency --- the
barriers for some of them may be insurmountable. ``A lot of people
suffered because of the government's lack of responsiveness,'' Albrecht
adds. ``What I mean is, they lost their right to vote.''

\textbf{A national election} is a giant pop-up event, larger in scale
and significance than any other private or public occasion. Two-thirds
of Americans expect the Covid-19 outbreak to disrupt voting in November,
according to a
\href{https://www.people-press.org/2020/04/28/two-thirds-of-americans-expect-presidential-election-will-be-disrupted-by-covid-19/}{late-April
survey by the Pew Research Center.} A successful election will require
some Covid-era changes. The main one is enabling tens of millions more
people to
\href{https://www.nytimes3xbfgragh.onion/2020/05/25/us/vote-by-mail-coronavirus.html}{vote
by mail} (also called absentee balloting --- the terms are synonymous)
than have ever done so before. It's also important to make adjustments
to keep polling places open for people who don't have stable mailing
addresses --- a group that increases as people are uprooted during an
economic downturn --- or whose disabilities, like blindness, make it
hard to fill out a ballot unassisted.

The outcome of the presidential contest will most likely be decided in a
handful of swing states. This year, the likeliest prospects are
Wisconsin, Michigan, Pennsylvania, Florida, North Carolina and Arizona.
All of them, along with 23 other states and the District of Columbia,
already have laws on the books that give voters the right to request an
absentee ballot without an excuse. But only one swing state is already
set up for most people to vote by mail --- Arizona, where 79 percent did
so in 2018. In Florida and Michigan, about 25 to 30 percent voted by
mail that year. In Wisconsin, Pennsylvania and North Carolina, very few
voters have voted absentee in a general election; in 2018, the range was
from 3 to 6 percent,
\href{https://www.brennancenter.org/our-work/research-reports/preparing-your-state-election-under-pandemic-conditions}{according
to The Brennan Center for Justice at New York University Law School.}(A
total of 27 states fell below 10 percent, including Georgia and New
Hampshire, which also may see close presidential results.)

To fundamentally change the way voting has been done in those states,
they will have to move quickly to sign contracts with vendors and then
order supplies, like specially certified paper for envelopes and
ballots, high-speed scanners to count votes and secure drop-off boxes.
If they wait, they'll risk running into shortages like the ones that
have troubled the country's efforts to fight the virus. In Wisconsin in
April, when voting by mail rose to more than 70 percent, totaling over a
million, from around 6 percent in previous elections, many people didn't
get to vote because counties ran out of envelopes for a time and then
couldn't fill all the applications for absentee ballots fast enough.
``Wisconsin shows that you can't adopt vote-by-mail overnight,'' says
Nathaniel Persily, a Stanford law professor and the head of the Healthy
Elections Project, a new effort by Stanford and the Massachusetts
Institute of Technology to address the threat of Covid-19. ``It's not as
easy as people think. The boring stuff matters --- the scut work of
supply chain and logistics and management is crucial.''

Significantly changing how elections are carried out will cost money,
and all states face a giant funding gap as they scramble to prepare for
the unknowns of November.
\href{https://www.brennancenter.org/our-work/research-reports/estimated-costs-covid-19-election-resiliency-measures}{The
Brennan Center for Justice estimates} the pandemic-associated costs of
properly running the 2020 elections (including the primaries as well as
the general) at \$4 billion. So far, Congress has promised \$400
million, with Democrats pushing for more and Republicans blocking their
bills. The debate over funding the Postal Service, which
\href{https://www.nytimes3xbfgragh.onion/2020/04/14/opinion/usps-coronavirus.html}{warns
it could run out of operating funds at the end of September}, is
similarly split.

In a different world, preparation for the election and its accompanying
costs would be nonpolitical. \href{https://www.voteathome.org/}{Five
states currently have universal vote-by-mail}, the system of sending all
registered voters a ballot without requiring them to request one first:
Utah, dominated by Republicans; Hawaii, Oregon and Washington, where
Democrats tend to win; and Colorado, where members of both parties hold
major statewide offices.
\href{https://www.reuters.com/article/us-usa-election-poll-idUSKBN21P3G0}{A
Reuters poll in April found} that 72 percent of Americans want the
government to require mail-in ballots in November to protect voters if
the coronavirus continues to pose a threat, including 65 percent support
among Republicans. Some Republican officials share the majority view: In
Ohio, Gov. Mike DeWine and Secretary of State Frank LaRose
\href{https://www.youtube.com/watch?v=gYliBlyROxo\&feature=youtu.be}{made
a video promoting the state's first primary by mail in June.} ``I wanted
to see as much participation as we could get,'' LaRose told me. Chris
Sununu, the Republican governor of New Hampshire,
\href{https://www.washingtonpost.com/politics/new-hampshire-gov-sununu-to-allow-absentee-voting-in-november-because-of-coronavirus-outbreak/2020/04/09/d0aa21c8-7aa2-11ea-a130-df573469f094_story.html}{promised
voting by mail for all in November}, if the coronavirus is still an
issue, despite the state's usual rule that voters can only receive an
absentee ballot if they have an excuse like travel or illness.

Researchers have found that vote-by-mail hasn't obviously helped one
party or the other. Nationwide, about the same share of Republicans and
Democrats voted by mail in 2016, Charles Stewart III, a
political-science professor at M.I.T., found. In partisan terms, ``it is
remarkably neutral,''
\href{https://twitter.com/andrewbhall/status/1250104293866098688}{wrote
Andrew Hall}, a political-science professor at Stanford University and
\href{http://www.andrewbenjaminhall.com/Thompson_et_al_VBM.pdf}{an
author of a 2020 study} (which hasn't yet been published) on voting by
mail. Hall's study found that shifting to mailed ballots has modestly
increased turnout --- by about 2 percent --- for each party;
\href{https://isps.yale.edu/research/publications/isps13-012}{a 2013
study} found similar results.

Image

Dunn, Wis.Credit...John Hart/Wisconsin State Journal, via Associated
Press

But even if
\href{https://www.nytimes3xbfgragh.onion/2020/04/10/us/politics/vote-by-mail.html}{vote-by-mail
hasn't hurt them}, conservatives have long focused on increased turnout
as a threat and have worked to minimize it. In the days of Jim Crow,
conservatives in the South (who were then generally Democrats) used the
blunt tools of poll taxes and literacy tests to prevent
African-Americans from voting. In the decades after the Voting Rights
Act of 1965 stamped out those forms of overt suppression,
\href{https://www.nytimes3xbfgragh.onion/2015/07/29/magazine/voting-rights-act-dream-undone.html}{newly
elected black legislators and their allies increased registration} with
state laws that let people register at the Department of Motor Vehicles
and public-assistance offices, or register at the polls on the same day
they voted. They also increased access by opening polling sites in the
weeks before Election Day.

Republicans generally opposed these efforts. ``I don't want everybody to
vote,'' Paul Weyrich, the conservative activist and co-founder of the
Heritage Foundation, said at a meeting in Dallas in 1980. ``As a matter
of fact, our leverage in the elections quite candidly goes up as the
voting populace goes down.'' In the 2000s,
\href{https://www.ncsl.org/research/elections-and-campaigns/voter-id-history.aspx}{Republicans
began passing strict voter-identification laws}, which could be
justified as a way to prevent fraud --- though in-person voting fraud is
extremely rare. In 2010, after taking control of most state
legislatures, Republicans eliminated early voting and same-day
registration where they could. Since
\href{https://www.nytimes3xbfgragh.onion/2016/11/07/magazine/the-supreme-court-ruled-that-voting-restrictions-were-a-bygone-problem-early-voting-results-suggest-otherwise.html}{the
Supreme Court effectively gutted} a
\href{http://www.slate.com/articles/news_and_politics/jurisprudence/2013/02/the_voting_rights_act_and_shelby_county_v_holder_how_the_supreme_court_could.html}{key
provision} of the Voting Rights Act in 2013,
\href{http://civilrightsdocs.info/pdf/reports/Democracy-Diverted.pdf}{more
than 1,600 polling places have been closed} across the country.

Trump benefited from decreased turnout in 2016, especially in the vital
swing states of Wisconsin, Pennsylvania and Michigan, where
participation by black and Democratic voters declined from the historic
levels that lifted Barack Obama. Wisconsin's voter-ID law accounted for
some of the decline in turnout in Milwaukee,
\href{https://www.motherjones.com/politics/2017/10/voter-suppression-wisconsin-election-2016/}{according
to Neil Albrecht}, the city election director.

In March, Trump announced his opposition to a Democratic bid to include
at least \$2 billion for state election preparation in the \$2 trillion
coronavirus relief bill. Republicans usually don't talk openly about
suppressing turnout in the way that Paul Weyrich did 40 years ago. Trump
broke that rule, saying at a news briefing that he thought his party
would lose if more people voted. The Democrats' proposals, he said,
``had things --- levels of voting that, if you ever agreed to it, you'd
never have a Republican elected in this country again.''

In the weeks that followed, Trump shifted to the preferred Republican
justification for making it harder to vote --- preventing fraud. With
the threat of the pandemic rising, he called voting by mail ``corrupt,''
imagining ``thousands of votes are gathered, and they come in, and
they're dumped in a location, and then, all of a sudden, you lose
elections you think you're going to win.'' In some states, Republicans
following Trump's messaging
\href{https://www.nytimes3xbfgragh.onion/2020/04/10/us/politics/vote-by-mail.html}{have
denounced vote-by-mail} as ``devastating to Republicans'' (David
Ralston, the Republican speaker of the Georgia House), ``the
apocalypse''
(\href{https://www.theatlantic.com/politics/archive/2020/04/voting-mail-2020-race-between-biden-and-trump/609799/}{Jennifer
Carnahan, chairwoman of the Minnesota Republican Party}) and ``the end
of our republic as we know it''
(\href{https://twitter.com/RepThomasMassie/status/1242573156776378371}{Representative
Thomas Massie of Kentucky}).

In February, the Trump campaign and the
\href{https://www.politico.com/news/2020/02/20/michigan-voting-lawsuits-rnc-trump-116142}{Republican
National Committee announced} they would spend \$10 million on
litigation and election monitoring in the 2020 cycle. Soon after, legal
attacks on expanding vote-by-mail began. In March,
\href{https://www.krwg.org/post/new-mexico-republicans-file-lawsuit-have-supreme-court-block-mail-primary}{the
Republican Party in New Mexico sued} to prevent 27 county clerks from
shifting to vote-by-mail for the June primary. In April, three voters
affiliated with the conservative group
\href{https://thenevadaindependent.com/article/another-lawsuit-filed-against-planned-mail-only-primary-election-by-conservative-voting-monitoring-group}{True
the Vote filed a lawsuit to stop Nevada} from conducting an all-mail
primary election planned by the secretary of state.
(\href{https://www.courthousenews.com/wp-content/uploads/2020/04/order.pdf}{A
federal court rejected the suit at the end of the month}, calling its
claim of voter fraud ``without any factual basis.'') In Texas,
\href{https://www.texasattorneygeneral.gov/sites/default/files/images/admin/2020/Press/4.14.20\%20Letter\%20to\%20Rep.\%20Klick.pdf}{Attorney
General Ken Paxton interpreted the state law} that requires an excuse
like illness for absentee voting to mean that a voter must actually be
sick rather than simply be concerned about becoming infected. Paxton
threatened ``criminal sanctions'' for anyone advising voters to apply
for a mail-in ballot based ``solely on fear of contracting Covid-19.''
When a state judge ruled in April that all Texas registered voters could
qualify for an absentee ballot because of the pandemic, Paxton appealed
the ruling, leaving the matter in limbo.

Before the coronavirus, the 2020 election was already vulnerable to
disinformation campaigns, foreign interference and the country's
increasing polarization. The pandemic creates other challenges. In a
nightmare scenario, officials could use the virus as an excuse to shut
the polls selectively, to the benefit of their party. Or state
legislatures could invoke the power the Constitution gives them to
choose the electors who cast votes in the Electoral College, and thus
actually select the president. (The states turned this power over to the
voters in the 19th century, but they could try to take it back.) Any
move like that would surely land in the Supreme Court, which has its own
deepening groove of ideological division --- and the dubious history of
Bush v. Gore, the case in which the court intervened to effectively
decide the outcome of the 2000 election.

With six months to go until the election (the date, the Tuesday after
the first Monday in November,
\href{https://www.loc.gov/law/help/statutes-at-large/28th-congress/session-2/c28s2ch1.pdf}{is
set by an 1845 law}, and both houses of Congress would have to agree to
change it) the chances of a breakdown in its administration seem high.
And this is a year when accusations of a stolen or broken election have
more potential than they've had for decades to rip the country apart.
It's hard to overstate the importance of seeing the election done right.
``It's this simple: A disputed election in this environment poses an
existential threat to American democracy,'' Persily says. ``It is that
serious.''

\textbf{Wisconsin shows how} politically divisive basic access to voting
could be in November. Three other swing states --- Michi­gan,
Pennsylvania and North Carolina --- have the same kind of divided
government, with Democratic governors and Republican-led legislatures
wrestling for control, the dynamic that caused so much trouble in April.
Wisconsin, Michigan and Pennsylvania also have major cities (Milwaukee,
Detroit and Philadelphia) where African-Americans could play a decisive
role in the election and have also suffered disproportionate Covid-19
infections and deaths. The combination could especially imperil their
constitutional right to vote.

The cities and counties of Wisconsin are learning from their experience
in April. State officials can advise them on preparing for the pandemic,
but it's the local clerks and commissioners who have to make the
logistics work. In Milwaukee, the City Council responded to the chaos
and disenfranchisement by passing a resolution asking Albrecht, the
election director, to send Milwaukee's 300,000 registered voters an
application for a mail-in ballot for November. Albrecht told me he would
spend the summer overhauling operations. ``I'm talking about all of
it,'' he said. He has submitted a request to the Postal Service for an
investigation. Many Milwaukee voters who applied for absentee ballots on
two particular dates, March 22 and 23, did not receive them. ``Our
forensic review shows we responded and sent them out,'' Albrecht said.
``Did the post office mess up? We don't know.'' Albrecht is also making
sure he has the supplies of paper for added ballots and envelopes that
he needs. Finally, Albrecht said, he is concentrating on voter
education. People who were accustomed to going to the polls made
mistakes, like dropping ballots through the book-drop slot at the
library without the certified envelope, which disqualified their votes.

In Pennsylvania, the presidential primary scheduled for June 2 will be
the first test of whether large numbers of people can successfully vote
by mail.
\href{https://triblive.com/news/pennsylvania/sweeping-changes-to-pennsylvania-election-law-could-make-early-voting-a-norm/}{The
Legislature last year passed a law} that provides for absentee ballots
for anyone who requests it without requiring an excuse. ``We've had
160,000 applications for mail-in ballots for the primary in the last
week,'' Secretary of the Commonwealth Kathy Boockvar said when I spoke
to her in mid-April. ``For comparison, in 2016, we got 19,000 in the
same period.'' She stressed that federal funding would be crucial for
preparing for November. In the long run, voting by mail can be less
expensive. Counties that adopted it in Colorado, one of five states that
sends ballots by mail to every registered voter,
\href{https://www.pewtrusts.org/en/research-and-analysis/issue-briefs/2016/03/colorado-voting-reforms-early-results}{spent
less than \$10 per voter in 2014}compared with about \$16 per voter six
years earlier. But in the present, states need help to make the switch.

Three elected city commissioners are responsible for directing the
logistics in Philadelphia. ``To be honest, everything we were planning
to do for November is on hold as we navigate through the virus,'' Lisa
Deeley, one of the three commissioners and the commission's chairwoman,
said when I called her in April. ``All our focus right now is on the
primary.''

A few days later, the National Association of Presort Mailers held a
teleconference for vendors across the country that are in the niche
business of printing and packaging bulk mail, including mail-in ballots.
They specialize in details like ensuring that the paper for the ballots
and envelopes is certified so the ink printed on it will scan correctly.

On the call,
\href{https://talkingpointsmemo.com/news/mail-in-elections-covid-19-supply}{according
to the news site Talking Points Memo}, companies warned that they were
already at capacity for November, filling orders from longtime
vote-by-mail states like California and Colorado. They could expand, but
they would need to buy costly equipment that takes several months to
obtain, a step they would only take with orders from states and counties
in hand. ``For example, the machine that folds and inserts the ballot
into the envelope can cost up to \$1 million,'' Richard Gebbie, chief
executive of Midwest Presort Mailing Services and president of the
national association, told me. ``It normally takes 90 days to order one
piece of gear. Then you have to get it installed and check everything,
because the security and quality control has to be very, very high.''
Gebbie's company has been contacting county boards of election in the
region, including in Pennsylvania, but he says so far it has received a
cool response. ``I think with the Covid, they're not sure what they can
do. We have one county in Pennsylvania, Mercer, that said, Let's get a
quote. The others said, Call us back in a month. The Catch-22 is: That
could be too late.''

Deeley called me back later in April to assure me that Philadelphia
would be ready for the fall election but gave few specifics. ``Her heart
is in the right place, but this is just a huge challenge,'' says David
Thornburgh, the president and C.E.O. of the Committee of Seventy, a
good-government group in Philadelphia founded in 1904. ``We are at the
house-is-burning level of alarm in some cities,'' says another voting
rights advocate, who didn't want to be identified criticizing local
election officials. As of the end of April, Philadelphia had a backlog
of almost 9,000 absentee applications waiting to be processed for the
June primary. Voting rights advocates have filed a lawsuit asking the
Pennsylvania Supreme Court to require the state to let all absentee
ballots sent or postmarked by Election Day in June and November to be
counted if they are received within seven days of each election.

In Michigan, where voters passed a 2018 referendum that allows voting by
mail without an excuse, a big increase is also expected. ``We are
planning for 70 to 90 percent voting by mail in Detroit,'' Secretary of
State Jocelyn Benson, who lives in the city, told me. ``That means
allocating resources, ordering supplies, developing educational
materials.''

For a set of local elections throughout the state in May, Benson's
office is mailing applications for absentee ballots to all registered
voters, with return postage prepaid by the state. But Michigan doesn't
pay return postage for voters' ballots for either the primary or general
election. Stamps are a particular barrier for young people who have
grown up communicating digitally, elections officials say. Most other
states --- including Florida and Pennsylvania --- don't pay return
postage for applications or ballots. Mailing costs and other
Covid-19-related expenses for the general election (and another election
in August) would cost Michigan \$40 million, Benson estimates. The state
has so far only received \$11 million for all election expenses related
to the pandemic.

A coalition of more than 200 public-interest groups are pushing hard for
Congress to include \$3.6 billion for the 2020 election cycle in the
next coronavirus relief bill. They also want all states to offer online
and same-day voter registration and to extend in-person early voting to
avoid crowding on Election Day. Chuck Schumer, the Senate majority
leader, called the funding a top priority on an April conference call
with 20 civil rights groups. Some Republican secretaries of state, like
LaRose from Ohio, support additional funding, but don't want the federal
government to tell them how to run their elections. Some Republican
senators continue to see the funding proposal as an effort to give
Democrats an advantage.

Image

Voters in Milwaukee.Credit...Daniel Acker/Reuters

In the coming months, in the swing states and elsewhere, partisan fights
could break out over whether to allow voters to request an absentee
ballot online instead of by mail (many states currently don't allow
this), or waive the requirement that voters obtain witness signatures
before returning their ballots (as North Carolina and Wisconsin, among
others, mandate) because some voters are self-isolating during the
pandemic.

Absentee-ballot fraud, the recent focus of Republicans, has occasionally
taken place in isolated instances in states where low numbers of people
typically vote by mail. ``There's a history of tampering with absentee
ballots, mostly in pockets in Appalachia (including Kentucky), South
Texas and sometimes in cities with party machines,'' says Richard Hasen,
author of the recent book ``Election Meltdown'' and a law and
political-science professor at the University of California, Irvine. The
most prominent modern-day case of absentee fraud occurred in rural
Bladen County, N.C., in 2018. North Carolina, like a lot of states, bars
people from collecting and turning in absentee ballots of voters outside
their family. (Other states cap the number that people can collect.)
Nonetheless, in Bladen County, after Mark Harris, a Republican candidate
for Congress, won his election by 905 votes, evidence emerged that a
political operative working for him may have collected as many as 800
absentee votes, many from African-American voters, filled some of them
in for Harris and perhaps tossed others away. The bipartisan state Board
of Elections threw out the results and ordered a new election.

States that have adopted universal vote-by-mail have shown it can be
done securely. ``They have very strong track records,'' Hasen says.
Election officials create a clear, unhackable paper trail for ballots,
sending them to voters with a bar code that can be tracked. Voters must
sign the ballots, which means signatures can be checked, and send them
back in a certified inner envelope, also signed and also with a bar
code. ``The claim of fraud is a distraction,'' Jena Griswold, the
secretary of state in Colorado, where 95 percent of people voted by mail
in 2018, told me. ``We have a history of clean elections. When we think
there is the \emph{possibility} of double voting, we send every case to
the attorney general. Our number for 2018 was 0.0027 percent.''

One big question for 2020 is how states will verify absentee ballots to
guard against fraud while also ensuring that voters are treated fairly.
Many states lack uniform criteria or training for matching the signature
on a ballot with the copy of the voter's signature that the state has on
file. As a result, rejection rates can vary a great deal from county to
county. States including Pennsylvania and Michigan don't require
election officials to notify voters if their signatures are missing or
have been rejected, so those voters don't have a chance to fix the
problem. The gaps in the law leave the decision up to county and local
officials.

There are certain best practices. It's better for counties to use
databases that chart the evolution of voters' signatures over time
rather than relying on a registration file that may be decades old. In
Washington,
\href{https://crosscut.com/2020/04/washingtons-successful-vote-mail-system-wasnt-built-overnight}{which
instituted universal vote-by-mail in 2011}, state patrol officers who
investigate fraud train election workers on evaluating signatures,
according to Kim Wyman, the secretary of state. ``They teach us to look
at the slant of the letters or the path of how the signer moves the
pen,'' she says. ``After the training, you have more confidence that a
signature can be a match even if it's not identical.'' If a signature
fails a first check, it goes through another round of review and then to
a three-member elected canvassing board, which examines any flagged
ballots in a public session. ``You have to be open and transparent about
how you're verifying, or people will think you're just throwing out
Democratic or Republican votes to win,'' Wyman says.

It's also important to give voters clear instructions about filling out
mail-in ballots. ``We had to educate the voters, and we also had graphic
designers come in and help us,'' Wyman says. ``A lot is in the design
--- for example, putting a big red X with `sign here' next to the
signature line.'' The fate of thousands of ballots --- and the outcome
of a close election --- can depend on the choices states make. ``The
problem of uniform standards can be easily overcome,'' says Nathaniel
Persily, the Stanford election expert. ``But if states don't address it
ahead of time, you can imagine absentee signatures being the hanging
chads of 2020.''

\textbf{Before the pandemic,} candidates rarely focused on vote-by-mail
in their campaigns. One exception is Stacey Abrams, the Democratic
candidate for governor of Georgia in 2018. Her campaign sent 1.6 million
applications for absentee ballots to registered voters who signaled they
supported her. ``I think we were the first modern Democratic campaign to
run a really aggressive vote-by-mail operation,'' says Lauren
Groh-Wargo, who was Abrams's campaign manager. ``It was integrated with
our voter education, our ads, our field operation. We could track the
delivery of the absentee ballots and also whether they'd been returned.
We staffed a hotline to walk people through any issues they had filling
them out.''

Abrams won the absentee-ballot count by about 53,000 votes. But in the
end, her opponent, Brian Kemp, who was the Georgia secretary of state
responsible for managing elections during the race, defeated her by
close to 55,000 votes.

After the election, Abrams founded a voting rights group, Fair Fight
Action, which sued the state later that November, along with a
domestic-worker advocacy group, for suppressing the vote in several
ways. One of them involved absentee ballots. Election officials had
rejected thousands of them, often for errors like writing the date of
the election in the field for a birth date. Daniel Smith, a
political-science professor at the University of Florida, analyzed
Georgia's absentee-ballot data as an expert for Fair Fight Action in the
lawsuit. He found a higher rate of rejection for voters of color, who
tended to support Abrams, than for white voters.

Image

Poll workers in Kenosha recognized the danger on Election Day --- but
showed up regardless.Credit...Derek R. Henkle/Agence France-Presse, via
Getty Images

Georgia now has a new secretary of state, Brad Raffensperger, a
Republican who has raised the specter of fraud by announcing an
``Absentee-Ballot Fraud Task Force'' for 2020 that prosecutors will help
lead. The task force, nine of whose 12 members are Republican, would
investigate, among other things, ``every signature mismatch'' on a
mail-in ballot, Raffensperger said in a news conference. Groh-Wargo of
Fair Fight Action called the task force ``a submission to the Trump
voter-suppression machine.'' In her view, Raffensperger's intention is
clear: Intimidate and deter voters.

Republican officials have also increasingly pursued a practice that will
matter in November no matter how voters cast their ballots, because it
affects eligibility to vote by mail as well as in person --- mass cuts
to the voter-registration rolls. ``Purges in and of themselves aren't
bad,'' Kevin Morris and Myrna Pérez of the Brennan Center
\href{https://www.brennancenter.org/our-work/analysis-opinion/florida-georgia-north-carolina-still-purging-voters-high-rates}{wrote
in a 2018 analysis.} ``They're commonly used to clean up voter lists
when someone has moved, passed away and more. But too often, names
identified for removal are determined by faulty criteria that wrongly
suggests a voter be deleted from the rolls.'' Purging often
disproportionally shaves away black and Latino voters.

Before she ran for governor, Abrams worked for years to register
hundreds of thousands of new voters, many of them African-American and
Latino, hoping to make Georgia (where people of color make up 40 percent
of the population) more competitive for Democrats. Between 2016 and
2018, Kemp purged more than 700,000 registered voters, more than 10
percent of the state total. Most people Kemp cut hadn't responded to a
notice sent by the state after they didn't vote in the last few
elections. The state presumed the voters it cut from the rolls had moved
away or died, but in 2019,
\href{https://www.apmreports.org/story/2019/10/29/georgia-voting-registration-records-removed}{an
investigation by APM Reports} from Ameri­can Public Media estimated that
at least 107,000 of them remained eligible to vote. Like many states,
Georgia does not permit same-day registration, so people who show up to
vote and find they can't are not simply allowed back on the rolls.

Secretary Raffensperger purged another 309,000 voters in December (and
then restored 22,000 of them, saying they were eliminated in error).
Last year, Ohio took the unusual step of releasing to advocacy groups in
advance a list of 235,000 voters it planned to purge. A watchdog group
called the
\href{https://www.nytimes3xbfgragh.onion/2019/10/14/us/politics/ohio-voter-purge.html}{Ohio
Voter Project discovered} that about 40,000 voters were being cut in
error, about half of them from a heavily Democratic county with one of
the highest percentages of people of color in the state.

If the 2020 election is close, purges in swing states could shape the
results.
\href{https://www.brennancenter.org/our-work/analysis-opinion/florida-georgia-north-carolina-still-purging-voters-high-rates}{According
to the Brennan Center}, in the two years leading up to the 2018
election, North Carolina, which has a Republican Legislature and at the
time had a Republican governor, purged 11.7 percent of its voters; and
Florida, also a Republican-controlled state, purged more than 7 percent,
compared with 0.2 percent from 2008 to 2010. (In 2000, Florida's
wrongful purge of thousands of voters, a disproportionate number of whom
were black, probably contributed to George W. Bush's presidential
victory, according to the general counsel of the U.S. Commission on
Civil Rights at the time,
\href{https://www.thenation.com/article/archive/how-the-2000-election-in-florida-led-to-a-new-wave-of-voter-disenfranchisement/}{in
a 2015 article in The Nation}.) In Wisconsin, a legal battle over
purging voter rolls is continuing. Concerned about errors, state
election officials tried to delay cutting 234,000 voters they identified
as having changed addresses until after the November election. But a
conservative group, the Wisconsin Institute for Law and Liberty, sued to
force the state to make the cuts before voting takes place. The
Wisconsin Supreme Court deadlocked 3 to 3 over the case in March, with
the seventh justice declining to participate because he was the one who
was running in the April election. After he lost, he wrote that it
appeared that the reason for his recusal ``no longer obtains,''
signaling that he would rejoin the case, which could then be decided
before the newly elected liberal justice takes her seat on August 1.

\textbf{On the day} of Wisconsin's April election, photos of people
lining up at the Milwaukee polls, many of them African-American,
streamed through social media feeds and were featured in press reports.
The images reminded people that voting matters, that it's a right so
precious that your political opponents will try to prevent you from
exercising it. Three years ago in Alabama, after a divisive Senate
campaign, African-American voters turned out for the Democratic
candidate, Doug Jones, and achieved a higher share of the vote than they
did for Barack Obama in 2008 and 2012. When Jones won, activists took
pride with social media posts like ``\#BlackVotesMatter and don't anyone
tell you different.'' Regrouping with her organization in Milwaukee,
Angela Lang said she was hearing simi­lar determination. ``We talked to
an older woman, in her 70s, and I think she ended up not voting, but she
said, No matter what, I'm voting in November. People can see how
important it is to have a say in how decisions are made.''

When the results were announced several days later, they showed that
encouraging voting by mail could in fact help Democrats. The liberal
State Supreme Court candidate beat her conservative opponent by a margin
of 10 percentage points more in the absentee-ballot count than at
polling places.

A conservative advocacy group, the Honest Elections Project, responded
to the Wisconsin election by spending \$250,000 on an online ad that
blasted ``record absentee voting.'' The ad showed photos of long lines
of masked voters with the line, ``It's wrong,'' and then pivoted to a
``responsible solution,'' with a photo of elderly white people in a
sunny room: ``Vulnerable people protected with expanded absentee voting.
Fraud, prevented.'' The mixed messages illustrate the difficulty of
railing against voting-by-mail while also promoting it among the party's
supporters.

It is possible to hold a successful and orderly election during the
pandemic. In April, South Korea recorded the highest turnout, 66
percent, for a parliamentary election in 28 years. The government's
handling of the coronavirus --- far more successful than that of the
United States in reducing deaths and infections --- dominated the
political discourse. But on Election Day, people in masks calmly lined
up at the polls, moving step by step between lines of tape marking off
one-meter distances. Poll workers took their temperatures, and those
with a fever went to a separate area to vote. Voters received hand
sanitizer and disposable gloves before entering the booths. People who
were self-quarantining received a text from the government permitting
them to leave their homes for 1 hour 40 minutes to vote at 6 p.m., when
the polls were closed to everyone else. Only about 40 percent of voters
cast their ballots early or by mail.

The United States prides itself on its democracy in theory --- but this
year, not necessarily in practice. What if Philadelphia runs out of
absentee ballots? What if a swing state can't count its avalanche of
mail-in ballots on election night, and the media races to call a winner,
and then the final tabulation changes it --- and then there's a dispute
over signature-matching? The 2020 results may well be too early to call
for days. A candidate who warns now about fraud and chaos, as Trump is
ceaselessly doing, is sowing the seeds for his supporters to distrust
the results if he loses.

``You've heard the election administrator's prayer, right?'' Persily
asked me. ``Whatever happens, dear Lord, please let it not be close.''

\hypertarget{our-2020-election-guide}{%
\section{Our 2020 Election Guide}\label{our-2020-election-guide}}

Updated ~Sept. 12, 2020

\begin{center}\rule{0.5\linewidth}{\linethickness}\end{center}

\begin{itemize}
\item ~
  \hypertarget{the-latest}{%
  \subsection{The Latest}\label{the-latest}}

  \begin{itemize}
  \item
    President Trump has failed to erase Joseph R. Biden Jr.'s lead
    across a set of key swing states,
    \href{https://www.nytimes3xbfgragh.onion/2020/09/12/us/politics/biden-trump-poll-wisconsin-minnesota.html?action=click\&pgtype=Article\&state=default\&region=BELOW_MAIN_CONTENT\&context=storylines_guide}{according
    to a poll}~conducted by The Times and Siena College.
  \end{itemize}
\item ~
  \hypertarget{paths-to-270}{%
  \subsection{Paths to 270}\label{paths-to-270}}

  \begin{itemize}
  \item
    Joe Biden and Donald Trump need 270 electoral votes to reach the
    White House. Try building
    \href{https://www.nytimes3xbfgragh.onion/interactive/2020/us/elections/election-states-biden-trump.html?action=click\&pgtype=Article\&state=default\&region=BELOW_MAIN_CONTENT\&context=storylines_guide}{your
    own coalition of battleground states}~to see potential outcomes.
  \end{itemize}
\item ~
  \hypertarget{voting-deadlines}{%
  \subsection{Voting Deadlines}\label{voting-deadlines}}

  \begin{itemize}
  \item
    Early voting for the presidential election starts in September~in
    some states. Take a look at
    \href{https://www.nytimes3xbfgragh.onion/interactive/2019/us/elections/2020-presidential-election-calendar.html?action=click\&pgtype=Article\&state=default\&region=BELOW_MAIN_CONTENT\&context=storylines_guide}{key
    dates}\href{https://www.nytimes3xbfgragh.onion/interactive/2019/us/elections/2020-presidential-election-calendar.html?action=click\&pgtype=Article\&state=default\&region=BELOW_MAIN_CONTENT\&context=storylines_guide}{where
    you
    liv}\href{https://www.nytimes3xbfgragh.onion/interactive/2019/us/elections/2020-presidential-election-calendar.html?action=click\&pgtype=Article\&state=default\&region=BELOW_MAIN_CONTENT\&context=storylines_guide}{e}.
    If you're voting by
    mail,~\href{https://www.nytimes3xbfgragh.onion/interactive/2020/08/31/us/politics/vote-by-mail-deadlines.html?action=click\&pgtype=Article\&state=default\&region=BELOW_MAIN_CONTENT\&context=storylines_guide}{it's
    risky to procrastinate}.
  \item
    \href{https://www.nytimes3xbfgragh.onion/interactive/2020/us/elections/joe-biden.html?action=click\&pgtype=Article\&state=default\&region=BELOW_MAIN_CONTENT\&context=storylines_guide}{}

    \hypertarget{joe-biden}{%
    \section{Joe Biden}\label{joe-biden}}

    \hypertarget{democrat}{%
    \subsection{Democrat}\label{democrat}}

    \href{https://www.nytimes3xbfgragh.onion/interactive/2020/us/elections/donald-trump.html?action=click\&pgtype=Article\&state=default\&region=BELOW_MAIN_CONTENT\&context=storylines_guide}{}

    \hypertarget{donald-trump}{%
    \section{Donald Trump}\label{donald-trump}}

    \hypertarget{republican}{%
    \subsection{Republican}\label{republican}}
  \end{itemize}
\item
  \hypertarget{keep-up-with-our-coverage}{%
  \subsection{Keep Up With Our
  Coverage}\label{keep-up-with-our-coverage}}

  \begin{itemize}
  \item
    Get an
    \href{https://www.nytimes3xbfgragh.onion/newsletters/politics?action=click\&pgtype=Article\&state=default\&region=BELOW_MAIN_CONTENT\&context=storylines_guide}{email}~recapping
    the day's news
  \item
    Download our mobile app on
    \href{https://apps.apple.com/us/app/nytimes/id284862083?ls=1\&mat_click_id=5c79ae7455014fd1bd66b5610c05b8f2-20191112-16948\&referrer=mat_click_id\%3D5c79ae7455014fd1bd66b5610c05b8f2-20191112-16948\%26link_click_id\%3D722930677036718082}{iOS}~and
    \href{http://a.localytics.com/android?id=com.nytimes.android\&referrer=utm_source\%3Dother_nyt_mobile_web\%26utm_medium\%3DWeb\%2520page\%26utm_term\%3DGeneral\%2520Mobile\%2520Page\%26utm_campaign\%3DNYT\%2520Mobile\%2520General\%2520Page}{Android}~and
    turn on Breaking News and Politics alerts
  \end{itemize}
\end{itemize}

Advertisement

\protect\hyperlink{after-bottom}{Continue reading the main story}

\hypertarget{site-index}{%
\subsection{Site Index}\label{site-index}}

\hypertarget{site-information-navigation}{%
\subsection{Site Information
Navigation}\label{site-information-navigation}}

\begin{itemize}
\tightlist
\item
  \href{https://help.nytimes3xbfgragh.onion/hc/en-us/articles/115014792127-Copyright-notice}{©~2020~The
  New York Times Company}
\end{itemize}

\begin{itemize}
\tightlist
\item
  \href{https://www.nytco.com/}{NYTCo}
\item
  \href{https://help.nytimes3xbfgragh.onion/hc/en-us/articles/115015385887-Contact-Us}{Contact
  Us}
\item
  \href{https://www.nytco.com/careers/}{Work with us}
\item
  \href{https://nytmediakit.com/}{Advertise}
\item
  \href{http://www.tbrandstudio.com/}{T Brand Studio}
\item
  \href{https://www.nytimes3xbfgragh.onion/privacy/cookie-policy\#how-do-i-manage-trackers}{Your
  Ad Choices}
\item
  \href{https://www.nytimes3xbfgragh.onion/privacy}{Privacy}
\item
  \href{https://help.nytimes3xbfgragh.onion/hc/en-us/articles/115014893428-Terms-of-service}{Terms
  of Service}
\item
  \href{https://help.nytimes3xbfgragh.onion/hc/en-us/articles/115014893968-Terms-of-sale}{Terms
  of Sale}
\item
  \href{https://spiderbites.nytimes3xbfgragh.onion}{Site Map}
\item
  \href{https://help.nytimes3xbfgragh.onion/hc/en-us}{Help}
\item
  \href{https://www.nytimes3xbfgragh.onion/subscription?campaignId=37WXW}{Subscriptions}
\end{itemize}
