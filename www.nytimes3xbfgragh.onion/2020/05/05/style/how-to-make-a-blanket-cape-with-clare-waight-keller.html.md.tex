Sections

SEARCH

\protect\hyperlink{site-content}{Skip to
content}\protect\hyperlink{site-index}{Skip to site index}

\href{https://www.nytimes3xbfgragh.onion/section/style}{Style}

\href{https://myaccount.nytimes3xbfgragh.onion/auth/login?response_type=cookie\&client_id=vi}{}

\href{https://www.nytimes3xbfgragh.onion/section/todayspaper}{Today's
Paper}

\href{/section/style}{Style}\textbar{}How to Make a Blanket Cape, With
Clare Waight Keller

\url{https://nyti.ms/3clZSoJ}

\begin{itemize}
\item
\item
\item
\item
\item
\item
\end{itemize}

\hypertarget{the-coronavirus-outbreak}{%
\subsubsection{\texorpdfstring{\href{https://www.nytimes3xbfgragh.onion/news-event/coronavirus?name=styln-coronavirus-national\&region=TOP_BANNER\&block=storyline_menu_recirc\&action=click\&pgtype=Article\&impression_id=c75236c0-f279-11ea-b3ae-754b2373ddeb\&variant=undefined}{The
Coronavirus
Outbreak}}{The Coronavirus Outbreak}}\label{the-coronavirus-outbreak}}

\begin{itemize}
\tightlist
\item
  live\href{https://www.nytimes3xbfgragh.onion/2020/09/08/world/covid-19-coronavirus.html?name=styln-coronavirus-national\&region=TOP_BANNER\&block=storyline_menu_recirc\&action=click\&pgtype=Article\&impression_id=c75236c1-f279-11ea-b3ae-754b2373ddeb\&variant=undefined}{Latest
  Updates}
\item
  \href{https://www.nytimes3xbfgragh.onion/interactive/2020/us/coronavirus-us-cases.html?name=styln-coronavirus-national\&region=TOP_BANNER\&block=storyline_menu_recirc\&action=click\&pgtype=Article\&impression_id=c75236c2-f279-11ea-b3ae-754b2373ddeb\&variant=undefined}{Maps
  and Cases}
\item
  \href{https://www.nytimes3xbfgragh.onion/interactive/2020/science/coronavirus-vaccine-tracker.html?name=styln-coronavirus-national\&region=TOP_BANNER\&block=storyline_menu_recirc\&action=click\&pgtype=Article\&impression_id=c75236c3-f279-11ea-b3ae-754b2373ddeb\&variant=undefined}{Vaccine
  Tracker}
\item
  \href{https://www.nytimes3xbfgragh.onion/2020/09/02/your-money/eviction-moratorium-covid.html?name=styln-coronavirus-national\&region=TOP_BANNER\&block=storyline_menu_recirc\&action=click\&pgtype=Article\&impression_id=c75236c4-f279-11ea-b3ae-754b2373ddeb\&variant=undefined}{Eviction
  Moratorium}
\item
  \href{https://www.nytimes3xbfgragh.onion/interactive/2020/09/02/magazine/food-insecurity-hunger-us.html?name=styln-coronavirus-national\&region=TOP_BANNER\&block=storyline_menu_recirc\&action=click\&pgtype=Article\&impression_id=c75236c5-f279-11ea-b3ae-754b2373ddeb\&variant=undefined}{American
  Hunger}
\end{itemize}

Advertisement

\protect\hyperlink{after-top}{Continue reading the main story}

Supported by

\protect\hyperlink{after-sponsor}{Continue reading the main story}

Designer D.I.Y. series

\includegraphics{https://static01.graylady3jvrrxbe.onion/images/2020/05/07/style/05diydesigners-waightkeller-profileimg/oakImage-1588602447696-articleLarge.jpg?quality=75\&auto=webp\&disable=upscale}

\hypertarget{how-to-make-a-blanket-cape-with-clare-waight-keller}{%
\section{How to Make a Blanket Cape, With Clare Waight
Keller}\label{how-to-make-a-blanket-cape-with-clare-waight-keller}}

Image

In the first of our Designer D.I.Y. at Home series, the former Givenchy
artistic director shows us how to turn an old fleece blanket into a
coverall.

By \href{https://www.nytimes3xbfgragh.onion/by/vanessa-friedman}{Vanessa
Friedman}

Illustrations by Samantha Hahn

May 5, 2020

\begin{center}\rule{0.5\linewidth}{\linethickness}\end{center}

What to do with those old picnic or children's bed blankets gathering
dust in a closet? Here the British designer Clare Waight Keller, who
recently stepped down as the artistic director of Givenchy, where she
was known for her elegant mix of the heritage and the contemporary (as
exemplified by the Meghan Markle wedding dress she designed), guides us
through how to turn an old blanket into a blanket cape --- one of her
trademark styles.

``I liked the idea of a cape, firstly because I love them,'' Ms. Waight
Keller said. ``Throughout my career at Givenchy and Chloé, I have
designed everything from checked blanket capes to haute couture evening
capes.

``In times like these, I believe we are all looking to feel wrapped and
protected. Capes are one of the most democratic fashion items: They work
for everyone, regardless of age, shape, size or height.''

Image

\textbf{Your tool kit:}

\begin{itemize}
\item
  One old blanket, approximately 5-by-6 feet. If a blanket is longer,
  that's OK; it just means the cape will be longer, too.
\item
  One pair of sharp scissors
\item
  Measuring tape
\item
  Marking pen
\item
  Wool yarn
\item
  Tapestry needle with a very large eye
\end{itemize}

Image

STEP 1

\hypertarget{lay-the-blanket-flat}{%
\subsubsection{\texorpdfstring{\textbf{Lay the blanket
flat.}}{Lay the blanket flat.}}\label{lay-the-blanket-flat}}

Fold it in half lengthwise so that it's 5 feet in width. That will
eventually become the shoulder line.

Image

STEP 2

\hypertarget{measure-halfway-across-the}{%
\subsubsection{\texorpdfstring{\textbf{Measure halfway across
the}}{Measure halfway across the}}\label{measure-halfway-across-the}}

\textbf{width at both the top of the fold}\\
\textbf{and the bottom.}

Mark the measurements on one side of the folded blanket with the pen,
then join the two dots vertically. The line should run straight up the
middle of the folded blanket.

Image

STEP 3

\hypertarget{cut-vertically-up-the-line}{%
\subsubsection{\texorpdfstring{\textbf{Cut vertically up the
line.}}{Cut vertically up the line.}}\label{cut-vertically-up-the-line}}

Remember: You are cutting only one side of the folded blanket.

Image

STEP 4

\hypertarget{measure-4-inches-horizontally}{%
\subsubsection{\texorpdfstring{\textbf{Measure 4 inches
horizontally}}{Measure 4 inches horizontally}}\label{measure-4-inches-horizontally}}

\textbf{on either side of the cut at the top}\\
\textbf{of the fold.}

Mark each side with the pen, then cut along the fold to each point.
Those cuts form the opening for the neck.

Image

STEP 5

\hypertarget{finish-the-edges}{%
\subsubsection{\texorpdfstring{\textbf{Finish the
edges.}}{Finish the edges.}}\label{finish-the-edges}}

Most picnic and fleece blankets are made of fabrics that don't fray, so
you can leave the edges raw if you like. But a blanket stitch adds an
additional visual element, if you have the time and inclination. To
create this finish, follow these instructions:

\begin{itemize}
\item
  Collect your blanket, needle and thread. Thread the needle with a long
  length of wool (approximately one arm's length) and make a knot at the
  end.
\item
  Begin on the right side of your fabric. Bring the thread up from the
  back so the knot is hidden.
\item
  Pull the thread all the way through. Hold on to the thread as you pull
  so the short tail doesn't become unthreaded.
\item
  Take the thread around to the back of the fabric and bring it up to
  the front again through the same hole. The loop should be parallel to
  the vertical edge of the cape.
\item
  Bring the needle through the stitched loop, from left to right, along
  the edge of the fabric.
\item
  Pull the thread straight up to tighten the stitch. Lay the thread atop
  your hand, so it remains in front of the stitches you are making.
\item
  From the back, make another stitch about ¾ inch to the left.
\item
  Pull the thread almost all the way through, leaving a loop. Bring the
  needle through the loop from front to back.
\item
  Gently pull the thread tight to complete the stitch. Continue sewing
  stitches to the left until your edge is finished.
\item
  Continue around the perimeter of the cape.
\end{itemize}

Image

\hypertarget{styling-options}{%
\subsubsection{\texorpdfstring{\textbf{Styling
Options}}{Styling Options}}\label{styling-options}}

Your new cape can be worn in two ways:

1. As a wrap: Drape one side over your shoulder so it falls straight and
then pull the corner of the other side across your neck to form a cowl,
so the edge crosses your body diagonally.

2. Belted: Using a rope tie or belt, circle your waist beneath the cape
in the back, through the sides and around the front so that it is
secured to your body in front and left free to fly out behind.

\begin{center}\rule{0.5\linewidth}{\linethickness}\end{center}

Photograph by Julien Mignot for The New York Times

Advertisement

\protect\hyperlink{after-bottom}{Continue reading the main story}

\hypertarget{site-index}{%
\subsection{Site Index}\label{site-index}}

\hypertarget{site-information-navigation}{%
\subsection{Site Information
Navigation}\label{site-information-navigation}}

\begin{itemize}
\tightlist
\item
  \href{https://help.nytimes3xbfgragh.onion/hc/en-us/articles/115014792127-Copyright-notice}{©~2020~The
  New York Times Company}
\end{itemize}

\begin{itemize}
\tightlist
\item
  \href{https://www.nytco.com/}{NYTCo}
\item
  \href{https://help.nytimes3xbfgragh.onion/hc/en-us/articles/115015385887-Contact-Us}{Contact
  Us}
\item
  \href{https://www.nytco.com/careers/}{Work with us}
\item
  \href{https://nytmediakit.com/}{Advertise}
\item
  \href{http://www.tbrandstudio.com/}{T Brand Studio}
\item
  \href{https://www.nytimes3xbfgragh.onion/privacy/cookie-policy\#how-do-i-manage-trackers}{Your
  Ad Choices}
\item
  \href{https://www.nytimes3xbfgragh.onion/privacy}{Privacy}
\item
  \href{https://help.nytimes3xbfgragh.onion/hc/en-us/articles/115014893428-Terms-of-service}{Terms
  of Service}
\item
  \href{https://help.nytimes3xbfgragh.onion/hc/en-us/articles/115014893968-Terms-of-sale}{Terms
  of Sale}
\item
  \href{https://spiderbites.nytimes3xbfgragh.onion}{Site Map}
\item
  \href{https://help.nytimes3xbfgragh.onion/hc/en-us}{Help}
\item
  \href{https://www.nytimes3xbfgragh.onion/subscription?campaignId=37WXW}{Subscriptions}
\end{itemize}
