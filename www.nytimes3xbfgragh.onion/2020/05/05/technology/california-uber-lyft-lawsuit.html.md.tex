Sections

SEARCH

\protect\hyperlink{site-content}{Skip to
content}\protect\hyperlink{site-index}{Skip to site index}

\href{https://www.nytimes3xbfgragh.onion/section/technology}{Technology}

\href{https://myaccount.nytimes3xbfgragh.onion/auth/login?response_type=cookie\&client_id=vi}{}

\href{https://www.nytimes3xbfgragh.onion/section/todayspaper}{Today's
Paper}

\href{/section/technology}{Technology}\textbar{}California Sues Uber and
Lyft, Claiming Workers Are Misclassified

\url{https://nyti.ms/2zd1Etz}

\begin{itemize}
\item
\item
\item
\item
\item
\end{itemize}

\hypertarget{the-coronavirus-outbreak}{%
\subsubsection{\texorpdfstring{\href{https://www.nytimes3xbfgragh.onion/news-event/coronavirus?name=styln-coronavirus-markets\&region=TOP_BANNER\&block=storyline_menu_recirc\&action=click\&pgtype=Article\&impression_id=4c388c30-f1f4-11ea-9ee4-4b0f9d7721fd\&variant=undefined}{The
Coronavirus
Outbreak}}{The Coronavirus Outbreak}}\label{the-coronavirus-outbreak}}

\begin{itemize}
\tightlist
\item
  live\href{https://www.nytimes3xbfgragh.onion/2020/09/08/world/covid-19-coronavirus.html?name=styln-coronavirus-markets\&region=TOP_BANNER\&block=storyline_menu_recirc\&action=click\&pgtype=Article\&impression_id=4c38b340-f1f4-11ea-9ee4-4b0f9d7721fd\&variant=undefined}{Latest
  Updates}
\item
  \href{https://www.nytimes3xbfgragh.onion/interactive/2020/us/coronavirus-us-cases.html?name=styln-coronavirus-markets\&region=TOP_BANNER\&block=storyline_menu_recirc\&action=click\&pgtype=Article\&impression_id=4c38b341-f1f4-11ea-9ee4-4b0f9d7721fd\&variant=undefined}{Maps
  and Cases}
\item
  \href{https://www.nytimes3xbfgragh.onion/interactive/2020/science/coronavirus-vaccine-tracker.html?name=styln-coronavirus-markets\&region=TOP_BANNER\&block=storyline_menu_recirc\&action=click\&pgtype=Article\&impression_id=4c38b342-f1f4-11ea-9ee4-4b0f9d7721fd\&variant=undefined}{Vaccine
  Tracker}
\item
  \href{https://www.nytimes3xbfgragh.onion/2020/09/02/your-money/eviction-moratorium-covid.html?name=styln-coronavirus-markets\&region=TOP_BANNER\&block=storyline_menu_recirc\&action=click\&pgtype=Article\&impression_id=4c38b343-f1f4-11ea-9ee4-4b0f9d7721fd\&variant=undefined}{Eviction
  Moratorium}
\item
  \href{https://www.nytimes3xbfgragh.onion/interactive/2020/09/02/magazine/food-insecurity-hunger-us.html?name=styln-coronavirus-markets\&region=TOP_BANNER\&block=storyline_menu_recirc\&action=click\&pgtype=Article\&impression_id=4c38b344-f1f4-11ea-9ee4-4b0f9d7721fd\&variant=undefined}{American
  Hunger}
\end{itemize}

Advertisement

\protect\hyperlink{after-top}{Continue reading the main story}

Supported by

\protect\hyperlink{after-sponsor}{Continue reading the main story}

\hypertarget{california-sues-uber-and-lyft-claiming-workers-are-misclassified}{%
\section{California Sues Uber and Lyft, Claiming Workers Are
Misclassified}\label{california-sues-uber-and-lyft-claiming-workers-are-misclassified}}

The ride-hailing companies are accused of defying a new state law that
says gig workers should be treated as employees.

\includegraphics{https://static01.graylady3jvrrxbe.onion/images/2020/05/05/business/05uber/merlin_160662738_6f343b7e-7f38-45b1-b07b-606b13f6ddc6-articleLarge.jpg?quality=75\&auto=webp\&disable=upscale}

By \href{https://www.nytimes3xbfgragh.onion/by/kate-conger}{Kate Conger}

\begin{itemize}
\item
  Published May 5, 2020Updated July 14, 2020
\item
  \begin{itemize}
  \item
  \item
  \item
  \item
  \item
  \end{itemize}
\end{itemize}

OAKLAND, Calif. --- California's attorney general and a coalition of
city attorneys in the state sued
\href{https://www.nytimes3xbfgragh.onion/2020/08/20/technology/uber-lyft-california-shutdown.html}{Uber
and Lyft} on Tuesday, claiming the companies wrongfully classified their
drivers as independent contractors in violation of a state law that
makes them employees.

The law, known as Assembly Bill 5, requires companies to treat their
workers as employees instead of contractors if they control how workers
perform tasks or if the work is a routine part of a company's business.

At least one million gig workers in the state are affected by the law,
which is supposed to give them a path to benefits like a minimum wage
and unemployment insurance that have been traditionally withheld from
independent contractors.

Although A.B. 5 took effect on Jan. 1, Uber, Lyft and other gig economy
companies that operate in California have resisted and are not taking
steps to reclassify their drivers. Uber, Lyft and DoorDash have
poured\href{https://www.nytimes3xbfgragh.onion/2019/08/29/technology/uber-lyft-ballot-initiative.html}{\$90
million} into a campaign for a ballot initiative that would exempt them
from complying with the law. Uber has also argued that
\href{https://www.nytimes3xbfgragh.onion/2019/09/11/business/economy/uber-california-bill.html}{its
core business is technology, not rides,} and therefore drivers are not a
key part of its business.

\href{https://www.sfcityattorney.org/wp-content/uploads/2020/05/2020-05-05-Complaint-Filed.pdf}{The
lawsuit} also claims the ride-hailing companies are engaging in an
unfair business practice that harms other California companies that
follow the law. By avoiding payroll taxes and not paying minimum wage,
Uber and Lyft are able to provide rides at ``an artificially low cost,''
the suit claims, giving them a competitive advantage over other
businesses. The suit seeks civil penalties and back wages for workers
that could add up to hundreds of millions of dollars.

``California has ground rules with rights and protections for workers
and their employers. We intend to make sure that Uber or Lyft play by
the rules,'' Xavier Becerra, California's attorney general, said in a
statement. The city attorneys of San Francisco, Los Angeles and San
Diego joined in the lawsuit.

\hypertarget{latest-updates-the-coronavirus-outbreak-and-the-economy}{%
\section{\texorpdfstring{\href{https://www.nytimes3xbfgragh.onion/live/2020/09/08/business/stock-market-today-coronavirus?action=click\&pgtype=Article\&state=default\&region=MAIN_CONTENT_1\&context=storylines_live_updates}{Latest
Updates: The Coronavirus Outbreak and the
Economy}}{Latest Updates: The Coronavirus Outbreak and the Economy}}\label{latest-updates-the-coronavirus-outbreak-and-the-economy}}

\href{https://www.nytimes3xbfgragh.onion/live/2020/09/08/business/stock-market-today-coronavirus?action=click\&pgtype=Article\&state=default\&region=MAIN_CONTENT_1\&context=storylines_live_updates\#boeing-warns-of-delays-in-787-dreamliner-deliveries-as-it-contends-with-quality-control-problems}{49m
ago}

\href{https://www.nytimes3xbfgragh.onion/live/2020/09/08/business/stock-market-today-coronavirus?action=click\&pgtype=Article\&state=default\&region=MAIN_CONTENT_1\&context=storylines_live_updates\#boeing-warns-of-delays-in-787-dreamliner-deliveries-as-it-contends-with-quality-control-problems}{Boeing
warns of delays in 787 Dreamliner deliveries as it contends with quality
control problems.}

\href{https://www.nytimes3xbfgragh.onion/live/2020/09/08/business/stock-market-today-coronavirus?action=click\&pgtype=Article\&state=default\&region=MAIN_CONTENT_1\&context=storylines_live_updates\#oil-prices-fall-close-to-9-percent-this-whole-summer-of-bullishness-is-over}{60m
ago}

\href{https://www.nytimes3xbfgragh.onion/live/2020/09/08/business/stock-market-today-coronavirus?action=click\&pgtype=Article\&state=default\&region=MAIN_CONTENT_1\&context=storylines_live_updates\#oil-prices-fall-close-to-9-percent-this-whole-summer-of-bullishness-is-over}{Oil
prices fall close to 9 percent: `This whole summer of bullishness is
over.'}

\href{https://www.nytimes3xbfgragh.onion/live/2020/09/08/business/stock-market-today-coronavirus?action=click\&pgtype=Article\&state=default\&region=MAIN_CONTENT_1\&context=storylines_live_updates\#tesla-shares-continue-to-retreat-from-their-recent-peak}{1h
ago}

\href{https://www.nytimes3xbfgragh.onion/live/2020/09/08/business/stock-market-today-coronavirus?action=click\&pgtype=Article\&state=default\&region=MAIN_CONTENT_1\&context=storylines_live_updates\#tesla-shares-continue-to-retreat-from-their-recent-peak}{Tesla
shares continue to retreat from their recent peak.}

\href{https://www.nytimes3xbfgragh.onion/live/2020/09/08/business/stock-market-today-coronavirus?action=click\&pgtype=Article\&state=default\&region=MAIN_CONTENT_1\&context=storylines_live_updates}{See
more updates}

More live coverage:
\href{https://www.nytimes3xbfgragh.onion/2020/09/08/world/covid-19-coronavirus.html?action=click\&pgtype=Article\&state=default\&region=MAIN_CONTENT_1\&context=storylines_live_updates}{Global}

California's move is a significant threat to the gig companies and could
influence other states with similar laws to take action against them,
labor experts said.

``Uber and Lyft have lived a kind of charmed life in terms of escaping
law enforcement generally, and particularly with regard to employment
law,'' said William B. Gould IV, a law professor at Stanford University
and the former chairman of the National Labor Relations Board. ``The
attorney general's action can't help but have a positive influence on
law enforcement generally against them.''

Although Uber and Lyft have argued that their drivers have independence
and decide when to work, the lawsuit claims that both ride-hailing
companies exert enough control over drivers to make them employees. In
remarks after the lawsuit was filed, Gov. Gavin Newsom of California
said he would seek funding for enforcement of A.B. 5 in the state budget
and that California had a responsibility to enforce the law.

``Uber and Lyft are traditional employers of these misclassified
employees. They hire and fire them. They control which drivers have
access to which possible assignments,'' the lawsuit says. ``Uber and
Lyft are transportation companies in the business of selling rides to
customers, and their drivers are the employees who provide the rides
they sell.''

Because ride-hailing companies and app-based food delivery services do
not employ drivers, they avoid the costs of insurance and vehicle
maintenance, sick leave and unemployment. But the coronavirus pandemic
has exposed gaps in the gig economy, as
\href{https://www.nytimes3xbfgragh.onion/2020/03/18/technology/gig-economy-pandemic.html}{drivers
have abruptly lost their income} and struggled to get unemployment
insurance, or fallen sick without access to paid sick leave.

``Uber and Lyft are breaking the law. We are going to put a stop to
it,'' said Dennis Herrera, the city attorney of San Francisco. ``This
pandemic just highlights the danger of the work these essential workers
are doing.''

Lyft said in a statement that it was ``looking forward'' to working with
the attorney general and mayors ``to bring all the benefits of
California's innovation economy to as many workers as possible,
especially during this time when the creation of good jobs with access
to affordable health care and other benefits is more important than
ever.''

Gig companies have responded to the outbreak by offering limited
quarantine pay to drivers who receive a positive coronavirus diagnosis
or a doctor's recommendation to isolate themselves. The companies have
also distributed hand sanitizer and other cleaning supplies to drivers.

Uber has worried that providing those things could expose it to
misclassification claims from workers, and the company has
\href{https://www.nytimes3xbfgragh.onion/2020/04/03/technology/virus-tech-lobbyists-gains.html}{asked
lawmakers to shield it from lawsuits} over how its drivers are
classified if it provides the drivers with medical supplies or
compensation. Its chief executive, Dara Khosrowshahi, wrote a letter to
President Trump recently asking for a new classification for drivers
that would make them neither employees nor contractors.

Mr. Khosrowshahi has called for a so-called third way of classifying
workers, which would provide some health benefits to drivers without
making them employees who could receive full employment benefits.

The lawsuit criticized Uber and Lyft's lobbying efforts, including a
ballot initiative that would exempt the companies from complying with
A.B. 5. ``Amid a once-in-a-century pandemic, they have gone to
extraordinary lengths to convince the public that their unlawful
misclassification scheme is in the public interest,'' the suit said.

Uber said on Tuesday it would press forward with its ballot initiative.
``We will contest this action in court, while at the same time pushing
to raise the standard of independent work for drivers in California,''
said Matt Wing, an Uber spokesman.

The lawsuit comes at a fraught moment for Uber and Lyft, as the
businesses struggle to adapt to the sudden decline in demand caused by
the pandemic. Consumer data suggests that
\href{https://www.nytimes3xbfgragh.onion/2020/04/17/technology/uber-lift-coronavirus.html}{spending
on ride-hailing has dropped} as much as 83 percent. Lyft is expected to
report its first-quarter earnings on Wednesday, while Uber reports on
Thursday.

Before the pandemic, Uber and Lyft were racing to become profitable
after their 2019 initial public offerings stumbled.

But as demand has declined, both companies have tried to cut costs. Last
week, Lyft laid off 17 percent of its work force, furloughed 5 percent,
and cut pay for executives and remaining staff members. Uber is also
considering layoffs and on Monday announced the closing of its food
delivery business in Egypt, Honduras, Saudi Arabia and several other
markets where it was deeply unprofitable.

The stock prices of both companies declined on news of the lawsuit.
Uber's shares fell about 1 percent and Lyft's fell 4 percent.

Still, Uber and Lyft have reported that they have substantial cash
reserves to weather the downturn caused by the pandemic. Uber said it
had more than \$8 billion, while Lyft said it had more than \$2 billion.

``In what world can you not pay your fair share?'' said Lorena Gonzalez,
a California Assembly member who represents southern San Diego and
drafted A.B. 5.

The city and state attorneys involved in the suit said they believed it
would succeed despite Uber's ballot measure. ``They are not going to
succeed in any event. The voters are too smart for that,'' said Mike
Feuer, the city attorney of Los Angeles.

Noam Scheiber contributed reporting from Evanston, Ill.

Advertisement

\protect\hyperlink{after-bottom}{Continue reading the main story}

\hypertarget{site-index}{%
\subsection{Site Index}\label{site-index}}

\hypertarget{site-information-navigation}{%
\subsection{Site Information
Navigation}\label{site-information-navigation}}

\begin{itemize}
\tightlist
\item
  \href{https://help.nytimes3xbfgragh.onion/hc/en-us/articles/115014792127-Copyright-notice}{©~2020~The
  New York Times Company}
\end{itemize}

\begin{itemize}
\tightlist
\item
  \href{https://www.nytco.com/}{NYTCo}
\item
  \href{https://help.nytimes3xbfgragh.onion/hc/en-us/articles/115015385887-Contact-Us}{Contact
  Us}
\item
  \href{https://www.nytco.com/careers/}{Work with us}
\item
  \href{https://nytmediakit.com/}{Advertise}
\item
  \href{http://www.tbrandstudio.com/}{T Brand Studio}
\item
  \href{https://www.nytimes3xbfgragh.onion/privacy/cookie-policy\#how-do-i-manage-trackers}{Your
  Ad Choices}
\item
  \href{https://www.nytimes3xbfgragh.onion/privacy}{Privacy}
\item
  \href{https://help.nytimes3xbfgragh.onion/hc/en-us/articles/115014893428-Terms-of-service}{Terms
  of Service}
\item
  \href{https://help.nytimes3xbfgragh.onion/hc/en-us/articles/115014893968-Terms-of-sale}{Terms
  of Sale}
\item
  \href{https://spiderbites.nytimes3xbfgragh.onion}{Site Map}
\item
  \href{https://help.nytimes3xbfgragh.onion/hc/en-us}{Help}
\item
  \href{https://www.nytimes3xbfgragh.onion/subscription?campaignId=37WXW}{Subscriptions}
\end{itemize}
