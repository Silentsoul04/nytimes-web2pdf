Sections

SEARCH

\protect\hyperlink{site-content}{Skip to
content}\protect\hyperlink{site-index}{Skip to site index}

\href{https://www.nytimes3xbfgragh.onion/section/us}{U.S.}

\href{https://myaccount.nytimes3xbfgragh.onion/auth/login?response_type=cookie\&client_id=vi}{}

\href{https://www.nytimes3xbfgragh.onion/section/todayspaper}{Today's
Paper}

\href{/section/us}{U.S.}\textbar{}With New Hot Spots Emerging, No Sign
of a Respite

\url{https://nyti.ms/2zfksbi}

\begin{itemize}
\item
\item
\item
\item
\item
\item
\end{itemize}

\hypertarget{the-coronavirus-outbreak}{%
\subsubsection{\texorpdfstring{\href{https://www.nytimes3xbfgragh.onion/news-event/coronavirus?name=styln-coronavirus-national\&region=TOP_BANNER\&block=storyline_menu_recirc\&action=click\&pgtype=Article\&impression_id=0ddc8940-f4bb-11ea-bb8e-792101acfed7\&variant=undefined}{The
Coronavirus
Outbreak}}{The Coronavirus Outbreak}}\label{the-coronavirus-outbreak}}

\begin{itemize}
\tightlist
\item
  live\href{https://www.nytimes3xbfgragh.onion/2020/09/11/world/covid-19-coronavirus.html?name=styln-coronavirus-national\&region=TOP_BANNER\&block=storyline_menu_recirc\&action=click\&pgtype=Article\&impression_id=0ddc8941-f4bb-11ea-bb8e-792101acfed7\&variant=undefined}{Latest
  Updates}
\item
  \href{https://www.nytimes3xbfgragh.onion/interactive/2020/us/coronavirus-us-cases.html?name=styln-coronavirus-national\&region=TOP_BANNER\&block=storyline_menu_recirc\&action=click\&pgtype=Article\&impression_id=0ddc8942-f4bb-11ea-bb8e-792101acfed7\&variant=undefined}{Maps
  and Cases}
\item
  \href{https://www.nytimes3xbfgragh.onion/interactive/2020/science/coronavirus-vaccine-tracker.html?name=styln-coronavirus-national\&region=TOP_BANNER\&block=storyline_menu_recirc\&action=click\&pgtype=Article\&impression_id=0ddc8943-f4bb-11ea-bb8e-792101acfed7\&variant=undefined}{Vaccine
  Tracker}
\item
  \href{https://www.nytimes3xbfgragh.onion/2020/09/10/us/politics/fda-coronavirus-vaccine.html?name=styln-coronavirus-national\&region=TOP_BANNER\&block=storyline_menu_recirc\&action=click\&pgtype=Article\&impression_id=0ddcb050-f4bb-11ea-bb8e-792101acfed7\&variant=undefined}{F.D.A.
  Regulators' Self-Defense}
\item
  \href{https://www.nytimes3xbfgragh.onion/2020/09/09/upshot/coronavirus-surprise-test-fees.html?name=styln-coronavirus-national\&region=TOP_BANNER\&block=storyline_menu_recirc\&action=click\&pgtype=Article\&impression_id=0ddcb051-f4bb-11ea-bb8e-792101acfed7\&variant=undefined}{Surprise
  Test Fees}
\end{itemize}

Advertisement

\protect\hyperlink{after-top}{Continue reading the main story}

Supported by

\protect\hyperlink{after-sponsor}{Continue reading the main story}

\hypertarget{with-new-hot-spots-emerging-no-sign-of-a-respite}{%
\section{With New Hot Spots Emerging, No Sign of a
Respite}\label{with-new-hot-spots-emerging-no-sign-of-a-respite}}

While cities like New York have seen a hopeful drop in cases, upticks in
other major cities and smaller communities have offset those decreases.

\includegraphics{https://static01.graylady3jvrrxbe.onion/images/2020/05/06/us/05stateofthevirus-01-jump/merlin_172181223_1696a681-6e53-422b-bbf3-d00639bb9412-articleLarge.jpg?quality=75\&auto=webp\&disable=upscale}

\href{https://www.nytimes3xbfgragh.onion/by/julie-bosman}{\includegraphics{https://static01.graylady3jvrrxbe.onion/images/2018/11/09/multimedia/author-julie-bosman/author-julie-bosman-thumbLarge.png}}\href{https://www.nytimes3xbfgragh.onion/by/mitch-smith}{\includegraphics{https://static01.graylady3jvrrxbe.onion/images/2018/09/10/multimedia/author-mitch-smith/author-mitch-smith-thumbLarge.png}}\href{https://www.nytimes3xbfgragh.onion/by/amy-harmon}{\includegraphics{https://static01.graylady3jvrrxbe.onion/images/2020/04/29/reader-center/author-amy-harmon/author-amy-harmon-thumbLarge-v2.png}}

By \href{https://www.nytimes3xbfgragh.onion/by/julie-bosman}{Julie
Bosman}, \href{https://www.nytimes3xbfgragh.onion/by/mitch-smith}{Mitch
Smith} and \href{https://www.nytimes3xbfgragh.onion/by/amy-harmon}{Amy
Harmon}

\begin{itemize}
\item
  Published May 5, 2020Updated May 28, 2020
\item
  \begin{itemize}
  \item
  \item
  \item
  \item
  \item
  \item
  \end{itemize}
\end{itemize}

In New York City, the daily onslaught of death from the
\href{https://www.nytimes3xbfgragh.onion/2020/05/14/health/talking-coronavirus-infect.html}{coronavirus}
has dropped to half of what it was. In Chicago, a makeshift hospital in
a lakefront convention center is closing, deemed no longer needed. And
in New Orleans, new cases have dwindled to a handful each day.

Yet across America, those signs of progress obscure a darker reality.

The country is still in the firm grip of a pandemic with little hope of
release. For every indication of improvement in controlling the virus,
new outbreaks have emerged elsewhere, leaving the nation stuck in a
steady, unrelenting march of deaths and
\href{https://www.nytimes3xbfgragh.onion/2020/05/14/health/talking-coronavirus-infect.html}{infections}.

\href{https://www.nytimes3xbfgragh.onion/interactive/2020/us/states-reopen-map-coronavirus.html}{As
states continue to lift restrictions} meant to stop the virus, impatient
Americans are freely returning to shopping, lingering in restaurants and
gathering in parks. Regular new flare-ups and super-spreader events are
expected to be close behind.

Any notion that the coronavirus threat is fading away appears to be
magical thinking, at odds with
\href{https://www.nytimes3xbfgragh.onion/interactive/2020/us/coronavirus-us-cases.html}{what
the latest numbers show}.

Coronavirus in America now looks like this: More than a month has passed
since there was a day with fewer than 1,000 deaths from the virus.
Almost every day, at least 25,000 new coronavirus cases are identified,
meaning that the total in the United States --- which has the
\href{https://www.nytimes3xbfgragh.onion/interactive/2020/world/coronavirus-maps.html}{highest
number of known cases in the world} with more than a million --- is
expanding by between 2 and 4 percent daily.

Rural towns that one month ago were unscathed are suddenly hot spots for
the virus. It is rampaging through nursing homes, meatpacking plants and
prisons, killing the medically vulnerable and the poor, and new
outbreaks keep emerging in grocery stores, Walmarts or factories, an
ominous harbinger of what a full reopening of the economy will bring.

While dozens of rural counties have no known coronavirus cases, a
panoramic view of the country reveals a grim and distressing picture.

\hypertarget{new-reported-cases-by-day}{%
\subsubsection{New Reported Cases by
Day}\label{new-reported-cases-by-day}}

\hypertarget{as-the-new-york-metro-area-has-seen-a-recent-decline-in-new-cases-the-number-of-cases-in-the-rest-of-the-united-states-has-steadily-increased}{%
\paragraph{As the New York metro area has seen a recent decline in new
cases, the number of cases in the rest of the United States has steadily
increased.}\label{as-the-new-york-metro-area-has-seen-a-recent-decline-in-new-cases-the-number-of-cases-in-the-rest-of-the-united-states-has-steadily-increased}}

New York metro area

Rest of the United States

Source:
\href{https://www.nytimes3xbfgragh.onion/interactive/2020/us/coronavirus-us-cases.html}{New
York Times database} of reports from state and local health agencies and
hospitals.·The New York City metropolitan area is defined by the U.S.
Census Bureau and includes nearby cities and suburbs in Westchester,
Long Island and northern New Jersey.

``If you include New York, it looks like a plateau moving down,'' said
Andrew Noymer, an associate professor of public health at the University
of California, Irvine. ``If you exclude New York, it's a plateau slowly
moving up.''

In early April, more than 5,000 new cases were regularly being added in
New York City on a daily basis. Those numbers have dropped significantly
over the past few weeks, but that progress has been largely offset by
increases in other major cities.

\includegraphics{https://static01.graylady3jvrrxbe.onion/images/2020/05/05/us/05stateofthevirus-02/merlin_172176717_1e0524bd-d03c-4662-964d-27626da9a290-articleLarge.jpg?quality=75\&auto=webp\&disable=upscale}

Consider Chicago and Los Angeles, which have flattened their curves and
avoided the explosive growth of New York City. Even so, coronavirus
cases in their counties have more than doubled since April 18. Cook
County, home to Chicago, is now sometimes adding more than 2,000 new
cases in a day, and Los Angeles County has often been adding at least
1,000.

\hypertarget{latest-updates-the-coronavirus-outbreak}{%
\section{\texorpdfstring{\href{https://www.nytimes3xbfgragh.onion/2020/09/11/world/covid-19-coronavirus.html?action=click\&pgtype=Article\&state=default\&region=MAIN_CONTENT_1\&context=storylines_live_updates}{Latest
Updates: The Coronavirus
Outbreak}}{Latest Updates: The Coronavirus Outbreak}}\label{latest-updates-the-coronavirus-outbreak}}

Updated 2020-09-12T05:29:13.829Z

\begin{itemize}
\tightlist
\item
  \href{https://www.nytimes3xbfgragh.onion/2020/09/11/world/covid-19-coronavirus.html?action=click\&pgtype=Article\&state=default\&region=MAIN_CONTENT_1\&context=storylines_live_updates\#link-dfb8a16}{Fauci
  cautions the virus could disrupt life in the U.S. until `maybe even
  towards the end of 2021.'}
\item
  \href{https://www.nytimes3xbfgragh.onion/2020/09/11/world/covid-19-coronavirus.html?action=click\&pgtype=Article\&state=default\&region=MAIN_CONTENT_1\&context=storylines_live_updates\#link-7104d154}{From
  Asia to Africa, China promotes its vaccine candidates to win friends.}
\item
  \href{https://www.nytimes3xbfgragh.onion/2020/09/11/world/covid-19-coronavirus.html?action=click\&pgtype=Article\&state=default\&region=MAIN_CONTENT_1\&context=storylines_live_updates\#link-393ad215}{The
  other way the virus will kill: hunger.}
\end{itemize}

\href{https://www.nytimes3xbfgragh.onion/2020/09/11/world/covid-19-coronavirus.html?action=click\&pgtype=Article\&state=default\&region=MAIN_CONTENT_1\&context=storylines_live_updates}{See
more updates}

More live coverage:
\href{https://www.nytimes3xbfgragh.onion/live/2020/09/11/business/stock-market-today-coronavirus?action=click\&pgtype=Article\&state=default\&region=MAIN_CONTENT_1\&context=storylines_live_updates}{Markets}

Dallas County in Texas has been adding about 100 more cases a day than
it was a month ago, and the counties that include Boston and
Indianapolis have also reported higher numbers.

It is not just the major cities. Smaller towns and rural counties in the
Midwest and South have suddenly been hit hard, underscoring the
capriciousness of the pandemic.

Dakota County, Neb., which has the third-most cases per capita in the
country, had no known cases as recently as April 11. Now the county is a
hot zone for the virus.

Dakota City is home to a major Tyson beef-processing plant, where cases
have been reported. And the region, which spreads across the borders of
both Iowa and South Dakota, is dotted with meat-processing plants that
have been a major source of work for generations.
\href{https://www.nytimes3xbfgragh.onion/2020/04/22/us/coronavirus-workplaces-midwest.html}{The
pattern has repeated all over}: Federal authorities say that at least
4,900 meat and poultry processing workers have been infected across 19
states.

The Tyson plant in Dakota City has temporarily closed for deep cleaning.
Now the workers wait, afraid to go back to work but fearful not to.

``They need money and they want to go back of course,'' said Qudsia
Hussaini, whose husband is an imam helping counsel the families of
workers who have been sickened or have died. With many businesses
shuttered or suffering financially because of the pandemic, she said,
``There's no other place they can work.''

Image

Diners visited a restaurant in San Antonio last week after Texas rolled
back its stay-at-home orders.Credit...Christopher Lee for The New York
Times

Trousdale County, Tenn., another rural area, suddenly finds itself with
the nation's highest per capita infection rate by far. A prison appears
responsible for a huge spike in cases; in 10 days, this county of about
11,000 residents saw its known cases skyrocket to 1,344 from 27.

As of last week, more than half of the inmates and staff members tested
at Trousdale Turner Correctional Center in Hartsville, Tenn., were
positive for the virus, officials said.

``It's been my worst nightmare since the beginning of this that this
would happen,'' said Dwight Jewell, chairman of the Trousdale County
Commission. ``I've been expecting this. You put that many people in a
contained environment and all it takes is one.''

Everyone in town knows about the outbreak. But they are defiant:
Businesses in the county are reopening this week. On Monday evening,
county commissioners held an in-person meeting, with chairs spaced six
feet apart. They have a budget to pass and other issues facing the
county, Mr. Jewell said.

``We've got to get back to the business of the community,'' he said.

Infectious-disease experts are troubled by perceptions that the United
States has seen the worst of the virus, and have sought to caution
against misplaced optimism.

``I don't see why we expect large declines in daily case counts over the
next month,'' Trevor Bedford, a scientist at the Fred Hutchinson Cancer
Research Center who has studied the spread and evolution of the virus,
\href{https://twitter.com/trvrb/status/1255976683770806272}{wrote on
Twitter}. He added, ``There may well be cities / counties that achieve
suppression locally, but nationally I expect things to be messy with
flare-ups in various geographies followed by responses to these
flare-ups.''

The outbreak in the United States has already killed more than 70,000
people, and epidemiologists say the nation will not see fewer than 5,000
coronavirus-related deaths a week until after June 20, according to a
survey conducted by researchers at the University of Massachusetts at
Amherst. A federal projection, based on government modeling pulled
together by the Federal Emergency Management Agency,
\href{https://www.nytimes3xbfgragh.onion/2020/05/04/us/coronavirus-live-updates.html}{forecasts
a steady rise} in deaths in the next several weeks, to a daily death
toll of 3,000 on June 1.

Image

Friends caught up, from a distance, in Council Bluffs, Iowa, on
Friday.Credit...Calla Kessler/The New York Times

Across the country, scientists tried to project the virus's future
course, and the results have been a range of shifting models.
\href{https://reichlab.io/covid19-forecast-hub/}{An aggregate} of
several models assembled by Nicholas Reich, a biostatistician at the
University of Massachusetts, predicts there will be an average of 10,000
deaths per week for the next few weeks. That is fewer than in previous
weeks, but it does not mean a peak has been passed, Dr. Reich said. In
the seven-day period that ended on Sunday, about 12,700 deaths tied to
the virus occurred across the country.

``There's this idea that it's going to go up and it's going to come down
in a symmetric curve,'' Dr. Reich said. ``It doesn't have to do that. It
could go up and we could have several thousand deaths per week for many
weeks.''

The deaths have
\href{https://www.nytimes3xbfgragh.onion/2020/04/17/us/coronavirus-nursing-homes.html}{hit
few places harder than America's nursing homes} and other long-term care
facilities. More than a quarter of the deaths have been linked to those
facilities, and more than 118,000 residents and staff members in at
least 6,800 homes have contracted the virus.

There is no escaping some basic epidemic math.

In the absence of a vaccine, stopping the spread of the virus requires
about two-thirds of the population to have been infected. And some
experts have argued that before what is known as herd immunity kicks in,
the number of people infected nationwide could reach a staggering 90
percent if social distancing is relaxed and transmission rates climb.
(It is also not clear how long immunity will last among those who have
been infected.)

As testing capacity has increased, so has the number of cases being
counted. But many jurisdictions are still missing cases
\href{https://www.nytimes3xbfgragh.onion/interactive/2020/04/28/us/coronavirus-death-toll-total.html}{and
undercounting deaths}. Many epidemiologists assume that roughly 10 times
as many people have been infected with the coronavirus than the number
of known cases.

Because of the time it will take for infections to spread, incubate and
cause people to die, the effects of reopening states may not be known
until at least six weeks after the fact. One model used by the Centers
for Disease Control and Prevention includes an assumption that the
infection rate will increase up to 20 percent in states that reopen.

Under \href{https://covid19-projections.com/}{that model}, by early
August, the most likely outcome is 3,000 more deaths in Georgia than the
state has right now, 10,000 more each in New York and New Jersey, and
around 7,000 more each in Pennsylvania, Illinois and Massachusetts.
Under the model's most likely forecast, the nation will see about
100,000 additional deaths by Aug. 4.

``Even if we're past the first peak, that doesn't mean the worst is
behind us,'' said Youyang Gu, the data scientist who created the model.
``It goes up quickly but it's a slow decline down.''

Reporting was contributed by John Eligon, Robert Gebeloff, Danielle
Ivory, Dionne Searcey, Timothy Williams and Karen Yourish.

Advertisement

\protect\hyperlink{after-bottom}{Continue reading the main story}

\hypertarget{site-index}{%
\subsection{Site Index}\label{site-index}}

\hypertarget{site-information-navigation}{%
\subsection{Site Information
Navigation}\label{site-information-navigation}}

\begin{itemize}
\tightlist
\item
  \href{https://help.nytimes3xbfgragh.onion/hc/en-us/articles/115014792127-Copyright-notice}{©~2020~The
  New York Times Company}
\end{itemize}

\begin{itemize}
\tightlist
\item
  \href{https://www.nytco.com/}{NYTCo}
\item
  \href{https://help.nytimes3xbfgragh.onion/hc/en-us/articles/115015385887-Contact-Us}{Contact
  Us}
\item
  \href{https://www.nytco.com/careers/}{Work with us}
\item
  \href{https://nytmediakit.com/}{Advertise}
\item
  \href{http://www.tbrandstudio.com/}{T Brand Studio}
\item
  \href{https://www.nytimes3xbfgragh.onion/privacy/cookie-policy\#how-do-i-manage-trackers}{Your
  Ad Choices}
\item
  \href{https://www.nytimes3xbfgragh.onion/privacy}{Privacy}
\item
  \href{https://help.nytimes3xbfgragh.onion/hc/en-us/articles/115014893428-Terms-of-service}{Terms
  of Service}
\item
  \href{https://help.nytimes3xbfgragh.onion/hc/en-us/articles/115014893968-Terms-of-sale}{Terms
  of Sale}
\item
  \href{https://spiderbites.nytimes3xbfgragh.onion}{Site Map}
\item
  \href{https://help.nytimes3xbfgragh.onion/hc/en-us}{Help}
\item
  \href{https://www.nytimes3xbfgragh.onion/subscription?campaignId=37WXW}{Subscriptions}
\end{itemize}
