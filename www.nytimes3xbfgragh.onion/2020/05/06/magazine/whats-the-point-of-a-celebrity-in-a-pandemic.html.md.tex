Sections

SEARCH

\protect\hyperlink{site-content}{Skip to
content}\protect\hyperlink{site-index}{Skip to site index}

\href{https://myaccount.nytimes3xbfgragh.onion/auth/login?response_type=cookie\&client_id=vi}{}

\href{https://www.nytimes3xbfgragh.onion/section/todayspaper}{Today's
Paper}

What's the Point of a Celebrity in a Pandemic?

\url{https://nyti.ms/2zWxGtY}

\begin{itemize}
\item
\item
\item
\item
\item
\item
\end{itemize}

Advertisement

\protect\hyperlink{after-top}{Continue reading the main story}

Supported by

\protect\hyperlink{after-sponsor}{Continue reading the main story}

\href{/column/screenland}{Screenland}

\hypertarget{whats-the-point-of-a-celebrity-in-a-pandemic}{%
\section{What's the Point of a Celebrity in a
Pandemic?}\label{whats-the-point-of-a-celebrity-in-a-pandemic}}

\includegraphics{https://static01.graylady3jvrrxbe.onion/images/2020/05/10/magazine/10mag-screenland-1/10mag-screenland-1-articleLarge.jpg?quality=75\&auto=webp\&disable=upscale}

By Carina Chocano

\begin{itemize}
\item
  May 6, 2020
\item
  \begin{itemize}
  \item
  \item
  \item
  \item
  \item
  \item
  \end{itemize}
\end{itemize}

Jimmy Fallon's appeal, such as it is, resides in his air of benighted,
puppyish cluelessness --- but in times of crisis, this sort of thing can
quickly lose its charm. Fallon himself seemed to learn this in 2016,
after he chummily tousled Donald Trump's hair during a ``Tonight Show''
appearance, sparking a public reaction so negative the host was pushed
into a rare sulk. ``What do you want me to do? You want me to kill
myself?'' he asked in The Hollywood Reporter. ``I'm sorry. I don't want
to make anyone angry --- I never do, and I never will. It's all in the
fun of the show.'' Here he had a point. Jimmy Fallon is not like Stephen
Colbert or Trevor Noah or John Oliver or Samantha Bee. His role is not
to engage, however wryly, with real life. It is to show up every night
and distract us from it, as though it's not really there.

Jimmy Fallon performs "The Safety Dance" with the Roots. \textbar{} "One
World: Together at Home"Credit...CreditVideo by Global Citizen

Last month, though, Fallon joined Colbert and Jimmy Kimmel as the hosts
of ``One World: Together at Home,'' a concert organized to show support
for the World Health Organization and very much a response to real life.
His role was the same as usual: goofy, innocent, like a child trying to
get a smile out of his worried mother. At one point in the program ---
which aired for two hours on broadcast networks and six on streaming
services and online --- he and the scattered members of his show's house
band, the Roots, performed a slightly modified cover of Men Without
Hats' 1983 hit ``The Safety Dance'' (``We can dance/We can
dance/Everybody's washing their hands''), cut together with images of
health care workers around the country. The medical personnel looked
calm and can-do in fresh protective gear, dancing and singing as they
scrubbed their hands and sprayed surfaces with disinfectants the
president would soon recommend as potential intravenous remedies.
Afterward, the actor Henry Golding spoke about the importance of
vaccines. Later, Beyoncé talked about how the virus was
disproportionately affecting African-Americans. Michelle Obama and Laura
Bush spoke as a duo. Appearing just days after the president halted U.S.
funding of the W.H.O., its director general, Dr. Tedros Adhanom
Ghebreyesus, affirmed the need for global cooperation.

This being a concert with a mission, the celebrities performing from
home continually redirected the spotlight onto health care workers,
medical researchers, teachers, delivery people and other essential
workers. They talked about poverty, homelessness and the failures of the
American health care system. They nodded to every last one of the
humanistic, evergreen, model-U.N. values the bulk of Americans usually
claim to aspire to: science, democracy, cooperation, responsibility,
hygiene, reason and kindness.

Given that most of us have been hearing this talk since preschool, there
should have been nothing remarkable about the broadcast. But so far in
this crisis, and even before it, there has been a strange vacuum of
basic shared-values platitudes. The president spends his time snapping
at journalists, shifting blame or promising miraculous cures; other
politicians snipe at states they see as enemies or idly muse about what
number of dead seniors might not be so terrible. We citizens, meanwhile,
are free to consume our separate, personalized diets of information or
disinformation. What the ``One World'' broadcast managed was to
replicate something old-fashioned: a big, corny, mass-media
entertainment, one that piously affirmed what feel like bedrock,
mainstream values. And this time, instead of feeling like a stifling
monoculture, it felt reassuring, like a dispatch from some consensus
reality we'd all forgotten we'd ever inhabited.

\textbf{The pandemic has} lifted the veil on all sorts of obvious but
seldom-acknowledged societal issues, but few have been as comical as the
distance between ordinary people and the celebrities who pretend to
relate to them. Some stars' behavior has come across blinkered and
clueless, as when Ellen DeGeneres compared staying in her palatial home
to being in jail. Others have seemed pandering and fake, as Gal Gadot
did when she rounded up famous friends and made them sing ``Imagine,''
as though we were all living in a 1970s Coke commercial. Arnold
Schwarzenegger enjoying a cigar in his hot tub while telling people to
stay home was not a great look; nor was doing it from his lavish
kitchen, alongside a miniature horse and donkey. This is the kind of
thing that makes people post memes of fake Ikea manuals for flat-pack
guillotines in the comments.

Over the past weeks, many people's contact with the outside world has
shrunk to the size of a screen. But much of the artifice that helps
those screens manipulate our feelings --- lights, swooping cameras, the
laughter of studio audiences --- has been stripped away. In some cases,
seeing stars exposed and unmediated has left them looking silly; in
others it has made them seem relatable, trustworthy. We've seen hosts
and anchors broadcasting from their kitchens, attics or sheds, telling
jokes that are met with empty silence. We've watched Pete Davidson make
a video for ``Saturday Night Live'' on a beige couch in his mom's living
room, and we've seen stars flood the internet with homemade video, from
the well-meaning (John Krasinski starting a ``Some Good News'' YouTube
channel) to the engagingly odd (January Jones sweeping her foyer in a
creepy Venetian mask).

Early in his job as the host of ``The Daily Show,'' Jon Stewart made a
decision to ``straighten out our point of view here'' and write ``about
how we really feel'' --- and spent the years that followed, in the wake
of 9/11, excoriating the growing tendency for politicians to create
their own realities when facts became inconvenient. The show that
resulted may have felt pointed or partisan, but it was animated, Stewart
told The New Yorker, by a sense that the nation was forgetting core
principles of ``fairness, common sense and moderation.''

Something similar seems to have happened for ``One World.'' In creating
a mass cultural event, the celebrities and the vast entertainment
apparatus behind it have discovered a purpose they can, for the moment,
usefully put themselves to: standing for values that used to feel more
uncontroversial than they do now. Amid a vacuum of leadership, they have
taken up the sorts of jobs once done by U.S.O. tours and propaganda
posters: boosting morale, stumping for shared beliefs, rallying people
to action. Plant a victory garden! Wash your hands! We'll prevail
together!

Thus: Jimmy Fallon, dancing innocuously amid clips of medical
professionals. These were not the drained and overburdened doctors and
nurses of the country's hardest-hit areas. They were friendly and
confident, chins up and ready to do their part, like masked modern Rosie
the Riveters. The program they appeared in was a parade of competence
and calm, a six-hour balm --- a reminder of what we were, or thought we
were, or were supposed to be.

Advertisement

\protect\hyperlink{after-bottom}{Continue reading the main story}

\hypertarget{site-index}{%
\subsection{Site Index}\label{site-index}}

\hypertarget{site-information-navigation}{%
\subsection{Site Information
Navigation}\label{site-information-navigation}}

\begin{itemize}
\tightlist
\item
  \href{https://help.nytimes3xbfgragh.onion/hc/en-us/articles/115014792127-Copyright-notice}{©~2020~The
  New York Times Company}
\end{itemize}

\begin{itemize}
\tightlist
\item
  \href{https://www.nytco.com/}{NYTCo}
\item
  \href{https://help.nytimes3xbfgragh.onion/hc/en-us/articles/115015385887-Contact-Us}{Contact
  Us}
\item
  \href{https://www.nytco.com/careers/}{Work with us}
\item
  \href{https://nytmediakit.com/}{Advertise}
\item
  \href{http://www.tbrandstudio.com/}{T Brand Studio}
\item
  \href{https://www.nytimes3xbfgragh.onion/privacy/cookie-policy\#how-do-i-manage-trackers}{Your
  Ad Choices}
\item
  \href{https://www.nytimes3xbfgragh.onion/privacy}{Privacy}
\item
  \href{https://help.nytimes3xbfgragh.onion/hc/en-us/articles/115014893428-Terms-of-service}{Terms
  of Service}
\item
  \href{https://help.nytimes3xbfgragh.onion/hc/en-us/articles/115014893968-Terms-of-sale}{Terms
  of Sale}
\item
  \href{https://spiderbites.nytimes3xbfgragh.onion}{Site Map}
\item
  \href{https://help.nytimes3xbfgragh.onion/hc/en-us}{Help}
\item
  \href{https://www.nytimes3xbfgragh.onion/subscription?campaignId=37WXW}{Subscriptions}
\end{itemize}
