Sections

SEARCH

\protect\hyperlink{site-content}{Skip to
content}\protect\hyperlink{site-index}{Skip to site index}

\href{https://www.nytimes3xbfgragh.onion/section/health}{Health}

\href{https://myaccount.nytimes3xbfgragh.onion/auth/login?response_type=cookie\&client_id=vi}{}

\href{https://www.nytimes3xbfgragh.onion/section/todayspaper}{Today's
Paper}

\href{/section/health}{Health}\textbar{}With Red Tape Lifted, Dr. Zoom
Will See You Now

\url{https://nyti.ms/2SKGBFF}

\begin{itemize}
\item
\item
\item
\item
\item
\item
\end{itemize}

\href{https://www.nytimes3xbfgragh.onion/spotlight/at-home?action=click\&pgtype=Article\&state=default\&region=TOP_BANNER\&context=at_home_menu}{At
Home}

\begin{itemize}
\tightlist
\item
  \href{https://www.nytimes3xbfgragh.onion/2020/09/07/travel/route-66.html?action=click\&pgtype=Article\&state=default\&region=TOP_BANNER\&context=at_home_menu}{Cruise
  Along: Route 66}
\item
  \href{https://www.nytimes3xbfgragh.onion/2020/09/04/dining/sheet-pan-chicken.html?action=click\&pgtype=Article\&state=default\&region=TOP_BANNER\&context=at_home_menu}{Roast:
  Chicken With Plums}
\item
  \href{https://www.nytimes3xbfgragh.onion/2020/09/04/arts/television/dark-shadows-stream.html?action=click\&pgtype=Article\&state=default\&region=TOP_BANNER\&context=at_home_menu}{Watch:
  Dark Shadows}
\item
  \href{https://www.nytimes3xbfgragh.onion/interactive/2020/at-home/even-more-reporters-editors-diaries-lists-recommendations.html?action=click\&pgtype=Article\&state=default\&region=TOP_BANNER\&context=at_home_menu}{Explore:
  Reporters' Google Docs}
\end{itemize}

Advertisement

\protect\hyperlink{after-top}{Continue reading the main story}

Supported by

\protect\hyperlink{after-sponsor}{Continue reading the main story}

the new old age

\hypertarget{with-red-tape-lifted-dr-zoom-will-see-you-now}{%
\section{With Red Tape Lifted, Dr. Zoom Will See You
Now}\label{with-red-tape-lifted-dr-zoom-will-see-you-now}}

The pandemic pushed Medicare to make telemedicine more financially
attractive. Now doctors, patients and regulators will see if they want
to stick with it.

\includegraphics{https://static01.graylady3jvrrxbe.onion/images/2020/05/12/science/12SCI-SPAN-TELEMEDICINE/12SCI-SPAN-TELEMEDICINE-articleLarge.jpg?quality=75\&auto=webp\&disable=upscale}

By \href{https://www.nytimes3xbfgragh.onion/by/paula-span}{Paula Span}

\begin{itemize}
\item
  May 8, 2020
\item
  \begin{itemize}
  \item
  \item
  \item
  \item
  \item
  \item
  \end{itemize}
\end{itemize}

In late March, Mary Jane Sturgis got a call from her primary-care
physician's office, saying that her doctor was working from home during
the Covid-19 crisis and suggesting an alternative for her scheduled
checkup. Would Ms. Sturgis agree to a video appointment on Zoom?

``I didn't know what Zoom was,'' Ms. Sturgis recalled. ``But I said if I
could figure it out, sure.''

A retired college administrator, she contends with ailments that put her
at high risk from the new coronavirus. Several autoimmune conditions.
Damaged lungs, caused by radiation for breast cancer and requiring daily
nebulizer and inhaler use. At 77, age itself.

She already found it tiring to drive half an hour from her home in
Media, Pa., to Dr. Lisa Sardanopoli's office at Lankenau Medical Center;
now, walking into a hospital also seemed dangerous.

The transition to telemedicine initially proved a bit rocky. Ms. Sturgis
could see her doctor on Zoom. ``But I couldn't hear her,'' Ms. Sturgis
said. ``And she couldn't see or hear me.''

So at her doctor's texted suggestion, they switched to FaceTime,
familiar to Ms. Sturgis from video chats with her grandchildren. ``I was
surprised at how well it worked,'' she said.

Ms. Sturgis missed the way Dr. Sardanopoli sometimes placed a reassuring
hand on hers when she worried. Otherwise, ``It felt like we were sitting
and talking together as usual.''

At the end of their appointment, ``I said: `Do I send you money? How do
I pay for this?''' Ms. Sturgis recalled. ``She said, `It's covered by
Medicare.'''

Just weeks earlier, that would not have been true. For years, advocates
and researchers have urged greater use of telemedicine --- delivered by
video or phone, through online patient portals or remote monitoring
devices --- particularly for older adults.

But Medicare
\href{https://www.healthaffairs.org/doi/abs/10.1377/hlthaff.2018.05151}{adoption
was slow}. The Government Accountability Office reported in 2017 that
just \href{https://www.gao.gov/products/GAO-17-365}{one percent of
beneficiaries}, most in rural areas, received care through telemedicine
(a term used interchangeably with telehealth).

Then came Covid-19 and its lockdowns, sending both small practices and
major health systems scrambling to give patients access to health care
without face-to-face contact. In response, federal agencies loosened
restrictions and regulations, at least temporarily, that had stalled
telemedicine for decades.

``This crisis has forced us to change how we deliver health care more in
20 days than we had in 20 years,'' said Dr. Robert McLean, a past
president of the American College of Physicians, as well as an internist
and rheumatologist with Northeast Medical Group in Connecticut.

Though some practices and systems never acquired the necessary
technology, Dr. McLean said, the major barrier to telehealth had been
financial. ``It just wasn't getting paid for adequately,'' he said.

In traditional Medicare, payment had been lower than for in-person
visits, a sure way to discourage use. (Most Medicare Advantage plans
already covered some telehealth services; each plan determines what it
pays for them.)

``Health care systems and hospitals are businesses,'' said Dr. Sirina
Keesara, who researches health system design at Stanford University and
co-authored \href{https://www.nejm.org/doi/full/10.1056/NEJMp2005835}{a
recent editorial} in The New England Journal of Medicine. ``If they
don't have a financial incentive to change, they'll stick with what they
know.''

Medicare's restrictions hampered adoption in other ways, too. It limited
services to rural patients and usually required them to travel to a
clinic or office, rather than participate from home. It covered some
services for ``established'' patients but not new ones, or insisted on
office visits before it would reimburse for subsequent telehealth.

But in March, citing the need for flexibility in face of the coronavirus
pandemic, the Centers for Medicare and Medicaid Services
\href{https://www.cms.gov/newsroom/fact-sheets/additional-backgroundsweeping-regulatory-changes-help-us-healthcare-system-address-covid-19-patient}{removed
those barriers}. It also added scores of new telehealth services it
would cover, including emergency-room visits, initial and discharge
visits at nursing homes and remote monitoring for chronic conditions.
And it agreed to pay the same rates as for in-person care.

It maintained a lower rate, at first, for audio-only phone visits.
Professional associations objected, arguing that this policy reinforced
the so-called digital divide, depriving older adults of remote care if
they lacked computers, smartphones or broadband. ``People who rely on a
landline cannot do video visits,'' Dr. McLean said.

On April 30, that obstacle fell, too, as Medicare
\href{https://www.cms.gov/newsroom/press-releases/trump-administration-issues-second-round-sweeping-changes-support-us-healthcare-system-during-covid}{agreed
to reimburse} equally for visits in person, by video or by phone.

And another major hurdle was removed by the Department of Health and
Human Services, which, in March, temporarily relaxed enforcement of
HIPAA, the federal patient privacy law. It will waive penalties when
providers use everyday platforms like FaceTime or Skype, which aren't
HIPAA-compliant.

Doctors and patients still need to be in the same room for some
appointments, of course. Certain conditions mandate physical
examination. ``Sometimes we need to have life-changing conversations,''
added Dr. Andrea Jonas, a pulmonologist and critical-care specialist at
Stanford University and a co-author of The New England Journal
editorial. ``Those are harder to do via telehealth.''

Still, by mid-April more than 20 percent of people over 70 had
experienced a telehealth appointment since the start of the pandemic,
\href{https://www.norc.org/NewsEventsPublications/PressReleases/Pages/more-than-half-of-older-adults-in-the-us-have-experienced-disruptions-in-care-due-to-coronavirus.aspx}{a
nationwide survey} by NORC at the University of Chicago found. Almost
half said they found the experience equivalent to an in-person visit;
about 40 percent said it was worse.

In interviews, patients told me of similarly mixed reactions.

Last month, Debra Reed, a management consultant in Ojai, Calif., sat in
on her husband's Zoom visit with his internist in Santa Barbara. Her
husband, 86, has dementia and is recovering from a stroke. ``It was odd
and unsatisfying, unsettling,'' Ms. Reed said of the encounter. ``It
leaves one lacking.''

Diana Hamlet-Cox felt differently. Her 89-year-old father, who recently
moved in with her and her husband in Goodyear, Ariz., has had half a
dozen video or phone appointments --- with a urologist, a
psychotherapist, a neurosurgeon.

``I was glad we didn't have to drive 25 miles to wait in a building with
other people and all those surfaces to touch,'' Ms. Hamlet-Cox said. ``I
thought, why didn't they do this sooner?''

Whether Medicare will stick with these changes --- temporary measures
allowed during the public health emergency --- is uncertain. A press
officer said the agency would assess its policies after the pandemic
ebbs. It will need to address concerns about privacy and fraud.

``I think there's going to be huge pressure to abandon all this,'' said
Dr. Kevin Schulman, a hospitalist and economist at Stanford University
and a co-author of The New England Journal editorial. ``Providers will
want to go back to the way we used to do it.''

The authors called for research to determine how well telehealth works
during the pandemic.

Previous studies have shown that patients with chronic obstructive
pulmonary disease \href{https://bjgp.org/content/62/604/e739}{don't fare
better} using telehealth, for instance. ``Perhaps patients were
encouraged to stay home instead of going to emergency rooms with more
severe symptoms,'' Dr. Jonas said.

For now, though, expanded Medicare telehealth coverage is giving
patients a glimpse of a different future, and some of them like it.

Mary Jane Sturgis, for instance. Last month, she began to fear that if
she contracted Covid-19, she'd be hospitalized and placed on a
ventilator without her consent; she asked Dr. Sardanopoli for an
appointment to discuss her end-of-life wishes.

They spent half an hour on FaceTime, talking through the options,
untroubled by their physical distance.

``I knew what I wanted, and she was completely respectful of that,'' Ms.
Sturgis said. ``I felt better and calmer afterward.''

\textbf{\emph{{[}}\href{http://on.fb.me/1paTQ1h}{\emph{Like the Science
Times page on Facebook.}}} ****** \emph{\textbar{} Sign up for the}
\textbf{\href{http://nyti.ms/1MbHaRU}{\emph{Science Times
newsletter.}}\emph{{]}}}

Advertisement

\protect\hyperlink{after-bottom}{Continue reading the main story}

\hypertarget{site-index}{%
\subsection{Site Index}\label{site-index}}

\hypertarget{site-information-navigation}{%
\subsection{Site Information
Navigation}\label{site-information-navigation}}

\begin{itemize}
\tightlist
\item
  \href{https://help.nytimes3xbfgragh.onion/hc/en-us/articles/115014792127-Copyright-notice}{©~2020~The
  New York Times Company}
\end{itemize}

\begin{itemize}
\tightlist
\item
  \href{https://www.nytco.com/}{NYTCo}
\item
  \href{https://help.nytimes3xbfgragh.onion/hc/en-us/articles/115015385887-Contact-Us}{Contact
  Us}
\item
  \href{https://www.nytco.com/careers/}{Work with us}
\item
  \href{https://nytmediakit.com/}{Advertise}
\item
  \href{http://www.tbrandstudio.com/}{T Brand Studio}
\item
  \href{https://www.nytimes3xbfgragh.onion/privacy/cookie-policy\#how-do-i-manage-trackers}{Your
  Ad Choices}
\item
  \href{https://www.nytimes3xbfgragh.onion/privacy}{Privacy}
\item
  \href{https://help.nytimes3xbfgragh.onion/hc/en-us/articles/115014893428-Terms-of-service}{Terms
  of Service}
\item
  \href{https://help.nytimes3xbfgragh.onion/hc/en-us/articles/115014893968-Terms-of-sale}{Terms
  of Sale}
\item
  \href{https://spiderbites.nytimes3xbfgragh.onion}{Site Map}
\item
  \href{https://help.nytimes3xbfgragh.onion/hc/en-us}{Help}
\item
  \href{https://www.nytimes3xbfgragh.onion/subscription?campaignId=37WXW}{Subscriptions}
\end{itemize}
