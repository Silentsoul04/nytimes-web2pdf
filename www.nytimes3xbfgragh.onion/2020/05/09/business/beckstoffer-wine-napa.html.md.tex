\href{/section/business}{Business}\textbar{}The Grapelord of Napa Faces
a Threat Worse Than Plague

\url{https://nyti.ms/3cihTUL}

\begin{itemize}
\item
\item
\item
\item
\item
\end{itemize}

\hypertarget{the-coronavirus-outbreak}{%
\subsubsection{\texorpdfstring{\href{https://www.nytimes3xbfgragh.onion/news-event/coronavirus?name=styln-coronavirus-markets\&region=TOP_BANNER\&block=storyline_menu_recirc\&action=click\&pgtype=Article\&impression_id=879e4a20-f52e-11ea-89d9-339d62ecbc9c\&variant=undefined}{The
Coronavirus
Outbreak}}{The Coronavirus Outbreak}}\label{the-coronavirus-outbreak}}

\begin{itemize}
\tightlist
\item
  live\href{https://www.nytimes3xbfgragh.onion/2020/09/12/world/covid-19-coronavirus.html?name=styln-coronavirus-markets\&region=TOP_BANNER\&block=storyline_menu_recirc\&action=click\&pgtype=Article\&impression_id=879e4a21-f52e-11ea-89d9-339d62ecbc9c\&variant=undefined}{Latest
  Updates}
\item
  \href{https://www.nytimes3xbfgragh.onion/interactive/2020/us/coronavirus-us-cases.html?name=styln-coronavirus-markets\&region=TOP_BANNER\&block=storyline_menu_recirc\&action=click\&pgtype=Article\&impression_id=879e4a22-f52e-11ea-89d9-339d62ecbc9c\&variant=undefined}{Maps
  and Cases}
\item
  \href{https://www.nytimes3xbfgragh.onion/interactive/2020/science/coronavirus-vaccine-tracker.html?name=styln-coronavirus-markets\&region=TOP_BANNER\&block=storyline_menu_recirc\&action=click\&pgtype=Article\&impression_id=879e7130-f52e-11ea-89d9-339d62ecbc9c\&variant=undefined}{Vaccine
  Tracker}
\item
  \href{https://www.nytimes3xbfgragh.onion/2020/09/10/us/politics/fda-coronavirus-vaccine.html?name=styln-coronavirus-markets\&region=TOP_BANNER\&block=storyline_menu_recirc\&action=click\&pgtype=Article\&impression_id=879e7131-f52e-11ea-89d9-339d62ecbc9c\&variant=undefined}{F.D.A.
  Regulators' Self-Defense}
\item
  \href{https://www.nytimes3xbfgragh.onion/2020/09/09/upshot/coronavirus-surprise-test-fees.html?name=styln-coronavirus-markets\&region=TOP_BANNER\&block=storyline_menu_recirc\&action=click\&pgtype=Article\&impression_id=879e7132-f52e-11ea-89d9-339d62ecbc9c\&variant=undefined}{Surprise
  Test Fees}
\end{itemize}

\includegraphics{https://static01.graylady3jvrrxbe.onion/images/2020/05/10/business/08BECKSTOFFER-01/merlin_172212930_d00da425-6e96-41d3-8431-8cf516d201ce-articleLarge.jpg?quality=75\&auto=webp\&disable=upscale}

Sections

\protect\hyperlink{site-content}{Skip to
content}\protect\hyperlink{site-index}{Skip to site index}

\hypertarget{the-grapelord-of-napa-faces-a-threat-worse-than-plague}{%
\section{The Grapelord of Napa Faces a Threat Worse Than
Plague}\label{the-grapelord-of-napa-faces-a-threat-worse-than-plague}}

For 50 years, Andy Beckstoffer drove up the price of wine. Did the
strategy work too well?

Andy Beckstoffer in Rutherford, Calif. Napa Valley has unquestionably
the best American terroir, and within it, the choicest land has his name
on a billboard next to it.Credit...Cayce Clifford for The New York Times

Supported by

\protect\hyperlink{after-sponsor}{Continue reading the main story}

By Ben Ryder Howe

\begin{itemize}
\item
  May 9, 2020
\item
  \begin{itemize}
  \item
  \item
  \item
  \item
  \item
  \end{itemize}
\end{itemize}

One balmy winter afternoon, Andy Beckstoffer, a grape grower who has
done more than nearly anyone to shape the premium U.S. wine industry,
was sitting in Mustard's, a restaurant in Napa Valley that is a kind of
clubhouse for the vintner class. Although Beckstoffer Vineyards, the
largest private grower in California, had recently set a sales record
with a blockbuster harvest of \$55 million worth of cabernet sauvignon,
its founder was not in the mood to celebrate. The wine industry was in
trouble, facing its worst outlook in generations --- and that was before
the coronavirus struck.

The litany of plagues was merciless: too many grapes, thanks to an epic
haul in California and Washington. Too many wildfires and weird bugs
unleashed by
\href{https://www.nytimes3xbfgragh.onion/interactive/2019/10/14/dining/drinks/climate-change-wine.html}{climate
change}. Too many new wineries in Napa, upsetting the balance of
agriculture and hospitality. And then there were the millennials, or
millenniums, as the 80-year-old Mr. Beckstoffer likes to call them.

The generation born between 1981 and 1996 has been blamed for killing
everything from napkins to homeownership, and thanks to its passion for
hard seltzer, liquid marijuana and other drinkable novelties, it's been
cast as the
\href{https://www.nytimes3xbfgragh.onion/2020/05/02/us/asian-giant-hornet-washington.html}{murder
hornet} of the wine industry as well. Mr. Beckstoffer finds their
health-crazed rituals (Drynuary?) puzzling.

``Wine is plant-based,'' he said, shaking his head and picking
mirthlessly at a spinach and mushroom burger. ``Why don't the
millenniums drink it?''

A few weeks later, Mr. Beckstoffer's anxiety was borne out by the
publication of Silicon Valley Bank's
\href{https://www.svb.com/wine-report}{annual report} on the U.S. wine
industry --- probably the most influential analysis of its kind. For
years, its author, Rob McMillan, has preached about the alarming
convergence of two trends: higher and higher bottle prices at the
premium end of the market, and millennial indifference. Some farmers and
winemakers have brushed Mr. McMillan off, and this time, he amped up the
urgency, writing plainly: ``The issue of greatest concern for the wine
business today is the lack of participation in the premium wine category
by the large millennial generation.''

The people who make wine don't just age grape juice, they ripen
customers, too, helping them evolve from undergraduate jug-swillers into
middle-aged buyers of prestige labels. That process, Mr. McMillan says,
appears to have stalled. Even as their purchases of other luxury goods
have increased, millennials have balked at high-priced cabernets, which
combined with the coronavirus makes 2020 ``the worst time since
Prohibition for fine wine producers in the United States,'' he said in
an interview.

The home of fine American wine is Napa Valley, where few benefit more
from high prices than Mr. Beckstoffer. The region has unquestionably the
best terroir in the United States, and within it, the choicest land has
his name on a billboard next to it. Starting in the 1980s, Mr.
Beckstoffer began seeking out what he calls ``the good stuff'' --- the
vineyards with records of success going back a century or longer. He now
owns six, including To Kalon, a plot in the center of the valley
considered the crown jewel of American viticulture. For the privilege of
squeezing Beckstoffer grapes, winemakers behind labels like Stag's Leap,
Schrader and Realm pay up to \$25,000 per ton --- more than five times
the Napa average.

All of which is to say: If the \$71 billion California wine industry
topples, then Mr. Beckstoffer, who values his empire at \$500 million,
may have the farthest to fall.

\includegraphics{https://static01.graylady3jvrrxbe.onion/images/2020/05/10/business/08BECKSTOFFER-02/merlin_172213275_e8bcabf1-11ea-44b2-8487-6158c11f887d-articleLarge.jpg?quality=75\&auto=webp\&disable=upscale}

\hypertarget{from-navy-grog-to-an-american-first-growth}{%
\subsection{From Navy Grog to an American `First
Growth'}\label{from-navy-grog-to-an-american-first-growth}}

Most Napa farmers hold some of their grapes back from market, in order
to press them into wine themselves. Not Mr. Beckstoffer, who sells every
last orb of fruit he can.

``Andy's different --- he has no interest in making wine,'' said Curtis
Strohl, the general manager of B Cellars, a Napa winery. In fact, Mr.
Beckstoffer finds the vinifying process a bore, and he doesn't care
about drinking great vintages himself. His rivals in Napa say he cares
only about money. Mr. Beckstoffer says he cares about farmers and the
land. But it seems clear that over the course of his 50-year career, as
the valley transformed from a drowsy agricultural community into an
inland yacht club, the two motivations have worked in concert.

\hypertarget{latest-updates-the-coronavirus-outbreak-and-the-economy}{%
\section{\texorpdfstring{\href{https://www.nytimes3xbfgragh.onion/live/2020/09/11/business/stock-market-today-coronavirus?action=click\&pgtype=Article\&state=default\&region=MAIN_CONTENT_1\&context=storylines_live_updates}{Latest
Updates: The Coronavirus Outbreak and the
Economy}}{Latest Updates: The Coronavirus Outbreak and the Economy}}\label{latest-updates-the-coronavirus-outbreak-and-the-economy}}

\href{https://www.nytimes3xbfgragh.onion/live/2020/09/11/business/stock-market-today-coronavirus?action=click\&pgtype=Article\&state=default\&region=MAIN_CONTENT_1\&context=storylines_live_updates\#the-nyse-may-move-its-data-center-out-of-new-jersey-in-response-to-a-proposed-tax}{23h
ago}

\href{https://www.nytimes3xbfgragh.onion/live/2020/09/11/business/stock-market-today-coronavirus?action=click\&pgtype=Article\&state=default\&region=MAIN_CONTENT_1\&context=storylines_live_updates\#the-nyse-may-move-its-data-center-out-of-new-jersey-in-response-to-a-proposed-tax}{The
N.Y.S.E. may move its data center out of New Jersey in response to a
proposed tax.}

\href{https://www.nytimes3xbfgragh.onion/live/2020/09/11/business/stock-market-today-coronavirus?action=click\&pgtype=Article\&state=default\&region=MAIN_CONTENT_1\&context=storylines_live_updates\#the-federal-budget-deficit-hit-3-trillion-as-of-august}{26h
ago}

\href{https://www.nytimes3xbfgragh.onion/live/2020/09/11/business/stock-market-today-coronavirus?action=click\&pgtype=Article\&state=default\&region=MAIN_CONTENT_1\&context=storylines_live_updates\#the-federal-budget-deficit-hit-3-trillion-as-of-august}{The
federal budget deficit hit \$3 trillion as of August.}

\href{https://www.nytimes3xbfgragh.onion/live/2020/09/11/business/stock-market-today-coronavirus?action=click\&pgtype=Article\&state=default\&region=MAIN_CONTENT_1\&context=storylines_live_updates\#warner-bros-pushes-the-release-of-wonder-woman-1984-to-christmas}{26h
ago}

\href{https://www.nytimes3xbfgragh.onion/live/2020/09/11/business/stock-market-today-coronavirus?action=click\&pgtype=Article\&state=default\&region=MAIN_CONTENT_1\&context=storylines_live_updates\#warner-bros-pushes-the-release-of-wonder-woman-1984-to-christmas}{Warner
Bros. pushes the release of `Wonder Woman 1984' to Christmas.}

\href{https://www.nytimes3xbfgragh.onion/live/2020/09/11/business/stock-market-today-coronavirus?action=click\&pgtype=Article\&state=default\&region=MAIN_CONTENT_1\&context=storylines_live_updates}{See
more updates}

More live coverage:
\href{https://www.nytimes3xbfgragh.onion/2020/09/11/world/covid-19-coronavirus.html?action=click\&pgtype=Article\&state=default\&region=MAIN_CONTENT_1\&context=storylines_live_updates}{Global}

Mr. Beckstoffer --- a courtly native of Richmond, Va., who pronounces
vineyard ``vin-yuhd'' and ``wine'' as if it had three syllables ---
readily concedes that what brought him to Napa was the chance to make a
killing. In 1967, recently graduated from Dartmouth's Tuck School of
Business, he was working for Heublein, an East Coast food and beverage
conglomerate with products such as Smirnoff, Jose Cuervo and a pre-mixed
tiki drink called Navy Grog.

The American palate was developing an appreciation for ``quality wine.''
That year, for the first time, more dry wine was sold than sweet. Mr.
Beckstoffer helped Heublein acquire Inglenook, a cherished, family-owned
winery that soon began pumping out ``oceans of plonk,'' as the novelist
and wine critic Jay McInerney once wrote. Heublein also bought Beaulieu,
where a similar transformation occurred.

``The arrogance,'' Mr. Beckstoffer said. ``We bought the two best
wineries in the valley and screwed it up.'' Though still in operation,
neither has returned to glory. Looking back ``makes your heart hurt,''
he said.

Heublein's bet swiftly turned sour. Spooked by labor issues, the company
gave up on quality wine after a few years and started selling its Napa
farmland --- to Mr. Beckstoffer, who had resigned from the company and
moved his family to the valley. By the 1980s, he had developed an
ambitious agenda that would take decades to unfold.

Like Robert Mondavi and a few others, Mr. Beckstoffer came to believe
that a once-in-a-lifetime opportunity lay dormant in Napa's soil. For
more than a century, the potential of the valley's wine had been
recognized even by Europeans. But the quality was uneven and financial
acumen was lacking, and as Mr. Beckstoffer saw it, the chance to create
an American equivalent of the First Growths of France was being
squandered --- like a great but unknown painter in need of a
sharp-elbowed dealer.

``The farmers were good at farming, but bad businessmen,'' Mr.
Beckstoffer said. ``You couldn't make any money owning land and selling
grapes.'' He believed that the local wine would only reach its potential
if it was strategically elevated into a luxury product --- scarce,
expensive, vigilantly branded --- even if that meant leaving behind an
Arcadian era centered on small family farms and affordability.

``In every agricultural area, there is a citizen hierarchy,'' Mr.
Beckstoffer said. ``Here, winemakers are at the top, and farmers used to
be at the bottom.'' He once told an interviewer that his overriding goal
was to give grape growers more clout.

With his Ivy League M.B.A. and corporate pedigree, Mr. Beckstoffer is
not exactly a typical farmer. In the 1980s, when Napa was still oriented
toward relatively humble varietals like zinfandel, an epidemic of
phylloxera --- a rapacious insect that feeds on the roots and leaves of
grape vines --- wiped out crops. Mr. Beckstoffer and others led the
charge for a valley-wide replanting with the more glamorous cabernet,
while introducing data analysis and other elements of industrial farming
that magnified yields enormously.

He also wielded back-room political skills to outmaneuver opponents.
Like any good luxury item, Napa land is in short supply --- 300 square
miles, most of it owned by a few families and corporations. The question
of whether to farm it, preserve it or use it to attract tourists is
never far from any conversation. In 1990, as wine drinkers were
developing a voracious appetite for Napa cabs at seemingly any price,
Mr. Beckstoffer was the driving force behind a landmark piece of
legislation, the Winery Definition Ordinance, requiring any wine with
the word ``Napa'' on it to be made from 75 percent local grapes.

The statute also limited what sort of social and commercial activities,
such as weddings, could take place at wineries. A generation later,
vintners still complain that the bill funneled business to its champion
and crippled the rest of the valley.

Image

In the 1980s, when Napa was still oriented toward relatively humble
varietals like zinfandel, Mr. Beckstoffer helped drive a valley-wide
replanting with the more glamorous cabernet sauvignon.Credit...Cayce
Clifford for The New York Times

Image

Not everyone appreciates his strategy. ``If you're not making enemies,''
Mr. Beckstoffer said, ``you're just taking up space.''Credit...Cayce
Clifford for The New York Times

As Mr. Beckstoffer became a land baron among land barons, he also
regularly enraged the winemakers at the top of Napa society, whom he
dismisses as ``blenders'' and ``media stars.'' One article from 1990
describes an incident at a country club, in which a winery owner
realizes that she has been seated near him and asks to be moved. Another
quotes a winemaker calling Mr. Beckstoffer ``a real snake.''

Mr. Beckstoffer displays both articles on his website. ``If you're not
making enemies,'' he said, smiling innocently, ``you're just taking up
space.''

For good measure, he has also stymied real estate developers. As one of
the sightlier parts of Northern California, with its majestic oaks and
gaudy colors, Napa has some of the highest real estate prices in America
and some of its most expensive hotel rooms. But Mr. Beckstoffer has long
sought to choke off the development of Napa as a ``lifestyle resort.''

Chuck Wagner, the founder of Caymus Vineyards, a prominent Napa winery,
is one of many proponents of building up the region --- more wineries,
more hotels, more tourists. ``People want to experience the beauty of
the valley,'' he said. ``Andy is against additional business.'' He
added, ``A lot of people believe that Andy does things for personal
financial gain.''

Mr. Beckstoffer insists he has higher principles, and despite his
corporate sheen, when he talks about securing ``agriculture in
perpetuity'' for Napa Valley, he has the unmistakable zeal of an
ideological convert. The heritage vineyards he bought are now in trusts
that cannot be developed or sold.

``You have to ask yourself, what do you want to leave for your
children?'' he told me. ``Someday, some spouse of a grandchild of mine
will want to build a hotel on one of our vineyards --- and they will
hate me, because they can't.''

\hypertarget{strong-arm-tactics--and-customers-happy-to-pay-the-price}{%
\subsection{Strong-arm tactics --- and customers happy to pay the
price}\label{strong-arm-tactics--and-customers-happy-to-pay-the-price}}

The Dr. Crane Vineyard is not what you think of when imagining
world-class terroir. Wedged between a ready-mix concrete plant and a
grade school, it is nevertheless one of Napa's oldest vineyards,
originally planted in the 1850s by George Belden Crane, the first grower
to transplant European viticulture to Napa. Ignore the immediate
surroundings --- the drooping electrical wires and paved yards --- and
at the end of a winter day, with a pink sun falling behind the Mayacamas
Mountains in the background, the rows of trellised vines look as
picturesque as any \#winecountry social media post.

``Look at the uniformity of the rocks!'' Mr. Beckstoffer said, cupping
one the size of an apple. Beckstoffer grapes are renowned for their
consistency, the result of exacting and technology-driven farming,
including heavy use of fertilizers. But most of what makes the heritage
vineyards superlative is a mystery. ``People say it's the soil, or the
climate,'' Mr. Beckstoffer likes to say. ``The truth is, we don't
know.''

As he drove away, he gestured with the back of his hand at the valley's
suburban sprawl --- light, by California standards. ``There are very few
places in the world where agriculture is the long-term, highest economic
value, best use of the land,'' he said. Napa used to have many vineyards
as exceptional as Dr. Crane, he said, but now ``they have a big house on
top of them.'' Asked how long he intends to own the property, he said,
``Forever. Even if disease wipes it out, it will be a field.''

When he talks of Napa farming, Mr. Beckstoffer speaks loftily, invoking
paragons of American culture like skyscrapers and jazz. ``You have to
have a larger cause, something bigger than money,'' he said. ``This
place is a national treasure. Napa Valley put American food and wine on
the map.''

Mr. Beckstoffer's holdings here total only about 1,000 acres, or roughly
2 percent of the valley's planted area. But thanks to his
near-stranglehold on prime vineyards like To Kalon and Dr. Crane, he can
demand almost whatever price he wants for his product. Decades ago, he
settled on a formula borrowed from Burgundy: For a ton of grapes, he
would charge 100 times the price of a bottle made with them. In other
words, if a bottle made from cabernet sauvignon grown at Dr. Crane
retails for \$150, the cost of buying the fruit equals \$15,000 per ton.

He also requires winemakers to put his name on their labels --- in
effect, making them do his marketing for him. Some find it coercive, but
Mr. Beckstoffer compares the arrangement to the ``Intel Inside'' logo
found on Windows PCs.

``He's in a position where he can do that,'' said Tor Kenward, a
winemaker who makes cabernets with Beckstoffer grapes, retailing for
\$200 to \$300. ``Some winemakers are uncomfortable with the terms, but
most think it's worth the price.'' At B Cellars, Mr. Strohl's wine cave
features a shrine-like Beckstoffer Heritage Room. ``I'll pay the price
because I know I'll get consistently excellent grapes, and I can make
stellar wine,'' he said.

The price of Napa bottles has risen year after year to ever-more
incomprehensible heights --- \$1,000 for cult brands such as Screaming
Eagle and Colgin --- creating a seemingly invincible aura of prestige.
As Mr. Beckstoffer likes to say: ``You put `Napa Valley' on a
toothpaste, you can sell it as a luxury product.''

The question is whether the category will continue to thrive as its most
lucrative demographic, the baby boomers, ages out of its prime
consumption years and a new cohort takes their place --- or doesn't.

\hypertarget{plagues-come-plagues-go}{%
\subsection{Plagues come, plagues go}\label{plagues-come-plagues-go}}

Every year, Napa awaits the publication of Silicon Valley Bank's ``State
of the U.S. Wine Industry'' analysis. In an interview, Mr. McMillan said
his increasingly vocal warnings of the millennial threat to the industry
were finally being heard.

``Every piece of research shows they're lagging,'' he said. ``It's not
that they don't drink wine. There are just other choices. In the 1990s,
there was incredible wage growth, but beer sucked. Now, guess what? Beer
is good. And so are spirits.''

Mr. McMillan said he thought that in the short term, the coronavirus
pandemic might benefit the premium wine industry, with data showing
locked-down consumers
``\href{https://svbwine.blogspot.com/2020/05/post-lock-down-opportunity-for-wineries.html}{willing
to spend up},'' perhaps as they try to recreate the restaurant
experience at home. But the larger picture is not encouraging. When
national crises come, so does a sense that we are all going to be more
serious, more responsible, and stop buying expensive bottles of
cabernet. After Sept. 11, 2001, terrorist attacks and the 2008 financial
crisis, the luxury wine market took painful hits. In such times, people
don't stop drinking; they just buy less of the expensive stuff.

When I caught up with Mr. Beckstoffer again in March, by phone, he was
sequestered at home. The pandemic had, if anything, helped him come to
terms with his basket of concerns, putting them in perspective.

I asked him to compare the crisis to earlier troubles in his career of
half a century. In the 1970s, to buy Heublein's land, he went into debt,
then saw the price of grapes crash, causing him to default on loans and
go into forfeiture. As he was beginning to recover, phylloxera hit and
many wineries went under. He saw it as an opportunity. ``When hard times
hit, people sold,'' he said. ``That's when we bought a lot of our
vineyards.''

He seemed sanguine about the Covid-19 economic crash. ``In this
business, we tend to get seven or eight good years, then two or three
bad ones,'' he said. In the mind of a farmer, plagues come and go.

The millenniums, however, still haunted him. ``Millennials,'' he
corrected himself.

``We'll figure it out,'' he continued. ``A well-managed business will
always weather the storm. Nobody in Napa Valley is panicking. No land
that's really good is for sale that I've seen. If something comes up,
I'll probably buy it.''

Image

Mr. Beckstoffer~values his empire at \$500 million.Credit...Cayce
Clifford for The New York Times

Advertisement

\protect\hyperlink{after-bottom}{Continue reading the main story}

\hypertarget{site-index}{%
\subsection{Site Index}\label{site-index}}

\hypertarget{site-information-navigation}{%
\subsection{Site Information
Navigation}\label{site-information-navigation}}

\begin{itemize}
\tightlist
\item
  \href{https://help.nytimes3xbfgragh.onion/hc/en-us/articles/115014792127-Copyright-notice}{©~2020~The
  New York Times Company}
\end{itemize}

\begin{itemize}
\tightlist
\item
  \href{https://www.nytco.com/}{NYTCo}
\item
  \href{https://help.nytimes3xbfgragh.onion/hc/en-us/articles/115015385887-Contact-Us}{Contact
  Us}
\item
  \href{https://www.nytco.com/careers/}{Work with us}
\item
  \href{https://nytmediakit.com/}{Advertise}
\item
  \href{http://www.tbrandstudio.com/}{T Brand Studio}
\item
  \href{https://www.nytimes3xbfgragh.onion/privacy/cookie-policy\#how-do-i-manage-trackers}{Your
  Ad Choices}
\item
  \href{https://www.nytimes3xbfgragh.onion/privacy}{Privacy}
\item
  \href{https://help.nytimes3xbfgragh.onion/hc/en-us/articles/115014893428-Terms-of-service}{Terms
  of Service}
\item
  \href{https://help.nytimes3xbfgragh.onion/hc/en-us/articles/115014893968-Terms-of-sale}{Terms
  of Sale}
\item
  \href{https://spiderbites.nytimes3xbfgragh.onion}{Site Map}
\item
  \href{https://help.nytimes3xbfgragh.onion/hc/en-us}{Help}
\item
  \href{https://www.nytimes3xbfgragh.onion/subscription?campaignId=37WXW}{Subscriptions}
\end{itemize}
