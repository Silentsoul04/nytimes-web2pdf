Sections

SEARCH

\protect\hyperlink{site-content}{Skip to
content}\protect\hyperlink{site-index}{Skip to site index}

\href{/section/politics}{Politics}\textbar{}Fight Over Virus's Death
Toll Opens Grim New Front in Election Battle

\url{https://nyti.ms/2Lg6U2h}

\begin{itemize}
\item
\item
\item
\item
\item
\item
\end{itemize}

\begin{itemize}
\item
  \href{https://www.nytimes3xbfgragh.onion/2020/09/12/us/politics/biden-trump-poll-wisconsin-minnesota.html?action=click\&pgtype=Article\&state=default\&region=TOP_BANNER\&context=storylines_menu}{New
  York Times Poll}
\item
  \href{https://www.nytimes3xbfgragh.onion/interactive/2020/us/elections/election-states-biden-trump.html?action=click\&pgtype=Article\&state=default\&region=TOP_BANNER\&context=storylines_menu}{Paths
  to 270}
\item
  \href{https://www.nytimes3xbfgragh.onion/interactive/2019/us/elections/2020-presidential-election-calendar.html?action=click\&pgtype=Article\&state=default\&region=TOP_BANNER\&context=storylines_menu}{Voting
  Deadlines}
\item
  \href{https://www.nytimes3xbfgragh.onion/interactive/2020/08/31/us/politics/vote-by-mail-deadlines.html?action=click\&pgtype=Article\&state=default\&region=TOP_BANNER\&context=storylines_menu}{Voting
  by Mail}
\item
  \href{https://www.nytimes3xbfgragh.onion/newsletters/politics?action=click\&pgtype=Article\&state=default\&region=TOP_BANNER\&context=storylines_menu}{Politics
  Newsletter}
\end{itemize}

\includegraphics{https://static01.graylady3jvrrxbe.onion/images/2020/05/08/us/00virus-deathtoll-1/merlin_171341274_380d6d07-bc72-4992-a360-a69cc0796728-articleLarge.jpg?quality=75\&auto=webp\&disable=upscale}

\hypertarget{fight-over-viruss-death-toll-opens-grim-new-front-in-election-battle}{%
\section{Fight Over Virus's Death Toll Opens Grim New Front in Election
Battle}\label{fight-over-viruss-death-toll-opens-grim-new-front-in-election-battle}}

Elements of the right have sought to bolster President Trump's political
standing by turning scientific questions into political issues.

Bodies were moved into a refrigerated truck last month outside Wyckoff
Heights Medical Center in Brooklyn.Credit...Dave Sanders for The New
York Times

Supported by

\protect\hyperlink{after-sponsor}{Continue reading the main story}

By
\href{https://www.nytimes3xbfgragh.onion/by/matthew-rosenberg}{Matthew
Rosenberg} and
\href{https://www.nytimes3xbfgragh.onion/by/jim-rutenberg}{Jim
Rutenberg}

\begin{itemize}
\item
  Published May 9, 2020Updated May 11, 2020
\item
  \begin{itemize}
  \item
  \item
  \item
  \item
  \item
  \item
  \end{itemize}
\end{itemize}

The claim was tailor-made for President Trump's most steadfast backers:
Federal guidelines are coaching doctors to mark Covid-19 as the cause of
death even when it is not, inflating the pandemic's death toll.

That the claim came from a doctor, Scott Jensen, who also happens to be
a Republican state senator in Minnesota, made it all the more alluring
to the president's allies. Never mind the experts who said that, if
anything,
\href{https://www.nytimes3xbfgragh.onion/interactive/2020/04/28/us/coronavirus-death-toll-total.html}{the
death toll was being vastly undercounted}.

``SHOCKING,'' tweeted Chris Berg, a conservative television show host on
West Dakota Fox, a Fox affiliate in North Dakota, after interviewing Dr.
Jensen last month. Soon after, Laura Ingraham, the Fox News host,
\href{https://www.youtube.com/watch?v=4kId_Di-wZA}{invited Dr. Jensen
onto her show}. His assertions were picked up by Infowars, the
conspiracy-oriented website founded by Alex Jones. They were shared by
followers of Qanon, who subscribe to a web of vague, baseless theories
that a secret cabal in the government is trying to take down the
president.

\includegraphics{https://static01.graylady3jvrrxbe.onion/images/2020/05/08/reader-center/00virus-deathtoll-2/merlin_172331409_a7eba88a-4701-4be9-b867-c7eafb3e2ca0-articleLarge.jpg?quality=75\&auto=webp\&disable=upscale}

``What is the primary benefit to keep public in mass-hysteria re:
Covid-19? Think voting. Are you awake yet?'' a Qanon follower known as
John the White wrote on Twitter, saying the pandemic was being used to
manipulate the electorate.

The likes of John the White may view the world through the most
conspiratorial of lenses, but they are hardly the only people weighing
the political impact of the virus's death toll. With implications for
how quickly businesses and their employees return to something like
normalcy, the fight to shape the official record is adding a grim new
front to the presidential campaign.

Since the outset of the crisis, elements of the right have sought to
bolster the president's political standing and justify reopening the
economy by questioning the death toll. Climate-change skeptics have
employed techniques perfected in the fight over global warming to raise
doubts about the deadliness of the virus. Others, including Mr. Trump's
media allies as well as some in the anti-vaccine movement, have
repurposed fringe theories about ``deep state'' bureaucrats undermining
the president to argue that the official numbers should not be trusted.

They have a found a receptive audience, and a booster of their ideas, in
Mr. Trump himself. For the president, the death toll has become a
pivotal political indicator, as important to his re-election prospects
as his approval ratings and his standing against former Vice President
\href{https://www.nytimes3xbfgragh.onion/interactive/2020/us/elections/joe-biden.html}{Joseph
R. Biden Jr.} in swing-state polls.

Late last month, with the number of dead in the United States
approaching 75,000, according to figures compiled by The New York Times,
projections foresaw another spike in Covid-19 cases and deaths as
social-distancing rules relaxed. One draft government report projected
as many as 3,000 deaths a day by the end of May. Yet according to
administration officials, Mr. Trump has begun privately questioning the
models and the official death statistics.

His skepticism is shared by others in an administration that has
regularly disregarded the advice of scientists. On Tuesday, the White
House Council of Economic Advisers released a model that
\href{https://twitter.com/WhiteHouseCEA/status/1257680258364555264?s=20}{showed
deaths dropping to zero} by the middle of May. The projection, which the
council suggested was to ''inform policymakers,'' appeared to ignore the
conventions used by epidemiologists and was roundly dismissed by
experts. But it did provide a useful counterpoint to those who argue it
is too soon to reopen the economy.

At the same time, the president has increasingly picked up on talk from
the political fringes of inflated death counts and plots to ensure his
defeat in November.

Image

Dr. Deborah Birx, the coronavirus task force coordinator, at a White
House briefing on the state of the pandemic.Credit...Erin Schaff/The New
York Times

In late April, as the toll approached 60,000, Mr. Trump retweeted a post
by a former New York City police official that claimed the number
\href{https://twitter.com/johncardillo/status/1254481206936326145?s=20}{was
being inflated by the same people} behind the ``failed coup attempts''
of the Mueller investigation and Mr. Trump's impeachment.

``Do you really think these lunatics wouldn't inflate the mortality
rates by underreporting the infection rates in an attempt to steal the
election?'' the post said.

\hypertarget{a-familiar-playbook}{%
\subsection{A Familiar Playbook}\label{a-familiar-playbook}}

At the forefront of the fight are a number of climate skeptics who have
long exploited the imperfections of scientific research --- statistical
margins of error, the subjective elements of projective modeling --- to
cast doubt on the conclusive finding that humans have contributed to
global warming.

\href{https://webcache.googleusercontent.com/search?q=cache:yq9FTH6O0LIJ:https://twitter.com/JunkScience/status/1232675902229811201+\&cd=11\&hl=en\&ct=clnk\&gl=us}{Steven
J. Milloy}, a fervent denier of that scientific consensus, was early to
play down the coronavirus threat. He compared it to the flu, an argument
that
\href{https://www.salon.com/2020/03/22/we-have-to-stop-comparing-this-coronavirus-to-the-flu_partner/}{public
health}
\href{https://www.foxnews.com/politics/fauci-explains-why-coronavirus-is-worse-than-flu-warns-against-americans-fleeing-europe-immediately}{officials}
say dangerously underestimates how deadly the virus is.

One policy group that has expressed skepticism about climate change, the
Heartland Institute, pointed to a widely used projection of 60,000
deaths to attack earlier models predicting up to two million fatalities.
The critique, posted on its website on April 17, ignored the fact that
the lower estimate took into account social-distancing measures, and
that the high estimate and
\href{https://www.nytimes3xbfgragh.onion/2020/03/13/us/coronavirus-deaths-estimate.html}{others
close to it} were presented as worst-case scenarios if no steps were
taken to mitigate the virus's spread. (The 60,000-death projection was
rendered null and void 13 days later, when the death toll surpassed that
number.)

Few of those who tacked from climate skepticism to Covid-19 denialism
have any real expertise in tracking pandemics. But several are funded by
industries that have long sought to question the work of scientists,
such as big oil companies like Exxon Mobil and tobacco companies like
Philip Morris. They are also backed by conservative groups like the
Mercer Family Foundation that hold immense sway inside the Trump White
House, and are deeply invested in the president's political future.

Image

Rebekah Mercer and her father, Robert, at a conference in 2017
questioning the science of climate change. The Mercer Family Foundation
has been involved in the conversation around Covid-19.Credit...Oliver
Contreras/The Washington Post, via Getty Images

``It's the same individuals. It's the same modus operandi, the same
organizations and the same backers,'' said Michael E. Mann, who directs
the Earth System Science Center at Pennsylvania State University.
``Right-wing conservative interests that are benefiting from the Trump
presidency obviously want to see a continuation with the Trump
presidency.''

The lines of attack against the conclusions of health experts are
familiar to those who have studied the climate-change denial movement,
which has long relied on what Naomi Oreskes, a science historian at
Harvard, called ``motivated reasoning.''

``It's, `I don't like what this implies; therefore I'm going to deny the
evidence, and I'm going to question the models, and I'm going to
question the motivations of the people who do it,''' Dr. Oreskes said.

For instance, Todd Starnes, a conservative radio host who has likened
climate change to
``\href{https://www.newsweek.com/robert-jeffress-greta-thunberg-rainbow-flood-climate-1461326}{the
Tooth Fairy},'' fed the virus ``truther'' movement when he
\href{https://www.nbcnews.com/tech/social-media/coronavirus-deniers-take-aim-hospitals-pandemic-grows-n1172336}{argued}
that the crisis was overstated because he did not see crowds outside the
Brooklyn Hospital Center in New York. It was the pandemic-era equivalent
of pointing to a snowstorm as evidence that the planet is not warming.
Days later, reporting from inside the hospital found a
\href{https://www.nytimes3xbfgragh.onion/2020/03/26/nyregion/coronavirus-brooklyn-hospital.html}{staff
overwhelmed by critical Covid cases}.

Image

Doctors treating a patient with Covid-19 at the Brooklyn Hospital
Center, which saw a surge in coronavirus patients.Credit...Victor J.
Blue for The New York Times

In an interview, James Taylor, who wrote the Heartland critique, drew a
direct line between problems he saw in the modeling of Covid-19 deaths
and climate science, arguing that in both instances ``we don't have
perfect information'' with which to make projections. ``The coronavirus
models' failure to make accurate predictions to this point should be
instructive when we are told to blindly accept certain climate models,''
he said.

Mr. Trump, for his part, has at times sought to use the uncertainty to
his advantage. Last month, after his most ardent supporters had attacked
the worst-case death estimates for weeks as evidence of hysteria, Mr.
Trump began calling attention to the two million figure --- as a
benchmark against which to judge his handling of the crisis.

Then he went further, pointing to 100,000 deaths as the number against
which to judge him.

``We will be lower than that number,'' Mr. Trump told reporters as the
death count kept by Johns Hopkins University approached 38,000. ``But I
really believe it could have been millions of people had we not done
what we did.''

Last Sunday, though, Mr. Trump acknowledged that the toll could hit
100,000. Still, he said, it could have been much worse had his
administration not acted. ``If we didn't do it, the minimum we would
have lost was a million two, a million four, a million five, that's the
minimum. We would have lost probably higher. It's possible higher than
2.2.''

\hypertarget{limited-data}{%
\subsection{Limited Data}\label{limited-data}}

Even under the best circumstances, modeling how a pandemic will play
out, like modeling the pace and impact of climate change, is an
imperfect science. And there is indeed great uncertainty about what the
death toll is now --- and what it will be --- given limited data about
the new coronavirus and the different counting methods jurisdictions are
using.

``There's a real set of challenges around the statistics --- let's be
clear,'' said Dr. Ezekiel Emanuel of the University of Pennsylvania, who
helped design Obamacare.

Image

A testing tent in Brooklyn, one of the regions hit hardest by the
virus.Credit...James Estrin/The New York Times

But in his estimation, conservatives questioning official statistics are
mostly seeking evidence that the numbers are exaggerated. ``They're not
looking at the full range of data, and if anything, there's an
undercount, not an overcount,'' he said.

Many of those conservatives have zeroed in on a recommendation by the
Centers for Disease Control and Prevention to add Covid-19 as a
``presumed'' cause of death even if the diagnosis is not confirmed by a
test. The recommendation was partly necessitated by the nationwide lag
in testing. Public health officials across the country say that even
with the additional ``presumed'' classifications on death certificates,
the actual toll is probably much higher.

It was those recommendations that Dr. Jensen, the Minnesota state
senator, seized on when he questioned the death toll in a series of
social media posts. He also questioned whether hospitals were
overreporting cases because Medicare was offering
\href{https://www.factcheck.org/2020/04/hospital-payments-and-the-covid-19-death-count/}{higher
payments for treating coronavirus patients}.

The posts, and subsequent media appearances, prompted the Minnesota
health commissioner, Jan Malcolm,
\href{https://www.fox9.com/news/minnesota-health-commissioner-criticizes-claims-of-inflated-covid-19-death-counts}{to
call Dr. Jensen's claim ``misinformation.''} Dr. Anthony S. Fauci, the
federal government's leading infectious disease specialist, called it a
conspiracy theory.

Dr. Jensen has continued to question the death toll. In a recent
interview, he bristled at being called a conspiracy theorist. ``I'm
surprised by the vehemence, surprised by the viciousness,'' he said.

Yet Dr. Jensen chose to air his concerns in partisan venues that are
hardly known for measured and thoughtful debate. After his first
television appearance, the host, Mr. Berg, pointedly asked on Twitter,
``Why is \#MN inflating Covid-19 death numbers?''

Ms. Ingraham invited Dr. Jensen on Fox News to repeat his claim and
address Dr. Fauci's charge, asking incredulously, ``Conspiracy theories,
doctor --- so you're engaging in conspiracy theories?''

Fox News's prime-time lineup has often been a clarion for doubt about
the pandemic's severity and the credibility of the nation's leading
health experts.

Image

Dr. Birx in March showed the projected number of deaths per
day.Credit...Erin Schaff/The New York Times

Beyond her segment with Dr. Jensen, Ms. Ingraham gave a platform to a
false and misleading claim by Dr. Phil McGraw, the television therapist,
that Covid-19 posed less of a public health threat than swimming pools.
While calling for reopening the economy, she has seized on discrepancies
in projections to argue that social-distancing measures have gone too
far.

Others on Fox, like Brit Hume, have pointed to New York as evidence that
numbers were being inflated, citing the city's decision to add presumed
cases to its count.

While Mr. Trump has proved receptive to such arguments, they appear to
be having less of an impact on public opinion. The vast majority of
Americans --- Republicans and Democrats alike --- are following
social-distancing guidelines, and recent polls have found
\href{https://www.washingtonpost.com/politics/americans-support-state-restrictions-on-businesses-and-halt-to-immigration-during-virus-outbreak-post-u-md-poll-finds/2020/04/27/763249ee-88af-11ea-9dfd-990f9dcc71fc_story.html?utm_campaign=wp_the_daily_202\&utm_medium=email\&utm_source=newsletter\&wpisrc=nl_daily202}{broad
support for restrictions} on businesses imposed by state governments.

To Dr. Mann, the seeming inability of Covid skeptics to sow doubts among
the public is cause for optimism. ``This is sort of a test case for
combating denialism and exposing the danger of denialism,'' he said.

Dr. Jensen, though, has stayed true to his skepticism in his own life.
Last week, he plugged into a remote State Senate hearing on easing
restrictions on telemedicine for addiction disorders while playing a
round of golf,
\href{https://www.startribune.com/gop-state-senator-in-spotlight-for-golfing-during-zoom-hearing/570133912/}{without
a mask}.

Image

Dr. Jensen called in to a hearing last month from a golf course where he
was not wearing a face mask.

Though he appeared to be trying to hide his location by holding his
phone camera close to his face, the background
\href{https://www.youtube.com/watch?v=Lmgc1MIuHUA}{whooshing of a golf
club and visible canopy of his cart} gave him away.

``I just want to ask the senator --- how's he hitting them out there?''
a Democratic senator, Jeff Hayden, broke in to ask him.

\hypertarget{our-2020-election-guide}{%
\section{Our 2020 Election Guide}\label{our-2020-election-guide}}

Updated ~Sept. 12, 2020

\begin{center}\rule{0.5\linewidth}{\linethickness}\end{center}

\begin{itemize}
\item ~
  \hypertarget{the-latest}{%
  \subsection{The Latest}\label{the-latest}}

  \begin{itemize}
  \item
    President Trump has failed to erase Joseph R. Biden Jr.'s lead
    across a set of key swing states,
    \href{https://www.nytimes3xbfgragh.onion/2020/09/12/us/politics/biden-trump-poll-wisconsin-minnesota.html?action=click\&pgtype=Article\&state=default\&region=BELOW_MAIN_CONTENT\&context=storylines_guide}{according
    to a poll}~conducted by The Times and Siena College.
  \end{itemize}
\item ~
  \hypertarget{paths-to-270}{%
  \subsection{Paths to 270}\label{paths-to-270}}

  \begin{itemize}
  \item
    Joe Biden and Donald Trump need 270 electoral votes to reach the
    White House. Try building
    \href{https://www.nytimes3xbfgragh.onion/interactive/2020/us/elections/election-states-biden-trump.html?action=click\&pgtype=Article\&state=default\&region=BELOW_MAIN_CONTENT\&context=storylines_guide}{your
    own coalition of battleground states}~to see potential outcomes.
  \end{itemize}
\item ~
  \hypertarget{voting-deadlines}{%
  \subsection{Voting Deadlines}\label{voting-deadlines}}

  \begin{itemize}
  \item
    Early voting for the presidential election starts in September~in
    some states. Take a look at
    \href{https://www.nytimes3xbfgragh.onion/interactive/2019/us/elections/2020-presidential-election-calendar.html?action=click\&pgtype=Article\&state=default\&region=BELOW_MAIN_CONTENT\&context=storylines_guide}{key
    dates}\href{https://www.nytimes3xbfgragh.onion/interactive/2019/us/elections/2020-presidential-election-calendar.html?action=click\&pgtype=Article\&state=default\&region=BELOW_MAIN_CONTENT\&context=storylines_guide}{where
    you
    liv}\href{https://www.nytimes3xbfgragh.onion/interactive/2019/us/elections/2020-presidential-election-calendar.html?action=click\&pgtype=Article\&state=default\&region=BELOW_MAIN_CONTENT\&context=storylines_guide}{e}.
    If you're voting by
    mail,~\href{https://www.nytimes3xbfgragh.onion/interactive/2020/08/31/us/politics/vote-by-mail-deadlines.html?action=click\&pgtype=Article\&state=default\&region=BELOW_MAIN_CONTENT\&context=storylines_guide}{it's
    risky to procrastinate}.
  \item
    \href{https://www.nytimes3xbfgragh.onion/interactive/2020/us/elections/joe-biden.html?action=click\&pgtype=Article\&state=default\&region=BELOW_MAIN_CONTENT\&context=storylines_guide}{}

    \hypertarget{joe-biden}{%
    \section{Joe Biden}\label{joe-biden}}

    \hypertarget{democrat}{%
    \subsection{Democrat}\label{democrat}}

    \href{https://www.nytimes3xbfgragh.onion/interactive/2020/us/elections/donald-trump.html?action=click\&pgtype=Article\&state=default\&region=BELOW_MAIN_CONTENT\&context=storylines_guide}{}

    \hypertarget{donald-trump}{%
    \section{Donald Trump}\label{donald-trump}}

    \hypertarget{republican}{%
    \subsection{Republican}\label{republican}}
  \end{itemize}
\item
  \hypertarget{keep-up-with-our-coverage}{%
  \subsection{Keep Up With Our
  Coverage}\label{keep-up-with-our-coverage}}

  \begin{itemize}
  \item
    Get an
    \href{https://www.nytimes3xbfgragh.onion/newsletters/politics?action=click\&pgtype=Article\&state=default\&region=BELOW_MAIN_CONTENT\&context=storylines_guide}{email}~recapping
    the day's news
  \item
    Download our mobile app on
    \href{https://apps.apple.com/us/app/nytimes/id284862083?ls=1\&mat_click_id=5c79ae7455014fd1bd66b5610c05b8f2-20191112-16948\&referrer=mat_click_id\%3D5c79ae7455014fd1bd66b5610c05b8f2-20191112-16948\%26link_click_id\%3D722930677036718082}{iOS}~and
    \href{http://a.localytics.com/android?id=com.nytimes.android\&referrer=utm_source\%3Dother_nyt_mobile_web\%26utm_medium\%3DWeb\%2520page\%26utm_term\%3DGeneral\%2520Mobile\%2520Page\%26utm_campaign\%3DNYT\%2520Mobile\%2520General\%2520Page}{Android}~and
    turn on Breaking News and Politics alerts
  \end{itemize}
\end{itemize}

Advertisement

\protect\hyperlink{after-bottom}{Continue reading the main story}

\hypertarget{site-index}{%
\subsection{Site Index}\label{site-index}}

\hypertarget{site-information-navigation}{%
\subsection{Site Information
Navigation}\label{site-information-navigation}}

\begin{itemize}
\tightlist
\item
  \href{https://help.nytimes3xbfgragh.onion/hc/en-us/articles/115014792127-Copyright-notice}{©~2020~The
  New York Times Company}
\end{itemize}

\begin{itemize}
\tightlist
\item
  \href{https://www.nytco.com/}{NYTCo}
\item
  \href{https://help.nytimes3xbfgragh.onion/hc/en-us/articles/115015385887-Contact-Us}{Contact
  Us}
\item
  \href{https://www.nytco.com/careers/}{Work with us}
\item
  \href{https://nytmediakit.com/}{Advertise}
\item
  \href{http://www.tbrandstudio.com/}{T Brand Studio}
\item
  \href{https://www.nytimes3xbfgragh.onion/privacy/cookie-policy\#how-do-i-manage-trackers}{Your
  Ad Choices}
\item
  \href{https://www.nytimes3xbfgragh.onion/privacy}{Privacy}
\item
  \href{https://help.nytimes3xbfgragh.onion/hc/en-us/articles/115014893428-Terms-of-service}{Terms
  of Service}
\item
  \href{https://help.nytimes3xbfgragh.onion/hc/en-us/articles/115014893968-Terms-of-sale}{Terms
  of Sale}
\item
  \href{https://spiderbites.nytimes3xbfgragh.onion}{Site Map}
\item
  \href{https://help.nytimes3xbfgragh.onion/hc/en-us}{Help}
\item
  \href{https://www.nytimes3xbfgragh.onion/subscription?campaignId=37WXW}{Subscriptions}
\end{itemize}
