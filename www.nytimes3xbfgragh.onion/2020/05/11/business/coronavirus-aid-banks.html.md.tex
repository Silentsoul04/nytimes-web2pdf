Sections

SEARCH

\protect\hyperlink{site-content}{Skip to
content}\protect\hyperlink{site-index}{Skip to site index}

\href{https://www.nytimes3xbfgragh.onion/section/business}{Business}

\href{https://myaccount.nytimes3xbfgragh.onion/auth/login?response_type=cookie\&client_id=vi}{}

\href{https://www.nytimes3xbfgragh.onion/section/todayspaper}{Today's
Paper}

\href{/section/business}{Business}\textbar{}As Banks Stumble in
Delivering Aid, Congress Weighs Other Options

\url{https://nyti.ms/3bmRZ0X}

\begin{itemize}
\item
\item
\item
\item
\item
\end{itemize}

\hypertarget{the-coronavirus-outbreak}{%
\subsubsection{\texorpdfstring{\href{https://www.nytimes3xbfgragh.onion/news-event/coronavirus?name=styln-coronavirus-markets\&region=TOP_BANNER\&block=storyline_menu_recirc\&action=click\&pgtype=Article\&impression_id=ae7ae740-f27f-11ea-8789-45ae9f4c38e2\&variant=undefined}{The
Coronavirus
Outbreak}}{The Coronavirus Outbreak}}\label{the-coronavirus-outbreak}}

\begin{itemize}
\tightlist
\item
  live\href{https://www.nytimes3xbfgragh.onion/2020/09/08/world/covid-19-coronavirus.html?name=styln-coronavirus-markets\&region=TOP_BANNER\&block=storyline_menu_recirc\&action=click\&pgtype=Article\&impression_id=ae7b0e50-f27f-11ea-8789-45ae9f4c38e2\&variant=undefined}{Latest
  Updates}
\item
  \href{https://www.nytimes3xbfgragh.onion/interactive/2020/us/coronavirus-us-cases.html?name=styln-coronavirus-markets\&region=TOP_BANNER\&block=storyline_menu_recirc\&action=click\&pgtype=Article\&impression_id=ae7b0e51-f27f-11ea-8789-45ae9f4c38e2\&variant=undefined}{Maps
  and Cases}
\item
  \href{https://www.nytimes3xbfgragh.onion/interactive/2020/science/coronavirus-vaccine-tracker.html?name=styln-coronavirus-markets\&region=TOP_BANNER\&block=storyline_menu_recirc\&action=click\&pgtype=Article\&impression_id=ae7b0e52-f27f-11ea-8789-45ae9f4c38e2\&variant=undefined}{Vaccine
  Tracker}
\item
  \href{https://www.nytimes3xbfgragh.onion/2020/09/02/your-money/eviction-moratorium-covid.html?name=styln-coronavirus-markets\&region=TOP_BANNER\&block=storyline_menu_recirc\&action=click\&pgtype=Article\&impression_id=ae7b0e53-f27f-11ea-8789-45ae9f4c38e2\&variant=undefined}{Eviction
  Moratorium}
\item
  \href{https://www.nytimes3xbfgragh.onion/interactive/2020/09/02/magazine/food-insecurity-hunger-us.html?name=styln-coronavirus-markets\&region=TOP_BANNER\&block=storyline_menu_recirc\&action=click\&pgtype=Article\&impression_id=ae7b0e54-f27f-11ea-8789-45ae9f4c38e2\&variant=undefined}{American
  Hunger}
\end{itemize}

Advertisement

\protect\hyperlink{after-top}{Continue reading the main story}

Supported by

\protect\hyperlink{after-sponsor}{Continue reading the main story}

\hypertarget{as-banks-stumble-in-delivering-aid-congress-weighs-other-options}{%
\section{As Banks Stumble in Delivering Aid, Congress Weighs Other
Options}\label{as-banks-stumble-in-delivering-aid-congress-weighs-other-options}}

Accounts provided by the Federal Reserve and distributions handled by
payroll processors are among the ideas floating around Capitol Hill.

\includegraphics{https://static01.graylady3jvrrxbe.onion/images/2020/05/11/business/11virus-bank2/11virus-bank2-articleLarge.jpg?quality=75\&auto=webp\&disable=upscale}

\href{https://www.nytimes3xbfgragh.onion/by/emily-flitter}{\includegraphics{https://static01.graylady3jvrrxbe.onion/images/2019/06/19/reader-center/author-emily-flitter/author-emily-flitter-thumbLarge.png}}\href{https://www.nytimes3xbfgragh.onion/by/emily-cochrane}{\includegraphics{https://static01.graylady3jvrrxbe.onion/images/2018/11/28/multimedia/author-emily-cochrane/author-emily-cochrane-thumbLarge-v3.png}}

By \href{https://www.nytimes3xbfgragh.onion/by/emily-flitter}{Emily
Flitter} and
\href{https://www.nytimes3xbfgragh.onion/by/emily-cochrane}{Emily
Cochrane}

\begin{itemize}
\item
  Published May 11, 2020Updated June 4, 2020
\item
  \begin{itemize}
  \item
  \item
  \item
  \item
  \item
  \end{itemize}
\end{itemize}

When the federal government agreed to funnel \$2.2 trillion in emergency
aid to Americans devastated by the economic shutdown, the nation's banks
were given a central role.

There were three main prongs of relief for taxpayers and American
businesses, all routed through the banks in various ways: stimulus
checks, a \$660 billion package for small businesses and unemployment
benefits. Confronted with an unprecedented crush of need as millions of
Americans lost their livelihoods, the banks stumbled in ways big and
small.

The small-business aid, the Paycheck Protection Program, had a chaotic
introduction and ran dry within days. Some banks
\href{https://www.nytimes3xbfgragh.onion/2020/04/16/business/stimulus-paychecks-garnish-banks.html}{withheld
stimulus cash} from people with overdrawn accounts. And some banks'
debit cards, used to distribute unemployment benefits, didn't work
properly.

Several lawmakers have begun exploring ways to sidestep banks to deliver
aid. Among the proposals, mainly from Democrats: using Internal Revenue
Service records and payroll processing companies, as well as the Federal
Reserve, to help distribute money more swiftly.

Democrats are also pushing for additional stimulus funds, but it's not
clear that more aid will win approval, especially with top Republicans
urging restraint. Top Democrats in the House, who are working on their
own coronavirus legislation, want to expand the existing programs,
including a provision for lending exclusively through nonprofits or
mission-focused lenders known as
\href{https://www.nytimes3xbfgragh.onion/2020/06/04/business/minority-businesses-damage-lenders.html}{community
development financial institutions}, which lend to poor communities not
served by banks.

Senator Doug Jones, Democrat of Alabama, is pushing a plan to give small
businesses another round of help in paying employees by using the
services of payroll processors, which already distribute wages for close
to 40 percent of U.S. businesses. And companies that don't use payroll
processors could get payouts directly from the I.R.S.

``Another option makes it easier and takes a little pressure off the
banks,'' Mr. Jones said. ``They've been overwhelmed.'' He had urged
fellow lawmakers to consider using payroll companies rather than banks
when the first installment of the paycheck program was taking shape.

\includegraphics{https://static01.graylady3jvrrxbe.onion/images/2020/05/11/business/11virrus-banks1/merlin_171588537_87890593-c5ab-446f-9236-580b84d02e6e-articleLarge.jpg?quality=75\&auto=webp\&disable=upscale}

The Paycheck Protection Program was a highlight of the CARES Act, but
its image suffered after several publicly traded companies exploited
loopholes to gain access to funds --- partly because of their close ties
to big banks. Many of those companies, including the restaurant operator
Ruth's Chris Steak House, have since returned the money.

``We're hoping that it's really going to get better now that Ruth's
Chris isn't supposed to be front and center,'' said Senator Ron Wyden of
Oregon, the top Democrat on the Senate Finance Committee. ``But it's
still going through a set of banking channels.''

\hypertarget{latest-updates-the-coronavirus-outbreak-and-the-economy}{%
\section{\texorpdfstring{\href{https://www.nytimes3xbfgragh.onion/live/2020/09/08/business/stock-market-today-coronavirus?action=click\&pgtype=Article\&state=default\&region=MAIN_CONTENT_1\&context=storylines_live_updates}{Latest
Updates: The Coronavirus Outbreak and the
Economy}}{Latest Updates: The Coronavirus Outbreak and the Economy}}\label{latest-updates-the-coronavirus-outbreak-and-the-economy}}

\href{https://www.nytimes3xbfgragh.onion/live/2020/09/08/business/stock-market-today-coronavirus?action=click\&pgtype=Article\&state=default\&region=MAIN_CONTENT_1\&context=storylines_live_updates\#the-latest-under-armour-announces-layoffs-and-lubys-will-liquidate}{12h
ago}

\href{https://www.nytimes3xbfgragh.onion/live/2020/09/08/business/stock-market-today-coronavirus?action=click\&pgtype=Article\&state=default\&region=MAIN_CONTENT_1\&context=storylines_live_updates\#the-latest-under-armour-announces-layoffs-and-lubys-will-liquidate}{The
latest: Under Armour announces layoffs, and Luby's will liquidate.}

\href{https://www.nytimes3xbfgragh.onion/live/2020/09/08/business/stock-market-today-coronavirus?action=click\&pgtype=Article\&state=default\&region=MAIN_CONTENT_1\&context=storylines_live_updates\#lululemon-reports-a-quarterly-profit-as-consumers-flock-to-yoga-pants}{12h
ago}

\href{https://www.nytimes3xbfgragh.onion/live/2020/09/08/business/stock-market-today-coronavirus?action=click\&pgtype=Article\&state=default\&region=MAIN_CONTENT_1\&context=storylines_live_updates\#lululemon-reports-a-quarterly-profit-as-consumers-flock-to-yoga-pants}{Lululemon
reports a quarterly profit as consumers flock to yoga pants.}

\href{https://www.nytimes3xbfgragh.onion/live/2020/09/08/business/stock-market-today-coronavirus?action=click\&pgtype=Article\&state=default\&region=MAIN_CONTENT_1\&context=storylines_live_updates\#the-work-from-home-challenge-for-employees-of-color}{14h
ago}

\href{https://www.nytimes3xbfgragh.onion/live/2020/09/08/business/stock-market-today-coronavirus?action=click\&pgtype=Article\&state=default\&region=MAIN_CONTENT_1\&context=storylines_live_updates\#the-work-from-home-challenge-for-employees-of-color}{The
work-from-home challenge for employees of color.}

\href{https://www.nytimes3xbfgragh.onion/live/2020/09/08/business/stock-market-today-coronavirus?action=click\&pgtype=Article\&state=default\&region=MAIN_CONTENT_1\&context=storylines_live_updates}{See
more updates}

More live coverage:
\href{https://www.nytimes3xbfgragh.onion/2020/09/08/world/covid-19-coronavirus.html?action=click\&pgtype=Article\&state=default\&region=MAIN_CONTENT_1\&context=storylines_live_updates}{Global}

Instead, Mr. Wyden is proposing that businesses with under \$1 million
in gross receipts --- what the business brings in before taking out
costs --- are reimbursed for a portion of payroll expenses directly from
the I.R.S.

There is also growing support among Democrats for a plan put forward by
Representative Pramila Jayapal, Democrat of Washington, that would have
the federal government directly pay 100 percent of weekly wages for
workers making up to \$100,000 and their benefits. A narrower version
proposed by Senator Josh Hawley, Republican of Missouri, would have the
government provide for 80 percent of wages,
\href{https://www.hawley.senate.gov/getting-america-back-work}{up to the
national median wage}.

Some Democrats on the Senate Banking Committee want to see the Fed
create accounts for every American, similar to how Social Security
numbers are assigned. Any future aid could then be distributed through
the central bank.

Although the architects of the CARES Act acknowledge the early missteps,
particularly with the start of the paycheck program, some lawmakers say
they would prefer to work within the confines of what they have already
created.

``We need to look at the programs that are out there, and tweak them to
get them to work better,'' said Senator Rob Portman, Republican of Ohio.
``I would hate to take it away from banks and try something else that we
haven't tried yet.''

Complicated rules and glitches in the tech platform operated by the
Small Business Administration, which is running the paycheck program,
contributed to its chaotic start. But the behavior of many banks added
to the mayhem, especially since they had wide latitude to determine whom
to lend to, and how much.

Connor Crosby runs Ignition Visuals, a video production company in
Lowell, Mass. After painstakingly calculating how much he could receive
as the sole proprietor, he asked Citizens Financial Group for \$15,000
under the paycheck program. Citizens, however, gave him only \$9,000,
without explaining why it had reduced the amount. An S.B.A. employee
told Mr. Crosby that it was his bank, not the government agency, that
determined how much money to give him. After The New York Times asked
Citizens about his case, Mr. Crosby got a call from a bank
representative saying he would be receiving the remaining \$6,000.

``Given the scale and complexity of the program and the very short time
frame available to put it in place, it is not surprising that some
errors may have been made in the early days of its execution,'' said
Peter Lucht, a Citizens spokesman. ``We are committed to identifying and
addressing such cases so that we can get these applicants the maximum
possible loan amount.''

Another Citizens customer, Alexander Ball, was told on April 20 by a
bank representative that he wouldn't get stimulus money because he had
overdrawn his account trying to pay bills after losing his piecemeal
work as a delivery driver. A week later, in response to an inquiry by
The Times, Mr. Lucht said the bank's policy was to grant overdrawn
customers access by temporarily crediting their accounts to zero out the
overdrafts --- but only if they called and asked. He said the
representative's response to Mr. Ball appeared to have been a mistake
and suggested he try again to get the bank to release his funds. Mr.
Ball eventually got the money on April 28.

Bank of America has also faced complaints that it gave small-business
owners a fraction of what they asked for.

``We have funded more than 250,000 loans so far and continue to process
applications,'' Bill Halldin, a Bank of America spokesman, said in an
email. ``When a client has a concern about their loan amount, we review
it.''

Jacob Leibenluft, a senior fellow at the Center for American Progress
and an economic adviser to President Barack Obama, said he could
understand the motivation for enlisting banks to help distribute aid.
But ``there's no question that any form of additional small-business
assistance needs to be both better targeted to businesses that are
experiencing the greatest need and take that rationing or
decision-making processes out of the banks.''

Many banks, including the country's four largest lenders and several
regional and community institutions, are now fighting lawsuits filed by
groups of would-be customers claiming they were mistreated or even
defrauded when they tried to apply for help.

Complaints have also piled up from out-of-work Americans against two
banks responsible for distributing unemployment benefits in 23 states
--- U.S. Bank and KeyBank. Many states put unemployment money on debit
cards, but those cards haven't always worked as intended, prompting the
outcry.

Last month, after complaints about nonfunctional cards issued by
KeyBank, which is handling benefit distributions for New York, the
state's labor commissioner, Roberta Reardon, advised people to have
their unemployment benefits deposited directly into their bank accounts
instead.

A KeyBank spokeswoman, Kimberly Kowalski, said that the lender had
issued benefit cards at 20 times its normal rate over the past two
months, and that its call centers were taking 10 times the normal number
of calls.

U.S. Bank has not had any systems outages affecting the distribution of
unemployment benefits, but some states had experienced problems
processing claims, a bank spokesman, David Palombi, said.

``The pandemic has created a situation that is unprecedented, and has
presented significant challenges,'' Mr. Palombi said.

Advertisement

\protect\hyperlink{after-bottom}{Continue reading the main story}

\hypertarget{site-index}{%
\subsection{Site Index}\label{site-index}}

\hypertarget{site-information-navigation}{%
\subsection{Site Information
Navigation}\label{site-information-navigation}}

\begin{itemize}
\tightlist
\item
  \href{https://help.nytimes3xbfgragh.onion/hc/en-us/articles/115014792127-Copyright-notice}{©~2020~The
  New York Times Company}
\end{itemize}

\begin{itemize}
\tightlist
\item
  \href{https://www.nytco.com/}{NYTCo}
\item
  \href{https://help.nytimes3xbfgragh.onion/hc/en-us/articles/115015385887-Contact-Us}{Contact
  Us}
\item
  \href{https://www.nytco.com/careers/}{Work with us}
\item
  \href{https://nytmediakit.com/}{Advertise}
\item
  \href{http://www.tbrandstudio.com/}{T Brand Studio}
\item
  \href{https://www.nytimes3xbfgragh.onion/privacy/cookie-policy\#how-do-i-manage-trackers}{Your
  Ad Choices}
\item
  \href{https://www.nytimes3xbfgragh.onion/privacy}{Privacy}
\item
  \href{https://help.nytimes3xbfgragh.onion/hc/en-us/articles/115014893428-Terms-of-service}{Terms
  of Service}
\item
  \href{https://help.nytimes3xbfgragh.onion/hc/en-us/articles/115014893968-Terms-of-sale}{Terms
  of Sale}
\item
  \href{https://spiderbites.nytimes3xbfgragh.onion}{Site Map}
\item
  \href{https://help.nytimes3xbfgragh.onion/hc/en-us}{Help}
\item
  \href{https://www.nytimes3xbfgragh.onion/subscription?campaignId=37WXW}{Subscriptions}
\end{itemize}
