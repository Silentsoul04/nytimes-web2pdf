Sections

SEARCH

\protect\hyperlink{site-content}{Skip to
content}\protect\hyperlink{site-index}{Skip to site index}

\href{https://www.nytimes3xbfgragh.onion/section/health}{Health}

\href{https://myaccount.nytimes3xbfgragh.onion/auth/login?response_type=cookie\&client_id=vi}{}

\href{https://www.nytimes3xbfgragh.onion/section/todayspaper}{Today's
Paper}

\href{/section/health}{Health}\textbar{}Talking Can Generate Coronavirus
Droplets That Linger Up to 14 Minutes

\url{https://nyti.ms/2zISHbp}

\begin{itemize}
\item
\item
\item
\item
\item
\item
\end{itemize}

\hypertarget{the-coronavirus-outbreak}{%
\subsubsection{\texorpdfstring{\href{https://www.nytimes3xbfgragh.onion/news-event/coronavirus?name=styln-coronavirus-national\&region=TOP_BANNER\&block=storyline_menu_recirc\&action=click\&pgtype=Article\&impression_id=50fe31c0-f1cc-11ea-96a6-ef6e07581fac\&variant=undefined}{The
Coronavirus
Outbreak}}{The Coronavirus Outbreak}}\label{the-coronavirus-outbreak}}

\begin{itemize}
\tightlist
\item
  live\href{https://www.nytimes3xbfgragh.onion/2020/09/08/world/covid-19-coronavirus.html?name=styln-coronavirus-national\&region=TOP_BANNER\&block=storyline_menu_recirc\&action=click\&pgtype=Article\&impression_id=50fe31c1-f1cc-11ea-96a6-ef6e07581fac\&variant=undefined}{Latest
  Updates}
\item
  \href{https://www.nytimes3xbfgragh.onion/interactive/2020/us/coronavirus-us-cases.html?name=styln-coronavirus-national\&region=TOP_BANNER\&block=storyline_menu_recirc\&action=click\&pgtype=Article\&impression_id=50fe58d0-f1cc-11ea-96a6-ef6e07581fac\&variant=undefined}{Maps
  and Cases}
\item
  \href{https://www.nytimes3xbfgragh.onion/interactive/2020/science/coronavirus-vaccine-tracker.html?name=styln-coronavirus-national\&region=TOP_BANNER\&block=storyline_menu_recirc\&action=click\&pgtype=Article\&impression_id=50fe58d1-f1cc-11ea-96a6-ef6e07581fac\&variant=undefined}{Vaccine
  Tracker}
\item
  \href{https://www.nytimes3xbfgragh.onion/2020/09/02/your-money/eviction-moratorium-covid.html?name=styln-coronavirus-national\&region=TOP_BANNER\&block=storyline_menu_recirc\&action=click\&pgtype=Article\&impression_id=50fe58d2-f1cc-11ea-96a6-ef6e07581fac\&variant=undefined}{Eviction
  Moratorium}
\item
  \href{https://www.nytimes3xbfgragh.onion/interactive/2020/09/02/magazine/food-insecurity-hunger-us.html?name=styln-coronavirus-national\&region=TOP_BANNER\&block=storyline_menu_recirc\&action=click\&pgtype=Article\&impression_id=50fe58d3-f1cc-11ea-96a6-ef6e07581fac\&variant=undefined}{American
  Hunger}
\end{itemize}

Advertisement

\protect\hyperlink{after-top}{Continue reading the main story}

Supported by

\protect\hyperlink{after-sponsor}{Continue reading the main story}

\hypertarget{talking-can-generate-coronavirus-droplets-that-linger-up-to-14-minutes}{%
\section{Talking Can Generate Coronavirus Droplets That Linger Up to 14
Minutes}\label{talking-can-generate-coronavirus-droplets-that-linger-up-to-14-minutes}}

A new study shows how respiratory droplets produced during normal
conversation may be just as important in transmitting disease,
especially indoors.

\includegraphics{https://static01.graylady3jvrrxbe.onion/images/2020/05/14/science/14VIRUS-DROPLETS1/merlin_172291812_41bdcc67-faf6-47b4-94b1-b5ceee51ee47-articleLarge.jpg?quality=75\&auto=webp\&disable=upscale}

\href{https://www.nytimes3xbfgragh.onion/by/knvul-sheikh}{\includegraphics{https://static01.graylady3jvrrxbe.onion/images/2020/01/03/reader-center/author-knvul-sheikh/author-knvul-sheikh-thumbLarge.png}}

By \href{https://www.nytimes3xbfgragh.onion/by/knvul-sheikh}{Knvul
Sheikh}

\begin{itemize}
\item
  Published May 14, 2020Updated June 2, 2020
\item
  \begin{itemize}
  \item
  \item
  \item
  \item
  \item
  \item
  \end{itemize}
\end{itemize}

Coughs or sneezes may not be the only way people transmit infectious
pathogens like the novel coronavirus to one another. Talking can also
launch thousands of droplets so small they can remain suspended in the
air for eight to 14 minutes, according to a new
\href{https://www.nytimes3xbfgragh.onion/2020/06/02/health/coronavirus-study.html}{study}.

The research, published Wednesday in
\href{https://www.pnas.org/content/early/2020/05/12/2006874117}{The
Proceedings of the National Academy of Sciences}, could help explain how
people with mild or no symptoms may infect others in close quarters such
as offices, nursing homes, cruise ships and other confined spaces. The
study's experimental conditions will need to be replicated in more
real-world circumstances, and researchers still don't know how much
virus has to be transmitted from one person to another to cause
infection. But its findings strengthen the case for wearing masks and
taking other precautions in such environments to reduce the spread of
the coronavirus.

Scientists agree that the coronavirus jumps from person to person most
often by hitching a ride inside tiny respiratory droplets. These
droplets tend to fall to the ground within a few feet of the person who
emits them. They may
\href{https://www.nytimes3xbfgragh.onion/2020/03/17/health/coronavirus-surfaces-aerosols.html}{land
on surfaces} like doorknobs, where people can touch lingering virus
particles and transfer them to their face. But some droplets can remain
aloft, and be inhaled by others.

Elaborate experiments have revealed how coughing or sneezing can produce
a crackling burst of air mixed with saliva or mucus that can force
hundreds of millions of influenza and other virus particles into the air
if a person is sick. A single cough can propel about 3,000 respiratory
droplets, while sneezing can generate
\href{https://www.ncbi.nlm.nih.gov/pmc/articles/PMC7132666/}{as many as
40,000}.

\href{https://www.nytimes3xbfgragh.onion/interactive/2020/04/14/science/coronavirus-transmission-cough-6-feet-ar-ul.html}{}

\includegraphics{https://static01.graylady3jvrrxbe.onion/images/2020/04/13/science/cough-image-still-promo/cough-image-still-promo-articleLarge-v2.jpg}

\hypertarget{this-3-d-simulation-shows-why-social-distancing-is-so-important}{%
\subsection{This 3-D Simulation Shows Why Social Distancing Is So
Important}\label{this-3-d-simulation-shows-why-social-distancing-is-so-important}}

We visualized a cough to show how far respiratory droplets can spread.
If you haven't been keeping your distance to fight the coronavirus, this
may persuade you.

To see how many droplets are produced during normal conversation,
researchers at the National Institute of Diabetes and Digestive and
Kidney Diseases and the University of Pennsylvania, who study the
kinetics of biological molecules inside the human body, asked volunteers
to repeat the words ``stay healthy'' several times. While the
participants spoke into the open end of a cardboard box, the researchers
illuminated its inside with green lasers, and tracked bursts of droplets
produced by the speaker.

The laser scans showed that about 2,600 small droplets were produced per
second while talking. When researchers projected the amount and size of
droplets produced at different volumes based on previous studies, they
found that speaking louder could generate larger droplets, as well as
greater quantities of them.

\hypertarget{latest-updates-the-coronavirus-outbreak}{%
\section{\texorpdfstring{\href{https://www.nytimes3xbfgragh.onion/2020/09/08/world/covid-19-coronavirus.html?action=click\&pgtype=Article\&state=default\&region=MAIN_CONTENT_1\&context=storylines_live_updates}{Latest
Updates: The Coronavirus
Outbreak}}{Latest Updates: The Coronavirus Outbreak}}\label{latest-updates-the-coronavirus-outbreak}}

Updated 2020-09-08T12:02:09.491Z

\begin{itemize}
\tightlist
\item
  \href{https://www.nytimes3xbfgragh.onion/2020/09/08/world/covid-19-coronavirus.html?action=click\&pgtype=Article\&state=default\&region=MAIN_CONTENT_1\&context=storylines_live_updates\#link-46162376}{Trillions
  of dollars separate lawmakers' proposals for virus relief.}
\item
  \href{https://www.nytimes3xbfgragh.onion/2020/09/08/world/covid-19-coronavirus.html?action=click\&pgtype=Article\&state=default\&region=MAIN_CONTENT_1\&context=storylines_live_updates\#link-679303d7}{Nine
  drugmakers pledge to thoroughly vet any coronavirus vaccine.}
\item
  \href{https://www.nytimes3xbfgragh.onion/2020/09/08/world/covid-19-coronavirus.html?action=click\&pgtype=Article\&state=default\&region=MAIN_CONTENT_1\&context=storylines_live_updates\#link-1c973131}{`The
  lockdown killed my father': Farmer suicides add to India's virus
  misery.}
\end{itemize}

\href{https://www.nytimes3xbfgragh.onion/2020/09/08/world/covid-19-coronavirus.html?action=click\&pgtype=Article\&state=default\&region=MAIN_CONTENT_1\&context=storylines_live_updates}{See
more updates}

More live coverage:
\href{https://www.nytimes3xbfgragh.onion/live/2020/09/08/business/stock-market-today-coronavirus?action=click\&pgtype=Article\&state=default\&region=MAIN_CONTENT_1\&context=storylines_live_updates}{Markets}

Although the scientists did not record speech droplets produced by
people who were sick,
\href{https://www.nature.com/articles/s41586-020-2196-x}{previous
studies} have calculated exactly how much coronavirus genetic material
can be found in oral fluids in the average patient. Based on this
knowledge, the researchers estimated that a single minute of loud
speaking could generate at least 1,000 virus-containing droplets.

\includegraphics{https://static01.graylady3jvrrxbe.onion/images/2020/05/14/science/14VIRUS-DROPLETS2/14VIRUS-DROPLETS2-articleLarge.jpg?quality=75\&auto=webp\&disable=upscale}

The scientists also found that while droplets start shrinking from
dehydration as soon as they leave a person's mouth, they can still float
in the air for eight to 14 minutes.

``These observations confirm that there is a substantial probability
that normal speaking causes airborne virus transmission in confined
environments,'' the authors wrote in the study.

The researchers acknowledged that the experiment was performed in a
controlled environment with stagnant air that may not reflect what
happens in rooms with good ventilation. But they still had reason to
believe their reported values were ``conservative lower limit
estimates'' because some people have a higher viral load, meaning they
may produce droplets with several thousand more virus particles than
average.

The Centers for Disease Control and Prevention says keeping
\href{https://www.nytimes3xbfgragh.onion/2020/04/14/health/coronavirus-six-feet.html}{at
least six feet away} from others can help people avoid contact with
respiratory droplets and lower the risk of infection. But many
scientists have argued that droplets can travel farther than six feet,
depending on the force with which droplets are launched, the surrounding
temperature, whether there are air currents that can carry them farther
and other conditions.

\href{https://www.nytimes3xbfgragh.onion/news-event/coronavirus?action=click\&pgtype=Article\&state=default\&region=MAIN_CONTENT_3\&context=storylines_faq}{}

\hypertarget{the-coronavirus-outbreak-}{%
\subsubsection{The Coronavirus Outbreak
›}\label{the-coronavirus-outbreak-}}

\hypertarget{frequently-asked-questions}{%
\paragraph{Frequently Asked
Questions}\label{frequently-asked-questions}}

Updated September 4, 2020

\begin{itemize}
\item ~
  \hypertarget{what-are-the-symptoms-of-coronavirus}{%
  \paragraph{What are the symptoms of
  coronavirus?}\label{what-are-the-symptoms-of-coronavirus}}

  \begin{itemize}
  \tightlist
  \item
    In the beginning, the coronavirus
    \href{https://www.nytimes3xbfgragh.onion/article/coronavirus-facts-history.html?action=click\&pgtype=Article\&state=default\&region=MAIN_CONTENT_3\&context=storylines_faq\#link-6817bab5}{seemed
    like it was primarily a respiratory illness}~--- many patients had
    fever and chills, were weak and tired, and coughed a lot, though
    some people don't show many symptoms at all. Those who seemed
    sickest had pneumonia or acute respiratory distress syndrome and
    received supplemental oxygen. By now, doctors have identified many
    more symptoms and syndromes. In April,
    \href{https://www.nytimes3xbfgragh.onion/2020/04/27/health/coronavirus-symptoms-cdc.html?action=click\&pgtype=Article\&state=default\&region=MAIN_CONTENT_3\&context=storylines_faq}{the
    C.D.C. added to the list of early signs}~sore throat, fever, chills
    and muscle aches. Gastrointestinal upset, such as diarrhea and
    nausea, has also been observed. Another telltale sign of infection
    may be a sudden, profound diminution of one's
    \href{https://www.nytimes3xbfgragh.onion/2020/03/22/health/coronavirus-symptoms-smell-taste.html?action=click\&pgtype=Article\&state=default\&region=MAIN_CONTENT_3\&context=storylines_faq}{sense
    of smell and taste.}~Teenagers and young adults in some cases have
    developed painful red and purple lesions on their fingers and toes
    --- nicknamed ``Covid toe'' --- but few other serious symptoms.
  \end{itemize}
\item ~
  \hypertarget{why-is-it-safer-to-spend-time-together-outside}{%
  \paragraph{Why is it safer to spend time together
  outside?}\label{why-is-it-safer-to-spend-time-together-outside}}

  \begin{itemize}
  \tightlist
  \item
    \href{https://www.nytimes3xbfgragh.onion/2020/05/15/us/coronavirus-what-to-do-outside.html?action=click\&pgtype=Article\&state=default\&region=MAIN_CONTENT_3\&context=storylines_faq}{Outdoor
    gatherings}~lower risk because wind disperses viral droplets, and
    sunlight can kill some of the virus. Open spaces prevent the virus
    from building up in concentrated amounts and being inhaled, which
    can happen when infected people exhale in a confined space for long
    stretches of time, said Dr. Julian W. Tang, a virologist at the
    University of Leicester.
  \end{itemize}
\item ~
  \hypertarget{why-does-standing-six-feet-away-from-others-help}{%
  \paragraph{Why does standing six feet away from others
  help?}\label{why-does-standing-six-feet-away-from-others-help}}

  \begin{itemize}
  \tightlist
  \item
    The coronavirus spreads primarily through droplets from your mouth
    and nose, especially when you cough or sneeze. The C.D.C., one of
    the organizations using that measure,
    \href{https://www.nytimes3xbfgragh.onion/2020/04/14/health/coronavirus-six-feet.html?action=click\&pgtype=Article\&state=default\&region=MAIN_CONTENT_3\&context=storylines_faq}{bases
    its recommendation of six feet}~on the idea that most large droplets
    that people expel when they cough or sneeze will fall to the ground
    within six feet. But six feet has never been a magic number that
    guarantees complete protection. Sneezes, for instance, can launch
    droplets a lot farther than six feet,
    \href{https://jamanetwork.com/journals/jama/fullarticle/2763852}{according
    to a recent study}. It's a rule of thumb: You should be safest
    standing six feet apart outside, especially when it's windy. But
    keep a mask on at all times, even when you think you're far enough
    apart.
  \end{itemize}
\item ~
  \hypertarget{i-have-antibodies-am-i-now-immune}{%
  \paragraph{I have antibodies. Am I now
  immune?}\label{i-have-antibodies-am-i-now-immune}}

  \begin{itemize}
  \tightlist
  \item
    As of right
    now,\href{https://www.nytimes3xbfgragh.onion/2020/07/22/health/covid-antibodies-herd-immunity.html?action=click\&pgtype=Article\&state=default\&region=MAIN_CONTENT_3\&context=storylines_faq}{~that
    seems likely, for at least several months.}~There have been
    frightening accounts of people suffering what seems to be a second
    bout of Covid-19. But experts say these patients may have a
    drawn-out course of infection, with the virus taking a slow toll
    weeks to months after initial exposure.~People infected with the
    coronavirus typically
    \href{https://www.nature.com/articles/s41586-020-2456-9}{produce}~immune
    molecules called antibodies, which are
    \href{https://www.nytimes3xbfgragh.onion/2020/05/07/health/coronavirus-antibody-prevalence.html?action=click\&pgtype=Article\&state=default\&region=MAIN_CONTENT_3\&context=storylines_faq}{protective
    proteins made in response to an
    infection}\href{https://www.nytimes3xbfgragh.onion/2020/05/07/health/coronavirus-antibody-prevalence.html?action=click\&pgtype=Article\&state=default\&region=MAIN_CONTENT_3\&context=storylines_faq}{.
    These antibodies may}~last in the body
    \href{https://www.nature.com/articles/s41591-020-0965-6}{only two to
    three months}, which may seem worrisome, but that's~perfectly normal
    after an acute infection subsides, said Dr. Michael Mina, an
    immunologist at Harvard University. It may be possible to get the
    coronavirus again, but it's highly unlikely that it would be
    possible in a short window of time from initial infection or make
    people sicker the second time.
  \end{itemize}
\item ~
  \hypertarget{what-are-my-rights-if-i-am-worried-about-going-back-to-work}{%
  \paragraph{What are my rights if I am worried about going back to
  work?}\label{what-are-my-rights-if-i-am-worried-about-going-back-to-work}}

  \begin{itemize}
  \tightlist
  \item
    Employers have to provide
    \href{https://www.osha.gov/SLTC/covid-19/standards.html}{a safe
    workplace}~with policies that protect everyone equally.
    \href{https://www.nytimes3xbfgragh.onion/article/coronavirus-money-unemployment.html?action=click\&pgtype=Article\&state=default\&region=MAIN_CONTENT_3\&context=storylines_faq}{And
    if one of your co-workers tests positive for the coronavirus, the
    C.D.C.}~has said that
    \href{https://www.cdc.gov/coronavirus/2019-ncov/community/guidance-business-response.html}{employers
    should tell their employees}~-\/- without giving you the sick
    employee's name -\/- that they may have been exposed to the virus.
  \end{itemize}
\end{itemize}

There is also debate about whether the coronavirus can also be
transmitted through even smaller droplets --- less than a tenth the
width of a human hair --- that are known as aerosols, and can remain
suspended or travel through the air for longer.

\href{https://www.nejm.org/doi/full/10.1056/NEJMc2007800}{In another
recent study}, the same authors showed that just articulating certain
sounds can produce significantly higher amounts of respiratory
particles. The ``th'' sound in the word ``healthy,'' for example, was a
very efficient generator of speech droplets.
\href{https://journals.plos.org/plosone/article?id=10.1371/journal.pone.0227699}{Another
paper}, published in January by researchers from the University of
California, Davis, found the vowel sound ``e'' in ``need'' produces more
droplets than the ``a'' in ``saw,'' or ``o'' in ``mood.''

What researchers don't yet know is whether all speech, cough and sneeze
droplets carrying virus particles are equally infectious, or if a
specific amount of virus needs to be transmitted for a person to get
sick by breathing it in.

But the new study adds to the case for maintaining a physical distance
from other people to help slow the spread of coronavirus, said Linsey
Marr, a professor of civil and environmental engineering at Virginia
Tech who was not involved with the paper.

``Based on this and other evidence, it would be wise to avoid extended
face-to-face conversations with other people unless you are far apart
and in a well-ventilated space, including outdoors,'' Dr. Marr said.

The study also highlights the importance of wearing masks during social
and other interactions.

``The risk of talking to one another will probably be lower than being
exposed to a person who is not wearing a mask and openly coughs and
sneezes,'' said Dr. Werner E. Bischoff, the medical director of
infection prevention and health system epidemiology at the Wake Forest
School of Medicine. ``Normal talking to a person while keeping the
recommended social distance will be fine. Putting on a mask will be even
better.''

Advertisement

\protect\hyperlink{after-bottom}{Continue reading the main story}

\hypertarget{site-index}{%
\subsection{Site Index}\label{site-index}}

\hypertarget{site-information-navigation}{%
\subsection{Site Information
Navigation}\label{site-information-navigation}}

\begin{itemize}
\tightlist
\item
  \href{https://help.nytimes3xbfgragh.onion/hc/en-us/articles/115014792127-Copyright-notice}{©~2020~The
  New York Times Company}
\end{itemize}

\begin{itemize}
\tightlist
\item
  \href{https://www.nytco.com/}{NYTCo}
\item
  \href{https://help.nytimes3xbfgragh.onion/hc/en-us/articles/115015385887-Contact-Us}{Contact
  Us}
\item
  \href{https://www.nytco.com/careers/}{Work with us}
\item
  \href{https://nytmediakit.com/}{Advertise}
\item
  \href{http://www.tbrandstudio.com/}{T Brand Studio}
\item
  \href{https://www.nytimes3xbfgragh.onion/privacy/cookie-policy\#how-do-i-manage-trackers}{Your
  Ad Choices}
\item
  \href{https://www.nytimes3xbfgragh.onion/privacy}{Privacy}
\item
  \href{https://help.nytimes3xbfgragh.onion/hc/en-us/articles/115014893428-Terms-of-service}{Terms
  of Service}
\item
  \href{https://help.nytimes3xbfgragh.onion/hc/en-us/articles/115014893968-Terms-of-sale}{Terms
  of Sale}
\item
  \href{https://spiderbites.nytimes3xbfgragh.onion}{Site Map}
\item
  \href{https://help.nytimes3xbfgragh.onion/hc/en-us}{Help}
\item
  \href{https://www.nytimes3xbfgragh.onion/subscription?campaignId=37WXW}{Subscriptions}
\end{itemize}
