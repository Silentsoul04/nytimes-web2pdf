Sections

SEARCH

\protect\hyperlink{site-content}{Skip to
content}\protect\hyperlink{site-index}{Skip to site index}

\href{https://myaccount.nytimes3xbfgragh.onion/auth/login?response_type=cookie\&client_id=vi}{}

\href{https://www.nytimes3xbfgragh.onion/section/todayspaper}{Today's
Paper}

Laurie Anderson and Michael Stipe on Music, Art and New Chapters

\url{https://nyti.ms/2WUAvDH}

\begin{itemize}
\item
\item
\item
\item
\item
\item
\end{itemize}

Advertisement

\protect\hyperlink{after-top}{Continue reading the main story}

Supported by

\protect\hyperlink{after-sponsor}{Continue reading the main story}

Admiration Society

\hypertarget{laurie-anderson-and-michael-stipe-on-music-art-and-new-chapters}{%
\section{Laurie Anderson and Michael Stipe on Music, Art and New
Chapters}\label{laurie-anderson-and-michael-stipe-on-music-art-and-new-chapters}}

``You've got to fall on your face to sit at the table,'' says the
erstwhile R.E.M. frontman.

\includegraphics{https://static01.graylady3jvrrxbe.onion/images/2020/05/13/t-magazine/13tmag-stipe/13tmag-stipe-articleLarge.jpg?quality=75\&auto=webp\&disable=upscale}

By \href{https://www.nytimes3xbfgragh.onion/by/joe-coscarelli}{Joe
Coscarelli}

\begin{itemize}
\item
  May 14, 2020
\item
  \begin{itemize}
  \item
  \item
  \item
  \item
  \item
  \item
  \end{itemize}
\end{itemize}

At 72, no one would fault
\href{https://www.nytimes3xbfgragh.onion/topic/person/laurie-anderson}{Laurie
Anderson}, the avant-garde multimedia icon --- an inventor, performance
artist, musician, writer, filmmaker, painter and more --- for being
nostalgic. She is best known for her 1981 surprise hit song,
``\href{https://www.youtube.com/watch?v=Vkfpi2H8tOE}{O Superman},'' a
visionary tirade against American imperialism, built on looped vocal
effects that seemed to predict the next three decades or so of pop
music. In recent years, Anderson has been moonlighting as an archivist,
shaping the legacy of her husband and collaborator,
\href{https://www.nytimes3xbfgragh.onion/topic/person/lou-reed}{Lou
Reed}, who
\href{https://www.nytimes3xbfgragh.onion/2013/10/28/arts/music/lou-reed-dies-at-71.html}{died
in 2013}. (His archives, which include letters, photographs and
recordings, are now \href{https://www.nypl.org/loureed}{housed at} the
New York Public Library for the Performing Arts and are available to
anyone with a library card.) But the experience of looking backward,
immersed in another's oeuvre, has only made Anderson's own work more
forward-looking.

From her dead dog and Guantánamo Bay to Tibetan Buddhism and the moon,
and from virtual reality to recorded music to oil on canvas, Anderson
finds inspiration from and expression in a wide range of subjects and
mediums, as she has since she began her career as a performance artist
in early 1970s New York. As the music critic John Rockwell
\href{https://www.nytimes3xbfgragh.onion/1988/07/23/arts/review-pop-laurie-anderson-returns.html}{wrote}
in The Times in 1988, ``She didn't so much emerge on the scene as create
it.'' Anderson's
\href{https://hirshhorn.si.edu/exhibitions/laurie-anderson-the-weather/}{upcoming
exhibition} at the Smithsonian's
\href{https://www.nytimes3xbfgragh.onion/2016/11/05/arts/design/hirshhorn-museum-and-sculpture-garden-goes-international.html}{Hirshhorn
Museum and Sculpture Garden} in Washington, D.C. has been postponed, but
when it does open it will consist mostly of new work, including
wall-size paintings that she warned last November were still works in
progress. (``They are really bad,'' she deadpanned.) Worrying about her
own legacy, she added, is just a waste of time.

For
\href{https://www.nytimes3xbfgragh.onion/2015/08/14/t-magazine/my-10-favorite-books-michael-stipe.html}{Michael
Stipe}, however, the past has become a preoccupation of late. The
60-year-old singer and songwriter fronted R.E.M. for more than three
decades, from 1980 until
\href{https://www.nytimes3xbfgragh.onion/2011/09/22/arts/music/rem-breaks-up-after-31-years-as-a-band.html}{their
breakup} in 2011, and the band has been reissuing some of its classic
1990s albums, including ``Automatic for the People'' and ``Monster,''
while at the same time revisiting live recordings and concert footage.
But Stipe has also largely moved on from music, concentrating instead on
art and photography (his most recent work is gathered in his 2019
monograph ``\href{https://www.damianieditore.com/en-US/product/735}{Our
Interference Times: A Visual Record}'').

Anderson and Stipe met for a conversation in a photo studio on the
Bowery this past November to discuss their current projects, their early
experiences in the city and Reed's leather jackets.

\includegraphics{https://static01.graylady3jvrrxbe.onion/images/2020/05/13/t-magazine/13tmag-anderson-05/13tmag-anderson-05-articleLarge.jpg?quality=75\&auto=webp\&disable=upscale}

\textbf{Laurie Anderson:} What are you making now?

\textbf{Michael Stipe:} When I first started talking about stuff that I
was making that wasn't music, I said sculpture, because I wanted people
to imagine something tangible that you could hold. The truth is, I
wasn't really making sculpture, I was just making things, and a lot of
it is photo-based. The book is an extension of that.

\textbf{LA:} Weren't you doing that as a songwriter, making people
imagine things? I mean, in sculpture, you actually make things that
people can hold. Imagining it is even wilder. That's what I love about
the art --- nothing's there. There's not even a fence. There's nothing.
I love it.

\textbf{MS:} ``O Superman'' was my introduction to you. I was surrounded
by art students {[}Stipe studied art at the University of Georgia before
dropping out soon after R.E.M. formed{]} who said, ``This is a new
day.'' I have to say, the work that came after it was, for me, more
enthralling, but in terms of really throwing down the gauntlet
sonically, that song was like ---

\textbf{LA:} I just do wonder why more people didn't do stuff like that.
More political or more electronic.

\textbf{MS:} Music now is post-hip-hop, which is really interesting to
me, because hip-hop came from the same culture as punk rock, and then
has this whole other aspect to it because it's coming from black
America.

\textbf{LA:} It has more language than punk rock, though. Although you
used a lot of language in your early things, and that's what I loved
about it: unfurling words.

\textbf{MS:} I was a vulnerable man. I'm so proud that I never pretended
to be something I wasn't, even before I spoke publicly about my
sexuality {[}Stipe came out in 1994{]}. Vulnerability and sensitivity in
a public figure and as a man was something that in my generation was not
common at all.

\textbf{LA:} A friend of mine just sent me a video of Lou when he was
talking about what it was like to be Lou and not be anybody else. The
interview ended with this amazing sentence, about how he felt about his
efforts as a writer. He said, ``I've always been true to him.''

\textbf{MS:} That's so beautiful and true. He allowed immense
vulnerability as a writer. But then he was kind of curmudgeonly as a
person, or he could be. He did not suffer fools gladly, and he made that
abundantly clear in the most beautiful way.

\textbf{LA:} Doing the archive was a really exciting thing for me. My
idea from the very beginning was to let anyone come in. You didn't have
to have white gloves or be a scholar. And I'm so proud of that. I'm
still giving things away --- I have endless things to give away.
Actually, you would fit in Lou's jackets. I've got to give you some.

\textbf{MS:} I would love that.

Image

Stipe performs with R.E.M. in Perth, Australia, on Jan. 13,
1994.Credit...AP Photo/Perth West Australian

Image

Stipe performs with R.E.M. at the Tibetan Freedom Concert in Washington,
D.C., on June 14, 1998.Credit...Rahav Segev for The New York Times

\textbf{LA:} Would you call what you were writing pop songs?

\textbf{MS:} The one time I met
\href{https://www.nytimes3xbfgragh.onion/topic/person/andy-warhol}{Warhol}
he told me, ``You're a pop star.'' And I said, ``I'm a singer in a
band.'' And he insisted that I was a pop star, and finally a few years
later, I was like, ``Well, he was actually right.'' I was a pop star all
along. And it's something I didn't mind at all. There's nothing
reluctant about being a pop star, I don't think, for either one of us.
It was really fun. I liked being a pop star.

\textbf{LA:} Oh, I was not a pop star. Maybe for four seconds.

\textbf{MS:} But those were four very important seconds.

\textbf{LA:} I approached it as an anthropologist. I was so observant.
It was always absurd to me. But I decided, ``This is going to be fun.''
And it \emph{was} fun. I didn't think I deserved it, or wanted more of
it. In fact, it was horrible in other ways.

\textbf{MS:} It's also grueling work. You've got to work to put out good
work.

\textbf{LA:} So what's radical now?

\textbf{MS:} Radical now might be what's mainstream, in fact. So, it's
\href{https://www.nytimes3xbfgragh.onion/2019/03/28/arts/music/billie-eilish-debut-album.html}{Billie
Eilish}, it's music that has been formed by the algorithms that are
\href{https://www.nytimes3xbfgragh.onion/2019/06/23/business/media/stream-classical-music-spotify.html}{dictating
what we listen to} on the platforms that are available to us.

\textbf{LA:} I want a sense of groups, community, sweat, instead of
bigger and bigger things sold to bigger and bigger markets, which is
what's going on now. Any time any big corporation --- whether it's
Google or Facebook or any cultural organization you can think of ---
decides to monetize art or music or whatever, the first real buzzword
they use is ``community'': ``We love your community.'' And then they buy
it, they brand it and they sell it back to you curated for a huge,
expanded audience of people with their cellphones.

\textbf{MS:} It's been watered down.

Image

From left: Anderson, Lou Reed and Bill T. Jones perform ``Save the Last
Dance for Me'' in New York on June 11, 2002.Credit...Rahav Segev for The
New York Times

\textbf{LA:} How'd you choose New York?

\textbf{MS:} I first came to New York in 1979, as a teenager. It was all
about opportunity and possibility. It was really through the punk rock
scene and CBGB. It was
\href{https://www.nytimes3xbfgragh.onion/topic/person/patti-smith}{Patti
Smith} who spoke of, or sang of, or wrote of --- she did all of those,
how about that? --- New York as a place of opportunity and possibility.
I took that very literally. I was like, ``This is where I find myself.''

\textbf{LA:} Let's say you were that age now, can you imagine coming to
New York?

\textbf{MS:} No.

\textbf{LA:} Where would you go?

\textbf{MS:} I'm trying to think of where my group of people would be in
2019. I'd have been born in 2000. My people would probably be obsessed
with somewhere in South America or somewhere like Porto, Portugal.

\textbf{LA:} I almost bought a sheep farm there. I used to have a hobby
of shopping for sheep farms: You're on tour, it's Sunday, there's
nothing happening and you want to see the place, so you look through
brochures.

\textbf{MS:} Did you really ever intend to buy a sheep farm, or was it
just a way to see the countryside?

\textbf{LA:} Initially, it was just a way to get someone to drive me
around. Then I found one near Porto. I was looking at brochures and
said, ``Oh, sheep farm, \$2,000.'' I called the agent and said, ``Is
that a typo?'' He said, ``It is. It's \$1,000.'' So I said, ``I'm going
to get a sheep farm for all my friends.'' We went out there, now with
the intention of buying this. I think the thing that made me reconsider
was he said, ``You can use all the prisoners here for labor because the
prison is right over there.'' I said, ``That's handy.'' There was no
water or electricity or anything like that, of course, which didn't
bother me. I didn't have that when I moved to New York. I had no roof. I
was basically squatting.

\textbf{MS:} Are you a futurist?

\textbf{LA:} Oh, I aspire to be. Don't you?

\textbf{MS:} Well, I like to surround myself with futurists. I like to
hear what they have to say.

\textbf{LA:} I was part of a group of futurists who were supposed to be
coming up with what people would be doing 20 years from now. With, oh
God, what's his name, the co-founder of Microsoft?
\href{https://www.nytimes3xbfgragh.onion/2018/10/15/obituaries/paul-allen-dead.html}{Paul
Allen}. I'd been on Paul's boat and knew him a little bit. I thought,
``What would I do if I could not have to start from zero every time?''
So I put all of my stuff in a box --- films, videos, websites, movies,
records --- and I sent them to Paul Allen. I said, ``I'm looking for a
Medici.'' He called me up, said, ``Let's talk.'' I'm a really proud
person, so it's hard for me to actually ask people for things. So we
were having this beer in a bar and it crashed on the floor, all over his
California white pants. I was like, ``Babe, your dream is over.''
Anyway, he had this thing called Interval Research, and he hired a
hundred people to tell him what people would be doing in the future. I
was able to do a lot of things. I formed a media company and we did a
lot of big projects. He funded a lot of the work. And it was a terrible
mistake. Some artists are good at opening the London office and becoming
the mega this, mega something. I just never had the ambition after that.

Image

Stipe and Patti Smith perform at Carnegie Hall in New York on March 11,
2009.Credit...Richard Termine for The New York Times

\textbf{MS:} I don't like having people that are on the clock to help me
attain whatever it is I want to attain.

\textbf{LA:} Is that part of what made you want to be a visual artist,
that you don't have to have a giant array of people?

\textbf{MS:} Yes, unless I was going to be the type of visual artist you
were just thinking about, the type who had a London office. I'm not
interested in that at all. I'm fortunate to be in a position where I
don't have to pay the rent with whatever I'm making, so I can just make
stuff to make myself happy.

\textbf{LA:} I do not think about my own legacy. Not in the least, not
for one second. And yet, I think about Lou's all the time. I think it's
important to put it together --- I'm an archivist. But for my own
things? I actually couldn't care less. That's not what's important to
me. What's important is love. That's the only thing. The stuff? I really
don't care. I'm happy with now. I'm trained as a Buddhist to think
that's all there is. But it's weird, I'm doing it obsessively in Lou's
case.

\textbf{MS:} There's a deflection there.

\textbf{LA:} Maybe something like that. I couldn't say no. But I might
have just burned out on the whole idea. After you do it so much, you're
like, ``Eh, I just want to make new things.'' That's why the Hirshhorn
chose me. There's almost no old work in it. I wish there was. I actually
have a meeting today, and I want to say we should put something old in
there, but we're making a lot of new stuff. It's terrifying.

\textbf{MS:} Because you don't know whether it's going to be good work
or not?

\textbf{LA:} It might be terrible.

\textbf{MS:} That, to me, is the sign of a true artist. You've got to
fall on your face to sit at the table.

\textbf{LA:} I'm trying to make really big paintings. They're really
big, and they're really bad.

\textbf{MS:} Acrylic or oil?

\textbf{LA:} Oil. They're the size of this wall.

\textbf{MS:} I failed this morning. I failed yesterday morning, for
sure. I woke up and my first thoughts were really dark and horrible.

\textbf{LA:} But so are mine. My first thoughts every day are very, very
dark.

\textbf{MS:} Are you going to keep making the giant paintings?

\textbf{LA:} Yeah, I'm going out there tomorrow morning. They're so bad.

\textbf{MS:} Keep working.

\emph{This interview has been edited and condensed.}

Advertisement

\protect\hyperlink{after-bottom}{Continue reading the main story}

\hypertarget{site-index}{%
\subsection{Site Index}\label{site-index}}

\hypertarget{site-information-navigation}{%
\subsection{Site Information
Navigation}\label{site-information-navigation}}

\begin{itemize}
\tightlist
\item
  \href{https://help.nytimes3xbfgragh.onion/hc/en-us/articles/115014792127-Copyright-notice}{©~2020~The
  New York Times Company}
\end{itemize}

\begin{itemize}
\tightlist
\item
  \href{https://www.nytco.com/}{NYTCo}
\item
  \href{https://help.nytimes3xbfgragh.onion/hc/en-us/articles/115015385887-Contact-Us}{Contact
  Us}
\item
  \href{https://www.nytco.com/careers/}{Work with us}
\item
  \href{https://nytmediakit.com/}{Advertise}
\item
  \href{http://www.tbrandstudio.com/}{T Brand Studio}
\item
  \href{https://www.nytimes3xbfgragh.onion/privacy/cookie-policy\#how-do-i-manage-trackers}{Your
  Ad Choices}
\item
  \href{https://www.nytimes3xbfgragh.onion/privacy}{Privacy}
\item
  \href{https://help.nytimes3xbfgragh.onion/hc/en-us/articles/115014893428-Terms-of-service}{Terms
  of Service}
\item
  \href{https://help.nytimes3xbfgragh.onion/hc/en-us/articles/115014893968-Terms-of-sale}{Terms
  of Sale}
\item
  \href{https://spiderbites.nytimes3xbfgragh.onion}{Site Map}
\item
  \href{https://help.nytimes3xbfgragh.onion/hc/en-us}{Help}
\item
  \href{https://www.nytimes3xbfgragh.onion/subscription?campaignId=37WXW}{Subscriptions}
\end{itemize}
