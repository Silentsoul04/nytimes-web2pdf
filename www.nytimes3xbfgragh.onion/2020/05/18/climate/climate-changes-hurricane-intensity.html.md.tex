Sections

SEARCH

\protect\hyperlink{site-content}{Skip to
content}\protect\hyperlink{site-index}{Skip to site index}

\href{https://www.nytimes3xbfgragh.onion/section/climate}{Climate}

\href{https://myaccount.nytimes3xbfgragh.onion/auth/login?response_type=cookie\&client_id=vi}{}

\href{https://www.nytimes3xbfgragh.onion/section/todayspaper}{Today's
Paper}

\href{/section/climate}{Climate}\textbar{}Climate Change Is Making
Hurricanes Stronger, Researchers Find

\url{https://nyti.ms/2zbuLxv}

\begin{itemize}
\item
\item
\item
\item
\item
\end{itemize}

\hypertarget{climate-and-environment}{%
\subsubsection{\texorpdfstring{\href{https://www.nytimes3xbfgragh.onion/section/climate?name=styln-climate\&region=TOP_BANNER\&block=storyline_menu_recirc\&action=click\&pgtype=Article\&impression_id=e62354b0-f2bc-11ea-886d-f70f43d98c6e\&variant=undefined}{Climate
and
Environment}}{Climate and Environment}}\label{climate-and-environment}}

\begin{itemize}
\tightlist
\item
  \href{https://www.nytimes3xbfgragh.onion/interactive/2020/08/24/climate/racism-redlining-cities-global-warming.html?name=styln-climate\&region=TOP_BANNER\&block=storyline_menu_recirc\&action=click\&pgtype=Article\&impression_id=e62354b1-f2bc-11ea-886d-f70f43d98c6e\&variant=undefined}{Environmental
  Racism}
\item
  \href{https://www.nytimes3xbfgragh.onion/interactive/2020/climate/trump-environment-rollbacks.html?name=styln-climate\&region=TOP_BANNER\&block=storyline_menu_recirc\&action=click\&pgtype=Article\&impression_id=e62354b2-f2bc-11ea-886d-f70f43d98c6e\&variant=undefined}{Trump's
  Changes}
\item
  \href{https://www.nytimes3xbfgragh.onion/interactive/2020/04/19/climate/climate-crash-course-1.html?name=styln-climate\&region=TOP_BANNER\&block=storyline_menu_recirc\&action=click\&pgtype=Article\&impression_id=e62354b3-f2bc-11ea-886d-f70f43d98c6e\&variant=undefined}{Climate
  101}
\item
  \href{https://www.nytimes3xbfgragh.onion/interactive/2018/08/30/climate/how-much-hotter-is-your-hometown.html?name=styln-climate\&region=TOP_BANNER\&block=storyline_menu_recirc\&action=click\&pgtype=Article\&impression_id=e62354b4-f2bc-11ea-886d-f70f43d98c6e\&variant=undefined}{Is
  Your Hometown Hotter?}
\end{itemize}

Advertisement

\protect\hyperlink{after-top}{Continue reading the main story}

Supported by

\protect\hyperlink{after-sponsor}{Continue reading the main story}

\hypertarget{climate-change-is-making-hurricanes-stronger-researchers-find}{%
\section{Climate Change Is Making Hurricanes Stronger, Researchers
Find}\label{climate-change-is-making-hurricanes-stronger-researchers-find}}

An analysis of satellite imagery from the past four decades suggests
that global warming has increased the chances of storms reaching
Category 3 or higher.

\includegraphics{https://static01.graylady3jvrrxbe.onion/images/2020/05/18/science/18CLI-HURRICANES/merlin_143651790_5feec6b1-4c48-45b3-96ad-8cf30513ad9f-articleLarge.jpg?quality=75\&auto=webp\&disable=upscale}

By \href{https://www.nytimes3xbfgragh.onion/by/henry-fountain}{Henry
Fountain}

\begin{itemize}
\item
  Published May 18, 2020Updated Aug. 23, 2020
\item
  \begin{itemize}
  \item
  \item
  \item
  \item
  \item
  \end{itemize}
\end{itemize}

Hurricanes have become stronger worldwide during the past four decades,
an analysis of observational data shows, supporting what theory and
computer models have long suggested: climate change is making these
storms more intense and destructive.

The analysis, of satellite images dating to 1979, shows that warming has
increased the likelihood of a
\href{https://www.nytimes3xbfgragh.onion/2020/08/23/us/tropical-storm-laura-marco-louisiana.html}{hurricane}
developing into a major one of Category 3 or higher, with sustained
winds greater than 110 miles an hour, by about 8 percent a decade.

``The trend is there and it is real,'' said James P. Kossin, a
researcher with the National Oceanic and Atmospheric Administration and
lead author of the
\href{https://www.pnas.org/cgi/doi/10.1073/pnas.1920849117}{study,}
published Monday in Proceedings of the National Academy of Sciences.
``There's this remarkable building of this body of evidence that we're
making these storms more deleterious.''

Kerry Emanuel, a
\href{https://www.nytimes3xbfgragh.onion/interactive/2020/07/25/us/hurricane-hanna-tracker-map.html}{hurricane}
expert at the Massachusetts Institute of Technology who was not involved
in the study, said the findings were ``much in line with what's
expected.''

``When you see things going up all over the globe like that, the ducks
are kind of in order,'' he said.

But in the North Atlantic, where hurricane activity has
\href{https://nca2014.globalchange.gov/report/our-changing-climate/changes-hurricanes}{increased
in recent decades} and storms have
caused\href{https://coast.noaa.gov/states/fast-facts/hurricane-costs.html}{tens
of billions of dollars of damage} in the United States and the
Caribbean, factors other than climate change may have played more of a
role in the increase in intensity, Dr. Emanuel said.

Physics suggests that as the world warms, hurricanes and other tropical
cyclones should get stronger, because warmer water provides more of the
energy that fuels these storms. And climate simulations have long showed
an increase in stronger hurricanes as warming continues.

\href{\%3Ca\%20href=\%22https://www.nytimes3xbfgragh.onion/section/climate?action=click\&pgtype=Article\&state=default\&region=MAIN_CONTENT_1\&context=storylines_keepup\%22\%3Ehttps://www.nytimes3xbfgragh.onion/section/climate?action=click\&pgtype=Article\&state=default\&region=MAIN_CONTENT_1\&context=storylines_keepup\%3C/a\%3E}{}

\hypertarget{climate-and-environment-}{%
\subsubsection{Climate and Environment
›}\label{climate-and-environment-}}

\hypertarget{keep-up-on-the-latest-climate-news}{%
\paragraph{Keep Up on the Latest Climate
News}\label{keep-up-on-the-latest-climate-news}}

Updated Sept. 6, 2020

Here's what you need to know this week:

\begin{itemize}
\item
  \begin{itemize}
  \tightlist
  \item
    Americans back
    \href{https://www.nytimes3xbfgragh.onion/2020/09/04/climate/flood-fire-building-restrictions.html?action=click\&pgtype=Article\&state=default\&region=MAIN_CONTENT_1\&context=storylines_keepup}{tough
    limits on building in fire and flood zones}, new research shows.
  \item
    California's wildfires are driving another crisis: More and more
    \href{https://www.nytimes3xbfgragh.onion/2020/09/02/climate/wildfires-insurance.html?action=click\&pgtype=Article\&state=default\&region=MAIN_CONTENT_1\&context=storylines_keepup}{homeowners
    can't get insurance}.
  \item
    The Trump administration has relaxed Obama-era rules limiting the
    release of
    \href{https://www.nytimes3xbfgragh.onion/2020/08/31/climate/trump-coal-plants.html?action=click\&pgtype=Article\&state=default\&region=MAIN_CONTENT_1\&context=storylines_keepup}{toxic
    waste from coal plants}.
  \end{itemize}
\end{itemize}

But confirming that through observations has been problematic, because
of the relatively small
\href{https://www.nytimes3xbfgragh.onion/2020/05/21/climate/hurricane-season-2020-noaa.html}{number
of hurricanes every year} and the difficulty of obtaining data on their
wind speeds and other characteristics. Even in the United States, storms
that do not potentially threaten populations are measured less than
others.

``We're doing collectively a bad job of measuring tropical cyclones
around the world,'' Dr. Emanuel said. ``We've all believed we should see
more intense hurricanes. But it's very very tricky to find it in the
data.''

Dr. Kossin and his colleagues got around the limitations by using
satellite images of storms worldwide and using computers to interpret
them with a long-accepted pattern-matching algorithm, or set of
instructions. They had done this before, in
\href{https://journals.ametsoc.org/doi/full/10.1175/JCLI-D-13-00262.1}{a
study published in 201}3, but that analysis only included imagery from
1982 to 2009 and the findings, while similar, were not statistically
significant.

In the new study the researchers extended the data set by 11 years,
using imagery from 1979 to 2017.

``The first time through we found trends but they hadn't risen to the
level of confidence that we would require,'' Dr. Kossin said. The
findings of the new study are statistically significant.

``This is saying, OK now, the historical observations are also in
agreement'' with the theory and models, he added.

The study looked at tropical storms worldwide because that provided a
lot more data than looking at those in just one region. And every region
has natural variability or other factors that can affect storm intensity
and make it more difficult to tease out the effects of warming.

``When you look at the picture globally, it tends to wash away that
regional variability,'' Dr. Kossin said. ``The trend rises above the
noise.''

The North Atlantic has seen increased hurricane activity in recent
decades, by a measure that combines intensity with other characteristics
like duration and frequency of storms. On Thursday, NOAA will issue its
forecast of activity for this season, which officially runs from June 1
to November 30. Forecasts by other organizations have suggested that
this year may be an active one.

But the North Atlantic is one region where climate change may be
overshadowed by other factors, Dr. Emanuel said.

``We do see clear signals and strong trends in the North Atlantic,'' he
said. ``The problem is we can't uniquely attribute that to greenhouse
gases.''

Some scientists say that long-term natural variability in sea surface
temperatures, on a time scale of decades, has played the major role in
affecting North Atlantic storm activity. Others say that mandated
reductions in sulfur emissions from fossil-fuel burning over the past
few decades may be more important, by affecting ocean temperatures
through a series of atmospheric connections.

Whatever the main factors are, the study suggests that climate change
will play a long-term role in increasing the strength of storms in the
North Atlantic and elsewhere, Dr. Kossin said. Planning for how to
mitigate the impact of major storms must take this into account.

``From a short time scale, these trends are not going to change the risk
landscape,'' Dr. Kossin said. But over the long term, he said, ``the
risk landscape could change, and in a bad way, not in a good way.''

Advertisement

\protect\hyperlink{after-bottom}{Continue reading the main story}

\hypertarget{site-index}{%
\subsection{Site Index}\label{site-index}}

\hypertarget{site-information-navigation}{%
\subsection{Site Information
Navigation}\label{site-information-navigation}}

\begin{itemize}
\tightlist
\item
  \href{https://help.nytimes3xbfgragh.onion/hc/en-us/articles/115014792127-Copyright-notice}{©~2020~The
  New York Times Company}
\end{itemize}

\begin{itemize}
\tightlist
\item
  \href{https://www.nytco.com/}{NYTCo}
\item
  \href{https://help.nytimes3xbfgragh.onion/hc/en-us/articles/115015385887-Contact-Us}{Contact
  Us}
\item
  \href{https://www.nytco.com/careers/}{Work with us}
\item
  \href{https://nytmediakit.com/}{Advertise}
\item
  \href{http://www.tbrandstudio.com/}{T Brand Studio}
\item
  \href{https://www.nytimes3xbfgragh.onion/privacy/cookie-policy\#how-do-i-manage-trackers}{Your
  Ad Choices}
\item
  \href{https://www.nytimes3xbfgragh.onion/privacy}{Privacy}
\item
  \href{https://help.nytimes3xbfgragh.onion/hc/en-us/articles/115014893428-Terms-of-service}{Terms
  of Service}
\item
  \href{https://help.nytimes3xbfgragh.onion/hc/en-us/articles/115014893968-Terms-of-sale}{Terms
  of Sale}
\item
  \href{https://spiderbites.nytimes3xbfgragh.onion}{Site Map}
\item
  \href{https://help.nytimes3xbfgragh.onion/hc/en-us}{Help}
\item
  \href{https://www.nytimes3xbfgragh.onion/subscription?campaignId=37WXW}{Subscriptions}
\end{itemize}
