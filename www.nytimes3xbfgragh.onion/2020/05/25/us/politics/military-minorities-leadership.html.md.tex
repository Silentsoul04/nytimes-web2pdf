\href{/section/politics}{Politics}\textbar{}African-Americans Are Highly
Visible in the Military, but Almost Invisible at the Top

\url{https://nyti.ms/2XpLDca}

\begin{itemize}
\item
\item
\item
\item
\item
\item
\end{itemize}

\includegraphics{https://static01.graylady3jvrrxbe.onion/images/2020/02/19/us/politics/00dc-militaryrace-promo/00dc-militaryrace-promo-superJumbo.jpg}

Photographs by Nate Palmer and Ilana Panich-Linsman for The New York
Times

Sections

\protect\hyperlink{site-content}{Skip to
content}\protect\hyperlink{site-index}{Skip to site index}

\hypertarget{african-americans-are-highly-visible-in-the-military-but-almost-invisible-at-the-top}{%
\section{African-Americans Are Highly Visible in the Military, but
Almost Invisible at the
Top}\label{african-americans-are-highly-visible-in-the-military-but-almost-invisible-at-the-top}}

Seventy-five years after integration, the military's upper echelons
remain the domain of white men.

Photographs by Nate Palmer and Ilana Panich-Linsman for The New York
TimesCredit...

Supported by

\protect\hyperlink{after-sponsor}{Continue reading the main story}

\href{https://www.nytimes3xbfgragh.onion/by/helene-cooper}{\includegraphics{https://static01.graylady3jvrrxbe.onion/images/2018/08/24/multimedia/author-helene-cooper/author-helene-cooper-thumbLarge.png}}

By \href{https://www.nytimes3xbfgragh.onion/by/helene-cooper}{Helene
Cooper}

\begin{itemize}
\item
  Published May 25, 2020Updated June 9, 2020
\item
  \begin{itemize}
  \item
  \item
  \item
  \item
  \item
  \item
  \end{itemize}
\end{itemize}

WASHINGTON --- A photograph of President Trump and his top four-star
generals and admirals,
\href{https://twitter.com/EsperDoD/status/1181572097384095745}{tweeted}
in October by Defense Secretary Mark T. Esper, was meant as a thank-you
to the commander in chief. But it angered a lot of others, and not just
those who erupted on Twitter.

``You would have thought it was 1950,'' said Lt. Col. Walter J. Smiley
Jr., who is African-American and fought in Iraq and Afghanistan before
retiring last year after 25 years in the Army. Dana Pittard, a retired
major general, also African-American, was equally frustrated. ``It's
America's military,'' he said. ``Why doesn't this photo look like
America?''

Yet the picture of the president surrounded by a sea of white faces in
full military dress is an accurate portrait of the top commanders who
lead an otherwise diverse institution.

Some 43 percent of the 1.3 million men and women on active duty in the
United States military are people of color. But the people making
crucial decisions, such as how to respond to the coronavirus crisis and
how many troops to send to Afghanistan or Syria, are almost entirely
white and male.

Of the 41 most senior commanders in the military --- those with
four-star rank in the Army, Navy,
\href{https://www.nytimes3xbfgragh.onion/2020/06/09/us/politics/general-charles-brown-air-force.html}{Air
Force}, Marines and Coast Guard --- only two are black: Gen. Michael X.
Garrett, who leads the Army's Forces Command, and Gen. Charles Q. Brown
Jr, the commander of Pacific Air Forces.

Gen. Paul M. Nakasone, whose father is second-generation
Japanese-American, leads the United States Cyber Command. The Army has
sometimes counted Gen. Stephen J. Townsend, the head of Africa Command
and the son of a German mother and an Afghan father, as a minority
commander. There is only one woman in the group: Gen. Maryanne Miller,
the chief of the Air Force's Air Mobility Command, who is white.

The reasons there are so few people of color at the top lie deep in the
history and culture of the United States military.
\href{https://www.fdrlibrary.org/documents/356632/390886/tusk_doc_a.pdf/4693156a-8844-4361-ae17-03407e7a3dee}{A
1925 guidance} for Army officers stated that black service members were
a class ``from which we cannot expect to draw leadership material.'' The
armed forces were not fully integrated until after World War II, a
legacy that has left African-Americans without the same history of
generations of family service shared by so many white enlistees.

The elite service academies that feed the officer class --- the United
States Military Academy at West Point, the Naval Academy in Annapolis,
Md., and the Air Force Academy in Colorado Springs --- have increased
their enrollment of minority recruits in recent years but remain largely
white. The African-Americans who do become officers are often steered to
specialize in logistics and transportation rather than the marquee
combat arms specialties that lead to the top jobs.

Interviews with more than three dozen white, black and Hispanic service
members and officers depict an entrenched and clubby system with near
cement ceilings for minority groups.

\includegraphics{https://static01.graylady3jvrrxbe.onion/images/2020/05/25/us/politics/00dc-militaryrace1/merlin_162339105_963c8f17-35f5-4d1f-9f8a-d5cb48b5ac4d-articleLarge.jpg?quality=75\&auto=webp\&disable=upscale}

The Trump presidency, minority service members said, has only magnified
the sense of isolation they have long felt in a stratified system. ``You
had the feeling with Obama, that people were looking up'' and trying to
impress the country's first black president, General Pittard said,
adding that similar sentiments existed under Presidents George W. Bush
and Bill Clinton. That pressure, he said, has disappeared with Mr.
Trump. ``There's not somebody pushing it,'' he said.

Racism within the military appears to be on the rise. A survey last fall
of 1,630 active-duty subscribers to Military Times found that 36 percent
of those polled and 53 percent of minority service members said they had
seen examples of white nationalism or ideologically driven racism among
their fellow troops. The numbers were up significantly from the same
poll conducted in 2018, when 22 percent of all respondents reported
personally witnessing white nationalism.

In recent years, the Pentagon has faced intensifying criticism for a
series of racist episodes. A lawsuit filed in federal court in February
by a Navy fighter pilot accused airmen and officers at the Naval Air
Station Oceana in Virginia Beach of seeking to cover up institutional
racism directed against African-American aviators, which he said
resulted in their wrongful removal from pilot training programs. The
pilot's lawyer said in an interview that black airmen at the base were,
among other things, given racially derogatory call signs like ``8-Ball''
and referred to as ``eggplants'' in group chats on social media.

In December, West Point announced that its Black Knights football team
had removed from its flag the initials G.F.B.D., for ``God Forgives,
Brothers Don't,'' after learning that it was a slogan demanding loyalty
by the Aryan Brotherhood of Texas, a white supremacist prison gang.

The small sniper community in the Marine Corps has often used a Nazi
symbol, the lightning bolt insignia of Hitler's SS units, as a stand-in
for ``Scout Sniper.'' Although the Marine Corps leadership moved quickly
to stamp out the symbol after a photo of a unit posing with an SS flag
surfaced in Afghanistan in 2012, it still persists, Marines say, much
like a secret handshake.

``The absence of minorities at the top means the absence of a voice to
point to things that should have been addressed a long time ago,'' said
Brandy Baxter, an Air Force veteran who served in Iraq and Afghanistan
and is African-American. ``And from a human standpoint, this absence
sends another message that here's another space where we are not
accepted.''

Minority service members applaud two recent changes: In March, General
Brown was nominated to be the next Air Force chief of staff. And in
January, the Navy announced that its
\href{https://www.nytimes3xbfgragh.onion/2020/01/18/us/doris-miller-aircraft-carrier.html}{newest
aircraft carrier would be the first to be named after a black seaman},
the African-American World War II hero Doris Miller, who manned
antiaircraft guns during the Japanese attack on Pearl Harbor and helped
save the wounded. But service members note that two other aircraft
carriers retain the names of segregationists, John C. Stennis and Carl
Vinson.

\includegraphics{https://static01.graylady3jvrrxbe.onion/images/2020/02/19/us/politics/00dc-militaryrace2/00dc-militaryrace2-jumbo.jpg}Credit...Ilana
Panich-Linsman for The New York Times

``The absence of minorities at the top means the absence of a voice to
point to things that should have been addressed a long time ago.''

Brandy Baxter, Air Force veteran

One of the biggest problems, service members say, is that white men in
the top ranks do not see the problem. In July, Gen. John E. Hyten, the
second-highest officer in the military, told a Senate committee that
racism in the military was a thing of the past compared with the issue
of sexism.

``When I came into the military, I came in from Alabama, and racism was
a huge problem in the military --- overt racism,'' said General Hyten,
the vice chairman of the Joint Chiefs of Staff. ``I watched commander
after commander after commander take charge, own that, and any time they
saw it, eliminated it from the formation.''

He added, ``Now when I am in uniform, I feel colorblind, which is
amazing.''

\hypertarget{a-lack-of-mentors}{%
\subsection{A Lack of Mentors}\label{a-lack-of-mentors}}

If you enter the Pentagon at the Potomac River entrance, where foreign
dignitaries are greeted by the defense secretary, you will walk down the
E Ring hall with its portraits of the men who have led the United States
armed forces for the past century. To nearly a one, the African-American
service members interviewed for this article said they paused when they
walked by the painting of Gen. Colin L. Powell, the first and only black
chairman of the Joint Chiefs of Staff. His portrait, they said, came as
both a relief --- that he was there at all --- and a reminder that no
one else with their skin color had made it.

``I walk their halls, and nobody on their wall looks like me,'' said
Lila Holley, a former Army chief warrant officer. Until she gets to the
portrait. ``I exhale when I see Colin Powell.'' she said.

Mr. Powell, who became President George W. Bush's first secretary of
state, declined to be interviewed about his military service for this
article. But in a 1995 article for The New Yorker, he spoke about the
subtle racism he had experienced. ``When I was a young lieutenant, I
would have commanders come up to me and say, `Powell, you're doing great
--- goddamn, you're the best black lieutenant I've ever seen,''' Mr.
Powell told Henry Louis Gates Jr., a Harvard professor and the author of
the article. ``And I'd say, `Thank you.' Just file it away.''

Image

Graduates of the United States Military Academy at West Point during
their commencement ceremony last year. Powerful relationships are often
formed early in the military.Credit...David Dee Delgado/Getty Images

Jonathan Rath Hoffman, a Pentagon spokesman, said that ``we are keenly
aware of the importance of cultural and ethnic diversity in our
senior-level positions.''

Officially, the military insists that generals and admirals are chosen
by strict criteria assessed by service selection boards. But in
practice, almost all of those interviewed said that finding a mentor
remained crucial.

The top Army officers --- Gen. James C. McConville, the Army chief of
staff; Gen. John M. Murray, the head of the Army's Futures Command; and
Gen. Paul E. Funk II, the head of the Army's Training and Doctrine
Command --- are all white and were all mentored by the same man, Gen.
Peter Chiarelli, a former Army vice chief of staff.

``The Army in particular is a pretty bubba-oriented system,'' said Derek
Chollet, a former assistant secretary of defense. ``It's about who's
going to take care of you. So if you don't have senior leadership that
makes fixing this a priority, it's very hard to see it happening.''

General Chiarelli said in an interview that the problem in advancing
African-Americans into leadership positions began long before the
promotion boards started choosing top officers.

``If I'm a C.E.O., I can go outside to look for a person if I don't have
one internally in my organization,'' he said. The Army, he said, can
choose from only the colonels before them. ``I can't go on the street
and hire somebody.''

Image

Midshipmen scaling the Herndon Monument, an annual tradition, at the
Naval Academy in Annapolis, Md. The service academies often start
high-ranking careers, while historically black colleges do
not.Credit...Win Mcnamee/Getty Images

Michael Williams, a retired Marine who wrote his doctoral dissertation
on the topic, said ``the reality is, the individuals in that room with
the secretary of defense represent decisions that were made 35 years
ago.''

Rising to the top of the military means enduring a four-decade career of
often being the only minority service member in the room, platoon or
meeting. ``If I had to go to work every day, for 38 years, where I was
the only person of color in the room --- wow,'' General Chiarelli said.
``I don't know how I would feel about that.''

\hypertarget{generals-of-logistics}{%
\subsection{Generals of Logistics}\label{generals-of-logistics}}

Equally crucial is where you come from. Graduates of West Point,
Annapolis and Colorado Springs are typically destined for military
leadership, but graduates of historically black colleges and
universities are not.

Graduates from black colleges who had successful military careers
typically specialized in logistics and transportation, like moving
supplies or driving trucks, and not in combat arms specialties like
infantry or artillery. Logistics and transportation are an outgrowth of
the segregated military, when many black troops were quartermasters and
truck drivers. But it is the combat postings, particularly during the
nearly two decades of war in Iraq and Afghanistan, that lead to the top
leadership jobs.

``From the historically black colleges, what people do is what others
who have been successful before them have done,'' said General Garrett,
the head of the Army's Forces Command. ``The students there see generals
of logistics,'' and so ``that's what they want to do, too.''

And yet African-Americans have a history of combat, from the Buffalo
Soldiers who served on the Western frontier after the Civil War to the
Tuskegee Airmen in World War II to the black soldiers who fought in
Vietnam. They were all fighting for a country, African-Americans have
pointed out, that has a long legacy of not treating them as equal
citizens.

The history of some of the military's most storied combat units --- the
soldiers who landed on Omaha Beach or the Marines who stormed Iwo Jima
--- has largely excised the black and brown troops who fought alongside
the white men. This casting of military history heightens the sense
among African-Americans, they say, that they are still not welcome in
such units.

The elite Special Operations forces --- Navy SEALs, Army Green Berets,
Rangers and Delta Force commandos --- tend to be as white as the
military's top ranks. ``I remember sitting in a review of a Ranger
regiment,'' General Chiarelli said. ``I was blown away, looking at six
to seven hundred young men, and I was straining to see if I could find a
single person of color.''

General Garrett said the lack of minority leadership at the top ranks
was ``something I spend a lot of lot of time thinking about. There are
no perfect answers.'' To get ahead, he said, African-Americans must move
away from support areas and into combat.

\includegraphics{https://static01.graylady3jvrrxbe.onion/images/2020/02/19/us/politics/00dc-militaryrace8/merlin_169125708_27d5294a-7f1a-48f2-b97e-8fab4241435b-jumbo.jpg}Credit...Nate
Palmer for The New York Times

``There are no perfect answers.''

Gen. Michael X. Garrett

``I just know that one of the denominators is combat arms,'' he said in
an interview. ``Generally speaking, those are the folks that run the
Army. Those are the folks that, throughout their career, have more
opportunity to be in charge.''

Some African-Americans are discouraged from combat by their families.
Tes Solomon Kifle, an African-American who worked in the Marines
mortuary affairs department, said his mother did not want him joining
the military to begin with, let alone going into combat arms. ``My mom
was crying when I joined,'' he said in an interview. ``She was deathly
against it.''

Other black men in the military offer similar accounts of terrified
mothers battered by years of trying to protect their sons from a society
in which being young, black and male can be a death sentence. In this
view, combat arms in the military was yet another threat.

Many African-Americans saw military service not as a career but as a way
to help pay for education or to help compete later in the civilian job
market. By contrast, many white service members with long family
histories of service sign up for what they call the ``warrior culture,''
because that is what is expected, and it is what their fathers,
grandfathers and great-grandfathers did.

Image

Tuskegee Airmen in Alabama in 1942.Credit...Getty Images

When news broke in October 2017 that one black service member was among
three Green Berets and a mechanic killed in an
\href{https://www.nytimes3xbfgragh.onion/interactive/2018/02/17/world/africa/niger-ambush-american-soldiers.htm}{ambush
in Niger}, several African-American colonels who were interviewed for
this article said that they knew immediately that the black service
member, Sgt. La David T. Johnson, was the mechanic.

But even though Sergeant Johnson did not have the Green Beret patch on
his sleeve, he died firing his weapon in the scrub of remote Niger,
surrounded by advancing militants.

``Something else is happening,'' said Reuben E. Brigety, a former Navy
submarine officer who is now the dean of George Washington University's
Elliott School of International Affairs. ``Unless you presume that
ethnic minorities are just not as good as their white male counterparts,
there has to be another reason.''

\hypertarget{the-other-reason}{%
\subsection{The Other Reason}\label{the-other-reason}}

In the Marines, the term for a black Marine is ``nonswimmer.'' In the
Army Rangers, it is ``night ranger.''

``I heard the name `night ranger,' '' said General Pittard, who did his
Ranger training in the North Georgia mountains. ```Come here, Night
Ranger.' That doesn't make you feel very welcome.''

The ``nonswimmer'' name, meant as a slur, refers to the ages-old trope
that black people cannot swim. Like any trope, there is just enough of a
glimmer of truth to make it hard to shake. General Pittard, who made it
as far as the commander of land forces for the American-led coalition
battling the Islamic State in Iraq and Syria in 2014, said that when he
entered West Point in 1977, fewer than 10 out of 100 black freshmen knew
how to swim. To graduate, they had to learn.

``We graduated 42'' black cadets, General Pittard recalled. ``So we lost
58.''

\includegraphics{https://static01.graylady3jvrrxbe.onion/images/2020/02/19/us/politics/00dc-militaryrace11/00dc-militaryrace11-jumbo.jpg}Credit...Nate
Palmer for The New York Times

``I heard the name `night ranger.' `Come here, Night Ranger.' That
doesn't make you feel very welcome.''

Maj. General Dana Pittard

General Pittard retired in 2015 after he was reprimanded after a
three-year investigation by the Army inspector general for the
``perception of favoritism'' in a defense contract award that went to a
firm run by two of his former West Point classmates.

In interviews, African-American, Asian and Hispanic officers and
enlisted service members described a feeling of not being accepted that
was sometimes so intangible that many grew frustrated trying to describe
it. In ways large and small, they said, they felt constantly challenged
over their right to be in elite units, let alone lead them.

After graduating from Prairie View A\&M University in 1993, Colonel
Smiley, one of the African-American retired officers offended by Mr.
Esper's photograph on Twitter, went into artillery, a combat arms
specialty. Over 25 years, he had multiple tours in South Korea, Iraq and
Afghanistan. When an African-American battalion commander called him
into his office and told him to lose his mustache because there were no
senior Army leaders with mustaches, he quickly shaved.

Colonel Smiley thought he was on the right track until 2011, when ``the
story changes,'' he said in an interview. His evaluation from his time
in Afghanistan, in 2009 and 2010, had been stellar, he said. But after
returning home, he received a second evaluation that was mediocre. And
that was it for his chances of being promoted from lieutenant colonel to
full colonel, let alone to general. In the Army's promotion system, one
mediocre evaluation is enough to kill your chance for advancement.

A one-star general later expressed surprise that Colonel Smiley was
still just a lieutenant colonel and called him into his office. ``You've
got a great file except for this one evaluation,'' he told Colonel
Smiley. ``What did you do?''

Colonel Smiley did not know. Almost a decade later, he still does not
know, although he said he thought race played a part. He left the Army
in September as a lieutenant colonel. ``I would have stayed if I had
made 06,'' he said, in reference to the rank of colonel.

When he saw the photo of Mr. Trump with his all-white military
leadership in October, he said he felt both frustrated and sad. ``All
those men are qualified,'' he said. ``But there are a great many others,
not in that picture, who are qualified, too.''

\includegraphics{https://static01.graylady3jvrrxbe.onion/images/2020/02/19/us/politics/00dc-militaryrace12/merlin_169125654_84a1bc89-560a-4bff-ba4c-dba99cd55c33-jumbo.jpg}Credit...Nate
Palmer for The New York Times

``I had aspirations of at least being a brigade-level commander, and
being able to mentor other African-American soldiers.''

Lt. Col. Walter Smiley

African-American officers said they had no room for error, and that
episodes that had little consequence for their white counterparts ended
careers for them. Consider the cases of Col. Gus Benton and Col. Bradley
D. Moses, two commanding officers, at different times, of the same elite
Army Green Beret unit, the Third Special Forces Group. Colonel Benton is
black, and Colonel Moses is white.

On Feb. 21, 2010, when Colonel Benton was the commander of the unit, his
group was involved in an episode in Afghanistan in which American
warplanes struck three vehicles full of Afghan civilians in Uruzgan
Province, killing 21 people, including children. Colonel Benton, who
took part in approving the strikes, received a career-ending letter of
reprimand. In the unit, he had often talked about his black college
fraternity and was viewed as an outlier in the largely white Green Beret
world. He retired from the military in 2014.

In October 2017, Colonel Moses was the commanding officer of the unit
when the ambush in Niger occurred, killing the four American service
members. Colonel Moses approved the Niger mission, including a change in
plans that made the mission more dangerous and led to the ambush.

The Army has since put Colonel Moses forward to the Senate Armed
Services Committee for promotion to brigadier general, although the
\href{https://www.nytimes3xbfgragh.onion/2020/03/14/world/africa/niger-ambush-promotion.html}{nomination
was blocked by lawmakers} in March.

Colonel Moses declined to comment.

Colonel Moses ``was part of the protected crew, and that's how it played
out,'' said retired Brig. Gen. Donald C. Bolduc, who replaced Colonel
Benton after the episode in Afghanistan. He called it ``the same `good
old boy' system."

\hypertarget{nasa-but-not-the-marines}{%
\subsection{NASA, but Not the Marines}\label{nasa-but-not-the-marines}}

The United States Marine Corps has never in its 244 years had a
four-star general who was not a white male.

Consider the case of Maj. Gen. Charles F. Bolden Jr., who managed to
break barriers on land and in the air. In 1963, after South Carolina's
congressional delegation turned him down for an appointment to the Naval
Academy, General Bolden wrote a letter to President Lyndon B. Johnson. A
recruiter came to his house a few weeks later, and he got into
Annapolis.

Image

Charles F. Bolden Jr., second from left, preparing to launch the Space
Shuttle Discovery at Cape Canaveral, Fla., in 1990.Credit...Bettmann,
via Getty Images

General Bolden flew more than 100 sorties over Vietnam, Laos and
Cambodia as a Marine fighter pilot during the Vietnam War. He went on to
NASA to pilot two space shuttles, the Columbia in 1986 and the Discovery
in 1990, and command two more, the Atlantis in 1992 and the Discovery in
1994.

Although he made it to the rank of major general, he never got that
third or fourth star, and he left the Marines in 2004. Five years later,
President Obama appointed him the head of NASA.

Lt. Gen. Ronald L. Bailey could not do it either. The first black man to
command the First Marine Division, from 2011 to 2013, General Bailey
retired in 2017 after 40 years in the Marines, one star short of
breaking the four-star barrier.

``The Marine Corps actually has given this a great deal of thought
because we have struggled,'' said Gen. Kenneth F. McKenzie Jr., the
Marine who is head of United States Central Command. ``We've struggled
to do it with minorities. We've struggled to do it with women. It is a
continuing problem for us.''

\hypertarget{it-undermines-you}{%
\subsection{`It Undermines You'}\label{it-undermines-you}}

In June, Lt. Col. Kimberly Barr was about to receive her first
leadership posting after 26 years in the Air Force: command of the 318th
Recruiting Squadron in Mechanicsburg, Pa. The ceremony, attended by her
friends and family and some 50 to 60 of the Air Force personnel who
would be reporting to Colonel Barr, was supposed to be a celebration of
her accomplishments.

In her neatly pressed blue dress uniform, Colonel Barr adopted the full
at-attention stance to accept her orders and take the oath: Chin up,
shoulders back, stomach in, arms fixed at the side, thumb parallel to
her skirt seam.

Directly behind her, her white predecessor, Lt. Col. Ernest T. Bice, was
supposed to be at attention, too. But just before Colonel Barr's right
hand went to her forehead in a salute, Colonel Bice touched his thumb
and forefinger together and stretched his other three fingers downward,
adopting the sign that the Anti-Defamation League says can be used to
denote white supremacy.

\includegraphics{https://static01.graylady3jvrrxbe.onion/images/2020/05/25/us/politics/25dc-military-holley-sub/25dc-military-holley-sub-jumbo.jpg}Credit...Ilana
Panich-Linsman for The New York Times

``I walk their halls, and nobody on their wall looks like me.''

Lila Holley, former Army chief warrant officer

Colonel Barr's friends posted a video of the actions on Facebook, and
the Air Force investigated. Colonel Bice told investigators that he was
playing a game with his son and had no racist intent. The Air Force
investigation ruled that he was not displaying a racist sign, although
he was issued a letter of counseling for ``unprofessional behavior'' and
went ahead with a planned retirement. Both Colonel Barr and Colonel Bice
declined to comment for this article.

``We always overlook things,'' said Tiffeny Young, a friend of Colonel
Barr, who was at the ceremony. ``But even if it wasn't meant to be
racist, it undermines the seriousness of the situation. He's telling
people `this is your new boss,' and he's not being respectful of her.
When a white dude is behind you doing stuff like that, it undermines
you.''

There are people at top levels of the Pentagon who would like to see a
military leadership that is more reflective of America. Ryan McCarthy,
the secretary of the Army and a former Army Ranger, is one of those who
is trying to increase the number of minority leaders in the Army's top
ranks. Last summer he traveled to Philadelphia for the annual convention
of the black fraternity Kappa Alpha Psi, stumping for more
African-Americans to join the Army's officer corps.

``If we don't get greater diversification in each officer cohort, we
will never catch up,'' Mr. McCarthy said.

It was sunny and windy in Philadelphia as Mr. McCarthy, along with a
majority black delegation from his office, got off the plane and
traveled to the convention center downtown. As he headed up the
escalator to the convention hall for his speech to the Kappas, Mr.
McCarthy looked up at a sea of black faces.

It was a turnaround from what usually faces him in meetings at the
Pentagon. This time, he was the minority in the room.

Eric Schmitt, Thomas Gibbons-Neff, Jennifer Steinhauer and Chris Cameron
contributed reporting.

Advertisement

\protect\hyperlink{after-bottom}{Continue reading the main story}

\hypertarget{site-index}{%
\subsection{Site Index}\label{site-index}}

\hypertarget{site-information-navigation}{%
\subsection{Site Information
Navigation}\label{site-information-navigation}}

\begin{itemize}
\tightlist
\item
  \href{https://help.nytimes3xbfgragh.onion/hc/en-us/articles/115014792127-Copyright-notice}{©~2020~The
  New York Times Company}
\end{itemize}

\begin{itemize}
\tightlist
\item
  \href{https://www.nytco.com/}{NYTCo}
\item
  \href{https://help.nytimes3xbfgragh.onion/hc/en-us/articles/115015385887-Contact-Us}{Contact
  Us}
\item
  \href{https://www.nytco.com/careers/}{Work with us}
\item
  \href{https://nytmediakit.com/}{Advertise}
\item
  \href{http://www.tbrandstudio.com/}{T Brand Studio}
\item
  \href{https://www.nytimes3xbfgragh.onion/privacy/cookie-policy\#how-do-i-manage-trackers}{Your
  Ad Choices}
\item
  \href{https://www.nytimes3xbfgragh.onion/privacy}{Privacy}
\item
  \href{https://help.nytimes3xbfgragh.onion/hc/en-us/articles/115014893428-Terms-of-service}{Terms
  of Service}
\item
  \href{https://help.nytimes3xbfgragh.onion/hc/en-us/articles/115014893968-Terms-of-sale}{Terms
  of Sale}
\item
  \href{https://spiderbites.nytimes3xbfgragh.onion}{Site Map}
\item
  \href{https://help.nytimes3xbfgragh.onion/hc/en-us}{Help}
\item
  \href{https://www.nytimes3xbfgragh.onion/subscription?campaignId=37WXW}{Subscriptions}
\end{itemize}
