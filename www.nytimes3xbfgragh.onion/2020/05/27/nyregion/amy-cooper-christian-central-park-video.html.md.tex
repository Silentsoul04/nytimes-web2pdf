Sections

SEARCH

\protect\hyperlink{site-content}{Skip to
content}\protect\hyperlink{site-index}{Skip to site index}

\href{https://www.nytimes3xbfgragh.onion/section/nyregion}{New York}

\href{https://myaccount.nytimes3xbfgragh.onion/auth/login?response_type=cookie\&client_id=vi}{}

\href{https://www.nytimes3xbfgragh.onion/section/todayspaper}{Today's
Paper}

\href{/section/nyregion}{New York}\textbar{}The Bird Watcher, That
Incident and His Feelings on the Woman's Fate

\url{https://nyti.ms/2ZM4ReH}

\begin{itemize}
\item
\item
\item
\item
\item
\item
\end{itemize}

Advertisement

\protect\hyperlink{after-top}{Continue reading the main story}

Supported by

\protect\hyperlink{after-sponsor}{Continue reading the main story}

\hypertarget{the-bird-watcher-that-incident-and-his-feelings-on-the-womans-fate}{%
\section{The Bird Watcher, That Incident and His Feelings on the Woman's
Fate}\label{the-bird-watcher-that-incident-and-his-feelings-on-the-womans-fate}}

Christian Cooper is already back birding at Central Park. ``I'm not
excusing the racism,'' he said. ``But I don't know if her life needed to
be torn apart.''

\includegraphics{https://static01.graylady3jvrrxbe.onion/images/2020/05/27/nyregion/27CENTRALPARK/27CENTRALPARK-articleLarge.jpg?quality=75\&auto=webp\&disable=upscale}

\href{https://www.nytimes3xbfgragh.onion/by/sarah-maslin-nir}{\includegraphics{https://static01.graylady3jvrrxbe.onion/images/2018/06/13/multimedia/author-sarah-maslin-nir/author-sarah-maslin-nir-thumbLarge.jpg}}

By \href{https://www.nytimes3xbfgragh.onion/by/sarah-maslin-nir}{Sarah
Maslin Nir}

\begin{itemize}
\item
  Published May 27, 2020Updated Sept. 9, 2020
\item
  \begin{itemize}
  \item
  \item
  \item
  \item
  \item
  \item
  \end{itemize}
\end{itemize}

His binoculars around his neck,
\href{https://www.nytimes3xbfgragh.onion/2020/09/09/nyregion/christian-cooper-amy-comic-graphic-novel.html}{Christian
Cooper}, an avid birder, was back in his happy place on Wednesday:
Central Park during migration season. He was trying to focus on the
olive-sided flycatchers and red-bellied woodpeckers --- not on what had
happened there two days earlier.

That was when Mr. Cooper, who is black, asked a white woman to put her
dog on a leash. When she did not, he began filming. In response, the
woman said she would tell the police that ``an African-American man is
threatening my life'' before dialing 911.

On Tuesday,
\href{https://www.nytimes3xbfgragh.onion/2020/05/26/nyregion/amy-cooper-dog-central-park.html}{the
video went viral on Twitter and garnered over 40 million views}, setting
off a painful discourse about the history of
\href{https://www.nytimes3xbfgragh.onion/2020/05/30/nyregion/central-park-video.html}{dangerous
false accusations against black people made to police}.

The birds were a welcome distraction from thinking about what had
happened next: By that day's end, the woman,
\href{https://www.nytimes3xbfgragh.onion/2020/09/09/nyregion/christian-cooper-amy-comic-graphic-novel.html}{Amy
Cooper} (no relation) had surrendered her dog and had been fired from
her high-level finance job. As he wandered the park's North Woods on
Wednesday shortly after dawn, he said he felt exhausted, exposed and
profoundly conflicted, particularly about Ms. Cooper's fate.

\includegraphics{https://static01.graylady3jvrrxbe.onion/images/2020/05/27/nyregion/27CENTRALPARK2/merlin_172843749_b00ebb38-78c0-4173-bb4a-29b9f5b0ba6c-articleLarge.jpg?quality=75\&auto=webp\&disable=upscale}

``Any of us can make --- not necessarily a racist mistake, but a
mistake,'' Mr. Cooper said, ``And to get that kind of tidal wave in such
a compressed period of time, it's got to hurt. It's got to hurt.''

A gray catbird darted around his hiking boots.

``I'm not excusing the racism,'' he said. ``But I don't know if her life
needed to be torn apart.''

He opened his mouth to speak further and then stopped himself. He had
been about to say the phrase, ``that poor woman,'' he later
acknowledged, but he could not bring himself to complete the thought.

``She went racial. There are certain dark societal impulses that she, as
a white woman facing in a conflict with a black man, that she thought
she could marshal to her advantage,'' he said.

``I don't know if it was a conscious thing or not,'' he added. ``But she
did it, and she went there.''

Mr. Cooper's love of birding began at age 10, he said, when his parents,
two Long Island schoolteachers, enrolled him in a 4-H program. There, in
a woodworking class, he crafted a bird feeder that he set in his lawn.

The creatures that flocked to it set off a fascination that has endured
for four decades, through his time at Harvard, where he graduated with a
degree in political science, and into his years as an editor for Marvel
Comics, where he is credited with creating one of the
\href{https://www.out.com/news/2020/5/26/man-center-viral-park-video-made-gay-star-trek-history?utm_source=facebook}{first
gay characters}in the Star Trek comic universe.

A northern rough-winged swallow alighted on a branch and Mr. Cooper, 57,
trained his lenses on it for a while.

Image

Mr. Cooper's interest in bird watching began when he was 10 years old on
Long Island.Credit...Brittainy Newman/The New York Times

Then he resumed. ``If we are going to make progress, we've got to
address these things, and if this painful process is going to help us
address this --- there's the yellow warbler!'' Mr. Cooper said, cutting
himself off to peer around with his binoculars.

At length, he turned his eyes away from the tops of the London plane
trees and continued where he had left off:

``If this painful process --- oh, a Baltimore oriole just flew
across!--- helps to correct, or takes us a step further toward
addressing the underlying racial, horrible assumptions that we
African-Americans have to deal with, and have dealt with for centuries,
that this woman tapped into, then it's worth it,'' he said, setting his
binoculars down on his chest.

``Sadly, it has to come at her expense,'' he added.

On Tuesday, Ms. Cooper was fired by her employer, Franklin Templeton,
where she had been a head of insurance portfolio management, according
to her LinkedIn page.

Ms. Cooper, who graduated from the University of Chicago Booth School of
Business, also surrendered her dog, Henry, to the rescue organization
she had adopted him from, the same day, according to a Facebook post by
the group.

She issued a public apology to Mr. Cooper, whom she had encountered in a
semi-wild part of the park called The Ramble, where dogs must be
leashed.

After she refused to tether her dog on Memorial Day, Mr. Cooper said, he
attempted to lure the dog with treats, to induce her to restrain her
pet. In a statement, Ms. Cooper said she had misread his intent.

``I reacted emotionally and made false assumptions about his intentions
when, in fact, I was the one who was acting inappropriately by not
having my dog on a leash,'' Ms. Cooper said in the statement.

She did not respond to multiple requests for comment.

On Wednesday, New York City's Commission on Human Rights began an
investigation into Ms. Cooper's actions.

On his birding walk Wednesday, Mr. Cooper said he had read her apology.

He called it ``a start.'' He said he was not interested in meeting her
or in any face-to-face reconciliation.

What he was interested in were birds, like the
\href{https://gothamist.com/arts-entertainment/rare-sighting-of-kirtlands-warbler-in-central-park-sends-birdwatchers-into-hushed-frenzy}{sighting
in 2018 of a rare Kirtland's warbler} that led him to sprint from his
office in Midtown Manhattan to the park to catch a glimpse.

Mr. Cooper, who now works in communications and lives on the Lower East
Side, has fed his passion with birding trips to Central Park and around
the world, and he is on the board of the
\href{http://www.nycaudubon.org/leadership-a-governance/board-of-directors}{New
York City Audubon Society}.

He has developed a virtuoso's ear for their birdsong, and can identify
them by chirp. (``There's a myth that I have the best ears in the
park,'' he said. ``It's a myth.'')

As he has pursued his passion, he has been keenly aware of the fact that
there appear to be few other African-American men invested in the hobby,
excluded by the same subtle messaging he gets when he is followed around
in shops, he said.

And he is aware that the image he cuts --- as a man often shuffling the
undergrowth after a rare bird, with a metal object, the binoculars, in
his hand --- can read differently for a black person than for a white
person.

It doesn't stop him.

``We should be out here. The birds belong to all of us,'' he said. ``The
birds don't care what color you are.''

Advertisement

\protect\hyperlink{after-bottom}{Continue reading the main story}

\hypertarget{site-index}{%
\subsection{Site Index}\label{site-index}}

\hypertarget{site-information-navigation}{%
\subsection{Site Information
Navigation}\label{site-information-navigation}}

\begin{itemize}
\tightlist
\item
  \href{https://help.nytimes3xbfgragh.onion/hc/en-us/articles/115014792127-Copyright-notice}{©~2020~The
  New York Times Company}
\end{itemize}

\begin{itemize}
\tightlist
\item
  \href{https://www.nytco.com/}{NYTCo}
\item
  \href{https://help.nytimes3xbfgragh.onion/hc/en-us/articles/115015385887-Contact-Us}{Contact
  Us}
\item
  \href{https://www.nytco.com/careers/}{Work with us}
\item
  \href{https://nytmediakit.com/}{Advertise}
\item
  \href{http://www.tbrandstudio.com/}{T Brand Studio}
\item
  \href{https://www.nytimes3xbfgragh.onion/privacy/cookie-policy\#how-do-i-manage-trackers}{Your
  Ad Choices}
\item
  \href{https://www.nytimes3xbfgragh.onion/privacy}{Privacy}
\item
  \href{https://help.nytimes3xbfgragh.onion/hc/en-us/articles/115014893428-Terms-of-service}{Terms
  of Service}
\item
  \href{https://help.nytimes3xbfgragh.onion/hc/en-us/articles/115014893968-Terms-of-sale}{Terms
  of Sale}
\item
  \href{https://spiderbites.nytimes3xbfgragh.onion}{Site Map}
\item
  \href{https://help.nytimes3xbfgragh.onion/hc/en-us}{Help}
\item
  \href{https://www.nytimes3xbfgragh.onion/subscription?campaignId=37WXW}{Subscriptions}
\end{itemize}
