Sections

SEARCH

\protect\hyperlink{site-content}{Skip to
content}\protect\hyperlink{site-index}{Skip to site index}

\href{https://www.nytimes3xbfgragh.onion/section/business/economy}{Economy}

\href{https://myaccount.nytimes3xbfgragh.onion/auth/login?response_type=cookie\&client_id=vi}{}

\href{https://www.nytimes3xbfgragh.onion/section/todayspaper}{Today's
Paper}

\href{/section/business/economy}{Economy}\textbar{}`Still Catching Up':
Jobless Numbers May Not Tell Full Story

\url{https://nyti.ms/3c5h00Q}

\begin{itemize}
\item
\item
\item
\item
\item
\item
\end{itemize}

\hypertarget{the-coronavirus-outbreak}{%
\subsubsection{\texorpdfstring{\href{https://www.nytimes3xbfgragh.onion/news-event/coronavirus?name=styln-coronavirus-markets\&region=TOP_BANNER\&block=storyline_menu_recirc\&action=click\&pgtype=Article\&impression_id=b8528d20-f52e-11ea-bd61-6d19986efa33\&variant=undefined}{The
Coronavirus
Outbreak}}{The Coronavirus Outbreak}}\label{the-coronavirus-outbreak}}

\begin{itemize}
\tightlist
\item
  live\href{https://www.nytimes3xbfgragh.onion/2020/09/12/world/covid-19-coronavirus.html?name=styln-coronavirus-markets\&region=TOP_BANNER\&block=storyline_menu_recirc\&action=click\&pgtype=Article\&impression_id=b852b430-f52e-11ea-bd61-6d19986efa33\&variant=undefined}{Latest
  Updates}
\item
  \href{https://www.nytimes3xbfgragh.onion/interactive/2020/us/coronavirus-us-cases.html?name=styln-coronavirus-markets\&region=TOP_BANNER\&block=storyline_menu_recirc\&action=click\&pgtype=Article\&impression_id=b852b431-f52e-11ea-bd61-6d19986efa33\&variant=undefined}{Maps
  and Cases}
\item
  \href{https://www.nytimes3xbfgragh.onion/interactive/2020/science/coronavirus-vaccine-tracker.html?name=styln-coronavirus-markets\&region=TOP_BANNER\&block=storyline_menu_recirc\&action=click\&pgtype=Article\&impression_id=b852b432-f52e-11ea-bd61-6d19986efa33\&variant=undefined}{Vaccine
  Tracker}
\item
  \href{https://www.nytimes3xbfgragh.onion/2020/09/10/us/politics/fda-coronavirus-vaccine.html?name=styln-coronavirus-markets\&region=TOP_BANNER\&block=storyline_menu_recirc\&action=click\&pgtype=Article\&impression_id=b852b433-f52e-11ea-bd61-6d19986efa33\&variant=undefined}{F.D.A.
  Regulators' Self-Defense}
\item
  \href{https://www.nytimes3xbfgragh.onion/2020/09/09/upshot/coronavirus-surprise-test-fees.html?name=styln-coronavirus-markets\&region=TOP_BANNER\&block=storyline_menu_recirc\&action=click\&pgtype=Article\&impression_id=b852b434-f52e-11ea-bd61-6d19986efa33\&variant=undefined}{Surprise
  Test Fees}
\end{itemize}

Advertisement

\protect\hyperlink{after-top}{Continue reading the main story}

Supported by

\protect\hyperlink{after-sponsor}{Continue reading the main story}

\hypertarget{still-catching-up-jobless-numbers-may-not-tell-full-story}{%
\section{`Still Catching Up': Jobless Numbers May Not Tell Full
Story}\label{still-catching-up-jobless-numbers-may-not-tell-full-story}}

Unemployment claims exceed 40 million since the start of the pandemic,
with 2.1 million added last week, but a backlog may be leaving many
uncounted.

\includegraphics{https://static01.graylady3jvrrxbe.onion/images/2020/05/28/business/28virus-jobless1/merlin_172898607_9cd928ec-f3a5-4324-9cc5-47667286e8a0-articleLarge.jpg?quality=75\&auto=webp\&disable=upscale}

\href{https://www.nytimes3xbfgragh.onion/by/patricia-cohen}{\includegraphics{https://static01.graylady3jvrrxbe.onion/images/2018/02/16/multimedia/author-patricia-cohen/author-patricia-cohen-thumbLarge.jpg}}

By \href{https://www.nytimes3xbfgragh.onion/by/patricia-cohen}{Patricia
Cohen}

\begin{itemize}
\item
  Published May 28, 2020Updated July 9, 2020
\item
  \begin{itemize}
  \item
  \item
  \item
  \item
  \item
  \item
  \end{itemize}
\end{itemize}

More than 40 million people --- the equivalent of one out of every four
American workers --- have filed for
\href{https://www.nytimes3xbfgragh.onion/2020/06/11/us/politics/unemployment-benefits-coronavirus.html}{unemployment
benefits} since the coronavirus pandemic grabbed hold in mid-March,
\href{https://www.dol.gov/ui/data.pdf}{the government reported} on
Thursday, an astounding tally that rivals the bleakest years of the
\href{https://www.thebalance.com/unemployment-rate-by-year-3305506}{Great
Depression.}

The latest additions --- the 2.1 million people who filed
\href{https://www.nytimes3xbfgragh.onion/2020/07/09/business/economy/unemployment-claims-coronavirus.html}{state
unemployment claims} last week --- may not be only a result of fresh
layoffs, but also evidence that states are working their way through
some of the choking backlog.

``We're still catching up,'' \href{https://twitter.com/DianeSwonk}{Diane
Swonk}, chief economist at the accounting firm Grant Thornton, said of
the newest claims. ``The lags have been long.''

The Labor Department report marks the eighth week in a row that new
jobless claims, on a seasonally adjusted basis, dipped from the
\href{https://oui.doleta.gov/press/2020/052820.pdf}{peak of almost 6.9
million} --- but the level is still far above any other historical
highs.

6

million

40.8 million

5

Claims were filed in

the last 10 weeks

4

Initial jobless claims, per week

Seasonally adjusted

3

2

RECESSION

1

'06

'08

'09

'12

'16

'20

6

million

40.8 million

5

Claims were filed in

the last 10 weeks

4

3

RECESSION

2

Initial jobless claims, per week

Seasonally adjusted

1

'06

'08

'09

'12

'16

'20

Source: Department of Labor

By The New York Times

At the same time, overcounting in some places and undercounting in
others makes it difficult to precisely measure the number of layoffs
caused by the pandemic --- and in devising a policy response.

``When we think about what to do when benefits expire, it would be
helpful to know how many people are actually getting them,'' said
\href{https://twitter.com/ENPancotti}{Elizabeth Pancotti,} a research
assistant at the National Bureau of Economic Research. While the Labor
Department reports may be the best source of information, she said, they
offer an ``incomplete picture.''

\hypertarget{latest-updates-the-coronavirus-outbreak-and-the-economy}{%
\section{\texorpdfstring{\href{https://www.nytimes3xbfgragh.onion/live/2020/09/11/business/stock-market-today-coronavirus?action=click\&pgtype=Article\&state=default\&region=MAIN_CONTENT_1\&context=storylines_live_updates}{Latest
Updates: The Coronavirus Outbreak and the
Economy}}{Latest Updates: The Coronavirus Outbreak and the Economy}}\label{latest-updates-the-coronavirus-outbreak-and-the-economy}}

\href{https://www.nytimes3xbfgragh.onion/live/2020/09/11/business/stock-market-today-coronavirus?action=click\&pgtype=Article\&state=default\&region=MAIN_CONTENT_1\&context=storylines_live_updates\#the-nyse-may-move-its-data-center-out-of-new-jersey-in-response-to-a-proposed-tax}{23h
ago}

\href{https://www.nytimes3xbfgragh.onion/live/2020/09/11/business/stock-market-today-coronavirus?action=click\&pgtype=Article\&state=default\&region=MAIN_CONTENT_1\&context=storylines_live_updates\#the-nyse-may-move-its-data-center-out-of-new-jersey-in-response-to-a-proposed-tax}{The
N.Y.S.E. may move its data center out of New Jersey in response to a
proposed tax.}

\href{https://www.nytimes3xbfgragh.onion/live/2020/09/11/business/stock-market-today-coronavirus?action=click\&pgtype=Article\&state=default\&region=MAIN_CONTENT_1\&context=storylines_live_updates\#the-federal-budget-deficit-hit-3-trillion-as-of-august}{26h
ago}

\href{https://www.nytimes3xbfgragh.onion/live/2020/09/11/business/stock-market-today-coronavirus?action=click\&pgtype=Article\&state=default\&region=MAIN_CONTENT_1\&context=storylines_live_updates\#the-federal-budget-deficit-hit-3-trillion-as-of-august}{The
federal budget deficit hit \$3 trillion as of August.}

\href{https://www.nytimes3xbfgragh.onion/live/2020/09/11/business/stock-market-today-coronavirus?action=click\&pgtype=Article\&state=default\&region=MAIN_CONTENT_1\&context=storylines_live_updates\#warner-bros-pushes-the-release-of-wonder-woman-1984-to-christmas}{26h
ago}

\href{https://www.nytimes3xbfgragh.onion/live/2020/09/11/business/stock-market-today-coronavirus?action=click\&pgtype=Article\&state=default\&region=MAIN_CONTENT_1\&context=storylines_live_updates\#warner-bros-pushes-the-release-of-wonder-woman-1984-to-christmas}{Warner
Bros. pushes the release of `Wonder Woman 1984' to Christmas.}

\href{https://www.nytimes3xbfgragh.onion/live/2020/09/11/business/stock-market-today-coronavirus?action=click\&pgtype=Article\&state=default\&region=MAIN_CONTENT_1\&context=storylines_live_updates}{See
more updates}

More live coverage:
\href{https://www.nytimes3xbfgragh.onion/2020/09/11/world/covid-19-coronavirus.html?action=click\&pgtype=Article\&state=default\&region=MAIN_CONTENT_1\&context=storylines_live_updates}{Global}

Shelter-in-place orders and business restrictions have been lifting
across the country, and there is evidence in the report that some
workers are being called back: The number of people receiving state
jobless benefits dropped by roughly 3.8 million to 21.1 million for the
week ended May 16.

But as Ms. Swonk noted, ``it's not enough to offset the extraordinary
economic devastation and job losses associated with Covid-19.''

And while rehiring certainly accounts for a chunk of that decline,
workers who had exhausted their weekly state benefits would also be
reflected.

Reopenings remain bumpy and incomplete, and flare-ups of the coronavirus
continue to disrupt business. On Tuesday, Ford Motor
\href{https://www.freep.com/story/money/cars/ford/2020/05/26/kansas-city-plant-production-claymoco-ford-worker-postive-covid/5262884002/}{temporarily
halted production at the Kansas City assembly plant} in Missouri to deep
clean after an employee tested positive for the virus. Two other Ford
plants --- in Chicago and Dearborn, Mich. --- were also temporarily
closed.

In Thursday's report, the department offered two sets of figures. One
includes the more than 40 million people who have applied for state
benefits and is seasonally adjusted. The other includes those who
applied under the new federal emergency program called Pandemic
Unemployment Assistance and is not seasonally adjusted. So far, more
than 10 million have applied.

Pandemic Unemployment Assistance, part of an expanded palette of jobless
benefits passed by Congress two months ago, is meant to help
freelancers, gig workers, the self-employed and others who would not
normally qualify under state rules.

But several economists suspect that there is a lot of double counting
and warn against simply adding figures from the two programs together.

Some states, flooded with applicants, were slow to put the pandemic
program into effect. Initially, many people were mistakenly told they
were ineligible. Others were instructed to apply for state benefits
first and be rejected before applying for the federal benefits.

That confused application process has caused potentially millions of
laid-off workers to be counted twice. States are also weeding out
duplicate applications from frustrated filers who had trouble getting
through or did not receive any response after weeks of waiting.

Then there are the mistakes. A data entry slip-up caused Massachusetts
to pump up the number of federal claims by nearly a million last week.
The previous week, a similar flub in Connecticut mistakenly inflated its
total by a quarter of a million.

``It's unclear if states are including duplicate claims due to error,
\href{https://www.nytimes3xbfgragh.onion/2020/05/21/us/coronavirus-news-tracker.html?searchResultPosition=2}{fraud}
or the Pandemic Unemployment Assistance program,'' said
\href{https://twitter.com/ernietedeschi}{Ernie Tedeschi}, a policy
economist at Evercore ISI in Washington.

But that is not all. That system is also probably missing millions of
other laid-off workers.

As of Tuesday, three states had not put the pandemic unemployment
insurance program into effect, and several others have yet to report any
claims. Thirteen states have not started another federal emergency
relief program, to provide an additional 13 weeks of benefits to workers
who have exhausted their state benefits.

Laid-off workers who have not applied for benefits and those who have
left the labor force entirely are not included in the claims numbers.
Nor are any of the eight million undocumented workers who lost their
jobs. They are not eligible for any benefits. Neither are new graduates
just entering the labor force.

Matthew Wilson, 24, who lost his barista job in Philadelphia, was turned
down because he had been working in the state for less than a year.

``It doesn't make any sense --- I moved, and now I'm magically not
qualified for unemployment?'' said Mr. Wilson, who relocated to
Pennsylvania after graduating from Tufts University in Massachusetts
last year. He appealed the decision and heard last week that his claim
had been approved, but he hasn't received any money. His partner, who
also lost her job as a barista, has applied four times but has yet to
collect benefits.

As for regular unemployment benefits, states draw up their own rules and
administer benefits. The result is that in some places, like Florida,
Texas and Arizona, only a small fraction of jobless workers are
receiving benefits, while other states offer much broader coverage.

The way ``initial claims'' are counted may also vary by state, with some
excluding claims that have not been processed.

Laurie Yadoff, a lawyer at Coast to Coast Legal Aid of South Florida,
said she had about 100 clients who qualify for regular state benefits
but have had trouble filing. Many are poor and older, with limited or no
access to the internet. ``A lot of them fall into the regular state
benefit program, and a lot of them are straightforward, and a lot of
them are still not getting money,'' she said.

Even when Ms. Yadoff has been able to get someone on the phone, the
person at the other end often doesn't know the answer. ``People are
desperate and frustrated,'' she said. ``They don't know what to do.''

Monitoring
\href{https://www.fiscal.treasury.gov/reports-statements/dts/index.html}{withdrawals
from the Treasury Departmen}t, Mr. Tedeschi of Evercore estimated that
by early May, roughly three-quarters of those eligible for benefits had
started to receive them.

Allison Hester, who is 50 and lives in Little Rock, Ark., applied for
unemployment after being laid off from her job in content marketing at a
window cleaning supply company in March, but was never able to get
through. ``I tried on and off for a month, but our system was so
overwhelmed.''

She returned to her job this month. ``It feels good, but I don't feel
secure anymore,'' Ms. Hester said. ``I don't take anything for
granted.''

Nelson D. Schwartz and Tiffany Hsu contributed reporting.

Advertisement

\protect\hyperlink{after-bottom}{Continue reading the main story}

\hypertarget{site-index}{%
\subsection{Site Index}\label{site-index}}

\hypertarget{site-information-navigation}{%
\subsection{Site Information
Navigation}\label{site-information-navigation}}

\begin{itemize}
\tightlist
\item
  \href{https://help.nytimes3xbfgragh.onion/hc/en-us/articles/115014792127-Copyright-notice}{©~2020~The
  New York Times Company}
\end{itemize}

\begin{itemize}
\tightlist
\item
  \href{https://www.nytco.com/}{NYTCo}
\item
  \href{https://help.nytimes3xbfgragh.onion/hc/en-us/articles/115015385887-Contact-Us}{Contact
  Us}
\item
  \href{https://www.nytco.com/careers/}{Work with us}
\item
  \href{https://nytmediakit.com/}{Advertise}
\item
  \href{http://www.tbrandstudio.com/}{T Brand Studio}
\item
  \href{https://www.nytimes3xbfgragh.onion/privacy/cookie-policy\#how-do-i-manage-trackers}{Your
  Ad Choices}
\item
  \href{https://www.nytimes3xbfgragh.onion/privacy}{Privacy}
\item
  \href{https://help.nytimes3xbfgragh.onion/hc/en-us/articles/115014893428-Terms-of-service}{Terms
  of Service}
\item
  \href{https://help.nytimes3xbfgragh.onion/hc/en-us/articles/115014893968-Terms-of-sale}{Terms
  of Sale}
\item
  \href{https://spiderbites.nytimes3xbfgragh.onion}{Site Map}
\item
  \href{https://help.nytimes3xbfgragh.onion/hc/en-us}{Help}
\item
  \href{https://www.nytimes3xbfgragh.onion/subscription?campaignId=37WXW}{Subscriptions}
\end{itemize}
