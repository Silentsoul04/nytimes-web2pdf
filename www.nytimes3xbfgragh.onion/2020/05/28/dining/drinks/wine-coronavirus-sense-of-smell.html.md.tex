Sections

SEARCH

\protect\hyperlink{site-content}{Skip to
content}\protect\hyperlink{site-index}{Skip to site index}

\href{https://www.nytimes3xbfgragh.onion/section/food/drinks}{Wine, Beer
\& Cocktails}

\href{https://myaccount.nytimes3xbfgragh.onion/auth/login?response_type=cookie\&client_id=vi}{}

\href{https://www.nytimes3xbfgragh.onion/section/todayspaper}{Today's
Paper}

\href{/section/food/drinks}{Wine, Beer \&
Cocktails}\textbar{}Rediscovering Wine After Covid-19

\url{https://nyti.ms/2M81AOS}

\begin{itemize}
\item
\item
\item
\item
\item
\item
\end{itemize}

\href{https://www.nytimes3xbfgragh.onion/spotlight/at-home?action=click\&pgtype=Article\&state=default\&region=TOP_BANNER\&context=at_home_menu}{At
Home}

\begin{itemize}
\tightlist
\item
  \href{https://www.nytimes3xbfgragh.onion/2020/09/07/travel/route-66.html?action=click\&pgtype=Article\&state=default\&region=TOP_BANNER\&context=at_home_menu}{Cruise
  Along: Route 66}
\item
  \href{https://www.nytimes3xbfgragh.onion/2020/09/04/dining/sheet-pan-chicken.html?action=click\&pgtype=Article\&state=default\&region=TOP_BANNER\&context=at_home_menu}{Roast:
  Chicken With Plums}
\item
  \href{https://www.nytimes3xbfgragh.onion/2020/09/04/arts/television/dark-shadows-stream.html?action=click\&pgtype=Article\&state=default\&region=TOP_BANNER\&context=at_home_menu}{Watch:
  Dark Shadows}
\item
  \href{https://www.nytimes3xbfgragh.onion/interactive/2020/at-home/even-more-reporters-editors-diaries-lists-recommendations.html?action=click\&pgtype=Article\&state=default\&region=TOP_BANNER\&context=at_home_menu}{Explore:
  Reporters' Google Docs}
\end{itemize}

Advertisement

\protect\hyperlink{after-top}{Continue reading the main story}

Supported by

\protect\hyperlink{after-sponsor}{Continue reading the main story}

\hypertarget{rediscovering-wine-after-covid-19}{%
\section{Rediscovering Wine After
Covid-19}\label{rediscovering-wine-after-covid-19}}

Aside from its toll on human life, the pandemic has also stolen little
things, like the ability to smell and taste. How a neurologist found a
deeper relationship with wine.

\includegraphics{https://static01.graylady3jvrrxbe.onion/images/2020/06/03/dining/28pour/merlin_172722315_1777fcdf-30fc-4bc6-bedc-15a30b784d5a-articleLarge.jpg?quality=75\&auto=webp\&disable=upscale}

\href{https://www.nytimes3xbfgragh.onion/by/eric-asimov}{\includegraphics{https://static01.graylady3jvrrxbe.onion/images/2018/06/13/multimedia/author-eric-asimov/author-eric-asimov-thumbLarge.jpg}}

By \href{https://www.nytimes3xbfgragh.onion/by/eric-asimov}{Eric Asimov}

\begin{itemize}
\item
  May 28, 2020
\item
  \begin{itemize}
  \item
  \item
  \item
  \item
  \item
  \item
  \end{itemize}
\end{itemize}

This is a story about what happens when one of life's joys is taken
away, perhaps forever. In this case it's wine, but it could as easily
have been painting, cooking, dancing, or playing golf or tennis.

The potential loss of these pleasures, of course, is trivial compared
with the social and personal catastrophes the coronavirus pandemic has
inflicted. It has taken friends and loved ones, destroyed jobs and
businesses, and shaken up lives. The human cost has been immense.

Yet people still want to savor what they love, what has shaped their
personalities and lives. They want to return to bars and restaurants, to
date and find romance, to play softball on the weekends and dive once
more into the wild surf.

\includegraphics{https://static01.graylady3jvrrxbe.onion/images/2020/05/28/dining/28pour2/merlin_172722282_f0ceb036-58d1-4523-b036-4d39ecfb7702-articleLarge.jpg?quality=75\&auto=webp\&disable=upscale}

Dr. Michael Pourfar's pleasure was wine, particularly on the weekends
when he and his wife, Jennifer, retreated from their workaday lives in
Manhattan to the Hudson Valley with their children, Alex, 13, and
Caroline, 9.

His loss of that pleasure traces back to one morning in mid-March when
his wife told him she could not smell her coffee.

Dr. Pourfar, 49, a neurologist who specializes in treating people with
Parkinson's disease and other nerve disorders, had not been treating
Covid-19 patients directly, but he knew about its symptoms.

His hospital,
\href{https://www.nytimes3xbfgragh.onion/2020/04/29/nyregion/coronavirus-nyc-hospitals.html}{N.Y.U.
Langone Health}, on the East Side of Manhattan, was hit hard in the
pandemic's early stages, and Dr. Pourfar had seen enough coronavirus
patients to understand that
\href{https://www.nytimes3xbfgragh.onion/2020/03/22/health/coronavirus-symptoms-smell-taste.html}{losing
one's sense of smell} was a possible first
\href{https://www.the-scientist.com/news-opinion/loss-of-smell-taste-may-be-reliable-predictor-of-covid-19-study-67528}{sign
of infection}.

He also realized that if his wife was infected with the coronavirus, he
had a greater chance of getting it, too.

As anyone might, he at first pondered the most morbid possibilities. He
was particularly worried about their children.

But his medical training soon kicked in. After rationally assessing the
situation, he concluded that while they might all get sick, the chances
of grave illness were low. For now, he and his wife needed to maintain a
calm routine for the sake of the children, as well as for their own
peace of mind.

That evening, routine meant choosing a bottle of wine from the cellar.
It was their weekend custom, and Ms. Pourfar wanted a glass even though
she was unable to smell anything.

Knowing that this might be the last bottle they would enjoy for a while,
he pondered his selection.

He considered a few of the most precious bottles he owned --- a
\href{https://www.sothebys.com/en/articles/5-things-to-know-about-domaine-de-la-romanee-conti}{Domaine
de la Romanée-Conti}, one of the great Burgundies, perhaps, or a
\href{https://www.thewinecellarinsider.com/bordeaux-wine-producer-profiles/bordeaux/st-emilion/cheval-blanc/}{Cheval
Blanc}, an equally hallowed Bordeaux. But he settled on a bottle of
\href{http://www.princeofpinot.com/article/851/}{Williams Selyem} pinot
noir from the Russian River Valley, a wine he and his wife had
discovered early in their marriage and enjoyed together regularly.

Within a few days of opening the Williams Selyem, the couple were
feverish, with aches and chills and relentless coughs. They could not
smell a thing, nor taste the food they forced themselves to eat.

But they were not sick enough for the hospital. Instead, they
quarantined themselves in their home, where they were able to care in
shifts for their children. Their son had mild symptoms, their daughter
none at all. But for the parents, the illness dragged on.

``You'd think you were getting better, then evening would come, and
you'd realize you're not out of it yet,'' he said. ``It wasn't really a
dragon, but it had a long tail.''

After a full month, they began to feel much better; Dr. Pourfar's
symptoms did not disappear entirely until mid-May. His sense of smell,
though, did not return. He understood that losing the ability to enjoy
wine was a small price to pay for one's life and health. Still, he could
not help but feel that in a small way he had been diminished.

Like many wine lovers, he had constructed what he called ``life's
comforting rituals'' around fetching a bottle: ``The considered
selection, the careful handling, the slow, deliberate opening and
thoughtful smelling, the little smile, they were gone,'' he said.

Dr. Pourfar, who grew up in Monroe, N.Y., near West Point, discovered
wine when, as a high school student, he spent a year in Alsace, France.
There, he lived with a family who always had wine on the table. He found
himself paying attention to it, and wine became entwined with his time
there.

``You don't realize what a powerful connection these sorts of flavors
can have with your life's experiences and memories,'' he said.

From there, in fits and starts, Dr. Pourfar set out on his exploration.
In medical school, he fell in with some fans of German wines, and then,
when he decided to study wine seriously, he began with Bordeaux, a
customary point of departure because of its rich history and the
relative simplicity of its structure and geography.

Like many whose wine journey began in the 1990s, Dr. Pourfar first
embraced the bold, fruity bottles that were popular and critically
acclaimed at the time. As he became more confident in his own tastes, he
gravitated toward subtler, more nuanced wines. Eventually, his arc of
discovery led him to Burgundy.

``It's where everybody ends up in this world, and it took me a long time
before I got it,'' he said.

Any wine at all, however, seemed unthinkable as he recovered from
Covid-19. So much of the pleasure of wine and the ability to taste are
dependent on the nose. But he could not smell much of anything.

Shortly after he had fallen ill, he gave himself a daily exercise,
partly in hopes of rehabilitating his olfactory sense, and partly out of
scientific curiosity. Because of its relative subtlety, wine was beyond
his capability, but he began taking daily whiffs of coffee in the
morning and of Rémy Martin X.O., a particularly aromatic Cognac, in the
afternoon, in order to gauge his sensitivity.

Early on, he could smell nothing. But slowly the sense began to return.
Each day he tracked his progression, and rated his ability using a scale
derived from Cognac's
\href{https://www.eater.com/drinks/2015/11/17/9747068/what-is-cognac}{hierarchy
of classifications}: V.S. would represent a trace return of smell,
V.S.O.P. a moderate return and X.O. a complete recovery.

The trajectory, like the overall recovery, was frustrating and erratic.
After two weeks of peaks and valleys, he found himself plateauing at the
V.S.O.P. level. Entire realms of aromas seemed beyond his reach, yet his
taste for wine was returning.

``Only when you start to get better do you realize you want part of your
sense of self back,'' he said. ``It's a joy that's part of something
bigger. Not everybody feels this way about wine, but they feel this way
about something.''

He found that he could not appreciate the subtleties of wines he had
come to love, like good Burgundies. At first he considered this a sort
of wine purgatory, a limbo where the desire had returned, but not the
means for satisfaction.

In his diminished state, he found his tastes beginning to change. He was
being drawn to the sorts of bolder, more effusive wines that he had once
enjoyed but believed he had outgrown.

\href{https://www.nytimes3xbfgragh.onion/2014/07/09/dining/your-next-lesson-zinfandel.html}{Zinfandel,}
which he had come to think of as exaggerated, he now perceived as
vibrant and alive. New Zealand
\href{https://www.nytimes3xbfgragh.onion/2017/07/27/dining/wine-school-new-zealand-sauvignon-blanc.html}{sauvignon
blanc}, which he had dismissed as overpowering, now seemed distinctive
and welcome.

Most especially, he said, he found renewed love and respect for
\href{https://www.nytimes3xbfgragh.onion/2016/10/26/dining/wine-review-bordeaux-2011.html}{Bordeaux},
another old favorite he had largely abandoned.

``These wines I thought I'd moved on from, I've found I'm grateful for
them now,'' he said. Enjoying Bordeaux again, he said, was like ``a
\href{https://www.youtube.com/watch?v=O4mQqVqRB7I}{Rosebud} moment.''
But where he might have craved one of the more exclusive labels, if only
to try to understand the appeal, he now found good bistro bottles like a
\href{https://www.thewinecellarinsider.com/bordeaux-wine-producer-profiles/bordeaux/haut-medoc-lesser-appellations/poujeaux/}{Château
Poujeaux} delightful and satisfying.

The rediscovery and acceptance of wines past, particularly those not
considered in the top echelon, he decided, was an indication that
perhaps he has become a little less judgmental about wine, a little more
tolerant.

``You don't have to put down what you liked at a certain time in your
life because you are different now,'' Dr. Pourfar said. ``I hope I will
have the ability not to be so binary. All of these things are wonderful
in the right context. If somebody's excited about it, there's probably
something to it.''

Image

As he recovered, Dr. Pourfar's relationship with wine changed. He found
himself becoming less judgmental.Credit...Sasha Maslov for The New York
Times

His path toward recovery has also made him consider the role wine came
to play in his life, not just as an enjoyable beverage, but as an
essential component of his character. He wonders whether his altered
experience of wine has changed him as a person.

``We all compose a sensory kaleidoscope out of our life experiences that
shapes our appreciation of the world,'' he said. ``Losing an
appreciation of wine's flavors was for me like losing the color red from
my kaleidoscope. The world was still beautiful and I was grateful for
the greens, blues and other colors that remained, but I realized
something important and familiar was missing, and the world just isn't
quite the same.''

As he recovered, Dr. Pourfar gingerly returned to work, first practicing
telemedicine from his country house, then heading into New York a few
times a week.

He has thought about the advice he has given in the past to some of his
Parkinson's patients who enjoy golf.

``I say, `You won't play golf like you did in your 30s, but you can
still play and enjoy the game,''' he said.

And he has continued to measure his recovery on what he calls the
Cognac-o-meter. The most recent report was positive.

``Gamay, which tasted all out of whack with shrill tartness a few weeks
ago, has fallen back in line,'' he said. ``Maybe not X.O., but getting
there.''

\emph{Follow} \href{https://twitter.com/nytfood}{\emph{NYT Food on
Twitter}} \emph{and}
\href{https://www.instagram.com/nytcooking/}{\emph{NYT Cooking on
Instagram}}\emph{,}
\href{https://www.facebookcorewwwi.onion/nytcooking/}{\emph{Facebook}}\emph{,}
\href{https://www.youtube.com/nytcooking}{\emph{YouTube}} \emph{and}
\href{https://www.pinterest.com/nytcooking/}{\emph{Pinterest}}\emph{.}
\href{https://www.nytimes3xbfgragh.onion/newsletters/cooking}{\emph{Get
regular updates from NYT Cooking, with recipe suggestions, cooking tips
and shopping advice}}\emph{.}

Advertisement

\protect\hyperlink{after-bottom}{Continue reading the main story}

\hypertarget{site-index}{%
\subsection{Site Index}\label{site-index}}

\hypertarget{site-information-navigation}{%
\subsection{Site Information
Navigation}\label{site-information-navigation}}

\begin{itemize}
\tightlist
\item
  \href{https://help.nytimes3xbfgragh.onion/hc/en-us/articles/115014792127-Copyright-notice}{©~2020~The
  New York Times Company}
\end{itemize}

\begin{itemize}
\tightlist
\item
  \href{https://www.nytco.com/}{NYTCo}
\item
  \href{https://help.nytimes3xbfgragh.onion/hc/en-us/articles/115015385887-Contact-Us}{Contact
  Us}
\item
  \href{https://www.nytco.com/careers/}{Work with us}
\item
  \href{https://nytmediakit.com/}{Advertise}
\item
  \href{http://www.tbrandstudio.com/}{T Brand Studio}
\item
  \href{https://www.nytimes3xbfgragh.onion/privacy/cookie-policy\#how-do-i-manage-trackers}{Your
  Ad Choices}
\item
  \href{https://www.nytimes3xbfgragh.onion/privacy}{Privacy}
\item
  \href{https://help.nytimes3xbfgragh.onion/hc/en-us/articles/115014893428-Terms-of-service}{Terms
  of Service}
\item
  \href{https://help.nytimes3xbfgragh.onion/hc/en-us/articles/115014893968-Terms-of-sale}{Terms
  of Sale}
\item
  \href{https://spiderbites.nytimes3xbfgragh.onion}{Site Map}
\item
  \href{https://help.nytimes3xbfgragh.onion/hc/en-us}{Help}
\item
  \href{https://www.nytimes3xbfgragh.onion/subscription?campaignId=37WXW}{Subscriptions}
\end{itemize}
