Sections

SEARCH

\protect\hyperlink{site-content}{Skip to
content}\protect\hyperlink{site-index}{Skip to site index}

\href{https://www.nytimes3xbfgragh.onion/spotlight/at-home}{At Home}

\href{https://myaccount.nytimes3xbfgragh.onion/auth/login?response_type=cookie\&client_id=vi}{}

\href{https://www.nytimes3xbfgragh.onion/section/todayspaper}{Today's
Paper}

\href{/spotlight/at-home}{At Home}\textbar{}If You've Got Lemons, Make
Limoncello

\url{https://nyti.ms/3gzY5ib}

\begin{itemize}
\item
\item
\item
\item
\item
\end{itemize}

\href{https://www.nytimes3xbfgragh.onion/spotlight/at-home?action=click\&pgtype=Article\&state=default\&region=TOP_BANNER\&context=at_home_menu}{At
Home}

\begin{itemize}
\tightlist
\item
  \href{https://www.nytimes3xbfgragh.onion/2020/09/07/travel/route-66.html?action=click\&pgtype=Article\&state=default\&region=TOP_BANNER\&context=at_home_menu}{Cruise
  Along: Route 66}
\item
  \href{https://www.nytimes3xbfgragh.onion/2020/09/04/dining/sheet-pan-chicken.html?action=click\&pgtype=Article\&state=default\&region=TOP_BANNER\&context=at_home_menu}{Roast:
  Chicken With Plums}
\item
  \href{https://www.nytimes3xbfgragh.onion/2020/09/04/arts/television/dark-shadows-stream.html?action=click\&pgtype=Article\&state=default\&region=TOP_BANNER\&context=at_home_menu}{Watch:
  Dark Shadows}
\item
  \href{https://www.nytimes3xbfgragh.onion/interactive/2020/at-home/even-more-reporters-editors-diaries-lists-recommendations.html?action=click\&pgtype=Article\&state=default\&region=TOP_BANNER\&context=at_home_menu}{Explore:
  Reporters' Google Docs}
\end{itemize}

Advertisement

\protect\hyperlink{after-top}{Continue reading the main story}

Supported by

\protect\hyperlink{after-sponsor}{Continue reading the main story}

\hypertarget{if-youve-got-lemons-make-limoncello}{%
\section{If You've Got Lemons, Make
Limoncello}\label{if-youve-got-lemons-make-limoncello}}

You'll also need alcohol, sugar and, most of all, patience.

\includegraphics{https://static01.graylady3jvrrxbe.onion/images/2020/05/31/multimedia/31ah-limoncello1/merlin_172902267_e41e2116-2d0f-42ce-9cc4-b9ae4edea236-articleLarge.jpg?quality=75\&auto=webp\&disable=upscale}

By \href{https://www.nytimes3xbfgragh.onion/by/adriana-balsamo}{Adriana
Balsamo}

\begin{itemize}
\item
  Published May 29, 2020Updated June 1, 2020
\item
  \begin{itemize}
  \item
  \item
  \item
  \item
  \item
  \end{itemize}
\end{itemize}

Limoncello is a vibrantly colored digestif that goes down easy on a hot
summer's day. Although the sweet liqueur is most often served, chilled,
in a shot glass, it is meant to be sipped slowly after your meal.

And although easy to make, limoncello has a complicated --- and
contested --- history. Some people **** believe that
\href{https://www.nytimes3xbfgragh.onion/2016/06/24/realestate/an-italian-retreat-with-a-monastic-past.html}{Italian
monks} **** first **** made the spirit as early as the Middle Ages;
others credit Southern Italian fishermen who were thought to drink it
upon returning to shore to warm themselves and fight off colds. But most
accounts attribute its creation to Maria Antonia Farace, who reportedly
lived on a small island off Italy's southern coast in the early 1900s.
One of Farace's descendants registered a small limoncello brand with
Federvini, an Italian trade group, in 1988, using her original recipe.

You can use any kind of organic lemon to make your own limoncello.
(You'll want to avoid fruit that has been treated with pesticides or
other unwanted chemicals, since you'll be using the lemon peels to make
the elixir.) Note that the spirit needs to sit for at least two weeks
before you indulge in a cool glass.

You can also experiment with using other citrus fruits: By following the
same process, you can make 'cellos from limes, grapefruits, oranges ---
the list goes on. Whatever you decide to use, you'll have a lot of
leftover of peeled fruit, and that provides its own
\href{https://cooking.nytimes3xbfgragh.onion/68861692-nyt-cooking/10757465-our-20-best-lemon-desserts}{culinary
opportunities}. You could squeeze your lemons into a
\href{https://cooking.nytimes3xbfgragh.onion/recipes/8703-sweet-or-salty-lemonade}{lemonade}
or stick with the Italian theme and make granita, a Sicilian water ice.
Or try your hand at a baked good, like Melissa Clark's
\href{https://cooking.nytimes3xbfgragh.onion/recipes/1014800-lemon-poppy-seed-pound-cake?action=click\&module=Collection\%20Page\%20Recipe\%20Card\&region=16\%20of\%20Our\%20Most\%20Popular\%20Lemon\%20Desserts\&pgType=collection\&rank=11}{Lemon
Pound Cake}.

The following recipe is one that my family, which hails from
\href{https://www.nytimes3xbfgragh.onion/interactive/2015/12/18/travel/what-to-do-in-36-hours-in-palermo-sicily.html}{Palermo},
has perfected for years, and is based on different formulas and notes
taken by diligent relatives. Salute!

\hypertarget{ingredients}{%
\subsubsection{\texorpdfstring{\textbf{Ingredients}}{Ingredients}}\label{ingredients}}

\begin{itemize}
\item
  3 cups 100 proof vodka (the higher quality, the smoother your 'cello
  will be)
\item
  1 cup 190 proof Everclear
\item
  12-18 lemons (the more lemons you use, the stronger your lemon flavor
  will be)
\item
  2 cups white sugar

  \textbf{Yield} About two and a half 750 ml bottles.
\end{itemize}

\hypertarget{steps}{%
\subsubsection{\texorpdfstring{\textbf{Steps}}{Steps}}\label{steps}}

\textbf{Get the most from your lemons}

1. Pour the vodka and Everclear into a large mason jar.

2. Using a very sharp vegetable peeler, remove the peels of your lemons.
Take care to avoid any pith on your peels. The thinner your peels, the
better. (You can remove any piths from your peels using a small spoon or
paring knife.)

3. Add the peels to the alcohol mixture, and make sure they are fully
submerged. Seal and allow to sit in a cool place for at least five days
and up to a month. The longer you let this mixture sit, the stronger the
lemon flavor will be.

\textbf{Making the simple syrup}

1. In a saucepan over medium heat, simmer equal parts sugar and water (2
cups each).

2. Allow the mixture to simmer for an additional five to eight minutes
and let cool.

3. Seal in a glass container and store.

\textbf{Straining your solution}

1. Once your alcohol and lemon peel mixture has sat for at least five
days, remove the lemon peels by straining them through cheesecloth and a
conical sieve, and save the solution.

2. Keep the lemon peels in the cheesecloth and use your hands to squeeze
any excess liquid from the peels.

3. Repeat the straining process until the solution is clear and all
particles are removed.

\textbf{Getting it just right}

1. In small increments, add the simple syrup to your alcoholic mixture.
Mix well before adding more simple syrup and taste. (Add too much sugar
and you'll dilute the citrus taste; add too little and the lemon will be
overpowering.)

2. If the alcohol level is too overwhelming, add a small amount of water
(¼ cup) and mix well. Taste after each adjustment. Once your spirit
tastes to your liking, store in the fridge overnight and taste again.

3. If you're satisfied with your limoncello, seal and store in the
freezer for at least one week before serving.

Advertisement

\protect\hyperlink{after-bottom}{Continue reading the main story}

\hypertarget{site-index}{%
\subsection{Site Index}\label{site-index}}

\hypertarget{site-information-navigation}{%
\subsection{Site Information
Navigation}\label{site-information-navigation}}

\begin{itemize}
\tightlist
\item
  \href{https://help.nytimes3xbfgragh.onion/hc/en-us/articles/115014792127-Copyright-notice}{©~2020~The
  New York Times Company}
\end{itemize}

\begin{itemize}
\tightlist
\item
  \href{https://www.nytco.com/}{NYTCo}
\item
  \href{https://help.nytimes3xbfgragh.onion/hc/en-us/articles/115015385887-Contact-Us}{Contact
  Us}
\item
  \href{https://www.nytco.com/careers/}{Work with us}
\item
  \href{https://nytmediakit.com/}{Advertise}
\item
  \href{http://www.tbrandstudio.com/}{T Brand Studio}
\item
  \href{https://www.nytimes3xbfgragh.onion/privacy/cookie-policy\#how-do-i-manage-trackers}{Your
  Ad Choices}
\item
  \href{https://www.nytimes3xbfgragh.onion/privacy}{Privacy}
\item
  \href{https://help.nytimes3xbfgragh.onion/hc/en-us/articles/115014893428-Terms-of-service}{Terms
  of Service}
\item
  \href{https://help.nytimes3xbfgragh.onion/hc/en-us/articles/115014893968-Terms-of-sale}{Terms
  of Sale}
\item
  \href{https://spiderbites.nytimes3xbfgragh.onion}{Site Map}
\item
  \href{https://help.nytimes3xbfgragh.onion/hc/en-us}{Help}
\item
  \href{https://www.nytimes3xbfgragh.onion/subscription?campaignId=37WXW}{Subscriptions}
\end{itemize}
