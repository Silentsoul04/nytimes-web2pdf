Sections

SEARCH

\protect\hyperlink{site-content}{Skip to
content}\protect\hyperlink{site-index}{Skip to site index}

\href{https://www.nytimes3xbfgragh.onion/section/politics}{Politics}

\href{https://myaccount.nytimes3xbfgragh.onion/auth/login?response_type=cookie\&client_id=vi}{}

\href{https://www.nytimes3xbfgragh.onion/section/todayspaper}{Today's
Paper}

\href{/section/politics}{Politics}\textbar{}Border Wall Land Grabs
Accelerate as Owners Shelter From Pandemic

\url{https://nyti.ms/2XIXmCu}

\begin{itemize}
\item
\item
\item
\item
\item
\item
\end{itemize}

\hypertarget{the-coronavirus-outbreak}{%
\subsubsection{\texorpdfstring{\href{https://www.nytimes3xbfgragh.onion/news-event/coronavirus?name=styln-coronavirus-national\&region=TOP_BANNER\&block=storyline_menu_recirc\&action=click\&pgtype=Article\&impression_id=79168eb0-f4b9-11ea-b51e-55332169c079\&variant=undefined}{The
Coronavirus
Outbreak}}{The Coronavirus Outbreak}}\label{the-coronavirus-outbreak}}

\begin{itemize}
\tightlist
\item
  live\href{https://www.nytimes3xbfgragh.onion/2020/09/11/world/covid-19-coronavirus.html?name=styln-coronavirus-national\&region=TOP_BANNER\&block=storyline_menu_recirc\&action=click\&pgtype=Article\&impression_id=79168eb1-f4b9-11ea-b51e-55332169c079\&variant=undefined}{Latest
  Updates}
\item
  \href{https://www.nytimes3xbfgragh.onion/interactive/2020/us/coronavirus-us-cases.html?name=styln-coronavirus-national\&region=TOP_BANNER\&block=storyline_menu_recirc\&action=click\&pgtype=Article\&impression_id=79168eb2-f4b9-11ea-b51e-55332169c079\&variant=undefined}{Maps
  and Cases}
\item
  \href{https://www.nytimes3xbfgragh.onion/interactive/2020/science/coronavirus-vaccine-tracker.html?name=styln-coronavirus-national\&region=TOP_BANNER\&block=storyline_menu_recirc\&action=click\&pgtype=Article\&impression_id=7916b5c0-f4b9-11ea-b51e-55332169c079\&variant=undefined}{Vaccine
  Tracker}
\item
  \href{https://www.nytimes3xbfgragh.onion/2020/09/10/us/politics/fda-coronavirus-vaccine.html?name=styln-coronavirus-national\&region=TOP_BANNER\&block=storyline_menu_recirc\&action=click\&pgtype=Article\&impression_id=7916b5c1-f4b9-11ea-b51e-55332169c079\&variant=undefined}{F.D.A.
  Regulators' Self-Defense}
\item
  \href{https://www.nytimes3xbfgragh.onion/2020/09/09/upshot/coronavirus-surprise-test-fees.html?name=styln-coronavirus-national\&region=TOP_BANNER\&block=storyline_menu_recirc\&action=click\&pgtype=Article\&impression_id=7916b5c2-f4b9-11ea-b51e-55332169c079\&variant=undefined}{Surprise
  Test Fees}
\end{itemize}

Advertisement

\protect\hyperlink{after-top}{Continue reading the main story}

Supported by

\protect\hyperlink{after-sponsor}{Continue reading the main story}

\hypertarget{border-wall-land-grabs-accelerate-as-owners-shelter-from-pandemic}{%
\section{Border Wall Land Grabs Accelerate as Owners Shelter From
Pandemic}\label{border-wall-land-grabs-accelerate-as-owners-shelter-from-pandemic}}

With private property proving hard to acquire, the administration has
stepped up efforts to secure land on the Mexican border for President
Trump's wall.

\includegraphics{https://static01.graylady3jvrrxbe.onion/images/2020/05/28/us/politics/28dc-wall1/merlin_164028420_0a517eaa-112d-41de-85ac-e39666870263-articleLarge.jpg?quality=75\&auto=webp\&disable=upscale}

\href{https://www.nytimes3xbfgragh.onion/by/zolan-kanno-youngs}{\includegraphics{https://static01.graylady3jvrrxbe.onion/images/2019/12/13/reader-center/author-zolan-kanno-youngs/author-zolan-kanno-youngs-thumbLarge.png}}

By \href{https://www.nytimes3xbfgragh.onion/by/zolan-kanno-youngs}{Zolan
Kanno-Youngs}

\begin{itemize}
\item
  Published May 29, 2020Updated June 15, 2020
\item
  \begin{itemize}
  \item
  \item
  \item
  \item
  \item
  \item
  \end{itemize}
\end{itemize}

WASHINGTON --- The Trump administration is accelerating efforts to seize
private property for
\href{https://www.nytimes3xbfgragh.onion/2020/03/31/us/coronavirus-border-wall-arizona.html}{President
Trump's border wall}, taking advantage of the
\href{https://www.nytimes3xbfgragh.onion/news-event/coronavirus}{coronavirus
pandemic} to survey land while its owners are confined indoors,
residents along the Rio Grande say.

``Is that essential business?'' asked Nayda Alvarez, 49, who recently
found construction markers on the land in Starr County, Texas, that has
been in her family for five generations. ``That didn't stop a single
minute during the shelter in place or stay at home.''

The federal government brought a flurry of
\href{https://www.nytimes3xbfgragh.onion/2019/12/26/us/politics/trump-border-wall.html}{lawsuits
against landowners in South Texas} to survey, seize and potentially
begin construction on private property in the first five months of the
year as the administration rushed to deliver on Mr. Trump's promise to
build 450 miles of wall by the end of the year, which he downgraded on
Thursday to 400. While Mr. Trump has built less than 200 of those miles,
his administration has brought 78 lawsuits against landowners on the
border, 30 of them this year.

Negotiations and lawsuits are proving to be arduous. The administration
has acquired just 10 of the 213 miles of private property that the
border wall is projected to pass through in the Laredo and Rio Grande
Valley sectors, according to Customs and Border Protection data from May
19 obtained by The Times, an increase of seven miles since December. In
recent months, the president's son-in-law, Jared Kushner, has stepped in
to oversee the effort.

The increased litigation against the landowners, despite the pandemic,
is evidence of the administration's sense of urgency to deliver on a
symbol of Mr. Trump's crackdown on immigration. The president has said
\href{https://www.nytimes3xbfgragh.onion/2020/02/29/us/politics/trump-rally-coronavirus.html}{the
pandemic is proof of the wall's necessity}, though there is no real
evidence it will have any effect on public health.

``Mexico is having a very, very hard time, as you know, with Covid,
especially along the border,'' Mr. Trump told reporters on Thursday,
though Mexico's 8,600 deaths and 78,000 infections are a fraction of the
toll in the United States. ``Fortunately,'' he added, ``we have a
brand-new wall along there, and the wall is saving us.''

The government filed 13 lawsuits in March alone to access and acquire
land, the highest single-month total since Mr. Trump took office,
according to the Texas Civil Rights Project.

\includegraphics{https://static01.graylady3jvrrxbe.onion/images/2020/05/28/us/politics/28dc-wall2/merlin_169303773_6c3fe21a-5453-44a8-a249-5b40d6389d63-articleLarge.jpg?quality=75\&auto=webp\&disable=upscale}

Some of the landowners sued have kept the properties in their families
for generations. But the Texans say the government's timing has left
them further disadvantaged in a process in which the administration
already has the law on its side. Landowners adhering to coronavirus
guidelines have been unable to meet with their relatives to discuss the
government's offers, to confer with lawyers on how to fight the
government or to consult appraisers on the accurate value of their land.
Some have questioned why the push to access their properties is coming
as the coronavirus spreads, and they try to avoid social contact.

``They want to do it all obviously prior to November'' and the election,
said Steven Kobernat, a 61-year-old landowner in Starr County who said
he felt hounded by the Department of Justice. ``But here we are in a
pandemic. We can't meet, we can't meet with our families. And then
D.O.J. says it's time-sensitive in a time of pandemic. It's just
absurd.''

\hypertarget{latest-updates-the-coronavirus-outbreak}{%
\section{\texorpdfstring{\href{https://www.nytimes3xbfgragh.onion/2020/09/11/world/covid-19-coronavirus.html?action=click\&pgtype=Article\&state=default\&region=MAIN_CONTENT_1\&context=storylines_live_updates}{Latest
Updates: The Coronavirus
Outbreak}}{Latest Updates: The Coronavirus Outbreak}}\label{latest-updates-the-coronavirus-outbreak}}

Updated 2020-09-12T05:29:13.829Z

\begin{itemize}
\tightlist
\item
  \href{https://www.nytimes3xbfgragh.onion/2020/09/11/world/covid-19-coronavirus.html?action=click\&pgtype=Article\&state=default\&region=MAIN_CONTENT_1\&context=storylines_live_updates\#link-dfb8a16}{Fauci
  cautions the virus could disrupt life in the U.S. until `maybe even
  towards the end of 2021.'}
\item
  \href{https://www.nytimes3xbfgragh.onion/2020/09/11/world/covid-19-coronavirus.html?action=click\&pgtype=Article\&state=default\&region=MAIN_CONTENT_1\&context=storylines_live_updates\#link-7104d154}{From
  Asia to Africa, China promotes its vaccine candidates to win friends.}
\item
  \href{https://www.nytimes3xbfgragh.onion/2020/09/11/world/covid-19-coronavirus.html?action=click\&pgtype=Article\&state=default\&region=MAIN_CONTENT_1\&context=storylines_live_updates\#link-393ad215}{The
  other way the virus will kill: hunger.}
\end{itemize}

\href{https://www.nytimes3xbfgragh.onion/2020/09/11/world/covid-19-coronavirus.html?action=click\&pgtype=Article\&state=default\&region=MAIN_CONTENT_1\&context=storylines_live_updates}{See
more updates}

More live coverage:
\href{https://www.nytimes3xbfgragh.onion/live/2020/09/11/business/stock-market-today-coronavirus?action=click\&pgtype=Article\&state=default\&region=MAIN_CONTENT_1\&context=storylines_live_updates}{Markets}

The Justice Department said in a statement, ``We are following all
local, state and federal Covid protocols for all phases of land
acquisition and court work.''

Raini Brunson, a spokeswoman for the Army Corps of Engineers, which is
leading construction, said the agency was committed to the safety of
employees, contractors and ``the people in communities in which we
work.'' The agency ``continues to execute its border barrier
infrastructure mission in order to safeguard national security
capabilities,'' she added.

Ms. Alvarez had just come back to Starr County in March when she noticed
something strange on her property: construction markers jammed into the
earth to measure elevation. Crews had come to her land while she was in
Washington to testify to Congress against the border wall.

Months before, Ms. Alvarez had encountered government contractors on her
property who claimed they had received permission from a relative to
survey the land. She refused them access, but her family has continued
to see construction crews driving around their land.

Mr. Kobernat feels similar pressure. When the Army Corps of Engineers
pressed him to accept an offer for seven acres of his family's farm, he
pleaded for time to allow the pandemic to ebb.

``There is a sudden mad rush to obtain our property by pushing us to
sign and sell immediately. But due to the extraordinary current pandemic
crisis, we simply need more time,'' Mr. Kobernat wrote in an April 27
email to Army Corps and Border Patrol officials. ``Our family is
presently unable to safely confer with each other or our attorney as we
need to --- due to my mayor, my governor and your boss's
shelter-in-place rules.''

Shortly after that, he began getting calls from the Justice Department
telling him to cooperate with the Army Corps or risk a lawsuit.

Lawyers and government officials agree that landowners already had few
options.

They can choose to voluntarily allow the government to access and survey
their land and, if the administration wants it, accept compensation that
is supposed to be based on fair market value. But if they refuse, they
are likely to be taken to court, where the government can use eminent
domain powers to argue that the wall is an emergency and eventually take
possession of their land. The government can then begin construction,
even while continuing to argue with the landowners over compensation.

Image

Private land near Donna, Texas, where the border wall is set to be
constructed.Credit...Ilana Panich-Linsman for The New York Times

Ricky Garza, a staff lawyer for the Texas Civil Rights Project, said the
timing of the government's push for private property had made what was
already an uphill battle for the landowners even more challenging.

``They've taken advantage of people sheltering in place. People have not
been able to seek out attorneys,'' Mr. Garza said. ``We haven't seen any
signs of it slowing down. The landowner is really at the mercy of what
the government is trying to do.''

Many of the property owners are still enlisting lawyers to negotiate.
Some hope they can delay the process beyond the election, when the
construction of the wall may not be as much of a priority.

But Mr. Trump is pressing forward as fast as possible. As the
coronavirus spread in March, he tweeted, ``We need the Wall more than
ever,'' despite a top health official saying he had not seen evidence
that physical barriers would
\href{https://www.politico.com/news/2020/03/10/cdc-director-border-wall-coronavirus-125007}{prevent
the spread of the virus}. The president's border agency recently started
\href{https://www.cbp.gov/border-security/along-us-borders/border-wall-system}{a
website} showcasing videos of the wall's construction, months after Mr.
Kushner and his allies pushed the Department of Homeland Security to
livestream the building of the project.

The administration has also waived federal contracting laws to speed
construction of the wall; 194 miles have been completed as of this week,
up from 93 in December. All but three of the miles are in areas where
dilapidated barriers existed or vehicle barriers once stood. The federal
government also recently gave a nearly \$1.3 billion contract to a North
Dakota company backed by Stephen K. Bannon to construct 42 miles of the
wall, despite the Office of Inspector General for the Department of
Defense
\href{https://www.nytimes3xbfgragh.onion/2019/12/12/us/politics/trump-border-wall-investigation.html}{examining
an earlier \$400 million contract} given to the company.

With recent funding transfers from the Department of Defense, the
administration now has \$15 billion to build 731 miles of border wall.
John B. Mennell, a Customs and Border Protection spokesman, pointed to
data that suggested the agency could build about 500 miles of wall on
federal border land, without private acquisitions. The agency has
apparently tried to lower expectations in recent months, removing
language from weekly border wall bulletins that said the administration
expected to have 450 miles completed by the end of the year.

By the Trump administration's own logic, private land in South Texas is
where the wall is most needed. The border agency recorded more than
34,000 illegal crossings in the Rio Grande Valley in fiscal year 2019,
the most of any border sector.

``It is the shortest land route from the Guatemala-Mexico border to the
U.S. It is an environment that is very difficult to enforce,'' said Gil
Kerlikowske, the Customs and Border Protection commissioner under
President Barack Obama. ``It should be the focal point.''

Image

Melissa Cigarroa's husband, Carlos, and daughter, Annalis, on their
property in Zapata County, Texas.Credit...via Melissa Cigarroa

Melissa Cigarroa, 53, said she had rarely thought of the border in
Zapata County, Texas, as dangerous. The land brings to mind skeet
shooting or watching
\href{https://www.britannica.com/animal/aoudad}{aoudads --- or Barbary
sheep} --- make their way through her family's 150-acre ranch.

So far, she has refused to sign documents allowing the government to
enter her property.

``Why are we going to be the guinea pigs?'' Ms. Cigarroa said. ``We're
such a little town on the border. It is ridiculous.''

Mr. Kobernat said he incorrectly thought that when he sold one acre to
the federal government a year ago to build a watchtower for Border
Patrol, he and his siblings might be spared from the border wall, which
could turn a portion of their farmland into a ``no man's land'' between
the wall and the Rio Grande.

He said he would not give the federal government an answer until he
could meet with his siblings, who share ownership of the property. All
of them are vulnerable to the coronavirus.

Advertisement

\protect\hyperlink{after-bottom}{Continue reading the main story}

\hypertarget{site-index}{%
\subsection{Site Index}\label{site-index}}

\hypertarget{site-information-navigation}{%
\subsection{Site Information
Navigation}\label{site-information-navigation}}

\begin{itemize}
\tightlist
\item
  \href{https://help.nytimes3xbfgragh.onion/hc/en-us/articles/115014792127-Copyright-notice}{©~2020~The
  New York Times Company}
\end{itemize}

\begin{itemize}
\tightlist
\item
  \href{https://www.nytco.com/}{NYTCo}
\item
  \href{https://help.nytimes3xbfgragh.onion/hc/en-us/articles/115015385887-Contact-Us}{Contact
  Us}
\item
  \href{https://www.nytco.com/careers/}{Work with us}
\item
  \href{https://nytmediakit.com/}{Advertise}
\item
  \href{http://www.tbrandstudio.com/}{T Brand Studio}
\item
  \href{https://www.nytimes3xbfgragh.onion/privacy/cookie-policy\#how-do-i-manage-trackers}{Your
  Ad Choices}
\item
  \href{https://www.nytimes3xbfgragh.onion/privacy}{Privacy}
\item
  \href{https://help.nytimes3xbfgragh.onion/hc/en-us/articles/115014893428-Terms-of-service}{Terms
  of Service}
\item
  \href{https://help.nytimes3xbfgragh.onion/hc/en-us/articles/115014893968-Terms-of-sale}{Terms
  of Sale}
\item
  \href{https://spiderbites.nytimes3xbfgragh.onion}{Site Map}
\item
  \href{https://help.nytimes3xbfgragh.onion/hc/en-us}{Help}
\item
  \href{https://www.nytimes3xbfgragh.onion/subscription?campaignId=37WXW}{Subscriptions}
\end{itemize}
