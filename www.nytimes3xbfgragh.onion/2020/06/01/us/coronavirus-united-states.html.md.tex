Sections

SEARCH

\protect\hyperlink{site-content}{Skip to
content}\protect\hyperlink{site-index}{Skip to site index}

\href{/section/us}{U.S.}\textbar{}Is America's Pandemic Waning or
Raging? Yes

\begin{itemize}
\item
\item
\item
\item
\item
\item
\end{itemize}

\hypertarget{the-coronavirus-outbreak}{%
\subsubsection{\texorpdfstring{\href{https://www.nytimes3xbfgragh.onion/news-event/coronavirus?name=styln-coronavirus-national\&region=TOP_BANNER\&block=storyline_menu_recirc\&action=click\&pgtype=Article\&impression_id=57902710-f4be-11ea-8f23-0589e0816c94\&variant=undefined}{The
Coronavirus
Outbreak}}{The Coronavirus Outbreak}}\label{the-coronavirus-outbreak}}

\begin{itemize}
\tightlist
\item
  live\href{https://www.nytimes3xbfgragh.onion/2020/09/11/world/covid-19-coronavirus.html?name=styln-coronavirus-national\&region=TOP_BANNER\&block=storyline_menu_recirc\&action=click\&pgtype=Article\&impression_id=57902711-f4be-11ea-8f23-0589e0816c94\&variant=undefined}{Latest
  Updates}
\item
  \href{https://www.nytimes3xbfgragh.onion/interactive/2020/us/coronavirus-us-cases.html?name=styln-coronavirus-national\&region=TOP_BANNER\&block=storyline_menu_recirc\&action=click\&pgtype=Article\&impression_id=57902712-f4be-11ea-8f23-0589e0816c94\&variant=undefined}{Maps
  and Cases}
\item
  \href{https://www.nytimes3xbfgragh.onion/interactive/2020/science/coronavirus-vaccine-tracker.html?name=styln-coronavirus-national\&region=TOP_BANNER\&block=storyline_menu_recirc\&action=click\&pgtype=Article\&impression_id=57904e20-f4be-11ea-8f23-0589e0816c94\&variant=undefined}{Vaccine
  Tracker}
\item
  \href{https://www.nytimes3xbfgragh.onion/2020/09/10/us/politics/fda-coronavirus-vaccine.html?name=styln-coronavirus-national\&region=TOP_BANNER\&block=storyline_menu_recirc\&action=click\&pgtype=Article\&impression_id=57904e21-f4be-11ea-8f23-0589e0816c94\&variant=undefined}{F.D.A.
  Regulators' Self-Defense}
\item
  \href{https://www.nytimes3xbfgragh.onion/2020/09/09/upshot/coronavirus-surprise-test-fees.html?name=styln-coronavirus-national\&region=TOP_BANNER\&block=storyline_menu_recirc\&action=click\&pgtype=Article\&impression_id=57904e22-f4be-11ea-8f23-0589e0816c94\&variant=undefined}{Surprise
  Test Fees}
\end{itemize}

\includegraphics{https://static01.graylady3jvrrxbe.onion/images/2020/05/29/us/00stateof-the-virus-chicago/merlin_172961754_cea3fb9e-13ce-4790-bc31-53b10695321a-articleLarge.jpg?quality=75\&auto=webp\&disable=upscale}

\hypertarget{is-americas-pandemic-waning-or-raging-yes}{%
\section{Is America's Pandemic Waning or Raging?
Yes}\label{is-americas-pandemic-waning-or-raging-yes}}

A single day reveals divergent realities across the country: As cases
drop in the Northeast and some cities reopen, other places report
stubbornly high numbers.

Chicago~has been shuttered since March, with little hope that life in
America's third-largest city will return to normal soon.Credit...Lyndon
French for The New York Times

Supported by

\protect\hyperlink{after-sponsor}{Continue reading the main story}

\href{https://www.nytimes3xbfgragh.onion/by/julie-bosman}{\includegraphics{https://static01.graylady3jvrrxbe.onion/images/2018/11/09/multimedia/author-julie-bosman/author-julie-bosman-thumbLarge.png}}\href{https://www.nytimes3xbfgragh.onion/by/mitch-smith}{\includegraphics{https://static01.graylady3jvrrxbe.onion/images/2018/09/10/multimedia/author-mitch-smith/author-mitch-smith-thumbLarge.png}}

By \href{https://www.nytimes3xbfgragh.onion/by/julie-bosman}{Julie
Bosman} and
\href{https://www.nytimes3xbfgragh.onion/by/mitch-smith}{Mitch Smith}

\begin{itemize}
\item
  Published June 1, 2020Updated June 5, 2020
\item
  \begin{itemize}
  \item
  \item
  \item
  \item
  \item
  \item
  \end{itemize}
\end{itemize}

CHICAGO --- In the weeks since America began reopening on a large scale,
the coronavirus has persisted on a stubborn but uneven path, with
meaningful progress in some cities and alarming new outbreaks in others.

A snapshot of the country on a single day last week revealed sharply
divergent realities. As the United States marked
\href{https://www.nytimes3xbfgragh.onion/interactive/2020/05/24/us/us-coronavirus-deaths-100000.html}{the
tragic milestone of 100,000 deaths} from the coronavirus on Wednesday,
the contrasting picture was unmistakable --- a murky, jumbled outlook
depending on one's location.

Around Chicago, Wednesday was one of the most lethal days of the
pandemic, with more than 100 deaths. Among the dead: a woman in her 30s,
and four men past their 90th birthdays.

In the Boston area, where an alarming crisis of a month ago has given
way to cautious optimism, businesses were reopening that day and new
cases numbered in the dozens, no longer the hundreds.

Around Rogers and Springdale in northwest Arkansas, which the virus had
barely touched in the pandemic's early weeks, poultry workers spent part
of Wednesday planning a protest as outbreaks in at least two plants were
driving a sudden surge in infection numbers.

The dizzying volatility from city to city and state to state could
continue indefinitely, with vastly different policy implications for
individual places and no single, unified course in sight.

Some states are seeing vast improvements. But as the pandemic
progresses, parts of the country may eventually need to reimpose
restrictions, Dr. Tom Inglesby, the director of the Center for Health
Security at Johns Hopkins University, said.

``The country is divided in terms of its overall trajectory,'' Dr.
Inglesby said. ``This virus is persistent. It hasn't changed.''

Understanding the coronavirus's spread depends on where in America one
is standing: New cases are on a small but steady decline over all, to
about 21,000 a day from more than 30,000 at its April peak, a somewhat
encouraging sign that the pandemic is waning in the United States.

\hypertarget{latest-updates-the-coronavirus-outbreak}{%
\section{\texorpdfstring{\href{https://www.nytimes3xbfgragh.onion/2020/09/11/world/covid-19-coronavirus.html?action=click\&pgtype=Article\&state=default\&region=MAIN_CONTENT_1\&context=storylines_live_updates}{Latest
Updates: The Coronavirus
Outbreak}}{Latest Updates: The Coronavirus Outbreak}}\label{latest-updates-the-coronavirus-outbreak}}

Updated 2020-09-12T05:29:13.829Z

\begin{itemize}
\tightlist
\item
  \href{https://www.nytimes3xbfgragh.onion/2020/09/11/world/covid-19-coronavirus.html?action=click\&pgtype=Article\&state=default\&region=MAIN_CONTENT_1\&context=storylines_live_updates\#link-dfb8a16}{Fauci
  cautions the virus could disrupt life in the U.S. until `maybe even
  towards the end of 2021.'}
\item
  \href{https://www.nytimes3xbfgragh.onion/2020/09/11/world/covid-19-coronavirus.html?action=click\&pgtype=Article\&state=default\&region=MAIN_CONTENT_1\&context=storylines_live_updates\#link-7104d154}{From
  Asia to Africa, China promotes its vaccine candidates to win friends.}
\item
  \href{https://www.nytimes3xbfgragh.onion/2020/09/11/world/covid-19-coronavirus.html?action=click\&pgtype=Article\&state=default\&region=MAIN_CONTENT_1\&context=storylines_live_updates\#link-393ad215}{The
  other way the virus will kill: hunger.}
\end{itemize}

\href{https://www.nytimes3xbfgragh.onion/2020/09/11/world/covid-19-coronavirus.html?action=click\&pgtype=Article\&state=default\&region=MAIN_CONTENT_1\&context=storylines_live_updates}{See
more updates}

More live coverage:
\href{https://www.nytimes3xbfgragh.onion/live/2020/09/11/business/stock-market-today-coronavirus?action=click\&pgtype=Article\&state=default\&region=MAIN_CONTENT_1\&context=storylines_live_updates}{Markets}

It does not feel that way, though, in Chicago, where new coronavirus
infections have remained steadily high. The city, America's
third-largest, has been shuttered since March, with little hope that
life will return to normal soon. On Wednesday, surrounding Cook County
added about 700 cases and about 100 deaths, its highest death toll in
two weeks.

Playgrounds have been wrapped protectively with yellow tape, sending
children away. Parks and beaches along Lake Michigan, a reliably popular
draw in the fleeting Chicago summer, are closed and under guard of the
police.

On Wednesday, there was an eerie calm at Marge's Still, one of the
oldest taverns in the city, which has remained open, offering carryout
dinner orders and drinks to go. Lisa Vakulin-Rose, a manager, arrived in
the afternoon as usual, but with no one to serve at the bar, she had no
fruit to slice for cocktails or glasses to polish. Instead, she opened
the windows and side door, let the summerlike breeze flow in and waited
for the phone to ring.

Regular customers have kept coming. They arrive and pick up their
dinners, their to-go cocktails, their bottles of wine. Some are becoming
impatient, asking: When are we going to be able to sit at this bar like
before?

``I'd love to be able to be in a normal state again,'' Ms. Vakulin-Rose
said. ``But we're not.''

Image

Marge's Still, one of the oldest taverns in the city, has remained open,
offering carryout dinner orders and drinks to go.Credit...Lyndon French
for The New York Times

Image

``I'd love to be able to be in a normal state again,'' said Lisa
Vakulin-Rose, a manager at Marge's Still. ``But we're
not.''Credit...Lyndon French for The New York Times

The Midwest is still troubled by persistent coronavirus outbreaks.
Hospitalizations from the virus are on the rise in Wisconsin, an
unnerving development after that state's Supreme Court
\href{https://www.nytimes3xbfgragh.onion/2020/05/13/us/coronavirus-wisconsin-supreme-court.html}{abruptly
overturned a stay-at-home order} in May. New cases are consistently high
in Minnesota, particularly around the Twin Cities, where health
officials have warned that escalating protests could increase the
infection risk.

Jan Malcolm, Minnesota's health commissioner, said in a statement that
``we are one of the communities most vulnerable to rapid increases in
the spread of the virus, given where we are in the course of the
epidemic.''

But in the Northeast, the outlook has seesawed in the other direction. A
glimpse of that region on the same day seemed hopeful.

In New York, the center of the outbreak in the United States, more than
1,000 deaths were announced on some of the worst April days. But that
number is now often below 100, and every region in the state has started
to reopen except New York City, which is expected to begin doing so on
June 8.

In New Jersey and Connecticut, case numbers have plunged considerably in
recent days. And in Massachusetts, Gov. Charlie Baker has given houses
of worship and many businesses permission to open again.

At the peak of the epidemic, Suffolk County, Mass., which includes
Boston, was reporting more than 300 confirmed coronavirus cases and 25
deaths on many days. On Wednesday, the county added 63 cases and six
deaths, a vast improvement from weeks ago.

The shift has left Bostonians wondering whether this means they are
ready to reopen.

That question was vexing Ray Hammond and Gloria White-Hammond, a husband
and wife who are co-pastors of the Bethel A.M.E. Church in Jamaica
Plain, a neighborhood in Boston.

Mr. Baker had already announced that churches could begin holding
in-person services again, with restrictions. But the pastors --- who are
also physicians --- were more surprised than elated by the decision.

They worried about the potential of church services to be
super-spreading events. They had also watched the pandemic take a
particularly devastating toll on black and Hispanic communities, and
their largely black congregation has had several dozen members fall ill.

Could they reopen safely? Or would they be putting their congregants and
everyone in the community at risk?

\includegraphics{https://static01.graylady3jvrrxbe.onion/images/2020/05/29/us/00stateofthevirus-fens/merlin_172953144_a781bdba-2bc7-4e48-9dfe-862373726ec5-articleLarge.jpg?quality=75\&auto=webp\&disable=upscale}

Image

The ticket windows were down at Fenway Park in Boston.Credit...Cassandra
Klos for The New York Times

On Wednesday, they said they had decided not to go ahead with reopening.

``I just don't think we have enough information to make that decision in
a way that I would feel --- I'm saying personally, I can't make that
decision for other people --- that I would feel meets the criteria for
love of neighbor,'' Pastor Hammond said.

His congregants have been uniformly supportive of the decision to move
slowly. Some people said they might be ready this summer, or in the
fall, or perhaps early next year.

Still others have told him they plan to attend church virtually until
there is a vaccine.

``Nary a soul has said, `We've got to get back right away,''' he said.

In the South, many states have been open for weeks, and officials there
are carefully monitoring the effects of lifting restrictions.

In some communities where the virus appeared to have been under control
only weeks ago, there are now small but fierce flare-ups. Rural pockets
of Alabama, Louisiana and Mississippi are struggling to control growing
outbreaks.

Image

Wade Ogle, owner of Block Street Records, at his store in Fayetteville,
Ark., on Wednesday. Mr. Ogle has a rope across the front door with rules
for entering the store.Credit...Beth Hall for The New York Times

Arkansas seemed to be on the rebound when May began. But as the month
wore on, any glimmer of recovery faded. By last week, daily reports of
new cases had spiked to near the highest levels since the epidemic
began. On Wednesday, the state added 97 new cases, down from previous
days. The progress did not last: More than 230 cases were announced both
Thursday and Friday.

Many of those new cases can be attributed to outbreaks in poultry
processing facilities, where employees work in close quarters with
little opportunity for social distancing.

In rural Yell County, the site of two poultry processing outbreaks,
cases grew tenfold over two weeks. In more densely populated northwest
Arkansas, home to the headquarters of Walmart and Tyson, the number of
known cases has more than tripled since the start of May, fueled in part
by outbreaks at poultry plants.

Magaly Licolli, a co-founder of Venceremos, an advocacy group for
Arkansas poultry workers, said employees at the plants had watched
nervously as food processing facilities in other states reported
outbreaks. Then their own co-workers, many of them immigrants, started
falling ill.

Image

A car is stopped for screening outside of Cargill's processing plant in
Springdale, Ark., on Wednesday.Credit...Beth Hall for The New York Times

``They are so terrified of going to work because they feel that they are
being led to slaughter,'' Ms. Licolli said. She added: ``It's a very
dark time for many of them. Many of them have pre-existing conditions.''

Given what happened at meatpacking plants elsewhere, Ms. Licolli said
the new spikes in northwest Arkansas seemed almost inevitable. On
Friday, state officials reported cases at a third poultry plant in the
region.

``We knew that we were going to get to this point,'' she said.

That lament was true across much of the South. By week's end, along the
virus's uneven path, cases were also rising in Virginia, North Carolina,
Alabama, Mississippi and South Carolina.

Julie Bosman reported from Chicago and Mitch Smith from Overland Park,
Kan. Kate Taylor contributed reporting from Cambridge, Mass.

Advertisement

\protect\hyperlink{after-bottom}{Continue reading the main story}

\hypertarget{site-index}{%
\subsection{Site Index}\label{site-index}}

\hypertarget{site-information-navigation}{%
\subsection{Site Information
Navigation}\label{site-information-navigation}}

\begin{itemize}
\tightlist
\item
  \href{https://help.nytimes3xbfgragh.onion/hc/en-us/articles/115014792127-Copyright-notice}{©~2020~The
  New York Times Company}
\end{itemize}

\begin{itemize}
\tightlist
\item
  \href{https://www.nytco.com/}{NYTCo}
\item
  \href{https://help.nytimes3xbfgragh.onion/hc/en-us/articles/115015385887-Contact-Us}{Contact
  Us}
\item
  \href{https://www.nytco.com/careers/}{Work with us}
\item
  \href{https://nytmediakit.com/}{Advertise}
\item
  \href{http://www.tbrandstudio.com/}{T Brand Studio}
\item
  \href{https://www.nytimes3xbfgragh.onion/privacy/cookie-policy\#how-do-i-manage-trackers}{Your
  Ad Choices}
\item
  \href{https://www.nytimes3xbfgragh.onion/privacy}{Privacy}
\item
  \href{https://help.nytimes3xbfgragh.onion/hc/en-us/articles/115014893428-Terms-of-service}{Terms
  of Service}
\item
  \href{https://help.nytimes3xbfgragh.onion/hc/en-us/articles/115014893968-Terms-of-sale}{Terms
  of Sale}
\item
  \href{https://spiderbites.nytimes3xbfgragh.onion}{Site Map}
\item
  \href{https://help.nytimes3xbfgragh.onion/hc/en-us}{Help}
\item
  \href{https://www.nytimes3xbfgragh.onion/subscription?campaignId=37WXW}{Subscriptions}
\end{itemize}
