Sections

SEARCH

\protect\hyperlink{site-content}{Skip to
content}\protect\hyperlink{site-index}{Skip to site index}

\href{https://www.nytimes3xbfgragh.onion/section/health}{Health}

\href{https://myaccount.nytimes3xbfgragh.onion/auth/login?response_type=cookie\&client_id=vi}{}

\href{https://www.nytimes3xbfgragh.onion/section/todayspaper}{Today's
Paper}

\href{/section/health}{Health}\textbar{}Scientists Question Medical Data
Used in Second Coronavirus Study

\url{https://nyti.ms/3gLi6lW}

\begin{itemize}
\item
\item
\item
\item
\item
\end{itemize}

\hypertarget{the-coronavirus-outbreak}{%
\subsubsection{\texorpdfstring{\href{https://www.nytimes3xbfgragh.onion/news-event/coronavirus?name=styln-coronavirus-national\&region=TOP_BANNER\&block=storyline_menu_recirc\&action=click\&pgtype=Article\&impression_id=bb6df570-f1e2-11ea-a641-3789e99198fa\&variant=undefined}{The
Coronavirus
Outbreak}}{The Coronavirus Outbreak}}\label{the-coronavirus-outbreak}}

\begin{itemize}
\tightlist
\item
  live\href{https://www.nytimes3xbfgragh.onion/2020/09/08/world/covid-19-coronavirus.html?name=styln-coronavirus-national\&region=TOP_BANNER\&block=storyline_menu_recirc\&action=click\&pgtype=Article\&impression_id=bb6df571-f1e2-11ea-a641-3789e99198fa\&variant=undefined}{Latest
  Updates}
\item
  \href{https://www.nytimes3xbfgragh.onion/interactive/2020/us/coronavirus-us-cases.html?name=styln-coronavirus-national\&region=TOP_BANNER\&block=storyline_menu_recirc\&action=click\&pgtype=Article\&impression_id=bb6df572-f1e2-11ea-a641-3789e99198fa\&variant=undefined}{Maps
  and Cases}
\item
  \href{https://www.nytimes3xbfgragh.onion/interactive/2020/science/coronavirus-vaccine-tracker.html?name=styln-coronavirus-national\&region=TOP_BANNER\&block=storyline_menu_recirc\&action=click\&pgtype=Article\&impression_id=bb6df573-f1e2-11ea-a641-3789e99198fa\&variant=undefined}{Vaccine
  Tracker}
\item
  \href{https://www.nytimes3xbfgragh.onion/2020/09/02/your-money/eviction-moratorium-covid.html?name=styln-coronavirus-national\&region=TOP_BANNER\&block=storyline_menu_recirc\&action=click\&pgtype=Article\&impression_id=bb6e1c80-f1e2-11ea-a641-3789e99198fa\&variant=undefined}{Eviction
  Moratorium}
\item
  \href{https://www.nytimes3xbfgragh.onion/interactive/2020/09/02/magazine/food-insecurity-hunger-us.html?name=styln-coronavirus-national\&region=TOP_BANNER\&block=storyline_menu_recirc\&action=click\&pgtype=Article\&impression_id=bb6e1c81-f1e2-11ea-a641-3789e99198fa\&variant=undefined}{American
  Hunger}
\end{itemize}

Advertisement

\protect\hyperlink{after-top}{Continue reading the main story}

Supported by

\protect\hyperlink{after-sponsor}{Continue reading the main story}

\hypertarget{scientists-question-medical-data-used-in-second-coronavirus-study}{%
\section{Scientists Question Medical Data Used in Second Coronavirus
Study}\label{scientists-question-medical-data-used-in-second-coronavirus-study}}

Medical records from a little-known company were used in two studies
published in major journals. The New England Journal of Medicine has
asked to see the data.

\includegraphics{https://static01.graylady3jvrrxbe.onion/images/2020/06/02/science/02VIRUS-STUDIES1/merlin_173036757_fe7f8f0e-a8c3-4109-b85f-828bc5724b4e-articleLarge.jpg?quality=75\&auto=webp\&disable=upscale}

\href{https://www.nytimes3xbfgragh.onion/by/roni-caryn-rabin}{\includegraphics{https://static01.graylady3jvrrxbe.onion/images/2018/02/20/multimedia/author-roni-caryn-rabin/author-roni-caryn-rabin-thumbLarge-v3.png}}

By \href{https://www.nytimes3xbfgragh.onion/by/roni-caryn-rabin}{Roni
Caryn Rabin}

\begin{itemize}
\item
  June 2, 2020
\item
  \begin{itemize}
  \item
  \item
  \item
  \item
  \item
  \end{itemize}
\end{itemize}

Since the outbreak began, researchers have rushed to publish new
findings about the coronavirus spreading swiftly through the world. On
Tuesday, for the second time in recent days, a group of clinicians and
researchers has questioned the data used in studies in two prominent
medical journals.

A group of scientists who raised questions last week about a study in
The Lancet about the use of antimalarial drugs in coronavirus patients
have now objected to another paper about blood pressure medicines in the
New England Journal of Medicine, which was published by some of the same
authors and relied on the same data registry.

Moments after their open letter was posted online Tuesday morning, the
editors of the N.E.J.M. posted an
\href{https://www.nejm.org/doi/full/10.1056/NEJMe2020822}{``expression
of concern''} about the paper, and said they had asked the paper's
authors to provide evidence that the data are reliable.

The Lancet followed later in the day with a statement about its own
concerns regarding the malarial drugs paper, saying that the editors
have commissioned an independent audit of the data.

Both of the studies relied on an analysis of patient outcomes from a
private database run by a company called Surgisphere, which says it has
granular information about nearly 100,000 Covid-19 patients from 1,200
hospitals and other health facilities on six continents. Many health
care data experts say they knew nothing about its existence until
recently.

Both papers were published in May within a few weeks of each other in
highly respected medical journals that subject studies to peer review
before publication. Both had considerable impact, halting clinical
trials of malaria drugs around the world and providing reassurance about
the risks of blood pressure medications taken by millions of patients.

But scientists have not seen the large data set that Surgisphere says it
has built, and questions about its provenance are rising in scientific
circles.

\hypertarget{latest-updates-the-coronavirus-outbreak}{%
\section{\texorpdfstring{\href{https://www.nytimes3xbfgragh.onion/2020/09/08/world/covid-19-coronavirus.html?action=click\&pgtype=Article\&state=default\&region=MAIN_CONTENT_1\&context=storylines_live_updates}{Latest
Updates: The Coronavirus
Outbreak}}{Latest Updates: The Coronavirus Outbreak}}\label{latest-updates-the-coronavirus-outbreak}}

Updated 2020-09-08T14:35:26.206Z

\begin{itemize}
\tightlist
\item
  \href{https://www.nytimes3xbfgragh.onion/2020/09/08/world/covid-19-coronavirus.html?action=click\&pgtype=Article\&state=default\&region=MAIN_CONTENT_1\&context=storylines_live_updates\#link-547feae1}{Senate
  Republicans plan to move forward with a scaled-back stimulus package.}
\item
  \href{https://www.nytimes3xbfgragh.onion/2020/09/08/world/covid-19-coronavirus.html?action=click\&pgtype=Article\&state=default\&region=MAIN_CONTENT_1\&context=storylines_live_updates\#link-679303d7}{Nine
  drugmakers pledge to thoroughly vet any coronavirus vaccine.}
\item
  \href{https://www.nytimes3xbfgragh.onion/2020/09/08/world/covid-19-coronavirus.html?action=click\&pgtype=Article\&state=default\&region=MAIN_CONTENT_1\&context=storylines_live_updates\#link-1c973131}{`The
  lockdown killed my father': Farmer suicides add to India's virus
  misery.}
\end{itemize}

\href{https://www.nytimes3xbfgragh.onion/2020/09/08/world/covid-19-coronavirus.html?action=click\&pgtype=Article\&state=default\&region=MAIN_CONTENT_1\&context=storylines_live_updates}{See
more updates}

More live coverage:
\href{https://www.nytimes3xbfgragh.onion/live/2020/09/08/business/stock-market-today-coronavirus?action=click\&pgtype=Article\&state=default\&region=MAIN_CONTENT_1\&context=storylines_live_updates}{Markets}

In the open letter to the authors of the N.E.J.M. paper and to the
journal's editor, Dr. Eric J. Rubin, more than 100 clinicians,
researchers and statisticians
\href{https://zenodo.org/record/3873178\#.XtZkDxNKjOS}{demanded more
detailed information about the patient data} that served as the basis of
the study, and called for independent validation of the work by a third
party.

The study was said to analyze 8,910 Covid-19 patients hospitalized
through mid-March at 169 medical centers in Asia, Europe and North
America. The authors concluded that cardiovascular disease increased
their risk of dying.

But the paper also appeared to put to rest any concerns that people with
high blood pressure might have about taking drugs called ACE inhibitors:
Some people had wondered whether the drugs were playing a role in
exacerbating the illness.

Instead, the patients taking these drugs were more likely to survive
than those who were not, the authors said. (Other studies have also
reported that
\href{https://www.nytimes3xbfgragh.onion/2020/05/01/health/blood-pressure-drugs-coronavirus.html}{blood
pressure drugs do not make people more susceptible} to infection with
the coronavirus, and do not increase the risk of more severe illness.)

In the paper published in The Lancet, the authors said they had analyzed
data gathered from 671 hospitals on six continents that shared granular
medical information about nearly 15,000 patients who had received the
antimalarial drugs and 81,000 who had not, while shielding their
identities.

The papers concluded that use of chloroquine and hydroxychloroquine may
have increased the risk of death in these patients.

The first author on both of the papers is Dr. Mandeep R. Mehra, a
cardiovascular specialist and professor at Harvard Medical School. The
second author is Dr. Sapan S. Desai, the owner and founder of
Surgisphere.

\includegraphics{https://static01.graylady3jvrrxbe.onion/images/2020/06/02/science/02VIRUS-STUDIES2/02VIRUS-STUDIES2-articleLarge.jpg?quality=75\&auto=webp\&disable=upscale}

On Tuesday morning, Dr. Desai, who has vigorously defended both the
studies and his database, said he and his co-authors on The Lancet study
have agreed to a voluntary third-party audit done in collaboration with
the journal.

He also said he was arranging the terms of a nondisclosure agreement
that would allow the editors of the N.E.J.M. to see the data they had
requested.

Dr. Desai had previously said that his contractual agreements with
hospitals prevented him from disclosing any hospital-level patient data,
even though it was anonymized. ``Surgisphere stands behind the integrity
of our studies and our scientific researchers, clinical partners and
data analysts,'' he said in a statement.

In their letter to the N.E.J.M., critics of the work wrote: ``Serious,
and as yet unanswered, concerns have been raised about the integrity and
provenance of these data.''

\href{https://www.nytimes3xbfgragh.onion/news-event/coronavirus?action=click\&pgtype=Article\&state=default\&region=MAIN_CONTENT_3\&context=storylines_faq}{}

\hypertarget{the-coronavirus-outbreak-}{%
\subsubsection{The Coronavirus Outbreak
›}\label{the-coronavirus-outbreak-}}

\hypertarget{frequently-asked-questions}{%
\paragraph{Frequently Asked
Questions}\label{frequently-asked-questions}}

Updated September 4, 2020

\begin{itemize}
\item ~
  \hypertarget{what-are-the-symptoms-of-coronavirus}{%
  \paragraph{What are the symptoms of
  coronavirus?}\label{what-are-the-symptoms-of-coronavirus}}

  \begin{itemize}
  \tightlist
  \item
    In the beginning, the coronavirus
    \href{https://www.nytimes3xbfgragh.onion/article/coronavirus-facts-history.html?action=click\&pgtype=Article\&state=default\&region=MAIN_CONTENT_3\&context=storylines_faq\#link-6817bab5}{seemed
    like it was primarily a respiratory illness}~--- many patients had
    fever and chills, were weak and tired, and coughed a lot, though
    some people don't show many symptoms at all. Those who seemed
    sickest had pneumonia or acute respiratory distress syndrome and
    received supplemental oxygen. By now, doctors have identified many
    more symptoms and syndromes. In April,
    \href{https://www.nytimes3xbfgragh.onion/2020/04/27/health/coronavirus-symptoms-cdc.html?action=click\&pgtype=Article\&state=default\&region=MAIN_CONTENT_3\&context=storylines_faq}{the
    C.D.C. added to the list of early signs}~sore throat, fever, chills
    and muscle aches. Gastrointestinal upset, such as diarrhea and
    nausea, has also been observed. Another telltale sign of infection
    may be a sudden, profound diminution of one's
    \href{https://www.nytimes3xbfgragh.onion/2020/03/22/health/coronavirus-symptoms-smell-taste.html?action=click\&pgtype=Article\&state=default\&region=MAIN_CONTENT_3\&context=storylines_faq}{sense
    of smell and taste.}~Teenagers and young adults in some cases have
    developed painful red and purple lesions on their fingers and toes
    --- nicknamed ``Covid toe'' --- but few other serious symptoms.
  \end{itemize}
\item ~
  \hypertarget{why-is-it-safer-to-spend-time-together-outside}{%
  \paragraph{Why is it safer to spend time together
  outside?}\label{why-is-it-safer-to-spend-time-together-outside}}

  \begin{itemize}
  \tightlist
  \item
    \href{https://www.nytimes3xbfgragh.onion/2020/05/15/us/coronavirus-what-to-do-outside.html?action=click\&pgtype=Article\&state=default\&region=MAIN_CONTENT_3\&context=storylines_faq}{Outdoor
    gatherings}~lower risk because wind disperses viral droplets, and
    sunlight can kill some of the virus. Open spaces prevent the virus
    from building up in concentrated amounts and being inhaled, which
    can happen when infected people exhale in a confined space for long
    stretches of time, said Dr. Julian W. Tang, a virologist at the
    University of Leicester.
  \end{itemize}
\item ~
  \hypertarget{why-does-standing-six-feet-away-from-others-help}{%
  \paragraph{Why does standing six feet away from others
  help?}\label{why-does-standing-six-feet-away-from-others-help}}

  \begin{itemize}
  \tightlist
  \item
    The coronavirus spreads primarily through droplets from your mouth
    and nose, especially when you cough or sneeze. The C.D.C., one of
    the organizations using that measure,
    \href{https://www.nytimes3xbfgragh.onion/2020/04/14/health/coronavirus-six-feet.html?action=click\&pgtype=Article\&state=default\&region=MAIN_CONTENT_3\&context=storylines_faq}{bases
    its recommendation of six feet}~on the idea that most large droplets
    that people expel when they cough or sneeze will fall to the ground
    within six feet. But six feet has never been a magic number that
    guarantees complete protection. Sneezes, for instance, can launch
    droplets a lot farther than six feet,
    \href{https://jamanetwork.com/journals/jama/fullarticle/2763852}{according
    to a recent study}. It's a rule of thumb: You should be safest
    standing six feet apart outside, especially when it's windy. But
    keep a mask on at all times, even when you think you're far enough
    apart.
  \end{itemize}
\item ~
  \hypertarget{i-have-antibodies-am-i-now-immune}{%
  \paragraph{I have antibodies. Am I now
  immune?}\label{i-have-antibodies-am-i-now-immune}}

  \begin{itemize}
  \tightlist
  \item
    As of right
    now,\href{https://www.nytimes3xbfgragh.onion/2020/07/22/health/covid-antibodies-herd-immunity.html?action=click\&pgtype=Article\&state=default\&region=MAIN_CONTENT_3\&context=storylines_faq}{~that
    seems likely, for at least several months.}~There have been
    frightening accounts of people suffering what seems to be a second
    bout of Covid-19. But experts say these patients may have a
    drawn-out course of infection, with the virus taking a slow toll
    weeks to months after initial exposure.~People infected with the
    coronavirus typically
    \href{https://www.nature.com/articles/s41586-020-2456-9}{produce}~immune
    molecules called antibodies, which are
    \href{https://www.nytimes3xbfgragh.onion/2020/05/07/health/coronavirus-antibody-prevalence.html?action=click\&pgtype=Article\&state=default\&region=MAIN_CONTENT_3\&context=storylines_faq}{protective
    proteins made in response to an
    infection}\href{https://www.nytimes3xbfgragh.onion/2020/05/07/health/coronavirus-antibody-prevalence.html?action=click\&pgtype=Article\&state=default\&region=MAIN_CONTENT_3\&context=storylines_faq}{.
    These antibodies may}~last in the body
    \href{https://www.nature.com/articles/s41591-020-0965-6}{only two to
    three months}, which may seem worrisome, but that's~perfectly normal
    after an acute infection subsides, said Dr. Michael Mina, an
    immunologist at Harvard University. It may be possible to get the
    coronavirus again, but it's highly unlikely that it would be
    possible in a short window of time from initial infection or make
    people sicker the second time.
  \end{itemize}
\item ~
  \hypertarget{what-are-my-rights-if-i-am-worried-about-going-back-to-work}{%
  \paragraph{What are my rights if I am worried about going back to
  work?}\label{what-are-my-rights-if-i-am-worried-about-going-back-to-work}}

  \begin{itemize}
  \tightlist
  \item
    Employers have to provide
    \href{https://www.osha.gov/SLTC/covid-19/standards.html}{a safe
    workplace}~with policies that protect everyone equally.
    \href{https://www.nytimes3xbfgragh.onion/article/coronavirus-money-unemployment.html?action=click\&pgtype=Article\&state=default\&region=MAIN_CONTENT_3\&context=storylines_faq}{And
    if one of your co-workers tests positive for the coronavirus, the
    C.D.C.}~has said that
    \href{https://www.cdc.gov/coronavirus/2019-ncov/community/guidance-business-response.html}{employers
    should tell their employees}~-\/- without giving you the sick
    employee's name -\/- that they may have been exposed to the virus.
  \end{itemize}
\end{itemize}

The letter points out ``major inconsistencies'' between the number of
coronavirus cases recorded in some countries during the study period and
the number of patient outcomes reported by the researchers over the same
period.

In particular, they said, it is ``difficult to reconcile'' the
Surgisphere data from the United Kingdom with government reports. The
paper reported on 706 patients hospitalized with confirmed Covid-19 in
just seven of the U.K.'s 1,257 National Health Service hospitals.

Yet a high proportion of coronavirus patients hospitalized in the U.K.
early on were in London, and no London borough or hospital had more than
100 confirmed cases by March 16, the critics said.

``The numbers from Turkey also appear incorrect,'' the letter says,
adding that the first Covid-19 case in Turkey was diagnosed at Istanbul
Faculty of Medicine on March 9, and the hospital did not see another
case until March 16.

By March 18, the Turkish Ministry of Health reported a total of 191
confirmed cases, yet Surgisphere reported data on 346 Covid-19 patients
admitted by March 15 to just three Turkish hospitals.

\textbf{\emph{{[}}\href{http://on.fb.me/1paTQ1h}{\emph{Like the Science
Times page on Facebook.}}} ****** \emph{\textbar{} Sign up for the}
\textbf{\href{http://nyti.ms/1MbHaRU}{\emph{Science Times
newsletter.}}\emph{{]}}}

Many of the scientists who first raised concerns about the database are
involved in clinical trials of chloroquine and hydroxychloroquine, and
they were forced to pause the studies for safety reviews after The
Lancet study was published.

James Watson, a senior scientist with MORU Tropical Health Network, said
his unit had to immediately suspend work on a large randomized clinical
trial to see if chloroquine or hydroxychloroquine can protect health
care workers exposed on the job to the coronavirus from infection.

``I saw very quickly this paper didn't hold up to much scrutiny at
all,'' he said. ``We started wondering, `Who's been collecting this
data, and where did it come from?' We were quite surprised to see a
global study with only four authors listed and no acknowledgment of
anyone else.''

The scientists then turned their attention to the paper about
cardiovascular disease and blood pressure drugs that had been published
in the N.E.J.M. on May 1. ``We immediately thought, `If there's
something wrong with the database, it's going to affect both
publications,''' he said.

David Glidden, a professor of biostatistics at University of California,
San Francisco, who reads all new publications about Covid-19 antiviral
therapies as a member of a National Institutes of Health clinical
guidelines panel, said he was immediately struck by the vagueness of the
descriptions in both papers.

There is a frenzy to publish research, he added: ``Medical journals
often feel pressure to be relevant and to be carrying the story that's
going to be talked about, and I think they need to be responsive to the
urgency of this pandemic but also to maintain their standards, which
require caution.''

Advertisement

\protect\hyperlink{after-bottom}{Continue reading the main story}

\hypertarget{site-index}{%
\subsection{Site Index}\label{site-index}}

\hypertarget{site-information-navigation}{%
\subsection{Site Information
Navigation}\label{site-information-navigation}}

\begin{itemize}
\tightlist
\item
  \href{https://help.nytimes3xbfgragh.onion/hc/en-us/articles/115014792127-Copyright-notice}{©~2020~The
  New York Times Company}
\end{itemize}

\begin{itemize}
\tightlist
\item
  \href{https://www.nytco.com/}{NYTCo}
\item
  \href{https://help.nytimes3xbfgragh.onion/hc/en-us/articles/115015385887-Contact-Us}{Contact
  Us}
\item
  \href{https://www.nytco.com/careers/}{Work with us}
\item
  \href{https://nytmediakit.com/}{Advertise}
\item
  \href{http://www.tbrandstudio.com/}{T Brand Studio}
\item
  \href{https://www.nytimes3xbfgragh.onion/privacy/cookie-policy\#how-do-i-manage-trackers}{Your
  Ad Choices}
\item
  \href{https://www.nytimes3xbfgragh.onion/privacy}{Privacy}
\item
  \href{https://help.nytimes3xbfgragh.onion/hc/en-us/articles/115014893428-Terms-of-service}{Terms
  of Service}
\item
  \href{https://help.nytimes3xbfgragh.onion/hc/en-us/articles/115014893968-Terms-of-sale}{Terms
  of Sale}
\item
  \href{https://spiderbites.nytimes3xbfgragh.onion}{Site Map}
\item
  \href{https://help.nytimes3xbfgragh.onion/hc/en-us}{Help}
\item
  \href{https://www.nytimes3xbfgragh.onion/subscription?campaignId=37WXW}{Subscriptions}
\end{itemize}
