Sections

SEARCH

\protect\hyperlink{site-content}{Skip to
content}\protect\hyperlink{site-index}{Skip to site index}

\href{https://www.nytimes3xbfgragh.onion/section/politics}{Politics}

\href{https://myaccount.nytimes3xbfgragh.onion/auth/login?response_type=cookie\&client_id=vi}{}

\href{https://www.nytimes3xbfgragh.onion/section/todayspaper}{Today's
Paper}

\href{/section/politics}{Politics}\textbar{}Hey @jack, Here Are More
Questionable Tweets From @realdonaldtrump

\url{https://nyti.ms/3crH7zm}

\begin{itemize}
\item
\item
\item
\item
\item
\end{itemize}

\hypertarget{race-and-america}{%
\subsubsection{\texorpdfstring{\href{https://www.nytimes3xbfgragh.onion/news-event/george-floyd-protests-minneapolis-new-york-los-angeles?name=styln-george-floyd\&region=TOP_BANNER\&block=storyline_menu_recirc\&action=click\&pgtype=Article\&impression_id=6f933830-f2ab-11ea-b152-11fbf9e92c26\&variant=undefined}{Race
and America}}{Race and America}}\label{race-and-america}}

\begin{itemize}
\tightlist
\item
  \href{https://www.nytimes3xbfgragh.onion/2020/09/04/nyregion/rochester-police-daniel-prude.html?name=styln-george-floyd\&region=TOP_BANNER\&block=storyline_menu_recirc\&action=click\&pgtype=Article\&impression_id=6f935f40-f2ab-11ea-b152-11fbf9e92c26\&variant=undefined}{What
  Happened in Rochester, N.Y.}
\item
  \href{https://www.nytimes3xbfgragh.onion/2020/09/01/us/politics/trump-fact-check-protests.html?name=styln-george-floyd\&region=TOP_BANNER\&block=storyline_menu_recirc\&action=click\&pgtype=Article\&impression_id=6f935f41-f2ab-11ea-b152-11fbf9e92c26\&variant=undefined}{Trump
  Fact Check}
\item
  \href{https://www.nytimes3xbfgragh.onion/2020/08/30/us/portland-shooting-explained.html?name=styln-george-floyd\&region=TOP_BANNER\&block=storyline_menu_recirc\&action=click\&pgtype=Article\&impression_id=6f935f42-f2ab-11ea-b152-11fbf9e92c26\&variant=undefined}{Portland
  Shooting}
\item
  \href{https://www.nytimes3xbfgragh.onion/2020/08/30/us/breonna-taylor-police-killing.html?name=styln-george-floyd\&region=TOP_BANNER\&block=storyline_menu_recirc\&action=click\&pgtype=Article\&impression_id=6f935f43-f2ab-11ea-b152-11fbf9e92c26\&variant=undefined}{Breonna
  Taylor's Life and Death}
\end{itemize}

Advertisement

\protect\hyperlink{after-top}{Continue reading the main story}

Supported by

\protect\hyperlink{after-sponsor}{Continue reading the main story}

Fact Check

\hypertarget{hey-jack-here-are-more-questionable-tweets-from-realdonaldtrump}{%
\section{Hey @jack, Here Are More Questionable Tweets From
@realdonaldtrump}\label{hey-jack-here-are-more-questionable-tweets-from-realdonaldtrump}}

We analyzed the president's Twitter feed for a week. A third of his
posts contained falsehoods or murky accusations, underscoring the
challenge to Twitter's chief, Jack Dorsey, of policing him.

\includegraphics{https://static01.graylady3jvrrxbe.onion/images/2020/06/03/business/03dc-factcheck/03dc-factcheck-articleLarge.jpg?quality=75\&auto=webp\&disable=upscale}

\href{https://www.nytimes3xbfgragh.onion/by/linda-qiu}{\includegraphics{https://static01.graylady3jvrrxbe.onion/images/2018/06/12/multimedia/author-linda-qiu/author-linda-qiu-thumbLarge.png}}

By \href{https://www.nytimes3xbfgragh.onion/by/linda-qiu}{Linda Qiu}

\begin{itemize}
\item
  Published June 3, 2020Updated June 10, 2020
\item
  \begin{itemize}
  \item
  \item
  \item
  \item
  \item
  \end{itemize}
\end{itemize}

\href{https://www.nytimes3xbfgragh.onion/2020/06/09/us/politics/trump-twitter.html}{Twitter}
and its chief executive, Jack Dorsey, placed warnings on three of
\href{https://www.nytimes3xbfgragh.onion/2020/06/09/us/politics/trump-twitter.html}{President
Trump's} tweets last week, taking a measured but hotly debated step to
place some limit on the president's use of social media to spread
falsehoods and incite his followers.

Twitter attached labels refuting two of Mr. Trump's tweets on
\href{https://www.nytimes3xbfgragh.onion/2020/05/26/technology/twitter-trump-mail-in-ballots.html}{voter
fraud} and
\href{https://www.nytimes3xbfgragh.onion/2020/05/29/technology/trump-twitter-minneapolis-george-floyd.html}{restricted
one}that implied protesters in Minneapolis could be shot. But it left
countless others unchallenged, including those
\href{https://www.nytimes3xbfgragh.onion/2020/05/26/us/politics/klausutis-letter-jack-dorsey.html}{baselessly}
insinuating that the MSNBC host Joe Scarborough killed a former staff
member.

A New York Times review of the president's 139 Twitter posts from
Sunday, May 24, to Saturday, May 30, found at least 26 contained clearly
false claims, including five about mail-in voting that were not flagged,
five promoting the false conspiracy theory about Mr. Scarborough and
three about Twitter itself. Another 24 were misleading, lacked context
or traded in innuendo. (This analysis did not include dozens of Mr.
Trump's retweets.)

To put it another way, more than a third of the president's tweets over
the course of a week contained dubious information. That presents a
challenge both to Twitter and to the millions of people who are exposed
to Mr. Trump on social media, especially now, with the nation facing the
triple challenge of a pandemic, economic dislocation and nationwide
protests over systemic racism.

\hypertarget{unsubstantiated-charges-of-fraud-in-mail-in-voting}{%
\subsection{Unsubstantiated charges of fraud in mail-in
voting}\label{unsubstantiated-charges-of-fraud-in-mail-in-voting}}

Twitter\href{https://www.nytimes3xbfgragh.onion/2020/05/26/technology/twitter-trump-mail-in-ballots.html}{attached
information} to refute two of Mr. Trump's posts about mail-in voting
that falsely claimed that California was sending ballots to ``anyone
living in the state no matter who they are or how they got there.''
State officials will mail ballots to registered voters only.

Mr. Dorsey
\href{https://twitter.com/jack/status/1265837143827468289}{said} that
those tweets on May 26 specifically violated the company's civic
integrity policy as they ``may mislead people into thinking they don't
need to register to get a ballot.''

Five other posts by Mr. Trump repeated his general falsehoods about
mail-in voting but did not specify any one state's election process or
ballot distribution plans --- and were not affixed with a label.

Two days before Twitter labeled the two tweets on May 26, the president
\href{https://twitter.com/realDonaldTrump/status/1264717545787006976}{warned}
\href{https://twitter.com/realDonaldTrump/status/1264558926021959680}{twice}
that the upcoming presidential election will be ``rigged'' through
mail-in ballots. After Twitter's actions, Mr. Trump continued to claim
that mail-in voting would lead to a
``\href{https://twitter.com/realDonaldTrump/status/1265601615261827072}{free
for all on cheating, forgery and the theft of Ballots},'' a
\href{https://twitter.com/realDonaldTrump/status/1266047584038256640}{''tainted''
election process},
``\href{https://twitter.com/realDonaldTrump/status/1266172570983940101}{MASSIVE
FRAUD AND ABUSE'' and ``THE END OF OUR GREAT REPUBLICAN PARTY.}''

There is
\href{https://www.nytimes3xbfgragh.onion/2020/05/20/us/politics/trump-mail-in-voting-absentee-ballots.html}{no
evidence} for any of these claims. Voter fraud in general is extremely
rare. While mail-in voting is less secure than in-person voting, fraud
incidence rates remain extremely low --- one study found an improper
voting rate of 0.004 percent. Studies have found little evidence that
mail-in voting and so-called no-excuse absentee voting benefit one
political party over another.

\hypertarget{inaccurate-claims-about-twitter}{%
\subsection{Inaccurate claims about
Twitter}\label{inaccurate-claims-about-twitter}}

Mr. Trump responded to Twitter's actions
\href{https://www.nytimes3xbfgragh.onion/2020/05/28/us/politics/trump-jack-dorsey.html}{by
issuing an executive order} that seeks to strip liability protection in
certain cases for companies like
\href{https://www.nytimes3xbfgragh.onion/2020/05/29/technology/trump-twitter.html}{Twitter},
Google and Facebook for the content on their sites. If carried out, the
order could lead to the companies facing legal liability for false and
defamatory statements posted on their sites.

The president also took to Twitter to castigate the platform in three
false posts.

He accused the company of
``\href{https://twitter.com/realDonaldTrump/status/1265427539008380928}{completely
stifling FREE SPEECH}.'' But this is a misreading of the First
Amendment, which prohibits Congress from
``\href{https://constitutioncenter.org/interactive-constitution/amendment/amendment-i}{abridging
the freedom of speech}.'' While the Supreme Court has held that this
applies to all government agencies, it does
\href{https://constitutioncenter.org/interactive-constitution/interpretation/amendment-i/interps/266}{not
apply} to private companies like Twitter.

In
\href{https://twitter.com/realDonaldTrump/status/1266326065833824257}{two}
\href{https://twitter.com/realDonaldTrump/status/1266346957611708417}{tweets}
on Friday morning, Mr. Trump complained that Twitter had not flagged
``China's propaganda.'' But the company had attached similar warning
labels two days earlier to
\href{https://twitter.com/zlj517/status/1238111898828066823?ref_src=twsrc\%5Etfw\%7Ctwcamp\%5Etweetembed\%7Ctwterm\%5E1238111898828066823\%7Ctwgr\%5E\&ref_url=https\%3A\%2F\%2Fwww.hindustantimes.com\%2Fworld-news\%2Ftwitter-flags-china-spokesman-s-tweet-on-covid-19\%2Fstory-FzULY6y660dBJ1coEl2khI.html}{tweets}
\href{https://twitter.com/zlj517/status/1238269193427906560?lang=en}{posted}
by Zhao Lijian, a spokesman for China's foreign ministry, that suggested
that the coronavirus originated in the United States.

\hypertarget{baseless-claims-about-ongoing-protests}{%
\subsection{Baseless claims about ongoing
protests}\label{baseless-claims-about-ongoing-protests}}

Mr. Trump has dismissed the nationwide protests over the killing of
George Floyd by suggesting they are not organic or legitimate.

In three tweets, he said without evidence that the protests across the
country and in front of the White House were
``\href{https://twitter.com/realDonaldTrump/status/1266724553620930561}{professionally}
\href{https://twitter.com/realDonaldTrump/status/1266711221836931072}{organized}''
and
``\href{https://twitter.com/realDonaldTrump/status/1266731291917062145}{have
nothing to do with George Floyd}.''

While it is impossible to know the motivation of every person
participating in these demonstrations, Mr. Trump's broad generalization
discounts the thousands who have taken to the streets specifically to
protest the killing of Mr. Floyd, a black man who died after being
handcuffed and pinned to the ground by a white police officer.

In another
\href{https://twitter.com/realDonaldTrump/status/1266800113260703744}{tweet}
on Saturday, Mr. Trump said that ``80\% of the RIOTERS in Minneapolis
last night were from OUT OF STATE.'' This was
\href{https://www.youtube.com/watch?v=jla3cyeuOI0}{an estimate first
offered} by Gov. Tim Walz of Minnesota, a Democrat. But after reporting
by
\href{https://www.twincities.com/2020/05/30/majority-of-those-arrested-in-connection-with-protests-riots-from-minnesota/}{local}
\href{https://www.kare11.com/article/news/investigations/kare-11-investigates-records-show-arrests-mostly-minnesotans-as-george-floyd-protests-riots-continue-minneapolis-st-paul/89-73f3e0e8-0664-41d5-8d3e-4467d04da7cb}{news
outlets} that suggested the opposite was true, Mr. Walz
\href{https://youtu.be/zkpOYn4J0nw?t=1387}{declined to repeat} the
estimate and said that more data was needed to properly characterize the
proportion of those arrested who were not local residents.

These theories echoed Mr. Trump's
\href{https://twitter.com/realDonaldTrump/status/1048196883464818688}{previous
claims} that those who protested the confirmation hearings of Justice
Brett M. Kavanaugh in 2018 were paid by George Soros, the billionaire
investor and Democratic donor.

Mr. Trump did not name the supposed organizers of the current protests.

\hypertarget{other-falsehoods}{%
\subsection{Other falsehoods}\label{other-falsehoods}}

Other inaccuracies from a week of Mr. Trump's tweets centered on
familiar foes and oft-repeated boasts.

Mr. Trump
\href{https://twitter.com/realDonaldTrump/status/1265301261903106054}{tweeted},
with no evidence, that Speaker Nancy Pelosi had complained that he had
moved too quickly in imposing some restrictions on travel from China in
response to the coronavirus. The Times was unable to find an instance of
Ms. Pelosi publicly addressing or criticizing that decision.

He repeated his claim that former Vice President Joseph R. Biden Jr.,
the presumptive Democratic presidential nominee, had
``\href{https://twitter.com/realDonaldTrump/status/1265009852516110336}{apologized}''
for opposing the policy. No record of an apology exists.

He
\href{https://twitter.com/realDonaldTrump/status/1265257673039249408}{falsely
accused}Representative Conor Lamb, Democrat of Pennsylvania, of breaking
his campaign promise to vote against Ms. Pelosi for speaker. Mr. Lamb,
who won his seat in 2018,
\href{http://clerk.house.gov/evs/2019/roll002.xml\#Kennedy}{voted} for
Representative Joseph P. Kennedy III, Democrat of Massachusetts, when
the House held its election for speaker last year.

He
\href{https://twitter.com/realDonaldTrump/status/1266711224391213056}{falsely
claimed} on Saturday that Washington's Democratic mayor, Muriel E.
Bowser, ``wouldn't let the D.C. Police get involved'' in monitoring
protests outside of the White House on Friday. But the Secret Service,
which Mr. Trump praised in the same tweet,
\href{https://www.secretservice.gov/data/press/releases/2020/20-MAY/Secret-Service-Statement-on-Pennsylvania-Avenue-Demonstrations.pdf}{said
in a statement} that it had made six arrests that night and ``the
Metropolitan Police Department and the U.S. Park Police were on the
scene.''

He again
\href{https://twitter.com/realDonaldTrump/status/1265834167272603649}{took
undue credit} for the Veterans Choice health care program. The program
was created in 2014, developed by Senators John McCain and Bernie
Sanders and signed by former President Barack Obama --- three of Mr.
Trump's political enemies.

Mr. Trump's declaration that ``it was me who shattered 100\% of the ISIS
Caliphate'' was also not true. About a
\href{https://news.ihsmarkit.com/prviewer/release_only/slug/aerospace-defense-security-islamic-state-territory-down-60-percent-and-revenue-down-80}{third}
to a
\href{https://web.archive.org/web/20171222214247/https://www.state.gov/r/pa/prs/ps/2017/12/276746.htm}{half}
of the territory formerly held by the Islamic State was regained under
Mr. Obama's administration, according to military and independent
estimates. And officials and experts had always
\href{https://www.nytimes3xbfgragh.onion/2017/10/17/us/politics/trump-islamic-state-raqqa-fact-check.html}{anticipated}
that the campaign, started in 2014, long before Mr. Trump took office,
would result in pushing the extremist group out of its self-declared
caliphate.

\hypertarget{half-truths-and-murky-accusations}{%
\subsection{Half truths and murky
accusations}\label{half-truths-and-murky-accusations}}

Three dozen tweets from the president occupied a factual gray zone. Some
were typical examples of political spin, neither completely true nor
totally wrong.

Twice, he claimed to have
\href{https://twitter.com/realDonaldTrump/status/1265301249630654467}{ban}ned
travel from China and to have done so
``\href{https://twitter.com/realDonaldTrump/status/1265633761024188417}{before
anybody thought necessary}'' to contain the spread of the coronavirus.
These were exaggerations. The restrictions did not amount to full ban.
They did not apply to American citizens or green card holders, and they
contained other exemptions. Numerous other countries had taken similar
actions before Mr. Trump did.

He misleadingly boasted of the United States having carried out 15
million coronavirus tests,
``\href{https://twitter.com/realDonaldTrump/status/1265613475809763328}{by
far the most in the World},'' and the number of cases and deaths
``\href{https://twitter.com/realDonaldTrump/status/1264564734860427265}{going
down all over the Country}.'' The raw number of tests, while accurate,
did not reflect that the United States continues to lag other countries
in testing per capita. Cases and deaths were decreasing across the
country as a whole, but not in some states.

Ten tweets were devoted to the announcement of grants to local transit
agencies from the Department of Transportation. Left unsaid was that
these grants have been routinely awarded since the 2013 federal fiscal
year and in the first two years of his presidency, Mr. Trump's proposed
budgets
\href{https://crsreports.congress.gov/product/pdf/R/R46191\#page=10}{called
for the grants to be phased out}.

Other tweets were ambiguously worded, making them difficult to fact
check even as they hint at nefarious activity.

Mr. Trump
\href{https://twitter.com/realDonaldTrump/status/1265819308699070464}{twice}
\href{https://twitter.com/realDonaldTrump/status/1265601611310739456}{said}
that social media companies ``attempted'' and ``failed'' to do something
in the 2016 election, but never specified what exactly the companies
attempted. (In the past, he has
\href{https://www.nytimes3xbfgragh.onion/2019/08/19/us/politics/google-votes-election-trump.html}{mischaracterized
research} to mount a baseless suggestion that Google ``manipulated''
votes.)

Perhaps there's no better example of how Mr. Trump trades in vague
claims than his repeated allegations of the ``greatest political crime''
or scandal in history, committed by the Obama administration to undercut
his 2016 campaign and the start of his presidency. In
\href{https://twitter.com/realDonaldTrump/status/1265976095209373696}{four}
\href{https://twitter.com/realDonaldTrump/status/1266017512162037761}{tweets},
the president
\href{https://twitter.com/realDonaldTrump/status/1265768877427851265}{echoed}
\href{https://twitter.com/realDonaldTrump/status/1264754622830444544}{this}
but never specified what that crime was. In others, he simply referred
to ``Obamagate.''

\textbf{Curious about the accuracy of a claim? Email}
\textbf{\href{mailto:factcheck@NYTimes.com}{\nolinkurl{factcheck@NYTimes.com}}.}

Advertisement

\protect\hyperlink{after-bottom}{Continue reading the main story}

\hypertarget{site-index}{%
\subsection{Site Index}\label{site-index}}

\hypertarget{site-information-navigation}{%
\subsection{Site Information
Navigation}\label{site-information-navigation}}

\begin{itemize}
\tightlist
\item
  \href{https://help.nytimes3xbfgragh.onion/hc/en-us/articles/115014792127-Copyright-notice}{©~2020~The
  New York Times Company}
\end{itemize}

\begin{itemize}
\tightlist
\item
  \href{https://www.nytco.com/}{NYTCo}
\item
  \href{https://help.nytimes3xbfgragh.onion/hc/en-us/articles/115015385887-Contact-Us}{Contact
  Us}
\item
  \href{https://www.nytco.com/careers/}{Work with us}
\item
  \href{https://nytmediakit.com/}{Advertise}
\item
  \href{http://www.tbrandstudio.com/}{T Brand Studio}
\item
  \href{https://www.nytimes3xbfgragh.onion/privacy/cookie-policy\#how-do-i-manage-trackers}{Your
  Ad Choices}
\item
  \href{https://www.nytimes3xbfgragh.onion/privacy}{Privacy}
\item
  \href{https://help.nytimes3xbfgragh.onion/hc/en-us/articles/115014893428-Terms-of-service}{Terms
  of Service}
\item
  \href{https://help.nytimes3xbfgragh.onion/hc/en-us/articles/115014893968-Terms-of-sale}{Terms
  of Sale}
\item
  \href{https://spiderbites.nytimes3xbfgragh.onion}{Site Map}
\item
  \href{https://help.nytimes3xbfgragh.onion/hc/en-us}{Help}
\item
  \href{https://www.nytimes3xbfgragh.onion/subscription?campaignId=37WXW}{Subscriptions}
\end{itemize}
