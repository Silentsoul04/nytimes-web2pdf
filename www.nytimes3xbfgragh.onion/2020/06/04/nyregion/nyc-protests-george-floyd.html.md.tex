Sections

SEARCH

\protect\hyperlink{site-content}{Skip to
content}\protect\hyperlink{site-index}{Skip to site index}

\href{https://www.nytimes3xbfgragh.onion/section/nyregion}{New York}

\href{https://myaccount.nytimes3xbfgragh.onion/auth/login?response_type=cookie\&client_id=vi}{}

\href{https://www.nytimes3xbfgragh.onion/section/todayspaper}{Today's
Paper}

\href{/section/nyregion}{New York}\textbar{}After Curfew, Protesters Are
Again Met With Strong Police Response in New York City

\url{https://nyti.ms/2Ud85EC}

\begin{itemize}
\item
\item
\item
\item
\item
\end{itemize}

\hypertarget{race-and-america}{%
\subsubsection{\texorpdfstring{\href{https://www.nytimes3xbfgragh.onion/news-event/george-floyd-protests-minneapolis-new-york-los-angeles?name=styln-george-floyd\&region=TOP_BANNER\&block=storyline_menu_recirc\&action=click\&pgtype=Article\&impression_id=7ed72ce0-f4cf-11ea-9a82-8ff3d0b97313\&variant=undefined}{Race
and America}}{Race and America}}\label{race-and-america}}

\begin{itemize}
\tightlist
\item
  \href{https://www.nytimes3xbfgragh.onion/2020/09/11/us/black-police-chiefs-reform.html?name=styln-george-floyd\&region=TOP_BANNER\&block=storyline_menu_recirc\&action=click\&pgtype=Article\&impression_id=7ed753f0-f4cf-11ea-9a82-8ff3d0b97313\&variant=undefined}{Black
  Police Chiefs}
\item
  \href{https://www.nytimes3xbfgragh.onion/2020/09/04/nyregion/rochester-police-daniel-prude.html?name=styln-george-floyd\&region=TOP_BANNER\&block=storyline_menu_recirc\&action=click\&pgtype=Article\&impression_id=7ed753f1-f4cf-11ea-9a82-8ff3d0b97313\&variant=undefined}{What
  Happened in Rochester, N.Y.}
\item
  \href{https://www.nytimes3xbfgragh.onion/2020/08/30/us/portland-shooting-explained.html?name=styln-george-floyd\&region=TOP_BANNER\&block=storyline_menu_recirc\&action=click\&pgtype=Article\&impression_id=7ed753f2-f4cf-11ea-9a82-8ff3d0b97313\&variant=undefined}{Portland
  Shooting}
\item
  \href{https://www.nytimes3xbfgragh.onion/2020/08/30/us/breonna-taylor-police-killing.html?name=styln-george-floyd\&region=TOP_BANNER\&block=storyline_menu_recirc\&action=click\&pgtype=Article\&impression_id=7ed753f3-f4cf-11ea-9a82-8ff3d0b97313\&variant=undefined}{Breonna
  Taylor's Life and Death}
\end{itemize}

Advertisement

\protect\hyperlink{after-top}{Continue reading the main story}

Supported by

\protect\hyperlink{after-sponsor}{Continue reading the main story}

\hypertarget{after-curfew-protesters-are-again-met-with-strong-police-response-in-new-york-city}{%
\section{After Curfew, Protesters Are Again Met With Strong Police
Response in New York
City}\label{after-curfew-protesters-are-again-met-with-strong-police-response-in-new-york-city}}

Officers were forceful in dispersing demonstrators who stayed out past
curfew in the Bronx, Manhattan and Brooklyn.

Published June 4, 2020Updated July 17, 2020

\begin{itemize}
\item
\item
\item
\item
\item
\end{itemize}

\emph{{[}This briefing has ended. For the latest updates on the George
Floyd protests in New York City,}
\href{https://www.nytimes3xbfgragh.onion/2020/06/05/nyregion/nyc-protests-george-floyd.html}{\emph{read
Friday's live coverage}}\emph{.{]}}

\hypertarget{heres-what-you-need-to-know}{%
\subsubsection{Here's what you need to
know:}\label{heres-what-you-need-to-know}}

\begin{itemize}
\tightlist
\item
  \protect\hyperlink{link-2c2fd029}{As protesters defied the curfew,
  police again surrounded crowds and charged in.}
\item
  \protect\hyperlink{link-5f1d02d0}{The police pinned protesters in the
  Bronx, then charged at them with batons.}
\item
  \protect\hyperlink{link-1a01dcc0}{Despite the threat of altercations,
  thousands showed up, energized to demand change.}
\item
  \protect\hyperlink{link-3835e093}{A large group of Brooklyn protesters
  policed themselves, with positive results.}
\item
  \protect\hyperlink{link-2b6fee88}{Buffalo police officers are
  suspended after a video showed them shoving a 75-year-old protester.}
\end{itemize}

\includegraphics{https://static01.graylady3jvrrxbe.onion/images/2020/07/04/nyregion/04nyunrest-briefing26/04nyunrest-briefing26-articleLarge.jpg?quality=75\&auto=webp\&disable=upscale}

\hypertarget{as-protesters-defied-the-curfew-police-again-surrounded-crowds-and-charged-in}{%
\subsection{As protesters defied the curfew, police again surrounded
crowds and charged
in.}\label{as-protesters-defied-the-curfew-police-again-surrounded-crowds-and-charged-in}}

A week has passed since protesters, upset, angry and energized by the
killing in police custody of George Floyd in Minneapolis, first poured
into New York City's streets.

On Thursday, despite a citywide curfew that was aggressively enforced by
police officers the night before, thousands of people were still on the
streets in large gatherings in Brooklyn, Manhattan and the Bronx after 8
p.m. came.

The intensity of the police crackdown appeared somewhat lesser than the
night before, with fewer violent confrontations, but officers continued
to surround
\href{https://www.nytimes3xbfgragh.onion/2020/07/17/us/portland-protests.html}{protesters}
and box them in before charging aggressively at crowds to break them up.

In the Bronx, just after the curfew began, rows of officers confined
protesters from both sides, pinning them in
\href{https://www.nytimes3xbfgragh.onion/2020/06/05/nyregion/police-kettling-protests-nyc.html}{using
a tactic known as kettling}, before running at the group with batons and
striking several demonstrators. At least one person was taken away on a
stretcher.

Around 9:15 p.m., in Brooklyn, police officers charged into a group of
demonstrators in the Williamsburg neighborhood, tackling several people
and making multiple arrests.

On a dimly lit side street there, officers stood over at least a
half-dozen protesters, zipping plastic handcuffs between screams and
cries for help. One man, pinned face down near the curb, said he was a
member of the press. Another lay on his back, motionless and not yet
cuffed, while officers leaned over him, assessing his condition. A woman
who swore at officers was flung to the ground and cuffed.

Around the corner, on Bedford Avenue, at least three officers tackled a
male cyclist, pulling him off his bike and onto the pavement. A police
supervisor took the bicycle and flung it into a pile of garbage bags out
for collection on the curb.

Colin Herlihy, 33, of Brooklyn, said the arrests were more violent than
anything he saw on Wednesday night in Downtown Brooklyn.
\href{https://twitter.com/AliWatkins/status/1268354710139961347}{Officers'
tactics} there have
\href{https://twitter.com/JumaaneWilliams/status/1268351817374674951}{drawn
scrutiny}.

``There, at least the cops warned people to go home,'' said Mr. Herlihy,
who was bicycling in the march and taking photographs. ``For this there
was absolutely no instigation.''

In Manhattan, the police stopped one group of protesters on the Upper
East Side that was heading south from what was a peaceful rally near
Gracie Mansion and began to make arrests. New York Times reporters at
the scene observed at least 10 people with their hands tied behind their
backs, sitting on a curb in police custody.

One man who had been pinned to the ground had his neck pressed against a
helmet that he repeatedly asked an officer to move. The officer
eventually did, telling him to relax and using an expletive in
admonishing him to stop squirming.

What appeared to be hundreds of officers later dispersed a second group
of protesters near 59th Street and Fifth Avenue by moving in from
several directions before effectively splitting the crowd in half. The
police arrested a number of people, and the rest broke off soon after.

\hypertarget{the-police-pinned-protesters-in-the-bronx-then-charged-at-them-with-batons}{%
\subsection{The police pinned protesters in the Bronx, then charged at
them with
batons.}\label{the-police-pinned-protesters-in-the-bronx-then-charged-at-them-with-batons}}

Image

The scene in the Mott Haven section of the Bronx on Thursday after the
police arrested protesters on Thursday night.Credit...Gabriela Bhaskar
for The New York Times

A protest in the South Bronx had proceeded peacefully, moving from the
Hub, one of the borough's most dynamic commercial centers, south through
the Mott Haven neighborhood.

Hundreds of demonstrators marched down Willis Avenue chanting slogans
demanding police reforms and changes to immigration policy. But police
barricades blocked their route, and minutes before the 8 p.m. curfew was
to take effect, the marchers headed east on 136th Street. Residents
cheered them as they passed.

Image

Credit...Gabriela Bhaskar for The New York Times

As the group proceeded down the street, a row of armored officers on
bicycles blocked its path. The officers shouted at the protesters to
move back. When 8 p.m. arrived, a second group of officers came in from
behind. They played a recorded message over loudspeakers advising the
group that the curfew was in effect, and that the protesters needed to
leave the streets.

But the protesters were pinned in.

Officers charged into the crowd and began to arrest people who had been
protesting peacefully moments earlier. With no apparent provocation,
officers shoved protesters onto sidewalks. Many people tried to leave,
shouting that they would willingly go home. But with officers on all
sides, they had no way out.

Image

Credit...Gabriela Bhaskar for The New York Times

Around 8:30, officers charged in again, swinging their batons and
striking protesters. Dozens of people were arrested, and then forced to
sit on the street with their hands cuffed. One person was taken away on
a stretcher.

\hypertarget{despite-the-threat-of-altercations-thousands-showed-up-energized-to-demand-change}{%
\subsection{Despite the threat of altercations, thousands showed up,
energized to demand
change.}\label{despite-the-threat-of-altercations-thousands-showed-up-energized-to-demand-change}}

Image

Protestors after curfew in Manhattan on Thursday.Credit...Chang W.
Lee/The New York Times

Despite the prospect of altercations with officers enforcing the curfew
as happened on Wednesday night, many protesters were undeterred in their
desire to march again on Thursday and urge change in the
criminal-justice system.

For some of them, the recent tensions had only deepened their
commitment.

``I saw the videos and just had to come out myself and do something,
anything, whatever I could,'' Linda Shapford, 47, said at a memorial for
Mr. Floyd that was held at Cadman Plaza in Brooklyn on Thursday. ``It
just wasn't a question after what I saw from last night.''

Mayor Bill de Blasio and the police commissioner, Dermot F. Shea, had
defended officers' aggressive actions earlier in the day. Gov. Andrew M.
Cuomo also lent his support, though an aide later said he had asked the
attorney general to review any possible misconduct.

For many of the protesters who assembled on Thursday, the officials'
defensiveness was further justification for staying in the streets and
pushing for change.

``The response has been way too dramatic,'' Anjali Jamin, 31, said at a
rally at Grand Army Plaza in Brooklyn.

Ms. Jamin, a medical student, marched with a group of about 60 health
care workers and students from SUNY Downstate Medical Center in
Brooklyn. At a ``die-in'' earlier, medical residents gave speeches and
lay in silence for 8 minutes and 46 seconds --- the length of time that
prosecutors say that Mr. Floyd, a 46-year-old black man, was pinned by
the neck under a Minneapolis police officer's knee before he died.

Shakaa Chaiban, a 20-year-old from Brooklyn, said he believed that
anyone who looked at a teenager's social media feed would come across a
video showing police brutality. And the nature of the New York protests,
he added, had been mischaracterized.

``Mainstream media coverage doesn't do justice to how these protests
actually are,'' said Mr. Chaiban, who was standing atop a planter near
Barclays Center offering bags of snacks to protesters as he had for the
past two days. ``They're not portraying the unity and the demands and
whose responsibility the violence is.''

On Manhattan's Upper East Side, hundreds of people protested peacefully
again near Gracie Mansion, the mayor's official residence.

David J. Hamilton III, 35, addressed the crowd, using his remarks to
press Mr. de Blasio to urge the police to adjust their tactics.

Mr. Hamilton said he initially felt ``a little mixture of exhaustion and
anger'' after hearing about the deaths of Mr. Floyd and Breonna Taylor,
a 26-year-old African-American emergency room technician who was shot by
police in Louisville, Ky.

After debating whether to join for the protests, he had joined a rally
for the first time on Thursday.

``Floyd and Taylor, they are showing us that tomorrow is not promised,''
he said. ``If we are treating each breath as a gift, then we are
definitely not out here wasting it.''

\hypertarget{a-large-group-of-brooklyn-protesters-policed-themselves-with-positive-results}{%
\subsection{A large group of Brooklyn protesters policed themselves,
with positive
results.}\label{a-large-group-of-brooklyn-protesters-policed-themselves-with-positive-results}}

In Brooklyn, over 1,000 demonstrators marched down Atlantic Avenue
blocking traffic in both directions after curfew.

``The people united will never be defeated!'' they chanted as police
vehicles and dozens of officers trailed them.

As the demonstrators kept moving, a small group of officers blocked them
from entering a cross street. A few protesters stopped and began
screaming angrily at the police. Leaders of the group rushed to the
intersection and guided them away.

``Keep walking!'' they shouted. ``Keep walking!''

For over an hour, the protesters evaded officers' efforts to cut off or
corner them. Leaders sent people on bicycles ahead when they saw the
police gathering to form a protective barrier, quickly directed marchers
onto side streets to avoid police blockades and kept the group packed
tightly together to prevent it from being broken into smaller clusters.

When the demonstration reached the Barclays Center, protest leaders
guided the group away from the plaza and into an adjacent neighborhood.

``No vandalism!'' one person yelled into a megaphone. ``No one is
getting hurt tonight!''

But when the protest reached Washington Avenue, officers caught up to
them. With police vans closing in from behind, people trying to reach
the other end of the street encountered blockade and were finally boxed
in.

Tensions rose as officers descended. But organizers intervened, as did
the city's public advocate, Jumaane Williams, and Brad Lander, a
Brooklyn city councilman, who urged the police to allow protesters to
disperse peacefully.

Instead of moving into the crowd and arresting protesters as officers
did elsewhere, the police began to corral small groups of protesters
onto the sidewalk.

``Come on, it's time to go, it's time to go, I'm asking you nicely,''
one officer yelled.

The group split up, and people began to make their way home.

``Tomorrow, guys, tomorrow,'' one person yelled. ``Tomorrow's another
day!''

\hypertarget{buffalo-police-officers-are-suspended-after-a-video-showed-them-shoving-a-75-year-old-protester}{%
\subsection{Buffalo police officers are suspended after a video showed
them shoving a 75-year-old
protester.}\label{buffalo-police-officers-are-suspended-after-a-video-showed-them-shoving-a-75-year-old-protester}}

\includegraphics{https://static01.graylady3jvrrxbe.onion/images/2020/06/06/autossell/Police-Push1/Police-Push-videoSixteenByNineJumbo1600.jpg}

Two Buffalo police officers were suspended without pay on Thursday night
after a video showed them shoving a 75-year-old protester, who was
hospitalized with a head injury as a result, the authorities said.

Byron Brown, the city's mayor, said the man was in serious but stable
condition. A video showed him motionless on the ground and bleeding from
his right ear after being shoved.

Mr. Cuomo, in a statement late Thursday, condemned the officers'
actions.

``The incident in Buffalo is wholly unjustified and utterly
disgraceful,'' Mr. Cuomo said. ``I've spoken with City of Buffalo Mayor
Byron Brown and we agree that the officers involved should be
immediately suspended. Police officers must enforce --- NOT ABUSE ---
the law.''

In the video, the officer who pushed the man appeared to start to check
on him but was nudged to leave by another officer. Someone could be
overheard saying, ``Get a medic, right now.''

The Buffalo Police Department initially told local media that
\href{https://twitter.com/JeffRussoWKBW/status/1268712651292643334?s=20}{``one
person was injured when he tripped and fell''} and that there had been
five arrests during the protest.

\hypertarget{the-mayor-was-criticized-at-a-memorial-where-george-floyds-brother-spoke}{%
\subsection{The mayor was criticized at a memorial where George Floyd's
brother
spoke.}\label{the-mayor-was-criticized-at-a-memorial-where-george-floyds-brother-spoke}}

Image

Terrence Floyd spoke at a memorial for his brother George at Cadman
Plaza in Brooklyn on Thursday.Credit...Demetrius Freeman for The New
York Times

Mr. de Blasio was met with hostility on Thursday at a memorial for
George Floyd in Brooklyn, the first time he had appeared in person
before protesters who have been marching in New York City's streets for
a week.

The demonstrators chanted ``I can't breathe,'' ``resign'' and ``defund
the police,'' and used a profanity to decry the curfew Mr. de Blasio
imposed after several days of rallies touched off by the killing Mr.
Floyd.

When the mayor was introduced by his wife, Chirlane McCray, the crowd
immediately jeered. A local pastor raised his hand in a peace sign,
urging that Mr. de Blasio be given a chance speak at the memorial, where
Mr. Floyd's brother Terrence spoke.

``For all of us who know white privilege, we need do more,'' Mr. de
Blasio said, ``because we don't even fully recognize the daily pain that
the racism in this society causes.''

But after struggling to be heard, the mayor quickly cut off his remarks.
The scene unfolded at Cadman Plaza, where the police used aggressive
tactics on Wednesday night against protesters who had defied the curfew.

\includegraphics{https://static01.graylady3jvrrxbe.onion/images/2020/06/04/autossell/bdb/bdb-videoSixteenByNineJumbo1600.jpg}

Terrence Floyd, wearing a face mask bearing an image of his brother and
a Yankees cap, teared up for about two minutes before regaining his
composure and addressing the crowd.

``I want to thank God,'' he said, adding, ``It wasn't his fault. It was
his will.''

``I thank God for you all showing love to my brother,'' Mr. Floyd
continued.

He also criticized some of the violence that has broken out at some
protests.

``I'm proud of the protest, but I'm not proud of the destruction,'' he
said. ``My brother wasn't about that.''

After the memorial ended, thousands of those who had gathered left the
plaza and marched across the Brooklyn Bridge into Manhattan ---~a goal
of rallies on previous nights that was thwarted by officers who would
not let protesters travel between the boroughs.

The group on Thursday swarmed the bridge's north-side lane and
pedestrian walkway, pausing regularly to take a knee as they crossed.
Drivers in the opposite lane honked horns and raised fists in shows of
support.

\hypertarget{for-many-people-the-protests-are-family-affairs}{%
\subsection{For many people, the protests are family
affairs.}\label{for-many-people-the-protests-are-family-affairs}}

For the past several days and again on Thursday, New Yorkers took to the
streets for protests that for some have evolved into multigenerational
family events.

Many older people who have taken part in the demonstrations said they
had seen and felt powerful echoes of past civil rights movements. Some
have sought to use what is happening now to teach lessons to children
and grandchildren.

``We marched in the '60s,'' Joseph Phillips, 71, said, as he looked on
as a protest at Grand Army Plaza in Brooklyn swelled early Thursday
evening, ``and it doesn't get any better.''

On Wednesday, Celia Oliver, 30, a nurse practitioner, had brought her
9-month-old son, Elliot, to a rally on Roosevelt Island. He peered out,
wide-eyed, from under his stroller's clear rain cover.

``I think it's important to show that all of us --- and every person who
has been killed --- started out as babies,'' Ms. Oliver said.

The protest, she said, would be ``his first lesson in anti-racism.''

On Bedford Avenue in Brooklyn the same day, Sarah Edwards, 54, scooped
up her grandson on the way to a rally. She said she had decided to come
because of him and her young nephews.

``I want them to live and see goodness and greatness in this world,''
she said, adding that ``this world is in trouble!''

Sheree Murphy, 40, was among several hundred people gathered at the
entrance of Carl Schurz Park on Manhattan's Upper East Side on Thursday
for a silent vigil in honor of Mr. Floyd. She brought her three
daughters --- 10, 7 and 20 months --- with her.

One of the girls lifted a homemade sign with a picture of the earth and
the words ``unity'' scrawled across it.

Ms. Murphy said she had been trying to speak to her children about
racial prejudice and police brutality ``so they can be aware and mindful
of it and speak out against it.''

Not all of the children at the park had had such conversations.

Fifteen minutes into the vigil, a toddler in a bike helmet ran between
protesters who were sitting on the ground.

``Why are you all here?'' he squealed.

\hypertarget{i-am-sorry-new-york-citys-police-commissioner-said-are-you}{%
\subsection{`I am sorry,' New York City's police commissioner said. `Are
you?'}\label{i-am-sorry-new-york-citys-police-commissioner-said-are-you}}

Commissioner Shea, apologized on Thursday for any instances of police
misconduct over the past several days.

But during a brief news conference at Police Headquarters, he also
demanded that demonstrators stop insulting and attacking his officers
and he warned that anti-police rhetoric could lead to continued violence
against those he oversees.

``For there to be calm, there must also be contrition,'' Commissioner
Shea said. ``So I am sorry. Sometimes even the best --- and the N.Y.P.D.
is the goddamned best police department in the country --- but sometimes
even the best fall down.''

``So for our part in the damage to civility, for our part in racial
bias, in excessive force, unacceptable behavior, unacceptable language
and many other mistakes, we are human,'' he said. ``I am sorry. Are
you?''

The commissioner,
\href{https://www.nytimes3xbfgragh.onion/2020/05/29/nyregion/nypd-officers-charged-social-distancing-arrest.html}{who
has condemned the killing of Mr. Floyd}, said he knew of at least seven
possible episodes of misconduct by New York officers in the course of
the demonstrations. There would, he said, ``probably be a couple of
officers suspended'' as a result.

But he also argued that videos of some incidents that had been shared
online were presented out of context and that in many cases, officers'
use of force had been ``completely justified.''

Overall, he said, the vast majority of New York officers had been
professional and had exercised ``extreme restraint'' in what he called a
``riot situation.''

On that point and others, the commissioner was defensive and defiant. He
opened the briefing by playing several video clips of
\href{https://www.nytimes3xbfgragh.onion/2020/06/04/nyregion/nypd-officers-shot-brooklyn.html}{officers
being attacked} in what he called acts of ``intentional violence.''

Specifically, he said people had run police officers over with vehicles,
bashed them in the head with fire extinguishers and shot at them. He
also highlighted episodes over the previous 24 hours where, he said,
attackers had tried to stab officers.

In the past several days, New York Times journalists embedded in
protests
\href{https://twitter.com/AliWatkins/status/1268530046173618178}{have
reported} that officers had charged at demonstrators with seemingly
little provocation, shoved them onto sidewalks, struck them with batons
and used other aggressive tactics.

Commissioner Shea also criticized elected officials who, he said, were
``spewing complaint after complaint after complaint to help their
careers.''

``Do not tell me that words don't matter,'' he said after recounting the
stories and names of N.Y.P.D.
\href{https://www.nytimes3xbfgragh.onion/2014/12/21/nyregion/two-police-officers-shot-in-their-patrol-car-in-brooklyn.html}{police
officers}
\href{https://www.nytimes3xbfgragh.onion/2017/07/05/nyregion/officer-shot-bronx-miosotis-familia.html}{killed}
in recent years. ``Don't tell me that it doesn't stir up hate.''

``If you don't have something nice to say, keep your mouth closed,'' he
added.

He also argued that ``demonstrators are not all on the same page,'' and
separated them into three camps: ``anarchists,'' ``core protesters'' and
``looters.''

``I honestly can't tell you a protester that was seriously injured,'' he
said, an assertion that
\href{https://twitter.com/AliWatkins/status/1268530567689183234}{reporting
by The Times} appeared to contradict.

By comparison, he added: ``I've been to the hospital a number times this
week for police officers.''

He also said he had decided to hold the briefing to call for civil
discourse and dialogue.

``You have to have calm in this city,'' he said. ``We need healing.''

\hypertarget{despite-a-pandemic-those-arrested-are-being-held-in-cramped-jails}{%
\subsection{Despite a pandemic, those arrested are being held in cramped
jails.}\label{despite-a-pandemic-those-arrested-are-being-held-in-cramped-jails}}

In the past week, hundreds of people who have been arrested in New York
City --- some for suspicion of looting; others after clashing with the
police amid largely peaceful protests --- have been detained in cramped
cells for more than 24 hours, their health at risk in the midst of a
pandemic, defense lawyers said.

As of Thursday morning, more than 380 people who were being held either
in cells at Police Headquarters, at local precincts or in a Manhattan
jail had yet to appear before a judge.

Nearly 70 percent of them had been waiting more than 24 hours, including
one defendant who had been waiting 80 hours, according to court
officials and the Legal Aid Society.

Prosecutors, the police and court officials said that they were doing
what they could to process people quickly, but that they faced
logistical hurdles because of
\href{https://www.nytimes3xbfgragh.onion/2020/06/04/world/coronavirus-us-update.html?module=inline}{the
coronavirus shutdown} and the unusually high number of arrests.

The Legal Aid Society filed a lawsuit this week accusing the Police
Department of illegally detaining people who had not seen a judge after
waiting three days, a violation of a state law requiring that a person
be arraigned within 24 hours of arrest.

On Thursday, Justice James M. Burke of State Supreme Court in Manhattan
denied Legal Aid's demand that the city release people held for more
than a day. Justice Burke noted that the police were contending with
widespread civil unrest in the middle of a pandemic.

``It is a crisis within a crisis,'' he said.

\hypertarget{the-mayor-and-governor-defended-the-polices-actions-against-protesters}{%
\subsection{The mayor and governor defended the police's actions against
protesters.}\label{the-mayor-and-governor-defended-the-polices-actions-against-protesters}}

Image

Protesters in Manhattan on Wednesday night.Credit...Hiroko Masuike/The
New York Times

After a day and night of mostly peaceful protests that culminated with
officers aggressively arresting demonstrators who remained on the
streets after 8 p.m. on Wednesday, Mr. de Blasio on Thursday
\href{https://www.nytimes3xbfgragh.onion/2020/06/04/nyregion/De-blasio-protests-curfew.html}{emphatically
defended} the Police Department's actions.

The crowds in Brooklyn and Manhattan who continued to rally against
police brutality and racism after the clampdown took effect on Wednesday
were bigger than they had been the night before.

And the police moved more swiftly to disperse demonstrators from the
city's rainy streets and to arrest those who failed to clear out.

``When people are instructed by the N.Y.P.D., especially after curfew,
they must follow those instructions,'' the mayor told reporters at a
news briefing.

Referring to officers who aggressively dispersed crowds, he added,
``It's not an unfair action to say, in the context of this crisis, in
the context of curfew, there is a point where enough is enough.''

Mr. Cuomo lent his support to the police, saying they were doing ``an
impossible job.''

Mr. Cuomo bristled when asked about the police using batons to disperse
peaceful protesters, despite reporting and widely seen videos that
captured just that.

``That's not a fact,'' he told reporters at a briefing. ``They don't do
that. Anyone who did do that would be obviously reprehensible, if not
criminal.''

Later in the day, a top aide to Mr. Cuomo said that the governor had
asked the attorney general, Letitia James, to review the police's
altercations with protesters in Brooklyn.

Mr. de Blasio said he had not seen videos or reports of the police using
batons to hit peaceful protesters, but he promised investigations if
they were warranted.

And just after midnight on Friday, the mayor
\href{https://twitter.com/NYCMayor/status/1268756177988485121}{expressed
alarm on Twitter} about the police apparently violating the rights of
essential workers performing their jobs after curfew.

\includegraphics{https://static01.graylady3jvrrxbe.onion/images/2020/06/04/nyregion/04vid-Cuomo-Live/merlin_172889781_432d3f30-e54f-4a4b-b915-7bc72f8047d1-videoSixteenByNine3000.jpg}

The mayor has sought to strike a balance on how the nightly curfew would
be enforced. He reiterated that peaceful protests would be allowed to
continue even after curfew, but that the police would have the
discretion to decide when to disperse crowds even if all was calm.

Mr. de Blasio, who campaigned on a promise to reform the city's police
force, the largest in the United States, has focused on defending the
department while also vowing to investigate reports of misconduct.

The mayor said the police should apply as much restraint as possible.

``I want the absolute least use of force,'' he said. ``Ideally, no use
of pepper spray or batons. There are certain situations where it's
necessary.''

As happened on Wednesday, there were few reports on Thursday of the
looting and vandalism the curfew was meant to curb.

\hypertarget{de-blasio-faces-his-one-of-the-toughest-moments-of-his-tenure}{%
\subsection{De Blasio faces his one of the toughest moments of his
tenure.}\label{de-blasio-faces-his-one-of-the-toughest-moments-of-his-tenure}}

Image

Protesters in Harlem in Manhattan on Thursday.Credit...Simbarashe Cha
for The New York Times

The aggression displayed by officers against protesters has created a
new crisis for Mr. de Blasio, who has praised the police for their
``tremendous restraint'' in handling the unrest.

Former allies have denounced his leadership. New Yorkers have called for
his resignation and that of Mr. Shea. Onetime aides have bickered online
about Mr. de Blasio's performance, and their perhaps self-interested
desire to distance themselves from him.

And when it seemed like Mr. de Blasio's day could not get worse on
Wednesday, an assailant stabbed a police officer in the neck, setting
off a gunfight that left two other officers wounded, and pulled the
mayor to a Brooklyn hospital for an early-morning news conference.

Mr. de Blasio is facing what may be the worst moment of his tenure since
the 2014 fatal shootings of two police officers by a man seeking
retribution for the police killing of Eric Garner on Staten Island.

``The most charitable assessment is that his mayoralty is currently on
life support,'' Neal Kwatra, a former adviser to both Mr. de Blasio and
Mr. Cuomo, said in a text message.

Mr. de Blasio said his detractors did not fully appreciate what the city
was dealing with.

``For anyone out there who is concerned or criticizing, I'm not sure
they understand the depth of the reality of what we've faced,'' he said
on Thursday. ``We have to keep the peace. We have to keep order. We have
to protect our democracy and our democratic rights. We're striking that
balance all the time.''

\hypertarget{he-was-wearing-a-mask-because-of-the-coronavirus-a-police-officer-pulled-it-down-and-maced-him}{%
\subsection{He was wearing a mask because of the coronavirus. A police
officer pulled it down and maced
him.}\label{he-was-wearing-a-mask-because-of-the-coronavirus-a-police-officer-pulled-it-down-and-maced-him}}

Andrew Smith, a 31-year-old Brooklyn resident, was among the those who
attended a protest near the corner of Bedford and Tilden Avenues last
Saturday.

He wanted to do his part to prevent police brutality, as well as the
spread of the coronavirus, so he made sure to attend the rally wearing a
mask. His was black, red and white, the colors of the national flag of
his native Trinidad.

At one point, officers near Mr. Smith and other protesters wanted to
clear a path for a police vehicle to depart. A video showed Mr. Smith
with his hands raised in the air as an officer, who was also wearing a
mask, put his hand on Mr. Smith's chest.

Mr. Smith said in an interview on Thursday that he told the officer not
to touch him.

``I guess he took that as offense,'' Mr. Smith said.

Then, as a widely circulated video showed, the officer pulled down Mr.
Smith's mask and sprayed him in the face with mace.

``It hurts,'' Mr. Smith recalled later. ``Hurts like hell. It's
blinding.''

``He got a clean spray in there,'' he added. ``I didn't swipe at him or
the spray, so I kind of got a good amount of it.''

The incident was among several that were captured on video and have
prompted investigations of possible misconduct by New York City officers
against protesters.

Mr. Smith, a hedge fund recruiter who wears contact lenses, said that he
had wanted to continue protesting that day but that volunteers providing
medical care had told him to seek treatment.

``My skin was just burning from the mace so I had to go take care of
that,'' he said.

Alain V. Massena, a lawyer for Mr. Smith, said that he and other lawyers
were cooperating in investigations being conducted by the state attorney
general's office, the Police Department and the Brooklyn district
attorney's office.

After being sprayed by the officer, Mr. Smith said he went home to
recover. He rejoined the protests the next two days before realizing
that he might have been exposed to the coronavirus.

``His hands touched my face, you know, as he grabbed for my mask,'' Mr.
Smith said of the officer. ``You don't know what he's been touching or
where he's been.''

Since Tuesday, Mr. Smith has been avoiding protests and marches for fear
of inadvertently spreading the virus to other people at those events.
The decision to stay away ``is aggravating,'' he said. He added that he
was ``hopeful I can get a Covid test and get that squared away sooner
than later.''

``I could be at potential risk,'' he added. ``I'd rather be respectful
of folks and try to support in other ways.''

Reporting was contributed by Anne Barnard, Gabriela Bhaskar, Julia
Carmel, Annie Correal, Luis Ferré-Sadurní, Alan Feuer, Michael Gold,
Christina Goldbaum, Corey Kilgannon, Jeffery C. Mays, Terence McGinley,
Andy Newman, Derek M. Norman, Azi Paybarah, Pia Peterson, Sean Piccoli,
Jan Ransom, Dana Rubinstein, Eliza Shapiro, Ashley Southall, Liam Stack,
Matt Stevens and Anjali Tsui.

Advertisement

\protect\hyperlink{after-bottom}{Continue reading the main story}

\hypertarget{site-index}{%
\subsection{Site Index}\label{site-index}}

\hypertarget{site-information-navigation}{%
\subsection{Site Information
Navigation}\label{site-information-navigation}}

\begin{itemize}
\tightlist
\item
  \href{https://help.nytimes3xbfgragh.onion/hc/en-us/articles/115014792127-Copyright-notice}{©~2020~The
  New York Times Company}
\end{itemize}

\begin{itemize}
\tightlist
\item
  \href{https://www.nytco.com/}{NYTCo}
\item
  \href{https://help.nytimes3xbfgragh.onion/hc/en-us/articles/115015385887-Contact-Us}{Contact
  Us}
\item
  \href{https://www.nytco.com/careers/}{Work with us}
\item
  \href{https://nytmediakit.com/}{Advertise}
\item
  \href{http://www.tbrandstudio.com/}{T Brand Studio}
\item
  \href{https://www.nytimes3xbfgragh.onion/privacy/cookie-policy\#how-do-i-manage-trackers}{Your
  Ad Choices}
\item
  \href{https://www.nytimes3xbfgragh.onion/privacy}{Privacy}
\item
  \href{https://help.nytimes3xbfgragh.onion/hc/en-us/articles/115014893428-Terms-of-service}{Terms
  of Service}
\item
  \href{https://help.nytimes3xbfgragh.onion/hc/en-us/articles/115014893968-Terms-of-sale}{Terms
  of Sale}
\item
  \href{https://spiderbites.nytimes3xbfgragh.onion}{Site Map}
\item
  \href{https://help.nytimes3xbfgragh.onion/hc/en-us}{Help}
\item
  \href{https://www.nytimes3xbfgragh.onion/subscription?campaignId=37WXW}{Subscriptions}
\end{itemize}
