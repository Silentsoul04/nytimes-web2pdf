Sections

SEARCH

\protect\hyperlink{site-content}{Skip to
content}\protect\hyperlink{site-index}{Skip to site index}

\href{https://www.nytimes3xbfgragh.onion/section/nyregion}{New York}

\href{https://myaccount.nytimes3xbfgragh.onion/auth/login?response_type=cookie\&client_id=vi}{}

\href{https://www.nytimes3xbfgragh.onion/section/todayspaper}{Today's
Paper}

\href{/section/nyregion}{New York}\textbar{}Despite Virus, Hundreds
Arrested in Unrest Are Held in Cramped Jails

\url{https://nyti.ms/3eO054x}

\begin{itemize}
\item
\item
\item
\item
\item
\end{itemize}

\hypertarget{race-and-america}{%
\subsubsection{\texorpdfstring{\href{https://www.nytimes3xbfgragh.onion/news-event/george-floyd-protests-minneapolis-new-york-los-angeles?name=styln-george-floyd\&region=TOP_BANNER\&block=storyline_menu_recirc\&action=click\&pgtype=Article\&impression_id=89952270-f4bd-11ea-ad23-071b6226df98\&variant=undefined}{Race
and America}}{Race and America}}\label{race-and-america}}

\begin{itemize}
\tightlist
\item
  \href{https://www.nytimes3xbfgragh.onion/2020/09/11/us/black-police-chiefs-reform.html?name=styln-george-floyd\&region=TOP_BANNER\&block=storyline_menu_recirc\&action=click\&pgtype=Article\&impression_id=89952271-f4bd-11ea-ad23-071b6226df98\&variant=undefined}{Black
  Police Chiefs}
\item
  \href{https://www.nytimes3xbfgragh.onion/2020/09/04/nyregion/rochester-police-daniel-prude.html?name=styln-george-floyd\&region=TOP_BANNER\&block=storyline_menu_recirc\&action=click\&pgtype=Article\&impression_id=89952272-f4bd-11ea-ad23-071b6226df98\&variant=undefined}{What
  Happened in Rochester, N.Y.}
\item
  \href{https://www.nytimes3xbfgragh.onion/2020/08/30/us/portland-shooting-explained.html?name=styln-george-floyd\&region=TOP_BANNER\&block=storyline_menu_recirc\&action=click\&pgtype=Article\&impression_id=89952273-f4bd-11ea-ad23-071b6226df98\&variant=undefined}{Portland
  Shooting}
\item
  \href{https://www.nytimes3xbfgragh.onion/2020/08/30/us/breonna-taylor-police-killing.html?name=styln-george-floyd\&region=TOP_BANNER\&block=storyline_menu_recirc\&action=click\&pgtype=Article\&impression_id=89954980-f4bd-11ea-ad23-071b6226df98\&variant=undefined}{Breonna
  Taylor's Life and Death}
\end{itemize}

Advertisement

\protect\hyperlink{after-top}{Continue reading the main story}

Supported by

\protect\hyperlink{after-sponsor}{Continue reading the main story}

\hypertarget{despite-virus-hundreds-arrested-in-unrest-are-held-in-cramped-jails}{%
\section{Despite Virus, Hundreds Arrested in Unrest Are Held in Cramped
Jails}\label{despite-virus-hundreds-arrested-in-unrest-are-held-in-cramped-jails}}

A flood of arrests has caused a backlog in New York City's courts,
forcing many to wait for more than 24 hours before seeing a judge.

\includegraphics{https://static01.graylady3jvrrxbe.onion/images/2020/06/03/nyregion/03nyunrest-arrests/merlin_173119836_ff94e887-aa5a-42a6-87b1-ff7d4e151659-articleLarge.jpg?quality=75\&auto=webp\&disable=upscale}

By \href{https://www.nytimes3xbfgragh.onion/by/jan-ransom}{Jan Ransom}

\begin{itemize}
\item
  June 4, 2020
\item
  \begin{itemize}
  \item
  \item
  \item
  \item
  \item
  \end{itemize}
\end{itemize}

They were held for more than a day in crowded New York City jail cells,
some without masks. Cheerios and food wrappers littered the floor. In
one holding pen, detainees spread tissue over a clogged toilet to try to
reduce the stench emanating from it.

In the week since the police killing of George Floyd in Minneapolis,
hundreds of people arrested in New York City --- some while looting,
others while clashing with the police during largely peaceful
demonstrations --- have been detained in cramped cells for more than 24
hours, their health at risk in the midst of a pandemic, defense lawyers
said.

On Thursday morning, more than 380 people --- waiting either in cells at
Police Headquarters, in local precincts and in a Manhattan jail --- had
yet to be brought before a judge. Nearly 70 percent of them had been
waiting for more than 24 hours, including one defendant who had been
waiting 80 hours, according to court officials and the Legal Aid
Society.

Police, prosecutors and court officials say they are doing what they can
to process people quickly, but they are facing logistical hurdles
because of the coronavirus shutdown and an unusually high number of
arrests.

But public defenders say prolonged detention of defendants violates
state law and their constitutional rights. They say the police have
clogged up the system by putting people through the courts who should
have instead received summonses for minor offenses during the protests.

``Rather than allowing people to protest peacefully, the N.Y.P.D. are
violently arresting them and holding them for hours,'' Stan Germán,
executive director of New York County Defender Services, said in a
statement. ``They are unlawfully and unnecessarily sending people
through the criminal arraignment process, where they face tight quarters
and exposure to the coronavirus.''

Lawyers for the city deny that protesters are punitively being held
longer than necessary. Mayor Bill de Blasio and his police commissioner,
Dermot F. Shea, have repeatedly defended the actions of rank-and file
police, saying they have shown tremendous restraint in dealing with
protesters under trying circumstances.

While the protests have been mostly peaceful, Commissioner Shea and Mr.
de Blasio have said some people have tried to incite violence against
the police, throwing bricks and bottles, while others have taken
advantage of the unrest to loot stores.

So far, more than 2,000 people have been arrested on charges such as
disorderly conduct, resisting arrest, unlawful assembly, assault on a
police officer and burglary, according to the police and prosecutors.
Most were released with a desk appearance ticket, which requires them to
return to court at a later date.

From Sunday to Tuesday, about 500 of those people have been charged with
looting shops, mostly in Manhattan, where bands of people have broken
into stores throughout SoHo and the central business district, including
\href{https://www.nytimes3xbfgragh.onion/2020/06/02/nyregion/nyc-looting-protests.html}{Macy's
flagship store in Herald Square} and Bloomingdale's on Third Avenue.
Many stores on
\href{https://www.nytimes3xbfgragh.onion/2020/06/03/nyregion/george-floyd-bronx-protests-looting.html}{Fordham
Road in the Bronx were also ransacked.}

The Legal Aid Society filed a lawsuit this week accusing the Police
Department of illegally detaining people who had not seen a judge after
waiting three days, which violates a state law requiring that a person
be arraigned within 24 hours of an arrest.

On Thursday, Justice James M. Burke of State Supreme Court in Manhattan
denied Legal Aid's demand that the city release people held for more
than a day, noting the Police Department was coping with widespread
civil unrest in the middle of a pandemic. ``It is a crisis within a
crisis,'' Justice Burke said. ``All writs are denied.''

Court officials blamed the delay on the police and on prosecutors, who
they say have been slow to complete necessary paperwork.

``To docket the case and arraign someone, the court needs the arrest
paperwork to be processed by the police and in conjunction with the
district attorney to write up the criminal complaint,'' Lucian Chalfen,
a spokesman for the Office of Court Administration, said in a statement
on Wednesday. ``It is a process that is taking a far longer time frame
than is customary.''

One protester, Khelon Robinson-Fraser, 21, of Atlanta, said he heard
officers threaten to delay processing several people's paperwork if they
asked too many questions or became fussy. ```We can keep you for however
long we want,''' he recalled an officer tell one detainee. ``It was just
pretty horrible.''

But Patricia Miller, chief of the city's Special Federal Litigation
Division, denied that officers were retaliating against protesters and
said that the agency was doing the best it could given the
circumstances. She called the allegations ``exceptionally unfair.''

``The N.Y.P.D., as well as the entire criminal court system, is working
within the confines of a pandemic and is now suddenly called upon not
only to secure orderly protesting, but also to address rioters who are
committing burglaries, destroying private property and assaulting fellow
New Yorkers,'' Ms. Miller said.

During the court hearing on Legal Aid's request, Janine Gilbert, an
assistant deputy police commissioner, acknowledged that social
distancing has been impossible in city lockups with so many arrests. She
said, for instance, it was common for up to two dozen people to be held
for hours on buses before being taken to be booked.

``And I might note that these protesters while they are out on the
street are not social distancing either,'' she said. She said each
detainee was given a mask and offered food while in the holding pen.

The court system, which
\href{https://www.nytimes3xbfgragh.onion/2020/03/20/nyregion/coronavirus-new-york-courts.html}{had
already been disrupted by the coronavirus}, has been largely virtual
since March. Since then, judges, defense lawyers and prosecutors have
conducted arraignments from their homes over video chat systems.

But the recent flood of arrests has added even more pressure, officials
said.

To handle the backlog, court officials added a second virtual
arraignment part and an overnight arraignment session from 1 a.m. to 9
a.m., Mr. Chalfen said. Most of the people awaiting arraignment were
charged with burglary, and in most cases must be released without bail
under the state's new bail law, Mr. Chalfen said.

Mr. de Blasio initially blamed the widespread looting and instances of
violence on out-of-towners and anarchists, but in recent days he and his
police commissioner acknowledged that most of the people accused of
breaking into stores were city residents using the protests as cover.
Some had even used cars and trucks --- in one instance a U-Haul truck
--- to transport stolen merchandise, police said.

Mr. de Blasio said the protests have been largely peaceful, but added
that there was ``an organized group of criminals doing things like
looting for pure financial gain, pure criminal gain, nothing to do with
protests whatsoever.''

At least a quarter of the people arrested have been accused of burglary,
which is the charge brought against looters, according to preliminary
figures from the police.

As he left the Manhattan Criminal Courts Building, Marcos Parker, 19,
said he had been detained on a burglary charge on Monday and was not
released until Wednesday.

``I won't lie. I was looting,'' he said. He said he had stolen
merchandise because he had lost his job. ``It was really this
coronavirus,'' he said. ``I was working before corona.''

Nearby, a 26-year-old woman, who requested anonymity, said her boyfriend
waited two days to see a judge after he was charged with looting.

But others have faced charges of assaulting a police officer. One
26-year-old was accused of throwing a garbage can into a crowd, striking
an officer. A 33-year-old man was charged with assault; according to the
criminal complaint, he punched an officer in the face.

Prosecutors in Brooklyn and Manhattan say they are investigating several
allegations of police using unnecessary force, including an instance ---
captured on video --- when two police S.U.V.s drove into a crowd of
protesters blocking a street in Brooklyn and a separate incident in
which a Wall Street Journal reporter says an officer assaulted him in
Manhattan.

Federal prosecutors have brought charges against two sisters from the
Catskills region who investigators say
\href{https://www.nytimes3xbfgragh.onion/aponline/2020/06/01/us/ap-us-america-protests-firebombing-charges.html}{tried
to firebomb a police vehicle} with officers inside it.

Some of the protesters, who were released this week after long
detentions, said the police officers were the aggressors.

Clarence Johnson, a 24-year-old chef from Harlem, said he and his
brother were protesting on 34th Street in Manhattan on Monday, when
officers tackled them, used pepper spray on them and then hit them with
batons. Mr. Johnson said he had bruising on his hip and that his
brother's face was swollen.

The officers had told them to go home, he said, but then had boxed them
in on the street before arresting them on charges of unlawful assembly.

Mr. Johnson said he waited 15 hours to be arraigned. By Wednesday
evening, his brother still had not appeared before a judge. Mr. Johnson
was held in a cell with about 30 people spaced only about two feet
apart, with a clogged toilet, and no soap. Some detainees were coughing
and others appeared sickly, he said.

Mr. Johnson said he had experienced many disturbing encounters with the
police while growing up in New York City. He was stopped frequently by
them as a teenager and had stared down the barrel of an officer's gun
more than once. For him, the protests against police brutality were
extremely personal.

``I have a daughter, a little girl, that's going to have to grow up in
this world,'' Mr. Johnson said. ``I just want it to be how it's supposed
to be. I shouldn't have to fear the police.''

Another protester, Dorthley Beaval, 20, a nursing student from Long
Island, emerged from the courthouse on Wednesday evening with his left
arm in a dark blue sling. The left side of his face was bruised.

He said a police officer picked him up, slammed him first against a wall
and then face down on the ground. The officer then began punching him
repeatedly until another officer stopped him, he said. One tooth had
been loosened and he coughed up blood, he said. He spent two days in a
cell waiting to see a judge. He was charged with burglary and criminal
mischief.

``I was there peacefully protesting, because of this type of behavior,''
he said. That night he had held a sign that read: For My Future Black
Children.

Ashley Southall and Colin Moynihan contributed reporting.

Advertisement

\protect\hyperlink{after-bottom}{Continue reading the main story}

\hypertarget{site-index}{%
\subsection{Site Index}\label{site-index}}

\hypertarget{site-information-navigation}{%
\subsection{Site Information
Navigation}\label{site-information-navigation}}

\begin{itemize}
\tightlist
\item
  \href{https://help.nytimes3xbfgragh.onion/hc/en-us/articles/115014792127-Copyright-notice}{©~2020~The
  New York Times Company}
\end{itemize}

\begin{itemize}
\tightlist
\item
  \href{https://www.nytco.com/}{NYTCo}
\item
  \href{https://help.nytimes3xbfgragh.onion/hc/en-us/articles/115015385887-Contact-Us}{Contact
  Us}
\item
  \href{https://www.nytco.com/careers/}{Work with us}
\item
  \href{https://nytmediakit.com/}{Advertise}
\item
  \href{http://www.tbrandstudio.com/}{T Brand Studio}
\item
  \href{https://www.nytimes3xbfgragh.onion/privacy/cookie-policy\#how-do-i-manage-trackers}{Your
  Ad Choices}
\item
  \href{https://www.nytimes3xbfgragh.onion/privacy}{Privacy}
\item
  \href{https://help.nytimes3xbfgragh.onion/hc/en-us/articles/115014893428-Terms-of-service}{Terms
  of Service}
\item
  \href{https://help.nytimes3xbfgragh.onion/hc/en-us/articles/115014893968-Terms-of-sale}{Terms
  of Sale}
\item
  \href{https://spiderbites.nytimes3xbfgragh.onion}{Site Map}
\item
  \href{https://help.nytimes3xbfgragh.onion/hc/en-us}{Help}
\item
  \href{https://www.nytimes3xbfgragh.onion/subscription?campaignId=37WXW}{Subscriptions}
\end{itemize}
