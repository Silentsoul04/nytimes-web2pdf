Sections

SEARCH

\protect\hyperlink{site-content}{Skip to
content}\protect\hyperlink{site-index}{Skip to site index}

\href{https://www.nytimes3xbfgragh.onion/section/nyregion}{New York}

\href{https://myaccount.nytimes3xbfgragh.onion/auth/login?response_type=cookie\&client_id=vi}{}

\href{https://www.nytimes3xbfgragh.onion/section/todayspaper}{Today's
Paper}

\href{/section/nyregion}{New York}\textbar{}`Kettling' of Peaceful
Protesters Shows Aggressive Shift by N.Y. Police

\url{https://nyti.ms/372PTCR}

\begin{itemize}
\item
\item
\item
\item
\item
\end{itemize}

\hypertarget{race-and-america}{%
\subsubsection{\texorpdfstring{\href{https://www.nytimes3xbfgragh.onion/news-event/george-floyd-protests-minneapolis-new-york-los-angeles?name=styln-george-floyd\&region=TOP_BANNER\&block=storyline_menu_recirc\&action=click\&pgtype=Article\&impression_id=a10364b0-f4ba-11ea-a6df-cbaea4376604\&variant=undefined}{Race
and America}}{Race and America}}\label{race-and-america}}

\begin{itemize}
\tightlist
\item
  \href{https://www.nytimes3xbfgragh.onion/2020/09/11/us/black-police-chiefs-reform.html?name=styln-george-floyd\&region=TOP_BANNER\&block=storyline_menu_recirc\&action=click\&pgtype=Article\&impression_id=a10364b1-f4ba-11ea-a6df-cbaea4376604\&variant=undefined}{Black
  Police Chiefs}
\item
  \href{https://www.nytimes3xbfgragh.onion/2020/09/04/nyregion/rochester-police-daniel-prude.html?name=styln-george-floyd\&region=TOP_BANNER\&block=storyline_menu_recirc\&action=click\&pgtype=Article\&impression_id=a10364b2-f4ba-11ea-a6df-cbaea4376604\&variant=undefined}{What
  Happened in Rochester, N.Y.}
\item
  \href{https://www.nytimes3xbfgragh.onion/2020/08/30/us/portland-shooting-explained.html?name=styln-george-floyd\&region=TOP_BANNER\&block=storyline_menu_recirc\&action=click\&pgtype=Article\&impression_id=a10364b3-f4ba-11ea-a6df-cbaea4376604\&variant=undefined}{Portland
  Shooting}
\item
  \href{https://www.nytimes3xbfgragh.onion/2020/08/30/us/breonna-taylor-police-killing.html?name=styln-george-floyd\&region=TOP_BANNER\&block=storyline_menu_recirc\&action=click\&pgtype=Article\&impression_id=a10364b4-f4ba-11ea-a6df-cbaea4376604\&variant=undefined}{Breonna
  Taylor's Life and Death}
\end{itemize}

Advertisement

\protect\hyperlink{after-top}{Continue reading the main story}

Supported by

\protect\hyperlink{after-sponsor}{Continue reading the main story}

\hypertarget{kettling-of-peaceful-protesters-shows-aggressive-shift-by-ny-police}{%
\section{`Kettling' of Peaceful Protesters Shows Aggressive Shift by
N.Y.
Police}\label{kettling-of-peaceful-protesters-shows-aggressive-shift-by-ny-police}}

Officers have charged and swung batons at demonstrators after curfew
with seemingly little provocation. The mayor said he would review any
reports of inappropriate enforcement.

\includegraphics{https://static01.graylady3jvrrxbe.onion/images/2020/06/05/nyregion/05NYUNREST-POLICING-top/merlin_173179854_0538c71e-cfc6-4018-a7bc-79a7fdb444ea-articleLarge.jpg?quality=75\&auto=webp\&disable=upscale}

\href{https://www.nytimes3xbfgragh.onion/by/ali-watkins}{\includegraphics{https://static01.graylady3jvrrxbe.onion/images/2019/02/20/multimedia/author-ali-watkins/author-ali-watkins-thumbLarge.png}}

By \href{https://www.nytimes3xbfgragh.onion/by/ali-watkins}{Ali Watkins}

\begin{itemize}
\item
  Published June 5, 2020Updated July 17, 2020
\item
  \begin{itemize}
  \item
  \item
  \item
  \item
  \item
  \end{itemize}
\end{itemize}

It was about 45 minutes past New York City's 8 p.m. curfew on Wednesday
when a peaceful protest march encountered a line of riot police near
Cadman Plaza in Brooklyn.

Hundreds of demonstrators stopped and chanted for 10 minutes, arms
raised, until their leaders decided to turn the group around and leave
the area.

The protesters had not seen that
\href{https://www.nytimes3xbfgragh.onion/2020/07/17/us/portland-protests.html}{riot
police} had flooded the plaza behind them, boxing them in. The maneuver
was a law enforcement tactic called kettling. The police encircle
protesters so that they have no way to exit from a park, city block or
other public space, and then charge in and make
\href{https://www.nytimes3xbfgragh.onion/2020/07/17/us/portland-protests.html}{arrests}.

For the next 20 minutes in Downtown Brooklyn, officers swinging batons
turned a demonstration that had been largely peaceful into a scene of
chaos.

The kettling operations carried out by the police department after
curfew have become among the most unsettling symbols of its use of force
against peaceful protests, and
\href{https://www.nytimes3xbfgragh.onion/2020/06/04/nyregion/De-blasio-protests-curfew.html}{have
touched off a fierce backlash} against Mayor Bill de Blasio and the
police commissioner, Dermot F. Shea.

In the past several days, New York Times journalists covering the
protests have seen officers repeatedly charge at demonstrators after
curfew with seemingly little provocation, shoving them onto sidewalks,
striking them with batons and using other rough tactics.

\includegraphics{https://static01.graylady3jvrrxbe.onion/images/2020/06/05/nyregion/05NYUNREST-POLICING3/merlin_173178462_2eba4484-a65c-40ea-a8ec-0bb28e4f7e3f-articleLarge.jpg?quality=75\&auto=webp\&disable=upscale}

The escalation in the use of force in New York is part of a national
trend. Across the country,
\href{https://www.nytimes3xbfgragh.onion/2020/06/05/us/police-violence-george-floyd.html}{local
police have resorted to increasingly violent crowd control techniques}
to control the protests ignited by the death of George Floyd, a black
man, as he was being held down by a white officer in Minneapolis.

In Minneapolis, the police have used tear gas, rubber bullets and
projectiles to deter peaceful protesters and journalists. In Los
Angeles, the police were
\href{https://abc7.com/lapd-video-batons-protest/6231194/}{recorded}
using batons to strike demonstrators, and in
\href{https://whyy.org/articles/philly-police-say-tear-gas-used-because-676-protest-turned-hostile-but-theres-no-evidence-that-happened/}{Philadelphia},
police officers corralled and tear-gassed an entire crowd.

Two Buffalo officers were suspended after they were filmed by a local
news outlet shoving a 75-year-old protester to the ground. And in
Atlanta,
\href{https://abcnews.go.com/US/wireStory/officers-charged-college-students-pulled-car-71022535}{six
police officers were charged} after they were recorded pulling two
college students out of their car, and using a stun gun on them.

Several incidents are under investigation in New York, too, the
authorities said, including a moment when two police S.U.V.s
\href{https://twitter.com/pgarapon/status/1266885414016688134}{drove
forward into a crowd} that had been blocking them, knocking several
people to the ground.

The kettling strategy has been broadly defended by both Mr. de Blasio
and Mr. Shea, who said it was necessary escalation to deter looters who
\href{https://www.nytimes3xbfgragh.onion/2020/06/02/nyregion/nyc-looting-protests.html}{ransacked
parts of Manhattan} over the weekend. ``There comes a point where enough
is enough,'' Mr. de Blasio said on Thursday.

But there have been
\href{https://www.nytimes3xbfgragh.onion/2020/06/05/nyregion/nyc-protests-george-floyd.html}{few
reports of looting in the last three days of unrest}, and the police are
deploying their more aggressive tactics against protesters who have done
little beyond continuing to march after the city's 8 p.m. curfew. About
270 people were arrested on Thursday night.

The police department's crackdown suffered a blow on Friday from the
district attorneys in Manhattan and Brooklyn, who announced they would
not prosecute anyone arrested during the protests on low-level charges.

The Brooklyn district attorney, Eric Gonzalez, said he would not
prosecute those charged with violating curfew or unlawful assembly,
while Manhattan's prosecutor, Cyrus R. Vance Jr., said that ``in the
interest of justice'' he would decline to pursue convictions for
unlawful assembly and disorderly conduct.

``Our office has a moral imperative to enact public policies which
assure all New Yorkers that in our justice system and our society, black
lives matter and police violence is a crime,'' Mr. Vance said in a
statement. Mr. Gonzalez's said, ``We stand for the right of people to
protest.'' Both said they would continue to prosecute people accused of
violence against officers and of looting.

As images of police officers using force to arrest seemingly peaceful
demonstrators have circulated online, Mr. de Blasio, who ran on a
platform to reform the police, has come under sharp criticism from some
elected officials, community leaders and even his former aides. He was
jeered and booed at a memorial for Mr. Floyd on Thursday.

By Friday, after more than a week of protests, the mayor had softened
his tone, pledging to review reports of police officers behaving
inappropriately and promising he would announce disciplinary measures
against some officers shortly.

Later, in an interview on WNYC, the public radio station, the mayor said
that the encircling of protesters was sometimes necessary for public
safety, and that the police were charging into crowds only when their
commanders had evidence of imminent violence.

``I don't want to see protesters hemmed in if they don't need to be,''
he said, but he added ``that sometimes there's a legitimate problem and
it's not visible to protesters.''

On
\href{https://www.nbcnews.com/news/us-news/new-york-officers-could-face-suspension-after-street-clashes-commissioner-n1225391}{Thursday},
the police commissioner said he was reviewing at least seven videos that
showed potential police misconduct and promised he would hold the
officers accountable if the allegations were proven. On Friday night, he
announced the suspension of two officers: one who
\href{https://newyork.cbslocal.com/2020/06/03/woman-seen-on-video-shoved-to-ground-by-nypd-officer-demands-accountability/}{violently
pushed a woman} to the ground, and another who pulled down a protester's
face mask and
\href{https://www.tmz.com/2020/05/31/mask-cops-nypd-protester-george-floyd-death-police/}{then
pepper sprayed him}.

But Mr. Shea also stressed that some protesters had come to the
demonstrations with the intent to attack the police. He also said the
anti-police rhetoric of the demonstrators --- and some elected officials
--- was encouraging attacks on officers, several of whom had been
injured with sticks, or thrown bottles and bricks.

``We need healing,'' Mr. Shea said. ``We need dialogue. We need peace.''

For many protesters, however, the hard-nosed tactics the police have
employed to shut down marches after curfew have only exacerbated the
violence.

Axel Hernandez, 30, was protesting at Cadman Plaza on Wednesday night
when officers rushed into the crowd. Mr. Hernandez, who had marched
several times this week, said that up until that point it had been one
of the calmest demonstrations he had attended.

``That was the most peaceful, no bottles thrown, no anything,'' he said.
``The next thing I know, police rush in, with batons, and started moving
people, and start hitting people.''

Experts on crowd control say kettling is a technique the police have
used for decades, not just in New York City, but around the world,
including Northern Ireland. In theory, officers surround protesters,
cutting off exits until they tire, then let them disperse in small
groups.

But because demonstrators have nowhere to go, the maneuver often ends
with a charge and mass arrests. Since the city put a curfew in place
this week, the police have used the technique in
\href{https://www.nytimes3xbfgragh.onion/interactive/2020/06/04/burst/brooklyn-protesters-police-confrontation.html}{Brooklyn},
Manhattan, and the
\href{https://www.nytimes3xbfgragh.onion/2020/06/04/nyregion/nyc-protests-george-floyd.html}{Bronx}.

``Kettling is basically when you take the crowd and drive it into a box,
which is a great idea if you're wanting to capture people,'' said Dennis
Kenney, a criminal justice professor at John Jay College of Criminal
Justice. ``It's generally a way to greatly increase the likelihood of
conflict.''

Image

The police used bicycles to encircle a protest in the Bronx on
Thursday.Credit...Gabriela Bhaskar for The New York Times

On Thursday night, in the Bronx, rows of officers surrounded protesters
from all sides, pinning them in before running at them with batons and
striking several people. At least one was taken away in a stretcher.

Asked about the incident on WNYC, Mr. de Blasio said the police believed
some in the crowd had intended to be destructive.

``The groups organizing that event advertised their desire to do
violence and create violence,'' Mr. de Blasio said.

``If any protesters were there peacefully and not associated with that,
and they got hemmed in at all, that's something I don't accept that and
we have to fix,'' Mr. de Blasio said, promising a full review of the
incident.

Mr. Shea
\href{https://nypost.com/2020/06/05/nypd-commissioner-says-violent-nyc-protest-was-only-about-mayhem/}{said
Friday} that police officers recovered gasoline and weapons, including a
firearm, from the crowd.

Many demonstrators whom the police have trapped in kettle formations
have had no way to disperse before being arrested, witnesses and
protesters said. On Wednesday night, for instance, the police would not
let protesters encircled near Gracie Mansion, the mayor's official
residence in Manhattan, comply with an order to leave.

``We were asking them, `Where should we go?' Everyone's hands were in
the air,''' one of the protesters, Lucas Zwirner, said. Many
demonstrators told the police they would disperse and go home, Mr.
Zwirner said, but officers would not let them through.

Police officers have used the maneuver to end some marches, but not
others. At one demonstration n Brooklyn on Wednesday, the police waited
until 9 p.m. --- an hour beyond the 8 p.m. curfew --- to surround
protesters and charge.

The day before, they allowed thousands to march peacefully across the
Manhattan Bridge hours after curfew had ended, and escorted a group of
thousands back to Brooklyn before letting them disperse. It was a
different story in the Bronx on Thursday, when officers surrounded a
group of demonstrators and began making arrests just minutes after the 8
p.m. curfew.

``We are continuing to exercise discretion,'' Mr. Shea said Thursday
evening. ``Where we have made arrests, we have made them
strategically.''

Jan Ransom contributed reporting.

Advertisement

\protect\hyperlink{after-bottom}{Continue reading the main story}

\hypertarget{site-index}{%
\subsection{Site Index}\label{site-index}}

\hypertarget{site-information-navigation}{%
\subsection{Site Information
Navigation}\label{site-information-navigation}}

\begin{itemize}
\tightlist
\item
  \href{https://help.nytimes3xbfgragh.onion/hc/en-us/articles/115014792127-Copyright-notice}{©~2020~The
  New York Times Company}
\end{itemize}

\begin{itemize}
\tightlist
\item
  \href{https://www.nytco.com/}{NYTCo}
\item
  \href{https://help.nytimes3xbfgragh.onion/hc/en-us/articles/115015385887-Contact-Us}{Contact
  Us}
\item
  \href{https://www.nytco.com/careers/}{Work with us}
\item
  \href{https://nytmediakit.com/}{Advertise}
\item
  \href{http://www.tbrandstudio.com/}{T Brand Studio}
\item
  \href{https://www.nytimes3xbfgragh.onion/privacy/cookie-policy\#how-do-i-manage-trackers}{Your
  Ad Choices}
\item
  \href{https://www.nytimes3xbfgragh.onion/privacy}{Privacy}
\item
  \href{https://help.nytimes3xbfgragh.onion/hc/en-us/articles/115014893428-Terms-of-service}{Terms
  of Service}
\item
  \href{https://help.nytimes3xbfgragh.onion/hc/en-us/articles/115014893968-Terms-of-sale}{Terms
  of Sale}
\item
  \href{https://spiderbites.nytimes3xbfgragh.onion}{Site Map}
\item
  \href{https://help.nytimes3xbfgragh.onion/hc/en-us}{Help}
\item
  \href{https://www.nytimes3xbfgragh.onion/subscription?campaignId=37WXW}{Subscriptions}
\end{itemize}
