Sections

SEARCH

\protect\hyperlink{site-content}{Skip to
content}\protect\hyperlink{site-index}{Skip to site index}

\href{https://www.nytimes3xbfgragh.onion/section/sports/football}{Pro
Football}

\href{https://myaccount.nytimes3xbfgragh.onion/auth/login?response_type=cookie\&client_id=vi}{}

\href{https://www.nytimes3xbfgragh.onion/section/todayspaper}{Today's
Paper}

\href{/section/sports/football}{Pro Football}\textbar{}As Trump
Rekindles N.F.L. Fight, Goodell Sides With Players

\url{https://nyti.ms/3eS6JqD}

\begin{itemize}
\item
\item
\item
\item
\item
\end{itemize}

\hypertarget{race-and-america}{%
\subsubsection{\texorpdfstring{\href{https://www.nytimes3xbfgragh.onion/news-event/george-floyd-protests-minneapolis-new-york-los-angeles?name=styln-george-floyd\&region=TOP_BANNER\&block=storyline_menu_recirc\&action=click\&pgtype=Article\&impression_id=8c8f6a60-f1c2-11ea-86ae-c7e0aefa31f3\&variant=undefined}{Race
and America}}{Race and America}}\label{race-and-america}}

\begin{itemize}
\tightlist
\item
  \href{https://www.nytimes3xbfgragh.onion/2020/09/04/nyregion/rochester-police-daniel-prude.html?name=styln-george-floyd\&region=TOP_BANNER\&block=storyline_menu_recirc\&action=click\&pgtype=Article\&impression_id=8c8f6a61-f1c2-11ea-86ae-c7e0aefa31f3\&variant=undefined}{How
  Police Handled Death of Daniel Prude}
\item
  \href{https://www.nytimes3xbfgragh.onion/2020/09/01/us/politics/trump-fact-check-protests.html?name=styln-george-floyd\&region=TOP_BANNER\&block=storyline_menu_recirc\&action=click\&pgtype=Article\&impression_id=8c8f6a62-f1c2-11ea-86ae-c7e0aefa31f3\&variant=undefined}{Trump
  Fact Check}
\item
  \href{https://www.nytimes3xbfgragh.onion/2020/08/30/us/portland-shooting-explained.html?name=styln-george-floyd\&region=TOP_BANNER\&block=storyline_menu_recirc\&action=click\&pgtype=Article\&impression_id=8c8f6a63-f1c2-11ea-86ae-c7e0aefa31f3\&variant=undefined}{Portland
  Shooting}
\item
  \href{https://www.nytimes3xbfgragh.onion/2020/08/30/us/breonna-taylor-police-killing.html?name=styln-george-floyd\&region=TOP_BANNER\&block=storyline_menu_recirc\&action=click\&pgtype=Article\&impression_id=8c8f6a64-f1c2-11ea-86ae-c7e0aefa31f3\&variant=undefined}{Breonna
  Taylor's Life and Death}
\end{itemize}

Advertisement

\protect\hyperlink{after-top}{Continue reading the main story}

Supported by

\protect\hyperlink{after-sponsor}{Continue reading the main story}

\hypertarget{as-trump-rekindles-nfl-fight-goodell-sides-with-players}{%
\section{As Trump Rekindles N.F.L. Fight, Goodell Sides With
Players}\label{as-trump-rekindles-nfl-fight-goodell-sides-with-players}}

The president tweeted to say it was disrespectful to kneel during the
national anthem, as Colin Kaepernick and other players began doing in
2016 to protest racial injustice. N.F.L. Commissioner Roger Goodell then
said the league supported players peacefully protesting.

\includegraphics{https://static01.graylady3jvrrxbe.onion/images/2020/06/05/sports/05nfl-video/merlin_143690931_ff9da7e9-092e-4489-b2a1-2ad3ac103b32-articleLarge.jpg?quality=75\&auto=webp\&disable=upscale}

\href{https://www.nytimes3xbfgragh.onion/by/ken-belson}{\includegraphics{https://static01.graylady3jvrrxbe.onion/images/2018/02/16/multimedia/author-ken-belson/author-ken-belson-thumbLarge.jpg}}

By \href{https://www.nytimes3xbfgragh.onion/by/ken-belson}{Ken Belson}

\begin{itemize}
\item
  June 5, 2020
\item
  \begin{itemize}
  \item
  \item
  \item
  \item
  \item
  \end{itemize}
\end{itemize}

The protracted debate in the N.F.L. over
\href{https://www.nytimes3xbfgragh.onion/2019/02/15/sports/nfl-colin-kaepernick-protests-timeline.html}{players
protesting racial injustice during the national anthem} reignited with
force on Friday as President Trump rekindled his war with the league
over the issue and the league's commissioner, Roger Goodell, issued his
strongest support yet for the players seeking to fight racism and police
brutality.

In a swift response to a video montage featuring star players asking the
league to address systemic racism, Goodell said he apologized for not
listening to the concerns of African-American players earlier and said
he supports the players' right to protest peacefully. During the 2016
season,
\href{https://www.nytimes3xbfgragh.onion/2020/06/12/sports/football/nfl-protest-colin-kaepernick-activism.html}{Colin
Kaepernick} started the movement within the league when he knelt to call
attention to racial injustice and violence by police, but no team has
offered him a contract since then.

Goodell did not directly name Kaepernick in his video. Still, his
comments were diametrically opposed to those made by the president.
Trump defended New Orleans Saints quarterback Drew Brees, who said this
week that it was disrespectful to kneel during the pregame playing of
the ``The Star-Spangled Banner.''

After a swift rebuke from fellow N.F.L. players, including some of his
teammates,
\href{https://www.nytimes3xbfgragh.onion/2020/06/04/sports/football/drew-brees-apology-comments.html}{Brees
apologized on Thursday} and then, in a social media post Friday night
that was addressed to Trump, issued a more forceful repudiation of his
remark about disrespect.
\href{https://www.instagram.com/p/CBE4y_9Hj2S/?igshid=7ejunv7ktpcn}{The
president had already said on Twitter} that Brees should not have bowed
to pressure and that everyone should stand when the national anthem is
played.

``We should be standing up straight and tall, ideally with a salute, or
a hand on heart,'' the president wrote. ``There are other things you can
protest, but not our Great American Flag --- NO KNEELING!''

Trump first attacked the N.F.L. over
\href{https://www.nytimes3xbfgragh.onion/2017/09/23/sports/football/trump-nfl-kaepernick.html}{protests
during the national anthem in September 2017}. During a campaign rally,
he called on owners to fire any players who knelt during the anthem, and
used a vulgarity to describe quarterback Colin Kaepernick, who started
the movement when he knelt through the previous season to call attention
to racial injustice and police brutality.

Kaepernick adopted the kneeling gesture on the advice of a former Green
Beret he had met, who suggested it would be a respectful way to call
attention to his cause.

A spokesman for the N.F.L. declined to comment on the president's
comments about Brees and the national anthem.

The president's admonishment comes as the N.F.L., like the rest of the
country, grapples with how to respond to the killings of black Americans
at the hands of police, and to the protests that have engulfed the
nation for nearly two weeks since the death of George Floyd in
Minneapolis.

More than any other major sports league, the N.F.L. has wrestled in
recent years with the issue of race, the lack of African-Americans and
other people of color in positions of power in the league and the rights
of players to protest social issues on the field. While three-quarters
of the league's players are African-American, nearly every owner is
white and several of the most prominent owners are strong supporters of
the president.

In this latest wave of civil unrest,
\href{https://www.nytimes3xbfgragh.onion/2020/05/31/sports/football/colin-kaepernick-george-floyd.html}{many
players, coaches and owners have spoken out} against racism, and have
pledged to become more involved in finding solutions.

Last Saturday, Roger Goodell was the first big league commissioner to
issue a statement of concern in response to Floyd's death, but his words
were panned as hypocritical because of the league owners' rejection of
Kaepernick, who has not found another job in the league since the end of
the 2016 season.

On Thursday, some of the league's biggest stars, including Kansas City
Chiefs quarterback Patrick Mahomes,
\href{https://twitter.com/Cantguardmike/status/1268712743860928513}{released
a video calling on the league} to condemn the oppression of black people
and to apologize for not supporting players who protested peacefully.

\href{https://www.instagram.com/p/CBE4y_9Hj2S/?igshid=7ejunv7ktpcn}{In
his Instagram} post directed at the president, Brees wrote that after
conversations with many people, he understood that ``this is not an
issue about the American flag. It has never been. We can no longer use
the flag to turn people away or distract them from the real issues that
face our black communities.''

On Friday, \href{https://twitter.com/NFL/status/1269026096034398208}{the
N.F.L. also tweeted} the video that the players made and said,
``Players, we hear you.''

Soon after,
\href{https://twitter.com/NFL/status/1269034074552721408}{Goodell
responded with his own video} in which he made his strongest and most
specific support of the demands and goals of African-American players.
In the one minute, 21-second video, the commissioner condemned the
oppression of black people, apologized for not listening to the concerns
of African-American players and encouraged the league's athletes to
protest peacefully.

After offering his condolences to ``the families who have endured police
brutality,'' Goodell said that ``We, the National Football League,
condemn racism and the systematic oppression of black people.''

In an apparent allusion to Kaepernick, who
\href{https://www.nytimes3xbfgragh.onion/2019/03/21/sports/colin-kaepernick-nfl-settlement.html}{settled
a grievance with the league last year} in which he accused the league of
blackballing him because of his political protests, Goodell added: ``We,
the National Football League, admit we were wrong for not listening to
N.F.L. players earlier and encourage all to speak out and peacefully
protest.''

``We, the National Football League, believe black lives matter,'' he
added. ``I personally protest with you and want to be a part of the
much-needed change in this country. Without black players, there would
be no National Football League.''

In earlier statements,
\href{https://www.instagram.com/p/CBB6ewBA-aO/?igshid=1w2oif5h7qxgi}{Goodell
has pointed to the league's social activism campaign}, Inspire Change,
which has donated tens of millions of dollars to groups working in
communities and pushing for the eradication of social injustice.

But given the political volatility of the issue of the national anthem,
Goodell's statement in support of peaceful protests could inflame the
relationship between the league and the president, who has used the
issue of protests during the national anthem to galvanize his
supporters.

After the president first criticized the N.F.L. for not cracking down on
protesters, owners voted to tighten the league's policy to prohibit
players from kneeling during the national anthem. After the
\href{https://www.nytimes3xbfgragh.onion/2018/07/10/sports/nfl-anthem.html}{N.F.L.
Players Association filed a grievance} to reverse the policy, the league
backed off and has never penalized a player for protesting.

Now, Goodell has spoken in support of the players' right to protest and
many more players have publicly called for the need for action against
racism and police brutality. Some players,
\href{https://profootballtalk.nbcsports.com/2020/06/05/adrian-peterson-will-without-a-doubt-kneel-during-national-anthem/}{including
running back Adrian Peterson}, have already said they intend to kneel
during the national anthem this coming season, which does not begin
until September.

While the commissioner has pledged to listen and ``move forward together
for a better and more united N.F.L. family,'' he will have to convince
owners, broadcasters, sponsors and fans who are uncomfortable with
player protests, experts said.

``What he needs to articulate to people who buy commercials and own
teams and anyone who might push back against the players is, are there
things they are doing that are negative?'' said Charles K. Ross, the
author of ``Outside the Lines: African Americans and the Integration of
the N.F.L.'' ``You can also stand up to individuals who are going to
push back and remind them we have the First Amendment.''

Advertisement

\protect\hyperlink{after-bottom}{Continue reading the main story}

\hypertarget{site-index}{%
\subsection{Site Index}\label{site-index}}

\hypertarget{site-information-navigation}{%
\subsection{Site Information
Navigation}\label{site-information-navigation}}

\begin{itemize}
\tightlist
\item
  \href{https://help.nytimes3xbfgragh.onion/hc/en-us/articles/115014792127-Copyright-notice}{©~2020~The
  New York Times Company}
\end{itemize}

\begin{itemize}
\tightlist
\item
  \href{https://www.nytco.com/}{NYTCo}
\item
  \href{https://help.nytimes3xbfgragh.onion/hc/en-us/articles/115015385887-Contact-Us}{Contact
  Us}
\item
  \href{https://www.nytco.com/careers/}{Work with us}
\item
  \href{https://nytmediakit.com/}{Advertise}
\item
  \href{http://www.tbrandstudio.com/}{T Brand Studio}
\item
  \href{https://www.nytimes3xbfgragh.onion/privacy/cookie-policy\#how-do-i-manage-trackers}{Your
  Ad Choices}
\item
  \href{https://www.nytimes3xbfgragh.onion/privacy}{Privacy}
\item
  \href{https://help.nytimes3xbfgragh.onion/hc/en-us/articles/115014893428-Terms-of-service}{Terms
  of Service}
\item
  \href{https://help.nytimes3xbfgragh.onion/hc/en-us/articles/115014893968-Terms-of-sale}{Terms
  of Sale}
\item
  \href{https://spiderbites.nytimes3xbfgragh.onion}{Site Map}
\item
  \href{https://help.nytimes3xbfgragh.onion/hc/en-us}{Help}
\item
  \href{https://www.nytimes3xbfgragh.onion/subscription?campaignId=37WXW}{Subscriptions}
\end{itemize}
