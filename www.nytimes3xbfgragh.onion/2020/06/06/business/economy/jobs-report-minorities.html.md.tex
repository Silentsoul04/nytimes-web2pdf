Sections

SEARCH

\protect\hyperlink{site-content}{Skip to
content}\protect\hyperlink{site-index}{Skip to site index}

\href{https://www.nytimes3xbfgragh.onion/section/business/economy}{Economy}

\href{https://myaccount.nytimes3xbfgragh.onion/auth/login?response_type=cookie\&client_id=vi}{}

\href{https://www.nytimes3xbfgragh.onion/section/todayspaper}{Today's
Paper}

\href{/section/business/economy}{Economy}\textbar{}Minority Workers Who
Lagged in a Boom Are Hit Hard in a Bust

\url{https://nyti.ms/2YaPUAm}

\begin{itemize}
\item
\item
\item
\item
\item
\end{itemize}

\hypertarget{the-coronavirus-outbreak}{%
\subsubsection{\texorpdfstring{\href{https://www.nytimes3xbfgragh.onion/news-event/coronavirus?name=styln-coronavirus-markets\&region=TOP_BANNER\&block=storyline_menu_recirc\&action=click\&pgtype=Article\&impression_id=167365d0-f4b9-11ea-8ebe-179a99a195cb\&variant=undefined}{The
Coronavirus
Outbreak}}{The Coronavirus Outbreak}}\label{the-coronavirus-outbreak}}

\begin{itemize}
\tightlist
\item
  live\href{https://www.nytimes3xbfgragh.onion/2020/09/11/world/covid-19-coronavirus.html?name=styln-coronavirus-markets\&region=TOP_BANNER\&block=storyline_menu_recirc\&action=click\&pgtype=Article\&impression_id=16738ce0-f4b9-11ea-8ebe-179a99a195cb\&variant=undefined}{Latest
  Updates}
\item
  \href{https://www.nytimes3xbfgragh.onion/interactive/2020/us/coronavirus-us-cases.html?name=styln-coronavirus-markets\&region=TOP_BANNER\&block=storyline_menu_recirc\&action=click\&pgtype=Article\&impression_id=16738ce1-f4b9-11ea-8ebe-179a99a195cb\&variant=undefined}{Maps
  and Cases}
\item
  \href{https://www.nytimes3xbfgragh.onion/interactive/2020/science/coronavirus-vaccine-tracker.html?name=styln-coronavirus-markets\&region=TOP_BANNER\&block=storyline_menu_recirc\&action=click\&pgtype=Article\&impression_id=16738ce2-f4b9-11ea-8ebe-179a99a195cb\&variant=undefined}{Vaccine
  Tracker}
\item
  \href{https://www.nytimes3xbfgragh.onion/2020/09/10/us/politics/fda-coronavirus-vaccine.html?name=styln-coronavirus-markets\&region=TOP_BANNER\&block=storyline_menu_recirc\&action=click\&pgtype=Article\&impression_id=16738ce3-f4b9-11ea-8ebe-179a99a195cb\&variant=undefined}{F.D.A.
  Regulators' Self-Defense}
\item
  \href{https://www.nytimes3xbfgragh.onion/2020/09/09/upshot/coronavirus-surprise-test-fees.html?name=styln-coronavirus-markets\&region=TOP_BANNER\&block=storyline_menu_recirc\&action=click\&pgtype=Article\&impression_id=16738ce4-f4b9-11ea-8ebe-179a99a195cb\&variant=undefined}{Surprise
  Test Fees}
\end{itemize}

Advertisement

\protect\hyperlink{after-top}{Continue reading the main story}

Supported by

\protect\hyperlink{after-sponsor}{Continue reading the main story}

\hypertarget{minority-workers-who-lagged-in-a-boom-are-hit-hard-in-a-bust}{%
\section{Minority Workers Who Lagged in a Boom Are Hit Hard in a
Bust}\label{minority-workers-who-lagged-in-a-boom-are-hit-hard-in-a-bust}}

African-Americans and Latinos are especially vulnerable to job losses in
the pandemic and at a disadvantage in getting government support.

\includegraphics{https://static01.graylady3jvrrxbe.onion/images/2020/06/05/business/05virus-minorityjobs/merlin_171550962_41a72fe2-3ced-4e0f-a132-c6acced0dc2f-articleLarge.jpg?quality=75\&auto=webp\&disable=upscale}

\href{https://www.nytimes3xbfgragh.onion/by/patricia-cohen}{\includegraphics{https://static01.graylady3jvrrxbe.onion/images/2018/02/16/multimedia/author-patricia-cohen/author-patricia-cohen-thumbLarge.jpg}}\href{https://www.nytimes3xbfgragh.onion/by/ben-casselman}{\includegraphics{https://static01.graylady3jvrrxbe.onion/images/2018/11/09/multimedia/author-ben-casselman/author-ben-casselman-thumbLarge.png}}

By \href{https://www.nytimes3xbfgragh.onion/by/patricia-cohen}{Patricia
Cohen} and
\href{https://www.nytimes3xbfgragh.onion/by/ben-casselman}{Ben
Casselman}

\begin{itemize}
\item
  June 6, 2020
\item
  \begin{itemize}
  \item
  \item
  \item
  \item
  \item
  \end{itemize}
\end{itemize}

When Illinois shut down businesses to slow the spread of the coronavirus
in March and the state's unemployment system jammed from the overload,
Bridget Altenburg, chief executive of a Chicago-based nonprofit group,
visited one of the organization's work force centers. Two things stood
out: the sheer number of people lined up to apply for unemployment
benefits, and how few faces were white.

``The thing that struck me was how un-diverse it was,'' Ms. Altenburg
said. ``All people of color. Latino, African-American, and the stories I
heard were just gut wrenching. People went to work Monday morning and
the doors were closed and they were told to go get unemployment.''

Black Americans have always had a more difficult time in the job market.
The latest evidence arrived Friday when the government reported that
\href{https://www.nytimes3xbfgragh.onion/2020/06/05/business/economy/jobs-report.html}{21
million Americans were unemployed} in May. Though the jobless rate for
whites dipped, to 12.4 percent, the rate for African-Americans inched up
to 16.8 percent, meaning that nearly 1.4 million black men and nearly
1.7 million black women were part of the labor force but without work.
The Hispanic jobless rate improved from April but was 17.6 percent.

\hypertarget{unemployment-rates-in-may-2020-by-gender-race-and-ethnicity}{%
\subsubsection{Unemployment rates in May 2020, by gender, race and
ethnicity}\label{unemployment-rates-in-may-2020-by-gender-race-and-ethnicity}}

\hypertarget{white-men-were-among-the-groups-with-lower-unemployment-than-the-national-rate-while-hispanic-women-and-others-had-notably-higher-unemployment}{%
\paragraph{White men were among the groups with lower unemployment than
the national rate, while Hispanic women and others had notably higher
unemployment.}\label{white-men-were-among-the-groups-with-lower-unemployment-than-the-national-rate-while-hispanic-women-and-others-had-notably-higher-unemployment}}

By Ella Koeze·Source: Bureau of Labor Statistics

Hiring prospects for African-American and Latino workers have long been
hobbled by factors that stretch from poorer educational options and
lopsided incarceration rates to outright discrimination by employers.

Even last year, as the national jobless rate fell below 4 percent to its
lowest level in half a century, the
\href{https://www.bls.gov/lau/ptable14full2019.htm}{rate for black men}
in Illinois was nearly 10 percent. African-Americans also earn less, are
quicker to be laid off, are slower to be rehired and are less likely to
be promoted. Historically, the black unemployment rate is twice that of
whites.

Even before the pandemic, most clients at Ms. Altenburg's group, the
National Able Network, were black or Latino. ``It doesn't surprise me,''
she said of the disparities she witnessed during a recent visit to
another work force center, in Omaha. ``But it makes me angry, and it
makes me tired.''

As Jerome H. Powell, chair of the Federal Reserve, explained at a news
conference in April, ``Unemployment has tended to go up much faster for
minorities, and for others who tend to be at the low end of the income
spectrum.'' The coronavirus pandemic has only amplified the problem.

``Everyone is suffering here,'' Mr. Powell added. ``But I think those
who are least able to bear it are the ones who are losing their jobs,
and losing their incomes and have little cushion to protect them in
times like that.''

The current economic crisis has struck black and Latino families
particularly hard in several ways. They are more likely to work in the
service industries that were the first to be hit by layoffs, and less
likely to work in white-collar jobs that can be done safely from home.
They have, on average, significantly less in savings to help them
weather a period of unemployment, and are less likely to have families
with the resources to help out.

\hypertarget{latest-updates-the-coronavirus-outbreak-and-the-economy}{%
\section{\texorpdfstring{\href{https://www.nytimes3xbfgragh.onion/live/2020/09/11/business/stock-market-today-coronavirus?action=click\&pgtype=Article\&state=default\&region=MAIN_CONTENT_1\&context=storylines_live_updates}{Latest
Updates: The Coronavirus Outbreak and the
Economy}}{Latest Updates: The Coronavirus Outbreak and the Economy}}\label{latest-updates-the-coronavirus-outbreak-and-the-economy}}

\href{https://www.nytimes3xbfgragh.onion/live/2020/09/11/business/stock-market-today-coronavirus?action=click\&pgtype=Article\&state=default\&region=MAIN_CONTENT_1\&context=storylines_live_updates\#the-nyse-may-move-its-data-center-out-of-new-jersey-in-response-to-a-proposed-tax}{9h
ago}

\href{https://www.nytimes3xbfgragh.onion/live/2020/09/11/business/stock-market-today-coronavirus?action=click\&pgtype=Article\&state=default\&region=MAIN_CONTENT_1\&context=storylines_live_updates\#the-nyse-may-move-its-data-center-out-of-new-jersey-in-response-to-a-proposed-tax}{The
N.Y.S.E. may move its data center out of New Jersey in response to a
proposed tax.}

\href{https://www.nytimes3xbfgragh.onion/live/2020/09/11/business/stock-market-today-coronavirus?action=click\&pgtype=Article\&state=default\&region=MAIN_CONTENT_1\&context=storylines_live_updates\#the-federal-budget-deficit-hit-3-trillion-as-of-august}{11h
ago}

\href{https://www.nytimes3xbfgragh.onion/live/2020/09/11/business/stock-market-today-coronavirus?action=click\&pgtype=Article\&state=default\&region=MAIN_CONTENT_1\&context=storylines_live_updates\#the-federal-budget-deficit-hit-3-trillion-as-of-august}{The
federal budget deficit hit \$3 trillion as of August.}

\href{https://www.nytimes3xbfgragh.onion/live/2020/09/11/business/stock-market-today-coronavirus?action=click\&pgtype=Article\&state=default\&region=MAIN_CONTENT_1\&context=storylines_live_updates\#warner-bros-pushes-the-release-of-wonder-woman-1984-to-christmas}{12h
ago}

\href{https://www.nytimes3xbfgragh.onion/live/2020/09/11/business/stock-market-today-coronavirus?action=click\&pgtype=Article\&state=default\&region=MAIN_CONTENT_1\&context=storylines_live_updates\#warner-bros-pushes-the-release-of-wonder-woman-1984-to-christmas}{Warner
Bros. pushes the release of `Wonder Woman 1984' to Christmas.}

\href{https://www.nytimes3xbfgragh.onion/live/2020/09/11/business/stock-market-today-coronavirus?action=click\&pgtype=Article\&state=default\&region=MAIN_CONTENT_1\&context=storylines_live_updates}{See
more updates}

More live coverage:
\href{https://www.nytimes3xbfgragh.onion/2020/09/11/world/covid-19-coronavirus.html?action=click\&pgtype=Article\&state=default\&region=MAIN_CONTENT_1\&context=storylines_live_updates}{Global}

Since the pandemic, fewer than half the blacks who are 16 and older have
a job. Latino unemployment rates are higher than any other racial or
ethnic group.

Minorities also had a harder time taking advantage of government support
efforts --- less likely to have computers to file for unemployment
benefits and less likely to have bank accounts, slowing the time it took
to receive government stimulus checks or making it harder for
small-business owners to apply for emergency loans.

\includegraphics{https://static01.graylady3jvrrxbe.onion/images/2020/06/05/business/05virus-minorityworkers2/merlin_172429950_a1df1197-da15-468c-af2e-77c121654ec2-articleLarge.jpg?quality=75\&auto=webp\&disable=upscale}

``Stark inequalities that existed and were exacerbated by the Great
Recession have been further exacerbated by the pandemic,'' said Ray
Boshara, director of the Center for Household Financial Stability at the
Federal Reserve Bank of St. Louis. ``The level of financial fragility is
much higher.''

Different challenges face those who have hung on to their jobs as part
of the nation's essential work force or in frontline occupations in
health care and social services, at grocery and drugstores, in public
transit and trucking, and in warehouses and cleaning services.

Minority women are more likely than any other group to be part of the
large underpaid work force that has been deemed necessary to keep the
country cared for and fed.

Still, the lives of these workers are insufficiently valued and
appreciated, said Rhonda Vonshay Sharpe, an economist and the president
of the Women's Institute for Science, Equity and Race in Mechanicsville,
Va. ``It's not the workers who are essential --- it's the jobs that are
essential,'' she said, pointing to the long delays in getting proper
protective equipment and taking other lifesaving measures.

``It suggests that the workers are expendable,'' Ms. Sharpe said. ``What
we're more concerned about is that the job is getting done.''

That partly explains why black Americans have suffered a
disproportionate share of coronavirus deaths.

``One of the reasons that African-Americans and Latinos are more
affected is we are in those jobs,'' said Stephanie James, who had been
taking care of a woman with dementia. ``We are the bus drivers, we are
the people who pick up your groceries, we are the people who work in the
stores, we are all of those folks.''

Ms. James, who lives in a suburb of Washington, is now out of a job. So
are two of her three siblings and many of her neighbors. She has
underlying health issues, and nearly all of the available jobs seem too
risky.

Image

``We are the bus drivers, we are the people who pick up your groceries,
we are the people who work in the stores,'' Stephanie James said. She
lost her job as a caregiver.Credit...Ting Shen for The New York Times

``I am scared to death of coming back to work,'' she said. ``I don't
think I should have to make the choice between having a livelihood and
having a life.''

\emph{{[}How do you feel about going back to work?}
\href{https://www.nytimes3xbfgragh.onion/2020/05/15/us/as-cities-reopen-how-do-you-feel-about-going-out.html}{\emph{Share
your story}}\emph{.{]}}

Ms. James knows that a spate of joblessness, especially during an
economic downturn, can have a lifelong impact. She spent 13 years
working for a government contractor, rising up the ranks, before losing
her job in 2010. Ms. James was unemployed for six months before she took
a job at a grocery store to get by. She eventually got back into her
field, but has not found the kind of steady work she enjoyed before the
last recession.

The pattern is familiar --- blacks tend to be out of a job longer than
whites.

``What we saw with the Great Recession was that it took much longer for
black and Latino workers and black and Latino households to recover from
that recession,'' said Valerie Wilson, an economist at the left-leaning
Economic Policy Institute who was a co-author of a
\href{https://www.epi.org/publication/black-workers-covid/}{recent
report} on the impact of the virus on black workers. ``And in fact some
would argue that we didn't see a recovery for those communities until
the last three years.''

Owning a business or being self-employed has not insulated
African-Americans from the pandemic's economic fallout, because they are
\href{https://socialequity.duke.edu/wp-content/uploads/2019/10/Entering-Entrepreneurship.pdf}{often
concentrated in personal service} activities, running barbershops and
beauty shops that have had to close so as not to become sources of
infection.

The next wave of the crisis could hit one of the underpinnings of the
black middle class:
\href{https://www.nytimes3xbfgragh.onion/2018/04/22/business/economy/public-employees.html}{state
and local government jobs}. Even as other sectors recorded some gains
last month, an additional 571,000 state and local government employees,
many of them teachers, lost their jobs.

In April, there were nearly a million job losses, and economists say
many more are expected as the collapse in tax revenue ripples through
statehouses and city halls.

African-Americans --- particularly women --- are disproportionately
employed in those positions, said Christian E. Weller, an economist at
the University of Massachusetts, Boston, who wrote a
\href{https://www.americanprogress.org/issues/economy/reports/2019/12/05/478150/african-americans-face-systematic-obstacles-getting-good-jobs/}{report}
on the systemic obstacles facing black job seekers for the Center for
American Progress.

``You don't get rich, but these are stable jobs with good benefits,'' he
said, ``and there isn't anything comparable.''

The loss of a job is particularly devastating for black and Hispanic
workers because a paycheck is often the sole lifeline. Even those with a
comfortable income may have little to fall back on. For every \$1 of
wealth that a white household has, a black one has 10 cents.

``In more normal times, blacks who are working full time
\href{https://socialequity.duke.edu/wp-content/uploads/2020/01/what-we-get-wrong.pdf}{have
a lower median level of wealth} than whites who are unemployed,'' said
William A. Darity Jr., a public policy professor at Duke University.
``And blacks who have a college degree who are heads of households have
a median net worth about two-thirds of white heads of households who
never finished high school.''

At this point, most employers and employees are assuming that jobs will
return. But there are signs that many layoffs will be permanent.

Image

Freddy Wiggins expected to go back to his job at Neiman Marcus. Then the
retailer went into bankruptcy.Credit...Nate Palmer for The New York
Times

Freddy Wiggins was laid off from his job as a customer service
representative at Neiman Marcus in Washington during the first week in
April. ``The assumption was that once this is over, we'll go back to
business as usual,'' he said, adding that he was paid his final weeks of
salary and any owed vacation time and sick leave.

A month later, the retailer
\href{https://www.nytimes3xbfgragh.onion/2020/05/07/business/neiman-marcus-bankruptcy.html}{filed
for bankruptcy protection}. He got a form letter soon after, explaining
the bankruptcy process, but he doesn't know what it means for his job.
``I have no clue,'' he said. ``I haven't heard anything from them.''

``It's one of the scary things about this whole situation,'' he added.
``When it's over, you still don't know if things will fall into place.''

Advertisement

\protect\hyperlink{after-bottom}{Continue reading the main story}

\hypertarget{site-index}{%
\subsection{Site Index}\label{site-index}}

\hypertarget{site-information-navigation}{%
\subsection{Site Information
Navigation}\label{site-information-navigation}}

\begin{itemize}
\tightlist
\item
  \href{https://help.nytimes3xbfgragh.onion/hc/en-us/articles/115014792127-Copyright-notice}{©~2020~The
  New York Times Company}
\end{itemize}

\begin{itemize}
\tightlist
\item
  \href{https://www.nytco.com/}{NYTCo}
\item
  \href{https://help.nytimes3xbfgragh.onion/hc/en-us/articles/115015385887-Contact-Us}{Contact
  Us}
\item
  \href{https://www.nytco.com/careers/}{Work with us}
\item
  \href{https://nytmediakit.com/}{Advertise}
\item
  \href{http://www.tbrandstudio.com/}{T Brand Studio}
\item
  \href{https://www.nytimes3xbfgragh.onion/privacy/cookie-policy\#how-do-i-manage-trackers}{Your
  Ad Choices}
\item
  \href{https://www.nytimes3xbfgragh.onion/privacy}{Privacy}
\item
  \href{https://help.nytimes3xbfgragh.onion/hc/en-us/articles/115014893428-Terms-of-service}{Terms
  of Service}
\item
  \href{https://help.nytimes3xbfgragh.onion/hc/en-us/articles/115014893968-Terms-of-sale}{Terms
  of Sale}
\item
  \href{https://spiderbites.nytimes3xbfgragh.onion}{Site Map}
\item
  \href{https://help.nytimes3xbfgragh.onion/hc/en-us}{Help}
\item
  \href{https://www.nytimes3xbfgragh.onion/subscription?campaignId=37WXW}{Subscriptions}
\end{itemize}
