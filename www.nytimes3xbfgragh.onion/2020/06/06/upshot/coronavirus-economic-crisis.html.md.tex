Sections

SEARCH

\protect\hyperlink{site-content}{Skip to
content}\protect\hyperlink{site-index}{Skip to site index}

\href{https://myaccount.nytimes3xbfgragh.onion/auth/login?response_type=cookie\&client_id=vi}{}

\href{https://www.nytimes3xbfgragh.onion/section/todayspaper}{Today's
Paper}

\href{/section/upshot}{The Upshot}\textbar{}Don't Lose the Thread. The
Economy Is Experiencing an Epic Collapse of Demand.

\url{https://nyti.ms/2Y6Cw0f}

\begin{itemize}
\item
\item
\item
\item
\item
\item
\end{itemize}

\hypertarget{the-coronavirus-outbreak}{%
\subsubsection{\texorpdfstring{\href{https://www.nytimes3xbfgragh.onion/news-event/coronavirus?name=styln-coronavirus-national\&region=TOP_BANNER\&block=storyline_menu_recirc\&action=click\&pgtype=Article\&impression_id=7c275de0-f4c9-11ea-a65e-239a33d4e80b\&variant=undefined}{The
Coronavirus
Outbreak}}{The Coronavirus Outbreak}}\label{the-coronavirus-outbreak}}

\begin{itemize}
\tightlist
\item
  live\href{https://www.nytimes3xbfgragh.onion/2020/09/11/world/covid-19-coronavirus.html?name=styln-coronavirus-national\&region=TOP_BANNER\&block=storyline_menu_recirc\&action=click\&pgtype=Article\&impression_id=7c275de1-f4c9-11ea-a65e-239a33d4e80b\&variant=undefined}{Latest
  Updates}
\item
  \href{https://www.nytimes3xbfgragh.onion/interactive/2020/us/coronavirus-us-cases.html?name=styln-coronavirus-national\&region=TOP_BANNER\&block=storyline_menu_recirc\&action=click\&pgtype=Article\&impression_id=7c2784f0-f4c9-11ea-a65e-239a33d4e80b\&variant=undefined}{Maps
  and Cases}
\item
  \href{https://www.nytimes3xbfgragh.onion/interactive/2020/science/coronavirus-vaccine-tracker.html?name=styln-coronavirus-national\&region=TOP_BANNER\&block=storyline_menu_recirc\&action=click\&pgtype=Article\&impression_id=7c2784f1-f4c9-11ea-a65e-239a33d4e80b\&variant=undefined}{Vaccine
  Tracker}
\item
  \href{https://www.nytimes3xbfgragh.onion/2020/09/10/us/politics/fda-coronavirus-vaccine.html?name=styln-coronavirus-national\&region=TOP_BANNER\&block=storyline_menu_recirc\&action=click\&pgtype=Article\&impression_id=7c2784f2-f4c9-11ea-a65e-239a33d4e80b\&variant=undefined}{F.D.A.
  Regulators' Self-Defense}
\item
  \href{https://www.nytimes3xbfgragh.onion/2020/09/09/upshot/coronavirus-surprise-test-fees.html?name=styln-coronavirus-national\&region=TOP_BANNER\&block=storyline_menu_recirc\&action=click\&pgtype=Article\&impression_id=7c2784f3-f4c9-11ea-a65e-239a33d4e80b\&variant=undefined}{Surprise
  Test Fees}
\end{itemize}

Advertisement

\protect\hyperlink{after-top}{Continue reading the main story}

Upshot

Supported by

\protect\hyperlink{after-sponsor}{Continue reading the main story}

\hypertarget{dont-lose-the-thread-the-economy-is-experiencing-an-epic-collapse-of-demand}{%
\section{Don't Lose the Thread. The Economy Is Experiencing an Epic
Collapse of
Demand.}\label{dont-lose-the-thread-the-economy-is-experiencing-an-epic-collapse-of-demand}}

A rip in the fabric of the economy won't be healed easily, and denial of
the severity of the crisis won't solve it.

\includegraphics{https://static01.graylady3jvrrxbe.onion/images/2020/06/07/business/07Up-thread/07Up-thread-articleLarge.jpg?quality=75\&auto=webp\&disable=upscale}

\href{https://www.nytimes3xbfgragh.onion/by/neil-irwin}{\includegraphics{https://static01.graylady3jvrrxbe.onion/images/2016/11/15/upshot/neil-irwin/neil-irwin-thumbLarge-v3.jpg}}

By \href{https://www.nytimes3xbfgragh.onion/by/neil-irwin}{Neil Irwin}

\begin{itemize}
\item
  Published June 6, 2020Updated June 7, 2020
\item
  \begin{itemize}
  \item
  \item
  \item
  \item
  \item
  \item
  \end{itemize}
\end{itemize}

Despite it all --- a nation on edge, with an untamed pandemic and
convulsive protests over police brutality --- for the first time in
three months there is a scent of economic optimism in the air.

Employers added millions of jobs to their payrolls in May, and the
jobless rate fell, a big surprise to forecasters who expected further
losses. Businesses are reopening, and the rate of coronavirus deaths has
edged down. The Trump administration has begun pointing to what are
likely to be impressive growth numbers as the
\href{https://www.nytimes3xbfgragh.onion/2020/07/15/business/economy/economic-recovery-coronavirus-resurgence.html}{economy}
starts to pull out of its deep hole.

All of that is good news, and far better than the alternative of a
continuing collapse in
\href{https://www.nytimes3xbfgragh.onion/2020/07/15/business/economy/economic-recovery-coronavirus-resurgence.html}{economic}
activity. But it also creates a risk: distraction and complacency.

You can already sense in the public debate over the economy that people
are starting to lose the thread --- viewing the slight rebound from epic
collapse as a sign that a crisis has been averted. That certainly is the
kind of optimism evident in the stock market, which is now down a mere
1.1 percent for the year.

But there are clear signs that the collapse of economic activity has set
in motion problems that will play out over many months, or maybe many
years. If not contained, they could cause human misery on a mass scale
and create lasting scars for families.

The fabric of the economy has been ripped, with damage done to millions
of interconnections --- between workers and employers, companies and
their suppliers, borrowers and lenders. Both the historical evidence
from severe economic crises and the data available today point to
enormous delayed effects.

``There's a lot of denial here, as there was in the 1930s,'' said Eric
Rauchway, a historian at the University of California, Davis, who has
written extensively about the Great Depression. ``At the beginning of
the Depression, nobody wanted to admit that it was a crisis. The actions
the government took were not adequate to the scope of the problem, yet
they were very quick to say there had been a turnaround.''

Though it may not attract the attention that reopening beaches and a
soaring stock market might, the evidence is everywhere if you look
closely.

Consider those seemingly great new employment numbers. It is clear that
many workers who were temporarily laid off in March and April returned
to work in May, such as employees at once-closed restaurants that opened
up, or construction workers who returned to job sites.

But it still left the economy with 19.55 million fewer jobs than existed
in February. And the rebound came in part thanks to more than \$500
billion in federal aid to small businesses offered on the condition that
workers be retained, under the Paycheck Protection Program.

\hypertarget{latest-updates-the-coronavirus-outbreak}{%
\section{\texorpdfstring{\href{https://www.nytimes3xbfgragh.onion/2020/09/11/world/covid-19-coronavirus.html?action=click\&pgtype=Article\&state=default\&region=MAIN_CONTENT_1\&context=storylines_live_updates}{Latest
Updates: The Coronavirus
Outbreak}}{Latest Updates: The Coronavirus Outbreak}}\label{latest-updates-the-coronavirus-outbreak}}

Updated 2020-09-12T07:09:04.082Z

\begin{itemize}
\tightlist
\item
  \href{https://www.nytimes3xbfgragh.onion/2020/09/11/world/covid-19-coronavirus.html?action=click\&pgtype=Article\&state=default\&region=MAIN_CONTENT_1\&context=storylines_live_updates\#link-dfb8a16}{Fauci
  cautions the virus could disrupt life in the U.S. until `maybe even
  towards the end of 2021.'}
\item
  \href{https://www.nytimes3xbfgragh.onion/2020/09/11/world/covid-19-coronavirus.html?action=click\&pgtype=Article\&state=default\&region=MAIN_CONTENT_1\&context=storylines_live_updates\#link-7104d154}{From
  Asia to Africa, China promotes its vaccine candidates to win friends.}
\item
  \href{https://www.nytimes3xbfgragh.onion/2020/09/11/world/covid-19-coronavirus.html?action=click\&pgtype=Article\&state=default\&region=MAIN_CONTENT_1\&context=storylines_live_updates\#link-393ad215}{The
  other way the virus will kill: hunger.}
\end{itemize}

\href{https://www.nytimes3xbfgragh.onion/2020/09/11/world/covid-19-coronavirus.html?action=click\&pgtype=Article\&state=default\&region=MAIN_CONTENT_1\&context=storylines_live_updates}{See
more updates}

More live coverage:
\href{https://www.nytimes3xbfgragh.onion/live/2020/09/11/business/stock-market-today-coronavirus?action=click\&pgtype=Article\&state=default\&region=MAIN_CONTENT_1\&context=storylines_live_updates}{Markets}

Other data points to a severe but slower-moving crisis of collapsing
demand that will affect many more corners of the economy than those that
were forced to close because of the pandemic.

New orders for manufactured goods, for example, remained in starkly
negative territory in May, according to the Institute for Supply
Management; its index came in at 31.8, far below the level of 50 that is
the line between expansion and contraction.

And despite the net gain in employment in May, there have been many
announced layoffs at companies outside sectors directly affected by the
pandemic. This suggests that the forced shutdown of travel, restaurant
and related industries is rippling out into a broad-based shortage of
demand in the economy.

Consider just a partial list of large well-known companies unaffected by
the direct first-round effects of pandemic-induced shutdowns, but which
have since announced layoffs:
\href{https://www.reuters.com/article/us-chevron-layoffs-exclusive/exclusive-chevron-to-cut-up-to-15-of-staff-amid-restructuring-idUSKBN2332P3}{Chevron},
\href{https://www.wsj.com/articles/ibm-announces-first-job-cuts-under-new-chief-executive-11590113061}{I.B.M.}
and
\href{https://www.cnbc.com/2020/05/15/office-depot-plans-store-closures-13100-job-cuts-by-2023.html}{Office
Depot}.

Last week, the Congressional Budget Office tried to put a number on the
aggregate economic activity that will be lost over the next decade
compared with what was projected at the start of the year. That number
is \$15.7 trillion, reflecting both less economic activity and
deflationary forces that reduce prices.

That is 5.3 percent less ``nominal'' output, meaning not adjusted for
inflation, than had been forecast. For comparison, from 2008 to 2018,
total nominal output came in 6 percent below the level the C.B.O. had
forecast at the start of 2008.

We know how miserable that economic crisis and sluggish recovery were,
with long-term costs to earnings and well-being. The C.B.O. is now
forecasting that the next decade will be nearly as bad --- but
\href{https://www.cbo.gov/publication/56376}{emphasizes} that policy
choices will shape how things actually evolve.

The economy is
\href{https://www.nytimes3xbfgragh.onion/2020/03/17/upshot/coronavirus-economy-crisis-demand-shock.html}{a
gigantic machine} in which one person's consumption spending generates
someone else's income. The pandemic began by crushing the economy's
productive capacity --- a shock to the supply side of the economy, as
many types of business activity were shut down for public health
concerns.

In normal times, when there is a negative supply shock (say, a year of
drought that reduces agricultural crops, or new tariffs that make
imports more expensive), the pain can be intense for people in sectors
directly affected, yet the economy as a whole adjusts.

But this crisis is so large and so sudden that the usual adjustment
mechanisms aren't working very well.

The people losing their jobs because of shutdowns cannot easily find new
ones, because so much of the economy is shuttered at the same time. The
businesses in danger of closing have cut every possible expense: A hotel
isn't going to invest in new furniture or new reservation software right
now. And consumer demand for some seemingly safe goods falls because
those goods are complements to the sectors that are shut down.

``Hotels are locked down, so people buy fewer cars because they don't
need to travel as much,'' said Veronica Guerrieri, an economist at the
University of Chicago Booth School of Business. ``Restaurants are locked
down, so people don't need fancy clothes because they don't want to go
out as much.''

The result is that what started as a disruption to the supply side of
the economy has metastasized into a collapse of the demand side, she and
co-authors say in a
\href{https://bfi.uchicago.edu/wp-content/uploads/BFI_WP_202035.pdf}{recent
working paper}. They call it a Keynesian supply shock: an inversion of
the demand-driven crisis of the Great Depression described by the great
economist of that era, John Maynard Keynes.

``Demand is interrelated with supply,'' said Iván Werning, an M.I.T.
economist and a co-author of the paper. ``It's not a separate concept.''

The demand shock, with
\href{https://www.nytimes3xbfgragh.onion/2020/05/11/upshot/virus-lasting-economic-effects.html}{lagged
effects}, is only beginning to hurt major segments of the economy, like
sellers of capital goods that are experiencing plunging sales; state and
local governments that are seeing tax revenues crater; and landlords who
are seeing
\href{https://www.nytimes3xbfgragh.onion/2020/06/05/business/economy/coronavirus-commercial-real-estate.html}{rent
payments dry up}.

The government can't wave a wand and bring back industries that are
semi-permanently shuttered. That original supply shock can be fixed only
as public health conditions allow sports arenas and the like to reopen.

But the government can act --- and has acted --- to try to keep demand
for goods and services at pre-crisis levels. That, in turn, can smooth
the path for other sectors to grow so that there is not a prolonged
depression of jobs, income and investment, with a resulting reduction in
the economy's long-term potential.

In the early phase of the crisis, Congress expanded unemployment
benefits, funneled hundreds of billions of dollars toward small
businesses to keep workers on their payrolls, and supported state
governments, among other steps. But much of this help is
\href{https://www.nytimes3xbfgragh.onion/2020/05/28/business/economy/coronavirus-stimulus-unemployment.html}{scheduled
to expire this summer}, absent further action --- and the positive jobs
numbers Friday led many Republicans on Capitol Hill allied with the
Trump administration to suggest that they were reluctant to do more.

It is against this backdrop that some of the most influential --- and
fiscally conservative --- voices in economic policy are saying that
further aggressive spending is needed to prevent this shock from causing
long-lasting damage to the economy.

``This is the time to use the great fiscal power of the United States to
do what we can to support the economy and try to get through this with
as little damage to the longer-run productive capacity of the economy as
possible,'' Jerome Powell, the Federal Reserve chair and a longtime
fiscal hawk, said
\href{https://www.federalreserve.gov/mediacenter/files/FOMCpresconf20200429.pdf}{at
a news conference} in late April.

``Please, spend wisely, but spend as much as you can!'' Kristalina
Georgieva, the managing director of the International Monetary Fund,
implored the world's governments at
\href{https://www.politico.com/newsletters/politico-nightly-coronavirus-special-edition/2020/05/15/spend-as-much-as-you-can-489240}{an
event} in May. ``And then, spend a bit more for your doctors, for your
nurses, for the vulnerable people in your society.''

Both the Fed and the I.M.F. more typically act as brakes on fiscal
profligacy. For Mr. Powell and Ms. Georgieva to effectively beg elected
officials to stop a spiraling crisis reflects the unusual circumstances
of this moment and the extraordinary risk they see if government action
is inadequate to the job. Their comments are the equivalent of a
normally debt-averse financial adviser urging a family to borrow more
money to ride out a period of illness without suffering long-term
financial damage.

When the crisis we now know as the Great Depression began in 1929,
President Herbert Hoover started with denial, then tried blaming other
countries, then argued that there was nothing the government could
really do to contain the damage.

Eventually, the Hoover administration took more aggressive action,
creating a large federal program of mass employment. ``He gave a speech
and said that 700,000 Americans were at work on federal public works,
and it was bigger than anything that had done before,'' Mr. Rauchway
said. ``And that was true, but it was at a time when more than seven
million people were out of work.''

That crisis showed how when there are profound rips in the economic
fabric, repairing them isn't a simple job, it isn't quick, and even what
seems like a huge response often isn't enough.

It's great that the economy is ticking up from its shutdown of March and
April. And the world right now is confusing and chaotic. But that makes
it all the more important not to lose focus on fundamental forces that
risk holding back the economy and that, if unchecked, could mean a
second lost decade in this young century.

Advertisement

\protect\hyperlink{after-bottom}{Continue reading the main story}

\hypertarget{site-index}{%
\subsection{Site Index}\label{site-index}}

\hypertarget{site-information-navigation}{%
\subsection{Site Information
Navigation}\label{site-information-navigation}}

\begin{itemize}
\tightlist
\item
  \href{https://help.nytimes3xbfgragh.onion/hc/en-us/articles/115014792127-Copyright-notice}{©~2020~The
  New York Times Company}
\end{itemize}

\begin{itemize}
\tightlist
\item
  \href{https://www.nytco.com/}{NYTCo}
\item
  \href{https://help.nytimes3xbfgragh.onion/hc/en-us/articles/115015385887-Contact-Us}{Contact
  Us}
\item
  \href{https://www.nytco.com/careers/}{Work with us}
\item
  \href{https://nytmediakit.com/}{Advertise}
\item
  \href{http://www.tbrandstudio.com/}{T Brand Studio}
\item
  \href{https://www.nytimes3xbfgragh.onion/privacy/cookie-policy\#how-do-i-manage-trackers}{Your
  Ad Choices}
\item
  \href{https://www.nytimes3xbfgragh.onion/privacy}{Privacy}
\item
  \href{https://help.nytimes3xbfgragh.onion/hc/en-us/articles/115014893428-Terms-of-service}{Terms
  of Service}
\item
  \href{https://help.nytimes3xbfgragh.onion/hc/en-us/articles/115014893968-Terms-of-sale}{Terms
  of Sale}
\item
  \href{https://spiderbites.nytimes3xbfgragh.onion}{Site Map}
\item
  \href{https://help.nytimes3xbfgragh.onion/hc/en-us}{Help}
\item
  \href{https://www.nytimes3xbfgragh.onion/subscription?campaignId=37WXW}{Subscriptions}
\end{itemize}
