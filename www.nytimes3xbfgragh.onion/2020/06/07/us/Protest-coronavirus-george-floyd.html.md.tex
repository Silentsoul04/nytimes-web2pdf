Sections

SEARCH

\protect\hyperlink{site-content}{Skip to
content}\protect\hyperlink{site-index}{Skip to site index}

\href{https://www.nytimes3xbfgragh.onion/section/us}{U.S.}

\href{https://myaccount.nytimes3xbfgragh.onion/auth/login?response_type=cookie\&client_id=vi}{}

\href{https://www.nytimes3xbfgragh.onion/section/todayspaper}{Today's
Paper}

\href{/section/us}{U.S.}\textbar{}A Delicate Balance: Weighing Protest
Against the Risks of the Coronavirus

\url{https://nyti.ms/30ftp01}

\begin{itemize}
\item
\item
\item
\item
\item
\item
\end{itemize}

\hypertarget{race-and-america}{%
\subsubsection{\texorpdfstring{\href{https://www.nytimes3xbfgragh.onion/news-event/george-floyd-protests-minneapolis-new-york-los-angeles?name=styln-george-floyd\&region=TOP_BANNER\&block=storyline_menu_recirc\&action=click\&pgtype=Article\&impression_id=cbd57f50-f4c6-11ea-b359-5f738db007af\&variant=undefined}{Race
and America}}{Race and America}}\label{race-and-america}}

\begin{itemize}
\tightlist
\item
  \href{https://www.nytimes3xbfgragh.onion/2020/09/11/us/black-police-chiefs-reform.html?name=styln-george-floyd\&region=TOP_BANNER\&block=storyline_menu_recirc\&action=click\&pgtype=Article\&impression_id=cbd57f51-f4c6-11ea-b359-5f738db007af\&variant=undefined}{Black
  Police Chiefs}
\item
  \href{https://www.nytimes3xbfgragh.onion/2020/09/04/nyregion/rochester-police-daniel-prude.html?name=styln-george-floyd\&region=TOP_BANNER\&block=storyline_menu_recirc\&action=click\&pgtype=Article\&impression_id=cbd57f52-f4c6-11ea-b359-5f738db007af\&variant=undefined}{What
  Happened in Rochester, N.Y.}
\item
  \href{https://www.nytimes3xbfgragh.onion/2020/08/30/us/portland-shooting-explained.html?name=styln-george-floyd\&region=TOP_BANNER\&block=storyline_menu_recirc\&action=click\&pgtype=Article\&impression_id=cbd57f53-f4c6-11ea-b359-5f738db007af\&variant=undefined}{Portland
  Shooting}
\item
  \href{https://www.nytimes3xbfgragh.onion/2020/08/30/us/breonna-taylor-police-killing.html?name=styln-george-floyd\&region=TOP_BANNER\&block=storyline_menu_recirc\&action=click\&pgtype=Article\&impression_id=cbd57f54-f4c6-11ea-b359-5f738db007af\&variant=undefined}{Breonna
  Taylor's Life and Death}
\end{itemize}

Advertisement

\protect\hyperlink{after-top}{Continue reading the main story}

Supported by

\protect\hyperlink{after-sponsor}{Continue reading the main story}

\hypertarget{a-delicate-balance-weighing-protest-against-the-risks-of-the-coronavirus}{%
\section{A Delicate Balance: Weighing Protest Against the Risks of the
Coronavirus}\label{a-delicate-balance-weighing-protest-against-the-risks-of-the-coronavirus}}

As the protests against police brutality continue, public officials are
warily watching for signs that mass demonstrations are leading to virus
outbreaks.

\includegraphics{https://static01.graylady3jvrrxbe.onion/images/2020/06/07/us/07unrest-virus-1/merlin_173202873_ad6fc13c-8a3c-40c8-8f8c-7715712f4177-articleLarge.jpg?quality=75\&auto=webp\&disable=upscale}

By \href{https://www.nytimes3xbfgragh.onion/by/amy-harmon}{Amy Harmon}
and \href{https://www.nytimes3xbfgragh.onion/by/rick-rojas}{Rick Rojas}

\begin{itemize}
\item
  Published June 7, 2020Updated June 10, 2020
\item
  \begin{itemize}
  \item
  \item
  \item
  \item
  \item
  \item
  \end{itemize}
\end{itemize}

None of the plans for how the nation might safely emerge from the
coronavirus lockdown involved thousands of Americans standing shoulder
to shoulder in the streets of major cities or coughing uncontrollably
when the authorities used tear gas to disperse them. No one planned on
protesters being herded into crowded prison buses or left in crowded
cells.

Before the eruption of outrage over the killing of George Floyd in
police custody in Minneapolis, debates about reopening centered on
whether states had adequate systems in place to detect and treat cases
of the coronavirus, which has killed more than 110,000 people in the
United States since the beginning of the year.

But as the protests against police brutality continue for a second week,
public officials are warily watching for signs that an unanticipated end
to social distancing on a mass scale has led to new cases of the virus.
The question has become part of the politicized debate over the economic
repercussions of the lockdown, which some critics have argued went too
far.

And on Sunday, infectious disease experts on Twitter debated how to
supply a reliable estimate of the
\href{https://www.nytimes3xbfgragh.onion/2020/07/01/nyregion/nyc-coronavirus-protests.html}{impact
of the protests on virus transmission} --- or whether trying to do so
may wrongly be seen as discouraging participation in the growing racial
justice movement.

In what he called a back-of-the-envelope estimate, Trevor Bedford, an
expert on the virus at the Fred Hutchinson Cancer Research Center, wrote
on Twitter that each day of protests would result in about 3,000 new
infections. Over several weeks, as each infected person infected just
under one other person on average --- the current U.S. transmission rate
--- those infections would in turn lead to 15,000 to 50,000 more, and
\href{https://twitter.com/trvrb/status/1269786156616433664}{50 to 500
eventual deaths}. Given the racial disparities so far in the pandemic,
he noted, those deaths will be disproportionately among black people.
``Societal benefit of continued protests must be weighed against
substantial potential impacts to health,'' he wrote.

\includegraphics{https://static01.graylady3jvrrxbe.onion/images/2020/06/07/us/07unrest-virus-3/merlin_172977357_351afc32-12a4-4748-80bd-f6c4ba1e6cb7-articleLarge.jpg?quality=75\&auto=webp\&disable=upscale}

Marc Lipsitch, an epidemiologist at Harvard, agreed that those
projections were reasonable, and said in an email that Dr. Bedford had
done ``a service'' by making an approximate estimate with explicit
assumptions.

But he also noted that if states where the virus was still spreading
managed to rein it in, that would
``\href{https://twitter.com/mlipsitch/status/1269668219196932098}{massively
overshadow the effects of the protests}.'' About 20,000 new cases are
identified across the country on most days, and about 1,000 new deaths
are announced. If all communities were performing enough tests and
contact tracing to bring down those numbers, fewer of those acquiring
infections at protests would infect others, shortening the transmission
chain and reducing the number of eventual deaths.

Dr. Bedford wrote that his estimates contained a lot of uncertainty.
There is no official estimate for how many people are protesting on an
average day, for instance. Still, he thought it was important, he said,
to provide a framework grounded in epidemiologic principles to counter
the offhand assumptions being made by political pundits. But,
\href{https://twitter.com/NathanGrubaugh/status/1269699851438247937}{in
response}, other scientists
\href{https://twitter.com/mlipsitch/status/1269666394100162562}{voiced
concern} that Dr. Bedford's posts would ``give fodder to those opposing
civil rights.''

Many epidemiologists,
\href{https://twitter.com/trvrb/status/1269786159032397824}{including
Dr. Bedford}, have noted in recent days that America's entrenched racial
inequalities themselves translate into disproportionate early deaths and
illness among African-Americans. That has been especially evident in the
coronavirus pandemic, in which black Americans are dying at about twice
the rate of white Americans. A group of more than 1,000 people working
in health and medicine
\href{https://drive.google.com/file/d/1Jyfn4Wd2i6bRi12ePghMHtX3ys1b7K1A/view}{signed
a letter} recently that said protests were, in fact, vital to public
health.

``Racism and police violence are major threats to public health in this
country, and protest is one of the only options available to people who
have been systematically disenfranchised,'' said Eleanor Murray, an
epidemiologist at Boston University.

Because it can take up to two weeks for a newly infected person to show
symptoms and the protests started shortly after Mr. Floyd's death,
health experts expect that any increase in cases will begin to surface
this week. Demonstrators in several places have contracted the virus,
including in Lawrence, Kan., where someone who attended a protest last
weekend tested positive on Friday. That person did not wear a mask while
protesting, local officials said. In Athens, Ga., a county commissioner
who attended a protest said that she had
\href{https://slack-redir.net/link?url=https\%3A\%2F\%2Fwww.onlineathens.com\%2Fnews\%2F20200603\%2Fathens-commissioner-mariah-parker-tests-positive-for-coronavirus}{tested
positive}.

In Oklahoma, a college football player who participated in a
demonstration revealed that he later tested positive for the virus.

``After attending a protest in Tulsa AND being well protective of
myself,
\href{https://twitter.com/closedprayer/status/1267971181715632129?ref_src=twsrc\%5Etfw}{I
have tested positive for Covid-19},'' Amen Ogbongbemiga, a linebacker at
Oklahoma State University, wrote on Twitter. ``Please, if you are going
to protest, take care of yourself and stay safe.''

Many did their best. In Califoria, Jarrion Harris, 32, wore a cloth mask
for a march in Hollywood on Sunday.

``I'm definitely not out here because I think Covid-19 has gone into the
shadows,'' he said. ``It's worth the risk.''

Politicians and public health officials have urged demonstrators to wear
face coverings and to maintain social distancing. In some places,
including Atlanta, Illinois, Los Angeles and Minnesota, officials have
also urged protesters to seek coronavirus tests to make sure they have
not become infected.

The surge of protests throughout American cities and the seeming support
of some political leaders have not gone unnoticed on the right. Pundits
and commentators have made the point in recent days that mayors and
governors who sternly warned against public gatherings are now giving
their blessing to the protests.

Dr. Tom Frieden, the former director of the Centers for Disease Control
and Prevention who was appointed by President Barack Obama, tweeted last
week that the threat to controlling the coronavirus was ``tiny''
compared with the threat ``created when governments act in ways that
lose community trust.''

That struck Jonah Goldberg, a writer for The Dispatch, an online
conservative magazine, as an unfair double standard.

``You know what erodes public trust in people like Frieden?'' Mr.
Goldberg
\href{https://gfile.thedispatch.com/p/the-treason-of-epidemiologists}{wrote}.
``When they say that you're a fool or monster who will get people killed
for wanting to go to church or keep your business open but you're a hero
when you join a protest they approve of.''

Image

Protesters in Washington on Sunday at times found that social distancing
was impossible.Credit...Anna Moneymaker/The New York Times

As the virus has spread across the country, many conservatives have
rebelled against the stringent measures supported by public health
experts, and there have been skirmishes over wearing masks in public,
allowing in-person religious services and placing a premium on restoring
the economy over other concerns.

If the virus surges again, said Christopher F. Rufo, director of the
Center on Wealth and Poverty at the Discovery Institute, a conservative
think tank, the experts' standing will have been undermined. ``You're
going to have half the country that has lost faith almost completely in
the public health establishment,'' he said.

But it may be hard to trace infections to other protesters, who are
often marching side by side with strangers. And infectious disease
experts have said that any increase in cases may well be the result of
the continued reopening of restaurants, workplaces and mass transit.

In Las Vegas, casinos are reopening. In New York City, long the center
of the coronavirus outbreak in the United States, as many as 400,000
workers on Monday can begin returning to construction jobs,
manufacturing sites and retail stores as the city enters the first phase
of its reopening.

``You cannot pin this on the protests,'' said Jeffrey Shaman, an
epidemiologist at Columbia University, whose projections of the virus's
path suggest that U.S. coronavirus deaths will increase in the coming
weeks. ``The protests are not in and of themselves going to drive the
resurgence in cases. This is associated with all the new opportunities
that are providing a way for people to get together and pass the virus
to one another.''

For their part, activists said the drive to protest in the face of the
virus reflected the larger gravity of the moment and the intensity of
their passion.

``If I get infected fighting for justice, my soul can sit with that,''
said Sara Semi, 27, a protester in Minneapolis who wears a mask with a
filter and carries cans of disinfectant spray. ``I can't sit at home
protected by my privileges if others aren't. I can't sit inside my house
safe while my friends and neighbors are not. Yes, corona is happening.
It's real, it's deadly. But racism kills way more lives.''

Vidal Guzman, 29, a protester in New York, said: ``People are more
scared of the police than Covid-19. They are willing to do anything.''

Different conditions and the many remaining unknowns about how the virus
is transmitted can make it difficult to predict the spread of the virus
at any given protest. Outdoor gatherings are much lower risk than indoor
ones. How many people are wearing masks, whether they are shouting or
using noisemakers and even the weather may all affect the risk of
infection.

Still, expert say it is within the power of law enforcement authorities
to minimize or eliminate some of the chief risk factors, like arrests
and tear gas. And they add that protesters, if they choose to take to
the streets, should do everything they can to stay safe.

``If you attended a protest,'' Gov. Andrew M. Cuomo of New York said on
Sunday, ``assume you may have been exposed to Covid. Get tested.''

Mitch Smith, Sabrina Tavernise, Anjali Tsui, Jill Cowan and Kimiko de
Freytas-Tamura contributed reporting.

Advertisement

\protect\hyperlink{after-bottom}{Continue reading the main story}

\hypertarget{site-index}{%
\subsection{Site Index}\label{site-index}}

\hypertarget{site-information-navigation}{%
\subsection{Site Information
Navigation}\label{site-information-navigation}}

\begin{itemize}
\tightlist
\item
  \href{https://help.nytimes3xbfgragh.onion/hc/en-us/articles/115014792127-Copyright-notice}{©~2020~The
  New York Times Company}
\end{itemize}

\begin{itemize}
\tightlist
\item
  \href{https://www.nytco.com/}{NYTCo}
\item
  \href{https://help.nytimes3xbfgragh.onion/hc/en-us/articles/115015385887-Contact-Us}{Contact
  Us}
\item
  \href{https://www.nytco.com/careers/}{Work with us}
\item
  \href{https://nytmediakit.com/}{Advertise}
\item
  \href{http://www.tbrandstudio.com/}{T Brand Studio}
\item
  \href{https://www.nytimes3xbfgragh.onion/privacy/cookie-policy\#how-do-i-manage-trackers}{Your
  Ad Choices}
\item
  \href{https://www.nytimes3xbfgragh.onion/privacy}{Privacy}
\item
  \href{https://help.nytimes3xbfgragh.onion/hc/en-us/articles/115014893428-Terms-of-service}{Terms
  of Service}
\item
  \href{https://help.nytimes3xbfgragh.onion/hc/en-us/articles/115014893968-Terms-of-sale}{Terms
  of Sale}
\item
  \href{https://spiderbites.nytimes3xbfgragh.onion}{Site Map}
\item
  \href{https://help.nytimes3xbfgragh.onion/hc/en-us}{Help}
\item
  \href{https://www.nytimes3xbfgragh.onion/subscription?campaignId=37WXW}{Subscriptions}
\end{itemize}
