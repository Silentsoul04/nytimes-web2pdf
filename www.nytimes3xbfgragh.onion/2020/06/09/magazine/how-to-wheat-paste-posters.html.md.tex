Sections

SEARCH

\protect\hyperlink{site-content}{Skip to
content}\protect\hyperlink{site-index}{Skip to site index}

\href{https://myaccount.nytimes3xbfgragh.onion/auth/login?response_type=cookie\&client_id=vi}{}

\href{https://www.nytimes3xbfgragh.onion/section/todayspaper}{Today's
Paper}

How to Wheat-Paste Posters

\url{https://nyti.ms/3dQCZdC}

\begin{itemize}
\item
\item
\item
\item
\item
\end{itemize}

\href{https://www.nytimes3xbfgragh.onion/spotlight/at-home?action=click\&pgtype=Article\&state=default\&region=TOP_BANNER\&context=at_home_menu}{At
Home}

\begin{itemize}
\tightlist
\item
  \href{https://www.nytimes3xbfgragh.onion/2020/09/07/travel/route-66.html?action=click\&pgtype=Article\&state=default\&region=TOP_BANNER\&context=at_home_menu}{Cruise
  Along: Route 66}
\item
  \href{https://www.nytimes3xbfgragh.onion/2020/09/04/dining/sheet-pan-chicken.html?action=click\&pgtype=Article\&state=default\&region=TOP_BANNER\&context=at_home_menu}{Roast:
  Chicken With Plums}
\item
  \href{https://www.nytimes3xbfgragh.onion/2020/09/04/arts/television/dark-shadows-stream.html?action=click\&pgtype=Article\&state=default\&region=TOP_BANNER\&context=at_home_menu}{Watch:
  Dark Shadows}
\item
  \href{https://www.nytimes3xbfgragh.onion/interactive/2020/at-home/even-more-reporters-editors-diaries-lists-recommendations.html?action=click\&pgtype=Article\&state=default\&region=TOP_BANNER\&context=at_home_menu}{Explore:
  Reporters' Google Docs}
\end{itemize}

Advertisement

\protect\hyperlink{after-top}{Continue reading the main story}

Supported by

\protect\hyperlink{after-sponsor}{Continue reading the main story}

\href{/column/magazine-tip}{Tip}

\hypertarget{how-to-wheat-paste-posters}{%
\section{How to Wheat-Paste Posters}\label{how-to-wheat-paste-posters}}

\includegraphics{https://static01.graylady3jvrrxbe.onion/images/2020/06/14/magazine/14Mag-Tip-01/14Mag-Tip-01-articleLarge-v2.jpg?quality=75\&auto=webp\&disable=upscale}

By Malia Wollan

\begin{itemize}
\item
  Published June 9, 2020Updated June 10, 2020
\item
  \begin{itemize}
  \item
  \item
  \item
  \item
  \item
  \end{itemize}
\end{itemize}

``Wheat paste is a way to get art and messaging into public spaces,''
says Mark Strandquist, an artist from Philadelphia, who was walking
around his neighborhood in early March, past storefronts covered over in
plywood, when he realized those boards and empty windows could serve as
canvases. He put out a call to artists, and within a few weeks the
project, called \href{https://coverthewallswithhope.weebly.com}{Fill the
Walls with Hope}, had collected dozens of downloadable images from them.
Strandquist and his fellow artists have put up more than 1,000 posters
with a home-cooked glue concoction made from flour and water. Many of
the posters share messages of public health and safety, among them,
``Mask Up'' and ``Don't Let Racism Go Viral.''

To cook wheat paste, use four parts water and one part flour. Start by
boiling your water. In a separate bowl, whisk the flour with enough cold
water to create a lump-free goo, then pour this mix into the boiling
water. Turn the burner down low, and stir continuously for 20 minutes.
Remove from the heat, and stir another 10 minutes. Wait until cooled to
use.

Print your poster on thin paper so the paste can seep through,
preferably 20-pound, uncoated bond paper. Once on site, use a wide
paintbrush to add a layer of paste just slightly larger than your image.
Press your poster into the paste, pushing out any air pockets. Paint
another layer over the top to seal.

Consider your safety and your privilege. ``Don't go out alone,'' says
Strandquist, who, as a 34-year-old white man, can poster the city in
broad daylight with an impunity he knows won't be extended to artists of
color. Many of the group's posters were put up with permission, but some
are what he calls ``uncommissioned.'' Amplify the work and words of
artists from the neighborhoods where you're postering.

A wheat-pasted poster might be ripped down after an hour or slowly fade
over a year. Think of the layered aesthetic of a collection of posters
as an ongoing public conversation. ``It's a way for artists to respond
to a crisis,'' Strandquist says. ``And help us imagine a better, more
equitable world.''

Advertisement

\protect\hyperlink{after-bottom}{Continue reading the main story}

\hypertarget{site-index}{%
\subsection{Site Index}\label{site-index}}

\hypertarget{site-information-navigation}{%
\subsection{Site Information
Navigation}\label{site-information-navigation}}

\begin{itemize}
\tightlist
\item
  \href{https://help.nytimes3xbfgragh.onion/hc/en-us/articles/115014792127-Copyright-notice}{©~2020~The
  New York Times Company}
\end{itemize}

\begin{itemize}
\tightlist
\item
  \href{https://www.nytco.com/}{NYTCo}
\item
  \href{https://help.nytimes3xbfgragh.onion/hc/en-us/articles/115015385887-Contact-Us}{Contact
  Us}
\item
  \href{https://www.nytco.com/careers/}{Work with us}
\item
  \href{https://nytmediakit.com/}{Advertise}
\item
  \href{http://www.tbrandstudio.com/}{T Brand Studio}
\item
  \href{https://www.nytimes3xbfgragh.onion/privacy/cookie-policy\#how-do-i-manage-trackers}{Your
  Ad Choices}
\item
  \href{https://www.nytimes3xbfgragh.onion/privacy}{Privacy}
\item
  \href{https://help.nytimes3xbfgragh.onion/hc/en-us/articles/115014893428-Terms-of-service}{Terms
  of Service}
\item
  \href{https://help.nytimes3xbfgragh.onion/hc/en-us/articles/115014893968-Terms-of-sale}{Terms
  of Sale}
\item
  \href{https://spiderbites.nytimes3xbfgragh.onion}{Site Map}
\item
  \href{https://help.nytimes3xbfgragh.onion/hc/en-us}{Help}
\item
  \href{https://www.nytimes3xbfgragh.onion/subscription?campaignId=37WXW}{Subscriptions}
\end{itemize}
