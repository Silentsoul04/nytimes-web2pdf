Sections

SEARCH

\protect\hyperlink{site-content}{Skip to
content}\protect\hyperlink{site-index}{Skip to site index}

\href{https://myaccount.nytimes3xbfgragh.onion/auth/login?response_type=cookie\&client_id=vi}{}

\href{https://www.nytimes3xbfgragh.onion/section/todayspaper}{Today's
Paper}

These Crispy Kimchi Pancakes Are Unbelievably Fun to Eat

\url{https://nyti.ms/2MHMWhH}

\begin{itemize}
\item
\item
\item
\item
\item
\end{itemize}

\href{https://www.nytimes3xbfgragh.onion/spotlight/at-home?action=click\&pgtype=Article\&state=default\&region=TOP_BANNER\&context=at_home_menu}{At
Home}

\begin{itemize}
\tightlist
\item
  \href{https://www.nytimes3xbfgragh.onion/2020/09/07/travel/route-66.html?action=click\&pgtype=Article\&state=default\&region=TOP_BANNER\&context=at_home_menu}{Cruise
  Along: Route 66}
\item
  \href{https://www.nytimes3xbfgragh.onion/2020/09/04/dining/sheet-pan-chicken.html?action=click\&pgtype=Article\&state=default\&region=TOP_BANNER\&context=at_home_menu}{Roast:
  Chicken With Plums}
\item
  \href{https://www.nytimes3xbfgragh.onion/2020/09/04/arts/television/dark-shadows-stream.html?action=click\&pgtype=Article\&state=default\&region=TOP_BANNER\&context=at_home_menu}{Watch:
  Dark Shadows}
\item
  \href{https://www.nytimes3xbfgragh.onion/interactive/2020/at-home/even-more-reporters-editors-diaries-lists-recommendations.html?action=click\&pgtype=Article\&state=default\&region=TOP_BANNER\&context=at_home_menu}{Explore:
  Reporters' Google Docs}
\end{itemize}

Advertisement

\protect\hyperlink{after-top}{Continue reading the main story}

Supported by

\protect\hyperlink{after-sponsor}{Continue reading the main story}

\href{/column/magazine-eat}{Eat}

\hypertarget{these-crispy-kimchi-pancakes-are-unbelievably-fun-to-eat}{%
\section{These Crispy Kimchi Pancakes Are Unbelievably Fun to
Eat}\label{these-crispy-kimchi-pancakes-are-unbelievably-fun-to-eat}}

\includegraphics{https://static01.graylady3jvrrxbe.onion/images/2020/06/14/magazine/14mag-eat/14mag-eat-articleLarge.jpg?quality=75\&auto=webp\&disable=upscale}

By Samin Nosrat

\begin{itemize}
\item
  June 10, 2020
\item
  \begin{itemize}
  \item
  \item
  \item
  \item
  \item
  \end{itemize}
\end{itemize}

I've always been curious --- my mom would call it nosy --- about how
other people live, and that interest extends into the kitchen. It's a
gift to get to pursue my curiosity in this column. When I'm out in the
world reporting, cooks open the doors to their kitchens. I watch them
knead a dough or stir a sauce until it's just right, and then try to
translate those instinctual or sensory cues into words.

But I'm at home now, like many of you, eating mostly peanut butter
sandwiches and quesadillas and trying to escape the taste of my own
boring cooking. Within weeks of the stay-at-home order, I couldn't
stomach another bowl of pasta or chickpea soup. Usually when I get into
a cooking rut, I go out to eat --- to be inspired and learn something
new. Now I find myself looking back, toward all of the dishes I'm
nostalgic for. Not the food of my childhood, but rather the taste of the
neighborhood restaurants where I've been eating my entire adult life.

One of them, Pyeong Chang Tofu House, is a family-run spot in Oakland
where I've been a regular since 2000. In the winter, it's where I go
when I want to be warmed by their \emph{sundubu-jigae}, a spicy, silky
soft tofu soup that arrives at the table at a rolling boil. In the
summer, I go for \emph{bibim guksu} --- spicy, sweet, cold noodles ---
or \emph{bibimbap}. And no matter which season, I start my meal with
\emph{kimchijeon}. Every time these golden kimchi pancakes arrive at the
table, I greedily take the first piece before anyone else can. I wonder
aloud how it can be both so satisfyingly chewy and so shatteringly
crisp. Whenever I eat the pancake with other cooks, we try to dissect it
--- is it made with glutinous rice flour? Does it have eggs? What
exactly makes this pancake, tart with pungent kimchi and fried to a
glorious crisp, so unbelievably fun to eat? I promised myself I would
knock on the kitchen door to ask the cooks their secret, but I shied
away every time.

Then, a few weeks ago, when I just couldn't take any more of my own
Italo-Mediterranean-ish cooking, I pulled down my Korean cookbooks and
took to the internet in search of a recipe that might lead me to some
approximation of my beloved kimchi pancake. For a week, I made variation
after variation, but none turned out right. The first one I made --- the
one when I went with my gut and deviated from every single traditional
recipe and sneaked in some eggs --- was outright disgusting. Once I left
the eggs out, things improved, but I still couldn't get that perfect
balance of chewy and crisp. I tweaked the combinations and proportions
of various starches, including flour, tapioca starch and potato starch,
until I arrived at a respectable pancake. But it still didn't taste
right.

Eventually it occurred to me that I could simply ask the folks at the
tofu house for their recipe. I put on my mask and headed over. No
manager or chef was present when I got there, so I left a note. To my
delight, the manager, Peter Hak Sun Kim, called me back a few hours
later. Translating for the chef and owner, Young S. Kim, who is also his
mother, Peter told me, ``My mom is happy to share her recipe, but she
wants me to tell you that it's our homemade kimchi that makes the
pancake so good.''

I had perfectly tasty kimchi in my fridge, so I brushed off his comment;
I just wanted to know exactly how they made that pancake! I was thrilled
to discover Mrs. Kim's secret: a combination of tempura mix and Korean
pancake mix. So I bought some of each from the Korean market, certain
that would solve everything. After analyzing the ingredients lists on
both packages (which conveniently listed the percentage of each
ingredient), I added baking powder and garlic powder to my own dry
mixture. I also added some \emph{gochujang}, or fermented pepper paste,
after recalling that Mrs. Kim's kimchi is some of the spiciest and most
flavorful I've ever tasted. Then I made a control pancake using the
premade mixes, and another using my own ratios.

They were fine. Good, even. The neighbors I shared them with thought
they were amazing --- I'd finally achieved that ideal texture and
reached a new depth of flavor. But they were virtually
indistinguishable. And neither of them tasted like my memory of Mrs.
Kim's version.

A few days later, I remembered that Koreans have a word for the
specific, irreplicable taste of someone's cooking: \emph{son-mat}. I
called Peter and Mrs. Kim to ask them about it. ``For my mom,'' he said,
``\emph{son-mat} is composed of three things: experience, the passion of
the individual and the knowledge gained from constant cooking. For her,
it's like a never-ending learning goal, like an art. She doesn't use
measuring cups or a scale when she makes kimchi.''

In Korea, kimchi-making is typically a family or neighborhood activity.
Groups of women spend two or three days making enormous batches of
fermented vegetables to last their families through the season or year.
Finding herself in Oakland with far less space and far fewer helping
hands, Mrs. Kim adapted those traditions when she took over the tofu
house in 2004. These days, she transforms 1,400 pounds of napa cabbage
into kimchi each month, and for the first time, it's available to
purchase. ``It takes so much time and work to make, we never really
wanted to sell it,'' Peter explained. But now that the restaurant has
pivoted to takeout only, its customers have been begging for it.
``Because of the lockdown,'' he explained, ``we decided to let people
take our \emph{son-mat} home.''

I bought a huge container of Mrs. Kim's kimchi and made the pancake
again. This time, it tasted exactly as I remembered. But funnily enough,
my neighbors struggled to detect any difference between this pancake and
the previous one. Both, they said, were sweet and spicy, tart and
crunchy, chewy and delightfully savory. I thought I'd been chasing a
precise formula to satisfy my craving. But as it turns out, what I miss
most right now can't quite be captured in a recipe.

Recipe:
\href{https://cooking.nytimes3xbfgragh.onion/recipes/1021158-kimchijeon-kimchi-pancake}{Kimchijeon
(Kimchi Pancake)}

Advertisement

\protect\hyperlink{after-bottom}{Continue reading the main story}

\hypertarget{site-index}{%
\subsection{Site Index}\label{site-index}}

\hypertarget{site-information-navigation}{%
\subsection{Site Information
Navigation}\label{site-information-navigation}}

\begin{itemize}
\tightlist
\item
  \href{https://help.nytimes3xbfgragh.onion/hc/en-us/articles/115014792127-Copyright-notice}{©~2020~The
  New York Times Company}
\end{itemize}

\begin{itemize}
\tightlist
\item
  \href{https://www.nytco.com/}{NYTCo}
\item
  \href{https://help.nytimes3xbfgragh.onion/hc/en-us/articles/115015385887-Contact-Us}{Contact
  Us}
\item
  \href{https://www.nytco.com/careers/}{Work with us}
\item
  \href{https://nytmediakit.com/}{Advertise}
\item
  \href{http://www.tbrandstudio.com/}{T Brand Studio}
\item
  \href{https://www.nytimes3xbfgragh.onion/privacy/cookie-policy\#how-do-i-manage-trackers}{Your
  Ad Choices}
\item
  \href{https://www.nytimes3xbfgragh.onion/privacy}{Privacy}
\item
  \href{https://help.nytimes3xbfgragh.onion/hc/en-us/articles/115014893428-Terms-of-service}{Terms
  of Service}
\item
  \href{https://help.nytimes3xbfgragh.onion/hc/en-us/articles/115014893968-Terms-of-sale}{Terms
  of Sale}
\item
  \href{https://spiderbites.nytimes3xbfgragh.onion}{Site Map}
\item
  \href{https://help.nytimes3xbfgragh.onion/hc/en-us}{Help}
\item
  \href{https://www.nytimes3xbfgragh.onion/subscription?campaignId=37WXW}{Subscriptions}
\end{itemize}
