Sections

SEARCH

\protect\hyperlink{site-content}{Skip to
content}\protect\hyperlink{site-index}{Skip to site index}

\href{https://www.nytimes3xbfgragh.onion/section/travel}{Travel}

\href{https://myaccount.nytimes3xbfgragh.onion/auth/login?response_type=cookie\&client_id=vi}{}

\href{https://www.nytimes3xbfgragh.onion/section/todayspaper}{Today's
Paper}

\href{/section/travel}{Travel}\textbar{}2020 Is the Summer of the Road
Trip. Unless You're Black.

\url{https://nyti.ms/2MNr0lf}

\begin{itemize}
\item
\item
\item
\item
\item
\item
\end{itemize}

\hypertarget{race-and-america}{%
\subsubsection{\texorpdfstring{\href{https://www.nytimes3xbfgragh.onion/news-event/george-floyd-protests-minneapolis-new-york-los-angeles?name=styln-george-floyd\&region=TOP_BANNER\&block=storyline_menu_recirc\&action=click\&pgtype=Article\&impression_id=e0ac4b40-f1f9-11ea-8c13-f5b785c7d6f4\&variant=undefined}{Race
and America}}{Race and America}}\label{race-and-america}}

\begin{itemize}
\tightlist
\item
  \href{https://www.nytimes3xbfgragh.onion/2020/09/04/nyregion/rochester-police-daniel-prude.html?name=styln-george-floyd\&region=TOP_BANNER\&block=storyline_menu_recirc\&action=click\&pgtype=Article\&impression_id=e0ac4b41-f1f9-11ea-8c13-f5b785c7d6f4\&variant=undefined}{What
  Happened in Rochester, N.Y.}
\item
  \href{https://www.nytimes3xbfgragh.onion/2020/09/01/us/politics/trump-fact-check-protests.html?name=styln-george-floyd\&region=TOP_BANNER\&block=storyline_menu_recirc\&action=click\&pgtype=Article\&impression_id=e0ac4b42-f1f9-11ea-8c13-f5b785c7d6f4\&variant=undefined}{Trump
  Fact Check}
\item
  \href{https://www.nytimes3xbfgragh.onion/2020/08/30/us/portland-shooting-explained.html?name=styln-george-floyd\&region=TOP_BANNER\&block=storyline_menu_recirc\&action=click\&pgtype=Article\&impression_id=e0ac4b43-f1f9-11ea-8c13-f5b785c7d6f4\&variant=undefined}{Portland
  Shooting}
\item
  \href{https://www.nytimes3xbfgragh.onion/2020/08/30/us/breonna-taylor-police-killing.html?name=styln-george-floyd\&region=TOP_BANNER\&block=storyline_menu_recirc\&action=click\&pgtype=Article\&impression_id=e0ac4b44-f1f9-11ea-8c13-f5b785c7d6f4\&variant=undefined}{Breonna
  Taylor's Life and Death}
\end{itemize}

Advertisement

\protect\hyperlink{after-top}{Continue reading the main story}

Supported by

\protect\hyperlink{after-sponsor}{Continue reading the main story}

\hypertarget{2020-is-the-summer-of-the-road-trip-unless-youre-black}{%
\section{2020 Is the Summer of the Road Trip. Unless You're
Black.}\label{2020-is-the-summer-of-the-road-trip-unless-youre-black}}

The family road trip is making a comeback in the wake of the
coronavirus, but for African-American motorists, it's never been a
source of unfettered freedom.

\includegraphics{https://static01.graylady3jvrrxbe.onion/images/2020/06/10/travel/10roadtrips/10roadtrips-articleLarge.jpg?quality=75\&auto=webp\&disable=upscale}

\href{https://www.nytimes3xbfgragh.onion/by/tariro-mzezewa}{\includegraphics{https://static01.graylady3jvrrxbe.onion/images/2018/08/24/opinion/tariro-headshot/tariro-headshot-thumbLarge-v2.png}}

By \href{https://www.nytimes3xbfgragh.onion/by/tariro-mzezewa}{Tariro
Mzezewa}

\begin{itemize}
\item
  Published June 10, 2020Updated June 26, 2020
\item
  \begin{itemize}
  \item
  \item
  \item
  \item
  \item
  \item
  \end{itemize}
\end{itemize}

\emph{We want to hear about black travelers' experiences of taking a
road trip around America. See the form at the end of the article.}

If there's one thing the people behind car and R.V. companies, state
tourism boards, national and state parks and hotels agree on right now,
it's that the summer of 2020 will be the summer of the road trip.

With the country reopening, travel industry experts say people are
planning short trips to destinations relatively close to home. By
driving they can control the number of people they interact with, how
many stops they make on the way and whether to take a detour or not ---
all things they can't control on a plane.

``I've always loved the freedom of the road trip,'' one hotel owner said
in an email in April. ``It feels familiar, nostalgic, and very American.
Now, more than ever, when we are allowed to travel again, we expect to
see families, friends, and couples jumping into their cars and hitting
the open road.''

For many black travelers, however, the road trip has long conjured fear,
not freedom. Victor Hugo Green published the first version of his
now-famous ``Green Book'' in 1936; it listed towns, motels, restaurants
and homes where black drivers were welcome and would be safe. At the
time, state and local laws enforced racial segregation, primarily in the
South, a racial caste system known as Jim Crow that was legally undone
by the passage of Civil Rights legislation in the 1960s. The ``Green
Book'' was updated and published through the 1960s and inspired
\href{https://www.nytimes3xbfgragh.onion/2019/01/23/arts/green-book-interracial-friendship.html}{the
2018 film} of the same name that won an Oscar but was widely criticized
for making a white character's emotional journey its focus.

And while white travelers might convince themselves that the dangers the
``Green Book'' addressed have faded --- places where there is a high
likelihood of being stopped by the police, being harassed by fellow
travelers, or where it could be fatal to be seen after sundown --- for
many black travelers these dangers remain all too vivid.

Following the deaths of
\href{https://www.nytimes3xbfgragh.onion/2020/05/31/us/george-floyd-investigation.html}{George
Floyd} and
\href{https://www.nytimes3xbfgragh.onion/article/breonna-taylor-police.html}{Breonna
Taylor} at the hands of the police, and
\href{https://www.nytimes3xbfgragh.onion/article/ahmaud-arbery-shooting-georgia.html}{Ahmaud
Arbery} at the hands of armed white residents, and coming on the heels
of the coronavirus and its heavy toll --- both in terms of
\href{https://slack-redir.net/link?url=https\%3A\%2F\%2Fwww.nytimes3xbfgragh.onion\%2F2020\%2F04\%2F07\%2Fus\%2Fcoronavirus-race.html}{health}
and
\href{https://slack-redir.net/link?url=https\%3A\%2F\%2Fwww.nytimes3xbfgragh.onion\%2F2020\%2F06\%2F01\%2Fbusiness\%2Feconomy\%2Fblack-workers-inequality-economic-risks.html}{employment}
--- on African-Americans, some black travelers worry that they will face
even more discrimination on the road this summer.

``Travel is supposed to be a reprieve from all the hard things we are
usually dealing with, but it often doesn't feel that way for us,'' said
Damon Lawrence, co-founder of Homage Hospitality Group, a hotel company
that draws \href{http://www.stayhomage.com/themoor}{inspiration for its
properties} from black history and caters particularly to black
travelers. ``Having to constantly be on high alert adds extra anxiety,
and it's always hard, but right now, it's an exhausting task to even
leave the house, let alone go on a road trip.''

Mr. Lawrence, like many other African-Americans, said that he always
shares his location with friends and family on his phone, so that
someone can check in and know where he is.

``If something goes wrong, I need someone to know where I am or where
I've been,'' he said.

\hypertarget{a-lot-of-planning-and-no-detours}{%
\subsection{A lot of planning, and no
detours}\label{a-lot-of-planning-and-no-detours}}

Nisha Parker, a special-education teacher in Bakersfield, Calif., loves
to drive and doesn't want to allow fear about what could go wrong to
stop her. She also wants her two children to see America's landscapes,
she said. So this summer her family will drive across the country from
their hometown to New York.

But Ms. Parker, 32, said that she can't imagine just being able to pack
up and go without a plan, like some white families might be able to do.

So for the last six months, she has been meticulously planning their
journey. She knows which towns her family will stop in, which they'll
drive straight through, and which they'll avoid entirely. She also knows
which stretches of the road her children won't be allowed to drink juice
or water on, to avoid bathroom breaks in towns where the family could
encounter racism or violence based on their race.

``We try not to stop in places that are desolate and we try to only stop
in cities for gas,'' she said. ``If we have to stop for gas in a rural
area, we use a debit card so we don't have to go into the gas station
store. If we are going to stay somewhere overnight, we look at the
demographics to make sure we aren't going to a place where we would be
the only black people or where we would be targeted, especially at
night.''

Ms. Parker grew up road tripping with family between New York and North
Carolina, and her parents took similar precautions. She and her husband
have also considered getting a dashboard camera, so that if they are
stopped by police and things turn deadly there is some record of it.

In a way, Facebook groups for black travelers and group chats have
become the 21st-century version of the ``Green Book.'' People talk about
where they've been and follow in each other's footsteps, sharing where
they were treated well and where they felt uncomfortable or unsafe. Many
stay in the same hotels, eat at the same restaurants or skip the same
towns.

``We go where our friends and family have gone because we know that it's
safe,'' said Dianelle Rivers-Mitchell, founder of
\href{https://www.blackgirlstraveltoo.com/}{Black Girls Travel Too}, a
group tour company for black women. ``During this moment, with the
protests as a backdrop, and as our community deals with how we were
harder hit by coronavirus and we risk facing even more discrimination
based on that, I just don't see road-tripping being it for us.''

The so-called sundown towns --- where black people were effectively
banned after dark and where those who stayed too late were attacked by
white mobs --- no longer exist, but, for some black drivers, the fear of
getting lost or stuck in a town where being black could lead to violence
is a real concern that affects how a road trip is planned.

Monica Jackson, a medical biller for a hospital network in Texas said
that she loves to drive, but as a rule she will not go on a trip that
requires driving for more than six hours, so that she doesn't have to
consider spending the night in a town where she could be targeted for
being black.

\hypertarget{deadly-police-stops}{%
\subsection{Deadly police stops}\label{deadly-police-stops}}

Ms. Jackson, 42, said that she feels anxious when she passes through
areas --- including Texas's Williamson and Denton counties --- where
she's had unnerving interactions with white police officers.

``I always feel worried on the road in some counties because I've been
stopped for no reason,'' she said. ``I always pray and say, `OK, Lord
please protect me. I don't want to end up in jail for no reason.' It's
always in the back of my mind that I could be the next Sandra Bland.''

Ms. Bland was a vocal civil rights activist
\href{https://www.nytimes3xbfgragh.onion/2019/05/07/us/sandra-bland-brian-encinia.html}{who
was found hanged} in a Texas jail cell in July 2015 after she was
arrested during a traffic stop.

Brian Oliver, founder of \href{https://bmoreseemore.org/}{BMore See
More}, a nonprofit that works with black male students and encourages
them to travel, said that he used to be worried about driving in the
Deep South, but videos of black men being killed by the police or
targeted by white Americans have shown him that racist violence can
occur anywhere.

``There used to be a sense of some places being less safe for black
people, but from seeing the news lately, I don't think there's any place
that's guaranteed to be safe for us,'' he said.

Mr. Oliver, 35, who also runs the blog
``\href{https://beyondbmore.com/}{Beyond Bmore},'' said that in 2013 he
and three friends --- all black men --- drove overnight from Atlanta to
New Orleans for Super Bowl XLVII, and for the duration of the drive,
they veiled their anxieties and fears in jokes about whether they would
arrive safely.

``We were in the car laughing about not stopping in this place or that
place, but the sad truth is that we all knew that we really couldn't
stop in some of those places,'' Mr. Oliver said. ``It's crazy to try and
describe the kind of threat and fear you feel at the prospect of getting
lost, losing signal and not really knowing where you are.''

Mr. Lawrence of Homage Hospitality said that a desire to make black
travelers feel welcome and able to relax without worrying that someone
might
\href{https://www.nytimes3xbfgragh.onion/2018/05/08/us/airbnb-black-women-police.html}{call
the police on them simply for checking in} was one of the main reasons
he created his company.

\hypertarget{the-right-to-be-on-the-road}{%
\subsection{The right to be on the
road}\label{the-right-to-be-on-the-road}}

Too often, black travelers say, they are made to feel like they don't
belong. That feeling was heightened recently by
\href{https://www.nytimes3xbfgragh.onion/article/ahmaud-arbery-shooting-georgia.html}{the
killing of Mr. Arbery} while running in a Georgia suburb and by the
false accusation that a black
\href{https://www.nytimes3xbfgragh.onion/2020/05/27/nyregion/amy-cooper-christian-central-park-video.html}{birdwatcher
in New York's Central Park} was assaulting a white woman after he'd
asked her to leash her dog.

And that feeling extends to the open road, so celebrated in American
literature and film, but where many black drivers said that when stopped
by the police, the first questions they're met with are often ``Where
are you going?'' or ``Why are you here?''

``It's as if they are saying why are you in this area?'' Mr. Oliver
said. ``I have every right to be here just as much as you have the right
to be on this road.''

Jeff Jenkins, a travel blogger who runs
\href{https://www.chubbydiaries.com/}{Chubby Diaries}, a travel company
for plus-size people, said that his anxiety about being targeted by the
police ran so deep that it affected his choice of car. The recent
killings of black men by the police have only added to his anxiety.

``I go for soccer-mom cars because they seem to be less intimidating to
the police,'' he said. ``A typical sedan or something that sort of just
says, `I'm safe and boring, don't look at me.'''

Mr. Jenkins, 34, is planning on driving from Austin to visit several
national parks this summer, he said, adding that in recent weeks he has
become ``an R.V. savant.''

``These opportunities, these parks, these roads are meant for me as
well,'' Mr. Jenkins said. ``They are not meant to just be shared with
one ethnicity. I have pride that this is my country, and I have every
right to bask in the wonders of America, like any white American.''

\hypertarget{we-want-to-hear-about-black-travelers-experiences-of-taking-a-road-trip-around-america}{%
\subsection{We want to hear about black travelers' experiences of taking
a road trip around
America.}\label{we-want-to-hear-about-black-travelers-experiences-of-taking-a-road-trip-around-america}}

\emph{\textbf{Follow New York Times Travel}} \emph{on}
\href{https://www.instagram.com/nytimestravel/}{\emph{Instagram}}\emph{,}
\href{https://twitter.com/nytimestravel}{\emph{Twitter}} \emph{and}
\href{https://www.facebookcorewwwi.onion/nytimestravel/}{\emph{Facebook}}\emph{.
And}
\href{https://www.nytimes3xbfgragh.onion/newsletters/traveldispatch}{\emph{sign
up for our weekly Travel Dispatch newsletter}} \emph{to receive expert
tips on traveling smarter and inspiration for your next vacation.}

Advertisement

\protect\hyperlink{after-bottom}{Continue reading the main story}

\hypertarget{site-index}{%
\subsection{Site Index}\label{site-index}}

\hypertarget{site-information-navigation}{%
\subsection{Site Information
Navigation}\label{site-information-navigation}}

\begin{itemize}
\tightlist
\item
  \href{https://help.nytimes3xbfgragh.onion/hc/en-us/articles/115014792127-Copyright-notice}{©~2020~The
  New York Times Company}
\end{itemize}

\begin{itemize}
\tightlist
\item
  \href{https://www.nytco.com/}{NYTCo}
\item
  \href{https://help.nytimes3xbfgragh.onion/hc/en-us/articles/115015385887-Contact-Us}{Contact
  Us}
\item
  \href{https://www.nytco.com/careers/}{Work with us}
\item
  \href{https://nytmediakit.com/}{Advertise}
\item
  \href{http://www.tbrandstudio.com/}{T Brand Studio}
\item
  \href{https://www.nytimes3xbfgragh.onion/privacy/cookie-policy\#how-do-i-manage-trackers}{Your
  Ad Choices}
\item
  \href{https://www.nytimes3xbfgragh.onion/privacy}{Privacy}
\item
  \href{https://help.nytimes3xbfgragh.onion/hc/en-us/articles/115014893428-Terms-of-service}{Terms
  of Service}
\item
  \href{https://help.nytimes3xbfgragh.onion/hc/en-us/articles/115014893968-Terms-of-sale}{Terms
  of Sale}
\item
  \href{https://spiderbites.nytimes3xbfgragh.onion}{Site Map}
\item
  \href{https://help.nytimes3xbfgragh.onion/hc/en-us}{Help}
\item
  \href{https://www.nytimes3xbfgragh.onion/subscription?campaignId=37WXW}{Subscriptions}
\end{itemize}
