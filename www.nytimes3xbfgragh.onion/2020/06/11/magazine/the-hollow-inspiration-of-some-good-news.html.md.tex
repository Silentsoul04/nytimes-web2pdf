Sections

SEARCH

\protect\hyperlink{site-content}{Skip to
content}\protect\hyperlink{site-index}{Skip to site index}

\href{https://myaccount.nytimes3xbfgragh.onion/auth/login?response_type=cookie\&client_id=vi}{}

\href{https://www.nytimes3xbfgragh.onion/section/todayspaper}{Today's
Paper}

The Hollow Inspiration of `Some Good News'

\begin{itemize}
\item
\item
\item
\item
\item
\item
\end{itemize}

\hypertarget{race-and-america}{%
\subsubsection{\texorpdfstring{\href{https://www.nytimes3xbfgragh.onion/news-event/george-floyd-protests-minneapolis-new-york-los-angeles?name=styln-george-floyd\&region=TOP_BANNER\&block=storyline_menu_recirc\&action=click\&pgtype=Article\&impression_id=8acbecd0-f1e0-11ea-9518-95db608f788b\&variant=undefined}{Race
and America}}{Race and America}}\label{race-and-america}}

\begin{itemize}
\tightlist
\item
  \href{https://www.nytimes3xbfgragh.onion/2020/09/04/nyregion/rochester-police-daniel-prude.html?name=styln-george-floyd\&region=TOP_BANNER\&block=storyline_menu_recirc\&action=click\&pgtype=Article\&impression_id=8acbecd1-f1e0-11ea-9518-95db608f788b\&variant=undefined}{What
  Happened in Rochester, N.Y.}
\item
  \href{https://www.nytimes3xbfgragh.onion/2020/09/01/us/politics/trump-fact-check-protests.html?name=styln-george-floyd\&region=TOP_BANNER\&block=storyline_menu_recirc\&action=click\&pgtype=Article\&impression_id=8acc13e0-f1e0-11ea-9518-95db608f788b\&variant=undefined}{Trump
  Fact Check}
\item
  \href{https://www.nytimes3xbfgragh.onion/2020/08/30/us/portland-shooting-explained.html?name=styln-george-floyd\&region=TOP_BANNER\&block=storyline_menu_recirc\&action=click\&pgtype=Article\&impression_id=8acc13e1-f1e0-11ea-9518-95db608f788b\&variant=undefined}{Portland
  Shooting}
\item
  \href{https://www.nytimes3xbfgragh.onion/2020/08/30/us/breonna-taylor-police-killing.html?name=styln-george-floyd\&region=TOP_BANNER\&block=storyline_menu_recirc\&action=click\&pgtype=Article\&impression_id=8acc13e2-f1e0-11ea-9518-95db608f788b\&variant=undefined}{Breonna
  Taylor's Life and Death}
\end{itemize}

Advertisement

\protect\hyperlink{after-top}{Continue reading the main story}

Supported by

\protect\hyperlink{after-sponsor}{Continue reading the main story}

\href{/column/screenland}{Screenland}

\hypertarget{the-hollow-inspiration-of-some-good-news}{%
\section{The Hollow Inspiration of `Some Good
News'}\label{the-hollow-inspiration-of-some-good-news}}

\includegraphics{https://static01.graylady3jvrrxbe.onion/images/2020/06/14/magazine/14Mag-Screenland-01/14Mag-Screenland-01-articleLarge.jpg?quality=75\&auto=webp\&disable=upscale}

By Alex Norcia

\begin{itemize}
\item
  Published June 11, 2020Updated June 12, 2020
\item
  \begin{itemize}
  \item
  \item
  \item
  \item
  \item
  \item
  \end{itemize}
\end{itemize}

For two months, John Krasinski looked on the bright side. Beginning in
late March and ending in late May, the actor served as the anchor for
eight episodes of a YouTube show he called ``Some Good News,'' or
``S.G.N.'' for short --- about 20 minutes, each week, focused entirely
on feel-good content. The show was a never-ending D-block, as if that
final, uplifting section of a news broadcast had been stocked with
dancing health professionals and put on an endless loop.

The final episode of "Some Good News."Credit...CreditVideo by
SomeGoodNews

It was intended as relief for a pent-up nation, an antidote to the
country's depressing reality: the increasing Covid-19 death toll, the
rising unemployment and food insecurity, the social isolation. Krasinski
enlisted his best-known friends to cheer people up. Lin-Manuel Miranda
serenaded a little girl who missed a performance of ``Hamilton'' in
Florida. Oprah Winfrey gave advice to a graduating college student.
Nurses in Boston tossed out the ``first pitch'' in an empty Fenway Park.
Millions, sheltered in place, tuned in every week.

There was a ``Sesame Street'' quality to the production. The set had the
aesthetics of upper-middle-class quarantine --- a single room, decorated
by Krasinski's young daughters with handcrafted posters --- and the
segments, while often mawkish, could be genuinely sweet. Krasinski could
be amusing, too, leavening the sentiment with a bit of Ron Burgundy in
his voice. He even developed a signature signoff, a running gag. Each
episode found him seated and wearing --- as far as anyone could tell
from above the desk --- a suit. But before the credits rolled, he would
stand up to reveal he wasn't in the proper trousers. ``I guarantee you
the bottom half of what you have on does not match the top,'' Steve
Carell said, appearing in the first episode. He was right: Krasinski was
in swim trunks. After hosting a virtual wedding, Krasinski unveiled a
``Just Married'' sign tied across his waist. After a Zoom prom for a
group of high schoolers, he rose to reveal that what had seemed to be a
pink shirt was actually a flowing dress.

And then, in the eighth and final episode, Krasinski stood to reveal,
finally, the matching pants. He buttoned his jacket. The suit was
complete. And a few days later, it was announced that he had sold
``S.G.N.'' to ViacomCBS.

The arc of ``S.G.N.'' parallels, in its way, the path of the nation over
those same eight weeks. The show started out offering a charming break
from bleakness --- a sign of a nation improvising new ways to help one
another through a crisis. It ended as a reminder of how little had
really changed. Soon, there would be fire in the streets.

When the first episode ran, the United States was close to a full-on
lockdown, warily bracing for the pandemic's impact. The crisis wasn't
entirely partisan yet. Homebound optimists stepped out each night to
applaud essential workers; they looked for ways to be together, apart.
If we were careful, if we kept washing our hands, we could defeat the
common enemy. Yes, the pandemic was further exposing so much we already
knew --- that generations of young people would be left in financial
ruin, that inequality and racial inequity were expanding. But perhaps
things had to get bad before they got better. ``S.G.N.'' fit perfectly
into that padded narrative: Good would prevail, love would conquer all.

By the time Krasinski managed to get his pants on, though, the old order
of things had firmly reasserted itself --- and not just in the easing of
restrictions in many states. The coronavirus, it turned out, had
reshuffled very little about how we do things. It had taken Americans
mere weeks to arrange themselves into vocal, ideological and
occasionally armed factions over it. Corporations that had briefly gone
silent figured out how to advertise their way through a pandemic,
crafting sentimental commercials about honoring front-line heroes ---
the same workers who, in many instances, were speaking out against their
unsafe working conditions. ``S.G.N.'' was no stranger to this
phenomenon: By the third episode, AT\&T had swooped in to provide free
cellphone service to nurses and doctors, and in a display of gratitude,
Krasinski's daughters drew the company's logo, a product placement as
cute as it was blatant.

Eventually, the homespun quarantine-relief show would become an ordinary
corporate property. The sale was clearly a disappointment to the
community Krasinski had fostered, or at least to the many who commented
to accuse him of selling out. The actor said previous commitments, and
the unsustainable cost of production, meant he had never intended to
make more than a handful of episodes.

Whatever the reason, Krasinski bowed out at the most opportune possible
moment. It is nearly impossible to imagine how ``S.G.N.'' would function
right now. Cities have erupted in protest against police brutality.
Curfews have been instituted; the National Guard has been deployed.
Federal agents laid a path of tear gas so the president could have a
photo opportunity outside a church. Iconic image replaces iconic image:
journalists shot with rubber bullets and arrested on live television,
peaceful protesters doused with chemicals, looters busting into stores
along Fifth Avenue. What would it look like now for Krasinski to show us
Brad Pitt cameos and cooped-up families playing household sports? We
have some idea: He recently tweeted a video of a child singing and
protesters embracing police, reminding us about the power of learning
and growth. It doesn't land as the broadcast did.

The difficulty of imagining another ``S.G.N.'' exposes what was nagging
about it all along. It was a well-intended distraction, but it was, as
with so much else, insufficient to the circumstances. When civil unrest
is worsening, it's difficult not to become keenly aware of when, and how
often, we're choosing to soothe ourselves and bury our heads in sand.
This even felt true early in Krasinski's run. Each night's applause for
health workers was dogged by the knowledge that we could not seem to
provide them with basic protections. We could improvise ceremonies for
our graduates, but we could not help them with the colossal burden of
their student debt. So often the sentimental gesture, deeply satisfying
on the surface, only covers our inability to act. By early this month,
for instance, many Instagram users were eagerly expressing their
solidarity with protesters by posting blank squares --- an action that
ended up drowning out actual information about the demonstrations. The
same day, Nancy Pelosi rebuked the president's waving a Bible like a
prop by reading to reporters from her own.

There was only ever one outcome available to ``S.G.N.'': Krasinski was
always going to put on his pants. Because before the good news, we may
have to hear all the bad, and act on it.

Advertisement

\protect\hyperlink{after-bottom}{Continue reading the main story}

\hypertarget{site-index}{%
\subsection{Site Index}\label{site-index}}

\hypertarget{site-information-navigation}{%
\subsection{Site Information
Navigation}\label{site-information-navigation}}

\begin{itemize}
\tightlist
\item
  \href{https://help.nytimes3xbfgragh.onion/hc/en-us/articles/115014792127-Copyright-notice}{©~2020~The
  New York Times Company}
\end{itemize}

\begin{itemize}
\tightlist
\item
  \href{https://www.nytco.com/}{NYTCo}
\item
  \href{https://help.nytimes3xbfgragh.onion/hc/en-us/articles/115015385887-Contact-Us}{Contact
  Us}
\item
  \href{https://www.nytco.com/careers/}{Work with us}
\item
  \href{https://nytmediakit.com/}{Advertise}
\item
  \href{http://www.tbrandstudio.com/}{T Brand Studio}
\item
  \href{https://www.nytimes3xbfgragh.onion/privacy/cookie-policy\#how-do-i-manage-trackers}{Your
  Ad Choices}
\item
  \href{https://www.nytimes3xbfgragh.onion/privacy}{Privacy}
\item
  \href{https://help.nytimes3xbfgragh.onion/hc/en-us/articles/115014893428-Terms-of-service}{Terms
  of Service}
\item
  \href{https://help.nytimes3xbfgragh.onion/hc/en-us/articles/115014893968-Terms-of-sale}{Terms
  of Sale}
\item
  \href{https://spiderbites.nytimes3xbfgragh.onion}{Site Map}
\item
  \href{https://help.nytimes3xbfgragh.onion/hc/en-us}{Help}
\item
  \href{https://www.nytimes3xbfgragh.onion/subscription?campaignId=37WXW}{Subscriptions}
\end{itemize}
