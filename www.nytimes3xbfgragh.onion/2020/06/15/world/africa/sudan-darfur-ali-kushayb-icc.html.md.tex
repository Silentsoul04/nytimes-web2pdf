Sections

SEARCH

\protect\hyperlink{site-content}{Skip to
content}\protect\hyperlink{site-index}{Skip to site index}

\href{https://www.nytimes3xbfgragh.onion/section/world/africa}{Africa}

\href{https://myaccount.nytimes3xbfgragh.onion/auth/login?response_type=cookie\&client_id=vi}{}

\href{https://www.nytimes3xbfgragh.onion/section/todayspaper}{Today's
Paper}

\href{/section/world/africa}{Africa}\textbar{}After 13 Years on the Run,
a Sudanese Militia Leader Appears in Court

\url{https://nyti.ms/3e8LMYw}

\begin{itemize}
\item
\item
\item
\item
\item
\end{itemize}

Advertisement

\protect\hyperlink{after-top}{Continue reading the main story}

Supported by

\protect\hyperlink{after-sponsor}{Continue reading the main story}

\hypertarget{after-13-years-on-the-run-a-sudanese-militia-leader-appears-in-court}{%
\section{After 13 Years on the Run, a Sudanese Militia Leader Appears in
Court}\label{after-13-years-on-the-run-a-sudanese-militia-leader-appears-in-court}}

A case in the International Criminal Court against Ali Kushayb is the
first to address the mass killings of civilians by Sudan's armed forces
and government-backed militias in Darfur.

\includegraphics{https://static01.graylady3jvrrxbe.onion/images/2020/06/15/world/15sudan-icc1/merlin_173554353_b0baadff-cc60-429a-a5af-ccf9ad6aa6c6-articleLarge.jpg?quality=75\&auto=webp\&disable=upscale}

By \href{https://www.nytimes3xbfgragh.onion/by/marlise-simons}{Marlise
Simons}

\begin{itemize}
\item
  Published June 15, 2020Updated July 30, 2020
\item
  \begin{itemize}
  \item
  \item
  \item
  \item
  \item
  \end{itemize}
\end{itemize}

A Sudanese militia leader appeared before the International Criminal
Court on Monday, after 13 years on the run, to face charges of
\href{https://www.nytimes3xbfgragh.onion/2019/12/22/world/africa/sudan-darfur-investigation.html}{war
crimes and crimes against humanity} committed during the bloody conflict
in the
\href{https://www.nytimes3xbfgragh.onion/2020/07/30/world/middleeast/darfur-sudan.html}{Sudanese
region of Darfu}r from 2003 to 2004.

The expected trial of
\href{https://www.nytimes3xbfgragh.onion/2020/06/09/world/africa/ali-kushayb-in-custody.html}{the
militia leader, identified in court documents as Ali Kushayb}, is the
first to address the destruction of several hundred villages and the
mass killings of civilians as Sudan's armed forces and government-backed
militias crushed a rebellion that over the past two decades has taken an
estimated 300,000 lives.

Fatou Bensouda, the court's chief prosecutor, called Mr. Kushayb's
detention ``pivotal'' at the U.N. Security Council last week in
discussing the court's pursuit of justice for victims of the Darfur
conflict and stressed the court's ``unwavering'' commitment to them.

``There should be no escape from justice for perpetrators of the world's
most serious crimes under international law,'' she said.

The developments are a victory of sorts for an international court that
has been the target of scorn by the United States --- which is not a
member of the court~--- most notably over its efforts to investigate
Americans over potential war crimes in Afghanistan. Last week, Secretary
of State Mike Pompeo called it ``a kangaroo court.''

The militia leader surrendered to the authorities in the Central African
Republic last week and asked to be handed over to the International
Criminal Court at its office there, more than a decade after the court
issued a warrant for his arrest.

His apparent motive, according to Sudan's public prosecutor, was an
attempt to save his life after Sudan issued an arrest warrant for him
last year in the wake of the ouster of the country's longtime ruler,
\href{https://www.nytimes3xbfgragh.onion/2020/07/30/world/middleeast/darfur-sudan.html}{President
Omar Hassan al-Bashir.}

The warrant charged him with murder, theft, rape and violence against
women. A~conviction in Sudan could potentially bring the death penalty,
whereas the maximum punishment at the International Criminal Court is a
life sentence.

For his initial appearance on Monday at the international court in The
Hague, where he arrived last week, he was not brought into the courtroom
because of health precautions related to the coronavirus.

Instead, he participated via a video link from the court's jail, telling
the judge that he wanted to be known by his real name, Ali Abd-Al-Rahman
and indicating that he was 70 years old.

After the prosecution read out the charges against him --- 50 counts of
war crimes and crimes against humanity --- he briefly said, ``What I
heard does not apply to me.'' He also requested a minute of silence
``for all the victims,'' but the judge refused.

He was not asked to enter any plea during the hourlong procedural
hearing.

The case against Mr. Kushayb at the international court focuses on his
activities as a leader of the janjaweed militias, whom Mr. al-Bashir's
government recruited to crush a rebel movement in the Darfur region.

The prosecution has accused Mr. Kushayb of commanding thousands of
janjaweed in 2003 and 2004, in addition to arming, funding and providing
food and other supplies to them.

\includegraphics{https://static01.graylady3jvrrxbe.onion/images/2020/06/15/world/15sudan-icc2/merlin_173554368_64acc884-3d47-4d6f-a65a-dd521273e857-articleLarge.jpg?quality=75\&auto=webp\&disable=upscale}

The prosecution said in its accusation that Mr. Kushayb had ``personally
participated in some of the attacks against civilians'' in several towns
``where the killing of civilians, rape, torture and other cruel
treatments occurred.''

Court documents say he also coordinated a campaign that involved
bombings by the Sudanese air force and attacks by other armed units that
joined the militias in carrying out scorched-earth tactics, burning
villages and forcing an estimated 2.7 million people to flee. Many still
live in refugee camps, including in Darfur and across the border in
Chad.

The International Criminal Court issued arrest warrants for Mr.
al-Bashir in 2009 and 2010 on charges of genocide and multiple
atrocities.

Mr. al-Bashir had traveled abroad despite the warrants from the court,
and avoided arrest until he was overthrown in a military coup in 2019.
He was convicted of corruption in December.

His ouster reawakened hope in The Hague that he would be sent there for
trial. Sudan's new military-backed government initially remained silent
on the issue, but more recently, amid peace talks with Darfur rebels,
\href{https://www.hrw.org/news/2020/02/12/sudan-opens-door-icc-prosecutions}{news
reports} said that
\href{https://apnews.com/c6698024bdd7f1cade89b9b4101d25c1}{senior
military officials} planned to hand over Mr. al-Bashir and two other
officials wanted by the court.

Ms. Bensouda, who said last week that she had not received direct
confirmation of Khartoum's intentions, has renewed calls for Sudan to
hand over Mr. al-Bashir; the former ministers Abdel-Rahim Hussein and
Ahmad Harun; and the rebel leader Abdallah Banda.

The arrest of Mr. Kushayb comes against the backdrop of a dispute
between the court and the Trump administration, which said last week
that court officials who participated in investigating possible war
crimes by Americans in Afghanistan would face economic sanctions and
travel restrictions, as would their families.

In response to Mr. Pompeo, the court said in a statement that ``these
attacks constitute an escalation and an unacceptable attempt to
interfere with the rule of law and the court's judicial proceedings.''

It said that ``an attack on the I.C.C. also represents an attack against
the interests of victims of atrocity crimes, for many of whom the court
represents the last hope for justice.''

The exchange came after an attack on the court by Mr. Pompeo in March
drew outrage both at the court and in international legal circles. At
that time, he stunned lawyers in The Hague when he singled out two
senior staff members in the prosecution office and said that they and
others were ``putting Americans at risk.''

At the hearing on Monday, the judge set the next hearing for Dec. 7 to
give the prosecution and defense time to prepare their cases. At that
time, the court will decide whether the prosecution has enough evidence
to proceed to trial.

Advertisement

\protect\hyperlink{after-bottom}{Continue reading the main story}

\hypertarget{site-index}{%
\subsection{Site Index}\label{site-index}}

\hypertarget{site-information-navigation}{%
\subsection{Site Information
Navigation}\label{site-information-navigation}}

\begin{itemize}
\tightlist
\item
  \href{https://help.nytimes3xbfgragh.onion/hc/en-us/articles/115014792127-Copyright-notice}{©~2020~The
  New York Times Company}
\end{itemize}

\begin{itemize}
\tightlist
\item
  \href{https://www.nytco.com/}{NYTCo}
\item
  \href{https://help.nytimes3xbfgragh.onion/hc/en-us/articles/115015385887-Contact-Us}{Contact
  Us}
\item
  \href{https://www.nytco.com/careers/}{Work with us}
\item
  \href{https://nytmediakit.com/}{Advertise}
\item
  \href{http://www.tbrandstudio.com/}{T Brand Studio}
\item
  \href{https://www.nytimes3xbfgragh.onion/privacy/cookie-policy\#how-do-i-manage-trackers}{Your
  Ad Choices}
\item
  \href{https://www.nytimes3xbfgragh.onion/privacy}{Privacy}
\item
  \href{https://help.nytimes3xbfgragh.onion/hc/en-us/articles/115014893428-Terms-of-service}{Terms
  of Service}
\item
  \href{https://help.nytimes3xbfgragh.onion/hc/en-us/articles/115014893968-Terms-of-sale}{Terms
  of Sale}
\item
  \href{https://spiderbites.nytimes3xbfgragh.onion}{Site Map}
\item
  \href{https://help.nytimes3xbfgragh.onion/hc/en-us}{Help}
\item
  \href{https://www.nytimes3xbfgragh.onion/subscription?campaignId=37WXW}{Subscriptions}
\end{itemize}
