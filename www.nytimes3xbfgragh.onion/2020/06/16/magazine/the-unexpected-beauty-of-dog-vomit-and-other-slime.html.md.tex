Sections

SEARCH

\protect\hyperlink{site-content}{Skip to
content}\protect\hyperlink{site-index}{Skip to site index}

\href{https://myaccount.nytimes3xbfgragh.onion/auth/login?response_type=cookie\&client_id=vi}{}

\href{https://www.nytimes3xbfgragh.onion/section/todayspaper}{Today's
Paper}

The Unexpected Beauty of `Dog Vomit' and Other Slime Molds

\begin{itemize}
\item
\item
\item
\item
\item
\end{itemize}

\hypertarget{the-coronavirus-outbreak}{%
\subsubsection{\texorpdfstring{\href{https://www.nytimes3xbfgragh.onion/news-event/coronavirus?name=styln-coronavirus-national\&region=TOP_BANNER\&block=storyline_menu_recirc\&action=click\&pgtype=Article\&impression_id=1d0d90d0-f1be-11ea-b289-89f7aaa81ae4\&variant=undefined}{The
Coronavirus
Outbreak}}{The Coronavirus Outbreak}}\label{the-coronavirus-outbreak}}

\begin{itemize}
\tightlist
\item
  live\href{https://www.nytimes3xbfgragh.onion/2020/09/08/world/covid-19-coronavirus.html?name=styln-coronavirus-national\&region=TOP_BANNER\&block=storyline_menu_recirc\&action=click\&pgtype=Article\&impression_id=1d0d90d1-f1be-11ea-b289-89f7aaa81ae4\&variant=undefined}{Latest
  Updates}
\item
  \href{https://www.nytimes3xbfgragh.onion/interactive/2020/us/coronavirus-us-cases.html?name=styln-coronavirus-national\&region=TOP_BANNER\&block=storyline_menu_recirc\&action=click\&pgtype=Article\&impression_id=1d0d90d2-f1be-11ea-b289-89f7aaa81ae4\&variant=undefined}{Maps
  and Cases}
\item
  \href{https://www.nytimes3xbfgragh.onion/interactive/2020/science/coronavirus-vaccine-tracker.html?name=styln-coronavirus-national\&region=TOP_BANNER\&block=storyline_menu_recirc\&action=click\&pgtype=Article\&impression_id=1d0db7e0-f1be-11ea-b289-89f7aaa81ae4\&variant=undefined}{Vaccine
  Tracker}
\item
  \href{https://www.nytimes3xbfgragh.onion/2020/09/02/your-money/eviction-moratorium-covid.html?name=styln-coronavirus-national\&region=TOP_BANNER\&block=storyline_menu_recirc\&action=click\&pgtype=Article\&impression_id=1d0db7e1-f1be-11ea-b289-89f7aaa81ae4\&variant=undefined}{Eviction
  Moratorium}
\item
  \href{https://www.nytimes3xbfgragh.onion/interactive/2020/09/02/magazine/food-insecurity-hunger-us.html?name=styln-coronavirus-national\&region=TOP_BANNER\&block=storyline_menu_recirc\&action=click\&pgtype=Article\&impression_id=1d0db7e2-f1be-11ea-b289-89f7aaa81ae4\&variant=undefined}{American
  Hunger}
\end{itemize}

Advertisement

\protect\hyperlink{after-top}{Continue reading the main story}

Supported by

\protect\hyperlink{after-sponsor}{Continue reading the main story}

\href{/column/letter-of-recommendation}{Letter of Recommendation}

\hypertarget{the-unexpected-beauty-of-dog-vomit-and-other-slime-molds}{%
\section{The Unexpected Beauty of `Dog Vomit' and Other Slime
Molds}\label{the-unexpected-beauty-of-dog-vomit-and-other-slime-molds}}

\includegraphics{https://static01.graylady3jvrrxbe.onion/images/2020/06/21/magazine/21Mag-LOR-01/21Mag-LOR-01-articleLarge.png?quality=75\&auto=webp\&disable=upscale}

By Daniel Mason

\begin{itemize}
\item
  June 16, 2020
\item
  \begin{itemize}
  \item
  \item
  \item
  \item
  \item
  \end{itemize}
\end{itemize}

They say everybody needs a hobby, and for the past six months, mine has
been taking videos and photographs of slime molds in the oak and redwood
forests near my home in California. Perhaps we've met somewhere on the
trail? I was the guy on the ground with a lens clipped to his iPhone,
the patient wife, the children asking worriedly from a distance whether
a plant was poison oak. Perhaps you may have recognized my
``slime-molding outfit'': the frayed Y2K jacket and Old Navy cargo pants
caked with mud of many vintages. Perhaps, if you dared ask what I was
doing, I popped up and (in those days when strangers stood close to one
another) showed you my pictures.

I must have heard of slime molds at some point before the rabbit hole
opened violently beneath me last December. For a tiny organism that
doesn't kill people, it hits the news with some regularity. Since making
\href{https://timesmachine.nytimes3xbfgragh.onion/timesmachine/1973/05/31/99147556.html?pageNumber=82}{national
headlines} almost 50 years ago as a ``mysterious, encompassing ooze'' in
a Texas backyard, slime molds have recurrently earned their 15 minutes
for such improbable feats as recreating a
\href{https://www.livescience.com/8035-slime-mold-beats-humans-perfecting-traffic-networks.html}{map
of the Tokyo subway system} and modeling the distribution of
\href{https://www.nasa.gov/feature/goddard/2020/slime-mold-simulations-used-to-map-dark-matter-holding-universe-together}{dark
matter in the universe.}

What sent me into the woods that month was not the creeping blobs of
horror movies but rather a photo of iridescent purple Lamproderma posted
to the citizen-science project iNaturalist. It was unlike anything I'd
seen: Imagine ranks of glass-head pins in the belly of a leaf, and now
imagine that those pinheads were made of frosted amethyst. This, I would
learn, was the sporangial form of a slime mold. The pulsating blobs of
Texas-backyard mystery are plasmodia --- multinucleate masses that
course the forest until conditions are just right for them to form the
colorful fruiting bodies full of spores.

My fascination --- immediate, convulsive --- came not only from the
beauty but also from my realization that such beauty had surrounded me
during decades of wandering these woods, without my ever having noticed
it. In my defense, slime molds make no appearance in my parents' old
Audubon guides, and the sole image in my mille-feuille college biology
textbook is a blurry yellow smear. The only literary slime molds I know
of are meditations in the notebooks of Octavia Butler and a telepathic
invention of Philip K. Dick.

Still, enough excuses: \emph{I'd walked right past them.} Past the
bursts of carnival-candy slime mold, the salmon pearls of wolf's milk,
the tussled sheaves of chocolate tube slimes. My new awareness was
nothing less than transportive. I was a sudden convert, an acolyte of
leaf duff and rotting logs. Children were dispatched in tense
reconnaissance --- being of keen eyesight and closer to the ground, they
make fine scouts. Part of the joy, we learned, was returning to a spot
to track a discovery, such as a dog-vomit slime mold we encountered
wandering in the redwoods, a dead ringer for a brand-new kitchen sponge.
A \emph{crawling} sponge, for we returned a few days later to find that
it had journeyed on.

At times, we have followed a slime mold's movements as one might a
favorite team. Early in our adventures, my youngest son spotted a swath
of gold so bright I thought it might be an abandoned signpost; closer
inspection revealed a Badhamia plasmodium marching down a rotting log
(like most slime molds, Badhamia, by virtue of obscurity, has no common
name). Racing back the next day, we found that, having reached the south
pole of the log, it had reversed course and was moving northward at such
velocity that we could see it moving with the time-lapse function on my
iPhone. The following week it began its apotheosis. The gold was gone;
now bunches of grapelike sporangia glittered. A week later, they began
to rupture, spilling twilight-colored spores. Two weeks later, only
hollowed orbs remained, crystalline and breath-shivered. By then I had
discovered a group of people posting their discoveries on Instagram with
the hashtag \#slimemoldsunday: intergalactic gem collections from the
forests of
\href{https://www.instagram.com/sarah.lloyd.tasmania/?hl=en}{Tasmania,}
\href{https://www.instagram.com/sir.myxo.lot/?hl=en}{Southern
California} and
\href{https://www.instagram.com/nycslimemolds/?hl=en}{Prospect Park.}

\href{https://www.nytimes3xbfgragh.onion/news-event/coronavirus?action=click\&pgtype=Article\&state=default\&region=MAIN_CONTENT_3\&context=storylines_faq}{}

\hypertarget{the-coronavirus-outbreak-}{%
\subsubsection{The Coronavirus Outbreak
›}\label{the-coronavirus-outbreak-}}

\hypertarget{frequently-asked-questions}{%
\paragraph{Frequently Asked
Questions}\label{frequently-asked-questions}}

Updated September 4, 2020

\begin{itemize}
\item ~
  \hypertarget{what-are-the-symptoms-of-coronavirus}{%
  \paragraph{What are the symptoms of
  coronavirus?}\label{what-are-the-symptoms-of-coronavirus}}

  \begin{itemize}
  \tightlist
  \item
    In the beginning, the coronavirus
    \href{https://www.nytimes3xbfgragh.onion/article/coronavirus-facts-history.html?action=click\&pgtype=Article\&state=default\&region=MAIN_CONTENT_3\&context=storylines_faq\#link-6817bab5}{seemed
    like it was primarily a respiratory illness}~--- many patients had
    fever and chills, were weak and tired, and coughed a lot, though
    some people don't show many symptoms at all. Those who seemed
    sickest had pneumonia or acute respiratory distress syndrome and
    received supplemental oxygen. By now, doctors have identified many
    more symptoms and syndromes. In April,
    \href{https://www.nytimes3xbfgragh.onion/2020/04/27/health/coronavirus-symptoms-cdc.html?action=click\&pgtype=Article\&state=default\&region=MAIN_CONTENT_3\&context=storylines_faq}{the
    C.D.C. added to the list of early signs}~sore throat, fever, chills
    and muscle aches. Gastrointestinal upset, such as diarrhea and
    nausea, has also been observed. Another telltale sign of infection
    may be a sudden, profound diminution of one's
    \href{https://www.nytimes3xbfgragh.onion/2020/03/22/health/coronavirus-symptoms-smell-taste.html?action=click\&pgtype=Article\&state=default\&region=MAIN_CONTENT_3\&context=storylines_faq}{sense
    of smell and taste.}~Teenagers and young adults in some cases have
    developed painful red and purple lesions on their fingers and toes
    --- nicknamed ``Covid toe'' --- but few other serious symptoms.
  \end{itemize}
\item ~
  \hypertarget{why-is-it-safer-to-spend-time-together-outside}{%
  \paragraph{Why is it safer to spend time together
  outside?}\label{why-is-it-safer-to-spend-time-together-outside}}

  \begin{itemize}
  \tightlist
  \item
    \href{https://www.nytimes3xbfgragh.onion/2020/05/15/us/coronavirus-what-to-do-outside.html?action=click\&pgtype=Article\&state=default\&region=MAIN_CONTENT_3\&context=storylines_faq}{Outdoor
    gatherings}~lower risk because wind disperses viral droplets, and
    sunlight can kill some of the virus. Open spaces prevent the virus
    from building up in concentrated amounts and being inhaled, which
    can happen when infected people exhale in a confined space for long
    stretches of time, said Dr. Julian W. Tang, a virologist at the
    University of Leicester.
  \end{itemize}
\item ~
  \hypertarget{why-does-standing-six-feet-away-from-others-help}{%
  \paragraph{Why does standing six feet away from others
  help?}\label{why-does-standing-six-feet-away-from-others-help}}

  \begin{itemize}
  \tightlist
  \item
    The coronavirus spreads primarily through droplets from your mouth
    and nose, especially when you cough or sneeze. The C.D.C., one of
    the organizations using that measure,
    \href{https://www.nytimes3xbfgragh.onion/2020/04/14/health/coronavirus-six-feet.html?action=click\&pgtype=Article\&state=default\&region=MAIN_CONTENT_3\&context=storylines_faq}{bases
    its recommendation of six feet}~on the idea that most large droplets
    that people expel when they cough or sneeze will fall to the ground
    within six feet. But six feet has never been a magic number that
    guarantees complete protection. Sneezes, for instance, can launch
    droplets a lot farther than six feet,
    \href{https://jamanetwork.com/journals/jama/fullarticle/2763852}{according
    to a recent study}. It's a rule of thumb: You should be safest
    standing six feet apart outside, especially when it's windy. But
    keep a mask on at all times, even when you think you're far enough
    apart.
  \end{itemize}
\item ~
  \hypertarget{i-have-antibodies-am-i-now-immune}{%
  \paragraph{I have antibodies. Am I now
  immune?}\label{i-have-antibodies-am-i-now-immune}}

  \begin{itemize}
  \tightlist
  \item
    As of right
    now,\href{https://www.nytimes3xbfgragh.onion/2020/07/22/health/covid-antibodies-herd-immunity.html?action=click\&pgtype=Article\&state=default\&region=MAIN_CONTENT_3\&context=storylines_faq}{~that
    seems likely, for at least several months.}~There have been
    frightening accounts of people suffering what seems to be a second
    bout of Covid-19. But experts say these patients may have a
    drawn-out course of infection, with the virus taking a slow toll
    weeks to months after initial exposure.~People infected with the
    coronavirus typically
    \href{https://www.nature.com/articles/s41586-020-2456-9}{produce}~immune
    molecules called antibodies, which are
    \href{https://www.nytimes3xbfgragh.onion/2020/05/07/health/coronavirus-antibody-prevalence.html?action=click\&pgtype=Article\&state=default\&region=MAIN_CONTENT_3\&context=storylines_faq}{protective
    proteins made in response to an
    infection}\href{https://www.nytimes3xbfgragh.onion/2020/05/07/health/coronavirus-antibody-prevalence.html?action=click\&pgtype=Article\&state=default\&region=MAIN_CONTENT_3\&context=storylines_faq}{.
    These antibodies may}~last in the body
    \href{https://www.nature.com/articles/s41591-020-0965-6}{only two to
    three months}, which may seem worrisome, but that's~perfectly normal
    after an acute infection subsides, said Dr. Michael Mina, an
    immunologist at Harvard University. It may be possible to get the
    coronavirus again, but it's highly unlikely that it would be
    possible in a short window of time from initial infection or make
    people sicker the second time.
  \end{itemize}
\item ~
  \hypertarget{what-are-my-rights-if-i-am-worried-about-going-back-to-work}{%
  \paragraph{What are my rights if I am worried about going back to
  work?}\label{what-are-my-rights-if-i-am-worried-about-going-back-to-work}}

  \begin{itemize}
  \tightlist
  \item
    Employers have to provide
    \href{https://www.osha.gov/SLTC/covid-19/standards.html}{a safe
    workplace}~with policies that protect everyone equally.
    \href{https://www.nytimes3xbfgragh.onion/article/coronavirus-money-unemployment.html?action=click\&pgtype=Article\&state=default\&region=MAIN_CONTENT_3\&context=storylines_faq}{And
    if one of your co-workers tests positive for the coronavirus, the
    C.D.C.}~has said that
    \href{https://www.cdc.gov/coronavirus/2019-ncov/community/guidance-business-response.html}{employers
    should tell their employees}~-\/- without giving you the sick
    employee's name -\/- that they may have been exposed to the virus.
  \end{itemize}
\end{itemize}

This was early March. As with any love, the seeds of loss were also
planted. These are the very forests that have been burning with
increasing violence each year; when the woods go, these worlds go with
them. Then: Covid-19 and the progressive closing of the parks. And then:
3,000 miles away, in another woods,
\href{https://www.nytimes3xbfgragh.onion/2020/06/14/nyregion/central-park-amy-cooper-christian-racism.html}{Christian
Cooper walked into the Ramble to look for birds.}

Before the shelter-in-place, my favorite spot was a rotting log at the
intersection of a pair of trails used heavily by runners who barely
registered me in the trailside gully with my cellphone. The past weeks
have made increasingly clear that not everyone is granted such peaceful
indifference. As Melody Cooper wrote movingly about her brother in an
\href{https://www.nytimes3xbfgragh.onion/2020/05/31/opinion/chris-cooper-central-park.html}{Op-Ed
for The New York Times,} even sneaking through the trees ``to catch a
glimpse of a beautiful warbler'' carries a risk for a nature lover who
is black.

It has been heartening, during these painful days, to see the
environmental movement stand behind Black Lives Matter and other
movements for social reform, to further recognize that environmentalism
and racial justice can't be unlinked. My hope is that this new awakening
will be durable, matched by a greater equality in open spaces --- a
freedom to rustle through the leaf-litter and peer into the damp of
fallen logs.

Advertisement

\protect\hyperlink{after-bottom}{Continue reading the main story}

\hypertarget{site-index}{%
\subsection{Site Index}\label{site-index}}

\hypertarget{site-information-navigation}{%
\subsection{Site Information
Navigation}\label{site-information-navigation}}

\begin{itemize}
\tightlist
\item
  \href{https://help.nytimes3xbfgragh.onion/hc/en-us/articles/115014792127-Copyright-notice}{©~2020~The
  New York Times Company}
\end{itemize}

\begin{itemize}
\tightlist
\item
  \href{https://www.nytco.com/}{NYTCo}
\item
  \href{https://help.nytimes3xbfgragh.onion/hc/en-us/articles/115015385887-Contact-Us}{Contact
  Us}
\item
  \href{https://www.nytco.com/careers/}{Work with us}
\item
  \href{https://nytmediakit.com/}{Advertise}
\item
  \href{http://www.tbrandstudio.com/}{T Brand Studio}
\item
  \href{https://www.nytimes3xbfgragh.onion/privacy/cookie-policy\#how-do-i-manage-trackers}{Your
  Ad Choices}
\item
  \href{https://www.nytimes3xbfgragh.onion/privacy}{Privacy}
\item
  \href{https://help.nytimes3xbfgragh.onion/hc/en-us/articles/115014893428-Terms-of-service}{Terms
  of Service}
\item
  \href{https://help.nytimes3xbfgragh.onion/hc/en-us/articles/115014893968-Terms-of-sale}{Terms
  of Sale}
\item
  \href{https://spiderbites.nytimes3xbfgragh.onion}{Site Map}
\item
  \href{https://help.nytimes3xbfgragh.onion/hc/en-us}{Help}
\item
  \href{https://www.nytimes3xbfgragh.onion/subscription?campaignId=37WXW}{Subscriptions}
\end{itemize}
