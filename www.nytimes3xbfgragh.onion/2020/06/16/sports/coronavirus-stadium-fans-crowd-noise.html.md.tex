Sections

SEARCH

\protect\hyperlink{site-content}{Skip to
content}\protect\hyperlink{site-index}{Skip to site index}

\href{https://www.nytimes3xbfgragh.onion/section/sports}{Sports}

\href{https://myaccount.nytimes3xbfgragh.onion/auth/login?response_type=cookie\&client_id=vi}{}

\href{https://www.nytimes3xbfgragh.onion/section/todayspaper}{Today's
Paper}

\href{/section/sports}{Sports}\textbar{}We Hope Your Cheers for This
Article Are for Real

\url{https://nyti.ms/30YeE27}

\begin{itemize}
\item
\item
\item
\item
\item
\item
\end{itemize}

Advertisement

\protect\hyperlink{after-top}{Continue reading the main story}

Supported by

\protect\hyperlink{after-sponsor}{Continue reading the main story}

\hypertarget{we-hope-your-cheers-for-this-article-are-for-real}{%
\section{We Hope Your Cheers for This Article Are for
Real}\label{we-hope-your-cheers-for-this-article-are-for-real}}

Most sporting events coming back in the pandemic have not permitted
fans, leading broadcasters to use fake crowd noise, for better or worse.

\includegraphics{https://static01.graylady3jvrrxbe.onion/images/2020/06/16/sports/16virus-crowdnoise-1/merlin_173465208_f9fa4a59-65b5-4f7a-a91c-425942d15d84-articleLarge.jpg?quality=75\&auto=webp\&disable=upscale}

By \href{https://www.nytimes3xbfgragh.onion/by/andrew-keh}{Andrew Keh}

\begin{itemize}
\item
  June 16, 2020
\item
  \begin{itemize}
  \item
  \item
  \item
  \item
  \item
  \item
  \end{itemize}
\end{itemize}

Andy Phillips, a soccer fan from Kent, England, has a modest expectation
for the games he watches on television: that what he is seeing and
hearing is real and actually happening.

The coronavirus pandemic has made this complicated.

Watching a German soccer game at home on a recent weekend afternoon,
Phillips, 53, was ``aghast'' to find that the TV network had layered
artificial crowd noise over the live broadcast from the stadium, which
had been closed to spectators because of the pandemic and was therefore
mostly silent.

He listened, ``psychologically annoyed,'' as the fake crowd cheered for
goals, booed for rough fouls and hummed with anticipation when the ball
drifted close to the penalty area.

``It was horrendous, to be honest,'' he said. ``Not because I don't
enjoy the sound of crowd noise, but the fact it was fake.''

As professional sports have tiptoed back to the playing field, league
officials and television executives around the world seem to have come
to a consensus: that sporting events without the accompaniment of crowd
noise are simply too jarring, too unfamiliar and too boring for the
typical fan to endure.

And so prerecorded crowd audio tracks have quickly become the go-to
solution for live showings of such disparate sports as Hungarian soccer,
South Korean baseball and Australian rugby.

For every fan like Phillips, who finds the embrace of aural artifice
bizarre and existentially troubling --- ``Who needs people in the
ground, when you create your own atmosphere?'' he said --- there are
also those for whom the simulated noise provides feelings of comfort and
normalcy.

``Anything is better than hearing the echoes around a quiet stadium,''
said Hunter Fauci, 24, of Highlands, N.Y., a member of the American fan
club of the German team Borussia Mönchengladbach who appreciated the
artificial noise. ``Silence would make a lot of fans depressed.''

These sonic sleights of hand, then, can be polarizing. But they are
about to become even more prominent in the coming weeks as other major
leagues inch back to competition.

For instance, Joe Buck, the Fox Sports play-by-play announcer, said last
month on SiriusXM Radio that it was ``pretty much a done deal'' that the
N.F.L. would use artificial fan noise for its live game broadcasts this
year if games were played in empty stadiums.

When it returns this week, England's Premier League will offer viewers
simulated crowd noise
\href{https://www.skysports.com/football/news/11661/12004035/crowd-noise-available-for-sky-sports-premier-league-games-how-it-works}{with
help from the Electronic Arts's ``FIFA'' soccer video game series}.
(While audiences in the Premier League and Bundesliga's home countries
have the option to switch between audio feeds on parallel channels,
television viewers in the United States watching on NBC and Fox networks
will get the augmented audio as the default for these leagues.)

Spain's La Liga returned last week, also with virtual stadium sounds
borrowed from ``FIFA.'' Similarly, The Athletic reported earlier this
month that the N.B.A. had discussed the possibility of using audio from
the ``N.B.A. 2K'' video games to enliven its own broadcasts.

Reactions to having the quietude of real life smothered by manufactured
noise have ranged from dystopian anxiety to resignation to relief.

Twenty years ago, CBS drew criticism when the network used taped nature
sounds to brighten up a broadcast of the PGA Championships; avian
experts noticed some non-indigenous bird calls chirping out of their
speakers. But today's circumstances seem to have created a more
welcoming environment for experimentation.

\hypertarget{sports-and-the-virus}{%
\subsubsection{Sports and the Virus}\label{sports-and-the-virus}}

\paragraph{}

Updated Sept. 8, 2020

Here's what's happening as the world of sports slowly comes back to
life:

\begin{itemize}
\item
  \begin{itemize}
  \tightlist
  \item
    As the United States Open enters its second week without fans, an
    Italian restaurateur stands outside the gates and
    \href{https://www.nytimes3xbfgragh.onion/2020/09/06/sports/tennis/US-Open-Matteo-Berrettini-fan.html?action=click\&pgtype=Article\&state=default\&region=MAIN_CONTENT_2\&context=storylines_keepup}{bellows
    his support}~for his favorite player.
  \item
    The coronavirus pandemic has had an
    \href{https://www.nytimes3xbfgragh.onion/2020/09/03/sports/ncaafootball/high-school-football-coronavirus-pandemic.html?action=click\&pgtype=Article\&state=default\&region=MAIN_CONTENT_2\&context=storylines_keepup}{uneven
    impact on high school football}~across the United States.
  \item
    The
    \href{https://www.nytimes3xbfgragh.onion/2020/09/02/sports/ncaafootball/coronavirus-cal-athletics-season.html?action=click\&pgtype=Article\&state=default\&region=MAIN_CONTENT_2\&context=storylines_keepup}{most
    complicated puzzle in sports is the return of college
    athletics}~during a pandemic. The University of California, Berkeley
    is allowing The Times an inside look at their journey's ups and
    downs.
  \end{itemize}
\end{itemize}

``We're kind of in a try-anything mode,'' said Bob Costas, the longtime
sports announcer. ``You just don't want it to sound like the laugh track
on a bad `60s sitcom.''

But old-school canned laughter may be the most fitting reference point
for what is happening now.

Alessandro Reitano, vice president of sports production at Sky Germany,
said the goal of the Bundesliga's ``enhanced audio'' initiative was to
``forget a little bit that you're seeing an empty stadium'' --- an
effort that has also involved the increased use of up-close camera
angles --- and to elevate the atmosphere beyond the feeling of ``kids
playing in the park.''

Viewers, this way, could get immersed again in the narrative of a game.
Emotions could be stimulated.

\includegraphics{https://static01.graylady3jvrrxbe.onion/images/2020/06/16/sports/16virus-crowdnoise-3/merlin_172561929_bd5eacb1-834b-44d8-8e06-312d139e0c89-articleLarge.jpg?quality=75\&auto=webp\&disable=upscale}

Still, Bundesliga officials were hesitant about the project. Fans in
Germany take particular pride in the organic and democratic quality of
sports in the country, and in recent years anything that has appeared to
de-emphasize the importance of live audiences, especially in the service
of television, has drawn an intense backlash.

But because of the unprecedented circumstances, the league went ahead
crafting a proprietary system in which a soundboard with more than a
dozen carefully selected audio samples --- as specific as a nervous
crescendo of applause while a team chases an equalizing goal or lusty
jeers for a call overturned by video review --- sits at the disposal of
an operator watching from a studio in Munich.

``They have this imagined sense of what the spectacle should be and how
the consumer should experience it, and they manipulate the
representations of it to produce that for the consumer, and it's just
taken to the nth degree,'' David Andrews, a professor of sports culture
at the University of Maryland, said of these leagues and television
networks. ``Baudrillard would have gone mad with this.''

Jean Baudrillard, the French theoretician, postulated that simulated
experiences were replacing real life in postindustrial society.

He described a media-saturated culture moving toward the realm of what
he and other critics called hyper-reality, a state where the simulated
can be more prominent than the authentic and where images and copies can
be considered realer than real life.

(It may be worth remembering, as well, that Baudrillard once described
Disneyland as ``a perfect model of all the entangled orders of
simulation,'' a fantasy representation of an idealized image of American
life, as the N.B.A. and Major League Soccer finalize plans to resume
their seasons this summer at Disney World.)

``We can look at sports and see how close we are moving toward that
model,'' said Richard Giulianotti, a sports sociologist at Loughborough
University in England.

Once, long ago, watching a game on TV felt akin to eavesdropping on a
party happening at some faraway place. Now games are specifically
tailored as made-for-TV spectacles, and the screen --- in your living
room, on your phone --- is where the action is.

Examples of sports' long journey toward hyper-reality abound: electronic
screens that instruct fans to cheer; luxury boxes that recreate the
plush feeling of a living room inside a stadium; instant replay and
video-assisted referee systems; digital strike zones and glowing first
down markers; e-sports.

Image

TV broadcasters believe canned crowd noise helps distract from
cavernous, empty stadiums.Credit...Pool photo by Eric Dobias

Whether this is a good or bad thing is left to the observer to decide.

Two months ago, Ross Hawkins, 44, a software developer from Auckland,
New Zealand, sat down to watch WrestleMania 36, the professional
wrestling event, which took place this year without fans.

The absence of crowd noise, he said, ``killed sports'' for him.

Several weeks later, Hawkins tuned in to watch Australia's National
Rugby League, which restarted play late last month with fake crowd
sounds. The gentle hum of the fake crowd washed over him, and his mind
felt suddenly at ease. He forgot the world had been turned upside down
by a virus. He could enjoy sports again.

``As a reasonably intelligent person, I knew it was fake, and I didn't
expect it to make such a difference, but it did,'' Hawkins said. ``It
feels like it's the brain clamoring for some normalcy in 2020.''

Advertisement

\protect\hyperlink{after-bottom}{Continue reading the main story}

\hypertarget{site-index}{%
\subsection{Site Index}\label{site-index}}

\hypertarget{site-information-navigation}{%
\subsection{Site Information
Navigation}\label{site-information-navigation}}

\begin{itemize}
\tightlist
\item
  \href{https://help.nytimes3xbfgragh.onion/hc/en-us/articles/115014792127-Copyright-notice}{©~2020~The
  New York Times Company}
\end{itemize}

\begin{itemize}
\tightlist
\item
  \href{https://www.nytco.com/}{NYTCo}
\item
  \href{https://help.nytimes3xbfgragh.onion/hc/en-us/articles/115015385887-Contact-Us}{Contact
  Us}
\item
  \href{https://www.nytco.com/careers/}{Work with us}
\item
  \href{https://nytmediakit.com/}{Advertise}
\item
  \href{http://www.tbrandstudio.com/}{T Brand Studio}
\item
  \href{https://www.nytimes3xbfgragh.onion/privacy/cookie-policy\#how-do-i-manage-trackers}{Your
  Ad Choices}
\item
  \href{https://www.nytimes3xbfgragh.onion/privacy}{Privacy}
\item
  \href{https://help.nytimes3xbfgragh.onion/hc/en-us/articles/115014893428-Terms-of-service}{Terms
  of Service}
\item
  \href{https://help.nytimes3xbfgragh.onion/hc/en-us/articles/115014893968-Terms-of-sale}{Terms
  of Sale}
\item
  \href{https://spiderbites.nytimes3xbfgragh.onion}{Site Map}
\item
  \href{https://help.nytimes3xbfgragh.onion/hc/en-us}{Help}
\item
  \href{https://www.nytimes3xbfgragh.onion/subscription?campaignId=37WXW}{Subscriptions}
\end{itemize}
