Sections

SEARCH

\protect\hyperlink{site-content}{Skip to
content}\protect\hyperlink{site-index}{Skip to site index}

\href{https://www.nytimes3xbfgragh.onion/section/us}{U.S.}

\href{https://myaccount.nytimes3xbfgragh.onion/auth/login?response_type=cookie\&client_id=vi}{}

\href{https://www.nytimes3xbfgragh.onion/section/todayspaper}{Today's
Paper}

\href{/section/us}{U.S.}\textbar{}Coronavirus Cases Rise Sharply in
Prisons Even as They Plateau Nationwide

\url{https://nyti.ms/3hxhdOs}

\begin{itemize}
\item
\item
\item
\item
\item
\item
\end{itemize}

\hypertarget{the-coronavirus-outbreak}{%
\subsubsection{\texorpdfstring{\href{https://www.nytimes3xbfgragh.onion/news-event/coronavirus?name=styln-coronavirus-national\&region=TOP_BANNER\&block=storyline_menu_recirc\&action=click\&pgtype=Article\&impression_id=088a7ed0-f4ba-11ea-8fe6-a15add8e2357\&variant=undefined}{The
Coronavirus
Outbreak}}{The Coronavirus Outbreak}}\label{the-coronavirus-outbreak}}

\begin{itemize}
\tightlist
\item
  live\href{https://www.nytimes3xbfgragh.onion/2020/09/11/world/covid-19-coronavirus.html?name=styln-coronavirus-national\&region=TOP_BANNER\&block=storyline_menu_recirc\&action=click\&pgtype=Article\&impression_id=088aa5e0-f4ba-11ea-8fe6-a15add8e2357\&variant=undefined}{Latest
  Updates}
\item
  \href{https://www.nytimes3xbfgragh.onion/interactive/2020/us/coronavirus-us-cases.html?name=styln-coronavirus-national\&region=TOP_BANNER\&block=storyline_menu_recirc\&action=click\&pgtype=Article\&impression_id=088aa5e1-f4ba-11ea-8fe6-a15add8e2357\&variant=undefined}{Maps
  and Cases}
\item
  \href{https://www.nytimes3xbfgragh.onion/interactive/2020/science/coronavirus-vaccine-tracker.html?name=styln-coronavirus-national\&region=TOP_BANNER\&block=storyline_menu_recirc\&action=click\&pgtype=Article\&impression_id=088aa5e2-f4ba-11ea-8fe6-a15add8e2357\&variant=undefined}{Vaccine
  Tracker}
\item
  \href{https://www.nytimes3xbfgragh.onion/2020/09/10/us/politics/fda-coronavirus-vaccine.html?name=styln-coronavirus-national\&region=TOP_BANNER\&block=storyline_menu_recirc\&action=click\&pgtype=Article\&impression_id=088aa5e3-f4ba-11ea-8fe6-a15add8e2357\&variant=undefined}{F.D.A.
  Regulators' Self-Defense}
\item
  \href{https://www.nytimes3xbfgragh.onion/2020/09/09/upshot/coronavirus-surprise-test-fees.html?name=styln-coronavirus-national\&region=TOP_BANNER\&block=storyline_menu_recirc\&action=click\&pgtype=Article\&impression_id=088accf0-f4ba-11ea-8fe6-a15add8e2357\&variant=undefined}{Surprise
  Test Fees}
\end{itemize}

Advertisement

\protect\hyperlink{after-top}{Continue reading the main story}

Supported by

\protect\hyperlink{after-sponsor}{Continue reading the main story}

\hypertarget{coronavirus-cases-rise-sharply-in-prisons-even-as-they-plateau-nationwide}{%
\section{Coronavirus Cases Rise Sharply in Prisons Even as They Plateau
Nationwide}\label{coronavirus-cases-rise-sharply-in-prisons-even-as-they-plateau-nationwide}}

Prison officials have been reluctant to do widespread virus testing even
as infection rates are escalating.

\includegraphics{https://static01.graylady3jvrrxbe.onion/images/2020/06/10/us/00virus-prisons01/merlin_172294968_fd0b6407-d2b4-4be4-866f-615546015351-articleLarge.jpg?quality=75\&auto=webp\&disable=upscale}

By \href{https://www.nytimes3xbfgragh.onion/by/timothy-williams}{Timothy
Williams}, Libby Seline and Rebecca Griesbach

\begin{itemize}
\item
  Published June 16, 2020Updated June 30, 2020
\item
  \begin{itemize}
  \item
  \item
  \item
  \item
  \item
  \item
  \end{itemize}
\end{itemize}

Cases of the
\href{https://www.nytimes3xbfgragh.onion/2020/07/30/nyregion/New-jersey-inmate-release-Covid.html}{coronavirus
in prisons} and jails across the United States have soared in recent
weeks, even as the overall daily infection rate in the nation has
remained relatively flat.

The number of prison inmates known to be infected has doubled during the
past month to more than 68,000. Prison deaths tied to the coronavirus
have also risen, by 73 percent since mid-May. By now, the
\href{https://www.nytimes3xbfgragh.onion/interactive/2020/us/coronavirus-us-cases.html\#clusters}{five
largest known clusters of the virus} in the United States are not at
nursing homes or meatpacking plants, but inside correction institutions,
according to data The New York Times has been collecting about confirmed
coronavirus cases since the pandemic reached American shores.

And the risk of more cases appears imminent: The swift growth in virus
cases behind bars comes as demonstrators arrested as part of large
\href{https://www.nytimes3xbfgragh.onion/news-event/george-floyd-protests-minneapolis-new-york-los-angeles}{police
brutality protests} across the nation have often been placed in
\href{https://www.nytimes3xbfgragh.onion/2020/06/04/nyregion/nyc-protests-jail.html}{crowded
holding cells} in local jails.

A muddled, uneven response by corrections officials to testing and care
for inmates and workers is complicating the spread of the coronavirus.
In interviews, prison and jail officials acknowledged that their
approach has largely been based on trial and error, and that an
effective, consistent response for U.S. correctional facilities remains
elusive.

``If there was clearly a right strategy, we all would have done it,''
said Dr. Owen Murray, a University of Texas Medical Branch physician who
oversees correctional health care at dozens of Texas prisons. ``There is
no clear-cut right strategy here. There are a lot of different choices
that one could make that are going to be in-the-moment decisions.''

The inconsistent response to the spread of the coronavirus in
correctional facilities is in contrast with efforts to halt its spread
in other known incubators of the virus: Much of the
\href{https://www.federalregister.gov/documents/2020/04/15/2020-07930/no-sail-order-and-suspension-of-further-embarkation-notice-of-modification-and-extension-and-other}{cruise
ship industry} has been closed down. Staff members and residents of
\href{https://apnews.com/1a169a537c6fb7f9ab824c49a6757b0c}{nursing
homes} in several states now face compulsory testing. Many
\href{https://www.foodprocessing.com/articles/2020/how-coronavirus-slammed-meat-and-poultry/}{meat
processing plants} have been shuttered for extensive cleaning.

As the toll in prisons has increased, so has fear among inmates who say
the authorities have done too little to protect them. There have been
riots and hunger strikes in correctional facilities from
\href{https://www.heraldnet.com/news/major-disturbance-at-monroe-prison-follows-covid-19-outbreak/}{Washington
State} to New York. And even the known case numbers are most likely a
significant undercount because testing has been extremely limited inside
prisons and because some places that test do not release the results to
the public.

``It's like a sword hanging over my head,'' said Fred Roehler, 77, an
inmate at a California prison who has chronic inflammatory lung disease
and other respiratory ailments. ``Any officer can bring it in.''

Public officials have long warned that the nation's correctional
facilities would most likely become vectors in the pandemic because they
are often overcrowded, unsanitary places where social distancing is
impractical, bathrooms and day rooms are shared by hundreds of inmates,
and access to cleaning supplies is tightly controlled. Many inmates are
60 or older, and many suffer from respiratory illnesses or heart
conditions.

\hypertarget{latest-updates-the-coronavirus-outbreak}{%
\section{\texorpdfstring{\href{https://www.nytimes3xbfgragh.onion/2020/09/11/world/covid-19-coronavirus.html?action=click\&pgtype=Article\&state=default\&region=MAIN_CONTENT_1\&context=storylines_live_updates}{Latest
Updates: The Coronavirus
Outbreak}}{Latest Updates: The Coronavirus Outbreak}}\label{latest-updates-the-coronavirus-outbreak}}

Updated 2020-09-12T05:29:13.829Z

\begin{itemize}
\tightlist
\item
  \href{https://www.nytimes3xbfgragh.onion/2020/09/11/world/covid-19-coronavirus.html?action=click\&pgtype=Article\&state=default\&region=MAIN_CONTENT_1\&context=storylines_live_updates\#link-dfb8a16}{Fauci
  cautions the virus could disrupt life in the U.S. until `maybe even
  towards the end of 2021.'}
\item
  \href{https://www.nytimes3xbfgragh.onion/2020/09/11/world/covid-19-coronavirus.html?action=click\&pgtype=Article\&state=default\&region=MAIN_CONTENT_1\&context=storylines_live_updates\#link-7104d154}{From
  Asia to Africa, China promotes its vaccine candidates to win friends.}
\item
  \href{https://www.nytimes3xbfgragh.onion/2020/09/11/world/covid-19-coronavirus.html?action=click\&pgtype=Article\&state=default\&region=MAIN_CONTENT_1\&context=storylines_live_updates\#link-393ad215}{The
  other way the virus will kill: hunger.}
\end{itemize}

\href{https://www.nytimes3xbfgragh.onion/2020/09/11/world/covid-19-coronavirus.html?action=click\&pgtype=Article\&state=default\&region=MAIN_CONTENT_1\&context=storylines_live_updates}{See
more updates}

More live coverage:
\href{https://www.nytimes3xbfgragh.onion/live/2020/09/11/business/stock-market-today-coronavirus?action=click\&pgtype=Article\&state=default\&region=MAIN_CONTENT_1\&context=storylines_live_updates}{Markets}

In response, local jails have discharged thousands of inmates since
February, many of whom had been awaiting trials to have charges heard or
serving time for nonviolent crimes. State prison systems, where people
convicted of more serious crimes are housed, have been more reluctant to
release inmates.

Testing for the virus within the nation's penal institutions varies
widely, and has become a matter of significant debate.

Republican-led states like Texas, Tennessee and Arkansas --- which
generally spend less on prisoners than the
\href{https://www.vera.org/publications/price-of-prisons-2015-state-spending-trends/price-of-prisons-2015-state-spending-trends/price-of-prisons-2015-state-spending-trends-prison-spending}{national
average} --- have found themselves at the forefront of testing inmates.

In Texas, the number of prisoners and staff members known to be infected
has more than quadrupled to 7,900 during the past three weeks after the
state began to test every inmate.

Yet states that typically spend far more on prisons have carried out
significantly less testing.

California, which spends \href{https://www.cdcr.ca.gov/budget/}{\$12
billion annually} on its prison system, has tested fewer than 7 percent
of inmates in several of its largest, most crowded facilities, according
to the state's data. Other Democratic-led states that also spend heavily
on prisons, including New York, Oregon and Colorado, have also conducted
limited testing despite large outbreaks in their facilities.

New York has tested about 3 percent of its 40,000 prison inmates; more
than 40 percent of those tested were infected.

Critics say that the dearth of testing in some facilities has meant that
prison and public health officials have only vague notions about the
spread of the virus, which has allowed some elected officials to suggest
that it is not present at all.

``We have really no true idea of how bad the problem is because most
places are not yet testing the way they should,'' said Dr. Homer
Venters, who served as chief medical officer for the New York City jail
system and now works for a group called Community Oriented Correctional
Health Services, which strives to improve health care services in local
jails. ``I think a lot of times some of the operational challenges of
either not having adequate quarantine policies or adequate medical
isolation policies are so vexing that places simply decide that they can
just throw up their hands.''

Most state prison systems have conducted few tests. Systems in Illinois,
Mississippi and Alabama have tested fewer than 2.5 percent of inmates.
And in Louisiana, officials had tested several dozen of its 31,000
inmates in March when the warden and medical director at one of the
state's largest prisons
\href{https://www.theadvocate.com/baton_rouge/news/coronavirus/article_2c8b1d96-7074-11ea-8eb7-57170ff59a6d.html}{died
of the coronavirus.} The state has since announced plans to test every
inmate.

Prison officials in states where only a limited number of inmates have
been tested say they are following federal guidelines. The Centers for
Disease Control and Prevention
\href{https://www.cdc.gov/coronavirus/2019-nCoV/hcp/clinical-criteria.html}{recommend
that only prisoners with symptoms be tested}.

Prisons that have conducted mass testing have found that about one in
seven tests of inmates have come back positive, the Times database
shows. The vast majority of inmates who have tested positive have been
asymptomatic.

Public health officials say that indicates the virus has been present in
prison populations for far longer than had previously been understood.

``If you don't do testing, you're flying blind,'' said Carlos
Franco-Paredes, an infectious-disease specialist at the University of
Colorado School of Medicine.

But in California, there continues to be reluctance to test each of the
state's 114,000 inmates, despite growing criticism to take a more
aggressive approach. One in six inmates in the state's prisons have been
tested, and the state has released some inmates who were later found to
have the virus, raising fears that prison systems could seed new
infections outside penal institutions.

``Nothing significant had been done to protect those most vulnerable to
the virus,'' said Marie Waldron, the Republican minority leader of the
California State Assembly.

\includegraphics{https://static01.graylady3jvrrxbe.onion/images/2020/06/10/us/00virus-prisons02/00virus-prisons02-articleLarge.jpg?quality=75\&auto=webp\&disable=upscale}

But J. Clark Kelso, who oversees prison health care in California, said
that mass testing would provide only a snapshot of the virus's spread.

``Testing's not a complete solution,'' Mr. Kelso said. ``It gives you
better information, but you don't want to get a false sense of
security.''

California's health department has
\href{https://www.cdph.ca.gov/Programs/CID/DCDC/Pages/COVID-19/Expanding-Access-to-Testing-Updated-Guidance-on-Prioritization-for-COVID-19-Testing.aspx}{recommended}
that a facility's prison inmates and staff members be given priority for
testing once an infection has been identified there.

But the state prison system has conducted mass testing at only a handful
of institutions where infections have been found, according to state
data. In one of those facilities, the California Institution for Men in
Chino, nearly 875 people have tested positive and 13 inmates have died.

Instead, California has employed surveillance testing, which involves
testing a limited number of inmates at each state prison regardless of
the known infection rate.

That method, Mr. Kelso said, had led officials to conclude that a vast
majority of its prisons are free of the virus.

``We're not 100 percent confident because we're not testing everyone,''
he said. ``As we learn every single day from what we're doing, we may
suddenly decide, `No, we actually have to test all of them.' We're not
at that point yet.''

In interviews, California prison inmates say prison staff have sometimes
refused to test them, even after they complained about symptoms similar
to the coronavirus. Several prisoners said they had been too weak to
move for weeks at a time, but were never permitted to see a nurse and
had never been tested.

``I had chest pains. I couldn't breathe,'' said Althea Housley, 43, an
inmate at Folsom State Prison, where no inmates have tested positive,
according to state data. ``They told us it was the flu going around, but
I ain't never had a flu like that.''

Mr. Kelso did not dispute the prisoners' accounts.

In Texas, mass testing has found that nearly 8,000 inmates and guards
have been infected. Sixty-two people have died, including some who had
not exhibited symptoms.

Dr. Murray, the physician who oversees much of Texas' prison health care
system, said the disparate approaches taken by prison authorities might
actually be beneficial as officials compare notes.

``I'm glad we've got 50 states and everyone is trying to do something a
little different --- whether that's by intent or not,'' he said,
``because it's really the only basis that we're going to have for
comparison later on.''

But Baleegh Brown, 31, an inmate at a California prison, said he was
displeased about being part of what he considered a science experiment.
His prison has had more than 170 infections.

He said that he and his cellmate were confined to a 6-by-9-foot space
for about 22 hours each day as the prison tries to prevent the virus
from spreading further. Mr. Brown said he had a weakened immune system
after a case of non-Hodgkin's lymphoma, making him particularly
vulnerable to illness.

``We need more testing here so everyone knows for sure,'' he said. ``And
for me, my body has been compromised, so I don't know how it is going to
react. That makes all you don't know even scarier.''

Brendon Derr, Danya Issawi and Maura Turcotte contributed reporting.

Advertisement

\protect\hyperlink{after-bottom}{Continue reading the main story}

\hypertarget{site-index}{%
\subsection{Site Index}\label{site-index}}

\hypertarget{site-information-navigation}{%
\subsection{Site Information
Navigation}\label{site-information-navigation}}

\begin{itemize}
\tightlist
\item
  \href{https://help.nytimes3xbfgragh.onion/hc/en-us/articles/115014792127-Copyright-notice}{©~2020~The
  New York Times Company}
\end{itemize}

\begin{itemize}
\tightlist
\item
  \href{https://www.nytco.com/}{NYTCo}
\item
  \href{https://help.nytimes3xbfgragh.onion/hc/en-us/articles/115015385887-Contact-Us}{Contact
  Us}
\item
  \href{https://www.nytco.com/careers/}{Work with us}
\item
  \href{https://nytmediakit.com/}{Advertise}
\item
  \href{http://www.tbrandstudio.com/}{T Brand Studio}
\item
  \href{https://www.nytimes3xbfgragh.onion/privacy/cookie-policy\#how-do-i-manage-trackers}{Your
  Ad Choices}
\item
  \href{https://www.nytimes3xbfgragh.onion/privacy}{Privacy}
\item
  \href{https://help.nytimes3xbfgragh.onion/hc/en-us/articles/115014893428-Terms-of-service}{Terms
  of Service}
\item
  \href{https://help.nytimes3xbfgragh.onion/hc/en-us/articles/115014893968-Terms-of-sale}{Terms
  of Sale}
\item
  \href{https://spiderbites.nytimes3xbfgragh.onion}{Site Map}
\item
  \href{https://help.nytimes3xbfgragh.onion/hc/en-us}{Help}
\item
  \href{https://www.nytimes3xbfgragh.onion/subscription?campaignId=37WXW}{Subscriptions}
\end{itemize}
