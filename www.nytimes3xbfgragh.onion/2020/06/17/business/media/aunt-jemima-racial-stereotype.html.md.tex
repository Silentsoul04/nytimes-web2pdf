Sections

SEARCH

\protect\hyperlink{site-content}{Skip to
content}\protect\hyperlink{site-index}{Skip to site index}

\href{https://www.nytimes3xbfgragh.onion/section/business/media}{Media}

\href{https://myaccount.nytimes3xbfgragh.onion/auth/login?response_type=cookie\&client_id=vi}{}

\href{https://www.nytimes3xbfgragh.onion/section/todayspaper}{Today's
Paper}

\href{/section/business/media}{Media}\textbar{}Aunt Jemima Brand to
Change Name and Image Over `Racial Stereotype'

\url{https://nyti.ms/3ftzYAA}

\begin{itemize}
\item
\item
\item
\item
\item
\item
\end{itemize}

\hypertarget{race-and-america}{%
\subsubsection{\texorpdfstring{\href{https://www.nytimes3xbfgragh.onion/news-event/george-floyd-protests-minneapolis-new-york-los-angeles?name=styln-george-floyd\&region=TOP_BANNER\&block=storyline_menu_recirc\&action=click\&pgtype=Article\&impression_id=bcc8d2e0-f1d4-11ea-9a4f-2bebfff0e82c\&variant=undefined}{Race
and America}}{Race and America}}\label{race-and-america}}

\begin{itemize}
\tightlist
\item
  \href{https://www.nytimes3xbfgragh.onion/2020/09/04/nyregion/rochester-police-daniel-prude.html?name=styln-george-floyd\&region=TOP_BANNER\&block=storyline_menu_recirc\&action=click\&pgtype=Article\&impression_id=bcc8f9f0-f1d4-11ea-9a4f-2bebfff0e82c\&variant=undefined}{How
  Police Handled Death of Daniel Prude}
\item
  \href{https://www.nytimes3xbfgragh.onion/2020/09/01/us/politics/trump-fact-check-protests.html?name=styln-george-floyd\&region=TOP_BANNER\&block=storyline_menu_recirc\&action=click\&pgtype=Article\&impression_id=bcc8f9f1-f1d4-11ea-9a4f-2bebfff0e82c\&variant=undefined}{Trump
  Fact Check}
\item
  \href{https://www.nytimes3xbfgragh.onion/2020/08/30/us/portland-shooting-explained.html?name=styln-george-floyd\&region=TOP_BANNER\&block=storyline_menu_recirc\&action=click\&pgtype=Article\&impression_id=bcc8f9f2-f1d4-11ea-9a4f-2bebfff0e82c\&variant=undefined}{Portland
  Shooting}
\item
  \href{https://www.nytimes3xbfgragh.onion/2020/08/30/us/breonna-taylor-police-killing.html?name=styln-george-floyd\&region=TOP_BANNER\&block=storyline_menu_recirc\&action=click\&pgtype=Article\&impression_id=bcc8f9f3-f1d4-11ea-9a4f-2bebfff0e82c\&variant=undefined}{Breonna
  Taylor's Life and Death}
\end{itemize}

Advertisement

\protect\hyperlink{after-top}{Continue reading the main story}

Supported by

\protect\hyperlink{after-sponsor}{Continue reading the main story}

\hypertarget{aunt-jemima-brand-to-change-name-and-image-over-racial-stereotype}{%
\section{Aunt Jemima Brand to Change Name and Image Over `Racial
Stereotype'}\label{aunt-jemima-brand-to-change-name-and-image-over-racial-stereotype}}

Quaker Oats, the owner of the 131-year-old brand, said it would retire
the name as it worked ``to make progress toward racial equality.''

\includegraphics{https://static01.graylady3jvrrxbe.onion/images/2020/06/17/business/17unrest-jemima-01/17unrest-jemima-01-articleLarge.jpg?quality=75\&auto=webp\&disable=upscale}

\href{https://www.nytimes3xbfgragh.onion/by/tiffany-hsu}{\includegraphics{https://static01.graylady3jvrrxbe.onion/images/2018/12/06/multimedia/author-tiffany-hsu/author-tiffany-hsu-thumbLarge.png}}

By \href{https://www.nytimes3xbfgragh.onion/by/tiffany-hsu}{Tiffany Hsu}

\begin{itemize}
\item
  Published June 17, 2020Updated July 10, 2020
\item
  \begin{itemize}
  \item
  \item
  \item
  \item
  \item
  \item
  \end{itemize}
\end{itemize}

For decades, Quaker Oats knew that one of its major brands, Aunt Jemima,
was built on racist imagery. The company inched toward fixing the
problem over the years, replacing the kerchief on the Aunt Jemima
character's head with a plaid headband in 1968, and adding pearl
earrings and a lace collar in 1989. But it was not until Wednesday that
Quaker Oats announced it would drop the Aunt Jemima name and change the
packaging.

The decision to remake the pancake-mix and syrup brand, which was
founded in 1889, came as widespread protests against
\href{https://www.nytimes3xbfgragh.onion/2020/07/10/sports/football/washington-redskins-name-change-mascots.html}{racism}
have reverberated throughout the country, leading to changes in the
corporate world and
\href{https://www.nytimes3xbfgragh.onion/2020/06/11/us/Jefferson-Davis-Statue-Richmond.html}{the
toppling} of statues depicting Confederate leaders.

Quaker Oats, which has been owned by PepsiCo since 2001, announced its
decision on Aunt Jemima days after a TikTok video describing the brand's
history was shared widely on social media. In retiring the name and
character, the company acknowledged that Aunt Jemima's origins were
``based on a racial stereotype.''

In a statement, the company said it was working ``to make progress
toward
\href{https://www.nytimes3xbfgragh.onion/2020/07/10/sports/football/washington-redskins-name-change-mascots.html}{racial}
equality through several initiatives.'' The packaging changes will
appear toward the end of the year, with the name change to follow.

The founders of the brand hired a former slave to portray Aunt Jemima at
the 1893 World's Fair in Chicago. In the 1930s, after Quaker Oats bought
the brand, the character was played in a radio series by a white actress
who had performed in blackface on Broadway. A 1954 magazine ad showed
Aunt Jemima superimposed over an image of a plantation and a riverboat.

Quaker Oats considered doing away with the logo in recent years, said
Dominique Wilburn, who worked as an executive assistant at PepsiCo for
several years. Ms. Wilburn said she joined an effort to come up with a
rebranding campaign for Aunt Jemima in 2016. In her group of six people,
she was the only person of color, she said.

The group tossed around ideas, all the while ``very aware of the broader
implications, and what would happen if we got this wrong,'' Ms. Wilburn
said. The plan was to introduce changes during a tranquil period, when
PepsiCo was not embroiled in any controversy.

Suggestions ranged from changing the character's name to ``Aunt J'' to
making Aunt Jemima's straightened hair more natural and building out her
back story, Ms. Wilburn said. One proposal involved asking artists to
remake the character's image; another called for sending Quaker Oats
employees to a Southern plantation to help them understand the legacy of
slavery, she said.

A team member suggested that Indra Nooyi, who was then PepsiCo's chief
executive, release a contrite letter on the brand's troubled history;
Ms. Wilburn said she criticized the idea that one of the few women of
color leading a major corporation should have to apologize for her
predecessors' mistakes.

At the end of the process, Ms. Wilburn's group agreed that the Aunt
Jemima name should be changed and the image removed, Ms. Wilburn said.
But gaining approval from top executives was difficult, partly because
PepsiCo found itself in a controversy after
\href{https://www.nytimes3xbfgragh.onion/2017/04/05/business/kendall-jenner-pepsi-ad.html}{running
a commercial} that showed Kendall Jenner, a white model, delivering a
can of Pepsi to a law enforcement officer at a Black Lives Matter
protest, she said.

PepsiCo said in a statement Wednesday that there were ``several
workstreams'' reviewing the brand in 2016 and that ``due to personnel
changes and shifting priorities, the workstream was eventually put on
hold.''

Since then, Quaker Oats has not given Aunt Jemima significant promotion.
Last year, Quaker Oats spent \$245,000 marketing the brand, compared
with \$6.2 million it spent on Life Cereal, excluding social media,
according to the research firm Kantar.

Ms. Wilburn said the company tried to avoid heavily promoting Aunt
Jemima. ``They were constantly being told, `Let's not over-promote it or
do a lot of partnerships' --- nobody wanted to call attention to it,''
she said. ``Aunt Jemima was a category leader, and nobody wanted to mess
with that stream of revenue.''

Wednesday's announcement stemmed from several weeks of meetings between
top PepsiCo leaders, employees and community leaders, the company said.

The decision to change the brand,
\href{https://www.nbcnews.com/news/us-news/aunt-jemima-brand-will-change-name-remove-image-quaker-says-n1231260?cid=sm_npd_ms_tw_ma}{reported
earlier by NBC News}, came during the widespread protests against racism
and police brutality prompted by the killing last month in Minneapolis
of George Floyd, a black man who died after a white police officer
pinned him to the ground.

The protests have led to statements of support from companies like Nike
and Twitter, which have declared Juneteenth on Friday an employee
holiday. The sportswear giant
\href{https://www.nytimes3xbfgragh.onion/2020/06/10/business/adidas-black-employees-discrimination.html}{Adidas}
pledged that 30 percent of new hires would be black or Latino and said
it would fund 50 university scholarships a year for black students over
the next five years. The makeup brand Sephora has pledged that
\href{https://www.nytimes3xbfgragh.onion/2020/06/10/business/sephora-black-owned-brands.html}{15
percent of the shelf space} in its stores will feature products made by
black-owned businesses.

In other cases, the reckoning has prompted the ousting of executives,
including the C.E.O. of CrossFit, who was dismissive of the uproar over
Mr. Floyd's death. High-level editors at
\href{https://www.nytimes3xbfgragh.onion/2020/06/07/business/media/james-bennet-resigns-nytimes-op-ed.html}{The
New York Times} and
\href{https://www.nytimes3xbfgragh.onion/2020/06/06/business/media/editor-philadephia-inquirer-resigns.html}{The
Philadelphia Inquirer} resigned after staff members criticized content
related to the protests in those papers. The NBC late-night host Jimmy
Fallon has apologized for
\href{https://www.nytimes3xbfgragh.onion/2020/05/26/us/jimmy-fallon-chris-rock-blackface.html}{performing
in blackface} on ``Saturday Night Live'' in 2000.

Last week, the streaming service
\href{https://www.nytimes3xbfgragh.onion/2020/06/10/business/media/gone-with-the-wind-hbo-max.html}{HBO
Max temporarily removed} the 1939 film ``Gone With the Wind'' from its
catalog because of its glorification of the antebellum South, a
depiction that included a subservient black character named Mammy.

The Aunt Jemima brand has its roots in a 19th-century minstrel song,
``Old Aunt Jemima.'' The character is one of ``many racialized
caricatures'' that were ``the creation of the white imagination'' during
the rise of the marketing industry, said Gregory D. Smithers, an
American history professor at Virginia Commonwealth University.

``Marketing companies used racism to sell everything from soap,
children's board games and food,'' said Mr. Smithers, who wrote a book
about the use of racist imagery in popular media.

Riché Richardson, an associate professor of African-American literature
at Cornell University, called for an end to the Aunt Jemima character in
\href{https://www.nytimes3xbfgragh.onion/roomfordebate/2015/06/24/besides-the-confederate-flag-what-other-symbols-should-go/can-we-please-finally-get-rid-of-aunt-jemima}{a
2015 opinion} essay in The Times. In an interview, she said, ``It is a
symbol that is rooted in the `Mammy' stereotype, that is premised on
notions of black otherness and inferiority, that harkens back to a time
when black people were thought of and idealized mainly in relation to
servant positions.''

On Monday, the singer Kirby described the history of the brand
\href{https://www.tiktok.com/@singkirbysing/video/6838642500052274438}{in
a TikTok video} that has been viewed more than 1.8 million times. The
video, ``How to Make a Non Racist Breakfast,'' ends with her pouring a
box of the pancake mix into a sink. The Reddit co-founder Alexis Ohanian
amplified Kirby's message on Twitter, where he has more than 334,000
followers. ``How is Aunt Jemima not canceled??''
\href{https://twitter.com/alexisohanian/status/1273071420017385473?s=20}{he
wrote} on Tuesday, linking to the TikTok video.

Kristin Kroepfl, the Quaker Oats chief marketing officer, said in a
statement on Wednesday, ``While work has been done over the years to
update the brand in a manner intended to be appropriate and respectful,
we realize those changes are not enough.''

Nancy Green, who played Aunt Jemima at the 1893 World's Fair in Chicago,
was born into slavery in Kentucky in 1834. In magazine ads throughout
much of the 20th century, some by the artist N.C. Wyeth, the character
was shown serving white families. From 1955 to 1970, Disneyland had an
Aunt Jemima restaurant. It featured an actress costumed in a plaid
dress, apron and kerchief who served food, sang and posed for photos
with patrons, according to the
\href{https://www.ferris.edu/HTMLS/news/jimcrow/question/2019/april_may.htm\#}{Jim
Crow Museum of Racist Memorabilia} in Michigan.

Black artists, including
\href{https://www.tate.org.uk/whats-on/tate-modern/exhibition/ey-exhibition-world-goes-pop/artist-interview/joe-overstreet}{Joe
Overstreet} and
\href{https://www.nytimes3xbfgragh.onion/video/opinion/100000006923399/betye-saar-taking-care-of-business.html}{Betye
Saar}, have challenged the character for decades. Mr. Overstreet painted
Aunt Jemima wielding a machine gun in 1964 and created an expanded
version of the work, called ``New Jemima,'' in 1970. Ms. Saar's 1972
mixed-media sculpture, ``The Liberation of Aunt Jemima,'' presented a
``mammy'' figurine armed with a rifle and a hand grenade against a
backdrop of repeated images of Aunt Jemima's face.

In 1980, in a commentary for National Public Radio, the black writer and
culinary historian Vertamae Smart-Grosvenor called on Quaker Oats to
retire the character.

Other food brands, including Cream of Wheat, Land O'Lakes and Uncle
Ben's, marketed themselves in the last century with racist stereotypes.

After the Quaker Oats announcement on Wednesday, the food and candy
giant Mars, the owner of
\href{https://www.nytimes3xbfgragh.onion/2007/03/30/business/media/30adco.html}{Uncle
Ben's}, said it was ``evaluating all possibilities'' concerning the
brand. Mars said it did not yet know the changes it would make or when
they would go into effect, but added that it had a responsibility ``to
take a stand in helping to put an end to racial bias and injustices.''

Also on Wednesday, the syrup brand Mrs. Butterworth's said it was
starting ``a complete brand and packaging review'' after acknowledging
that its bottle, which is ``intended to evoke the images of a loving
grandmother,'' could ``be interpreted in a way that is wholly
inconsistent with our values.'' The brand, owned by ConAgra Foods, said
that ``it's heartbreaking and unacceptable that racism and racial
injustices exist around the world'' and pledged to ``be part of the
solution.''

B\&G Foods also said on Wednesday that it was initiating a review of its
Cream of Wheat packaging to ``take steps to ensure that we and our
brands do not inadvertently contribute to systemic racism.''

Land O'Lakes had started
\href{https://www.nytimes3xbfgragh.onion/2020/04/17/business/land-o-lakes-butter.html}{removing
stereotypical Native American imagery} from many of its products before
the recent protests.

Quaker Oats' discomfort with the Aunt Jemima brand was apparent decades
ago, said Scott Buckley, who worked on advertising projects for Quaker
Oats in the 1980s and 1990s, when he was an account supervisor at the
Jordan McGrath Case \& Taylor agency. He said the company was often
reluctant to spend heavily to market Aunt Jemima, believing ``it wasn't
worth the blowback.''

Quaker Oats said on Wednesday that it would donate at least \$5 million
over the next five years ``to create meaningful, ongoing support and
engagement in the Black community.''

Maria Cramer contributed reporting.

Advertisement

\protect\hyperlink{after-bottom}{Continue reading the main story}

\hypertarget{site-index}{%
\subsection{Site Index}\label{site-index}}

\hypertarget{site-information-navigation}{%
\subsection{Site Information
Navigation}\label{site-information-navigation}}

\begin{itemize}
\tightlist
\item
  \href{https://help.nytimes3xbfgragh.onion/hc/en-us/articles/115014792127-Copyright-notice}{©~2020~The
  New York Times Company}
\end{itemize}

\begin{itemize}
\tightlist
\item
  \href{https://www.nytco.com/}{NYTCo}
\item
  \href{https://help.nytimes3xbfgragh.onion/hc/en-us/articles/115015385887-Contact-Us}{Contact
  Us}
\item
  \href{https://www.nytco.com/careers/}{Work with us}
\item
  \href{https://nytmediakit.com/}{Advertise}
\item
  \href{http://www.tbrandstudio.com/}{T Brand Studio}
\item
  \href{https://www.nytimes3xbfgragh.onion/privacy/cookie-policy\#how-do-i-manage-trackers}{Your
  Ad Choices}
\item
  \href{https://www.nytimes3xbfgragh.onion/privacy}{Privacy}
\item
  \href{https://help.nytimes3xbfgragh.onion/hc/en-us/articles/115014893428-Terms-of-service}{Terms
  of Service}
\item
  \href{https://help.nytimes3xbfgragh.onion/hc/en-us/articles/115014893968-Terms-of-sale}{Terms
  of Sale}
\item
  \href{https://spiderbites.nytimes3xbfgragh.onion}{Site Map}
\item
  \href{https://help.nytimes3xbfgragh.onion/hc/en-us}{Help}
\item
  \href{https://www.nytimes3xbfgragh.onion/subscription?campaignId=37WXW}{Subscriptions}
\end{itemize}
