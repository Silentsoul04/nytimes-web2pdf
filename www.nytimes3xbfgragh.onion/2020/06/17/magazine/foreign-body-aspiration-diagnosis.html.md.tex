Sections

SEARCH

\protect\hyperlink{site-content}{Skip to
content}\protect\hyperlink{site-index}{Skip to site index}

\href{https://myaccount.nytimes3xbfgragh.onion/auth/login?response_type=cookie\&client_id=vi}{}

\href{https://www.nytimes3xbfgragh.onion/section/todayspaper}{Today's
Paper}

She Struggled With Asthma, but It Had Never Been This Bad

\url{https://nyti.ms/37zpFbl}

\begin{itemize}
\item
\item
\item
\item
\item
\item
\end{itemize}

\hypertarget{the-coronavirus-outbreak}{%
\subsubsection{\texorpdfstring{\href{https://www.nytimes3xbfgragh.onion/news-event/coronavirus?name=styln-coronavirus-national\&region=TOP_BANNER\&block=storyline_menu_recirc\&action=click\&pgtype=Article\&impression_id=81441c30-f1c4-11ea-9f8d-c5c4a80db310\&variant=undefined}{The
Coronavirus
Outbreak}}{The Coronavirus Outbreak}}\label{the-coronavirus-outbreak}}

\begin{itemize}
\tightlist
\item
  live\href{https://www.nytimes3xbfgragh.onion/2020/09/08/world/covid-19-coronavirus.html?name=styln-coronavirus-national\&region=TOP_BANNER\&block=storyline_menu_recirc\&action=click\&pgtype=Article\&impression_id=81444340-f1c4-11ea-9f8d-c5c4a80db310\&variant=undefined}{Latest
  Updates}
\item
  \href{https://www.nytimes3xbfgragh.onion/interactive/2020/us/coronavirus-us-cases.html?name=styln-coronavirus-national\&region=TOP_BANNER\&block=storyline_menu_recirc\&action=click\&pgtype=Article\&impression_id=81444341-f1c4-11ea-9f8d-c5c4a80db310\&variant=undefined}{Maps
  and Cases}
\item
  \href{https://www.nytimes3xbfgragh.onion/interactive/2020/science/coronavirus-vaccine-tracker.html?name=styln-coronavirus-national\&region=TOP_BANNER\&block=storyline_menu_recirc\&action=click\&pgtype=Article\&impression_id=81444342-f1c4-11ea-9f8d-c5c4a80db310\&variant=undefined}{Vaccine
  Tracker}
\item
  \href{https://www.nytimes3xbfgragh.onion/2020/09/02/your-money/eviction-moratorium-covid.html?name=styln-coronavirus-national\&region=TOP_BANNER\&block=storyline_menu_recirc\&action=click\&pgtype=Article\&impression_id=81446a50-f1c4-11ea-9f8d-c5c4a80db310\&variant=undefined}{Eviction
  Moratorium}
\item
  \href{https://www.nytimes3xbfgragh.onion/interactive/2020/09/02/magazine/food-insecurity-hunger-us.html?name=styln-coronavirus-national\&region=TOP_BANNER\&block=storyline_menu_recirc\&action=click\&pgtype=Article\&impression_id=81446a51-f1c4-11ea-9f8d-c5c4a80db310\&variant=undefined}{American
  Hunger}
\end{itemize}

Advertisement

\protect\hyperlink{after-top}{Continue reading the main story}

Supported by

\protect\hyperlink{after-sponsor}{Continue reading the main story}

\href{/column/diagnosis}{Diagnosis}

\hypertarget{she-struggled-with-asthma-but-it-had-never-been-this-bad}{%
\section{She Struggled With Asthma, but It Had Never Been This
Bad}\label{she-struggled-with-asthma-but-it-had-never-been-this-bad}}

\includegraphics{https://static01.graylady3jvrrxbe.onion/images/2020/06/21/magazine/21mag-diagnosis-1/21mag-diagnosis-1-articleLarge.jpg?quality=75\&auto=webp\&disable=upscale}

By \href{https://www.nytimes3xbfgragh.onion/by/lisa-sanders-md}{Lisa
Sanders, M.D.}

\begin{itemize}
\item
  June 17, 2020
\item
  \begin{itemize}
  \item
  \item
  \item
  \item
  \item
  \item
  \end{itemize}
\end{itemize}

The 58-year-old woman struggled to get out of her car at the West
Roxbury Veterans Affairs Medical Center just outside Boston. When she
finally made it to her feet, she leaned on the trunk of her car as she
gasped for air. She was wheezing so loudly that she could hear it as
well as feel it rumble in her chest. This was in 2019 --- a time before
the novel coronavirus that causes Covid-19 turned this kind of
respiratory distress into an everyday event.

In the E.R., the wheezing woman was immediately moved to the treatment
area. A mask hissing with a watery mist was placed over her nose and
mouth, and only then did her breathing begin to ease.

She had a long history of asthma, the woman told her E.R. doctor, but it
had never been this bad. Four weeks earlier, after the start of the new
year, she came down with a terrible pneumonia in her right lung. The
middle lobe had collapsed completely, she was told. She spent two days
in a community hospital, but even after she was sent home, she didn't
feel well. Her chest was tight, and the slightest effort turned her
breaths into wheezes. Her friends at work were worried. Go back to the
hospital, they urged. But she hated hospitals, so she waited as long as
she could. By the time she decided to drive herself to the V.A. hospital
where she got much of her care, her entire body was exhausted simply
from the effort it took for her to breathe.

\hypertarget{exploring-the-airways}{%
\subsection{\texorpdfstring{\textbf{Exploring the
Airways}}{Exploring the Airways}}\label{exploring-the-airways}}

In the E.R., a chest X-ray showed that she had another pneumonia. She
was started on antibiotics and admitted to the hospital. A CT scan
showed one possible reason for the back-to-back pneumonias. Deep in her
right lung something --- it wasn't clear what --- was blocking one of
the main airways. Pneumonias frequently occur when an airway is
occluded. The patient had a distant history of smoking, which made lung
cancer a possibility. Her medical team reached out to the pulmonology
service, the lung specialists, in case the patient needed a bronchoscopy
--- a bronch, for short. In that procedure, a small camera embedded at
the end of a long tube is snaked through the nose or mouth, down the
throat and into the lungs to get a closer look at the airways or
something inside the airways.

Dr. Justin Rucci was the pulmonologist in training assigned to the
patient's case. If she needed a bronch, he'd be the one to do it. Rucci
carefully went over the patient's records. She didn't have a fever, but
she needed supplemental oxygen to keep her in the normal range. That was
new. Her chest X-ray showed a pneumonia, and the CT scan clearly
revealed a blockage, cutting off the airways into the lower lobe.

\hypertarget{latest-updates-the-coronavirus-outbreak}{%
\section{\texorpdfstring{\href{https://www.nytimes3xbfgragh.onion/2020/09/08/world/covid-19-coronavirus.html?action=click\&pgtype=Article\&state=default\&region=MAIN_CONTENT_1\&context=storylines_live_updates}{Latest
Updates: The Coronavirus
Outbreak}}{Latest Updates: The Coronavirus Outbreak}}\label{latest-updates-the-coronavirus-outbreak}}

Updated 2020-09-08T11:04:36.368Z

\begin{itemize}
\tightlist
\item
  \href{https://www.nytimes3xbfgragh.onion/2020/09/08/world/covid-19-coronavirus.html?action=click\&pgtype=Article\&state=default\&region=MAIN_CONTENT_1\&context=storylines_live_updates\#link-4a77847f}{As
  senators return to Washington, an impasse over a virus relief package
  looms.}
\item
  \href{https://www.nytimes3xbfgragh.onion/2020/09/08/world/covid-19-coronavirus.html?action=click\&pgtype=Article\&state=default\&region=MAIN_CONTENT_1\&context=storylines_live_updates\#link-679303d7}{Nine
  drugmakers pledge to thoroughly vet any coronavirus vaccine.}
\item
  \href{https://www.nytimes3xbfgragh.onion/2020/09/08/world/covid-19-coronavirus.html?action=click\&pgtype=Article\&state=default\&region=MAIN_CONTENT_1\&context=storylines_live_updates\#link-1c973131}{`The
  lockdown killed my father': Farmer suicides add to India's virus
  misery.}
\end{itemize}

\href{https://www.nytimes3xbfgragh.onion/2020/09/08/world/covid-19-coronavirus.html?action=click\&pgtype=Article\&state=default\&region=MAIN_CONTENT_1\&context=storylines_live_updates}{See
more updates}

More live coverage:
\href{https://www.nytimes3xbfgragh.onion/live/2020/09/08/business/stock-market-today-coronavirus?action=click\&pgtype=Article\&state=default\&region=MAIN_CONTENT_1\&context=storylines_live_updates}{Markets}

Rucci had access to the records from the hospital where the patient was
treated for pneumonia the month before. He immediately clicked on the
imaging. In the chest X-ray done during that hospital stay, Rucci easily
identified the bright white disk against the textured gray of the lungs
that indicated a pneumonia; it was in the same lung but in the middle
lobe, not the lower lobe. At that time, her lower lobe looked fine. If
the pneumonia she had now was caused by the obstruction they saw in the
CT scan, could the same problem also have somehow caused the earlier
pneumonia in the middle lobe?

\includegraphics{https://static01.graylady3jvrrxbe.onion/images/2020/06/21/magazine/21mag-diagnosis-2/21mag-diagnosis-2-articleLarge.jpg?quality=75\&auto=webp\&disable=upscale}

\hypertarget{a-scary-episode}{%
\subsection{\texorpdfstring{\textbf{A Scary
Episode}}{A Scary Episode}}\label{a-scary-episode}}

Rucci had seen plenty of patients with blocked airways. The most common
culprits were cancer, a mucus plug or an aspirated foreign body. A
blockage caused by cancer shouldn't move, and it shouldn't grow fast
enough to reach the lower lobe in the time between the two pneumonias.
So cancer seemed unlikely. And she didn't have any problems in her lungs
like scarring or diseases like cystic fibrosis that make mucus plugs
common. Foreign bodies are not often found in adults, but Rucci was
pretty sure that's what she had.

He went to see the patient late that afternoon. She was in bed and
cheerful, despite the plastic tubing that delivered oxygen-enriched air
to her nostrils. After hearing her story, Rucci had a question. Did she
ever have problems swallowing? In fact, she did. She always had to have
a big glass of water at hand to help her get her food down.

What about choking? Did she recall if she recently had a really bad
choking episode --- when her food had gone down the wrong pipe? She most
certainly did. Maybe three months earlier, she was eating a salad and
something hard dropped into her airway. She was home alone, and suddenly
she couldn't breathe at all. She couldn't even cough, though she could
feel herself trying to. She jumped up and ran out of the old farmhouse.
She lived alone and the only other person she could think of on the
property was her elderly landlord, and she couldn't see him anywhere.
Dark spots appeared before her eyes, and she wondered if she'd be found
dead with a piece of her salad stuck in her throat. After what seemed
like forever but was probably less than a minute, something shifted, and
the airway popped open. Her heart raced. She was an Army veteran, but
she'd never felt closer to death than she had right then.

\href{https://www.nytimes3xbfgragh.onion/news-event/coronavirus?action=click\&pgtype=Article\&state=default\&region=MAIN_CONTENT_3\&context=storylines_faq}{}

\hypertarget{the-coronavirus-outbreak-}{%
\subsubsection{The Coronavirus Outbreak
›}\label{the-coronavirus-outbreak-}}

\hypertarget{frequently-asked-questions}{%
\paragraph{Frequently Asked
Questions}\label{frequently-asked-questions}}

Updated September 4, 2020

\begin{itemize}
\item ~
  \hypertarget{what-are-the-symptoms-of-coronavirus}{%
  \paragraph{What are the symptoms of
  coronavirus?}\label{what-are-the-symptoms-of-coronavirus}}

  \begin{itemize}
  \tightlist
  \item
    In the beginning, the coronavirus
    \href{https://www.nytimes3xbfgragh.onion/article/coronavirus-facts-history.html?action=click\&pgtype=Article\&state=default\&region=MAIN_CONTENT_3\&context=storylines_faq\#link-6817bab5}{seemed
    like it was primarily a respiratory illness}~--- many patients had
    fever and chills, were weak and tired, and coughed a lot, though
    some people don't show many symptoms at all. Those who seemed
    sickest had pneumonia or acute respiratory distress syndrome and
    received supplemental oxygen. By now, doctors have identified many
    more symptoms and syndromes. In April,
    \href{https://www.nytimes3xbfgragh.onion/2020/04/27/health/coronavirus-symptoms-cdc.html?action=click\&pgtype=Article\&state=default\&region=MAIN_CONTENT_3\&context=storylines_faq}{the
    C.D.C. added to the list of early signs}~sore throat, fever, chills
    and muscle aches. Gastrointestinal upset, such as diarrhea and
    nausea, has also been observed. Another telltale sign of infection
    may be a sudden, profound diminution of one's
    \href{https://www.nytimes3xbfgragh.onion/2020/03/22/health/coronavirus-symptoms-smell-taste.html?action=click\&pgtype=Article\&state=default\&region=MAIN_CONTENT_3\&context=storylines_faq}{sense
    of smell and taste.}~Teenagers and young adults in some cases have
    developed painful red and purple lesions on their fingers and toes
    --- nicknamed ``Covid toe'' --- but few other serious symptoms.
  \end{itemize}
\item ~
  \hypertarget{why-is-it-safer-to-spend-time-together-outside}{%
  \paragraph{Why is it safer to spend time together
  outside?}\label{why-is-it-safer-to-spend-time-together-outside}}

  \begin{itemize}
  \tightlist
  \item
    \href{https://www.nytimes3xbfgragh.onion/2020/05/15/us/coronavirus-what-to-do-outside.html?action=click\&pgtype=Article\&state=default\&region=MAIN_CONTENT_3\&context=storylines_faq}{Outdoor
    gatherings}~lower risk because wind disperses viral droplets, and
    sunlight can kill some of the virus. Open spaces prevent the virus
    from building up in concentrated amounts and being inhaled, which
    can happen when infected people exhale in a confined space for long
    stretches of time, said Dr. Julian W. Tang, a virologist at the
    University of Leicester.
  \end{itemize}
\item ~
  \hypertarget{why-does-standing-six-feet-away-from-others-help}{%
  \paragraph{Why does standing six feet away from others
  help?}\label{why-does-standing-six-feet-away-from-others-help}}

  \begin{itemize}
  \tightlist
  \item
    The coronavirus spreads primarily through droplets from your mouth
    and nose, especially when you cough or sneeze. The C.D.C., one of
    the organizations using that measure,
    \href{https://www.nytimes3xbfgragh.onion/2020/04/14/health/coronavirus-six-feet.html?action=click\&pgtype=Article\&state=default\&region=MAIN_CONTENT_3\&context=storylines_faq}{bases
    its recommendation of six feet}~on the idea that most large droplets
    that people expel when they cough or sneeze will fall to the ground
    within six feet. But six feet has never been a magic number that
    guarantees complete protection. Sneezes, for instance, can launch
    droplets a lot farther than six feet,
    \href{https://jamanetwork.com/journals/jama/fullarticle/2763852}{according
    to a recent study}. It's a rule of thumb: You should be safest
    standing six feet apart outside, especially when it's windy. But
    keep a mask on at all times, even when you think you're far enough
    apart.
  \end{itemize}
\item ~
  \hypertarget{i-have-antibodies-am-i-now-immune}{%
  \paragraph{I have antibodies. Am I now
  immune?}\label{i-have-antibodies-am-i-now-immune}}

  \begin{itemize}
  \tightlist
  \item
    As of right
    now,\href{https://www.nytimes3xbfgragh.onion/2020/07/22/health/covid-antibodies-herd-immunity.html?action=click\&pgtype=Article\&state=default\&region=MAIN_CONTENT_3\&context=storylines_faq}{~that
    seems likely, for at least several months.}~There have been
    frightening accounts of people suffering what seems to be a second
    bout of Covid-19. But experts say these patients may have a
    drawn-out course of infection, with the virus taking a slow toll
    weeks to months after initial exposure.~People infected with the
    coronavirus typically
    \href{https://www.nature.com/articles/s41586-020-2456-9}{produce}~immune
    molecules called antibodies, which are
    \href{https://www.nytimes3xbfgragh.onion/2020/05/07/health/coronavirus-antibody-prevalence.html?action=click\&pgtype=Article\&state=default\&region=MAIN_CONTENT_3\&context=storylines_faq}{protective
    proteins made in response to an
    infection}\href{https://www.nytimes3xbfgragh.onion/2020/05/07/health/coronavirus-antibody-prevalence.html?action=click\&pgtype=Article\&state=default\&region=MAIN_CONTENT_3\&context=storylines_faq}{.
    These antibodies may}~last in the body
    \href{https://www.nature.com/articles/s41591-020-0965-6}{only two to
    three months}, which may seem worrisome, but that's~perfectly normal
    after an acute infection subsides, said Dr. Michael Mina, an
    immunologist at Harvard University. It may be possible to get the
    coronavirus again, but it's highly unlikely that it would be
    possible in a short window of time from initial infection or make
    people sicker the second time.
  \end{itemize}
\item ~
  \hypertarget{what-are-my-rights-if-i-am-worried-about-going-back-to-work}{%
  \paragraph{What are my rights if I am worried about going back to
  work?}\label{what-are-my-rights-if-i-am-worried-about-going-back-to-work}}

  \begin{itemize}
  \tightlist
  \item
    Employers have to provide
    \href{https://www.osha.gov/SLTC/covid-19/standards.html}{a safe
    workplace}~with policies that protect everyone equally.
    \href{https://www.nytimes3xbfgragh.onion/article/coronavirus-money-unemployment.html?action=click\&pgtype=Article\&state=default\&region=MAIN_CONTENT_3\&context=storylines_faq}{And
    if one of your co-workers tests positive for the coronavirus, the
    C.D.C.}~has said that
    \href{https://www.cdc.gov/coronavirus/2019-ncov/community/guidance-business-response.html}{employers
    should tell their employees}~-\/- without giving you the sick
    employee's name -\/- that they may have been exposed to the virus.
  \end{itemize}
\end{itemize}

\hypertarget{a-new-understanding-of-an-old-event}{%
\subsection{\texorpdfstring{\textbf{A New Understanding of an Old
Event}}{A New Understanding of an Old Event}}\label{a-new-understanding-of-an-old-event}}

Afterward, her chest was sore, but her breathing was back to normal. So
she hadn't thought of it months later when the wheezing started. Even
when the doctors at that first hospital told her she might have a mass,
her thoughts went to cancer and not to that choking episode.

But after she was discharged that first time, she was still coughing up
a storm. After one bad bout of hacking, she brought up something solid.
When she fished it out of her mouth, she saw what looked like a piece of
walnut and recalled those terrible moments when she thought the thing
might kill her. She figured she'd gotten rid of the problem. Perhaps
she'd been wrong.

On Day 5 of this second hospital stay, she was scheduled for the bronch.
She was positioned in a chair that reminded her of the one in her
dentist's office. Once she was sedated, Rucci gently introduced the
endoscope into her mouth, through her vocal cords and into her lungs. He
directed the camera through the complex intersections of the large
airways until he was all the way down to the lower lobe. And there it
was --- wedged in tight, blocking off the entire section. He could see a
sliver of free space near the top of the object. He slid a tiny tool
through the tubing past the camera, and then to the far side of the
object. Once there he moved a switch and felt, rather than saw, a small
net open behind the obstruction. He coaxed the net forward until he was
certain he'd captured the thing. It was too large to be pulled out
through the scope, so he slowly withdrew the entire instrument, keeping
an eye on the captured object. The retrieved item clattered into the
specimen container. Rucci squinted at the object. It was beige and hard.
It was the rest of the snorted walnut. Suddenly it made sense. The
patient had inhaled the nut, which got stuck in the middle lobe. Her
violent cough broke it in two, and one part came up and the other, now
smaller, piece dropped farther down the progressively narrower airways.

\hypertarget{swallowed-foreign-objects}{%
\subsection{\texorpdfstring{\textbf{Swallowed Foreign
Objects}}{Swallowed Foreign Objects}}\label{swallowed-foreign-objects}}

While food is what's usually aspirated, a surprisingly wide variety of
items manage to make their way into the lungs. Chevalier Jackson, a
physician during the late 19th and early 20th centuries, devoted his
career to developing instruments and techniques to retrieve these
misplaced items. During Jackson's 75-year career, he extracted 2,374
inhaled or swallowed foreign bodies from patients' throats, esophagi and
lungs, including safety pins, buttons, screws, dentures and lots and
lots of toys. More than 80 percent of those objects were found in
children. The entire collection, along with details of the patients from
whom they were retrieved and the techniques used, is housed in the
Mütter Museum in Philadelphia.

Like most patients, this one did well after the object was retrieved.
Once the airway was opened, the pneumonia cleared up easily. She went
home a couple of days later. The patient tells me that she still has
trouble swallowing. She recently heard about a kind of physical therapy
that might help, and plans to try that --- once her doctors start seeing
patients again.

Advertisement

\protect\hyperlink{after-bottom}{Continue reading the main story}

\hypertarget{site-index}{%
\subsection{Site Index}\label{site-index}}

\hypertarget{site-information-navigation}{%
\subsection{Site Information
Navigation}\label{site-information-navigation}}

\begin{itemize}
\tightlist
\item
  \href{https://help.nytimes3xbfgragh.onion/hc/en-us/articles/115014792127-Copyright-notice}{©~2020~The
  New York Times Company}
\end{itemize}

\begin{itemize}
\tightlist
\item
  \href{https://www.nytco.com/}{NYTCo}
\item
  \href{https://help.nytimes3xbfgragh.onion/hc/en-us/articles/115015385887-Contact-Us}{Contact
  Us}
\item
  \href{https://www.nytco.com/careers/}{Work with us}
\item
  \href{https://nytmediakit.com/}{Advertise}
\item
  \href{http://www.tbrandstudio.com/}{T Brand Studio}
\item
  \href{https://www.nytimes3xbfgragh.onion/privacy/cookie-policy\#how-do-i-manage-trackers}{Your
  Ad Choices}
\item
  \href{https://www.nytimes3xbfgragh.onion/privacy}{Privacy}
\item
  \href{https://help.nytimes3xbfgragh.onion/hc/en-us/articles/115014893428-Terms-of-service}{Terms
  of Service}
\item
  \href{https://help.nytimes3xbfgragh.onion/hc/en-us/articles/115014893968-Terms-of-sale}{Terms
  of Sale}
\item
  \href{https://spiderbites.nytimes3xbfgragh.onion}{Site Map}
\item
  \href{https://help.nytimes3xbfgragh.onion/hc/en-us}{Help}
\item
  \href{https://www.nytimes3xbfgragh.onion/subscription?campaignId=37WXW}{Subscriptions}
\end{itemize}
