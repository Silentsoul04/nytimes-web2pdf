Sections

SEARCH

\protect\hyperlink{site-content}{Skip to
content}\protect\hyperlink{site-index}{Skip to site index}

\href{https://www.nytimes3xbfgragh.onion/section/politics}{Politics}

\href{https://myaccount.nytimes3xbfgragh.onion/auth/login?response_type=cookie\&client_id=vi}{}

\href{https://www.nytimes3xbfgragh.onion/section/todayspaper}{Today's
Paper}

\href{/section/politics}{Politics}\textbar{}Bernie Sanders Predicted
Revolution, Just Not This One

\url{https://nyti.ms/3daE1jn}

\begin{itemize}
\item
\item
\item
\item
\item
\end{itemize}

\hypertarget{race-and-america}{%
\subsubsection{\texorpdfstring{\href{https://www.nytimes3xbfgragh.onion/news-event/george-floyd-protests-minneapolis-new-york-los-angeles?name=styln-george-floyd\&region=TOP_BANNER\&block=storyline_menu_recirc\&action=click\&pgtype=Article\&impression_id=4c15be80-f28f-11ea-8c44-5749c33336ea\&variant=undefined}{Race
and America}}{Race and America}}\label{race-and-america}}

\begin{itemize}
\tightlist
\item
  \href{https://www.nytimes3xbfgragh.onion/2020/09/04/nyregion/rochester-police-daniel-prude.html?name=styln-george-floyd\&region=TOP_BANNER\&block=storyline_menu_recirc\&action=click\&pgtype=Article\&impression_id=4c15be81-f28f-11ea-8c44-5749c33336ea\&variant=undefined}{What
  Happened in Rochester, N.Y.}
\item
  \href{https://www.nytimes3xbfgragh.onion/2020/09/01/us/politics/trump-fact-check-protests.html?name=styln-george-floyd\&region=TOP_BANNER\&block=storyline_menu_recirc\&action=click\&pgtype=Article\&impression_id=4c15be82-f28f-11ea-8c44-5749c33336ea\&variant=undefined}{Trump
  Fact Check}
\item
  \href{https://www.nytimes3xbfgragh.onion/2020/08/30/us/portland-shooting-explained.html?name=styln-george-floyd\&region=TOP_BANNER\&block=storyline_menu_recirc\&action=click\&pgtype=Article\&impression_id=4c15e590-f28f-11ea-8c44-5749c33336ea\&variant=undefined}{Portland
  Shooting}
\item
  \href{https://www.nytimes3xbfgragh.onion/2020/08/30/us/breonna-taylor-police-killing.html?name=styln-george-floyd\&region=TOP_BANNER\&block=storyline_menu_recirc\&action=click\&pgtype=Article\&impression_id=4c15e591-f28f-11ea-8c44-5749c33336ea\&variant=undefined}{Breonna
  Taylor's Life and Death}
\end{itemize}

Advertisement

\protect\hyperlink{after-top}{Continue reading the main story}

Supported by

\protect\hyperlink{after-sponsor}{Continue reading the main story}

\hypertarget{bernie-sanders-predicted-revolution-just-not-this-one}{%
\section{Bernie Sanders Predicted Revolution, Just Not This
One}\label{bernie-sanders-predicted-revolution-just-not-this-one}}

The politician who talked the most about progressive change is now
wrestling with a nationwide movement he didn't start.

\includegraphics{https://static01.graylady3jvrrxbe.onion/images/2020/06/14/us/politics/00Progessives1/00Progessives1-articleLarge-v2.jpg?quality=75\&auto=webp\&disable=upscale}

\href{https://www.nytimes3xbfgragh.onion/by/sydney-ember}{\includegraphics{https://static01.graylady3jvrrxbe.onion/images/2018/06/12/multimedia/author-sydney-ember/author-sydney-ember-thumbLarge.png}}

By \href{https://www.nytimes3xbfgragh.onion/by/sydney-ember}{Sydney
Ember}

\begin{itemize}
\item
  June 19, 2020
\item
  \begin{itemize}
  \item
  \item
  \item
  \item
  \item
  \end{itemize}
\end{itemize}

People are in the streets, confronting injustice and demanding
fundamental change.

It is the kind of moment that Senator Bernie Sanders spoke about on the
2020 presidential campaign trail, and for decades before that. But when
the revolution finally came, it wasn't his.

The rise of revolutionary sentiment, like many things, is about timing.
The coronavirus outbreak had already renewed support for progressive
policy proposals, including ``Medicare for all.'' But the
disproportionate impact of the coronavirus on black Americans, combined
with the galvanizing death of George Floyd at the hands of the police in
Minneapolis, have heightened the call to address systemic racism and
police brutality, uniting Democrats --- and the country --- in a
campaign for action in a way that Mr. Sanders's message of economic
equality did not.

``People are sick and tired of police murders of African-Americans,''
Mr. Sanders said in an interview. ``People are saying enough is
enough.''

Mr. Sanders, whose slogan on his campaign was ``Not me, us,'' described
the protests as a validation of his theory of social change: ``What I
have said for a very long time is that real change is never going to
come from the top on down, it's always from the bottom on up.''

But during his presidential bid, Mr. Sanders at times seemed
uncomfortable speaking overtly about race. At
\href{https://www.nytimes3xbfgragh.onion/2019/04/24/us/politics/she-the-people-forum-2020-women.html}{a
presidential forum} in April 2019 for women of color, he offered few
specific policy details, and drew some groans from the audience when he
referred to marching with the Rev. Dr. Martin Luther King Jr. in
response to a question about how he would handle current challenges.

And more recently, during
\href{https://www.nytimes3xbfgragh.onion/2020/03/08/us/politics/bernie-sanders-michigan.html}{an
event in Flint, Mich.}, in March that campaign aides had billed as an
opportunity for him to speak directly to black voters, he decided not to
deliver a planned speech and instead largely ceded the stage to
panelists including the academic Cornel West.

When Mr. Sanders spoke about racial equality, it was often in the
context of economic equality, championing proposals and prescriptions
that he believed would improve the lives of all working Americans. He
said that policies like single-payer health care would address higher
maternal and infant mortality rates in black communities. And he wanted
to legalize marijuana and end cash bail, policies he said were aimed in
particular at helping black Americans and other people of color.

These proposals, however, also amounted to an implicit expectation that
\href{https://www.nytimes3xbfgragh.onion/2020/01/19/us/politics/bernie-sanders-black-vote-elizabeth-warren.html}{voters
trust the government} --- an especially difficult sell for those
including older black voters who feel they have been historically let
down by the government.

They were shortcomings that help explain why Mr. Sanders lost to former
Vice President
\href{https://www.nytimes3xbfgragh.onion/interactive/2020/us/elections/joe-biden.html}{Joseph
R. Biden Jr.} in the Democratic primary race: Unable to win over older
black voters, he came in a distant second to Mr. Biden in South
Carolina, then went on to lose to Mr. Biden in every Southern state on
Super Tuesday. Those defeats, Mr. Sanders's allies say, contributed to
the perception that Mr. Biden was more electable and would fare better
against
\href{https://www.nytimes3xbfgragh.onion/interactive/2020/us/elections/donald-trump.html}{President
Trump} in the general election in November --- a notion that helped
propel Mr. Biden to victory in the primary.

Now, some of the same progressive leaders who never quite figured out
how to mobilize such a broad coalition in the primary are considering
how to not only support the movement, but also harness its energy as
they look toward November.

Many Sanders allies maintain that racial justice was and continues to be
a central tenet of his ideology. ``If you do parse back the things that
our campaign was fighting for, those policies, if they were able to come
to life, would have certainly changed the positions for
African-Americans in this country,'' said Nina Turner, a national
co-chair of the Sanders campaign and one of his most prominent black
surrogates.

Though she praised his agenda regarding black Americans, she conceded
that he did not articulate it forcefully enough. ``It had that kind of
tone to it, but it wasn't as piercing as this moment demands,'' she
said.

\includegraphics{https://static01.graylady3jvrrxbe.onion/images/2020/06/14/us/politics/00Progessives2/merlin_173350497_6ea8ce6f-c8c5-4154-8cb7-72390ddda69d-articleLarge.jpg?quality=75\&auto=webp\&disable=upscale}

Yet amid a national movement for racial justice that took hold after
high-profile killings of black men and women, there is also an
acknowledgment among some progressives that their discussion of racism,
including from their standard-bearer, did not seem to meet or anticipate
the forcefulness of these protests.

Kimberlé Crenshaw, the legal scholar who pioneered the concept of
intersectionality to describe how various forms of discrimination can
overlap, said that Mr. Sanders struggled with the reality that talking
forcefully about racial injustice has traditionally alienated white
voters --- especially the working-class white voters he was aiming to
win over. But that is where thinking of class as a ``colorblind
experience'' limits white progressives. ``Class cannot help you see the
specific contours of race disparity,'' she said.

Many other institutions, she noted, have now gone further faster than
the party that is the political base of most African-American voters.
``You basically have a moment where every corporation worth its salt is
saying something about structural racism and anti-blackness, and that
stuff is even outdistancing what candidates in the Democratic Party were
actually saying,'' she said.

Already, a split has emerged in the way progressive leaders and
protesters approach systemic racism and police reform, raising broader
questions about whether elected officials are in sync with what is
happening on the ground. While some activists have embraced the
protesters' rallying cry to ``defund the police,'' many progressive
leaders, including Mr. Sanders, are calibrating their approach.

Unlike during the primary season, when he often took the most leftward
position, Mr. Sanders has disagreed with protesters' demands to
eliminate funding for police departments, staking out a careful position
on police reform.

``Anyone who thinks that we should abolish all police departments in
America, I don't agree,'' Mr. Sanders
\href{https://www.newyorker.com/news/the-new-yorker-interview/bernie-sanders-is-not-done-fighting}{told
The New Yorker}. In keeping with his stance when he
\href{https://www.nytimes3xbfgragh.onion/2015/11/26/us/politics/as-mayor-bernie-sanders-was-more-pragmatic-than-socialist.html}{was
mayor of Burlington, Vt.}, he supported paying police officers more.

At the same time, progressive organizations like the Sunrise Movement, a
youth-led liberal environmental group that
\href{https://www.nytimes3xbfgragh.onion/2020/01/09/us/politics/bernie-sanders-sunrise-movement-endorsement.html}{endorsed
Mr. Sanders in the primary}, have aggressively pushed to defund the
police, adopting the policy as one of their own. When Mr. Biden released
a statement last week that took a more cautious position on police
overhaul, the Sunrise Movement
\href{https://twitter.com/sunrisemvmt/status/1270130352544235521}{denounced
his stance} on Twitter. ``\href{https://twitter.com/JoeBiden}{@JoeBiden}
you're hurting any chance you have at defeating Trump by taking these
centrist stances,'' the group said. ``We need someone fighting with us
to create bold change, not someone to maintain the status-quo
\href{https://twitter.com/hashtag/DefundPolice?src=hashtag_click}{\#DefundPolice}.''

But while most progressives might not have seen this revolution coming,
they are catching up.

Rahna Epting, the executive director of the progressive group MoveOn,
said the protests were a time for national groups like hers to listen to
the grass roots. ``In terms of what we do, we see the people on the
streets right now, this is completely organic,'' she said. ``This is
beyond any one organization or institution.''

She added: ``We're recognizing the moment is not ours, it's the
people's, and we need to flank the people right now.''

The protests are not directly connected to partisan politics, even
though there are some similarities between their broad demands and the
revolutionary sentiment embodied by Mr. Sanders's campaign. But if there
is overlap, it is not yet clear whether the energy on the ground,
particularly among young progressives who supported Mr. Sanders but
remain dissatisfied with Mr. Biden, will translate to enthusiasm at the
ballot box in November.

Progressives at both the national and grass-roots level are still trying
to push Mr. Biden to the left even as he has
\href{https://www.nytimes3xbfgragh.onion/2020/05/13/us/politics/joe-biden-trump.html}{begun
to adopt} the language of systemic disruption. His willingness to
satisfy their demands --- on policing perhaps most urgently but also on
issues like climate change and health care --- could help determine
whether he is successful in the general election.

Despite persistent ideological disagreements, some progressive leaders
are optimistic that the mass social movement will become an animating
force in the upcoming election, especially for voters on the left who
may have been unhappy initially with Mr. Biden. In a
\href{https://www.pewsocialtrends.org/2020/06/12/amid-protests-majorities-across-racial-and-ethnic-groups-express-support-for-the-black-lives-matter-movement/}{Pew
Research Center survey} released last week, 91 percent of Democrats and
those who lean Democratic said they supported the Black Lives Matter
movement.

``At some point, many of the people on the street will view this
election as a referendum on black lives,'' said Maurice Mitchell, the
national director of the left-wing Working Families Party and a leader
in the Movement for Black Lives, a coalition of rights groups. In
addition to elevating the demands of protesters, he said, his group
plans to support candidates who are ``brave enough to say that this is a
time that we take on the police.''

Some progressives point to the demonstrations occurring around the
country, in big cities and small towns, as proof that many Americans did
support the idea of systemic, revolutionary change, even if it did not
succeed as an electoral argument for Mr. Sanders in the primary.

``We see on television wonderful evidence of the kinetic energy of the
Democratic Party,'' said Faiz Shakir, who served as Mr. Sanders's
campaign manager, ``that was obviously evident during the Bernie Sanders
campaign, too.''

Image

Jamaal Bowman, a progressive congressional candidate in New York State,
said,~``What we need now from Democrats and the country is a deep
analysis of structural racism.''~Credit...Desiree Rios for The New York
Times

Many political observers are eagerly awaiting the results of the June 23
Democratic primary in New York's 16th Congressional District, a
predominantly black and Hispanic district that includes parts of
Westchester and the Bronx, where Jamaal Bowman, a middle school
principal and outspoken advocate for racial justice, is running a tight
race against the longtime incumbent, Representative Eliot L. Engel. In
recent days, Mr. Bowman has received endorsements from several top
progressives, including Representative Alexandria Ocasio-Cortez of New
York, Senator Elizabeth Warren and Mr. Sanders.

In an interview, Mr. Bowman was reluctant to speak about his electoral
odds. But he said he had received messages from supporters thanking him
for speaking out against racism and racial inequality. ``What we need
now from Democrats and the country is a deep analysis of structural
racism,'' he said.

It is a sentiment that has taken on new urgency among many progressives,
who stress their longstanding support for racial justice but who have
also come to realize that they must more directly confront racism and
police brutality than they have before.

To support the protest movement, Mr. Sanders has endorsed a slate of
progressive candidates who are fighting explicitly for racial justice.
Using his email list, he has raised more than \$2 million for racial
justice organizations, according to a spokesman.

``It's not good enough to sit back,'' Mr. Sanders said.

Although he said he thought there would be ``a broadening of discussion
about what we mean by justice in America,'' he also held firm to his
political philosophy.

``If you're serious about racial justice --- if you're serious about
criminal justice,'' he said, ``you have got to be serious about economic
justice.''

Giovanni Russonello contributed reporting. Kitty Bennett contributed
research.

Advertisement

\protect\hyperlink{after-bottom}{Continue reading the main story}

\hypertarget{site-index}{%
\subsection{Site Index}\label{site-index}}

\hypertarget{site-information-navigation}{%
\subsection{Site Information
Navigation}\label{site-information-navigation}}

\begin{itemize}
\tightlist
\item
  \href{https://help.nytimes3xbfgragh.onion/hc/en-us/articles/115014792127-Copyright-notice}{©~2020~The
  New York Times Company}
\end{itemize}

\begin{itemize}
\tightlist
\item
  \href{https://www.nytco.com/}{NYTCo}
\item
  \href{https://help.nytimes3xbfgragh.onion/hc/en-us/articles/115015385887-Contact-Us}{Contact
  Us}
\item
  \href{https://www.nytco.com/careers/}{Work with us}
\item
  \href{https://nytmediakit.com/}{Advertise}
\item
  \href{http://www.tbrandstudio.com/}{T Brand Studio}
\item
  \href{https://www.nytimes3xbfgragh.onion/privacy/cookie-policy\#how-do-i-manage-trackers}{Your
  Ad Choices}
\item
  \href{https://www.nytimes3xbfgragh.onion/privacy}{Privacy}
\item
  \href{https://help.nytimes3xbfgragh.onion/hc/en-us/articles/115014893428-Terms-of-service}{Terms
  of Service}
\item
  \href{https://help.nytimes3xbfgragh.onion/hc/en-us/articles/115014893968-Terms-of-sale}{Terms
  of Sale}
\item
  \href{https://spiderbites.nytimes3xbfgragh.onion}{Site Map}
\item
  \href{https://help.nytimes3xbfgragh.onion/hc/en-us}{Help}
\item
  \href{https://www.nytimes3xbfgragh.onion/subscription?campaignId=37WXW}{Subscriptions}
\end{itemize}
