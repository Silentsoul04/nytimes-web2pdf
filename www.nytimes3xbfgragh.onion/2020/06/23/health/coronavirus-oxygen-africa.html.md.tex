Sections

SEARCH

\protect\hyperlink{site-content}{Skip to
content}\protect\hyperlink{site-index}{Skip to site index}

\href{https://www.nytimes3xbfgragh.onion/section/health}{Health}

\href{https://myaccount.nytimes3xbfgragh.onion/auth/login?response_type=cookie\&client_id=vi}{}

\href{https://www.nytimes3xbfgragh.onion/section/todayspaper}{Today's
Paper}

\href{/section/health}{Health}\textbar{}A Simple Way to Save Lives as
Covid-19 Hits Poorer Nations.

\url{https://nyti.ms/3fM9kCM}

\begin{itemize}
\item
\item
\item
\item
\item
\item
\end{itemize}

\hypertarget{the-coronavirus-outbreak}{%
\subsubsection{\texorpdfstring{\href{https://www.nytimes3xbfgragh.onion/news-event/coronavirus?name=styln-coronavirus-national\&region=TOP_BANNER\&block=storyline_menu_recirc\&action=click\&pgtype=Article\&impression_id=13887c50-f1c3-11ea-9feb-8792ec8b6371\&variant=undefined}{The
Coronavirus
Outbreak}}{The Coronavirus Outbreak}}\label{the-coronavirus-outbreak}}

\begin{itemize}
\tightlist
\item
  live\href{https://www.nytimes3xbfgragh.onion/2020/09/08/world/covid-19-coronavirus.html?name=styln-coronavirus-national\&region=TOP_BANNER\&block=storyline_menu_recirc\&action=click\&pgtype=Article\&impression_id=13887c51-f1c3-11ea-9feb-8792ec8b6371\&variant=undefined}{Latest
  Updates}
\item
  \href{https://www.nytimes3xbfgragh.onion/interactive/2020/us/coronavirus-us-cases.html?name=styln-coronavirus-national\&region=TOP_BANNER\&block=storyline_menu_recirc\&action=click\&pgtype=Article\&impression_id=13887c52-f1c3-11ea-9feb-8792ec8b6371\&variant=undefined}{Maps
  and Cases}
\item
  \href{https://www.nytimes3xbfgragh.onion/interactive/2020/science/coronavirus-vaccine-tracker.html?name=styln-coronavirus-national\&region=TOP_BANNER\&block=storyline_menu_recirc\&action=click\&pgtype=Article\&impression_id=1388a360-f1c3-11ea-9feb-8792ec8b6371\&variant=undefined}{Vaccine
  Tracker}
\item
  \href{https://www.nytimes3xbfgragh.onion/2020/09/02/your-money/eviction-moratorium-covid.html?name=styln-coronavirus-national\&region=TOP_BANNER\&block=storyline_menu_recirc\&action=click\&pgtype=Article\&impression_id=1388a361-f1c3-11ea-9feb-8792ec8b6371\&variant=undefined}{Eviction
  Moratorium}
\item
  \href{https://www.nytimes3xbfgragh.onion/interactive/2020/09/02/magazine/food-insecurity-hunger-us.html?name=styln-coronavirus-national\&region=TOP_BANNER\&block=storyline_menu_recirc\&action=click\&pgtype=Article\&impression_id=1388a362-f1c3-11ea-9feb-8792ec8b6371\&variant=undefined}{American
  Hunger}
\end{itemize}

Advertisement

\protect\hyperlink{after-top}{Continue reading the main story}

Supported by

\protect\hyperlink{after-sponsor}{Continue reading the main story}

Global health

\hypertarget{a-simple-way-to-save-lives-as-covid-19-hits-poorer-nations}{%
\section{A Simple Way to Save Lives as Covid-19 Hits Poorer
Nations.}\label{a-simple-way-to-save-lives-as-covid-19-hits-poorer-nations}}

Aid agencies are scrambling to get oxygen equipment to low-income
countries where the coronavirus is rapidly spreading.

\includegraphics{https://static01.graylady3jvrrxbe.onion/images/2020/06/25/science/23VIRUS-OXYGEN1/23VIRUS-OXYGEN1-articleLarge.jpg?quality=75\&auto=webp\&disable=upscale}

\href{https://www.nytimes3xbfgragh.onion/by/donald-g-mcneil-jr}{\includegraphics{https://static01.graylady3jvrrxbe.onion/images/2018/06/13/multimedia/author-donald-g-mcneil-jr/author-donald-g-mcneil-jr-thumbLarge-v4.png}}

By
\href{https://www.nytimes3xbfgragh.onion/by/donald-g-mcneil-jr}{Donald
G. McNeil Jr.}

\begin{itemize}
\item
  Published June 23, 2020Updated June 24, 2020
\item
  \begin{itemize}
  \item
  \item
  \item
  \item
  \item
  \item
  \end{itemize}
\end{itemize}

As the coronavirus pandemic hits more impoverished countries with
fragile health care systems, global health authorities are scrambling
for supplies of a simple treatment that saves lives: oxygen.

Many patients severely ill with Covid-19, the illness caused by the
coronavirus, require help with breathing at some point. But now the
epidemic is spreading rapidly in South Asia, Latin America and parts of
Africa, regions of the world where many hospitals are poorly equipped
and lack the ventilators, tanks and other equipment necessary to save
patients whose lungs are failing.

The World Health Organization is hoping to raise \$250 million to
increase oxygen delivery to those regions. The World Bank and the
African Union are contributing to the effort, and some medical charities
are seeking donations for the cause.

The W.H.O. estimated on Wednesday that with about one million new
coronavirus cases worldwide per week, the world will need 620,000 cubic
meters of oxygen per day, or 88,000 large cylinders.

By a stroke of luck, the W.H.O., UNICEF and the Bill \& Melinda Gates
Foundation in 2017 began searching for ways to increase oxygen delivery
in poor and middle-income countries --- not in anticipation of a
pandemic, but because oxygen can save the lives of premature infants and
children with pneumonia.

The organizations began ordering equipment in January, but within weeks
suppliers were swamped by the sudden surge in demand created by the
pandemic.

The W.H.O. has purchased 14,000 oxygen concentrators, which filter
oxygen from the air, to be sent to 120 countries, and the agency hopes
to buy 170,000 more concentrators over the next six months at a cost of
\$100 million.

Dr. Tedros Adhanom Ghebreyesus, the director-general of the W.H.O.,
warned that ``80 percent of the market is owned by just a few companies,
and demand is outstripping supply.''

Although the machinery needed to generate oxygen is relatively simple,
it must be sturdy enough to withstand the dust, humidity and other
hazards common in rural hospitals in poor countries. Some companies
produce relatively rugged equipment, but prices are rising and
restrictions on international flights are complicating deliveries.

The machines cannot come too soon, doctors working in the field said.

\includegraphics{https://static01.graylady3jvrrxbe.onion/images/2020/06/23/science/23VIRUS-OXYGEN2/merlin_171607428_03894edc-74a7-4f6c-8e55-6c3ce3d5d8ef-articleLarge.jpg?quality=75\&auto=webp\&disable=upscale}

In May, the Alliance for International Medical Action, or Alima, treated
123 Covid-19 patients in the Democratic Republic of Congo, said Dr.
Baweye Mayoum Barka, the charity's representative in Kinshasa, the
country's capital. Fifty-six of them needed oxygen, but not enough
equipment was available.

\hypertarget{latest-updates-the-coronavirus-outbreak}{%
\section{\texorpdfstring{\href{https://www.nytimes3xbfgragh.onion/2020/09/08/world/covid-19-coronavirus.html?action=click\&pgtype=Article\&state=default\&region=MAIN_CONTENT_1\&context=storylines_live_updates}{Latest
Updates: The Coronavirus
Outbreak}}{Latest Updates: The Coronavirus Outbreak}}\label{latest-updates-the-coronavirus-outbreak}}

Updated 2020-09-08T10:37:22.362Z

\begin{itemize}
\tightlist
\item
  \href{https://www.nytimes3xbfgragh.onion/2020/09/08/world/covid-19-coronavirus.html?action=click\&pgtype=Article\&state=default\&region=MAIN_CONTENT_1\&context=storylines_live_updates\#link-4a77847f}{As
  senators return to Washington, an impasse over a virus relief package
  looms.}
\item
  \href{https://www.nytimes3xbfgragh.onion/2020/09/08/world/covid-19-coronavirus.html?action=click\&pgtype=Article\&state=default\&region=MAIN_CONTENT_1\&context=storylines_live_updates\#link-679303d7}{Nine
  drugmakers pledge to thoroughly vet any coronavirus vaccine.}
\item
  \href{https://www.nytimes3xbfgragh.onion/2020/09/08/world/covid-19-coronavirus.html?action=click\&pgtype=Article\&state=default\&region=MAIN_CONTENT_1\&context=storylines_live_updates\#link-1c973131}{`The
  lockdown killed my father': Farmer suicides add to India's virus
  misery.}
\end{itemize}

\href{https://www.nytimes3xbfgragh.onion/2020/09/08/world/covid-19-coronavirus.html?action=click\&pgtype=Article\&state=default\&region=MAIN_CONTENT_1\&context=storylines_live_updates}{See
more updates}

More live coverage:
\href{https://www.nytimes3xbfgragh.onion/live/2020/09/08/business/stock-market-today-coronavirus?action=click\&pgtype=Article\&state=default\&region=MAIN_CONTENT_1\&context=storylines_live_updates}{Markets}

``So, unfortunately, there were 26 deaths, 70 percent of them in less
than 24 hours,'' Dr. Barka said. ``I can't say they were all from a lack
of oxygen, but it played a role.''

Alima needs 40 oxygen concentrators, but the agency has just eight, he
said. Because it is hard to move patients from one hospital to another,
some die waiting, gasping for air.

In Congo, many Covid-19 patients arrive at hospitals with critically low
blood oxygen levels --- sometimes as low as 60 percent, a level at which
patients must normally be put on a ventilator to survive. (Normal oxygen
saturation levels are 95 percent or more.)

One such patient was a doctor who had for a while refused to go to the
hospital and instead stayed home, taking chloroquine, which is still in
Congo's national treatment guidelines.

``Then, when his condition deteriorated and he did come, just as he was
nearing the Covid building, he developed convulsions,'' Dr. Barka
recalled. ``They stopped to give him a drug for them, and he died just
at the gate.''

Nigeria is also grappling with an oxygen shortage, said Dr. Sanjana
Bhardwaj, UNICEF's chief of health there. Since May, hospitals in Lagos
and Kano have seen a steady stream of older patients with Covid-19
symptoms who need oxygen.

In nearly every country the virus has hit, rich or poor, about 15
percent of all symptomatic patients develop pneumonia severe enough to
require extra oxygen, the W.H.O. estimates, but not so dire that they
must be put on a ventilator.

Ventilators are rare in poor countries; they can cost up to \$50,000,
and patients must be heavily sedated the whole time the breathing tube
is lodged deep in their airways; also, the pressure must be constantly
monitored to prevent lung damage. That requires anesthesiologists and
trained respiratory technicians, positions that many hospitals lack.

Image

Marie Louise Diouf Ndour, a nurse working with the charity Alima,
preparing to enter the coronavirus treatment area of the Hospital Center
University De Fann in Dakar, Senegal, with an oxygen
concentrator.Credit...Sylvain Cherkaoui/Alima

Oxygen can be delivered in two ways. Tanks contain nearly pure oxygen.
For patients who need large volumes and help keeping the air sacs in
their lungs open, tanks can deliver oxygen at high pressure through a
mask strapped tightly over the nose and mouth.

But tanks are heavy, must be refilled at central stations and delivered
by truck, and they pose some risk of explosion and fire. While many poor
countries have plants making industrial-grade oxygen for construction
jobs like welding, it cannot be used on patients because the tanks often
contain rust or oily water that could lodge in the lungs, said Paul
Molinaro, chief of operations support and logistics at the W.H.O.

An alternative is an
\href{https://en.wikipedia.org/wiki/Oxygen_concentrator}{oxygen
concentrator}, which is usually the size of a suitcase or even a
briefcase. Concentrators pull oxygen out of ambient air by forcing it
under pressure through a ``molecular sieve'' filled with the mineral
zeolite, which adsorbs nitrogen.

Most concentrators cost only \$1,000 to \$2,000. They need electricity
but can run on a generator or batteries, using about as much power as a
small refrigerator.

Typically concentrators can produce about 90 percent pure oxygen. They
do not deliver it under pressure, but the thin tube through which the
oxygen streams can be connected to a continuous positive airway pressure
machine, or CPAP, to enrich the air it blows into the lungs.

Alima has started a campaign, ``Oxygen for Africa,'' to raise money to
send about 500 concentrators to six poor countries, Jennifer Lazuta, a
spokeswoman, said.

UNICEF \href{https://www.unicef.org/innovation/oxygen-therapy}{has
ordered about 16,000 concentrators} for about 90 countries, but thus far
has been able to deliver only about 700, said Jonathan Howard-Brand, an
innovation specialist at UNICEF's procurement center in Copenhagen.

\href{https://www.nytimes3xbfgragh.onion/news-event/coronavirus?action=click\&pgtype=Article\&state=default\&region=MAIN_CONTENT_3\&context=storylines_faq}{}

\hypertarget{the-coronavirus-outbreak-}{%
\subsubsection{The Coronavirus Outbreak
›}\label{the-coronavirus-outbreak-}}

\hypertarget{frequently-asked-questions}{%
\paragraph{Frequently Asked
Questions}\label{frequently-asked-questions}}

Updated September 4, 2020

\begin{itemize}
\item ~
  \hypertarget{what-are-the-symptoms-of-coronavirus}{%
  \paragraph{What are the symptoms of
  coronavirus?}\label{what-are-the-symptoms-of-coronavirus}}

  \begin{itemize}
  \tightlist
  \item
    In the beginning, the coronavirus
    \href{https://www.nytimes3xbfgragh.onion/article/coronavirus-facts-history.html?action=click\&pgtype=Article\&state=default\&region=MAIN_CONTENT_3\&context=storylines_faq\#link-6817bab5}{seemed
    like it was primarily a respiratory illness}~--- many patients had
    fever and chills, were weak and tired, and coughed a lot, though
    some people don't show many symptoms at all. Those who seemed
    sickest had pneumonia or acute respiratory distress syndrome and
    received supplemental oxygen. By now, doctors have identified many
    more symptoms and syndromes. In April,
    \href{https://www.nytimes3xbfgragh.onion/2020/04/27/health/coronavirus-symptoms-cdc.html?action=click\&pgtype=Article\&state=default\&region=MAIN_CONTENT_3\&context=storylines_faq}{the
    C.D.C. added to the list of early signs}~sore throat, fever, chills
    and muscle aches. Gastrointestinal upset, such as diarrhea and
    nausea, has also been observed. Another telltale sign of infection
    may be a sudden, profound diminution of one's
    \href{https://www.nytimes3xbfgragh.onion/2020/03/22/health/coronavirus-symptoms-smell-taste.html?action=click\&pgtype=Article\&state=default\&region=MAIN_CONTENT_3\&context=storylines_faq}{sense
    of smell and taste.}~Teenagers and young adults in some cases have
    developed painful red and purple lesions on their fingers and toes
    --- nicknamed ``Covid toe'' --- but few other serious symptoms.
  \end{itemize}
\item ~
  \hypertarget{why-is-it-safer-to-spend-time-together-outside}{%
  \paragraph{Why is it safer to spend time together
  outside?}\label{why-is-it-safer-to-spend-time-together-outside}}

  \begin{itemize}
  \tightlist
  \item
    \href{https://www.nytimes3xbfgragh.onion/2020/05/15/us/coronavirus-what-to-do-outside.html?action=click\&pgtype=Article\&state=default\&region=MAIN_CONTENT_3\&context=storylines_faq}{Outdoor
    gatherings}~lower risk because wind disperses viral droplets, and
    sunlight can kill some of the virus. Open spaces prevent the virus
    from building up in concentrated amounts and being inhaled, which
    can happen when infected people exhale in a confined space for long
    stretches of time, said Dr. Julian W. Tang, a virologist at the
    University of Leicester.
  \end{itemize}
\item ~
  \hypertarget{why-does-standing-six-feet-away-from-others-help}{%
  \paragraph{Why does standing six feet away from others
  help?}\label{why-does-standing-six-feet-away-from-others-help}}

  \begin{itemize}
  \tightlist
  \item
    The coronavirus spreads primarily through droplets from your mouth
    and nose, especially when you cough or sneeze. The C.D.C., one of
    the organizations using that measure,
    \href{https://www.nytimes3xbfgragh.onion/2020/04/14/health/coronavirus-six-feet.html?action=click\&pgtype=Article\&state=default\&region=MAIN_CONTENT_3\&context=storylines_faq}{bases
    its recommendation of six feet}~on the idea that most large droplets
    that people expel when they cough or sneeze will fall to the ground
    within six feet. But six feet has never been a magic number that
    guarantees complete protection. Sneezes, for instance, can launch
    droplets a lot farther than six feet,
    \href{https://jamanetwork.com/journals/jama/fullarticle/2763852}{according
    to a recent study}. It's a rule of thumb: You should be safest
    standing six feet apart outside, especially when it's windy. But
    keep a mask on at all times, even when you think you're far enough
    apart.
  \end{itemize}
\item ~
  \hypertarget{i-have-antibodies-am-i-now-immune}{%
  \paragraph{I have antibodies. Am I now
  immune?}\label{i-have-antibodies-am-i-now-immune}}

  \begin{itemize}
  \tightlist
  \item
    As of right
    now,\href{https://www.nytimes3xbfgragh.onion/2020/07/22/health/covid-antibodies-herd-immunity.html?action=click\&pgtype=Article\&state=default\&region=MAIN_CONTENT_3\&context=storylines_faq}{~that
    seems likely, for at least several months.}~There have been
    frightening accounts of people suffering what seems to be a second
    bout of Covid-19. But experts say these patients may have a
    drawn-out course of infection, with the virus taking a slow toll
    weeks to months after initial exposure.~People infected with the
    coronavirus typically
    \href{https://www.nature.com/articles/s41586-020-2456-9}{produce}~immune
    molecules called antibodies, which are
    \href{https://www.nytimes3xbfgragh.onion/2020/05/07/health/coronavirus-antibody-prevalence.html?action=click\&pgtype=Article\&state=default\&region=MAIN_CONTENT_3\&context=storylines_faq}{protective
    proteins made in response to an
    infection}\href{https://www.nytimes3xbfgragh.onion/2020/05/07/health/coronavirus-antibody-prevalence.html?action=click\&pgtype=Article\&state=default\&region=MAIN_CONTENT_3\&context=storylines_faq}{.
    These antibodies may}~last in the body
    \href{https://www.nature.com/articles/s41591-020-0965-6}{only two to
    three months}, which may seem worrisome, but that's~perfectly normal
    after an acute infection subsides, said Dr. Michael Mina, an
    immunologist at Harvard University. It may be possible to get the
    coronavirus again, but it's highly unlikely that it would be
    possible in a short window of time from initial infection or make
    people sicker the second time.
  \end{itemize}
\item ~
  \hypertarget{what-are-my-rights-if-i-am-worried-about-going-back-to-work}{%
  \paragraph{What are my rights if I am worried about going back to
  work?}\label{what-are-my-rights-if-i-am-worried-about-going-back-to-work}}

  \begin{itemize}
  \tightlist
  \item
    Employers have to provide
    \href{https://www.osha.gov/SLTC/covid-19/standards.html}{a safe
    workplace}~with policies that protect everyone equally.
    \href{https://www.nytimes3xbfgragh.onion/article/coronavirus-money-unemployment.html?action=click\&pgtype=Article\&state=default\&region=MAIN_CONTENT_3\&context=storylines_faq}{And
    if one of your co-workers tests positive for the coronavirus, the
    C.D.C.}~has said that
    \href{https://www.cdc.gov/coronavirus/2019-ncov/community/guidance-business-response.html}{employers
    should tell their employees}~-\/- without giving you the sick
    employee's name -\/- that they may have been exposed to the virus.
  \end{itemize}
\end{itemize}

The W.H.O. has ordered another 14,000, of which 2,000 have been
delivered and 2,000 are in transit, Mr. Molinaro said.

Image

Pallets of health supplies delivered to UNICEF workers in Nigeria,
including 15 oxygen concentrators, in April.Credit...UNICEF

He and Mr. Howard-Brand described severe delivery problems created by
the epidemic, including delays of up to five weeks. When possible, the
aid agencies ship through the World Food Program, which
\href{https://www.wfp.org/publications/wfp-aviation-2018}{has dozens of
planes}. But the concentrators must compete for space with shipments of
food, personal protective gear and other lifesaving goods.

Also, some countries are far from cargo hub cities, while others
restrict all flights, even those containing aid, for fear of the virus
being introduced.

``We need more planes in the air,'' Mr. Howard-Brand said.

UNICEF is also buying tens of thousands of pulse oximeters, fingertip
devices to measure blood-oxygen saturation.

In deciding how much equipment to buy, the aid agencies are, to some
extent, flying blind. As New York State learned when it was desperately
collecting ventilators in March, how great the need will be is
unpredictable.

\textbf{\emph{{[}}\href{http://on.fb.me/1paTQ1h}{\emph{Like the Science
Times page on Facebook.}}} ****** \emph{\textbar{} Sign up for the}
\textbf{\href{http://nyti.ms/1MbHaRU}{\emph{Science Times
newsletter.}}\emph{{]}}}

Younger Covid-19 patients and those without other health problems often
survive without supplemental oxygen. Populations in Africa skew younger,
because vaccination and anti-malaria campaigns over the past two decades
have saved many children who otherwise would have died. Wide swaths of
older Africans died of AIDS before H.I.V. therapy became widely
available in the mid-2000s.

(It is
\href{https://www.hiv.gov/hiv-basics/staying-in-hiv-care/other-related-health-issues/coronavirus-covid-19}{still
unclear whether being on H.I.V. treatment} increases risks of death from
Covid-19, according to the Centers for Disease Control and Prevention.
But U.N.AIDS, the U.N. program fighting the disease,
\href{https://www.voanews.com/science-health/un-warns-risk-low-distribution-aids-drug-amid-covid-lockdowns}{worries
that lockdowns and border closings will disrupt supplies} of H.I.V.
medicines, which would undoubtedly put H.I.V. patients at high risk.)

Image

People lined up to refill medical oxygen tanks in Lima this
month.Credit...Raul Sifuentes/Getty Images

The agencies seek advice from other aid personnel in each country to
estimate how much equipment is needed, Mr. Molinaro said. If he had more
money and time, he added, he would concentrate on ways to increase
supplies of tanked oxygen, which is dangerous to ship and so must be
produced on site.

In recent years, some public-private partnerships have sprung up to do
that. In East Africa, for example, an aid organization,
\href{https://assistinternational.org/global-health/access-to-oxygen/}{Assist
International}, set out several years ago to break local corporate
monopolies producing medical oxygen that many public hospitals in Africa
could not afford.

With equipment supplied by the GE Foundation and money from Grand
Challenges Canada and other donors, Assist now has a network of
oxygen-making plants in Rwanda, Kenya and Ethiopia.

The U.N.'s oxygen-concentrator procurement effort, begun in April, was a
natural extension of the U.N.'s Oxygen Therapy Project, which began in
2017 with Gates Foundation support in an effort to save babies and
children.

By January, the project had found four manufacturers --- two in China
and two in the United States --- whose machines could stand up to harsh
conditions and which could add voltage stabilizers to prevent damage
from power spikes, which are common in the electrical systems of poor
countries and anywhere that relies on generators for power.

The agencies were just beginning to place orders when the pandemic
began.

``Our timing was immaculate,'' said Mr. Howard-Brand, who helped write
the specifications for the new machines. ``Now maybe the market will
open up.''

David Waldstein contributed reporting from New Rochelle, N.Y.

Advertisement

\protect\hyperlink{after-bottom}{Continue reading the main story}

\hypertarget{site-index}{%
\subsection{Site Index}\label{site-index}}

\hypertarget{site-information-navigation}{%
\subsection{Site Information
Navigation}\label{site-information-navigation}}

\begin{itemize}
\tightlist
\item
  \href{https://help.nytimes3xbfgragh.onion/hc/en-us/articles/115014792127-Copyright-notice}{©~2020~The
  New York Times Company}
\end{itemize}

\begin{itemize}
\tightlist
\item
  \href{https://www.nytco.com/}{NYTCo}
\item
  \href{https://help.nytimes3xbfgragh.onion/hc/en-us/articles/115015385887-Contact-Us}{Contact
  Us}
\item
  \href{https://www.nytco.com/careers/}{Work with us}
\item
  \href{https://nytmediakit.com/}{Advertise}
\item
  \href{http://www.tbrandstudio.com/}{T Brand Studio}
\item
  \href{https://www.nytimes3xbfgragh.onion/privacy/cookie-policy\#how-do-i-manage-trackers}{Your
  Ad Choices}
\item
  \href{https://www.nytimes3xbfgragh.onion/privacy}{Privacy}
\item
  \href{https://help.nytimes3xbfgragh.onion/hc/en-us/articles/115014893428-Terms-of-service}{Terms
  of Service}
\item
  \href{https://help.nytimes3xbfgragh.onion/hc/en-us/articles/115014893968-Terms-of-sale}{Terms
  of Sale}
\item
  \href{https://spiderbites.nytimes3xbfgragh.onion}{Site Map}
\item
  \href{https://help.nytimes3xbfgragh.onion/hc/en-us}{Help}
\item
  \href{https://www.nytimes3xbfgragh.onion/subscription?campaignId=37WXW}{Subscriptions}
\end{itemize}
