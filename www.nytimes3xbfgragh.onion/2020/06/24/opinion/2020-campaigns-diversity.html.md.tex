Sections

SEARCH

\protect\hyperlink{site-content}{Skip to
content}\protect\hyperlink{site-index}{Skip to site index}

\href{https://myaccount.nytimes3xbfgragh.onion/auth/login?response_type=cookie\&client_id=vi}{}

\href{https://www.nytimes3xbfgragh.onion/section/todayspaper}{Today's
Paper}

\href{/section/opinion}{Opinion}\textbar{}The People Running Campaigns
Don't Look Like Me

\url{https://nyti.ms/2VfLF5W}

\begin{itemize}
\item
\item
\item
\item
\item
\end{itemize}

Advertisement

\protect\hyperlink{after-top}{Continue reading the main story}

\href{/section/opinion}{Opinion}

Supported by

\protect\hyperlink{after-sponsor}{Continue reading the main story}

\hypertarget{the-people-running-campaigns-dont-look-like-me}{%
\section{The People Running Campaigns Don't Look Like
Me}\label{the-people-running-campaigns-dont-look-like-me}}

Having a seat at the table is good, but people of color need to be able
to set the agenda too.

By Chuck Rocha

Mr. Rocha is a former senior adviser to Senator Bernie Sanders.

\begin{itemize}
\item
  June 24, 2020
\item
  \begin{itemize}
  \item
  \item
  \item
  \item
  \item
  \end{itemize}
\end{itemize}

\includegraphics{https://static01.graylady3jvrrxbe.onion/images/2020/06/24/opinion/24rocha1/merlin_170349615_13d79968-0ac5-4e3e-844d-5dea7c6f869a-articleLarge.jpg?quality=75\&auto=webp\&disable=upscale}

I have worked in politics for over 30 years. For most of those years,
there have rarely been black or brown faces at the top level of most
campaigns I've worked on. I am often the lone voice telling a room full
of campaign professionals what it's like to be pulled over by the police
when you're a brown man.

People of color make up
\href{https://suppliertynews.com/2017/07/27/the-national-association-of-diverse-consultants-looks-to-increase-diversity-in-political-consulting/}{less
than 1 percent of political consultants}. The real decision makers in
campaigns are too often a small group of white male consultants and
senior advisers, who then go on to be informal advisers to elected
officials after the campaign is over.

The problem is not restricted to campaigns. A lack of diversity is
glaringly apparent in the halls of Congress. In the most racially
diverse House in history, just over
\href{https://www.nytimes3xbfgragh.onion/2019/03/15/us/house-staff-minorities-democrats.html}{20
percent} of the congressional staff is of color.

The nation's growing diversity is also not reflected in state
legislatures. Black people, who make up roughly 13 percent of the
country's population, account for only
\href{https://www.ncsl.org/research/about-state-legislatures/who-we-elect.aspx}{9
percen}t of all state legislators. Latinos, who are about 18 percent of
the population, only
\href{https://www.ncsl.org/research/about-state-legislatures/who-we-elect.aspx}{account
for 5 percent of state lawmakers}.

And yet we expect politicians and candidates to be on the right side of
criminal justice reform, policing, cash bail and ex-felon voting
restoration. You see the problem here?

The lack of diversity among our elected officials and the top aides who
help them win office impairs their ability to understand the diverse
perspectives in their districts. If we are more intentional about the
way that we ensure diversity in political campaigns, public offices and
the rooms where decisions are made, it will transform the way that
political leaders show up during moments of crisis. It is also how we
can effect change that is inclusive and meaningful.

Many members of the House, Senate and state legislatures are up for
election every two to six years. Who's in the room advising them? Who's
advising the candidate running to become sheriff? Who's advising the
district attorney candidate?

Campaigns and legislative offices have been built around the racist
infrastructure that was created centuries ago when only an elite group
of white, male, Protestant landowners had access to power. It's
reflected in the challenges to breaking through the hiring hierarchy if
you're a person of color or poor.

The entry point for most of these jobs is at the intern or staff
assistant level, jobs that are either unpaid or don't pay a living wage.
For young people of color who graduate from college with large debts,
this rigged system keeps them out of opportunity after opportunity.

In most campaigns, Latino and black staff members are often relegated to
minor positions and never given any power over budgets, strategy,
staffing or hiring. They are merely assigned to talk to other black or
brown people.

I never went to college, and I have a
\href{https://www.dol.gov/olms/regs/compliance/enforce_2013.htm\#:~:text=On\%20July\%2030\%2C\%202013\%2C\%20in,pay\%20a\%20fine\%20of\%20\%242\%2C000.}{criminal
record}. I also founded one of the largest Latino-owned and -operated
political consulting firms in the nation. Most recently, I served as a
senior adviser to Senator Bernie Sanders's 2020 presidential campaign.
Senator Sanders gave me a say over budgets, hiring and many aspects of
the operation. At our headquarters, interns were paid \$20 an hour with
full medical benefits.

The campaign ensured pay equity across all departments regardless of
race or gender. Latinos were integrated into every aspect of the
campaign. They intimately understood that Latinos in California do not
have the same problems and concerns as Latinos in Iowa or Nevada, and we
adjusted our outreach accordingly --- a first in the history of
presidential campaigns and a prime example of empowering people of color
intentionally. As a result, Senator Sanders won Iowa's four
\href{https://thehill.com/latino/482030-analysis-sanders-ran-the-table-with-latinos-in-iowa}{Spanish-language
satellite caucuses} and
\href{https://www.dailykos.com/stories/2020/2/23/1921481/-UCLA-data-on-Nevada-more-than-70-of-Latinos-voted-for-Bernie-Sanders}{over
70 percent} of the Latino vote in Nevada's primary.

It's a case study in diversity, equity and inclusion, sure; but it's
also about what happens when you put people of color in leadership
positions and empower them. It went beyond just giving me a seat at the
table --- we were all part of the conversation and often leaders of the
meetings. But this type of change is not happening in enough rooms where
political decisions are being made.

Reforming the criminal justice system and our government as a whole has
to happen through public policy at the local and federal levels. We can
change the power dynamics in our political system, but to do that we
must first address the role race plays in that system and in public
policy.

When I advise campaigns, I'm looking through a lens of cultural
understanding and life experience that is undeniably necessary. It's a
perspective that's often lost when the room is accessible only to people
with a certain pedigree and connections. Having a seat at the table is
good, but we need to be able to set the agenda, too.

Chuck Rocha
(\href{https://twitter.com/ChuckRocha?ref_src=twsrc\%5Egoogle\%7Ctwcamp\%5Eserp\%7Ctwgr\%5Eauthor}{@ChuckRocha})
is the founder and president of
\href{https://www.solidaritystrategies.com/}{Solidarity Strategies} and
a former senior adviser to Senator Bernie Sanders.

\emph{The Times is committed to publishing}
\href{https://www.nytimes3xbfgragh.onion/2019/01/31/opinion/letters/letters-to-editor-new-york-times-women.html}{\emph{a
diversity of letters}} \emph{to the editor. We'd like to hear what you
think about this or any of our articles. Here are some}
\href{https://help.nytimes3xbfgragh.onion/hc/en-us/articles/115014925288-How-to-submit-a-letter-to-the-editor}{\emph{tips}}\emph{.
And here's our email:}
\href{mailto:letters@NYTimes.com}{\emph{letters@NYTimes.com}}\emph{.}

\emph{Follow The New York Times Opinion section on}
\href{https://www.facebookcorewwwi.onion/nytopinion}{\emph{Facebook}}\emph{,}
\href{http://twitter.com/NYTOpinion}{\emph{Twitter (@NYTopinion)}}
\emph{and}
\href{https://www.instagram.com/nytopinion/}{\emph{Instagram}}\emph{.}

Advertisement

\protect\hyperlink{after-bottom}{Continue reading the main story}

\hypertarget{site-index}{%
\subsection{Site Index}\label{site-index}}

\hypertarget{site-information-navigation}{%
\subsection{Site Information
Navigation}\label{site-information-navigation}}

\begin{itemize}
\tightlist
\item
  \href{https://help.nytimes3xbfgragh.onion/hc/en-us/articles/115014792127-Copyright-notice}{©~2020~The
  New York Times Company}
\end{itemize}

\begin{itemize}
\tightlist
\item
  \href{https://www.nytco.com/}{NYTCo}
\item
  \href{https://help.nytimes3xbfgragh.onion/hc/en-us/articles/115015385887-Contact-Us}{Contact
  Us}
\item
  \href{https://www.nytco.com/careers/}{Work with us}
\item
  \href{https://nytmediakit.com/}{Advertise}
\item
  \href{http://www.tbrandstudio.com/}{T Brand Studio}
\item
  \href{https://www.nytimes3xbfgragh.onion/privacy/cookie-policy\#how-do-i-manage-trackers}{Your
  Ad Choices}
\item
  \href{https://www.nytimes3xbfgragh.onion/privacy}{Privacy}
\item
  \href{https://help.nytimes3xbfgragh.onion/hc/en-us/articles/115014893428-Terms-of-service}{Terms
  of Service}
\item
  \href{https://help.nytimes3xbfgragh.onion/hc/en-us/articles/115014893968-Terms-of-sale}{Terms
  of Sale}
\item
  \href{https://spiderbites.nytimes3xbfgragh.onion}{Site Map}
\item
  \href{https://help.nytimes3xbfgragh.onion/hc/en-us}{Help}
\item
  \href{https://www.nytimes3xbfgragh.onion/subscription?campaignId=37WXW}{Subscriptions}
\end{itemize}
