\href{/section/travel}{Travel}\textbar{}The Haunting Beauty of a
Hut-to-Hut Hike in the Dolomites

\url{https://nyti.ms/380zlM3}

\begin{itemize}
\item
\item
\item
\item
\item
\item
\end{itemize}

\includegraphics{https://static01.graylady3jvrrxbe.onion/images/2020/06/24/travel/24travel-dolomites-4/merlin_173732964_ba1a1544-78b1-4fec-aae0-19d0a448407a-articleLarge.jpg?quality=75\&auto=webp\&disable=upscale}

Sections

\protect\hyperlink{site-content}{Skip to
content}\protect\hyperlink{site-index}{Skip to site index}

The World Through a Lens

\hypertarget{the-haunting-beauty-of-a-hut-to-hut-hike-in-the-dolomites}{%
\section{The Haunting Beauty of a Hut-to-Hut Hike in the
Dolomites}\label{the-haunting-beauty-of-a-hut-to-hut-hike-in-the-dolomites}}

With their colossal limestone walls and gloriously green valleys,
Italy's Dolomites are home to some of the world's most majestic scenery
--- and mountain huts called rifugios make it all the more accessible.

Credit...

Supported by

\protect\hyperlink{after-sponsor}{Continue reading the main story}

Photographs and Text by Mónica R. Goya

\begin{itemize}
\item
  June 24, 2020
\item
  \begin{itemize}
  \item
  \item
  \item
  \item
  \item
  \item
  \end{itemize}
\end{itemize}

\emph{With travel restrictions in place worldwide, we've launched a new
series,}
\href{https://www.nytimes3xbfgragh.onion/column/the-world-through-a-lens}{\emph{The
World Through a Lens}}\emph{, in which photojournalists help transport
you, virtually, to some of our planet's most beautiful and intriguing
places. This week, Mónica R. Goya shares a series of photographs taken
on an extended hike through the Dolomites.}

\begin{center}\rule{0.5\linewidth}{\linethickness}\end{center}

Last August, long before the coronavirus pandemic
\href{https://www.nytimes3xbfgragh.onion/2020/03/21/world/europe/italy-coronavirus-center-lessons.html}{descended
across Italy}, I set off on a hike following the
\href{https://www.alpineexploratory.com/walking-guides/alta-via-1.html}{Alta
Via 1}, a long-distance footpath that traverses the Dolomites from north
to south.

A monumental mountain range in northeastern Italy,
\href{https://www.nytimes3xbfgragh.onion/2018/08/30/travel/what-to-do-in-the-dolomites.html}{the
Dolomites} --- a World Heritage Site since 2009 --- are home to some of
the world's most majestic scenery: colossal vertical limestone walls,
gloriously green valleys. There are several Alta Via routes, but the
AV1, with fewer exposed sections, is ideal for less experienced hikers.

\includegraphics{https://static01.graylady3jvrrxbe.onion/images/2020/06/24/travel/24travel-dolomites-18/24travel-dolomites-18-articleLarge.jpg?quality=75\&auto=webp\&disable=upscale}

Image

The Dolomites are also known as the Pale Mountains because of their
unique colors.

The trail runs south from Lago di Braies, a chilly Alpine lake in South
Tyrol, to Belluno, a town in Italy's Veneto region. The first few miles
include both a ferocious ascent up a slope covered in scree and broad
views of a vast plateau --- a fitting preview of the striking contrasts
to come.

Image

Daisy, a farm dog, followed us along one of the sections of the trail.

Image

Religious markers are a common sight along the trail.

The trail's northern terminus lies less than 20 miles from the Austrian
border, and many villages in its vicinity have both an Italian and an
Austrian name --- a reminder of the
\href{https://www.nytimes3xbfgragh.onion/2014/03/25/world/europe/italys-historic-multicultural-compromise.html}{region's
linguistic peculiarities}. (In addition to speaking Italian and German,
many residents of the Dolomites also speak a language called
\href{https://www.altabadia.org/en/italian-alps-dolomites/about-alta-badia/the-ladin-language-and-culture.html}{Ladin}.)

Over the course of my nine-day hike, the trail --- mostly well marked
--- snaked its way up jagged bare peaks in picturesque formations:
pinnacles, spires, towers. It also wound through lush Alpine grazing
lands and valley floors carpeted with pine and fir trees. Largely
because of the beauty of the pale dolomitic limestone, panoramic vistas
were a constant.

Image

The trail near Cima Ambrizzola.

Idyllic mountain huts, called rifugios, are spaced at day-hike intervals
along the trail; there are about 30 altogether. (The 75-mile trek
typically takes about 10 days to complete.) The trail reaches a maximum
elevation of over 9,000 feet and includes a total elevation gain of more
than 20,000 feet --- which means that arriving early at the rifugios and
catching up on rest often feels more like a necessity than a luxury.

Image

An early morning outside Rifugio Coldai.

Image

A deck at Rifugio Lagazuoi offers panoramic views over some of the most
famous mountains in the Dolomites, including Pelmo and Civetta.

Once, while traversing a stretch of trail on my way to a stunning rock
formation called the
\href{https://www.lonelyplanet.com/italy/cinque-torri}{Cinque Torri}, I
found myself enraptured by the lofty views of
\href{https://goo.gl/maps/CXjADMPCPTcKYEKM8}{Lago di Lagazuoi}, a small
mountain lake. But my wonderment didn't last long: Soon after I sat
down, apple in hand, the skies went dark with storm clouds.

Image

Lago di Lagazuoi is surrounded by the imposing walls of the Cima del
Lago, the Cima Scotoni and Fànis.

\href{https://www.rifugiolagazuoi.com/index_en.php}{Rifugio Lagazuoi},
my destination for the night, was visible in the distance and appeared
close at hand --- less than two miles away as the crow flies. But,
finding it separated from me by a very steep descent on switchback
paths, plus one last backbreaking ascent, I panicked slightly, realizing
there was no way to reach shelter before the storm would break. I pulled
out my rain gear and soldiered on.

Image

A steep descent.

Image

A flock of sheep grazing in the high pastures near Forcella Ambrizzola.

This mountainous heart of Europe, its trails now evoking sublime
grandeur, was once the scene of one of the most
\href{https://www.smithsonianmag.com/history/most-treacherous-battle-world-war-i-italian-mountains-180959076/}{treacherous
battles of World War I} --- which is now commemorated at the
\href{https://lagazuoi.it/EN/Discover-History-page4-The-Open-Air-Museum-of-Mt-Lagazuoi}{Open
Air Museum of Mount Lagazuoi}. Andrea, a re-enactor dressed in a
historical Tyrolean Rifle Regiment uniform, led us on a guided tour
through various trenches and tunnels, describing how the Italian and
Austro-Hungarian armies had turned the mountain into a fortress.

Image

Wartime tunnels are preserved at the Open Air Museum of Mount Lagazuoi.

Down in \href{http://www.rifugiocittadifiume.it/?lang=en}{Rifugio Città
di Fiume}, and back at tree-line level, after leaving behind the
picture-perfect Alpine meadows of Cinque Torri, with its scattered
sheep, cattle and marmots, the air was heavy with the refreshing scent
of pine trees. There, gazing at the dramatic peak of Monte Civetta, I
first experienced what a local hiker called ``enrosadira,'' an exquisite
glow that happens at sunrise and sunset, when the dolomitic limestone is
bathed in gorgeous peachy-pink hues.

Image

Mount Civetta at sunrise, as seen from Rifugio Città di Fiume.

Rifugios come in all shapes and forms, from spartan rustic buildings
with cracking wooden floors to charming Alpine mountain lodges. But
there are common threads among them --- calorie-dense dinners (and
breakfasts), affable service, the chance to experience camaraderie with
fellow backpackers from around the world. Facilities are basic, but most
of them have a drying room and a coin-operated hot shower --- which runs
for two or three minutes, to prevent waste. And, yes, there's Wi-Fi.

The rifugios are normally open from June to September --- and they
remain open this year, in spite of the coronavirus. But, since some are
now operating at reduced capacity, advanced booking is mandatory. New
regulations also require visitors to bring their own sleeping bags,
slippers and masks. (In normal circumstances, only a sleeping bag liner
is required, as blankets are provided.) And be prepared to have your
temperature taken before checking in; hut wardens can deny access if
your temperature is too high.

Image

An Alpine meadows near Cinque Torri, with the iconic Croda da Lago
mountain chain visible in the background.

Nearly 150 years have passed since
\href{https://www.nytimes3xbfgragh.onion/2019/11/11/travel/Egypt-Nile-cruise-women.html}{Amelia
Edwards}, an accomplished English journalist,
\href{https://www.google.com/books/edition/Untrodden_Peaks_and_Unfrequented_Valleys/ECFZAAAAcAAJ?hl=en\&gbpv=1\&dq=amelia\%20edwards\%20untrodden\&pg=PA20\&printsec=frontcover}{wrote
about} being haunted by the Dolomites' ``strange outlines and still
stranger colouring.'' Much has changed since then --- but much has
endured, too.

The rhythm of a long-distance trek here --- the exhaustion, the
challenging simplicity of the routine --- washes away mundane worries.
Visitors are dwarfed by the ever-changing and imposing surroundings.
And, all these years later, the splendor of these unique mountains still
enchants, and haunts, those who take to its paths.

\begin{center}\rule{0.5\linewidth}{\linethickness}\end{center}

\href{http://www.monicargoya.com/}{\emph{Mónica R. Goya}} \emph{is a
London-based journalist and photographer. You can follow her work on}
\href{https://www.instagram.com/monicargoya/}{\emph{Instagram}}\emph{.}

\emph{\textbf{Follow New York Times Travel}} \emph{on}
\href{https://www.instagram.com/nytimestravel/}{\emph{Instagram}}\emph{,}
\href{https://twitter.com/nytimestravel}{\emph{Twitter}} \emph{and}
\href{https://www.facebookcorewwwi.onion/nytimestravel/}{\emph{Facebook}}\emph{.
And}
\href{https://www.nytimes3xbfgragh.onion/newsletters/traveldispatch}{\emph{sign
up for our weekly Travel Dispatch newsletter}} \emph{to receive expert
tips on traveling smarter and inspiration for your next vacation.}

Advertisement

\protect\hyperlink{after-bottom}{Continue reading the main story}

\hypertarget{site-index}{%
\subsection{Site Index}\label{site-index}}

\hypertarget{site-information-navigation}{%
\subsection{Site Information
Navigation}\label{site-information-navigation}}

\begin{itemize}
\tightlist
\item
  \href{https://help.nytimes3xbfgragh.onion/hc/en-us/articles/115014792127-Copyright-notice}{©~2020~The
  New York Times Company}
\end{itemize}

\begin{itemize}
\tightlist
\item
  \href{https://www.nytco.com/}{NYTCo}
\item
  \href{https://help.nytimes3xbfgragh.onion/hc/en-us/articles/115015385887-Contact-Us}{Contact
  Us}
\item
  \href{https://www.nytco.com/careers/}{Work with us}
\item
  \href{https://nytmediakit.com/}{Advertise}
\item
  \href{http://www.tbrandstudio.com/}{T Brand Studio}
\item
  \href{https://www.nytimes3xbfgragh.onion/privacy/cookie-policy\#how-do-i-manage-trackers}{Your
  Ad Choices}
\item
  \href{https://www.nytimes3xbfgragh.onion/privacy}{Privacy}
\item
  \href{https://help.nytimes3xbfgragh.onion/hc/en-us/articles/115014893428-Terms-of-service}{Terms
  of Service}
\item
  \href{https://help.nytimes3xbfgragh.onion/hc/en-us/articles/115014893968-Terms-of-sale}{Terms
  of Sale}
\item
  \href{https://spiderbites.nytimes3xbfgragh.onion}{Site Map}
\item
  \href{https://help.nytimes3xbfgragh.onion/hc/en-us}{Help}
\item
  \href{https://www.nytimes3xbfgragh.onion/subscription?campaignId=37WXW}{Subscriptions}
\end{itemize}
