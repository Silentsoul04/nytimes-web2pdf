Sections

SEARCH

\protect\hyperlink{site-content}{Skip to
content}\protect\hyperlink{site-index}{Skip to site index}

\href{https://myaccount.nytimes3xbfgragh.onion/auth/login?response_type=cookie\&client_id=vi}{}

\href{https://www.nytimes3xbfgragh.onion/section/todayspaper}{Today's
Paper}

\href{/section/opinion}{Opinion}\textbar{}Place Your Bets on Biden's
V.P.

\url{https://nyti.ms/31gTj46}

\begin{itemize}
\item
\item
\item
\item
\item
\end{itemize}

Advertisement

\protect\hyperlink{after-top}{Continue reading the main story}

transcript

Back to The Argument

bars

0:00/0:00

-0:00

transcript

\hypertarget{place-your-bets-on-bidens-vp}{%
\subsection{Place Your Bets on Biden's
V.P.}\label{place-your-bets-on-bidens-vp}}

\hypertarget{with-frank-bruni-ross-douthat-and-michelle-goldberg}{%
\subsubsection{With Frank Bruni, Ross Douthat and Michelle
Goldberg}\label{with-frank-bruni-ross-douthat-and-michelle-goldberg}}

\hypertarget{plus-a-supreme-court-mystery}{%
\paragraph{Plus, a Supreme Court
mystery.}\label{plus-a-supreme-court-mystery}}

Thursday, June 25th, 2020

\begin{itemize}
\item
  michelle goldberg\\
  I'm Michelle Goldberg.
\item
  frank bruni\\
  I'm Frank Bruni.
\item
  ross douthat\\
  I'm Ross Douthat. And this is ``The Argument.'' {[}THEME MUSIC{]}

  Today on the show, it's Veep stakes. {[}BELL DINGING{]} Where Michelle
  and Frank go to bat for their top Biden V.P. picks and I get to watch,
  or listen. And then we're going to talk about some weird stuff going
  on at the Supreme Court. It's supposed to be a coequal branch of
  government, so why did the judges keep taking over for Congress?
\item
  amy klobuchar\\
  I think the right thing to do right now --- and I told this to Vice
  President Biden --- is to put a woman of color on the ticket as the
  next vice president of our country. All I know is that Joe Biden needs
  to win. And I think---
\end{itemize}

ross douthat

Last week, Minnesota Senator Amy Klobuchar withdrew her name from Joe
Biden's list of possible VP picks and urged the Democratic nominee to
choose a woman of color as his running mate. Biden has already pledged
to pick a woman, but he hasn't gotten more specific. And the field of
possibilities remains pretty big. Kamala Harris, Val Demings, Gretchen
Whitmer, Susan Rice, Elizabeth Warren--- the list goes on. So who should
he pick? Who is the person most likely to help him win the presidency?
To answer this question, we'll pit my co-hosts against each other, and
then maybe I'll weigh in. Michelle, let's start with you. Who's your
pick and why?

michelle goldberg

So it won't surprise listeners of this show that I'm hoping that Joe
Biden picks Elizabeth Warren, who --- full disclosure, my husband
consulted on Warren's presidential campaign. And I'm hoping that he
picks her for both electoral reasons and substantive governing reasons.
She is the favorite of progressives. She's the favorite of young people
in polling. She's the favorite of young Black people in polling. She's
the favorite of a lot of the parts of the Democratic Party that felt
most disappointed with how things played out and with Joe Biden getting
the nomination --- people whose enthusiasm is going to be really, really
important in November. And so it's going to be important to get young
people excited and enthused. And Warren gives them something to be
excited about. Even if you--- if you look on Twitter, you might think
that the Bernie Sanders movement hates Elizabeth Warren, because there's
still a lot of bad blood among some of his high profile surrogates. But
in polls, Bernie supporters want Warren to be the vice president, as do,
like I said, other groups of the progressive base. But beyond that,
Biden is coming into office inheriting these historic crises, these
historic overlapping crises. And he's going to need a governing partner.
And he's talking about kind of needing a New Deal-sized presidency, kind
of New Deal-sized programs to address all of these overlapping
catastrophes. And just nobody has thought more clearly about how you do
that and how you do that within the constraints of a gridlocked
government than Elizabeth Warren. And so I just think she vastly
increases the chances that Joe Biden's presidency would be successful.

ross douthat

Alright, Frank. Tell us why Michelle is wrong and the perfect candidate
is somebody entirely different.

frank bruni

{[}CHUCKLES{]}

I would never say flatly that Michelle is wrong. I would say I have a
difference of opinion with her here. She talks in a compelling way about
Elizabeth Warren as a governing pick, is one of the phrases that's used
for that kind of thing. I think only one thing matters in 2020, and it
is making sure there are not four more years for Donald Trump. So I'm
thinking of this completely and totally in strategic terms. And while
Elizabeth Warren is someone who has some promise in turning out
progressives, there's the question of whether you lose a bunch of White
working class voters, especially in key states, for every voter she
brings in. My pick--- I think a better one in those regards --- is
Senator Tammy Duckworth of Illinois. I think of several things, but
right now I'll just focus on a few. She has an amazing, amazing,
inspiring personal story that I think will resonate with voters at a
time when people really want to feel good about politicians and politics
in a way they haven't in a long time. She's Asian-American ---
Thai-American, to be specific. She is an Iraq war veteran, who lost both
of her legs during her combat service as a helicopter pilot and speaks
of this with such humility and such grace. She is a mother of two
children under the age of six. You want to talk about relatable with
voters? She and her husband are an economically humble couple. She is
someone also who I think would be impossible for Trump and the
Republicans to campaign against. Trump is only comfortable when he is
mocking and insulting people. And his relationship with Elizabeth Warren
--- and we're not going to get into the nickname that he calls her ---
is a great example of that. He will be comfortable campaigning against a
ticket that includes Elizabeth Warren. He won't know what to do with
Tammy Duckworth, who has a Purple Heart, who interestingly coined the
term ``Cadet Bone Spurs'' for him. So he's going to stand there, sit
there, whatever, tongue tied about how to talk about this half of the
Democratic ticket while she brings the case to him in a very, very
winning way. I think she'd be an extraordinary pick.

michelle goldberg

I love Tammy Duckworth. So I'm not going to argue against her. I would
argue against the idea that we can sort of project into the minds of
White working class voters what will and won't resonate with them. And I
think that in the past, when particularly people like us have tried to
do that, we have gone really wide of the mark, right? I remember when
Democrats decided that there was no way George W. Bush, a draft dodger,
was going to be able to campaign effectively against war hero John
Kerry--- they found a way. I remember when people thought that it would
be very hard to smear Max Cleland, the former senator, a triple amputee,
as some sort of terrorist apologist: They did that. And so I'm reluctant
to think about this choice as sort of four-dimensional chess in terms of
what sort of attack ads they will be able to gin up, right? There is no
bottom to what they will say about any candidate. And obviously, Tammy
Duckworth's service to her country is going to mean precisely nothing to
the people who are making attack ads for Donald Trump. I also think
there's something to be said for a candidate who's already run for
president, so we have some idea of how they perform under that
spotlight. And again, I don't doubt that Tammy Duckworth is up to the
job. And if he picks her, it seems to me like a great idea. I accept
your argument that defeating Trump is so important that electoral
concerns should trump those. Joe Biden is there to appeal to White
people in the Midwest and White people in swing states, right? That's
substantially why a lot of Democratic primary voters picked him. And so
we obviously have to worry about those people, but where he has real
weakness, where he really needs to balance the ticket is among young
progressives.

frank bruni

Let me --- let me just quickly push back on two things ---

ross douthat

Yes, do it. Do it.

frank bruni

---in a collegial and friendly way. One is, yes, Elizabeth Warren has
already run for president. And it didn't turn out so well. Now, I have
enormous respect for her. And I think she has given more thought to the
problems that face this nation and how to solve them than probably
anyone we're going to discuss today. But at the end of the day, she was
not able to translate that into the kind of excitement among voters that
would validate her choice as his running mate. She came in third in the
Democratic primary in Massachusetts.

michelle goldberg

Yeah. No, you're right. Although if you look at the polling of people
being asked, if you could just choose who was going to be president as
opposed to who was going to be the nominee, who would you pick, then
Elizabeth Warren does a lot better, right? So she suffered a lot from
people who liked her the best but have been traumatized by 2016 ---
didn't believe that a woman could beat Donald Trump; wanted the safest
possible candidate. She's still, in polls, the most popular with
Democrats of all the people who are being considered for vice president.

frank bruni

One thing we haven't mentioned that I think we have to about Elizabeth
Warren before we move on--- she's 70. Joe Biden is 77. Joe Biden would
be 78 on Inauguration Day, the oldest president ever. Well, Elizabeth
Warren has actually shown quite a bit of appeal to younger voters---
voters of color, that's a whole other story. There is something to me
un-ideal and potentially not a great idea about having a Democratic
ticket, in particular, of two people in their 70s. It goes against the
Democratic Party's brand. It goes against when they've had the most
success in presidential elections, with candidates like Bill Clinton,
with candidates like Barack Obama, who were a lot younger. It feels to
me like a real risk to put out a Democratic ticket --- a ticket for the
party that's about youth, and optimism, and change --- that has a
70-year-old and a 77-year-old on it.

ross douthat

Working off Frank's last point, that sort of the image of a ticket of
two older candidates isn't ideal for the Democrats, both of you are
suggesting candidates who have many virtues but are not
African-American. And I think while Amy Klobuchar in her comment did say
``woman of color'' and Tammy Duckworth is Asian-American, I think the
assumption is that she was suggesting that, given the Black Lives Matter
moment--- the protests, everything that's going on within America and
within liberalism--- that there is some kind of an obligation for Joe
Biden to pick an African-American running mate. And I think you can see
that in the rumors around his campaign right Now. There's a lot of talk
about Kamala Harris, Val Demings, and Susan Rice as frontrunners. So
what do you guys think of that challenge to each of your preferred
candidates?

michelle goldberg

To me, that's the most significant challenge. There was a lot to be said
for him having a Black woman running mate before this moment. And now,
with everything that's going on, it seems even more urgent. I'm not sure
how strong the electoral case is for that. There's a piece in
FiveThirtyEight about the arguments that people are making on behalf of
various vice presidential candidates that's a little bit skeptical about
the electoral arguments--- that having a Black woman on the ticket would
boost Black turnout. The data seems to be mixed about that. Maybe more
significant is the representational argument--- just that it's time. But
all of these women are obviously not interchangeable. And so I think
rather than just make a case for a Black woman, there's cases to be made
for specific Black women, right? And so each of these candidates has
things that make them formidable and things that make them less so,
right? I'm a huge, huge, huge Stacey Abrams fan and yet really worry
about having someone on the ticket who I think won the Georgia
governor's race and was cheated out of it, but the fact remains that she
didn't serve as governor, doesn't have executive experience, and is
stepping into a role when, because of Joe Biden's age, she could end up
being called on to be president almost immediately. So in that respect,
I see more of an argument for Kamala Harris. But then I really like
Kamala Harris. And like Tammy Duckworth, I would be delighted if Kamala
Harris is on the ticket. But think about one of her big weak spots when
she was running for president. People were saying Kamala was a cop. And
she was being --- she was attacked a lot on her criminal justice
background. So I think it's possible to see that in this environment as
being a real liability.

ross douthat

What about the argument, which I think is closer to Michelle's take
here, that vice presidential picks, for all that we obsess over them,
rarely actually have a strong effect on the election --- that how Joe
Biden performs in the fall debates, for instance, is going to be much
more important than who he picks as his running mate? And given Biden's
advanced age and I think a little bit at least of deterioration in his
public speechifying, isn't this a vice presidential pick that's closer
to picking another president, or a co-president, or a future president
than most picks? Do you guys think that's fair?

michelle goldberg

I think that's fair. Although even if vice presidential picks don't have
a large effect, right, this election will likely be one around the
margins. And because there's some question as to how Joe Biden is sort
of identifying and positioning himself --- it's his very interesting
phenomenon where he sort of ran to the center during the primary and has
moved left since winning the nomination, which is just something that
doesn't usually happen --- picking Elizabeth Warren would signal to
people that that move is sincere and that when he talks about governing
in a kind of New Deal fashion that he's serious. But yeah, I do agree
with you that whoever he picks is extremely likely to be the candidate
in 2024.

frank bruni

I think the fact that Biden could be a one-term president has changed
the Veep stakes profoundly to the extent that you see people auditioning
for this job, for lack of a different term, more aggressively than I can
remember in previous presidential cycles. And I think that's because all
of those women under consideration understand that, because of Joe
Biden's age, whoever is his vice president might be the Democratic
nominee in four years rather than eight years, might quickly become the
de facto leader of the party. And that I think has made the job of vice
president triply, if not quadruply, appealing to the people who were in
contention for it. I don't think that voters, who tend to do this more
emotionally than in a hyper-intellectual fashion, are going to say, I've
got to look at this vice presidential pick in a different way than I did
in previous years because of his age because maybe this person is going
to be the Democratic candidate in 2024. I don't think they'll go through
all of those hoops. I agree that the vice presidential choice usually
matters way, way less than those of us who are in the business of
churning out copy ---

ross douthat

Right.

frank bruni

--- and opinions make it seem. But going back to what I view as the
stakes of this election, going back to the fact that I think it is do or
die for this country that Trump, Pence not get another four years, I
think you have to assume as you go into this pick that it could be the
difference. Because if at the margins, as Michelle said, it made the
barest difference and that barest difference was a win/loss difference,
well, you've got to plan for that.

ross douthat

So let me now just before we wrap up throw out my own sort of wildcard
pick, which is not someone he's actually going to pick because it's not
a woman. But I've been struck by the sort of --- the mood on the left
and sort of the centering of urban politics at the moment by what a
strong vice presidential candidate Cory Booker would make at this
moment. And I'm curious what you guys think about that. It seems to me
that Booker is someone who has --- he's African-American, obviously. He
was mayor of Newark, a successful mayor --- a city that has weathered
the current era of protest a lot better than a lot of other cities. He
has a sort of mixed appeal to left and center, with sort of records that
can play to either side. He didn't win the primary obviously, but I
think he came through it better liked by just about everybody than
Harris did. And so I'm wondering if, in deciding for understandable
reasons to say he was going to pick a woman, Joe Biden preemptively
deprived himself of the guy who would have been the right choice for the
political moment we've ended up living in right now. What do you guys
think of that?

michelle goldberg

I think there's something to that. I wrote a piece during the general
election saying, why not Cory Booker, right? That if voters wanted
somebody who was more centrist than Bernie or Warren, that Booker was
really the person they should look to, right? He has this history of
executive experience. He brings a lot to the table. But I think that
you're right that Biden has preemptively narrowed the field. That said,
I think it was a good call at the time when he was asked if he would
make this pledge and he did. And so--- and there's no shortage of really
compelling choices.

frank bruni

Ross, I think you're right that Cory Booker would have been a very, very
compelling choice. But I do think Biden did the right thing, not just at
the time, but forever more in terms of saying, you know what, I'm going
to choose a woman this time. I think putting a woman on the ticket makes
sense in all sorts of ways. I think it still leaves him with an enormous
range of choices, as you can see if you look at the list. Ross, I
thought you were going to ask us --- I thought you were going to make
Michelle and I really go out on a limb and ask us not just who we
wanted, which we've told you, but if we had to place a bet, who we think
Joe Biden will pick. But maybe I'll start by asking you that, and then
you can turn it around, huh?

ross douthat

I think he's going to pick Kamala. I think that she --- in spite of the
issues on criminal justice, I think the campaign's assumption will be
that she is the safest, most vetted African-American female candidate
and that she was always sort of a natural pick for him in certain
ways--- which is boring, which is why I brought up Cory Booker.
{[}LAUGHTER{]} So what---

michelle goldberg

Although Kama--- right. Although actually, a Kamala Harris nomination
would not be boring. It would be really exciting for a lot of people.
And there are a lot of people who love her. If you ever run afoul of the
K-Hive on Twitter, you will quickly learn that she has a pretty
fanatical base of her own.

frank bruni

{[}LAUGHS{]}

ross douthat

I only meant ``boring'' in the sense of being something that we could
see coming a long time ago. So all right. Both of you, who will he pick?

michelle goldberg

I think the same thing. If I had to bet, I would bet on Kamala.

frank bruni

And sadly for a show called ``The Argument,'' what we have here is the
unanimity or the consensus. I think he'll pick Kamala too. And honestly,
if I weren't recommending Tammy Duckworth, she would be my second choice
in terms of our personal recommendation. And I think in both of them
what's important to note--- and we haven't said this about some of the
other contenders --- is if you're trying to play a safe game and if
you're trying to make sure also that you're consistent with your own
argument that you don't want the nation to take a risk after having
taken such a risk on Trump --- and that is a big part of Joe Biden's
argument --- ``look at me, I'm tested, I'm conventionally competent,''
and all those things--- I think choosing someone who's in the Senate,
like Kamala or like Tammy Duckworth, makes more sense than choosing,
say, a mayor or someone who's never been above the status of state
lawmaker. I just think it is a more prudent and a safer way to go.

ross douthat

I think that's absolutely right, but it will be quite something at a
certain level if Joe Biden's running mate is someone who was arguably
beaten decisively in the Democratic primary because she was too far to
the right on criminal justice. Again, I don't think that will prevent
her from being picked, but it will be kind of a remarkable turn given
the kind of debates and arguments we've been having in the country and
that liberals certainly have been arguing for the last month. So we'll
leave it there. And we'll be right back to talk about the Supreme Court.

frank bruni

And we're back. I'm going to take the lead from here. And we will let
Michelle go get on with the rest of her day. In her place, we've invited
our colleague Jesse Wegman. Hi, Jesse.

jesse wegman

Hey, Frank. How are you?

frank bruni

Great. It's great to have you here. Jesse writes about the Supreme Court
and legal issues for The Times editorial board. He's also written a book
about abolishing the electoral college. And he's a big proponent of
killing the filibuster. In fact, don't get him started about Mitch
McConnell stealing a Supreme Court seat from Barack Obama. We asked
Jesse here to talk with us about SCOTUS. And we're going to start with
two recent big decisions --- about L.G.B.T. workplace rights and about
the DREAM Act, or D.A.C.A. A lot of people were surprised given the
Court's conservative majority, Jesse. Were you surprised by these
decisions?

jesse wegman

Not as much as I think most people were. I anticipated the L.G.B.T.
ruling; less so the immigration ruling. But I think it helps to go back
and refresh our memories about what these two cases are about to
understand why they maybe aren't as surprising or as predictive as some
people think they might be. The first ruling, on the L.G.B.T.Q. rights,
held that a federal civil rights law prohibits employers from firing
workers for being gay or transgender. The plaintiffs sued under Title 7
of the Civil Rights Act, which prohibits discrimination in the workplace
on the basis of, among other things, sex. So what does ``sex'' refer to?
The employers who had fired these workers argued, and several of the
conservative justices on the Court agreed, that it could only possibly
refer to men and women in their capacity as men and women. This is how
lawmakers thought about it in 1964. That's clear. And it couldn't be
thought of in any other way. But of course, that can't be true, because
over the years the Supreme Court has expanded Title 7 to cover all kinds
of situations that those original lawmakers wouldn't have thought of.
For example, male-on-male sexual harassment is covered under Title 7.
It's prohibited under Title 7. That certainly wasn't a thing the law
paid attention to in 1964. So the question here was, does sex refer to
sexual orientation and gender identity? Six justices said that it did.
And that opinion, the majority opinion for the Court, was written by not
some bleeding heart liberal, but by the rock-ribbed conservative Neil
Gorsuch. Gorsuch famously adheres to an interpretive approach known as
textualism, and that means, when you're reading a statute, the plain
words of the statute control. And as he pointed out, obviously you
cannot talk about sexual orientation or gender identity without talking
about sex. So there you have it.

frank bruni

Jesse, did that reasoning surprise you? Would you have ever predicted
that Neil Gorsuch would come up with that interpretation of, quote
unquote, textualism?

jesse wegman

Yes, I think that was --- that was actually from the beginning what
people thought was one of the more likely outcomes of this case, was
that Justice Gorsuch would join with the liberals. I think the small
surprise in this case was that Chief Justice John Roberts also joined.
He didn't write separately, so we don't know what his thinking was. But
he has been on the other side of gay rights cases since he's been on the
Court. Now, conservatives generally were apoplectic over this ruling,
saying it's the end of the conservative legal movement, if not of the
world itself. I think that's an overstatement in the extreme. This court
has had a majority of Republican-appointed justices for about a half a
century now. And it is more solidly conservative than it has been since
the 1930s. The liberals on the Court are older on average than the
conservatives. And really, when you look at this opinion, Justice
Gorsuch's decision was purely about a single word--- ``sex.'' It was not
about equality or dignity of gay and transgender people, as it would
have been had it been written by Justice Anthony Kennedy, who is no
longer on the Court, but who wrote all of the Court's gay rights
decisions before this one. In fact, it was actually Justice Kavanaugh,
one of the other conservative justices, who dissented from the ruling,
but ended with a gracious note, I would say, congratulating gay and
transgender people for winning this victory. He's the only one who
really said anything about gay and transgender dignity.

frank bruni

{[}LAUGHS{]}

jesse wegman

It was not justice. Gorsuch himself.

frank bruni

I--- {[}CHUCKLES{]} I'm laughing, Jesse, because, you don't want us to
have our rights, but you want to congratulate us and let us know how
nice you feel about us. Thank you, Justice Kavanaugh. Now, Jesse, you
mentioned conservative apoplexy--- a great phrase. It was intensified by
the other Supreme Court decision that we're going to talk about. Explain
that decision for our listeners.

jesse wegman

So that case involved an executive order which we refer to as DACA,
which is Deferred Action for Childhood Arrivals. And that's a fancy
administrative language to say that people who were brought to America
as children by undocumented immigrant parents, and therefore are not
citizens, but came here through no fault of their own and broke the law
through no fault of their own were allowed to stay without fear of being
immediately deported. There's about 650,000-700,000 of them. It's a lot
of people. And it very much angered, I would say, for lack of a better
term, the White revanchist base that Trump relies on for his electoral
success. He had promised to end it. He said he would wind it down. And
the problem is it's a very popular program. It has support, like, in the
80th percentile, somewhere around there. There's nothing in this country
that that many people agree about. But the DREAMers, as they're called
--- these young people who were brought here as children--- are an
extremely sympathetic group. And Donald Trump knows that. He has said it
himself. So he was in a tough position, right? He wants to satisfy his
angry White base, but he doesn't want the political blowback of shutting
down a popular program. So what do they do? They try to thread the
needle and say, well, we're not shutting it down. We're saying that
President Obama didn't have the authority to issue this order in the
first place --- he exceeded his legal authority. That was their entire
argument. And the Court ruled in an opinion by Chief Justice Roberts
that that wasn't enough. The administration had failed to provide a
reasoned explanation for its decision. Now, once again, as with the
L.G.B.T. case, this was not an emphatic victory that liberals might have
thought it was. It's good, obviously, for the Dreamers because they get
protected for a couple of more years. But essentially, it's a function
of how bad the Trump administration is at governing that Chief Justice
Roberts would say, you can't even do what you're trying to do right. And
so he gave them more opportunities to go back. They can--- technically,
the Trump administration can go back and come up with a better
explanation. Realistically, that's not going to happen before the
November elections.

frank bruni

So even if not emphatic victories for liberals, these were certainly
defeats, temporary or otherwise, for President Trump. But Ross, you
think these are defeats in a larger sense. You see something going on
here that's about the Court's trajectory and role over time that
concerns you greatly. Can you talk about that?

ross douthat

Sure. And first, Jesse, thanks so much for joining us. And I think
you're totally wrong--- {[}FRANK LAUGHS{]} --- about --- about the
conservative legal movement. I think the conservative reaction to this
decision was less apoplexy and more weary resignation.

jesse wegman

I don't know. Did you read Twitter? {[}CHUCKLES{]}

ross douthat

Well, if you're reading Twitter, it's all apoplexy all the time. I'm
just --- I'm not --- I'm talking about the top secret conversations and
the layers of legal conservatives. But ---

jesse wegman

Oh, sorry. Fair enough. I am not privy to this.

ross douthat

Not yet. We'll indoctrinate you later. {[}JESSE LAUGHS{]} But the
independent sort of specific theories of interpretation, the animating
force, the political force behind the conservative legal movement going
back 50 years really now has been the view that unelected judges ---
justices of the Supreme Court --- are taking it upon themselves to
effectively rule on and settle what we call culture war issues/social
issue debates rather than leave them to the Democratic process. So
really, this goes back well before Roe v. Wade. It goes all the way back
to the school prayer rulings in the 1940s and `50s. But obviously, it
continues with rulings on abortion, rulings on same sex marriage, all
the way down to the present day. And again and again, what's happened---
and it happened with Anthony Kennedy, and now it's happened again with
Neil Gorsuch --- is that the Republican appointees put in a position to
rule on contested culture war questions decide to set themselves up as
the arbiters of the controversy rather than leaving it to the democratic
process to work out. And so in that sense, given that pattern, it's not
particularly surprising that Neil Gorsuch came up with a totally absurd
texturalist explanation for why the word ``sex'' actually means ``sexual
orientation.'' It's part of a larger pattern, and one that extends, I
should say, beyond culture war cases. This is a big story that
encompasses all kinds of other debates --- sometimes debates where
liberals are more on the losing side --- where the Supreme Court has
basically increased its powers dramatically as Congress has ceased
legislating and has sort of deferred to the Supreme Court, which you saw
in this case. Republicans in the Senate were very happy to have this
decision because it means that they don't actually have to sit down and
work out the hard questions of, if you pass national anti-discrimination
protections, what are the implications for transgender athletes and
youth sports, what are the implications for religious institutions'
hirings. These are all the kinds of questions that, in a healthy
democracy, legislators would address. In our democracy, Gorsuch issues
this ruling and then says, well, I haven't addressed any of these other
questions, so why don't you file a bunch more lawsuits, and I'll tell
you who wins and loses in those. That's literally what the ruling comes
down to. There is a general sense among conservatives that the country
supports anti-discrimination protections for gays and lesbians. I think
the hard questions around this issue are the ones that Gorsuch didn't
take up, which is, what does it mean when those rights conflict with
religious liberty, what does it mean for single sex institutions.
There's a whole range of these questions. But the bottom line is where
the conservative legal movement has failed is in its quest to have a
Supreme Court that leaves culture war battles to legislatures, to
states, to Congress. That's the failure. It's not that there aren't
Republicans on the bench--- obviously there are. But the whole purpose,
the whole multi-decade push has just ended up once again with a
Republican-appointed justice saying, don't worry, kids, I'm going to
tell you how this culture war controversy is going to turn out.

frank bruni

Ross, I want to ask you this question, but then I want to get Jesse's
response to it too. You mentioned I think accurately that Congress has
ceased to legislate. We basically have a void there. And I would add to
that that Congress --- {[}CHUCKLES{]}

--- the current Congress, due to a whole bunch of forces we could spend
many shows talking about, doesn't do a great job often of reflecting
where the American public is, and what the American public has learned,
and how it's evolved. Given that void, is it such a terrible thing that
the Supreme Court has stepped into it, has stepped up? Why isn't this
evolution of the Court appropriate for this moment in time when Congress
is, as I said, a kind of void?

ross douthat

Well, it might be appropriate for the time in the sense that maybe it
was appropriate when Caesar Augustus did away with senatorial powers
because the Roman Senate had ceased to govern the Roman Republic
{[}FRANK LAUGHS{]} going on 150 years, right? That's --- that's --- I'm
being sort of deliberately overstated, but the historical reality is,
yes, power abhors a vacuum. If a republican institution doesn't fulfill
its duties, some other force or institution will take up that power. But
the design of our system is supposed to leave substantial powers in the
hands of an elected legislature. And so it is a bit of a problem for the
idea of the American republic if policy is increasingly negotiated
between the executive branch and the courts. And you can see that in the
other case we're talking about here, right, where the reason we don't
have a enduring settlement on DACA is that Congress didn't come up with
one; Barack Obama decided to pre-empt Congress, creating a debate about
whether his move was constitutional; then Donald Trump elected, tried to
undo what Obama did; and then the Court came up with somewhat convoluted
reasons to say, not so fast, let's postpone this through the next
election, basically. Again, it's a negotiation between justices
appointed by the president and the executive branch. And that's a
political system; it's just not the one that we're taught about in
civics class, if there are civics classes anymore.

frank bruni

Jesse, do you agree with Ross that the Court is usurping its bounds? And
do you share his concern about that if so?

jesse wegman

Well, first, let me just say, Ross, you mentioned that I had perhaps
overstated the conservative unhappiness. But you did invoke the fall of
Rome. So--- {[}LAUGHTER{]}

ross douthat

I have --- I have--- I have one-upped myself.

frank bruni

Touché.

ross douthat

Yep, touché. Touché.

jesse wegman

So I'll say this, I absolutely--- I wouldn't use the term ``usurp,'' but
I would --- I certainly agree with Ross that the court is stepping into
a vacuum that has been left by Congress. And to be fair, the executive
is also stepping into that vacuum.

ross douthat

Absolutely.

jesse wegman

Some of the results are not as concerning; some of the results are much
more concerning. And I think we're right to be asking these questions. I
would say this is not new. This is not a feature of the current
Congress, although the current Congress is particularly derelict in its
duties. de Tocqueville said this 200 years ago --- there's hardly any
political question in the United States that sooner or later does not
turn into a judicial question. So this is something --- the Court's
centrality to our policy debates has always been there and it's always
been debated. And I would say the separation of powers itself is
really--- it is an ongoing project. It is a journey, not a destination,
right, where there is always going to be push and pull among the
branches over who has what power and who can step in when somebody else
doesn't wield their power. I think that will be with us until the end,
whether the end is in November or in another 200 years. So I do agree
with Ross in the bigger picture, that this is happening right now. So
the extent to which this court will go in both directions, I think, to
take powers that may more appropriately be left with Congress is
remarkable.

ross douthat

I think that's right. I think that what you see, particularly with John
Roberts, is a sense of himself as managing the Court's credibility but
also of himself as a policymaker. And I think that if you read his
decision, he sounds like a legislator making a sort of argument for why
changing conditions have changed the need for this law. That's not
ultimately a constitutional argument. And you can see the same thing in
Obamacare, where basically he had four justices who wanted to strike
down Obamacare, and Roberts decided to save it by rewriting it,
declaring the mandate a tax, changing the --- we have --- we've changed
the Medicaid provisions in Obama's here because of John Roberts' say so.
This is not just a left-wing phenomenon. What it is though is though ---
the pattern with culture war controversies in particular is that there
always seems to be a Republican appointee, whether it's Anthony Kennedy
or now Gorsuch, who wants to sort of take the issue into his own hands
and sort of finesse a national compromise. And that I do think is novel
--- that if you go back and look at the culture wars of the 19th
century, the culture wars of the early 20th century, certainly the
courts were involved throughout. But you have a long pattern of
hard-fought constitutional amendments getting passed. The fact that
we're having this argument at all --- we're having an argument about a
portion of the Civil Rights Act that was itself passed as part of a long
legislative battle. And there just isn't that equivalent today. We don't
have constitutional amendments; no one can imagine passing them. And
Congress doesn't want to make difficult choices on culture war issues if
the Court is willing to take them off its hands.

frank bruni

You two essentially agree that the Court is seizing powers or a kind of
role that it hasn't always done in the past. Assuming Congress doesn't
rouse itself from its stupor or break free of its partisan gridlock, is
there any remedy, any push back, anything one can do to reverse this
trend in the Court or to make sure that it doesn't go further in this
direction?

jesse wegman

I think Ross makes an excellent point about the chief justice, who does
find himself often in a bind. And I think some of that is an artifact of
his position as chief. And I think he considers himself to be more of a
guardian of the Court as an institution than he might were he only an
associate justice. I think he probably would be coming further from the
right if he were only an associate justice. But it's true that when you
have --- when you have justices trying to make policy, you're going to
get bad case law, or confusing rulings, or vaporous language that nobody
knows how to apply in a future case.

frank bruni

But so what's the --- what's the correction of that? How do we get away
from judges making policy then?

jesse wegman

Well, you kill the filibuster for one thing. And then Congress can pass
more laws, because the Senate is where --- what is it? The Senate is the
graveyard of legislation. So if you have a Senate that can actually be
empowered to pass more laws, then you get more laws passed, and you
don't have the Supreme Court making policy on the side.

frank bruni

I look at this from much more of a distance and in way less of a
sophisticated and granular way than the two of you. And I see a sort of
broken Congress. I see a sort of broken system in a lot of ways that
have turned this Congress into somewhat of a joke when it comes to
addressing the nation's needs and when it comes to reflecting the
nation's needs. And so some --- so much of what we're talking about in
terms of the Court may not be how it was originally envisioned, what it
was originally designed for. But in certain instances --- and I would
argue both decisions last week fit into this category. In certain
instances, I think it's --- {[}CHUCKLES{]}

--- performing a vital lifesaving role --- lifesaving in terms of the
democracy and in terms of a system of government that is often broken
these days. Is that such a crazy thought to have, Ross?

ross douthat

My comparison to the fall of the Roman Republic, obviously sounded
apocalyptic. But I do think that there are arguments, certainly that
when systems decay power has to be exercised. And where it needs to be
exercised, the institutions capable of exercising it will do so no
matter what people wringing their hands about constitutional niceties
may think. That's sort of my basic view of the situation, which is why I
used the phrase ``weary resignation'' rather than ``apoplexy.'' I do
think though that there is a pattern here that does not begin with the
decay of Congress over the last 20 years --- that it actually begins
with Court decisions that only social conservatives opposed that clearly
overrode Democratic lawmaking, and that that beginning is more important
in the long run to understanding the Court's peculiar role right now
than the particular decisions of Mitch McConnell. But it might well be
that when the filibuster goes --- and I expect it will go at some point
in the next four to eight years --- you will have some sort of
congressional reassertion. I guess I just also think though that
Congress has sort of trained itself not to want to touch some of these
issues. And that I do think is a very unhealthy thing for democracy.
It's very unhealthy that congressmen are comfortable writing laws about
the budget but not writing laws about gay rights or abortion. And I
think --- and I think that there is no better way to work out a
contested issue. If you have sort of guaranteed civil rights not to be
fired or --- in hiring for gays and lesbians, what does that mean for
Catholic adoption agencies and so on? This is a case that the court will
probably take up next term. I think in a healthy society that would
be--- or a healthy democratic society, which we're supposed to be, that
would be the kind of question that requires debate and compromise that a
legislature would handle. And it's just --- it tells us something about
where we are, I think, that Neil Gorsuch is going to handle it for us
--- and I say that thinking that he may rule on my side of the argument,
right? I think it's totally possible that Gorsuch sees himself as doing
a series of rulings that will on the one hand grant anti-discrimination
protections and on the other hand protect religious liberty. And that's
an outcome that I --- I will happily accept. But it's just --- it's
another sign of a society that is in some--- that is sort of ruled by
its lawyers in a way that has always--- again, I take the Tocqueville
point. It's always been very American to have very powerful lawyers. But
it's still a depressing place for our experiment to end up.

frank bruni

Well, these are issues obviously that Americans care deeply about. So
we'll be keeping a close eye on the Supreme Court. In the meantime,
Jesse, we can't thank you enough for joining us today. And I want to
remind listeners that, if they want to hear more of your thoughts, you
have a book out this year, ``Let the People Pick the President,'' about
the electoral college and your belief that its time has passed. Thank
you, Jesse.

ross douthat

Thanks, Jesse.

jesse wegman

Thanks so much for having me.

frank bruni

That's our show this week. Thanks for listening. {[}THEME MUSIC{]} The
Argument is a production of The New York Times opinion section. The team
includes Phoebe Lett, Lauren Kelley, Paula Schuzman, and Pedro Rafael
Rosado. Special thanks to Brad Fisher, Constanza Gallardo, and James T.
Green. We'll see you next week.

\href{https://www.nytimes3xbfgragh.onion/column/the-argument}{\includegraphics{https://static01.graylady3jvrrxbe.onion/images/2018/10/03/opinion/the-argument-album-art/the-argument-album-art-square320-v3.png}The
Argument}Subscribe:

\begin{itemize}
\tightlist
\item
  \href{https://itunes.apple.com/us/podcast/id1438024613}{Apple
  Podcasts}
\item
  \href{https://www.google.com/podcasts?feed=aHR0cHM6Ly9yc3MuYXJ0MTkuY29tL3RoZS1hcmd1bWVudA\%3D\%3D}{Google
  Podcasts}
\end{itemize}

\hypertarget{place-your-bets-on-bidens-vp-1}{%
\section{Place Your Bets on Biden's
V.P.}\label{place-your-bets-on-bidens-vp-1}}

\hypertarget{plus-a-supreme-court-mystery-1}{%
\subsection{Plus, a Supreme Court
mystery.}\label{plus-a-supreme-court-mystery-1}}

With Frank Bruni, Ross Douthat and Michelle Goldberg

Transcript

transcript

Back to The Argument

bars

0:00/0:00

-0:00

transcript

\hypertarget{place-your-bets-on-bidens-vp-2}{%
\subsection{Place Your Bets on Biden's
V.P.}\label{place-your-bets-on-bidens-vp-2}}

\hypertarget{with-frank-bruni-ross-douthat-and-michelle-goldberg-1}{%
\subsubsection{With Frank Bruni, Ross Douthat and Michelle
Goldberg}\label{with-frank-bruni-ross-douthat-and-michelle-goldberg-1}}

\hypertarget{plus-a-supreme-court-mystery-2}{%
\paragraph{Plus, a Supreme Court
mystery.}\label{plus-a-supreme-court-mystery-2}}

Thursday, June 25th, 2020

\begin{itemize}
\item
  michelle goldberg\\
  I'm Michelle Goldberg.
\item
  frank bruni\\
  I'm Frank Bruni.
\item
  ross douthat\\
  I'm Ross Douthat. And this is ``The Argument.'' {[}THEME MUSIC{]}

  Today on the show, it's Veep stakes. {[}BELL DINGING{]} Where Michelle
  and Frank go to bat for their top Biden V.P. picks and I get to watch,
  or listen. And then we're going to talk about some weird stuff going
  on at the Supreme Court. It's supposed to be a coequal branch of
  government, so why did the judges keep taking over for Congress?
\item
  amy klobuchar\\
  I think the right thing to do right now --- and I told this to Vice
  President Biden --- is to put a woman of color on the ticket as the
  next vice president of our country. All I know is that Joe Biden needs
  to win. And I think---
\end{itemize}

ross douthat

Last week, Minnesota Senator Amy Klobuchar withdrew her name from Joe
Biden's list of possible VP picks and urged the Democratic nominee to
choose a woman of color as his running mate. Biden has already pledged
to pick a woman, but he hasn't gotten more specific. And the field of
possibilities remains pretty big. Kamala Harris, Val Demings, Gretchen
Whitmer, Susan Rice, Elizabeth Warren--- the list goes on. So who should
he pick? Who is the person most likely to help him win the presidency?
To answer this question, we'll pit my co-hosts against each other, and
then maybe I'll weigh in. Michelle, let's start with you. Who's your
pick and why?

michelle goldberg

So it won't surprise listeners of this show that I'm hoping that Joe
Biden picks Elizabeth Warren, who --- full disclosure, my husband
consulted on Warren's presidential campaign. And I'm hoping that he
picks her for both electoral reasons and substantive governing reasons.
She is the favorite of progressives. She's the favorite of young people
in polling. She's the favorite of young Black people in polling. She's
the favorite of a lot of the parts of the Democratic Party that felt
most disappointed with how things played out and with Joe Biden getting
the nomination --- people whose enthusiasm is going to be really, really
important in November. And so it's going to be important to get young
people excited and enthused. And Warren gives them something to be
excited about. Even if you--- if you look on Twitter, you might think
that the Bernie Sanders movement hates Elizabeth Warren, because there's
still a lot of bad blood among some of his high profile surrogates. But
in polls, Bernie supporters want Warren to be the vice president, as do,
like I said, other groups of the progressive base. But beyond that,
Biden is coming into office inheriting these historic crises, these
historic overlapping crises. And he's going to need a governing partner.
And he's talking about kind of needing a New Deal-sized presidency, kind
of New Deal-sized programs to address all of these overlapping
catastrophes. And just nobody has thought more clearly about how you do
that and how you do that within the constraints of a gridlocked
government than Elizabeth Warren. And so I just think she vastly
increases the chances that Joe Biden's presidency would be successful.

ross douthat

Alright, Frank. Tell us why Michelle is wrong and the perfect candidate
is somebody entirely different.

frank bruni

{[}CHUCKLES{]}

I would never say flatly that Michelle is wrong. I would say I have a
difference of opinion with her here. She talks in a compelling way about
Elizabeth Warren as a governing pick, is one of the phrases that's used
for that kind of thing. I think only one thing matters in 2020, and it
is making sure there are not four more years for Donald Trump. So I'm
thinking of this completely and totally in strategic terms. And while
Elizabeth Warren is someone who has some promise in turning out
progressives, there's the question of whether you lose a bunch of White
working class voters, especially in key states, for every voter she
brings in. My pick--- I think a better one in those regards --- is
Senator Tammy Duckworth of Illinois. I think of several things, but
right now I'll just focus on a few. She has an amazing, amazing,
inspiring personal story that I think will resonate with voters at a
time when people really want to feel good about politicians and politics
in a way they haven't in a long time. She's Asian-American ---
Thai-American, to be specific. She is an Iraq war veteran, who lost both
of her legs during her combat service as a helicopter pilot and speaks
of this with such humility and such grace. She is a mother of two
children under the age of six. You want to talk about relatable with
voters? She and her husband are an economically humble couple. She is
someone also who I think would be impossible for Trump and the
Republicans to campaign against. Trump is only comfortable when he is
mocking and insulting people. And his relationship with Elizabeth Warren
--- and we're not going to get into the nickname that he calls her ---
is a great example of that. He will be comfortable campaigning against a
ticket that includes Elizabeth Warren. He won't know what to do with
Tammy Duckworth, who has a Purple Heart, who interestingly coined the
term ``Cadet Bone Spurs'' for him. So he's going to stand there, sit
there, whatever, tongue tied about how to talk about this half of the
Democratic ticket while she brings the case to him in a very, very
winning way. I think she'd be an extraordinary pick.

michelle goldberg

I love Tammy Duckworth. So I'm not going to argue against her. I would
argue against the idea that we can sort of project into the minds of
White working class voters what will and won't resonate with them. And I
think that in the past, when particularly people like us have tried to
do that, we have gone really wide of the mark, right? I remember when
Democrats decided that there was no way George W. Bush, a draft dodger,
was going to be able to campaign effectively against war hero John
Kerry--- they found a way. I remember when people thought that it would
be very hard to smear Max Cleland, the former senator, a triple amputee,
as some sort of terrorist apologist: They did that. And so I'm reluctant
to think about this choice as sort of four-dimensional chess in terms of
what sort of attack ads they will be able to gin up, right? There is no
bottom to what they will say about any candidate. And obviously, Tammy
Duckworth's service to her country is going to mean precisely nothing to
the people who are making attack ads for Donald Trump. I also think
there's something to be said for a candidate who's already run for
president, so we have some idea of how they perform under that
spotlight. And again, I don't doubt that Tammy Duckworth is up to the
job. And if he picks her, it seems to me like a great idea. I accept
your argument that defeating Trump is so important that electoral
concerns should trump those. Joe Biden is there to appeal to White
people in the Midwest and White people in swing states, right? That's
substantially why a lot of Democratic primary voters picked him. And so
we obviously have to worry about those people, but where he has real
weakness, where he really needs to balance the ticket is among young
progressives.

frank bruni

Let me --- let me just quickly push back on two things ---

ross douthat

Yes, do it. Do it.

frank bruni

---in a collegial and friendly way. One is, yes, Elizabeth Warren has
already run for president. And it didn't turn out so well. Now, I have
enormous respect for her. And I think she has given more thought to the
problems that face this nation and how to solve them than probably
anyone we're going to discuss today. But at the end of the day, she was
not able to translate that into the kind of excitement among voters that
would validate her choice as his running mate. She came in third in the
Democratic primary in Massachusetts.

michelle goldberg

Yeah. No, you're right. Although if you look at the polling of people
being asked, if you could just choose who was going to be president as
opposed to who was going to be the nominee, who would you pick, then
Elizabeth Warren does a lot better, right? So she suffered a lot from
people who liked her the best but have been traumatized by 2016 ---
didn't believe that a woman could beat Donald Trump; wanted the safest
possible candidate. She's still, in polls, the most popular with
Democrats of all the people who are being considered for vice president.

frank bruni

One thing we haven't mentioned that I think we have to about Elizabeth
Warren before we move on--- she's 70. Joe Biden is 77. Joe Biden would
be 78 on Inauguration Day, the oldest president ever. Well, Elizabeth
Warren has actually shown quite a bit of appeal to younger voters---
voters of color, that's a whole other story. There is something to me
un-ideal and potentially not a great idea about having a Democratic
ticket, in particular, of two people in their 70s. It goes against the
Democratic Party's brand. It goes against when they've had the most
success in presidential elections, with candidates like Bill Clinton,
with candidates like Barack Obama, who were a lot younger. It feels to
me like a real risk to put out a Democratic ticket --- a ticket for the
party that's about youth, and optimism, and change --- that has a
70-year-old and a 77-year-old on it.

ross douthat

Working off Frank's last point, that sort of the image of a ticket of
two older candidates isn't ideal for the Democrats, both of you are
suggesting candidates who have many virtues but are not
African-American. And I think while Amy Klobuchar in her comment did say
``woman of color'' and Tammy Duckworth is Asian-American, I think the
assumption is that she was suggesting that, given the Black Lives Matter
moment--- the protests, everything that's going on within America and
within liberalism--- that there is some kind of an obligation for Joe
Biden to pick an African-American running mate. And I think you can see
that in the rumors around his campaign right Now. There's a lot of talk
about Kamala Harris, Val Demings, and Susan Rice as frontrunners. So
what do you guys think of that challenge to each of your preferred
candidates?

michelle goldberg

To me, that's the most significant challenge. There was a lot to be said
for him having a Black woman running mate before this moment. And now,
with everything that's going on, it seems even more urgent. I'm not sure
how strong the electoral case is for that. There's a piece in
FiveThirtyEight about the arguments that people are making on behalf of
various vice presidential candidates that's a little bit skeptical about
the electoral arguments--- that having a Black woman on the ticket would
boost Black turnout. The data seems to be mixed about that. Maybe more
significant is the representational argument--- just that it's time. But
all of these women are obviously not interchangeable. And so I think
rather than just make a case for a Black woman, there's cases to be made
for specific Black women, right? And so each of these candidates has
things that make them formidable and things that make them less so,
right? I'm a huge, huge, huge Stacey Abrams fan and yet really worry
about having someone on the ticket who I think won the Georgia
governor's race and was cheated out of it, but the fact remains that she
didn't serve as governor, doesn't have executive experience, and is
stepping into a role when, because of Joe Biden's age, she could end up
being called on to be president almost immediately. So in that respect,
I see more of an argument for Kamala Harris. But then I really like
Kamala Harris. And like Tammy Duckworth, I would be delighted if Kamala
Harris is on the ticket. But think about one of her big weak spots when
she was running for president. People were saying Kamala was a cop. And
she was being --- she was attacked a lot on her criminal justice
background. So I think it's possible to see that in this environment as
being a real liability.

ross douthat

What about the argument, which I think is closer to Michelle's take
here, that vice presidential picks, for all that we obsess over them,
rarely actually have a strong effect on the election --- that how Joe
Biden performs in the fall debates, for instance, is going to be much
more important than who he picks as his running mate? And given Biden's
advanced age and I think a little bit at least of deterioration in his
public speechifying, isn't this a vice presidential pick that's closer
to picking another president, or a co-president, or a future president
than most picks? Do you guys think that's fair?

michelle goldberg

I think that's fair. Although even if vice presidential picks don't have
a large effect, right, this election will likely be one around the
margins. And because there's some question as to how Joe Biden is sort
of identifying and positioning himself --- it's his very interesting
phenomenon where he sort of ran to the center during the primary and has
moved left since winning the nomination, which is just something that
doesn't usually happen --- picking Elizabeth Warren would signal to
people that that move is sincere and that when he talks about governing
in a kind of New Deal fashion that he's serious. But yeah, I do agree
with you that whoever he picks is extremely likely to be the candidate
in 2024.

frank bruni

I think the fact that Biden could be a one-term president has changed
the Veep stakes profoundly to the extent that you see people auditioning
for this job, for lack of a different term, more aggressively than I can
remember in previous presidential cycles. And I think that's because all
of those women under consideration understand that, because of Joe
Biden's age, whoever is his vice president might be the Democratic
nominee in four years rather than eight years, might quickly become the
de facto leader of the party. And that I think has made the job of vice
president triply, if not quadruply, appealing to the people who were in
contention for it. I don't think that voters, who tend to do this more
emotionally than in a hyper-intellectual fashion, are going to say, I've
got to look at this vice presidential pick in a different way than I did
in previous years because of his age because maybe this person is going
to be the Democratic candidate in 2024. I don't think they'll go through
all of those hoops. I agree that the vice presidential choice usually
matters way, way less than those of us who are in the business of
churning out copy ---

ross douthat

Right.

frank bruni

--- and opinions make it seem. But going back to what I view as the
stakes of this election, going back to the fact that I think it is do or
die for this country that Trump, Pence not get another four years, I
think you have to assume as you go into this pick that it could be the
difference. Because if at the margins, as Michelle said, it made the
barest difference and that barest difference was a win/loss difference,
well, you've got to plan for that.

ross douthat

So let me now just before we wrap up throw out my own sort of wildcard
pick, which is not someone he's actually going to pick because it's not
a woman. But I've been struck by the sort of --- the mood on the left
and sort of the centering of urban politics at the moment by what a
strong vice presidential candidate Cory Booker would make at this
moment. And I'm curious what you guys think about that. It seems to me
that Booker is someone who has --- he's African-American, obviously. He
was mayor of Newark, a successful mayor --- a city that has weathered
the current era of protest a lot better than a lot of other cities. He
has a sort of mixed appeal to left and center, with sort of records that
can play to either side. He didn't win the primary obviously, but I
think he came through it better liked by just about everybody than
Harris did. And so I'm wondering if, in deciding for understandable
reasons to say he was going to pick a woman, Joe Biden preemptively
deprived himself of the guy who would have been the right choice for the
political moment we've ended up living in right now. What do you guys
think of that?

michelle goldberg

I think there's something to that. I wrote a piece during the general
election saying, why not Cory Booker, right? That if voters wanted
somebody who was more centrist than Bernie or Warren, that Booker was
really the person they should look to, right? He has this history of
executive experience. He brings a lot to the table. But I think that
you're right that Biden has preemptively narrowed the field. That said,
I think it was a good call at the time when he was asked if he would
make this pledge and he did. And so--- and there's no shortage of really
compelling choices.

frank bruni

Ross, I think you're right that Cory Booker would have been a very, very
compelling choice. But I do think Biden did the right thing, not just at
the time, but forever more in terms of saying, you know what, I'm going
to choose a woman this time. I think putting a woman on the ticket makes
sense in all sorts of ways. I think it still leaves him with an enormous
range of choices, as you can see if you look at the list. Ross, I
thought you were going to ask us --- I thought you were going to make
Michelle and I really go out on a limb and ask us not just who we
wanted, which we've told you, but if we had to place a bet, who we think
Joe Biden will pick. But maybe I'll start by asking you that, and then
you can turn it around, huh?

ross douthat

I think he's going to pick Kamala. I think that she --- in spite of the
issues on criminal justice, I think the campaign's assumption will be
that she is the safest, most vetted African-American female candidate
and that she was always sort of a natural pick for him in certain
ways--- which is boring, which is why I brought up Cory Booker.
{[}LAUGHTER{]} So what---

michelle goldberg

Although Kama--- right. Although actually, a Kamala Harris nomination
would not be boring. It would be really exciting for a lot of people.
And there are a lot of people who love her. If you ever run afoul of the
K-Hive on Twitter, you will quickly learn that she has a pretty
fanatical base of her own.

frank bruni

{[}LAUGHS{]}

ross douthat

I only meant ``boring'' in the sense of being something that we could
see coming a long time ago. So all right. Both of you, who will he pick?

michelle goldberg

I think the same thing. If I had to bet, I would bet on Kamala.

frank bruni

And sadly for a show called ``The Argument,'' what we have here is the
unanimity or the consensus. I think he'll pick Kamala too. And honestly,
if I weren't recommending Tammy Duckworth, she would be my second choice
in terms of our personal recommendation. And I think in both of them
what's important to note--- and we haven't said this about some of the
other contenders --- is if you're trying to play a safe game and if
you're trying to make sure also that you're consistent with your own
argument that you don't want the nation to take a risk after having
taken such a risk on Trump --- and that is a big part of Joe Biden's
argument --- ``look at me, I'm tested, I'm conventionally competent,''
and all those things--- I think choosing someone who's in the Senate,
like Kamala or like Tammy Duckworth, makes more sense than choosing,
say, a mayor or someone who's never been above the status of state
lawmaker. I just think it is a more prudent and a safer way to go.

ross douthat

I think that's absolutely right, but it will be quite something at a
certain level if Joe Biden's running mate is someone who was arguably
beaten decisively in the Democratic primary because she was too far to
the right on criminal justice. Again, I don't think that will prevent
her from being picked, but it will be kind of a remarkable turn given
the kind of debates and arguments we've been having in the country and
that liberals certainly have been arguing for the last month. So we'll
leave it there. And we'll be right back to talk about the Supreme Court.

frank bruni

And we're back. I'm going to take the lead from here. And we will let
Michelle go get on with the rest of her day. In her place, we've invited
our colleague Jesse Wegman. Hi, Jesse.

jesse wegman

Hey, Frank. How are you?

frank bruni

Great. It's great to have you here. Jesse writes about the Supreme Court
and legal issues for The Times editorial board. He's also written a book
about abolishing the electoral college. And he's a big proponent of
killing the filibuster. In fact, don't get him started about Mitch
McConnell stealing a Supreme Court seat from Barack Obama. We asked
Jesse here to talk with us about SCOTUS. And we're going to start with
two recent big decisions --- about L.G.B.T. workplace rights and about
the DREAM Act, or D.A.C.A. A lot of people were surprised given the
Court's conservative majority, Jesse. Were you surprised by these
decisions?

jesse wegman

Not as much as I think most people were. I anticipated the L.G.B.T.
ruling; less so the immigration ruling. But I think it helps to go back
and refresh our memories about what these two cases are about to
understand why they maybe aren't as surprising or as predictive as some
people think they might be. The first ruling, on the L.G.B.T.Q. rights,
held that a federal civil rights law prohibits employers from firing
workers for being gay or transgender. The plaintiffs sued under Title 7
of the Civil Rights Act, which prohibits discrimination in the workplace
on the basis of, among other things, sex. So what does ``sex'' refer to?
The employers who had fired these workers argued, and several of the
conservative justices on the Court agreed, that it could only possibly
refer to men and women in their capacity as men and women. This is how
lawmakers thought about it in 1964. That's clear. And it couldn't be
thought of in any other way. But of course, that can't be true, because
over the years the Supreme Court has expanded Title 7 to cover all kinds
of situations that those original lawmakers wouldn't have thought of.
For example, male-on-male sexual harassment is covered under Title 7.
It's prohibited under Title 7. That certainly wasn't a thing the law
paid attention to in 1964. So the question here was, does sex refer to
sexual orientation and gender identity? Six justices said that it did.
And that opinion, the majority opinion for the Court, was written by not
some bleeding heart liberal, but by the rock-ribbed conservative Neil
Gorsuch. Gorsuch famously adheres to an interpretive approach known as
textualism, and that means, when you're reading a statute, the plain
words of the statute control. And as he pointed out, obviously you
cannot talk about sexual orientation or gender identity without talking
about sex. So there you have it.

frank bruni

Jesse, did that reasoning surprise you? Would you have ever predicted
that Neil Gorsuch would come up with that interpretation of, quote
unquote, textualism?

jesse wegman

Yes, I think that was --- that was actually from the beginning what
people thought was one of the more likely outcomes of this case, was
that Justice Gorsuch would join with the liberals. I think the small
surprise in this case was that Chief Justice John Roberts also joined.
He didn't write separately, so we don't know what his thinking was. But
he has been on the other side of gay rights cases since he's been on the
Court. Now, conservatives generally were apoplectic over this ruling,
saying it's the end of the conservative legal movement, if not of the
world itself. I think that's an overstatement in the extreme. This court
has had a majority of Republican-appointed justices for about a half a
century now. And it is more solidly conservative than it has been since
the 1930s. The liberals on the Court are older on average than the
conservatives. And really, when you look at this opinion, Justice
Gorsuch's decision was purely about a single word--- ``sex.'' It was not
about equality or dignity of gay and transgender people, as it would
have been had it been written by Justice Anthony Kennedy, who is no
longer on the Court, but who wrote all of the Court's gay rights
decisions before this one. In fact, it was actually Justice Kavanaugh,
one of the other conservative justices, who dissented from the ruling,
but ended with a gracious note, I would say, congratulating gay and
transgender people for winning this victory. He's the only one who
really said anything about gay and transgender dignity.

frank bruni

{[}LAUGHS{]}

jesse wegman

It was not justice. Gorsuch himself.

frank bruni

I--- {[}CHUCKLES{]} I'm laughing, Jesse, because, you don't want us to
have our rights, but you want to congratulate us and let us know how
nice you feel about us. Thank you, Justice Kavanaugh. Now, Jesse, you
mentioned conservative apoplexy--- a great phrase. It was intensified by
the other Supreme Court decision that we're going to talk about. Explain
that decision for our listeners.

jesse wegman

So that case involved an executive order which we refer to as DACA,
which is Deferred Action for Childhood Arrivals. And that's a fancy
administrative language to say that people who were brought to America
as children by undocumented immigrant parents, and therefore are not
citizens, but came here through no fault of their own and broke the law
through no fault of their own were allowed to stay without fear of being
immediately deported. There's about 650,000-700,000 of them. It's a lot
of people. And it very much angered, I would say, for lack of a better
term, the White revanchist base that Trump relies on for his electoral
success. He had promised to end it. He said he would wind it down. And
the problem is it's a very popular program. It has support, like, in the
80th percentile, somewhere around there. There's nothing in this country
that that many people agree about. But the DREAMers, as they're called
--- these young people who were brought here as children--- are an
extremely sympathetic group. And Donald Trump knows that. He has said it
himself. So he was in a tough position, right? He wants to satisfy his
angry White base, but he doesn't want the political blowback of shutting
down a popular program. So what do they do? They try to thread the
needle and say, well, we're not shutting it down. We're saying that
President Obama didn't have the authority to issue this order in the
first place --- he exceeded his legal authority. That was their entire
argument. And the Court ruled in an opinion by Chief Justice Roberts
that that wasn't enough. The administration had failed to provide a
reasoned explanation for its decision. Now, once again, as with the
L.G.B.T. case, this was not an emphatic victory that liberals might have
thought it was. It's good, obviously, for the Dreamers because they get
protected for a couple of more years. But essentially, it's a function
of how bad the Trump administration is at governing that Chief Justice
Roberts would say, you can't even do what you're trying to do right. And
so he gave them more opportunities to go back. They can--- technically,
the Trump administration can go back and come up with a better
explanation. Realistically, that's not going to happen before the
November elections.

frank bruni

So even if not emphatic victories for liberals, these were certainly
defeats, temporary or otherwise, for President Trump. But Ross, you
think these are defeats in a larger sense. You see something going on
here that's about the Court's trajectory and role over time that
concerns you greatly. Can you talk about that?

ross douthat

Sure. And first, Jesse, thanks so much for joining us. And I think
you're totally wrong--- {[}FRANK LAUGHS{]} --- about --- about the
conservative legal movement. I think the conservative reaction to this
decision was less apoplexy and more weary resignation.

jesse wegman

I don't know. Did you read Twitter? {[}CHUCKLES{]}

ross douthat

Well, if you're reading Twitter, it's all apoplexy all the time. I'm
just --- I'm not --- I'm talking about the top secret conversations and
the layers of legal conservatives. But ---

jesse wegman

Oh, sorry. Fair enough. I am not privy to this.

ross douthat

Not yet. We'll indoctrinate you later. {[}JESSE LAUGHS{]} But the
independent sort of specific theories of interpretation, the animating
force, the political force behind the conservative legal movement going
back 50 years really now has been the view that unelected judges ---
justices of the Supreme Court --- are taking it upon themselves to
effectively rule on and settle what we call culture war issues/social
issue debates rather than leave them to the Democratic process. So
really, this goes back well before Roe v. Wade. It goes all the way back
to the school prayer rulings in the 1940s and `50s. But obviously, it
continues with rulings on abortion, rulings on same sex marriage, all
the way down to the present day. And again and again, what's happened---
and it happened with Anthony Kennedy, and now it's happened again with
Neil Gorsuch --- is that the Republican appointees put in a position to
rule on contested culture war questions decide to set themselves up as
the arbiters of the controversy rather than leaving it to the democratic
process to work out. And so in that sense, given that pattern, it's not
particularly surprising that Neil Gorsuch came up with a totally absurd
texturalist explanation for why the word ``sex'' actually means ``sexual
orientation.'' It's part of a larger pattern, and one that extends, I
should say, beyond culture war cases. This is a big story that
encompasses all kinds of other debates --- sometimes debates where
liberals are more on the losing side --- where the Supreme Court has
basically increased its powers dramatically as Congress has ceased
legislating and has sort of deferred to the Supreme Court, which you saw
in this case. Republicans in the Senate were very happy to have this
decision because it means that they don't actually have to sit down and
work out the hard questions of, if you pass national anti-discrimination
protections, what are the implications for transgender athletes and
youth sports, what are the implications for religious institutions'
hirings. These are all the kinds of questions that, in a healthy
democracy, legislators would address. In our democracy, Gorsuch issues
this ruling and then says, well, I haven't addressed any of these other
questions, so why don't you file a bunch more lawsuits, and I'll tell
you who wins and loses in those. That's literally what the ruling comes
down to. There is a general sense among conservatives that the country
supports anti-discrimination protections for gays and lesbians. I think
the hard questions around this issue are the ones that Gorsuch didn't
take up, which is, what does it mean when those rights conflict with
religious liberty, what does it mean for single sex institutions.
There's a whole range of these questions. But the bottom line is where
the conservative legal movement has failed is in its quest to have a
Supreme Court that leaves culture war battles to legislatures, to
states, to Congress. That's the failure. It's not that there aren't
Republicans on the bench--- obviously there are. But the whole purpose,
the whole multi-decade push has just ended up once again with a
Republican-appointed justice saying, don't worry, kids, I'm going to
tell you how this culture war controversy is going to turn out.

frank bruni

Ross, I want to ask you this question, but then I want to get Jesse's
response to it too. You mentioned I think accurately that Congress has
ceased to legislate. We basically have a void there. And I would add to
that that Congress --- {[}CHUCKLES{]}

--- the current Congress, due to a whole bunch of forces we could spend
many shows talking about, doesn't do a great job often of reflecting
where the American public is, and what the American public has learned,
and how it's evolved. Given that void, is it such a terrible thing that
the Supreme Court has stepped into it, has stepped up? Why isn't this
evolution of the Court appropriate for this moment in time when Congress
is, as I said, a kind of void?

ross douthat

Well, it might be appropriate for the time in the sense that maybe it
was appropriate when Caesar Augustus did away with senatorial powers
because the Roman Senate had ceased to govern the Roman Republic
{[}FRANK LAUGHS{]} going on 150 years, right? That's --- that's --- I'm
being sort of deliberately overstated, but the historical reality is,
yes, power abhors a vacuum. If a republican institution doesn't fulfill
its duties, some other force or institution will take up that power. But
the design of our system is supposed to leave substantial powers in the
hands of an elected legislature. And so it is a bit of a problem for the
idea of the American republic if policy is increasingly negotiated
between the executive branch and the courts. And you can see that in the
other case we're talking about here, right, where the reason we don't
have a enduring settlement on DACA is that Congress didn't come up with
one; Barack Obama decided to pre-empt Congress, creating a debate about
whether his move was constitutional; then Donald Trump elected, tried to
undo what Obama did; and then the Court came up with somewhat convoluted
reasons to say, not so fast, let's postpone this through the next
election, basically. Again, it's a negotiation between justices
appointed by the president and the executive branch. And that's a
political system; it's just not the one that we're taught about in
civics class, if there are civics classes anymore.

frank bruni

Jesse, do you agree with Ross that the Court is usurping its bounds? And
do you share his concern about that if so?

jesse wegman

Well, first, let me just say, Ross, you mentioned that I had perhaps
overstated the conservative unhappiness. But you did invoke the fall of
Rome. So--- {[}LAUGHTER{]}

ross douthat

I have --- I have--- I have one-upped myself.

frank bruni

Touché.

ross douthat

Yep, touché. Touché.

jesse wegman

So I'll say this, I absolutely--- I wouldn't use the term ``usurp,'' but
I would --- I certainly agree with Ross that the court is stepping into
a vacuum that has been left by Congress. And to be fair, the executive
is also stepping into that vacuum.

ross douthat

Absolutely.

jesse wegman

Some of the results are not as concerning; some of the results are much
more concerning. And I think we're right to be asking these questions. I
would say this is not new. This is not a feature of the current
Congress, although the current Congress is particularly derelict in its
duties. de Tocqueville said this 200 years ago --- there's hardly any
political question in the United States that sooner or later does not
turn into a judicial question. So this is something --- the Court's
centrality to our policy debates has always been there and it's always
been debated. And I would say the separation of powers itself is
really--- it is an ongoing project. It is a journey, not a destination,
right, where there is always going to be push and pull among the
branches over who has what power and who can step in when somebody else
doesn't wield their power. I think that will be with us until the end,
whether the end is in November or in another 200 years. So I do agree
with Ross in the bigger picture, that this is happening right now. So
the extent to which this court will go in both directions, I think, to
take powers that may more appropriately be left with Congress is
remarkable.

ross douthat

I think that's right. I think that what you see, particularly with John
Roberts, is a sense of himself as managing the Court's credibility but
also of himself as a policymaker. And I think that if you read his
decision, he sounds like a legislator making a sort of argument for why
changing conditions have changed the need for this law. That's not
ultimately a constitutional argument. And you can see the same thing in
Obamacare, where basically he had four justices who wanted to strike
down Obamacare, and Roberts decided to save it by rewriting it,
declaring the mandate a tax, changing the --- we have --- we've changed
the Medicaid provisions in Obama's here because of John Roberts' say so.
This is not just a left-wing phenomenon. What it is though is though ---
the pattern with culture war controversies in particular is that there
always seems to be a Republican appointee, whether it's Anthony Kennedy
or now Gorsuch, who wants to sort of take the issue into his own hands
and sort of finesse a national compromise. And that I do think is novel
--- that if you go back and look at the culture wars of the 19th
century, the culture wars of the early 20th century, certainly the
courts were involved throughout. But you have a long pattern of
hard-fought constitutional amendments getting passed. The fact that
we're having this argument at all --- we're having an argument about a
portion of the Civil Rights Act that was itself passed as part of a long
legislative battle. And there just isn't that equivalent today. We don't
have constitutional amendments; no one can imagine passing them. And
Congress doesn't want to make difficult choices on culture war issues if
the Court is willing to take them off its hands.

frank bruni

You two essentially agree that the Court is seizing powers or a kind of
role that it hasn't always done in the past. Assuming Congress doesn't
rouse itself from its stupor or break free of its partisan gridlock, is
there any remedy, any push back, anything one can do to reverse this
trend in the Court or to make sure that it doesn't go further in this
direction?

jesse wegman

I think Ross makes an excellent point about the chief justice, who does
find himself often in a bind. And I think some of that is an artifact of
his position as chief. And I think he considers himself to be more of a
guardian of the Court as an institution than he might were he only an
associate justice. I think he probably would be coming further from the
right if he were only an associate justice. But it's true that when you
have --- when you have justices trying to make policy, you're going to
get bad case law, or confusing rulings, or vaporous language that nobody
knows how to apply in a future case.

frank bruni

But so what's the --- what's the correction of that? How do we get away
from judges making policy then?

jesse wegman

Well, you kill the filibuster for one thing. And then Congress can pass
more laws, because the Senate is where --- what is it? The Senate is the
graveyard of legislation. So if you have a Senate that can actually be
empowered to pass more laws, then you get more laws passed, and you
don't have the Supreme Court making policy on the side.

frank bruni

I look at this from much more of a distance and in way less of a
sophisticated and granular way than the two of you. And I see a sort of
broken Congress. I see a sort of broken system in a lot of ways that
have turned this Congress into somewhat of a joke when it comes to
addressing the nation's needs and when it comes to reflecting the
nation's needs. And so some --- so much of what we're talking about in
terms of the Court may not be how it was originally envisioned, what it
was originally designed for. But in certain instances --- and I would
argue both decisions last week fit into this category. In certain
instances, I think it's --- {[}CHUCKLES{]}

--- performing a vital lifesaving role --- lifesaving in terms of the
democracy and in terms of a system of government that is often broken
these days. Is that such a crazy thought to have, Ross?

ross douthat

My comparison to the fall of the Roman Republic, obviously sounded
apocalyptic. But I do think that there are arguments, certainly that
when systems decay power has to be exercised. And where it needs to be
exercised, the institutions capable of exercising it will do so no
matter what people wringing their hands about constitutional niceties
may think. That's sort of my basic view of the situation, which is why I
used the phrase ``weary resignation'' rather than ``apoplexy.'' I do
think though that there is a pattern here that does not begin with the
decay of Congress over the last 20 years --- that it actually begins
with Court decisions that only social conservatives opposed that clearly
overrode Democratic lawmaking, and that that beginning is more important
in the long run to understanding the Court's peculiar role right now
than the particular decisions of Mitch McConnell. But it might well be
that when the filibuster goes --- and I expect it will go at some point
in the next four to eight years --- you will have some sort of
congressional reassertion. I guess I just also think though that
Congress has sort of trained itself not to want to touch some of these
issues. And that I do think is a very unhealthy thing for democracy.
It's very unhealthy that congressmen are comfortable writing laws about
the budget but not writing laws about gay rights or abortion. And I
think --- and I think that there is no better way to work out a
contested issue. If you have sort of guaranteed civil rights not to be
fired or --- in hiring for gays and lesbians, what does that mean for
Catholic adoption agencies and so on? This is a case that the court will
probably take up next term. I think in a healthy society that would
be--- or a healthy democratic society, which we're supposed to be, that
would be the kind of question that requires debate and compromise that a
legislature would handle. And it's just --- it tells us something about
where we are, I think, that Neil Gorsuch is going to handle it for us
--- and I say that thinking that he may rule on my side of the argument,
right? I think it's totally possible that Gorsuch sees himself as doing
a series of rulings that will on the one hand grant anti-discrimination
protections and on the other hand protect religious liberty. And that's
an outcome that I --- I will happily accept. But it's just --- it's
another sign of a society that is in some--- that is sort of ruled by
its lawyers in a way that has always--- again, I take the Tocqueville
point. It's always been very American to have very powerful lawyers. But
it's still a depressing place for our experiment to end up.

frank bruni

Well, these are issues obviously that Americans care deeply about. So
we'll be keeping a close eye on the Supreme Court. In the meantime,
Jesse, we can't thank you enough for joining us today. And I want to
remind listeners that, if they want to hear more of your thoughts, you
have a book out this year, ``Let the People Pick the President,'' about
the electoral college and your belief that its time has passed. Thank
you, Jesse.

ross douthat

Thanks, Jesse.

jesse wegman

Thanks so much for having me.

frank bruni

That's our show this week. Thanks for listening. {[}THEME MUSIC{]} The
Argument is a production of The New York Times opinion section. The team
includes Phoebe Lett, Lauren Kelley, Paula Schuzman, and Pedro Rafael
Rosado. Special thanks to Brad Fisher, Constanza Gallardo, and James T.
Green. We'll see you next week.

Previous

More episodes ofThe Argument

\href{https://www.nytimes3xbfgragh.onion/2020/09/11/opinion/the-argument-latino-2020-vote.html?action=click\&module=audio-series-bar\&region=header\&pgtype=Article}{\includegraphics{https://static01.graylady3jvrrxbe.onion/images/2020/09/12/opinion/10argumentWeb/10argumentWeb-thumbLarge-v2.jpg}}

September 11, 2020How to Win the Latino Vote

\href{https://www.nytimes3xbfgragh.onion/2020/09/03/opinion/the-argument-trump-biden-kenosha-portland.html?action=click\&module=audio-series-bar\&region=header\&pgtype=Article}{\includegraphics{https://static01.graylady3jvrrxbe.onion/images/2020/09/05/opinion/03argumentWeb/03argumentWeb-thumbLarge.jpg}}

September 3, 2020Is `American Carnage' Campaign Gold?

\href{https://www.nytimes3xbfgragh.onion/2020/08/27/opinion/the-argument-republican-convention-trump.html?action=click\&module=audio-series-bar\&region=header\&pgtype=Article}{\includegraphics{https://static01.graylady3jvrrxbe.onion/images/2020/08/28/opinion/27argument-ninetytwo1-print/27argument-ninetytwo1-thumbLarge.jpg}}

August 27, 2020Can the Republicans Sell a Whole New Trump?

\href{https://www.nytimes3xbfgragh.onion/2020/08/20/opinion/the-argument-democratic-convention-biden.html?action=click\&module=audio-series-bar\&region=header\&pgtype=Article}{\includegraphics{https://static01.graylady3jvrrxbe.onion/images/2020/08/20/opinion/20argument-ninetyone1/20argument-ninetyone1-thumbLarge.jpg}}

August 20, 2020What Biden Must Do

\href{https://www.nytimes3xbfgragh.onion/2020/08/13/opinion/the-argument-coronavirus-catholic-covid.html?action=click\&module=audio-series-bar\&region=header\&pgtype=Article}{\includegraphics{https://static01.graylady3jvrrxbe.onion/images/2020/08/13/opinion/13argument1/merlin_173532477_02e02102-92e6-4f5a-82bf-5394265f898b-thumbLarge.jpg}}

August 13, 2020Is Individualism America's Religion?

\href{https://www.nytimes3xbfgragh.onion/2020/08/06/opinion/the-argument-trump-coronavirus-election.html?action=click\&module=audio-series-bar\&region=header\&pgtype=Article}{\includegraphics{https://static01.graylady3jvrrxbe.onion/images/2020/08/06/opinion/06argSub/06argSub-thumbLarge.jpg}}

August 6, 2020Trump Supporters Make Their Case for 2020

\href{https://www.nytimes3xbfgragh.onion/2020/07/30/opinion/the-argument-authoritarianism-anne-applebaum.html?action=click\&module=audio-series-bar\&region=header\&pgtype=Article}{\includegraphics{https://static01.graylady3jvrrxbe.onion/images/2020/07/31/opinion/30argumentWeb-print/30argumentWeb-thumbLarge.jpg}}

July 30, 2020When Conservatives Fall for Demagogues

\href{https://www.nytimes3xbfgragh.onion/2020/07/23/opinion/the-argument-israel-palestinian.html?action=click\&module=audio-series-bar\&region=header\&pgtype=Article}{\includegraphics{https://static01.graylady3jvrrxbe.onion/images/2020/07/25/opinion/25audio/21argumentWeb-thumbLarge.jpg}}

July 23, 2020The Case for a One-State Solution

\href{https://www.nytimes3xbfgragh.onion/2020/07/16/opinion/the-argument-tammy-duckworth.html?action=click\&module=audio-series-bar\&region=header\&pgtype=Article}{\includegraphics{https://static01.graylady3jvrrxbe.onion/images/2020/07/17/opinion/16argumentWeb-print/16argumentWeb-thumbLarge.jpg}}

July 16, 2020A Conversation With Tammy Duckworth

\href{https://www.nytimes3xbfgragh.onion/2020/07/09/opinion/is-trumps-fate-sealed.html?action=click\&module=audio-series-bar\&region=header\&pgtype=Article}{\includegraphics{https://static01.graylady3jvrrxbe.onion/images/2020/07/10/opinion/10a2_audio/09argument1-thumbLarge.jpg}}

July 9, 2020Is Trump's Fate Sealed?

\href{https://www.nytimes3xbfgragh.onion/2020/07/02/opinion/the-argument-protest-statue-revolution.html?action=click\&module=audio-series-bar\&region=header\&pgtype=Article}{\includegraphics{https://static01.graylady3jvrrxbe.onion/images/2020/07/05/opinion/02argument-eightyfive1/02argument-eightyfive1-thumbLarge.jpg}}

July 2, 2020Whose Statue Must Fall?

\href{https://www.nytimes3xbfgragh.onion/2020/06/25/opinion/the-argument-biden-vice-president-supreme-court.html?action=click\&module=audio-series-bar\&region=header\&pgtype=Article}{\includegraphics{https://static01.graylady3jvrrxbe.onion/images/2020/06/28/opinion/25argument-eightyfour1/25argument-eightyfour1-thumbLarge.jpg}}

June 25, 2020Place Your Bets on Biden's V.P.

\href{https://www.nytimes3xbfgragh.onion/column/the-argument}{See All
Episodes ofThe Argument}

Next

June 25, 2020

\begin{itemize}
\item
\item
\item
\item
\item
\end{itemize}

\emph{\textbf{Listen and subscribe to our podcast from your mobile
device:}}

\textbf{\href{https://itunes.apple.com/us/podcast/the-argument/id1438024613?mt=2}{\emph{Apple
Podcasts}}} \emph{\textbf{\textbar{}}}
\textbf{\href{https://open.spotify.com/show/6bmhSFLKtApYClEuSH8q42}{\emph{Spotify}}}
\emph{\textbf{\textbar{}}}
\textbf{\href{https://play.google.com/music/m/Idxib4hsg3yviao4gtym76knjjy?t=The_Argument}{\emph{Google
Play}}} \emph{\textbf{\textbar{}}}
\textbf{\href{https://radiopublic.com/the-argument-Wdbepr}{\emph{RadioPublic}}}
\emph{\textbf{\textbar{}}}
\textbf{\href{https://www.stitcher.com/podcast/the-new-york-times/the-argument}{\emph{Stitcher}}}

Joe Biden has vowed to pick a woman as his running mate. But of the many
qualified contenders, who should win the veepstakes? Michelle and Frank
have different ideas as to whose name on the ticket could help push Mr.
Biden to victory in November.

Then, editorial board member Jesse Wegman joins Ross and Frank for a
Supreme Court battle: has SCOTUS usurped Congress when it comes to
legislating America's culture wars?

\includegraphics{https://static01.graylady3jvrrxbe.onion/images/2020/06/28/opinion/25argument-eightyfour1/merlin_173479032_d7c52256-c13a-4103-9aca-2109e27c1d1c-articleLarge.jpg?quality=75\&auto=webp\&disable=upscale}

\begin{center}\rule{0.5\linewidth}{\linethickness}\end{center}

\textbf{Background Reading:}

\begin{itemize}
\item
  Ross on
  the\href{https://www.nytimes3xbfgragh.onion/2020/06/20/opinion/sunday/neil-gorsuch-supreme-court.html}{juristocracy
  and the Supreme Court as arbiter of the ``culture war'' debates}
\item
  Michelle on
  \href{https://www.nytimes3xbfgragh.onion/2020/06/15/opinion/lgbt-supreme-court-gorsuch.html}{Justice
  Neil Gorsuch and the L.G.B.T.Q. decision}
\item
  Frank on the
  \href{https://www.nytimes3xbfgragh.onion/2020/06/20/opinion/sunday/trump-supreme-court.html}{court's
  L.G.B.T.Q. civil rights ruling}
\item
  Perry Bacon Jr. for FiveThirtyEight:
  ``\href{https://fivethirtyeight.com/features/the-debate-over-bidens-vp-pick-is-full-of-half-truths-and-misleading-arguments/}{The
  Debate Over Biden's V.P. Pick Is Full Of Half-Truths And Misleading
  Arguments}''
\end{itemize}

\begin{center}\rule{0.5\linewidth}{\linethickness}\end{center}

\hypertarget{meet-the-hosts}{%
\section{Meet the Hosts}\label{meet-the-hosts}}

\hypertarget{frank-bruni}{%
\subsection{Frank Bruni}\label{frank-bruni}}

Image

I've been an Op-Ed columnist for The Times since 2011, but my career
with the newspaper stretches back to 1995 and includes many twists and
turns that reflect my embarrassingly scattered interests. I covered
Congress, the White House and several political campaigns; I also spent
five years in the role of chief restaurant critic. As the Rome bureau
chief, I reported on the Vatican; as a staff writer for The Times's
Sunday magazine, I wrote many celebrity profiles. That jumble has
informed my various books, which focus on the Roman Catholic Church,
George W. Bush, my strange eating life, the college admissions process
and meatloaf. Politically, I'm grief-stricken over the way President
Trump has governed and I'm left of center, but I don't think that the
center is a bad place or ``compromise'' a dirty word. I'm
Italian-American, I'm gay and I write a
\href{https://www.nytimes3xbfgragh.onion/newsletters/frank-bruni}{weekly
Times newsletter} in which you'll occasionally encounter my dog, Regan,
who has the run of our Manhattan apartment.

\hypertarget{ross-douthat}{%
\subsection{Ross Douthat}\label{ross-douthat}}

Image

I've been an Op-Ed columnist since 2009, and I write about politics,
religion, pop culture, sociology and the places where they all
intersect. I'm a Catholic and a conservative, in that order, which means
that I'm against abortion and critical of the sexual revolution, but I
tend to agree with liberals that the Republican Party is too friendly to
the rich. I was against Donald Trump in 2016 for reasons specific to
Donald Trump, but in general I think the populist movements in Europe
and America have legitimate grievances and I often prefer the populists
to the ``reasonable'' elites. I've written books about Harvard, the
G.O.P., American Christianity and Pope Francis, and decadence. Benedict
XVI was my favorite pope. I review movies for National Review and have
strong opinions about many prestige television shows. I have four small
children, three girls and a boy, and I live in New Haven with my wife.

\hypertarget{michelle-goldberg}{%
\subsection{Michelle Goldberg}\label{michelle-goldberg}}

Image

I've been an Op-Ed columnist at The New York Times since 2017, writing
mainly about politics, ideology and gender. These days people on the
right and the left both use ``liberal'' as an epithet, but that's
basically what I am, though the nightmare of Donald Trump's presidency
has radicalized me and pushed me leftward. I've written three books,
including one, in 2006, about the danger of right-wing populism in its
religious fundamentalist guise. (My other two were about the global
battle over reproductive rights and, in a brief detour from politics,
about an adventurous Russian émigré who helped bring yoga to the West.)
I love to travel; a long time ago, after my husband and I eloped, we
spent a year backpacking through Asia. Now we live in Brooklyn with our
son and daughter.

\begin{center}\rule{0.5\linewidth}{\linethickness}\end{center}

\hypertarget{how-do-i-listen}{%
\subsection{How do I listen?}\label{how-do-i-listen}}

\emph{Tune in on}
\href{https://itunes.apple.com/us/podcast/the-argument/id1438024613?mt=2}{\emph{iTunes}}\emph{,}
\href{https://play.google.com/music/listen?u=0\#/ps/Idxib4hsg3yviao4gtym76knjjy}{\emph{Google
Play}}\emph{,}
\href{https://open.spotify.com/episode/5fIsHqqunLBwoxPSUUSGre?si=Rz5D9VnlRFKdGMu8ixzBOw}{\emph{Spotify}}\emph{,}
\href{https://www.stitcher.com/podcast/the-new-york-times/the-argument}{\emph{Stitcher}}
\emph{or wherever you listen to podcasts. Tell us what you think at}
\href{mailto:argument@NYTimes.com}{\emph{argument@NYTimes.com.}}
\emph{Follow Frank Bruni
(}\href{https://twitter.com/FrankBruni}{\emph{@FrankBruni}}\emph{), Ross
Douthat
(}\href{https://twitter.com/DouthatNYT}{\emph{@DouthatNYT}}\emph{) and
Michelle Goldberg
(}\href{https://twitter.com/michelleinbklyn}{\emph{@michelleinbklyn}}\emph{)
on Twitter.}

``The Argument'' Is a production of the New York Times Opinion section.
The team includes Phoebe Lett, Lauren Kelley, Paula Schuzman and Pedro
Rafael Rosado. Special thanks to Brad Fisher, Constanza Gallardo, Sara
Nics and James T. Green.

Advertisement

\protect\hyperlink{after-bottom}{Continue reading the main story}

\hypertarget{site-index}{%
\subsection{Site Index}\label{site-index}}

\hypertarget{site-information-navigation}{%
\subsection{Site Information
Navigation}\label{site-information-navigation}}

\begin{itemize}
\tightlist
\item
  \href{https://help.nytimes3xbfgragh.onion/hc/en-us/articles/115014792127-Copyright-notice}{©~2020~The
  New York Times Company}
\end{itemize}

\begin{itemize}
\tightlist
\item
  \href{https://www.nytco.com/}{NYTCo}
\item
  \href{https://help.nytimes3xbfgragh.onion/hc/en-us/articles/115015385887-Contact-Us}{Contact
  Us}
\item
  \href{https://www.nytco.com/careers/}{Work with us}
\item
  \href{https://nytmediakit.com/}{Advertise}
\item
  \href{http://www.tbrandstudio.com/}{T Brand Studio}
\item
  \href{https://www.nytimes3xbfgragh.onion/privacy/cookie-policy\#how-do-i-manage-trackers}{Your
  Ad Choices}
\item
  \href{https://www.nytimes3xbfgragh.onion/privacy}{Privacy}
\item
  \href{https://help.nytimes3xbfgragh.onion/hc/en-us/articles/115014893428-Terms-of-service}{Terms
  of Service}
\item
  \href{https://help.nytimes3xbfgragh.onion/hc/en-us/articles/115014893968-Terms-of-sale}{Terms
  of Sale}
\item
  \href{https://spiderbites.nytimes3xbfgragh.onion}{Site Map}
\item
  \href{https://help.nytimes3xbfgragh.onion/hc/en-us}{Help}
\item
  \href{https://www.nytimes3xbfgragh.onion/subscription?campaignId=37WXW}{Subscriptions}
\end{itemize}
