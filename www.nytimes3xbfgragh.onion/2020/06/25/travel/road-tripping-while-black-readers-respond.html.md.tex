Sections

SEARCH

\protect\hyperlink{site-content}{Skip to
content}\protect\hyperlink{site-index}{Skip to site index}

\href{https://www.nytimes3xbfgragh.onion/section/travel}{Travel}

\href{https://myaccount.nytimes3xbfgragh.onion/auth/login?response_type=cookie\&client_id=vi}{}

\href{https://www.nytimes3xbfgragh.onion/section/todayspaper}{Today's
Paper}

\href{/section/travel}{Travel}\textbar{}Road Tripping While Black:
Readers Respond

\url{https://nyti.ms/31eYFwz}

\begin{itemize}
\item
\item
\item
\item
\item
\item
\end{itemize}

Advertisement

\protect\hyperlink{after-top}{Continue reading the main story}

Supported by

\protect\hyperlink{after-sponsor}{Continue reading the main story}

\hypertarget{road-tripping-while-black-readers-respond}{%
\section{Road Tripping While Black: Readers
Respond}\label{road-tripping-while-black-readers-respond}}

Hitting the open road can be fraught for some black Americans, who share
their anxieties of racist targeting. For others, getting behind the
wheel is freedom.

\includegraphics{https://static01.graylady3jvrrxbe.onion/images/2020/06/25/travel/25road-trip-grid/25road-trip-grid-articleLarge.jpg?quality=75\&auto=webp\&disable=upscale}

\href{https://www.nytimes3xbfgragh.onion/by/tariro-mzezewa}{\includegraphics{https://static01.graylady3jvrrxbe.onion/images/2018/08/24/opinion/tariro-headshot/tariro-headshot-thumbLarge-v2.png}}\href{https://www.nytimes3xbfgragh.onion/by/tacey-rychter}{\includegraphics{https://static01.graylady3jvrrxbe.onion/images/2019/01/29/multimedia/author-tacey-rychter/author-tacey-rychter-thumbLarge.png}}

By \href{https://www.nytimes3xbfgragh.onion/by/tariro-mzezewa}{Tariro
Mzezewa} and
\href{https://www.nytimes3xbfgragh.onion/by/tacey-rychter}{Tacey
Rychter}

\begin{itemize}
\item
  Published June 25, 2020Updated July 13, 2020
\item
  \begin{itemize}
  \item
  \item
  \item
  \item
  \item
  \item
  \end{itemize}
\end{itemize}

The road trip has long been considered part of the great American
experience, but the feeling of freedom often associated with getting in
a car and seeing the country hasn't always been extended to
African-American travelers, who often worry about discrimination and
racism while traveling.

After publishing a story on the concerns and anxieties many
\href{https://www.nytimes3xbfgragh.onion/2020/06/10/travel/road-trip-black.html}{black
travelers experience when driving}, we asked readers to share their
experiences. In hundreds of responses, people told us about their fears
and worries, but they also shared stories of a pastime they've enjoyed
with grandparents, parents, children and friends.

Many people said that their families used the Green Book, a guidebook
first published in 1936 that listed welcoming towns, motels and
restaurants for black travelers; others said that trips require a lot of
planning and little spontaneity. Fear of being stopped by the police or
encountering racism on the road has kept some people from venturing out.
But for those who have traveled, seeing the country has been rewarding
and they plan on continuing to explore.

The following responses have been lightly edited for length and clarity.

\hypertarget{it-is-a-lifestyle-this-worry}{%
\subsection{`It is a lifestyle, this
worry'}\label{it-is-a-lifestyle-this-worry}}

Image

Thalia Rodriguez with her brother on a trip to the Jersey
Shore.~Credit...Thalia Rodriguez

I've taken several road trips to the Poconos with my mother and siblings
and we have been stopped by the police, unfortunately, almost every
time. We are from the Bronx and we would take road trips to indoor water
parks like Great Wolf Lodge and were always stopped. It would cause us
extreme anxiety. The cops would be so aggressive toward our mother and
we would be still in the car waiting to see what would happen. To this
day, as young adults, when going for road trips to the lakes or Lake
George we are always looking out the window, actively worried. It is a
lifestyle, this worry and anxiety about the cops.

\emph{--- Thalia Rodriguez, Bronx, N.Y.}

\hypertarget{this-land-is-my-land-and-i-plan-to-drive-every-mile-of-it}{%
\subsection{`This land is my land, and I plan to drive every mile of
it.'}\label{this-land-is-my-land-and-i-plan-to-drive-every-mile-of-it}}

\includegraphics{https://static01.graylady3jvrrxbe.onion/images/2020/06/24/travel/oakImage-1593019440077/oakImage-1593019440077-articleLarge-v2.jpg?quality=75\&auto=webp\&disable=upscale}

My parents always told me stories of the perils of ``road tripping while
Black,'' but it never stopped us from going on road trips as a family.
To this day, I have a pretty good interstate map in my head. At least
twice a year I end up taking a trip of 12 hours or more. I probably
worry about being a solo traveler more than being a black traveler. I'm
always aware of my surroundings because being solo just makes me more of
a target. I generally stick to the Interstate System, and when I do
venture on back roads, I try to avoid stopping for gas or bathroom
breaks. The prevalence of Confederate flags mounted on porches and
flying in yards in the rural South still chills me to my core. I have
logged miles through the dense fog of the Appalachian Mountains, the
humid heat of the Deep South and the frozen prairies of Illinois, and I
don't see any reason to stop now. This land is my land, and I plan to
drive every mile of it.

--- \emph{Joshua Lawrence Lazard, Raleigh, N.C.}

\hypertarget{it-makes-you-wary-all-the-time}{%
\subsection{`It makes you wary, all the
time'}\label{it-makes-you-wary-all-the-time}}

Image

Valerie Johnson (bottom-left) on vacation with her
family.Credit...Valerie Johnson

In about 1971 in Georgia, the skies opened up on our family trip back
home to North Carolina from our vacation in Florida, so when we saw the
glowing red vacancy sign, we were elated. We pulled up close to the door
and a man came out in a trench coat with a big umbrella, a welcoming
smile on his face. But as soon as the inside light popped on, revealing
my mother, father, brother and me as black, the man's smile vanished. He
signaled ``no'' with his hand and retreated. The last thing I saw
through the rain-streaked window was the ``no'' flickering on in front
of ``vacancy'' as we pulled out of the mostly empty parking lot. I was
only 6 or 7 but I knew what had happened.

This summer I need to accompany my daughter on her move to California.
There are states where I am not comfortable on the roads. I am always
afraid that the police will stop me. I am afraid that I will be accused
of doing something and the fact that I am a middle-aged lawyer won't
save me. And I am wary of stopping at establishments in towns where I
might not be wanted. That is what racism does. It makes you wary, all
the time, so we will likely fly despite the risks of airplane travel.

--- \emph{Valerie Johnson, Durham, N.C.}

\hypertarget{unless-im-with-my-white-friends-i-wont-stop}{%
\subsection{`Unless I'm with my white friends, I won't
stop'}\label{unless-im-with-my-white-friends-i-wont-stop}}

Image

Tyler Beckworth, with his fiancee at Zion National Park in Utah,
February~ 2018.~Credit...Tyler Beckworth

Road trips have always been a passion of mine and I will certainly plan
a few this summer. However, being a black male has and will continue to
keep certain places off limits in my mind. Rolling through a small town
and stopping in the local bar has always intrigued me, but unless I'm
with my white friends, I won't stop. There's simply no knowing who is in
there. I'm sure I've missed out on a great bar with great people. But
it's not a risk I'm willing to take.

\emph{--- Tyler Beckworth., Los Angeles, Calif.}

\hypertarget{i-sweat-every-detail}{%
\subsection{`I sweat every detail'}\label{i-sweat-every-detail}}

My friends, who are mostly white, have a freedom in making travel plans
I never will. I sweat every detail from where we stop for gas, spend the
night, even the side roads and detours we take.

I refuse to let fear dictate the choices I make, but there are limits to
how much protection even the most abundant caution and vigilance can
provide. I've been incredibly lucky and never had a racial incident
while driving, but I wonder with every trip whether or not this is the
time my luck runs out.

\emph{--- Spencer Gilbert, New York, N.Y.}

\hypertarget{we-did-not-experience-any-overt-hatred}{%
\subsection{`We did not experience any overt
hatred'}\label{we-did-not-experience-any-overt-hatred}}

Image

April Banks during her 2010 cross-country bicycle ride from San
Francisco to Washington D.C..

I have taken multiple short and long road trips across America,
including one bicycle ride from San Francisco to Washington D.C.,
exactly 10 years ago this June.

Of all my international travels, it was the bicycle trip across America
that worried my friends and family the most. Two black women on bicycles
riding through Middle America. We were highly visible and had lots of
interactions with other travelers and folks in small-town America. Our
awareness was heightened, but we did not experience any overt hatred.

We did experience many acts of kindness, like a ride into town during a
hailstorm. Or when a white police chief in a small town let us camp in
his backyard because the only motel in town said they had ``no
vacancy.'' We heard many opinions from native, black, white and
immigrant communities about our ride and about their politics. We saw
signs protesting Obama. Mostly people were shocked and curious to see
us.

\emph{--- April Banks, Los Angeles, Calif.}

\hypertarget{never-had-any-issues}{%
\subsection{`Never had any issues'}\label{never-had-any-issues}}

Image

Sam Tyler on a trip to Yosemite National Park in 2017.

I am a 39-year-old black male who has traveled all around the U.S.A. and
in 13 countries. In all of my road trips, I have never really been
afraid to travel because of my race. There are times when I stop in
small rural towns (especially in the South) for gas or at welcome stops
where I can tell my presence as a black male is not welcome. The biggest
risks, in my opinion, for black travelers are state troopers who see us
as targets for stops. I will never let the racism of a few deter me from
traveling, as during most of my traveling I have never had any issues. I
think it is important to stay aware while not being afraid.

\emph{--- Sam Tyler, Atlanta, Ga.}

\hypertarget{weve-always-made-it-beautiful}{%
\subsection{`We've always made it
beautiful'}\label{weve-always-made-it-beautiful}}

Image

Jerilyn D. Williams as a child (center) during a family reunion in Ohio
in the summer of 1973.

As a young girl in the 1970s, I always looked forward to our annual
summer road trip to Tennessee for my grandfather's family reunion. One
of the best treats of the weekend was the huge picnic lunch my
grandmother would prepare for our trip down I-65. Mom (as my sisters and
I called our grandmother) would fry chicken and make potato salad,
meatloaf, poundcake and more. There would be six or seven cars full of
Walkers and Wards heading south to Centerville, and we'd all stop at
that same rest area in Kentucky for lunch. Then we'd pile back into our
respective cars and make our way down the interstate.

It didn't dawn on me until I was in my 20s why my family had this
established tradition. My grandmother made and packed lunch for her
three children during the 1950s and '60s when they would trek to the
South. My grandparents were painfully aware of the South's Jim Crow
customs and unwritten laws (even in the 1970s and '80s). They knew,
although laws had changed since the Civil Rights Movement, laws don't
change hearts and minds.

The mid-trip picnics I enjoyed so much as a child began out of
necessity, not from choice.

Even though racism and bigotry were the catalysts for my family's
tradition, we have lovingly turned it into something warming and
familial. That's what my ancestors have always done: We've always been
given the scraps, the leftovers, the remnants; and we've always made it
beautiful, covetous, delicious.

\emph{--- Jerilyn D. Williams, Lansing, Mich.}

\begin{center}\rule{0.5\linewidth}{\linethickness}\end{center}

\emph{\textbf{Follow New York Times Travel}}
\emph{on}\href{https://www.instagram.com/nytimestravel/}{\emph{Instagram}}\emph{,}\href{https://twitter.com/nytimestravel}{\emph{Twitter}}
\emph{and}\href{https://www.facebookcorewwwi.onion/nytimestravel/}{\emph{Facebook}}\emph{.
And}\href{https://www.nytimes3xbfgragh.onion/newsletters/traveldispatch}{\emph{sign
up for our weekly Travel Dispatch newsletter}} \emph{to receive expert
tips on traveling smarter and inspiration for your next vacation.
Dreaming up a future getaway or just armchair traveling? Check out
our}\href{https://www.nytimes3xbfgragh.onion/interactive/2020/travel/places-to-visit.html}{\emph{52
Places list}}\emph{.}

Advertisement

\protect\hyperlink{after-bottom}{Continue reading the main story}

\hypertarget{site-index}{%
\subsection{Site Index}\label{site-index}}

\hypertarget{site-information-navigation}{%
\subsection{Site Information
Navigation}\label{site-information-navigation}}

\begin{itemize}
\tightlist
\item
  \href{https://help.nytimes3xbfgragh.onion/hc/en-us/articles/115014792127-Copyright-notice}{©~2020~The
  New York Times Company}
\end{itemize}

\begin{itemize}
\tightlist
\item
  \href{https://www.nytco.com/}{NYTCo}
\item
  \href{https://help.nytimes3xbfgragh.onion/hc/en-us/articles/115015385887-Contact-Us}{Contact
  Us}
\item
  \href{https://www.nytco.com/careers/}{Work with us}
\item
  \href{https://nytmediakit.com/}{Advertise}
\item
  \href{http://www.tbrandstudio.com/}{T Brand Studio}
\item
  \href{https://www.nytimes3xbfgragh.onion/privacy/cookie-policy\#how-do-i-manage-trackers}{Your
  Ad Choices}
\item
  \href{https://www.nytimes3xbfgragh.onion/privacy}{Privacy}
\item
  \href{https://help.nytimes3xbfgragh.onion/hc/en-us/articles/115014893428-Terms-of-service}{Terms
  of Service}
\item
  \href{https://help.nytimes3xbfgragh.onion/hc/en-us/articles/115014893968-Terms-of-sale}{Terms
  of Sale}
\item
  \href{https://spiderbites.nytimes3xbfgragh.onion}{Site Map}
\item
  \href{https://help.nytimes3xbfgragh.onion/hc/en-us}{Help}
\item
  \href{https://www.nytimes3xbfgragh.onion/subscription?campaignId=37WXW}{Subscriptions}
\end{itemize}
