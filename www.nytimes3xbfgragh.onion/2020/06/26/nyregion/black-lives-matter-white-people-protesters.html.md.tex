Sections

SEARCH

\protect\hyperlink{site-content}{Skip to
content}\protect\hyperlink{site-index}{Skip to site index}

\href{https://www.nytimes3xbfgragh.onion/section/nyregion}{New York}

\href{https://myaccount.nytimes3xbfgragh.onion/auth/login?response_type=cookie\&client_id=vi}{}

\href{https://www.nytimes3xbfgragh.onion/section/todayspaper}{Today's
Paper}

\href{/section/nyregion}{New York}\textbar{}Black Activists Wonder: Is
Protesting Just Trendy for White People?

\url{https://nyti.ms/3eCToCV}

\begin{itemize}
\item
\item
\item
\item
\item
\item
\end{itemize}

\hypertarget{race-and-america}{%
\subsubsection{\texorpdfstring{\href{https://www.nytimes3xbfgragh.onion/news-event/george-floyd-protests-minneapolis-new-york-los-angeles?name=styln-george-floyd\&region=TOP_BANNER\&block=storyline_menu_recirc\&action=click\&pgtype=Article\&impression_id=326d0cc0-f1ba-11ea-bd45-53512389bd45\&variant=undefined}{Race
and America}}{Race and America}}\label{race-and-america}}

\begin{itemize}
\tightlist
\item
  \href{https://www.nytimes3xbfgragh.onion/2020/09/04/nyregion/rochester-police-daniel-prude.html?name=styln-george-floyd\&region=TOP_BANNER\&block=storyline_menu_recirc\&action=click\&pgtype=Article\&impression_id=326d33d0-f1ba-11ea-bd45-53512389bd45\&variant=undefined}{How
  Police Handled Death of Daniel Prude}
\item
  \href{https://www.nytimes3xbfgragh.onion/2020/09/01/us/politics/trump-fact-check-protests.html?name=styln-george-floyd\&region=TOP_BANNER\&block=storyline_menu_recirc\&action=click\&pgtype=Article\&impression_id=326d33d1-f1ba-11ea-bd45-53512389bd45\&variant=undefined}{Trump
  Fact Check}
\item
  \href{https://www.nytimes3xbfgragh.onion/2020/08/30/us/portland-shooting-explained.html?name=styln-george-floyd\&region=TOP_BANNER\&block=storyline_menu_recirc\&action=click\&pgtype=Article\&impression_id=326d33d2-f1ba-11ea-bd45-53512389bd45\&variant=undefined}{Portland
  Shooting}
\item
  \href{https://www.nytimes3xbfgragh.onion/2020/08/30/us/breonna-taylor-police-killing.html?name=styln-george-floyd\&region=TOP_BANNER\&block=storyline_menu_recirc\&action=click\&pgtype=Article\&impression_id=326d33d3-f1ba-11ea-bd45-53512389bd45\&variant=undefined}{Breonna
  Taylor's Life and Death}
\end{itemize}

Advertisement

\protect\hyperlink{after-top}{Continue reading the main story}

Supported by

\protect\hyperlink{after-sponsor}{Continue reading the main story}

\hypertarget{black-activists-wonder-is-protesting-just-trendy-for-white-people}{%
\section{Black Activists Wonder: Is Protesting Just Trendy for White
People?}\label{black-activists-wonder-is-protesting-just-trendy-for-white-people}}

Though black protesters have been heartened by the many white people
joining them in the streets, some wonder if this newfound commitment
will last.

\includegraphics{https://static01.graylady3jvrrxbe.onion/images/2020/06/18/nyregion/00nyunrest-whiteprotesters-patton/00nyunrest-whiteprotesters-patton-articleLarge.jpg?quality=75\&auto=webp\&disable=upscale}

\href{https://www.nytimes3xbfgragh.onion/by/nikita-stewart}{\includegraphics{https://static01.graylady3jvrrxbe.onion/images/2018/09/25/multimedia/author-nikita-stewart/author-nikita-stewart-thumbLarge-v2.png}}

By \href{https://www.nytimes3xbfgragh.onion/by/nikita-stewart}{Nikita
Stewart}

\begin{itemize}
\item
  Published June 26, 2020Updated June 27, 2020
\item
  \begin{itemize}
  \item
  \item
  \item
  \item
  \item
  \item
  \end{itemize}
\end{itemize}

\hypertarget{listen-to-this-article}{%
\subsubsection{Listen to This Article}\label{listen-to-this-article}}

Audio Recording by Audm

\emph{To hear more audio stories from publishers like The New York
Times,
download}\href{https://www.audm.com/?utm_source=nytmag\&utm_medium=embed\&utm_campaign=left_behind_draper}{**}\href{https://www.audm.com/?utm_source=nyt\&utm_medium=embed\&utm_campaign=black_trendy_white}{\emph{Audm
for iPhone or Android}}\emph{.}

Cherish Patton recalled springing into action when a friend sent her a
message that a New York City police officer had grabbed a petite
protester by her hood and had flung her to the pavement.

\href{https://www.instagram.com/cherishluvsyou/?hl=en}{Ms. Patton}, who
has organized several Black Lives Matter protests, posted a plea on
social media for help identifying the officer. She also called her
friend for details on the protester, who had been whisked to the
emergency room. ``Oh, it's Michelle,'' her friend told her.

``Wait, white Michelle who I argued with for three years? White
Michelle?'' asked an astonished, and confused, Ms. Patton, who is black.
The hurt protester was a former classmate, Michelle Moran, 18, whose
conservative commentary on politics and social issues had made Ms.
Patton, 18, cringe in high school in Manhattan.

George Floyd's death in police custody in Minneapolis pushed anguished
black people into the streets, as had happened countless times after
police killings of black people. But this time, the black protesters
have been joined en masse by white people, in rallies across New York
City and around the country.

Now, though, the protests in New York City are ebbing somewhat, though
they are still drawing thousands of people to some events, particularly
on weekends. And outside City Hall, there is a growing encampment of
diverse demonstrators who are demanding deep cuts in the police budget.

And so that naturally raises a question for black activists who have
long been dedicated to the movement: Will the commitment of white
protesters endure?

Some of the white protesters identify as liberal and said they had long
been sympathetic to the \href{https://blacklivesmatter.com/}{Black Lives
Matter} movement but had not done much, if anything, before to show it.
Other white people said they had once believed that the police did not
discriminate against black people but had changed their minds because of
Mr. Floyd's killing.

Some black people have responded to the influx of white protesters with
a mix of hope, I-told-you-so sentiment and skepticism. For longtime
activists, there is a frustration that it took a global pandemic and yet
another death at the hands of the police to push white people to
publicly embrace the movement. They wonder how long white people will
keep showing up.

``We see so many white people who hate us, absolutely hate us for the
way that we look,'' Ms. Patton said, adding: ``To see white people on
the front lines, it's exciting to know that these younger generations of
white people care.

``This is a different level of protest.''

\includegraphics{https://static01.graylady3jvrrxbe.onion/images/2020/06/14/nyregion/00nyunrest-whiteprotesters-1/merlin_173386638_df2fca3e-ed86-44b1-94cf-9076040d3962-articleLarge.jpg?quality=75\&auto=webp\&disable=upscale}

Still, some black protesters and activists expressed ambivalence about
the shift.

\href{https://www.opaltometi.org/}{Opal Tometi}, 35, a co-founder of
Black Lives Matter, called the outpouring ``beautiful,'' but she added,
``I have minor trepidation, like most, that this could end up being a
trend.''

``When the social media posts die down, will the actions and people's
conviction for change die down too? '' she said in written responses to
questions. ``I have been waiting for this moment since I was 12 years
old as the only Black kid on the block. I've always known I've been a
part of something bigger than myself. I didn't know how it would unfold,
but here we are.''

Anthony Beckford, president of Black Lives Matter Brooklyn, recalled
being at a protest in Brooklyn and feeling uneasy about the large
numbers of white people who had shown up.

``I looked around and I was like: `I feel outnumbered. Is my life in
danger?''' said Mr. Beckford, 38, who added that he feared that some of
the protesters were white nationalists infiltrating the march.

He said he and his friends have had to tell some white protesters that
they could not just show up and take over.

``Our fight is our fight. Their privilege can amplify the message, but
they can never speak for us,'' Mr. Beckford said. ``There have been
moments where some have wanted to be in the front. I've told them to go
to the back.''

Image

``Our fight is our fight,'' said Anthony Beckford, president of Black
Lives Matter Brooklyn,~Credit...Demetrius Freeman for The New York Times

Two young white people new to the movement tried to organize a protest
in Bay Ridge that Mr. Beckford found out about from other white people.
He said he shut it down. ``Their messaging was, `Yes, black lives
matter, and police lives matter, too.' I was, like, no. You can think of
the `Kumbaya' moment when we get our mission accomplished,'' he said.

Research
\href{https://www.nytimes3xbfgragh.onion/2020/06/12/us/george-floyd-white-protesters.html}{does
seem to confirm black protesters' sense} that they have been joined for
the first time at demonstrations against police brutality by large
numbers of white protesters.

One study of the Floyd protests on one weekend this month found
overwhelmingly young crowds, with large numbers of white and highly
educated people. White protesters made up 61 percent of those surveyed
in New York, according to the researchers, and 65 percent of protesters
in Washington. In Los Angeles, 53 percent of protesters were white.

Opinion polls have also shown that racial attitudes among white
Americans have been shifting, with a sharp turn by white liberals toward
a more sympathetic view of black people.

Ms. Moran, the injured white protester whose plight was noticed by Ms.
Patton, said she was a newcomer to the movement. She said her parents
and a childhood in a predominantly white block of Woodlawn, in the
Bronx, initially shaped her worldview and politics.

``I slowly but surely opened my eyes to the horrors of the criminal
justice system,'' said Ms. Moran, who said she turned a corner a year
ago, influenced by readings, the news and the documentary
``\href{https://www.nytimes3xbfgragh.onion/2016/01/29/movies/review-noam-chomsky-focuses-on-financial-inequality.html}{Requiem
for the American Dream}'' about income inequality.

As for her parents, Ms. Moran said, ``I'm still trying to change them,
but they're not budging.''

Ms. Patton, her voice hoarse from daily chants and speeches, said she
remains skeptical of some white protesters who she believes are showing
up to ``wreak havoc.''

But talking now with Ms. Moran, Ms. Patton said she saw that some white
people were willing to be allies.

The teenagers have gone from barely speaking to now having a mutual
respect for each other, they said.

These issues are playing out in school settings across the city as well.

When Theo Schimmel, 14, who identifies as white and Indian, decided to
hold a protest for children in Washington Heights, where he lives, he
reached out to his classmates from Bank Street School for Children,
Melany Linton, who identifies as Afro-Latina, and Stella Tillery-Lee,
who is black.

Asked whether he chose them because they were black, Theo paused and
then said: ``Yeah, but I didn't really focus on that aspect of it. I
knew how important this was to them in classes.''

Image

From left, Stella Tillery-Lee, Theo Schimmel and Melany Linton,
classmates at Bank Street School for Children, held a protest in Fort
Tryon Park.Credit...Simbarashe Cha for The New York Times

Stella, 14, who lives in Harlem, said she appreciated that Theo took the
step that he did. ``We definitely need more people that are not
necessarily African-American or black helping to support our community
because so many people are being bystanders, which is great, but it's
not enough at all,'' she said.

About 300 people showed up to join Stella, Melany and Theo on a lawn in
Fort Tryon Park.

``Throughout history, people see black people as inhuman or as objects
and that's ridiculous,'' Melany said in an interview. ``The fact that so
many things, like what happened to George Floyd, continue to go on in
our country is so upsetting and disturbing that it really does strike a
certain nerve in people, as it should.''

Among the protesters were the teachers Ever Ramirez, who is Asian, and
Shelby Brody, who is white. They held signs reading, ``DEFUND THE
POLICE. INVEST IN SCHOOLS'' and ``ASIANS FOR BLACK LIVES MATTER.''

Mx. Brody said they had learned more about themselves and racism by
reading the book
``\href{https://www.nytimes3xbfgragh.onion/2020/07/15/magazine/white-fragility-robin-diangelo.html}{White
Fragility}'' by Robin DiAngelo and taking part in a group at school
where white employees explored racism and their role in it.

Mx. Brody had initially steered clear of the group. ``I was called in by
a colleague of color who rightly said, `White people sitting out is part
of the problem,''' Mx. Brody said.

Also at the park protest was one of Melany's family friends, April
Dinwoodie, 48, who splits her time between Harlem and Westerly, R.I.,
where 95 percent of the residents are white.

A biracial woman raised in the town by her white adoptive parents with
white siblings, Ms. Dinwoodie said she moved to Harlem years ago as she
searched for a connection to ``my blackness.''

Image

April Dinwoodie was stunned to see residents of her largely white
hometown in Rhode Island protesting.Credit...Simbarashe Cha for The New
York Times

Driving through the town recently, she said she could not believe what
she saw. There they were, dozens of Westerly residents holding a Black
Lives Matter protest.

``I was like, `Oh, my gosh,''' she said, almost giddy. ``I had to stop
and pull over because I was crying, because my little town was having a
protest. And I said, `Well look at that. That's new. That's new to
me.'''

``Quite frankly,'' she said, ``I didn't expect much from my town.''

For years now, mainly black people have been on the front lines of
issues that affect black people, said Adilka Pimentel, 30, a lead
organizer at Make the Road New York who identifies as black Dominican.

Ms. Pimentel has been involved in activism for a long time, since she
was 14 years old. She pointed out that with the Floyd protests, more
white people have the advantages of reliable health care, higher incomes
and savings to take to the streets at a time when black people have been
especially hard hit by the coronavirus outbreak.

``The same way that essential workers are mostly black and brown and
account for most of the deaths of Covid, they can't be out there because
they have to feed their families,'' she said.

She said she realized that social justice movements ebb and flow, and
hoped that the new protesters remained part of the movement.

``I worry about all the support dying down mostly because it's what
happens. Eric Garner. It died down. Mike Brown. It died down. Ferguson.
It died down,'' Ms. Pimentel said. ``The hope is that it stays. Those of
us who have been doing the work are going to continue to do the work. If
we feel like it starts to slip, we can be here to pick it up.''

Ms. Patton, the protest organizer, stood on 125th Street in Harlem
recently at yet another gathering she had organized, this one to
recognize Breonna Taylor, who was killed by police in Louisville, Ky.

As she looked over the crowd and prepared to welcome them, a white man,
a stranger, handed her a megaphone.

``Could the white man who brought this help us figure it out?'' she
asked, laughing. The crowd laughed with her.

The man walked up and hit a button to amplify her voice.

Ms. Patton put the megaphone to her mouth. The crowd had grown to
hundreds in just a few minutes.

``I am so overwhelmed at how many of you came out!'' she shouted.
``Thank you for coming!''

Advertisement

\protect\hyperlink{after-bottom}{Continue reading the main story}

\hypertarget{site-index}{%
\subsection{Site Index}\label{site-index}}

\hypertarget{site-information-navigation}{%
\subsection{Site Information
Navigation}\label{site-information-navigation}}

\begin{itemize}
\tightlist
\item
  \href{https://help.nytimes3xbfgragh.onion/hc/en-us/articles/115014792127-Copyright-notice}{©~2020~The
  New York Times Company}
\end{itemize}

\begin{itemize}
\tightlist
\item
  \href{https://www.nytco.com/}{NYTCo}
\item
  \href{https://help.nytimes3xbfgragh.onion/hc/en-us/articles/115015385887-Contact-Us}{Contact
  Us}
\item
  \href{https://www.nytco.com/careers/}{Work with us}
\item
  \href{https://nytmediakit.com/}{Advertise}
\item
  \href{http://www.tbrandstudio.com/}{T Brand Studio}
\item
  \href{https://www.nytimes3xbfgragh.onion/privacy/cookie-policy\#how-do-i-manage-trackers}{Your
  Ad Choices}
\item
  \href{https://www.nytimes3xbfgragh.onion/privacy}{Privacy}
\item
  \href{https://help.nytimes3xbfgragh.onion/hc/en-us/articles/115014893428-Terms-of-service}{Terms
  of Service}
\item
  \href{https://help.nytimes3xbfgragh.onion/hc/en-us/articles/115014893968-Terms-of-sale}{Terms
  of Sale}
\item
  \href{https://spiderbites.nytimes3xbfgragh.onion}{Site Map}
\item
  \href{https://help.nytimes3xbfgragh.onion/hc/en-us}{Help}
\item
  \href{https://www.nytimes3xbfgragh.onion/subscription?campaignId=37WXW}{Subscriptions}
\end{itemize}
