Sections

SEARCH

\protect\hyperlink{site-content}{Skip to
content}\protect\hyperlink{site-index}{Skip to site index}

\href{https://www.nytimes3xbfgragh.onion/section/politics}{Politics}

\href{https://myaccount.nytimes3xbfgragh.onion/auth/login?response_type=cookie\&client_id=vi}{}

\href{https://www.nytimes3xbfgragh.onion/section/todayspaper}{Today's
Paper}

\href{/section/politics}{Politics}\textbar{}How Americans Feel About the
Country Right Now: Anxious. Hopeful.

\url{https://nyti.ms/2NBKLg4}

\begin{itemize}
\item
\item
\item
\item
\item
\item
\end{itemize}

\begin{itemize}
\item
  \href{https://www.nytimes3xbfgragh.onion/live/2020/09/07/us/trump-vs-biden?action=click\&pgtype=Article\&state=default\&region=TOP_BANNER\&context=storylines_menu}{Election
  Updates}
\item
  \href{https://www.nytimes3xbfgragh.onion/interactive/2020/us/elections/election-states-biden-trump.html?action=click\&pgtype=Article\&state=default\&region=TOP_BANNER\&context=storylines_menu}{Paths
  to 270}
\item
  \href{https://www.nytimes3xbfgragh.onion/interactive/2020/08/31/us/politics/vote-by-mail-deadlines.html?action=click\&pgtype=Article\&state=default\&region=TOP_BANNER\&context=storylines_menu}{Voting
  by Mail}
\item
  \href{https://www.nytimes3xbfgragh.onion/interactive/2019/us/elections/2020-presidential-election-calendar.html?action=click\&pgtype=Article\&state=default\&region=TOP_BANNER\&context=storylines_menu}{Key
  Dates}
\item
  \href{https://www.nytimes3xbfgragh.onion/newsletters/politics?action=click\&pgtype=Article\&state=default\&region=TOP_BANNER\&context=storylines_menu}{Politics
  Newsletter}
\end{itemize}

Advertisement

\protect\hyperlink{after-top}{Continue reading the main story}

Supported by

\protect\hyperlink{after-sponsor}{Continue reading the main story}

\hypertarget{how-americans-feel-about-the-country-right-now-anxious-hopeful}{%
\section{How Americans Feel About the Country Right Now: Anxious.
Hopeful.}\label{how-americans-feel-about-the-country-right-now-anxious-hopeful}}

Exhausted and angry as they face a series of crises, many voters
nevertheless remain optimistic about the future, a poll shows, viewing
this moment as a chance for progress, one they can help shape.

\includegraphics{https://static01.graylady3jvrrxbe.onion/images/2020/06/26/us/politics/26mood-poll1/26mood-poll1-articleLarge.jpg?quality=75\&auto=webp\&disable=upscale}

By \href{https://www.nytimes3xbfgragh.onion/by/lisa-lerer}{Lisa Lerer},
\href{https://www.nytimes3xbfgragh.onion/by/elaina-plott}{Elaina Plott}
and \href{https://www.nytimes3xbfgragh.onion/by/lazaro-gamio}{Lazaro
Gamio}

\begin{itemize}
\item
  Published June 27, 2020Updated June 28, 2020
\item
  \begin{itemize}
  \item
  \item
  \item
  \item
  \item
  \item
  \end{itemize}
\end{itemize}

Amanda Vibelius, a stay-at-home mother in rural Arizona, is angry and
overwhelmed.

​Her father is diabetic, a condition that cost him work because of the
coronavirus. As cases skyrocket in her state, she's nervous about
allowing her 11-year-old daughter to join friends at the playground. And
she has warned her husband, a doctor, that if he contracts the virus,
she will kick him out of the house to quarantine.

But, like a striking number of frustrated Americans, Ms. Vibelius says
she is also hopeful. A Republican-turned-independent, who is ``leaning
more and more Democrat every day,'' Ms. Vibelius thinks a rebound may
come quickly --- as long as President Trump loses in November.

``It took too long to take precautions and it reopened too soon, and
that's why we're getting these spikes,'' she said. The country will come
back, she said, ``when we get rid of the current administration.''

Nearly six months after the first case of coronavirus reached the United
States, a majority of registered voters say they are anxious, exhausted
and angry, according to a poll by The New York Times and Siena College.
Yet even as they brace themselves for months of challenges from the
virus, many remain optimistic about the country's future, viewing this
moment of pandemic, economic devastation and social unrest as an
opportunity for progress --- one they can help shape.

The poll and follow-up interviews with respondents reveal an electorate
acutely attuned to the ways in which the health crisis and economic
hardships have seeped into their lives, and to the idea that the
political process --- and their vote --- might improve things. The usual
personality contests and ideological showdowns of presidential campaigns
have given way to immediate shocks, like losing a job or knowing someone
who died from Covid-19, and deciding whether to hold Mr. Trump
ultimately responsible.

For other voters, the decision is not so complicated: They are rejecting
the president because of his divisive rhetoric and his assault on
democratic norms.

The mood of the country has rarely been so enmeshed in the country's
politics. Nearly every four years, politicians try to energize
supporters by describing the presidential election as the most important
of their lifetime. For once, voters may actually agree.

``As Americans, I mean, for centuries, we've overcome things,'' said
Troy Howard, a general manager from Charlotte, N.C. ``And we will
overcome this. It's who we are.''

Mr. Howard said in an interview that he was frustrated about the current
state of the country but hopeful about the long run --- not least
because he thinks Mr. Trump will be beaten in November.

\includegraphics{https://static01.graylady3jvrrxbe.onion/newsgraphics/2020/06/24/june-poll/5075e913f5eec9f7fb63b1b42dcad6f1c42f48ff/t_1.png}
\includegraphics{https://static01.graylady3jvrrxbe.onion/newsgraphics/2020/06/24/june-poll/5075e913f5eec9f7fb63b1b42dcad6f1c42f48ff/t_2.png}
\includegraphics{https://static01.graylady3jvrrxbe.onion/newsgraphics/2020/06/24/june-poll/5075e913f5eec9f7fb63b1b42dcad6f1c42f48ff/t_3.png}
\includegraphics{https://static01.graylady3jvrrxbe.onion/newsgraphics/2020/06/24/june-poll/5075e913f5eec9f7fb63b1b42dcad6f1c42f48ff/t_4.png}
\includegraphics{https://static01.graylady3jvrrxbe.onion/newsgraphics/2020/06/24/june-poll/5075e913f5eec9f7fb63b1b42dcad6f1c42f48ff/b_1.png}
\includegraphics{https://static01.graylady3jvrrxbe.onion/newsgraphics/2020/06/24/june-poll/5075e913f5eec9f7fb63b1b42dcad6f1c42f48ff/b_2.png}
\includegraphics{https://static01.graylady3jvrrxbe.onion/newsgraphics/2020/06/24/june-poll/5075e913f5eec9f7fb63b1b42dcad6f1c42f48ff/b_3.png}
\includegraphics{https://static01.graylady3jvrrxbe.onion/newsgraphics/2020/06/24/june-poll/5075e913f5eec9f7fb63b1b42dcad6f1c42f48ff/b_4.png}
\includegraphics{https://static01.graylady3jvrrxbe.onion/newsgraphics/2020/06/24/june-poll/5075e913f5eec9f7fb63b1b42dcad6f1c42f48ff/h_1.png}
\includegraphics{https://static01.graylady3jvrrxbe.onion/newsgraphics/2020/06/24/june-poll/5075e913f5eec9f7fb63b1b42dcad6f1c42f48ff/h_2.png}
\includegraphics{https://static01.graylady3jvrrxbe.onion/newsgraphics/2020/06/24/june-poll/5075e913f5eec9f7fb63b1b42dcad6f1c42f48ff/h_3.png}
\includegraphics{https://static01.graylady3jvrrxbe.onion/newsgraphics/2020/06/24/june-poll/5075e913f5eec9f7fb63b1b42dcad6f1c42f48ff/h_4.png}
\includegraphics{https://static01.graylady3jvrrxbe.onion/newsgraphics/2020/06/24/june-poll/5075e913f5eec9f7fb63b1b42dcad6f1c42f48ff/sd_1.png}
\includegraphics{https://static01.graylady3jvrrxbe.onion/newsgraphics/2020/06/24/june-poll/5075e913f5eec9f7fb63b1b42dcad6f1c42f48ff/sd_2.png}
\includegraphics{https://static01.graylady3jvrrxbe.onion/newsgraphics/2020/06/24/june-poll/5075e913f5eec9f7fb63b1b42dcad6f1c42f48ff/sd_3.png}
\includegraphics{https://static01.graylady3jvrrxbe.onion/newsgraphics/2020/06/24/june-poll/5075e913f5eec9f7fb63b1b42dcad6f1c42f48ff/sd_4.png}
\includegraphics{https://static01.graylady3jvrrxbe.onion/newsgraphics/2020/06/24/june-poll/5075e913f5eec9f7fb63b1b42dcad6f1c42f48ff/sd_5.png}
\includegraphics{https://static01.graylady3jvrrxbe.onion/newsgraphics/2020/06/24/june-poll/5075e913f5eec9f7fb63b1b42dcad6f1c42f48ff/sd_6.png}
\includegraphics{https://static01.graylady3jvrrxbe.onion/newsgraphics/2020/06/24/june-poll/5075e913f5eec9f7fb63b1b42dcad6f1c42f48ff/s_1.png}
\includegraphics{https://static01.graylady3jvrrxbe.onion/newsgraphics/2020/06/24/june-poll/5075e913f5eec9f7fb63b1b42dcad6f1c42f48ff/s_2.png}
\includegraphics{https://static01.graylady3jvrrxbe.onion/newsgraphics/2020/06/24/june-poll/5075e913f5eec9f7fb63b1b42dcad6f1c42f48ff/s_3.png}
\includegraphics{https://static01.graylady3jvrrxbe.onion/newsgraphics/2020/06/24/june-poll/5075e913f5eec9f7fb63b1b42dcad6f1c42f48ff/s_4.png}
\includegraphics{https://static01.graylady3jvrrxbe.onion/newsgraphics/2020/06/24/june-poll/5075e913f5eec9f7fb63b1b42dcad6f1c42f48ff/s_5.png}
\includegraphics{https://static01.graylady3jvrrxbe.onion/newsgraphics/2020/06/24/june-poll/5075e913f5eec9f7fb63b1b42dcad6f1c42f48ff/s_6.png}

\paragraph{}

How Voters Feel About the State of America Today

◄

►

...

Each figure represents one poll respondent.

The shift in the national mood has been swift and striking. After years
of economic growth, only one-third of poll respondents give the economy
positive marks. The virus has become so far-reaching that nearly one in
five say they know someone who has died of it --- including one-third of
African-Americans, who have been disproportionately affected by the
virus. Fifty-seven percent of registered voters believe the worst of the
pandemic is yet to come.

Families that once debated educational choices now face discussions
about whether attending school will even be an option. Once-routine
trips to pick up a gallon of milk are loaded with the politics of
whether to wear a mask. Protests of police killings have injected new
and sometimes difficult discussions about race into daily conversations.

\hypertarget{-1}{%
\paragraph{}\label{-1}}

How Voters Feel About the Coronavirus

◄

►

...

Each figure represents one poll respondent.

At the same time, most voters describe themselves as optimistic about
America. Even as unemployment rates reach some of the highest levels
since the Great Depression, more than seven in 10 voters believe
economic conditions will be better in a year. Sixty-eight percent of
voters say they feel hopeful about the state of the country.

Many Republicans are angry, too, and hopeful that the country will
rebound within a year --- but they have very different perspective than
Democrats. Republicans largely believe the president's claims that the
virus is ``fading away'' and that skyrocketing cases are a result of
increased testing. The Times/Siena poll shows that expectations for the
pandemic break along partisan lines. More than three-quarters of
Democrats think the worst is still to come, a view shared by less than a
third of Republicans.

Even as cases surge in her home state, Sandra Derleth, 59, of Melbourne
Beach, Fla., said she thought the country ``overreacted'' to the virus
in the spring.

``We're overdoing a lot of precautions,'' said Ms. Derleth, who lost her
job as an administrative assistant at a local university. ``I just feel
like with any illness or disease or flu or bug there's going to be some
people that get it.''

Florida set
\href{https://www.nytimes3xbfgragh.onion/interactive/2020/us/florida-coronavirus-cases.html}{a
new daily record} for single-day coronavirus cases
\href{https://www.nytimes3xbfgragh.onion/2020/06/27/world/coronavirus-updates.html?action=click\&module=Spotlight\&pgtype=Homepage}{on
Saturday}, with the total number now exceeding 130,000 in the state.

``Once fall hits and once Trump gets re-elected and is pushing the
economy forward again, maybe we'll start to see some new jobs coming
up,'' said Ms. Derleth, who plans to vote for Mr. Trump again in
November.

As Americans mark days by death rates, protests and waves of illness,
the instability of the moment leaves open the possibility that public
opinion could shift before Election Day.

Already, sentiment splits sharply around partisan lines. More than
three-quarters of Biden supporters say they feel ``angry'' at the state
of the country right now, the Times/Siena poll shows, while only 47
percent of Mr. Trump backers say they feel the same. Nearly two-thirds
of Biden supporters say they feel ``scared'' about the state of the
country, compared to about half as many Trump backers who say the same.

Still, a consensus has emerged around the broad strokes the country must
take to combat the pandemic.

\includegraphics{https://static01.graylady3jvrrxbe.onion/images/2020/06/26/us/politics/26mood-poll2/merlin_173717493_10f16a4c-5fd0-413a-a546-905f1619932e-articleLarge.jpg?quality=75\&auto=webp\&disable=upscale}

Despite double digit unemployment, majorities across demographic groups
say the federal government's priority should be to contain the spread of
the virus, even if it hurts the economy. Younger voters and black voters
take the most stringent view of the social distancing rules, with more
than four in 10 saying the guidance is being lifted too quickly. Only
backers of Mr. Trump overwhelmingly believe government should prioritize
the economy.

More than three-quarters of registered voters say they always or mostly
wear a mask in public when they expect to be within six feet of another
person, including 60 percent who support Mr. Trump and 79 percent of
those under 30.

Men are more likely to go barefaced. Only 46 percent of men say they
always wear a mask, compared with 61 percent of women. The findings
confirm \href{https://psyarxiv.com/tg7vz}{academic research} suggesting
that men are more likely to opt out of wearing masks, believing them to
be ``not cool'' or ``a sign of weakness,'' even though men are at a
higher risk of dying from Covid-19 than women.

In the most heavily impacted states, voters feel even more strongly
about taking measures to stop the spread of the virus. A higher
percentage of voters in Arizona and Florida, where infections are
spiking, say restrictions don't go far enough and that businesses are
reopening too quickly.

Scott Bertoglio, 33, of suburban Phoenix, said he is considering not
sending his three young children back to school in the fall because he
was worried that the state government has failed to adequately implement
rules protecting their health, like mandating mask wearing.

``We've essentially been holed up at home,'' he said, taking a break
from watching a PowerPoint presentation in his home office. ``But
Arizona is not taking it seriously and the schools are saying we're
going to open.''

Still, the poll shows that voters overwhelmingly believe that any
economic pain stemming from the virus will be temporary. Even among
those living in a household with coronavirus-related job losses, 81
percent say they expect to find work within the next few months or have
already regained it.

Majorities or near majorities in six key swing states --- Arizona,
Florida, Michigan, North Carolina, Pennsylvania and Wisconsin --- feel
slightly more anxiety about the recovery. Largely facing higher
unemployment rates than much of the rest of the country, registered
voters in those states say the economy will take a long time to recover
once the virus is gone. Still, only about a third of those voters
support protests against coronavirus-related restrictions.

The economic and health impacts have fallen disproportionately on voters
of color. One-third of black voters and 21 percent of Hispanics say they
know someone who has died from the coronavirus, compared with only 16
percent of white voters.

\hypertarget{-2}{%
\paragraph{}\label{-2}}

How Voters Feel About the Economy

◄

►

...

Each figure represents one poll respondent.

Black and Hispanic voters also take a bleaker view of the country. Only
a quarter of black Americans and 34 percent of Hispanics describe
themselves as ``proud'' of the state of America today, a view shared by
nearly half of whites. More than eight in 10 black voters say they feel
exhausted, compared with 63 percent of whites.

Cherri Hampton, 62, a retiree from Milwaukee, said it was a ``sad time''
for the country, describing the world as in a state of unrest.

``Right now with Donald Trump being the leader of this country, we've
got to have a whole lot of prayer,'' she said, citing a general lack of
respect among Americans.

She said she planned to vote for Mr. Biden, though she wasn't totally
sure how she felt about him.

``We don't know who we can trust, that's the bad part,'' she said,
describing the mentality of much of the black community in her area.
``But I trust God. That's the only thing getting me through this.''

\hypertarget{our-2020-election-guide}{%
\section{Our 2020 Election Guide}\label{our-2020-election-guide}}

Updated ~Sept. 7, 2020

\begin{center}\rule{0.5\linewidth}{\linethickness}\end{center}

\begin{itemize}
\item ~
  \hypertarget{the-latest}{%
  \subsection{The Latest}\label{the-latest}}

  \begin{itemize}
  \item
    The unofficial Labor Day kickoff to the fall presidential campaign
    centered on Pennsylvania and Wisconsin,
    \href{https://www.nytimes3xbfgragh.onion/2020/09/07/us/politics/wisconsin-biden-harris-trump-pence.html?action=click\&pgtype=Article\&state=default\&region=BELOW_MAIN_CONTENT\&context=storylines_guide}{two
    pivotal states for both President Trump and Joseph R. Biden Jr}.
  \end{itemize}
\item ~
  \hypertarget{how-to-win-270}{%
  \subsection{How to Win 270}\label{how-to-win-270}}

  \begin{itemize}
  \item
    Joe Biden and Donald Trump need 270 electoral votes to reach the
    White House. Try building
    \href{https://www.nytimes3xbfgragh.onion/interactive/2020/us/elections/election-states-biden-trump.html?action=click\&pgtype=Article\&state=default\&region=BELOW_MAIN_CONTENT\&context=storylines_guide}{your
    own coalition of battleground states}~to see potential outcomes.
  \end{itemize}
\item ~
  \hypertarget{voting-by-mail}{%
  \subsection{Voting by Mail}\label{voting-by-mail}}

  \begin{itemize}
  \item
    Will you have enough time to vote by mail in your state? Yes, but
    it's risky to procrastinate.
    \href{https://www.nytimes3xbfgragh.onion/interactive/2020/08/31/us/politics/vote-by-mail-deadlines.html?action=click\&pgtype=Article\&state=default\&region=BELOW_MAIN_CONTENT\&context=storylines_guide}{Check
    your state's deadline.}
  \item
    \href{https://www.nytimes3xbfgragh.onion/interactive/2020/us/elections/joe-biden.html?action=click\&pgtype=Article\&state=default\&region=BELOW_MAIN_CONTENT\&context=storylines_guide}{}

    \hypertarget{joe-biden}{%
    \section{Joe Biden}\label{joe-biden}}

    \hypertarget{democrat}{%
    \subsection{Democrat}\label{democrat}}

    \href{https://www.nytimes3xbfgragh.onion/interactive/2020/us/elections/donald-trump.html?action=click\&pgtype=Article\&state=default\&region=BELOW_MAIN_CONTENT\&context=storylines_guide}{}

    \hypertarget{donald-trump}{%
    \section{Donald Trump}\label{donald-trump}}

    \hypertarget{republican}{%
    \subsection{Republican}\label{republican}}
  \end{itemize}
\item
  \hypertarget{keep-up-with-our-coverage}{%
  \subsection{Keep Up With Our
  Coverage}\label{keep-up-with-our-coverage}}

  \begin{itemize}
  \item
    Get an
    \href{https://www.nytimes3xbfgragh.onion/newsletters/politics?action=click\&pgtype=Article\&state=default\&region=BELOW_MAIN_CONTENT\&context=storylines_guide}{email}~recapping
    the day's news
  \item
    Download our mobile app on
    \href{https://apps.apple.com/us/app/nytimes/id284862083?ls=1\&mat_click_id=5c79ae7455014fd1bd66b5610c05b8f2-20191112-16948\&referrer=mat_click_id\%3D5c79ae7455014fd1bd66b5610c05b8f2-20191112-16948\%26link_click_id\%3D722930677036718082}{iOS}~and
    \href{http://a.localytics.com/android?id=com.nytimes.android\&referrer=utm_source\%3Dother_nyt_mobile_web\%26utm_medium\%3DWeb\%2520page\%26utm_term\%3DGeneral\%2520Mobile\%2520Page\%26utm_campaign\%3DNYT\%2520Mobile\%2520General\%2520Page}{Android}~and
    turn on Breaking News and Politics alerts
  \end{itemize}
\end{itemize}

Advertisement

\protect\hyperlink{after-bottom}{Continue reading the main story}

\hypertarget{site-index}{%
\subsection{Site Index}\label{site-index}}

\hypertarget{site-information-navigation}{%
\subsection{Site Information
Navigation}\label{site-information-navigation}}

\begin{itemize}
\tightlist
\item
  \href{https://help.nytimes3xbfgragh.onion/hc/en-us/articles/115014792127-Copyright-notice}{©~2020~The
  New York Times Company}
\end{itemize}

\begin{itemize}
\tightlist
\item
  \href{https://www.nytco.com/}{NYTCo}
\item
  \href{https://help.nytimes3xbfgragh.onion/hc/en-us/articles/115015385887-Contact-Us}{Contact
  Us}
\item
  \href{https://www.nytco.com/careers/}{Work with us}
\item
  \href{https://nytmediakit.com/}{Advertise}
\item
  \href{http://www.tbrandstudio.com/}{T Brand Studio}
\item
  \href{https://www.nytimes3xbfgragh.onion/privacy/cookie-policy\#how-do-i-manage-trackers}{Your
  Ad Choices}
\item
  \href{https://www.nytimes3xbfgragh.onion/privacy}{Privacy}
\item
  \href{https://help.nytimes3xbfgragh.onion/hc/en-us/articles/115014893428-Terms-of-service}{Terms
  of Service}
\item
  \href{https://help.nytimes3xbfgragh.onion/hc/en-us/articles/115014893968-Terms-of-sale}{Terms
  of Sale}
\item
  \href{https://spiderbites.nytimes3xbfgragh.onion}{Site Map}
\item
  \href{https://help.nytimes3xbfgragh.onion/hc/en-us}{Help}
\item
  \href{https://www.nytimes3xbfgragh.onion/subscription?campaignId=37WXW}{Subscriptions}
\end{itemize}
