Sections

SEARCH

\protect\hyperlink{site-content}{Skip to
content}\protect\hyperlink{site-index}{Skip to site index}

\href{https://www.nytimes3xbfgragh.onion/section/style}{Style}

\href{https://myaccount.nytimes3xbfgragh.onion/auth/login?response_type=cookie\&client_id=vi}{}

\href{https://www.nytimes3xbfgragh.onion/section/todayspaper}{Today's
Paper}

\href{/section/style}{Style}\textbar{}TikTok Is Shaping Politics. But
How?

\url{https://nyti.ms/3g6oHq1}

\begin{itemize}
\item
\item
\item
\item
\item
\item
\end{itemize}

\hypertarget{race-and-america}{%
\subsubsection{\texorpdfstring{\href{https://www.nytimes3xbfgragh.onion/news-event/george-floyd-protests-minneapolis-new-york-los-angeles?name=styln-george-floyd\&region=TOP_BANNER\&block=storyline_menu_recirc\&action=click\&pgtype=Article\&impression_id=9ca56230-f1cf-11ea-bed1-dd85a44a6708\&variant=undefined}{Race
and America}}{Race and America}}\label{race-and-america}}

\begin{itemize}
\tightlist
\item
  \href{https://www.nytimes3xbfgragh.onion/2020/09/04/nyregion/rochester-police-daniel-prude.html?name=styln-george-floyd\&region=TOP_BANNER\&block=storyline_menu_recirc\&action=click\&pgtype=Article\&impression_id=9ca56231-f1cf-11ea-bed1-dd85a44a6708\&variant=undefined}{How
  Police Handled Death of Daniel Prude}
\item
  \href{https://www.nytimes3xbfgragh.onion/2020/09/01/us/politics/trump-fact-check-protests.html?name=styln-george-floyd\&region=TOP_BANNER\&block=storyline_menu_recirc\&action=click\&pgtype=Article\&impression_id=9ca56232-f1cf-11ea-bed1-dd85a44a6708\&variant=undefined}{Trump
  Fact Check}
\item
  \href{https://www.nytimes3xbfgragh.onion/2020/08/30/us/portland-shooting-explained.html?name=styln-george-floyd\&region=TOP_BANNER\&block=storyline_menu_recirc\&action=click\&pgtype=Article\&impression_id=9ca56233-f1cf-11ea-bed1-dd85a44a6708\&variant=undefined}{Portland
  Shooting}
\item
  \href{https://www.nytimes3xbfgragh.onion/2020/08/30/us/breonna-taylor-police-killing.html?name=styln-george-floyd\&region=TOP_BANNER\&block=storyline_menu_recirc\&action=click\&pgtype=Article\&impression_id=9ca56234-f1cf-11ea-bed1-dd85a44a6708\&variant=undefined}{Breonna
  Taylor's Life and Death}
\end{itemize}

Advertisement

\protect\hyperlink{after-top}{Continue reading the main story}

Supported by

\protect\hyperlink{after-sponsor}{Continue reading the main story}

\hypertarget{tiktok-is-shaping-politics-but-how}{%
\section{TikTok Is Shaping Politics. But
How?}\label{tiktok-is-shaping-politics-but-how}}

Two researchers have studied political expression on the app since the
Musical.ly era. Here's what they found.

\includegraphics{https://static01.graylady3jvrrxbe.onion/images/2020/06/25/fashion/25TIKTOK-POLITICS-teens-protest/25TIKTOK-POLITICS-teens-protest-articleLarge.jpg?quality=75\&auto=webp\&disable=upscale}

By \href{https://www.nytimes3xbfgragh.onion/by/john-herrman}{John
Herrman}

\begin{itemize}
\item
  June 28, 2020
\item
  \begin{itemize}
  \item
  \item
  \item
  \item
  \item
  \item
  \end{itemize}
\end{itemize}

As a place where millions of young Americans perform and explore their
identities in public,
\href{https://www.nytimes3xbfgragh.onion/2020/08/03/technology/trump-tiktok-microsoft.html}{TikTok}
has become a prominent venue for ideological formation, political
activism and trolling. It has
\href{https://www.nytimes3xbfgragh.onion/2020/02/27/style/tiktok-politics-bernie-trump.html}{homegrown
pundits}, and despite its parent company's reluctance to be involved
with politics --- the service does not allow political ads ---~it has
attracted interest from campaigns. It is also a space where people can
be gathered and pressed into action quickly.

TikTok was instrumental in the organization of a
\href{https://www.nytimes3xbfgragh.onion/2020/06/21/style/tiktok-trump-rally-tulsa.html}{mass
false-registration drive} ahead of a Trump rally in Tulsa, Okla., where
many seats were unfilled. It has amplified footage of police brutality
as well as scenes and commentary from Black Lives Matter protests around
the world, with videos created and shared on the platform frequently
moving beyond it. They carry TikTok's distinctive and wide-ranging
audiovisual vernacular: often playfully disorienting, carefully edited,
arch and musical. It has been suggested by many, including The New York
Times,
\href{https://www.nytimes3xbfgragh.onion/2020/06/22/opinion/trump-protest-gen-z.html}{that
TikTok teens will save the world}.

The truth is more complicated. A team of researchers has been analyzing
political expression on TikTok since, well, before it was TikTok. While
nonusers of TikTok may think it's bursting onto the political stage
rather suddenly, and that it has something like a collective political
identity, the research gives a different picture.

It depicts a diverse, diffuse and not nearly united community of
millions of young people discovering the capabilities and limits of a
platform that is, despite its many similarities with predecessors, a
unique and strange place.

In an email exchange, Ioana Literat, an assistant professor of
communication and media at Teachers College, Columbia University, and
Neta Kligler-Vilenchik, an assistant professor of communication at the
Hebrew University of Jerusalem, discussed the characteristics of
political expression on TikTok and why it feels like a novel phenomenon.

This interview has been edited.

\textbf{The idea that TikTok is an engine for progressive young politics
is gaining some currency among people who don't use the platform. What
might outsiders be surprised to find on TikTok, in terms of youth
political expression? Is there anything resembling consensus?}

\textbf{Ioana Literat:} I've noticed this tendency recently, not only on
older social media like Twitter but also in the press. It plays into
larger debates about youth civic attitudes --- and especially youth
civic attitudes online --- which tend to verge between utopia and
dystopia.

On the one hand, youth are hailed (or tokenized --- think Greta Thunberg
and the Parkland youth) as the future of democracy, for whom political
expression comes easy. But on the other hand, people are worried about
how they don't show up at the polls, or fall prey to misinformation, or
don't care about newspapers anymore. And all of these are true; it's not
an either/or kind of situation.

\textbf{Neta Kligler-Vilenchik:} Extreme views, ranging from dystopian
to utopian, are voiced not only in regard to youth, but also in regard
to any media phenomenon that is significant and new. As early as
Socrates's concern that the written word would eradicate wisdom, every
new technology has been believed to either be our savior (the internet
will bring people around the world into one global community!) or our
doom (robots will make us all unemployed!).

To me, this continuity is quite reassuring, because it shows us that our
fears and hopes are not so much around the traits of the specific new
technology, rather they are broad societal fears and hopes that are
projected onto whatever technology is new and not yet understood. To
most of its adult commenters, TikTok is a big unknown.

\textbf{Dr. Literat:} In terms of youth political expression, while
there's a dynamic and influential liberal activist community on TikTok,
there's actually plenty of conservative political expression, and
\href{https://www.nytimes3xbfgragh.onion/2019/05/13/style/trump-tiktok.html}{pro-Trump
voices} definitely find an audience on the platform.

We found this to be true
\href{https://journals.sagepub.com/doi/10.1177/1461444819837571}{in our
early research on}\href{http://musical.ly/}{Musical.ly}, in the
aftermath of the 2016 election, and it's still true today on TikTok, as
we're gearing up for the 2020 election. On TikTok, you can find powerful
political statements and activist organizing. You can find young people
lip-syncing speeches by Trump or Obama (both earnestly and
sarcastically). You can also find plenty of racist and sexist content,
conspiracy theories and misinformation, and kids showing off their gun
collections and posing with Confederate flags.

It's hard to refer to what we see on the platform as consensus. Rather,
we find that TikTok enables collective political expression for youth
--- that is, it allows them to deliberately connect to a like-minded
audience by using shared symbolic resources.

\textbf{Dr. Kligler-Vilenchik:} Shared symbolic resources can be
physical (MAGA hats), visual (the closed fist for the Black Lives Matter
movement) or hashtags (\#alllivesmatter). TikTok-specific elements like
viral dances, popular soundtracks, etc. are also shared symbolic
resources that help facilitate connections and foreground the collective
aspects of youth political expression.

\textbf{Are there novel ways in which political conflict unfolds on
TikTok? It doesn't seem to be especially well suited to the sorts of
conflict we're familiar with on some older platforms.}

\textbf{Dr. Literat:} There's relatively little crosscutting political
talk (i.e. across partisan lines, with politically heterogeneous
others). And when it does happen, it's not very productive. It's still a
very polarized discussion of us v. them.

Something that's pretty special about TikTok in terms of both political
expression and political dialogue/conflict is that it's all filtered
through young people's personal identities and experiences. Political
dialogue on the platform is very personal, and youth will often state
diverse social identities --- e.g. Black, Mexican, L.G.B.T.Q., redneck,
country --- in direct relation to their political views.

Not to say that political talk on other social media platforms is not
personal, but having done comparative analyses, we're really struck by
just how front-and-center youth identities are on TikTok.

\textbf{Dr. Kligler-Vilenchik:} If we return to the idea of collective
political expression as the ability to speak to a like-minded audience
through shared symbolic resources, we see that this enables at least the
potential for a conversation across political views.

So, some users may choose to tag their video with \#bluelivesmatter and
speak to a certain audience. But they can also choose to tag their video
with \#blacklivesmatter, and that way reach a different audience, with a
different view. Often this is done ironically, as a parody of others'
views (e.g., a video tagged \#whitelivesmatter that goes on to explain
the idea of white privilege), but it may also be a way to spark
conversation between sides.

\textbf{Lastly, if you've been able to check in, have you noticed
anything surprising about youth expression on TikTok around BLM, racism
and policing in the last few weeks?}

\textbf{Dr. Literat:} The collective aspects of youth political
expression --- which materialize, for instance, in frequently used songs
like Childish Gambino's ``This Is America'' --- are very salient in the
context of BLM-related expression on TikTok.

Like hashtags, these songs function as connective threads among the
videos. At the same time, there is such a wide variety in terms of style
and ethos of expression, from anger to silliness to humor, from
confessionals to original songs to footage of protests to memes to
interviews or oral histories.

There's also a sense of generational awareness and generational
solidarity, which is connected to this concept of collective political
expression. On footage of protests, you see a lot of comments like ``Gen
Z is changing the world,'' ``our generation is so powerful,'' ``I love
our generation with all my heart'' --- which is really interesting
because generations, and especially terms like Gen Z or Gen Alpha, are
how outsiders (academics, commenters, brands, etc.) usually refer to
youth.

It may be that youth are reclaiming these terms to assert their agency,
or perhaps these larger societal discourses are seeping into youth
discourse too.

\textbf{Dr. Kligler-Vilenchik:} Looking at what's going on in the U.S.
right now from outside (I'm in Israel), I'm struck by how these same
hashtags are also used by people from outside the U.S. to support the
Black Lives Matter movement and also connect it to localized instances
of racism and anti-government protest.

In Israel, protests in solidarity with BLM were infused with the protest
of Ethiopian-origin Israelis who suffer from racial discrimination and
police brutality. This speaks to how TikTok enables young people to
connect a personalized political message to a broader political moment.

Advertisement

\protect\hyperlink{after-bottom}{Continue reading the main story}

\hypertarget{site-index}{%
\subsection{Site Index}\label{site-index}}

\hypertarget{site-information-navigation}{%
\subsection{Site Information
Navigation}\label{site-information-navigation}}

\begin{itemize}
\tightlist
\item
  \href{https://help.nytimes3xbfgragh.onion/hc/en-us/articles/115014792127-Copyright-notice}{©~2020~The
  New York Times Company}
\end{itemize}

\begin{itemize}
\tightlist
\item
  \href{https://www.nytco.com/}{NYTCo}
\item
  \href{https://help.nytimes3xbfgragh.onion/hc/en-us/articles/115015385887-Contact-Us}{Contact
  Us}
\item
  \href{https://www.nytco.com/careers/}{Work with us}
\item
  \href{https://nytmediakit.com/}{Advertise}
\item
  \href{http://www.tbrandstudio.com/}{T Brand Studio}
\item
  \href{https://www.nytimes3xbfgragh.onion/privacy/cookie-policy\#how-do-i-manage-trackers}{Your
  Ad Choices}
\item
  \href{https://www.nytimes3xbfgragh.onion/privacy}{Privacy}
\item
  \href{https://help.nytimes3xbfgragh.onion/hc/en-us/articles/115014893428-Terms-of-service}{Terms
  of Service}
\item
  \href{https://help.nytimes3xbfgragh.onion/hc/en-us/articles/115014893968-Terms-of-sale}{Terms
  of Sale}
\item
  \href{https://spiderbites.nytimes3xbfgragh.onion}{Site Map}
\item
  \href{https://help.nytimes3xbfgragh.onion/hc/en-us}{Help}
\item
  \href{https://www.nytimes3xbfgragh.onion/subscription?campaignId=37WXW}{Subscriptions}
\end{itemize}
