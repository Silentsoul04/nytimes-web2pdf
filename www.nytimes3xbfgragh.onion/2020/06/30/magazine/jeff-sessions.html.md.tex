The Fall of Jeff Sessions, and What Came After

\url{https://nyti.ms/3ghzX33}

\begin{itemize}
\item
\item
\item
\item
\item
\item
\end{itemize}

\includegraphics{https://static01.graylady3jvrrxbe.onion/images/2020/07/05/magazine/05mag-sessions-07/05mag-sessions-07-articleLarge.jpg?quality=75\&auto=webp\&disable=upscale}

Sections

\protect\hyperlink{site-content}{Skip to
content}\protect\hyperlink{site-index}{Skip to site index}

Feature

\hypertarget{the-fall-of-jeff-sessions-and-what-came-after}{%
\section{The Fall of Jeff Sessions, and What Came
After}\label{the-fall-of-jeff-sessions-and-what-came-after}}

The former attorney general is fighting for his political life in
Alabama's Senate race, in the shadow of a president who still despises
him.

Credit...Joshua Rashaad McFadden for The New York Times

Supported by

\protect\hyperlink{after-sponsor}{Continue reading the main story}

By \href{https://www.nytimes3xbfgragh.onion/by/elaina-plott}{Elaina
Plott}

\begin{itemize}
\item
  Published June 30, 2020Updated July 14, 2020
\item
  \begin{itemize}
  \item
  \item
  \item
  \item
  \item
  \item
  \end{itemize}
\end{itemize}

\hypertarget{listen-to-this-article}{%
\subsubsection{Listen to This Article}\label{listen-to-this-article}}

Audio Recording by Audm

\emph{To hear more audio stories from publishers like The New York
Times, download}
\emph{\href{https://www.audm.com/?utm_source=nytmag\&utm_medium=embed\&utm_campaign=outcast_jeff_sessions}{Audm
for iPhone or Android}.}

For several months before he fired
\href{https://www.nytimes3xbfgragh.onion/2020/07/14/us/politics/Election-primary-runoff-results.html}{Jeff
Sessions}, Donald Trump had telegraphed that his attorney general would
leave following the 2018 midterm elections. Still, Justice Department
aides were surprised when the call came quite literally the next
morning. At about 10 a.m., on Nov. 7, a few of them gathered in
Sessions's fifth-floor office as John Kelly, then Trump's chief of
staff, delivered the news. Sessions asked Kelly if he could at least
hold off until the end of the week. Kelly said he could not; it was
either resign now, or await a presidential tweet. So Sessions's
communications director pulled out her phone and tapped out a statement
from the notes she prepared the day before, just in case. ``Thank you
for the opportunity, Mr. President,'' it concluded. Two aides grabbed it
off the printer and carried it to the West Wing.

The previous three years had transpired for Jeff Sessions like a
malarial dream. There he was in early 2016, beaming from the campaign
stage in the Huntsville, Ala., suburb of Madison before a crowd of more
than 10,000, Trump's prized opening act, extolling the inception of a
``movement.'' There he was one year later in his dream job at the
Justice Department, one gear shy of skipping as he zagged through the
corridors of the West Wing, greeting old campaign and congressional
acquaintances as they settled into their quarters, ``like a kid in a
candy store,'' one former White House official recalled. And there he
was, just 22 days after his confirmation, issuing
\href{https://www.nytimes3xbfgragh.onion/2017/03/02/us/politics/jeff-sessions-russia-trump-investigation-democrats.html}{the
terse statement recusing himself} from any investigation his department
might undertake into charges that Russia had interfered in the 2016
presidential election --- the action that would send the dream spiraling
into still weirder territory.

\includegraphics{https://static01.graylady3jvrrxbe.onion/images/2020/07/05/magazine/05mag-sessions-08/05mag-sessions-08-articleLarge.jpg?quality=75\&auto=webp\&disable=upscale}

It had all happened with astonishing speed: the reports, in January
2017, that counterintelligence agents were investigating communications
between Michael Flynn, Trump's national security adviser, and the
Russian ambassador, Sergei Kislyak; Trump's conversation with the F.B.I.
director, James Comey, six days after Sessions's confirmation in which
Trump suggested Comey drop the investigation; the revelation that
Sessions, too, had met with Kislyak during the campaign, despite his
claims during his confirmation hearings, under oath, that he had not. On
March 2, Sessions appeared as briefly as possible before reporters to
announce that he would be recusing himself from the looming Russia
investigation. It was what virtually all Democrats, and some
Republicans, in Congress believed he should have done. It also left
Sessions a dead man walking in the halls of the White House that he had
so recently skipped through, the unwitting protagonist of the era's most
vivid cautionary tale about crossing Donald Trump.

It was in his hour of darkness, after his firing, that Sessions received
a call from Trent Lott. The former Republican senator from Mississippi
knew something about unceremonious downfalls, his tenure as Senate
majority leader cut short in 2002 following a toast to the past
presidential aspirations of one Strom Thurmond. (``If the rest of the
country'' had voted for Thurmond in 1948, when he ran on the
pro-segregation Dixiecrat ticket, Lott said, ``we wouldn't have had all
these problems over all these years, either.'') But Lott had rebounded
with ease, slinking back into Senate leadership before exiting politics
on his own terms and settling into the life of a lobbyist. He had since
acted as a kind of life coach for Senate friends --- Kit Bond of
Missouri, the late Arlen Specter of Pennsylvania --- who were
considering what might come after public service, and he suggested
Sessions come by his office for a talk.

When he did, Lott gave Sessions a copy of a visual aid he put together
several years earlier called ``The Wheel of Fortune.'' The wheel, Lott
told me, had a series of ``spokes,'' all of which represent things you
might do upon leaving politics. You could join a law firm! Give
speeches! Write a book! Many lawmakers became professors or sat on
corporate boards. Lott walked Sessions through the pros and cons of
each. And so Sessions left K Street that day encouraged anew by the wide
world before him.

The problem was that, as he commenced to spin the proverbial wheel as
advised, the wide world only seemed to narrow. Much of 2019 unfurled for
Sessions in a series of small indignities, a continual reminder that
Trump's disfavor could cast its shadow over even a man who won his
fourth Senate term entirely unopposed. According to three people
familiar with the matter, shortly after leaving the Justice Department,
Sessions entered talks to join the law firm Maynard Cooper \& Gale,
which was founded in Birmingham, Ala. With a longtime friend of
Sessions's pulling for him on the inside, the deal seemed all but done.
But ultimately, the firm's leadership decided against bringing him in,
the news of which was broken to Sessions over dinner at Charlie Palmer's
in Washington. ``People at Maynard obviously respect Jeff,'' said one
person with direct knowledge of the decision, speaking on the condition
of anonymity. ``But I don't think, given the manner in which he left''
the Justice Department, ``he could make the business case for how the
work would follow.'' (A spokeswoman for Maynard Cooper confirmed it had
been in talks with Sessions, but declined further comment.)

Another heave of the wheel. Sessions considered starting a think tank,
an institution that would endeavor to lend a scholarly heft to the
right-wing populism that he had long espoused and that was now
co-defined with Trump, but he was unable to find financing for the
project. At one point, he agreed to meet with agents about writing a
book about the Trump agenda, but decided against it.

One spoke still beckoned. ``And I have to confess, you know,'' Lott
recalled, ``without being asked, I said, `Let me just say right here at
the beginning: I hope that you will not think about running again for
the Senate. It's just not what it used to be.'''

\textbf{On a recent} June afternoon, after a long day of running for the
Senate, Jeff Sessions retired to a corner booth at a Ruby Tuesday in the
south Alabama town of Bay Minette. He wore a blue-and-white gingham
button shirt and gray slacks. His eyes were a touch bloodshot and
bleary. He ordered a glass of peach tea and, for the second time that
day, dessert. ``I don't know when I've had a pineapple upside-down
cake,'' he mused to the waitress, studying the menu. ``I don't have to
eat all of it, do I?''

The day, his first on the nonvirtual campaign trail since March, began
at Mac and Jerry's, a homespun breakfast spot in Robertsdale, where
Sessions seemed pleasantly surprised by the modest crowd awaiting him.
Ducking in from the rain, he placed his hands on his hips and looked
around for one private moment, like a birthday celebrant who couldn't
quite believe his guests had shown. ``At least three people, maybe four,
said: `Our whole family voted for you,''' he told me at Ruby Tuesday.
``I like to hear that.''

Opportunities for affirmation had been few since Sessions, who is now
73, declared his candidacy for his old Senate seat last November.
Despite early polls that showed him as the favorite, Sessions did not
anticipate an easy primary. The field was wide, and he hoped in part to
outspend his way to the top before moving on to what would likely be a
race in name only against Doug Jones, the Democrat who won a special
election for the seat in 2017. Instead, Sessions finished a narrow
second in the primary and, per Alabama's election rules, advanced to a
runoff against the former Auburn University football coach Tommy
Tuberville. The spread of the coronavirus delayed the election until
July 14. Polls have since showed Sessions trailing his opponent by as
many as 20 points.

Image

Sessions campaigning in Alabama in February.Credit...Vasha
Hunt/Associated Press

Sessions can probably thank Trump for this. The president remains more
popular in Alabama than in virtually any other state, and on March 10,
he endorsed Tuberville in a pair of tweets, calling him a ``REAL
LEADER.'' He has been increasingly vocal in his contempt for his former
attorney general, a contempt that seems to have only sharpened with
time. ``Jeff, you had your chance \& you blew it,'' Trump tweeted in
late May. ``Recused yourself ON DAY ONE (you never told me of a
problem), and ran for the hills. You had no courage \& ruined many
lives.'' There had been flashes of life for Sessions in recent weeks; a
few surveys, and Sessions's internal polling, showed him closing the gap
with Tuberville. Even so, he is still running behind a political novice
in a Republican primary runoff for a seat he held for two decades, the
loss of which would be tantamount to his final consignment to the
political abyss.

\emph{\href{https://www.nytimes3xbfgragh.onion/2020/02/10/magazine/alabama-republican-primary-senate.html}{{[}How
Alabama's Senate Primary Became a Trump Loyalty Contest{]}}}

Before voters, Sessions's voice can seem vaguely strained, flecked with
irritation, even, when unwinding the events of the past four years. It
is not so much that he is tired of rehashing his decision to recuse
himself, the D.O.J. regulations and whatnot that required it, though
undoubtedly that is part of it. Rather, he seems cosmically bewildered
as to how he got to this point: fighting for his political life just as
the Republican base appears more in thrall than ever to his brand of
conservatism, fielding questions about his loyalty to a president who
found acceptance in the G.O.P. establishment largely through Sessions.

If many elected Republicans ultimately came to support Trump out of
convenience or opportunism or fear, Sessions was --- is --- a true
believer. The Republican Party, and even Trump's own administration, are
littered with those who, when talking to reporters, squirm to telegraph
their \emph{great personal distaste} for the MAGA enterprise. Not
Sessions. Today, when cornered in Capitol corridors by reporters, most
G.O.P. lawmakers profess ignorance as to Trump's latest social-media
activity. But unlike the bulk of his former colleagues, the gentleman
from Alabama saw the tweet. He probably loved it too.

Even in his exile, perhaps no one is as eager as Sessions to hold forth
on why he likes Trump, why his party --- why the \emph{country} --- so
desperately needs him. Nearly every tangent in our two-hour conversation
eventually arrived at this view. At one point, we were discussing
Syria's descent into anarchy over the past decade. ``A banker I know
from Greece,'' Sessions said, ``he said you could go to Aleppo, you
could do business deals, you could even buy whiskey. Cross Assad, you're
in big trouble, but you could do business'' before the Arab Spring. ``He
said, `There's a difference between freedom and democracy. You need to
understand this.'''

Sessions continued: ``And you know who we want to run Syria? Assad. We
are hoping that somehow he can get back in control. And there was no
terrorism, no ISIS when he ran the place.'' (ISIS emerged in Syria under
Bashar al-Assad's rule, which, while diminished, is ongoing.) ``He'd
kill 'em. And if you didn't cross him, he wouldn't kill you. And he
protected Christians; they were a part of his coalition.''

Sessions referred back to an earlier moment in the conversation, when I
asked him how he considered his support of Trump from the standpoint of
his faith as an evangelical Christian. ``You asked how Christians could
support Trump,'' he said. Consider Egypt's Christian minority under
president Abdel Fattah el-Sisi, he said: ``It's not a democracy --- he's
a strongman, tough man, but he promised to protect them. And they
believed him, because they didn't want the Muslim Brotherhood taking
over Egypt. Because they knew they'd be vulnerable. They chose to
support somebody that would protect them. And that's basically what the
Christians in the United States did. They felt they were under attack,
and the strong guy promised to defend them. And he has.''

This reminded Sessions of the events of three days earlier, when U.S.
Park Police tear-gassed protesters in Washington to make way for Trump
as
he\href{https://www.nytimes3xbfgragh.onion/2020/06/02/us/politics/trump-walk-lafayette-square.html}{strode
to the front of St. John's Church}, the basement of which was set on
fire by rioters the night before. Stopping before the cameras, Trump
held up a Bible. (``Is that your Bible?'' one reporter asked. ``It's a
Bible,'' Trump responded.) ``He came out there with that Bible,''
Sessions said, pausing briefly to giggle, ``and so all the Episcopal
bishops said: `Ohhh! Horrible!' You know? But this was a defender of the
faith.'' He continued in a faux tone of dismay: `` `Ohhh, his heart's
not right. He shouldn't have held that Bible up. \ldots{}' Oh, that's
malarkey.'' Sessions rolled his eyes. ``Just a bunch of socialist
leftists.''

Here, then, was the central paradox of Sessions's plight. In ethos and
in substance, Sessions had long harbored the presentiments of Trumpism.
On immigration, trade and policing, the dusted-off rhetoric of ``law and
order,'' his stamp on the president's administration remains indelible.
And yet no figure has been more totally cast out of Trump's orbit.

Image

Sessions sitting before the Senate Judiciary Committee after being
nominated for a U.S. District Court judgeship by President Reagan in
1986.Credit...Terry Ashe/The LIFE Images Collection, via Getty Images

\textbf{It was Washington's} early rejection of Sessions that kindled
his political career to begin with. In 1986, Ronald Reagan nominated
Sessions, then a United States Attorney, to a federal district judgeship
in Alabama. During his confirmation hearings before the Senate Judiciary
Committee, a black assistant U.S. attorney testified that Sessions had
once called him ``boy'' (which Sessions denied) and said the Ku Klux
Klan was ``OK until I found out they smoked pot'' (which Sessions said
was a joke). Senators also questioned
\href{https://www.nytimes3xbfgragh.onion/2017/01/09/magazine/the-voter-fraud-case-jeff-sessions-lost-and-cant-escape.html}{Sessions
about his prosecution of three black civil rights activists, including
Albert Turner Jr.}, a former aide to Martin Luther King Jr. who helped
lead the march from Selma to Montgomery, for voter fraud in 1985. (After
the judge threw out several counts, the jury acquitted all three on the
rest.) Coretta Scott King and other civil rights leaders accused
Sessions of having deliberately targeted the defendants, and she urged
against his confirmation.

Sessions denied claims of unfair targeting and still stands by the case.
But taken together, the accusations were enough to make him the first
federal district court nominee in more than 30 years not to be
confirmed. And denying Sessions the critical vote needed to advance his
cause to the Senate floor was the Democratic senior senator from his own
state, Howell Heflin.

As for the events that followed, Sessions would never confess to
something so uncouth as revenge. But in 1996, when Heflin announced his
retirement, and Sessions announced his intentions not just to succeed
him, but, upon election, to vie for appointment to the very Judiciary
Committee that had spurned him, Sessions could not suppress a stray grin
when asked to reflect on the chance of it all. ``I don't know that
`vindication' is the word,'' he told The Montgomery Advertiser as he
settled into his new office, taking a seat for the first time in
Heflin's old chair, at Heflin's old desk. ``But there is a sense that
life is a wonderful thing and things do work out in the end if you keep
your head up and try to do right.''

Sessions often told reporters at the start of his Senate career that he
had no intention of being a ``potted plant'' while in office. He made
the most of his coveted seat on the Judiciary Committee --- where he
would eventually serve as ranking member --- occasionally asking
judicial nominees: ``Are you a member of the American Civil Liberties
Union, or have you ever been?'' Nineteen years before
\href{https://www.nytimes3xbfgragh.onion/2016/03/17/us/politics/supreme-court-merrick-garland-senate.html}{the
Senate majority leader Mitch McConnell blocked Merrick Garland's
nomination to the Supreme Court}, Sessions tried to upend Garland's
confirmation to the U.S. Court of Appeals for the District of Columbia
on the grounds that the seat itself was a ``rip-off'' to the taxpayer
--- exasperating even the committee's Republican chairman, Orrin Hatch
of Utah, who snapped at Sessions for ``playing politics with judges.''

Image

Sessions with Mitch McConnell on the Senate Judiciary Committee in
2001.Credit...Scott J. Ferrell/Congressional Quarterly, via Getty Images

But it was immigration that preoccupied Sessions above all else. He
fulminated about it in his floor speeches, often delivered on Friday
afternoons when most of his colleagues had long since flown home for the
weekend. In 2007, he led the opposition to George W. Bush's attempt at
immigration reform, calling it ``no illegal alien left behind.'' In
2013, as ranking member of the Senate Budget Committee, he called on
Republicans to quash the so-called Gang of Eight's bipartisan
immigration bill in favor of a ``humble and honest populism.'' ``The
same set of G.O.P. strategists, lobbyists and donors who have always
favored a proposal like the Gang of Eight immigration bill argue that
the great lesson of the 2012 election is that the G.O.P. needs to push
for immediate amnesty and a drastic surge in low-skill immigration,'' he
wrote in a
\href{https://www.washingtonexaminer.com/weekly-standard/sessions-to-republicans-gop-elite-view-on-immigration-is-nonsense}{memo}.
``This is nonsense.''

For these tirades, Sessions was largely written off by his colleagues as
a backbencher with fringe views and little influence. He earned admirers
in the conservative media, however, such as Laura Ingraham and Michelle
Malkin. National Review deemed him
``\href{https://www.nationalreview.com/2014/08/amnestys-worst-enemy-eliana-johnson/}{Amnesty's
Worst Enemy}.'' On the matters of refugees, civil rights and prison
reform, ``you knew exactly who he was,'' said Al Franken, the former
Democratic senator from Minnesota, who was friendly with Sessions during
his tenure. ``I mean, he took some really strange stances.''

And then, finally, those stances met their moment --- and their
candidate. Perhaps more than anything else in his political life,
Sessions treasures having been the first senator to endorse Trump, in
February 2016. He traveled the country as the Trump campaign's
national-security chairman, forming what felt like a preordained
relationship with the man he was certain God planned to use for good. He
helped craft the campaign's immigration platform and advised Trump on
whom to select as his running mate. His devotion was so total that, when
Trump won, Sessions was a ``shoo-in'' for whatever cabinet position he
wanted, according to a former senior White House official who helped
lead the transition. Attorney general was his one request.

Image

Sessions at Madison City Stadium, where he publicly endorsed
then-candidate Donald Trump on Feb. 28, 2016, in Madison,
Ala.Credit...Taylor Hill/WireImage, via Getty Images

\textbf{``Well, I'll say} this,'' Sessions told me: ``I was surprised at
how comfortable I felt about being attorney general.''

Sessions told me he was moved by the chance to act on his and Trump's
shared belief that the police were ``demoralized'' during the Obama
years. ``I said, `We're going to embrace this as our mission, we're
going to \emph{back} the police and we're going to \emph{reduce}
crime.''' He began laying the groundwork for a zero-tolerance policy for
illegal immigration, a crackdown on MS-13 gang members and a rollback of
the civil rights agenda advanced through the Justice Department during
the Obama years. But these efforts were still in their infancy when, in
March 2017, he made his fateful decision.

As Sessions still maintains, he believed that in recusing himself, he
was doing what anyone in his position would have been obligated to do.
``There's one term that he used to use a lot,'' recalled Rod Rosenstein,
who served as deputy attorney general under Sessions: `` `regular
order.' And what he meant by that was, let's make sure we figure out
what the rules are, and let's make sure we're following the rules, and
let's make sure we're not getting distracted by inappropriate political
considerations.''

But it was in the aftermath of his recusal that White House officials,
particularly those who had not worked on the campaign, were suddenly
enlightened to Trump's capacity for rage. ``It was really the first time
I think any of us had ever seen him really blow up,'' the former
official recalled. ``He was frustrated with press coverage of crowd
sizes --- yes, he was angry about that --- but he had never really
raised his voice or shouted. But I remember him really laying into
McGahn'' --- Don McGahn, then the White House counsel --- ``and
\emph{shouting}. It was very much like: `How did you let this happen?
How did this {[}expletive{]} happen?'''

At the time, Sessions had a small collection of friends and former
colleagues in the White House, including Stephen K. Bannon, the chief
executive of Trump's campaign and then his chief strategist in the
administration, who has called Sessions his mentor and once pushed him
to run for president. Bannon, as well as Reince Priebus, then the chief
of staff, got in touch with Sessions and advised him to make himself
scarce for a while, to lie low until Trump's attentions inevitably
shifted elsewhere.

But for once, they didn't. For a time, Trump kept his frustrations off
Twitter, his fixation on what he called ``the ultimate betrayal''
manifesting itself in venting sessions with aides instead. Officials
recalled how, after Robert Mueller's appointment as special counsel,
meetings about any number of unrelated issues were derailed the moment
Trump glanced at the television and saw a chyron related to the Russia
investigation.

Even some aides who agreed with Sessions's decision found themselves
sympathizing with the president's view that the existential terror of
the Mueller investigation would never have emerged were it not for
Sessions. (``I would have put him at the border if I'd known,'' Trump
would often mutter, referring to the Department of Homeland Security.
``I would have put him at the border.'') More awkward for these aides
was the digression into mockery of Sessions that sometimes followed.
Trump would deride his accent --- the slow drawl, the fact that he often
paused for several seconds, sometimes midconversation, to think through
his next words. Sessions also has a tendency to raise slightly up and
down on the balls of his feet while standing and talking, a small tic
onto which Trump gleefully latched.

Sessions's defenders in such moments were few. Bannon, who considered
Sessions to be Trump's most effective ally from a policy standpoint,
says he would try to press his case, to no avail. Some senators,
including Lindsey Graham, stressed to Trump that Sessions had had no
choice. Altogether silent, however, was Stephen Miller, a former
Sessions aide turned Trump adviser who by now had emerged as an
influential force in the White House, advancing the
immigration-restriction agenda he and Sessions shared. Officials I spoke
with had the impression that Miller at first retained affection for his
former boss, even if he disagreed with Sessions's decision to recuse
himself. Nevertheless, ``he was never going to get caught defending the
guy,'' a second former White House official said. ``He never wanted
Trump to view him as a `Sessions guy,' and so whenever it came up, he
just wouldn't talk. Sometimes it even seemed like he'd find a way to
leave the room.''

This appeared to stem in part from Jared Kushner's example. Trump's
son-in-law despised Sessions, who came to be the chief opponent of
Kushner's vision for criminal-justice reform, and at least once referred
to him to colleagues as a racist. It was Miller's correct understanding
early on that an alliance with Kushner was the ticket to longevity in
Trump's White House. But as Bannon pointed out to me: ``Stephen Miller
and the rest of the immigration gang would have gotten zero done were it
not for what Sessions did at D.O.J.''

Indeed, during the first two years of Trump's presidency, Sessions was
arguably more successful than anyone else in Trump's cabinet in
advancing the president's professed goals. If anything, Sessions told
me, his only regret was not more forcefully advocating them. He
recounted the
\href{https://www.nytimes3xbfgragh.onion/2018/06/15/us/sessions-bible-verse-romans.html}{outrage}
over his use of Scripture to defend border agents
\href{https://www.nytimes3xbfgragh.onion/2019/07/16/magazine/immigration-department-of-homeland-security.html}{separating
migrant children from their families}, calling it ``totally
ridiculous.'' ``I was right about that,'' he said. ``I wish I'd fought
it.'' Then, in a disturbing, guttural voice, he mocked much of the
nation's reaction: ``Nooooo, this is a poor child! They just want a
job!'' From law enforcement to immigration to the war on drugs,
Sessions's conviction that the Obama administration had coddled
criminals motivated much of his agenda. And unlike many of the
president's appointees, Sessions ``actually understood what the levers
of power were to effect change,'' said Vanita Gupta, who led the Civil
Rights Division during Obama's second term. ``So he was actually pretty
effective at killing big areas of some of the highest-profile work the
Civil Rights Division had been doing.''

Image

Sessions at the White House in 2018 during a meeting hosted by President
Trump on sanctuary cities.Credit...Manuel Balce Ceneta/Associated Press

Sessions reversed an Obama-era policy that specified protections for
transgender workers against discrimination under the Civil Rights Act of
1964. Under Sessions, the department also filed briefs in support of
states fighting court orders to curb potential voting rights
infringements such as voter-ID laws. On his last day in office, Sessions
formalized a policy that made it harder for the Justice Department to
enter into consent decrees with local governments --- policing reforms
enforced by a federal judge, which were a cornerstone of Barack Obama's
police-reform agenda and central to the role the federal government
played in police-brutality cases in Ferguson, Mo., Baltimore and
elsewhere.

The mantra was: ``Back to the men and women in blue,'' Sessions told me.
``The police had been demoralized. There was all the Obama --- there's a
riot, and he has a beer at the White House with some criminal, to listen
to him. Wasn't having a beer with the police officers. So we said,
`We're on your side. We've got your back, you got our thanks.''' (Asked
whether this was a confused reference to the
\href{https://www.nytimes3xbfgragh.onion/interactive/2020/02/03/magazine/henry-louis-gates-jr-interview.html}{meeting
Obama had with the scholar Henry Louis Gates Jr.}, who had been
wrongfully arrested entering his own home, and the police officer
involved in the arrest, a Sessions spokesman declined to elaborate.)

Sessions seemed annoyed when I asked if he would support measures to
reform law enforcement if he were re-elected. ``I \emph{suppose} we
could do a survey about police ---'' he began. He paused for nine
seconds and sighed, slumping slightly against the booth. ``And see how
they --- whether their training is at the highest level or not.'' A few
minutes later he returned to the subject: ``I think you should probably
have some money for actually training for riots,'' he said. ``That's
what really needs to be done. Not tell the police, `If you were just
more \emph{sensitive}, riots wouldn't occur.'''

He called Secretary of Defense Mark Esper ``immature'' for
\href{https://www.nytimes3xbfgragh.onion/2020/06/03/us/politics/esper-milley-trump-protest.html}{saying}
he did not support Trump's threat, amid the nationwide protests
following George Floyd's death, to invoke the Insurrection Act, which
allows a president to domestically deploy military troops to restore
order. ``Who cares what he thinks?'' Sessions said. ``The president can
ask for his advice, or not ask for it. There's one commander in chief of
the United States military, Mr. Secretary. Not Esper.'' It was every
civil servant's duty, he went on, to obey his or her commander with
enthusiasm, or quit. ``Who do you think runs this country?''

\textbf{One theory holds} that Sessions's extreme fealty to the
president was, in fact, what prolonged his problems with him. Sessions
was willing to endure Trump's personal derision in order to realize
their shared vision for the country. Trump, on the other hand, seemed
unnerved that anyone's policy goals could outweigh their pride. And so
with every sunny response to his insults, Trump's disdain for Sessions
deepened. ``So many people in the White House thought the way to build a
better relationship with Trump was just to agree with him on everything
and praise him to the hilt and be sycophantic and plug those gaping
insecurities that fuel his narcissism,'' the first former White House
official said. ``When the reality is that once you actually give in to
him like that, he detests you for it.'' (The White House did not respond
to multiple requests for comment.)

That dynamic has continued to plague Sessions in Alabama, where many
Republican voters will brook no dissent of Trump but also question a man
who appears disinclined to defend his own honor. On May 22, after Trump
excoriated Sessions yet again on Twitter (``Alabama, do not trust Jeff
Sessions. He let our Country down''), Sessions decided, for the first
time, to push back. ``Look, I know your anger, but recusal was required
by law. I did my duty \& you're damn fortunate I did,'' Sessions
tweeted. ``It protected the rule of law \& resulted in your exoneration.
Your personal feelings don't dictate who Alabama picks as their senator,
the people of Alabama do.'' All told, one campaign aide told me, the
composition of the tweet involved perhaps a dozen advisers and
approximately 100 emails.

Sessions's former colleagues, caught up in their own delicate dances
with Trump, apparently see little upside to encouraging Sessions
publicly, or even discussing their friendship with him. Of the many
Republican senators I reached out to for this article, only Richard
Shelby, Sessions's old colleague in Alabama's Senate delegation, agreed
to talk. ``I think Alabama would do well by sending him back, but you
know, that's ultimately up to the people,'' Shelby told me. ``We'll see
what happens in July. I have no idea.''

In the past four months, meanwhile, Trump and Tuberville have spoken
frequently by phone, sometimes as often as twice a week. In mid-June,
Tuberville joined the president on Air Force One when it landed in
Dallas. When we spoke at Ruby Tuesday, Sessions acknowledged
Tuberville's appeal. College-football coaches, particularly in the
Southeastern Conference, know how to pitch, how to sit in a living room
with a skeptical recruit and his family and sell them on a future. And
Tuberville's pitch now, as Sessions, describing one recent campaign ad,
characterizes it, is as follows: `` `I support Donald Trump. God sent
Donald Trump, and I'm going up there and I'm going to do something. I'm
strong; I yelled at the referee.''' (Tuberville's ad used a clip of him,
in his Auburn days, berating an official on the field.) ``Well, people
like that. That's a Trump --- a Trump \emph{thing}.''

But in Sessions's eyes, Tuberville is poised to be yet another
Republican who claims to support Trump in public while actively working
against Trumpism. Congress is full of them now, Sessions says, lawmakers
nursing an ``ideological obsession'' with free markets and free trade
and open borders. ``Like Tommy Tuberville says, `I'm 100 percent free
market, I don't believe in tariffs. \ldots{}' There are a lot of
Republican senators that believe that. Some of 'em have probably
\emph{said} it. Most of them are too devious and gutless to say it.''

``Our moral duty is to citizens of the U.S.,'' Sessions said, picking up
the theme again later in the conversation. ``Nation-states are
\emph{not} gone, they're not out of date. America is not an \emph{idea},
Paul Ryan --- it's a nation.'' He began to bang his fists as he spoke,
sending the silverware and ice in his peach tea aquiver. ``It's a
\emph{secular nation-state.} It has'' --- another bang ---
``\emph{rules}.''

I had pointed out earlier that Trump's hatred of Sessions stemmed from
Sessions's following the rules. ``Well, he's not a lawyer --- he's a
doer,'' Sessions replied. ``I knew that when I signed on. But he's been
law-and-order for the most part, about supporting police.''

``You get to pick and choose in what areas?'' I ventured.

``I didn't expect him to be perfect. Nobody's perfect. He's new to
Washington; he's not a lawyer. He has less confidence in this legal
system than I do, I'll acknowledge that.''

``Did you think he had confidence in you?'' I asked.

``Mm-hmm,'' Sessions said. He paused to eat a forkful of pineapple cake.
``He thinks what was done to him was wrong, and he thinks I could have
stopped it. And he's not interested in details.'' He blotted his mouth
with a napkin and laughed.

\textbf{Sessions's current} existential tremors are not limited to
regret for losing the president. Trump's show of force in Washington in
early June, and the police crackdowns on protesters in cities across the
country, were a maximal expression of the law-and-order vision advanced
by Sessions. But as the nation reckons with that vision, it is difficult
to deduce any great rallying around it; even Republicans voters, who
broadly do not support the Black Lives Matter movement, are more likely
to support it now than they were before the protests following George
Floyd's death, according to recent polling. Trump left Sessions, yes,
but there seems about the candidate a dim unease that his country may
have left him, too.

And so, having spent the past hour and a half recounting scenes from his
life that perhaps did not make sense to him, he suddenly seemed anxious
to anchor himself in the few that still did. Toward the end of the
conversation, he commenced upon recollections of Camden, Ala., where he
grew up. ``It was an idyllic period,'' he said. ``Sort of a window. End
of an age.''

I had driven to Camden the week before, a Black Belt town of some 2,000
people just off the Alabama River. When I arrived, I met Fleet
Hollinger, a childhood friend of Sessions's. In his black pickup, a
heavy rain thrashing against the windshield, we steered through the
sliver of downtown where the two movie theaters once stood, where,
Hollinger recalled, you could catch a Gene Autry double feature for a
dime. About 10 miles down the road was the Sessions family home, the
slightly pitched roof and kerosene heater and mildew creeping up the
front-porch screen. There was Bell's Landing Presbyterian, est. 1819,
behind which his parents are buried. For Sessions and his friends,
summer Sundays began with the morning service there and ended in the
swimming hole nearby. ``Nobody didn't really have anything, but we
didn't know it,'' Hollinger said. ``We were happy like we were.''

As Sessions tells it, life in Camden was ordered and disciplined and
dependable. No wheels of fortune spinning in the wind. He careened from
one reminiscence to the next --- going barefoot to school, the beautiful
Beth Jones, Miss Watson's trigonometry class. At one point, his
communications director had nudged him to wrap up. But 20 minutes later,
Sessions was still there, seemingly in a daze, plucking at footnotes
from a past life. ``It was segregated,'' he acknowledged, ``so we had
those, we had advantages from \ldots{} '' He trailed off. ``I don't
know, I don't know,'' he said, his voice barely above a mumble, trying
to articulate what had made that time so singular. ``I'm at a loss,
actually. I haven't quite got --- figured that out yet.''

He seemed quietly desperate to reaffirm the conviction borne out of his
upbringing, that in politics as in Eagle Scouts, there is still regular
order, and all things work together for the good of those who follow it.
``Like, well, a federal retirement will pay me just about as much as
being a senator,'' he said. ``I've got 10 grandchildren and they're all
doing well. I've got a home in Alabama, a place in the country to hide
out in if I need to. How much better can it get than this?''

``So I don't care what they say,'' he went on. Then, with a faint laugh:
``Sometimes I don't.''

Advertisement

\protect\hyperlink{after-bottom}{Continue reading the main story}

\hypertarget{site-index}{%
\subsection{Site Index}\label{site-index}}

\hypertarget{site-information-navigation}{%
\subsection{Site Information
Navigation}\label{site-information-navigation}}

\begin{itemize}
\tightlist
\item
  \href{https://help.nytimes3xbfgragh.onion/hc/en-us/articles/115014792127-Copyright-notice}{©~2020~The
  New York Times Company}
\end{itemize}

\begin{itemize}
\tightlist
\item
  \href{https://www.nytco.com/}{NYTCo}
\item
  \href{https://help.nytimes3xbfgragh.onion/hc/en-us/articles/115015385887-Contact-Us}{Contact
  Us}
\item
  \href{https://www.nytco.com/careers/}{Work with us}
\item
  \href{https://nytmediakit.com/}{Advertise}
\item
  \href{http://www.tbrandstudio.com/}{T Brand Studio}
\item
  \href{https://www.nytimes3xbfgragh.onion/privacy/cookie-policy\#how-do-i-manage-trackers}{Your
  Ad Choices}
\item
  \href{https://www.nytimes3xbfgragh.onion/privacy}{Privacy}
\item
  \href{https://help.nytimes3xbfgragh.onion/hc/en-us/articles/115014893428-Terms-of-service}{Terms
  of Service}
\item
  \href{https://help.nytimes3xbfgragh.onion/hc/en-us/articles/115014893968-Terms-of-sale}{Terms
  of Sale}
\item
  \href{https://spiderbites.nytimes3xbfgragh.onion}{Site Map}
\item
  \href{https://help.nytimes3xbfgragh.onion/hc/en-us}{Help}
\item
  \href{https://www.nytimes3xbfgragh.onion/subscription?campaignId=37WXW}{Subscriptions}
\end{itemize}
