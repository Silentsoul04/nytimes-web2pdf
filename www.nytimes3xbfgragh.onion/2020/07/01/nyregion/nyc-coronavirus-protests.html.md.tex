Sections

SEARCH

\protect\hyperlink{site-content}{Skip to
content}\protect\hyperlink{site-index}{Skip to site index}

\href{https://www.nytimes3xbfgragh.onion/section/nyregion}{New York}

\href{https://myaccount.nytimes3xbfgragh.onion/auth/login?response_type=cookie\&client_id=vi}{}

\href{https://www.nytimes3xbfgragh.onion/section/todayspaper}{Today's
Paper}

\href{/section/nyregion}{New York}\textbar{}Did Floyd Protests Lead to a
Virus Surge? Here's What We Know

\begin{itemize}
\item
\item
\item
\item
\item
\item
\end{itemize}

\hypertarget{race-and-america}{%
\subsubsection{\texorpdfstring{\href{https://www.nytimes3xbfgragh.onion/news-event/george-floyd-protests-minneapolis-new-york-los-angeles?name=styln-george-floyd\&region=TOP_BANNER\&block=storyline_menu_recirc\&action=click\&pgtype=Article\&impression_id=555e01a0-f4ba-11ea-b9f9-d5e9868d2682\&variant=undefined}{Race
and America}}{Race and America}}\label{race-and-america}}

\begin{itemize}
\tightlist
\item
  \href{https://www.nytimes3xbfgragh.onion/2020/09/11/us/black-police-chiefs-reform.html?name=styln-george-floyd\&region=TOP_BANNER\&block=storyline_menu_recirc\&action=click\&pgtype=Article\&impression_id=555e01a1-f4ba-11ea-b9f9-d5e9868d2682\&variant=undefined}{Black
  Police Chiefs}
\item
  \href{https://www.nytimes3xbfgragh.onion/2020/09/04/nyregion/rochester-police-daniel-prude.html?name=styln-george-floyd\&region=TOP_BANNER\&block=storyline_menu_recirc\&action=click\&pgtype=Article\&impression_id=555e01a2-f4ba-11ea-b9f9-d5e9868d2682\&variant=undefined}{What
  Happened in Rochester, N.Y.}
\item
  \href{https://www.nytimes3xbfgragh.onion/2020/08/30/us/portland-shooting-explained.html?name=styln-george-floyd\&region=TOP_BANNER\&block=storyline_menu_recirc\&action=click\&pgtype=Article\&impression_id=555e01a3-f4ba-11ea-b9f9-d5e9868d2682\&variant=undefined}{Portland
  Shooting}
\item
  \href{https://www.nytimes3xbfgragh.onion/2020/08/30/us/breonna-taylor-police-killing.html?name=styln-george-floyd\&region=TOP_BANNER\&block=storyline_menu_recirc\&action=click\&pgtype=Article\&impression_id=555e01a4-f4ba-11ea-b9f9-d5e9868d2682\&variant=undefined}{Breonna
  Taylor's Life and Death}
\end{itemize}

Advertisement

\protect\hyperlink{after-top}{Continue reading the main story}

Supported by

\protect\hyperlink{after-sponsor}{Continue reading the main story}

\hypertarget{did-floyd-protests-lead-to-a-virus-surge-heres-what-we-know}{%
\section{Did Floyd Protests Lead to a Virus Surge? Here's What We
Know}\label{did-floyd-protests-lead-to-a-virus-surge-heres-what-we-know}}

Epidemiologists have braced for a surge of coronavirus cases. But it has
not come yet.

\includegraphics{https://static01.graylady3jvrrxbe.onion/images/2020/06/30/nyregion/00nyvirus-protests1/merlin_173138814_cd9f862d-4a26-47d5-b9f1-7bb40d6a3dc8-articleLarge.jpg?quality=75\&auto=webp\&disable=upscale}

\href{https://www.nytimes3xbfgragh.onion/by/joseph-goldstein}{\includegraphics{https://static01.graylady3jvrrxbe.onion/images/2018/07/16/multimedia/author-joseph-goldstein/author-joseph-goldstein-thumbLarge.png}}

By \href{https://www.nytimes3xbfgragh.onion/by/joseph-goldstein}{Joseph
Goldstein}

\begin{itemize}
\item
  July 1, 2020
\item
  \begin{itemize}
  \item
  \item
  \item
  \item
  \item
  \item
  \end{itemize}
\end{itemize}

For more than two months, the authorities had been urging New Yorkers to
stay indoors and keep their distance from others. But after the police
killed George Floyd in Minneapolis, tens of thousands of New Yorkers
poured into the streets, day and night, to
\href{https://www.nytimes3xbfgragh.onion/2020/07/20/us/politics/portland-federal-agents-trump.html}{protest
police brutality} and racism.

Epidemiologists braced for a surge of new
\href{https://www.nytimes3xbfgragh.onion/2020/07/21/health/coronavirus-infections-us.html}{coronavirus
cases}. But it has not come yet.

On May 27, the day before the protests began in New York City, some 754
Covid-19 cases were diagnosed, according to the city's Department of
Health. That was the last time the city recorded more than 700 cases on
a single day.

By the end of the first week of protests, the city was recording
slightly more than 500 cases a day. By the end of the second week of
protests, the case counts were in the low 400s or high 300s a day.
They've continued to drop slightly. According to revised numbers the
city released on Wednesday, the last time New York City recorded more
than 300 cases was on June 23.

``We've been looking very closely at the number of positive cases every
day to see if there is an uptick in the context of the protests,'' Ted
Long, executive director of the city's contact tracing program, said.
``We have not seen that.''

In interviews, several epidemiologists expressed either surprise or
relief, and offered theories for what occurred. This is what we know:

\hypertarget{the-virus-spread-in-new-york-city-was-already-slowing-down}{%
\subsection{The virus spread in New York City was already slowing
down.}\label{the-virus-spread-in-new-york-city-was-already-slowing-down}}

The lockdown enacted in March worked. By the end of May, when the
protests began, the virus was not as prevalent in New York as it had
been when the lockdown began.

``It seems we in New York City did achieve a substantial decrease in the
number of cases so that made the odds of encountering a case of Covid-19
in these protests quite low,'' said Wafaa El-Sadr, an epidemiology
professor at Columbia University.

Exactly how low is tough to gauge. Throughout June, somewhere between
10,000 and 35,000 New Yorkers per day were tested. The percentage of
coronavirus tests in New York City consistently turning up positive
\href{https://www1.nyc.gov/site/doh/covid/covid-19-data.page}{declined}
in June, from about 3 percent at the start to 2 percent. But New York
City has released little specific information about current hot spots or
clusters, or current infection rates among different age groups.

Some cities and states have made a point of testing demonstrators and
released their findings.

\includegraphics{https://static01.graylady3jvrrxbe.onion/images/2020/06/30/nyregion/00nyvirus-protests2/merlin_173725254_a9b1371f-c9a6-450b-9be8-db1aabafb1fa-articleLarge.jpg?quality=75\&auto=webp\&disable=upscale}

In Minnesota, an initiative that targeted demonstrators
\href{https://www.startribune.com/minnesota-virus-cases-decline-despite-mass-protest-fears/571421142/?refresh=true}{found
that 1.5 percent of them} tested positive. In Massachusetts,
\href{https://www.bostonglobe.com/2020/06/23/nation/baker-25-percent-protesters-tested-positive-coronavirus/}{fewer
than 3 percent} of protesters did. A positive test does not necessarily
mean a
\href{https://www.statnews.com/2020/06/08/viral-shedding-covid19-pcr-montreal-baby/}{person
is likely to still be contagious}; people can continue to test positive
for weeks after becoming ill and starting to recover.

In New York, Gov. Andrew M. Cuomo pledged in early June to
\href{https://www.nytimes3xbfgragh.onion/2020/06/07/world/coronavirus-updates-us-usa.html}{dedicate
15 testing sites} in the city exclusively to people who attended
protests. But a state Health Department spokesman said that data is not
yet available.

Kitaw Demissie, an epidemiologist and dean of SUNY Downstate School of
Public Health in Brooklyn, said it was possible that in areas with
accelerating outbreaks --- such as some southern and western states ---
the mass demonstrations could well play a role in the spread of the
virus.

\hypertarget{outdoor-transmission-is-more-rare}{%
\subsection{Outdoor transmission is more
rare.}\label{outdoor-transmission-is-more-rare}}

Conditions at the demonstrations may not have been conducive for
transmission, mainly because the protests occurred outdoors,
epidemiologists said.

The virus spreads far more
\href{https://www.medrxiv.org/content/10.1101/2020.02.28.20029272v2.full.pdf}{easily}
indoors than outdoors, and close contact indoors is believed to be
\href{https://www.medrxiv.org/content/10.1101/2020.04.04.20053058v1.full.pdf}{the
main}
\href{https://wwwnc.cdc.gov/eid/article/26/6/20-0495_article}{driver of
transmission}, epidemiologists say.
\href{https://doi.org/10.1101/2020.02.28.20029272}{One study} based on a
review of 110 cases in Japan found that the odds of transmission were
18.7 times higher in closed environments --- everything from fitness
studios to tents --- than in open-air environments. Another study
involving a review of thousands of cases in China found
\href{https://www.medrxiv.org/content/10.1101/2020.04.04.20053058v1.full.pdf}{only
a single instance} of outdoor transmission.

In Minnesota, where Mr. Floyd was killed, cases among young adults
climbed substantially over June. But officials said that
\href{https://www.twincities.com/2020/06/26/coronavirus-friday-update-cases-jump-by-493-as-five-more-minnesotans-die-of-covid-19/}{gatherings
in re-opened} bars were partly to blame.

The virus
\href{https://www.cdc.gov/coronavirus/2019-ncov/prevent-getting-sick/how-covid-spreads.html}{is
thought} to be spread primarily through the virus-laden droplets emitted
when a contagious person coughs, sneezes or talks. When outdoors, this
virus-laden air is more
\href{https://www.nytimes3xbfgragh.onion/2020/05/15/us/coronavirus-what-to-do-outside.html}{quickly
diluted} and carried away than it would be in a poorly ventilated room.
Because a
\href{https://www.statnews.com/2020/04/14/how-much-of-the-coronavirus-does-it-take-to-make-you-sick/}{certain
quantity of virus} is needed for an infectious dose, the dilution can
make a significant difference, epidemiologists say.

Another potential factor: Demonstrators were often on the move, marching
at a brisk pace. That may have promoted dilution and also spaced people
out from each other.

Image

Virus-laden droplets --- like the kind that are spewed when an infected
person is talking, singing, shouting or chanting --- are thought to be
the primary way the coronavirus is spread.Credit...Simbarashe Cha for
The New York Times

``This doesn't say that being in a crowd is not risky,'' said Howard
Markel, a physician and historian of medicine at the University of
Michigan who has written on past epidemics. He said that protesters in
New York may have just been ``incredibly lucky.''

He noted that outdoor crowds can accelerate the spread of respiratory
viruses ---
\href{https://www.nytimes3xbfgragh.onion/2020/04/04/us/coronavirus-spanish-flu-philadelphia-pennsylvania.html}{most
notoriously during a war bond parade} in Philadelphia during the 1918
influenza pandemic.

\hypertarget{most-protesters-wore-masks}{%
\subsection{Most protesters wore
masks.}\label{most-protesters-wore-masks}}

New York City's Health Department had gone so far as to urge protesters
not to chant or yell --- which can increase the likelihood of
transmission --- but to instead carry signs and consider bringing a
drum.

But while that bit of advice went largely unheeded, most protesters
adhered to another: Wear a mask.

Carlos Polanco, 21, from Brooklyn, who protested for 22 or 23 days
straight, often out front at protests with a bullhorn, said that
organizers made a point of bringing extra masks and distributing them to
demonstrators. Mr. Polanco, a rising senior at Dartmouth College, said
that he tried to wear a mask except when he was delivering a speech or
leading chants --- during which time he tried to keep his six feet of
distance from others, he said.

And many protesters complained when
\href{https://www.nytimes3xbfgragh.onion/2020/06/11/nyregion/nypd-face-masks-nyc-protests.html}{police
officers at protests did not wear masks}.

\hypertarget{we-could-still-see-a-wave-of-infections-tied-to-the-protests}{%
\subsection{We could still see a wave of infections tied to the
protests.}\label{we-could-still-see-a-wave-of-infections-tied-to-the-protests}}

Some scientists say it's still too early to tell how much transmission
occurred at the demonstrations in New York. One reason is that many
protesters were young adults --- a demographic in whom severe cases and
hospitalizations are less common. As a result, a rise in cases that
started within this demographic might remain undetected by public health
officials for longer.

``We don't know the impact. We'll see that in the next two weeks,''
Florian Krammer, a virologist at the Icahn School of Medicine at Mount
Sinai, said in an interview last week.

Moreover, city officials have instructed contact tracers not to ask new
Covid-19 patients if they attended protests, according
\href{https://www.thecity.nyc/coronavirus/2020/6/14/21290963/nyc-covid-19-trackers-skipping-floyd-protest-questions-even-amid-fears-of-new-wave}{to
a report in The City}, a nonprofit news organization.

And the protests continue. Hundreds gathered
\href{https://www.nytimes3xbfgragh.onion/2020/07/22/nyregion/occupy-city-hall-protest-nypd.html}{at
City Hall Park} over the past week,
\href{https://www.nytimes3xbfgragh.onion/2020/06/28/nyregion/occupy-city-hall-nyc.html}{to
demand deep cuts} to the New York Police Department budget. Some
protesters are camping out in sleeping bags or under tarps. The
gathering is drawing some comparisons to the Occupy Wall Street
encampment at nearby Zuccotti Park in 2011. Rarely remembered, a vicious
cough, called
``\href{https://www.nytimes3xbfgragh.onion/2020/06/28/nyregion/occupy-city-hall-nyc.html}{Zuccotti
lung},'' circulated around that encampment.

So far there has been no clear increase in patients in emergency rooms
complaining of pneumonia or flu-like symptoms --- a metric the city's
Health Department tracks as an early warning system for Covid-19
transmission.

But public health experts cautioned against drawing too much reassurance
from New York's experience. ``Like most every other aspect of this
pandemic the most predictable thing is the unpredictability,'' said
Professor Markel, the historian and physician from the University of
Michigan.

Advertisement

\protect\hyperlink{after-bottom}{Continue reading the main story}

\hypertarget{site-index}{%
\subsection{Site Index}\label{site-index}}

\hypertarget{site-information-navigation}{%
\subsection{Site Information
Navigation}\label{site-information-navigation}}

\begin{itemize}
\tightlist
\item
  \href{https://help.nytimes3xbfgragh.onion/hc/en-us/articles/115014792127-Copyright-notice}{©~2020~The
  New York Times Company}
\end{itemize}

\begin{itemize}
\tightlist
\item
  \href{https://www.nytco.com/}{NYTCo}
\item
  \href{https://help.nytimes3xbfgragh.onion/hc/en-us/articles/115015385887-Contact-Us}{Contact
  Us}
\item
  \href{https://www.nytco.com/careers/}{Work with us}
\item
  \href{https://nytmediakit.com/}{Advertise}
\item
  \href{http://www.tbrandstudio.com/}{T Brand Studio}
\item
  \href{https://www.nytimes3xbfgragh.onion/privacy/cookie-policy\#how-do-i-manage-trackers}{Your
  Ad Choices}
\item
  \href{https://www.nytimes3xbfgragh.onion/privacy}{Privacy}
\item
  \href{https://help.nytimes3xbfgragh.onion/hc/en-us/articles/115014893428-Terms-of-service}{Terms
  of Service}
\item
  \href{https://help.nytimes3xbfgragh.onion/hc/en-us/articles/115014893968-Terms-of-sale}{Terms
  of Sale}
\item
  \href{https://spiderbites.nytimes3xbfgragh.onion}{Site Map}
\item
  \href{https://help.nytimes3xbfgragh.onion/hc/en-us}{Help}
\item
  \href{https://www.nytimes3xbfgragh.onion/subscription?campaignId=37WXW}{Subscriptions}
\end{itemize}
