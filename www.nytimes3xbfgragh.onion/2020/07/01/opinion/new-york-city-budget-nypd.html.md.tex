Sections

SEARCH

\protect\hyperlink{site-content}{Skip to
content}\protect\hyperlink{site-index}{Skip to site index}

\href{https://myaccount.nytimes3xbfgragh.onion/auth/login?response_type=cookie\&client_id=vi}{}

\href{https://www.nytimes3xbfgragh.onion/section/todayspaper}{Today's
Paper}

\href{/section/opinion}{Opinion}\textbar{}Can the N.Y.P.D. Handle a \$1
Billion Cut? Yes

\url{https://nyti.ms/3dRRnl8}

\begin{itemize}
\item
\item
\item
\item
\item
\end{itemize}

Advertisement

\protect\hyperlink{after-top}{Continue reading the main story}

\href{/section/opinion}{Opinion}

Supported by

\protect\hyperlink{after-sponsor}{Continue reading the main story}

\hypertarget{can-the-nypd-handle-a-1-billion-cut-yes}{%
\section{Can the N.Y.P.D. Handle a \$1 Billion Cut?
Yes}\label{can-the-nypd-handle-a-1-billion-cut-yes}}

Protests and a fiscal crisis shine a new light on the biggest Police
Department budget in America.

\href{https://www.nytimes3xbfgragh.onion/by/mara-gay}{\includegraphics{https://static01.graylady3jvrrxbe.onion/images/2018/05/07/opinion/mara-gay-circular/mara-gay-circular-thumbLarge.png}}

By \href{https://www.nytimes3xbfgragh.onion/by/mara-gay}{Mara Gay}

Ms. Gay is a member of the editorial board.

\begin{itemize}
\item
  July 1, 2020
\item
  \begin{itemize}
  \item
  \item
  \item
  \item
  \item
  \end{itemize}
\end{itemize}

\includegraphics{https://static01.graylady3jvrrxbe.onion/images/2020/07/01/opinion/01Gay/merlin_135398274_931cf341-fb93-4a86-8044-3bd4a5fe2dbc-articleLarge.jpg?quality=75\&auto=webp\&disable=upscale}

In a city staring down \$9 billion in lost revenue, it is hardly
revolutionary to ask the Police Department to share the pain.

But this is New York, home to the largest and probably most powerful
Police Department in the country. For the city's police officials, an
agreement to cut nearly
\href{https://www.nytimes3xbfgragh.onion/2020/06/29/nyregion/nyc-budget-police.html}{\$1
billion} from the department is considered a radical act.

Police Commissioner Dermot Shea said on Tuesday ahead of the budget vote
that the cut presents an ``extreme challenge'' to keeping the city safe.
``It's probably going to impact people of color more than anyone else,''
Mr. Shea
\href{https://www.pix11.com/news/america-in-crisis/police-commissioner-weighs-in-on-clash-with-protesters-nypd-budget-cuts}{told}
WPIX.

The Police Department was not the only target of cuts. After years of
heady spending and a hiring spree of city workers under Mayor Bill de
Blasio, New York was left scrambling to balance its budget this year,
reeling from losses stemming from the coronavirus pandemic. An \$88.2
billion budget approved by the City Council late on Tuesday uses
billions in reserves to close the gap, along with deep cuts across the
board. The Department of Education has taken a hit of more than \$400
million in recent months. Mr. de Blasio has also warned that New York
may have to lay off 22,000 city workers in the coming months.

Against that backdrop, the agreed-upon cut to the Police Department is a
modest ask. The department's annual budget was \$6 billion, up from
nearly \$5 billion in 2015. New York has one police officer for every
162 residents, among the highest ratios in the nation, according to an
\href{https://www.vera.org/publications/what-policing-costs-in-americas-biggest-cities}{analysis}
by the Vera Institute of Justice last month.

In the midst of a national debate over how to reshape Police
Departments, and with the Department of Education and most other city
offices under a hiring freeze, the Police Department cuts agreed upon by
Mayor Bill de Blasio and the City Council speaker, Corey Johnson, are
timid. Saying they add up to \$1 billion is generous.

Under the budget agreement, one class of 1,163 new recruits would be
canceled, a saving of \$55 million. The cuts would reduce the size of
the force to about 35,000 --- which is still slightly larger than the
force the department had in 2015, when Mr. de Blasio and the City
Council added 1,300 officers.

The budget also calls for overtime pay to be reduced by more than \$350
million. But that spending has a tendency to rise over the course of a
year, regardless of what is budgeted, so these savings are far from
guaranteed.

The mayor's \$1 billion figure also includes more than \$133 million in
benefit savings that aren't part of the Police Department's operating
budget. And it includes the nearly \$350 million the department spends
on roughly 7,500 school safety agents and crossing guards. The budget
agreement moves these workers from the Police Department to the
Department of Education.

The budget cuts won't bring the kind of systemic changes in policing
demanded by protesters and others in New York and in cities like
\href{https://www.nytimes3xbfgragh.onion/2020/06/07/us/minneapolis-police-abolish.html}{Minneapolis}.
That larger debate, over whether and how New York should fundamentally
remake its police force, is almost certain to be a central focus of the
mayor's race next year to replace Mr. de Blasio, who is term-limited. In
a preview of just how complicated that debate could be, the Council's
Black, Latino and Asian Caucus has resisted pulling officers from the
neighborhoods it represents because of longtime concerns over crime,
while also supporting larger efforts at reform.

For the activists camped outside City Hall, though, and others who want
more sweeping changes, the cuts to the Police Department in this year's
budget don't go nearly far enough. But they are significant. They
represent a rejection of the department's once unquestioned political
power in New York.

For more than two decades, the New York Police Department has enjoyed
outsized influence in city government, its ballooning budgets nurtured
by mayors of both parties who feared a return to the ``bad old days'' of
the 1970s and the crack epidemic of the early 1990s. New York had become
one of the safest cities in the country, and the Police Department
received most of the credit.

The Sept. 11 attack intensified the effect, drawing federal dollars that
helped
\href{https://www.nytimes3xbfgragh.onion/2011/12/05/nyregion/drone-submarines-add-eyes-for-nyc-harbor-police.html}{militarize}
the department. Over the next decade, the department conducted
\href{https://www.nytimes3xbfgragh.onion/2017/03/06/nyregion/nypd-spying-muslims-surveillance-lawsuit.html}{surveillance}
on Muslim communities. It also made more than
\href{https://www.nytimes3xbfgragh.onion/2019/11/17/nyregion/bloomberg-stop-and-frisk-new-york.html}{five
million street stops} between 2002 and 2013, part of the practice known
as stop-and-frisk that overwhelmingly targeted Black and Hispanic boys
and men. Police officials defended stop-and-frisk for years, saying it
was necessary to get guns off the street. But nearly 90 percent of the
stops resulted in no arrests, and when the number of stops plummeted,
crime continued to decline.

In the past decade, the biggest challenge facing the city was housing,
rather than crime, which remained at historic lows. But the Police
Department's operating budget continued to grow. The nonpartisan
Citizens Budget Commission
\href{https://cbcny.org/research/seven-facts-about-nypd-budget}{estimates}
that total city spending on the police, including pensions and health
care benefits, is about \$11 billion a year. The department's
aggressive, often violent, response to peaceful protests in recent
months made it impossible to ignore what too many New Yorkers already
knew firsthand: that the same Police Department that helped sustain
years of record-low crime was also too big and too unaccountable to the
people it served.

In the end, it took sustained nationwide protests over police brutality,
along with the worst fiscal crisis in a generation, before New York was
willing to even consider a different approach.

Given the grim economic outlook, New York may soon be forced to
reimagine other parts of its government as well. Even if it imposes
serious fiscal discipline, as many cities are, it is almost certain to
need to borrow money to meet its obligations. Because of controls put in
place in the wake of
\href{https://www.nytimes3xbfgragh.onion/2017/05/05/books/review/fear-city-new-york-fiscal-crisis-kim-phillips-fein.html}{the
1970s fiscal crisis}, the city cannot do so without approval from
Albany. The State Legislature and Gov. Andrew Cuomo should give the city
that borrowing authority. It is the best way to avoid austerity measures
sure to cause long-term harm to New York and its most vulnerable
communities, which have already been ravaged by the coronavirus.

\emph{The Times is committed to publishing}
\href{https://www.nytimes3xbfgragh.onion/2019/01/31/opinion/letters/letters-to-editor-new-york-times-women.html}{\emph{a
diversity of letters}} \emph{to the editor. We'd like to hear what you
think about this or any of our articles. Here are some}
\href{https://help.nytimes3xbfgragh.onion/hc/en-us/articles/115014925288-How-to-submit-a-letter-to-the-editor}{\emph{tips}}\emph{.
And here's our email:}
\href{mailto:letters@NYTimes.com}{\emph{letters@NYTimes.com}}\emph{.}

\emph{Follow The New York Times Opinion section on}
\href{https://www.facebookcorewwwi.onion/nytopinion}{\emph{Facebook}}\emph{,}
\href{http://twitter.com/NYTOpinion}{\emph{Twitter (@NYTopinion)}}
\emph{and}
\href{https://www.instagram.com/nytopinion/}{\emph{Instagram}}\emph{.}

Advertisement

\protect\hyperlink{after-bottom}{Continue reading the main story}

\hypertarget{site-index}{%
\subsection{Site Index}\label{site-index}}

\hypertarget{site-information-navigation}{%
\subsection{Site Information
Navigation}\label{site-information-navigation}}

\begin{itemize}
\tightlist
\item
  \href{https://help.nytimes3xbfgragh.onion/hc/en-us/articles/115014792127-Copyright-notice}{©~2020~The
  New York Times Company}
\end{itemize}

\begin{itemize}
\tightlist
\item
  \href{https://www.nytco.com/}{NYTCo}
\item
  \href{https://help.nytimes3xbfgragh.onion/hc/en-us/articles/115015385887-Contact-Us}{Contact
  Us}
\item
  \href{https://www.nytco.com/careers/}{Work with us}
\item
  \href{https://nytmediakit.com/}{Advertise}
\item
  \href{http://www.tbrandstudio.com/}{T Brand Studio}
\item
  \href{https://www.nytimes3xbfgragh.onion/privacy/cookie-policy\#how-do-i-manage-trackers}{Your
  Ad Choices}
\item
  \href{https://www.nytimes3xbfgragh.onion/privacy}{Privacy}
\item
  \href{https://help.nytimes3xbfgragh.onion/hc/en-us/articles/115014893428-Terms-of-service}{Terms
  of Service}
\item
  \href{https://help.nytimes3xbfgragh.onion/hc/en-us/articles/115014893968-Terms-of-sale}{Terms
  of Sale}
\item
  \href{https://spiderbites.nytimes3xbfgragh.onion}{Site Map}
\item
  \href{https://help.nytimes3xbfgragh.onion/hc/en-us}{Help}
\item
  \href{https://www.nytimes3xbfgragh.onion/subscription?campaignId=37WXW}{Subscriptions}
\end{itemize}
