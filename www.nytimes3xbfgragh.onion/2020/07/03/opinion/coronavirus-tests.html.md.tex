Sections

SEARCH

\protect\hyperlink{site-content}{Skip to
content}\protect\hyperlink{site-index}{Skip to site index}

\href{https://myaccount.nytimes3xbfgragh.onion/auth/login?response_type=cookie\&client_id=vi}{}

\href{https://www.nytimes3xbfgragh.onion/section/todayspaper}{Today's
Paper}

\href{/section/opinion}{Opinion}\textbar{}A Cheap, Simple Way to Control
the Coronavirus

\url{https://nyti.ms/3glq8AR}

\begin{itemize}
\item
\item
\item
\item
\item
\end{itemize}

Advertisement

\protect\hyperlink{after-top}{Continue reading the main story}

\href{/section/opinion}{Opinion}

Supported by

\protect\hyperlink{after-sponsor}{Continue reading the main story}

\hypertarget{a-cheap-simple-way-to-control-the-coronavirus}{%
\section{A Cheap, Simple Way to Control the
Coronavirus}\label{a-cheap-simple-way-to-control-the-coronavirus}}

With easy-to-use tests, everyone can check themselves every day.

By Laurence J. Kotlikoff and Michael Mina

Mr. Kotlikoff is a professor of economics at Boston University and Dr.
Mina is an assistant professor of epidemiology at the Harvard T.H. Chan
School of Public Health.

\begin{itemize}
\item
  July 3, 2020
\item
  \begin{itemize}
  \item
  \item
  \item
  \item
  \item
  \end{itemize}
\end{itemize}

\includegraphics{https://static01.graylady3jvrrxbe.onion/images/2020/07/03/opinion/03Mina1/03Mina1-articleLarge.jpg?quality=75\&auto=webp\&disable=upscale}

Simple at-home tests for the coronavirus, some that involve spitting
into a small tube of solution, could be the key to expanding testing and
impeding the spread of the pandemic. The Food and Drug Administration
should encourage their development and then fast track approval.

One variety, paper-strip tests, are inexpensive and easy enough to make
that Americans could test themselves every day. You would simply spit
into a tube of saline solution and insert a small piece of paper
embedded with a strip of protein. If you are infected with enough of the
virus, the strip will change color within 15 minutes.

Your next step would be to self-quarantine, notify your doctor and
confirm the result with a standard swab test --- the polymerase chain
reaction nasal swab. Confirmation would give public health officials key
information on the virus's spread and confirm that you should remain in
quarantine until your daily test turned negative.

\href{https://e25bio.com/}{E25Bio},
\href{https://sherlock.bio/technology/}{Sherlock Biosciences},
\href{https://mammoth.bio/}{Mammoth Biosciences}, and an increasing
number of academic
\href{http://news.mit.edu/2020/sherlock-based-one-step-test-provides-rapid-sensitive-covid-19-detection-0505}{research
laboratories} are in the late stages of
\href{https://sciencebusiness.technewslit.com/?p=36279}{developing}
\href{https://e25bio.com/products/}{paper-strip} and other simple, daily
Covid-19 tests. Some of the daily tests are in trials and proving highly
\href{https://www.nature.com/articles/s41587-020-0513-4}{effective}.

The strips could be mass produced in a matter of weeks and freely
supplied by the government to everyone in the country. The price per
person would be from \$1 to \$5 a day, a considerable sum for the entire
population, but remarkably cost effective.

Screening the population for infection, however, is different from
determining whether someone is infected.

The Food and Drug Administration has recently approved group P.C.R.
testing to screen large numbers of people.
(\href{https://www.nytimes3xbfgragh.onion/2020/05/07/opinion/coronavirus-group-testing.html}{Group
testing}, which is used in other countries, assays multiple swab samples
at once and if the virus is found, individuals are tested.) So there is
reason to hope that the F.D.A. will also approve paper-strip tests as a
way to find out where the virus has spread.

Hope needs to be replaced with surety. Biotech companies are reluctant
to take these tests to market for fear that the F.D.A. will disparage
them for being less sensitive than the nasal swab tests. The nasal swab
test can detect extremely small quantities of viral particles.

But the problem with the nasal swab tests is their cost, which ranges
from \$50 to \$150. They also require laboratory assessment, which can
take days. That is why, the Centers for Disease Control and Prevention
\href{https://www.washingtonpost.com/health/2020/06/25/coronavirus-cases-10-times-larger/}{reports},
nine of 10 infected Americans never get tested. It's also why those who
do get tested, generally are tested only once.

Clearly, if you're infected and never tested, you can unwittingly spread
the virus. And if you are tested, but just once, and the test comes back
negative, you may still later become infectious. Finally, if your
polymerase chain reaction swab is positive, but it takes five days to
learn the result, you may spend those days transmitting the disease.

Group testing can dramatically lower nasal-swab-testing costs for
\href{https://www.wsj.com/articles/why-cornell-will-reopen-in-the-fall-11593535516}{universities}
and large companies. But absent federal coordination, it can't be used
routinely to test all Americans.

We need the best means of detecting and containing the virus, not a
perfect test that no one can use. That is where paper-strip testing
would have the advantage. Their ability to be used more frequently would
trump the nasal swab test's higher sensitivity. Paper-strip testing
would also sharply improve diagnosis as those with a positive
paper-strip test would still be given a nasal swab test.

Would everyone take a paper-strip test every day? Here market incentives
will surely help. Once they are provided to all, employers would likely
require their workers to take time-dated pictures of their negative test
results before coming to work. Colleges would require students to do the
same before coming to class. Restaurants could accept reservations only
if accompanied by negative-test pictures. In short, everyone will have
an incentive to test themselves daily to participate fully in the
economy and return to normal life.

Once paper strips' efficacy is definitively proved and they are cleared
by the F.D.A., Congress can quickly authorize the production and
distribution, for free, of a year's supply to all Americans. Then we'll
have not only a true day-to-day sense of Covid-19's path. We'll also
have a far better means to quickly contain and end this terrible plague.

Laurence Kotlikoff is a professor of economics at Boston University, and
Michael Mina is an assistant professor of epidemiology at the Harvard
T.H. Chan School of Public Health.

\emph{The Times is committed to publishing}
\href{https://www.nytimes3xbfgragh.onion/2019/01/31/opinion/letters/letters-to-editor-new-york-times-women.html}{\emph{a
diversity of letters}} \emph{to the editor. We'd like to hear what you
think about this or any of our articles. Here are some}
\href{https://help.nytimes3xbfgragh.onion/hc/en-us/articles/115014925288-How-to-submit-a-letter-to-the-editor}{\emph{tips}}\emph{.
And here's our email:}
\href{mailto:letters@NYTimes.com}{\emph{letters@NYTimes.com}}\emph{.}

\emph{Follow The New York Times Opinion section on}
\href{https://www.facebookcorewwwi.onion/nytopinion}{\emph{Facebook}}\emph{,}
\href{http://twitter.com/NYTOpinion}{\emph{Twitter (@NYTopinion)}}
\emph{and}
\href{https://www.instagram.com/nytopinion/}{\emph{Instagram}}\emph{.}

Advertisement

\protect\hyperlink{after-bottom}{Continue reading the main story}

\hypertarget{site-index}{%
\subsection{Site Index}\label{site-index}}

\hypertarget{site-information-navigation}{%
\subsection{Site Information
Navigation}\label{site-information-navigation}}

\begin{itemize}
\tightlist
\item
  \href{https://help.nytimes3xbfgragh.onion/hc/en-us/articles/115014792127-Copyright-notice}{©~2020~The
  New York Times Company}
\end{itemize}

\begin{itemize}
\tightlist
\item
  \href{https://www.nytco.com/}{NYTCo}
\item
  \href{https://help.nytimes3xbfgragh.onion/hc/en-us/articles/115015385887-Contact-Us}{Contact
  Us}
\item
  \href{https://www.nytco.com/careers/}{Work with us}
\item
  \href{https://nytmediakit.com/}{Advertise}
\item
  \href{http://www.tbrandstudio.com/}{T Brand Studio}
\item
  \href{https://www.nytimes3xbfgragh.onion/privacy/cookie-policy\#how-do-i-manage-trackers}{Your
  Ad Choices}
\item
  \href{https://www.nytimes3xbfgragh.onion/privacy}{Privacy}
\item
  \href{https://help.nytimes3xbfgragh.onion/hc/en-us/articles/115014893428-Terms-of-service}{Terms
  of Service}
\item
  \href{https://help.nytimes3xbfgragh.onion/hc/en-us/articles/115014893968-Terms-of-sale}{Terms
  of Sale}
\item
  \href{https://spiderbites.nytimes3xbfgragh.onion}{Site Map}
\item
  \href{https://help.nytimes3xbfgragh.onion/hc/en-us}{Help}
\item
  \href{https://www.nytimes3xbfgragh.onion/subscription?campaignId=37WXW}{Subscriptions}
\end{itemize}
