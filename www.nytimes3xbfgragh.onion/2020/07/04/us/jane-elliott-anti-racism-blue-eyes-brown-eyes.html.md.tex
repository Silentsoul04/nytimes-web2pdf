Sections

SEARCH

\protect\hyperlink{site-content}{Skip to
content}\protect\hyperlink{site-index}{Skip to site index}

\href{https://www.nytimes3xbfgragh.onion/section/us}{U.S.}

\href{https://myaccount.nytimes3xbfgragh.onion/auth/login?response_type=cookie\&client_id=vi}{}

\href{https://www.nytimes3xbfgragh.onion/section/todayspaper}{Today's
Paper}

\href{/section/us}{U.S.}\textbar{}A Teacher Held a Famous Racism
Exercise in 1968. She's Still at It.

\url{https://nyti.ms/2Z1oeQm}

\begin{itemize}
\item
\item
\item
\item
\item
\end{itemize}

Advertisement

\protect\hyperlink{after-top}{Continue reading the main story}

Supported by

\protect\hyperlink{after-sponsor}{Continue reading the main story}

In her words

\hypertarget{a-teacher-held-a-famous-racism-exercise-in-1968-shes-still-at-it}{%
\section{A Teacher Held a Famous Racism Exercise in 1968. She's Still at
It.}\label{a-teacher-held-a-famous-racism-exercise-in-1968-shes-still-at-it}}

The day after Rev. Dr. Martin Luther King Jr. was assassinated, Jane
Elliott carried out the ``Blue Eyes, Brown Eyes'' exercise in her
classroom. Now, people are returning to her work.

\includegraphics{https://static01.graylady3jvrrxbe.onion/images/2020/06/26/us/IHW-BLUEEYESBROWNEYES/IHW-BLUEEYESBROWNEYES-articleLarge.jpg?quality=75\&auto=webp\&disable=upscale}

\href{https://www.nytimes3xbfgragh.onion/by/alisha-haridasani-gupta}{\includegraphics{https://static01.graylady3jvrrxbe.onion/images/2018/09/10/multimedia/author-alisha-haridasani-gupta/author-alisha-haridasani-gupta-thumbLarge-v3.png}}

By
\href{https://www.nytimes3xbfgragh.onion/by/alisha-haridasani-gupta}{Alisha
Haridasani Gupta}

\begin{itemize}
\item
  Published July 4, 2020Updated July 15, 2020
\item
  \begin{itemize}
  \item
  \item
  \item
  \item
  \item
  \end{itemize}
\end{itemize}

\begin{center}\rule{0.5\linewidth}{\linethickness}\end{center}

\hypertarget{it-makes-me-really-angry-that-ive-been-saying-these-things-for-52-years}{%
\subsection{``It makes me really angry that I've been saying these
things for 52
years.''}\label{it-makes-me-really-angry-that-ive-been-saying-these-things-for-52-years}}

\emph{--- Jane Elliott, a schoolteacher turned anti-racism educator}

\begin{center}\rule{0.5\linewidth}{\linethickness}\end{center}

\emph{{[}In Her Words is available as a newsletter.}
\href{https://www.nytimes3xbfgragh.onion/newsletters/in-her-words}{\emph{Sign
up here to get it delivered to your inbox}}\emph{.{]}}

As
\href{https://www.nytimes3xbfgragh.onion/news-event/george-floyd-protests-minneapolis-new-york-los-angeles}{protests
against racism} started sweeping across America and rest of the world,
clips of
\href{https://www.nytimes3xbfgragh.onion/2020/07/15/style/jane-elliott-anti-racism.html}{Jane
Elliott}, a schoolteacher turned anti-racism educator, began circulating
on social media.

Perhaps you've seen them.

In
\href{https://twitter.com/adilray/status/1269196871534284802?s=20}{one
grainy clip from 2001}, Ms. Elliott, with her signature round glasses
and clipped white hair, gets into such a heated argument with a white
female college student during an educational exercise about racism that
the uncomfortable and distraught woman starts crying and storms out of
the classroom.

``You just exercised a freedom that none of these people of color
have,'' Ms. Elliott tells the student, sternly. ``When these people of
color get tired of racism, they can't just walk out.''

Or maybe you've seen
\href{https://twitter.com/PadmaLakshmi/status/1274380187086606337?s=20}{the
2018 video} of Ms. Elliott in a
\href{https://www.facebookcorewwwi.onion/538649879867825/posts/699300837136061/}{round-table
discussion} on racism with the actress and producer Jada Pinkett Smith,
Ms. Pinkett Smith's daughter, Willow, and Ms. Pinkett Smith's mother,
\href{https://www.nytimes3xbfgragh.onion/2020/03/27/style/self-care/adrienne-banfield-norris-red-table-talk.html}{Adrienne
Banfield-Norris}.

``I'm not a white woman. I'm a faded Black person,'' Ms. Elliott says,
stunning the hosts. ``My people moved far from the Equator, and that's
the only reason my skin is lighter.''

``Wow,'' Ms. Pinkett Smith says back. ``I'm with you, Jane!''

Ms. Elliott, now 87, said she started teaching about racism on April 5,
1968 --- the day after the Rev. Dr. Martin Luther King Jr. was
assassinated.

At the time, she was a third-grade schoolteacher in the all-white Iowa
town of Riceville, and the news of Dr. King's death so shocked and moved
her that she threw out the lesson plan for the next day and came up with
a new one that would force the children to experience prejudice and
discrimination firsthand.

In what is now known as the ``Blue Eyes, Brown Eyes'' exercise, she
split up her class into two groups based on an arbitrary characteristic:
eye color. Those with blue eyes were better, smarter and superior to
those with brown eyes, she told her students, and therefore they were
entitled to perks, like more recess time and access to the water
fountain.

Quickly, the dynamic of the room shifted. ``I watched wonderful,
thoughtful children turn into nasty, vicious, discriminating little
third graders,'' she explained, in a
\href{https://www.youtube.com/watch?v=1mcCLm_LwpE}{PBS documentary}
about her work.

The next day, she reversed the roles. Now the brown-eyed students were
superior and had perks and the blue-eyed students were inferior.

For decades, Ms. Elliott repeated the exercise around the country,
including in 1992 \href{https://www.youtube.com/watch?v=ebPoSMULI5U}{on
the Oprah Winfrey Show}, and she would witness more or less the same
outcome: people turning on each other on the basis of eye color.

Now, as the recent wave of demonstrations reaches
\href{https://www.nytimes3xbfgragh.onion/interactive/2020/06/13/us/george-floyd-protests-cities-photos.html}{every
corner of the U.S}., drawing more
\href{https://www.nytimes3xbfgragh.onion/2020/06/12/us/george-floyd-white-protesters.html}{white
people} than previous protests against racism, Ms. Elliott's work is
thrust back in the spotlight.

And she's sick and tired of it.

``I keep trying to tell people why racism has to stop, and they keep
asking the same questions, like `How do we do that?' and then continue
to ignore the answers,'' she said in a phone interview.

``It makes me really angry that I've been saying these things for 52
years.''

I caught up with Ms. Elliott to discuss the persistence of racism in
America and how things have evolved since 1968 --- if at all.

\emph{The interview has been condensed and edited for clarity.}

\textbf{What are your thoughts about what's happening in this country at
this moment?}

What I want to know is why the networks keep showing that video of
George Floyd over and over and over again. How dare they do that? There
should be fines for every one of those networks that keep showing that.
Do they show it so that white people will see how awful it is? Or do
they show it so that young Black boys and their mothers will realize
what could happen to them?

It is insensitive to the point where the networks don't even realize the
message that they are sending to Black women and their sons.

\textbf{For the past few decades, you've been doing anti-racism lectures
and workshops around the country. Have you noticed a shift in how they
have been received over the years?}

I've been doing the exercise with adults for about 35 years. But in the
last few years, I've only been doing speeches about it because we now
live in a situation where people turn off immediately if they think
they're going to learn something counter to their beliefs, and I don't
want to be threatened with death anymore. I'm tired of receiving death
threats.

\textbf{You've been receiving death threats?}

Yes. Just recently, a year ago, I was giving a speech at a college in
Southern California and my daughter was there and these kids behind her
--- three white males --- said, ``Wouldn't you like to go up and just
shoot her?'' And the other one said, ``I'd like to go up and beat her
and then rape her.''

When my daughter told the security person, those three boys jumped up
and ran out of the building.

\textbf{In} \textbf{\href{https://www.youtube.com/watch?v=f2z-ahJ4uws}{a
recent interview with Jimmy Fallon}, you said it frustrates you when
people say ``I don't see Black or brown.'' Can you elaborate on why that
makes you angry?}

Teachers will stand up in front of classrooms and say, ``I don't see
people as Black or brown, I just see people.'' What these teachers are
actually saying is that \emph{they} have the freedom to ignore the
largest organ on your body.

There is only one race --- the human race. It's time to get over the
idea of a number of races. We are all of the same species, just of
different colors, different shapes, different sizes, different genders.

When somebody says to me, ``I am biracial,\textbf{''} I say, ``Which of
your parents came from outer space? Do you have one parent who isn't a
human being? You're trying to tell me that your parents are of two
different color groups, but that doesn't mean there are two different
races.''

\textbf{So we shouldn't be colorblind?}

See skin color, but don't see it as a negative or as a positive --- it
is just your body's reaction to the natural environment!

\textbf{Where did you grow up, and when did you come to truly understand
the problem of racism in this country?}

I was raised on a farm in northeast Iowa. When I went to school, I
started to learn the standard elementary curriculum, which is that white
men did all the inventing and discovering and civilizing because they
are oh-so-superior. We just didn't talk about race, but we knew that we
were OK because no matter how poor you were --- and we were dirt poor,
by the way --- at least we were white.

Then I went to college, and in my first social studies education class,
the white professor stood up in front of that group of students and
said, ``When you get into the classroom, you must not teach in
opposition to local mores.''

And I thought, well, that's wrong. But I didn't stand up and say
anything. I did what white people do all day, every day: I went along to
get along.

\textbf{So when did you start pushing back?}

In the 1950s, my husband and I were living in Waterloo, Iowa, and he got
transferred to another town. So we had to put our house up for rent. I
put an ad in the paper, and somebody called to ask if we rented to
people of color. I can still remember what I did: I thought that if we
rent to people of color, then when we come back, our white neighbors
won't speak to us. So I said, ``This is an all-white neighborhood.'' And
that person said thanks and hung up.

But I realized immediately what I had done: I let money buy my ethics in
that moment. And I decided this will never happen again. I will never
ever put money above my principles. And I haven't, and I won't.

\textbf{A lot of white people are trying to reassess their own biases.
Based on the work you've done, what can white people do to actually help
in this moment?}

First of all, you have to realize what I do isn't hard work. What Black
people do is hard work. I get paid for the work that I do. They don't
get paid for taking this crap every day --- they have to take it. They
don't volunteer for it. It was forced on them.

And second, white people need to stop referring to themselves as
``allies'' --- as if we can make it all right. They need to educate away
the ignorance that was poured into them when they were in school and
realize that \emph{they} are the reason everyone is so angry.

\begin{center}\rule{0.5\linewidth}{\linethickness}\end{center}

Advertisement

\protect\hyperlink{after-bottom}{Continue reading the main story}

\hypertarget{site-index}{%
\subsection{Site Index}\label{site-index}}

\hypertarget{site-information-navigation}{%
\subsection{Site Information
Navigation}\label{site-information-navigation}}

\begin{itemize}
\tightlist
\item
  \href{https://help.nytimes3xbfgragh.onion/hc/en-us/articles/115014792127-Copyright-notice}{©~2020~The
  New York Times Company}
\end{itemize}

\begin{itemize}
\tightlist
\item
  \href{https://www.nytco.com/}{NYTCo}
\item
  \href{https://help.nytimes3xbfgragh.onion/hc/en-us/articles/115015385887-Contact-Us}{Contact
  Us}
\item
  \href{https://www.nytco.com/careers/}{Work with us}
\item
  \href{https://nytmediakit.com/}{Advertise}
\item
  \href{http://www.tbrandstudio.com/}{T Brand Studio}
\item
  \href{https://www.nytimes3xbfgragh.onion/privacy/cookie-policy\#how-do-i-manage-trackers}{Your
  Ad Choices}
\item
  \href{https://www.nytimes3xbfgragh.onion/privacy}{Privacy}
\item
  \href{https://help.nytimes3xbfgragh.onion/hc/en-us/articles/115014893428-Terms-of-service}{Terms
  of Service}
\item
  \href{https://help.nytimes3xbfgragh.onion/hc/en-us/articles/115014893968-Terms-of-sale}{Terms
  of Sale}
\item
  \href{https://spiderbites.nytimes3xbfgragh.onion}{Site Map}
\item
  \href{https://help.nytimes3xbfgragh.onion/hc/en-us}{Help}
\item
  \href{https://www.nytimes3xbfgragh.onion/subscription?campaignId=37WXW}{Subscriptions}
\end{itemize}
