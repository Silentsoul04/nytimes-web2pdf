Sections

SEARCH

\protect\hyperlink{site-content}{Skip to
content}\protect\hyperlink{site-index}{Skip to site index}

\href{https://www.nytimes3xbfgragh.onion/section/health}{Health}

\href{https://myaccount.nytimes3xbfgragh.onion/auth/login?response_type=cookie\&client_id=vi}{}

\href{https://www.nytimes3xbfgragh.onion/section/todayspaper}{Today's
Paper}

\href{/section/health}{Health}\textbar{}A New Generation of Fast
Coronavirus Tests Is Coming

\url{https://nyti.ms/2Z30ivD}

\begin{itemize}
\item
\item
\item
\item
\item
\item
\end{itemize}

\hypertarget{the-coronavirus-outbreak}{%
\subsubsection{\texorpdfstring{\href{https://www.nytimes3xbfgragh.onion/news-event/coronavirus?name=styln-coronavirus-national\&region=TOP_BANNER\&block=storyline_menu_recirc\&action=click\&pgtype=Article\&impression_id=cadc1e70-f4b6-11ea-90f6-17a6587a4d59\&variant=undefined}{The
Coronavirus
Outbreak}}{The Coronavirus Outbreak}}\label{the-coronavirus-outbreak}}

\begin{itemize}
\tightlist
\item
  live\href{https://www.nytimes3xbfgragh.onion/2020/09/11/world/covid-19-coronavirus.html?name=styln-coronavirus-national\&region=TOP_BANNER\&block=storyline_menu_recirc\&action=click\&pgtype=Article\&impression_id=cadc4580-f4b6-11ea-90f6-17a6587a4d59\&variant=undefined}{Latest
  Updates}
\item
  \href{https://www.nytimes3xbfgragh.onion/interactive/2020/us/coronavirus-us-cases.html?name=styln-coronavirus-national\&region=TOP_BANNER\&block=storyline_menu_recirc\&action=click\&pgtype=Article\&impression_id=cadc4581-f4b6-11ea-90f6-17a6587a4d59\&variant=undefined}{Maps
  and Cases}
\item
  \href{https://www.nytimes3xbfgragh.onion/interactive/2020/science/coronavirus-vaccine-tracker.html?name=styln-coronavirus-national\&region=TOP_BANNER\&block=storyline_menu_recirc\&action=click\&pgtype=Article\&impression_id=cadc4582-f4b6-11ea-90f6-17a6587a4d59\&variant=undefined}{Vaccine
  Tracker}
\item
  \href{https://www.nytimes3xbfgragh.onion/2020/09/10/us/politics/fda-coronavirus-vaccine.html?name=styln-coronavirus-national\&region=TOP_BANNER\&block=storyline_menu_recirc\&action=click\&pgtype=Article\&impression_id=cadc4583-f4b6-11ea-90f6-17a6587a4d59\&variant=undefined}{F.D.A.
  Regulators' Self-Defense}
\item
  \href{https://www.nytimes3xbfgragh.onion/2020/09/09/upshot/coronavirus-surprise-test-fees.html?name=styln-coronavirus-national\&region=TOP_BANNER\&block=storyline_menu_recirc\&action=click\&pgtype=Article\&impression_id=cadc4584-f4b6-11ea-90f6-17a6587a4d59\&variant=undefined}{Surprise
  Test Fees}
\end{itemize}

Advertisement

\protect\hyperlink{after-top}{Continue reading the main story}

Supported by

\protect\hyperlink{after-sponsor}{Continue reading the main story}

\hypertarget{a-new-generation-of-fast-coronavirus-tests-is-coming}{%
\section{A New Generation of Fast Coronavirus Tests Is
Coming}\label{a-new-generation-of-fast-coronavirus-tests-is-coming}}

New technologies, like the gene-editing tool Crispr, can spot the virus
in less than an hour. But it will likely be months before these tests
hit clinics.

\includegraphics{https://static01.graylady3jvrrxbe.onion/images/2020/06/30/science/00virus-new-tests01/30virus-new-tests01-articleLarge.jpg?quality=75\&auto=webp\&disable=upscale}

By
\href{https://www.nytimes3xbfgragh.onion/by/katherine-j--wu}{Katherine
J. Wu}

\begin{itemize}
\item
  Published July 6, 2020Updated Aug. 7, 2020
\item
  \begin{itemize}
  \item
  \item
  \item
  \item
  \item
  \item
  \end{itemize}
\end{itemize}

Researchers around the world are working on the next generation of
\href{https://www.nytimes3xbfgragh.onion/2020/08/07/us/covid-test-accuracy-governor-dewine-ohio.html}{coronavirus
tests} that give answers in less than an hour, without onerous equipment
or highly trained personnel.

The latest so-called point-of-care tests, which could be done in a
doctor's office or even at home, would be a welcome upgrade from today's
status quo: uncomfortable swabs that snake up the nose and can take
several days to produce results.

The handful of point-of-care devices now on the market are
\href{https://www.nytimes3xbfgragh.onion/2020/05/13/health/coronavirus-testing-white-house.html}{frequently
inaccurate}. The up-and-coming
\href{https://www.nytimes3xbfgragh.onion/2020/07/15/parenting/kids-covid-19-test.html}{tests}
could yield more reliable results, researchers say, potentially leading
to on-the-spot testing nationwide. But most of the new contenders are
still in early stages, and won't be available in clinics for months.

Some of the tests in development swap brain-tickling swabs for plastic
tubes that collect spit. Others dunk patient samples into chemical
cocktails that light up when they detect coronavirus genes. Another type
of test identifies
\href{https://www.nytimes3xbfgragh.onion/2020/07/15/parenting/kids-covid-19-test.html}{coronavirus}
proteins in minutes, using a cheap device that's easy to produce in bulk
and deploy in low-resource settings.

``To combat this virus, we need to test widely and frequently, and get
the results back quickly,'' said Dr. Zev Williams at Columbia
University, who is developing a
\href{https://www.medrxiv.org/content/10.1101/2020.06.13.20129841v1.full.pdf}{coronavirus
spit test} that can run in about 30 minutes. ``That requires a genuine
paradigm shift in the way we go about testing for it.''

Once scaled up and distributed, faster tests could be used in hospitals
to quickly screen emergency room patients. Schools and workplaces could
buy them to monitor the health of children and employees. With
additional tinkering, some tests could even be developed to work as
simply as a pregnancy test, yielding a clean-cut positive or negative
result in the comfort of a person's home.

``The quicker and easier tests can be done,'' the more ubiquitous they
can be, said Dr. Amesh Adalja at the Johns Hopkins University Center for
Health Security. ``That's going to help people get back to some
semblance of normalcy.''

Most diagnostic tests on the market now hunt for bits of genetic
material specific to the coronavirus. (This distinguishes diagnostic
tests from antibody tests, which sample the blood and show if a person
has been exposed to the virus in the past.)

The gold-standard method involves funneling a long, absorbent swab a few
inches into the nose until it hits the nasopharynx, the part of the
airway where the nasal passage meets the throat and a common target of
the coronavirus.

\hypertarget{latest-updates-the-coronavirus-outbreak}{%
\section{\texorpdfstring{\href{https://www.nytimes3xbfgragh.onion/2020/09/11/world/covid-19-coronavirus.html?action=click\&pgtype=Article\&state=default\&region=MAIN_CONTENT_1\&context=storylines_live_updates}{Latest
Updates: The Coronavirus
Outbreak}}{Latest Updates: The Coronavirus Outbreak}}\label{latest-updates-the-coronavirus-outbreak}}

Updated 2020-09-12T04:56:54.924Z

\begin{itemize}
\tightlist
\item
  \href{https://www.nytimes3xbfgragh.onion/2020/09/11/world/covid-19-coronavirus.html?action=click\&pgtype=Article\&state=default\&region=MAIN_CONTENT_1\&context=storylines_live_updates\#link-dfb8a16}{Fauci
  cautions the virus could disrupt life in the U.S. until `maybe even
  towards the end of 2021.'}
\item
  \href{https://www.nytimes3xbfgragh.onion/2020/09/11/world/covid-19-coronavirus.html?action=click\&pgtype=Article\&state=default\&region=MAIN_CONTENT_1\&context=storylines_live_updates\#link-7104d154}{From
  Asia to Africa, China promotes its vaccine candidates to win friends.}
\item
  \href{https://www.nytimes3xbfgragh.onion/2020/09/11/world/covid-19-coronavirus.html?action=click\&pgtype=Article\&state=default\&region=MAIN_CONTENT_1\&context=storylines_live_updates\#link-393ad215}{The
  other way the virus will kill: hunger.}
\end{itemize}

\href{https://www.nytimes3xbfgragh.onion/2020/09/11/world/covid-19-coronavirus.html?action=click\&pgtype=Article\&state=default\&region=MAIN_CONTENT_1\&context=storylines_live_updates}{See
more updates}

More live coverage:
\href{https://www.nytimes3xbfgragh.onion/live/2020/09/11/business/stock-market-today-coronavirus?action=click\&pgtype=Article\&state=default\&region=MAIN_CONTENT_1\&context=storylines_live_updates}{Markets}

``The moment you see the swab, you're like, `Oh no, my face isn't that
deep,''' said Fernanda Ferreira, a virologist at Harvard University who
took a nasopharyngeal swab test in April. ``Turns out it is.''

The virus's genes must be extracted from the sample with a specific
suite of chemicals. The material is then processed through a laboratory
technique called polymerase chain reaction, or PCR, in which a machine
cycles through several temperature changes to amplify genetic material.
This step is key to these tests' success: Copying genetic material over
and over means that even tiny amounts of the virus can be spotted.

But the process can bog down at multiple points. Swabs and chemicals
necessary for processing are
\href{https://www.nytimes3xbfgragh.onion/2020/03/18/health/coronavirus-test-shortages-face-masks-swabs.html}{often
in short supply}, and invasive sampling requires trained health care
workers who quickly drain precious supplies of gowns, gloves and masks.
Additionally, many community testing centers lack PCR machines and must
outsource their samples to large laboratories, leading to delays of days
or even weeks.

Rachel Coker, the director of research advancement at Binghamton
University --- one of many institutions nationwide that have begun to
reopen --- had to wait 10 days for her results after being sampled at a
drive-through testing site.

\includegraphics{https://static01.graylady3jvrrxbe.onion/images/2020/07/06/science/00virus-fast-tests04/merlin_174146376_0a230ac0-cd77-4926-a6be-a3eb728aec64-articleLarge.jpg?quality=75\&auto=webp\&disable=upscale}

``The good news was it was negative,'' she said. But she could have been
exposed while waiting for results. ``By the time I knew,'' Ms. Coker
said, ``it was almost useless.''

Researchers are attempting to streamline every part of the diagnostic
pipeline. One timesaving tactic that's already been rolled out
nationwide involves sampling areas other than the nasopharynx, such as
swabbing the nostrils or throat, or collecting gobs of saliva.

These tests are painless, and avoid putting health care workers in
harm's way. But they
\href{https://www.cebm.net/covid-19/comparative-accuracy-of-oropharyngeal-and-nasopharyngeal-swabs-for-diagnosis-of-covid-19/}{aren't
always accurate}. ``Unfortunately, this virus doesn't hang around in the
nose or throat so much,'' said Dr. Ravindra Gupta, a clinical
microbiologist at the University of Cambridge.

To avoid mistakenly declaring infected people virus-free, Dr. Gupta and
his colleagues are developing a point-of-care test that can
\href{https://www.medrxiv.org/content/10.1101/2020.06.16.20133157v3.full.pdf}{simultaneously
screen patients for the coronavirus and antibodies that recognize it}.
Antibodies often start to appear by the second week of infection.

At the Broad Institute in Cambridge, Ma., another team of researchers is
tackling the next plodding step in the work flow: amplifying the sample.
In the lab, the scientists use a technique that, unlike PCR, can copy
genetic material at a single temperature. If the virus is present,
\href{https://www.biorxiv.org/content/10.1101/2020.05.28.119131v1.full.pdf}{a
gene-editing tool called Crispr} will make the tube's contents glow at a
wavelength detectable by a smartphone. The entire procedure takes less
than an hour, and
\href{https://www.biorxiv.org/content/10.1101/2020.05.28.119131v1.full.pdf}{correctly
identifies active infections about 90 percent of the time}.

Laboratory experiments that use Crispr are
\href{https://www.ncbi.nlm.nih.gov/pmc/articles/PMC5915479/}{thought to
be very precise}, potentially giving these tests a low rate of false
positives, said Catherine Freije, one of the scientists developing the
Crispr test. The molecular machinery in the test is specific to the new
coronavirus, and doesn't react to its close viral relatives.

The test cooked up by Columbia University's Dr. Williams and his
colleagues might be simpler still: Spit is added to a premixed slew of
chemicals, which then gets incubated at 145 degrees Fahrenheit for half
an hour. If the tube turns yellow, the test is positive; if it's red,
negative. The test can detect even tiny amounts of virus, making it more
sensitive than similar tests, and
\href{https://www.medrxiv.org/content/10.1101/2020.06.13.20129841v1.full.pdf}{gives
false negatives less than 5 percent of the time}, according to a study
that has not yet been published in a scientific journal. Dr. Williams
and his team are seeking authorization from the F.D.A.

Image

In Dr. Zev Williams's laboratory at Columbia University, researchers are
developing a coronavirus spit test that can yield color-based results in
about 30 minutes.Credit...Dr. Zev Williams/Columbia University

Researchers are
\href{https://jamanetwork.com/journals/jama/fullarticle/2765837}{still
gauging} how the accuracy of spit tests
\href{https://pubmed.ncbi.nlm.nih.gov/32310815/}{stacks up} against that
of the deep nasal swabs, but
\href{https://www.medrxiv.org/content/10.1101/2020.04.16.20067835v1.full.pdf}{early
results} are promising. ``You put it in a tube --- that's hard to mess
up,'' said Anne Wyllie, an epidemiologist at Yale's School of Public
Health who is studying the saliva tests.

Still, the quick tests available now are frequently inaccurate. Although
they ``ensure we can get an answer faster,'' said Dr. Ibukun Akinboyo, a
pediatrician and infectious disease specialist at Duke University's
School of Medicine, ``you lose some sensitivity,'' she said. ``It's hard
to win at both.''

In May, a swab-based point-of-care test called Abbott ID Now
\href{https://www.nytimes3xbfgragh.onion/interactive/2020/05/12/us/coronavirus-testing-white-house.html}{made
headlines} when an analysis found that it might miss infections up to 48
percent of the time, despite being promoted by President Trump as
``highly accurate.''

Sensitivity issues also plague antigen tests, which detect pieces of
proteins made by the virus, rather than its genes. Antigen tests have
been used to detect other airway infections, such as the flu, in less
than an hour, and are easy to manufacture en masse. But the convenience
comes at a cost: Unlike genetic material, antigens can't be amplified
easily. Some antigen tests, including a few that search for
\href{https://www.cdc.gov/flu/professionals/diagnosis/clinician_guidance_ridt.htm}{influenza
viruses}, fail to pick up on active infections
\href{https://www.cdc.gov/flu/professionals/diagnosis/clinician_guidance_ridt.htm}{around
50 percent of the time}.

``If a Covid antigen test performs like an influenza antigen test, I
don't think they will have much utility,'' said Dr. David Alland, the
director of the Center for Emerging Pathogens at Rutgers New Jersey
Medical School. Still, he noted, ``if improved, they could be very
promising.''

Even imprecise tests have their place in this pandemic, as long as
they're easy to use and distributed widely enough. Should a test ``miss
someone on Monday, maybe you'll get them a day or two later,'' Dr.
Wyllie said.

So far, only two companies have received emergency
\href{https://www.fda.gov/news-events/press-announcements/coronavirus-covid-19-update-fda-authorizes-first-antigen-test-help-rapid-detection-virus-causes}{authorization}
from the F.D.A. for coronavirus antigen tests. One is
\href{https://www.nytimes3xbfgragh.onion/2020/05/09/health/antigen-testing-fda-coronavirus.html}{Quidel},
which is, according to a representative, producing millions of tests
each month, many of which have been distributed to urgent care centers
and medical clinics in the United States. On Monday, a second firm,
Becton Dickinson \& Company, also entered the fray with a point-of-care
antigen test that can reportedly
\href{https://www.nytimes3xbfgragh.onion/reuters/2020/07/06/us/06reuters-health-coronavirus-becton-dickinson.html}{produce
results in 15 minutes}. While speedy, both Quidel's and BD's tests may
produce false negatives
\href{https://investors.bd.com/news-releases/news-release-details/bd-launches-portable-rapid-point-care-antigen-test-detect-sars}{between
15}
\href{https://www.sciencemag.org/news/2020/05/coronavirus-antigen-tests-quick-and-cheap-too-often-wrong}{and
20 percent} of the time.

Other antigen tests have made headway overseas, and experts estimated
that several more will likely seek clearance in the United States in
coming months.

One will likely come from medical device manufacturer OraSure, which has
made antigen tests for H.I.V. and Ebola. Stephen Tang, OraSure's
president and chief executive officer, said his team is brewing up a
``secret sauce'' that will make their coronavirus test highly accurate,
while still producing results within half an hour, but declined to
specify details.

Until these experimental tests are widely available, many people will
still need the nasty nasal swab.

``For any kind of normal life to resume, I think all of us need to get
this idea that we're going to have to get tested all the time,''
Binghamton's Ms. Coker said.

A faster, less invasive test would be nice. But even an unpleasant test
is better than no test at all, she said. ``If it's this painful one, so
be it.''

\textbf{\emph{{[}}\href{http://on.fb.me/1paTQ1h}{\emph{Like the Science
Times page on Facebook.}}} ****** \emph{\textbar{} Sign up for the}
\textbf{\href{http://nyti.ms/1MbHaRU}{\emph{Science Times
newsletter.}}\emph{{]}}}

Advertisement

\protect\hyperlink{after-bottom}{Continue reading the main story}

\hypertarget{site-index}{%
\subsection{Site Index}\label{site-index}}

\hypertarget{site-information-navigation}{%
\subsection{Site Information
Navigation}\label{site-information-navigation}}

\begin{itemize}
\tightlist
\item
  \href{https://help.nytimes3xbfgragh.onion/hc/en-us/articles/115014792127-Copyright-notice}{©~2020~The
  New York Times Company}
\end{itemize}

\begin{itemize}
\tightlist
\item
  \href{https://www.nytco.com/}{NYTCo}
\item
  \href{https://help.nytimes3xbfgragh.onion/hc/en-us/articles/115015385887-Contact-Us}{Contact
  Us}
\item
  \href{https://www.nytco.com/careers/}{Work with us}
\item
  \href{https://nytmediakit.com/}{Advertise}
\item
  \href{http://www.tbrandstudio.com/}{T Brand Studio}
\item
  \href{https://www.nytimes3xbfgragh.onion/privacy/cookie-policy\#how-do-i-manage-trackers}{Your
  Ad Choices}
\item
  \href{https://www.nytimes3xbfgragh.onion/privacy}{Privacy}
\item
  \href{https://help.nytimes3xbfgragh.onion/hc/en-us/articles/115014893428-Terms-of-service}{Terms
  of Service}
\item
  \href{https://help.nytimes3xbfgragh.onion/hc/en-us/articles/115014893968-Terms-of-sale}{Terms
  of Sale}
\item
  \href{https://spiderbites.nytimes3xbfgragh.onion}{Site Map}
\item
  \href{https://help.nytimes3xbfgragh.onion/hc/en-us}{Help}
\item
  \href{https://www.nytimes3xbfgragh.onion/subscription?campaignId=37WXW}{Subscriptions}
\end{itemize}
