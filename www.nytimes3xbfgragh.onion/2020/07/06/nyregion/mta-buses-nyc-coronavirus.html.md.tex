Sections

SEARCH

\protect\hyperlink{site-content}{Skip to
content}\protect\hyperlink{site-index}{Skip to site index}

\href{https://www.nytimes3xbfgragh.onion/section/nyregion}{New York}

\href{https://myaccount.nytimes3xbfgragh.onion/auth/login?response_type=cookie\&client_id=vi}{}

\href{https://www.nytimes3xbfgragh.onion/section/todayspaper}{Today's
Paper}

\href{/section/nyregion}{New York}\textbar{}Why New York Buses Are on
the Rise in a Subway City

\url{https://nyti.ms/2Z2W5IE}

\begin{itemize}
\item
\item
\item
\item
\item
\item
\end{itemize}

\hypertarget{the-coronavirus-outbreak}{%
\subsubsection{\texorpdfstring{\href{https://www.nytimes3xbfgragh.onion/news-event/coronavirus?name=styln-coronavirus-national\&region=TOP_BANNER\&block=storyline_menu_recirc\&action=click\&pgtype=Article\&impression_id=4db92960-f4c0-11ea-8629-2f0d22bd94c8\&variant=undefined}{The
Coronavirus
Outbreak}}{The Coronavirus Outbreak}}\label{the-coronavirus-outbreak}}

\begin{itemize}
\tightlist
\item
  live\href{https://www.nytimes3xbfgragh.onion/2020/09/11/world/covid-19-coronavirus.html?name=styln-coronavirus-national\&region=TOP_BANNER\&block=storyline_menu_recirc\&action=click\&pgtype=Article\&impression_id=4db95070-f4c0-11ea-8629-2f0d22bd94c8\&variant=undefined}{Latest
  Updates}
\item
  \href{https://www.nytimes3xbfgragh.onion/interactive/2020/us/coronavirus-us-cases.html?name=styln-coronavirus-national\&region=TOP_BANNER\&block=storyline_menu_recirc\&action=click\&pgtype=Article\&impression_id=4db95071-f4c0-11ea-8629-2f0d22bd94c8\&variant=undefined}{Maps
  and Cases}
\item
  \href{https://www.nytimes3xbfgragh.onion/interactive/2020/science/coronavirus-vaccine-tracker.html?name=styln-coronavirus-national\&region=TOP_BANNER\&block=storyline_menu_recirc\&action=click\&pgtype=Article\&impression_id=4db95072-f4c0-11ea-8629-2f0d22bd94c8\&variant=undefined}{Vaccine
  Tracker}
\item
  \href{https://www.nytimes3xbfgragh.onion/2020/09/10/us/politics/fda-coronavirus-vaccine.html?name=styln-coronavirus-national\&region=TOP_BANNER\&block=storyline_menu_recirc\&action=click\&pgtype=Article\&impression_id=4db95073-f4c0-11ea-8629-2f0d22bd94c8\&variant=undefined}{F.D.A.
  Regulators' Self-Defense}
\item
  \href{https://www.nytimes3xbfgragh.onion/2020/09/09/upshot/coronavirus-surprise-test-fees.html?name=styln-coronavirus-national\&region=TOP_BANNER\&block=storyline_menu_recirc\&action=click\&pgtype=Article\&impression_id=4dc00730-f4c0-11ea-8629-2f0d22bd94c8\&variant=undefined}{Surprise
  Test Fees}
\end{itemize}

Advertisement

\protect\hyperlink{after-top}{Continue reading the main story}

Supported by

\protect\hyperlink{after-sponsor}{Continue reading the main story}

\hypertarget{why-new-york-buses-are-on-the-rise-in-a-subway-city}{%
\section{Why New York Buses Are on the Rise in a Subway
City}\label{why-new-york-buses-are-on-the-rise-in-a-subway-city}}

During the coronavirus pandemic, daily ridership on buses has surpassed
the subway for the first time in over half a century.

\includegraphics{https://static01.graylady3jvrrxbe.onion/images/2020/06/30/nyregion/00nyvirus-buses1/merlin_172230249_66000bfe-c5be-4872-b085-85cd8fa5ce60-articleLarge.jpg?quality=75\&auto=webp\&disable=upscale}

\href{https://www.nytimes3xbfgragh.onion/by/christina-goldbaum}{\includegraphics{https://static01.graylady3jvrrxbe.onion/images/2019/11/22/reader-center/author-christina-goldbaum/author-christina-goldbaum-thumbLarge.png}}\href{https://www.nytimes3xbfgragh.onion/by/winnie-hu}{\includegraphics{https://static01.graylady3jvrrxbe.onion/images/2018/06/13/multimedia/author-winnie-hu/author-winnie-hu-thumbLarge.jpg}}

By
\href{https://www.nytimes3xbfgragh.onion/by/christina-goldbaum}{Christina
Goldbaum} and
\href{https://www.nytimes3xbfgragh.onion/by/winnie-hu}{Winnie Hu}

\begin{itemize}
\item
  July 6, 2020
\item
  \begin{itemize}
  \item
  \item
  \item
  \item
  \item
  \item
  \end{itemize}
\end{itemize}

In the battle for riders,
\href{https://www.nytimes3xbfgragh.onion/2020/07/21/nyregion/mta-subway-financial-cuts.html}{New
York City's subway} has always trounced buses. By a lot.

But at the height of the coronavirus pandemic the equation was flipped
on its head --- average daily ridership in April and May was 444,000 on
the subway and 505,000 on the buses.

It was the first time that happened since transit officials started
keeping such records more than half a century ago.

Buses have held on to their lead even as the city has begun reopening
after a three-month shutdown and more commuters return to work. Average
daily counts in June were 752,000 riders for the subway --- and 830,000
riders for the buses.

The city's sprawling bus system, which has long been overshadowed by the
subway, has emerged as a crucial part of its recovery. Buses are being
counted on to keep people out of cars and to relieve subway crowding as
more commuters come back, drawing many riders who said they felt buses
were a safer and less-stressful alternative because riders can wait
outside for the bus, see how clean or crowded a bus is before boarding,
and hop off at any time and be back outside again.

``I'm more comfortable on the bus,'' said Arturo Carrion, 52, who works
as a cleaner for Uber. ``The train is tight with a lot of people like
sardines.''

Buses also reach into parts of the city where the subway doesn't and
serve a less well-off ridership. And when the subway started shutting
down overnight for cleaning in May some workers turned to buses to get
to their jobs.

To speed up buses, Mayor Bill de Blasio said the city would
\href{https://www.nytimes3xbfgragh.onion/2020/06/08/nyregion/coronavirus-nyc-reopen-phase-1.html}{install
five busways} that would push cars off some of New York's busiest
arteries, including Fifth Avenue in Manhattan and Main Street in
Flushing, Queens.

\includegraphics{https://static01.graylady3jvrrxbe.onion/images/2020/06/30/nyregion/00nyvirus-buses3/merlin_172224606_23094a75-24a7-4e44-ad25-b4761d11cee9-articleLarge.jpg?quality=75\&auto=webp\&disable=upscale}

The mayor had been under growing pressure from advocates and bus riders
to create more busways before the pandemic, and the outbreak has
intensified a focus on how better public transit can reduce car traffic
as the city slowly resumes normal life and more people return to work.

The city opened a busway last fall
\href{https://www.nytimes3xbfgragh.onion/2019/10/03/nyregion/car-ban-14th-street-manhattan.html}{on
14th Street} in Lower Manhattan that has significantly boosted bus
speeds and ridership. Before the pandemic, the average time it took to
complete a trip had dropped by 36 percent. Weekday ridership had
increased 19 percent and as much as 25 percent during morning rush hour.

\hypertarget{latest-updates-the-coronavirus-outbreak}{%
\section{\texorpdfstring{\href{https://www.nytimes3xbfgragh.onion/2020/09/11/world/covid-19-coronavirus.html?action=click\&pgtype=Article\&state=default\&region=MAIN_CONTENT_1\&context=storylines_live_updates}{Latest
Updates: The Coronavirus
Outbreak}}{Latest Updates: The Coronavirus Outbreak}}\label{latest-updates-the-coronavirus-outbreak}}

Updated 2020-09-12T06:16:33.399Z

\begin{itemize}
\tightlist
\item
  \href{https://www.nytimes3xbfgragh.onion/2020/09/11/world/covid-19-coronavirus.html?action=click\&pgtype=Article\&state=default\&region=MAIN_CONTENT_1\&context=storylines_live_updates\#link-dfb8a16}{Fauci
  cautions the virus could disrupt life in the U.S. until `maybe even
  towards the end of 2021.'}
\item
  \href{https://www.nytimes3xbfgragh.onion/2020/09/11/world/covid-19-coronavirus.html?action=click\&pgtype=Article\&state=default\&region=MAIN_CONTENT_1\&context=storylines_live_updates\#link-7104d154}{From
  Asia to Africa, China promotes its vaccine candidates to win friends.}
\item
  \href{https://www.nytimes3xbfgragh.onion/2020/09/11/world/covid-19-coronavirus.html?action=click\&pgtype=Article\&state=default\&region=MAIN_CONTENT_1\&context=storylines_live_updates\#link-393ad215}{The
  other way the virus will kill: hunger.}
\end{itemize}

\href{https://www.nytimes3xbfgragh.onion/2020/09/11/world/covid-19-coronavirus.html?action=click\&pgtype=Article\&state=default\&region=MAIN_CONTENT_1\&context=storylines_live_updates}{See
more updates}

More live coverage:
\href{https://www.nytimes3xbfgragh.onion/live/2020/09/11/business/stock-market-today-coronavirus?action=click\&pgtype=Article\&state=default\&region=MAIN_CONTENT_1\&context=storylines_live_updates}{Markets}

Still, the new busways have angered some business owners who are already
struggling to survive the economic fallout of the virus.

``This is Queens --- people here drive,'' said Dian Song, executive
director of the downtown Flushing Business Improvement District, which
serves around 2,000 businesses. ``Adding driving restrictions on Main
Street, you will scare away those customers. You are really going to
bankrupt those businesses.''

The health crisis that has changed so much about New York has upended
its transit patterns and unexpectedly allowed buses to shine. By most
measures, buses have received far less attention and resources than the
subway. If the subway was slow and crowded, the buses were usually
worse. The buses were often the ride of last resort for those moving
about the city.

When the subway plunged into a crisis in 2017, Gov. Andrew M. Cuomo
\href{https://www.nytimes3xbfgragh.onion/2017/06/29/nyregion/cuomo-declares-a-state-of-emergency-for-the-subway.html}{declared
a state of emergency}. Yet the buses have been on a steady decline for
more than a decade. Bus speeds dropped year after year --- to almost
four miles per hour --- as congestion worsened. Riders fled for faster
options, including Uber, Lyft and Citi Bike.

Image

During the height of the pandemic, subway ridership fell below bus
ridership for the first time since the Metropolitan Transportation
Authority started recording such data in 1963.Credit...Demetrius Freeman
for The New York Times

But during the worst of the pandemic, as subway ridership was wiped out,
buses still carried as many as half their riders, including essential
workers.

``This is what I have to do to get to work,'' said Jackie Inabinet, 57,
a security guard who never stopped riding the bus from her home in Crown
Heights, Brooklyn, to her job in Long Island City, Queens.

Other riders are newcomers to the bus like Toddara Galimore, 23, a
junior office manager in Brooklyn who traded in the J train for the B44
bus. ``I can see the outside, said Ms. Galimore, who never took the bus
to work until the pandemic. ``If I need to get off quick, I can get off
fast.''

Bus service even improved. With the city nearly shut down, buses zoomed
down empty streets --- at speeds up to 19 percent faster than normal ---
in a tantalizing glimpse of just how much better service could be.

``Buses are no longer seen as second tier anymore,'' said Tom Wright,
the president of the Regional Plan Association, an influential planning
group.

Across the nation, buses have lost ground to subways and trains for
decades even as city populations grew and local economies boomed. Bus
ridership fell every year for the last seven years, reaching its lowest
level last year since the early 1970s.

Bus systems have been battered by reductions in service, competition
from ride-hail services and bike share programs, and low gas prices and
car loan interest rates that made car ownership more appealing,
according to
\href{https://www.nytimes3xbfgragh.onion/interactive/2020/03/13/upshot/mystery-of-missing-bus-riders.html}{transit
researchers}.

But since the pandemic, buses have increasingly emerged as a reliable,
flexible and efficient way to bolster public transit systems that face
their worst financial crisis in generations.

``This may be the start of a comeback for buses,'' said Joseph P.
Schwieterman, a professor of public service at DePaul University.
``Buses are versatile in a time of crisis. They serve a wider range of
riders than trains. As transit agencies pinch pennies to stop the flow
of red ink, the bus may take center stage.''

Image

The city created a busway last fall along 14th Street that largely
banned cars and has sped up travel times and attracted more
riders.~Credit...Kirsten Luce for The New York Times

In Los Angeles and Washington, bus ridership dropped only about a third
during the pandemic, according to analysis from the Eno Center for
Transportation, a nonpartisan research group in Washington.

In Seattle, extra buses had to be added to a half-dozen routes to ensure
that riders, including many essential workers, had enough room for
social distancing. ``The bus system in Seattle is deeply ingrained in
our culture,'' said Bill Bryant, a King County Metro official.

New York's buses carried more riders than the subway every day for more
than two months --- the first time that has happened since the
Metropolitan Transportation Authority began keeping track in 1963.

Bus ridership dropped to a low of 430,000 riders one day in April, or 20
percent of pre-pandemic levels. Subway ridership hit bottom with 403,000
daily riders, a 93 percent drop.

Transportation experts said expanding the city's bus network was
essential to attracting more riders, especially in transit deserts
outside Manhattan without subway lines. Improvements to buses can also
be made faster and cheaper than to the subway, they said.

``The city's bus system has always sort of been the unwanted step
sibling of transit in New York,'' said Janette Sadik-Khan, a former city
transportation commissioner. ``But buses are a more attractive option
when they can operate above ground just like subways operate
underground.''

Kate Slevin, a senior vice president of the Regional Plan Association,
said faster bus trips would also mean less time that riders could be
potentially exposed to the virus. ``It's a public health issue and an
equity issue as well,'' she said. ``The last thing you want is essential
workers stuck in traffic behind single-occupant vehicles.

Currently, there are 144 miles of bus lanes across the city, and M.T.A.
officials have called for an additional 60 miles. City officials have
promised a total of 20 miles of new bus lanes, including the five new
busways and four new bus lanes that together will serve about 750,000
daily riders.

In recent years, the city has also expanded a program that gives buses
priority at traffic signals to 1,025 intersections, or about 13 percent
of all such intersections along bus routes.

``We can do this --- we can have buses that are faster and more
reliable,'' said Polly Trottenberg, the city's transportation
commissioner.

Even before the pandemic, M.T.A. officials were overhauling New York's
bus system by redesigning the city's
\href{https://www.nytimes3xbfgragh.onion/2018/08/28/nyregion/bus-routes-nyc-transit.html}{outdated
and inefficient bus network} borough by borough, and installing
\href{http://www.mta.info/press-release/nyc-transit/mta-bus-mounted-camera-program-begins-issuing-bus-lane-violations-m15-sbs}{bus-mounted
cameras} to issue tickets to cars that blocked bus lanes.

``When people are getting to where they want to go in a safe and
reliable manner, the bus is a viable alternative to subway or rail,''
said Craig Cipriano, acting president of the M.T.A. Bus Company. ``We
need to build upon this momentum.''

Nate Schweber contributed reporting.

Advertisement

\protect\hyperlink{after-bottom}{Continue reading the main story}

\hypertarget{site-index}{%
\subsection{Site Index}\label{site-index}}

\hypertarget{site-information-navigation}{%
\subsection{Site Information
Navigation}\label{site-information-navigation}}

\begin{itemize}
\tightlist
\item
  \href{https://help.nytimes3xbfgragh.onion/hc/en-us/articles/115014792127-Copyright-notice}{©~2020~The
  New York Times Company}
\end{itemize}

\begin{itemize}
\tightlist
\item
  \href{https://www.nytco.com/}{NYTCo}
\item
  \href{https://help.nytimes3xbfgragh.onion/hc/en-us/articles/115015385887-Contact-Us}{Contact
  Us}
\item
  \href{https://www.nytco.com/careers/}{Work with us}
\item
  \href{https://nytmediakit.com/}{Advertise}
\item
  \href{http://www.tbrandstudio.com/}{T Brand Studio}
\item
  \href{https://www.nytimes3xbfgragh.onion/privacy/cookie-policy\#how-do-i-manage-trackers}{Your
  Ad Choices}
\item
  \href{https://www.nytimes3xbfgragh.onion/privacy}{Privacy}
\item
  \href{https://help.nytimes3xbfgragh.onion/hc/en-us/articles/115014893428-Terms-of-service}{Terms
  of Service}
\item
  \href{https://help.nytimes3xbfgragh.onion/hc/en-us/articles/115014893968-Terms-of-sale}{Terms
  of Sale}
\item
  \href{https://spiderbites.nytimes3xbfgragh.onion}{Site Map}
\item
  \href{https://help.nytimes3xbfgragh.onion/hc/en-us}{Help}
\item
  \href{https://www.nytimes3xbfgragh.onion/subscription?campaignId=37WXW}{Subscriptions}
\end{itemize}
