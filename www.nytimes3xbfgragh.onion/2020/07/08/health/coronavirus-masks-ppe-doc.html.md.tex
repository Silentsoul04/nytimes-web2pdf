Sections

SEARCH

\protect\hyperlink{site-content}{Skip to
content}\protect\hyperlink{site-index}{Skip to site index}

\href{https://www.nytimes3xbfgragh.onion/section/health}{Health}

\href{https://myaccount.nytimes3xbfgragh.onion/auth/login?response_type=cookie\&client_id=vi}{}

\href{https://www.nytimes3xbfgragh.onion/section/todayspaper}{Today's
Paper}

\href{/section/health}{Health}\textbar{}Grave Shortages of Protective
Gear Flare Again as Covid Cases Surge

\url{https://nyti.ms/2ZSJExZ}

\begin{itemize}
\item
\item
\item
\item
\item
\end{itemize}

\hypertarget{the-coronavirus-outbreak}{%
\subsubsection{\texorpdfstring{\href{https://www.nytimes3xbfgragh.onion/news-event/coronavirus?name=styln-coronavirus-national\&region=TOP_BANNER\&block=storyline_menu_recirc\&action=click\&pgtype=Article\&impression_id=97a76520-f2d4-11ea-b9d1-d10fb06eae1c\&variant=undefined}{The
Coronavirus
Outbreak}}{The Coronavirus Outbreak}}\label{the-coronavirus-outbreak}}

\begin{itemize}
\tightlist
\item
  live\href{https://www.nytimes3xbfgragh.onion/2020/09/09/world/covid-coronavirus.html?name=styln-coronavirus-national\&region=TOP_BANNER\&block=storyline_menu_recirc\&action=click\&pgtype=Article\&impression_id=97a76521-f2d4-11ea-b9d1-d10fb06eae1c\&variant=undefined}{Latest
  Updates}
\item
  \href{https://www.nytimes3xbfgragh.onion/interactive/2020/us/coronavirus-us-cases.html?name=styln-coronavirus-national\&region=TOP_BANNER\&block=storyline_menu_recirc\&action=click\&pgtype=Article\&impression_id=97a78c30-f2d4-11ea-b9d1-d10fb06eae1c\&variant=undefined}{Maps
  and Cases}
\item
  \href{https://www.nytimes3xbfgragh.onion/interactive/2020/science/coronavirus-vaccine-tracker.html?name=styln-coronavirus-national\&region=TOP_BANNER\&block=storyline_menu_recirc\&action=click\&pgtype=Article\&impression_id=97a78c31-f2d4-11ea-b9d1-d10fb06eae1c\&variant=undefined}{Vaccine
  Tracker}
\item
  \href{https://www.nytimes3xbfgragh.onion/2020/09/02/your-money/eviction-moratorium-covid.html?name=styln-coronavirus-national\&region=TOP_BANNER\&block=storyline_menu_recirc\&action=click\&pgtype=Article\&impression_id=97a78c32-f2d4-11ea-b9d1-d10fb06eae1c\&variant=undefined}{Eviction
  Moratorium}
\item
  \href{https://www.nytimes3xbfgragh.onion/2020/09/09/upshot/coronavirus-surprise-test-fees.html?name=styln-coronavirus-national\&region=TOP_BANNER\&block=storyline_menu_recirc\&action=click\&pgtype=Article\&impression_id=97a78c33-f2d4-11ea-b9d1-d10fb06eae1c\&variant=undefined}{Surprise
  Test Fees}
\end{itemize}

Advertisement

\protect\hyperlink{after-top}{Continue reading the main story}

Supported by

\protect\hyperlink{after-sponsor}{Continue reading the main story}

\hypertarget{grave-shortages-of-protective-gear-flare-again-as-covid-cases-surge}{%
\section{Grave Shortages of Protective Gear Flare Again as Covid Cases
Surge}\label{grave-shortages-of-protective-gear-flare-again-as-covid-cases-surge}}

Five months into the pandemic, the U.S. still hasn't solved the problem.
The dearth of supplies is affecting a broad array of health facilities,
renewing pleas for White House intervention.

\includegraphics{https://static01.graylady3jvrrxbe.onion/images/2020/07/08/science/08virus-PPE/merlin_174331119_0dd57367-3954-4feb-9edc-741db800c07b-articleLarge.jpg?quality=75\&auto=webp\&disable=upscale}

By \href{https://www.nytimes3xbfgragh.onion/by/andrew-jacobs}{Andrew
Jacobs}

\begin{itemize}
\item
  July 8, 2020
\item
  \begin{itemize}
  \item
  \item
  \item
  \item
  \item
  \end{itemize}
\end{itemize}

As coronavirus cases surge across the country, hospitals, nursing homes
and private medical practices are facing a problem many had hoped would
be resolved by now: a dire shortage of respirator masks, isolation gowns
and disposable gloves that protect front-line medical workers from
infection.

Unlike the crisis that caught a handful of big city hospitals off guard
in March and April, the soaring demand for protective gear is now
affecting a broad range of medical facilities across the country, a
problem public health experts and major medical associations say could
have been avoided if the federal government had embraced a more
aggressive approach toward procuring and distributing critical supplies
in the early days of the pandemic.

Doctors at Memorial City Medical Center in Houston who treat Covid-19
patients have been told to reuse single-use N95 respirator masks for up
to 15 days before throwing them out. The country's largest organization
of registered nurses found in a
\href{https://www.nationalnursesunited.org/covid-19-survey}{survey of
its members in late June}that 85 percent had been forced to reuse
disposable N95 masks while treating coronavirus patients. In Florida,
some hospitals are handing out only loosefitting surgical masks to
workers treating newly admitted patients who may be asymptomatic
carriers.

The inability to find personal protective equipment, known as P.P.E., is
starting to impede other critical areas of medicine too. Neurologists,
cardiologists and cancer specialists around the country have been unable
to reopen their offices in recent weeks, leaving many patients without
care, according to the American Medical Association and other doctor
groups.

``We have kids living with grapefruit-sized abscesses for over three
months who can't eat or drink and there's nothing we can do for them
because we can't get P.P.E.,'' said Kay Kennel, the chief officer of
Lubbock Kids Dental, a clinic serving low-income families in Texas that
has a list of 50 children awaiting emergency surgery. ``It's been just
horrible, and given the growing number of infections here, I'm afraid
things are going to get worse.''

In a coronavirus briefing on Wednesday, Vice President Mike Pence
downplayed the shortages, but said the government was preparing to issue
new guidance on the preservation and reuse of protective gear. ``P.P.E.,
we hear, remains very strong,'' he said.

Many of the problems of early spring, when hospital workers in New York,
New Jersey, Michigan, California and other states first walloped by the
virus scrambled to obtain rudimentary protective gear, have only grown.
The United States remains dependent on overseas manufacturers and
fly-by-night middlemen who have jacked up prices sevenfold amid soaring
global demand, according to supply chain specialists and public health
experts, who warn that the problem will intensify as the pandemic
spreads. The handful of American companies still making protective
equipment domestically say they are already at maximum capacity.

``It's been chaos for us,'' said Randy Bury, president of the Good
Samaritan Society, which has struggled to keep its 200 nursing homes
supplied with hand sanitizer, masks and gowns. ``The supply chain in the
United States is not healthy, and we've learned we cannot depend on the
government.''

The crisis has reinvigorated calls for President Trump to invoke the
Defense Production Act and order American manufacturers to step in and
help. The presumptive Democratic presidential nominee, former Vice
President Joseph R. Biden Jr., said this week that he
\href{https://www.documentcloud.org/documents/6982369-Biden-Supply-Chain-Fact-Sheet-07-07-20.html?utm_source=newsletter\&utm_medium=email\&utm_campaign=newsletter_axiosvitals\&stream=top}{would
use that law} to boost domestic protection of medical gear if elected.

``It's incredibly frustrating because a lot of attention was paid to the
need for ventilators early on in the pandemic, but now we're realizing
that there's going to be a tremendous ongoing need for simple things
like masks, gowns and face shields,'' said Dr. Susan R. Bailey,
president of the American Medical Association, which last week
\href{https://searchlf.ama-assn.org/undefined/documentDownload?uri=\%2Funstructured\%2Fbinary\%2Fletter\%2FLETTERS\%2F2020-6-30-Letter-to-Pence-re-PPE.pdf}{wrote
a letter to Mr. Pence} urging the administration to use the Defense
Production Act. ``We need a national coordinated strategy.''

\hypertarget{latest-updates-the-coronavirus-outbreak}{%
\section{\texorpdfstring{\href{https://www.nytimes3xbfgragh.onion/2020/09/09/world/covid-coronavirus.html?action=click\&pgtype=Article\&state=default\&region=MAIN_CONTENT_1\&context=storylines_live_updates}{Latest
Updates: The Coronavirus
Outbreak}}{Latest Updates: The Coronavirus Outbreak}}\label{latest-updates-the-coronavirus-outbreak}}

Updated 2020-09-09T19:39:09.983Z

\begin{itemize}
\tightlist
\item
  \href{https://www.nytimes3xbfgragh.onion/2020/09/09/world/covid-coronavirus.html?action=click\&pgtype=Article\&state=default\&region=MAIN_CONTENT_1\&context=storylines_live_updates\#link-279e24e2}{Top
  U.S. health officials update Congress on vaccine development and
  distribution plans.}
\item
  \href{https://www.nytimes3xbfgragh.onion/2020/09/09/world/covid-coronavirus.html?action=click\&pgtype=Article\&state=default\&region=MAIN_CONTENT_1\&context=storylines_live_updates\#link-38cb0bfc}{The
  AstraZeneca vaccine trial was halted after a person enrolled in it
  developed a rare inflammatory condition.}
\item
  \href{https://www.nytimes3xbfgragh.onion/2020/09/09/world/covid-coronavirus.html?action=click\&pgtype=Article\&state=default\&region=MAIN_CONTENT_1\&context=storylines_live_updates\#link-792ae257}{Indoor
  dining in N.Y.C. will return with limits on Sept. 30, Cuomo says.}
\end{itemize}

\href{https://www.nytimes3xbfgragh.onion/2020/09/09/world/covid-coronavirus.html?action=click\&pgtype=Article\&state=default\&region=MAIN_CONTENT_1\&context=storylines_live_updates}{See
more updates}

More live coverage:
\href{https://www.nytimes3xbfgragh.onion/live/2020/09/09/business/stock-market-today-coronavirus?action=click\&pgtype=Article\&state=default\&region=MAIN_CONTENT_1\&context=storylines_live_updates}{Markets}

In recent weeks, congressional Democrats along with a growing number of
governors and medical associations have been urging the White House to
play a more muscular role in the production, procurement and
distribution of crucial supplies. They are also urging the
administration to tackle the flagrant price gouging that has frozen many
long-term care facilities, low-income health clinics and small hospitals
out of the market.

Mr. Trump has
\href{https://www.nytimes3xbfgragh.onion/2020/03/20/us/politics/trump-coronavirus-supplies.html}{resisted
using federal powers} to address the problem, saying in March that
individual governors should find their own gear because ``We're not a
shipping clerk.'' With the National Strategic Stockpile depleted, states
have been left to fend for themselves, though the Federal Emergency
Management Agency has been distributing modest shipments of gear to
nursing homes and long-term care facilities.

At \href{https://getusppe.org/}{GetUsPPE}, a volunteer organization that
helps health care facilities and workers find protective gear, demand
has been rising sharply in states experiencing a surge of infections. In
June, the amount of P.P.E. requested from medical providers **** in Iowa
jumped 440 percent from the previous month, and more than 200 percent in
Texas and Louisiana.

``I feel horrible for the health care workers and hospitals that are
dealing with this,'' said Dr. Ali Raja, a founder of the organization
and an emergency room doctor at Massachusetts General Hospital. ``They
are crying out for help.''

Members of National Nurses United, the country's largest organization of
registered nurses, said they were worried about the ability of reused
masks to filter out virus particles after so much wear and tear. Many
are also concerned about the health implications of a chemical
decontamination process recently approved for emergency use by the Food
and Drug Administration that involves spraying soiled masks with
hydrogen peroxide. The F.D.A. has also granted emergency authorization
for decontamination procedures that use ultraviolet irradiation and
moist heat, though
\href{https://www.cdc.gov/coronavirus/2019-ncov/hcp/ppe-strategy/decontamination-reuse-respirators.html}{regulators
acknowledge} that reusing disposable masks is less than ideal.

``Nurses and health care workers are being forced to reuse masks with an
unproven system,'' said Deborah Burger, the organization's co-president.
``It's almost five months into a pandemic in the richest country in the
world and we're putting people's lives at risk because we don't have
enough P.P.E.''

\includegraphics{https://static01.graylady3jvrrxbe.onion/images/2020/07/08/science/08virus-PPE02/08virus-PPE02-articleLarge.jpg?quality=75\&auto=webp\&disable=upscale}

The risks are not abstract. More than 900 health care workers have died
of Covid-19, according to a tally by the organization, and Ms. Burger
said many of the deaths have been linked to inadequate protective gear.

``There are tools at President Trump's disposal and he has failed us,''
she said. ``These deaths are entirely preventable.''

FEMA has been distributing 14-day supplies of gear to nursing homes, but
many providers have quickly burned through the shipments. There have
also been widespread complaints about defective equipment, including
child-size gloves, gowns without armholes and loosefitting cloth masks
that are ineffective for filtering out virus particles, according to
\href{https://www.leadingage.org/}{LeadingAge}, a national association
of nonprofit care providers. The dearth of protective equipment at
facilities serving older adults has prompted mounting alarm among public
health experts. More than 40 percent of all coronavirus deaths have been
\href{https://www.nytimes3xbfgragh.onion/interactive/2020/us/coronavirus-nursing-homes.html}{linked
to nursing homes} and long-term care centers, according to a tally by
The New York Times.

FEMA said in a statement that it had made changes to most recent
shipments in response to feedback from recipients.

The national free-for-all to obtain scarce protective gear has favored
large hospital chains with procurement professionals and established
supply chains, but even deep-pocketed institutions have been rationing
masks and gowns. At St. Petersburg General Hospital in Florida --- part
of HCA Healthcare, a for-profit chain that includes more than 2,000
hospitals, clinics and surgery centers --- medical staff members said
they were given a single surgical mask each day to make their rounds;
only those assigned to the Covid ward were allowed access to N95s, which
are kept under lock and key.

``If you sneeze in your mask, you still have to wear it your entire
shift,'' said Barbara Murray, a nurse at St. Petersburg General.

Ms. Murray said medical staff members worried that surgical masks
offered little protection when treating asymptomatic carriers of the
virus. She said she was increasingly seized with anxiety as the hospital
filled up with coronavirus patients, some of them sent from local
nursing homes, because staff members lacked even basic protective gear
and were unable to care for them.

Hospital administrators, she said, won't even allow employees to wear
N95 masks they have purchased with their own money. ``We're nurses ---
we want to take care of our patients and we want them to be safe,'' Ms.
Murray said. ``But at the end of the day, we want to go home to our
families and know that they are safe too.''

A spokeswoman for St. Petersburg General declined to comment on the
hospital's mask policies but said adequate supplies were available to
employees who needed them.

Across the country, private medical offices, especially those without
access to group purchasing networks, are struggling to get protective
gear on the open market. Even when they can find them on Amazon and
other websites, doctors say they are paying up to \$7 for N95 masks that
sold for less than a dollar before the pandemic.

``Community physicians have it worse because we are at the bottom of the
totem pole,'' said Dr. Inderpal S. Chhabra, an internal medicine
specialist in New Hyde Park, N.Y., who recently reopened his office but
could see only four or five patients a day because of limited supplies.
``Everyone is running around like crazy trying to get N95s, but no one
can get them. I afraid for my staff.''

At Arizona Community Physicians, a private health clinic in Tucson,
medical technicians are not given N95 masks but they are still required
to see Covid-19 patients, who arrive for nonemergency procedures like
mammograms, ultrasounds and chest X-rays, according to two employees who
asked to remain anonymous for fear they could lose their jobs. The
employees say they have been unable to buy medical grade N95 masks
online; some vendors have run out of supplies while others say they
won't sell to individuals. ``Every day I go into work and I am scared to
death --- not just for myself, but for my family,'' one worker said.

Arizona Community Physicians did not respond to emails and phone
messages seeking comment. A spokesman for the Arizona Department of
Health Services said state regulations for protective gear did not apply
to private clinics.

That's not the case in Texas, which requires health facilities to have
adequate equipment before reopening. State officials said they had
distributed 500,000 respirator masks to dental offices, but Ms. Kennel,
the chief officer at Lubbock Kids Dental, said her clinic was not among
the recipients.

Her employees spend much of their day on the phone trying to calm the
parents of children in severe pain. Others show up at the door with
their children and beg for help. With dental clinics across the state
facing the same problem, the staff can only prescribe antibiotics and
tell caregivers to sit tight. Her greatest fear is that an untreated
abscess will enter the bloodstream and turn fatal, a preventable death
that has claimed
\href{https://well.blogs.nytimes3xbfgragh.onion/2013/08/30/oral-infections-causing-more-hospitalizations/}{dozens
of lives} in recent years.

``We're not talking about silver crowns, teeth cleaning or veneers,''
Ms. Kennel said, her voice choking with emotion. ``These are children
with severe infections, and there is nothing we can do for them. It's
just heartbreaking.''

Advertisement

\protect\hyperlink{after-bottom}{Continue reading the main story}

\hypertarget{site-index}{%
\subsection{Site Index}\label{site-index}}

\hypertarget{site-information-navigation}{%
\subsection{Site Information
Navigation}\label{site-information-navigation}}

\begin{itemize}
\tightlist
\item
  \href{https://help.nytimes3xbfgragh.onion/hc/en-us/articles/115014792127-Copyright-notice}{©~2020~The
  New York Times Company}
\end{itemize}

\begin{itemize}
\tightlist
\item
  \href{https://www.nytco.com/}{NYTCo}
\item
  \href{https://help.nytimes3xbfgragh.onion/hc/en-us/articles/115015385887-Contact-Us}{Contact
  Us}
\item
  \href{https://www.nytco.com/careers/}{Work with us}
\item
  \href{https://nytmediakit.com/}{Advertise}
\item
  \href{http://www.tbrandstudio.com/}{T Brand Studio}
\item
  \href{https://www.nytimes3xbfgragh.onion/privacy/cookie-policy\#how-do-i-manage-trackers}{Your
  Ad Choices}
\item
  \href{https://www.nytimes3xbfgragh.onion/privacy}{Privacy}
\item
  \href{https://help.nytimes3xbfgragh.onion/hc/en-us/articles/115014893428-Terms-of-service}{Terms
  of Service}
\item
  \href{https://help.nytimes3xbfgragh.onion/hc/en-us/articles/115014893968-Terms-of-sale}{Terms
  of Sale}
\item
  \href{https://spiderbites.nytimes3xbfgragh.onion}{Site Map}
\item
  \href{https://help.nytimes3xbfgragh.onion/hc/en-us}{Help}
\item
  \href{https://www.nytimes3xbfgragh.onion/subscription?campaignId=37WXW}{Subscriptions}
\end{itemize}
