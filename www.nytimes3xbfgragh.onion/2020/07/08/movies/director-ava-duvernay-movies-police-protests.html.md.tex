Sections

SEARCH

\protect\hyperlink{site-content}{Skip to
content}\protect\hyperlink{site-index}{Skip to site index}

\href{https://www.nytimes3xbfgragh.onion/section/movies}{Movies}

\href{https://myaccount.nytimes3xbfgragh.onion/auth/login?response_type=cookie\&client_id=vi}{}

\href{https://www.nytimes3xbfgragh.onion/section/todayspaper}{Today's
Paper}

\href{/section/movies}{Movies}\textbar{}Ava DuVernay's Fight for Change,
Onscreen and Off

\url{https://nyti.ms/3gHADPs}

\begin{itemize}
\item
\item
\item
\item
\item
\end{itemize}

\hypertarget{race-and-america}{%
\subsubsection{\texorpdfstring{\href{https://www.nytimes3xbfgragh.onion/news-event/george-floyd-protests-minneapolis-new-york-los-angeles?name=styln-george-floyd\&region=TOP_BANNER\&block=storyline_menu_recirc\&action=click\&pgtype=Article\&impression_id=87479410-f2bc-11ea-9844-b5eaae65ed1c\&variant=undefined}{Race
and America}}{Race and America}}\label{race-and-america}}

\begin{itemize}
\tightlist
\item
  \href{https://www.nytimes3xbfgragh.onion/2020/09/04/nyregion/rochester-police-daniel-prude.html?name=styln-george-floyd\&region=TOP_BANNER\&block=storyline_menu_recirc\&action=click\&pgtype=Article\&impression_id=87479411-f2bc-11ea-9844-b5eaae65ed1c\&variant=undefined}{What
  Happened in Rochester, N.Y.}
\item
  \href{https://www.nytimes3xbfgragh.onion/2020/09/01/us/politics/trump-fact-check-protests.html?name=styln-george-floyd\&region=TOP_BANNER\&block=storyline_menu_recirc\&action=click\&pgtype=Article\&impression_id=87479412-f2bc-11ea-9844-b5eaae65ed1c\&variant=undefined}{Trump
  Fact Check}
\item
  \href{https://www.nytimes3xbfgragh.onion/2020/08/30/us/portland-shooting-explained.html?name=styln-george-floyd\&region=TOP_BANNER\&block=storyline_menu_recirc\&action=click\&pgtype=Article\&impression_id=8747bb20-f2bc-11ea-9844-b5eaae65ed1c\&variant=undefined}{Portland
  Shooting}
\item
  \href{https://www.nytimes3xbfgragh.onion/2020/08/30/us/breonna-taylor-police-killing.html?name=styln-george-floyd\&region=TOP_BANNER\&block=storyline_menu_recirc\&action=click\&pgtype=Article\&impression_id=8747bb21-f2bc-11ea-9844-b5eaae65ed1c\&variant=undefined}{Breonna
  Taylor's Life and Death}
\end{itemize}

Advertisement

\protect\hyperlink{after-top}{Continue reading the main story}

Supported by

\protect\hyperlink{after-sponsor}{Continue reading the main story}

In her words

\hypertarget{ava-duvernays-fight-for-change-onscreen-and-off}{%
\section{Ava DuVernay's Fight for Change, Onscreen and
Off}\label{ava-duvernays-fight-for-change-onscreen-and-off}}

The award-winning film director on the role of artists in a time of
widespread unrest, and how film can help hold police to account.

\includegraphics{https://static01.graylady3jvrrxbe.onion/images/2020/02/10/us/6ihw-duvernay/6ihw-duvernay-articleLarge-v2.jpg?quality=75\&auto=webp\&disable=upscale}

By \href{https://www.nytimes3xbfgragh.onion/by/emma-goldberg}{Emma
Goldberg}

\begin{itemize}
\item
  July 8, 2020
\item
  \begin{itemize}
  \item
  \item
  \item
  \item
  \item
  \end{itemize}
\end{itemize}

\hypertarget{an-artist-and-an-activist-are-not-so-far-apart}{%
\subsection{``An artist and an activist are not so far
apart.''}\label{an-artist-and-an-activist-are-not-so-far-apart}}

\emph{--- Ava DuVernay, award-winning writer, producer and film
director}

\begin{center}\rule{0.5\linewidth}{\linethickness}\end{center}

\emph{{[}In Her Words is available as a newsletter.}
\href{https://www.nytimes3xbfgragh.onion/newsletters/in-her-words}{\emph{Sign
up here to get it delivered to your inbox}}\emph{.{]}}

They might depict scenes from decades past, but movie sets featured in
films by the director
\href{https://www.nytimes3xbfgragh.onion/2020/08/06/movies/ava-duvernay-gish-prize.html}{Ava
DuVernay} are starting to look a lot like the United States today.

For ``Selma,'' her 2014 film about the 1965 marches for voting rights
and the Rev. Dr. Martin Luther King Jr.'s part in them, Ms. DuVernay
directed hundreds of Black and white actors in a restaging of civil
rights protests. ``When They See Us,'' her mini-series on the wrongful
conviction of teenage boys known as the Central Park Five, released last
year, had her grappling with the injustices Black men experience at the
hands of the police. And her Netflix documentary ``13th,'' from 2016,
traces the legacy of American slavery to the present day criminalization
of Black communities.

As hundreds of thousands across the United States march for Black Lives
Matter, Ms. DuVernay's films about Black histories and experiences have
come to feel more essential than ever.

But there aren't enough Black directors telling those stories.

For decades, few Black women have had access to the resources and
platforms to make major motion pictures. In 2018, Hollywood saw a record
high number of top films from Black directors --- and it was only
\href{https://www.cnbc.com/2019/01/04/record-number-of-black-directors-among-2018s-top-films.html}{14
percent}. Only one of them was a woman, and she was Ms. DuVernay.

The calls to break up Hollywood's entrenched disparities are building.
Five years ago, the
\href{https://www.nytimes3xbfgragh.onion/2020/02/06/movies/oscarssowhite-history.html}{hashtag
\#OscarsSoWhite} put a spotlight on the industry's lack of diversity,
and its following has since continued to
\href{https://www.nytimes3xbfgragh.onion/2020/02/06/movies/oscarssowhite-history.html}{hold
Hollywood} to account for its lack of representation. Two years later,
the \#MeToo movement erupted and dozens of women exposed the film titan
Harvey Weinstein's sexual abuses.

Today, industry leaders are listening to people of color protesting
films that romanticize the slavery era. For a brief moment, ``Gone With
the Wind,'' the highest-grossing film of all time when adjusted for
inflation, was
\href{https://www.nytimes3xbfgragh.onion/2020/06/10/business/media/gone-with-the-wind-hbo-max.html}{removed
from HBO Max}. (It was later
\href{https://variety.com/2020/digital/news/gone-with-the-wind-hbo-max-disclaimer-horrors-slavery-1234648726/}{restored}
with additional videos offering historical context.) Filmmakers, like
Ms. DuVernay, are working to ensure the momentum does not subside.

Last month, Ms. DuVernay's media company ARRAY introduced the Law
Enforcement Accountability Project in the wake of George Floyd's killing
while in police custody in Minneapolis, with the goal of commissioning,
funding and amplifying works from Black and female artists that focus on
police violence. One of the goals, she said, is to consider who is
writing the history of this moment.

Ms. DuVernay spoke with In Her Words about the role she sees for artists
in a time of widespread unrest, and whether problematic films --- like
problematic statues --- should be removed to make space for new voices.

The interview has been condensed and edited for clarity.

\textbf{We're in a moment of upheaval --- hundreds of thousands
marching, a pandemic, an upcoming U.S. presidential election. What's the
role of storytelling in this moment?}

The story has been told from one point of view for too long. And when we
say story, I don't just mean film or television. I mean the stories we
embrace as part of the criminalization of Black people. Every time an
officer writes a police report about an incident, they're telling a
story. Look at the case of Breonna Taylor and her police report. They
had nothing on it; it said she had no injuries. That is a story of those
officers saying, ``Nothing to look at here, nothing happened.'' But
that's not the story that happened because if she could speak for
herself, she would say, ``I was shot in the dark on a no-knock warrant
in my bed.''

So when you think of her story and multiply that times hundreds of
thousands of people over the years in communities of color, specifically
Black communities, a single story line has led the day and we need to
change that story line. And to do that, you have to change who the
storytellers are.

\includegraphics{https://static01.graylady3jvrrxbe.onion/images/2020/07/06/us/6ihw-duvernay/6ihw-duvernay-articleLarge-v2.jpg?quality=75\&auto=webp\&disable=upscale}

\textbf{This is a moment of grief and rage for so many. How can those
emotions be translated into art?}

The answer to your question for me personally was the creation of our
Law Enforcement Accountability Project --- LEAP --- which uses art to
hold police accountable.

It links to the idea that an artist and an activist are not so far
apart. Whether you call yourself an activist or not, artists use their
imagination to envision a world that does not exist and make it so.
Activists use their imagination to envision a world that does not exist
and make it so.

\textbf{Today's movement against police violence was, in large part,
prompted by the killing of George Floyd when it was captured}
\textbf{\href{https://www.nytimes3xbfgragh.onion/2020/06/03/arts/george-floyd-video-racism.html}{on
video}. When did you first make the connection between film and social
action?}

I began to make the link between art and social action in high school
when I went to my first Amnesty International concert. It was the first
time that I, a girl from Compton, started to link the things that I was
experiencing to the wider world through music, and what was being said
in those lyrics. And then as I got to college, I started to watch films
like ``The Battle of Algiers'' and the work of Haile Gerima, an
Ethiopian filmmaker, and started to see the link between images, film
and social justice, and what's possible in storytelling.

Image

Ms. DuVernay on the set of ``Selma.''Credit...Atsushi
Nishijima/Paramount Pictures

\textbf{You were nominated for an Academy Award for best picture the
same year that you helped start the movement \#OscarsSoWhite. What are
the challenges of changing an institution or community that you're
deeply a part of?}

It's a system that I work within. It's not a community as it relates to
the problem you're talking about. It's a system like the criminal
justice system. It's a system like the health care system, the education
system. It is the Hollywood system, the entertainment system, the way we
create images. It's a system that is over a hundred years old, and it's
built on a foundation of racism, exclusion and patriarchy.

So, at this point, I think of it less like ``rally the community,'' or
``how do you change people's minds.'' Changing people's minds doesn't
matter if those changed minds are working within a system that's still
diseased. It's just a Band-Aid. So the way I approach it is, yes, you
want to create awareness, you want to educate people, you want to make
people be less ignorant to the nuances of living in skin of color and as
a woman --- but, ultimately, the systems that we all work in are harmful
to a healthy industry. We need to be thinking more broadly about how we
not just reform that system but rebuild the system.

\textbf{You once}
\textbf{\href{https://www.glamour.com/story/ava-duvernay-shares-her-advice-for-women}{called}}
\textbf{Hollywood ``a patriarchy, headed by men and built for men.'' One
year after you said that, the \#MeToo movement was introduced to take
down some of Hollywood's worst abusers. Are you optimistic about the
industry's future?}

I'm hopeful about everything in the world because I believe in the power
of people. I'm a student of history, and there's too much precedent to
be hopeless. I understand there is a way forward, and the way forward is
as a united front. You're seeing some of that right now. Whenever you
get enough people with enough energy behind it, that's power exerted.

\textbf{Let's talk about ``Gone With the Wind.'' Statues are being torn
down. Should films like this be erased from the canon, or are they
important in some way?}

I don't think the film should be erased from the canon, because then you
erase past sins --- those cannot be erased. The damage that was done,
the foundational elements of that film that seeped into cinematic
culture worldwide, I don't think you want to erase those. But I do think
you need to give context to them so that they aren't used as propaganda
for ideas that are poison to our culture. They're really a lesson on our
dark past and what we need to do to get past it. So I feel like
contextualizing these films is important.

Now film is different from statues commemorating murderers and traitors
to the country who wanted to see human beings enslaved. An artist who
decided to promote a certain narrative about the inferiority of certain
people is different from a monument to murderers. I think these
monuments should come down and the films should have context.

\textbf{Many people in the United States are just beginning the fight
for racial and social justice. You've been in this battle a long time.
What's your advice for sustaining the fight long term?}

The battle is ongoing whether you keep it going or not. The question is
how are you going to react to it? That's up to everyone to decide for
themselves.

But the battle is not by choice. I would rather not do any of it. I'd
rather just make my films and go about my day. But if I don't buy into
the fight then I don't get to make my films. That's what privilege is.

\emph{In Her Words is available as a newsletter.}
\href{https://www.nytimes3xbfgragh.onion/newsletters/in-her-words}{\emph{Sign
up here to get it delivered to your inbox}}\emph{. Write to us at}
\href{mailto:inherwords@NYTimes.com}{\emph{inherwords@NYTimes.com}}\emph{.}

\begin{center}\rule{0.5\linewidth}{\linethickness}\end{center}

\begin{center}\rule{0.5\linewidth}{\linethickness}\end{center}

Advertisement

\protect\hyperlink{after-bottom}{Continue reading the main story}

\hypertarget{site-index}{%
\subsection{Site Index}\label{site-index}}

\hypertarget{site-information-navigation}{%
\subsection{Site Information
Navigation}\label{site-information-navigation}}

\begin{itemize}
\tightlist
\item
  \href{https://help.nytimes3xbfgragh.onion/hc/en-us/articles/115014792127-Copyright-notice}{©~2020~The
  New York Times Company}
\end{itemize}

\begin{itemize}
\tightlist
\item
  \href{https://www.nytco.com/}{NYTCo}
\item
  \href{https://help.nytimes3xbfgragh.onion/hc/en-us/articles/115015385887-Contact-Us}{Contact
  Us}
\item
  \href{https://www.nytco.com/careers/}{Work with us}
\item
  \href{https://nytmediakit.com/}{Advertise}
\item
  \href{http://www.tbrandstudio.com/}{T Brand Studio}
\item
  \href{https://www.nytimes3xbfgragh.onion/privacy/cookie-policy\#how-do-i-manage-trackers}{Your
  Ad Choices}
\item
  \href{https://www.nytimes3xbfgragh.onion/privacy}{Privacy}
\item
  \href{https://help.nytimes3xbfgragh.onion/hc/en-us/articles/115014893428-Terms-of-service}{Terms
  of Service}
\item
  \href{https://help.nytimes3xbfgragh.onion/hc/en-us/articles/115014893968-Terms-of-sale}{Terms
  of Sale}
\item
  \href{https://spiderbites.nytimes3xbfgragh.onion}{Site Map}
\item
  \href{https://help.nytimes3xbfgragh.onion/hc/en-us}{Help}
\item
  \href{https://www.nytimes3xbfgragh.onion/subscription?campaignId=37WXW}{Subscriptions}
\end{itemize}
