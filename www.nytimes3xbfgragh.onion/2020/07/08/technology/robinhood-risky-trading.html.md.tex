Sections

SEARCH

\protect\hyperlink{site-content}{Skip to
content}\protect\hyperlink{site-index}{Skip to site index}

\href{https://www.nytimes3xbfgragh.onion/section/technology}{Technology}

\href{https://myaccount.nytimes3xbfgragh.onion/auth/login?response_type=cookie\&client_id=vi}{}

\href{https://www.nytimes3xbfgragh.onion/section/todayspaper}{Today's
Paper}

\href{/section/technology}{Technology}\textbar{}Robinhood Has Lured
Young Traders, Sometimes With Devastating Results

\url{https://nyti.ms/2DorkW7}

\begin{itemize}
\item
\item
\item
\item
\item
\item
\end{itemize}

Advertisement

\protect\hyperlink{after-top}{Continue reading the main story}

Supported by

\protect\hyperlink{after-sponsor}{Continue reading the main story}

\hypertarget{robinhood-has-lured-young-traders-sometimes-with-devastating-results}{%
\section{Robinhood Has Lured Young Traders, Sometimes With Devastating
Results}\label{robinhood-has-lured-young-traders-sometimes-with-devastating-results}}

Its users buy and sell the riskiest financial products and do so more
frequently than customers at other retail brokerage firms, but their
inexperience can lead to staggering losses.

\includegraphics{https://static01.graylady3jvrrxbe.onion/images/2020/07/07/business/00robinhood-sub/merlin_174050700_059488cb-04ea-4120-84e4-d221da38a1fe-articleLarge.jpg?quality=75\&auto=webp\&disable=upscale}

\href{https://www.nytimes3xbfgragh.onion/by/nathaniel-popper}{\includegraphics{https://static01.graylady3jvrrxbe.onion/images/2018/10/22/multimedia/author-nathaniel-popper/author-nathaniel-popper-thumbLarge.png}}

By
\href{https://www.nytimes3xbfgragh.onion/by/nathaniel-popper}{Nathaniel
Popper}

\begin{itemize}
\item
  July 8, 2020
\item
  \begin{itemize}
  \item
  \item
  \item
  \item
  \item
  \item
  \end{itemize}
\end{itemize}

Richard Dobatse, a Navy medic in San Diego, dabbled infrequently in
stock trading. But his behavior changed in 2017 when he signed up for
Robinhood, a trading app that made buying and selling stocks simple and
seemingly free.

Mr. Dobatse, now 32, said he had been charmed by Robinhood's one-click
trading, easy access to complex investment products, and features like
falling confetti and emoji-filled phone notifications that made it feel
like a game. After funding his account with \$15,000 in credit card
advances, he began spending more time on the app.

As he repeatedly lost money, Mr. Dobatse took out two \$30,000 home
equity loans so he could buy and sell more speculative stocks and
options, hoping to pay off his debts. His account value shot above \$1
million this year --- but almost all of that recently disappeared. This
week, his balance was \$6,956.

``When he is doing his trading, he won't want to eat,'' said his wife,
Tashika Dobatse, with whom he has three children. ``He would have
nightmares.''

Millions of young Americans have begun investing in recent years through
Robinhood, which
was\href{https://www.nytimes3xbfgragh.onion/2017/02/18/business/robinhood-stock-trading-app.html}{founded
in 2013} with a sales pitch of no trading fees or account minimums. The
ease of trading has turned it into a cultural phenomenon and a Silicon
Valley darling, with the start-up climbing to an \$8.3 billion
valuation. It has been one of the tech industry's biggest growth stories
in the recent market turmoil.

But at least part of Robinhood's success appears to have been built on a
Silicon Valley playbook of behavioral nudges and push notifications,
which has drawn inexperienced investors into the riskiest trading,
according to an analysis of industry data and legal filings, as well as
interviews with nine current and former Robinhood employees and more
than a dozen customers. And the more that customers engaged in such
behavior, the better it was for the company, the data shows.

More than at any other retail brokerage firm, Robinhood's users trade
the riskiest products and at the fastest pace, according to an analysis
of new filings from nine brokerage firms by the research firm
Alphacution for The New York Times.

In the first three months of 2020, Robinhood users traded nine times as
many shares as E-Trade customers, and 40 times as many shares as Charles
Schwab customers, per dollar in the average customer account in the most
recent quarter. They also bought and sold 88 times as many risky options
contracts as Schwab customers, relative to the average account size,
according to the analysis.

The more often small investors trade stocks, the worse their returns are
likely to be,
\href{https://www.sciencedirect.com/science/article/abs/pii/S1386418113000190}{studies}
\href{http://citeseerx.ist.psu.edu/viewdoc/download?doi=10.1.1.408.1468\&rep=rep1\&type=pdf}{have}
\href{https://www.sciencedirect.com/science/article/abs/pii/S1386418113000190}{shown}.
The returns are even worse when they get involved with options,
\href{https://www.nytimes3xbfgragh.onion/2013/05/25/business/growth-in-options-trading-helps-brokers-but-not-small-investors.html}{research}
\href{https://papers.ssrn.com/sol3/papers.cfm?abstract_id=965810}{ha}s
\href{https://www.sciencedirect.com/science/article/abs/pii/S0378426614003501}{found}.

This kind of trading, where a few minutes can mean the difference
between winning and losing, was particularly hazardous on Robinhood
because the firm has experienced an unusual number of technology issues,
public records show. Some Robinhood employees, who declined to be
identified for fear of retaliation, said the company failed to provide
adequate guardrails and technology to support its customers.

Those dangers came into focus last month when Alex Kearns, 20, a college
student in Nebraska,
\href{https://www.forbes.com/sites/sergeiklebnikov/2020/06/17/20-year-old-robinhood-customer-dies-by-suicide-after-seeing-a-730000-negative-balance/\#bbc1cfb16384}{killed
himself} after he logged into the app and saw that his balance had
dropped to negative \$730,000. The figure was high partly because of
some incomplete trades.

``There was no intention to be assigned this much and take this much
risk,'' Mr. Kearns wrote in his suicide note, which a family member
\href{https://twitter.com/BillBrewsterSCG/status/1273292130769932288?s=20}{posted
on Twitter}.

Like Mr. Kearns, Robinhood's average customer is young and lacks
investing know-how. The average age is 31, the company said, and half of
its customers had never invested before.

Some have visited Robinhood's headquarters in Menlo Park, Calif., in
recent years to confront the staff about their losses, said four
employees who witnessed the incidents. This year, they said, the
start-up installed bulletproof glass at the front entrance.

``They encourage people to go from training wheels to driving
motorcycles,'' Scott Smith, who tracks brokerage firms at the financial
consulting firm Cerulli, said of Robinhood. ``Over the long term, it's
like trying to beat the casino.''

At the core of Robinhood's business is an incentive to encourage more
trading. It does not charge fees for trading, but it is still paid more
if its customers trade more.

That's because it makes money through a complex practice known as
``payment for order flow.'' Each time a Robinhood customer trades, Wall
Street firms actually buy or sell the shares and determine what price
the customer gets. These firms pay Robinhood for the right to do this,
because they then engage in a form of arbitrage by trying to buy or sell
the stock for a profit over what they give the Robinhood customer.

This practice is not new, and retail brokers such as E-Trade and Schwab
also do it. But Robinhood makes significantly more than they do for each
stock share and options contract sent to the professional trading firms,
the filings show.

For each share of stock traded, Robinhood made four to 15 times more
than Schwab in the most recent quarter, according to the filings. In
total, Robinhood got \$18,955 from the trading firms for every dollar in
the average customer account, while Schwab made \$195, the Alphacution
analysis shows. Industry experts said this was most likely because the
trading firms believed they could score the easiest profits from
Robinhood customers.

Vlad Tenev, a founder and co-chief executive of Robinhood, said in an
interview that even with some of its customers losing money,
\href{https://www.nytimes3xbfgragh.onion/2020/04/06/business/millennials-economic-crisis-virus.html}{young
Americans risked greater losses} by not investing in stocks at all. Not
participating in the markets ``ultimately contributed to the sort of the
massive inequalities that we're seeing in society,'' he said.

Mr. Tenev said only 12 percent of the traders active on Robinhood each
month used options, which allow people to bet on where the price of a
specific stock will be on a specific day and multiply that by 100. He
said the company had added educational content on how to invest safely.

He declined to comment on why Robinhood makes more than its competitors
from the Wall Street firms. The company also declined to provide data on
its customers' performance.

Robinhood does not force people to trade, of course. But its success at
getting them do so has been highlighted internally. In June, the actor
Ashton Kutcher, who has invested in Robinhood, attended one of the
company's weekly staff meetings on Zoom and celebrated its success by
comparing it to gambling websites, said three people who were on the
call.

Mr. Kutcher said in a statement that his comment ``was not intended to
be a comparison of business models nor the experience Robinhood provides
its customers'' and that it referred ``to the current growth metrics.''
He added that he was ``absolutely not insinuating that Robinhood was a
gambling platform.''

\hypertarget{democratizing-finance}{%
\subsection{Democratizing Finance}\label{democratizing-finance}}

\includegraphics{https://static01.graylady3jvrrxbe.onion/images/2020/07/07/business/00robinhood2/merlin_148100856_c8ccf700-86f2-4d82-b37c-0bd481aae3ae-articleLarge.jpg?quality=75\&auto=webp\&disable=upscale}

Robinhood was founded by Mr. Tenev and Baiju Bhatt, two children of
immigrants who met at Stanford University in 2005. After teaming up on
several ventures, including a high-speed trading firm, they were
inspired by the
\href{https://www.nytimes3xbfgragh.onion/topic/organization/occupy-movement-occupy-wall-street}{Occupy
Wall Street movement} to create a company that would make finance more
accessible, they said. They named the start-up Robinhood after
\href{https://en.wikipedia.org/wiki/Robin_Hood}{the English outlaw} who
stole from the rich and gave to the poor.

Robinhood eliminated trading fees while most brokerage firms charged
\$10 or more for a trade. **** It also added features to make investing
more like a game. New members were given a free share of stock, but only
after they scratched off images that looked like a lottery ticket.

The app is simple to use. The home screen has a list of trendy stocks.
If a customer touches one of them, a green button pops up with the word
``trade,'' skipping many of the steps that other firms require.

Robinhood initially offered only stock trading. Over time, it added
options trading and margin loans, which make it possible to turbocharge
investment gains --- and to supersize losses.

The app advertises options with the tagline ``quick, straightforward \&
free.'' Customers who want to trade options answer just a few
multiple-choice questions. Beginners are legally barred from trading
options, but those who click that they have no investing experience are
coached by the app on how to change the answer to ``not much''
experience. Then people can immediately begin trading.

Before Robinhood added options trading in 2017, Mr. Bhatt scoffed at the
idea that the company was letting investors take uninformed risks.

``The best thing we can say to those people is `Just do it,''' he
\href{https://www.businessinsider.com/robinhood-cofounder-baiju-bhatt-interview-2017-8}{told
Business Insider} at the time.

In May, Robinhood said it had 13 million accounts, up from 10 million at
the end of 2019. Schwab said it had 12.7 million brokerage accounts in
its latest filings; E-Trade reported 5.5 million.

That growth has kept the money flowing in from venture capitalists.
Sequoia Capital and New Enterprise Associates are among those that have
poured \$1.3 billion into Robinhood. In May, the company received a
fresh
\href{https://blog.robinhood.com/news/2020/5/4/robinhood-raises-280-million-in-series-f-funding-led-by-sequoia}{\$280
million}.

``Robinhood has made the financial markets accessible to the masses and,
in turn, revolutionized the decades-old brokerage industry,'' Andrew
Reed, a partner at Sequoia,
\href{https://blog.robinhood.com/news/2020/5/4/robinhood-raises-280-million-in-series-f-funding-led-by-sequoia}{said}
after last month's fund-raising.

\hypertarget{two-days-in-march}{%
\subsection{Two Days in March}\label{two-days-in-march}}

Image

Robinhood shows users that its options trading is free of commissions.~

Mr. Tenev
\href{https://www.wealthmanagement.com/technology/robinhood-ceo-schwab-fidelity-not-really-technology-companies}{has
said} Robinhood has invested in the best technology in the industry. But
the risks of trading through the app have been compounded by its tech
glitches.

In 2018, Robinhood released software that accidentally
\href{https://www.elitetrader.com/et/threads/robinhood-options-errors.327998/}{reversed
the direction} of options trades, giving customers the opposite outcome
from what they expected. Last year, it mistakenly allowed people to
borrow infinite money to multiply their bets, leading to some enormous
gains and losses.

Robinhood's website has also gone down more often than those of its
rivals --- 47 times since March for Robinhood and 10 times for Schwab
--- according to a Times analysis of data from Downdetector.com, which
tracks website reliability. In March, the
\href{https://www.nytimes3xbfgragh.onion/2020/03/03/technology/trading-app-robinhood-outage.html}{site
was down} for almost two days, just as
\href{https://www.nytimes3xbfgragh.onion/2020/03/11/business/bear-market-stocks-dow.html}{stock
prices were gyrating} because of the coronavirus pandemic. Robinhood's
customers were unable to make trades to blunt the damage to their
accounts.

Four Robinhood employees, who declined to be identified, said the outage
was rooted in issues with the company's phone app and servers. They said
the start-up had underinvested in technology and moved too quickly
rather than carefully.

Mr. Tenev said he could not talk about the outage beyond a company
\href{https://blog.robinhood.com/news/2020/3/3/an-update-from-robinhoods-founders}{blog
post} that said it was ``not acceptable.'' Robinhood had recently made
new technology investments, he said.

Plaintiffs who have sued over the outage said Robinhood had done little
to respond to their losses. Unlike other brokers, the company has no
phone number for customers to call.

Mr. Dobatse suffered his biggest losses in the March outage ---
\$860,000, his records show. Robinhood did not respond to his emails, he
said. A Robinhood spokesman said the company did respond.

Mr. Dobatse said he planned to take his case to financial regulators for
arbitration.

``They make it so easy for people that don't know anything about
stocks,'' he said. ``Then you go there and you start to lose money.''

Advertisement

\protect\hyperlink{after-bottom}{Continue reading the main story}

\hypertarget{site-index}{%
\subsection{Site Index}\label{site-index}}

\hypertarget{site-information-navigation}{%
\subsection{Site Information
Navigation}\label{site-information-navigation}}

\begin{itemize}
\tightlist
\item
  \href{https://help.nytimes3xbfgragh.onion/hc/en-us/articles/115014792127-Copyright-notice}{©~2020~The
  New York Times Company}
\end{itemize}

\begin{itemize}
\tightlist
\item
  \href{https://www.nytco.com/}{NYTCo}
\item
  \href{https://help.nytimes3xbfgragh.onion/hc/en-us/articles/115015385887-Contact-Us}{Contact
  Us}
\item
  \href{https://www.nytco.com/careers/}{Work with us}
\item
  \href{https://nytmediakit.com/}{Advertise}
\item
  \href{http://www.tbrandstudio.com/}{T Brand Studio}
\item
  \href{https://www.nytimes3xbfgragh.onion/privacy/cookie-policy\#how-do-i-manage-trackers}{Your
  Ad Choices}
\item
  \href{https://www.nytimes3xbfgragh.onion/privacy}{Privacy}
\item
  \href{https://help.nytimes3xbfgragh.onion/hc/en-us/articles/115014893428-Terms-of-service}{Terms
  of Service}
\item
  \href{https://help.nytimes3xbfgragh.onion/hc/en-us/articles/115014893968-Terms-of-sale}{Terms
  of Sale}
\item
  \href{https://spiderbites.nytimes3xbfgragh.onion}{Site Map}
\item
  \href{https://help.nytimes3xbfgragh.onion/hc/en-us}{Help}
\item
  \href{https://www.nytimes3xbfgragh.onion/subscription?campaignId=37WXW}{Subscriptions}
\end{itemize}
