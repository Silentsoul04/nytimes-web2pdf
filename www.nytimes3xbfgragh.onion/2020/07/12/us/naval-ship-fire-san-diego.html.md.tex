Sections

SEARCH

\protect\hyperlink{site-content}{Skip to
content}\protect\hyperlink{site-index}{Skip to site index}

\href{https://www.nytimes3xbfgragh.onion/section/us}{U.S.}

\href{https://myaccount.nytimes3xbfgragh.onion/auth/login?response_type=cookie\&client_id=vi}{}

\href{https://www.nytimes3xbfgragh.onion/section/todayspaper}{Today's
Paper}

\href{/section/us}{U.S.}\textbar{}Navy Ship Continues to Burn Off San
Diego After Fire Injures 57

\url{https://nyti.ms/2DBqFRh}

\begin{itemize}
\item
\item
\item
\item
\item
\item
\end{itemize}

Advertisement

\protect\hyperlink{after-top}{Continue reading the main story}

Supported by

\protect\hyperlink{after-sponsor}{Continue reading the main story}

\hypertarget{navy-ship-continues-to-burn-off-san-diego-after-fire-injures-57}{%
\section{Navy Ship Continues to Burn Off San Diego After Fire Injures
57}\label{navy-ship-continues-to-burn-off-san-diego-after-fire-injures-57}}

A Navy spokesman said that 34 sailors and 23 civilians were treated for
minor injuries after a fire and explosion aboard the U.S.S. Bonhomme
Richard.

\includegraphics{https://static01.graylady3jvrrxbe.onion/images/2020/07/12/multimedia/12xp-boat-fire-2d-sub/12xp-boat-fire-2d-sub-videoSixteenByNine3000.jpg}

\href{https://www.nytimes3xbfgragh.onion/by/neil-vigdor}{\includegraphics{https://static01.graylady3jvrrxbe.onion/images/2019/07/25/reader-center/author-neil-vigdor/author-neil-vigdor-thumbLarge.png}}\href{https://www.nytimes3xbfgragh.onion/by/john-ismay}{\includegraphics{https://static01.graylady3jvrrxbe.onion/images/2018/07/12/multimedia/author-john-ismay/author-john-ismay-thumbLarge.png}}

By \href{https://www.nytimes3xbfgragh.onion/by/neil-vigdor}{Neil Vigdor}
and \href{https://www.nytimes3xbfgragh.onion/by/john-ismay}{John Ismay}

\begin{itemize}
\item
  Published July 12, 2020Updated Aug. 26, 2020
\item
  \begin{itemize}
  \item
  \item
  \item
  \item
  \item
  \item
  \end{itemize}
\end{itemize}

\emph{{[}Read more on the}
\href{https://www.nytimes3xbfgragh.onion/2020/07/13/us/naval-ship-fire-san-diego.html}{\emph{Navy
warship that is still ablaze}} \emph{and now tilting to one side.{]}}

At least 57 people were injured when a fire and an explosion broke out
Sunday on a
\href{https://www.nytimes3xbfgragh.onion/2020/08/26/us/bonhomme-richard-arson-fire.html}{U.S.
Navy warship} that was docked in San Diego, causing heavy damage and
burning into the next day, officials said.

The ship, the U.S.S. Bonhomme Richard, was docked at the U.S. naval base
in San Diego when a fire was reported at 8:30 a.m. in a lower cargo hold
that is used for vehicle storage, the Navy said.

It was not immediately clear what had caused the fire or the explosion.

Rear Adm. Philip E. Sobeck
\href{https://www.facebookcorewwwi.onion/SurfaceWarriors/videos/2713973245550040/?__xts__\%5B0\%5D=68.ARBncjWQcC-BIBJ7BLcf6xp8GVwwqvxHmXAeYTYmawWEs0ib9dLS7PZGyOPW96yBSRYJjYQzCB7QjxqEWK15yeoJPNogUx_T572RzKgVWFegU4V5WKjsPQ7Dp-JfyO05yUZfXJnoGGp-vIhu_SqIiUVLhkQncsM1IR-APVF_lR4CQvc4MX6YKpnet-EBocLNqwhctd6mCFZgnkx-n9DIvsrBte_0YvcEWC5dpN-OuBXI7bpA23vIgDYwijEW1dpkoQQD_3TQRcwPk7Rsoe4wsEplReyr4CIv4-O5SskoWnxyGBtCr6TG1kSavnDwmJhlLz2v77MZcaVqnoLp4t2bvrW5B1gKtQ\&__tn__=-R}{said
at a news conference on Monday} that the ship was listing, though he did
not say by how much, and that 400 sailors had been working over the
previous 24 hours to save the ship.

He said he was ``absolutely hopeful'' about those prospects, adding that
``dewatering'' efforts were continuing.

Admiral Sobeck said temperatures in some places on the ship reached
1,000 degrees and that the fire may have been fueled by drywall,
heavy-duty cardboard boxes and other materials that were in storage.

He said the superstructure of the upper deck was continuing to burn and
had sustained damage. Helicopters had joined the firefighting efforts
and dropped more than 400 buckets of water.

About 160 sailors were aboard the ship at the time, according to the
Navy, which said that 34 sailors and 23 civilians were injured.

The injuries were not life-threatening, and included heat exhaustion and
smoke inhalation. All crew members had been accounted for, the Navy
said, and five sailors remained hospitalized in stable condition.

Fire Chief Colin Stowell of San Diego told CNN that the fire could burn
for several days,
\href{https://twitter.com/cnnbrk/status/1282424297701728262}{``down to
the waterline.''}

A spokeswoman for San Diego Fire-Rescue said later on Sunday that Chief
Stowell was not available for further interviews and referred questions
to the Navy, which could not say how extensive the damage was or how
much it would cost to repair the ship.

Admiral Sobeck vowed that the ship would return to service.

``We're absolutely going to make sure it sails again,'' he said on
Sunday.

The Bonhomme Richard is
\href{https://www.navy.mil/navydata/fact_display.asp?cid=4200\&tid=400\&ct=4}{one
of eight Wasp-class amphibious assault ships} and can carry more than
1,000 sailors, according to the Navy. Its cost has been
\href{https://fas.org/man/dod-101/sys/ship/lhd-1.htm}{estimated at \$761
million} by the Federation of American Scientists.

Two other warships that had been docked nearby
\href{https://twitter.com/SurfaceWarriors/status/1282427073835851777?s=20}{were
moved to other piers} as a precaution, the Navy said.

The Bonhomme Richard is capable of carrying helicopters and fixed-wing
aircraft. It was docked at the base in San Diego for scheduled
maintenance between deployments, said Krishna Jackson, a Navy
spokeswoman. The sailors assigned to it were staying in Navy or private
housing on shore and were not on board when the fire started, Ms.
Jackson said. A ``duty section'' of sailors trained to fight fires was
on board when the fire started, she said, and they were the first to
respond.

Admiral Sobeck said that there was no ordnance on the ship at the time
of the fire. The ship has light arms but would not normally have large
explosive munitions, such as airdropped bombs, on board while docked,
according to the Navy.

The fire was the latest in a series of disasters for the Navy in the
Pacific. In 2017, two Navy destroyers
\href{https://www.nytimes3xbfgragh.onion/2017/11/01/us/politics/navy-collisions-avoidable.html}{collided
with two commercial vessels}, near Japan and Singapore, killing a total
of 17 American sailors. And this year, the acting Navy secretary
\href{https://www.nytimes3xbfgragh.onion/2020/04/07/us/politics/coronavirus-navy-captain-firing.html}{resigned
over his bungled response to an outbreak} of the coronavirus aboard an
aircraft carrier. Hundreds of sailors were sickened, and the captain's
\href{https://www.nytimes3xbfgragh.onion/2020/04/02/us/politics/coronavirus-aircraft-carrier-roosevelt.html}{plea
for help} led to his removal and
\href{https://www.nytimes3xbfgragh.onion/2020/04/29/us/politics/navy-coronavirus-ship.html}{a
formal investigation}.

Experts said that although a fire posed special challenges on a naval
vessel, sailors were thoroughly equipped to handle fires on ships.

``Everyone gets trained to be a firefighter, flooding stopper --- all
the damage control --- and that's because when you're out at sea,
there's nobody coming to you,'' said
\href{https://usnwc.edu/Faculty-and-Departments/Directory/Eric-A-Dukat}{Eric
A. Dukat}, a retired U.S. Navy commander who is now an associate
professor in the College of Maritime Operational Warfare at the U.S.
Naval War College.

Professor Dukat, who worked as a damage control assistant aboard the
U.S.S. Wasp, said fires on ships ``are not like a house fire'' and
present a unique hazard because of the rising heat inside the vessel and
the intense steam that's produced when water is used.

``Imagine a fire inside of a ship, just imagine the inside of your
oven,'' he said.

In 1967, a fire on the aircraft carrier U.S.S. Forrestal
\href{https://www.history.com/this-day-in-history/rocket-causes-deadly-fire-on-aircraft-carrier}{killed
more than 130 sailors} after a rocket accidentally fired on the flight
deck and ignited several explosions. The episode has been used as a
lesson on how to tackle safety procedures aboard Navy vessels, Professor
Dukat said.

John Liddle, a lieutenant commander who retired from the Navy last year
after 20 years of service, said in an interview Sunday that fighting a
fire while a ship is in port can be harder because not all crew members
are available and safety standards are different when the vessel is not
at sea.

``The things you would normally do to keep a fire contained, you can't
do them,'' he said.

Mr. Liddle was second-in-command of the U.S.S. Hue City when a fire was
sparked onboard hundreds of miles off the coast of Bermuda in April
2014.

The fire on his ship was sparked by rags that sailors had stored in
``basically a chimney,'' he said. The Navy
\href{https://www.navy.mil/submit/display.asp?story_id=81920}{relieved
him of his duties after an investigation into the fire}.

``Even in a really horrible, awful, wartime fire, you're still going to
be able to keep the ship afloat,'' Mr. Liddle said. ``Where the problem
really comes, where a ship is lost for good, is normally actually
because of the water.''

``You're putting so much water into it in one place or another that all
of a sudden it's not buoyant in the same way that it was designed to
be,'' he said.

The Bonhomme Richard is
\href{https://www.navy.mil/navydata/fact_display.asp?cid=4200\&tid=400\&ct=4}{outfitted
to carry} landing craft to transport equipment and troops as well as
landing boats. It is
\href{https://www.nvr.navy.mil/SHIPDETAILS/SHIPSDETAIL_LHD_6_2395.HTML}{847
feet long} and has a crew of 102 officers and just over 1,000 sailors.

The ship was commissioned on Aug. 15, 1998, and is the third Navy
warship to bear the name Bonhomme Richard.

The first was a 900-ton merchant vessel fitted with 40 guns and placed
under the command of Capt. John Paul Jones during the Revolutionary War.
Jones named the ship after the French translation of
\href{https://www.public.navy.mil/surfor/lhd6/Pages/namesake.aspx}{the
pen name Benjamin Franklin} used as the author of ``Poor Richard's
Almanac.''

The original Bonhomme Richard gained fame for defeating a British
warship, the H.M.S. Serapis, in British home waters in 1779. The
present-day U.S.S. Bonhomme Richard has participated in numerous
operations, including Operation Enduring Freedom, Operation Anaconda and
Operation Iraqi Freedom.

Aimee Ortiz, Brian Pietsch and Allyson Waller contributed reporting.

Advertisement

\protect\hyperlink{after-bottom}{Continue reading the main story}

\hypertarget{site-index}{%
\subsection{Site Index}\label{site-index}}

\hypertarget{site-information-navigation}{%
\subsection{Site Information
Navigation}\label{site-information-navigation}}

\begin{itemize}
\tightlist
\item
  \href{https://help.nytimes3xbfgragh.onion/hc/en-us/articles/115014792127-Copyright-notice}{©~2020~The
  New York Times Company}
\end{itemize}

\begin{itemize}
\tightlist
\item
  \href{https://www.nytco.com/}{NYTCo}
\item
  \href{https://help.nytimes3xbfgragh.onion/hc/en-us/articles/115015385887-Contact-Us}{Contact
  Us}
\item
  \href{https://www.nytco.com/careers/}{Work with us}
\item
  \href{https://nytmediakit.com/}{Advertise}
\item
  \href{http://www.tbrandstudio.com/}{T Brand Studio}
\item
  \href{https://www.nytimes3xbfgragh.onion/privacy/cookie-policy\#how-do-i-manage-trackers}{Your
  Ad Choices}
\item
  \href{https://www.nytimes3xbfgragh.onion/privacy}{Privacy}
\item
  \href{https://help.nytimes3xbfgragh.onion/hc/en-us/articles/115014893428-Terms-of-service}{Terms
  of Service}
\item
  \href{https://help.nytimes3xbfgragh.onion/hc/en-us/articles/115014893968-Terms-of-sale}{Terms
  of Sale}
\item
  \href{https://spiderbites.nytimes3xbfgragh.onion}{Site Map}
\item
  \href{https://help.nytimes3xbfgragh.onion/hc/en-us}{Help}
\item
  \href{https://www.nytimes3xbfgragh.onion/subscription?campaignId=37WXW}{Subscriptions}
\end{itemize}
