Sections

SEARCH

\protect\hyperlink{site-content}{Skip to
content}\protect\hyperlink{site-index}{Skip to site index}

\href{https://www.nytimes3xbfgragh.onion/section/sports/football}{Pro
Football}

\href{https://myaccount.nytimes3xbfgragh.onion/auth/login?response_type=cookie\&client_id=vi}{}

\href{https://www.nytimes3xbfgragh.onion/section/todayspaper}{Today's
Paper}

\href{/section/sports/football}{Pro Football}\textbar{}Redskins to Drop
Name, Yielding to Pressure From Sponsors and Activists

\url{https://nyti.ms/2CwoZI2}

\begin{itemize}
\item
\item
\item
\item
\item
\item
\end{itemize}

Advertisement

\protect\hyperlink{after-top}{Continue reading the main story}

Supported by

\protect\hyperlink{after-sponsor}{Continue reading the main story}

\hypertarget{redskins-to-drop-name-yielding-to-pressure-from-sponsors-and-activists}{%
\section{Redskins to Drop Name, Yielding to Pressure From Sponsors and
Activists}\label{redskins-to-drop-name-yielding-to-pressure-from-sponsors-and-activists}}

The N.F.L. team in Washington announced the move on Monday and will
continue its search for a new name and logo.

\includegraphics{https://static01.graylady3jvrrxbe.onion/images/2020/07/13/sports/13nfl-redskins-1/merlin_174370611_df8e48a8-f8e3-4f32-86ff-6b5b6466979c-articleLarge.jpg?quality=75\&auto=webp\&disable=upscale}

\href{https://www.nytimes3xbfgragh.onion/by/ken-belson}{\includegraphics{https://static01.graylady3jvrrxbe.onion/images/2018/02/16/multimedia/author-ken-belson/author-ken-belson-thumbLarge.jpg}}\href{https://www.nytimes3xbfgragh.onion/by/kevin-draper}{\includegraphics{https://static01.graylady3jvrrxbe.onion/images/2018/07/18/multimedia/author-kevin-draper/author-kevin-draper-thumbLarge.png}}

By \href{https://www.nytimes3xbfgragh.onion/by/ken-belson}{Ken Belson}
and \href{https://www.nytimes3xbfgragh.onion/by/kevin-draper}{Kevin
Draper}

\begin{itemize}
\item
  Published July 13, 2020Updated July 16, 2020
\item
  \begin{itemize}
  \item
  \item
  \item
  \item
  \item
  \item
  \end{itemize}
\end{itemize}

First, the city of
\href{https://www.nytimes3xbfgragh.onion/2020/07/16/style/washington-redskins-name-change-merchandise.html}{Washington}
took down a tribute to George Preston Marshall, the founder of its
N.F.L. team, that was in front of the team's old home, Robert F. Kennedy
Memorial Stadium. Then the team
\href{https://www.nytimes3xbfgragh.onion/2020/06/19/sports/baseball/statue-removed-rfk-twins.html}{removed}
references to Marshall, who named his team the ``Redskins,''
\href{https://www.nytimes3xbfgragh.onion/2020/06/24/sports/football/redskins-ring-of-fame-marshall.html?action=click\&module=RelatedLinks\&pgtype=Article}{from
inside its stadium and at its training facility}.

On Monday, under pressure from corporate sponsors,
\href{https://twitter.com/Redskins/status/1282661063943651328?s=20}{the
team announced} its most dramatic step, that it would drop its logo and
``\href{https://www.nytimes3xbfgragh.onion/2020/07/16/sports/football/washington-sexual-assault-harassment-dan-snyder.html}{Redskins}''
from its name, an all but forced turnaround by
\href{https://www.nytimes3xbfgragh.onion/2020/08/10/sports/football/washington-nfl-snyder-lawsuit.html}{team
owner Daniel Snyder}, who for decades said he would never change the
name that had long been considered a racial slur.

``Today, we are announcing we will be retiring the
\href{https://www.nytimes3xbfgragh.onion/2020/07/16/style/washington-redskins-name-change-merchandise.html}{Redskins}
name and logo upon completion of this review,'' the team said in a
statement.

The team has long been in the spotlight, in part because of its
checkered racial past. Marshall was the last owner in the N.F.L. to sign
a Black player, and only under pressure from the federal government. He
named the team the Redskins, which he considered a nod to bravery.

But only in the past few weeks has there been movement to address his
legacy by removing the monument and his name from team facilities.
Monday's decision came just 10 days after the franchise said it would
review the 87-year-old team name under significant pressure from major
corporate partners including FedEx, which had threatened to end its
naming rights sponsorship of the team's stadium.

Snyder's shift from total resistance
to\href{https://www.nytimes3xbfgragh.onion/2020/06/24/sports/football/redskins-ring-of-fame-marshall.html}{grudging
acceptance} in just a few weeks has been remarkably swift in a league
that often moves forward deliberately, if at all. But after the killing
of George Floyd by police in Minneapolis in late May, much of the
country has moved rapidly to confront historical representations of
racist symbols.

The team did not announce a new name on Monday as it continues to
evaluate possibilities. Snyder said
\href{https://www.nfl.com/news/washington-redskins-to-undergo-thorough-review-of-team-s-name}{the
new name, when chosen}, would ``take into account not only the proud
tradition and history of the franchise but also input from our alumni,
the organization, sponsors, the National Football League and the local
community it is proud to represent on and off the field.''

His about-face comes after hundreds of universities and schools have in
recent years abandoned team names and mascots with Native American
imagery.

``This day of the retirement of the r-word slur and stereotypical logo
belongs to all those Native families,'' said Suzan Shown Harjo, a Native
American activist. She said that the change was a victory for all those
who ``bore the brunt of and carry the scars from the epithets, beatings,
death threats and other emotional and physical brutalities resulting
from all the `Native' sports names and images that cause harm and injury
to actual Native people.''

That Washington, among the N.F.L.'s most valuable franchises, was
compelled to change its name likely adds pressure on the remaining
professional teams with Native American mascots and logos to re-evaluate
their names and monikers. The
\href{https://www.nytimes3xbfgragh.onion/2020/01/29/sports/football/chiefs-tomahawk-chop.html}{Kansas
City Chiefs} of the N.F.L., the Chicago Blackhawks of the N.H.L. and the
Atlanta Braves and the Cleveland Indians of Major League Baseball have
long resisted changing their names and logos, though the Indians
\href{https://www.nytimes3xbfgragh.onion/2018/01/29/sports/baseball/cleveland-indians-chief-wahoo-logo.html}{dropped
the mascot Chief Wahoo} last year and recently said they would
\href{https://www.nytimes3xbfgragh.onion/2020/07/03/sports/baseball/cleveland-indians-name-change.html}{review
the team name}.

The Washington team's most immediate task is changing its official
branding, but it is unclear how the team will address fans who continue
to wear headdresses, war paint, and other stereotypical imagery to
games, and if it will replace its fight song, ``Hail to the Redskins,''
which contains references to ``braves on the warpath'' and is played
after touchdowns at home games. The team may get to delay making those
decisions if
\href{https://www.nytimes3xbfgragh.onion/2020/07/02/sports/football/nfl-salary-cap-no-fans.html}{fans
are not allowed to attend games} this season because of the coronavirus.

At the end of June, some of the team's biggest sponsors, including
FedEx, Nike and Pepsi, received letters from investors who called on the
companies to cut their ties with the team. On July 2,
\href{https://www.nytimes3xbfgragh.onion/2020/07/10/sports/football/dan-snyder-washington-redskins-name-fedex.html}{FedEx,
which pays about \$8 million a year to have its name on the team's
stadium in Landover, Md., told the Redskins in a letter} that if the
team did not change its name it would ask that its name be taken off the
stadium at the end of the coming season.

The next day, July 3, the team said a
\href{https://www.nytimes3xbfgragh.onion/2020/07/03/sports/football/washington-redskins-nickname-nfl.html?smid=tw-share}{change
was likely to be forthcoming}, when it began a ``thorough review of the
team's name,'' after weeks of discussions with the N.F.L. Nike stopped
selling the team's gear, and
\href{https://twitter.com/WalmartInc/status/1279141523867734016?s=20}{Walmart},
\href{https://www.barrons.com/articles/target-walmart-pull-washington-redskins-merchandise-from-online-stores-51594053993}{Target}
and
\href{https://www.usatoday.com/story/money/2020/07/09/washington-redskins-amazon-stop-selling-merchandise/5404092002/}{Amazon}
--- some of the country's largest retailers --- said they would stop
selling Washington's
\href{https://www.walmart.com/cp/nfl-fan-shop/1423455?search_redirect=true\&redirect_query=redskins\&redirectQuery=redskins}{merchandise
on their websites}.

The boycott came
\href{https://www.nytimes3xbfgragh.onion/2013/10/10/sports/football/redskins-name-change-remains-her-unfinished-business.html}{after
decades of pressure on the team} to change the name, which many people
(and some dictionaries) consider to be offensive. In 1992, Native
American activists began a campaign to compel the United States Patent
and Trademark Office to cancel the team's ``redskin'' trademark, a legal
battle that the
\href{https://www.nytimes3xbfgragh.onion/2017/06/19/sports/supreme-court-offensive-names-sports-teams.html}{Supreme
Court ended in 2017}, finding that even potentially disparaging
trademarks are protected by the First Amendment.

In 2014, 50 U.S. senators
\href{https://www.nytimes3xbfgragh.onion/2014/05/22/sports/football/citing-nba-example-senators-urge-nfl-to-act-on-redskins-name.html}{sent
a letter} to the N.F.L. urging the league to step in. And across the
country, waves of universities and schools abandoned mascots and sports
team names with Native American symbols.

But more than
\href{https://www.nytimes3xbfgragh.onion/2020/07/10/sports/football/washington-redskins-name-change-mascots.html}{2,200
high schools still use Native American imagery} in their names or
mascots, according to a database of mascot names.

All the while, Snyder, who purchased the Washington team in 1999,
remained steadfast. ``We will never change the name of the team,'' he
\href{https://www.usatoday.com/story/sports/nfl/redskins/2013/05/09/washington-redskins-daniel-snyder/2148127/}{said
in 2013}, a stance he maintained even in the face of pushback from
activists, politicians and some fans.

What finally changed was, seemingly, wider American society around the
team. After the death of Floyd, there has been a widespread
reconsideration of statues, flags, symbols and mascots considered to be
racist or celebrating racist history.

Snyder's move to drop his team's name could pave the way for his effort
to build a stadium inside the city, a potential relocation from its
current site in a Washington, D.C. suburb. Lawmakers in the city had
said they would not support his goal of building within the District
unless he renamed the team.

Now that the team has let go of its current name, it will have to find a
replacement, a process that requires navigating trademarks and the
league's many licensing deals with partners and which can often take
years. Teams use their name, logos and colors to forge a new identity, a
process that can include speaking with sponsors, fans and other
constituents.

Ed O'Hara, who has designed team names and logos for more than 30 years,
said that dropping the existing name first will buy time for Snyder to
find a replacement. The team's existing colors are unique and powerful,
he said. A good name, though, should have an easy connection to a
mascot, be easy to say and be connected to the market where the team
plays.

``The name is always the hardest part,'' he said. ``You get one chance
to make this right for the next 80 years.''

Gillian R. Brassil contributed reporting.

Advertisement

\protect\hyperlink{after-bottom}{Continue reading the main story}

\hypertarget{site-index}{%
\subsection{Site Index}\label{site-index}}

\hypertarget{site-information-navigation}{%
\subsection{Site Information
Navigation}\label{site-information-navigation}}

\begin{itemize}
\tightlist
\item
  \href{https://help.nytimes3xbfgragh.onion/hc/en-us/articles/115014792127-Copyright-notice}{©~2020~The
  New York Times Company}
\end{itemize}

\begin{itemize}
\tightlist
\item
  \href{https://www.nytco.com/}{NYTCo}
\item
  \href{https://help.nytimes3xbfgragh.onion/hc/en-us/articles/115015385887-Contact-Us}{Contact
  Us}
\item
  \href{https://www.nytco.com/careers/}{Work with us}
\item
  \href{https://nytmediakit.com/}{Advertise}
\item
  \href{http://www.tbrandstudio.com/}{T Brand Studio}
\item
  \href{https://www.nytimes3xbfgragh.onion/privacy/cookie-policy\#how-do-i-manage-trackers}{Your
  Ad Choices}
\item
  \href{https://www.nytimes3xbfgragh.onion/privacy}{Privacy}
\item
  \href{https://help.nytimes3xbfgragh.onion/hc/en-us/articles/115014893428-Terms-of-service}{Terms
  of Service}
\item
  \href{https://help.nytimes3xbfgragh.onion/hc/en-us/articles/115014893968-Terms-of-sale}{Terms
  of Sale}
\item
  \href{https://spiderbites.nytimes3xbfgragh.onion}{Site Map}
\item
  \href{https://help.nytimes3xbfgragh.onion/hc/en-us}{Help}
\item
  \href{https://www.nytimes3xbfgragh.onion/subscription?campaignId=37WXW}{Subscriptions}
\end{itemize}
