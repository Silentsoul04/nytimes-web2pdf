Sections

SEARCH

\protect\hyperlink{site-content}{Skip to
content}\protect\hyperlink{site-index}{Skip to site index}

\href{https://www.nytimes3xbfgragh.onion/section/world/asia}{Asia
Pacific}

\href{https://myaccount.nytimes3xbfgragh.onion/auth/login?response_type=cookie\&client_id=vi}{}

\href{https://www.nytimes3xbfgragh.onion/section/todayspaper}{Today's
Paper}

\href{/section/world/asia}{Asia Pacific}\textbar{}Caught in `Ideological
Spiral,' U.S. and China Drift Toward Cold War

\url{https://nyti.ms/3ethtLi}

\begin{itemize}
\item
\item
\item
\item
\item
\item
\end{itemize}

Advertisement

\protect\hyperlink{after-top}{Continue reading the main story}

Supported by

\protect\hyperlink{after-sponsor}{Continue reading the main story}

\hypertarget{caught-in-ideological-spiral-us-and-china-drift-toward-cold-war}{%
\section{Caught in `Ideological Spiral,' U.S. and China Drift Toward
Cold
War}\label{caught-in-ideological-spiral-us-and-china-drift-toward-cold-war}}

Relations are in free fall. Lines are being drawn. As the two
superpowers clash over technology, territory and clout, a new
geopolitical era is dawning.

\includegraphics{https://static01.graylady3jvrrxbe.onion/images/2020/07/14/world/00China-coldwar1/merlin_158742183_d5bd30f6-1743-4022-80ff-b96b324c4372-articleLarge.jpg?quality=75\&auto=webp\&disable=upscale}

\href{https://www.nytimes3xbfgragh.onion/by/steven-lee-myers}{\includegraphics{https://static01.graylady3jvrrxbe.onion/images/2018/10/15/multimedia/author-steven-lee-myers/author-steven-lee-myers-thumbLarge.png}}\href{https://www.nytimes3xbfgragh.onion/by/paul-mozur}{\includegraphics{https://static01.graylady3jvrrxbe.onion/images/2018/10/15/multimedia/author-paul-mozur/author-paul-mozur-thumbLarge.png}}

By \href{https://www.nytimes3xbfgragh.onion/by/steven-lee-myers}{Steven
Lee Myers} and
\href{https://www.nytimes3xbfgragh.onion/by/paul-mozur}{Paul Mozur}

\begin{itemize}
\item
  Published July 14, 2020Updated July 23, 2020
\item
  \begin{itemize}
  \item
  \item
  \item
  \item
  \item
  \item
  \end{itemize}
\end{itemize}

\href{https://cn.nytimes3xbfgragh.onion/asia-pacific/20200715/cold-war-china-us/}{阅读简体中文版}\href{https://cn.nytimes3xbfgragh.onion/asia-pacific/20200715/cold-war-china-us/zh-hant/}{閱讀繁體中文版}

One by one, the United States has hit at the core tenets of Xi Jinping's
vision for a rising China ready to assume the mantle of superpower.

In a matter of weeks, the Trump administration has imposed
\href{https://www.nytimes3xbfgragh.onion/2020/06/29/business/economy/us-halts-high-tech-exports-hong-kong.html}{sanctions
over punitive policies} in Hong Kong and China's western region of
Xinjiang. It took new measures to suffocate Chinese innovation by
cutting it off from American technology and pushing allies to look
elsewhere. On Monday, it
\href{https://www.nytimes3xbfgragh.onion/2020/07/13/world/asia/south-china-sea-pompeo.html}{challenged
China's claims} in the South China Sea, setting the stage for sharper
confrontation.

And President Trump said on Tuesday that he had signed into law
\href{https://www.nytimes3xbfgragh.onion/2020/07/02/us/politics/senate-china-hong-kong-sanctions.html}{a
bill to punish Chinese officials} for the
\href{https://www.nytimes3xbfgragh.onion/2020/06/29/world/asia/china-hong-kong-security-law-rules.html}{new
security law} that curbs the rights of Hong Kong residents, along with
an executive order ending preferential trade treatment for Hong Kong.

``The power gap is closing, and the ideological gap is widening,'' said
Rush Doshi, director of the China Strategy Initiative at the Brookings
Institution in Washington, adding that
\href{https://www.nytimes3xbfgragh.onion/2020/07/23/world/asia/us-china-consulate.html}{China
and the United States} had entered a downward ``ideological spiral''
years in the making.

``Where's the bottom?'' he asked.

For years, officials and historians have dismissed the idea that a new
Cold War was emerging between the United States and China. The contours
of today's world, the argument went, are simply incomparable to the
decades when the United States and the Soviet Union squared off in an
existential struggle for supremacy. The world was said to be too
interconnected to easily divide into ideological blocs.

Now, lines are being drawn and relations are in free fall, laying the
foundation for a confrontation that will have many of the
characteristics of the Cold War --- and the dangers. As the two
superpowers clash over technology, territory and clout, they face the
same risk of small disputes escalating into military conflict.

The relationship is increasingly imbued with deep distrust and
animosity, as well as the fraught tensions that come with two powers
jockeying for primacy, especially in areas where their interests
collide: in cyberspace and outer space, in
\href{https://www.nytimes3xbfgragh.onion/2020/07/01/world/asia/taiwan-china-hong-kong.html}{the
Taiwan Strait} and the South China Sea, and
\href{https://www.nytimes3xbfgragh.onion/2020/07/11/world/asia/china-iran-trade-military-deal.html}{even
in the Persian Gulf}.

And the coronavirus pandemic, coupled with China's recent aggressive
actions on its borders --- from the Pacific to
\href{https://www.nytimes3xbfgragh.onion/2020/06/17/world/asia/china-india-border.html}{the
Himalayas} --- has turned existing fissures into chasms that could be
difficult to overcome, no matter the outcome of this year's American
presidential election.

\includegraphics{https://static01.graylady3jvrrxbe.onion/images/2020/07/14/world/14china-coldwar2/merlin_174274827_114ed7da-8329-44b2-bb51-a5c212dc61a2-articleLarge.jpg?quality=75\&auto=webp\&disable=upscale}

From Beijing's perspective, it is the United States that has plunged
relations to what China's foreign minister, Wang Yi, said last week was
their lowest point since the countries re-established diplomatic
relations in 1979.

``The current China policy of the United States is based on ill-informed
strategic miscalculation and is fraught with emotions and whims and
McCarthyist bigotry,'' Mr. Wang said, evoking the Cold War himself to
describe the current level of tensions.

``It seems as if every Chinese investment is politically driven, every
Chinese student is a spy and every cooperation initiative is a scheme
with a hidden agenda,'' he added.

Domestic politics in both countries have hardened views and given
ammunition to hawks.

``What cooperation is there between China and the United States right
now?'' said Zheng Yongnian, director of the East Asian Institute at the
National University of Singapore. ``I can't see any substantial
cooperation.''

The pandemic, too, has inflamed tensions, especially in the United
States. Mr. Trump refers to the coronavirus
\href{https://www.nytimes3xbfgragh.onion/2020/06/23/us/politics/trump-race-racism-protests.html}{with
racist tropes}, while Beijing accuses his administration of attacking
China to detract from its failures to contain the virus.

Mr. Trump, in a statement delivered from the Rose Garden Tuesday evening
that focused harshly on China and his presidential rival, Joseph R.
Biden Jr., referred to the pandemic as ``the plague pouring in from
China,'' and said that the Chinese ``could have stopped it.''

Both countries are forcing other nations to take sides, even if they are
disinclined to do so. The Trump administration, for example, has pressed
allies --- with some success in Australia
\href{https://www.nytimes3xbfgragh.onion/2020/07/14/business/uk-bans-huawei-from-5g-network-raising-tensions-with-china.html}{and,
on Tuesday, in Britain} --- to forswear the Chinese tech giant Huawei as
they develop 5G networks. China, facing condemnation over its policies
in Xinjiang and Hong Kong, has rallied countries to make public
demonstrations of support for them.

Image

A protest in Hong Kong this month, hours after China imposed a new
security law on the city.Credit...Lam Yik Fei for The New York Times

At the United Nations Humans Rights Council in Geneva, 53 nations ---
from Belarus to Zimbabwe --- signed a statement supporting China's
\href{https://www.nytimes3xbfgragh.onion/2020/06/29/world/asia/china-hong-kong-security-law-rules.html}{new
security law for Hong Kong}. Only 27 nations on the council criticized
it, most of them European democracies, along with Japan, Australia and
New Zealand. Such blocs would not have been unfamiliar at the height of
the Cold War.

China has also wielded its vast economic power as a tool of political
coercion, cutting off imports of beef and barley
\href{https://www.nytimes3xbfgragh.onion/2020/06/30/world/australia/cyber-defense-china-hacking.html}{from
Australia} because its government called for an international
investigation into the origins of the pandemic. On Tuesday, Beijing said
it would sanction the American aerospace manufacturer Lockheed Martin
over recent weapons sales to Taiwan.

With the world distracted by the pandemic, China has also wielded its
military might, as it did by testing its disputed frontier with India in
April and May. That led to the
\href{https://www.nytimes3xbfgragh.onion/2020/06/17/world/asia/china-india-border.html}{first
deadly clash there} since 1975. The damage to the relationship could
take years to repair.

Increasingly, China seems willing to accept the risks of such actions.
Only weeks later, it asserted a new territorial claim in Bhutan, the
mountain kingdom that is closely allied with India.

With China
\href{https://www.nytimes3xbfgragh.onion/2020/05/24/world/asia/china-hong-kong-taiwan.html}{menacing
vessels}from Vietnam, Malaysia and Indonesia in the South China Sea, the
United States dispatched two aircraft carriers through the waters last
month in an aggressive show of strength. Further brinkmanship appears
inevitable now that the State Department has declared China's claims
there illegal.

Image

A Chinese Coast Guard ship near the Scarborough Shoal, a reef in the
South China Sea~claimed by both China and the Philippines, in
2016.Credit...Sergey Ponomarev for The New York Times

A spokesman for China's foreign ministry, Zhao Lijian, said on Tuesday
that the American declaration would undermine regional peace and
stability, asserting that China had controlled the islands in the sea
``for thousands of years,'' which is not true. As he stated, the
Republic of China --- then controlled by the Nationalist forces of
Chiang Kai-shek --- only made a formal claim in 1948.

``China is committed to resolving territorial and jurisdictional
disputes with directly related sovereign states through negotiations and
consultations,'' he said.

That is not how its neighbors see things. Japan warned this week that
China was attempting to ``alter the status quo in the East China Sea and
the South China Sea.'' It called China a more serious long-term threat
than a nuclear-armed North Korea.

Michael A. McFaul, a former American ambassador to Russia and professor
of international studies at Stanford University, said China's recent
maneuvering appeared to be ``overextended and overreaching,'' likening
it to one of the most fraught moments of the Cold War.

``It does remind me of Khrushchev,'' he said. ``He's lashing out, and
suddenly he's in a Cuban missile crisis with the U.S.''

A backlash against Beijing appears to be growing. The tensions are
particularly clear in tech, where China has sought to compete with the
world in cutting-edge technologies like artificial intelligence and
microchips, while harshly restricting what people can read, watch or
listen to inside the country.

Image

Protesters in Jammu, India, burned photos of Mr. Xi this
month.Credit...Channi Anand/Associated Press

If the Berlin Wall was the physical symbol of the first Cold War, the
Great Firewall could well be the virtual symbol of the new one.

What began as a divide in cyberspace to insulate Chinese citizens from
views not authorized by the Communist Party has now proved to be a
prescient indicator of the deeper fissures between China and much of the
Western world.

Mr. Wang, in his speech, said China had never sought to impose its way
on other countries. But it has done exactly that by
\href{https://www.nytimes3xbfgragh.onion/2020/06/11/technology/zoom-china-tiananmen-square.html}{getting
Zoom to censor talks} that were being held in the United States and by
launching cyberattacks on Uighurs across the globe.

Its controls have been hugely successful at home in stifling dissent and
helping to seed domestic internet giants, but they have won China little
influence abroad. India's move to
\href{https://www.nytimes3xbfgragh.onion/2020/06/29/world/asia/tik-tok-banned-india-china.html?searchResultPosition=1}{block
59 Chinese apps} threatens to hobble China's biggest overseas internet
success to date, the meme-laden short-video app TikTok.

Last week, TikTok also shut down in Hong Kong because of China's new
national security law there. The American tech giants Facebook, Google
and Twitter said they would stop reviewing data requests from the Hong
Kong authorities as they assessed the law's restrictions.

Image

A Huawei advertisement in Shanghai. The tech giant is at the center of
one of China's disputes with the United States.Credit...Lam Yik Fei for
The New York Times

``China is big, it will be successful, it will develop its own tech, but
there are limits to what it can do,'' said James A. Lewis, a former
American official who writes on cybersecurity and espionage for the
Center for Strategic and International Studies in Washington.

Even in places where China has succeeded in selling its technology, the
tide appears to be turning.

Beijing's recent truculence has now led the United Kingdom to block new
Huawei equipment from going into its networks, and the Trump
administration is determined to cut the company off from microchips and
other components it needs. To counter, Beijing has redoubled efforts to
build homegrown options.

Calls for a total decoupling of China's supply chain from American tech
companies are unrealistic in the short term, and would prove massively
expensive in the longer term. Still, the United States has moved to pull
Taiwan's microchip manufacturing --- crucial to the supply chains of
Huawei and other Chinese tech companies --- closer to its backyard, with
plans to support a new Taiwan Semiconductor Manufacturing plant in
Arizona.

Mr. Wang, the foreign minister, urged the United States to step back and
seek areas where the two countries can work together. Pessimism about
the relationship is nonetheless widespread, though most Chinese
officials and analysts blame the Trump administration for trying to
deflect attention from its failure to control the pandemic.

Image

American and Chinese flags at the site of the meeting between Mr. Trump
and Mr. Xi in Osaka.~Credit...Erin Schaff/The New York Times

``It is not difficult to see that under the impact of the coronavirus in
this U.S. election year various powers in the U.S. are focused on
China,'' Zhao Kejin, a professor of international relations at Tsinghua
University, wrote in a recent paper. ``The China-U.S. relationship faces
the most serious moment since the establishment of diplomatic
relations.''

While he eschewed the idea of a new Cold War, his alternative phrasing
was no more reassuring: ``The new reality is China-U.S. relations are
not entering `a new Cold War' but sliding into a `soft war.'''

Reporting and research were contributed by Claire Fu in Beijing, Lin
Qiqing in Shanghai and Motoko Rich in Tokyo.

Advertisement

\protect\hyperlink{after-bottom}{Continue reading the main story}

\hypertarget{site-index}{%
\subsection{Site Index}\label{site-index}}

\hypertarget{site-information-navigation}{%
\subsection{Site Information
Navigation}\label{site-information-navigation}}

\begin{itemize}
\tightlist
\item
  \href{https://help.nytimes3xbfgragh.onion/hc/en-us/articles/115014792127-Copyright-notice}{©~2020~The
  New York Times Company}
\end{itemize}

\begin{itemize}
\tightlist
\item
  \href{https://www.nytco.com/}{NYTCo}
\item
  \href{https://help.nytimes3xbfgragh.onion/hc/en-us/articles/115015385887-Contact-Us}{Contact
  Us}
\item
  \href{https://www.nytco.com/careers/}{Work with us}
\item
  \href{https://nytmediakit.com/}{Advertise}
\item
  \href{http://www.tbrandstudio.com/}{T Brand Studio}
\item
  \href{https://www.nytimes3xbfgragh.onion/privacy/cookie-policy\#how-do-i-manage-trackers}{Your
  Ad Choices}
\item
  \href{https://www.nytimes3xbfgragh.onion/privacy}{Privacy}
\item
  \href{https://help.nytimes3xbfgragh.onion/hc/en-us/articles/115014893428-Terms-of-service}{Terms
  of Service}
\item
  \href{https://help.nytimes3xbfgragh.onion/hc/en-us/articles/115014893968-Terms-of-sale}{Terms
  of Sale}
\item
  \href{https://spiderbites.nytimes3xbfgragh.onion}{Site Map}
\item
  \href{https://help.nytimes3xbfgragh.onion/hc/en-us}{Help}
\item
  \href{https://www.nytimes3xbfgragh.onion/subscription?campaignId=37WXW}{Subscriptions}
\end{itemize}
