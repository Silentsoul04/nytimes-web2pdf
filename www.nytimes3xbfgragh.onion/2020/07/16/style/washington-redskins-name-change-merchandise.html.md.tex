Sections

SEARCH

\protect\hyperlink{site-content}{Skip to
content}\protect\hyperlink{site-index}{Skip to site index}

\href{https://www.nytimes3xbfgragh.onion/section/style}{Style}

\href{https://myaccount.nytimes3xbfgragh.onion/auth/login?response_type=cookie\&client_id=vi}{}

\href{https://www.nytimes3xbfgragh.onion/section/todayspaper}{Today's
Paper}

\href{/section/style}{Style}\textbar{}What Happens Now to All That
Washington N.F.L. Merchandise?

\url{https://nyti.ms/2B3R2OL}

\begin{itemize}
\item
\item
\item
\item
\item
\item
\end{itemize}

Advertisement

\protect\hyperlink{after-top}{Continue reading the main story}

Supported by

\protect\hyperlink{after-sponsor}{Continue reading the main story}

\hypertarget{what-happens-now-to-all-that-washington-nfl-merchandise}{%
\section{What Happens Now to All That Washington N.F.L.
Merchandise?}\label{what-happens-now-to-all-that-washington-nfl-merchandise}}

The Redskins' decision to rebrand transforms decades of memorabilia into
artifacts of a racist name.

\includegraphics{https://static01.graylady3jvrrxbe.onion/images/2020/07/19/fashion/16washingtonmerch-1/16washingtonmerch-1-articleLarge.jpg?quality=75\&auto=webp\&disable=upscale}

\href{https://www.nytimes3xbfgragh.onion/by/jonah-engel-bromwich}{\includegraphics{https://static01.graylady3jvrrxbe.onion/images/2018/02/16/multimedia/author-jonah-engel-bromwich/author-jonah-engel-bromwich-thumbLarge.jpg}}

By
\href{https://www.nytimes3xbfgragh.onion/by/jonah-engel-bromwich}{Jonah
Engel Bromwich}

\begin{itemize}
\item
  July 16, 2020
\item
  \begin{itemize}
  \item
  \item
  \item
  \item
  \item
  \item
  \end{itemize}
\end{itemize}

Matt Pearson was born in
\href{https://www.nytimes3xbfgragh.onion/2020/07/16/sports/football/washington-sexual-assault-harassment-dan-snyder.html}{Washington},
D.C., in 1983, which means he is old enough to remember watching his
hometown football team win their second Super Bowl in 1988. He remembers
the smell of chips and beer, his parents screaming like crazy at the
television, the team's record-breaking comeback and victory.

After that, said Mr. Pearson, who is now an industrial designer, ``my
level of football fanhood was complete and devotional.''

By the time he could turn the television on himself, ``there was nothing
that would take me away from Sundays with the Washington football
team,'' he said. ``At that time I called them the
\href{https://www.nytimes3xbfgragh.onion/2020/07/16/sports/football/washington-sexual-assault-harassment-dan-snyder.html}{Redskins}.
I was obsessed.''

When first confronted with the idea that the team's name was racist,
around the time he was in middle school, Mr. Pearson, who is Black,
said: ``My brain couldn't handle it.'' But as he grew up, and his
politics began to develop, he turned against the name and the league,
both of which he found increasingly unacceptable. He no longer watches
professional football.

Still, his house, like the homes of many people who grew up in D.C. at
the time --- including that of my own family; we are friendly with the
Pearsons --- was once littered with jerseys and memorabilia, including
limited-edition Super Bowl victory Coke cans that featured the team's
logo. In my own childhood home, there is still a collector's edition
Redskins-branded Monopoly game, made in 2005.

\href{https://www.nytimes3xbfgragh.onion/2020/07/13/sports/football/washington-redskins-new-name.html}{The
team chose to change its name this week}, a decision that had been urged
for decades by
\href{https://www.washingtonpost.com/sports/2020/07/07/redskins-conduct-name-review-native-american-groups-say-they-havent-heard-team/}{Native
American activists and groups} but forced by major sponsors including
Nike and FedEx. The change will transform jerseys and memorabilia from
the merchandise of an active team to artifacts of its racist former
name. (On Thursday, multiple allegations of sexual harassment and other
forms of abuse within the Redskins organization were
\href{https://www.washingtonpost.com/sports/2020/07/16/redskins-sexual-harassment-larry-michael-alex-santos/?arc404=true}{reported
by The Washington Post}.)

Since the death of George Floyd in police custody in late May set off
protests against racism throughout the country, brands including
\href{https://www.nytimes3xbfgragh.onion/2020/06/17/business/media/aunt-jemima-racial-stereotype.html}{Aunt
Jemima} and
\href{https://www.nytimes3xbfgragh.onion/2020/06/17/business/aunt-jemima-mrs-butterworth-uncle-ben.html}{Uncle
Ben's} have reconsidered the racist imagery they use to sell products.
But while those images are affixed to disposable food items, the
Redskins name and logo appear on innumerable pieces of merchandise that
fans had no plans to get rid of. Some of them are old items that hadn't
been given much thought. Others are souvenirs and keepsakes with
significant meaning to their owners.

The millions of people from D.C., Maryland and Virginia who supported or
support the local team, whether enthusiastically or reluctantly, are
going to have to decide: What will happen to all their Redskins gear?

\includegraphics{https://static01.graylady3jvrrxbe.onion/images/2020/07/19/fashion/16washingtonmerch-3/16washingtonmerch-3-articleLarge.jpg?quality=75\&auto=webp\&disable=upscale}

\hypertarget{that-was-our-team}{%
\subsection{`That Was Our Team'}\label{that-was-our-team}}

Among those encouraging people to re-evaluate their relationships with
their branded belongings are former Redskins players. Darrell Green, a
Washington legend and pro football Hall-of-Famer who holds the league's
record for most consecutive seasons with an interception,
\href{https://twitter.com/DanGrazianoESPN/status/1282741424820310018}{told
ESPN} this week that he would consider throwing away his old jerseys and
memorabilia.

He elaborated in an interview with The New York Times, saying that now
was the time to look to a better future instead of clinging to
nostalgia.

``We shouldn't have longings for something that is past, that is a
jersey or a moment, that is to the detriment of living people,'' he
said, referring to Native American communities. ``My millions, my
trophies, my videos, my jerseys, my stuff, it's not worth it. My vantage
point is that human beings come first.''

Some fans are rushing to rid themselves of their gear. Patrick Casady,
25, lives in Maryville, Mo., but chose to root for the Washington N.F.L.
team when he was young because he liked their colors and connected with
the players of that era.

I contacted him through eBay, where he was selling a jersey of the
quarterback Robert Griffin III, who was for a
time\href{https://www.washingtonpost.com/sports/2020/06/25/rg-iii-his-time-with-redskins-being-an-african-american-quarterback-social-change/}{the
bright shining hope of the D.C. Metropolitan area}.

``I always knew growing up that the name was racist. No one ever did
anything about it for the longest time, it was just kind of there,'' Mr.
Casady said.

Eventually, he stopped wearing his many jerseys outside of the house.
``You kind of feel a little awkward in it honestly,'' he said. ``You
know that, I personally would never say it to a person of Indian descent
or anything. So that shows you that it's not good, it's racist.''

But many fans of the team say that they will continue to proudly sport
the team's logo and name. In
\href{https://es.redskins.com/topic/439661-will-you-still-wear-your-redskins-gear-if-the-name-changes/page/6/}{a
thread on the team's official message board} asking whether participants
would continue to wear their old gear, more than 40 respondents said
they would, about twice as many as those who said they would not.

``Yes absolutely will wear my Redskin gear! Never considered it racist!
And still don't!''
\href{https://es.redskins.com/topic/439661-will-you-still-wear-your-redskins-gear-if-the-name-changes/page/2/?tab=comments\#comment-11804256}{said
one}, who identified himself as an electrician who has been rooting for
the team since 1968.

Jeff Rinehart Jr., 40, from Virginia Beach, said that he inherited a
love of the team from his father, who died in 1998. He is a member of a
charitable foundation associated with the team, the
\href{https://www.hogshaven.com/2020/2/9/21129032/the-hogfarmers-supporting-both-the-redskins-and-kids-with-pediatric-cancer-and-their-families}{HogFarmers,}
which supports children with cancer and their families.

Mr. Rhinehart started attending games in 2001, and said that going made
him feel closer to his father. He does not plan to stop wearing his team
gear, he said.

Image

Jeff Rinehart Jr. has a room dedicated to the team's merchandise and
memorabilia.

Jesenia Clepper, 41, will not only continue to sport the logo and the
team name but is also making and selling her own shirts prominently
displaying them. When we spoke, she said she was wearing a T-shirt with
the team's logo over an American flag, as well as a branded wristwatch.

She said that she grew up extremely poor in Houston, and that she and
her nine siblings, including her brother Roosevelt, did what they could
to entertain themselves without spending money.

``We played football and we were the Redskins and the Cowboys,'' she
said. ``That was back in '83. We just adopted the Redskins into our
lives and that was our team.'' (The team won its first Super Bowl in
1983).

Then in 1989, Roosevelt had an aneurysm, which permanently changed his
behavior and cognitive abilities, Ms. Clepper said, choking up as she
told the story. For her, the team is a reminder of a happier time. She
does not see its name as racist. (Ms. Clepper is Hispanic.)

``I can't even put it in my head that people actually think that we as
fans wear Redskins attire as a form of hate or to disparage Native
Americans,'' she said, contrasting the symbol with the Confederate flag,
which she said was a racist symbol.

The decision to change the name has broken her heart, and she is now
``done with the N.F.L.,'' she said. ``No more football for me.''

Image

In the 1980s, Washington's offensive line was nicknamed The Hogs. Fans
have since incorporated pig imagery into shows of support for the team.~

\hypertarget{shame-and-preservation}{%
\subsection{Shame and Preservation}\label{shame-and-preservation}}

Innumerable pieces of branded merchandise are churned out on behalf of
professional football teams each year. A person familiar with the
N.F.L.'s licensing operation who requested anonymity because he was not
authorized to speak for the league, estimated that in the past 20 years,
about \$700 million worth of Washington N.F.L. gear had been sold
nationally.

Some of the merchandise associated with the team now belongs to the Jim
Crow Museum of Racist Memorabilia, in Big Rapids, Mich. The museum's
curator, David Pilgrim, said that he has collected memorabilia devoted
to various sports teams
\href{https://www.nytimes3xbfgragh.onion/2020/07/14/arts/design/washington-football-logo-native-american.html}{with
names that invoked Native American people}over the years, including
about a dozen items devoted to the Washington team.

Mr. Pilgrim said that the name change would provide fans with an
opportunity to question what was in their closets.

``A fan's attachment to a jersey can't be the compelling factor here,''
Mr. Pilgrim said. ``I'm not going to tell people what to do with them
but if they visited our museum, one of the questions I would ask is,
`Why do you keep them?'''

Museums don't have the capacity --- or necessarily the desire --- to
store
\href{https://www.nytimes3xbfgragh.onion/2020/07/14/arts/design/washington-football-logo-native-american.html}{unending
amounts}of Washington-branded merchandise. Hunter Old Elk, a curator at
the Indians Plains Museum in Cody, Wyo., and an Indigenous woman of the
Crow and Yakama Nations, said that what should be done with old gear was
a complicated question.

``I think there are a couple of examples that could be used in
interpretation,'' she said. But the vast majority, ``since they're still
mass produced, there's not a lot of value historically in them.''

She said that it would be left to fans to decide whether to save or
dispose of their belongings, which she called ``images of a continued
systemic racism.'' She didn't have high hopes that they would throw them
out. She expected them to become similar to symbols of the Confederacy,
proud taboos for those who continue to hold on to them.

Lyra Monteiro, a professor at Rutgers University who studies how
cultural artifacts can help us understand American history, also saw
that as a possible outcome. She said that when people were compelled to
recognize that something is offensive to others, particularly when that
thing has emotional import, ``what mostly comes out as a reaction to
that is anger and defensiveness.''

She said that it was a no-brainer that people would have a hard time
letting go of the gear, especially given what it meant to their
childhoods. Emotional bonds help a culture perpetuate itself.

But Ms. Monteiro, an archaeologist by training, said that the tendency
to erase all evidence of objects that had been recognized as racist or
otherwise offensive only contributed to further ignorance.

She cited the history of blackface, which she said could be difficult
for her students to understand as a problem because they so rarely
encountered it before it surfaced in recent political scandals.

She was hopeful that for some fans, the confrontation forced by the name
change ``can be the deepest and most profound learning opportunity.''

``You know the power of that cultural product, because of what it means
to you,'' she said. ``That can be what drives you to dig into it more
deeply.''

Mr. Pearson said that he thinks many of his old jerseys have been
donated. But he said that he could see some benefit to keeping an item
or two of apparel, even if it seemed like ``a relic from another time.''

Its purpose: ``To remind myself of my own fervor,'' he said. ``Just to
remember how much your views can change.''

Advertisement

\protect\hyperlink{after-bottom}{Continue reading the main story}

\hypertarget{site-index}{%
\subsection{Site Index}\label{site-index}}

\hypertarget{site-information-navigation}{%
\subsection{Site Information
Navigation}\label{site-information-navigation}}

\begin{itemize}
\tightlist
\item
  \href{https://help.nytimes3xbfgragh.onion/hc/en-us/articles/115014792127-Copyright-notice}{©~2020~The
  New York Times Company}
\end{itemize}

\begin{itemize}
\tightlist
\item
  \href{https://www.nytco.com/}{NYTCo}
\item
  \href{https://help.nytimes3xbfgragh.onion/hc/en-us/articles/115015385887-Contact-Us}{Contact
  Us}
\item
  \href{https://www.nytco.com/careers/}{Work with us}
\item
  \href{https://nytmediakit.com/}{Advertise}
\item
  \href{http://www.tbrandstudio.com/}{T Brand Studio}
\item
  \href{https://www.nytimes3xbfgragh.onion/privacy/cookie-policy\#how-do-i-manage-trackers}{Your
  Ad Choices}
\item
  \href{https://www.nytimes3xbfgragh.onion/privacy}{Privacy}
\item
  \href{https://help.nytimes3xbfgragh.onion/hc/en-us/articles/115014893428-Terms-of-service}{Terms
  of Service}
\item
  \href{https://help.nytimes3xbfgragh.onion/hc/en-us/articles/115014893968-Terms-of-sale}{Terms
  of Sale}
\item
  \href{https://spiderbites.nytimes3xbfgragh.onion}{Site Map}
\item
  \href{https://help.nytimes3xbfgragh.onion/hc/en-us}{Help}
\item
  \href{https://www.nytimes3xbfgragh.onion/subscription?campaignId=37WXW}{Subscriptions}
\end{itemize}
