Sections

SEARCH

\protect\hyperlink{site-content}{Skip to
content}\protect\hyperlink{site-index}{Skip to site index}

\href{https://www.nytimes3xbfgragh.onion/section/nyregion}{New York}

\href{https://myaccount.nytimes3xbfgragh.onion/auth/login?response_type=cookie\&client_id=vi}{}

\href{https://www.nytimes3xbfgragh.onion/section/todayspaper}{Today's
Paper}

\href{/section/nyregion}{New York}\textbar{}3 Weeks After Primary, N.Y.
Officials Still Can't Say Who Won Key Races

\url{https://nyti.ms/3hkNdo7}

\begin{itemize}
\item
\item
\item
\item
\item
\end{itemize}

Advertisement

\protect\hyperlink{after-top}{Continue reading the main story}

Supported by

\protect\hyperlink{after-sponsor}{Continue reading the main story}

\hypertarget{3-weeks-after-primary-ny-officials-still-cant-say-who-won-key-races}{%
\section{3 Weeks After Primary, N.Y. Officials Still Can't Say Who Won
Key
Races}\label{3-weeks-after-primary-ny-officials-still-cant-say-who-won-key-races}}

Tens of thousands of absentee ballots in New York are still uncounted
and many races have yet to be called. What will November look like?

\includegraphics{https://static01.graylady3jvrrxbe.onion/images/2020/07/16/nyregion/16nyabsentee/16nyabsentee-articleLarge.jpg?quality=75\&auto=webp\&disable=upscale}

\href{https://www.nytimes3xbfgragh.onion/by/jesse-mckinley}{\includegraphics{https://static01.graylady3jvrrxbe.onion/images/2018/02/20/multimedia/author-jesse-mckinley/author-jesse-mckinley-thumbLarge.jpg}}\href{https://www.nytimes3xbfgragh.onion/by/luis-ferre-sadurni}{\includegraphics{https://static01.graylady3jvrrxbe.onion/images/2018/06/22/multimedia/author-luis-ferre-sadurni/author-luis-ferre-sadurni-thumbLarge.png}}

By \href{https://www.nytimes3xbfgragh.onion/by/jesse-mckinley}{Jesse
McKinley} and
\href{https://www.nytimes3xbfgragh.onion/by/luis-ferre-sadurni}{Luis
Ferré-Sadurní}

\begin{itemize}
\item
  July 17, 2020
\item
  \begin{itemize}
  \item
  \item
  \item
  \item
  \item
  \end{itemize}
\end{itemize}

More than three weeks after the New York primaries, election officials
have
\href{https://www.nytimes3xbfgragh.onion/2020/08/03/nyregion/nyc-mail-ballots-voting.html}{not
yet counted an untold number of mail-in absentee ballots}, leaving
numerous closely watched races unresolved, including two key Democratic
congressional contests.

The absentee ballot count --- greatly inflated this year after the state
expanded the vote-by-mail option because of the coronavirus pandemic ---
has been painstakingly slow, and hard to track, with no running account
of the vote totals available.

In some cases, the tiny number of ballots counted has bordered on the
absurd: In the 12th Congressional District, where Representative Carolyn
B. Maloney is \href{https://web.enrboenyc.us/CD235630.html}{fighting for
her political life} against her challenger, Suraj Patel, only 800 of
some 65,000 absentee ballots had been tabulated as of Wednesday,
according to Mr. Patel, though thousands had been disqualified.

Another young insurgent candidate, Ritchie Torres, held a commanding
lead in his Democratic House primary race after a count of machine-cast
ballots on primary night. Mr. Torres, a New York City councilman, was
leading a pack of contenders in the 15th Congressional District in the
Bronx.

On Friday, one race was finally called, as Jamaal Bowman, a
middle-school principal from Yonkers who had the support of the
Democratic Party's progressive wing, was declared the winner in the 16th
Congressional District, which straddles the city's northern border. He
defeated Representative Eliot L. Engel, a longtime congressman with
ample establishment backing.

The delays in New York's primaries raise huge concerns about how the
state will handle the general election in November, and may offer a
cautionary note for other states as they weigh whether to embrace, and
how to implement, a vote-by-mail system because of the pandemic.

The primary reason for the delays is the sheer number of absentee
ballots: In New York City, 403,203 ballots were mailed for the June
primary; as a comparison, just 76,258 absentee and military ballots were
counted in New York City in the 2008 general election, when Barack Obama
was elected president**.**

But other factors also have played a part.

Election officials said they were left scrambling when Gov. Andrew M.
Cuomo decided in late April to send absentee ballot applications to
every registered voter; a May court decision that reinstituted a June
presidential primary also complicated matters.

Officials said they were also hamstrung by outdated technology,
including using toner-and-tray copiers, instead of computerized
scanners, to handle requests from candidates for copies of absentee
ballots; those copies are often used in legal challenges to try to
restore disqualified ballots or challenge the legitimacy of others.

``The board has received an unprecedented volume of absentee ballots,
and also an unprecedented number of requests for copies of those
absentee ballots from various campaigns,'' said Michael Ryan, the
executive director of the New York City Board of Elections.

``While I appreciate the public's desire to know the results, at the end
of the process we must ensure the integrity of the elections and the
accuracy of the results,'' he said.

\includegraphics{https://static01.graylady3jvrrxbe.onion/images/2020/07/16/nyregion/16nyabsentee1/merlin_173838546_7d4eecc8-651c-48e8-bd06-7913c7a84a03-articleLarge.jpg?quality=75\&auto=webp\&disable=upscale}

For the June primary, the elections board did not hire additional staff,
even as hundreds of thousands of absentee ballots were mailed to voters.

``The staff that we have is the staff that we have,'' Mr. Ryan said, who
added that ``you want people who are familiar with the process. This is
ultimately too important a task to leave to untrained people.''

The process for counting absentee ballots is labor- and time-intensive:
Before absentee ballots can begin to be counted, election officials have
to sift through mounds of ballots to determine which are valid and which
are not. The process is closed to the public, though campaigns are
allowed to challenge these decisions.

Once the ballots are determined to be valid, Board of Elections staff
members --- one Democrat, one Republican --- begin the actual counting,
sitting side-by-side. Even here, there's evidence of the impact of the
coronavirus era: The tables are spaced out and outfitted with clear
partitions to protect the workers from the virus.

The workers open each ballot's envelope and go through each ballot to
determine whether, for example, a ballot has an extraneous marking that
could disqualify it. They also hold up each ballot to the candidates or
their representatives --- known as watchers --- who are intently
monitoring the process from six feet away.

The ballots are then run through a machine that tallies the votes for
each candidate.

Mr. Ryan said his workers were ``working around the clock'' and had been
doing so ``throughout the Covid-19 emergency.''

Candidates and their campaigns have nonetheless been deeply frustrated
by the slow pace, and increasingly concerned about what it portends for
the general election in the fall.

``This is just a primary: Imagine November with the presidential race
and all the Senate and House races,'' said Rebecca Katz, a progressive
political consultant who serves as an adviser to Mr. Bowman's campaign.
``What's going to happen to our country?''

John Conklin, a spokesman for the New York State Board of Elections,
said that the ``astronomically high number of absentee ballots''
overwhelmed a system built to handle far, far fewer.

``The system is built to process 3 to 5 percent of the election in
absentee ballots, not 40 to 60 percent of the election,'' Mr. Conklin
said, adding that it is ``not possible to change this process
overnight.''

Moreover, Mr. Conklin said that the delays could be repeated in
November, if ``local boards are not given additional resources'' for
hiring and overtime pay.

``You will see a similar extended counting period, if we see an equally
high number of absentee ballots,'' he said.

Voting-rights groups have also been alarmed by reports of
\href{https://theintercept.com/2020/07/16/new-york-mail-in-ballots-thrown-out/}{thousands
of disqualified ballots}, raising the specter of widespread voter
disenfranchisement.

Preliminary data obtained by The New York Times shows that about 20
percent of ballots have been invalidated in the Manhattan and Queens
portions of the 12th District, for instance, and almost 30 percent in
the Brooklyn portion of the district. Mr. Patel said he believed some
ballots had been invalidated because voters dropped them off on June 23,
the deadline to postmark ballots, but they weren't postmarked until the
following day by the Postal Service.

Data compiled by \href{https://www.newreformers.org/candidates}{New
Reformers}, a Queens political organization, shows that election
officials have invalidated at least 22,000 out of about 89,000 absentee
ballots received in the borough, or about 25 percent, sometimes for
minor issues like an envelope's being sealed with tape or missing
signatures on ballot envelopes.

``The state was not ready for this,'' said Sochie Nnaemeka, the New York
director of the Working Families Party, a progressive group which backed
several challengers to Democratic incumbents. ``There is rightfully fear
that voter choice and voter participation will be eroded through this
process.''

Image

Jamaal Bowman, shown in June, was named the victor on Friday in his race
against Representative Eliot Engel.Credit...Stephanie Keith/Getty Images

The balky pace of the 2020 primary count has given new urgency to
supporters of \href{https://letnyvote.org/covid-19}{a batch of
voting-related bills} being considered in Albany, including a bill that
would allow any ballot received
\href{https://legislation.nysenate.gov/pdf/bills/2019/S8367}{within
seven days of Election Day to be considered valid}, regardless of
whether or not it had a postmark.

Another
\href{https://legislation.nysenate.gov/pdf/bills/2019/S8369}{bill} would
allow election officials to contact voters whose ballot envelopes were
incorrectly sealed --- with tape, for instance, rather than saliva ---
and would allow those voters to make written testimony that they had, in
fact, cast the ballot, thus allowing it to be counted.

Other states, such as Colorado, allow election officials to contact
voters
\href{https://www.ncsl.org/research/elections-and-campaigns/vopp-table-15-states-that-permit-voters-to-correct-signature-discrepancies.aspx}{to
``cure'' small problems} with ballots, like signature inconsistencies,
before their ballots are declared invalid.

``It gives a voter an opportunity to know their ballot is not going to
be counted,'' said Crisanta Duran, a Democrat and former speaker of the
Colorado House of Representatives, who added that she felt encouraged
that New York was at least trying to improve voter participation.

``Look, I think it's wonderful that this meaningful step was taken,
though it was unfortunately due to Covid,'' she said. ``But there are
going to be growing pains.''

New York, which passed
\href{https://www.nytimes3xbfgragh.onion/2019/01/10/nyregion/voting-reform-election-ny.html}{a
series of voting changes in 2019} after a wave of progressive Democrats
were swept into power in Albany, has long lagged behind many other
states in electoral innovations.

More than two dozen states allow some form of vote-by-mail elections,
and five states --- including early adopters like Oregon and Washington
--- conduct all their elections by mail, something that officials say
has increased both turnout and voter awareness. Other states have also
indicated that they will act to address the election issues presented by
the coronavirus in November: California, the nation's most populous
state, will send a general election ballot to every voter in accordance
with
\href{https://www.gov.ca.gov/wp-content/uploads/2020/05/05.08.2020-EO-N-64-20-signed.pdf}{an
executive order} issued in early May by Gov. Gavin Newsom.

But it is unclear whether New York will follow suit and allow all voters
to vote absentee in November --- especially after the snafus in the
primary. On Thursday, Mr. Cuomo did not answer that question directly,
but conceded that there were ``additional complications'' for local
boards of elections ``dealing with a greater administrative burden.''

``Did it slow the results? Yes,'' the governor said, adding, ``Life is
alternatives, and I don't know that we had a better alternative. And I
don't know that we're going to have a better alternative in November.''

At least one candidate on the November ballot said that he hoped things
would be better.

``I am very worried,'' said Mondaire Jones, the Democratic primary
winner in the 17th Congressional District, north of the city, where the
\href{https://www.nytimes3xbfgragh.onion/2020/07/14/nyregion/mondaire-jones-house-primary.html}{race
was called} on Tuesday, three weeks after Election Day.

``The volume of absentee ballots is going be exponentially larger,'' he
continued. ``It means our boards of elections have to hire adequate
number of staff. It means you have to have space and facilities for
lawyers and volunteers to observe and make challenges. It means you
can't have boards of elections that send out the wrong polling
address.''

Advertisement

\protect\hyperlink{after-bottom}{Continue reading the main story}

\hypertarget{site-index}{%
\subsection{Site Index}\label{site-index}}

\hypertarget{site-information-navigation}{%
\subsection{Site Information
Navigation}\label{site-information-navigation}}

\begin{itemize}
\tightlist
\item
  \href{https://help.nytimes3xbfgragh.onion/hc/en-us/articles/115014792127-Copyright-notice}{©~2020~The
  New York Times Company}
\end{itemize}

\begin{itemize}
\tightlist
\item
  \href{https://www.nytco.com/}{NYTCo}
\item
  \href{https://help.nytimes3xbfgragh.onion/hc/en-us/articles/115015385887-Contact-Us}{Contact
  Us}
\item
  \href{https://www.nytco.com/careers/}{Work with us}
\item
  \href{https://nytmediakit.com/}{Advertise}
\item
  \href{http://www.tbrandstudio.com/}{T Brand Studio}
\item
  \href{https://www.nytimes3xbfgragh.onion/privacy/cookie-policy\#how-do-i-manage-trackers}{Your
  Ad Choices}
\item
  \href{https://www.nytimes3xbfgragh.onion/privacy}{Privacy}
\item
  \href{https://help.nytimes3xbfgragh.onion/hc/en-us/articles/115014893428-Terms-of-service}{Terms
  of Service}
\item
  \href{https://help.nytimes3xbfgragh.onion/hc/en-us/articles/115014893968-Terms-of-sale}{Terms
  of Sale}
\item
  \href{https://spiderbites.nytimes3xbfgragh.onion}{Site Map}
\item
  \href{https://help.nytimes3xbfgragh.onion/hc/en-us}{Help}
\item
  \href{https://www.nytimes3xbfgragh.onion/subscription?campaignId=37WXW}{Subscriptions}
\end{itemize}
