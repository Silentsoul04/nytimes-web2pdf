Sections

SEARCH

\protect\hyperlink{site-content}{Skip to
content}\protect\hyperlink{site-index}{Skip to site index}

\href{https://www.nytimes3xbfgragh.onion/section/us}{U.S.}

\href{https://myaccount.nytimes3xbfgragh.onion/auth/login?response_type=cookie\&client_id=vi}{}

\href{https://www.nytimes3xbfgragh.onion/section/todayspaper}{Today's
Paper}

\href{/section/us}{U.S.}\textbar{}C.T. Vivian, Martin Luther King's
Field General, Dies at 95

\url{https://nyti.ms/2WxhKXL}

\begin{itemize}
\item
\item
\item
\item
\item
\item
\end{itemize}

Advertisement

\protect\hyperlink{after-top}{Continue reading the main story}

Supported by

\protect\hyperlink{after-sponsor}{Continue reading the main story}

\hypertarget{ct-vivian-martin-luther-kings-field-general-dies-at-95}{%
\section{C.T. Vivian, Martin Luther King's Field General, Dies at
95}\label{ct-vivian-martin-luther-kings-field-general-dies-at-95}}

A disciplined advocate of nonviolence, he was on the front lines in the
1960s movement for racial justice.

\includegraphics{https://static01.graylady3jvrrxbe.onion/images/2020/07/18/world/17vivian01/17vivian01-articleLarge.jpg?quality=75\&auto=webp\&disable=upscale}

By \href{https://www.nytimes3xbfgragh.onion/by/robert-d-mcfadden}{Robert
D. McFadden}

\begin{itemize}
\item
  Published July 17, 2020Updated July 19, 2020
\item
  \begin{itemize}
  \item
  \item
  \item
  \item
  \item
  \item
  \end{itemize}
\end{itemize}

The Rev. C.T. Vivian, an early civil rights organizer and field general
for the Rev. Dr. Martin Luther King Jr. and his Southern Christian
Leadership Conference in the historic struggle for racial justice a
half-century ago, died on Friday at his home in Atlanta. He was 95.

Two of his daughters, Kira Vivian and Denise Morse, confirmed the death.
Ms. Morse said he had been in hospice care.

In a nation trying to come to grips with racial inequality in the 1960s,
Mr. Vivian was a paladin of nonviolence on the front lines of bloody
confrontations. He led passive protesters through shrieking white mobs
and, with discipline and endurance, absorbed the blows of
segregationists and complicit law enforcement officials across the
South.

Mr. Vivian was a Baptist minister and a member of Dr. King's inner
circle of advisers, alongside the Rev.
\href{https://www.nytimes3xbfgragh.onion/2011/10/06/us/rev-fred-l-shuttlesworth-civil-rights-leader-dies-at-89.html\#:~:text=Shuttlesworth\%2C\%20a\%20storied\%20civil\%20rights,He\%20was\%2089.}{Fred
L. Shuttlesworth}, the Rev.
\href{https://www.nytimes3xbfgragh.onion/2018/01/23/obituaries/wyatt-tee-walker-dead.html}{Wyatt
Tee Walker}, the Rev.
\href{https://www.nytimes3xbfgragh.onion/1990/04/18/obituaries/ralph-david-abernathy-rights-pioneer-is-dead-at-64.html}{Ralph
Abernathy} and other civil rights luminaries. He was the national
director of some 85 local affiliate chapters of the S.C.L.C. from 1963
to 1966, directing protest activities and training in nonviolence as
well as coordinating voter registration and community development
projects.

In Selma and Birmingham, Ala.; St. Augustine, Fla.; Jackson, Miss.; and
other segregated cities, Mr. Vivian led sit-ins at lunch counters,
boycotts of businesses, and marches that continued for weeks or months,
raising tensions that often led to mass arrests and harsh repression.

Televised scenes of marchers attacked by police officers and
firefighters with cattle prods, snarling dogs, fire hoses and
nightsticks shocked the national conscience, legitimized the civil
rights movement and led to passage of the Civil Rights Act of 1964 and
the Voting Rights Act of 1965.

``Nonviolence is the only honorable way of dealing with social change,
because if we are wrong, nobody gets hurt but us,'' Mr. Vivian said in
an address to civil rights workers, as recounted in ``At Canaan's Edge:
America in the King Years, 1965-68'' (2006), by Taylor Branch. ``And if
we are right, more people will participate in determining their own
destinies than ever before.''

\includegraphics{https://static01.graylady3jvrrxbe.onion/images/2020/07/18/obituaries/17vivian-1961/17vivian-1961-articleLarge.jpg?quality=75\&auto=webp\&disable=upscale}

Like his followers, Mr. Vivian was arrested often, jailed and beaten. In
1961, at the end of a violence-plagued interracial Freedom Ride to
Jackson, he was dispatched to the Hinds County Prison Farm, where he was
beaten by guards.

In 1964, Mr. Vivian was nearly killed in St. Augustine, America's oldest
continuously inhabited city and, at the time, one of its most rigidly
segregated, where he had joined Dr. King in an extended campaign of
peaceful protest. On an Atlantic beach, ``roving gangs of whites whipped
Black bathers with chains and almost drowned C.T. Vivian,'' Stephen B.
Oates wrote in ``Let the Trumpet Sound,'' his 1982 biography of Dr.
King.

Accompanying Dr. King on a voter-registration drive in 1965, Mr. Vivian
confronted
\href{https://www.nytimes3xbfgragh.onion/2007/06/07/us/07clark.html}{Sheriff
Jim Clark} outside a courthouse in Selma, where 1,400 Black voters had
been barred from registering. As television cameras rolled, The New York
Times reported, Mr. Vivian asked Sheriff Clark to admit 100 Black people
lined up behind him --- just to get in out of a lightly falling rain.

Image

Mr. Vivian, front row left, at the head of a group of about 3,000
demonstrators taking part in a civil rights march in Nashville in
1960.Credit...Jack Corn/USA TODAY Network

Sheriff Clark refused.

Mr. Vivian, a tall, lanky, angular man, called Sheriff Clark a
``brute,'' ``fascist'' and ``Hitler.''

The 220-pound sheriff struck Mr. Vivian in the mouth with his right
fist, sending him reeling down the courthouse steps. Sheriff Clark then
ordered deputies to arrest him for ``criminal provocation.'' Mr. Vivian
was dragged away, blood streaming down his face.

Sheriff Clark later told reporters that he had no recollection of the
incident. ``Of course the camera might make me out a liar,'' he said.
``I do have a sore finger.''

After leaving Dr. King's staff, Mr. Vivian founded educational and civil
rights organizations, lectured widely, promoted jobs for Black
Chicagoans and wrote ``Black Power and the American Myth'' (1970), an
early assessment of the civil rights movement.

``It was Martin Luther King who removed the Black struggle from the
economic realm and placed it in a moral and spiritual context,'' he
wrote. ``It was on this plane that The Movement first confronted the
conscience of the nation.''

Cordy Tindell Vivian was born in Boonville, Mo., on July 30, 1924, the
only child of Robert and Euzetta Tindell Vivian. His family moved to
Macomb, Ill., when he was 6, and he later graduated from Macomb High
School in 1942. He studied history at Western Illinois University in
Macomb, but he dropped out and became a recreation worker in Peoria,
Ill., where he joined his first protest, in 1947, helping to desegregate
a cafeteria.

In 1945, Mr. Vivian married Jane Teague, who worked at a hardware store,
and they had one daughter, Jo Anna Walker, who survives him. The couple
separated amicably in the late 1940s and divorced later so that Mr.
Vivian could marry Octavia Geans, in 1952. She was the author of
``Coretta'' (1970), the first biography of Dr. King's wife, Coretta
Scott King. She died in 2011.

In addition to his daughters Kira, Denise and Jo Anna, Mr. Vivian is
survived by another daughter, Anita Charisse Thornton; two sons, Mark
Evans Vivian and Albert Louis Vivian; nine grandchildren; 10
great-grandchildren; 28 great-great-grandchildren; and two
great-great-great-grandchildren. Another son, Cordy Jr., died in 2010.

While studying for the ministry at the American Baptist College in
Nashville in 1957, Mr. Vivian joined services at a packed local church
and for the first time heard Dr. King speak on nonviolence.

Image

From left to right: Martin Luther King III; his wife, Andrea; Christine
King Farris; Bernice King; and Mr. Vivian at a private ceremony in
Washington in 2011 to inaugurate a memorial to the Rev. Dr. Martin
Luther King Jr.Credit...Bill O'Leary/The Washington Post, via Getty
Images

Mr. Vivian ``was spellbound,'' Mr. Oates recounted in his biography of
Dr. King. ``He had studied Gandhian techniques, but until now had never
understood the philosophy behind them.''

In 1959, Mr. Vivian met the Rev. James Lawson, who was teaching
nonviolent strategies to members of the Nashville Student Movement. His
students included Diane Nash, Bernard Lafayette, James Bevel and John
Lewis, all of whom became prominent civil rights organizers. (Mr. Lewis
went on to become a powerful voice for social justice in Congress
\href{https://www.nytimes3xbfgragh.onion/2020/07/17/us/john-lewis-dead.html}{and
also died on Friday, the same day as Mr. Vivian}.)

Those students became the nucleus of a successful three-month sit-in
campaign at lunch counters in Nashville in 1960. As 4,000 protesters
marched on City Hall, Mr. Vivian and Ms. Nash confronted Mayor Ben West,
who acknowledged that racial discrimination was morally wrong.

``In less than three weeks, the lunch counters were desegregated,''
David Halberstam, who covered the campaign for The Nashville Tennessean,
\href{https://www.nytimes3xbfgragh.onion/1995/05/01/opinion/nashville-revisited-lunchcounter-days.html}{later
wrote in The Times}.

A year after the Nashville campaign, Mr. Vivian replaced an injured
member of the Congress of Racial Equality on the Freedom Ride to
Mississippi and submitted to his first beating. It was a fearful
experience.

``Going to Mississippi in 1961 was a whole different world,'' he was
quoted as saying in ``King Remembered'' (1986), by Flip Schulke and
Penelope O. McPhee. ``You knew you could easily be killed there.''

After a year as a pastor in Chattanooga, Mr. Vivian helped organize
Tennessee's contingent for the 1963 March on Washington and was invited
to join Dr. King's staff.

His civil rights work continued for a half century. He became director
of the Urban Training Center for Christian Mission in Chicago in 1966
and dean of the Shaw University Divinity School in Raleigh, N.C., in
1972.

He later founded the Black Action Strategies and Information Center in
Atlanta to foster workplace race relations, and was a founder of the
National Anti-Klan Network, which monitored hate groups. It was later
renamed the Center for Democratic Renewal to reflect broader educational
goals.

Mr. Vivian was deputy director for clergy in the 1984 presidential
campaign of the Rev. Jesse Jackson; appeared on
``\href{https://americanarchive.org/catalog/cpb-aacip_151-z892806203}{Eyes
on the Prize}'' (1987), a 14-part PBS documentary on the civil rights
era; and was later the focus of a PBS special, ``The Healing Ministry of
Dr. C.T. Vivian.''

He received the Presidential Medal of Freedom, the nation's highest
civilian honor, from President Barack Obama in 2013.

Image

President Barack Obama awarded the Presidential Medal of Freedom to Mr.
Vivian in 2013.Credit...Bill O'Leary/The Washington Post, via Getty
Images

Alex Traub and Julia Carmel contributed reporting.

Advertisement

\protect\hyperlink{after-bottom}{Continue reading the main story}

\hypertarget{site-index}{%
\subsection{Site Index}\label{site-index}}

\hypertarget{site-information-navigation}{%
\subsection{Site Information
Navigation}\label{site-information-navigation}}

\begin{itemize}
\tightlist
\item
  \href{https://help.nytimes3xbfgragh.onion/hc/en-us/articles/115014792127-Copyright-notice}{©~2020~The
  New York Times Company}
\end{itemize}

\begin{itemize}
\tightlist
\item
  \href{https://www.nytco.com/}{NYTCo}
\item
  \href{https://help.nytimes3xbfgragh.onion/hc/en-us/articles/115015385887-Contact-Us}{Contact
  Us}
\item
  \href{https://www.nytco.com/careers/}{Work with us}
\item
  \href{https://nytmediakit.com/}{Advertise}
\item
  \href{http://www.tbrandstudio.com/}{T Brand Studio}
\item
  \href{https://www.nytimes3xbfgragh.onion/privacy/cookie-policy\#how-do-i-manage-trackers}{Your
  Ad Choices}
\item
  \href{https://www.nytimes3xbfgragh.onion/privacy}{Privacy}
\item
  \href{https://help.nytimes3xbfgragh.onion/hc/en-us/articles/115014893428-Terms-of-service}{Terms
  of Service}
\item
  \href{https://help.nytimes3xbfgragh.onion/hc/en-us/articles/115014893968-Terms-of-sale}{Terms
  of Sale}
\item
  \href{https://spiderbites.nytimes3xbfgragh.onion}{Site Map}
\item
  \href{https://help.nytimes3xbfgragh.onion/hc/en-us}{Help}
\item
  \href{https://www.nytimes3xbfgragh.onion/subscription?campaignId=37WXW}{Subscriptions}
\end{itemize}
