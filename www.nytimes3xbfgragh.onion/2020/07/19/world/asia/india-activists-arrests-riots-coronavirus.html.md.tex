Sections

SEARCH

\protect\hyperlink{site-content}{Skip to
content}\protect\hyperlink{site-index}{Skip to site index}

\href{https://www.nytimes3xbfgragh.onion/section/world/asia}{Asia
Pacific}

\href{https://myaccount.nytimes3xbfgragh.onion/auth/login?response_type=cookie\&client_id=vi}{}

\href{https://www.nytimes3xbfgragh.onion/section/todayspaper}{Today's
Paper}

\href{/section/world/asia}{Asia Pacific}\textbar{}India Rounds Up
Critics Under Shadow of Virus Crisis, Activists Say

\url{https://nyti.ms/2OFMi5f}

\begin{itemize}
\item
\item
\item
\item
\item
\end{itemize}

Advertisement

\protect\hyperlink{after-top}{Continue reading the main story}

Supported by

\protect\hyperlink{after-sponsor}{Continue reading the main story}

\hypertarget{india-rounds-up-critics-under-shadow-of-virus-crisis-activists-say}{%
\section{India Rounds Up Critics Under Shadow of Virus Crisis, Activists
Say}\label{india-rounds-up-critics-under-shadow-of-virus-crisis-activists-say}}

The Indian authorities arrested dozens of people during a nationwide
lockdown. Human rights groups say many of the detentions are based on
scant evidence.

\includegraphics{https://static01.graylady3jvrrxbe.onion/images/2020/07/09/world/00india-arrests4/merlin_172177905_0f4bed11-ee2e-45ea-9503-4a7100c13dca-articleLarge.jpg?quality=75\&auto=webp\&disable=upscale}

\href{https://www.nytimes3xbfgragh.onion/by/sameer-yasir}{\includegraphics{https://static01.graylady3jvrrxbe.onion/images/2019/11/22/reader-center/author-sameer-yasir/author-sameer-yasir-thumbLarge.png}}\href{https://www.nytimes3xbfgragh.onion/by/kai-schultz}{\includegraphics{https://static01.graylady3jvrrxbe.onion/images/2019/11/22/reader-center/author-kai-schultz/author-kai-schultz-thumbLarge.png}}

By \href{https://www.nytimes3xbfgragh.onion/by/sameer-yasir}{Sameer
Yasir} and \href{https://www.nytimes3xbfgragh.onion/by/kai-schultz}{Kai
Schultz}

\begin{itemize}
\item
  July 19, 2020
\item
  \begin{itemize}
  \item
  \item
  \item
  \item
  \item
  \end{itemize}
\end{itemize}

NEW DELHI --- After spending several anxious days in prison, Natasha
Narwal, a student activist accused of rioting by the New Delhi police,
thought her ordeal was nearing an end.

A judge ruled that Ms. Narwal had been exercising her democratic rights
when she participated in protests earlier this year against a divisive
citizenship law that
\href{https://www.nytimes3xbfgragh.onion/2019/12/22/world/asia/modi-india-citizenship-law.html}{incited
unrest} across India.

But shortly after the judge approved Ms. Narwal's release in late May,
the police
\href{https://scroll.in/latest/963353/delhi-violence-pinjra-tod-member-natasha-narwal-booked-under-uapa}{announced
fresh charges}: murder, terrorism and organizing protests that
instigated
\href{https://www.nytimes3xbfgragh.onion/2020/02/25/world/asia/new-delhi-hindu-muslim-violence.html}{deadly
religious violence} in India's capital. Ms. Narwal, 32, who has said
that she is innocent, was returned to her cell.

``I felt like crying,'' said her roommate, Vikramaditya Sahai. ``We are
grieving the country we grew up in.''

Image

Natasha Narwal

As India struggles to quell surging coronavirus infections, lawyers
accuse the authorities of rounding up government critics and keeping
them in detention in the middle of a pandemic. It is part of a strategy,
they say, to stifle activists who are protesting what they see as
iron-fisted and anti-minority policies under Prime Minister Narendra
Modi.

In recent weeks, Ms. Narwal and
\href{https://www.ohchr.org/EN/NewsEvents/Pages/DisplayNews.aspx?NewsID=26002\&LangID=E}{nearly
a dozen other prominent activists} --- along with potentially dozens of
other demonstrators, though police records are unclear --- have been
detained. They are being held under stringent sedition and antiterrorism
laws that have been used to criminalize everything from leading rallies
to posting political messages on social media.

India's coronavirus restrictions, some of which are still in effect,
\href{https://www.hrw.org/news/2020/06/15/india-end-bias-prosecuting-delhi-violence}{have
blocked pathways to justice}, lawyers and rights activists say. With
courts closed for weeks, lawyers have struggled to file bail
applications, and meeting privately with prisoners has been nearly
impossible.

Law enforcement officials in New Delhi, who are under the direct control
of India's home ministry, have denied any impropriety. But rights groups
say the arrests have been
\href{https://www.fidh.org/en/issues/human-rights-defenders/india-arbitrary-detention-of-several-defenders-for-protesting-against}{arbitrary},
based on scant evidence and in line with a broader deterioration of free
speech in India.

In a
\href{http://muslimmirror.com/eng/kapil-mishras-provocative-speech-responsible-for-delhi-violence-says-minorities-commission-report/}{lengthy
report} released this month, the Delhi Minorities Commission, a
government body, accused the police and politicians from Mr. Modi's
party of inciting brutal attacks on protesters and supporting a
``pogrom'' against minority Muslims.

\includegraphics{https://static01.graylady3jvrrxbe.onion/images/2020/07/09/world/00india-arrests6/merlin_168094227_6c1756e3-e35a-4af6-8ef1-7932cad960d4-articleLarge.jpg?quality=75\&auto=webp\&disable=upscale}

Meenakshi Ganguly, the South Asia director of Human Rights Watch, said
cases against the activists appeared to be ``politically motivated,''
and that the police had devised a formula for keeping people like Ms.
Narwal in jail: When a judge orders the release of a prisoner for lack
of evidence, new charges are introduced.

``The urgency to arrest rights activists and an obvious reluctance to
act against violent actions of the government's supporters show a
complete breakdown in the rule of law,'' she said.

Before the pandemic hit, Mr. Modi was in the throes of the most
significant challenge to his power since becoming prime minister in
2014. After Parliament
\href{https://www.nytimes3xbfgragh.onion/2019/12/11/world/asia/india-muslims-citizenship-narendra-modi.html}{passed
a law} last year that made it easier for non-Muslim migrants to become
Indian citizens,
\href{https://www.nytimes3xbfgragh.onion/2019/12/16/world/asia/india-citizenship-protests.html}{millions
protested across the country}.

To critics, the citizenship law was more evidence that Mr. Modi's Hindu
nationalist government planned to strip the country's Muslims of their
rights.

Tensions peaked in February when
\href{https://www.nytimes3xbfgragh.onion/2020/02/25/world/asia/new-delhi-hindu-muslim-violence.html}{sectarian
violence and rioting} broke out in New Delhi. The vast majority of
people killed, hurt or displaced were Muslim, and the police were
involved in many of those cases.

After Mr. Modi announced a nationwide lockdown in late March to contain
the coronavirus, shutting down businesses and ordering all 1.3 billion
Indians inside, the protests disbanded. Lawyers said the police then
moved to detain demonstrators while skirting complaints against
government allies.

Image

Bodies being taken to a mortuary in southern New Delhi after riots in
February.Credit...Atul Loke for The New York Times

Among those in custody are a youth activist who raised awareness about
police brutality against Muslims; an academic who gave a speech opposing
the citizenship law; and Ms. Narwal, a graduate student who co-founded
Pinjra Tod, or Break the Cage, a women's collective that organized some
of the largest rallies.

Nitika Khaitan, a criminal lawyer, said the crackdown had also pushed
beyond higher-profile critics to include ordinary residents of riot-hit
neighborhoods. She recently challenged those arrests in a jointly signed
letter to the Delhi High Court.

Lawyers have tracked a few dozen such arrests under the lockdown, though
Ms. Khaitan said the true figure could not be verified because police
reports had not been made public. Many detentions were ``not in
compliance with constitutional mandates,'' she said.

In a recent interview, Sachidanand Shrivastava, the police chief in New
Delhi, said his officers were conducting fair investigations.

In May, the authorities said they had detained
\href{https://www.thehindu.com/news/cities/Delhi/1300-persons-arrested-for-north-east-delhi-riots-police/article31604364.ece}{about
1,300 people} for involvement in the protests and riots, including an
equal number of Hindus and Muslims. Recently, the police
\href{https://www.thehindu.com/news/cities/Delhi/nine-of-those-killed-in-northeast-delhi-riots-were-forced-to-shout-jai-shri-ram/article31973715.ece}{arrested
a group of Hindus} for forcing nine Muslim men to chant ``Hail Lord
Ram,'' a reference to a Hindu god, before killing them and throwing
their bodies into a drain.

``It is very important that the police force remain impartial,'' Mr.
Shrivastava said. ``And we are following this principle from Day 1.''

But members of India's judiciary have questioned the official numbers,
accusing the police of withholding information about the arrests under
national security protections and singling out Muslims for many of the
harsher charges.

In court proceeding notes reviewed by The Times, a judge hearing a case
against a Muslim protester
\href{https://www.outlookindia.com/newsscroll/delhi-violence-investigation-in-case-seems-to-be-targeted-towards-one-end-says-court/1848821}{wrote}
that the police appeared to be targeting only ``one end'' without
probing the ``rival faction.'' During the riots, the police were accused
of
\href{https://www.nytimes3xbfgragh.onion/2020/03/12/world/asia/india-police-muslims.html}{abetting
Hindus and, in some cases, torturing Muslims}.

Image

Security personnel in northeastern Delhi after riots in February. The
authorities said they had arrested about 1,300 people in May for
involvement in the riots.Credit...Atul Loke for The New York Times

Khalid Saifi, a member of United Against Hate, a group that works with
victims of hate crimes, was arrested after he tried to mediate between
the police and protesters, according to his lawyers.

The police charged him with being a ``key conspirator'' of the riots.
His wife, Nargis Saifi, said he was tortured in custody.

``His only crime is he is a Muslim,'' she said.

M.S. Randhawa, a police spokesman, denied that Mr. Saifi had been
tortured, adding that he has regular opportunities to speak to a judge
if abuse occurs.

``These are just allegations,'' Mr. Randhawa said. ``He would have told
the magistrate if he had been tortured.''

But rights advocates accuse Mr. Modi's government of shielding party
officials --- and more broadly, of Hindus involved in the violence.

Ms. Narwal, who was detained in May, could face at least several years
in prison for helping organize demonstrations that blocked a busy road
in northeast Delhi, where February's bloodiest battles between Hindus
and Muslims broke out.

The police have accused her of
\href{https://thewire.in/rights/pinjra-tods-devangana-kalita-natasha-narwal-denied-bail-again-in-delhi-riots-case}{playing
a leading role} in the riots,
\href{https://www.livelaw.in/pdf_upload/pdf_upload-376649.pdf}{charging
her} with murder, attempt to murder and being part of a ``criminal
conspiracy.''

At the same time, the police have been accused of ignoring complaints
against Kapil Mishra, a local politician with Mr. Modi's Bharatiya
Janata Party who gave
\href{https://www.nytimes3xbfgragh.onion/2020/02/26/world/asia/delhi-riots-kapil-mishra.html}{a
fiery speech} threatening to remove Ms. Narwal and other protesters
forcibly if the authorities did not take action.

Image

Kapil Mishra addressing a crowd in New Delhi last year.Credit...Sonu
Mehta/Hindustan Times

Hours after the ultimatum, the streets erupted. But charges were never
filed against Mr. Mishra, who has denied a role in starting the riots.

A New Delhi police superintendent, who spoke on the condition of
anonymity, said that some officers had wanted to act against Mr. Mishra,
but that they were pressured by the force's leadership not to touch
``the warriors of the government.''

``We did not even try,'' the superintendent said. ``The directions were
clear: Don't lay your hands on him.''

Through an intermediary, Mr. Mishra declined to comment.

Ms. Narwal's father, Mahavir Narwal, said the government was moving
India closer to authoritarianism and demonizing anybody who questioned
its policies.

For weeks, prison officials ignored his calls and emails to Tihar Jail,
where Ms. Narwal is being held. With coronavirus restrictions in place,
she was moved into an isolation ward at one point, where she stayed for
17 days, said Mr. Narwal, a retired scientist.

Lately, communication has smoothed out. But Mr. Narwal said the subtext
of his daughter's arrest seemed clear: ``If you protest, you will be
called a terrorist.''

``All she did was fight to keep the soul of India alive,'' he said.

Karan Deep Singh contributed reporting.

Advertisement

\protect\hyperlink{after-bottom}{Continue reading the main story}

\hypertarget{site-index}{%
\subsection{Site Index}\label{site-index}}

\hypertarget{site-information-navigation}{%
\subsection{Site Information
Navigation}\label{site-information-navigation}}

\begin{itemize}
\tightlist
\item
  \href{https://help.nytimes3xbfgragh.onion/hc/en-us/articles/115014792127-Copyright-notice}{©~2020~The
  New York Times Company}
\end{itemize}

\begin{itemize}
\tightlist
\item
  \href{https://www.nytco.com/}{NYTCo}
\item
  \href{https://help.nytimes3xbfgragh.onion/hc/en-us/articles/115015385887-Contact-Us}{Contact
  Us}
\item
  \href{https://www.nytco.com/careers/}{Work with us}
\item
  \href{https://nytmediakit.com/}{Advertise}
\item
  \href{http://www.tbrandstudio.com/}{T Brand Studio}
\item
  \href{https://www.nytimes3xbfgragh.onion/privacy/cookie-policy\#how-do-i-manage-trackers}{Your
  Ad Choices}
\item
  \href{https://www.nytimes3xbfgragh.onion/privacy}{Privacy}
\item
  \href{https://help.nytimes3xbfgragh.onion/hc/en-us/articles/115014893428-Terms-of-service}{Terms
  of Service}
\item
  \href{https://help.nytimes3xbfgragh.onion/hc/en-us/articles/115014893968-Terms-of-sale}{Terms
  of Sale}
\item
  \href{https://spiderbites.nytimes3xbfgragh.onion}{Site Map}
\item
  \href{https://help.nytimes3xbfgragh.onion/hc/en-us}{Help}
\item
  \href{https://www.nytimes3xbfgragh.onion/subscription?campaignId=37WXW}{Subscriptions}
\end{itemize}
