Sections

SEARCH

\protect\hyperlink{site-content}{Skip to
content}\protect\hyperlink{site-index}{Skip to site index}

\href{https://www.nytimes3xbfgragh.onion/section/books/review}{Book
Review}

\href{https://myaccount.nytimes3xbfgragh.onion/auth/login?response_type=cookie\&client_id=vi}{}

\href{https://www.nytimes3xbfgragh.onion/section/todayspaper}{Today's
Paper}

\href{/section/books/review}{Book Review}\textbar{}Remembering the Time
Meat Fell From the Sky

\url{https://nyti.ms/39efcTo}

\begin{itemize}
\item
\item
\item
\item
\item
\end{itemize}

Advertisement

\protect\hyperlink{after-top}{Continue reading the main story}

Supported by

\protect\hyperlink{after-sponsor}{Continue reading the main story}

nonfiction

\hypertarget{remembering-the-time-meat-fell-from-the-sky}{%
\section{Remembering the Time Meat Fell From the
Sky}\label{remembering-the-time-meat-fell-from-the-sky}}

\includegraphics{https://static01.graylady3jvrrxbe.onion/images/2020/07/26/books/review/26Blum/26Blum-articleLarge.jpg?quality=75\&auto=webp\&disable=upscale}

Buy Book ▾

\begin{itemize}
\tightlist
\item
  \href{https://www.amazon.com/gp/search?index=books\&tag=NYTBSREV-20\&field-keywords=The+Unidentified+Colin+Dickey}{Amazon}
\item
  \href{https://du-gae-books-dot-nyt-du-prd.appspot.com/buy?title=The+Unidentified\&author=Colin+Dickey}{Apple
  Books}
\item
  \href{https://www.anrdoezrs.net/click-7990613-11819508?url=https\%3A\%2F\%2Fwww.barnesandnoble.com\%2Fw\%2F\%3Fean\%3D9780525557562}{Barnes
  and Noble}
\item
  \href{https://www.anrdoezrs.net/click-7990613-35140?url=https\%3A\%2F\%2Fwww.booksamillion.com\%2Fp\%2FThe\%2BUnidentified\%2FColin\%2BDickey\%2F9780525557562}{Books-A-Million}
\item
  \href{https://bookshop.org/a/3546/9780525557562}{Bookshop}
\item
  \href{https://www.indiebound.org/book/9780525557562?aff=NYT}{Indiebound}
\end{itemize}

When you purchase an independently reviewed book through our site, we
earn an affiliate commission.

By Deborah Blum

\begin{itemize}
\item
  Published July 21, 2020Updated July 22, 2020
\item
  \begin{itemize}
  \item
  \item
  \item
  \item
  \item
  \end{itemize}
\end{itemize}

\textbf{THE UNIDENTIFIED}\\
\textbf{Mythical Monsters, Alien Encounters, and Our Obsession With the
Unexplained}\\
By Colin Dickey

On a spring day in 1876 --- as reported by national newspapers in lurid
detail --- a Kentucky farm wife was caught in a sudden shower of meat,
gristly shreds tumbling from the clear sky above. The ``carnal rain,''
as The New York Herald called it, spattered across the yard for several
minutes, prompting the woman, who had been making soap, to abandon the
project and flee into the house.

The
\href{https://blogs.scientificamerican.com/running-ponies/the-great-kentucky-meat-shower-mystery-unwound-by-projectile-vulture-vomit/}{Kentucky
Meat Shower} (suspected to be vulture vomit) might have stayed a passing
oddity. But a conspiracy-minded American writer, Charles Fort,
highlighted it in
\href{https://www.sacred-texts.com/fort/damn/index.htm}{his 1919
publication, ``The Book of the Damned,''} along with accounts of frogs
falling from clouds in Ireland, rancid butter tumbling from the sky in
Missouri and fish dropped by storms in India. More than 100 years later,
Fort's screed remains a
\href{https://www.penguinrandomhouse.com/books/533872/the-book-of-the-damned-by-charles-fort/}{still-in-print}
cult classic, one that \href{http://www.colindickey.com/books}{Colin
Dickey} argues helped solidify the idea that inexplicable mysteries of
the world are as important as the tangible evidence of science --- and
maybe more so.

As becomes clear in his book, ``The Unidentified,'' Dickey is not a Fort
enthusiast. He describes Fort as a crank, ``a rogue historian of the
early 20th century'' and as a depressingly effective cultivator of
paranoid conspiracy theories. The phrase ``the damned'' in Fort's book
is not a reference to cursed denizens of the underworld. It refers to
events and experiences --- like meat showers --- that science has damned
or excluded as untrustworthy. Fort, in fact, bitterly referred to
members of the scientific establishment as ``the exclusionists.''

\emph{{[} Read an excerpt from}
\href{https://www.nytimes3xbfgragh.onion/2020/07/22/books/review/the-unidentified-by-colin-dickey-an-excerpt.html}{\emph{``The
Unidentified.''}} \emph{{]}}

But in Dickey's fascinating, troubling, compassionate and --- in the end
--- deeply thoughtful narrative, he also makes the case for why people
like Fort wield so much influence. Dickey has explored occult territory
before, in books like ``Ghostland'' (2016), but this time he sets
himself the goal of trying to explain why so many find paranormal events
and ideas so persuasively real. Surveys show that, between 2015 and
2018, belief in Bigfoot grew from 11 percent of the American population
to 21 percent, and acceptance of alien visitations rose from 20 percent
to 41 percent. ``We are more and more ignoring `experts' and embracing
the kinds of beliefs that were once relegated to cults,'' Dickey writes.

To understand why, he carefully traces lines of influence between early
flame-fanners like Fort and the earnest believers of today. Dickey's
story emphasizes the potency of the 19th century, a time when lovers of
myth and mystery, alienated by the rise of Darwinian science, proposed
possibilities of vanished civilizations like Atlantis or lost landscapes
of large aliens and small lemurs, like
\href{https://blogs.scientificamerican.com/history-of-geology/a-geologists-dream-the-lost-continent-of-lemuria/}{Lemuria.}
Many of us may not have considered, or even heard of, Lemurian
possibilities. But Dickey details the history of the idea, following it
from a 19th-century glimmer of a thought to its still faithful followers
in Northern California, who hold tight to the belief that Lemurians
shelter in caverns under Mount Shasta.

This is natural territory, of course, for curiosities. I also had no
idea that, until the 1860s, reputable scientists considered pandas a
ridiculous myth. Or that the American government was hunting Yeti
through the Himalayas in the 1950s, spooking the Soviets into
accusations that our monster hunt was all about espionage. Or even that
New York's old, acclaimed, alternative weekly, The Village Voice, once
fostered the idea of alien abductions.

Dickey uses such incidents not merely to tell good campfire stories but
to illustrate their shared darker themes --- a deep distrust of science
and government, amplified both by self-promoters and by conspiracy
lovers. And he notes that scientific arrogance and excessive government
secrecy have fueled these fires. The military's heavy-handed
classification of U.F.O. information, for instance, was a treasured gift
to those weaving tales of federal cover-ups and hidden spacecraft.

There's nothing startlingly new or transformative in these conclusions.
But Dickey's sense of history reminds us of the complex reasons our
odder beliefs endure. It's not that we necessarily want weirdness, he
suggests, but we do want wonder, we want the freedom of possibility. So
there's beauty and even comfort in the idea of ``a world beyond our
understanding, a world we can glimpse here and there but never fully
see.''

Advertisement

\protect\hyperlink{after-bottom}{Continue reading the main story}

\hypertarget{site-index}{%
\subsection{Site Index}\label{site-index}}

\hypertarget{site-information-navigation}{%
\subsection{Site Information
Navigation}\label{site-information-navigation}}

\begin{itemize}
\tightlist
\item
  \href{https://help.nytimes3xbfgragh.onion/hc/en-us/articles/115014792127-Copyright-notice}{©~2020~The
  New York Times Company}
\end{itemize}

\begin{itemize}
\tightlist
\item
  \href{https://www.nytco.com/}{NYTCo}
\item
  \href{https://help.nytimes3xbfgragh.onion/hc/en-us/articles/115015385887-Contact-Us}{Contact
  Us}
\item
  \href{https://www.nytco.com/careers/}{Work with us}
\item
  \href{https://nytmediakit.com/}{Advertise}
\item
  \href{http://www.tbrandstudio.com/}{T Brand Studio}
\item
  \href{https://www.nytimes3xbfgragh.onion/privacy/cookie-policy\#how-do-i-manage-trackers}{Your
  Ad Choices}
\item
  \href{https://www.nytimes3xbfgragh.onion/privacy}{Privacy}
\item
  \href{https://help.nytimes3xbfgragh.onion/hc/en-us/articles/115014893428-Terms-of-service}{Terms
  of Service}
\item
  \href{https://help.nytimes3xbfgragh.onion/hc/en-us/articles/115014893968-Terms-of-sale}{Terms
  of Sale}
\item
  \href{https://spiderbites.nytimes3xbfgragh.onion}{Site Map}
\item
  \href{https://help.nytimes3xbfgragh.onion/hc/en-us}{Help}
\item
  \href{https://www.nytimes3xbfgragh.onion/subscription?campaignId=37WXW}{Subscriptions}
\end{itemize}
