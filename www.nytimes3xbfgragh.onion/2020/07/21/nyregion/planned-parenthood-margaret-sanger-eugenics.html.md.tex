Sections

SEARCH

\protect\hyperlink{site-content}{Skip to
content}\protect\hyperlink{site-index}{Skip to site index}

\href{https://www.nytimes3xbfgragh.onion/section/nyregion}{New York}

\href{https://myaccount.nytimes3xbfgragh.onion/auth/login?response_type=cookie\&client_id=vi}{}

\href{https://www.nytimes3xbfgragh.onion/section/todayspaper}{Today's
Paper}

\href{/section/nyregion}{New York}\textbar{}Planned Parenthood in N.Y.
Disavows Margaret Sanger Over Eugenics

\url{https://nyti.ms/3hpxPGG}

\begin{itemize}
\item
\item
\item
\item
\item
\item
\end{itemize}

\hypertarget{race-and-america}{%
\subsubsection{\texorpdfstring{\href{https://www.nytimes3xbfgragh.onion/news-event/george-floyd-protests-minneapolis-new-york-los-angeles?name=styln-george-floyd\&region=TOP_BANNER\&block=storyline_menu_recirc\&action=click\&pgtype=Article\&impression_id=2947c640-f295-11ea-84d9-09ec1c57291a\&variant=undefined}{Race
and America}}{Race and America}}\label{race-and-america}}

\begin{itemize}
\tightlist
\item
  \href{https://www.nytimes3xbfgragh.onion/2020/09/04/nyregion/rochester-police-daniel-prude.html?name=styln-george-floyd\&region=TOP_BANNER\&block=storyline_menu_recirc\&action=click\&pgtype=Article\&impression_id=2947c641-f295-11ea-84d9-09ec1c57291a\&variant=undefined}{What
  Happened in Rochester, N.Y.}
\item
  \href{https://www.nytimes3xbfgragh.onion/2020/09/01/us/politics/trump-fact-check-protests.html?name=styln-george-floyd\&region=TOP_BANNER\&block=storyline_menu_recirc\&action=click\&pgtype=Article\&impression_id=2947c642-f295-11ea-84d9-09ec1c57291a\&variant=undefined}{Trump
  Fact Check}
\item
  \href{https://www.nytimes3xbfgragh.onion/2020/08/30/us/portland-shooting-explained.html?name=styln-george-floyd\&region=TOP_BANNER\&block=storyline_menu_recirc\&action=click\&pgtype=Article\&impression_id=2947c643-f295-11ea-84d9-09ec1c57291a\&variant=undefined}{Portland
  Shooting}
\item
  \href{https://www.nytimes3xbfgragh.onion/2020/08/30/us/breonna-taylor-police-killing.html?name=styln-george-floyd\&region=TOP_BANNER\&block=storyline_menu_recirc\&action=click\&pgtype=Article\&impression_id=2947c644-f295-11ea-84d9-09ec1c57291a\&variant=undefined}{Breonna
  Taylor's Life and Death}
\end{itemize}

Advertisement

\protect\hyperlink{after-top}{Continue reading the main story}

Supported by

\protect\hyperlink{after-sponsor}{Continue reading the main story}

\hypertarget{planned-parenthood-in-ny-disavows-margaret-sanger-over-eugenics}{%
\section{Planned Parenthood in N.Y. Disavows Margaret Sanger Over
Eugenics}\label{planned-parenthood-in-ny-disavows-margaret-sanger-over-eugenics}}

Ms. Sanger, a feminist icon and reproductive-rights pioneer, supported a
discredited belief in improving the human race through selective
breeding.

\includegraphics{https://static01.graylady3jvrrxbe.onion/images/2020/07/21/nyregion/21nysanger03/merlin_174799098_2c9e5db9-7c9e-40f2-a484-ff97459cf548-articleLarge.jpg?quality=75\&auto=webp\&disable=upscale}

\href{https://www.nytimes3xbfgragh.onion/by/nikita-stewart}{\includegraphics{https://static01.graylady3jvrrxbe.onion/images/2018/09/25/multimedia/author-nikita-stewart/author-nikita-stewart-thumbLarge-v2.png}}

By \href{https://www.nytimes3xbfgragh.onion/by/nikita-stewart}{Nikita
Stewart}

\begin{itemize}
\item
  July 21, 2020
\item
  \begin{itemize}
  \item
  \item
  \item
  \item
  \item
  \item
  \end{itemize}
\end{itemize}

Planned Parenthood of Greater New York will remove the name of Margaret
Sanger, a founder of the national organization, from its Manhattan
health clinic because of her ``harmful connections to the eugenics
movement,'' the group said on Tuesday.

Ms. Sanger, a public health nurse who opened the first birth control
clinic in the United States in Brooklyn in 1916, has long been lauded as
a feminist icon and reproductive-rights pioneer.

But her legacy also includes supporting eugenics, a discredited belief
in improving the human race through selective breeding, often targeted
at poor people, those with disabilities, immigrants and people of color.

``The removal of Margaret Sanger's name from our building is both a
necessary and overdue step to reckon with our legacy and acknowledge
Planned Parenthood's contributions to historical reproductive harm
within communities of color,'' Karen Seltzer, the chair of the New York
affiliate's board, said in a statement.

The group is also talking to city leaders about replacing Ms. Sanger's
name on a street sign that has hung near its offices on Bleecker Street
for more than two decades.

The actions thrust Ms. Sanger onto a growing list of historical figures
whose legacies are being re-evaluated amid both widespread protests
against systemic racism and a pandemic that has exposed racial and
economic inequalities in health care services.

\href{https://www.nytimes3xbfgragh.onion/2020/06/27/nyregion/princeton-university-woodrow-wilson.html}{Princeton
University} said last month that it would remove President Woodrow
Wilson's name from its public policy school and a residential college
because of his segregationist views. Just four years ago, Princeton
trustees voted against such a move.

\includegraphics{https://static01.graylady3jvrrxbe.onion/images/2020/07/21/nyregion/21nysanger01/merlin_148150425_93c17bc4-7f5c-42bc-90a7-28dac5f91ae7-articleLarge.jpg?quality=75\&auto=webp\&disable=upscale}

\href{https://www.plannedparenthood.org/}{Planned Parenthood Federation
of America}, the national organization, has defended Ms. Sanger in the
past, citing her work with Black leaders in the 1930s and 1940s. As
recently as 2016, the
group\href{https://www.plannedparenthood.org/uploads/filer_public/37/fd/37fdc7b6-de5f-4d22-8c05-9568268e92d8/sanger_opposition_claims_fact_sheet_2016.pdf}{issued
a fact sheet} saying that while it condemned some of her beliefs, she
had mostly been well intentioned in trying to make birth control
accessible for poor and immigrant communities.

The national organization said in the fact sheet that it disagreed with
Ms. Sanger's decision to speak to members of the Ku Klux Klan in 1926 as
she tried to spread her message about birth control.

It also condemned her support for policies to sterilize people who had
disabilities that could not be treated; for banning immigrants with
disabilities; and for ``placing so-called illiterates, paupers,
unemployables, criminals, prostitutes, and dope fiends on farms and in
open spaces as long as necessary for the strengthening and development
of moral conduct.''

In a statement, the national organization said it supported the New York
chapter's decision to strike Ms. Sanger's name from the clinic. There is
no sign on the facility, but it had been identified both internally and
publicly by Ms. Sanger's name. It will now be known as the Manhattan
Health Center.

``Planned Parenthood, like many other organizations that have existed
for a century or more, is reckoning with our history, and working to
address historical inequities to better serve patients and our
mission,'' Melanie Roussell Newman, a spokeswoman for the group, said in
the statement.

Ms. Sanger still has defenders who say the decision to repudiate her
lacks historical nuance.

Ellen Chesler, a senior fellow at the Roosevelt Institute, a think tank,
and the author of
\href{https://www.amazon.com/Woman-Valor-Margaret-Control-Movement/dp/1416540768}{a
biography of Ms. Sanger} and the birth control movement, said that while
the country is undergoing vast social change and reconsidering prominent
figures from the past, Ms. Sanger's views have been misinterpreted.

The eugenics movement had wide support at the time in both conservative
and liberal circles, Ms. Chesler said, and Ms. Sanger was squarely in
the latter camp. She rejected some eugenicists' belief that white
middle-class families should have more children than others, Ms. Chesler
said.

Image

Outside the birth control clinic in Brooklyn's Brownsville section that
Margaret Sanger started with her sister, Ethel Byrne, and a colleague,
Fania Mindell, in 1916.Credit...Social Press Association via Library of
Congress, via Associated Press

Instead, Ms. Sanger believed that the quality of all children's lives
could be improved if their parents had smaller families, Ms. Chesler
said, adding that Ms. Sanger believed Black people and immigrants had a
right to that better life.

``Her motives were the opposite of racism,'' Ms. Chesler said, citing
Ms. Sanger's relationships with prominent Black leaders like W.E.B. Du
Bois, a founder of the N.A.A.C.P.

Planned Parenthood of Greater New York, which was formed when five area
chapters merged in January, is now the national group's largest
affiliate and wants to recognize the Black women and others who also
championed the reproductive justice movement, said Merle McGee, the New
York chapter's chief equity and engagement officer.

The decision to drop Ms. Sanger's name from the clinic creates an
unusual alignment between the New York group and anti-abortion
conservatives like
\href{https://www.cruz.senate.gov/?p=press_release\&id=2476}{Senator Ted
Cruz, Republican of Texas,}and Ben Carson, the federal housing
secretary, who have pointed to Ms. Sanger's belief in eugenics to
criticize Planned Parenthood and its mission.

Ms. McGee said the group could not worry about how conservative figures
might react to the move, which she described as a response to the
concerns of patients and people of color.

``We're not going to obliterate her,'' Ms. McGee said. ``If we
obliterate her, we cannot reckon with her.''

The New York affiliate's effort to disavow Ms. Sanger comes as it
wrestles with internal turmoil, including
\href{https://www.nytimes3xbfgragh.onion/2020/06/23/nyregion/ny-planned-parenthood-laura-mcquade.html}{the
recent ouster of its executive director, Laura McQuade}, in part because
of complaints that she had mistreated Black employees.

Ms. McGee said there was no connection between Ms. McQuade's departure
and the decision to remove Ms. Sanger's name. The move, she said, arose
out of a three-year effort to tackle racism internally and to improve
relationships with groups led by Black women who have been wary of
Planned Parenthood's origins. Of the New York affiliate's 22 board
members, one is Black, two are Asian and two are Hispanic.

``The biggest concern with Margaret Sanger is her public support for the
eugenics medical philosophy which was rooted in racism, ableism and
classism,'' Ms. McGee said.

Ms. Sanger based her movement in New York and has been honored in
several places around the city. Her former clinic, on West 16th Street,
was designated a National Historic Landmark in 1993. The same year, the
City Council voted to name the corner of Mott and Bleecker Streets
Margaret Sanger Square. Planned Parenthood of Greater New York helped
push for the designation after moving its offices there.

The group's announcement on Tuesday focused on its current building and
the street sign.

A representative of the group, recounting an often repeated but
uncorroborated story, told city leaders in 1993 that it was fitting to
honor Ms. Sanger in the area given that she had helped start the birth
control movement nearby on the Lower East Side.

As the story goes, Ms. Sanger treated a woman named ``Sadie Sachs,'' who
had given herself an abortion. Sadie asked a doctor how she could avoid
having another baby, and the doctor recommended abstinence. A few months
later, Ms. Sanger was called to treat Sadie again after she had given
herself another abortion, and she died in Ms. Sanger's arms.

Ms. Sanger went on to start clinics, including one in Harlem. She pushed
for reproductive rights, even after she was arrested and sent to jail
for opening her first clinic, in the Brownsville section of Brooklyn.

Susan C. Beachy contributed research.

Advertisement

\protect\hyperlink{after-bottom}{Continue reading the main story}

\hypertarget{site-index}{%
\subsection{Site Index}\label{site-index}}

\hypertarget{site-information-navigation}{%
\subsection{Site Information
Navigation}\label{site-information-navigation}}

\begin{itemize}
\tightlist
\item
  \href{https://help.nytimes3xbfgragh.onion/hc/en-us/articles/115014792127-Copyright-notice}{©~2020~The
  New York Times Company}
\end{itemize}

\begin{itemize}
\tightlist
\item
  \href{https://www.nytco.com/}{NYTCo}
\item
  \href{https://help.nytimes3xbfgragh.onion/hc/en-us/articles/115015385887-Contact-Us}{Contact
  Us}
\item
  \href{https://www.nytco.com/careers/}{Work with us}
\item
  \href{https://nytmediakit.com/}{Advertise}
\item
  \href{http://www.tbrandstudio.com/}{T Brand Studio}
\item
  \href{https://www.nytimes3xbfgragh.onion/privacy/cookie-policy\#how-do-i-manage-trackers}{Your
  Ad Choices}
\item
  \href{https://www.nytimes3xbfgragh.onion/privacy}{Privacy}
\item
  \href{https://help.nytimes3xbfgragh.onion/hc/en-us/articles/115014893428-Terms-of-service}{Terms
  of Service}
\item
  \href{https://help.nytimes3xbfgragh.onion/hc/en-us/articles/115014893968-Terms-of-sale}{Terms
  of Sale}
\item
  \href{https://spiderbites.nytimes3xbfgragh.onion}{Site Map}
\item
  \href{https://help.nytimes3xbfgragh.onion/hc/en-us}{Help}
\item
  \href{https://www.nytimes3xbfgragh.onion/subscription?campaignId=37WXW}{Subscriptions}
\end{itemize}
