Sections

SEARCH

\protect\hyperlink{site-content}{Skip to
content}\protect\hyperlink{site-index}{Skip to site index}

\href{https://www.nytimes3xbfgragh.onion/section/parenting}{Parenting}

\href{https://myaccount.nytimes3xbfgragh.onion/auth/login?response_type=cookie\&client_id=vi}{}

\href{https://www.nytimes3xbfgragh.onion/section/todayspaper}{Today's
Paper}

\href{/section/parenting}{Parenting}\textbar{}Pods, Microschools and
Tutors: Can Parents Solve the Education Crisis on Their Own?

\url{https://nyti.ms/30Fksvg}

\begin{itemize}
\item
\item
\item
\item
\item
\end{itemize}

\hypertarget{school-reopenings}{%
\subsubsection{\texorpdfstring{\href{https://www.nytimes3xbfgragh.onion/spotlight/schools-reopening?name=styln-coronavirus-schools-reopening\&region=TOP_BANNER\&block=storyline_menu_recirc\&action=click\&pgtype=Article\&impression_id=ce3b7610-f4b6-11ea-b916-45f6d9eb4786\&variant=undefined}{School
Reopenings}}{School Reopenings}}\label{school-reopenings}}

\begin{itemize}
\tightlist
\item
  \href{https://www.nytimes3xbfgragh.onion/2020/09/08/us/school-districts-cyberattacks-glitches.html?name=styln-coronavirus-schools-reopening\&region=TOP_BANNER\&block=storyline_menu_recirc\&action=click\&pgtype=Article\&impression_id=ce3b7611-f4b6-11ea-b916-45f6d9eb4786\&variant=undefined}{Remote
  Learning Glitches}
\item
  \href{https://www.nytimes3xbfgragh.onion/2020/09/08/upshot/children-testing-shortfalls-virus.html?name=styln-coronavirus-schools-reopening\&region=TOP_BANNER\&block=storyline_menu_recirc\&action=click\&pgtype=Article\&impression_id=ce3b9d20-f4b6-11ea-b916-45f6d9eb4786\&variant=undefined}{Limited
  Testing for Children}
\item
  \href{https://www.nytimes3xbfgragh.onion/2020/09/10/us/des-moines-school-opening-coronavirus.html?name=styln-coronavirus-schools-reopening\&region=TOP_BANNER\&block=storyline_menu_recirc\&action=click\&pgtype=Article\&impression_id=ce3b9d21-f4b6-11ea-b916-45f6d9eb4786\&variant=undefined}{District
  Defies Reopening Order}
\item
  \href{https://www.nytimes3xbfgragh.onion/interactive/2020/us/covid-college-cases-tracker.html?name=styln-coronavirus-schools-reopening\&region=TOP_BANNER\&block=storyline_menu_recirc\&action=click\&pgtype=Article\&impression_id=ce3b9d22-f4b6-11ea-b916-45f6d9eb4786\&variant=undefined}{Tracking
  College Cases}
\end{itemize}

Advertisement

\protect\hyperlink{after-top}{Continue reading the main story}

Supported by

\protect\hyperlink{after-sponsor}{Continue reading the main story}

\hypertarget{pods-microschools-and-tutors-can-parents-solve-the-education-crisis-on-their-own}{%
\section{Pods, Microschools and Tutors: Can Parents Solve the Education
Crisis on Their
Own?}\label{pods-microschools-and-tutors-can-parents-solve-the-education-crisis-on-their-own}}

As school openings remain in flux, families grapple with big questions
about safety, money and politics.

\includegraphics{https://static01.graylady3jvrrxbe.onion/images/2020/07/22/multimedia/22parenting-learning-pods/22parenting-learning-pods-articleLarge.jpg?quality=75\&auto=webp\&disable=upscale}

By Melinda Wenner Moyer

\begin{itemize}
\item
  Published July 22, 2020Updated Aug. 18, 2020
\item
  \begin{itemize}
  \item
  \item
  \item
  \item
  \item
  \end{itemize}
\end{itemize}

In recent weeks, many parents have realized the agonizing truth about
\href{https://www.nytimes3xbfgragh.onion/2020/07/14/us/coronavirus-schools-fall.html}{school
this fall}: If it happens in person, it might not feel safe. And
\href{https://www.nytimes3xbfgragh.onion/2020/07/13/us/lausd-san-diego-school-reopening.html}{if
it happens remotely}, it will be inadequate, isolating and unable to
provide
\href{https://www.nytimes3xbfgragh.onion/2020/07/10/nyregion/nyc-school-daycare-reopening.html}{the
child care many working parents need}. Desperate for a better solution,
parents around the country have started organizing ``pandemic pods,'' or
home schooling pods, for the fall, in which groups of three to 10
students learn together in homes under the tutelage of the children's
parents or a hired teacher.

These pods could provide families with a schooling option that feels
safe --- yet also allows kids to have fun and build social skills. And,
depending on how the pods are set up, they may offer parents a break.
But given that pods can be pricey, complicated to organize and
self-selecting, they are likely to be most popular among families of
privilege, experts say, and may worsen educational inequality.

\hypertarget{pod-mania}{%
\subsection{Pod Mania}\label{pod-mania}}

For parents who can organize and afford them, pods seem like an easy
choice. Marissa Leitner, a mother of three and a school psychologist who
lives in Culver City, California, where schools have announced they will
be entirely remote in the fall, doesn't think an internet-based
curriculum would serve her kindergartner well. ``I don't believe that
the Zoom experience for that age group is appropriate,'' she said.
``Kids at this age really need that multimodal sensory learning.''

So Leitner and her brother-in-law, Daniel Zakowski, who also has three
kids, are organizing a pod involving three or four families in which
they will hire a teacher during the mornings and perhaps a college
student to help out in the afternoons. The pod will provide their kids
with a predictable schedule and structure, which they hope will make the
children feel anchored and safe, Zakowski said.

Seattle-based father Ivan Kerbel also plans to set up a learning pod for
his kids. In fact, he is hoping to organize these pods, what he calls
``nano schools,'' for many Seattle families, and he has started a
\href{https://www.facebookcorewwwi.onion/groups/seattle.micro.schools/}{Facebook
group} to facilitate it, which now has more than 3,000 members.

``If you think of coronavirus as a natural disaster, like a tsunami that
has swept over the land, it's actually left a lot of things intact,''
Kerbel said. ``The only rule is, you can't bring a lot of people
together in one enclosed space. The teachers are by and large available,
the content and the curriculum weren't destroyed,'' so there's no reason
kids can't go to school, he said --- it just has to happen in lower
numbers and different spaces.

Services have even popped up to match families with teachers and to
organize pods on behalf of families. In early July, the website
\href{https://families.getselected.com/}{Selected For Families} launched
to connect families with a professional teachers and tutors. Of the
first 60 families that filled out registration forms, 46 percent were
inquiring on behalf of learning pods, according to the company.
\href{https://www.getschoolhouse.com/}{Schoolhouse} provides a similar
service.

The Manhattan-based independent Portfolio School, along with the
independent Hudson Lab School in Westchester and Red Bridge School in
San Francisco, are also \href{https://www.learning-pods.com/}{organizing
home-based pods} for families around the country; they will hire and
manage teachers and even help families negotiate pod agreements. ``We
just saw the writing on the wall that people are going to start
podding,'' said Portfolio School co-founder Doug Schachtel. More than
300 people registered for an information session that ran on Tuesday.

A lot of interest, Schachtel says, seems to be stemming from New Jersey,
Los Angeles and San Francisco. The San Francisco-based Facebook group
\href{https://www.facebookcorewwwi.onion/groups/pandemicpodsf}{Pandemic
Pods and Microschools}, for instance, now has more than 9,500 members.
The group was started by Lian Chang, who wanted to find a safe way for
her 3-year-old to socialize during the pandemic.

Instead of hiring teachers, some families are hoping to share the
teaching among the parents. Meredith Phillips, a mother of an 8-year-old
and an 11-year-old who lives in Croton, N.Y., is hoping to create a pod
with three other families this fall that will rotate houses. One of the
dads, who owns a tech company, might teach coding, while Phillips, who
is an editor, will teach reading and writing. The parents will ideally
teach ``whatever they're good at, or know about or care about,''
Phillips said, and in doing so expose the kids to lots of different
subjects.

Some families are pulling their kids out of school for these learning
pods, while others are using pods as a supplement to their schools'
online curricula. ``Ideally, from our perspective, it would be
complementary, rather than a replacement,'' said Adam Davis, a
pediatrician in San Francisco who is hoping to create a learning pod
with a teacher or college-aged helper for his second grader and
kindergartener in the fall.

``We're pretty committed to our school --- we're involved in the Parent
Teacher Association,'' he added. Parents who pull their kids out of
school also have to contend with state homeschooling laws, although it's
not clear whether infractions would be enforced during the pandemic.

\hypertarget{pods-and-privilege}{%
\subsection{Pods and Privilege}\label{pods-and-privilege}}

Some parents argue that by pulling their kids out of public schools to
join pods, they are doing a public service because they leave more
resources for kids who stay in school. But that's ``not how education
finance works,'' said L'Heureux Lewis-McCoy, Ph.D., an educational
sociologist who studies educational inequality at New York University's
Steinhardt School of Culture, Education and Human Development. ``The
idea that if I pull out my child, it'll be better for the district, is
quite the opposite,'' he said.

In fact, if students leave public schools to join pods, funding for
already starved public schools could drop further. ``If dollars follow
students, and in many states they do, that can mean that school budgets
are directly reduced for each child that is no longer attending,'' said
Jessica Calarco, Ph.D., a sociologist who studies educational inequality
at Indiana University. Parents starting pods should ask their school
administrators how their departure will affect both short-term and
long-term school funding, Dr. Calarco said, and ideally donate any lost
funds to the school through the P.T.A. or a school foundation.

Given the financial and time costs of podding, they will likely be more
popular among privileged families. The pods organized by the Portfolio
School, Hudson Lab School and Red Bridge School cost \$2,500 per
elementary school child per month for a pod size of five, though
financial aid is available.

``What most families do is, they start from a place of self-interest.
They say, `all right, I've got to figure out what's best for my family,
got to figure out what's best for my child.' And the families who have
greater sets of resources usually use those resources to hoard
educational opportunities,'' Dr. Lewis-McCoy said. ``The truth of the
matter is, we're staring down the barrel at something that is going to
divide and widen the gaps between kids.''

Alexis Kushner de la Peña, a mother of a 5-year-old who is
Mexican-American, joined the Pandemic Pods and Microschools group in
July. She noticed that most families introducing themselves were white
and well-off. ``I noticed that the parents seemed to have nice careers,
spoke English, and their children didn't look particularly diverse,''
she said. So she wrote in a post to the group last week, ``We should all
think about how pods might further the gap between our students.
Families of underserved communities might not be engaging on Facebook,
might not speak English, might not have means to pay, to drive,
etcetera.'' It's also likely that children with disabilities, learning
differences or behavioral issues will be left out of pods, Dr. Calarco
said.

Well-meaning attempts to invite underprivileged kids into pods and
subsidizing their costs --- which has been suggested in pod-related
Facebook groups --- may create friction as well. ``That is much more a
model of charity than equity, and the fundamental issue with that is,
first, the person who was invited in is often viewed as the
beneficiary,'' Dr. Lewis-McCoy said. ``That child and that family don't
have equal share in what happens there.''

\href{https://www.nytimes3xbfgragh.onion/spotlight/schools-reopening?action=click\&pgtype=Article\&state=default\&region=MAIN_CONTENT_3\&context=storylines_keepup}{}

\hypertarget{school-reopenings-}{%
\subsubsection{School Reopenings ›}\label{school-reopenings-}}

\hypertarget{back-to-school}{%
\paragraph{Back to School}\label{back-to-school}}

Updated Sept. 11, 2020

The latest on how schools are reopening amid the pandemic.

\begin{itemize}
\item
  \begin{itemize}
  \tightlist
  \item
    School officials in Des Moines are refusing to hold in-person
    classes,
    \href{https://www.nytimes3xbfgragh.onion/2020/09/10/us/des-moines-school-opening-coronavirus.html?action=click\&pgtype=Article\&state=default\&region=MAIN_CONTENT_3\&context=storylines_keepup}{despite
    an order from Iowa's governor and a judge's ruling}, risking school
    funding and their jobs because they think it's unsafe.
  \item
    The University of Illinois at Urbana-Champaign had one of the most
    comprehensive plans by a major college to keep the virus under
    control. But it
    \href{https://www.nytimes3xbfgragh.onion/2020/09/10/health/university-illinois-covid.html?action=click\&pgtype=Article\&state=default\&region=MAIN_CONTENT_3\&context=storylines_keepup}{failed
    to account for students partying}.
  \item
    College students are
    \href{https://www.nytimes3xbfgragh.onion/2020/09/10/technology/coronavirus-quarantines-college.html?action=click\&pgtype=Article\&state=default\&region=MAIN_CONTENT_3\&context=storylines_keepup}{using
    apps to shame their schools}~into better coronavirus plans.
  \item
    For some families, the pandemic
    \href{https://www.nytimes3xbfgragh.onion/2020/09/10/parenting/family-second-language-coronavirus.html?action=click\&pgtype=Article\&state=default\&region=MAIN_CONTENT_3\&context=storylines_keepup}{has
    meant a return to their native languages}.
  \end{itemize}
\end{itemize}

Still, many parents feel they are in an impossible situation, given that
so many schools do not have the funding to provide safe and effective
options.

\hypertarget{making-pods-safe-effective--and-equitable}{%
\subsection{Making pods safe, effective --- and
equitable}\label{making-pods-safe-effective--and-equitable}}

Another concern about pods is that families may not know how to minimize
Covid risks. Ideally, pods shouldn't have more than five kids, said
Saskia Popescu, Ph.D., an infection prevention epidemiologist at George
Mason University. When you add together the teacher and all of the kids'
family members, a seemingly small pod ends up including dozens of
people, and the more people in it, the greater the risk for coronavirus
exposure.

Ideally, families in learning pods shouldn't be socializing with people
outside the pod unless they wear masks and remain socially distant, Dr.
Popescu said. It's also important for families to work through various
contingencies, such as what should happen if someone ends up in a
high-risk situation, like going to a hospital, or gets sick. Pods should
have clear rules on wearing masks and washing hands.

If parents are doing the teaching in a pod, educators say that there are
no right and wrong answers. ``Try multiple things, see how they go. Go
in thinking of it as an experiment,'' said Mordechai Levy-Eichel, Ph.D.,
an education historian at Yale University and a father of three who was
home schooled himself. Focus less on formal curricula and more on
projects and enriching discussions, he suggested.

If parents are hiring a teacher, they should make sure their credentials
include a bachelor's degree in education and that they meet state
requirements, said Meg Flanagan, an
\href{http://megflanagan.com/}{educational consultant} based in the
Washington, D.C., area. Parents might want to run a background check on
candidates and ``ask about how the home-school learning day will be
structured, additional costs beyond the teacher's rate, and teaching
philosophy.'' Consider also hiring a teacher who is Black, Indigenous or
a person of color (B.I.P.O.C.), and asking them to implement a social
justice-themed curriculum, said Nikolai Pizarro, an educator, author and
mother in Atlanta who runs the Facebook group
\href{https://www.facebookcorewwwi.onion/groups/876843739507884/?ref=share}{BIPOC-Led
Pandemic Pods and Microschools.}

If parents do want to invite less privileged families into their pods,
they should choose families with whom they already have a good and
trusting relationship, Dr. Calarco suggested, so that these families are
more likely to feel they have equal standing. ``Think about building on
existing connections, if possible,'' she said.

Even better, though: Petition your local public school to create its own
learning pods. The K-8 \href{https://rooftopk8.org/}{Rooftop School}, in
the San Francisco Unified School District, will be supplementing its
remote learning this fall with pods of seven to nine students that will
meet perhaps in local parks or outside on the school campus. The school
is still securing funding, but principal Nancy Bui is confident it will
come together.

Bui decided to create these school-wide pods after she received emails
from parents asking if the school's teachers were interested in private
tutoring. She also heard of local families who had rented studio
apartments just for their pods. She said that if she didn't step up and
create an equitable solution, the school's families would self-segregate
according to privilege, and the most well-off families would have access
to more educational resources. That's ``the antithesis of what Rooftop
is all about, which is inclusion and diversity,'' she said.

\hypertarget{are-you-creating-a-learning-pod}{%
\subsection{Are You Creating a Learning
Pod?}\label{are-you-creating-a-learning-pod}}

\begin{center}\rule{0.5\linewidth}{\linethickness}\end{center}

Melinda Wenner Moyer is a science and health writer and the author of a
forthcoming book on raising children.

Advertisement

\protect\hyperlink{after-bottom}{Continue reading the main story}

\hypertarget{site-index}{%
\subsection{Site Index}\label{site-index}}

\hypertarget{site-information-navigation}{%
\subsection{Site Information
Navigation}\label{site-information-navigation}}

\begin{itemize}
\tightlist
\item
  \href{https://help.nytimes3xbfgragh.onion/hc/en-us/articles/115014792127-Copyright-notice}{©~2020~The
  New York Times Company}
\end{itemize}

\begin{itemize}
\tightlist
\item
  \href{https://www.nytco.com/}{NYTCo}
\item
  \href{https://help.nytimes3xbfgragh.onion/hc/en-us/articles/115015385887-Contact-Us}{Contact
  Us}
\item
  \href{https://www.nytco.com/careers/}{Work with us}
\item
  \href{https://nytmediakit.com/}{Advertise}
\item
  \href{http://www.tbrandstudio.com/}{T Brand Studio}
\item
  \href{https://www.nytimes3xbfgragh.onion/privacy/cookie-policy\#how-do-i-manage-trackers}{Your
  Ad Choices}
\item
  \href{https://www.nytimes3xbfgragh.onion/privacy}{Privacy}
\item
  \href{https://help.nytimes3xbfgragh.onion/hc/en-us/articles/115014893428-Terms-of-service}{Terms
  of Service}
\item
  \href{https://help.nytimes3xbfgragh.onion/hc/en-us/articles/115014893968-Terms-of-sale}{Terms
  of Sale}
\item
  \href{https://spiderbites.nytimes3xbfgragh.onion}{Site Map}
\item
  \href{https://help.nytimes3xbfgragh.onion/hc/en-us}{Help}
\item
  \href{https://www.nytimes3xbfgragh.onion/subscription?campaignId=37WXW}{Subscriptions}
\end{itemize}
