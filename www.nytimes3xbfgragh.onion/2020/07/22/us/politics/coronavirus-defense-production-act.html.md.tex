Sections

SEARCH

\protect\hyperlink{site-content}{Skip to
content}\protect\hyperlink{site-index}{Skip to site index}

\href{https://www.nytimes3xbfgragh.onion/section/politics}{Politics}

\href{https://myaccount.nytimes3xbfgragh.onion/auth/login?response_type=cookie\&client_id=vi}{}

\href{https://www.nytimes3xbfgragh.onion/section/todayspaper}{Today's
Paper}

\href{/section/politics}{Politics}\textbar{}Virus Surge Brings Calls for
Trump to Invoke Defense Production Act

\url{https://nyti.ms/3fRDkO5}

\begin{itemize}
\item
\item
\item
\item
\item
\end{itemize}

\hypertarget{the-coronavirus-outbreak}{%
\subsubsection{\texorpdfstring{\href{https://www.nytimes3xbfgragh.onion/news-event/coronavirus?name=styln-coronavirus-national\&region=TOP_BANNER\&block=storyline_menu_recirc\&action=click\&pgtype=Article\&impression_id=639b7e90-f2c7-11ea-afab-d96c7c69d660\&variant=undefined}{The
Coronavirus
Outbreak}}{The Coronavirus Outbreak}}\label{the-coronavirus-outbreak}}

\begin{itemize}
\tightlist
\item
  live\href{https://www.nytimes3xbfgragh.onion/2020/09/09/world/covid-19-coronavirus.html?name=styln-coronavirus-national\&region=TOP_BANNER\&block=storyline_menu_recirc\&action=click\&pgtype=Article\&impression_id=639ba5a0-f2c7-11ea-afab-d96c7c69d660\&variant=undefined}{Latest
  Updates}
\item
  \href{https://www.nytimes3xbfgragh.onion/interactive/2020/us/coronavirus-us-cases.html?name=styln-coronavirus-national\&region=TOP_BANNER\&block=storyline_menu_recirc\&action=click\&pgtype=Article\&impression_id=639ba5a1-f2c7-11ea-afab-d96c7c69d660\&variant=undefined}{Maps
  and Cases}
\item
  \href{https://www.nytimes3xbfgragh.onion/interactive/2020/science/coronavirus-vaccine-tracker.html?name=styln-coronavirus-national\&region=TOP_BANNER\&block=storyline_menu_recirc\&action=click\&pgtype=Article\&impression_id=639ba5a2-f2c7-11ea-afab-d96c7c69d660\&variant=undefined}{Vaccine
  Tracker}
\item
  \href{https://www.nytimes3xbfgragh.onion/2020/09/02/your-money/eviction-moratorium-covid.html?name=styln-coronavirus-national\&region=TOP_BANNER\&block=storyline_menu_recirc\&action=click\&pgtype=Article\&impression_id=639ba5a3-f2c7-11ea-afab-d96c7c69d660\&variant=undefined}{Eviction
  Moratorium}
\item
  \href{https://www.nytimes3xbfgragh.onion/2020/09/09/upshot/coronavirus-surprise-test-fees.html?name=styln-coronavirus-national\&region=TOP_BANNER\&block=storyline_menu_recirc\&action=click\&pgtype=Article\&impression_id=639ba5a4-f2c7-11ea-afab-d96c7c69d660\&variant=undefined}{Surprise
  Test Fees}
\end{itemize}

Advertisement

\protect\hyperlink{after-top}{Continue reading the main story}

Supported by

\protect\hyperlink{after-sponsor}{Continue reading the main story}

\hypertarget{virus-surge-brings-calls-for-trump-to-invoke-defense-production-act}{%
\section{Virus Surge Brings Calls for Trump to Invoke Defense Production
Act}\label{virus-surge-brings-calls-for-trump-to-invoke-defense-production-act}}

Experts say the law could be more actively used to secure medical
supplies, but the Trump administration's reluctance has left states and
medical providers in ``chaos.''

\includegraphics{https://static01.graylady3jvrrxbe.onion/images/2020/07/17/us/politics/17dc-virus-dpa1/merlin_173583366_d67478ff-a021-480b-96ae-a1666c1bdb73-articleLarge.jpg?quality=75\&auto=webp\&disable=upscale}

By Aishvarya Kavi

\begin{itemize}
\item
  July 22, 2020
\item
  \begin{itemize}
  \item
  \item
  \item
  \item
  \item
  \end{itemize}
\end{itemize}

WASHINGTON --- Experts, medical workers and elected officials are
reviving their call for the Trump administration to ramp up its use of
the
\href{https://www.nytimes3xbfgragh.onion/2020/03/20/us/politics/defense-production-act-virus.html}{Defense
Production Act} to secure critical medical supplies.

In March, as the coronavirus pandemic took hold in the United States and
\href{https://www.nytimes3xbfgragh.onion/2020/03/27/us/politics/coronavirus-trump-ventilators-gm-ventec.html}{pressure
from cities and states} grew, President Trump used the act to
\href{https://www.nytimes3xbfgragh.onion/2020/03/30/business/gm-ventilators-coronavirus-trump.html}{press
General Motors} to begin production of ventilators. But four months
later, frustrated by what they describe as a lack of federal leadership
in the face of continued shortages, critics say the Trump administration
is not wielding the act to the extent that it can and should.

``What the federal government --- the president or secretaries
possessing delegated authority --- have not done yet is use the D.P.A.
to create a permanent, sustainable, redundant, domestic supply chain for
all things pandemic: testing, swabs,
\href{https://www.nytimes3xbfgragh.onion/2020/06/01/health/masks-surgical-N95-coronavirus.html}{N95
masks}, etc.,'' said Jamie Baker, a former legal adviser to the National
Security Council and a professor of national security law at Syracuse
University.

``I'm frustrated that there appears to be no national strategy,'' said
Larry Hall, who retired in August as the director of the Defense
Production Act program at the Federal Emergency Management Agency. ``Why
isn't this administration using the act to prevent shortages?''

The White House did not respond to a request for comment, but at a White
House coronavirus task force briefing last week,
\href{https://www.whitehouse.gov/briefings-statements/press-briefing-vice-president-pence-members-white-house-coronavirus-task-force-baton-rouge-la/}{Vice
President Mike Pence said} that the administration continued to work to
provide states with critical supplies. The federal government's
``ability to respond to this pandemic is substantially better than two
and three months ago,'' he said.

The Defense Production Act grants the federal government a range of
authorities such as issuing loans to expand a vendor's capacity,
controlling the distribution of a company's products and --- most
commonly --- compelling companies to prioritize the government's orders
over those of other clients, meaning the federal government essentially
skips to the front of the line.

Since taking office, Mr. Trump has
\href{https://www.nytimes3xbfgragh.onion/2020/03/31/us/politics/coronavirus-defense-production-act.html}{routinely
used} the Korean War-era law to procure defense equipment, but he
\href{https://www.nytimes3xbfgragh.onion/2020/03/20/us/politics/trump-coronavirus-supplies.html}{resisted
doing so} in the early days of the pandemic. States and medical
providers are still competing with one another to procure vital supplies
that are unevenly distributed, just as they did months ago.

\hypertarget{latest-updates-the-coronavirus-outbreak}{%
\section{\texorpdfstring{\href{https://www.nytimes3xbfgragh.onion/2020/09/09/world/covid-19-coronavirus.html?action=click\&pgtype=Article\&state=default\&region=MAIN_CONTENT_1\&context=storylines_live_updates}{Latest
Updates: The Coronavirus
Outbreak}}{Latest Updates: The Coronavirus Outbreak}}\label{latest-updates-the-coronavirus-outbreak}}

Updated 2020-09-09T18:04:33.466Z

\begin{itemize}
\tightlist
\item
  \href{https://www.nytimes3xbfgragh.onion/2020/09/09/world/covid-19-coronavirus.html?action=click\&pgtype=Article\&state=default\&region=MAIN_CONTENT_1\&context=storylines_live_updates\#link-279e24e2}{Top
  U.S. health officials update Congress on vaccine development and
  distribution plans.}
\item
  \href{https://www.nytimes3xbfgragh.onion/2020/09/09/world/covid-19-coronavirus.html?action=click\&pgtype=Article\&state=default\&region=MAIN_CONTENT_1\&context=storylines_live_updates\#link-792ae257}{Indoor
  dining in N.Y.C. will return with limits on Sept. 30, Cuomo says.}
\item
  \href{https://www.nytimes3xbfgragh.onion/2020/09/09/world/covid-19-coronavirus.html?action=click\&pgtype=Article\&state=default\&region=MAIN_CONTENT_1\&context=storylines_live_updates\#link-5b0bf0d1}{As
  drugmakers pledge to thoroughly vet vaccines, one company pauses its
  trials for a safety review.}
\end{itemize}

\href{https://www.nytimes3xbfgragh.onion/2020/09/09/world/covid-19-coronavirus.html?action=click\&pgtype=Article\&state=default\&region=MAIN_CONTENT_1\&context=storylines_live_updates}{See
more updates}

More live coverage:
\href{https://www.nytimes3xbfgragh.onion/live/2020/09/09/business/stock-market-today-coronavirus?action=click\&pgtype=Article\&state=default\&region=MAIN_CONTENT_1\&context=storylines_live_updates}{Markets}

``The reluctance to use this tool has simply led to chaos,'' said Andrew
Hunter, the director of the Defense-Industrial Initiatives Group at the
Center for Strategic and International Studies.

Deborah Burger, a president of National Nurses United, says many nurses
are forced to reuse N95 masks even though they should be discarded
between patients. Some
\href{https://www.nytimes3xbfgragh.onion/2020/07/08/health/coronavirus-masks-ppe-doc.html}{nurses
are also concerned} that because of wear and tear,
\href{https://www.nytimes3xbfgragh.onion/2020/04/11/business/coronavirus-n95-mask-decontaminating-reuse.html}{decontaminated
masks} will not properly filter out the virus. The Centers for Disease
Control and Prevention
\href{https://www.cdc.gov/coronavirus/2019-ncov/hcp/ppe-strategy/decontamination-reuse-respirators.html}{does
not approve of decontaminating} masks for reuse, but exceptions have
been made for ``times of shortage.''

``In all indications, the capacity of P.P.E. is lower than the demand,''
said Prakash Mirchandani, the director of the Center for Supply Chain
Management at the University of Pittsburgh, referring to personal
protective equipment. ``The total requirement of N95 masks, just for
physicians and nurses, is about four to five billion units a year.''

The federal government does not release
\href{https://www.fema.gov/media-library-data/1582898704576-dc44bbe61cce3cf763cc8a6b92617188/2018_DPAC_Report_to_Congress.pdf}{reports}
outlining each specific order placed under the Defense Production Act.
But based on contracts that have been announced, interviews with experts
who are tracking the distribution of medical supplies and interviews
with advocates for medical workers' needs, there is little evidence that
the administration has made widespread use of the act to control the
supply chain to combat the coronavirus.

``There have been myths that, at one point, all of a sudden, there was
enough of a stockpile of protective equipment,'' said Ms. Burger, who is
a nurse at a hospital in Sonoma County, Calif. ``But the honest truth is
that if you talk to any nurse in the country, they have never had enough
equipment.''

A spokeswoman for the Defense Department said it had awarded seven
contracts for medical supplies under the Defense Production Act since
the coronavirus pandemic began. But **** as cases
\href{https://www.nytimes3xbfgragh.onion/interactive/2020/us/coronavirus-us-cases.html}{surge
across the country}, the rate of infections has outpaced the production
of protective gear, especially for N95 masks.

Under five of those contracts, the Defense Department paid three
companies over \$200 million to expand and retool factories to produce
N95 masks in greater quantities --- a process that takes months. The
companies will not reach maximum production until late summer or fall.

A spokeswoman for 3M, a manufacturer of N95 masks that has received at
least four contracts from the federal government under the act, said it
had delivered 200 million N95 masks to U.S. hospitals, FEMA and the
federal stockpile since the pandemic began.

But FEMA
\href{https://www.hassan.senate.gov/imo/media/doc/SCTF\%20Demand\%20PPE\%20Chart.pdf}{estimates
that even in October}, when it predicts that the United States will
produce 180 million N95 masks a month, domestic production will not meet
demand. Still, the last time the Defense Department issued a contract
under the Defense Production Act to produce additional masks was in late
May.

For one of the Defense Department's two other contracts for medical
supplies under the act, deliveries of protective gear began the same
week the agreement was announced in May. More than 10 million masks and
gowns were included in kits of protective gear for workers at over
15,000 nursing homes. But FEMA estimated in May that the United States
needed more than 150 million gowns.

The other contract, announced in April, doubled production of the only
domestic manufacturer of swabs for coronavirus tests to 40 million per
month. But now, more that three months later, officials in some cities
are forced to
\href{https://www.nytimes3xbfgragh.onion/2020/07/06/us/coronavirus-test-shortage.html}{limit
testing} because of swab shortages.

In addition to the Defense Department's contracts, FEMA said it had
issued one contract under the act to 3M for the production of N95 masks.
The Department of Health and Human Services has
\href{https://www.hhs.gov/about/news/2020/04/13/hhs-announces-new-ventilator-contracts-orders-now-totaling-over-130000-ventilators.html}{announced
contracts} with at least seven companies to produce ventilators; it did
not respond to a request for more information.

In March, Neil Bradley, the executive vice president and chief policy
officer at the U.S. Chamber of Commerce, said the Defense Production Act
was not ``a magic wand.''

And testifying before the House Homeland Security Committee on
Wednesday, Peter T. Gaynor, the FEMA administrator, defended the
administration's approach to using the law.

``We used it when we needed it,'' Mr. Gaynor said. ``It's just not as
easy as flipping the switch.''

But Mr. Hall said that had not been his experience.

``I have used this like a switch,'' he said. ``The military uses this
like a switch. I can't understand why it can't be switched on to meet
these vital shortages.''

Mr. Mirchandani said that relying on the law to increase manufacturing
of ventilators helped significantly. There does not seem to be a
shortage after at least seven contracts to send almost 30,000
ventilators to the federal government's stockpile by June 1. But there
is far less demand for ventilators than for protective gear.

``There has to be support from some external source, probably from the
federal government, to increase capacity quickly,'' Mr. Mirchandani
said. ``And distribution also has to be done in a coordinated fashion.''

\includegraphics{https://static01.graylady3jvrrxbe.onion/images/2020/07/17/us/politics/17dc-virus-dpa2/merlin_174389439_401d5771-5268-4ea5-bb7b-6c865e9266bb-articleLarge.jpg?quality=75\&auto=webp\&disable=upscale}

Jessica R. Maxwell, a spokeswoman for the Defense Department, said that
all current contracts under the act were on track to meet production
goals, but she did not say how the department would work to fill the gap
in demand.

In testimony before Congress this month, Gov. J.B. Pritzker of Illinois
blamed the federal government for forcing states into ``a bidding war
for lifesaving supplies.''

``The Defense Production Act was not broadly invoked early enough,'' he
said.

Mr. Baker, the national security law professor, said he worried that the
federal government's struggle to supply swabs and protective gear might
portend difficulties in widely distributing a vaccine.

``Whatever vaccine is produced, it's going to involve a vial and a
needle,'' he said. ``If we cannot figure out how to produce enough swabs
or tests, will we figure out how to produce enough vaccine or
treatments?''

Advertisement

\protect\hyperlink{after-bottom}{Continue reading the main story}

\hypertarget{site-index}{%
\subsection{Site Index}\label{site-index}}

\hypertarget{site-information-navigation}{%
\subsection{Site Information
Navigation}\label{site-information-navigation}}

\begin{itemize}
\tightlist
\item
  \href{https://help.nytimes3xbfgragh.onion/hc/en-us/articles/115014792127-Copyright-notice}{©~2020~The
  New York Times Company}
\end{itemize}

\begin{itemize}
\tightlist
\item
  \href{https://www.nytco.com/}{NYTCo}
\item
  \href{https://help.nytimes3xbfgragh.onion/hc/en-us/articles/115015385887-Contact-Us}{Contact
  Us}
\item
  \href{https://www.nytco.com/careers/}{Work with us}
\item
  \href{https://nytmediakit.com/}{Advertise}
\item
  \href{http://www.tbrandstudio.com/}{T Brand Studio}
\item
  \href{https://www.nytimes3xbfgragh.onion/privacy/cookie-policy\#how-do-i-manage-trackers}{Your
  Ad Choices}
\item
  \href{https://www.nytimes3xbfgragh.onion/privacy}{Privacy}
\item
  \href{https://help.nytimes3xbfgragh.onion/hc/en-us/articles/115014893428-Terms-of-service}{Terms
  of Service}
\item
  \href{https://help.nytimes3xbfgragh.onion/hc/en-us/articles/115014893968-Terms-of-sale}{Terms
  of Sale}
\item
  \href{https://spiderbites.nytimes3xbfgragh.onion}{Site Map}
\item
  \href{https://help.nytimes3xbfgragh.onion/hc/en-us}{Help}
\item
  \href{https://www.nytimes3xbfgragh.onion/subscription?campaignId=37WXW}{Subscriptions}
\end{itemize}
