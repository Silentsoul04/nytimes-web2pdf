Sections

SEARCH

\protect\hyperlink{site-content}{Skip to
content}\protect\hyperlink{site-index}{Skip to site index}

\href{https://www.nytimes3xbfgragh.onion/section/food/drinks}{Wine, Beer
\& Cocktails}

\href{https://myaccount.nytimes3xbfgragh.onion/auth/login?response_type=cookie\&client_id=vi}{}

\href{https://www.nytimes3xbfgragh.onion/section/todayspaper}{Today's
Paper}

\href{/section/food/drinks}{Wine, Beer \& Cocktails}\textbar{}Anthony
Terlato, Who Brought Pinot Grigio to the U.S., Dies at 86

\url{https://nyti.ms/32P3tte}

\begin{itemize}
\item
\item
\item
\item
\item
\end{itemize}

Advertisement

\protect\hyperlink{after-top}{Continue reading the main story}

Supported by

\protect\hyperlink{after-sponsor}{Continue reading the main story}

\hypertarget{anthony-terlato-who-brought-pinot-grigio-to-the-us-dies-at-86}{%
\section{Anthony Terlato, Who Brought Pinot Grigio to the U.S., Dies at
86}\label{anthony-terlato-who-brought-pinot-grigio-to-the-us-dies-at-86}}

In a 60-year career as a wine importer and marketer, he introduced
Americans to lesser-known labels and shaped tastes.

\includegraphics{https://static01.graylady3jvrrxbe.onion/images/2020/07/27/obituaries/27terlato-obit1/merlin_174793932_369d588b-c7a9-4752-9920-f585799eeeae-articleLarge.jpg?quality=75\&auto=webp\&disable=upscale}

\href{https://www.nytimes3xbfgragh.onion/by/eric-asimov}{\includegraphics{https://static01.graylady3jvrrxbe.onion/images/2018/06/13/multimedia/author-eric-asimov/author-eric-asimov-thumbLarge.jpg}}

By \href{https://www.nytimes3xbfgragh.onion/by/eric-asimov}{Eric Asimov}

\begin{itemize}
\item
  July 23, 2020
\item
  \begin{itemize}
  \item
  \item
  \item
  \item
  \item
  \end{itemize}
\end{itemize}

Anthony Terlato, a visionary wine importer and marketer who introduced
Americans to enduringly popular European wines and sought to elevate the
wine market in the United States, died on June 28 at his home in Lake
Geneva, Wis. He was 86.

His family company, the \href{http://www.twg.com/}{Terlato Wine Group},
said he died in his sleep after a family gathering.

Over a 60-year career, Mr. Terlato was a retailer, wholesaler and
importer, and later a winery and vineyard owner. But he will probably be
best remembered as the man who introduced pinot grigio to Americans.

In 1979, Mr. Terlato was head of what was then called Paterno Wine
Imports. The company had already played a major role in popularizing
wines that would become ubiquitous in American culinary history,
including Mateus and Lancers from Portugal and Blue Nun from Germany. He
had also sold a lot of Corvo, an inexpensive white wine from Sicily.

But Mr. Terlato noted the rising interest in good white wines from
California and France, bottles that sold for considerably more than the
\$4 he could expect to fetch for leading Italian whites like Soave or
Orvieto.

He traveled to Italy in search of a white that he could sell for more
than \$10 a bottle, with Milan as his first stop. As he recounted in his
memoir, ``Taste: A Life in Wine'' (2008), he was offered a pinot grigio,
a wine made from a grape that was barely known then in the United
States, at his first meal.

He was so taken with the wine that he traveled the next day to
northeastern Italy, the main region for pinot grigio. At dinner in the
town of Portogruaro, he ordered all 18 bottles of pinot grigio on the
menu and concluded that the one produced by
\href{https://santamargheritawines.com/our-wines/pinot-grigio/}{Santa
Margherita} was the most interesting.

Shortly after, he negotiated a 10-year contract with Santa Margherita to
sell the wine in the United States. Over that time, Santa Margherita, a
crisp, light, easygoing alternative to oaky chardonnays, became one of
the most recognized wine brands in the United States and helped to make
pinot grigio a synonym for a glass of white at countless bars and
restaurants.

\includegraphics{https://static01.graylady3jvrrxbe.onion/images/2020/07/27/obituaries/27Terlato-obit2/22Terlato-Wine-articleLarge.png?quality=75\&auto=webp\&disable=upscale}

Although Mr. Terlato worked with the wines of many countries and
considered himself a Francophile, he decided in the late 1970s to focus
on Italian wines. He noted their improving quality and sensed a shift in
American tastes for Italian cuisine at the expense of French food.

He imported Il Poggione, a leading Brunello di Montalcino, and Renato
Ratti, a top Barolo. He also noted the rising popularity of certain
Tuscan red wines that were not conforming to the rules of the local
appellations and were put into the lowest official category --- vino di
tavola, or table wine. But because of their inherent quality, those
wines, with proprietary names like Sassicaia and Tignanello, came to be
known as Super-Tuscans and were highly popular in the 1980s and '90s.

Anthony John Terlato was born on May 11, 1934, in Brooklyn and grew up
in the Bensonhurst neighborhood. His father, Salvatore, sold insurance
and real estate, and his mother, Frances (Giarusso) Terlato, kept house.
They poured wine with dinner and, in the European fashion, mixed a
little wine with water for young Anthony.

Mr. Terlato briefly attended St. Francis College in Brooklyn. But, eager
to make his fortune, he left to work briefly in a bank and then held a
variety of jobs at hotels in Miami Beach.

His father, meanwhile, had retired and hoped to open a liquor store in
Brooklyn. When bureaucratic hurdles made his goal seem unreachable, an
old friend, Anthony Paterno, who ran a grocery and bottling business in
Chicago, persuaded him to try opening a shop there, where fewer
obstacles existed.

Mr. Terlato enlisted Anthony to help him, and together they opened the
shop, Leading Liquor Marts, in 1955. The encouragement from Mr. Paterno
came with additional benefits. In 1956, Mr. Terlato married his
daughter, Josephine Paterno.

She survives him, as do their sons, William Terlato, the chief executive
of Terlato Wine Group, and John Terlato, the group's vice chairman,
along with six grandchildren and three great-grandchildren.

After working briefly with his father, Mr. Terlato accepted his
father-in-law's invitation to work with him at Pacific Wine Company,
which bought wine in bulk and bottled it locally in Chicago.

Sensing accurately that local bottling was on its way out, Mr. Terlato
helped the company transition to importing and distributing. He also
shifted the focus from the sweetened and fortified wines that were
popular in the 1950s to the more expensive European-style bottles found
in fine restaurants.

Along the way he developed relationships with men like importer and
writer
\href{http://www.frankjohnsonselections.com/frank\%20schoonmaker.pdf}{Frank
Schoonmaker,} the wine writer and entrepreneur
\href{https://www.nytimes3xbfgragh.onion/1989/06/02/obituaries/alexis-lichine-76-an-author-and-expert-on-wine.html}{Alexis
Lichine} and
\href{https://www.nytimes3xbfgragh.onion/2008/05/17/business/17mondavi.html}{Robert
Mondavi}, the California vintner, who were all instrumental in guiding
Americans toward fine wines.

He would later work with French producers like Michel Chapoutier of the
Rhône Valley and distribute California wines like Williams Selyem,
Rochioli, Duckhorn and Silver Oak.

Pacific Wine Company became Paterno Imports and eventually, in 2007,
Terlato Wines International. (Terlato Wine Group is the umbrella
company.)

In the 1990s, Mr. Terlato began to acquire wineries and vineyards,
beginning with Rutherford Hill in Napa Valley. His stable of labels
would grow to include Chimney Rock in Napa, Sanford in Santa Barbara
County and Terlato Vineyards.

Mr. Terlato continued to work until he died, serving as chairman of
Terlato Wine Group.

``If you're not making decisions for quality reasons, you're going
backwards,'' he said in 2004, summing up his philosophy to Wine
Spectator magazine. ``Quality is the only thing that endures.''

Advertisement

\protect\hyperlink{after-bottom}{Continue reading the main story}

\hypertarget{site-index}{%
\subsection{Site Index}\label{site-index}}

\hypertarget{site-information-navigation}{%
\subsection{Site Information
Navigation}\label{site-information-navigation}}

\begin{itemize}
\tightlist
\item
  \href{https://help.nytimes3xbfgragh.onion/hc/en-us/articles/115014792127-Copyright-notice}{©~2020~The
  New York Times Company}
\end{itemize}

\begin{itemize}
\tightlist
\item
  \href{https://www.nytco.com/}{NYTCo}
\item
  \href{https://help.nytimes3xbfgragh.onion/hc/en-us/articles/115015385887-Contact-Us}{Contact
  Us}
\item
  \href{https://www.nytco.com/careers/}{Work with us}
\item
  \href{https://nytmediakit.com/}{Advertise}
\item
  \href{http://www.tbrandstudio.com/}{T Brand Studio}
\item
  \href{https://www.nytimes3xbfgragh.onion/privacy/cookie-policy\#how-do-i-manage-trackers}{Your
  Ad Choices}
\item
  \href{https://www.nytimes3xbfgragh.onion/privacy}{Privacy}
\item
  \href{https://help.nytimes3xbfgragh.onion/hc/en-us/articles/115014893428-Terms-of-service}{Terms
  of Service}
\item
  \href{https://help.nytimes3xbfgragh.onion/hc/en-us/articles/115014893968-Terms-of-sale}{Terms
  of Sale}
\item
  \href{https://spiderbites.nytimes3xbfgragh.onion}{Site Map}
\item
  \href{https://help.nytimes3xbfgragh.onion/hc/en-us}{Help}
\item
  \href{https://www.nytimes3xbfgragh.onion/subscription?campaignId=37WXW}{Subscriptions}
\end{itemize}
