Sections

SEARCH

\protect\hyperlink{site-content}{Skip to
content}\protect\hyperlink{site-index}{Skip to site index}

\href{https://www.nytimes3xbfgragh.onion/section/health}{Health}

\href{https://myaccount.nytimes3xbfgragh.onion/auth/login?response_type=cookie\&client_id=vi}{}

\href{https://www.nytimes3xbfgragh.onion/section/todayspaper}{Today's
Paper}

\href{/section/health}{Health}\textbar{}`It's Like Groundhog Day':
Coronavirus Testing Labs Again Lack Key Supplies

\url{https://nyti.ms/2EdrMHb}

\begin{itemize}
\item
\item
\item
\item
\item
\item
\end{itemize}

\hypertarget{the-coronavirus-outbreak}{%
\subsubsection{\texorpdfstring{\href{https://www.nytimes3xbfgragh.onion/news-event/coronavirus?name=styln-coronavirus-national\&region=TOP_BANNER\&block=storyline_menu_recirc\&action=click\&pgtype=Article\&impression_id=58f187a0-f4c5-11ea-9f76-ef806bade584\&variant=undefined}{The
Coronavirus
Outbreak}}{The Coronavirus Outbreak}}\label{the-coronavirus-outbreak}}

\begin{itemize}
\tightlist
\item
  live\href{https://www.nytimes3xbfgragh.onion/2020/09/11/world/covid-19-coronavirus.html?name=styln-coronavirus-national\&region=TOP_BANNER\&block=storyline_menu_recirc\&action=click\&pgtype=Article\&impression_id=58f1aeb0-f4c5-11ea-9f76-ef806bade584\&variant=undefined}{Latest
  Updates}
\item
  \href{https://www.nytimes3xbfgragh.onion/interactive/2020/us/coronavirus-us-cases.html?name=styln-coronavirus-national\&region=TOP_BANNER\&block=storyline_menu_recirc\&action=click\&pgtype=Article\&impression_id=58f1aeb1-f4c5-11ea-9f76-ef806bade584\&variant=undefined}{Maps
  and Cases}
\item
  \href{https://www.nytimes3xbfgragh.onion/interactive/2020/science/coronavirus-vaccine-tracker.html?name=styln-coronavirus-national\&region=TOP_BANNER\&block=storyline_menu_recirc\&action=click\&pgtype=Article\&impression_id=58f1aeb2-f4c5-11ea-9f76-ef806bade584\&variant=undefined}{Vaccine
  Tracker}
\item
  \href{https://www.nytimes3xbfgragh.onion/2020/09/10/us/politics/fda-coronavirus-vaccine.html?name=styln-coronavirus-national\&region=TOP_BANNER\&block=storyline_menu_recirc\&action=click\&pgtype=Article\&impression_id=58f1aeb3-f4c5-11ea-9f76-ef806bade584\&variant=undefined}{F.D.A.
  Regulators' Self-Defense}
\item
  \href{https://www.nytimes3xbfgragh.onion/2020/09/09/upshot/coronavirus-surprise-test-fees.html?name=styln-coronavirus-national\&region=TOP_BANNER\&block=storyline_menu_recirc\&action=click\&pgtype=Article\&impression_id=58f1d5c0-f4c5-11ea-9f76-ef806bade584\&variant=undefined}{Surprise
  Test Fees}
\end{itemize}

Advertisement

\protect\hyperlink{after-top}{Continue reading the main story}

Supported by

\protect\hyperlink{after-sponsor}{Continue reading the main story}

\hypertarget{its-like-groundhog-day-coronavirus-testing-labs-again-lack-key-supplies}{%
\section{`It's Like Groundhog Day': Coronavirus Testing Labs Again Lack
Key
Supplies}\label{its-like-groundhog-day-coronavirus-testing-labs-again-lack-key-supplies}}

Just weeks after resolving shortages in swabs, researchers are
struggling to find the chemicals and plastic pieces they need to carry
out coronavirus tests in the lab --- leading to long waiting times.

\includegraphics{https://static01.graylady3jvrrxbe.onion/images/2020/07/23/science/23VIRUS-TESTINGSHORTAGES1/merlin_171396006_c59eb8ae-293a-4a1e-ab7a-94a9defa3df0-articleLarge.jpg?quality=75\&auto=webp\&disable=upscale}

\href{https://www.nytimes3xbfgragh.onion/by/katherine-j--wu}{\includegraphics{https://static01.graylady3jvrrxbe.onion/images/2020/08/11/reader-center/author-katherine-j-wu/author-katherine-j-wu-thumbLarge.png}}

By
\href{https://www.nytimes3xbfgragh.onion/by/katherine-j--wu}{Katherine
J. Wu}

\begin{itemize}
\item
  Published July 23, 2020Updated Aug. 15, 2020
\item
  \begin{itemize}
  \item
  \item
  \item
  \item
  \item
  \item
  \end{itemize}
\end{itemize}

Labs across the country are facing backlogs in
\href{https://www.nytimes3xbfgragh.onion/2020/08/15/us/coronavirus-testing-decrease.html}{coronavirus
testing} thanks in part to a shortage of tiny pieces of tapered plastic.

Researchers need these little disposables, called pipette tips, to
quickly and precisely move liquid between vials as they process the
tests.

As the number of known coronavirus cases in the United States
\href{https://www.nytimes3xbfgragh.onion/interactive/2020/us/coronavirus-us-cases.html}{passes
4 million}, these new shortages of pipette tips and other lab supplies
are once again stymieing efforts to track and curb the spread of
disease. Some people are
\href{https://larremorelab.github.io/covid19testgroup}{waiting days or
even weeks for results}, and labs are vying for crucial materials.

``That's the crazy part,'' said Dr. Alexander McAdam, director of the
infectious diseases diagnostic laboratory at Boston Children's Hospital,
one of many institutions seeking the prized pipette tips. ``Whenever
there's a shortage, it's lab versus lab, city versus city, state versus
state, competing for supplies.''

Fed into automated devices, pipette tips can help researchers blaze
through hundreds of coronavirus tests in a matter of hours, sparing them
grueling manual labor.

The Swiss company Tecan, which supplies pipette tips for machines used
by hundreds of laboratories in the United States, has been slammed with
orders from U.S. customers in recent months, according to Martin
Brändle, the firm's senior vice president of corporate communications
and investor relations. The demand has been so high, he said, that Tecan
has tapped into an emergency stash, and is racing to install new
production lines that he hopes will double the company's output by fall.

Pipette tips aren't the only laboratory items in short supply. Dwindling
stocks of machines, containers and chemicals needed to extract or
amplify the coronavirus's genetic material have clogged almost every
point along the testing workflow.

The crisis is an eerie echo of the early days of the pandemic, when
researchers scrambled to find the swabs and liquids needed to collect
and store samples en route to laboratories.

``It's like Groundhog Day,'' said Scott Shone, director of the North
Carolina State Laboratory of Public Health. ``I feel like I lived this
day four or five months ago.''

In New York, researchers running low on chemicals are running machines
at half capacity as test specimens pile up at the door. In Florida,
where cases are spiking, labs are reporting turnaround times of seven to
10 days.

\hypertarget{latest-updates-the-coronavirus-outbreak}{%
\section{\texorpdfstring{\href{https://www.nytimes3xbfgragh.onion/2020/09/11/world/covid-19-coronavirus.html?action=click\&pgtype=Article\&state=default\&region=MAIN_CONTENT_1\&context=storylines_live_updates}{Latest
Updates: The Coronavirus
Outbreak}}{Latest Updates: The Coronavirus Outbreak}}\label{latest-updates-the-coronavirus-outbreak}}

Updated 2020-09-12T06:16:33.399Z

\begin{itemize}
\tightlist
\item
  \href{https://www.nytimes3xbfgragh.onion/2020/09/11/world/covid-19-coronavirus.html?action=click\&pgtype=Article\&state=default\&region=MAIN_CONTENT_1\&context=storylines_live_updates\#link-dfb8a16}{Fauci
  cautions the virus could disrupt life in the U.S. until `maybe even
  towards the end of 2021.'}
\item
  \href{https://www.nytimes3xbfgragh.onion/2020/09/11/world/covid-19-coronavirus.html?action=click\&pgtype=Article\&state=default\&region=MAIN_CONTENT_1\&context=storylines_live_updates\#link-7104d154}{From
  Asia to Africa, China promotes its vaccine candidates to win friends.}
\item
  \href{https://www.nytimes3xbfgragh.onion/2020/09/11/world/covid-19-coronavirus.html?action=click\&pgtype=Article\&state=default\&region=MAIN_CONTENT_1\&context=storylines_live_updates\#link-393ad215}{The
  other way the virus will kill: hunger.}
\end{itemize}

\href{https://www.nytimes3xbfgragh.onion/2020/09/11/world/covid-19-coronavirus.html?action=click\&pgtype=Article\&state=default\&region=MAIN_CONTENT_1\&context=storylines_live_updates}{See
more updates}

More live coverage:
\href{https://www.nytimes3xbfgragh.onion/live/2020/09/11/business/stock-market-today-coronavirus?action=click\&pgtype=Article\&state=default\&region=MAIN_CONTENT_1\&context=storylines_live_updates}{Markets}

And in New Mexico, researchers at TriCore Reference Laboratories ---~the
state's largest medical laboratory --- have revved up testing in the
days after deliveries arrive, only to find themselves hamstrung by
faltering supplies at week's end.

``It's a merry-go-round of shortages,'' said Karissa Culbreath, the
laboratory's scientific director of infectious disease, research and
development. ``Just when we think we've dealt with one issue, another
challenge pops up.''

\includegraphics{https://static01.graylady3jvrrxbe.onion/images/2020/07/23/science/23VIRUS-TESTINGSHORTAGES3/23VIRUS-TESTINGSHORTAGES3-articleLarge.jpg?quality=75\&auto=webp\&disable=upscale}

TriCore and many other laboratories are now having to prioritize testing
for the sickest patients, a trend that has troubled many as evidence
mounts of the
\href{https://www.nytimes3xbfgragh.onion/2020/06/27/world/europe/coronavirus-spread-asymptomatic.html}{virus's
ability to spread from infected people} before symptoms appear, if they
do at all. For months, experts have underscored the need for more
widespread testing, particularly among elderly people and the most
vulnerable racial and ethnic groups, to slacken the coronavirus's grip
on the nation.

In interviews, public and private lab staffers in a dozen states said
they were exhausted from marathon days of running tests on a shoestring
supply chain. Some are regularly working 12-hour days. Others are taking
overnight shifts to babysit machines running never-ending batches of
tests at full capacity.

``I've come in at 4 a.m., I've come in at 3 a.m.,'' said Felicia Rice, a
laboratory technologist who conducts coronavirus tests at Mayo Clinic
Arizona, where local demand has skyrocketed in lock step with
\href{https://www.nytimes3xbfgragh.onion/interactive/2020/us/arizona-coronavirus-cases.html}{the
recent crest in cases}.

More than 20 percent of the 72 institutions recently surveyed by the
Association of Public Health Laboratories have said they will run out of
at least one item required to do their tests within a week. About as
many **** said they were
\href{https://www.aphl.org/programs/preparedness/Crisis-Management/COVID-19-Response/Pages/COVID-19-Dashboard.aspx}{unable
to meet current testing needs}.

Soaring demand has also
\href{https://www.acla.com/covid-19/}{exacerbated} capacity issues at
private laboratories such as LabCorp and Quest Diagnostics, which
\href{https://www.labcorp.com/coronavirus-disease-covid-19/labcorp-newsroom}{together
have performed}
\href{https://newsroom.questdiagnostics.com/COVIDTestingUpdates}{one-third
of the nation's tests}.

Wait times for test results from both companies have ballooned to
several days --- in some cases stretching well over a week.

David Rohlfing, who took one of Quest's tests at a walk-in site in
Queens on July 6, said he still didn't have his results 17 days later.
That's far longer than the four days he waited the last time he was
tested at the same site, in early June. If he gets a negative result, it
won't help much, since he could have been exposed in the interim.

``If no one is getting test results,'' Mr. Rohlfing said, ``we do not
actually know how the opening up is going here.''

After a person's specimen is collected at a hospital, clinic or
community testing site, it can go on to be processed in a dizzying bevy
of places. Some are run in
\href{https://www.aphl.org/aboutAPHL/Pages/aboutphls.aspx}{public health
laboratories} operated by governments at the federal, state or local
level. Many of these facilities have been somewhat buffered from
shortages by their access to the Centers for Disease Control and
Prevention's
\href{https://www.internationalreagentresource.org/}{International
Reagent Resource}, which maintains stocks of supplies necessary to run
the agency's
\href{https://www.cdc.gov/coronavirus/2019-ncov/lab/testing.html}{in-house
coronavirus tests}.

But most state public health labs are not set up to perform diagnostics
en masse, Dr. Shone said. ``We're here for the initial emergency
response, then the clinical lab system of the country typically takes on
the lion's share of testing.''

As demand ratchets up, it's these commercial labs that have been left in
a lurch. In a
\href{https://newsroom.questdiagnostics.com/COVIDTestingUpdates}{statement}
released on July 20, Quest noted that the dearth of equipment and
chemicals comprised ``the most significant gating factor'' in its
testing pipeline. And members of the American Clinical Laboratory
Association, a group that represents many of the country's private labs,
have lamented the spotty availability of materials like chemicals and
pipette tips, Julie Khani, the organization's president, said in an
email.

``Any one constriction in the chain of supply can suddenly create a
bottleneck,'' Ms. Khani said. ``With more supplies and platforms, we
could perform more testing, but the global supply chain remains
constrained.''

Sputtering supply chains have started to shunt the onus of testing from
some private laboratories to their public counterparts, Peter Iwen,
director of Nebraska's Public Health Laboratory, said in an email. ``We
are now getting backlogged and will need to start rejecting specimens,''
he said.

Image

A drive-through testing site in Jericho, N.Y., in March.Credit...Johnny
Milano for The New York Times

In California, where the number of new cases has
\href{https://www.nytimes3xbfgragh.onion/interactive/2020/us/california-coronavirus-cases.html}{surged
above 10,000 per day}, regional public health laboratories, like the one
in Sonoma County, are fighting tooth and nail to keep pace. ``We've been
over capacity for a long time,'' said Rachel Rees, the institution's
director of laboratory services. Dr. Rees's lab is currently processing
samples from local hospitals that have run out of supplies.

To avoid halting testing entirely, many laboratories are maintaining
stocks to run multiple types of tests at once --- requiring technicians
to maintain both the materials and mental wherewithal to perform many
protocols, often at the same time. Researchers at Mayo Clinic Arizona
must juggle four or five protocols; at TriCore, in New Mexico, that
number has soared to seven.

This sort of bet-hedging wasn't the norm for laboratories before, said
Omai Garner, the director of clinical microbiology for the U.C.L.A.
Health System, where he runs a laboratory of more than 100 people.

``No single manufacturer can give a laboratory enough tests to cover the
entire volume they need to cover,'' said Dr. Garner, who is in the
process of adding a fifth type of coronavirus test to his team's
repertoire.

Shortages are so widespread that even backup options don't always pan
out.

Marilyn Freeman, who is deputy director of Virginia's D.C.L.S. public
health laboratory, said her team had been waiting months for its orders
of machines that can automate coronavirus test processing, which would
ease the burden on staff. Two of the devices in highest demand --- the
Hologic Panther and Hologic Panther Fusion, the same ultraefficient
robots that take Tecan's sought-after pipette tips --- most likely won't
ship to Dr. Freeman's lab until the fall.

What's more, some of the biggest issues from the early days of the
pandemic haven't yet resolved. Erin Graf, who regularly clocks 80-hour
weeks as the director of microbiology at Mayo Clinic Arizona, said her
laboratory was still strained by an inconsistent supply of the
specialized swabs needed to collect specimens --- an added stress on top
of the new round of obstacles her team is contending with.

``We're used to dealing with challenges. We welcome challenges,'' Dr.
Graf said. ``But it feels like the challenge is coming almost daily
now.''

As fall approaches, many researchers are growing increasingly worried
that the flu season will exacerbate shortages. The coronavirus isn't the
only pathogen circulating through the human population, or the only
infection that laboratories need tools to test for. Though the C.D.C.
and many private companies are currently developing tests that can
\href{https://www.cdc.gov/coronavirus/2019-ncov/lab/multiplex.html}{detect
multiple pathogens at once}, the sheer volume of autumn illnesses is
still expected to hit labs hard --- and may force some teams to delay
testing for other infections.

Already, labs like Dr. Graf's have had to cut corners with testing for
sexually transmitted infections, in part because several manufacturers
have had to pivot supply chains toward coronavirus testing. ``Some of
the most basic tests that we do, we can't do anymore,'' Dr. Graf said.
``Every resource is going toward Covid. That's something we never
would've thought would happen.''

Shifts toward point-of-care coronavirus tests, which are
\href{https://www.nytimes3xbfgragh.onion/2020/07/06/health/fast-coronavirus-tests.html}{fast
and simple enough to perform without the need for specialized
equipment}, could ease some of the burden on laboratories. Pooled
testing for the coronavirus, in which samples from multiple people are
\href{https://www.nytimes3xbfgragh.onion/2020/07/01/health/coronavirus-pooled-testing.html}{combined
and analyzed in batches}, could cut down on material consumption as
well. But these tests are not yet in widespread use, and depend on many
of the same manufacturing pipelines.

In New Mexico, TriCore's Dr. Culbreath worries that her next big
shortage may be the laboratory's most valuable supply of all: its
people.

``I worry about my own staff, and burnout. Their ability to take care of
themselves,'' said Dr. Culbreath, who has pulled many weekend shifts and
10-hour days.

Eventually, she'll ``find someone to manufacture a plastic pipette
tip,'' she said. ``But I can't find someone with the years of training
and certification of these amazing scientists.''

Advertisement

\protect\hyperlink{after-bottom}{Continue reading the main story}

\hypertarget{site-index}{%
\subsection{Site Index}\label{site-index}}

\hypertarget{site-information-navigation}{%
\subsection{Site Information
Navigation}\label{site-information-navigation}}

\begin{itemize}
\tightlist
\item
  \href{https://help.nytimes3xbfgragh.onion/hc/en-us/articles/115014792127-Copyright-notice}{©~2020~The
  New York Times Company}
\end{itemize}

\begin{itemize}
\tightlist
\item
  \href{https://www.nytco.com/}{NYTCo}
\item
  \href{https://help.nytimes3xbfgragh.onion/hc/en-us/articles/115015385887-Contact-Us}{Contact
  Us}
\item
  \href{https://www.nytco.com/careers/}{Work with us}
\item
  \href{https://nytmediakit.com/}{Advertise}
\item
  \href{http://www.tbrandstudio.com/}{T Brand Studio}
\item
  \href{https://www.nytimes3xbfgragh.onion/privacy/cookie-policy\#how-do-i-manage-trackers}{Your
  Ad Choices}
\item
  \href{https://www.nytimes3xbfgragh.onion/privacy}{Privacy}
\item
  \href{https://help.nytimes3xbfgragh.onion/hc/en-us/articles/115014893428-Terms-of-service}{Terms
  of Service}
\item
  \href{https://help.nytimes3xbfgragh.onion/hc/en-us/articles/115014893968-Terms-of-sale}{Terms
  of Sale}
\item
  \href{https://spiderbites.nytimes3xbfgragh.onion}{Site Map}
\item
  \href{https://help.nytimes3xbfgragh.onion/hc/en-us}{Help}
\item
  \href{https://www.nytimes3xbfgragh.onion/subscription?campaignId=37WXW}{Subscriptions}
\end{itemize}
