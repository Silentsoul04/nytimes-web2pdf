Sections

SEARCH

\protect\hyperlink{site-content}{Skip to
content}\protect\hyperlink{site-index}{Skip to site index}

\href{https://myaccount.nytimes3xbfgragh.onion/auth/login?response_type=cookie\&client_id=vi}{}

\href{https://www.nytimes3xbfgragh.onion/section/todayspaper}{Today's
Paper}

A Portrait of an Artist Not to Be Underestimated

\url{https://nyti.ms/3jEjJTM}

\begin{itemize}
\item
\item
\item
\item
\item
\end{itemize}

Advertisement

\protect\hyperlink{after-top}{Continue reading the main story}

Supported by

\protect\hyperlink{after-sponsor}{Continue reading the main story}

True Believers

\hypertarget{a-portrait-of-an-artist-not-to-be-underestimated}{%
\section{A Portrait of an Artist Not to Be
Underestimated}\label{a-portrait-of-an-artist-not-to-be-underestimated}}

The painter Howardena Pindell discusses her 1990 work ``Scapegoat.''

\includegraphics{https://static01.graylady3jvrrxbe.onion/images/2020/07/13/t-magazine/13tmag-pindell/13tmag-pindell-articleLarge.jpg?quality=75\&auto=webp\&disable=upscale}

By Howardena Pindell

\begin{itemize}
\item
  July 24, 2020
\item
  \begin{itemize}
  \item
  \item
  \item
  \item
  \item
  \end{itemize}
\end{itemize}

\emph{In this new series, The Artists, an installment of which will
publish every day this week and regularly thereafter, T will highlight a
recent or little-shown work by a Black artist, along with a few words
from that artist, putting the work into context. Today, we're looking at
a piece by Howardena Pindell, a painter and mixed-media artist whose
work explores themes of racism, sexism and xenophobia.}

\href{https://www.nytimes3xbfgragh.onion/issue/t-magazine/2020/07/02/true-believers-art-issue}{\includegraphics{https://static01.graylady3jvrrxbe.onion/newsgraphics/2020/06/29/tmag-art-embeds-new/assets/images/art_issue_gif_special_editon.gif}}

\textbf{Name:} \href{https://www.howardenapindell.org/}{Howardena
Pindell}

\textbf{Age:} 77

\textbf{Based in:} New York City

\textbf{Originally from:} Philadelphia

\textbf{When and where did you make this work?} In my studio in New York
City. I moved out of my loft to a large apartment on account of the rent
increase in 1987. I had a loft in SoHo on Broome Street. I worked on it
in the room I now use as a bedroom. My ceilings are about 10 feet high,
**** and I was able to install tract lighting.

\textbf{Can you describe what's going on in it?} It is part of my
``Autobiography'' series. (It is in the collection of the
\href{https://www.nytimes3xbfgragh.onion/topic/organization/studio-museum-in-harlem}{Studio
Museum in Harlem}.) I have four portraits of myself, one as a little
child holding a ball, which I find interesting since I use the circle,
and one with the foot of my ex-boss at the Modern stepping on my head
{[}Pindell worked at the Museum of Modern Art from 1967 to 1979{]}. She
was not pleasant. I lay down on the canvas and cut out my figure and put
in realistic portrait heads, which I painted from photographs of me at
various ages. It also deals with issues of **** racism in the upper
right area where there is a target. I enjoy using text and images. I
think my childhood exposure to ancient Egyptian art, in which text is
image and image is text, **** influenced me. A family friend said that
there was a mummy in the
\href{https://www.nytimes3xbfgragh.onion/topic/organization/philadelphia-museum-of-art}{Philadelphia
Museum} that looked like me. I was taken to the museum as a result. It
was a Fayum, an encaustic mummy with the portrait painted on a
linen-like cloth wrapping. I went to see the pyramids, in Luxor and
Thebes, and the tombs in the Valley of the Kings in 1974.

\textbf{What inspired you to make this work?} I had been in a car
accident in 1979 as a passenger and sustained a head injury. After that,
I moved from abstraction to works that **** had a personal narrative and
in some cases dealt with issues of racism and women's rights. One
painting in the series dealt with wife-burning in India. It was seen
many years ago as an honor to jump onto one's husband's burning funeral
pyre. It apparently still happens in more remote areas, although it has
been outlawed. The wife's hands were traced onto the Hindu temple's ****
outside walls. I **** lived in India for **** about four months and have
lived in the desert. I also have friends there.

\textbf{What's the work of art in any medium that changed your life?} I
feel that my transition **** from oil to acrylic changed my life. I was
trained as a figurative painter and used oil. I went to Yale's School of
Art and Architecture for my M.F.A. and was very influenced by the range
of work I saw there. After I graduated, my work gradually became more
abstract. I liked manipulating textures that I could get with acrylic.
Oil takes a year to completely dry and can crack. Acrylic can crack if
you freeze it, and you cannot heat it as the fumes are toxic. You can
heat up oil paint with wax to create works that are **** encaustic, just
do not hang it over the fireplace.

\hypertarget{true-believers-art-issue}{%
\subsubsection{\texorpdfstring{\href{https://www.nytimes3xbfgragh.onion/issue/t-magazine/2020/07/02/true-believers-art-issue}{True
Believers Art
Issue}}{True Believers Art Issue}}\label{true-believers-art-issue}}

Advertisement

\protect\hyperlink{after-bottom}{Continue reading the main story}

\hypertarget{site-index}{%
\subsection{Site Index}\label{site-index}}

\hypertarget{site-information-navigation}{%
\subsection{Site Information
Navigation}\label{site-information-navigation}}

\begin{itemize}
\tightlist
\item
  \href{https://help.nytimes3xbfgragh.onion/hc/en-us/articles/115014792127-Copyright-notice}{©~2020~The
  New York Times Company}
\end{itemize}

\begin{itemize}
\tightlist
\item
  \href{https://www.nytco.com/}{NYTCo}
\item
  \href{https://help.nytimes3xbfgragh.onion/hc/en-us/articles/115015385887-Contact-Us}{Contact
  Us}
\item
  \href{https://www.nytco.com/careers/}{Work with us}
\item
  \href{https://nytmediakit.com/}{Advertise}
\item
  \href{http://www.tbrandstudio.com/}{T Brand Studio}
\item
  \href{https://www.nytimes3xbfgragh.onion/privacy/cookie-policy\#how-do-i-manage-trackers}{Your
  Ad Choices}
\item
  \href{https://www.nytimes3xbfgragh.onion/privacy}{Privacy}
\item
  \href{https://help.nytimes3xbfgragh.onion/hc/en-us/articles/115014893428-Terms-of-service}{Terms
  of Service}
\item
  \href{https://help.nytimes3xbfgragh.onion/hc/en-us/articles/115014893968-Terms-of-sale}{Terms
  of Sale}
\item
  \href{https://spiderbites.nytimes3xbfgragh.onion}{Site Map}
\item
  \href{https://help.nytimes3xbfgragh.onion/hc/en-us}{Help}
\item
  \href{https://www.nytimes3xbfgragh.onion/subscription?campaignId=37WXW}{Subscriptions}
\end{itemize}
