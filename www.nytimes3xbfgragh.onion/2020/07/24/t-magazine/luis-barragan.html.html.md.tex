Sections

SEARCH

\protect\hyperlink{site-content}{Skip to
content}\protect\hyperlink{site-index}{Skip to site index}

\href{https://myaccount.nytimes3xbfgragh.onion/auth/login?response_type=cookie\&client_id=vi}{}

\href{https://www.nytimes3xbfgragh.onion/section/todayspaper}{Today's
Paper}

Luis Barragán's Forgotten Works, Revisited

\url{https://nyti.ms/3hybVRI}

\begin{itemize}
\item
\item
\item
\item
\item
\item
\end{itemize}

Advertisement

\protect\hyperlink{after-top}{Continue reading the main story}

Supported by

\protect\hyperlink{after-sponsor}{Continue reading the main story}

True Believers

\hypertarget{luis-barraguxe1ns-forgotten-works-revisited}{%
\section{Luis Barragán's Forgotten Works,
Revisited}\label{luis-barraguxe1ns-forgotten-works-revisited}}

After moving to Mexico City in 1935, the architect set about designing a
series of obscure functionalist residences that he would later disown.

\includegraphics{https://static01.graylady3jvrrxbe.onion/images/2020/07/13/t-magazine/13tmag-barragan-slide-MBMV/13tmag-barragan-slide-MBMV-articleLarge.jpg?quality=75\&auto=webp\&disable=upscale}

By Suleman Anaya

\begin{itemize}
\item
  July 24, 2020
\item
  \begin{itemize}
  \item
  \item
  \item
  \item
  \item
  \item
  \end{itemize}
\end{itemize}

\href{https://www.nytimes3xbfgragh.onion/2014/06/15/travel/finding-mexico-city-and-luis-barragan-again.html}{LUIS
BARRAGÁN}'S INCLUSION in the pantheon of the 20th century's most
influential architects rests on a strikingly limited output: foremost
his own house and studio in the west of Mexico City, a UNESCO World
Heritage site, followed by a handful of standout residences created
after 1945 for wealthy clients. The reception of Barragán's work is
similarly reduced to a concise class of qualities: In the global
imagination, his architecture became synonymous with evocatively vague
notions of silence, mystery, serenity and thick walls in sensual colors
considered to be redolent of some absolute sense of Mexican tradition.

No less an authority than
\href{https://www.nytimes3xbfgragh.onion/1998/04/21/books/octavio-paz-mexico-s-man-of-letters-dies-at-84.html}{Octavio
Paz}, the Mexican writer and Nobel laureate, summed up this reputation
in 1980, on the occasion of Barragán winning the Pritzker,
architecture's top prize: ``The art of Barragán is modern but not
modernist \ldots{} His architecture was inspired by two words: the word
magic and the word surprise \ldots{} The roots of his art are
traditional and popular \ldots{} stemming from Mexican pueblos where
walls are painted in vivid colors --- reds, ochres, blues --- unlike
those of Moorish and Mediterranean towns which are painted white.'' If
the encyclopedic mind of Paz, known for nuanced assessments, could help
cement a selective, idealized version of facts around Barragán, why
wouldn't everyone else blithely accept this new, more streamlined
historiography?

\href{https://www.nytimes3xbfgragh.onion/issue/t-magazine/2020/07/02/true-believers-art-issue}{\includegraphics{https://static01.graylady3jvrrxbe.onion/newsgraphics/2020/06/29/tmag-art-embeds-new/assets/images/art_issue_gif_special_editon.gif}}

In 1931, Barragán, a then unknown architect from Guadalajara, traveled
for the second time to Europe, where he visited several recent projects
by
\href{https://www.nytimes3xbfgragh.onion/topic/person/le-corbusier}{Le
Corbusier}, including the Villa Savoye in Poissy, France. In notes from
that trip, Barragán described the paradigmatic residence, which
epitomized Le Corbusier's radical theories for a new international
architecture --- characterized by whitewashed, rational ``machines for
living'' with flat, terraced roofs, purist forms and long horizontal
openings --- as ``very modern, like a beautiful sculpture.'' The young
Barragán, who was deeply affected by the Swiss father of Modernist
architecture, does not fit so tidily into today's prevalent reading of
him as the author of introverted, almost fortified domestic sanctuaries
known for their rich color schemes and locally inspired, joyfully
inefficient touches.

\includegraphics{https://static01.graylady3jvrrxbe.onion/images/2020/07/13/t-magazine/13tmag-barragan-slide-W3JD/13tmag-barragan-slide-W3JD-articleLarge.jpg?quality=75\&auto=webp\&disable=upscale}

But in fact, traces of Le Corbusier's influence would remain present
throughout Barragán's oeuvre. He starts incorporating Corbusian elements
here and there upon his return to Guadalajara, where until now his work
had consisted of Spanish-looking houses with round-arched openings,
rustic woodwork and other distinctly pre-Modern details. Nods to the
European master can even be found, albeit in more subtle manifestations,
in the Mexican's late heroic houses --- the famous floating staircase at
Barragán's own home, which he moved into in 1947, had its obvious
precursor on the roof terrace of a Champs-Élysées penthouse Le Corbusier
designed for a rich client. But Barragán's interest in Corbusian ideas
is nowhere more evident than in a seminal body of work he created in the
immediate years following his move to Mexico City in 1935.

For the first five years after arriving in the booming capital, where he
hoped to improve his prospects and would later stake his reputation,
Barragán designed almost two dozen apartment buildings and houses in
up-and-coming neighborhoods. Sometimes called Barragán's functionalist
years, these works have become unfairly forgotten footnotes in his
storied career. Barragán distanced himself from his early Mexico City
output. In a telling 1962 interview, he refers to his creations from
this period as ``edificitos'' (little buildings), ``nothing great.''

Last fall, I traveled to Mexico City to look at this unspoken corner of
Barragán. What could these buildings --- to the extent that they
survived --- tell us about the genesis of Barragán's mature phase that
followed? Were they really as insignificant as their hidden condition
suggests?

Not all of these buildings are masterpieces. A rental project Barragán
designed for his brother lacks the attention to detail and emotional
resonance of the rest of his work, its only point of interest a little
roof terrace featuring an unglazed stripe window to frame distant
mountains. A heavily modified apartment building on Calle Estocolmo,
where the architect doubled as landlord, is similarly anodyne. But most
of them contain elements --- a meticulously modulated staircase,
strategically placed skylights, in some cases just a simple,
unnecessarily elegant metal mail slot --- that speak to Barragán's
genius for imbuing space with wonder and enveloping even the most
pragmatic projects in a thought-out sort of invisible parallel function:
to provide the user with the most agreeable spatial experience possible.

Visiting these often unassuming buildings, one senses the architect's
inner conflicts and his unwillingness to compromise, endowing even the
most prosaic of works with extraordinary angles, emotionally affecting
progressions between rooms, abundant natural light and a wealth of other
sensory gratifications that no one asked from him, least of all the
people who employed him at this stage of his career.

Image

On the restored facade of Parque Melchor Ocampo 38, contrasting dark
\emph{recinto} stone on street level and an extremely light gray hue
chosen for the upper floors emphasize the purist, two-dimensional
appearance of Barragán and Max Cetto's design.Credit...Nin Solis

Barragán's most important work from this period, Parque Melchor Ocampo
38, in the neighborhood known as Colonia Cuauhtémoc, has recently
undergone a sensitive yet liberal restoration in the hands of
\href{http://www.vrtical.mx/about/}{Luis Beltrán del Río and Andrew
Sosa}, two of the young architects that are remaking the erstwhile
neighborhoods of Mexico City's bourgeoisie for a new generation. Even
before this, Melchor Ocampo 38 was the most interesting building
Barragán designed during this early period, mostly for its striking
Cubist appearance on the outside. The building is also noteworthy for
its illustrious inhabitants, among them the artist
\href{https://www.philamuseum.org/exhibitions/2008/285.html?page=2}{Juan
Soriano} and the Cuban-born designer Clara Porset, whose furniture
designs were part of a
\href{https://www.artic.edu/exhibitions/9198/in-a-cloud-in-a-wall-in-a-chair-six-modernists-in-mexico-at-midcentury}{recent
exhibition} at the Art Institute of Chicago
(\href{https://www.fundacionjumex.org/en/exposiciones/189-clara-porset-diseno-y-pensamiento}{another}
Porset show, this one focused on her writings, opened at Mexico City's
Museo Jumex on March 7).
\href{https://www.nytimes3xbfgragh.onion/2015/12/20/travel/pablo-neruda-chile.html}{Pablo
Neruda} and
\href{https://lens.blogs.nytimes3xbfgragh.onion/2017/08/24/tina-modotti-edward-weston-photography/}{Tina
Modotti} are said to have visited at this address.

Porset and her husband, the painter Xavier Guerrero, lived and worked in
one of Melchor Ocampo's four apartments for close to three decades. It's
likely it was here that Porset designed the Butaque chair that now sells
for upward of \$10,000, and a leading Porset scholar told me the
couple's apartment was physically surveilled by the F.B.I. in the 1950s
and '60s because of their Communist affiliations. Adding to its mystery,
Melchor Ocampo 38 forms part of a block of landmark Modernist buildings
that has somehow managed to withstand the turmoil surrounding them ---
earthquakes, traffic, corruption --- relatively intact, as if frozen in
time.

ONE OF MEXICO CITY'S central neighborhoods, the Colonia Cuauhtémoc is of
exceptional architectural significance. Developed to a great extent in
the 1940s, it is bordered to the south by Paseo de la Reforma, the
boulevard once lined by stately mansions that have gradually been
replaced by ever-taller office towers. While it lost some of its
cosmopolitan feel to a transient office population, the area retains
some of the discrete, slightly gloomy character that has always made it
a favorite of architects and intellectuals. Octavio Paz lived in the
area almost until his death, in 1998, as did the Swiss architect (and
second director of the Bauhaus) Hannes Meyer during his Mexican years
(from 1939 to 1949). Wondrously, the streets of Cuauhtémoc are littered
with early buildings by Modernist masters ---
\href{https://www.moma.org/artists/62771}{José Creixell}, Mario Pani and
Enrique del Moral, to name a few.

Image

Barragán and Cetto's building, shown here in the middle of the curved
block in 1942, forms part of an exemplary urban ensemble by some of
Mexico's leading architects of the mid-20th century. The street also
showcases early works by the largely forgotten Modernist masters Enrique
del Moral --- whose prow-like design can be seen in the foreground ---
and José Creixell, with whom Barragán designed the apartment building on
the opposite end, just to the left of Melchor Ocampo 38.Credit...Peter
Stackpole/The LIFE Picture Collection via Getty Images

It was here that, beginning in 1939, Barragán designed Melchor Ocampo 38
for a pair of sisters, Carmen and Paz Orozco, about whom little is known
besides the fact that the architect had already designed a since
demolished house for one of them in Guadalajara. From the onset, Melchor
Ocampo 38 was intended to contain four studio apartments for painters.
It is possible that the idea for the building --- and its strangely
specific purpose --- was Barragán's, and he somehow convinced the
sisters that it would be a good investment and source of income for
them.

But Barragán didn't design Melchor Ocampo 38 alone. A frequent corollary
of the Barragán myth is the assumption that he created without help. In
fact, throughout his career, he relied on a series of collaborators,
business partners and creative friends who served as soundboards and
executors of his vision, but also often gave him ideas he wouldn't have
had without their input, shaping his work in significant ways. Most
notably, Barragán's acclaimed sense of color and use of colonial objects
and folk art as counterpoints to modern spaces was directly indebted to
his close relationship to the artist and antiquarian Jesús ``Chucho''
Reyes.

Image

The entrance to Melchor Ocampo 38 with the original signage. The
building was also known as the Four Painters' Studios because of the
specific function for which it was conceived.Credit...Nin Solis

Image

The interior of one of the four studio apartments at Melchor Ocampo 38.
Barragán and Cetto achieved an extraordinary quality of space and light
on a compact, irregular plot. The recent restoration preserved many
period details, including the original pine wood floors.Credit...Nin
Solis

In the case of the Four Painters' Studios, as Melchor Ocampo 38 is known
among architecture historians, Barragán shared design responsibilities
with Max Cetto, a German émigré whose contribution to mid-20th-century
Mexican architecture culture has yet to be fully recognized. The
Koblenz-born architect had just arrived in Mexico, likely recommended by
\href{https://www.nytimes3xbfgragh.onion/1970/04/18/archives/richard-neutra-architect-dies-helped-shape-modern-outlook.html}{Richard
Neutra}, with whom he had worked in California. Before that, Cetto
studied under the Expressionist
\href{https://www.nytimes3xbfgragh.onion/1936/06/16/archives/hans-poelzig-dead-german-architect-placed-in-background-by-nazis.html}{Hans
Poelzig} in Berlin and was part of Ernst May's groundbreaking New
Frankfurt affordable-housing initiative. (Also a vocal critic of the
Nazi regime, in 1933 Cetto penned a letter to Joseph Goebbels that
remains a fascinating document of creative political engagement.) In
Mexico, Cetto's varied training and personal ideology alchemized into an
unusual appreciation for craftsmanship, site, local natural building
materials and the visible hand of his adopted country's highly skilled
manual labor.

Among many things that remain puzzling about Barragán and Cetto's
Melchor Ocampo project, strangest may be the choice to develop an
impractical piece of land for the most impractical use imaginable. This
didn't stop the two architects from investing an extraordinary level of
thought and detail in the building. Faced with a small, irregularly
shaped site, they devised a parti of astounding complexity. Rather than
standardize the unwieldy plot, the architects decided to match its
irregularity: The four apartments are stacked in two pairs on each side,
with two different floor plans per level and services clustered with
Teutonic efficiency around a central well that contains the communal
terrazzo stairs. Indeed, the strongest influence, besides Le Corbusier,
seems to be Germany's prototypical housing estates of the 1920s, where a
modern sensibility of space and living were combined with a pronounced
emphasis on optimization.

Image

The spiral staircase leading up to the \emph{tapanco}, a mezzanine in
each apartment intended as a bedroom. The volcanic stone steps have been
replaced to match the original design.Credit...Nin Solis

Image

In each apartment, a large \emph{ventanal}, characterized by a grid of
slender steel mullions, floods the double-height space with light, while
old trees keep the city's chaos and traffic from sight.Credit...Nin
Solis

If such an elaborate layout is unexpected in so small a space, the
details were equally nonstandard, from custom cabinets to invisible
golden ratios and the uncanny fact that the building contains almost no
right angles. The stairwell alone is a symphony of jagged corners, as
Barragán and Cetto sculpted the stone to appear dynamic, enhancing the
effect by subtly but precisely deploying shadows and small optical
illusions at every turn.

Still, the pair saved Melchor Ocampo 38's double pièce de résistance for
the inside of every apartment: Upon entering, a small vestibule,
deliberately compressed on all sides, opens up unexpectedly to a
double-height space dominated by a single large frame-like window
articulated with a grid of slender mullions. On most days, the
north-oriented \emph{ventanal} bathes the studio in an inordinate amount
of sunlight, making it feel twice the size it actually is. When the
building was completed, it faced open fields, a situation that has
radically changed. Still, the positioning of the windows manages to
erase the urban chaos outside, and the main view is the abstract
greenery of tree crowns. The sculptural spiral stairs --- cast in
concrete with volcanic rock steps --- that lead to the mezzanine are
another highlight, and a Cetto trademark.

On the outside, like the rest of the block, the building bends softly to
follow the edge of the park that it gets its name from, while its
asymmetric inner logic is hinted at in the purist, switchboard-like
front, a play of voids and solids dominated by the four large windows.
In line with Barragán's lifelong love of two-dimensional abstractions of
his work, the facade reads as an autonomous form as much as it does a
diagram of what is behind it.

AS MEXICO CITY has found itself in the middle of another wave of
unbridled construction, a lot of it speculative and poorly regulated,
it's miraculous any of the early Modernist buildings in Colonia
Cuauhtémoc survive. With a thriving real-estate market, investors have
been buying up entire swaths of buildings in historic Colonias that
trace the evolution of Mexican society and its design tastes.

Melchor Ocampo 38 illustrates the dilemma the booming Mexican capital
faces two decades into the 21st century. Overburdened with physical
riches spanning seven centuries, chronically lacking in resources and
systemically bogged down by bureaucracy and corruption, the overdue
rehabilitation of its Modernist heritage both poses a strain and isn't
an official priority. Any real chance to preserve these valuable
buildings depends on the good will of investors, who, in most cases, are
buying them for profit, not out of civic duty.

Image

A hand-drawn reproduction of the Melchor Ocampo 38 floor plans published
in Susanne Dussel's book ``Max Cetto, 1903-1980: Arquitecto
Mexicano-Alemán,'' shows the building's complex inner logic and
ultra-efficient layouts.Credit...Courtesy of Susanne Dussel. Original
plans courtesy of Archivo Max Cetto, UAM-Azc.

In the hands of the wrong buyers or architects, Melchor Ocampo 38 could
have been lost. As it stands, its exterior is newly radiant, clearly
recognizable as Barragán and Cetto's work, while the inside spaces are
also largely preserved, with the exception of minor contemporary
modifications, including new, decidedly 21st-century baths and kitchens.

The property manager says that from time to time a guest rents an
apartment in the building specifically for its architectural pedigree,
but more frequently, people --- young professionals, often foreign ---
are simply drawn to Melchor Ocampo's prime location and its airy,
light-filled interior, whose design remains conspicuously modern,
especially considering the building's age.

Seen from any angle, Melchor Ocampo 38 is revelatory. It proves that
even at his most commercial, Barragán was trying out essential hallmarks
of what would become his signature vocabulary: scenic framing, dramatic
changes in scale and other minimal gestures with maximum impact, all
while displaying unusual brilliance in handling space, light and volume
with a poet's precision and, perhaps above all, towering ambition.

So why has his early Mexico City work effectively been denied, and why
does most of it remain stuck in neglected anonymity? It's easy to assume
Barragán, who would edit his Wikipedia entry from his grave if he could,
wanted it this way. It may be more that \emph{we} have wanted it this
way.

One reason, perhaps, is **** that to talk about this phase of Barragán,
or really to talk honestly about any phase of Barragán's productivity,
means to acknowledge him as a visionary salesman as well as a
prodigiously gifted architect. The myth of Barragán often tends to leave
out his sharp entrepreneurial instincts. In truth, the monk-like
aesthete was also an avid businessman who engaged in speculative
real-estate development for most of his career and made no secret of it.
Even his greatest creative and aesthetic success, the exclusive
residential subdivision known as Jardines del Pedregal de San Ángel ---
envisioned in 1945 as a collection of Modernist homes designed to both
complement and contrast with the native vegetation and rock formations
of a millenary lava field --- was conceived of by Barragán as a business
opportunity.

Barragán didn't discover El Pedregal, which had enchanted travelers and
artists before him for its dramatic, purplish-black wilderness, but he
was the first to realize its commercial potential through a highly
refined \emph{Gesamtplan}, which encompassed selling it to the right
people before it even existed. Barragán cocreated (with Cetto) the
initial template for an innovative type of residence that integrated
signifiers of modern affluence and high-end architecture with an unusual
respect for the existing landscape, and oversaw the development's
defining design details --- high walls, winding roads that followed the
natural terrain, de Chirico-like plazas --- which together converted the
inhospitable terrain into one of the world's most spectacular
residential enclaves. But his achievement consisted just as much in
finding the right business partners to execute his brilliant
bigger-picture vision: To purchase inexpensive land with the intention
of selling it for a profit after dividing it into large parcels and
maximizing their perceived value through an elaborate promotional
campaign --- masterminded by Barragán himself --- that emphasized an
aura of exclusivity and otherworldly beauty. As Keith Eggener, a
renowned scholar who has written extensively on the subject, told me,
``I don't see anything preventing one from being a soulful,
sophisticated artist and savvy businessman. The peculiar way in which
Barragán combined these is at the heart of what I've long found so
fascinating about him.''

Image

Bougainvilleas climb up the structure containing the private living
quarters at Casa Ortega's Patio del Perol. The sprawling house and
garden marked a turning point in Barragán's career, the beginning of his
famous lyrical phase.Credit...Nin Solis

BEFORE I LEFT Mexico City, I visited Tacubaya, once a separate town and
weekend retreat on the outskirts of the capital, now a bustling barrio
fully incorporated into the metropolis. In contrast to Colonia
Cuauhtémoc --- with its cultured luster and proximity to high finance
--- Tacubaya offers a more modest and traditionally Mexican streetscape
of large, neglected 19th-century houses mixed with more recent,
anonymous working-class construction and a sprinkling of Art Deco gems.
This is where Barragán's own UNESCO-inscribed house and studio, Casa
Barragán, is located, rightly revered among architects and architecture
lovers from around the world for its alternatingly muted and startling,
exquisitely calibrated composition of fluidly connected, distinctly
appointed rooms, which together create a rich sensory whole that seems
to lock out the city. A conversation with
\href{https://www.nytimes3xbfgragh.onion/2002/12/05/garden/in-mexico-city-a-quiet-revelation.html}{Catalina
Corcuera}, Casa Barragán's longtime director, opened my eyes to an
oft-overlooked fact: Before this iconic residence, there was another
significant, now semi-forgotten Barragán house in the area --- the
missing piece to the puzzle of Barragán's early work.

It intrigued me that at the same time Barragán was actively engaged in
impeccably Corbusian experiments, his attention seemed to already be in
a different place --- figuratively and literally. I felt the key to
understanding Barragán's thinking around 1940 wasn't just in the white
apartment buildings of Colonia Cuauhtémoc. In fact, even before Melchor
Ocampo 38 was completed, around 1940, the architect had bought several
pieces of land in Tacubaya.

The first of these plots is now known as the Ortega house and garden,
after the family he eventually sold it to. While his focus at this site
initially was to design a rambling, multilevel garden, gradual additions
to an existing structure slowly coalesced into an expansive T-shaped
house. The project marked a decisive turning point for Barragán, the
place where his longstanding ideas and influences started being fully
expressed. Here, pared-down volumes in Mediterranean hues, loggias and
subtle references to the Alhambra --- which Barragán visited on his
first trip to Europe --- meet a desire to express something specific to
place and tradition, resulting in a complex succession of indoor and
outdoor spaces that combine textures, changes in height, exactingly
placed objects and other optical tricks to direct the visitor's eye and
create atmosphere. He moved into the completed building in 1943,
signaling the start of the phase of Barragán that everyone knows well,
the one that produced the works Barragán didn't dismiss as little
buildings.

The most lyrical phase of Barragán's career began here, at the
architect's first Tacubaya house, which became a laboratory of sorts,
where forms were tested and concepts explored. It's all a bit less
perfect and coherent than at the house-studio he moved to in 1947 --- on
the plot directly adjacent to the Ortega grounds --- which also made it
more intimate. It is here that Barragán started to reincorporate the
vernacular nods of his private dwellings in Guadalajara, to experiment
with the use of a precise shade of pink and to tinker with the
sophisticated synthesis of memories and references --- from the
haciendas of his childhood to gardens in the South of France --- that
is, in essence, the late style everyone associates with Barragán today.
Importantly, his Corbusian experience stayed with him --- under its
traditionalist trappings, Casa Ortega's sense of space is fundamentally
Modernist.

Image

A shaded loggia at Casa Ortega is used as an open dining room in the
warm season. Barragán left the table behind when he moved to his now
iconic house next door.Credit...Nin Solis

The majority of the hordes of tourists that have descended on the
Mexican capital in recent years have left the city without knowledge of
the existence of the Ortega house and gardens, which don't get even a
fraction of the attention its uber-famous neighbor receives. Not
realizing Barragán's architecture isn't always made of walls --- in
fact, he cared as much about garden design as he did about physical
rooms --- when people do come, many skip the garden. It's their loss.
Encompassing terraces and lawns on various levels, hidden paths,
different types of vegetation, sculptures, multiple sets of stairs and
underground remnants of the Tepetate quarry Barragán found when he
arrived, the Ortega garden is a self-contained territory designed to get
lost in. It's Barragán's best-kept secret.

Paradoxically, visiting Casa Ortega made me look at Melchor Ocampo 38
and the other buildings Barragán made between 1935 and 1940 in a new
light. Often treated as a parenthesis, Barragán's functionalist work now
revealed a continuity with what preceded it and what came after. I
realized that the vision of Corbusian Modernism Barragán expressed
during his first five years in Mexico City is as deeply personal a body
of work as are his earliest creations in Guadalajara and the iconic
postwar output.

In 1940, Barragán wrote a letter to his clients, informing them he was
quitting his profession as an architect. Years later, in the same 1962
interview in which he belittled his functionalist work, he explained how
the decision to step away had been motivated by feeling ``enormously
demoralized and humiliated by clients, who didn't pay my fees and
treated me patronizingly.'' Exhausted, and maybe a little bored,
Barragán longed for greater financial and creative freedom.

Seeing the Ortega property, and knowing that the year he started working
on it was the same year he temporarily retired as an architect, it's
impossible not to wonder how much his discovery of Tacubaya had to do
with his willingness to forsake a burgeoning career for an uncertain but
more satisfying future.

\hypertarget{true-believers-art-issue}{%
\subsubsection{\texorpdfstring{\href{https://www.nytimes3xbfgragh.onion/issue/t-magazine/2020/07/02/true-believers-art-issue}{True
Believers Art
Issue}}{True Believers Art Issue}}\label{true-believers-art-issue}}

Advertisement

\protect\hyperlink{after-bottom}{Continue reading the main story}

\hypertarget{site-index}{%
\subsection{Site Index}\label{site-index}}

\hypertarget{site-information-navigation}{%
\subsection{Site Information
Navigation}\label{site-information-navigation}}

\begin{itemize}
\tightlist
\item
  \href{https://help.nytimes3xbfgragh.onion/hc/en-us/articles/115014792127-Copyright-notice}{©~2020~The
  New York Times Company}
\end{itemize}

\begin{itemize}
\tightlist
\item
  \href{https://www.nytco.com/}{NYTCo}
\item
  \href{https://help.nytimes3xbfgragh.onion/hc/en-us/articles/115015385887-Contact-Us}{Contact
  Us}
\item
  \href{https://www.nytco.com/careers/}{Work with us}
\item
  \href{https://nytmediakit.com/}{Advertise}
\item
  \href{http://www.tbrandstudio.com/}{T Brand Studio}
\item
  \href{https://www.nytimes3xbfgragh.onion/privacy/cookie-policy\#how-do-i-manage-trackers}{Your
  Ad Choices}
\item
  \href{https://www.nytimes3xbfgragh.onion/privacy}{Privacy}
\item
  \href{https://help.nytimes3xbfgragh.onion/hc/en-us/articles/115014893428-Terms-of-service}{Terms
  of Service}
\item
  \href{https://help.nytimes3xbfgragh.onion/hc/en-us/articles/115014893968-Terms-of-sale}{Terms
  of Sale}
\item
  \href{https://spiderbites.nytimes3xbfgragh.onion}{Site Map}
\item
  \href{https://help.nytimes3xbfgragh.onion/hc/en-us}{Help}
\item
  \href{https://www.nytimes3xbfgragh.onion/subscription?campaignId=37WXW}{Subscriptions}
\end{itemize}
