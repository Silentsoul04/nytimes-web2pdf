Sections

SEARCH

\protect\hyperlink{site-content}{Skip to
content}\protect\hyperlink{site-index}{Skip to site index}

\href{https://www.nytimes3xbfgragh.onion/section/us}{U.S.}

\href{https://myaccount.nytimes3xbfgragh.onion/auth/login?response_type=cookie\&client_id=vi}{}

\href{https://www.nytimes3xbfgragh.onion/section/todayspaper}{Today's
Paper}

\href{/section/us}{U.S.}\textbar{}When Caring for Your Child's Needs
Becomes a Job All Its Own

\url{https://nyti.ms/30J1HY1}

\begin{itemize}
\item
\item
\item
\item
\item
\end{itemize}

\href{https://www.nytimes3xbfgragh.onion/spotlight/at-home?action=click\&pgtype=Article\&state=default\&region=TOP_BANNER\&context=at_home_menu}{At
Home}

\begin{itemize}
\tightlist
\item
  \href{https://www.nytimes3xbfgragh.onion/2020/09/07/travel/route-66.html?action=click\&pgtype=Article\&state=default\&region=TOP_BANNER\&context=at_home_menu}{Cruise
  Along: Route 66}
\item
  \href{https://www.nytimes3xbfgragh.onion/2020/09/04/dining/sheet-pan-chicken.html?action=click\&pgtype=Article\&state=default\&region=TOP_BANNER\&context=at_home_menu}{Roast:
  Chicken With Plums}
\item
  \href{https://www.nytimes3xbfgragh.onion/2020/09/04/arts/television/dark-shadows-stream.html?action=click\&pgtype=Article\&state=default\&region=TOP_BANNER\&context=at_home_menu}{Watch:
  Dark Shadows}
\item
  \href{https://www.nytimes3xbfgragh.onion/interactive/2020/at-home/even-more-reporters-editors-diaries-lists-recommendations.html?action=click\&pgtype=Article\&state=default\&region=TOP_BANNER\&context=at_home_menu}{Explore:
  Reporters' Google Docs}
\end{itemize}

Advertisement

\protect\hyperlink{after-top}{Continue reading the main story}

Supported by

\protect\hyperlink{after-sponsor}{Continue reading the main story}

\hypertarget{when-caring-for-your-childs-needs-becomes-a-job-all-its-own}{%
\section{When Caring for Your Child's Needs Becomes a Job All Its
Own}\label{when-caring-for-your-childs-needs-becomes-a-job-all-its-own}}

For some parents, work outside the home is impossible as they navigate
complicated and frustrating systems for help. But they don't have to go
it alone.

\includegraphics{https://static01.graylady3jvrrxbe.onion/images/2020/07/26/multimedia/00ADA-CAREGIVERS/00ADA-CAREGIVERS-articleLarge.jpg?quality=75\&auto=webp\&disable=upscale}

\href{https://www.nytimes3xbfgragh.onion/by/nikita-stewart}{\includegraphics{https://static01.graylady3jvrrxbe.onion/images/2018/09/25/multimedia/author-nikita-stewart/author-nikita-stewart-thumbLarge-v2.png}}

By \href{https://www.nytimes3xbfgragh.onion/by/nikita-stewart}{Nikita
Stewart}

\begin{itemize}
\item
  Published July 24, 2020Updated July 28, 2020
\item
  \begin{itemize}
  \item
  \item
  \item
  \item
  \item
  \end{itemize}
\end{itemize}

\hypertarget{listen-to-this-article}{%
\subsubsection{Listen to This Article}\label{listen-to-this-article}}

Audio Recording by Audm

\emph{To hear more audio stories from publishers like The New York
Times,
download}\href{https://www.audm.com/?utm_source=nytmag\&utm_medium=embed\&utm_campaign=left_behind_draper}{**}\href{https://www.audm.com/?utm_source=nyt\&utm_medium=embed\&utm_campaign=caring_child_job}{\emph{Audm
for iPhone or Android}}\emph{.}

Crystal Watson had to quit her job as a supervisor at the New York City
Parks Department because the school that her son, who has severe
attention deficit hyperactivity disorder, attended kept calling her ``to
come get him.'' Alicia Chapman, a cleaner for New York City Transit,
said she could not take a job for 14 years, unable to trust anyone to
properly care for two of her children with special needs. Shanae
Williams recently had to enlist the help of her oldest daughter to care
for her three youngest children, all of whom have autism, so she could
keep her job at a Brooklyn Walgreens.

``These last two years have been the most horrific of my life,'' said
Ms. Williams, 46, who lives in a homeless shelter in Brooklyn.

Parents of children with disabilities often face an agonizing choice:
working outside the home or caring for their children. Either option can
spiral a family into poverty or keep them there. Choosing both at the
same time can leave some parents feeling as if they are doing neither
well. It is a dilemma that the \href{https://www.legalaidnyc.org/}{Legal
Aid Society} in New York has made a priority, taking up the cases of
parents who cannot afford lawyers to help them through a complex system
of policies and laws.

Ms. Williams's son, Rasheed, 15, and her twins Isis and Adonis, 10, are
nonverbal; the twins are also incontinent. In 2018, they were evicted
from their Brooklyn apartment. The cooperative board gave several
reasons, among them that Isis, who had been playing with a faucet in the
apartment, caused a flood in the building. Shakeema, 26, her oldest
daughter, said that she believed that the eviction was illegal and that
the faucet was faulty.

After losing the apartment and running out of money for hotels, the
family entered the New York City shelter system.

Ms. Williams said she had a hard time seeing a way out, a weight that in
turn has affected her ability to press for help. The emotional and
physical fatigue parents experience, along with work schedules, makes it
more difficult for them to meet with the school officials who make
critical decisions on class placements and financial support for special
education.

(The responsibility for care most often falls to mothers. More study is
needed, but
\href{https://digitalcommons.usu.edu/cgi/viewcontent.cgi?article=3071\&context=etd}{researchers}
have found that fathers sometimes have difficulty making a connection
with their children with disabilities, often leaving the bulk of
parental duties to mothers.)

Advocates and researchers say that students from low-income households
are more likely to receive inaccurate diagnoses, and then are more
likely to be placed in classroom settings separate from other students,
than their peers from wealthier households.

Many legal protections and services for disabled children predate the
Americans With Disabilities Act. Section 504 of the Rehabilitation Act
of 1973 gives children with physical and mental disabilities the right
to equal access to education and extracurricular activities. The
Individuals With Disabilities Education Act, known as IDEA, was passed
in 1975, and provides free public education for eligible children with
disabilities up to age 21. A student often loses those services after
graduating from high school and then has to navigate college, employment
and other services with the assistance of anti-discrimination policies
laid out by the A.D.A.

The early years are critical for children with disabilities, and those
from low-income households often do not have a level playing field. For
example, accommodation (or 504) plans, which are generally used by
students with less severe disabilities, are disproportionately granted
to students from wealthier families, according to
\href{https://www.nytimes3xbfgragh.onion/2019/07/30/us/extra-time-504-sat-act.html}{a
New York Times analysis} of Department of Education data. Such plans can
give students extra time on tests used for placement in classes and
entrance to college.

Those inequalities often extend into adulthood. The
\href{https://ncd.gov/events/2020/30onada30}{National Council on
Disability} found that people with disabilities account for about 12
percent of working-age people in the United States, but that more than
half of working-age people with disabilities live in long-term poverty.
``People with disabilities live in poverty at more than twice the rate
of people without disabilities,'' according to a
\href{https://ncd.gov/progressreport/2017/national-disability-policy-progress-report-october-2017}{2017
report} published by the council.

The Century Foundation, a nonpartisan think tank with a focus on
reducing inequality, looked specifically at the disproportionality of
IDEA, recommending in
\href{https://tcf.org/content/report/students-low-income-families-special-education/}{a
report last year} that policymakers require that income status be
reported, as a way to better assure that children with disabilities are
treated equally in how they are disciplined, identified and placed. IDEA
was supposed to narrow the gap for children from low-income households,
but they continue to lag behind their peers from wealthier households on
assessment tests, the report found.

\href{https://www.pli.edu/faculty/susan-j.-horwitz-23354}{Susan J.
Horwitz}, a supervising attorney at the Legal Aid Society (Ms. Chapman
is a client), said spirit and intention differed greatly from
application. ``I love the IDEA, the A.D.A., 504. This is incredible
legislation to not discriminate against people with disabilities,'' she
said. ``But if there's no money support to provide everything? This is
not so much the legislation as much as the implementation.''

Ms. Chapman, the New York City Transit worker, has four children, and
the oldest and youngest have disabilities. Kevin Spalding, now 20, has
behavioral issues, which she described as ``disruptive personality.''
``He required so much attention that I couldn't work,'' said Ms.
Chapman, 46, who said she initially had financial support from her
partner. ``When I did work, I was so exhausted all the time that I
couldn't do anything with him.''

Ms. Chapman and her partner separated when their youngest son, Kyle
Spalding, was 4 months old, leaving her with no income and two children
in need of full-time attention. ``I had to keep running to the
hospital'' with Kyle, she said, who is allergic to most foods --- so
severely that his diet is limited to rice and chicken.

Day care centers refused to accept Kyle, so it became clear that she
would not be able to work and that she would have to rely on public
assistance.

Through the years, the Legal Aid Society helped her to ensure that Kevin
and Kyle, now 15, received adequate educations, at times at private
schools.

Kevin is now working for a T-shirt printing company, and Kyle is in high
school. With her sons in stable situations, Ms. Chapman was able to
apply for the transit job. In October, she got a call that a position
was hers if she was still interested.

``Out of nowhere came this job,'' she said with joy in her voice. ``I've
never been able to support my family like I am now. It's job security.
It's so much more. It means independence. It means confidence, being
able to carry your own.''

Kimberly Kashefsky, a mother of three in Springfield, Mass., said she
missed working outside the home. At one point, she held three jobs. ``I
would leave a job to go to a job,'' she recalled.

That work allowed her the extra money to take her children on weekend
trips, or to go bowling or to the movies. When her daughter Mya, now 13,
was 18 months old, they learned that she had severe autism. ``That's
when my whole life changed,'' said Ms. Kashefsky, 44. ``I had to make a
decision between my work and my children.''

Ms. Kashefsky's son, Skyler, 16, has attention deficit hyperactivity
disorder and a communication disability; he is in a special education
program. She said she had to quit her jobs to focus on making sure her
children, especially Mya, who is nonverbal and developmentally delayed,
were receiving the care they needed.

Skyler said he planned to be an electrician after high school, where he
is an honor student headed into his junior year. ``Even now, it's still
hard. It's not over yet. We are trying to help my sister any way she can
so that she can live a normal life,'' he said.

Ms. Watson, the former New York City Parks supervisor, said she was
proud of her job. A 31-year-old mother of three, she had to turn her
attention to Michael, her oldest son, who is now 12. When he was in
elementary school, she said: ``They would call me literally at least
about 16 times. I just had to let the job go.''

Her focus turned to talking to doctors, guidance counselors, teachers
and administrators. ``It's definitely a job and a half. You just have to
give your all to that situation,'' she said, explaining that a doctor
initially told her that her son was going through ``a stage.''

``I researched. I Googled it,'' she said. She finally learned that her
son had A.D.H.D. when he was 9.

With her son getting medication and proper care at school, Ms. Watson
was able to rejoin the job force and now works as a director in an
after-school program. ``I'm able to meet the standards with my kids,''
she said. ``I get them the things that are necessary, not the things
that they want.''

Ms. Williams, who now works at Walgreens, is still juggling the needs of
her children, hoping to one day return to running a personal training
business. That job had proved to be too much as she also juggled the
needs of three children with autism.

Shakeema filled in the gaps for her mother. ``I've always helped my
mother,'' she said. ``It wasn't really a made decision.''

She added that she became a teacher and therapist, studying A.D.A.
policies, because ``half the time, you can't get teachers who know what
they are talking about.''

Shakeema has a beauty company, but her job during the coronavirus
pandemic has been caring for her siblings --- and she said that she had
adapted. ``This is my life,'' she said. ``I don't take this as a joke.''

At the shelter in Brooklyn and at another shelter where the family lived
before, Shanae Williams has faced angry fellow residents and even staff
members, who she said were insensitive to her children's disabilities.

In the shelter, where the family lives in a single room, Shakeema helped
her brother and the twins with their remote learning each day during the
school year. She said she knew that they could one day have productive
adult lives.

``They are really frowned upon like they can't do anything. They are
marginalized,'' she said. ``My life looks like me giving them the tools
to survive. Wait, I don't like using that phrase --- to survive ---
because that sounds like life or death. These kids are going to make
it.''

Advertisement

\protect\hyperlink{after-bottom}{Continue reading the main story}

\hypertarget{site-index}{%
\subsection{Site Index}\label{site-index}}

\hypertarget{site-information-navigation}{%
\subsection{Site Information
Navigation}\label{site-information-navigation}}

\begin{itemize}
\tightlist
\item
  \href{https://help.nytimes3xbfgragh.onion/hc/en-us/articles/115014792127-Copyright-notice}{©~2020~The
  New York Times Company}
\end{itemize}

\begin{itemize}
\tightlist
\item
  \href{https://www.nytco.com/}{NYTCo}
\item
  \href{https://help.nytimes3xbfgragh.onion/hc/en-us/articles/115015385887-Contact-Us}{Contact
  Us}
\item
  \href{https://www.nytco.com/careers/}{Work with us}
\item
  \href{https://nytmediakit.com/}{Advertise}
\item
  \href{http://www.tbrandstudio.com/}{T Brand Studio}
\item
  \href{https://www.nytimes3xbfgragh.onion/privacy/cookie-policy\#how-do-i-manage-trackers}{Your
  Ad Choices}
\item
  \href{https://www.nytimes3xbfgragh.onion/privacy}{Privacy}
\item
  \href{https://help.nytimes3xbfgragh.onion/hc/en-us/articles/115014893428-Terms-of-service}{Terms
  of Service}
\item
  \href{https://help.nytimes3xbfgragh.onion/hc/en-us/articles/115014893968-Terms-of-sale}{Terms
  of Sale}
\item
  \href{https://spiderbites.nytimes3xbfgragh.onion}{Site Map}
\item
  \href{https://help.nytimes3xbfgragh.onion/hc/en-us}{Help}
\item
  \href{https://www.nytimes3xbfgragh.onion/subscription?campaignId=37WXW}{Subscriptions}
\end{itemize}
