Sections

SEARCH

\protect\hyperlink{site-content}{Skip to
content}\protect\hyperlink{site-index}{Skip to site index}

\href{/section/us}{U.S.}\textbar{}How One of America's Whitest Cities
Became the Center of B.L.M. Protests

\url{https://nyti.ms/3eX7huM}

\begin{itemize}
\item
\item
\item
\item
\item
\item
\end{itemize}

\hypertarget{race-and-america}{%
\subsubsection{\texorpdfstring{\href{https://www.nytimes3xbfgragh.onion/news-event/george-floyd-protests-minneapolis-new-york-los-angeles?name=styln-george-floyd\&region=TOP_BANNER\&block=storyline_menu_recirc\&action=click\&pgtype=Article\&impression_id=77a0fee0-f52b-11ea-8f57-43e30c511e3d\&variant=undefined}{Race
and America}}{Race and America}}\label{race-and-america}}

\begin{itemize}
\tightlist
\item
  \href{https://www.nytimes3xbfgragh.onion/2020/09/11/us/black-police-chiefs-reform.html?name=styln-george-floyd\&region=TOP_BANNER\&block=storyline_menu_recirc\&action=click\&pgtype=Article\&impression_id=77a0fee1-f52b-11ea-8f57-43e30c511e3d\&variant=undefined}{Black
  Police Chiefs}
\item
  \href{https://www.nytimes3xbfgragh.onion/2020/09/04/nyregion/rochester-police-daniel-prude.html?name=styln-george-floyd\&region=TOP_BANNER\&block=storyline_menu_recirc\&action=click\&pgtype=Article\&impression_id=77a0fee2-f52b-11ea-8f57-43e30c511e3d\&variant=undefined}{What
  Happened in Rochester, N.Y.}
\item
  \href{https://www.nytimes3xbfgragh.onion/2020/08/30/us/portland-shooting-explained.html?name=styln-george-floyd\&region=TOP_BANNER\&block=storyline_menu_recirc\&action=click\&pgtype=Article\&impression_id=77a0fee3-f52b-11ea-8f57-43e30c511e3d\&variant=undefined}{Portland
  Shooting}
\item
  \href{https://www.nytimes3xbfgragh.onion/2020/08/30/us/breonna-taylor-police-killing.html?name=styln-george-floyd\&region=TOP_BANNER\&block=storyline_menu_recirc\&action=click\&pgtype=Article\&impression_id=77a0fee4-f52b-11ea-8f57-43e30c511e3d\&variant=undefined}{Breonna
  Taylor's Life and Death}
\end{itemize}

\includegraphics{https://static01.graylady3jvrrxbe.onion/images/2020/07/24/us/24portland-race01/merlin_174891828_0efe9f26-2890-48cc-9786-9a3ad7006897-articleLarge.jpg?quality=75\&auto=webp\&disable=upscale}

\hypertarget{how-one-of-americas-whitest-cities-became-the-center-of-blm-protests}{%
\section{How One of America's Whitest Cities Became the Center of B.L.M.
Protests}\label{how-one-of-americas-whitest-cities-became-the-center-of-blm-protests}}

In a state with a brutal racist history, the Black Lives Matter protests
in Portland, Ore., have been overwhelmingly attended by white
demonstrators.

Protesters gathered to listen to Black Lives Matter activists in
Portland, Ore., on Thursday.Credit...Octavio Jones for The New York
Times

Supported by

\protect\hyperlink{after-sponsor}{Continue reading the main story}

\href{https://www.nytimes3xbfgragh.onion/by/thomas-fuller}{\includegraphics{https://static01.graylady3jvrrxbe.onion/images/2018/06/12/multimedia/author-thomas-fuller/author-thomas-fuller-thumbLarge.png}}

By \href{https://www.nytimes3xbfgragh.onion/by/thomas-fuller}{Thomas
Fuller}

\begin{itemize}
\item
  Published July 24, 2020Updated July 29, 2020
\item
  \begin{itemize}
  \item
  \item
  \item
  \item
  \item
  \item
  \end{itemize}
\end{itemize}

PORTLAND, Ore. --- Seyi Fasoranti, a chemist who moved to Oregon from
the East Coast six months ago, has watched the Black Lives Matter
\href{https://www.nytimes3xbfgragh.onion/2020/07/29/us/protests-portland-federal-withdrawal.html}{protests
in Portland} with fascination. A sea of white faces in one of the
whitest major American cities has cried out for racial justice every
night for nearly two months.

``It's something I joke about with my friends,'' Mr. Fasoranti, who is
Black, said over the din of protest chants this week. ``There are more
Black Lives Matter signs in Portland than Black people.''

Loud advocacy has been a hallmark of
\href{https://www.nytimes3xbfgragh.onion/2020/07/25/us/a-wall-of-vets-joins-the-front-lines-of-portland-protests.html}{Portland}
life for decades, but unlike past protests over environmental policies
or foreign wars, racism is a more complicated topic in Oregon, one that
is intertwined with demographics and the state's legacy of some of
\href{https://www.nytimes3xbfgragh.onion/2017/06/04/us/portland-killings-racist-laws-oregon.html}{the
most brutal anti-Black laws} in the nation.

During 56 straight nights of protests here, throngs of largely white
protesters have raised their fists in the air and chanted, ``This is not
a riot, it's a revolution.'' They have thrown water bottles at the
federal courthouse, tried to pry off the plywood that protects the
entrance and engaged in running battles with police officers through
clouds of tear gas. In recent nights, the number of protesters has
swollen into the thousands.

Damany Igwé, 43, a bath products salesman who is Black and has taken
part in dozens of the protests, says white crowds have shielded him from
the police, all the while yelling ``Black power!''

\includegraphics{https://static01.graylady3jvrrxbe.onion/images/2020/07/22/autossell/Portland-Still_01/Portland-Still_01-videoSixteenByNineJumbo1600-v4.jpg}

``I feel the most protected that I ever have in my city,'' Mr. Igwé said
during a Wednesday night protest that lasted well into Thursday morning.
``White people can't understand what we've been through completely, but
they are trying to empathize. That's a beginning.''

Of the 35 cities in the United States with populations larger than
500,000, Portland is the whitest, according to census data, with 71
percent of residents categorized as non-Latino whites.

Oregon's relative homogeneity --- the state is three-quarters white
compared with neighboring California, where white people make up 37
percent of the population --- was not accidental. The state was founded
on principles of white supremacy. A 19th-century lash law called for
whipping any Black person found in the state. In the early part of the
20th century Oregon's Legislature was dominated by members of the Ku
Klux Klan.

Today the average income level for Black families in Portland is nearly
half that of white residents, and police shootings of Black residents
are disproportionate to their 6 percent share of the population. Three
years ago, two good Samaritans were
\href{https://www.nytimes3xbfgragh.onion/2017/05/27/us/portland-train-attack-muslim-rant.html}{fatally
stabbed} while trying to stop a man from shouting slurs at two
African-American women on a commuter train, one of whom was wearing
Muslim dress.

``Really there are two Portlands that exist,'' said Walidah Imarisha, a
scholar of Black history in Oregon. ``There's white Portland and
Portland of color.''

The differences, she said, cover almost every aspect of life. ``There's
massive racial disparities around wealth, health care, schools and
criminal legal systems that white Portlanders just don't understand.''

Yet on the streets this week in Portland there was optimism among Black
protest leaders who generally spoke admiringly of the large white
crowds, which were reinvigorated last week after clashes with federal
riot police officers who are protecting a U.S. courthouse and other
buildings.

\includegraphics{https://static01.graylady3jvrrxbe.onion/images/2020/07/24/us/24portland-race02/24portland-race02-articleLarge.jpg?quality=75\&auto=webp\&disable=upscale}

Xavier Warner, a Black protest organizer, called the predominance of
white protesters ``a beautiful thing'' that speaks to the progressive
ethos in the city.

Teal Lindseth, another Black organizer, said she saw the irony in
predominantly white Portland having among the longest continuous
protests stemming from the police killing of George Floyd in Minneapolis
on May 25. But she said she was thankful for the strength in numbers.
``They hurt us less when there are more people,'' she said.

The role of white protesters has some detractors in the Black community.

In an
\href{https://www.washingtonpost.com/opinions/2020/07/23/portlands-protests-were-supposed-be-about-black-lives-now-theyre-white-spectacle/}{op-ed
published Thursday in The Washington Post}, the Rev. E.D. Mondainé, the
president of the Portland branch of the N.A.A.C.P., called the protests
a ``spectacle'' that distracted attention from the Black Lives Matter
movement.

``Are they really furthering the cause of justice, or is this another
example of white co-optation?'' he wrote.

But in a measure of the divided opinion on this question, Mr. Mondainé's
predecessor at the N.A.A.C.P., Jo Ann Hardesty, a city commissioner,
rejected his criticism.

``There's a lot of new, aware folks who have joined into the battle for
Black lives,'' she said during a news conference on Thursday.

Ms. Hardesty, who took office in 2019 as the first African-American
woman on the Portland City Council, said the protests were serving the
dual purposes of fighting racial injustice and rejecting the presence of
federal agents sent to the city by the Trump administration.

Both protest goals were important, she said. ``And one is not any more
important than the other.''

Image

The Wall of Moms, a group of mothers that regularly attends the
protests.Credit...Octavio Jones for The New York Times

Joe Lowndes, an expert on right-wing politics and race at the University
of Oregon, said the protests reflected an intertwining of interests in
recent years between racial justice advocates and the largely white
anti-fascist movement. Both are deeply distrustful of the police and
want police powers and budgets curtailed. The presence of far-right
groups in Oregon, emboldened during the Trump administration, has also
brought anti-racists and anti-fascists into closer alignment, he said.

Speeches and chants at the protests have touched on the legacy of
slavery and the stripping of lands from Native Americans. From a
historical perspective, the sight of hundreds of white protesters
chanting one of that movement's most popular refrains --- ``Stolen lands
and stolen people'' --- can be jarring.

As the destination of the Lewis and Clark expeditions, Oregon once
symbolized the conquest of the American West and the subjugation of
Native peoples.

Some white protesters said it was this white supremacist legacy that
helped spur them into the streets.

``Bringing that history to light is definitely a motivating factor,''
said Liza Lopetrone, a veterinary nurse who joined the Wall of Moms
protest this week that consisted mostly of white women locking arms in
the face of the federal agents. ``Oregon has an extremely racist
history. I'm not from here but I take responsibility for it now.''

Image

There is a sense of optimism among Black protest leaders, who generally
spoke admiringly of the largely white crowds.Credit...Octavio Jones for
The New York Times

Another woman at the protest, Julie Liggins, had a more immediate
connection to prejudice and racism in Portland. She is white and her
husband of three decades, Reginald, is Black.

During the years he drove his car to work, Mr. Liggins said, he was
pulled over by Portland police multiple times without cause. He said he
switched to riding the bus. But two years ago when Mr. Liggins, who is
60, ran to catch a bus, the police
\href{https://katu.com/news/local/beaverton-police-launch-investigation-into-possible-case-of-racial-profiling-reggie-liggins}{pulled
it over} after misidentifying him for a robbery suspect in his 20s.

Mr. Liggins said he was encouraged by the protests even if he wished the
reckoning over race in America had occurred earlier. And he loves his
life in Portland.

``You can literally go days without seeing people that look like you,''
he said. ``But I find Portland to be a very progressive city despite its
racist past. I can honestly say that as an interracial couple we haven't
had any problems here.''

Mr. Fasoranti, the chemist, says he has been impressed with the
awareness of racial issues in Portland and described the current round
of protests as something that ``feels genuine.''

He says he feels welcome in the city and was intrigued soon after he
arrived when a white motorist pulled over to the sidewalk and asked if
he needed a ride. He has been invited to conversations about
gentrification and the displacement of Black residents.

``There are less of these conversations in New York or New Jersey, where
I used to live,'' he said.

Advertisement

\protect\hyperlink{after-bottom}{Continue reading the main story}

\hypertarget{site-index}{%
\subsection{Site Index}\label{site-index}}

\hypertarget{site-information-navigation}{%
\subsection{Site Information
Navigation}\label{site-information-navigation}}

\begin{itemize}
\tightlist
\item
  \href{https://help.nytimes3xbfgragh.onion/hc/en-us/articles/115014792127-Copyright-notice}{©~2020~The
  New York Times Company}
\end{itemize}

\begin{itemize}
\tightlist
\item
  \href{https://www.nytco.com/}{NYTCo}
\item
  \href{https://help.nytimes3xbfgragh.onion/hc/en-us/articles/115015385887-Contact-Us}{Contact
  Us}
\item
  \href{https://www.nytco.com/careers/}{Work with us}
\item
  \href{https://nytmediakit.com/}{Advertise}
\item
  \href{http://www.tbrandstudio.com/}{T Brand Studio}
\item
  \href{https://www.nytimes3xbfgragh.onion/privacy/cookie-policy\#how-do-i-manage-trackers}{Your
  Ad Choices}
\item
  \href{https://www.nytimes3xbfgragh.onion/privacy}{Privacy}
\item
  \href{https://help.nytimes3xbfgragh.onion/hc/en-us/articles/115014893428-Terms-of-service}{Terms
  of Service}
\item
  \href{https://help.nytimes3xbfgragh.onion/hc/en-us/articles/115014893968-Terms-of-sale}{Terms
  of Sale}
\item
  \href{https://spiderbites.nytimes3xbfgragh.onion}{Site Map}
\item
  \href{https://help.nytimes3xbfgragh.onion/hc/en-us}{Help}
\item
  \href{https://www.nytimes3xbfgragh.onion/subscription?campaignId=37WXW}{Subscriptions}
\end{itemize}
