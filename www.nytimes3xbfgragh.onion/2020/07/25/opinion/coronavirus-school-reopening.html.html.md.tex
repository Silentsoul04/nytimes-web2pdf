Sections

SEARCH

\protect\hyperlink{site-content}{Skip to
content}\protect\hyperlink{site-index}{Skip to site index}

\href{https://myaccount.nytimes3xbfgragh.onion/auth/login?response_type=cookie\&client_id=vi}{}

\href{https://www.nytimes3xbfgragh.onion/section/todayspaper}{Today's
Paper}

\href{/section/opinion}{Opinion}\textbar{}`Home-Schooling Won't Kill Us.
Covid-19 Might.'

\url{https://nyti.ms/3huvcDy}

\begin{itemize}
\item
\item
\item
\item
\item
\item
\end{itemize}

Advertisement

\protect\hyperlink{after-top}{Continue reading the main story}

\href{/section/opinion}{Opinion}

Supported by

\protect\hyperlink{after-sponsor}{Continue reading the main story}

\hypertarget{home-schooling-wont-kill-us-covid-19-might}{%
\section{`Home-Schooling Won't Kill Us. Covid-19
Might.'}\label{home-schooling-wont-kill-us-covid-19-might}}

Parents and teachers struggle with how to reopen schools safely this
fall.

By Rachel L. Harris and Lisa Tarchak

Ms. Harris and Ms. Tarchak are senior editorial assistants.

\begin{itemize}
\item
  July 25, 2020
\item
  \begin{itemize}
  \item
  \item
  \item
  \item
  \item
  \item
  \end{itemize}
\end{itemize}

\includegraphics{https://static01.graylady3jvrrxbe.onion/images/2020/07/25/opinion/sunday/25opreaders-school/25opreaders-school-articleLarge.jpg?quality=75\&auto=webp\&disable=upscale}

``Of all the American institutions the pandemic has shut down, none face
pressure to reopen quite like schools do,'' Sarah Darville
\href{https://www.nytimes3xbfgragh.onion/2020/07/23/sunday-review/reopening-schools-coronavirus.html}{writes}
in this week's Sunday Review.

In the past few weeks we've heard from
multiple\href{https://www.nytimes3xbfgragh.onion/2020/07/18/opinion/sunday/covid-schools-reopen-teacher-safety.html}{contributors},
\href{https://www.nytimes3xbfgragh.onion/2020/07/15/opinion/schools-reopening.html}{columnists}
and the
\href{https://www.nytimes3xbfgragh.onion/2020/07/10/opinion/coronavirus-schools-reopening.html}{editorial
board} about whether or not schools should reopen for in-person classes
this fall. And in the thousands of comments on these pieces, parents and
teachers weighed the dangers and the repercussions of continued virtual
learning.

Wherever they landed, many agreed that the coronavirus crisis has
brought into acute focus how vital America's schools and child care
centers are to society and how crucial they are to helping our
diminished economy recover. A selection of those comments follows. They
have been edited for clarity and length.

\begin{center}\rule{0.5\linewidth}{\linethickness}\end{center}

\hypertarget{no-one-wants-to-go-back-to-school-more-than-i-do}{%
\subsection{`No one wants to go back to school more than I
do'}\label{no-one-wants-to-go-back-to-school-more-than-i-do}}

\hypertarget{it-is-imperative-that-we-have-a-real-plan-in-place}{%
\subsubsection{\texorpdfstring{\textbf{`It is imperative that we have a
real plan in
place'}}{`It is imperative that we have a real plan in place'}}\label{it-is-imperative-that-we-have-a-real-plan-in-place}}

I love my job. It is my calling, my life's work. I have done this for
more than twenty years at the same urban public school. My students
amuse me and amaze me on a daily basis. Yet the urgent desire of people
who are not in education to get schools up and running, frankly, amazes
me. Despite all my love for my students, I don't really want to die for
them or anyone else. Neither does my partner, who is living with cancer.
It is imperative that we have a real plan in place if we have school.
Teachers and students and cafeteria workers and secretaries and
custodians and librarians and bus drivers all deserve to be safe while
at their jobs. ---
\href{https://nyti.ms/3gSauxs\#permid=108077824}{\emph{Eva Lockhart,
Minneapolis}}

\hypertarget{returning-to-normal-requires-first-and-foremost-controlling-the-virus}{%
\subsubsection{\texorpdfstring{\textbf{`Returning to normal requires,
first and foremost, controlling the
virus'}}{`Returning to normal requires, first and foremost, controlling the virus'}}\label{returning-to-normal-requires-first-and-foremost-controlling-the-virus}}

Of course we need to reopen schools. However, here's a few things that
need to happen: Reliable, regular and random testing seems one important
criteria. In that vein, is the school nurse responsible for all this
testing? Where does all the P.P.E. (personal protective equipment) come
from? What about substitute teachers for regular old 24-hour bugs? How
many teachers receive combat pay while being forced into mortal heroics?
All of our society's shortcomings are being exacerbated by this
pandemic, and too many forget that returning to normal requires, first
and foremost, controlling the virus. \emph{---}
\href{https://nyti.ms/38ZtS8M\#permid=108139377}{\emph{James Siegel,
Maine}}

\hypertarget{remote-learning-was-adequate-at-best}{%
\subsubsection{\texorpdfstring{\textbf{`Remote learning was adequate at
best'}}{`Remote learning was adequate at best'}}\label{remote-learning-was-adequate-at-best}}

I'm the parent of a 12-year-old. Her experience with remote learning was
adequate at best. She is very shy and it was easy for her to be missed
by her teachers. Two of her teachers had problems transitioning. Does my
daughter want to go back to the classroom? Yes. Do I prefer that she
does? Yes. Do I want to risk her health in order for her to return to
the classroom? No! Luckily I live in Massachusetts, where decisions on
school reopenings are being made by intelligent, thinking people. Too
bad the same can't be said about the federal government. ---
\href{https://nyti.ms/32n4WXM\#permid=108138527}{\emph{Alan,
Massachusetts}}

\hypertarget{if-young-kids-are-home-one-parent-has-to-quit-their-job}{%
\subsubsection{\texorpdfstring{\textbf{`If young kids are home, one
parent has to quit their
job'}}{`If young kids are home, one parent has to quit their job'}}\label{if-young-kids-are-home-one-parent-has-to-quit-their-job}}

I'm a parent of a first grader and remote learning is a disaster. My kid
only had one hour of remote learning a day. The one hour was far from
smooth (interruptions, technology issues). I had to teach my child the
rest of the day while trying to keep up with my job. Not everyone has
one parent staying at home who can home-school the kids. Some parents
actually have to go to the office in the fall. Basically, if young kids
are home, one parent has to quit their job. Teachers should take the
proper precautions (masks and sanitizer) and come to school to teach.
--- \href{https://nyti.ms/2BeuqLJ\#permid=108205646}{\emph{DK, New
Jersey}}

\hypertarget{tell-me-how-to-have-a-socially-distanced-active-shooter-drill}{%
\subsubsection{\texorpdfstring{\textbf{`Tell me how to have a socially
distanced active shooter
drill'}}{`Tell me how to have a socially distanced active shooter drill'}}\label{tell-me-how-to-have-a-socially-distanced-active-shooter-drill}}

There isn't anyone involved in schools or children's lives who doesn't
want to see children return to school safely. But we are not yet safe.
Tell me how to get a 6-year-old to not sneeze on his friends let alone
play and work from a distance (mucus, saliva, pee, poop, this is all
part of our day at the lower levels of education). Tell me how each
child is going to have her own supplies for the day as shared supplies
are no longer an option. No more Legos, no more books. Tell me how to
comfort a hysterical child from a distance of six feet. Tell me how to
have a socially distanced active shooter drill. Seriously, tell me.
Because no one wants to go back to school more than I do. ---
\href{https://nyti.ms/2W8RrH7\#permid=108084485}{\emph{Anna B,
Westchester, N.Y}}\emph{.}

\hypertarget{give-remote-learning-another-chance}{%
\subsection{`Give remote learning another
chance'}\label{give-remote-learning-another-chance}}

\hypertarget{you-cant-expect-students-to-learn-if-they-arent-even-required-to-show-up}{%
\subsubsection{`You can't expect students to learn if they aren't even
required to show
up'}\label{you-cant-expect-students-to-learn-if-they-arent-even-required-to-show-up}}

I am baffled every time I read that remote learning was a ``joke'' or a
``disaster.'' I spent 12 or more hours a day teaching live lessons,
providing written feedback on student work, making instructional videos,
meeting remotely with students one on one, planning lessons, calling
students and parents, and even making socially distant ``house calls.''
Yes, it was hard. I know this was not the ideal way for many students to
learn. But the biggest barrier to online education was not the lack of
teacher effort; it was the absence of student accountability. You can't
expect students to learn if they aren't even required to show up. I say
give remote learning another chance, because when we start seeing the
consequences of a hasty return to school, we will have to anyway. ---
\href{https://nyti.ms/2DLfVjo\#permid=108078115}{\emph{Carolyn,
Princeton, N.J.}}

\hypertarget{parents-need-to-step-up-and-step-in-to-educate-their-kids}{%
\subsubsection{\texorpdfstring{\textbf{`Parents need to step up and step
in to educate their
kids'}}{`Parents need to step up and step in to educate their kids'}}\label{parents-need-to-step-up-and-step-in-to-educate-their-kids}}

Kids can indeed learn well remotely. My kids did online public school
for three years before we started home-schooling. They learned great and
did very well. Parents need to step up and step in to educate their
kids. This insanity that kids need to be in a classroom setting is
foolish. Home-schooled kids thrive every day. ---
\href{https://nyti.ms/30w0Yti\#permid=108204069}{\emph{Nikki G.,
Pahrump, Nev.}}

\hypertarget{the-kids-will-be-fine-its-the-adults-that-need-to-get-their-act-together}{%
\subsubsection{\texorpdfstring{\textbf{`The kids will be fine, it's the
adults that need to get their act
together'}}{`The kids will be fine, it's the adults that need to get their act together'}}\label{the-kids-will-be-fine-its-the-adults-that-need-to-get-their-act-together}}

I raised three children and the truth is all this nonsense about not
having traditional school being detrimental is just that, nonsense. Kids
are very resilient and they get their cues from adults. Education is far
more than just classroom instruction. I believe all this drama about
opening schools is causing far more damage to our children than the
quarantine itself. It's fairly obvious to most people that you cannot
open schools in high-rate Covid areas like South Florida. The kids will
be fine, it's the adults that need to get their act together. ---
\href{https://nyti.ms/2OQMFKd\#permid=108282017}{\emph{Mike L, South
Carolina}}

\hypertarget{kids-are-going-to-suffer-either-way}{%
\subsection{Kids are going to suffer either
way}\label{kids-are-going-to-suffer-either-way}}

\hypertarget{i-worry-about-many-students-who-have-unstable-homes}{%
\subsubsection{`I worry about many students who have unstable
homes'}\label{i-worry-about-many-students-who-have-unstable-homes}}

I'm a nurse who works in a school, and I believe we need to go back. The
kids are suffering, and many are ``non-engagers''; that is, they are not
participating in online classes. I worry about many students who have
unstable homes and need to have eyes on them. I'm a bit taken aback by
the author's background as a nurse. We go where we are needed most, and
I have a few hundred students who need me. The risk to the health of the
children appears to be minimal; severe illness is very rare. In
balancing that risk against the real risks of abuse, isolation and
neglect, I strongly believe it is better for us to return to school. ---
\href{https://nyti.ms/39s7rsZ\#permid=108203959}{\emph{Molly B.,
Pittsburgh}}

\hypertarget{we-are-going-to-be-facing-some-long-term-damage}{%
\subsubsection{\texorpdfstring{`\textbf{We are going to be facing some
long-term
damage'}}{`We are going to be facing some long-term damage'}}\label{we-are-going-to-be-facing-some-long-term-damage}}

I'm a 20-year middle and high school teacher. I have a job because your
children go to a public school. If we insist on continuing this
insanity, we are going to be facing some long-term damage, with
social-emotional being the category that takes the hardest hit. I went
into this profession to be of service to my students. If I have to wear
P.P.E. (which my school is providing) and we have to do things a bit
differently (labs will look different now, for instance), then so be it!
We need to suck it up or get out. ---
\href{https://nyti.ms/3eTsCp2\#permid=108207697}{\emph{Boni,
California}}

\hypertarget{the-surest-way-to-exacerbate-inequality}{%
\subsubsection{\texorpdfstring{\textbf{`\textbf{The surest way to
exacerbate
inequality}'}}{`The surest way to exacerbate inequality'}}\label{the-surest-way-to-exacerbate-inequality}}

The gaps between students will become more and more apparent the longer
schools stay closed. My son just finished first grade. He has progressed
an additional year in math and several years in reading ability since
school was canceled in March. Why? Because his mother has a B.S. in
chemical engineering, a J.D. from a T14 law school and is an experienced
teacher. **** But for his classmates whose parents don't speak English,
there's no way they can catch up. The surest way to exacerbate
inequality is to close schools. ---
\href{https://nyti.ms/3juPoXN\#permid=108208754}{\emph{MN, Portland}}

\hypertarget{this-isnt-the-first-time-americas-schools-have-been-neglected}{%
\subsection{This isn't the first time America's schools have been
neglected}\label{this-isnt-the-first-time-americas-schools-have-been-neglected}}

\hypertarget{this-is-about-protecting-people-in-a-marginalized-career}{%
\subsubsection{\texorpdfstring{\textbf{`This is about protecting people
in a marginalized
career'}}{`This is about protecting people in a marginalized career'}}\label{this-is-about-protecting-people-in-a-marginalized-career}}

I teach at a private school in New York City and in the last couple of
weeks we've been cc'd on a number of emails from the school's president
being sent out to parents about plans to reopen. Not a single email has
been directed to us --- the teachers --- about how we'll be protected
and compensated. The way politicians have been talking about the virus
and kids as carriers, you'd think the entire student population is in
elementary school. I teach high school students, kids that are
biologically inclined to share spit with each other. My particular
students are special education. The risks go far beyond families needing
to get back to work to stimulate the economy. This is about protecting
people in a marginalized career. Once again, public servants are forced
to martyr themselves for the good of the private sector. --- \emph{Dave,
New York City}

\hypertarget{if-i-were-a-teacher-i-would-leave}{%
\subsubsection{\texorpdfstring{\textbf{`If I were a teacher, I would
leave'}}{`If I were a teacher, I would leave'}}\label{if-i-were-a-teacher-i-would-leave}}

The government has never granted parents the right to child care. In the
eyes of the state, school and child care are different. Our teachers had
to strike a few years ago for higher wages. Our district is constantly
piling on more administrative requirements. We've decided it's OK that
teachers and students can be killed in their classrooms by a deranged
gunman, and we expect teachers to solve problems caused by poverty. If I
were a teacher, I would leave as well. I'd be tired of sacrificing
myself. \emph{---}
\href{https://nyti.ms/2D27dwA\#permid=108212644}{\emph{Alicia, Seattle}}

\hypertarget{lawmakers-put-allocating-the-money-school-districts-desperately-needed-way-down-at-the-bottom-of-the-priority-list}{%
\subsubsection{\texorpdfstring{\textbf{`\textbf{Lawmakers put allocating
the money school districts desperately needed way down at the bottom of
the priority
list}'}}{`Lawmakers put allocating the money school districts desperately needed way down at the bottom of the priority list'}}\label{lawmakers-put-allocating-the-money-school-districts-desperately-needed-way-down-at-the-bottom-of-the-priority-list}}

The needs of children and working parents were ignored by American
lawmakers in spectacular, unanimous, bipartisan fashion when the CARES
act was passed. Our lawmakers put allocating the money school districts
desperately needed way down at the bottom of the priority list and
slathered trillions on corporate America without even bothering to
ensure oversight. Democrats should have insisted that all the money
schools and our children would need be included in that first bill, no
matter how bitter the fight. Home-schooling our children was brutal this
spring. My wife and I will rise to the challenge and do better.
Home-schooling won't kill us. Covid-19 might. ---
\href{https://nyti.ms/32rAT17\#permid=108140607}{\emph{Joe, New York}}

\hypertarget{treat-us-like-babysitters-and-youll-lose-a-generation-of-teachers}{%
\subsubsection{\texorpdfstring{\textbf{`Treat us like babysitters and
you'll lose a generation of
teachers'}}{`Treat us like babysitters and you'll lose a generation of teachers'}}\label{treat-us-like-babysitters-and-youll-lose-a-generation-of-teachers}}

Third-generation teacher here, with a degree from M.I.T. Who will teach
all of those kids in all of those extra spaces? Do you picture me
running back and forth between the gym and my classroom? Do you care if
there are qualified teachers doing the work? Am I just a babysitter?
Keep me safe next year and you'll keep me in the profession. Treat us
like babysitters and you'll lose a generation of teachers who even
before Covid sacrificed a lot for children and their families. ---
\href{https://nyti.ms/2BQRDE8\#permid=108076999}{\emph{EWilson, Olympia,
Wash.}}

\emph{The Times is committed to publishing}
\href{https://www.nytimes3xbfgragh.onion/2019/01/31/opinion/letters/letters-to-editor-new-york-times-women.html}{\emph{a
diversity of letters}} \emph{to the editor. We'd like to hear what you
think about this or any of our articles. Here are some}
\href{https://help.nytimes3xbfgragh.onion/hc/en-us/articles/115014925288-How-to-submit-a-letter-to-the-editor}{\emph{tips}}\emph{.
And here's our email:}
\href{mailto:letters@NYTimes.com}{\emph{letters@NYTimes.com}}\emph{.}

\emph{Follow The New York Times Opinion section on}
\href{https://www.facebookcorewwwi.onion/nytopinion}{\emph{Facebook}}\emph{,}
\href{http://twitter.com/NYTOpinion}{\emph{Twitter (@NYTopinion)}}
\emph{and}
\href{https://www.instagram.com/nytopinion/}{\emph{Instagram}}\emph{.}

Advertisement

\protect\hyperlink{after-bottom}{Continue reading the main story}

\hypertarget{site-index}{%
\subsection{Site Index}\label{site-index}}

\hypertarget{site-information-navigation}{%
\subsection{Site Information
Navigation}\label{site-information-navigation}}

\begin{itemize}
\tightlist
\item
  \href{https://help.nytimes3xbfgragh.onion/hc/en-us/articles/115014792127-Copyright-notice}{©~2020~The
  New York Times Company}
\end{itemize}

\begin{itemize}
\tightlist
\item
  \href{https://www.nytco.com/}{NYTCo}
\item
  \href{https://help.nytimes3xbfgragh.onion/hc/en-us/articles/115015385887-Contact-Us}{Contact
  Us}
\item
  \href{https://www.nytco.com/careers/}{Work with us}
\item
  \href{https://nytmediakit.com/}{Advertise}
\item
  \href{http://www.tbrandstudio.com/}{T Brand Studio}
\item
  \href{https://www.nytimes3xbfgragh.onion/privacy/cookie-policy\#how-do-i-manage-trackers}{Your
  Ad Choices}
\item
  \href{https://www.nytimes3xbfgragh.onion/privacy}{Privacy}
\item
  \href{https://help.nytimes3xbfgragh.onion/hc/en-us/articles/115014893428-Terms-of-service}{Terms
  of Service}
\item
  \href{https://help.nytimes3xbfgragh.onion/hc/en-us/articles/115014893968-Terms-of-sale}{Terms
  of Sale}
\item
  \href{https://spiderbites.nytimes3xbfgragh.onion}{Site Map}
\item
  \href{https://help.nytimes3xbfgragh.onion/hc/en-us}{Help}
\item
  \href{https://www.nytimes3xbfgragh.onion/subscription?campaignId=37WXW}{Subscriptions}
\end{itemize}
