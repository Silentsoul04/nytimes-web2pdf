Sections

SEARCH

\protect\hyperlink{site-content}{Skip to
content}\protect\hyperlink{site-index}{Skip to site index}

\href{/section/opinion}{Opinion}\textbar{}A Visit to the Classrooms the
Kids Left Behind

\url{https://nyti.ms/3jFrdWR}

\begin{itemize}
\item
\item
\item
\item
\item
\end{itemize}

\includegraphics{https://static01.graylady3jvrrxbe.onion/images/2020/07/23/opinion/00Exposures-Gilbertson-slide-4JQX/00Exposures-Gilbertson-slide-4JQX-articleLarge.jpg?quality=75\&auto=webp\&disable=upscale}

\href{/section/opinion}{Opinion}

\hypertarget{a-visit-to-the-classrooms-the-kids-left-behind}{%
\section{A Visit to the Classrooms the Kids Left
Behind}\label{a-visit-to-the-classrooms-the-kids-left-behind}}

School administrators confront what one called ``the greatest challenge
of my career.''

A coronavirus lesson from March remained in a fifth grade classroom at
Speyer School.Credit...

Supported by

\protect\hyperlink{after-sponsor}{Continue reading the main story}

Photographs and Text by Ashley Gilbertson

Mr. Gilbertson is a photographer in New York.

\begin{itemize}
\item
  July 25, 2020
\item
  \begin{itemize}
  \item
  \item
  \item
  \item
  \item
  \end{itemize}
\end{itemize}

Over a million New York City students and teachers are still unsure of
when and how they might return to school this fall. Their classrooms are
capsules of those panicked final days in March, when schools abruptly
shut down to prevent the spread of the coronavirus.

As a parent, it feels impossible to keep up with the politicized debate
over reopening schools. I wanted to bring the focus back to the
classrooms and the voices of the people that occupy them, who care for
and educate our children. In July, I visited schools around the city to
photograph the spaces that children like mine abruptly left.

\includegraphics{https://static01.graylady3jvrrxbe.onion/images/2020/07/23/opinion/00Exposures-Gilbertson-slide-Y13B/00Exposures-Gilbertson-slide-Y13B-articleLarge.jpg?quality=75\&auto=webp\&disable=upscale}

Image

As the city debated school closures in March, many parents had already
stopped sending their kids to school.

Image

Larry Donovan, the head of Speyer School, shows plans for socially
distanced seating.

``It feels like walking into Pompeii --- everything frozen at a precise
time.'' Said Larry Donovan, the Head of School at the Speyer School.
``There was a great sense of urgency, and we had to be nimble. We had
kids grab their personal belongings from classrooms, lockers, and
cubbies, and tried to get computers and iPads into the students' hands,
too.''

Image

A middle school social studies classroom at P.S./M.S. 5 in the Bronx.

Overnight, principals and administrators had to negotiate between
anxious parents, city officials and teachers. They were forced to become
experts in personal protection equipment, conduct health screenings, and
support students through loss and grief.

At P.S./M.S. 5 public school in the South Bronx, Principal Danielle
Keane was asked to visit a young student, after her mother died from
Covid-19.

``Do you know what passed means?'' Principal Keane remembers asking the
girl.

``Well, my brothers are crying,'' she said, ``so it can't be good.''

``It means Mommy died.'' Ms. Keane said, her voice breaking as she told
me the story. ``Never in a million years could I have imagined that was
a role I'd be taking on as a principal. Schools are central to the
community --- we're the first call.''

As infections continue to spike nationwide, President Trump
\href{https://twitter.com/realDonaldTrump/status/1280853299600789505?s=20}{threatened}
to cut federal funding for districts that defied his demand to resume
classes in person. Education Secretary Betsy DeVos
\href{https://www.washingtonpost.com/politics/2020/07/23/devoss-claim-that-children-are-stoppers-covid-19/}{claimed}
that children are ``stoppers'' of the virus, despite health officials
saying there's no evidence of that. Public health and the well being of
our children appear to run a distant second to getting the economy
moving.

Image

A lesson on coronavirus is still up on the whiteboard at a science
classroom at a Brooklyn Prospect Charter School.

Image

Presidential politics have politicized school reopenings, putting aside
the health and welfare of students and educators.

But at the schools I visited, the administrators are working on the
practical issues and facing an uncertain future.

``The pandemic has, by a long shot, been the greatest challenge of my
career,'' said Bo Lauder, who has been the head of school at Friends
Seminary for the past 18 years. ``We've never encountered a situation
like this, with details changing almost daily, and serious danger to our
community's health seemingly lurking around every corner.''

Image

Plants, as well as two turtles, from areas throughout the school were
consolidated into one classroom for watering at~Washington
Heights~Expeditionary Learning School.

Image

A fourth grade classroom at Port Morris School of Community Leadership.~

Educators are about to be thrust into yet another role --- navigating
the nation's school systems through uncharted waters as we go about one
of the largest experiments in human history: How to reopen safely?

``This was an epicenter of the virus. We have a lot of teachers that
lost family and got sick themselves, and they're scared to death to come
back,'' said Principal Keane. ``I basically cried at my budget meeting
on Friday --- I've got teacher salaries and that's pretty much it. I
keep asking where the tape is for the floor, to measure six feet, but
it's not here. We might have to furlough cleaning staff though we only
have two now. We only have one nurse. We need more of these people, not
less.''

At schools throughout the New York City, preparations are taking place
around the clock. Classrooms are being mapped out for social distancing;
ventilation systems are being reconfigured; windows are being
retrofitted to open; in upper grades, students will be cycled between
in-person classes and distance learning. At public schools, entire
buildings have been deep cleaned, with custodians working throughout the
pandemic and the summer break.

Image

A high school science classroom at the Washington Heights Expeditionary
Learning School.

Image

The auditorium at Port Morris School of Community Leadership in the
Bronx. Using spaces like these for teaching can enable social
distancing.

Image

A seventh grade classroom at Port Morris School of Community Leadership.
In a way, each classroom still feels occupied, as though students are a
moment away from returning from lunch.

Image

The school assembly hall at Friends Seminary, founded by Quakers. ``The
pandemic has, by a long shot, been the greatest challenge of my
career,'' said Bo Lauder, the head of school.

Image

``We are all frightened to some degree, and the isolation that we are
experiencing adds pressure for us all. Humans aren't wired to live this
way,'' said Mr. Lauder.

Administrators work on strategies and multiple alternate approaches, all
while trying to communicate plans that change daily to school boards,
parents and children. With so little clarity on the virus, the health
risk of exposure and the developmental risks of staying at home, it
appears that every option will upset as many people as it appeases.

``I spend almost every minute in a state of concern,'' said Mr. Lauder.
``The only way I can see to err totally on the side of caution is to go
all distance learning all the time, but we know that is not good for
faculty, students or families.''

Image

An eighth grade classroom at the Washington Heights Expeditionary
Learning School.

In the South Bronx, Principal Keane has taken to parking out of sight
behind the building when she comes to work. If the neighborhood kids see
her car out front they try to come to school.

``It turns out the one place they didn't think they wanted to be,'' she
said, ``is the place they want to be more than anywhere else.''

Ashley Gilbertson
(\href{https://twitter.com/ashgilbertson?lang=en}{@AshGilbertson}) is a
member of the VII Photo agency.

\emph{The Times is committed to publishing}
\href{https://www.nytimes3xbfgragh.onion/2019/01/31/opinion/letters/letters-to-editor-new-york-times-women.html}{\emph{a
diversity of letters}} \emph{to the editor. We'd like to hear what you
think about this or any of our articles. Here are some}
\href{https://help.nytimes3xbfgragh.onion/hc/en-us/articles/115014925288-How-to-submit-a-letter-to-the-editor}{\emph{tips}}\emph{.
And here's our email:}
\href{mailto:letters@NYTimes.com}{\emph{letters@NYTimes.com}}\emph{.}

\emph{Follow The New York Times Opinion section on}
\href{https://www.facebookcorewwwi.onion/nytopinion}{\emph{Facebook}}\emph{,}
\href{http://twitter.com/NYTOpinion}{\emph{Twitter (@NYTopinion)}}
\emph{and}
\href{https://www.instagram.com/nytopinion/}{\emph{Instagram}}\emph{.}

Advertisement

\protect\hyperlink{after-bottom}{Continue reading the main story}

\hypertarget{site-index}{%
\subsection{Site Index}\label{site-index}}

\hypertarget{site-information-navigation}{%
\subsection{Site Information
Navigation}\label{site-information-navigation}}

\begin{itemize}
\tightlist
\item
  \href{https://help.nytimes3xbfgragh.onion/hc/en-us/articles/115014792127-Copyright-notice}{©~2020~The
  New York Times Company}
\end{itemize}

\begin{itemize}
\tightlist
\item
  \href{https://www.nytco.com/}{NYTCo}
\item
  \href{https://help.nytimes3xbfgragh.onion/hc/en-us/articles/115015385887-Contact-Us}{Contact
  Us}
\item
  \href{https://www.nytco.com/careers/}{Work with us}
\item
  \href{https://nytmediakit.com/}{Advertise}
\item
  \href{http://www.tbrandstudio.com/}{T Brand Studio}
\item
  \href{https://www.nytimes3xbfgragh.onion/privacy/cookie-policy\#how-do-i-manage-trackers}{Your
  Ad Choices}
\item
  \href{https://www.nytimes3xbfgragh.onion/privacy}{Privacy}
\item
  \href{https://help.nytimes3xbfgragh.onion/hc/en-us/articles/115014893428-Terms-of-service}{Terms
  of Service}
\item
  \href{https://help.nytimes3xbfgragh.onion/hc/en-us/articles/115014893968-Terms-of-sale}{Terms
  of Sale}
\item
  \href{https://spiderbites.nytimes3xbfgragh.onion}{Site Map}
\item
  \href{https://help.nytimes3xbfgragh.onion/hc/en-us}{Help}
\item
  \href{https://www.nytimes3xbfgragh.onion/subscription?campaignId=37WXW}{Subscriptions}
\end{itemize}
