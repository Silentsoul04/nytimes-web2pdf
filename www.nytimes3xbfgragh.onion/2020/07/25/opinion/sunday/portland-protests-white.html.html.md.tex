Sections

SEARCH

\protect\hyperlink{site-content}{Skip to
content}\protect\hyperlink{site-index}{Skip to site index}

\href{https://www.nytimes3xbfgragh.onion/section/opinion/sunday}{Sunday
Review}

\href{https://myaccount.nytimes3xbfgragh.onion/auth/login?response_type=cookie\&client_id=vi}{}

\href{https://www.nytimes3xbfgragh.onion/section/todayspaper}{Today's
Paper}

\href{/section/opinion/sunday}{Sunday Review}\textbar{}Who Gets to Be a
`Naked Athena'?

\url{https://nyti.ms/2CR7R0n}

\begin{itemize}
\item
\item
\item
\item
\item
\end{itemize}

Advertisement

\protect\hyperlink{after-top}{Continue reading the main story}

\href{/section/opinion}{Opinion}

Supported by

\protect\hyperlink{after-sponsor}{Continue reading the main story}

\hypertarget{who-gets-to-be-a-naked-athena}{%
\section{Who Gets to Be a `Naked
Athena'?}\label{who-gets-to-be-a-naked-athena}}

On weirdness, whiteness and federal agents in Portland.

By Mitchell S. Jackson

Mr. Jackson is a writer.

\begin{itemize}
\item
  July 25, 2020
\item
  \begin{itemize}
  \item
  \item
  \item
  \item
  \item
  \end{itemize}
\end{itemize}

\includegraphics{https://static01.graylady3jvrrxbe.onion/images/2020/07/26/opinion/25jackson1/merlin_174705171_8f925c80-a093-4aa2-80cb-c60cc9db99aa-articleLarge.jpg?quality=75\&auto=webp\&disable=upscale}

``Naked Athena.'' Have you heard of her? She's the woman who was so
christened after she strolled into a recent Portland protest --- one
that was ostensibly, crucially, about Black lives --- stark naked, save
a mask (kudos to that) and skullcap. She sat down with her legs wide,
and proceeded to do some yoga poses. Some say she was putting herself
between protesters and police, that she was turning the cultural
sacredness of a white (or at least a white-passing) woman's body into a
shield against rubber bullets and tear gas.

Naked Athena --- whose friend describes her as a light-skinned person of
color and outspoken feminist --- said nada during her demonstration and
hasn't been interviewed, so I can't know her intentions. What I can say
with confidence is that what she did was aligned with the ``weird'' that
Portland espouses in its beloved slogan: ``Keep Portland Weird.'' What I
can say with reasonable assurance is that, were she a Black woman, she
would've reaped a different public reaction than the ample awe and
admiration I've seen on social media. And what I \emph{must} say is that
no matter her intentions, for a moment at least, she might've upstaged
the movement, and not in a way I could discern as connected to its
stated objectives.

Don't get me wrong, I appreciate Naked Athena, and the
\href{https://uk.reuters.com/article/uk-global-race-protests-portland-veteran/navy-veteran-says-he-was-beaten-like-a-punching-bag-in-portland-idUKKCN24L2DN}{white
Navy veteran} whose passivity exposed the bellicose bent of federal
agents. I'm thankful for the passion and courage of other white allies
during this movement.

But I've also been musing on the subject of weirdness --- how that
quality requires freedom, or at least the belief that one possesses it.
How the ability to express passion and courage and weirdness is a
product of that privilege; how a sense of utopianism of the sort that
exists for white people in Portland, my hometown, leads to a certain
audacity when it comes to both self-expression and political radicalism;
how that audacity can make a city into a tempting target for a federal
government that's determined to look tough against a purported paragon
of eccentric liberalism.

Let's be clear: Oregon was intended as a white man's Zion. And since its
admission into the union, it has remained one. That isn't intended to
distract from, or in any way excuse, the ongoing state violence there;
it's just that there should be no serious discussion of my home state or
what's happening in my home city that excludes or forgets its founding
ethos.

Oregon Country's provisional government passed a law excluding Blacks
from the territory and, though it voted against slavery, thanks to a
member of its first provisional government --- a former slave holder
from Missouri --- it amended this law to disallow Blacks from remaining
within its borders beyond a three-year residence. You wouldn't know
unless you Sherlocked that Oregon once boasted the
\href{https://www.wweek.com/arts/2017/08/17/oregon-was-once-ku-ku-for-the-klan/}{largest
KKK chapter} west of the Mississippi, that it waited over 100 years
after the Civil War to ratify the 14th Amendment; it took almost 90
years to ratify the 15th.

In the years since, Oregon's largest city has done a bang-up job of
marketing itself as a bastion of lefty quirkiness as well as a place for
great food, beautiful landscapes, formidable cultural scenes and, of
course, Just Doing It. But the laws keeping black people out? Oregonians
didn't vote to scrub them from the state's books 'til 2002.

Per the
\href{https://www.census.gov/quickfacts/fact/table/portlandcityoregon,OR/PST045219}{latest
U.S. census statistics}, Oregon is 86.7 percent white, and 2.2 percent
Black. Portland itself is 77.1 percent white and 5.8 percent Black.
That's why the Black Lives Matter protests there look like they do ---
white. They have to; that's who lives there.

But in a monolith, it's even easier for white people to center
themselves at the expense of those they claim to support. That must make
it harder to know where the line is between amplifying a voice and
becoming the voice, between ardent allyship and white saviorship,
between the values of a cause and the culture of a city. But the
difficult thing, the complicated thing, is this movement can't afford to
be distorted by ``weird.''

\includegraphics{https://static01.graylady3jvrrxbe.onion/images/2020/07/25/opinion/25jackson2/merlin_174823803_54343c56-6a02-4240-8ad0-ba8f3130a69f-articleLarge.jpg?quality=75\&auto=webp\&disable=upscale}

My beloved City of Roses made a great showing at the outset of the Black
Lives Matter protests; you might've seen them gathered in a
\href{https://www.oregonlive.com/news/2020/06/the-world-needs-to-see-this-the-story-behind-the-iconic-photo-of-the-burnside-bridge-protest-in-portland.html}{thousands-strong
die}-in on the Burnside Bridge, a preponderance of white faces turned
downward in an apt symbol of George Floyd, pinned and pleading, under
the knee of Derek Chauvin. It made me proud to witness my city's
collective conscience over the tragic death of a Black man in far-off
Minneapolis.

But I've felt a bit more ambivalent about the past 50-some days of
protests since. A small few have employed
\href{https://abcnews.go.com/US/wireStory/portland-oregon-protest-turns-violent-arrested-71594643}{anarchist
tactics}, and/or seem to have lost the vision of a unified agenda. And
I've seen nary national coverage of the smaller marches or activism led
by Blacks and other people of color out in the Numbers: what we call the
part of the city that Black people were dispersed to when whites
gentrified my old neighborhood.

And now, the feds are there.

When I hear Keep Portland Weird, it always sounds to me a lot like Keep
Portland White. But I imagine for the
\href{https://www.census.gov/quickfacts/fact/table/US/PST045219}{76.3
percent of Americans who still claim white alone on the census}, it
sounds like Keep Portland a Symbol. Portland is Portlandia. Portland is
the new frontier for migrating Brooklyn hipsters. Portland is Bush Sr.'s
``Little Beirut,'' the same place where almost all-white Antifa
activists once battled neo-fascist Proud Boys. Portland whiteness: It
leans way left but stretches far right.

It's the opposite of ironic, isn't it? A president who has defended
white supremacists and championed white-power-esque policies sent
federal agents to a notable bulwark of liberal whiteness, a place
engaged in brazen support of a movement pursuing Black freedom. The
footage has been straight terrifying: Agents instigating violence,
abducting people into unmarked cars, providing more evidence of an
administration trooping double-time toward totalitarianism. Can you
imagine if Trump dispatches these tactics to Chicago and Albuquerque, to
blacker and browner cities elsewhere?

Let me back up: This ain't me arguing that whiteness always leads to
weirdness; that weirdness is necessarily connected to anarchy, and,
hell, even anarchists don't excuse fascism. People ringing the alarm
about what's happening in Portland are right. But Portland's racial
dynamics aren't a distraction from the real story of what's happening
there; they're at the heart of it. And what bothers me is that, amid the
naked woman, the brave white veterans, the heroic wall of lullabying
white moms, the
\href{https://www.nytimes3xbfgragh.onion/2020/07/23/us/portland-protest-tear-gas-mayor.html}{tear-gassed
mayor}, and the unidentified federal agents, we've once again stopped
discussing the fight against institutional racism and state-sponsored
violence against Black people in this country.

Those objectives were on my mind in mid-June when my homeboy forwarded
me a clip of Portland protesters toppling a statue of Thomas Jefferson
at his eponymous high school --- the Oregon high school with the largest
share of Black students, and where I graduated in 1993 (One time for the
Demos!). Go figure, white men were part of the small crowd that cheered
and tugged the statue on the ground and bashed it.

And peep this: I must've passed that statue a hundred-plus times over
the years, and not a once did it occur to me that I could do a damn
thing about its flagrance.

Amen that it did to those audacious few. These decades hence, I've
realized that, though born and raised in the Rose City, it has never
been my utopia, and in truth was never meant to be. It's only ever been
a home: where whiteness hovers over us Black folk, as perennial as old
Jefferson's duplicitous self-evident truth.

Mitchell S. Jackson is the author of \emph{The Residue Years} and
\emph{Survival Math}. His next novel \emph{John of Watts} is
forthcoming. He teaches creative writing at the University of Chicago.

\emph{The Times is committed to publishing}
\href{https://www.nytimes3xbfgragh.onion/2019/01/31/opinion/letters/letters-to-editor-new-york-times-women.html}{\emph{a
diversity of letters}} \emph{to the editor. We'd like to hear what you
think about this or any of our articles. Here are some}
\href{https://help.nytimes3xbfgragh.onion/hc/en-us/articles/115014925288-How-to-submit-a-letter-to-the-editor}{\emph{tips}}\emph{.
And here's our email:}
\href{mailto:letters@NYTimes.com}{\emph{letters@NYTimes.com}}\emph{.}

\emph{Follow The New York Times Opinion section on}
\href{https://www.facebookcorewwwi.onion/nytopinion}{\emph{Facebook}}\emph{,}
\href{http://twitter.com/NYTOpinion}{\emph{Twitter (@NYTopinion)}}
\emph{and}
\href{https://www.instagram.com/nytopinion/}{\emph{Instagram}}\emph{.}

Advertisement

\protect\hyperlink{after-bottom}{Continue reading the main story}

\hypertarget{site-index}{%
\subsection{Site Index}\label{site-index}}

\hypertarget{site-information-navigation}{%
\subsection{Site Information
Navigation}\label{site-information-navigation}}

\begin{itemize}
\tightlist
\item
  \href{https://help.nytimes3xbfgragh.onion/hc/en-us/articles/115014792127-Copyright-notice}{©~2020~The
  New York Times Company}
\end{itemize}

\begin{itemize}
\tightlist
\item
  \href{https://www.nytco.com/}{NYTCo}
\item
  \href{https://help.nytimes3xbfgragh.onion/hc/en-us/articles/115015385887-Contact-Us}{Contact
  Us}
\item
  \href{https://www.nytco.com/careers/}{Work with us}
\item
  \href{https://nytmediakit.com/}{Advertise}
\item
  \href{http://www.tbrandstudio.com/}{T Brand Studio}
\item
  \href{https://www.nytimes3xbfgragh.onion/privacy/cookie-policy\#how-do-i-manage-trackers}{Your
  Ad Choices}
\item
  \href{https://www.nytimes3xbfgragh.onion/privacy}{Privacy}
\item
  \href{https://help.nytimes3xbfgragh.onion/hc/en-us/articles/115014893428-Terms-of-service}{Terms
  of Service}
\item
  \href{https://help.nytimes3xbfgragh.onion/hc/en-us/articles/115014893968-Terms-of-sale}{Terms
  of Sale}
\item
  \href{https://spiderbites.nytimes3xbfgragh.onion}{Site Map}
\item
  \href{https://help.nytimes3xbfgragh.onion/hc/en-us}{Help}
\item
  \href{https://www.nytimes3xbfgragh.onion/subscription?campaignId=37WXW}{Subscriptions}
\end{itemize}
