\href{/section/opinion/sunday}{Sunday Review}\textbar{}The Pandemic
Could Make Political Conventions Less Terrible

\url{https://nyti.ms/2BuKN6S}

\begin{itemize}
\item
\item
\item
\item
\item
\end{itemize}

\includegraphics{https://static01.graylady3jvrrxbe.onion/images/2020/07/25/opinion/25convention_virtual/25convention_virtual-superJumbo.jpg}

Credit...Nicholas Konrad/the New York Times

Sections

\protect\hyperlink{site-content}{Skip to
content}\protect\hyperlink{site-index}{Skip to site index}

\href{/section/opinion}{Opinion}

\hypertarget{the-pandemic-could-make-political-conventions-less-terrible}{%
\section{The Pandemic Could Make Political Conventions Less
Terrible}\label{the-pandemic-could-make-political-conventions-less-terrible}}

The president did America a favor, in more ways than one, by canceling
much of this year's Republican National Convention.

Credit...Nicholas Konrad/the New York Times

Supported by

\protect\hyperlink{after-sponsor}{Continue reading the main story}

By
\href{https://www.nytimes3xbfgragh.onion/interactive/opinion/editorialboard.html}{The
Editorial Board}

The editorial board is a group of opinion journalists whose views are
informed by expertise, research, debate and certain longstanding ****
\href{https://www.nytimes3xbfgragh.onion/interactive/2018/opinion/editorialboard.html?module=inline}{values}.
It is separate from the newsroom.

\begin{itemize}
\item
  July 25, 2020
\item
  \begin{itemize}
  \item
  \item
  \item
  \item
  \item
  \end{itemize}
\end{itemize}

President Trump announced on Thursday that, in deference to the
pandemic, he was canceling the portion of the Republican National
Committee's nominating convention scheduled to take place in
Jacksonville, Fla., late next month.

``We won't do a big, crowded convention, per se --- it's not the right
time for that,'' the president said during his daily coronavirus
briefing, noting that he ``felt it was wrong'' to have hordes of people
heading into ``a hot spot.''
\href{https://www.nytimes3xbfgragh.onion/2020/07/23/us/politics/jacksonville-rnc.html}{Mr.
Trump added he'd told} his advisers, ``There's nothing more important in
our country than keeping our people safe.''

Better late than never.

Mr. Trump's coronation party
\href{https://www.nytimes3xbfgragh.onion/2020/06/06/us/politics/republican-convention-charlotte-2020.html}{originally
was planned} for Charlotte, N.C., which is where much of the
convention's official business will still take place. In June, however,
the president
\href{https://www.nytimes3xbfgragh.onion/2020/06/11/us/politics/trump-jacksonville-rnc-speech.html}{relocated
all the flashy bits}, including his acceptance speech, to Florida, after
North Carolina officials refused to guarantee him the overcrowded,
non-socially distanced spectacle he wanted.

Florida, however, is now in the throes of a Covid-19 spike. The state
reported on Thursday \href{https://floridahealthcovid19.gov/}{10,249 new
cases} and 173 deaths, a record. Bringing thousands of conventiongoers
into the mix would have been a recipe for more tragic outcomes.

Instead of an arena full of cheering fans, Mr. Trump must content
himself with ``tele-rallies,'' other virtual events and maybe some
smaller gatherings. This is surely a bitter pill for the president, who
draws energy from large, adoring crowds. But this moment of crisis also
provides his party --- both parties, for that matter --- with an
opportunity to reimagine and reshape their conventions into something
more engaging and possibly more relevant to the American public.

The convention of conventions is
\href{https://www.nytimes3xbfgragh.onion/2020/05/04/us/politics/democratic-republican-convention.html}{overdue
for an overhaul}. Why not make necessity the mother of reinvention?

Much of what goes on at national conventions is not meant for
consumption by the general public. Once upon a time, serious nominating
business was conducted at these gatherings, but those days are gone. And
for all the quadrennial chatter about the possibility of a brokered
convention, the parties knock themselves out to avoid that kind of
drama, even in cycles with ugly primaries.

Nowadays, conventions are in large part extended reunions, awash in
booze, food, music and elbow rubbing between elected officials,
lobbyists, activists, operatives, celebrities, fund-raisers, journalists
and other players. They are, in some ways, politics at its swampiest.

The parts produced for at-home viewers are dominated by speeches ---
many of them boring, vapid or even frightening, with an eye toward
whipping up the party faithful. The lineups typically feature political
stars, up-and-comers the party wants to spotlight
(\href{https://www.nytimes3xbfgragh.onion/2004/07/27/politics/campaign/barack-obamas-remarks-to-the-democratic-national.html}{Barack
Obama} in 2004,
\href{https://www.vox.com/2016/7/26/12285312/bill-clinton-dnc-1988-speaker-late-night}{Bill
Clinton} in 1988) and members of Congress. Former primary rivals often
appear as a show of party unity, and members of the nominee's family are
trotted out. Then there are the celebrities brought in for a dash of
pizazz, like Meryl Streep, will.i.am and Katy Perry. (Such appearances
don't always go over as planned, as when Clint Eastwood conducted a
\href{https://www.washingtonpost.com/news/morning-mix/wp/2016/08/04/clint-eastwood-explains-and-regrets-his-speech-to-an-empty-chair/}{much}-\href{https://www.washingtonpost.com/blogs/compost/post/clint-eastwood-delivers-greatest-speech-in-us-history-mitt-romney-also-there/2012/08/31/cd8485fe-f320-11e1-adc6-87dfa8eff430_blog.html?itid=lk_inline_manual_6}{mocked}
\href{https://www.nytimes3xbfgragh.onion/video/us/politics/100000001752472/clint-eastwoods-rnc-speech.html}{chat
with an empty chair} at the 2012 Republican convention.)

There has got to be a better way.

As it happens, Democrats have been working on this issue for some time,
having realized several weeks ago that they needed to shift to a largely
virtual gathering. The fine-tuning is still in progress, but some
details are available. Airtime will be slashed and the speaking lineup
shortened, Joe Solmonese, the chief executive of this year's convention,
told the editorial board. ``We want to be concise and respect people's
time.''

The proceedings will also be more geographically dispersed. Delegates
and public officials aren't gathering in the host city of Milwaukee. Joe
Biden will deliver his speech from there, and his vice-presidential pick
will be on site for part of the week. But many speakers will be
scattered across battleground states and other meaningful locales, based
on each evening's theme.

``We're going to be very much grounded in the moment we're in,'' said
Mr. Solmonese. ``So when it comes time to talk about education and the
tough decisions parents will make about their kids going back to school,
we're going to go to the places those conversations are happening.'' The
same holds for the public health responders dealing with Covid-19 and
the small businesses fighting for survival, he said, noting that having
to think beyond the convention location ``creates an opportunity for us
to go where we think there are important stories to be told.''

With a nod to social distancing, the stage will feature a multiscreen
Zoom layout on which political V.I.P.s and regular Americans will
participate in a remote roll call vote. Dreamers and union members and
activists will chime in from ``iconic or message-based locations in 57
states and territories across America,'' according to an
\href{https://www.thedailybeast.com/leaked-documents-show-the-dncs-plans-for-a-reimagined-convention?ref=wrap}{internal
party memo} obtained by The Daily Beast. These will include the Edmund
Pettus Bridge in Selma, Ala., the site of the Bloody Sunday civil rights
clash in 1965.

Using resonant locations and nonfamous faces to spotlight important
issues is a smart move. Message: This election is not about partisan
games or insiders' egos. It is about the nation's collective future.

As for the themes conveyed, anything that focuses on comforting and
healing the nation is likely to play well in these unsettling times ---
and speaks to Mr. Biden's particular brand. For nonincumbents,
conventions are about introducing the nominee to voters. There will, of
course, be gauzy videos telling Mr. Biden's life story. Cutting down on
the speechifying and focusing on real people's stories is also less
likely to put viewers to sleep.

The Republicans and Mr. Trump are facing a slightly different challenge
--- with significantly less time to adapt. At this point, most Americans
already have a clear view of the president. He will not be introducing
himself to the nation so much as he will be attempting to rebrand
himself.

With his polls numbers slipping, it's clear Mr. Trump needs a retool.
For starters, he could drop the self-pitying talk about how unfair
everyone has been to him and make a positive case for why he deserves to
be re-elected. Central to this: He needs to articulate his vision and
priorities for a second term. The president has been asked this question
repeatedly of late, and he has consistently failed to offer a coherent
answer. A (virtual) convention celebrating his renomination seems the
obvious place to correct that.

Pageantry and celebrities have their place. Who doesn't love a good
balloon drop? But this year, the entire nation is under enormous strain.
Americans want to know that the presidential contenders understand and
care about their problems --- and, more than that, that they are focused
intently on how to solve those problems.

\emph{The Times is committed to publishing}
\href{https://www.nytimes3xbfgragh.onion/2019/01/31/opinion/letters/letters-to-editor-new-york-times-women.html}{\emph{a
diversity of letters}} \emph{to the editor. We'd like to hear what you
think about this or any of our articles. Here are some}
\href{https://help.nytimes3xbfgragh.onion/hc/en-us/articles/115014925288-How-to-submit-a-letter-to-the-editor}{\emph{tips}}\emph{.
And here's our email:}
\href{mailto:letters@NYTimes.com}{\emph{letters@NYTimes.com}}\emph{.}

\emph{Follow The New York Times Opinion section on}
\href{https://www.facebookcorewwwi.onion/nytopinion}{\emph{Facebook}}\emph{,}
\href{http://twitter.com/NYTOpinion}{\emph{Twitter (@NYTopinion)}}
\emph{and}
\href{https://www.instagram.com/nytopinion/}{\emph{Instagram}}\emph{.}

Advertisement

\protect\hyperlink{after-bottom}{Continue reading the main story}

\hypertarget{site-index}{%
\subsection{Site Index}\label{site-index}}

\hypertarget{site-information-navigation}{%
\subsection{Site Information
Navigation}\label{site-information-navigation}}

\begin{itemize}
\tightlist
\item
  \href{https://help.nytimes3xbfgragh.onion/hc/en-us/articles/115014792127-Copyright-notice}{©~2020~The
  New York Times Company}
\end{itemize}

\begin{itemize}
\tightlist
\item
  \href{https://www.nytco.com/}{NYTCo}
\item
  \href{https://help.nytimes3xbfgragh.onion/hc/en-us/articles/115015385887-Contact-Us}{Contact
  Us}
\item
  \href{https://www.nytco.com/careers/}{Work with us}
\item
  \href{https://nytmediakit.com/}{Advertise}
\item
  \href{http://www.tbrandstudio.com/}{T Brand Studio}
\item
  \href{https://www.nytimes3xbfgragh.onion/privacy/cookie-policy\#how-do-i-manage-trackers}{Your
  Ad Choices}
\item
  \href{https://www.nytimes3xbfgragh.onion/privacy}{Privacy}
\item
  \href{https://help.nytimes3xbfgragh.onion/hc/en-us/articles/115014893428-Terms-of-service}{Terms
  of Service}
\item
  \href{https://help.nytimes3xbfgragh.onion/hc/en-us/articles/115014893968-Terms-of-sale}{Terms
  of Sale}
\item
  \href{https://spiderbites.nytimes3xbfgragh.onion}{Site Map}
\item
  \href{https://help.nytimes3xbfgragh.onion/hc/en-us}{Help}
\item
  \href{https://www.nytimes3xbfgragh.onion/subscription?campaignId=37WXW}{Subscriptions}
\end{itemize}
