Sections

SEARCH

\protect\hyperlink{site-content}{Skip to
content}\protect\hyperlink{site-index}{Skip to site index}

\href{https://www.nytimes3xbfgragh.onion/pages/opinion/index.html\#sundayreview}{Sunday
Review}

\href{https://myaccount.nytimes3xbfgragh.onion/auth/login?response_type=cookie\&client_id=vi}{}

\href{https://www.nytimes3xbfgragh.onion/section/todayspaper}{Today's
Paper}

\href{/pages/opinion/index.html}{Sunday Review}\textbar{}A.O.C. and the
Daughter Defense

\url{https://nyti.ms/2ZYK7jt}

\begin{itemize}
\item
\item
\item
\item
\item
\end{itemize}

Advertisement

\protect\hyperlink{after-top}{Continue reading the main story}

Supported by

\protect\hyperlink{after-sponsor}{Continue reading the main story}

NEWS ANALYSIS

\hypertarget{aoc-and-the-daughter-defense}{%
\section{A.O.C. and the Daughter
Defense}\label{aoc-and-the-daughter-defense}}

Sorry, Ted Yoho. Having daughters doesn't get you a sexism free pass.

\includegraphics{https://static01.graylady3jvrrxbe.onion/images/2020/07/25/opinion/25bennett/merlin_174872130_53f84652-52ff-43f8-806a-b8a717fadd02-articleLarge.jpg?quality=75\&auto=webp\&disable=upscale}

\href{https://www.nytimes3xbfgragh.onion/by/jessica-bennett}{\includegraphics{https://static01.graylady3jvrrxbe.onion/images/2018/07/12/multimedia/author-jessica-bennett/author-jessica-bennett-thumbLarge.png}}

By \href{https://www.nytimes3xbfgragh.onion/by/jessica-bennett}{Jessica
Bennett}

Ms. Bennett is a Times editor at large.

\begin{itemize}
\item
  July 25, 2020
\item
  \begin{itemize}
  \item
  \item
  \item
  \item
  \item
  \end{itemize}
\end{itemize}

Brett Kavanaugh
\href{https://www.vox.com/2018/9/27/17911474/supreme-court-brett-kavanaugh-crying}{invoked
it}. Mitch McConnell
\href{https://www.politico.com/story/2016/10/mcconnell-trump-must-apologize-for-repugnant-comments-229332}{used
it too}. Matt Damon and Ben Affleck have
\href{https://www.vulture.com/2017/10/damon-affleck-father-of-daughters-excuse-harvey-weinstein.html}{each
talked about it}, and this week, Representative Ted Yoho joined their
ranks: He, too, is now a member of the
having-a-daughter-makes-me-an-ally-to-women --- or at the very least,
should-excuse-my-bad-behavior --- club.

``Having been married for 45 years with two daughters, I'm very
cognizant of language,'' Representative Yoho said in a speech on the
House floor this week, denying that he called Alexandria Ocasio-Cortez,
the freshman Congresswoman from New York, a ``fucking bitch''
\href{https://thehill.com/homenews/house/508259-ocaasio-cortez-accosted-by-gop-lawmaker-over-remarks-that-kind-of}{after
a confrontation on the steps of the Capitol}.

Mr. Yoho later expressed regret for the ``abrupt manner of the
conversation,'' in which he told Ms. Ocasio-Cortez that her statements
about poverty and crime in New York City were ``disgusting.'' But, he
noted, ``I cannot apologize for my passion or for loving my God, my
family and my country.''

On Thursday, in
\href{https://www.nytimes3xbfgragh.onion/2020/07/23/us/politics/aoc-women-ted-yoho.html}{a
speech on the House floor} that has since gone viral --- in which she
read the vulgarity into the Congressional record --- Ms. Ocasio-Cortez
said, ``I am someone's daughter too.'' She said she'd planned to ignore
the insults --- it's ``just another day'' as a woman, she said --- but
changed her mind after Mr. Yoho decided to bring his wife and daughters
into the fray.

Our culture is full of platitudes about fathers and daughters: the
Hallmark card, the weeping dad at the wedding. But invoking daughters
and wives to deflect criticism is a particular kind of political trope
--- and one that's been used throughout history to ``excuse a host of
bad behavior,'' said the historian Barbara Berg.

The love a man has for the female members of his family, particularly
his offspring, is presumed to have special power --- to humanize the
other half of the population, to allow him to imagine the world his
daughter will inhabit. Sometimes, in fact, this happens. Other times,
the Daughter Excuse comes across mostly as cynical ploy.

``As if familial affiliation alone equals enlightened attitudes towards
women,'' said Susan Douglas, a professor of communication and media at
the University of Michigan. ``It's like claiming `I have a Black
friend‚' as if that makes you anti-racist.''

There is social science that's shown there is \emph{something} to being
the father of a daughter.

In a study called
``\href{https://academic.oup.com/poq/article-abstract/82/3/493/5095700}{The
First-Daughter Effect},'' Elizabeth Sharrow, an associate professor of
public policy and history at the University of Massachusetts, Amherst,
and her colleagues, determined that fathering daughters --- and
firstborn daughters, in particular --- indeed played a role in making
men's attitudes toward gender equality more progressive, particularly
when it came to policies like equal pay or sexual harassment protocols.
The researchers also determined that those dads of firstborn daughters
were, in 2016, more likely to support Hillary Clinton or a fictional
female congressional candidate delivering a similar pitch.

``Our argument is not that it is genetics or biology, but that it is
proximity,'' said Dr. Sharrow. In other words: The daughters help the
fathers see the problems they may have previously dismissed.

Witness basketball
star\href{https://www.washingtonpost.com/news/monkey-cage/wp/2018/09/14/yes-stephen-curry-is-right-having-a-daughter-does-change-fathers-political-outlooks-but-only-if-shes-the-firstborn/}{Stephen
Curry}, who has
\href{https://www.theplayerstribune.com/en-us/articles/stephen-curry-womens-equality}{written}
about how ``the idea of women's equality has become a little more
personal for me, lately, and a little more real,'' since having a
daughter.

Or Dick Cheney, whose views on same-sex marriage shifted earlier than
many might have expected
\href{https://www.washingtonpost.com/arts-entertainment/2018/12/27/vice-looks-back-how-dick-cheneys-daughter-mary-shaped-his-views-same-sex-marriage/}{because
of his daughter}, who is gay.

And yet.

Daughters influencing fathers' views for the better is far different
from fathers using their daughters as ``shields and excuses for poor
behavior,'' as Ms. Ocasio-Cortez described Mr. Yoho in her speech.

It's also different from fathers using them as ``props,'' as Dr. Berg
puts it, to emphasize their alignment with women's causes --- or, by
contrast, their disgust over behaviors perceived to be in opposition to
them.

Consider Justice Kavanaugh, who --- during his testimony before the
Senate Judiciary Committee about allegations of sexual assault by
Christine Blasey Ford --- spoke repeatedly of his daughters (as well as
his wife and mother) and noted that coaching his daughter's basketball
team was what he loved ``more than anything I've ever done in my whole
life'' --- as if loving coaching and allegedly treating women badly as a
teenager are mutually exclusive.

``Men have often pointed to their relationships with and love for
\emph{some} women --- especially wives and daughters --- to combat
claims that they have mistreated \emph{other} women,'' said Kelly
Dittmar, a scholar at the Center for American Women and Politics at
Rutgers University. ``We have seen this both inside and outside of
politics, especially when men are subject to accusations of sexual
harassment and assault.''

In the wake of the 2016 reports on comments made by Donald Trump on the
now-infamous ``Access Hollywood'' tape, a host of fathers-of-daughters
came out to condemn the behavior. Mr. McConnell noted that ``as the
father of three daughters'' he believed that Mr. Trump ``needs to
apologize directly to women and girls everywhere,'' while Mitt Romney
said that the comments
``\href{https://twitter.com/MittRomney/status/784546373525966849}{demean
our wives and daughters}.'' (It is perhaps worth noting that Mr. Trump,
too, has daughters.)

Similarly, in response to revelations of sexual misconduct by Harvey
Weinstein, both Ben Affleck and Matt Damon, who had worked with the
disgraced Hollywood producer, expressed their disgust on behalf of their
female offspring. We ``need to do better at protecting our friends,
sisters, co-workers and daughters,'' Mr. Affleck
\href{https://twitter.com/BenAffleck/status/917787533802655744}{said on
Twitter}, while Mr. Damon
\href{https://deadline.com/2017/10/matt-damon-harvey-weinstein-russell-crowe-sexual-abuse-scandal-interview-1202185574/}{explained}
that ``as the father of four daughters, this is the kind of sexual
predation that keeps me up at night.''

Women, too, have at times invoked men's daughters --- and other female
relatives --- in trying to appeal to some men. When asked about Mr.
Yoho's behavior, House Speaker Nancy Pelosi
\href{https://www.independent.co.uk/news/world/americas/us-politics/aoc-speech-today-nancy-pelosi-ted-yoho-congress-floor-a9635191.html}{said}:
``What's so funny is, you'd say to them, `Do you not have a daughter? Do
you not have a mother? Do you not have a sister? Do you not have a wife?
What makes you think that you can be so' --- and this is the word I use
for them --- `condescending, in addition to being disrespectful?'''

The caveat, of course, is the qualification. ``Qualifying your outrage
against misogyny as due to your role as a father or husband implies
that, absent those roles, you would be either unaware of or
unconcerned,'' said Dr. Dittmar.

Or as Ms. Ocasio-Cortez put it: ``Having a daughter does not make a man
decent. Having a wife does not make a decent man. Treating people with
dignity and respect makes a decent man.'' Why should daughters still
have to be a prerequisite to respect?

Jessica Bennett is a Times editor at large covering gender and culture.
She is the author of ``Feminist Fight Club'' and ``This Is 18.''

\emph{The Times is committed to publishing}
\href{https://www.nytimes3xbfgragh.onion/2019/01/31/opinion/letters/letters-to-editor-new-york-times-women.html}{\emph{a
diversity of letters}} \emph{to the editor. We'd like to hear what you
think about this or any of our articles. Here are some}
\href{https://help.nytimes3xbfgragh.onion/hc/en-us/articles/115014925288-How-to-submit-a-letter-to-the-editor}{\emph{tips}}\emph{.
And here's our email:}
\href{mailto:letters@NYTimes.com}{\emph{letters@NYTimes.com}}\emph{.}

\emph{Follow The New York Times Opinion section on}
\href{https://www.facebookcorewwwi.onion/nytopinion}{\emph{Facebook}}\emph{,}
\href{http://twitter.com/NYTOpinion}{\emph{Twitter (@NYTopinion)}}
\emph{and}
\href{https://www.instagram.com/nytopinion/}{\emph{Instagram}}\emph{.}

Advertisement

\protect\hyperlink{after-bottom}{Continue reading the main story}

\hypertarget{site-index}{%
\subsection{Site Index}\label{site-index}}

\hypertarget{site-information-navigation}{%
\subsection{Site Information
Navigation}\label{site-information-navigation}}

\begin{itemize}
\tightlist
\item
  \href{https://help.nytimes3xbfgragh.onion/hc/en-us/articles/115014792127-Copyright-notice}{©~2020~The
  New York Times Company}
\end{itemize}

\begin{itemize}
\tightlist
\item
  \href{https://www.nytco.com/}{NYTCo}
\item
  \href{https://help.nytimes3xbfgragh.onion/hc/en-us/articles/115015385887-Contact-Us}{Contact
  Us}
\item
  \href{https://www.nytco.com/careers/}{Work with us}
\item
  \href{https://nytmediakit.com/}{Advertise}
\item
  \href{http://www.tbrandstudio.com/}{T Brand Studio}
\item
  \href{https://www.nytimes3xbfgragh.onion/privacy/cookie-policy\#how-do-i-manage-trackers}{Your
  Ad Choices}
\item
  \href{https://www.nytimes3xbfgragh.onion/privacy}{Privacy}
\item
  \href{https://help.nytimes3xbfgragh.onion/hc/en-us/articles/115014893428-Terms-of-service}{Terms
  of Service}
\item
  \href{https://help.nytimes3xbfgragh.onion/hc/en-us/articles/115014893968-Terms-of-sale}{Terms
  of Sale}
\item
  \href{https://spiderbites.nytimes3xbfgragh.onion}{Site Map}
\item
  \href{https://help.nytimes3xbfgragh.onion/hc/en-us}{Help}
\item
  \href{https://www.nytimes3xbfgragh.onion/subscription?campaignId=37WXW}{Subscriptions}
\end{itemize}
