Sections

SEARCH

\protect\hyperlink{site-content}{Skip to
content}\protect\hyperlink{site-index}{Skip to site index}

\href{https://www.nytimes3xbfgragh.onion/section/us}{U.S.}

\href{https://myaccount.nytimes3xbfgragh.onion/auth/login?response_type=cookie\&client_id=vi}{}

\href{https://www.nytimes3xbfgragh.onion/section/todayspaper}{Today's
Paper}

\href{/section/us}{U.S.}\textbar{}John Lewis, Son of Alabama, Makes His
Last Journey Home

\url{https://nyti.ms/3g2ajPQ}

\begin{itemize}
\item
\item
\item
\item
\item
\item
\end{itemize}

Advertisement

\protect\hyperlink{after-top}{Continue reading the main story}

Supported by

\protect\hyperlink{after-sponsor}{Continue reading the main story}

\hypertarget{john-lewis-son-of-alabama-makes-his-last-journey-home}{%
\section{John Lewis, Son of Alabama, Makes His Last Journey
Home}\label{john-lewis-son-of-alabama-makes-his-last-journey-home}}

Days of memorials for the congressman and civil rights figure began in
Troy, the small Alabama town where he was raised. ``Rest well,'' one of
his sisters said.

\includegraphics{https://static01.graylady3jvrrxbe.onion/images/2020/07/25/us/25lewis-memorial-1/merlin_174939324_b99d22b9-07b3-46cd-b132-7406ba9638ec-videoSixteenByNine3000.jpg}

By \href{https://www.nytimes3xbfgragh.onion/by/rick-rojas}{Rick Rojas}

\begin{itemize}
\item
  July 25, 2020
\item
  \begin{itemize}
  \item
  \item
  \item
  \item
  \item
  \item
  \end{itemize}
\end{itemize}

TROY, Ala. --- In the coming days,
\href{https://www.nytimes3xbfgragh.onion/2020/07/25/us/john-lewis-memorial-service.html}{John
Lewis} will be brought to the halls of power. He will lie in state in
the Capitol in Washington, as well as in the statehouses in Alabama and
Georgia. He will be mourned by lawmakers and governors and the many
other influential figures he came to know during more than 30 years in
Congress.

But before all that, he came home.

``You know now when I look at all the accolades, the pictures I see all
the time, I think about where he came from,'' Ethel Mae Tyner, Mr.
Lewis's sister, said of her brother, the Georgia congressman and civil
rights leader, during a memorial service on Saturday in Troy, Ala., the
small town where he grew up on a farm raising cotton.

His brothers and sisters shared their pride in seeing how Mr. Lewis, who
died on July 17 at the age of 80, ascended and in the work he did along
the way. But while the world knew John Lewis the activist and
congressman, his family sought to memorialize the brother they called
Robert, his middle name, used only by those closest to him.

Robert, they said, was the boy who wanted to be a pastor and preached to
the chickens on the farm. Robert was one who was afraid of thunder and
lightning, dashing inside whenever storm clouds would fill the sky. They
saw Robert grow into the man who, as Mr. Lewis always put it, looked to
stir ``good trouble.''

His brother Samuel Lewis remembered when he left home. ``Mother told him
not to get in trouble, not to get in the way,'' he recalled. ``We all
know that John got in trouble, got in the way, but it was a good
trouble.''

The memorial service, which drew a crowd to the campus of Troy
University, was the start of a series of tributes that mirrored Mr.
Lewis's path through life. It began on Saturday with a final journey to
his home state of Alabama, and on Sunday, his body will be carried
across the Edmund Pettus Bridge in Selma, Ala., where he helped lead the
demonstrators beaten down by the authorities as they marched on March 7,
1965.

He will lie in state at the U.S. Capitol on Monday and Tuesday, and on
Wednesday, he will be brought to the Georgia Capitol in Atlanta. On
Thursday, his funeral will be held in Ebenezer Baptist Church, a
sanctuary in Atlanta with deep ties to the civil rights movement, as it
had been the home of the Rev. Dr. Martin Luther King Jr.

On Saturday, the crowd in Troy, a city of roughly 19,000 people
southeast of Montgomery, was most likely smaller than it would have been
had the coronavirus not been a factor. His family asked people to not
travel long distances to come. Still, there was a robust assembly, which
formed a line wrapping around the floor of the arena.

Bruce W. Griggs came from Atlanta with what he declared to be the
world's biggest sympathy card, standing eight feet tall. He called
people over as they walked inside, asking them to sign it. He would be
driving the card to Selma, then Washington and back to Atlanta.

He had made similar cards for George Floyd and Secoriea
Turner,\href{https://www.nytimes3xbfgragh.onion/2020/07/06/us/atlanta-mayor-8-year-old-killed.html}{the
8-year-old girl who was recently killed in Atlanta} near the same site
where Rayshard Brooks was fatally shot by the police in June, spurring
protests in the city.

This was different. ``I didn't know George Floyd,'' he said. ``I didn't
know Secoriea. This is personal.''

Mr. Griggs said that he had known the congressman since 1995, and that
Mr. Lewis had gone out of his way to support the young men who
participated in a mentoring program Mr. Griggs runs.

As one person after the next came up to sign, he pointed out the message
that stuck out to him the most, which was written in the form of a poem:

\emph{Time to come home, dear brother}\\
\emph{Your tour of duty through}\\
\emph{You've given as much as anyone}\\
\emph{Could be expected to do.}

``Now God has taken home another soldier, the last of the soldiers,''
Mr. Griggs said.

A roster of pastors and local officials were among the speakers.

``I have frankly never felt more unworthy to be in front of a
microphone,'' Jason Reeves, Troy's mayor, said during the memorial
service.

He said that he had seen some of Mr. Lewis's academic records, where a
teacher wrote that he ``appears shy but verbally says he is going on to
school to be somebody.''

``I thought about that word, `be,' and how `be' is not only a linking
verb, it's an action verb,'' he said. ``I think about all the actions he
had taken and the example he had been and the courage that it took to do
those things.''

Mr. Lewis's family also recalled a caring brother who regularly called
to check in, calls they welcomed even if they came late at night.

The night before he died, in the last conversations he had with some of
them, he asked about his nieces and nephews.

``To my brother Robert,'' Ms. Tyner, his sister, said, ``this is not a
goodbye. It's just a different kind of hello.''

``Rest well, Robert,'' she added. ``Rest well.''

Advertisement

\protect\hyperlink{after-bottom}{Continue reading the main story}

\hypertarget{site-index}{%
\subsection{Site Index}\label{site-index}}

\hypertarget{site-information-navigation}{%
\subsection{Site Information
Navigation}\label{site-information-navigation}}

\begin{itemize}
\tightlist
\item
  \href{https://help.nytimes3xbfgragh.onion/hc/en-us/articles/115014792127-Copyright-notice}{©~2020~The
  New York Times Company}
\end{itemize}

\begin{itemize}
\tightlist
\item
  \href{https://www.nytco.com/}{NYTCo}
\item
  \href{https://help.nytimes3xbfgragh.onion/hc/en-us/articles/115015385887-Contact-Us}{Contact
  Us}
\item
  \href{https://www.nytco.com/careers/}{Work with us}
\item
  \href{https://nytmediakit.com/}{Advertise}
\item
  \href{http://www.tbrandstudio.com/}{T Brand Studio}
\item
  \href{https://www.nytimes3xbfgragh.onion/privacy/cookie-policy\#how-do-i-manage-trackers}{Your
  Ad Choices}
\item
  \href{https://www.nytimes3xbfgragh.onion/privacy}{Privacy}
\item
  \href{https://help.nytimes3xbfgragh.onion/hc/en-us/articles/115014893428-Terms-of-service}{Terms
  of Service}
\item
  \href{https://help.nytimes3xbfgragh.onion/hc/en-us/articles/115014893968-Terms-of-sale}{Terms
  of Sale}
\item
  \href{https://spiderbites.nytimes3xbfgragh.onion}{Site Map}
\item
  \href{https://help.nytimes3xbfgragh.onion/hc/en-us}{Help}
\item
  \href{https://www.nytimes3xbfgragh.onion/subscription?campaignId=37WXW}{Subscriptions}
\end{itemize}
