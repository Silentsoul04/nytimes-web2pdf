Sections

SEARCH

\protect\hyperlink{site-content}{Skip to
content}\protect\hyperlink{site-index}{Skip to site index}

\href{https://www.nytimes3xbfgragh.onion/section/politics}{Politics}

\href{https://myaccount.nytimes3xbfgragh.onion/auth/login?response_type=cookie\&client_id=vi}{}

\href{https://www.nytimes3xbfgragh.onion/section/todayspaper}{Today's
Paper}

\href{/section/politics}{Politics}\textbar{}Anatomy of an Election
`Meltdown' in Georgia

\url{https://nyti.ms/3jBpqlv}

\begin{itemize}
\item
\item
\item
\item
\item
\item
\end{itemize}

\begin{itemize}
\item
  \href{https://www.nytimes3xbfgragh.onion/2020/07/24/us/elections/2020-presidential-election.html?action=click\&pgtype=Article\&state=default\&module=styln-elections-2020\&region=TOP_BANNER\&context=storylines_menu}{Election
  Updates}
\item
  \href{https://www.nytimes3xbfgragh.onion/interactive/2020/07/24/us/politics/trump-biden-campaign-donors.html?action=click\&pgtype=Article\&state=default\&module=styln-elections-2020\&region=TOP_BANNER\&context=storylines_menu}{Map
  of Donations}
\item
  \href{https://www.nytimes3xbfgragh.onion/article/biden-vice-president-2020.html?action=click\&pgtype=Article\&state=default\&module=styln-elections-2020\&region=TOP_BANNER\&context=storylines_menu}{Biden's
  V.P. Search}
\item
  \href{https://www.nytimes3xbfgragh.onion/interactive/2020/us/elections/delegate-count-primary-results.html?action=click\&pgtype=Article\&state=default\&module=styln-elections-2020\&region=TOP_BANNER\&context=storylines_menu}{Delegate
  Count}
\item
  \href{https://www.nytimes3xbfgragh.onion/interactive/2019/us/politics/2020-presidential-candidates.html?action=click\&pgtype=Article\&state=default\&module=styln-elections-2020\&region=TOP_BANNER\&context=storylines_menu}{The
  Candidates}
\item
  \href{https://www.nytimes3xbfgragh.onion/newsletters/politics?action=click\&pgtype=Article\&state=default\&module=styln-elections-2020\&region=TOP_BANNER\&context=storylines_menu}{Politics
  Newsletter}
\end{itemize}

Advertisement

\protect\hyperlink{after-top}{Continue reading the main story}

Supported by

\protect\hyperlink{after-sponsor}{Continue reading the main story}

\hypertarget{anatomy-of-an-election-meltdown-in-georgia}{%
\section{Anatomy of an Election `Meltdown' in
Georgia}\label{anatomy-of-an-election-meltdown-in-georgia}}

\includegraphics{https://static01.graylady3jvrrxbe.onion/images/2020/07/19/us/politics/00GEORGIA1/merlin_173310798_a778b947-25bc-4156-ba8b-5d45c29fa159-articleLarge.jpg?quality=75\&auto=webp\&disable=upscale}

By \href{https://www.nytimes3xbfgragh.onion/by/danny-hakim}{Danny
Hakim}, \href{https://www.nytimes3xbfgragh.onion/by/reid-j-epstein}{Reid
J. Epstein} and
\href{https://www.nytimes3xbfgragh.onion/by/stephanie-saul}{Stephanie
Saul}

\begin{itemize}
\item
  July 25, 2020, 5:00 a.m. ET
\item
  \begin{itemize}
  \item
  \item
  \item
  \item
  \item
  \item
  \end{itemize}
\end{itemize}

Last month, Daryl Marvin got his first taste of voting in Georgia.

Mr. Marvin had previously lived in Connecticut, where voting was a brisk
process measured in minutes. But on the day of the primary, June 9, he
and his wife waited four hours to vote at Park Tavern, an Atlanta
restaurant where more than 16,000 voters were consolidated into a single
precinct. An electrical engineer by training, Mr. Marvin was baffled by
what he saw when he finally got inside: a station with 15 to 20 touch
screens on which to vote but only a single scanner to process the
printed ballots.

``The scanner was the choke point,'' he said. ``Nobody thought about it,
and this is Operations Research 101. It's not very difficult to figure
it out.''

Captured in drone footage, beamed across airwaves and internet, the
interminable lines at Atlanta polling sites became an instant and
indelible omen of voting breakdown in this pandemic-challenged
presidential election year.

Elections workers described a cascade of failures as they struggled to
activate and operate Georgia's new high-tech voting system. Next came a
barrage of partisan blame-throwing: The Republican secretary of state,
Brad Raffensperger, accused the liberal-leaning Fulton County, which
includes most of Atlanta, of botching the election, while Democratic
leaders saw the fiasco as just the latest episode in Republicans'
yearslong effort to disenfranchise the state's minority voters.

Six weeks later, as the political calendar bends toward November and the
presidential campaigns look to Georgia as a possible battleground, the
faults in the state's balky elections system remain largely unresolved.
And it has become increasingly clear that what happened in June was a
collective collapse.

\includegraphics{https://static01.graylady3jvrrxbe.onion/images/2020/07/19/us/politics/00GEORGIA3/merlin_171330123_88087175-f1c6-435a-bcd1-2a1a06f7581e-articleLarge.jpg?quality=75\&auto=webp\&disable=upscale}

On-the-ground planning deficiencies emerged across the state, though
they were far and away direst in Fulton County, the state's most
populous. With a history of difficulty administering elections, Fulton
showed little ability to adjust three months into the pandemic,
struggling to process an unprecedented flood of absentee ballots and
putting out a frantic call for 250 poll workers just days before
in-person voting was held.

But an examination by The New York Times found that in the face of
repeated warnings about counties' readiness for the rollout of the
highly complex voting system, Georgia's top elections official, the
secretary of state, remained largely passive. With the clock ticking
fast toward Primary Day, Mr. Raffensperger and his office failed to
ensure that hard-pressed counties had adequate equipment or received
desperately needed support.

Training on the new \$107 million system --- a Rube Goldbergian
assemblage of interrelated components --- was widely described as
wanting. The state deployed little more than one technician per county.
And at most polling sites there was only a single scanner, with little
apparent regard for the expected turnout.

``What I experienced was a complete meltdown,'' Jacoria Borders, a
Fulton County poll worker hired the day before the election, testified
at a legislative hearing.

Questions have also emerged about the accuracy of the vote count. County
officials, good-government groups and elections experts expressed
concern that Georgia's new system failed to count some mail-in ballots
marked with check marks or X's instead of filled-in ovals. Some county
officials believe that thousands of votes could remain uncounted.

Mr. Raffensperger's office insisted it was following the guidance of the
federal Election Assistance Commission, which certified its voting
machines, on how much of an oval must be filled in for a ballot to be
reviewed. But the commission says no such guidelines exist.

``If the reports are true, something is wrong, I'm telling you,'' said a
senior elections official in another state.

It was the historical failings of the secretary of state's office that
got Georgia into this position in the first place. The state
\href{https://www.ajc.com/news/state—regional-govt--politics/companies-submit-confidential-bids-for-georgia-voting-system/SYsJ3vS3OxKYLKRw3BCDdK/}{moved
to the new system} after a federal judge found in 2018 that elections
officials ``had buried their
\href{https://www.documentcloud.org/documents/6989937-Curling-2018-Decision.html}{heads
in the sand}'' as evidence mounted that their old machines were plagued
by security flaws.

The story of Georgia's elections breakdown underscores the critical role
played by secretaries of state, generally low-profile officials whose
importance is magnified this year by twin challenges to the very act of
casting a ballot: the pandemic and the fevered legal battles in many
states over efforts to limit who can vote. Indeed, Mr. Raffensperger and
his Republican predecessors have long worked to tighten the state's
voting rules --- policies that have fallen heaviest on communities of
color, leading to continuing litigation with civil rights groups.

Central to November's election in Georgia, as the coronavirus surges
through the South, will be questions about the availability of mail-in
voting. Since the primary, Mr. Raffensperger has decided to stop sending
absentee-ballot applications to registered voters, which seems certain
to increase crowding at the polls. Instead, he plans to create a website
where voters can apply for absentee ballots, a step that will not help
many older Georgians or those without internet access.

He has also said he hopes to work with counties to deploy more
technically skilled poll workers. But legislative remedies stalled amid
partisan rancor.

\hypertarget{latest-updates-2020-election}{%
\section{\texorpdfstring{\href{https://www.nytimes3xbfgragh.onion/2020/07/24/us/elections/2020-presidential-election.html?action=click\&pgtype=Article\&state=default\&module=styln-elections-2020\&region=MAIN_CONTENT_1\&context=storylines_live_updates}{Latest
Updates: 2020
Election}}{Latest Updates: 2020 Election}}\label{latest-updates-2020-election}}

Updated 2020-07-25T01:46:31.732Z

\begin{itemize}
\tightlist
\item
  \href{https://www.nytimes3xbfgragh.onion/2020/07/24/us/elections/2020-presidential-election.html?action=click\&pgtype=Article\&state=default\&module=styln-elections-2020\&region=MAIN_CONTENT_1\&context=storylines_live_updates\#link-52155c51}{Trump
  saw his convention as a potential success story. Reality intervened.}
\item
  \href{https://www.nytimes3xbfgragh.onion/2020/07/24/us/elections/2020-presidential-election.html?action=click\&pgtype=Article\&state=default\&module=styln-elections-2020\&region=MAIN_CONTENT_1\&context=storylines_live_updates\#link-628390e8}{U.S.
  intelligence warns of foreign interference in the election. But
  Democrats say it's not enough.}
\item
  \href{https://www.nytimes3xbfgragh.onion/2020/07/24/us/elections/2020-presidential-election.html?action=click\&pgtype=Article\&state=default\&module=styln-elections-2020\&region=MAIN_CONTENT_1\&context=storylines_live_updates\#link-a66d253}{Trump
  signed several executive orders aimed at lowering U.S. drug prices.}
\end{itemize}

\href{https://www.nytimes3xbfgragh.onion/2020/07/24/us/elections/2020-presidential-election.html?action=click\&pgtype=Article\&state=default\&module=styln-elections-2020\&region=MAIN_CONTENT_1\&context=storylines_live_updates}{See
more updates}

In interviews, Mr. Raffensperger said repeatedly that he did not accept
any responsibility for hourslong lines or malfunctioning voting
equipment. He has begun an investigation of Fulton County's management
of the primary.

``This all lays on Fulton County,'' Mr. Raffensperger said. ``The
counties run their elections, and the problems in Fulton County are
problems with Fulton County and their management team, not with me.''

Georgia's Democratic leaders, though, regard Mr. Raffensperger from a
deep well of distrust.

``If there is an investigation, then the investigation should begin
where the buck stops, at the top,'' said Michael L. Thurmond, the chief
executive of DeKalb County, which encompasses parts of Atlanta and its
suburbs, and recently moved on its own to send voters absentee-ballot
applications. ``They need to investigate themselves.''

\hypertarget{a-court-intervenes}{%
\subsection{A Court Intervenes}\label{a-court-intervenes}}

Republicans took control of the secretary of state's office in 2007,
after decades of Democratic domination. They soon set about making it
harder to vote.

Image

After Karen Handel became Georgia's secretary of state, Republicans
introduced strict new voter requirements.Credit...Curtis Compton/Atlanta
Journal-Constitution, via Associated Press

The newly elected secretary, Karen Handel, implemented an ``exact
match'' system that could disqualify voters for minute differences
between their registration forms and other government documents. That
rule was overturned by the Justice Department, which found it
``\href{https://www.justice.gov/crt/voting-determination-letter-58}{seriously
flawed}'' and falling ``disproportionately on minority voters.'' But
Republicans reinstated such requirements after the Supreme Court
stripped the Justice Department of its mandate to approve changes in
voting rules.

Ms. Handel's successor, Brian Kemp,
\href{https://www.bloomberg.com/news/articles/2018-10-15/how-georgia-s-exact-match-program-was-made-possible}{aggressively
used the new powers}, and in addition
\href{https://www.nytimes3xbfgragh.onion/2019/03/06/us/politics/governor-brian-kemp-voter-suppression.html}{purged}
more than 1.4 million Georgians from the voter rolls, both bitter flash
points with civil rights groups. In 2014, he balked at accepting
thousands of registration forms collected by the New Georgia Project,
which promotes minority voting, instead starting a three-year
\href{https://www.wsbtv.com/news/local/state-launches-fraud-investigation-voter-registrat/137992052/}{investigation}
of allegations that the organization had forged voter registrations; no
wrongdoing by the group was found. Nearly all the registrations were
ultimately accepted, though many came too late for the 2014 election.

Mr. Kemp was subsequently elected governor, in 2018, in a contest marred
by charges of voter suppression. A group founded by his defeated
opponent, Stacey Abrams, has filed a lawsuit charging that the state's
electoral system is designed to be discriminatory.

But it was his office's lax oversight of Georgia's elections machinery
that was highlighted in the summer of 2016, when a cybersecurity expert
\href{https://www.politico.com/magazine/story/2017/06/14/will-the-georgia-special-election-get-hacked-215255}{named
Logan Lamb} found that he was easily able to obtain registration records
for the state's nearly seven million voters, along with passwords for
the state's central elections server.

Mr. Lamb informed state officials, but the problems were not fixed.

Image

Brian Kemp, who succeeded Ms. Handel as secretary of state, was elected
governor in 2018, in a contest marred by charges of voter
suppression.Credit...Audra Melton for The New York Times

There were other issues as well. Under Mr. Kemp, Georgia went years
without fixing a widely known security flaw in its old Diebold voting
machines that had been corrected in other states.

(The risks would be underscored in 2018 when Robert S. Mueller III, the
special counsel investigating Russian interference in the 2016 election,
indicted 12 Russian operatives who had targeted the voting systems of
Georgia and two other states.)

In 2017, an advocacy group
\href{https://www.courthousenews.com/wp-content/uploads/2017/07/voting-atlanta.pdf}{sued}
the secretary of state's office over the integrity of the voting system.
Within days, the state mysteriously
\href{https://slate.com/technology/2017/10/georgia-destroyed-election-data-right-after-a-lawsuit-alleged-the-system-was-vulnerable.html}{deleted
election data} critical to the case. The federal judge overseeing the
matter, Amy Totenberg, later said the state ``minimized, erased or
dodged'' underlying issues in the case, leaving ``critical deficiencies
and risks that impact the reliability and integrity of the voting
system.''

Last August, Judge Totenberg ordered the state to scrap its voting
machines and undertake the daunting task of starting over for 2020.

Amid the litigation, the state turned to Dominion Voting, a Denver-based
company whose lobbyists included a
\href{http://media.ethics.ga.gov/search/Lobbyist/Lobbyist_Name.aspx?\&FilerID=L20070103}{former
chief of staff} to Mr. Kemp and a former secretary of state.

While Dominion technology is widely used, both in the United States and
as far away as Mongolia, the particular system Georgia purchased is seen
by some experts
\href{https://qz.com/1661870/georgias-new-voting-tech-raises-questions-over-election-security/}{as
unnecessarily complex}, with a chain of components: a device to check in
voters, another to cast votes, another to print ballots and a fourth to
scan them.

Texas rejected a similar Dominion system, saying frequent problems
during demonstrations had raised doubts that it could be implemented
``without experiencing numerous and substantial errors,'' according to
\href{https://voterga.files.wordpress.com/2019/08/texas-sos-rejection-of-dominion.pdf}{a
state report}. But versions of the system are used in a number of
states, including Pennsylvania and California.

A primary attraction of the new machines is the paper record they create
--- an analog layer of security against a cyberattack. But some experts
see the multitude of components as more vulnerable to attack and to
technical problems.

Georgia's old system was ``absolutely one of the worst in the country,''
said J. Alex Halderman, a computer scientist who was an expert witness
for the plaintiffs in the lawsuit. The new system, he added, ``still
leaves a lot to be desired.''

\hypertarget{a-management-problem}{%
\subsection{`A Management Problem'}\label{a-management-problem}}

Cathy Cox walked into her old offices last fall for a 90-minute
demonstration of Georgia's new voting machines.

When Ms. Cox, Georgia's last Democratic secretary of state, introduced a
new voting system back in 2002, her office held demonstrations at
supermarkets, churches and county fairs. Mr. Raffensperger, she said,
had no similar agenda, despite the complexities of navigating the new
system. Instead, he planned a social-media campaign, which Ms. Cox
warned would fail to reach thousands of older and low-income Georgians
without internet access.

``Their response to that was that all older people have Facebook
accounts to talk to their grandchildren,'' she recalled.

Mr. Raffensperger's office disputed Ms. Cox's account, saying it had
modeled its approach on hers and had done regional demonstrations.

Hers was hardly the only voice of concern. In January, two months before
Georgia's originally scheduled presidential primary, county elections
administrators from across the state fretted they wouldn't have time to
train poll workers on the new, and still undelivered, machines.

Image

A poll worker at an Atlanta elementary school.Credit...Erik S
Lesser/EPA, via Shutterstock

``I'm getting a little worried,'' Sharon Gregg, the assistant elections
director in Walton County, wrote in an email thread with other state and
local elections officials. Robin Webb, the elections coordinator in Hart
County, wrote that she had yet to receive needed guidance on poll-worker
training from the state elections board. ``Some days I am in a panic
mode,'' she said.

The pandemic led Mr. Raffensperger to twice delay the presidential
primary, ultimately combining it with primaries for Georgia's other
federal races on June 9. To alleviate crowding at voting sites, he
mailed absentee ballot applications to all active registered voters, a
move supported by Democrats.

Come Election Day, the extra time afforded by the delay didn't help. At
a recent state House hearing, Danielle Wynn, a poll watcher in Floyd
County, which borders Alabama, testified that three of the four
ballot-marking devices at her location failed at one point. Poll workers
were also unprepared for a flood of questions about absentee ballots
that voters had requested but not received, and unsure what to tell
those who brought completed ballots to the polls. ``Many voters just
opted to leave without voting,'' she said.

Carol Beckham, manager of a small polling site in Carroll County, said
confusion over absentee ballots was ``just an abysmal failure'' that the
state might have helped with more public outreach. And problems she
faced getting a ballot-marking device to communicate with a printer
``would've caused chaos'' in larger precincts, she said.

Jonathan Banes, a precinct manager in DeKalb County, said he had had
only a rudimentary tutorial on the new voting machines in February,
followed by an online refresher. ``We didn't go into troubleshooting
scenarios on how to deal with technical issues,'' he said, adding that
he had been shown basics like how to ``turn the machines on, turn them
off --- that's it.''

That left him and a depleted crew of poll workers unable to start their
equipment without outside help. ``At the local and state level, there's
just not great coordination,'' he said.

The state's most populous county, Fulton, was overwhelmed by
absentee-ballot requests. Election offices also briefly closed after a
worker became fatally ill with the coronavirus. Richard L. Barron, the
county's elections director, likened the dual effort of mailing ballots
and conducting in-person voting to running two elections simultaneously
--- all with a pandemic-depleted staff.

Fulton voters waited weeks for absentee ballots from the county that
never came, or arrived damaged. After waiting a month for an absentee
ballot, Jon Ossoff, who would win the state's Democratic Senate primary,
waited four hours to vote early on June 5 at the C. T. Martin Natatorium
in Atlanta. He returned home to find that his absentee ballot had
finally arrived. Ms. Abrams said hers came with a return envelope that
was sealed.

``There are a myriad of things that happened,'' Robb Pitts, the chairman
of the Fulton County Board of Commissioners, said in an interview,
including that at the 11th hour some longtime polling venues decided
against welcoming voters amid the pandemic.

``We had to scramble about at the last minute to find new locations,''
Mr. Pitts said. ``And in some cases, we didn't have a chance to check
the power with respect to the machines, and the wattage.'' That led to
electrical failures at some sites.

Confidence has waned in Mr. Raffensperger and in Fulton's leadership.
One Democratic lawmaker in Fulton, Josh McLaurin, sought to give
lawmakers greater oversight over the county's elections, but the move
stalled in the State Senate.

Mr. Marvin, who was troubled by the lack of scanners when he voted in
the Democratic primary at Park Tavern, said, ``I don't blame the people
working there, they had to deal with grumpy voters --- and we
\emph{were} grumpy.'' It was ``a management problem,'' he added, that
``goes all the way to the top, which is the secretary of state of
Georgia.''

\hypertarget{checking-the-check-marks}{%
\subsection{Checking the Check Marks}\label{checking-the-check-marks}}

One challenge lingered after the polls closed: how mail-in ballots were
being counted. The Coalition for Good Governance is reviewing how
Dominion scanners count such ballots and is considering suing the state,
according to the group's executive director, Marilyn Marks.

``You'd be surprised how big the X's and check marks are that don't get
counted,'' she said.

Image

Some officials believe that thousands of votes could remain uncounted.

Workers examining mail-in ballots stumbled across the uncounted votes in
the days after the primary, discovering that the Dominion scanners were
programmed to ignore marks that filled in less than a certain percentage
of a ballot's black-outlined oval. The secretary of state's office said
the percentage thresholds used by Georgia ``were the same ones
certified'' by the federal Election Assistance Commission.

But Kristen Muthig, an agency spokeswoman, said there was no federal
standard. ``It varies at the state and local level and by the various
equipment,'' she said.

Mr. Raffensperger's aides seemed unsure of the threshold, saying at
various points that it was 12 percent, 13 percent or 14 percent. In
Colorado, by comparison, any black-colored ballot ovals that are less
than 9 percent filled in are automatically not counted by scanners.

In Georgia's Morgan County, Jeanne Dufort, a former textile importer who
served on an appointed vote review panel --- and who
\href{https://morgancountycitizen.com/2018/10/19/resident-joins-suit-against-kemp/}{has
challenged} state voting practices in the past --- reviewed 150 ballots
and counted about 20 unrecorded votes.

Adam Shirley, working on a similar panel in Athens-Clarke County, said
his group reviewed 76 ballots and found 12 that included votes clearly
visible to the human eye that had not been counted by the scanners. Mr.
Shirley, a science teacher, worries about what could happen statewide in
November.

``If you start talking about a few hundred votes per county, times
159,'' he said, ``that could swing the election.''

\hypertarget{the-next-election}{%
\subsection{The Next Election}\label{the-next-election}}

To Ms. Abrams, the Democrat who has become the state's most visible
voting-rights advocate, election problems were foreseeable, as was the
fact they would not be foreseen.

``He had months to get this election right,'' she said of Mr.
Raffensperger. ``He rescheduled the primary twice and he still failed.''

``He, of his own volition, invested in a complicated machine, a purchase
that he inadequately prepared the counties for adoption of,'' she added.
``He did not provide support when counties raised their hands and raised
their voices saying, `We are concerned about the deployment of these
brand-new machines in what will likely be among the most contentious
elections we've had in 20 years.'''

Since the primary, only limited action has been taken. Legislative
remedies were derailed after Republicans tried to bar counties from
mailing out ballot applications. (``That sent the bill into the ditch,''
said Mary Margaret Oliver, a Democratic member of the state House.)

Echoing President Trump, state Republicans have argued, without
evidence, that mail-in voting is a recipe for widespread fraud. This
spring,
\href{https://thehill.com/homenews/state-watch/490879-georgias-gop-house-speaker-says-vote-by-mail-system-would-be-devastating}{David
Ralston}, the speaker of Georgia's House, expressed opposition to
sending out mail-in ballot applications amid the pandemic, saying it
would ``be extremely devastating to Republicans and conservatives.''

Mr. Raffensperger was asked in the interviews whether he was concerned
about his office's poor relations with Black voters. ``I believe that
I've been very open-minded and fair,'' he said, noting that he had
demonstrated some voting machines at the King Center in Atlanta, a
foundation run by the family of the Rev. Dr. Martin Luther King Jr.
``Whenever we can reach out to groups, we've been very proactive.''

Image

State Farm Arena, home of the Atlanta Hawks NBA team, is an early-voting
site for the August primary runoff and the general
election.Credit...Erik S Lesser/EPA, via Shutterstock

Some of Georgia's best hopes might come from outside government. The
Atlanta Hawks NBA franchise recently announced that its State Farm Arena
would become an early-voting location for August's primary runoff and
the general election, potentially accommodating 250 voting machines.
(State law restricts Election Day voting to local precincts.)

Evan Malbrough, a 2020 graduate of Georgia State University, has started
recruiting and training a cavalry of college students to be Atlanta poll
workers. Spurred by what he calls a desire to ``fix voter suppression,''
he recently founded the Georgia Youth Poll Worker Project, figuring that
students would be more adept at voting technology than typically older
poll workers.

``It's like when you live with your parents, and they got a new phone or
a new TV, and you have to set it up,'' said Mr. Malbrough, who knows
from experience --- he lives with his parents, and has also been a poll
worker. ``Even without training, I feel like most young people could
intuitively fix a lot of these issues.''

In suburban DeKalb, Mr. Thurmond, the chief executive, hopes to avoid a
repeat of what he calls the ``meltdown'' in June. Under his plan, poll
workers will be designated ``front-line workers'' and receive hazard
pay. He is looking for new polling places with greater electrical
capacity.

For Mr. Thurmond, a former state labor commissioner who is one of the
few Black people ever elected statewide, the struggle over voting is
personal.

``It's embedded in the DNA, here in Georgia, to suppress --- first to
deny, and then to suppress --- the African-American vote,'' he said.
``It's a sad and embarrassing part of the political history of the state
of Georgia. It's unfortunate that in 2020 we're still engaged in this
fight.''

\hypertarget{our-2020-election-guide}{%
\section{Our 2020 Election Guide}\label{our-2020-election-guide}}

Updated July 24, 2020

\begin{itemize}
\item
  \begin{center}\rule{0.5\linewidth}{\linethickness}\end{center}

  \hypertarget{the-latest}{%
  \subsection{The Latest}\label{the-latest}}

  \begin{itemize}
  \tightlist
  \item
    Kimberly
    Guilfoyle's\href{https://www.nytimes3xbfgragh.onion/2020/07/24/us/politics/kimberly-guilfoyles-trump-fundraising.html?action=click\&pgtype=Article\&state=default\&module=styln-elections-2020\&region=BELOW_MAIN_CONTENT\&context=storylines_guide}{}\href{https://www.nytimes3xbfgragh.onion/2020/07/24/us/politics/kimberly-guilfoyles-trump-fundraising.html?action=click\&pgtype=Article\&state=default\&module=styln-elections-2020\&region=BELOW_MAIN_CONTENT\&context=storylines_guide}{fund-raising
    for President Trump} draws scrutiny.
  \end{itemize}
\item
  \begin{center}\rule{0.5\linewidth}{\linethickness}\end{center}

  \hypertarget{bidens-vp-search}{%
  \subsection{Biden's V.P. Search}\label{bidens-vp-search}}

  \begin{itemize}
  \tightlist
  \item
    \href{https://www.nytimes3xbfgragh.onion/article/biden-vice-president-2020.html?action=click\&pgtype=Article\&state=default\&module=styln-elections-2020\&region=BELOW_MAIN_CONTENT\&context=storylines_guide}{Here
    are 13 women} who have been under consideration to be Joe Biden's
    running mate, and why each might be chosen --- and might not be.
  \end{itemize}
\item
  \begin{center}\rule{0.5\linewidth}{\linethickness}\end{center}

  \hypertarget{keep-up-with-our-coverage}{%
  \subsection{Keep Up With Our
  Coverage}\label{keep-up-with-our-coverage}}

  \begin{itemize}
  \tightlist
  \item
    Get an
    \href{https://www.nytimes3xbfgragh.onion/newsletters/politics?action=click\&pgtype=Article\&state=default\&module=styln-elections-2020\&region=BELOW_MAIN_CONTENT\&context=storylines_guide}{email}
    recapping the day's news
  \end{itemize}

  \begin{itemize}
  \tightlist
  \item
    Download our mobile app on
    \href{https://apps.apple.com/us/app/nytimes/id284862083?ls=1\&mat_click_id=5c79ae7455014fd1bd66b5610c05b8f2-20191112-16948\&referrer=mat_click_id\%3D5c79ae7455014fd1bd66b5610c05b8f2-20191112-16948\%26link_click_id\%3D722930677036718082}{iOS}
    and
    \href{http://a.localytics.com/android?id=com.nytimes.android\&referrer=utm_source\%3Dother_nyt_mobile_web\%26utm_medium\%3DWeb\%2520page\%26utm_term\%3DGeneral\%2520Mobile\%2520Page\%26utm_campaign\%3DNYT\%2520Mobile\%2520General\%2520Page}{Android}
    and turn on Breaking News and Politics alerts
  \end{itemize}
\end{itemize}

Advertisement

\protect\hyperlink{after-bottom}{Continue reading the main story}

\hypertarget{site-index}{%
\subsection{Site Index}\label{site-index}}

\hypertarget{site-information-navigation}{%
\subsection{Site Information
Navigation}\label{site-information-navigation}}

\begin{itemize}
\tightlist
\item
  \href{https://help.nytimes3xbfgragh.onion/hc/en-us/articles/115014792127-Copyright-notice}{©~2020~The
  New York Times Company}
\end{itemize}

\begin{itemize}
\tightlist
\item
  \href{https://www.nytco.com/}{NYTCo}
\item
  \href{https://help.nytimes3xbfgragh.onion/hc/en-us/articles/115015385887-Contact-Us}{Contact
  Us}
\item
  \href{https://www.nytco.com/careers/}{Work with us}
\item
  \href{https://nytmediakit.com/}{Advertise}
\item
  \href{http://www.tbrandstudio.com/}{T Brand Studio}
\item
  \href{https://www.nytimes3xbfgragh.onion/privacy/cookie-policy\#how-do-i-manage-trackers}{Your
  Ad Choices}
\item
  \href{https://www.nytimes3xbfgragh.onion/privacy}{Privacy}
\item
  \href{https://help.nytimes3xbfgragh.onion/hc/en-us/articles/115014893428-Terms-of-service}{Terms
  of Service}
\item
  \href{https://help.nytimes3xbfgragh.onion/hc/en-us/articles/115014893968-Terms-of-sale}{Terms
  of Sale}
\item
  \href{https://spiderbites.nytimes3xbfgragh.onion}{Site Map}
\item
  \href{https://help.nytimes3xbfgragh.onion/hc/en-us}{Help}
\item
  \href{https://www.nytimes3xbfgragh.onion/subscription?campaignId=37WXW}{Subscriptions}
\end{itemize}
