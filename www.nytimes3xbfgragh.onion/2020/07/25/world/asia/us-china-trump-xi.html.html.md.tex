Sections

SEARCH

\protect\hyperlink{site-content}{Skip to
content}\protect\hyperlink{site-index}{Skip to site index}

\href{https://www.nytimes3xbfgragh.onion/section/world/asia}{Asia
Pacific}

\href{https://myaccount.nytimes3xbfgragh.onion/auth/login?response_type=cookie\&client_id=vi}{}

\href{https://www.nytimes3xbfgragh.onion/section/todayspaper}{Today's
Paper}

\href{/section/world/asia}{Asia Pacific}\textbar{}Officials Push
U.S.-China Relations Toward Point of No Return

\url{https://nyti.ms/39zKEfa}

\begin{itemize}
\item
\item
\item
\item
\item
\end{itemize}

Advertisement

\protect\hyperlink{after-top}{Continue reading the main story}

Supported by

\protect\hyperlink{after-sponsor}{Continue reading the main story}

\hypertarget{officials-push-us-china-relations-toward-point-of-no-return}{%
\section{Officials Push U.S.-China Relations Toward Point of No
Return}\label{officials-push-us-china-relations-toward-point-of-no-return}}

Top aides to President Trump want to leave a lasting legacy of ruptured
ties between the two powers. China's aggression has been helping their
cause.

\includegraphics{https://static01.graylady3jvrrxbe.onion/images/2020/07/26/world/00china-us-clash1/merlin_157181268_478b9364-1e98-4d34-a4af-7e959f4ae9a8-articleLarge.jpg?quality=75\&auto=webp\&disable=upscale}

\href{https://www.nytimes3xbfgragh.onion/by/edward-wong}{\includegraphics{https://static01.graylady3jvrrxbe.onion/images/2018/09/24/multimedia/author-edward-wong/author-edward-wong-thumbLarge-v5.png}}\href{https://www.nytimes3xbfgragh.onion/by/steven-lee-myers}{\includegraphics{https://static01.graylady3jvrrxbe.onion/images/2018/10/15/multimedia/author-steven-lee-myers/author-steven-lee-myers-thumbLarge.png}}

By \href{https://www.nytimes3xbfgragh.onion/by/edward-wong}{Edward Wong}
and \href{https://www.nytimes3xbfgragh.onion/by/steven-lee-myers}{Steven
Lee Myers}

\begin{itemize}
\item
  July 25, 2020
\item
  \begin{itemize}
  \item
  \item
  \item
  \item
  \item
  \end{itemize}
\end{itemize}

WASHINGTON --- Step by step, blow by blow, the United States and China
are dismantling decades of political, economic and social engagement,
setting the stage for
\href{https://www.nytimes3xbfgragh.onion/2019/06/26/world/asia/united-states-china-conflict.html}{a
new era of confrontation} shaped by the views of
\href{https://www.nytimes3xbfgragh.onion/2020/07/14/world/asia/cold-war-china-us.html}{the
most hawkish voices} on both sides.

With President Trump
\href{https://www.nytimes3xbfgragh.onion/2020/07/24/upshot/biden-polls-demographics.html}{trailing
badly in the polls} as the election nears, his national security
officials have intensified
\href{https://www.nytimes3xbfgragh.onion/2020/03/22/us/politics/coronavirus-us-china.html}{their
attack on China} in recent weeks, targeting its officials, diplomats and
executives. While the strategy has reinforced a key campaign message,
some American officials, worried Mr. Trump will lose, are also trying to
engineer irreversible changes, according to people familiar with the
thinking.

China's leader, Xi Jinping, has inflamed the fight, brushing aside
international concern about the country's rising authoritarianism to
consolidate his own political power and to crack down on basic freedoms,
from
\href{https://www.nytimes3xbfgragh.onion/interactive/2019/11/16/world/asia/china-xinjiang-documents.html}{Xinjiang}
to
\href{https://www.nytimes3xbfgragh.onion/2020/06/29/world/asia/china-hong-kong-security-law-rules.html}{Hong
Kong}. By doing so, he has hardened attitudes in Washington, fueling a
clash that at least some in China believe could be dangerous to the
country's interests.

The combined effect could prove to be Mr. Trump's most consequential
foreign policy legacy,
\href{https://www.nytimes3xbfgragh.onion/2020/06/18/us/politics/trump-china-bolton.html}{even
if it's not one he has consistently pursued}: the entrenchment of a
fundamental strategic and ideological confrontation between the world's
two largest economies.

A state of broad and intense competition is the end goal of the
president's hawkish advisers. In their view, confrontation and coercion,
aggression and antagonism should be the status quo with the Chinese
Communist Party, no matter who is leading the United States next year.
They call it ``reciprocity.''

Secretary of State Mike Pompeo
\href{https://www.state.gov/communist-china-and-the-free-worlds-future/}{declared
in a speech} on Thursday that the relationship should be based on the
principle of ``distrust and verify,'' saying that the
\href{https://www.nytimes3xbfgragh.onion/1972/02/24/archives/nixon-talks-further-with-chou-and-drives-to-view-great-wall-wider.html}{diplomatic
opening} orchestrated by President Richard M. Nixon nearly half a
century ago had ultimately undermined American interests.

``We must admit a hard truth that should guide us in the years and
decades to come: that if we want to have a free 21st century, and not
the Chinese century of which Xi Jinping dreams, the old paradigm of
blind engagement with China simply won't get it done,'' Mr. Pompeo said.
``We must not continue it and we must not return to it.''

\includegraphics{https://static01.graylady3jvrrxbe.onion/images/2020/07/24/world/00china-us-clash2/merlin_174903876_b3f9d7b1-ef12-4393-8331-d63d7201eae6-articleLarge.jpg?quality=75\&auto=webp\&disable=upscale}

The events of the last week brought relations to yet another low,
\href{https://www.nytimes3xbfgragh.onion/2020/07/14/world/asia/cold-war-china-us.html}{accelerating
the downward spiral}.

On Tuesday, the State Department ordered China to
\href{https://www.nytimes3xbfgragh.onion/2020/07/22/world/asia/us-china-houston-consulate.html}{shut
down its Houston consulate}, prompting diplomats there to burn documents
in a courtyard. On Friday, in retaliation, China ordered the United
States to close its consulate in the southwestern city of Chengdu. The
Chinese Foreign Ministry the next day denounced what it called ``forced
entry'' into the Houston consulate by U.S. law enforcement officers on
Friday afternoon.

In between, the Department of Justice announced
\href{https://www.justice.gov/opa/pr/researchers-charged-visa-fraud-after-lying-about-their-work-china-s-people-s-liberation-army}{criminal
charges} against four members of the People's Liberation Army for lying
about their status in order to operate as undercover intelligence
operatives in the United States. All four have been arrested. One, Tang
Juan, who was studying at the University of California, Davis, ignited a
diplomatic standoff when she sought refuge in the Chinese consulate in
San Francisco, but was taken into custody on Thursday night.

This comes on top of a month in which the administration
\href{https://www.nytimes3xbfgragh.onion/2020/07/09/world/asia/trump-china-sanctions-uighurs.html}{announced
sanctions} on senior Chinese officials, including a member of the ruling
Politburo, over the mass internment of Muslims;
\href{https://www.nytimes3xbfgragh.onion/2020/07/15/world/asia/china-trump-hong-kong.html}{revoked
the special status of Hong Kong} in diplomatic and trade relations; and
declared that China's vast maritime claims in the South China Sea
\href{https://www.nytimes3xbfgragh.onion/2020/07/13/world/asia/south-china-sea-pompeo.html}{were
illegal}.

Image

An indoctrination center in the western Chinese region of Xinjiang.
American sanctions have been imposed on Chinese officials over the mass
internment of Muslims there.Credit...Gilles Sabrié for The New York
Times

The administration has also
\href{https://www.nytimes3xbfgragh.onion/2020/05/28/us/politics/china-hong-kong-trump-student-visas.html}{imposed
a travel ban} on Chinese students at graduate level or higher with ties
to military institutions in China. Officials are discussing whether to
\href{https://www.nytimes3xbfgragh.onion/2020/07/15/us/politics/china-travel-ban.html}{do
the same to members of the Communist Party} and their families, a
sweeping move that could put 270 million people on a blacklist.

``Below the president, Secretary Pompeo and other members of the
administration appear to have broader goals,'' said Ryan Hass, a China
director on President Barack Obama's National Security Council who is
now at the Brookings Institution.

``They want to reorient the U.S.-China relationship toward an
all-encompassing systemic rivalry that cannot be reversed by the outcome
of the upcoming U.S. election,'' he said. ``They believe this
reorientation is needed to put the United States on a competitive
footing against its 21st-century geostrategic rival.''

From the start, Mr. Trump has vowed to change the relationship with
China, but mainly when it comes to trade. Early this year, the
\href{https://www.nytimes3xbfgragh.onion/2020/01/15/business/economy/china-trade-deal.html}{negotiated
truce} in the countries' trade war was hailed by some aides as a
signature accomplishment. That deal is still in effect, though hanging
by a thread, overshadowed by the broader fight.

Beyond China, few of the administration's foreign policy goals have been
fully achieved. Mr. Trump's personal diplomacy with Kim Jong-un, the
North Korean leader, has
\href{https://www.nytimes3xbfgragh.onion/2020/06/12/world/asia/korea-nuclear-trump-kim.html?searchResultPosition=5}{done
nothing to end the country's nuclear weapons program}.

His
\href{https://www.nytimes3xbfgragh.onion/2018/05/08/us/politics/trump-speech-iran-deal.html?searchResultPosition=40}{withdrawal
from the Iran nuclear deal} has further alienated allies and made that
country's leaders even more belligerent. His effort to change the
government
\href{https://www.nytimes3xbfgragh.onion/2020/03/31/world/americas/coronavirus-venezuela-maduro-guaido.html?searchResultPosition=2}{in
Venezuela} failed. His promised withdrawal of all American troops
f\href{https://www.nytimes3xbfgragh.onion/2020/05/26/world/asia/afghanistan-troop-withdrawal-election-day.html?searchResultPosition=2}{rom
Afghanistan} has yet to occur.

In Beijing, some officials and analysts have publicly dismissed many of
the Trump administration's moves as campaign politics, accusing Mr.
Pompeo and others of promoting a Cold War mentality to score points for
an uphill re-election fight. There is a growing recognition, though,
that the conflict's roots run deeper.

The breadth of the administration's campaign has vindicated those in
China --- and possibly Mr. Xi himself --- who have long suspected that
the United States will never accept the country's growing economic and
military might, or its authoritarian political system.

``It's not just electoral considerations,'' said Cheng Xiaohe, an
associate professor at the School of International Studies at Renmin
University in Beijing. ``It is also a natural escalation and a result of
the inherent contradictions between China and the United States.''

Already reeling from the coronavirus pandemic, some Chinese officials
have sought to avoid open conflict with the United States. They have
urged the Trump administration to reconsider each of its actions and
called for cooperation, not confrontation, albeit without offering
significant concessions of their own.

Image

Outside the United States Consulate in Chengdu, China, on Friday. China
ordered it closed in retaliation for the shutdown of its consulate in
Houston.Credit...Noel Celis/Agence France-Presse --- Getty Images

``With global anti-China sentiment at its highest level in decades,
Chinese officials have indicated an interest in exploring potential
offramps to the current death spiral in U.S.-China relations,'' said
Jessica Chen Weiss, a political scientist at Cornell University who
studies Chinese foreign policy and public opinion.

``Beijing isn't spoiling for an all-out fight with the United States,''
she said, ``but at a minimum the Chinese government will retaliate to
show the world --- and a prospective Biden administration --- that China
won't be intimidated or pushed around.''

Image

Supporters of Mr. Trump at a rally in Tulsa, Okla., last
month.Credit...Doug Mills/The New York Times

Given the size of each nation's economy and their entwinement, there are
limits to the unwinding of relations, or what some Trump officials call
``decoupling.'' In the United States, tycoons and business executives,
who exercise enormous sway among politicians of both parties, will
continue to push for a more moderate approach, as members of Mr. Trump's
cabinet who represent Wall Street interests
\href{https://www.nytimes3xbfgragh.onion/2020/04/02/us/politics/coronavirus-trump-china.html}{have
done}. China is making leaps in science, technology and education that
Americans and citizens of other Western nations will want to share in.
In his Thursday speech, even Mr. Pompeo acknowledged, ``China is deeply
integrated into the global economy.''

Only two weeks ago, the foreign minister, Wang Yi, called on the United
States to step back from confrontation and work with China. In reality,
officials in Beijing appear resigned to the likelihood that nothing will
change for the better before next year.

``There is very little China can do to take the initiative,'' said Wu
Qiang, an independent analyst in Beijing. ``It has very few proactive
options.''

Image

Secretary of State Mike Pompeo said in a speech on Thursday that
``distrust and verify'' should be the basis of the U.S.-China
relationship.Credit...David Mcnew/Getty Images

Mr. Trump whipsaws in his language on China. He has called Mr. Xi ``a
very, very good ****** friend'' and even privately encouraged him to
keep building mass internment camps for Muslims and handle the Hong Kong
pro-democracy protesters his way, according to a new book by John R.
Bolton, the former national security adviser. When he last spoke with
Mr. Xi, he expressed
\href{https://twitter.com/realdonaldtrump/status/1243407157321560071?lang=en}{``much
respect!''} on Twitter.

With the election looming, Mr. Trump's tone has changed. He has returned
to bashing China, as he did in 2016, blaming Beijing for the pandemic
and even referring to the coronavirus with a racist phrase, ``Kung
Flu.'' His campaign aides have made aggressive rhetoric on China a
pillar of their strategy, believing it could help energize voters.

The heated language, combined with the administration's policy actions,
could actually be having a galvanizing effect on Chinese citizens, some
analysts and political figures in Beijing say.

``I strongly urge American people to re-elect Trump because his team has
many crazy members like Pompeo,'' Hu Xijin, the editor of the
nationalist newspaper Global Times,
\href{https://twitter.com/huxijin_gt/status/1286362851527950336}{wrote
on Twitter} on Friday. ``They help China strengthen solidarity and
cohesion in a special way.''

The relationship might not change course even if former Vice President
Joseph R. Biden Jr. defeats Mr. Trump in November. The idea of orienting
American policy toward competition with China has had robust bipartisan
support over the last three-and-a-half years.

The Chinese government's initial mishandling of the coronavirus outbreak
and its actions in Hong Kong, which is widely seen as a beacon of
liberal values within China, have been signal moments this year,
contributing to the tectonic shift in views across the political
spectrum.

The China hawks in the administration have seized on them to publicly
push their perspective: that the Chinese Communist Party seeks to expand
its ideology and authoritarian vision worldwide, and that citizens of
liberal nations must wake up to the dangers and gird themselves for a
conflict that could last for decades.

Since late June, the administration has rolled out four top officials to
make that case.

Attorney General William P. Barr accused American companies of
``\href{https://www.justice.gov/opa/speech/attorney-general-william-p-barr-delivers-remarks-china-policy-gerald-r-ford-presidential}{corporate
appeasement},'' while Christopher Wray, the F.B.I. director, said his
agency was opening
\href{https://www.fbi.gov/news/speeches/the-threat-posed-by-the-chinese-government-and-the-chinese-communist-party-to-the-economic-and-national-security-of-the-united-states}{a
new China-related counterintelligence investigation} every 10 hours.

Mr. Trump's national security adviser, Robert O'Brien, warned that the
Chinese Communist Party
\href{https://www.whitehouse.gov/briefings-statements/chinese-communist-partys-ideology-global-ambitions/}{aimed
to remake the world} in its image. ``The effort to control thought
beyond the borders of China is well underway,'' he said.

Mr. Pompeo's speech on Thursday was meant as the punctuation mark. He
chose the presidential library of the man credited with opening up
U.S.-China relations
\href{https://www.state.gov/communist-china-and-the-free-worlds-future/}{to
declare the policy a failure}.

``President Nixon once said he feared he had created a `Frankenstein' by
opening the world to the C.C.P.,'' Mr. Pompeo said, referring to the
Chinese Communist Party, ``and here we are.''

Edward Wong reported from Washington, and Steven Lee Myers from Seoul,
South Korea. Claire Fu contributed research from Beijing.

Advertisement

\protect\hyperlink{after-bottom}{Continue reading the main story}

\hypertarget{site-index}{%
\subsection{Site Index}\label{site-index}}

\hypertarget{site-information-navigation}{%
\subsection{Site Information
Navigation}\label{site-information-navigation}}

\begin{itemize}
\tightlist
\item
  \href{https://help.nytimes3xbfgragh.onion/hc/en-us/articles/115014792127-Copyright-notice}{©~2020~The
  New York Times Company}
\end{itemize}

\begin{itemize}
\tightlist
\item
  \href{https://www.nytco.com/}{NYTCo}
\item
  \href{https://help.nytimes3xbfgragh.onion/hc/en-us/articles/115015385887-Contact-Us}{Contact
  Us}
\item
  \href{https://www.nytco.com/careers/}{Work with us}
\item
  \href{https://nytmediakit.com/}{Advertise}
\item
  \href{http://www.tbrandstudio.com/}{T Brand Studio}
\item
  \href{https://www.nytimes3xbfgragh.onion/privacy/cookie-policy\#how-do-i-manage-trackers}{Your
  Ad Choices}
\item
  \href{https://www.nytimes3xbfgragh.onion/privacy}{Privacy}
\item
  \href{https://help.nytimes3xbfgragh.onion/hc/en-us/articles/115014893428-Terms-of-service}{Terms
  of Service}
\item
  \href{https://help.nytimes3xbfgragh.onion/hc/en-us/articles/115014893968-Terms-of-sale}{Terms
  of Sale}
\item
  \href{https://spiderbites.nytimes3xbfgragh.onion}{Site Map}
\item
  \href{https://help.nytimes3xbfgragh.onion/hc/en-us}{Help}
\item
  \href{https://www.nytimes3xbfgragh.onion/subscription?campaignId=37WXW}{Subscriptions}
\end{itemize}
