Sections

SEARCH

\protect\hyperlink{site-content}{Skip to
content}\protect\hyperlink{site-index}{Skip to site index}

\href{https://www.nytimes3xbfgragh.onion/section/politics}{Politics}

\href{https://myaccount.nytimes3xbfgragh.onion/auth/login?response_type=cookie\&client_id=vi}{}

\href{https://www.nytimes3xbfgragh.onion/section/todayspaper}{Today's
Paper}

\href{/section/politics}{Politics}\textbar{}As Republicans Embrace Cut
in Jobless Aid, Divisions Weaken Their Leverage

\url{https://nyti.ms/3jGZB3L}

\begin{itemize}
\item
\item
\item
\item
\item
\end{itemize}

Advertisement

\protect\hyperlink{after-top}{Continue reading the main story}

Supported by

\protect\hyperlink{after-sponsor}{Continue reading the main story}

\hypertarget{as-republicans-embrace-cut-in-jobless-aid-divisions-weaken-their-leverage}{%
\section{As Republicans Embrace Cut in Jobless Aid, Divisions Weaken
Their
Leverage}\label{as-republicans-embrace-cut-in-jobless-aid-divisions-weaken-their-leverage}}

The proposal comes after Republicans struggled to iron out their policy
differences with the administration and each other. Democrats are all
but guaranteed to reject the offer.

\includegraphics{https://static01.graylady3jvrrxbe.onion/images/2020/07/27/us/politics/27dc-virus-cong-sub/27dc-virus-cong-articleLarge.jpg?quality=75\&auto=webp\&disable=upscale}

\href{https://www.nytimes3xbfgragh.onion/by/emily-cochrane}{\includegraphics{https://static01.graylady3jvrrxbe.onion/images/2018/11/28/multimedia/author-emily-cochrane/author-emily-cochrane-thumbLarge-v3.png}}\href{https://www.nytimes3xbfgragh.onion/by/jim-tankersley}{\includegraphics{https://static01.graylady3jvrrxbe.onion/images/2018/10/19/multimedia/author-jim-tankersley/author-jim-tankersley-thumbLarge.png}}

By \href{https://www.nytimes3xbfgragh.onion/by/emily-cochrane}{Emily
Cochrane} and
\href{https://www.nytimes3xbfgragh.onion/by/jim-tankersley}{Jim
Tankersley}

\begin{itemize}
\item
  July 27, 2020
\item
  \begin{itemize}
  \item
  \item
  \item
  \item
  \item
  \end{itemize}
\end{itemize}

WASHINGTON --- Senate Republicans and the White House on Monday threw
their support behind a substantial cut in jobless aid for tens of
millions of Americans laid off amid the pandemic, proposing a weekly
reduction of \$400 to a benefit that has cushioned the nation's economy
even as
\href{https://www.nytimes3xbfgragh.onion/2020/07/28/world/coronavirus-covid-19.html}{coronavirus}
cases continue to
\href{https://www.nytimes3xbfgragh.onion/interactive/2020/us/coronavirus-us-cases.html}{rise
across the country}.

The proposal was part of a \$1 trillion opening bid that would have to
be reconciled with Democrats, who were pushing a recovery package that
would spend three times as much and extend the \$600 per week in extra
unemployment aid through the end of the year.

Economists say that the money, slated to expire this week, has provided
a crucial economic buffer for the unemployed, and that lowering the
payments could have a cascade of damaging effects across the economy.
But Republicans contend that it is too generous, discouraging Americans
from returning to work and hampering a recovery.

The Senate Republicans' decision to embrace the decrease reflects the
predicament in which they find themselves during a worsening pandemic
and continued economic recession, little more than three months before
Election Day. With a small but vital bloc of conservative senators
opposed to providing any more federal coronavirus aid, the Republican
Party has struggled to agree on how to stabilize the battered economy,
leaving Democrats with crucial leverage for an intense set of
negotiations over the relief package.

Even as Republicans rolled out their proposal on Monday evening, Mark
Meadows, the White House chief of staff, and Steven Mnuchin, the
Treasury secretary, were huddled in Speaker Nancy Pelosi's Capitol
office suite, meeting with top Democratic leaders in a reflection of
their influence in the talks.

With the two sides far apart, it appeared unlikely that they could
bridge their differences in time to avert a lapse on Friday of the
supplemental jobless aid, nor was it guaranteed that they would be able
to do so at all. That left uncertain the fate of President Trump's best
hope of injecting one last shot of stimulus into the economy before the
general election in November.

Complicating the picture, Republicans and the White House continued to
bicker over the contents of the package even after it was announced,
with Senator Mitch McConnell, Republican of Kentucky and the majority
leader, appearing surprised that it included funding for a new F.B.I.
building that has long been an obsession of Mr. Trump's.

``I don't think there is funding, is there?'' Mr. McConnell said to
reporters who asked about the money, which is designated as a
coronavirus-related emergency in the draft bill. Assured that it was, he
said the administration would ``have to answer the question on why they
insisted on that provision.''

Republicans had hoped to avoid this situation altogether, knowing that
many in their ranks had grown exhausted with the torrent of federal
spending --- nearly \$3 trillion --- that Congress approved in rapid
succession in early spring. They resisted passing another package,
gambling that if they waited, the virus would dissipate and the economy
would rebound, and that they could push through a bare-bones package.

Instead, they are now staring down the beginning of the school year with
skyrocketing cases and record unemployment levels, with many in their
ranks unwilling to pour any more money into the economy. Their proposal
spends more than many Republicans are likely to support, and it will
most likely grow as Democrats place their stamp on it.

``There is significant resistance to yet another trillion dollars,''
Senator Ted Cruz, Republican of Texas, told reporters on Monday. ``As it
stands now, I think it's likely that you'll see a number of Republicans
in opposition to this bill and expressing serious concerns.''

The policy gulf between the two parties has widened in recent days to
the point where top White House officials have begun to float the
prospect of a narrow bill to address the unemployment benefits,
liability protections and school funding, eschewing Democratic
priorities and other objectives in an effort to address what they deem
to be more immediate needs. Democrats have rejected such a plan.

``We have produced a tailored and targeted draft that will cut right to
the heart of three distinct crises facing our country: getting kids back
in school, getting workers back to work and winning the health care
fight against the virus,'' Mr. McConnell said as he led about a dozen
Republican senators in unveiling the legislation on the Senate floor.
House Democrats' proposal, he said, amounted to a ``multitrillion dollar
socialist manifesto.''

The package of bills rolled out on Monday included a new round of
\$1,200 direct payments to Americans earning \$75,000 or less per year.
In line with Mr. Trump's demands, it would reserve tens of billions of
dollars in federal funding for schools that reopen for in-person
instruction.

It would limit legal liability for businesses that open amid the
pandemic, a top priority of business groups in Washington, for
coronavirus-related episodes that take place through October 2024. It
establishes a tax credit for companies to reconfigure their workplaces
to promote safety from the virus, and it would expand tax credits for
employers that hire and retain workers amid the outbreak. The proposal
would also offer tax certainty to Americans who work in one state and
live in another, and are facing the prospect of paying income taxes in
multiple states if they were forced to work from home.

The package would both extend government aid for small businesses
through the Paycheck Protection Program, which was established in March,
and narrow the set of companies eligible to receive it. It would also
create an alternate source of aid for businesses in low-income,
high-poverty areas: a 20-year loan, with an interest rate of 1 percent,
that would give those businesses enough cash to replace up to two years
of lost revenues.

It also includes a bipartisan proposal to force Congress to consider
future deficit and debt-reduction measures.

The introduction of the package had been delayed for days as Republicans
worked to resolve internal divisions with each other and Mr. Trump and
to put forward a united front before negotiations with Democrats. But on
Monday, the measure already faced resistance by some rank-and-file
Republicans and was scorned by Democrats in both chambers, who said it
failed to meet even a fraction of the country's economic and health
needs.

``If they're not even getting to the fundamentals of food and rent and
economic survival, they're not really ready to have a serious
negotiation,'' Ms. Pelosi said after a nearly two-hour meeting with Mr.
Mnuchin, Mr. Meadows and Senator Chuck Schumer of New York, the
Democratic leader.

Mr. Schumer added, ``We hope they can get their act together --- we very
much want to get something done for the needs of the people.''

Mr. Meadows, as he left Ms. Pelosi's office with Mr. Mnuchin, declared
it a good meeting and said the pair would return on Tuesday. Both
Democratic leaders said they planned to carefully review the Republican
offer overnight.

The disconnect between the parties has already allowed the additional
\$600-per-week unemployment benefit to expire for many workers, a move
that could prove more harmful if households opt to make precautionary
spending cuts without the guarantee of more relief.

Mr. Meadows and Mr. Mnuchin sought to leave an indelible mark on the
package on behalf of Mr. Trump, spending a weekend on Capitol Hill
meeting with Senate staff --- an unusual step for senior cabinet
officials --- to hammer out the technical details of the unemployment
proposal.

While the two men ultimately agreed to drop demands for a payroll tax
cut --- a presidential priority dismissed by members of both parties ---
they succeeded in securing \$1.75 billion for the design and
construction of a new building for the F.B.I. Headquarters across from
Mr. Trump's luxury hotel in downtown Washington, in which he
\href{https://www.nytimes3xbfgragh.onion/2018/10/18/us/politics/fbi-headquarters-building-trump.html}{has
repeatedly shown a personal interest}.

The proposal to cut the jobless aid by two-thirds is likely to be among
the most bitterly contested issues in the negotiations to come.

Many Republicans oppose the supplemental jobless aid, arguing that it is
a disincentive to returning to work because it exceeds what some workers
can earn in regular wages. The Republican plan envisions eventually
shifting to a new system of calculating federal aid that would cap
benefits at about 70 percent of a worker's prior income,
\href{https://www.nytimes3xbfgragh.onion/2020/07/23/business/economy/unemployment-benefits.html}{which
would also amount to about \$200 per week}. Most Democrats say state
unemployment systems were already struggling to handle distribution of
the \$600 lump sum, and would be challenged to adapt to a new system.

The Republican proposal would also establish a liability shield for
businesses, schools and hospitals open during the pandemic from facing
claims over episodes related to the coronavirus. Mr. McConnell has
repeatedly deemed such a provision to be a prerequisite for any further
aid bill, while Democrats have instead pushed for federal protections
for workers against the coronavirus.

Republicans also set aside \$105 billion in aid for schools, with \$70
billion going to elementary and secondary schools, two-thirds of it
reserved for institutions that have begun reopening and holding some
in-person classes. Another \$30 billion would go to colleges and
universities.

Overcoming opposition from the White House, Republicans also set aside
\$16 billion in new funding for states to conduct coronavirus testing
and contact tracing, as well as additional aid for top health agencies
and efforts to support the response to the pandemic and potentially
distribute a global vaccine.

Advertisement

\protect\hyperlink{after-bottom}{Continue reading the main story}

\hypertarget{site-index}{%
\subsection{Site Index}\label{site-index}}

\hypertarget{site-information-navigation}{%
\subsection{Site Information
Navigation}\label{site-information-navigation}}

\begin{itemize}
\tightlist
\item
  \href{https://help.nytimes3xbfgragh.onion/hc/en-us/articles/115014792127-Copyright-notice}{©~2020~The
  New York Times Company}
\end{itemize}

\begin{itemize}
\tightlist
\item
  \href{https://www.nytco.com/}{NYTCo}
\item
  \href{https://help.nytimes3xbfgragh.onion/hc/en-us/articles/115015385887-Contact-Us}{Contact
  Us}
\item
  \href{https://www.nytco.com/careers/}{Work with us}
\item
  \href{https://nytmediakit.com/}{Advertise}
\item
  \href{http://www.tbrandstudio.com/}{T Brand Studio}
\item
  \href{https://www.nytimes3xbfgragh.onion/privacy/cookie-policy\#how-do-i-manage-trackers}{Your
  Ad Choices}
\item
  \href{https://www.nytimes3xbfgragh.onion/privacy}{Privacy}
\item
  \href{https://help.nytimes3xbfgragh.onion/hc/en-us/articles/115014893428-Terms-of-service}{Terms
  of Service}
\item
  \href{https://help.nytimes3xbfgragh.onion/hc/en-us/articles/115014893968-Terms-of-sale}{Terms
  of Sale}
\item
  \href{https://spiderbites.nytimes3xbfgragh.onion}{Site Map}
\item
  \href{https://help.nytimes3xbfgragh.onion/hc/en-us}{Help}
\item
  \href{https://www.nytimes3xbfgragh.onion/subscription?campaignId=37WXW}{Subscriptions}
\end{itemize}
