Sections

SEARCH

\protect\hyperlink{site-content}{Skip to
content}\protect\hyperlink{site-index}{Skip to site index}

\href{https://www.nytimes3xbfgragh.onion/section/arts/design}{Art \&
Design}

\href{https://myaccount.nytimes3xbfgragh.onion/auth/login?response_type=cookie\&client_id=vi}{}

\href{https://www.nytimes3xbfgragh.onion/section/todayspaper}{Today's
Paper}

\href{/section/arts/design}{Art \& Design}\textbar{}Met Museum Acquires
Two Sculptures by Wangechi Mutu

\url{https://nyti.ms/332NQ1x}

\begin{itemize}
\item
\item
\item
\item
\item
\end{itemize}

\href{https://www.nytimes3xbfgragh.onion/spotlight/at-home?action=click\&pgtype=Article\&state=default\&region=TOP_BANNER\&context=at_home_menu}{At
Home}

\begin{itemize}
\tightlist
\item
  \href{https://www.nytimes3xbfgragh.onion/2020/09/07/travel/route-66.html?action=click\&pgtype=Article\&state=default\&region=TOP_BANNER\&context=at_home_menu}{Cruise
  Along: Route 66}
\item
  \href{https://www.nytimes3xbfgragh.onion/2020/09/04/dining/sheet-pan-chicken.html?action=click\&pgtype=Article\&state=default\&region=TOP_BANNER\&context=at_home_menu}{Roast:
  Chicken With Plums}
\item
  \href{https://www.nytimes3xbfgragh.onion/2020/09/04/arts/television/dark-shadows-stream.html?action=click\&pgtype=Article\&state=default\&region=TOP_BANNER\&context=at_home_menu}{Watch:
  Dark Shadows}
\item
  \href{https://www.nytimes3xbfgragh.onion/interactive/2020/at-home/even-more-reporters-editors-diaries-lists-recommendations.html?action=click\&pgtype=Article\&state=default\&region=TOP_BANNER\&context=at_home_menu}{Explore:
  Reporters' Google Docs}
\end{itemize}

Advertisement

\protect\hyperlink{after-top}{Continue reading the main story}

Supported by

\protect\hyperlink{after-sponsor}{Continue reading the main story}

\hypertarget{met-museum-acquires-two-sculptures-by-wangechi-mutu}{%
\section{Met Museum Acquires Two Sculptures by Wangechi
Mutu}\label{met-museum-acquires-two-sculptures-by-wangechi-mutu}}

The new additions are from the series that is on display on the museum's
Fifth Avenue facade.

\includegraphics{https://static01.graylady3jvrrxbe.onion/images/2020/08/01/arts/28mutu-item1/28mutu-item1-articleLarge.jpg?quality=75\&auto=webp\&disable=upscale}

By \href{https://www.nytimes3xbfgragh.onion/by/peter-libbey}{Peter
Libbey}

\begin{itemize}
\item
  July 28, 2020
\item
  \begin{itemize}
  \item
  \item
  \item
  \item
  \item
  \end{itemize}
\end{itemize}

The series of bronze statues by
\href{https://www.nytimes3xbfgragh.onion/2018/06/20/t-magazine/wangechi-mutu.html}{Wangechi
Mutu} that currently adorns the Metropolitan Museum of Art's facade is
scheduled to be on public display until early November. But two of the
four pieces, ``The Seated I'' and ``The Seated III,'' will remain at the
museum long after the exhibition closes as a part of its collections,
the Met announced on Tuesday.

``Sometimes when you do a site-specific commission it only works for the
specific site or in that particular context,'' Max Hollein, the museum's
director, said in an interview. ``In regard to Wangechi's works, it's
clear that on the facade they work as these four sculptures framing the
facade, transforming the facade, but they also work as singular
objects.''

When they were unveiled last September, Ms. Mutu's caryatid sculptures
--- traditionally female figures carved into architectural support
structures like columns --- were the first artworks to be presented from
the face of the Met's building on Fifth Avenue. In Ms. Mutu's
renderings, the figures are released from their supporting role. Instead
of helping to hold up roofs or balconies, they sit freely on pedestals.

Their style also separates them from the typical caryatids that visitors
might see elsewhere. ``With sloping eyes and long fingers expressive of
exceptional reach, they speak as messengers from an
Afrofuturist-inflected otherworld,'' Nancy Princenthal
\href{https://www.nytimes3xbfgragh.onion/2019/09/05/arts/design/wangechi-mutu-metropolitan-museum.html}{wrote
in the Times.}

The series, collectively titled ``The NewOnes, will free Us,'' was
commissioned by the Met, along with paintings by
\href{https://www.youtube.com/watch?v=qvofmOMDLMM}{Kent Monkman}. These
efforts, and the acquisition of Ms. Mutu's sculptures, are part of
\href{https://www.nytimes3xbfgragh.onion/2019/03/21/arts/design/the-met-will-use-its-facade-and-great-hall-to-showcase-contemporary-art.html}{a
larger push} by the museum to increase its engagement with contemporary
art.

Last November, the acclaimed exhibition of Ms. Mutu's sculptures
\href{https://www.nytimes3xbfgragh.onion/2019/11/06/arts/design/Wangechi-Mutu-metropolitan-museum-extension.html}{was
extended} from January 2020 to June. It was extended again when the
museum was forced to shut down because of the coronavirus pandemic. (In
June, the Met
\href{https://www.nytimes3xbfgragh.onion/2020/06/23/arts/design/met-museum-reopen-virus.html}{announced
plans} to reopen to the public on Aug. 29 pending
\href{https://www.nytimes3xbfgragh.onion/2020/07/17/arts/design/nyc-museums-phase-4.html}{state
and city approval}.)

Plans for how Ms. Mutu's sculptures will be shown when they move indoors
are still being formulated, Mr. Hollein said: ``They will be displayed,
certainly, as part of our contemporary collection but obviously we have
a strong context with our collection of African art as well as our
collection of European, classical sculpture.''

Carol Bove, a sculptor known for large-scale works that combine
modernist and minimalist elements, will be
\href{https://www.nytimes3xbfgragh.onion/2020/02/20/arts/design/met-museum-rooftop-commission.html}{the
next artist} to tackle the Met's frontispiece. Mr. Hollein anticipates
that her installation, originally scheduled to debut in September, will
be exhibited in 2021.

Advertisement

\protect\hyperlink{after-bottom}{Continue reading the main story}

\hypertarget{site-index}{%
\subsection{Site Index}\label{site-index}}

\hypertarget{site-information-navigation}{%
\subsection{Site Information
Navigation}\label{site-information-navigation}}

\begin{itemize}
\tightlist
\item
  \href{https://help.nytimes3xbfgragh.onion/hc/en-us/articles/115014792127-Copyright-notice}{©~2020~The
  New York Times Company}
\end{itemize}

\begin{itemize}
\tightlist
\item
  \href{https://www.nytco.com/}{NYTCo}
\item
  \href{https://help.nytimes3xbfgragh.onion/hc/en-us/articles/115015385887-Contact-Us}{Contact
  Us}
\item
  \href{https://www.nytco.com/careers/}{Work with us}
\item
  \href{https://nytmediakit.com/}{Advertise}
\item
  \href{http://www.tbrandstudio.com/}{T Brand Studio}
\item
  \href{https://www.nytimes3xbfgragh.onion/privacy/cookie-policy\#how-do-i-manage-trackers}{Your
  Ad Choices}
\item
  \href{https://www.nytimes3xbfgragh.onion/privacy}{Privacy}
\item
  \href{https://help.nytimes3xbfgragh.onion/hc/en-us/articles/115014893428-Terms-of-service}{Terms
  of Service}
\item
  \href{https://help.nytimes3xbfgragh.onion/hc/en-us/articles/115014893968-Terms-of-sale}{Terms
  of Sale}
\item
  \href{https://spiderbites.nytimes3xbfgragh.onion}{Site Map}
\item
  \href{https://help.nytimes3xbfgragh.onion/hc/en-us}{Help}
\item
  \href{https://www.nytimes3xbfgragh.onion/subscription?campaignId=37WXW}{Subscriptions}
\end{itemize}
