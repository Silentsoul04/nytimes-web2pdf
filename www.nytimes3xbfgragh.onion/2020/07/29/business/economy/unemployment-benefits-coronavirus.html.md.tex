Sections

SEARCH

\protect\hyperlink{site-content}{Skip to
content}\protect\hyperlink{site-index}{Skip to site index}

\href{https://www.nytimes3xbfgragh.onion/section/business/economy}{Economy}

\href{https://myaccount.nytimes3xbfgragh.onion/auth/login?response_type=cookie\&client_id=vi}{}

\href{https://www.nytimes3xbfgragh.onion/section/todayspaper}{Today's
Paper}

\href{/section/business/economy}{Economy}\textbar{}An Extra \$600 a Week
Kept Many Jobless Workers Afloat. Now What Will They Do?

\url{https://nyti.ms/3hOtJIq}

\begin{itemize}
\item
\item
\item
\item
\item
\item
\end{itemize}

\hypertarget{the-coronavirus-outbreak}{%
\subsubsection{\texorpdfstring{\href{https://www.nytimes3xbfgragh.onion/news-event/coronavirus?name=styln-coronavirus-markets\&region=TOP_BANNER\&block=storyline_menu_recirc\&action=click\&pgtype=Article\&impression_id=75c6fcd0-f52d-11ea-8817-2db458fa0a4d\&variant=undefined}{The
Coronavirus
Outbreak}}{The Coronavirus Outbreak}}\label{the-coronavirus-outbreak}}

\begin{itemize}
\tightlist
\item
  live\href{https://www.nytimes3xbfgragh.onion/2020/09/12/world/covid-19-coronavirus.html?name=styln-coronavirus-markets\&region=TOP_BANNER\&block=storyline_menu_recirc\&action=click\&pgtype=Article\&impression_id=75c6fcd1-f52d-11ea-8817-2db458fa0a4d\&variant=undefined}{Latest
  Updates}
\item
  \href{https://www.nytimes3xbfgragh.onion/interactive/2020/us/coronavirus-us-cases.html?name=styln-coronavirus-markets\&region=TOP_BANNER\&block=storyline_menu_recirc\&action=click\&pgtype=Article\&impression_id=75c6fcd2-f52d-11ea-8817-2db458fa0a4d\&variant=undefined}{Maps
  and Cases}
\item
  \href{https://www.nytimes3xbfgragh.onion/interactive/2020/science/coronavirus-vaccine-tracker.html?name=styln-coronavirus-markets\&region=TOP_BANNER\&block=storyline_menu_recirc\&action=click\&pgtype=Article\&impression_id=75c723e0-f52d-11ea-8817-2db458fa0a4d\&variant=undefined}{Vaccine
  Tracker}
\item
  \href{https://www.nytimes3xbfgragh.onion/2020/09/10/us/politics/fda-coronavirus-vaccine.html?name=styln-coronavirus-markets\&region=TOP_BANNER\&block=storyline_menu_recirc\&action=click\&pgtype=Article\&impression_id=75c723e1-f52d-11ea-8817-2db458fa0a4d\&variant=undefined}{F.D.A.
  Regulators' Self-Defense}
\item
  \href{https://www.nytimes3xbfgragh.onion/2020/09/09/upshot/coronavirus-surprise-test-fees.html?name=styln-coronavirus-markets\&region=TOP_BANNER\&block=storyline_menu_recirc\&action=click\&pgtype=Article\&impression_id=75c723e2-f52d-11ea-8817-2db458fa0a4d\&variant=undefined}{Surprise
  Test Fees}
\end{itemize}

Advertisement

\protect\hyperlink{after-top}{Continue reading the main story}

Supported by

\protect\hyperlink{after-sponsor}{Continue reading the main story}

\hypertarget{an-extra-600-a-week-kept-many-jobless-workers-afloat-now-what-will-they-do}{%
\section{An Extra \$600 a Week Kept Many Jobless Workers Afloat. Now
What Will They
Do?}\label{an-extra-600-a-week-kept-many-jobless-workers-afloat-now-what-will-they-do}}

A supplement to unemployment benefits is at an end, and Congress is
deadlocked over new aid. For some, that means hunger, evictions or
bankruptcies.

\includegraphics{https://static01.graylady3jvrrxbe.onion/images/2020/07/29/business/29virus-cliff1/29virus-cliff1-articleLarge.jpg?quality=75\&auto=webp\&disable=upscale}

\href{https://www.nytimes3xbfgragh.onion/by/patricia-cohen}{\includegraphics{https://static01.graylady3jvrrxbe.onion/images/2018/02/16/multimedia/author-patricia-cohen/author-patricia-cohen-thumbLarge.jpg}}\href{https://www.nytimes3xbfgragh.onion/by/ben-casselman}{\includegraphics{https://static01.graylady3jvrrxbe.onion/images/2018/11/09/multimedia/author-ben-casselman/author-ben-casselman-thumbLarge.png}}\href{http://nytimes3xbfgragh.onion/by/gillian-friedman}{\includegraphics{https://static01.graylady3jvrrxbe.onion/images/2020/08/28/reader-center/author-gillian-friedman/author-gillian-friedman-thumbLarge.png}}

By \href{https://www.nytimes3xbfgragh.onion/by/patricia-cohen}{Patricia
Cohen}, \href{https://www.nytimes3xbfgragh.onion/by/ben-casselman}{Ben
Casselman} and
\href{http://nytimes3xbfgragh.onion/by/gillian-friedman}{Gillian
Friedman}

\begin{itemize}
\item
  Published July 29, 2020Updated Sept. 10, 2020
\item
  \begin{itemize}
  \item
  \item
  \item
  \item
  \item
  \item
  \end{itemize}
\end{itemize}

For Sara Gard, the
\href{https://www.nytimes3xbfgragh.onion/article/coronavirus-stimulus-package-questions-answers.html}{government's
safety net} moved smoothly into place when the coronavirus pandemic
upended her family's lives. Jobless benefit checks began arriving a few
days after she was furloughed in April from an entertainment company in
Atlanta. A \$600 weekly supplement, part of an emergency federal
program, would cover the mortgage until her company resumed operations
--- probably in June.

June came and went, and the reopening was pushed to August. Now August
is near, the business is still shuttered and the
\href{https://www.nytimes3xbfgragh.onion/2020/07/21/business/economy/coronavirus-unemployment-benefits.html}{weekly
benefit booster has run out}.

``When the \$600 is gone, we're going to totally have to rethink our
lives because we don't have a way to pay the mortgage,'' Ms. Gard said.
Without it, her weekly benefits from the state total \$300. Her mortgage
is \$1,700 a month.

Ms. Gard is one of roughly 30 million Americans who are getting
\href{https://www.nytimes3xbfgragh.onion/2020/09/03/business/economy/unemployment-claims.html}{unemployment}
payments --- a staggering figure that reflects one of the country's most
calamitous economic events.

But the stark urgency that faces families perilously close to losing
their homes, skipping medical treatments or missing meals because they
can't afford food has not extended to Washington. More than two months
after House Democrats approved another round of emergency relief, Senate
Republicans and the White House put forward a proposal this week with
far different priorities. Rather than restoring the \$600 supplement,
they would replace it with a \$200 payment, saying the larger sum
discourages looking for work.

The Gards recognize that they and their two children are luckier than
many families. Already nearly 11 percent of Americans say they live in
households where there is not enough to eat, according to a
\href{https://www.census.gov/programs-surveys/household-pulse-survey/data.html?utm_campaign=20200727mspuls1ccdtanl\&utm_medium=email\&utm_source=govdelivery}{recent
survey by the Census Bureau}. More than a quarter have missed a rent or
mortgage payment and doubt they will make the next one. Forty percent of
adults have delayed getting medical care.

Ms. Gard's husband, Matt, has kept his hospital maintenance job, and her
employer of 15 years continues to pay its portion of the cost of her
medical insurance.

But she has to come up with her part --- \$350 a month --- while dealing
with several other bills. ``I am our family's major breadwinner,'' said
Ms. Gard, 39, who had just gotten a raise that lifted her annual salary
to \$80,000.

They also have some savings --- a comfort when more than 40 percent of
American households lack cash to cover an unexpected \$400 expense. That
cushion was crucial last week when the Gards' air-conditioning system
suddenly died. The repair gobbled up what would have been a few months'
worth of mortgage payments.

\hypertarget{latest-updates-the-coronavirus-outbreak-and-the-economy}{%
\section{\texorpdfstring{\href{https://www.nytimes3xbfgragh.onion/live/2020/09/11/business/stock-market-today-coronavirus?action=click\&pgtype=Article\&state=default\&region=MAIN_CONTENT_1\&context=storylines_live_updates}{Latest
Updates: The Coronavirus Outbreak and the
Economy}}{Latest Updates: The Coronavirus Outbreak and the Economy}}\label{latest-updates-the-coronavirus-outbreak-and-the-economy}}

\href{https://www.nytimes3xbfgragh.onion/live/2020/09/11/business/stock-market-today-coronavirus?action=click\&pgtype=Article\&state=default\&region=MAIN_CONTENT_1\&context=storylines_live_updates\#the-nyse-may-move-its-data-center-out-of-new-jersey-in-response-to-a-proposed-tax}{23h
ago}

\href{https://www.nytimes3xbfgragh.onion/live/2020/09/11/business/stock-market-today-coronavirus?action=click\&pgtype=Article\&state=default\&region=MAIN_CONTENT_1\&context=storylines_live_updates\#the-nyse-may-move-its-data-center-out-of-new-jersey-in-response-to-a-proposed-tax}{The
N.Y.S.E. may move its data center out of New Jersey in response to a
proposed tax.}

\href{https://www.nytimes3xbfgragh.onion/live/2020/09/11/business/stock-market-today-coronavirus?action=click\&pgtype=Article\&state=default\&region=MAIN_CONTENT_1\&context=storylines_live_updates\#the-federal-budget-deficit-hit-3-trillion-as-of-august}{25h
ago}

\href{https://www.nytimes3xbfgragh.onion/live/2020/09/11/business/stock-market-today-coronavirus?action=click\&pgtype=Article\&state=default\&region=MAIN_CONTENT_1\&context=storylines_live_updates\#the-federal-budget-deficit-hit-3-trillion-as-of-august}{The
federal budget deficit hit \$3 trillion as of August.}

\href{https://www.nytimes3xbfgragh.onion/live/2020/09/11/business/stock-market-today-coronavirus?action=click\&pgtype=Article\&state=default\&region=MAIN_CONTENT_1\&context=storylines_live_updates\#warner-bros-pushes-the-release-of-wonder-woman-1984-to-christmas}{26h
ago}

\href{https://www.nytimes3xbfgragh.onion/live/2020/09/11/business/stock-market-today-coronavirus?action=click\&pgtype=Article\&state=default\&region=MAIN_CONTENT_1\&context=storylines_live_updates\#warner-bros-pushes-the-release-of-wonder-woman-1984-to-christmas}{Warner
Bros. pushes the release of `Wonder Woman 1984' to Christmas.}

\href{https://www.nytimes3xbfgragh.onion/live/2020/09/11/business/stock-market-today-coronavirus?action=click\&pgtype=Article\&state=default\&region=MAIN_CONTENT_1\&context=storylines_live_updates}{See
more updates}

More live coverage:
\href{https://www.nytimes3xbfgragh.onion/2020/09/11/world/covid-19-coronavirus.html?action=click\&pgtype=Article\&state=default\&region=MAIN_CONTENT_1\&context=storylines_live_updates}{Global}

Delaying wasn't an option, Ms. Gard explained: ``Georgia in August.''

Without further information on when she might be rehired, Ms. Gard has
started updating her résumé, and reaching out to recruiters and contacts
on LinkedIn.

Then her school district announced that all teaching would be online in
the fall. Her mother, 71, used to pitch in to care for her children, 2
and 5, but Ms. Gard worries about the health risk, so child care is
another issue.

``I have the month of August to figure out where September's mortgage
payment and everything else will come from,'' she said.

As
the\href{https://www.nytimes3xbfgragh.onion/2020/07/23/business/economy/unemployment-economy-coronavirus.html}{economy
falters}, pain is everywhere. Assistance, though, is more uneven.

\href{https://www.cbpp.org/research/economy/policy-basics-unemployment-insurance}{Normally},
\href{https://www.cbpp.org/research/economy/policy-basics-how-many-weeks-of-unemployment-compensation-are-available}{individual
states run their own unemployment programs}, setting different benefit
levels and eligibility rules. On average, benefits replace about 45
percent of a worker's weekly paycheck. Freelance, self-employed and
part-time workers, who didn't qualify for state benefits but received
funds through the federal Pandemic Unemployment Assistance program,
tended to get a much smaller fraction of their previous earnings.

That is where the extra \$600 a week came in. It was meant to make up
for lost income and ensure recipients had enough money to buy food, pay
rent, keep the lights on, afford medical prescriptions or make car
payments. Lawmakers settled on a lump sum as the quickest and easiest
way to deliver assistance --- given the limited capabilities of already
overwhelmed state unemployment networks.

The money was crucial in supplying the economy with fuel to keep the
engine going, economists say. Like any one-size-fits-all measure,
however, the \$600 supplement fell outside the target zone in many
instances. Roughly two-thirds of workers ended up with more income than
they would have earned had they not lost their jobs. The windfalls
angered critics who warned of ballooning government expenditures and
disincentives to work --- despite a severe shortage of available jobs.

Some recipients said they could manage without the bonus. Kimberly
Zaiger, for example, lost her job as a convention services manager at a
hotel in San Antonio, Texas, in March. The extra money ``was helpful,''
she said, enabling her to offer some financial help to her grown
children, but ``not crucial.''

Ms. Zaiger, 52, will still get \$521 a week in regular state jobless
benefits in addition to a share of her ex-husband's military pension.
She also has savings and a fiancé who is working and splits some bills.

``I've been crunching the numbers and prioritizing and I'll be fine,''
she said.

But for others, the weekly \$600 made the difference between staying
afloat and ruin.

Rebecca Mallery, 46, was cobbling together a living from three jobs when
the coronavirus shut the economy. She lost them all on the same day:
March 15.

Her earnings had averaged less than \$250 a week --- compared with the
\$600 in supplemental pandemic unemployment assistance that arrived with
her unemployment insurance.

But without any supplement, she faces bankruptcy.

\includegraphics{https://static01.graylady3jvrrxbe.onion/images/2020/07/29/business/29virus-cliff/merlin_175058202_e20f9af3-9850-4411-a5f1-b6068c45d952-articleLarge.jpg?quality=75\&auto=webp\&disable=upscale}

She qualifies for unemployment benefits for only one of her jobs, a
part-time gig conducting surveys for the Las Vegas Convention and
Visitors Authority. That comes to \$96 a week. With that and a small
monthly disability check, she has enough to cover her \$815 in monthly
rent, but not much else.

A single mother with a 9-year-old son, Ms. Mallery lives just across the
Nevada border in Arizona and has been looking for work. But with the
tourism industry struggling, there isn't much available.

``There's just nothing left out there right now,'' she said. Even if
there were, she wonders how she would manage if schools don't fully
reopen and she has to look after her son during the day. ``How do you go
to work?'' she said. ``When you're a single parent, that leaves you with
nothing, there are no options.''

She worries that a job that involves contact with the public puts her at
higher risk of exposing her mother, who has cancer, to the virus.

When the Lowe's near her reopened, though, she quickly applied. ``I was
out in the garden center, shuffling around cactuses in 100-degree heat,
but it was great,'' she said. ``I was glad to be working.'' But she
picked up only a couple of shifts.

With the extra unemployment benefits running out and little hope of
finding steady work, Ms. Mallery is applying for subsidized housing,
even though she hates to leave her townhouse, which has three bedrooms
and a yard where her son can play.

``I can't use any of my credit cards anymore --- they're all maxed
out,'' she said. ``I'm going to have to declare bankruptcy.''

Congressional Democrats have pushed for another
\href{https://www.nytimes3xbfgragh.onion/2020/07/28/us/politics/coronavirus-relief-bills-house-senate.html}{\$3
trillion relief package} that would preserve the \$600 weekly
supplements through January. Senate Republicans and the administration
have countered with a \$1 trillion proposal that would reduce the extra
benefit to \$200.

That smaller sum would more than replace what Ms. Mallery earned from
her three jobs before the pandemic. Other workers, though, would be left
without enough to cover the essentials.

In Chicago, more than 1,700 miles away from Ms. Mallery's home, Grey
Parker has been trying to map out a budget for the next few months.

Before the pandemic, he had snagged his dream job, a quality control
engineer at Production Resource Group, one of the largest
live-entertainment production companies in the world.

As coronavirus lockdowns shut down one live event after another, Mr.
Parker was furloughed. His wife's part-time work cleaning houses dried
up as well.

His package of jobless benefits, including the supplement, replaced
about half of their \$80,000 to \$90,000 annual income.

Money was tight, said Mr. Parker, who has a 6-year-old daughter, but
``we weren't worried about food, and we weren't worried about rent.''

Without the extra weekly benefits, Mr. Parker will receive \$350 a week.
He contacted his utility company to set up a deferred payment plan and
arranged to start receiving food from local food banks.

But he can't figure out how to keep paying the \$1,800 rent for his
house beyond September.

``We are now facing potential ruin within a couple of months,'' he said.

He also worries about his health. Mr. Parker, 50, has a vascular disease
called thrombosis, a blood-clotting disorder that puts him in a
high-risk group for complications if he were to contract Covid-19. Even
with the \$600 supplement, he didn't have enough money for the \$240
monthly cost of continuing his health insurance.

Without insurance, though, the cost of the daily medication he takes to
prevent blood clots rose from \$10 a month to \$500 --- far more than he
could afford. In the first few weeks of his furlough, he rationed his
medication, taking only half the amount he needed, which gave him a
frightening series of symptoms: bruising, dizziness and an increased
risk of stroke. He recently qualified for emergency assistance from the
pharmaceutical company Bristol Myers Squibb, which will provide a 90-day
supply. After that, Mr. Parker is unsure of what to do --- maybe ask for
donations through GoFundMe.

This week, just after the final jobless benefit supplements were sent
out, Mr. Parker learned that his company was extending the furlough
through September. He hopes to return to work, but doubts that the
live-event industry will be back in the fall. Even if it is, he said,
his medical condition will make him think twice about returning to work
before a vaccine is available.

The weekly \$600 premium was a life preserver. ``It gave us our one
sense of security,'' he said. ``Now that's gone.''

Advertisement

\protect\hyperlink{after-bottom}{Continue reading the main story}

\hypertarget{site-index}{%
\subsection{Site Index}\label{site-index}}

\hypertarget{site-information-navigation}{%
\subsection{Site Information
Navigation}\label{site-information-navigation}}

\begin{itemize}
\tightlist
\item
  \href{https://help.nytimes3xbfgragh.onion/hc/en-us/articles/115014792127-Copyright-notice}{©~2020~The
  New York Times Company}
\end{itemize}

\begin{itemize}
\tightlist
\item
  \href{https://www.nytco.com/}{NYTCo}
\item
  \href{https://help.nytimes3xbfgragh.onion/hc/en-us/articles/115015385887-Contact-Us}{Contact
  Us}
\item
  \href{https://www.nytco.com/careers/}{Work with us}
\item
  \href{https://nytmediakit.com/}{Advertise}
\item
  \href{http://www.tbrandstudio.com/}{T Brand Studio}
\item
  \href{https://www.nytimes3xbfgragh.onion/privacy/cookie-policy\#how-do-i-manage-trackers}{Your
  Ad Choices}
\item
  \href{https://www.nytimes3xbfgragh.onion/privacy}{Privacy}
\item
  \href{https://help.nytimes3xbfgragh.onion/hc/en-us/articles/115014893428-Terms-of-service}{Terms
  of Service}
\item
  \href{https://help.nytimes3xbfgragh.onion/hc/en-us/articles/115014893968-Terms-of-sale}{Terms
  of Sale}
\item
  \href{https://spiderbites.nytimes3xbfgragh.onion}{Site Map}
\item
  \href{https://help.nytimes3xbfgragh.onion/hc/en-us}{Help}
\item
  \href{https://www.nytimes3xbfgragh.onion/subscription?campaignId=37WXW}{Subscriptions}
\end{itemize}
