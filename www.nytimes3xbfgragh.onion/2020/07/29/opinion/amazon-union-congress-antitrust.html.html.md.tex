Sections

SEARCH

\protect\hyperlink{site-content}{Skip to
content}\protect\hyperlink{site-index}{Skip to site index}

\href{https://myaccount.nytimes3xbfgragh.onion/auth/login?response_type=cookie\&client_id=vi}{}

\href{https://www.nytimes3xbfgragh.onion/section/todayspaper}{Today's
Paper}

\href{/section/opinion}{Opinion}\textbar{}Amazon Has Too Much Power.
Take It Back.

\url{https://nyti.ms/39BnnJH}

\begin{itemize}
\item
\item
\item
\item
\item
\item
\end{itemize}

Advertisement

\protect\hyperlink{after-top}{Continue reading the main story}

\href{/section/opinion}{Opinion}

Supported by

\protect\hyperlink{after-sponsor}{Continue reading the main story}

\hypertarget{amazon-has-too-much-power-take-it-back}{%
\section{Amazon Has Too Much Power. Take It
Back.}\label{amazon-has-too-much-power-take-it-back}}

The tech company's workers need a union.

By Tim Bray and Christy Hoffman

Mr. Bray is a former vice president at Amazon. Ms. Hoffman is the
general secretary of UNI Global Union.

\begin{itemize}
\item
  July 29, 2020, 5:00 a.m. ET
\item
  \begin{itemize}
  \item
  \item
  \item
  \item
  \item
  \item
  \end{itemize}
\end{itemize}

\includegraphics{https://static01.graylady3jvrrxbe.onion/images/2020/07/29/opinion/29BrayHoffman2/merlin_172114362_b9359e17-e151-45de-91ec-6e1a77ce4b46-articleLarge.jpg?quality=75\&auto=webp\&disable=upscale}

Covid-19 has created strange bedfellows. Six months ago, a labor leader
and an Amazon vice president would have been on opposite sides in
discussing the future of work at Big Tech in general and Amazon in
particular. Then on May 1, one of us, Tim, walked away from a senior
role at Amazon Web Services, and potentially millions in compensation,
in protest over the firing of workers who spoke out about conditions in
the company's warehouses.

During the pandemic, we've seen Big Tech share prices and revenue
rocket, while some of Amazon's warehouse workers say they fear coming to
work and catching the coronavirus. The company's decision to fire the
activists who demanded safer jobs is unacceptable.

Both of us now agree: Amazon --- and the rest of Big Tech --- must
change. And that includes allowing its workers to unionize.

We're not alone in wanting accountability from these companies:
\href{https://www.nytimes3xbfgragh.onion/2020/07/28/technology/amazon-apple-facebook-google-antitrust-hearing.html}{Jeff
Bezos} and the heads of Facebook, Google and Apple will appear before
the House judiciary's antitrust subcommittee today.

The coronavirus has killed over half a million people worldwide and
pushed global unemployment to rates not seen since the Great Depression.
Shared sacrifice is called for, yet the burden has been far from even.
Since mid-March, when quarantined shoppers turned to Amazon's vast
retail platform,
\href{https://ycharts.com/companies/AMZN/market_cap}{its shareholder
value increased} by nearly \$500 billion, to more than \$1.4 trillion.
Stock market shares are owned disproportionately by the richest people
in society, and by
\href{https://www.vox.com/recode/2020/7/21/21332166/tech-billionaires-wealth-elon-musk-steve-ballmer-jeff-bezos-pandemic-covid}{Mr.
Bezos in particular}; his lead over the other richest people on earth
has increased markedly.

This wealth is not shared with the workers who help create it. The
temporary Covid-19-related hourly raise
\href{https://www.nytimes3xbfgragh.onion/2020/07/14/business/coronavirus-essential-workers-pay-raises.html}{was
rolled back in June}, but
\href{https://finance.yahoo.com/news/amazon-amzn-beat-q2-earnings-125512326.html}{the
order flow remains high}, making the already stressful work of those who
sort, package and deliver Amazon goods even worse.

\includegraphics{https://static01.graylady3jvrrxbe.onion/images/2020/05/29/autossell/OP-AMZAON-THUMB2/OP-AMZAON-THUMB2-videoSixteenByNineJumbo1600.jpg}

As ** this was unfolding, most of Big Tech, including Amazon, sent
white-collar workers home to ``flatten the curve'' and fight the
pandemic. Tim saw company leadership go to great lengths to make sure
this new system was working and actively seek feedback from the remote
workers. Christy heard from a warehouse employee who said productivity
targets made it difficult for workers to take a break even for hand
washing without a mark on their record. Pay for warehouse workers
\href{https://www.washingtonpost.com/business/economy/amazons-15-minimum-wage-doesnt-end-debate-over-whether-its-creating-good-jobs/2018/10/05/b1da23a0-c802-11e8-9b1c-a90f1daae309_story.html}{starts
at \$15 an hour} with
\href{https://www.amazon.jobs/en/landing_pages/pto-overview-us}{minimal
access to time off}; in May Amazon
\href{https://www.bloomberg.com/news/articles/2020-04-24/amazon-asks-workers-sheltering-at-home-to-return-or-seek-leave?sref=ExbtjcSG}{ended
the unpaid leave policy} that for a few weeks allowed them to stay home
if they had Covid-19 symptoms. **** The contrast in the treatment of
knowledge and warehouse workers couldn't be starker. Equally clear is
the cause: One group has power, the other doesn't.

Amazon's decision to fire the activists was easy to make in the United
States, where Amazon workers have no union and are left to fend for
themselves. With no right to paid sick leave or protection from unfair
dismissal, American workers are among the most vulnerable in the world
to pressure from any employer, not just Amazon.

Union-represented Amazon workers in Spain, Italy, France and Germany
initially failed to resolve their concerns through negotiation, but with
court action, regulatory intervention and strikes, they got their needs
addressed.

Let's look at France: Unions there
\href{https://www.nytimes3xbfgragh.onion/2020/04/24/business/amazon-france-unions-coronavirus.html}{brought
a civil case} arguing that Amazon had taken inadequate steps to protect
workers from infection risk and that it had sidestepped the unions'
statutory role. The court ordered Amazon to limit its sales to only
``essential'' items, or face harsh penalties until it could reach a
safety agreement with the unions. Rather than negotiate, Amazon
\href{https://www.nytimes3xbfgragh.onion/2020/04/15/business/amazon-france-covid.html}{closed
its French operations} and appealed. But the appellate court also sided
with the workers, who ultimately
\href{https://www.nytimes3xbfgragh.onion/2020/05/16/business/amazon-france-unions-coronavirus.html}{negotiated
a settlement} including mandatory union consultation over safety
measures, union hiring of external experts to assess the measures'
effectiveness and a continued increase in workers' hourly pay. The news
from Europe shows that Amazon can work with unions and get good results.

Both of us want Amazon to share the wealth with workers and stop putting
the relentless pursuit of revenue growth ahead of all other concerns.
One way or another, this requires putting more power in the hands of
workers. Regulation and legislation are part of the solution. But
there's no need to wait; power can be taken, not just given. That's what
unions are for.

Amazon is a data-driven company. It should recognize the evidence
showing that countries with more collective bargaining have a stronger
social fabric and better growth, and are more able to weather economic
ups and downs. Businesses with collective bargaining relationships,
including Auchan Retail and Carrefour, navigated the Covid-19 crisis
with less disruption to their businesses and emerged with their
reputations intact and even enhanced.

For its own future and the future of the global economy, Amazon should
become more responsive to the women and men who've enriched shareholders
and be willing to recognize and bargain with their representatives. When
it comes to the rights of its workers, it should be a leader, not a
laggard.

It's not just Amazon: The need for more unionization is urgent across
Big Tech. Amazon stands out because it combines the extraordinary profit
margins of these companies with employing hundreds of thousands of
front-line workers. There are fewer of these workers at the other iconic
tech companies, but nevertheless their employees also deserve a voice
over the issues that matter to them.

The question for Mr. Bezos and the billionaires of the world is: Are
they ready to rise to the occasion? Will Big Tech listen to and work
with its employees to help the world overcome the worst economic and
social crisis in recent history?

Tim Bray is a former vice president at Amazon. Christy Hoffman is the
general secretary of UNI Global Union.

\emph{The Times is committed to publishing}
\href{https://www.nytimes3xbfgragh.onion/2019/01/31/opinion/letters/letters-to-editor-new-york-times-women.html}{\emph{a
diversity of letters}} \emph{to the editor. We'd like to hear what you
think about this or any of our articles. Here are some}
\href{https://help.nytimes3xbfgragh.onion/hc/en-us/articles/115014925288-How-to-submit-a-letter-to-the-editor}{\emph{tips}}\emph{.
And here's our email:}
\href{mailto:letters@NYTimes.com}{\emph{letters@NYTimes.com}}\emph{.}

\emph{Follow The New York Times Opinion section on}
\href{https://www.facebookcorewwwi.onion/nytopinion}{\emph{Facebook}}\emph{,}
\href{http://twitter.com/NYTOpinion}{\emph{Twitter (@NYTopinion)}}
\emph{and}
\href{https://www.instagram.com/nytopinion/}{\emph{Instagram}}\emph{.}

Advertisement

\protect\hyperlink{after-bottom}{Continue reading the main story}

\hypertarget{site-index}{%
\subsection{Site Index}\label{site-index}}

\hypertarget{site-information-navigation}{%
\subsection{Site Information
Navigation}\label{site-information-navigation}}

\begin{itemize}
\tightlist
\item
  \href{https://help.nytimes3xbfgragh.onion/hc/en-us/articles/115014792127-Copyright-notice}{©~2020~The
  New York Times Company}
\end{itemize}

\begin{itemize}
\tightlist
\item
  \href{https://www.nytco.com/}{NYTCo}
\item
  \href{https://help.nytimes3xbfgragh.onion/hc/en-us/articles/115015385887-Contact-Us}{Contact
  Us}
\item
  \href{https://www.nytco.com/careers/}{Work with us}
\item
  \href{https://nytmediakit.com/}{Advertise}
\item
  \href{http://www.tbrandstudio.com/}{T Brand Studio}
\item
  \href{https://www.nytimes3xbfgragh.onion/privacy/cookie-policy\#how-do-i-manage-trackers}{Your
  Ad Choices}
\item
  \href{https://www.nytimes3xbfgragh.onion/privacy}{Privacy}
\item
  \href{https://help.nytimes3xbfgragh.onion/hc/en-us/articles/115014893428-Terms-of-service}{Terms
  of Service}
\item
  \href{https://help.nytimes3xbfgragh.onion/hc/en-us/articles/115014893968-Terms-of-sale}{Terms
  of Sale}
\item
  \href{https://spiderbites.nytimes3xbfgragh.onion}{Site Map}
\item
  \href{https://help.nytimes3xbfgragh.onion/hc/en-us}{Help}
\item
  \href{https://www.nytimes3xbfgragh.onion/subscription?campaignId=37WXW}{Subscriptions}
\end{itemize}
