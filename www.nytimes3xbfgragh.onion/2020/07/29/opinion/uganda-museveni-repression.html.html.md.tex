Sections

SEARCH

\protect\hyperlink{site-content}{Skip to
content}\protect\hyperlink{site-index}{Skip to site index}

\href{https://myaccount.nytimes3xbfgragh.onion/auth/login?response_type=cookie\&client_id=vi}{}

\href{https://www.nytimes3xbfgragh.onion/section/todayspaper}{Today's
Paper}

\href{/section/opinion}{Opinion}\textbar{}My Torture at the Hands of
America's Favorite African Strongman

\url{https://nyti.ms/3geIjZu}

\begin{itemize}
\item
\item
\item
\item
\item
\end{itemize}

Advertisement

\protect\hyperlink{after-top}{Continue reading the main story}

\href{/section/opinion}{Opinion}

Supported by

\protect\hyperlink{after-sponsor}{Continue reading the main story}

\hypertarget{my-torture-at-the-hands-of-americas-favorite-african-strongman}{%
\section{My Torture at the Hands of America's Favorite African
Strongman}\label{my-torture-at-the-hands-of-americas-favorite-african-strongman}}

Yoweri Museveni, the country's president and the Pentagon's closest
military ally in Africa, deploys security forces to assault opposition
lawmakers.

By Bobi Wine

Mr. Wine is a musician and a member of the Ugandan Parliament.

\begin{itemize}
\item
  July 29, 2020
\item
  \begin{itemize}
  \item
  \item
  \item
  \item
  \item
  \end{itemize}
\end{itemize}

\includegraphics{https://static01.graylady3jvrrxbe.onion/images/2020/07/29/opinion/29Wine/29Wine-articleLarge.jpg?quality=75\&auto=webp\&disable=upscale}

KAMPALA, Uganda --- Brutal policing is a global crisis, but America's
favorite African strongman, Yoweri Museveni, Uganda's president since
1986, has deployed his own security forces to a particularly malign end:
assaulting opposition parliamentary lawmakers to crush the democratic
challenge he is facing.

I speak from experience. I am a member of Uganda's Parliament and also a
musician, activist and founder of the opposition
\href{https://www.thenation.com/article/archive/can-bobi-wine-unite-uganda-and-bring-down-a-dictator/}{People
Power movement}. For the past three years, we have been seeking social,
economic and political change with the support of Uganda's youth --- 80
percent of the population --- who face dire poverty.

On April 19, my colleague Francis Zaake, a 29-year-old member of
Parliament,
\href{https://www.buzzfeednews.com/article/lesterfeder/uganda-francis-zaake-coronavirus}{was
arrested and tortured}. Previously strapping and healthy, he now walks
with a cane from the beatings he received there.

Why torture an elected member of Parliament?

On March 31, the Ugandan government imposed a strict coronavirus
lockdown without notice, leaving many citizens unable to work. When some
parliamentarians began passing out relief food to constituents, Mr.
Museveni
\href{https://www.softpower.ug/covid-19-mps-nsereko-luttamaguzi-risk-being-arrested-for-distributing-food/}{threatened
to arrest them.} In theory, the ban was universal; in practice,
politicians from Mr. Museveni's ruling party continued passing out food.
The message was clear: Support the regime or starve.

Mr. Zaake's crime was delivering food to the hungry while being an
opposition member.

Assaulting elected members of Parliament and their supporters is an
assault on the very idea of democracy. Ugandans have lived under Mr.
Museveni's tyranny for 34 years. We have had elections, but their
legitimacy has been marred by
\href{https://www.nybooks.com/daily/2016/05/16/uganda-cost-of-fake-democracy/}{rigging}
and
\href{https://www.hrw.org/news/2005/12/19/uganda-respect-opposition-right-campaign}{the
killing}\href{https://www.theguardian.com/world/2006/feb/26/uganda.deniscampbell}{and
torture} of opposition supporters.

\includegraphics{https://static01.graylady3jvrrxbe.onion/images/2020/07/29/opinion/29Wine2/29Wine2-articleLarge.jpg?quality=75\&auto=webp\&disable=upscale}

Twice, Uganda's Supreme Court seemed
\href{https://allafrica.com/stories/200105090061.html}{on the verge of
overturning} Mr. Museveni's election. After the 2016 election, Mr.
Museveni placed his main challenger, Kizza Besigye, under house arrest
so that he couldn't petition the court within the constitutionally
mandated time period.

As support for our People Power movement has grown in recent years, Mr.
Museveni has increased the frequency and brutality of attacks on
lawmakers.

On Aug. 13, 2018, I was with colleagues in Arua, a town in northern
Uganda, after a long day of campaigning. We were there to support an
opposition colleague who was running for Parliament in a special
election. All of a sudden, Uganda's Special Forces Command besieged our
hotel. They
\href{https://www.amnesty.org/en/latest/news/2018/08/uganda-investigate-death-of-opposition-politicians-driver/}{shot
and killed my driver,} Yasin Kawuma, who was sitting in the passenger
seat of my vehicle. The bullets seemed to have been intended for me.
Thirty-four of us, including three other lawmakers and the candidate
Kassiano Wadri, who eventually won the Arua special election, were
arrested.

We were held for more than a week. I and several others were tortured
and couldn't walk unaided when released. I traveled on crutches to the
United States for medical treatment.

We aren't the only legislators to have suffered at the hands of Uganda's
security forces. In September 2017, opposition lawmakers filibustered to
block a parliamentary bill to remove
\href{https://www.aljazeera.com/news/2017/09/uganda-introduces-bill-remove-presidential-age-limit-170927172204813.html}{the
age limit for the presidency}, set by the 1995 Constitution at 75 years.
The change would allow Mr. Museveni, who says he was born in 1944, to
run in 2021. An Afrobarometer poll in September 2017 suggested that
\href{https://www.nytimes3xbfgragh.onion/2017/12/20/world/africa/uganda-president-museveni-age-limit.html}{75
percent} of the population opposed lifting the age limit.

On Sept. 19, 2017, when the bill was to be introduced, Mr. Museveni
deployed armored vehicles and heavily armed police around Parliament to
prevent protests. Our filibuster managed to delay the bill's
introduction for a week, but on Sept. 27 dozens of plainclothes
operatives appeared on the floor of the Parliament.

About 30 lawmakers were arrested, including me. During the mayhem, six
operatives escorted a parliamentarian, Betty Nambooze, into a room
without security cameras. They pressed her against the wall while one of
them
\href{https://www.thenation.com/article/archive/us-turns-blind-eye-ugandas-assault-democracy/}{shoved
a knee into her back}, severely injuring her spine. She was flown to
India for surgery, enabling her to walk again, but was tortured again in
June 2018 and now walks with a cane.

Fearful and despondent, we dropped the filibuster campaign, and the age
limit on the president was removed in December 2017. The attacks on our
People Power movement have continued, and we have lost dozens of
activists and supporters to violence on the part of the security forces.
Yet support for our movement has increased.

I grew up in poverty and was fortunate to have a successful career as a
musician. I soon found myself singing about corruption, poverty and
oppression. Music galvanizes people but they can be empowered only
through politics, so I decided to run for the Parliament.

Last week, my colleagues and I formed
\href{https://www.nytimes3xbfgragh.onion/aponline/2020/07/22/world/africa/ap-af-uganda-bobi-wine.html}{the
National Unity Platform}, a political party to challenge Mr. Museveni
and his party in Uganda's next election, expected in early 2021.

We stand for democratic rule; depoliticizing the security forces,
judiciary and other institutions; peace in our region; and fighting
Uganda's rampant corruption. We maintain that this will help create the
conditions for Uganda's economy to thrive.

We regret to say that we might not have suffered for so long had
Washington not chosen to ignore Mr. Museveni's abuses. He is among the
Pentagon's closest African security allies, with
\href{https://www.amazon.com/dp/B074CY9SZN/ref=dp-kindle-redirect?_encoding=UTF8\&btkr=1\#customerReviews}{troops
in Somalia} and guards
\href{https://theintercept.com/2016/02/22/wounded-in-iraq-ugandan-contractors-fight-for-compensation-in-america/}{under
U.S. command in Iraq}. However, he has also stoked conflict both within
\href{https://www.dw.com/en/ugandas-double-game-in-south-sudan-civil-war-revealed/a-46500925}{Uganda
and in neighboring countries}, while hoodwinking Washington into
trusting him on security matters.

The international community needs to rethink its financial, moral and
military assistance to our tormentors in Uganda and stand up for
democracy.

Bobi Wine is a musician and a member of the Parliament in Uganda.

\emph{The Times is committed to publishing}
\href{https://www.nytimes3xbfgragh.onion/2019/01/31/opinion/letters/letters-to-editor-new-york-times-women.html}{\emph{a
diversity of letters}} \emph{to the editor. We'd like to hear what you
think about this or any of our articles. Here are some}
\href{https://help.nytimes3xbfgragh.onion/hc/en-us/articles/115014925288-How-to-submit-a-letter-to-the-editor}{\emph{tips}}\emph{.
And here's our email:}
\href{mailto:letters@NYTimes.com}{\emph{letters@NYTimes.com}}\emph{.}

\emph{Follow The New York Times Opinion section on}
\href{https://www.facebookcorewwwi.onion/nytopinion}{\emph{Facebook}}\emph{,}
\href{http://twitter.com/NYTOpinion}{\emph{Twitter (@NYTopinion)}}
\emph{and}
\href{https://www.instagram.com/nytopinion/}{\emph{Instagram}}\emph{.}

Advertisement

\protect\hyperlink{after-bottom}{Continue reading the main story}

\hypertarget{site-index}{%
\subsection{Site Index}\label{site-index}}

\hypertarget{site-information-navigation}{%
\subsection{Site Information
Navigation}\label{site-information-navigation}}

\begin{itemize}
\tightlist
\item
  \href{https://help.nytimes3xbfgragh.onion/hc/en-us/articles/115014792127-Copyright-notice}{©~2020~The
  New York Times Company}
\end{itemize}

\begin{itemize}
\tightlist
\item
  \href{https://www.nytco.com/}{NYTCo}
\item
  \href{https://help.nytimes3xbfgragh.onion/hc/en-us/articles/115015385887-Contact-Us}{Contact
  Us}
\item
  \href{https://www.nytco.com/careers/}{Work with us}
\item
  \href{https://nytmediakit.com/}{Advertise}
\item
  \href{http://www.tbrandstudio.com/}{T Brand Studio}
\item
  \href{https://www.nytimes3xbfgragh.onion/privacy/cookie-policy\#how-do-i-manage-trackers}{Your
  Ad Choices}
\item
  \href{https://www.nytimes3xbfgragh.onion/privacy}{Privacy}
\item
  \href{https://help.nytimes3xbfgragh.onion/hc/en-us/articles/115014893428-Terms-of-service}{Terms
  of Service}
\item
  \href{https://help.nytimes3xbfgragh.onion/hc/en-us/articles/115014893968-Terms-of-sale}{Terms
  of Sale}
\item
  \href{https://spiderbites.nytimes3xbfgragh.onion}{Site Map}
\item
  \href{https://help.nytimes3xbfgragh.onion/hc/en-us}{Help}
\item
  \href{https://www.nytimes3xbfgragh.onion/subscription?campaignId=37WXW}{Subscriptions}
\end{itemize}
