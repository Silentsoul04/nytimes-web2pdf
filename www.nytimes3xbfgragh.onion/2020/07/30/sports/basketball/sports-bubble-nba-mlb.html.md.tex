Sections

SEARCH

\protect\hyperlink{site-content}{Skip to
content}\protect\hyperlink{site-index}{Skip to site index}

\href{https://www.nytimes3xbfgragh.onion/section/sports/basketball}{Pro
Basketball}

\href{https://myaccount.nytimes3xbfgragh.onion/auth/login?response_type=cookie\&client_id=vi}{}

\href{https://www.nytimes3xbfgragh.onion/section/todayspaper}{Today's
Paper}

\href{/section/sports/basketball}{Pro Basketball}\textbar{}`Bubbles' Are
Working. But How Long Can Sports Stay Inside?

\url{https://nyti.ms/2EsvUD8}

\begin{itemize}
\item
\item
\item
\item
\item
\end{itemize}

Advertisement

\protect\hyperlink{after-top}{Continue reading the main story}

Supported by

\protect\hyperlink{after-sponsor}{Continue reading the main story}

\hypertarget{bubbles-are-working-but-how-long-can-sports-stay-inside}{%
\section{`Bubbles' Are Working. But How Long Can Sports Stay
Inside?}\label{bubbles-are-working-but-how-long-can-sports-stay-inside}}

The restricted, campuslike environments used by soccer and pro
basketball have proved (mostly) impervious to the coronavirus. But not
every league fits inside one.

\includegraphics{https://static01.graylady3jvrrxbe.onion/images/2020/07/30/sports/30virus-bubbles1/merlin_175084758_6b939c67-508f-4795-b6e9-3adb40cdc8c3-articleLarge.jpg?quality=75\&auto=webp\&disable=upscale}

By \href{https://www.nytimes3xbfgragh.onion/by/andrew-keh}{Andrew Keh}

\begin{itemize}
\item
  July 30, 2020
\item
  \begin{itemize}
  \item
  \item
  \item
  \item
  \item
  \end{itemize}
\end{itemize}

Amid all the uncertainties saddling the resumption of sports in the
shadow of the coronavirus, this much seems clear: Bubbles work. How long
they will remain in use is another question.

While a
\href{https://twitter.com/Phillies/status/1288882323614846976?s=20}{spreading
coronavirus outbreak} has threatened to derail the abbreviated season in
Major League Baseball, which elected not to sequester its players when
it began play last week, it has been hard to ignore how serenely play
has continued inside American sports' so-called bubbles, the tightly
controlled campus environments where some leagues have elected to
operate.

The National Women's Soccer League completed a virus-free monthlong
tournament inside a Utah bubble --- albeit after one team
\href{https://www.nytimes3xbfgragh.onion/2020/06/22/sports/soccer/orlando-pride-nwsl-coronavirus.html}{dropped
out} before arriving because of an outbreak. Major League Soccer, after
losing two teams during its own early stumbles, has not recorded a
positive test since July 10 at its enclosed setup in Florida.

The pattern has continued with the N.B.A., which restarted its season on
Thursday at Walt Disney World and has not logged a case since July 13,
and the W.N.B.A., which opened play last weekend and recorded its last
positive test back on July 9. The N.H.L. will have similar hopes of
safety when it returns to play this weekend inside two bubble sites in
Canada.

``So far they have looked very intact and safe, and constant vigilance
is going to be required to make sure they stay that way,'' Dr. Zachary
Binney, an epidemiologist specializing in sports at Emory University,
said about the efficacy of bubbles. ``I was always optimistic, but this
has exceeded my expectations.''

\includegraphics{https://static01.graylady3jvrrxbe.onion/images/2020/07/31/sports/30virus-bubble3/merlin_174978843_89d0ed7b-0222-4875-a86b-b5da49799360-articleLarge.jpg?quality=75\&auto=webp\&disable=upscale}

But while bubbles are proving to be the best and safest way to conduct
the business of playing sports, they do not last forever. And it is what
comes next --- as teams and leagues attempt something resembling
normalcy in communities where the virus is still on the rise --- that
will be a riskier test for sports.

M.L.S., for example, will push ahead with plans to allow its teams to
resume play in their home stadiums later this summer, if local rules
allow it. The N.F.L. is expected to open its 2020 season this fall with
teams in their home markets as well. And baseball has vowed to push
forth as long as it can, even as it
\href{https://www.nytimes3xbfgragh.onion/2020/07/27/sports/baseball/coronavirus-yankees-marlins-phillies.html}{contorts
its competitive structures} at the whims of a capricious virus.

The only reason bubbles became necessary, of course, is because the
United States failed to wrest control of the virus in ways that other
developed nations have. In Europe, most of the world's top soccer
leagues finished their seasons with teams playing in their own stadiums.
Fans were allowed back into baseball stadiums in South Korea this week.

But M.L.B.'s stunning outbreak --- 17 players on the Miami Marlins
tested positive for the virus in recent days, causing a chain reaction
of scheduling disruptions --- served as a stark reminder of the risks
associated with resuming work in American communities right now.

Image

Social distancing hasn't saved baseball: The Phillies postponed a series
on Thursday after two staff members tested positive.Credit...Chris
Szagola/Associated Press

``We all read of what's going on in baseball,'' said Bob Bradley, the
coach of Los Angeles F.C, who credited M.L.S. players for their
discipline inside the league's Florida bubble and for enduring the
psychological challenges of being far from their homes and families.
``It's still hard to know whether that's something that is going to
close down the league, or whether it's just something that happens in a
team and spreads and has to be dealt with.''

M.L.S. officials, now pondering life after the bubble, have been
watching how baseball navigates its crisis.

\hypertarget{sports-and-the-virus}{%
\subsubsection{Sports and the Virus}\label{sports-and-the-virus}}

\paragraph{}

Updated Sept. 9, 2020

Here's what's happening as the world of sports slowly comes back to
life:

\begin{itemize}
\item
  \begin{itemize}
  \tightlist
  \item
    September Saturdays at Penn State are usually the apex of a week of
    hype. Now, as at other college football destinations, the approach
    of autumn has been
    \href{https://www.nytimes3xbfgragh.onion/2020/09/09/sports/penn-state-college-football-canceled.html?action=click\&pgtype=Article\&state=default\&region=MAIN_CONTENT_2\&context=storylines_keepup}{unusually
    quiet}~there.
  \item
    More than half the players who made the quarterfinals at the U.S.
    Open were not supposed to be there. It's a
    \href{https://www.nytimes3xbfgragh.onion/2020/09/09/sports/us-open-crowd.html?action=click\&pgtype=Article\&state=default\&region=MAIN_CONTENT_2\&context=storylines_keepup}{little
    bit easier}~when there are no fans, some say.
  \item
    In a pandemic,
    \href{https://www.nytimes3xbfgragh.onion/2020/09/08/sports/ironman-tallinn-triathlon-pandemic.html?action=click\&pgtype=Article\&state=default\&region=MAIN_CONTENT_2\&context=storylines_keepup}{getting
    to a triathlon}~is as hard as finishing it. The first Ironman race
    since March, in Tallinn, Estonia, included travel restrictions,
    temperature checks, masked volunteers and medals handed over in
    bags.
  \end{itemize}
\end{itemize}

But even as questions about the wisdom of returning to play in dozens of
virus-ridden communities have grown louder, there has been a level of
confidence internally that the lessons learned while inside the bubble
--- the importance of constant testing, mask-wearing and, more
important, the conscientious conduct of athletes --- will serve players
well outside of it. For now, though, M.L.S. remains committed to
returning teams to the field in their home markets later this summer.

``We have to be mindful of what's happening in the markets where we're
trying to play, but the commitment we'll have there is the same
commitment we've had here, which is that we're prioritizing the health
and safety of all of our participants,'' said M.L.S. deputy commissioner
Mark Abbott, who has been living and working inside the league's bubble
at the ESPN Wide World of Sports complex at Disney World near Orlando,
alongside the players and other staff members.

Image

Bob Bradley, the coach of Los Angeles F.C., credited the success of the
M.L.S. bubble in Florida to players' discipline inside
it.Credit...Carter Augustine

The drawbacks of bubbles, of course, are plain. They are difficult to
organize, expensive to maintain and emotionally taxing on players, who
cannot return to their homes for weeks or months at a time.

Michele Roberts, the executive director of the N.B.A. Players
Association, said in an interview this week that the league and union
were watching closely for any ``adverse consequences of being segregated
from family and community for extended periods of time.''

Roberts said that for all the safety afforded by a sport's bubble
environment, the emotional strain on the people inside it was obvious.

``Months of life in this bubble is not an extended vacation,'' said
Roberts, who has been on site at Disney World among the teams. ``I'm
reminded of this every time I see a player doing FaceTime with a young
child.''

And still, because the concept is working, Roberts told ESPN on Tuesday
that the ongoing spread of the virus could lead the N.B.A. to play the
2020-2021 season, which it hopes to start in December, inside a bubble,
too. ``So it may be that, if the bubble is the way to play, then that is
likely going to be the way we play next season, if things remain as they
are,'' she said.

The size and structural ambitions of certain leagues can make bubbles
seem impractical. The N.F.L., for instance, features 53-man rosters and
almost innumerable staff members, and it typically runs a six-month
season, making the sequestering of players a greater logistical,
financial and emotional challenge than the one tackled by the N.B.A.

Image

The N.H.L. will resume its season this weekend at two sites in
Canada.Credit...Darryl Dyck/The Canadian Press, via Associated Press

Keeping players inside a strictly controlled environment for more than a
few months also has limited appeal. The teams that reach the N.B.A.
Finals, for example, can expect to be inside theirs until October.

Lisa Baird, the commissioner of the N.W.S.L., said her league's compact
tournament schedule and the stresses of quarantining in a hotel had
required a high level of sustained intensity from players that would be
difficult to keep up over a longer period. She said the league was still
planning its next move, including a possible return to play this fall,
but that another restricted-campus setup was not on the table.

``There's the old adage, `It's a marathon, not a sprint,''' Baird said.
``But with our format, it was both a sprint and a marathon.''

The leagues' successes inside their respective bubbles will continue to
raise moral questions about their very existence, particularly in light
of the sheer number of daily coronavirus tests and laboratory resources
required to keep the operations running, all while testing logjams
persist around the country.

In this context, the contrasting fortunes of baseball and the sports
world's bubble-dwellers could lead one to a discomfiting conclusion
about the state of the industry:

Bubbles, Binney said, ``may be the only way you can safely have sports
in the U.S.A. right now.''

That premise will soon be put to the test.

Advertisement

\protect\hyperlink{after-bottom}{Continue reading the main story}

\hypertarget{site-index}{%
\subsection{Site Index}\label{site-index}}

\hypertarget{site-information-navigation}{%
\subsection{Site Information
Navigation}\label{site-information-navigation}}

\begin{itemize}
\tightlist
\item
  \href{https://help.nytimes3xbfgragh.onion/hc/en-us/articles/115014792127-Copyright-notice}{©~2020~The
  New York Times Company}
\end{itemize}

\begin{itemize}
\tightlist
\item
  \href{https://www.nytco.com/}{NYTCo}
\item
  \href{https://help.nytimes3xbfgragh.onion/hc/en-us/articles/115015385887-Contact-Us}{Contact
  Us}
\item
  \href{https://www.nytco.com/careers/}{Work with us}
\item
  \href{https://nytmediakit.com/}{Advertise}
\item
  \href{http://www.tbrandstudio.com/}{T Brand Studio}
\item
  \href{https://www.nytimes3xbfgragh.onion/privacy/cookie-policy\#how-do-i-manage-trackers}{Your
  Ad Choices}
\item
  \href{https://www.nytimes3xbfgragh.onion/privacy}{Privacy}
\item
  \href{https://help.nytimes3xbfgragh.onion/hc/en-us/articles/115014893428-Terms-of-service}{Terms
  of Service}
\item
  \href{https://help.nytimes3xbfgragh.onion/hc/en-us/articles/115014893968-Terms-of-sale}{Terms
  of Sale}
\item
  \href{https://spiderbites.nytimes3xbfgragh.onion}{Site Map}
\item
  \href{https://help.nytimes3xbfgragh.onion/hc/en-us}{Help}
\item
  \href{https://www.nytimes3xbfgragh.onion/subscription?campaignId=37WXW}{Subscriptions}
\end{itemize}
