Sections

SEARCH

\protect\hyperlink{site-content}{Skip to
content}\protect\hyperlink{site-index}{Skip to site index}

\href{https://www.nytimes3xbfgragh.onion/section/books}{Books}

\href{https://myaccount.nytimes3xbfgragh.onion/auth/login?response_type=cookie\&client_id=vi}{}

\href{https://www.nytimes3xbfgragh.onion/section/todayspaper}{Today's
Paper}

\href{/section/books}{Books}\textbar{}Isabel Wilkerson's `Caste' Is an
`Instant American Classic' About Our Abiding Sin

\url{https://nyti.ms/2Xfu0wc}

\begin{itemize}
\item
\item
\item
\item
\item
\end{itemize}

Advertisement

\protect\hyperlink{after-top}{Continue reading the main story}

Supported by

\protect\hyperlink{after-sponsor}{Continue reading the main story}

\href{/column/books-of-the-times}{Books of The Times}

\hypertarget{isabel-wilkersons-caste-is-an-instant-american-classic-about-our-abiding-sin}{%
\section{Isabel Wilkerson's `Caste' Is an `Instant American Classic'
About Our Abiding
Sin}\label{isabel-wilkersons-caste-is-an-instant-american-classic-about-our-abiding-sin}}

By \href{https://www.nytimes3xbfgragh.onion/by/dwight-garner}{Dwight
Garner}

\begin{itemize}
\item
  July 31, 2020
\item
  \begin{itemize}
  \item
  \item
  \item
  \item
  \item
  \end{itemize}
\end{itemize}

\includegraphics{https://static01.graylady3jvrrxbe.onion/images/2020/08/03/books/03BOOKWILKERSON1/03BOOKWILKERSON1-articleLarge.jpg?quality=75\&auto=webp\&disable=upscale}

Buy Book ▾

\begin{itemize}
\tightlist
\item
  \href{https://www.amazon.com/gp/search?index=books\&tag=NYTBSREV-20\&field-keywords=Caste+Isabel+Wilkerson}{Amazon}
\item
  \href{https://du-gae-books-dot-nyt-du-prd.appspot.com/buy?title=Caste\&author=Isabel+Wilkerson}{Apple
  Books}
\item
  \href{https://www.anrdoezrs.net/click-7990613-11819508?url=https\%3A\%2F\%2Fwww.barnesandnoble.com\%2Fw\%2F\%3Fean\%3D9780593230251}{Barnes
  and Noble}
\item
  \href{https://www.anrdoezrs.net/click-7990613-35140?url=https\%3A\%2F\%2Fwww.booksamillion.com\%2Fp\%2FCaste\%2FIsabel\%2BWilkerson\%2F9780593230251}{Books-A-Million}
\item
  \href{https://bookshop.org/a/3546/9780593230251}{Bookshop}
\item
  \href{https://www.indiebound.org/book/9780593230251?aff=NYT}{Indiebound}
\end{itemize}

When you purchase an independently reviewed book through our site, we
earn an affiliate commission.

A critic shouldn't often deal in superlatives. He or she is here to
explicate, to expand context and to make fine distinctions. But
sometimes a reviewer will shout as if into a mountaintop megaphone. I
recently came upon William Kennedy's review of ``One Hundred Years of
Solitude,'' which he called ``the first piece of literature since the
Book of Genesis that should be required reading for the entire human
race.'' Kennedy wasn't far off.

I had these thoughts while reading Isabel Wilkerson's new book, ``Caste:
The Origins of Our Discontents.'' It's an extraordinary document, one
that strikes me as an instant American classic and almost certainly the
keynote nonfiction book of the American century thus far. It made the
back of my neck prickle from its first pages, and that feeling never
went away.

I told more than one person, as I moved through my days this past week,
that I was reading one of the most powerful nonfiction books I'd ever
encountered.

Wilkerson's book is about how brutal misperceptions about race have
disfigured the American experiment. This is a topic that major
historians and novelists have examined from many angles, with care,
anger, deep feeling and sometimes simmering wit.

Wilkerson's book is a work of synthesis. She borrows from all that has
come before, and her book stands on many shoulders. ``Caste'' lands so
firmly because the historian, the sociologist and the reporter are not
at war with the essayist and the critic inside her. This book has the
reverberating and patriotic slap of the best American prose writing.

\emph{{[} This book is one of our most anticipated titles of August.}
\href{https://www.nytimes3xbfgragh.onion/2020/07/30/books/new-august-books.html}{\emph{See
the full list}}\emph{. {]}}

This is a complicated book that does a simple thing. Wilkerson, who won
a Pulitzer Prize for national reporting while at The New York Times and
whose previous book,
\href{https://www.nytimes3xbfgragh.onion/2010/09/05/books/review/Oshinsky-t.html}{``The
Warmth of Other Suns: The Epic Story of America's Great Migration,''}
won the National Book Critics Circle Award, avoids words like ``white''
and ``race'' and ``racism'' in favor of terms like ``dominant caste,''
``favored caste,'' ``upper caste'' and ``lower caste.''

Some will quibble with her conflation of race and caste. (Social class
is a separate matter, which Wilkerson addresses only rarely.) She does
not argue that the words are synonyms. She argues that they ``can and do
coexist in the same culture and serve to reinforce each other. Race, in
the United States, is the visible agent of the unseen force of caste.
Caste is the bones, race the skin.'' The reader does not have to follow
her all the way on this point to find her book a fascinating thought
experiment. She persuasively pushes the two notions together while
addressing the internal wounds that, in America, have failed to clot.

A caste system, she writes, is ``an artificial construction, a fixed and
embedded ranking of human value that sets the presumed supremacy of one
group against the presumed inferiority of other groups on the basis of
ancestry and often immutable traits, traits that would be neutral in the
abstract but are ascribed life-and-death meaning.''

``As we go about our daily lives, caste is the wordless usher in a
darkened theater, flashlight cast down in the aisles, guiding us to our
assigned seats for a performance,'' Wilkerson writes. She observes that
caste ``is about respect, authority and assumptions of competence ---
who is accorded these and who is not.''

Image

Isabel Wilkerson, whose new book is ``Caste: The Origins of Our
Discontents.''Credit...Joe Henson

Wilkerson's usages neatly lift the mind out of old ruts. They enable her
to make unsettling comparisons between India's treatment of its
untouchables, or Dalits, Nazi Germany's treatment of Jews and America's
treatment of African-Americans. Each country ``relied on stigmatizing
those deemed inferior to justify the dehumanization necessary to keep
the lowest-ranked people at the bottom and to rationalize the protocols
of enforcement.''

Wilkerson does not shy from the brutality that has gone hand in hand
with this kind of dehumanization. As if pulling from a deep reservoir,
she always has a prime example at hand. It takes resolve and a strong
stomach to stare at the particulars, rather than the generalities, of
lives under slavery and Jim Crow and recent American experience. To feel
the heat of the furnace of individual experience. It's the kind of
resolve Americans will require more of.

``Caste'' gets off to an uncertain start. Its first pages summon, in
dystopian-novel fashion, the results of the 2016 election alongside
anthrax trapped in the permafrost being released into the atmosphere
because of global warming. Wilkerson is making a point about old poisons
returning to haunt us. But by pulling in global warming (a subject she
never returns to in any real fashion) so early in her book, you wonder
if ``Caste'' will be a mere grab bag of nightmare impressions.

It isn't.

Her consideration of the 2016 election, and American politics in
general, is sobering. To anyone who imagined that the election of Barack
Obama was a sign that America had begun to enter a post-racial era, she
reminds us that the majority of whites did not vote for him.

She poses the question so many intellectuals and pundits on the left
have posed, with increasing befuddlement: Why do the white working
classes in America vote against their economic interests?

She runs further with the notion of white resentment than many
commentators have been willing to, and the juices of her argument follow
the course of her knife. What these pundits had not considered,
Wilkerson writes, ``was that the people voting this way were, in fact,
voting their interests. Maintaining the caste system as it had always
been was in their interest. And some were willing to accept short-term
discomfort, forgo health insurance, risk contamination of the water and
air, and even die to protect their long-term interest in the hierarchy
as they had known it.''

In her novel ``Americanah,'' Chimamanda Ngozi Adichie suggested that
``maybe it's time to just scrap the word `racist.' Find something new.
Like Racial Disorder Syndrome. And we could have different categories
for sufferers of this syndrome: mild, medium and acute.''

Wilkerson has written a closely-argued book that largely avoids the word
``racism,'' yet stares it down with more humanity and rigor than nearly
all but a few books in our literature.

``Caste'' deepens our tragic sense of American history. It reads like
watching the slow passing of a long and demented cortege. In its
suggestion that we need something akin to South Africa's Truth and
Reconciliation Commission, her book points the way toward an alleviation
of alienation. It's a book that seeks to shatter a paralysis of will.
It's a book that changes the weather inside a reader.

While reading ``Caste,'' I thought often of a pair of sentences from
Colson Whitehead's novel
\href{https://www.nytimes3xbfgragh.onion/2016/08/03/books/review-the-underground-railroad-colson-whitehead.html}{``The
Underground Railroad.''} ``The Declaration {[}of Independence{]} is like
a map,'' he wrote. ``You trust that it's right, but you only know by
going out and testing it for yourself.''

Advertisement

\protect\hyperlink{after-bottom}{Continue reading the main story}

\hypertarget{site-index}{%
\subsection{Site Index}\label{site-index}}

\hypertarget{site-information-navigation}{%
\subsection{Site Information
Navigation}\label{site-information-navigation}}

\begin{itemize}
\tightlist
\item
  \href{https://help.nytimes3xbfgragh.onion/hc/en-us/articles/115014792127-Copyright-notice}{©~2020~The
  New York Times Company}
\end{itemize}

\begin{itemize}
\tightlist
\item
  \href{https://www.nytco.com/}{NYTCo}
\item
  \href{https://help.nytimes3xbfgragh.onion/hc/en-us/articles/115015385887-Contact-Us}{Contact
  Us}
\item
  \href{https://www.nytco.com/careers/}{Work with us}
\item
  \href{https://nytmediakit.com/}{Advertise}
\item
  \href{http://www.tbrandstudio.com/}{T Brand Studio}
\item
  \href{https://www.nytimes3xbfgragh.onion/privacy/cookie-policy\#how-do-i-manage-trackers}{Your
  Ad Choices}
\item
  \href{https://www.nytimes3xbfgragh.onion/privacy}{Privacy}
\item
  \href{https://help.nytimes3xbfgragh.onion/hc/en-us/articles/115014893428-Terms-of-service}{Terms
  of Service}
\item
  \href{https://help.nytimes3xbfgragh.onion/hc/en-us/articles/115014893968-Terms-of-sale}{Terms
  of Sale}
\item
  \href{https://spiderbites.nytimes3xbfgragh.onion}{Site Map}
\item
  \href{https://help.nytimes3xbfgragh.onion/hc/en-us}{Help}
\item
  \href{https://www.nytimes3xbfgragh.onion/subscription?campaignId=37WXW}{Subscriptions}
\end{itemize}
