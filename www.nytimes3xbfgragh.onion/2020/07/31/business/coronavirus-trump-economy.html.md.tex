Sections

SEARCH

\protect\hyperlink{site-content}{Skip to
content}\protect\hyperlink{site-index}{Skip to site index}

\href{https://www.nytimes3xbfgragh.onion/section/business}{Business}

\href{https://myaccount.nytimes3xbfgragh.onion/auth/login?response_type=cookie\&client_id=vi}{}

\href{https://www.nytimes3xbfgragh.onion/section/todayspaper}{Today's
Paper}

\href{/section/business}{Business}\textbar{}Does Trump Want to Save His
Economy?

\url{https://nyti.ms/2D83gH7}

\begin{itemize}
\item
\item
\item
\item
\item
\item
\end{itemize}

\begin{itemize}
\item
  \href{https://www.nytimes3xbfgragh.onion/interactive/2020/09/08/us/elections/results-new-hampshire-primary-elections.html?action=click\&pgtype=Article\&state=default\&region=TOP_BANNER\&context=storylines_menu}{New
  Hampshire Results}
\item
  \href{https://www.nytimes3xbfgragh.onion/live/2020/09/08/us/trump-vs-biden?action=click\&pgtype=Article\&state=default\&region=TOP_BANNER\&context=storylines_menu}{Election
  Updates}
\item
  \href{https://www.nytimes3xbfgragh.onion/interactive/2020/us/elections/election-states-biden-trump.html?action=click\&pgtype=Article\&state=default\&region=TOP_BANNER\&context=storylines_menu}{Paths
  to 270}
\item
  \href{https://www.nytimes3xbfgragh.onion/interactive/2020/08/31/us/politics/vote-by-mail-deadlines.html?action=click\&pgtype=Article\&state=default\&region=TOP_BANNER\&context=storylines_menu}{Voting
  by Mail}
\item
  \href{https://www.nytimes3xbfgragh.onion/interactive/2019/us/elections/2020-presidential-election-calendar.html?action=click\&pgtype=Article\&state=default\&region=TOP_BANNER\&context=storylines_menu}{Key
  Dates}
\item
  \href{https://www.nytimes3xbfgragh.onion/newsletters/politics?action=click\&pgtype=Article\&state=default\&region=TOP_BANNER\&context=storylines_menu}{Politics
  Newsletter}
\end{itemize}

Advertisement

\protect\hyperlink{after-top}{Continue reading the main story}

Supported by

\protect\hyperlink{after-sponsor}{Continue reading the main story}

news analysis

\hypertarget{does-trump-want-to-save-his-economy}{%
\section{Does Trump Want to Save His
Economy?}\label{does-trump-want-to-save-his-economy}}

The president is showing little urgency or strategy as the economic
recovery stalls ahead of the November election.

\includegraphics{https://static01.graylady3jvrrxbe.onion/images/2020/07/31/us/politics/31dc--virus-econassess/merlin_175135461_bf96bd8c-cccb-4e5a-9833-3cf703af48c0-articleLarge.jpg?quality=75\&auto=webp\&disable=upscale}

\href{https://www.nytimes3xbfgragh.onion/by/jim-tankersley}{\includegraphics{https://static01.graylady3jvrrxbe.onion/images/2018/10/19/multimedia/author-jim-tankersley/author-jim-tankersley-thumbLarge.png}}

By \href{https://www.nytimes3xbfgragh.onion/by/jim-tankersley}{Jim
Tankersley}

\begin{itemize}
\item
  July 31, 2020
\item
  \begin{itemize}
  \item
  \item
  \item
  \item
  \item
  \item
  \end{itemize}
\end{itemize}

WASHINGTON --- The United States
\href{https://www.nytimes3xbfgragh.onion/2020/07/30/business/economy/q2-gdp-coronavirus-economy.html}{just
suffered its worst economic quarter} in nearly 75 years. Its recovery
from the depths of a pandemic-induced recession has stalled, as
coronavirus deaths rise again across the country. President Trump has
what appears to be one final chance to cut a deal with Congress to
ensure hard-hit workers and businesses do not collapse before the
November election.

He has shown little interest in taking it.

Rather than push for a comprehensive plan that could win support from
both Democrats and Republicans, Mr. Trump first embraced big-ticket
items that Senate Republicans did not want and that would do little to
help millions of struggling workers and businesses. That included a
payroll tax cut and an expanded tax break for business lunches, along
with \$1.75 billion to rebuild the F.B.I.'s headquarters in Washington.

He has since
\href{https://www.nytimes3xbfgragh.onion/2020/07/29/business/economy/virus-aid-trump.html?referringSource=articleShare}{derided
efforts} to find middle ground with Democratic leaders on a broad
economic rescue package, declaring on Wednesday that ``we really don't
care'' about several possible parts of it.

Mr. Trump and his aides waited until the 11th hour to engage Democrats
over expiring unemployment benefits that have been a lifeline to
millions of workers, and Democratic leaders
\href{https://www.nytimes3xbfgragh.onion/2020/07/30/us/politics/senate-virus-aid.html}{have
dismissed} his last-minute proposal to temporarily extend them. Over the
past week, the president has publicly called for stimulus measures that
were not included in the \$1 trillion proposal that his administration
and Senate Republicans
\href{https://www.nytimes3xbfgragh.onion/2020/07/27/us/politics/republicans-jobless-aid.html}{unveiled
on Monday}, like continuing a national moratorium on evictions.

On Friday, he used a series of Twitter posts to slam Democrats for not
agreeing to a temporary extension of the \$600-per-week unemployment
supplement, a plan that he and his aides opposed until very recently.
``The Do Nothing Democrats are more interested in playing politics than
in helping our deserving people,'' he wrote.

The president's approach to the negotiations over another round of
federal stimulus for the ailing economy has confounded economists,
lobbyists and lawmakers, who say they are baffled by Mr. Trump's
apparent lack of a plan to nail down another rescue package that he can
sign into law.

The economic rebound Mr. Trump still boasts about has stalled as virus
cases reach records and states reimpose lockdowns. Labor Department data
released on Thursday showed that new unemployment claims had risen for
the second consecutive week, and that the number of continuing claims
was also increasing. On Wednesday, the chair of the Federal Reserve,
Jerome H. Powell, warned the nascent recovery
\href{https://www.nytimes3xbfgragh.onion/2020/07/29/business/economy/federal-reserve-meeting-interest-rates.html}{that
was underway} in May and June was sputtering.

A wide range of polls show Mr. Trump's lead over the presumptive
Democratic presidential nominee, Joseph R. Biden Jr., on economic issues
has evaporated in recent months. Congress is set to leave for recess in
less than two weeks, with little prospect of major legislation remaining
from then through the election.

Strategists say there is a potential compromise to be had between
Democrats and the White House, most likely in the range of a \$2
trillion package, before Congress decamps for its summer recess. They do
not understand why Mr. Trump would decline to seize a deal and risk
watching Americans lose their homes and businesses as November
approaches.

The president's strategy to help the economy ``is hard to decipher,''
said Michael R. Strain, an economist at the conservative American
Enterprise Institute who has urged Congress to provide more aid to
people, businesses and hard-hit state and local governments. ``It seems
to me there isn't a clear strategy to support the economy right now
coming from the White House.''

Mr. Trump, he added, ``is just misreading how bad the economy is, and
how hard of a shape workers and families are in right now.''

Some members of Mr. Trump's inner circle, along with his allies in the
Senate, have urged the president to oppose a large new spending bill,
including some of the provisions included in the Senate Republican
proposal unveiled on Monday. The economists Arthur B. Laffer and Stephen
Moore, who informally advise Mr. Trump, have told him to focus on a
payroll tax cut for workers and businesses --- a move that few
Republicans support and that economists say would do little to help the
30 million Americans without a job.

Democrats in Congress say they are surprised to find themselves in the
position of pushing Mr. Trump and his party to agree to a stimulus
spending plan, given how central the economy has been to his presidency
and the dire straits it is facing. Democrats say the package of
proposals they are insisting upon --- including additional money to test
for and prevent the spread of the virus, extensions of supplemental
unemployment benefits and more aid to small business, states and local
governments --- could actually help Mr. Trump win a second term, by
lifting the economy and helping control the pandemic.

But the president has remained steadfast in his belief that the virus
will soon abate and continues to portray the economic slowdown as a
blip. During remarks at the White House on Thursday, he said that a
comeback ``won't take very long, based on everything that we're
seeing.''

Asked this week in North Carolina if he was worried about the state of
the recovery, Mr. Trump replied: ``I don't think so. I think the
recovery has been very strong. We've set record job numbers.''

Senator Ron Wyden of Oregon, the top Democrat on the Finance Committee,
said the president's rosy predictions were not going to help the
millions of Americans without jobs.

``Donald Trump thought he could just happy-talk his way through the
things people are talking about in their kitchens and their living
rooms, which is how to beat the virus and how to fix the economy,'' he
said.

When the Commerce Department reported the economy contracted in the
second quarter at its fastest rate since the immediate aftermath of
World War II, Mr. Trump's re-election campaign responded with a news
release promoting job gains in May and June and attacking Mr. Biden over
his economic record as vice president.

Outside economists are far less bullish than the president on the
economy's trajectory. Ian Shepherdson of Pantheon Macroeconomics wrote
in a research note on Thursday that ``we would not be at all surprised''
to see little job growth or a dip in the unemployment rate for August or
September, given current trends in the economy. Forecasters at TD
Securities have recently cut their projections for growth in the third
quarter and warned that growth for the end of the year will ``depend
significantly on Covid developments.''

Mr. Powell, the Fed chair, said on Wednesday that the stimulus spending
approved so far had been crucial to keeping ``people in their homes and
businesses in business'' but warned more support would be needed given
``a large number of people'' would struggle to regain employment in the
months to come.

The delays in negotiations have already pushed the country over a
benefit ``cliff'' --- the expiration of the additional \$600 a week that
unemployed workers were receiving under the economic rescue package Mr.
Trump
\href{https://www.nytimes3xbfgragh.onion/2020/03/25/us/politics/coronavirus-senate-deal.html}{signed
in March} expire on Friday. Many economists warn that the sudden
reduction in buying power for millions of Americans will only worsen the
current slowdown.

The expanded unemployment benefit is ``an incredibly important part of
our household consumption base in the United States right now,'' said
Kathryn Anne Edwards, an economist at the RAND Corporation. ``It's
keeping up consumption, it's keeping people in their homes.''

Administration officials tried on Thursday to sell Democrats on a
short-term extension of those benefits, while negotiations continue on a
larger deal. Democratic leaders said they wanted a longer-term
agreement.

Senate Republicans have begun to minimize Mr. Trump's role in hastening
the recovery with more federal spending.

``A strategy for the economy?'' asked Senator Todd Young, Republican of
Indiana. ``That's not how economies work. He's not the Wizard of Oz ---
you can print this, I hope you do --- he's not the Wizard of Oz, who
controls the economy. Growth is created by innovators and entrepreneurs
and rank and file workers, based on supply and demand.''

``I think he's doing what he can,'' Mr. Young added, ``so that the
economy can open up again.''

Emily Cochrane contributed reporting.

\hypertarget{our-2020-election-guide}{%
\section{Our 2020 Election Guide}\label{our-2020-election-guide}}

Updated ~Sept. 8, 2020

\begin{center}\rule{0.5\linewidth}{\linethickness}\end{center}

\begin{itemize}
\item ~
  \hypertarget{the-latest}{%
  \subsection{The Latest}\label{the-latest}}

  \begin{itemize}
  \item
    President Trump and his party are using a playbook that aims to
    alarm people about crime in their backyards. It didn't work in 2018,
    but
    \href{https://www.nytimes3xbfgragh.onion/2020/09/08/us/politics/trump-republicans-fear-strategy.html?action=click\&pgtype=Article\&state=default\&region=BELOW_MAIN_CONTENT\&context=storylines_guide}{both
    parties think it could resonate more this year}.
  \end{itemize}
\item ~
  \hypertarget{how-to-win-270}{%
  \subsection{How to Win 270}\label{how-to-win-270}}

  \begin{itemize}
  \item
    Joe Biden and Donald Trump need 270 electoral votes to reach the
    White House. Try building
    \href{https://www.nytimes3xbfgragh.onion/interactive/2020/us/elections/election-states-biden-trump.html?action=click\&pgtype=Article\&state=default\&region=BELOW_MAIN_CONTENT\&context=storylines_guide}{your
    own coalition of battleground states}~to see potential outcomes.
  \end{itemize}
\item ~
  \hypertarget{voting-by-mail}{%
  \subsection{Voting by Mail}\label{voting-by-mail}}

  \begin{itemize}
  \item
    Will you have enough time to vote by mail in your state? Yes, but
    it's risky to procrastinate.
    \href{https://www.nytimes3xbfgragh.onion/interactive/2020/08/31/us/politics/vote-by-mail-deadlines.html?action=click\&pgtype=Article\&state=default\&region=BELOW_MAIN_CONTENT\&context=storylines_guide}{Check
    your state's deadline.}
  \item
    \href{https://www.nytimes3xbfgragh.onion/interactive/2020/us/elections/joe-biden.html?action=click\&pgtype=Article\&state=default\&region=BELOW_MAIN_CONTENT\&context=storylines_guide}{}

    \hypertarget{joe-biden}{%
    \section{Joe Biden}\label{joe-biden}}

    \hypertarget{democrat}{%
    \subsection{Democrat}\label{democrat}}

    \href{https://www.nytimes3xbfgragh.onion/interactive/2020/us/elections/donald-trump.html?action=click\&pgtype=Article\&state=default\&region=BELOW_MAIN_CONTENT\&context=storylines_guide}{}

    \hypertarget{donald-trump}{%
    \section{Donald Trump}\label{donald-trump}}

    \hypertarget{republican}{%
    \subsection{Republican}\label{republican}}
  \end{itemize}
\item
  \hypertarget{keep-up-with-our-coverage}{%
  \subsection{Keep Up With Our
  Coverage}\label{keep-up-with-our-coverage}}

  \begin{itemize}
  \item
    Get an
    \href{https://www.nytimes3xbfgragh.onion/newsletters/politics?action=click\&pgtype=Article\&state=default\&region=BELOW_MAIN_CONTENT\&context=storylines_guide}{email}~recapping
    the day's news
  \item
    Download our mobile app on
    \href{https://apps.apple.com/us/app/nytimes/id284862083?ls=1\&mat_click_id=5c79ae7455014fd1bd66b5610c05b8f2-20191112-16948\&referrer=mat_click_id\%3D5c79ae7455014fd1bd66b5610c05b8f2-20191112-16948\%26link_click_id\%3D722930677036718082}{iOS}~and
    \href{http://a.localytics.com/android?id=com.nytimes.android\&referrer=utm_source\%3Dother_nyt_mobile_web\%26utm_medium\%3DWeb\%2520page\%26utm_term\%3DGeneral\%2520Mobile\%2520Page\%26utm_campaign\%3DNYT\%2520Mobile\%2520General\%2520Page}{Android}~and
    turn on Breaking News and Politics alerts
  \end{itemize}
\end{itemize}

Advertisement

\protect\hyperlink{after-bottom}{Continue reading the main story}

\hypertarget{site-index}{%
\subsection{Site Index}\label{site-index}}

\hypertarget{site-information-navigation}{%
\subsection{Site Information
Navigation}\label{site-information-navigation}}

\begin{itemize}
\tightlist
\item
  \href{https://help.nytimes3xbfgragh.onion/hc/en-us/articles/115014792127-Copyright-notice}{©~2020~The
  New York Times Company}
\end{itemize}

\begin{itemize}
\tightlist
\item
  \href{https://www.nytco.com/}{NYTCo}
\item
  \href{https://help.nytimes3xbfgragh.onion/hc/en-us/articles/115015385887-Contact-Us}{Contact
  Us}
\item
  \href{https://www.nytco.com/careers/}{Work with us}
\item
  \href{https://nytmediakit.com/}{Advertise}
\item
  \href{http://www.tbrandstudio.com/}{T Brand Studio}
\item
  \href{https://www.nytimes3xbfgragh.onion/privacy/cookie-policy\#how-do-i-manage-trackers}{Your
  Ad Choices}
\item
  \href{https://www.nytimes3xbfgragh.onion/privacy}{Privacy}
\item
  \href{https://help.nytimes3xbfgragh.onion/hc/en-us/articles/115014893428-Terms-of-service}{Terms
  of Service}
\item
  \href{https://help.nytimes3xbfgragh.onion/hc/en-us/articles/115014893968-Terms-of-sale}{Terms
  of Sale}
\item
  \href{https://spiderbites.nytimes3xbfgragh.onion}{Site Map}
\item
  \href{https://help.nytimes3xbfgragh.onion/hc/en-us}{Help}
\item
  \href{https://www.nytimes3xbfgragh.onion/subscription?campaignId=37WXW}{Subscriptions}
\end{itemize}
