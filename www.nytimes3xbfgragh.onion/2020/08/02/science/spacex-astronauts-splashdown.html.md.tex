Sections

SEARCH

\protect\hyperlink{site-content}{Skip to
content}\protect\hyperlink{site-index}{Skip to site index}

\href{https://www.nytimes3xbfgragh.onion/section/science}{Science}

\href{https://myaccount.nytimes3xbfgragh.onion/auth/login?response_type=cookie\&client_id=vi}{}

\href{https://www.nytimes3xbfgragh.onion/section/todayspaper}{Today's
Paper}

\href{/section/science}{Science}\textbar{}`Thanks for Flying SpaceX':
NASA Astronauts Safely Splash Down After Journey From Orbit

\url{https://nyti.ms/3k6gO6E}

\begin{itemize}
\item
\item
\item
\item
\item
\end{itemize}

\hypertarget{spacexs-astronaut-trip}{%
\subsubsection{\texorpdfstring{\href{https://www.nytimes3xbfgragh.onion/2020/08/02/science/spacex-astronauts-splashdown.html?name=styln-nasa-spacex\&region=TOP_BANNER\&block=storyline_menu_recirc\&action=click\&pgtype=Article\&impression_id=0ae11df0-f2d6-11ea-a5b3-8155027c1ab2\&variant=undefined}{SpaceX's
Astronaut
Trip}}{SpaceX's Astronaut Trip}}\label{spacexs-astronaut-trip}}

\begin{itemize}
\tightlist
\item
  \href{https://www.nytimes3xbfgragh.onion/2020/08/02/science/spacex-astronauts-splashdown.html?name=styln-nasa-spacex\&region=TOP_BANNER\&block=storyline_menu_recirc\&action=click\&pgtype=Article\&impression_id=0ae11df0-f2d6-11ea-a5b3-8155027c1ab2\&variant=undefined}{`Thanks
  for Flying SpaceX'}
\item
  \href{https://www.nytimes3xbfgragh.onion/2020/05/26/science/spacex-launch-nasa.html?name=styln-nasa-spacex\&region=TOP_BANNER\&block=storyline_menu_recirc\&action=click\&pgtype=Article\&impression_id=0ae11df1-f2d6-11ea-a5b3-8155027c1ab2\&variant=undefined}{Why
  NASA Picked SpaceX}
\item
  \href{https://www.nytimes3xbfgragh.onion/interactive/2020/05/26/science/spacex-nasa.html?name=styln-nasa-spacex\&region=TOP_BANNER\&block=storyline_menu_recirc\&action=click\&pgtype=Article\&impression_id=0ae14500-f2d6-11ea-a5b3-8155027c1ab2\&variant=undefined}{Inside
  the Capsule}
\item
  \href{https://www.nytimes3xbfgragh.onion/2020/05/27/science/bob-behnken-doug-hurley.html?name=styln-nasa-spacex\&region=TOP_BANNER\&block=storyline_menu_recirc\&action=click\&pgtype=Article\&impression_id=0ae14501-f2d6-11ea-a5b3-8155027c1ab2\&variant=undefined}{Meet
  the Astronauts}
\end{itemize}

Advertisement

\protect\hyperlink{after-top}{Continue reading the main story}

Supported by

\protect\hyperlink{after-sponsor}{Continue reading the main story}

\hypertarget{thanks-for-flying-spacex-nasa-astronauts-safely-splash-down-after-journey-from-orbit}{%
\section{`Thanks for Flying SpaceX': NASA Astronauts Safely Splash Down
After Journey From
Orbit}\label{thanks-for-flying-spacex-nasa-astronauts-safely-splash-down-after-journey-from-orbit}}

Bob Behnken and Doug Hurley returned to Earth in the first water landing
by an American space crew since 1975.

\includegraphics{https://static01.graylady3jvrrxbe.onion/images/2020/08/02/video/02vid-spacex-splash/02vid-spacex-splash-videoSixteenByNineJumbo1600.jpg}

\href{https://www.nytimes3xbfgragh.onion/by/kenneth-chang}{\includegraphics{https://static01.graylady3jvrrxbe.onion/images/2018/02/16/multimedia/author-kenneth-chang/author-kenneth-chang-thumbLarge.jpg}}

By \href{https://www.nytimes3xbfgragh.onion/by/kenneth-chang}{Kenneth
Chang}

\begin{itemize}
\item
  Aug. 2, 2020
\item
  \begin{itemize}
  \item
  \item
  \item
  \item
  \item
  \end{itemize}
\end{itemize}

The first astronaut trip to orbit by a private company parachuted to a
safe conclusion in the Gulf of Mexico on Sunday.

It was the first water landing by NASA astronauts since 1975, when the
agency's crews were still flying to and from orbit in the Apollo modules
used for the historic American moon missions.

Riding in a capsule built and operated by SpaceX, the rocket company
founded by Elon Musk, two NASA astronauts ---
\href{https://www.nytimes3xbfgragh.onion/2020/05/27/science/bob-behnken-doug-hurley.html}{Robert
L. Behnken and Douglas G. Hurley} --- splashed down near Pensacola,
Fla., on Sunday afternoon.

The Crew Dragon capsule, suspended under four giant billowing
orange-and-white parachutes, settled upright into the water at a gentle
pace of 15 miles per hour at 2:48 p.m. Eastern time.

``On behalf of the SpaceX and NASA teams, welcome back to planet
Earth,'' Michael Heiman, the SpaceX engineer communicating with the
astronauts, said after splashdown. ``And thanks for flying SpaceX.''

More than an hour later, after Mr. Behnken and Mr. Hurley were helped
out of the spacecraft, Mr. Hurley thanked the employees of NASA and
SpaceX who helped make the mission a success.

``You should take a moment to just cherish this day, especially given
all the things that have happened this year,'' he said.

Although NASA was the customer this time, the mission could be a first
step to more people going to space for a variety of new activities, like
sightseeing, corporate research and satellite repair. A goal of the
space agency is to turn over to private enterprise some things it used
to do.

``We are entering a new era of human spaceflight, where NASA is no
longer the purchaser, owner and operator of all the hardware,'' Jim
Bridenstine, the NASA administrator, said during a news conference after
the splashdown. ``We are going to be a customer, one customer of many
customers in a very robust commercial marketplace for human spaceflight
to low Earth orbit.''

NASA has hired two companies --- SpaceX and Boeing --- to provide
transportation of astronauts to and from the International Space
Station, and SpaceX was the first to be ready to take astronauts to
orbit,
\href{https://www.nytimes3xbfgragh.onion/2020/05/30/science/spacex-nasa-astronauts.html}{launching
Mr. Behnken and Mr. Hurley in May}.

Gwynne Shotwell, the president and chief operating officer of SpaceX,
said the mission was ``incredibly smooth'' and a step to more ambitious
trips.

''This is really just the beginning,'' she said. ``We are starting the
journey of bringing people regularly to and from low Earth orbit and
onto the moon and then ultimately onto Mars.''

\includegraphics{https://static01.graylady3jvrrxbe.onion/images/2020/08/02/science/02sci-splashdown-inside/merlin_175234710_19b643b6-2c86-4344-9acf-e971b5f239c3-articleLarge.jpg?quality=75\&auto=webp\&disable=upscale}

After two months on the space station, Mr. Behnken and Mr. Hurley
reboarded the Crew Dragon and undocked from the space station on
Saturday evening. The spacecraft autonomously maneuvered away from the
space station and, while Mr. Behnken and Mr. Hurley were sleeping,
performed a six-minute burn of the thrusters to line up with the
splashdown zone.

Earlier concerns about
\href{https://www.nytimes3xbfgragh.onion/2020/08/02/us/Hurricane-Isaias-track.html}{the
Isaias storm system} working its way up the Florida Atlantic coast
prompted splashdown near Pensacola, the westernmost of seven possible
landing sites, where calm weather cooperated to enable a safe return.

On Sunday morning, the astronauts woke up to familiar voices.

``I'm happy you went into space, but I'm even happier that you're coming
back home,'' said Mr. Hurley's son, Jack.

``Wake up, wake up, wake up, wake up, Daddy, wake up!'' said Mr.
Behnken's son, Theo. ``Don't worry, you can sleep in tomorrow. Hurry
home so we can go get my dog!''

About an hour before splashdown, the spacecraft began a final series of
maneuvers. As it passed over the Indian Ocean, just to the west of
Australia, it jettisoned a bottom piece, known as the trunk, which was
no longer needed. That exposed the capsule's heat shield.

``Oh yeah, we felt it,'' Mr. Hurley said after the maneuver was
confirmed on the ground.

An 11-minute firing of the thrusters set the Crew Dragon on a trajectory
to fall out of orbit at 17,500 miles per hour. The rush of air heated
the bottom of the capsule to 3,500 degrees Fahrenheit and, as expected,
cut off communications with the spacecraft for six minutes.

``I'm almost speechless as to how well things went today with the
deorbit,'' said Steve Stich, manager of the commercial crew program at
NASA.

Image

Crew aboard a recovery ship worked to clear toxic fumes they detected
before the astronauts could exit the capsule.Credit...NASA/Via Reuters

As SpaceX crews raced to attend to the capsule and its crew in the
water, they also had to contend with
\href{https://www.nytimes3xbfgragh.onion/2020/08/02/us/flag-boat-SpaceX.html}{a
flotilla of small boats} piloted by private onlookers seeking a closer
view of the spacecraft. One of them flew a banner supporting President
Trump.

``That was not what we were anticipating,'' Mr. Bridenstine said. The
Coast Guard cleared out the area for the splashdown.

``After they landed, the boats just came in, and we need to do a better
job next time for sure,'' he said.

\includegraphics{https://static01.graylady3jvrrxbe.onion/images/2020/08/02/multimedia/02xp-boats-pix-sub/02xp-boats-pix-sub-videoSixteenByNine3000.jpg}

SpaceX crews on the boats told them to move farther away, seeking to
maintain the safety zone around the capsule because toxic propellant
fumes from the spacecraft thrusters can endanger passengers on vessels
nearby. Detection of residual fumes once the spacecraft was pulled from
the sea delayed the opening of the hatch for the astronauts to exit.

Mr. Behnken addressed the SpaceX team just before he left the Crew
Dragon: ``Thank you for doing the most difficult parts and the most
important parts of human spaceflight --- getting us into orbit and
bringing us home, safely.''

Once back on land, the astronauts were flown from Pensacola to Ellington
Field, a military base in Houston.

By the time they walked off the Gulfstream plane, Mr. Behnken and Mr.
Hurley looked as if they had already largely acclimated to gravity
again, walking with only slight wobbling to seats on the tarmac. They
again thanked the people at SpaceX and NASA who had worked to make the
mission a success.

Mr. Hurley said the journey was still ``a lot to process,'' then joked
that he and Mr. Behnken had been in the capsule ``making prank satellite
phone calls to whoever we could get a hold of.''

He added that the phone bill should be sent to Mr. Musk, who had flown
from California, where he had watched the splashdown from SpaceX
headquarters, to Houston to welcome the astronauts back.

``I really came here because I just wanted to see Bob and Doug, to be
totally frank,'' Mr. Musk said during his brief remarks.

``I'm not very religious, but I prayed for this one,'' Mr. Musk said.

NASA has been busy in the past week. On Thursday, it launched
\href{https://www.nytimes3xbfgragh.onion/2020/07/30/science/nasa-mars-launch.html}{Perseverance,
its next robotic rover, on a six-and-half month journey to Mars}. Mr.
Bridenstine took the opportunity of Mr. Behnken's and Mr. Hurley's
return on Sunday to promote the space agency's next major push: to send
astronauts back to the moon.

The House of Representatives, controlled by Democrats, has been
reluctant to provide the money that NASA says it needs to meet a goal
set by the Trump administration of a moon landing in 2024.

``What I'm asking for our members of Congress to do is look at what
we've done with what we have,'' Mr. Bridenstine said at Ellington after
Mr. Behnken and Mr. Hurley had spoken. ``And if you fund us at our
budget request level, we will be on the moon.''

After the splashdown on Sunday,
\href{https://twitter.com/realDonaldTrump/status/1289997897723863040}{Mr.
Trump tweeted}, ``Great to have NASA Astronauts return to Earth after
very successful two month mission. Thank you to all!''

\href{https://www.nytimes3xbfgragh.onion/interactive/2020/05/26/science/spacex-nasa.html}{}

\includegraphics{https://static01.graylady3jvrrxbe.onion/images/2020/05/26/us/spacex-nasa-promo-1590499638707/spacex-nasa-promo-1590499638707-articleLarge-v2.jpg}

\hypertarget{now-boarding-spacexs-new-ride-to-orbit-for-nasa-astronauts}{%
\subsection{Now Boarding: SpaceX's New Ride to Orbit for NASA
Astronauts}\label{now-boarding-spacexs-new-ride-to-orbit-for-nasa-astronauts}}

The Crew Dragon launched successfully on Saturday.

Mr. Behnken and Mr. Hurley ended up with a longer and busier stay at the
space station than the two weeks originally planned. Because of repeated
delays by SpaceX and Boeing, NASA ended up short-handed, with only one
astronaut, Christopher J. Cassidy, aboard the space station when the
Crew Dragon and its two passengers docked.

They stayed two months. Mr. Behnken and Mr. Cassidy performed four
spacewalks to complete the installation of new batteries on the space
station. Mr. Hurley helped by operating the station's robotic arm. The
men also contributed to science experiments in low-Earth orbit.

Mr. Cassidy will remain aboard the station with two Russian astronauts,
Anatoly Ivanishin and Ivan Vagner. All three are to stay
\href{https://www.nasa.gov/sites/default/files/atoms/files/exp-63-summary.pdf}{on
board through October}, when another crew of one American and two
Russian astronauts
\href{https://www.nasa.gov/press-release/nasa-astronaut-kate-rubins-crewmates-to-discuss-upcoming-spaceflight}{will
replace them}.

Once the mission is formally certified as a success, the next flight of
the Crew Dragon will launch no earlier than late September. It will take
three NASA astronauts --- Michael S. Hopkins, Victor J. Glover and
Shannon Walker --- and one Japanese astronaut, Soichi Noguchi, to the
space station.

The second operational flight, tentatively scheduled for February 2021,
will use the same capsule that just returned with Mr. Behnken and Mr.
Hurley. It will carry two NASA astronauts, Robert S. Kimbrough and K.
Megan McArthur; Akihiko Hoshide of Japan; and Thomas Pesquet of the
European Space Agency.

Ms. McArthur is married to Mr. Behnken.

SpaceX's counterpart in the commercial crew program, Boeing, will almost
certainly not be able to launch astronauts until next year. An uncrewed
flight last year
\href{https://www.nytimes3xbfgragh.onion/2020/07/07/science/boeing-starliner-nasa.html}{suffered
significant software errors, which could have led to a loss of the
spacecraft} during its orbital test. Boeing will now repeat the uncrewed
test later this year before putting astronauts aboard.

Advertisement

\protect\hyperlink{after-bottom}{Continue reading the main story}

\hypertarget{site-index}{%
\subsection{Site Index}\label{site-index}}

\hypertarget{site-information-navigation}{%
\subsection{Site Information
Navigation}\label{site-information-navigation}}

\begin{itemize}
\tightlist
\item
  \href{https://help.nytimes3xbfgragh.onion/hc/en-us/articles/115014792127-Copyright-notice}{©~2020~The
  New York Times Company}
\end{itemize}

\begin{itemize}
\tightlist
\item
  \href{https://www.nytco.com/}{NYTCo}
\item
  \href{https://help.nytimes3xbfgragh.onion/hc/en-us/articles/115015385887-Contact-Us}{Contact
  Us}
\item
  \href{https://www.nytco.com/careers/}{Work with us}
\item
  \href{https://nytmediakit.com/}{Advertise}
\item
  \href{http://www.tbrandstudio.com/}{T Brand Studio}
\item
  \href{https://www.nytimes3xbfgragh.onion/privacy/cookie-policy\#how-do-i-manage-trackers}{Your
  Ad Choices}
\item
  \href{https://www.nytimes3xbfgragh.onion/privacy}{Privacy}
\item
  \href{https://help.nytimes3xbfgragh.onion/hc/en-us/articles/115014893428-Terms-of-service}{Terms
  of Service}
\item
  \href{https://help.nytimes3xbfgragh.onion/hc/en-us/articles/115014893968-Terms-of-sale}{Terms
  of Sale}
\item
  \href{https://spiderbites.nytimes3xbfgragh.onion}{Site Map}
\item
  \href{https://help.nytimes3xbfgragh.onion/hc/en-us}{Help}
\item
  \href{https://www.nytimes3xbfgragh.onion/subscription?campaignId=37WXW}{Subscriptions}
\end{itemize}
