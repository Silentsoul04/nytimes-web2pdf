Sections

SEARCH

\protect\hyperlink{site-content}{Skip to
content}\protect\hyperlink{site-index}{Skip to site index}

\href{https://www.nytimes3xbfgragh.onion/section/well/family}{Family}

\href{https://myaccount.nytimes3xbfgragh.onion/auth/login?response_type=cookie\&client_id=vi}{}

\href{https://www.nytimes3xbfgragh.onion/section/todayspaper}{Today's
Paper}

\href{/section/well/family}{Family}\textbar{}The Benefits of Talking to
Strangers

\url{https://nyti.ms/3i2Stgs}

\begin{itemize}
\item
\item
\item
\item
\item
\item
\end{itemize}

\href{https://www.nytimes3xbfgragh.onion/spotlight/at-home?action=click\&pgtype=Article\&state=default\&region=TOP_BANNER\&context=at_home_menu}{At
Home}

\begin{itemize}
\tightlist
\item
  \href{https://www.nytimes3xbfgragh.onion/2020/09/07/travel/route-66.html?action=click\&pgtype=Article\&state=default\&region=TOP_BANNER\&context=at_home_menu}{Cruise
  Along: Route 66}
\item
  \href{https://www.nytimes3xbfgragh.onion/2020/09/04/dining/sheet-pan-chicken.html?action=click\&pgtype=Article\&state=default\&region=TOP_BANNER\&context=at_home_menu}{Roast:
  Chicken With Plums}
\item
  \href{https://www.nytimes3xbfgragh.onion/2020/09/04/arts/television/dark-shadows-stream.html?action=click\&pgtype=Article\&state=default\&region=TOP_BANNER\&context=at_home_menu}{Watch:
  Dark Shadows}
\item
  \href{https://www.nytimes3xbfgragh.onion/interactive/2020/at-home/even-more-reporters-editors-diaries-lists-recommendations.html?action=click\&pgtype=Article\&state=default\&region=TOP_BANNER\&context=at_home_menu}{Explore:
  Reporters' Google Docs}
\end{itemize}

Advertisement

\protect\hyperlink{after-top}{Continue reading the main story}

Supported by

\protect\hyperlink{after-sponsor}{Continue reading the main story}

Personal Health

\hypertarget{the-benefits-of-talking-to-strangers}{%
\section{The Benefits of Talking to
Strangers}\label{the-benefits-of-talking-to-strangers}}

Casual connections with people we encounter in the course of daily life
can give us the sense of belonging to a community.

\includegraphics{https://static01.graylady3jvrrxbe.onion/images/2020/08/04/science/03BRODY-STRANGERS-illo/03BRODY-STRANGERS-illo-articleLarge.jpg?quality=75\&auto=webp\&disable=upscale}

\href{https://www.nytimes3xbfgragh.onion/by/jane-e-brody}{\includegraphics{https://static01.graylady3jvrrxbe.onion/images/2018/06/12/multimedia/jane-e-brody/jane-e-brody-thumbLarge.png}}

By \href{https://www.nytimes3xbfgragh.onion/by/jane-e-brody}{Jane E.
Brody}

\begin{itemize}
\item
  Aug. 3, 2020
\item
  \begin{itemize}
  \item
  \item
  \item
  \item
  \item
  \item
  \end{itemize}
\end{itemize}

I'm a lifelong extrovert who readily establishes and relishes casual
contacts with people I encounter during daily life: while walking my
dog, shopping for groceries, working out at the Y, even sweeping my
sidewalk. These ephemeral connections add variety to my life, are a
source of useful information and often provide needed emotional and
physical support. Equally important, they nearly always leave me with a
smile on my face (although now hidden under a mask!).

In recent months, under stay-at-home orders because of the coronavirus
pandemic, many people lost such daily encounters. I, on the other hand,
have done my best to maintain as many of them as possible while striving
to remain safe. With in-person time with family and close friends now
limited by a mutual desire to avoid exposure to Covid-19, the brief
socially distant contacts with people in my neighborhood, both those
I've known casually for years and others I just met, have been crucial
to my emotional and practical well-being and maybe even my health.

The benefits I associate with my casual connections were reinforced
recently by a fortuitous find. During a Covid-inspired cleanup I
stumbled upon a book in my library called ``Consequential Strangers: The
Power of People Who Don't Seem to Matter \ldots{} \emph{But Really
Do.''} Published 11 years ago, this enlightening tome was written by
Melinda Blau, a science writer, and Karen L. Fingerman, currently a
professor of psychology at the University of Texas, Austin, who studies
the nature and effects of so-called weak ties that people have with
others in their lives: the barista who fetches their coffee, the person
who cuts their hair, the proprietor of the local market, the folks they
see often at the gym or train station.

In an interview, Dr. Fingerman noted that casual connections with people
encountered in the course of daily life can give people a feeling that
they belong to a community, which she described as ``a basic human
need.''

As she and Ms. Blau wrote in their book, consequential strangers ``are
as vital to our well-being, growth, and day-to-day existence as family
and close friends. Consequential strangers anchor us in the world and
give us a sense of being plugged into something larger. They also
enhance and enrich our lives and offer us opportunities for novel
experiences and information that is beyond the purview of our inner
circles. They are vital social connections --- people who help you get
through the day and make life more interesting.''

My tendency to ``chat up'' total strangers I meet in the course of just
living has resulted in a slew of acquaintances who have filled my days
with pleasantries, advice, information, needed assistance and, most
important of all during this time of enforced semi-isolation, a valuable
sense of connections to people who share my environment.

Covid-19 lockdowns have reminded so many of us of how important our
relationships are to our quality of life --- not only relationships with
the friends and family members we love and know well and who know us
well, but also with more casual ones that help us maintain a positive
outlook during dark and distressing times.

Dr. Fingerman's research has also shown that people who are more
socially integrated are also more active physically. ``Being sedentary
kills you,'' she said. ``You have to get up and move to be with the
people you run into when exercising.'' Consequential strangers also help
your brain, she said, because ``conversations are more stimulating than
with people you know well.''

A fellow researcher in the field, Katherine L. Fiori, chairwoman of
undergraduate psychology at Adelphi University who studies social
networks of older adults, has found that activities that foster ``weaker
ties'' than are formed with family and close friends foster greater life
satisfaction and better emotional and physical health.

``The greater the number of weaker ties, the stronger the association
with positive feelings and fewer depressed feelings,'' Dr. Fiori said in
an interview. ``It's clearly not the case that close ties are all that
older adults need.''

And not just older adults, all adults. Dr. Fingerman said research has
shown that, in general, ``people do better when they have a more diverse
group of people in their lives.'' But as Dr. Fiori observed,
``Unfortunately, Covid has severely curtailed our ability to maintain
weaker ties. It can take a lot more effort to do this online.''

When Covid-19 descended with a fury on New York City, many people I knew
who had second homes ``escaped'' the city in hopes of avoiding the
virus. I, on the other hand, chose to stay in my Brooklyn neighborhood
where everyday I encountered people I knew casually as well as others in
my extended network of friends and acquaintances I'd made at the Y, in
local stores and when walking and cycling in Prospect Park.

In my country house, especially during the dark cold days of early
spring, I would have been far more isolated. Yes, I could walk my dog
and ride my bike without having to wear a mask because I would have met
almost no one else on route. But I would also have been deprived of
conversations with the many ``consequential strangers'' I encountered
daily during my outdoor excursions in Brooklyn, including the 7 p.m.
``shout-out'' in support of our essential workers.

To counter the loneliness and maintain her many casual connections, one
of my Y buddies started a group email that not only filled in for the
daily conversations she was missing but also gave her an ongoing support
system when faced with an injury and struggling with doom-and-gloom
isolation.

In their book, Ms. Blau and Dr. Fingerman emphasize the importance of
creating and being in environments that foster relationships with
consequential strangers. Decades ago when The New York Times erected
cubicles for its writers and editors, it destroyed an environment that
was conducive to sharing information and fostering camaraderie,
prompting me to work from home most days and save the time and effort
needed to dress for work and commute. I suspect that when Covid
limitations are finally lifted, many more office workers will do the
same and sacrifice casual work-based relationships.

As the authors wrote, ``Where we live, work, shop, and mingle has
everything to do with the weak ties we cultivate, and therefore our
quality of life.'' As they described a central theme of their book,
``Casual acquaintances inspire us to venture beyond our comfort zones.''
And until we do, we'll never know what we might gain from relationships
with ``people who don't seem to matter.''

Advertisement

\protect\hyperlink{after-bottom}{Continue reading the main story}

\hypertarget{site-index}{%
\subsection{Site Index}\label{site-index}}

\hypertarget{site-information-navigation}{%
\subsection{Site Information
Navigation}\label{site-information-navigation}}

\begin{itemize}
\tightlist
\item
  \href{https://help.nytimes3xbfgragh.onion/hc/en-us/articles/115014792127-Copyright-notice}{©~2020~The
  New York Times Company}
\end{itemize}

\begin{itemize}
\tightlist
\item
  \href{https://www.nytco.com/}{NYTCo}
\item
  \href{https://help.nytimes3xbfgragh.onion/hc/en-us/articles/115015385887-Contact-Us}{Contact
  Us}
\item
  \href{https://www.nytco.com/careers/}{Work with us}
\item
  \href{https://nytmediakit.com/}{Advertise}
\item
  \href{http://www.tbrandstudio.com/}{T Brand Studio}
\item
  \href{https://www.nytimes3xbfgragh.onion/privacy/cookie-policy\#how-do-i-manage-trackers}{Your
  Ad Choices}
\item
  \href{https://www.nytimes3xbfgragh.onion/privacy}{Privacy}
\item
  \href{https://help.nytimes3xbfgragh.onion/hc/en-us/articles/115014893428-Terms-of-service}{Terms
  of Service}
\item
  \href{https://help.nytimes3xbfgragh.onion/hc/en-us/articles/115014893968-Terms-of-sale}{Terms
  of Sale}
\item
  \href{https://spiderbites.nytimes3xbfgragh.onion}{Site Map}
\item
  \href{https://help.nytimes3xbfgragh.onion/hc/en-us}{Help}
\item
  \href{https://www.nytimes3xbfgragh.onion/subscription?campaignId=37WXW}{Subscriptions}
\end{itemize}
