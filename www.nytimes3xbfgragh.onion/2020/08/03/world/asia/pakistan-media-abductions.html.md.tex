Sections

SEARCH

\protect\hyperlink{site-content}{Skip to
content}\protect\hyperlink{site-index}{Skip to site index}

\href{https://www.nytimes3xbfgragh.onion/section/world/asia}{Asia
Pacific}

\href{https://myaccount.nytimes3xbfgragh.onion/auth/login?response_type=cookie\&client_id=vi}{}

\href{https://www.nytimes3xbfgragh.onion/section/todayspaper}{Today's
Paper}

\href{/section/world/asia}{Asia Pacific}\textbar{}Abductions, Censorship
and Layoffs: Pakistani Critics Are Under Siege

\url{https://nyti.ms/2XlJG0V}

\begin{itemize}
\item
\item
\item
\item
\item
\end{itemize}

Advertisement

\protect\hyperlink{after-top}{Continue reading the main story}

Supported by

\protect\hyperlink{after-sponsor}{Continue reading the main story}

\hypertarget{abductions-censorship-and-layoffs-pakistani-critics-are-under-siege}{%
\section{Abductions, Censorship and Layoffs: Pakistani Critics Are Under
Siege}\label{abductions-censorship-and-layoffs-pakistani-critics-are-under-siege}}

Recent abductions of a journalist and an activist have underscored
Pakistan's worsening rights conditions as the country's security forces
pressure the news media and human rights groups.

\includegraphics{https://static01.graylady3jvrrxbe.onion/images/2020/08/03/world/03pakistan-media-1/merlin_174798348_d0b91251-b688-4f8f-b671-a777613ea8fe-articleLarge.jpg?quality=75\&auto=webp\&disable=upscale}

\href{https://www.nytimes3xbfgragh.onion/by/maria-abi-habib}{\includegraphics{https://static01.graylady3jvrrxbe.onion/images/2018/10/08/multimedia/author-maria-abi-habib/author-maria-abi-habib-thumbLarge.png}}

By \href{https://www.nytimes3xbfgragh.onion/by/maria-abi-habib}{Maria
Abi-Habib}

\begin{itemize}
\item
  Published Aug. 3, 2020Updated Aug. 4, 2020
\item
  \begin{itemize}
  \item
  \item
  \item
  \item
  \item
  \end{itemize}
\end{itemize}

When Prime Minister Imran Khan boasted last year that Pakistan had one
of the ``freest presses in the world,''
\href{https://rsf.org/en/news/after-year-press-freedom-violations-rsf-writes-pakistans-premier}{journalists
were quick to object}, saying that intimidation of reporters across the
country was intensifying. It has only gotten worse since.

Two years into Mr. Khan's term, censorship is on the rise, journalists
and activists say, leaving the country's heavy-handed military and
security forces unchecked as they intimidate the news media to a degree
unseen since the
\href{https://tribune.com.pk/story/459782/when-musharraf-silenced-the-media}{country's
era of army juntas}.

The security forces frequently pressure editors to fire or muzzle
reporters, journalists say, while the government starves critical news
outlets of advertising funds and refuses to
\href{https://www.thenews.com.pk/print/659532-extraordinary-delay-in-payment-of-media-dues-apns-extremely-concerned-over-govt-inaction}{settle
previous bills worth millions of dollars}.

The \href{https://www.dawn.com/news/1570325}{abduction of a prominent
reporter}by state security officers in late July, coupled with the
\href{https://www.amnesty.org/en/get-involved/take-action/where-is-idris-khattak/}{disappearance
of a rights activist}in November, has heightened those concerns. In
June, Pakistan's Military Intelligence agency admitted that it had
detained the activist and that he is awaiting trial in a secret court on
undisclosed charges.

``Disappearances are a tool of terror, used not just to silence the
victim but to fill the wider community with fear,'' said Omar Waraich,
the head of South Asia for Amnesty International.

``In Pakistan, the military's intelligence apparatus has used
disappearances with impunity,'' Mr. Waraich said, adding: ``Civilian
politicians look on helplessly, affecting concern and promising to
investigate. Unable to uphold the rule of law as Imran Khan vowed to do,
their authority erodes.''

On July 21, the reporter, Matiullah Jan, had just dropped off his wife
at her job in an upscale neighborhood in Pakistan's capital Islamabad
when several men, some in plain clothes, others in counterterrorism
police uniforms, dragged him from his car, bundled him into one of their
vehicles and sped away.

Mr. Jan, 51, is a vocal critic of Mr. Khan's governing party, the
judiciary and the military, which critics accuse of working together to
preserve their power and stamp out dissent.

\includegraphics{https://static01.graylady3jvrrxbe.onion/images/2020/08/03/world/03pakistan-media-6/merlin_152695041_643e8084-8888-45f8-99ba-7b588a0706a7-articleLarge.jpg?quality=75\&auto=webp\&disable=upscale}

Footage from a security camera clearly shows the police's involvement in
the abduction, working alongside men in civilian clothes that many
believe are Pakistani intelligence officers. The footage culminated in a
pressure campaign on social media and Mr. Jan was released 12 hours
later. He released a vague statement saying he had been abducted by
forces that are ``against democracy.''

Multiple requests to the Pakistani government and military to comment
for this article went unanswered. Pakistan's security forces have not
publicly commented on Mr. Jan's abduction.

Under Pakistani law, state-directed abductions like Mr. Jan's are
lawful. The
\href{https://www.nytimes3xbfgragh.onion/2007/01/14/world/asia/14pakistan.html}{detentions
often go unexplained}, leaving the families of the victims wondering for
months or even years whether their loved one was killed in something as
commonplace as a hit-and-run accident or secretly detained by the
security forces.

While Pakistan has long had a poor track record on press freedom, it has
gotten notably worse under Mr. Khan's administration, which has been
widely seen as a high-water mark for military influence in the past
decade. Pakistan slipped six spots since 2017 --- the year before Mr.
Khan took office \textbf{---} to 145th place out of 180 countries in the
\href{https://rsf.org/en/pakistan}{2020 world press freedom index}
compiled by Reporters Without Borders.

In the last five years, 11 journalists have been
\href{https://cpj.org/asia/pakistan/}{killed in Pakistan}, seven of them
since Mr. Khan was sworn in as prime minister two years ago. Anchors
have frequently seen their newscasts cut off in the middle of
broadcasting --- a level of censorship not seen since the era of
military dictatorships in Pakistan.

Instead of establishing an outright dictatorship, human rights groups
say, Pakistan's generals are effectively imposing their will through
their allies in a government that they helped usher into office.

During the 2018
elections,\href{https://www.nytimes3xbfgragh.onion/2018/07/21/world/asia/pakistan-election-military.html}{the
military was accused of meddling} to ensure victory for Mr. Khan and his
Pakistan Tehreek-e-Insaf ****** party and to virtually dismantle the
party of former Prime Minister Nawaz Sharif, who had tried to curb the
military's powers. The military has denied those accusations.

Image

Prime Minister Imran Khan, center, during a military parade last year.
The military has been accused of tampering with the election to ensure
Mr. Khan's victory.~Credit...Akhtar Soomro/Reuters

As those elections drew near, the military accused reporters of being
anti-state, an allegation that was swiftly condemned by the
\href{https://cpj.org/2018/06/pakistan-army-spokesperson-accuses-journalists-of/}{Committee
to Protect Journalists.} After a series of articles detailing the
military's political and electoral interference, the security forces
disrupted the
\href{https://cpj.org/2018/05/pakistani-authorities-disrupt-distribution-of-dawn/}{distribution
of Dawn newspaper} across the country.

Over the past year, the country's remaining critical news outlets have
been gutted by the combination of a devastated national economy and the
sudden elimination of government advertising dollars. Media
organizations have laid off dozens of journalists, and the combination
of heavy pressure and job insecurity has led many reporters to avoid
critical or controversial subjects.

Like many Pakistani reporters, Mr. Jan claims that he lost his job as a
popular talk show host just months after the election because of his
hard-hitting reporting. He now runs his own YouTube channel.

``This is the first time in the 31 years of my career where I've seen a
structural takeover of the media industry,'' said Talat Hussain, a
former Geo TV news anchor who has been critical of the military and
government.

Mr. Hussain said his company fired him under pressure from the military
shortly after Mr. Khan's election. He has remained unemployed, with
newspapers and TV shows refusing to host his work.

``We have dealt with fairly tyrannical regimes that were elected and
dealt in repression, but it was episodic,'' Mr. Hussain said. ``This
time it is structural and complete and it's hard to breathe.''

Eventually, the authorities came after Mr. Hussain's former boss. In
March, Mir Shakil-ur-Rehman, the owner of the Jang Group, which owns Geo
TV and The News newspaper,
\href{https://www.nytimes3xbfgragh.onion/2020/03/12/world/asia/pakistan-journalist-jang.html}{was
detained over accusations of corruption}, which Mr. Rehman has denied.
Mr. Rehman has been held for over 100 days without charges, and several
bail hearings have been postponed.

Image

Mir Shakil-ur-Rehman, center, the owner of Jang media group, which
operates a TV station and newspapers critical of the government, has
been detained without charges for more than 100 days.~Credit...Reuters

When the rights activist Idris Khattak, 56, disappeared late last year,
there was no video footage to give his family the clarity that Mr. Jan's
family had.

In November, Mr. Khattak's 21-year-old daughter, Talia Khattak, left
Islamabad to go on a trip organized by her university. Her father told
her he would call her to check in multiple times a day.

During their last call, Ms. Khattak said her father sounded nervous and
there was a commotion in the background. He promised to call back in
``two or three days'' before hastily hanging up the phone, an unusual
gap of time for him.

``Those two or three days have turned into eight months,'' Ms. Khattak
said in an interview.

As the coronavirus rippled through Pakistan in the months after his
disappearance, the family's panic deepened --- Mr. Khattak's health
issues, including diabetes, have proved dangerous in those stricken by
the virus.

Mr. Khattak's disappearance was unusual. He retired about five years ago
from his advocacy work with groups including Human Rights Watch and
Amnesty International. Mr. Khattak's work focused on state-sponsored
abductions. But he had since lived a quiet and seemingly uncontroversial
life, his daughter said.

Finally, in June, a rare admission came from the Military Intelligence
agency: Mr. Khattak was in its custody and would be tried in a secret
military tribunal.

``When you take someone, when you abduct them, those people have
families behind them. You're ending all their lives,'' Ms. Khattak said.
``That they can just do this, with no repercussions, is
unconscionable.''

The authorities have not allowed Mr. Khattak's family members to speak
with him. In the eight months he has been gone, they have received no
word about his health or that he is getting his medication.

The abduction of her father has thrown Ms. Khattak into a murky
political game as she tries to challenge the most secretive and
repressive parts of the Pakistani state.

``Whenever there is a journalist or activist in Pakistan speaking up on
sensitive issues, they disappear like this,'' she said. ``Papa didn't do
anything illegal --- all Papa did was speak up.''

Image

The security forces frequently pressure editors to fire or muzzle
reporters, journalists say, while the government starves critical news
outlets of advertising funds.Credit...Rizwan Tabassum/Agence
France-Presse --- Getty Images

Advertisement

\protect\hyperlink{after-bottom}{Continue reading the main story}

\hypertarget{site-index}{%
\subsection{Site Index}\label{site-index}}

\hypertarget{site-information-navigation}{%
\subsection{Site Information
Navigation}\label{site-information-navigation}}

\begin{itemize}
\tightlist
\item
  \href{https://help.nytimes3xbfgragh.onion/hc/en-us/articles/115014792127-Copyright-notice}{©~2020~The
  New York Times Company}
\end{itemize}

\begin{itemize}
\tightlist
\item
  \href{https://www.nytco.com/}{NYTCo}
\item
  \href{https://help.nytimes3xbfgragh.onion/hc/en-us/articles/115015385887-Contact-Us}{Contact
  Us}
\item
  \href{https://www.nytco.com/careers/}{Work with us}
\item
  \href{https://nytmediakit.com/}{Advertise}
\item
  \href{http://www.tbrandstudio.com/}{T Brand Studio}
\item
  \href{https://www.nytimes3xbfgragh.onion/privacy/cookie-policy\#how-do-i-manage-trackers}{Your
  Ad Choices}
\item
  \href{https://www.nytimes3xbfgragh.onion/privacy}{Privacy}
\item
  \href{https://help.nytimes3xbfgragh.onion/hc/en-us/articles/115014893428-Terms-of-service}{Terms
  of Service}
\item
  \href{https://help.nytimes3xbfgragh.onion/hc/en-us/articles/115014893968-Terms-of-sale}{Terms
  of Sale}
\item
  \href{https://spiderbites.nytimes3xbfgragh.onion}{Site Map}
\item
  \href{https://help.nytimes3xbfgragh.onion/hc/en-us}{Help}
\item
  \href{https://www.nytimes3xbfgragh.onion/subscription?campaignId=37WXW}{Subscriptions}
\end{itemize}
