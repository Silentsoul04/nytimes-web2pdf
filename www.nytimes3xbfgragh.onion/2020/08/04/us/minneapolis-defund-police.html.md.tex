Sections

SEARCH

\protect\hyperlink{site-content}{Skip to
content}\protect\hyperlink{site-index}{Skip to site index}

\href{https://www.nytimes3xbfgragh.onion/section/us}{U.S.}

\href{https://myaccount.nytimes3xbfgragh.onion/auth/login?response_type=cookie\&client_id=vi}{}

\href{https://www.nytimes3xbfgragh.onion/section/todayspaper}{Today's
Paper}

\href{/section/us}{U.S.}\textbar{}Distrust of the Minneapolis Police,
and Also the Effort to Defund Them

\begin{itemize}
\item
\item
\item
\item
\item
\item
\end{itemize}

\hypertarget{race-and-america}{%
\subsubsection{\texorpdfstring{\href{https://www.nytimes3xbfgragh.onion/news-event/george-floyd-protests-minneapolis-new-york-los-angeles?name=styln-george-floyd\&region=TOP_BANNER\&block=storyline_menu_recirc\&action=click\&pgtype=Article\&impression_id=38371860-f2a2-11ea-985c-b51c0979f06f\&variant=undefined}{Race
and America}}{Race and America}}\label{race-and-america}}

\begin{itemize}
\tightlist
\item
  \href{https://www.nytimes3xbfgragh.onion/2020/09/04/nyregion/rochester-police-daniel-prude.html?name=styln-george-floyd\&region=TOP_BANNER\&block=storyline_menu_recirc\&action=click\&pgtype=Article\&impression_id=38373f70-f2a2-11ea-985c-b51c0979f06f\&variant=undefined}{What
  Happened in Rochester, N.Y.}
\item
  \href{https://www.nytimes3xbfgragh.onion/2020/09/01/us/politics/trump-fact-check-protests.html?name=styln-george-floyd\&region=TOP_BANNER\&block=storyline_menu_recirc\&action=click\&pgtype=Article\&impression_id=38373f71-f2a2-11ea-985c-b51c0979f06f\&variant=undefined}{Trump
  Fact Check}
\item
  \href{https://www.nytimes3xbfgragh.onion/2020/08/30/us/portland-shooting-explained.html?name=styln-george-floyd\&region=TOP_BANNER\&block=storyline_menu_recirc\&action=click\&pgtype=Article\&impression_id=38373f72-f2a2-11ea-985c-b51c0979f06f\&variant=undefined}{Portland
  Shooting}
\item
  \href{https://www.nytimes3xbfgragh.onion/2020/08/30/us/breonna-taylor-police-killing.html?name=styln-george-floyd\&region=TOP_BANNER\&block=storyline_menu_recirc\&action=click\&pgtype=Article\&impression_id=38373f73-f2a2-11ea-985c-b51c0979f06f\&variant=undefined}{Breonna
  Taylor's Life and Death}
\end{itemize}

Advertisement

\protect\hyperlink{after-top}{Continue reading the main story}

Supported by

\protect\hyperlink{after-sponsor}{Continue reading the main story}

\hypertarget{distrust-of-the-minneapolis-police-and-also-the-effort-to-defund-them}{%
\section{Distrust of the Minneapolis Police, and Also the Effort to
Defund
Them}\label{distrust-of-the-minneapolis-police-and-also-the-effort-to-defund-them}}

Residents on Minneapolis's North Side, which has a majority Black
population, have mixed opinions on the City Council's effort to
significantly reduce the police force.

\includegraphics{https://static01.graylady3jvrrxbe.onion/images/2020/08/05/us/00blackdefund-01/00blackdefund-01-articleLarge.jpg?quality=75\&auto=webp\&disable=upscale}

\href{https://www.nytimes3xbfgragh.onion/by/john-eligon}{\includegraphics{https://static01.graylady3jvrrxbe.onion/images/2018/06/12/multimedia/author-john-eligon/author-john-eligon-thumbLarge.png}}

By \href{https://www.nytimes3xbfgragh.onion/by/john-eligon}{John Eligon}

\begin{itemize}
\item
  Published Aug. 4, 2020Updated Aug. 10, 2020
\item
  \begin{itemize}
  \item
  \item
  \item
  \item
  \item
  \item
  \end{itemize}
\end{itemize}

MINNEAPOLIS --- The burgundy Oldsmobile sped through an intersection in
a tree-lined residential neighborhood on Minneapolis's North Side, and
Lisa Williams shook her head in disgust.

``Look at this,'' she said, surrounded by four of her young
grandchildren on the short stoop of her home. ``They ride as fast as
they can right down through here with no regard for the children.''

It is in such moments --- when she is reminded of the many dangers in
her community, from speeding cars to gunshots --- that Ms. Williams, 50,
would welcome the presence of the police.

But then she recalls the time several years ago when she and her husband
arrived home to find several police vehicles parked on their front lawn.
Officers told them to mind their own business when they asked what was
going on, leading to an argument that ended with her husband getting
handcuffed and taken to jail.

Minneapolis's North Side, with a majority Black population, has
decidedly mixed opinions on the City Council's effort, following the
police killing of George Floyd, to significantly reduce the size and
scope of Minneapolis's police force.

Residents complain of rampant police mistreatment, but also of
out-of-control crime and violence. That reality has left many Black
residents here unenthusiastic about what has become known as the defund
movement. Adding complexity to the debate, they say that they despise
the police but need someone to call when things go awry.

``It does seem like a no-win situation,'' Ms. Williams said.

Proponents of defunding argue that having considerably fewer --- or no
--- police officers could actually reduce crime because those resources
could instead be invested into communities struggling with poverty.

But that argument does not win over everybody.

In a
\href{https://runforsomething.net/wp-content/uploads/2020/07/PoliceReform_deck-d8.pdf}{survey
last month of likely voters in 10 battleground states}, just under half
of Black respondents said they would be more likely to support a
candidate who made defunding the police a priority, according to the
poll commissioned by Run for Something, which supports young,
progressive candidates, and Collective PAC, which backs Black
candidates.

Reducing police department budgets drew support from 70 percent of Black
Americans, according to
\href{https://news.gallup.com/poll/315962/americans-say-policing-needs-major-changes.aspx}{a
Gallup poll released in July}. Yet only 22 percent of Black respondents
supported the more drastic measure pushed by some activists of zeroing
out police department budgets altogether.

``What are they suggesting would be the answer if we didn't have
police?'' asked Bunny Beeks, whose
\href{https://www.mprnews.org/story/2017/12/19/random-bullet-killed-birdell-beeks-her-daughter-wouldnt-let-her-name-die}{mother
was fatally shot in North Minneapolis} four years ago. ``I just don't
understand what that would look like.''

The Minneapolis City Council's proposal would not totally eliminate the
Police Department. But some council members have said they would like to
replace the existing department, which has been widely criticized for
its aggressive attitudes toward Black residents.

Most North Side residents say they hope for major reforms, including
requiring officers to live in their community and better training them
to interact with residents.

Tiffany Roberson, whose brother, Jamar Clark,
\href{https://www.nytimes3xbfgragh.onion/2016/03/31/us/jamar-clark-shooting-minneapolis.html}{was
fatally shot by the police} five years ago, suggested creating a
community council that could work with and oversee the police in North
Minneapolis.

Though skeptics say that decades of reforms have failed to create
fundamental change, some residents said they had faith that Mr. Floyd's
death, and the outrage it has prompted, could make this time different.

Many residents say they have confidence in Chief Medaria Arradondo, the
first African-American to hold the position, saying he has shown an
appetite for change that past police leaders have not. But a
reform-minded chief cannot overhaul a department alone.

Speaking from a North Side street corner where young men sitting on lawn
chairs chat on sunny summer days, Royal Jones, 32, said he had had many
brushes with law enforcement. He compared his feelings about the police
to his relationship with his mother growing up. He said she might
``whoop'' him for doing something wrong, and he might get mad at her for
it, but at the end of the day, he still relied on her.

\includegraphics{https://static01.graylady3jvrrxbe.onion/images/2020/08/05/us/00blackdefund-02/00blackdefund-02-articleLarge.jpg?quality=75\&auto=webp\&disable=upscale}

Similarly, he said, if someone broke into his house, he would have to
rely on law enforcement to handle it rather than ``go the street way,''
which would just prompt more violence.

``Even a person like me might need the police,'' he said.

Still, Mr. Jones said he believed that a better approach might be to
employ community outreach workers to avert violence before it happens
and interact with police officers once it occurs. Such efforts already
exist, but Mr. Jones said they could be more robust.

Standing nearby, his friend Kentrell Grimes, a fellow North Sider, was
not necessarily buying that approach.

``At the end of the day, that is still policing,'' said Mr. Grimes, 25,
a cook. ``This is what I'm saying, though: How can you defund the police
and then bring another group to police? That's stupid. I'm sitting here
trying to wrap my brain around this.''

Minneapolis proponents of defunding the police have said that these are
the types of discussions that community members needed to have to decide
what works best for public safety in their neighborhoods.

Some may see the need for armed officers. Others may come up with a
different model. Kandace Montgomery, the director of Black Visions
Collective, a leading advocate of defunding in Minneapolis, acknowledged
the difficulty of getting people to envision a system of public safety
different from the only one they have always known.

``We do have to imagine,'' she said. ``I recognize that is deeply
scary.''

City Council members have worked closely with Black Visions Collective
and other Black-led organizations in an effort to defund the Police
Department. That has stirred tensions.

Many North Side civic leaders and legacy organizations, like the Urban
League and several Black churches, have accused elected officials of
ignoring the voices of their communities as they create a path forward
for policing. They point out that some of the defund movement's leaders
are based on the South Side --- where Mr. Floyd was killed by the police
--- which has a much smaller Black population.

``They've made this choice for us as Black people, when they don't
necessarily live or engage with Black people,'' said Raeisha Williams, a
community activist whose brother was fatally shot two years ago. ``When
my house is broken into, I want to be able to call the police. When my
security alarm goes off, I want to know they're going to arrive and
protect my family.''

The council has proposed amending the City Charter to eliminate the
Police Department as a core agency and replace it with a new public
safety department. That move alone would not eliminate the police, but
it would provide a blank canvas on which city leaders could create a new
mechanism for public safety that could include social services and
crime-prevention initiatives.

The two council members representing the North Side, Phillipe Cunningham
and Jeremiah Ellison, have supported the effort to change the charter
and defund the police.

``To say that Black North Siders have not had a voice erases the
existence of two Black North Side council members,'' Mr. Cunningham
said.

Image

Kentrell Grimes with his 2-year-old son, Kentrell Jr. ``How can you
defund the police and then bring another group to police?'' Mr. Grimes
said.Credit...Nina Robinson for The New York Times

He said his constituents have told him they want to see ``transformative
change in the way that the city keeps our community safe.''

He acknowledged that the police could not be eliminated in one fell
swoop.

``We will likely need some form of law enforcement for the foreseeable
future,'' Mr. Cunningham said. Yet he envisioned a system in which
greater investment in things like community workers, health, housing and
education would stabilize the community and drive down crime.

But that is difficult for many to envision right now as Minneapolis,
like many other urban areas across the country, is in the midst of a
spike in gun violence. The Police Department's Fourth Precinct, which
covers North Minneapolis,
\href{https://tableau.minneapolismn.gov/views/MPDMStatCrimeData/CrimeDashboard-byDate?\%3Aembed=y\&\%3AshowAppBanner=false\&\%3AshowShareOptions=true\&\%3Adisplay_count=no\&\%3AshowVizHome=no}{has
seen more murders} and violent crimes this year than any other precinct
in the city.

One of those victims of violence was Taona Mays, 24, who was sitting in
the back of a friend's sport utility vehicle on a Saturday in late July
when a man walked up alongside the car and began shooting. A bullet
struck her left hip, leaving her with a severe limp.

``The presence of the police is definitely needed because without it,
people definitely will just do anything,'' said Ms. Mays, who does
medical transport at a hospital.

Yet she also embraces elements of what defund activists have been
preaching. If there were fewer officers, she said, they would only be
able to focus on major crimes rather than harassing people for petty
things. She actually wants something to replace the police, she said,
but she cannot think of what that would be.

``It's good to have good police,'' she said. ``It's bad to have bad
police.''

Advertisement

\protect\hyperlink{after-bottom}{Continue reading the main story}

\hypertarget{site-index}{%
\subsection{Site Index}\label{site-index}}

\hypertarget{site-information-navigation}{%
\subsection{Site Information
Navigation}\label{site-information-navigation}}

\begin{itemize}
\tightlist
\item
  \href{https://help.nytimes3xbfgragh.onion/hc/en-us/articles/115014792127-Copyright-notice}{©~2020~The
  New York Times Company}
\end{itemize}

\begin{itemize}
\tightlist
\item
  \href{https://www.nytco.com/}{NYTCo}
\item
  \href{https://help.nytimes3xbfgragh.onion/hc/en-us/articles/115015385887-Contact-Us}{Contact
  Us}
\item
  \href{https://www.nytco.com/careers/}{Work with us}
\item
  \href{https://nytmediakit.com/}{Advertise}
\item
  \href{http://www.tbrandstudio.com/}{T Brand Studio}
\item
  \href{https://www.nytimes3xbfgragh.onion/privacy/cookie-policy\#how-do-i-manage-trackers}{Your
  Ad Choices}
\item
  \href{https://www.nytimes3xbfgragh.onion/privacy}{Privacy}
\item
  \href{https://help.nytimes3xbfgragh.onion/hc/en-us/articles/115014893428-Terms-of-service}{Terms
  of Service}
\item
  \href{https://help.nytimes3xbfgragh.onion/hc/en-us/articles/115014893968-Terms-of-sale}{Terms
  of Sale}
\item
  \href{https://spiderbites.nytimes3xbfgragh.onion}{Site Map}
\item
  \href{https://help.nytimes3xbfgragh.onion/hc/en-us}{Help}
\item
  \href{https://www.nytimes3xbfgragh.onion/subscription?campaignId=37WXW}{Subscriptions}
\end{itemize}
