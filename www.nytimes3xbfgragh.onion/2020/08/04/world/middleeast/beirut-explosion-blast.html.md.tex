Sections

SEARCH

\protect\hyperlink{site-content}{Skip to
content}\protect\hyperlink{site-index}{Skip to site index}

\href{https://www.nytimes3xbfgragh.onion/section/world/middleeast}{Middle
East}

\href{https://myaccount.nytimes3xbfgragh.onion/auth/login?response_type=cookie\&client_id=vi}{}

\href{https://www.nytimes3xbfgragh.onion/section/todayspaper}{Today's
Paper}

\href{/section/world/middleeast}{Middle East}\textbar{}Deadly Explosions
Shatter Beirut, Lebanon

\url{https://nyti.ms/33qudRi}

\begin{itemize}
\item
\item
\item
\item
\item
\item
\end{itemize}

\hypertarget{beirut-explosion}{%
\subsubsection{Beirut Explosion}\label{beirut-explosion}}

\begin{itemize}
\tightlist
\item
  \href{https://www.nytimes3xbfgragh.onion/2020/08/11/world/middleeast/lebanon-government-resigns-explainer.html?name=styln-beirut\&region=TOP_BANNER\&block=storyline_menu_recirc\&action=click\&pgtype=Article\&impression_id=b4d6dc20-f1c6-11ea-ab9d-b3a5fe44f351\&variant=undefined}{Lebanon
  in Political Limbo}
\item
  \href{https://www.nytimes3xbfgragh.onion/2020/08/05/world/middleeast/beirut-explosion-footage.html?name=styln-beirut\&region=TOP_BANNER\&block=storyline_menu_recirc\&action=click\&pgtype=Article\&impression_id=b4d6dc21-f1c6-11ea-ab9d-b3a5fe44f351\&variant=undefined}{Footage
  of the Blast}
\item
  \href{https://www.nytimes3xbfgragh.onion/2020/08/05/world/middleeast/beirut-explosion-ammonium-nitrate.html?name=styln-beirut\&region=TOP_BANNER\&block=storyline_menu_recirc\&action=click\&pgtype=Article\&impression_id=b4d70330-f1c6-11ea-ab9d-b3a5fe44f351\&variant=undefined}{What
  is Ammonium Nitrate?}
\item
  \href{https://www.nytimes3xbfgragh.onion/interactive/2020/08/04/world/middleeast/beirut-explosion-damage.html?name=styln-beirut\&region=TOP_BANNER\&block=storyline_menu_recirc\&action=click\&pgtype=Article\&impression_id=b4d70331-f1c6-11ea-ab9d-b3a5fe44f351\&variant=undefined}{Mapping
  the Damage}
\end{itemize}

Advertisement

\protect\hyperlink{after-top}{Continue reading the main story}

Supported by

\protect\hyperlink{after-sponsor}{Continue reading the main story}

\hypertarget{deadly-explosions-shatter-beirut-lebanon}{%
\section{Deadly Explosions Shatter Beirut,
Lebanon}\label{deadly-explosions-shatter-beirut-lebanon}}

Dozens are dead and thousands hurt. The cause is unclear but the
government said ``highly explosive materials'' had been stored at the
blast scene in the Lebanese capital.

\href{https://www.nytimes3xbfgragh.onion/by/ben-hubbard}{\includegraphics{https://static01.graylady3jvrrxbe.onion/images/2018/10/10/multimedia/author-ben-hubbard/author-ben-hubbard-thumbLarge.png}}\href{https://www.nytimes3xbfgragh.onion/by/maria-abi-habib}{\includegraphics{https://static01.graylady3jvrrxbe.onion/images/2018/10/08/multimedia/author-maria-abi-habib/author-maria-abi-habib-thumbLarge.png}}

By \href{https://www.nytimes3xbfgragh.onion/by/ben-hubbard}{Ben Hubbard}
and \href{https://www.nytimes3xbfgragh.onion/by/maria-abi-habib}{Maria
Abi-Habib}

\begin{itemize}
\item
  Aug. 4, 2020
\item
  \begin{itemize}
  \item
  \item
  \item
  \item
  \item
  \item
  \end{itemize}
\end{itemize}

\hypertarget{heres-what-we-know}{%
\subsubsection{Here's what we know:}\label{heres-what-we-know}}

\begin{itemize}
\tightlist
\item
  \protect\hyperlink{link-2e7a9219}{More than 70 are dead, the health
  ministry said.}
\item
  \protect\hyperlink{link-1971293e}{A huge cache of ammonium nitrate, an
  explosive compound, had been stored at the blast site.}
\item
  \protect\hyperlink{link-12ef1c10}{A smaller explosion was followed by
  a much larger one.}
\item
  \protect\hyperlink{link-1fdae9ed}{I was bloodied and dazed. Beirut
  strangers treated me like a friend.}
\item
  \protect\hyperlink{link-777231d6}{Health facilities were badly damaged
  at the moment they were most needed.}
\item
  \protect\hyperlink{link-1202af77}{The explosion hit the waterfront,
  near several important buildings.}
\item
  \protect\hyperlink{link-3a2e264f}{The blast stirred memories of war in
  a city that had been relatively calm in recent years.}
\end{itemize}

\includegraphics{https://static01.graylady3jvrrxbe.onion/images/2020/08/05/world/05lebanon-videocover/05lebanon-briefing-ledesub-videoSixteenByNine3000.jpg}

\hypertarget{more-than-70-are-dead-the-health-ministry-said}{%
\subsection{More than 70 are dead, the health ministry
said.}\label{more-than-70-are-dead-the-health-ministry-said}}

Lebanon's health ministry said that at least 78 people had died and
4,000 suffered injuries in the
\href{https://www.nytimes3xbfgragh.onion/2020/08/04/world/middleeast/lebanon-explosion-beirut.html}{explosions}
and fire that shook
\href{https://www.nytimes3xbfgragh.onion/2020/08/04/world/middleeast/lebanon-explosion-beirut.html}{Beirut}
on Tuesday.

The numbers climbed steadily through the day, and with the wounded still
streaming into hospitals and the search for missing people underway,
they were likely to go higher still.

The secretary-general of the Kataeb political party, Nizar Najarian, was
killed in the blast, and among those injured was Kamal Hayek, the
chairman of the state-owned electricity company, who was in critical
condition, the news agency reported.

\href{https://www.nytimes3xbfgragh.onion/interactive/2020/08/04/world/middleeast/beirut-explosion-damage.html}{}

\includegraphics{https://static01.graylady3jvrrxbe.onion/images/2020/08/04/us/beirut-explosion-damage-promo-1596586440536/beirut-explosion-damage-promo-1596586440536-articleLarge-v3.jpg}

\hypertarget{mapping-the-damage-from-the-beirut-explosion}{%
\subsection{Mapping the Damage From the Beirut
Explosion}\label{mapping-the-damage-from-the-beirut-explosion}}

Damage was seen at least two miles from the explosion, encompassing an
area with more than 750,000 residents.

Videos of the aftermath posted online showed wounded people bleeding
amid the dust and rubble, and damage where flying debris had punched
holes in walls and furniture. On social media, people reported damage to
homes and cars far from the port.

The Lebanese Red Cross said that every available ambulance from North
Lebanon, Bekaa and South Lebanon was being dispatched to Beirut to help
patients.

Hospitals were so overwhelmed that they were turning wounded people
away, including the American University Hospital. Patients were
transported to hospitals outside Beirut because those in the city were
at capacity.

Public Health Minister Hamad Hassan announced that his ministry would
cover the costs of treating the wounded at hospitals, the National News
Agency reported. It said the decision covered both hospitals that have
contracts with the ministry as well as those that don't.

Prime Minister Hassan Diab announced that Wednesday would be a national
day of mourning, the National News Agency reported. The Lebanese
presidency said on Twitter that President Michel Aoun had instructed the
military to aid in the response, and called an emergency meeting of the
Supreme Defense Council, which declared Beirut a disaster area.

\hypertarget{a-huge-cache-of-ammonium-nitrate-an-explosive-compound-had-been-stored-at-the-blast-site}{%
\subsection{A huge cache of ammonium nitrate, an explosive compound, had
been stored at the blast
site.}\label{a-huge-cache-of-ammonium-nitrate-an-explosive-compound-had-been-stored-at-the-blast-site}}

\includegraphics{https://static01.graylady3jvrrxbe.onion/images/2020/08/04/world/04lebanon6/merlin_175303926_4f704a95-e36c-43c2-917e-8280c64135f2-articleLarge.jpg?quality=75\&auto=webp\&disable=upscale}

A large cache of explosive material seized by the government years ago
was stored where the explosions occurred, according to top Lebanese
officials --- specifically ammonium nitrate, commonly used in both
fertilizer and bombs.

Accidental detonation of ammonium nitrate has caused a number of deadly
industrial accidents, including the worst in United States history: In
1947, a ship carrying ammonium nitrate caught fire and exploded in the
harbor of Texas City, Texas, starting a chain reaction of blasts and
blazes that killed 581 people.

The chemical has also been the primary ingredient in bombs used in
several terrorist attacks, including the destruction of the federal
office building in Oklahoma City in 1995, which killed 168 people.

In a televised statement, an official of the Lebanese Higher Defense
Council quoted Prime Minister Diab as saying: ``I will not relax until
we find the responsible party for what happened, hold it accountable and
apply the most serious punishments against it because it isn't
acceptable that a shipment of ammonium nitrate --- estimated to be 2,750
tons --- was in a depot for the past six years without precautionary
measures being taken.''

Hours earlier, Maj. Gen. Abbas Ibrahim, the head of Lebanon's general
security service, had said that ``highly explosive materials'' were
stored at the site, which Mr. Aoun then confirmed. At first, neither of
them said what those materials were, but General Ibrahim warned against
getting ``ahead of the investigation'' and speculating about a terrorist
act.

American military leaders ``seem to think it was an attack,'' President
Trump told reporters at the White House, which was at odds with what
Lebanese officials said. ``It was a bomb of some kind.''

Mr. Diab, the prime minister, said in a televised statement, ``Facts on
this dangerous depot, which has existed since 2014 or the past six
years, will be announced.''

``What happened today will not come to pass without accountability,''
Mr. Diab said. ``Those responsible will pay a price for this
catastrophe.'' he said. ``This is a promise to the martyrs and wounded
people. This is a national commitment.''

The explosions on Tuesday were preceded by a fire at a warehouse at
Beirut's port,
\href{http://nna-leb.gov.lb/en/show-news/118492/Fire-breaks-out-in-warehouse-at-Port-of-Beirut-causes-major-explosion}{according
to Lebanon's National News Agency}.

There were local reports that the warehouse contained fireworks, and in
several videos posted online, colored flashes could be seen in the dark
smoke rising from the fire, just before the major explosion.

The governor of Beirut, Marwan Abboud, speaking on television, could not
say what had caused the explosion. Breaking into tears, he called it a
national catastrophe.

\hypertarget{a-smaller-explosion-was-followed-by-a-much-larger-one}{%
\subsection{A smaller explosion was followed by a much larger
one.}\label{a-smaller-explosion-was-followed-by-a-much-larger-one}}

Image

Smoke rising from the scene of an explosion in Beirut on
Tuesday.Credit...Anwar Amro/Agence France-Presse --- Getty Images

Two explosions shook Beirut --- the second one much larger than the
first, carrying enough force to overturn cars, damage and shake
buildings across the city and strew, debris over a wide area.

The larger explosion, at 6:08 p.m., blew out the glass from balconies
and windows of buildings several miles away from the port and at least
one building collapsed from the force of the blast. One resident said
the streets looked like they were ``cobbled in glass.''

Videos posted online showed a shock wave erupting from the second
explosion, knocking people down and enveloping much of the center city
in a cloud of dust and smoke. Cars were overturned and streets were
blocked by debris, forcing many injured people to walk to hospitals.

Flames continued to rise from the rubble well after the explosions, and
a cloud of smoke, tinted pink in the sunset, rose thousands of feet into
the sky.

The larger blast was heard and felt in Cyprus, more than 100 miles away,
and
\href{https://www.emsc-csem.org/Earthquake/earthquake.php?id=882410\#summary}{registered
on seismographs at magnitude 3.3}.

\hypertarget{i-was-bloodied-and-dazed-beirut-strangers-treated-me-like-a-friend}{%
\subsection{I was bloodied and dazed. Beirut strangers treated me like a
friend.}\label{i-was-bloodied-and-dazed-beirut-strangers-treated-me-like-a-friend}}

\emph{Vivian Yee, a correspondent for The New York Times, was at home in
Beiru}t \emph{when two explosions convulsed the city. This is her
first-person account of what happened.}

I was just about to look at a video a friend had sent me on Tuesday
afternoon --- ``the port seems to be burning,'' she said --- when my
whole building shook. Uneasily, naïvely, I ran to the window, then back
to my desk to check for news.

Then came a much bigger boom, and the sound itself seemed to splinter.
There was shattered glass flying everywhere. Not thinking but moving, I
ducked under my desk.

When the world stopped cracking open, I couldn't see at first because of
the blood running down my face. After blinking the blood from my eyes, I
tried to take in the sight of my apartment turned into a demolition
site. My yellow front door had been hurled on top of my dining table. I
couldn't find my passport, or sturdy shoes.

Later, someone would tell me that Beirut is of her generation, raised
during Lebanon's 15-year civil war, instinctively ran into their
hallways as soon as they heard the first blast, to escape the glass they
knew would break.

I was not so well-trained, but the Lebanese who would help me in the
hours to come had the steadiness that comes from having lived through
countless previous disasters. Nearly all were strangers, yet they
treated me like a friend.

When I got downstairs, someone passing on a motorbike saw my bloody face
and told me to hop on.

Everyone on the street seemed to be either bleeding from open gashes or
swathed in makeshift bandages --- all except one woman in a chic,
backless top leading a small dog on a leash. Only an hour before, we had
all been walking dogs or checking email or grocery shopping. Only an
hour before, there had been no blood.

\href{https://www.nytimes3xbfgragh.onion/2020/08/04/world/middleeast/beirut-explosion-first-person.html}{\emph{Read
more of Ms. Yee's account}}\emph{.}

\hypertarget{health-facilities-were-badly-damaged-at-the-moment-they-were-most-needed}{%
\subsection{Health facilities were badly damaged at the moment they were
most
needed.}\label{health-facilities-were-badly-damaged-at-the-moment-they-were-most-needed}}

Image

Health-care workers moving an injured man from one hospital to another
in Beirut on Tuesday night.Credit...Nabil Mounzer/EPA, via Shutterstock

St. George Hospital in central Beirut, one of the city's biggest, was so
severely damaged that it had to shut down and send patients elsewhere.
Dozens of patients and visitors were wounded by falling debris and
flying glass.

``Every floor of the hospital is damaged,'' said Dr. Peter Noun, the
chief of pediatric hematology and oncology. ``I didn't see this even
during the war. It's a catastrophe.''

The 60-bed Bikhazi Medical Group hospital treated 500 patients in the
hours after the blast, despite extensive damage, said Rima Azar, the
hospital director and co-owner. One woman was already dead when she was
brought in.

``The hospital has lots of cracked glass, the door to entrance of the
hospital is completely shattered,'' Ms. Azar said. ``The full ceiling
fell on some patients in some rooms. The pressure was horrific. We heard
a boom, then everything was shaking.''

Health care workers worried about the fate of one of the country's main
vaccine and medication stockpiles, in the Karantina warehouse near the
port. They said hundreds of thousands of doses, used to supply health
centers across Lebanon, were stored on tall shelves in the warehouse, in
an area where other buildings were badly damaged.

Inside St. George Hospital, about six-tenths of a mile from the
explosion, ``everything just fell down, the windows destroyed, the
ceiling in pieces,'' Dr. Noun said. Several of his patients --- children
with cancer --- and their family members were among the injured.

Two parents of his patients were in critical condition, Dr. Noun said.
Shards of glass from a shattered window ripped into the face and body of
one of them, a father who was visiting his child. The man was intubated
and in critical condition at another hospital, Dr. Noun said.

He said the parents of four children being treated for cancer were so
panicked they grabbed their children, pulled out their intravenous
needles and bundled them into their cars, headed to other hospitals or
even went home.

A voice recording from Dr. Joseph Haddad of St. George Hospital was
shared with other doctors across Lebanon, who forwarded it to The New
York Times. In a follow-up phone call Dr. Haddad, director of intensive
care at the hospital, confirmed the recording's authenticity.

``My friends, my friends. This is Joseph Haddad calling you from St.
George Hospital. There is no St. George Hospital anymore. It's fallen,
it's on the floor,'' Dr. Haddad says, as broken glass is heard crackling
underfoot. ``It's all destroyed. All of it. Pray to God, pray to God.''

\hypertarget{the-explosion-hit-the-waterfront-near-several-important-buildings}{%
\subsection{The explosion hit the waterfront, near several important
buildings.}\label{the-explosion-hit-the-waterfront-near-several-important-buildings}}

Image

Wreckage from the explosion at the port in Beirut on
Tuesday.Credit...Mohamed Azakir/Reuters

The explosions hit Beirut's northern, industrial waterfront, little more
than a mile away from the Grand Serail palace, where Lebanon's prime
minister is based. Many landmarks, including hospitals, mosques,
churches and universities are nearby.

They erupted next to a tall building called Beirut Port Silos, at or
near a structure identified on maps as a warehouse. Videos showed only
twisted metal and chunks of concrete where that warehouse had been, some
of it identifiable as the remains of trucks and shipping containers.

\hypertarget{the-blast-stirred-memories-of-war-in-a-city-that-had-been-relatively-calm-in-recent-years}{%
\subsection{The blast stirred memories of war in a city that had been
relatively calm in recent
years.}\label{the-blast-stirred-memories-of-war-in-a-city-that-had-been-relatively-calm-in-recent-years}}

Image

Running through the streets in Beirut after the explosion on
Tuesday.Credit...Hassan Ammar/Associated Press

The severity of the explosions recalled the days when bombings and
mayhem were a regular fact of life in Beirut, both during its 1975-1990
civil war and its aftermath, including sporadic conflicts between Israel
and Hezbollah.

Among the worst were in 1983, when a
\href{https://www.history.com/this-day-in-history/suicide-bomber-destroys-u-s-embassy-in-beirut}{suicide
attack on the United States Embassy} killed 63 people in April, and
\href{https://www.history.com/this-day-in-history/beirut-barracks-blown-up}{bombing
in October} on the headquarters of international peacekeepers killed 241
U.S. Marines and 58 French troops. The attack on the Marines, the worst
loss for them since the invasion of Iwo Jima in World War II, was blamed
by American officials on Hezbollah, which the United States, Israel and
a number of other countries consider a terrorist organization.

Another bombing in Lebanon upended Middle East politics in February of
2005, when Rafik Hariri, a former prime minister, was killed along with
21 others by a car bombing of his motorcade.

The attack was blamed by many on Hezbollah, a Shiite militia and
political party, and its ally, Syria, which had deployed troops in
Lebanon for nearly three decades. Under enormous pressure, the
\href{https://www.nytimes3xbfgragh.onion/2005/04/26/international/middleeast/syrian-troops-leave-lebanon-after-29year-occupation.html}{Syrians
withdrew from Lebanon} two months later, though they retained close ties
to Hezbollah.

A United Nations-backed tribunal at The Hague has been trying four
operatives of Hezbollah, which is now part of Lebanon's government, in
absentia for the Hariri assassination
\href{https://www.reuters.com/article/us-lebanon-tribunal-hariri/crisis-weary-lebanon-braces-for-hariri-tribunal-verdict-idUSKCN2500JU}{and
is due to render a verdict this Friday.}

In the summer of 2006, Israel and Hezbollah engaged in a 34-day war
\href{https://www.hrw.org/report/2007/09/05/why-they-died/civilian-casualties-lebanon-during-2006-war\#}{that,
according to a tally by Human Rights Watch,}left more than 1,100
Lebanese and at least 55 Israelis dead, most of them civilians.

But if the explosions on Tuesday were intentional, they would shatter a
prolonged stretch of relative calm in the Lebanese capital. An Israeli
intelligence official denied any Israeli involvement in the incident.

Less than a week ago, Israel said it had thwarted a raid by a
``terrorist squad'' from Hezbollah, the Shiite group that is part of
Lebanon's government, in a disputed border area. Israeli military
officials said there was an exchange of gunfire, which Hezbollah denied.
Israeli military officials say Hezbollah has planted many rockets in
southern Lebanon that could threaten northern Israel.

But in recent years, the longtime enemies have sought to avoid another
war. Hezbollah has refrained from killing Israelis while Israel has
largely avoided killing Hezbollah fighters in Syria, where they are
fighting on the Syrian government's side.

\hypertarget{assessing-the-toll-a-political-party-waits-to-learn-whether-it-was-malice-or-neglect}{%
\subsection{Assessing the toll, a political party waits to learn whether
it was malice or
neglect.}\label{assessing-the-toll-a-political-party-waits-to-learn-whether-it-was-malice-or-neglect}}

Image

The devastation near the port in Beirut on Tuesday.Credit...Agence
France-Presse --- Getty Images

When the explosion struck, meetings were in full swing less than a mile
away, at the hillside headquarters of the Kataeb Party, a Christian
political group that was once one of Lebanon's most powerful.

The blast shook the building so badly that party members thought a bomb
had gone off inside. As they collected their nerves and their
belongings, they saw that the party's general secretary, Nazar Najarian,
had been wounded by falling debris. Mr. Najarian, known by the nickname
Nazo, died of his injuries.

``He had been through explosions, assassination attempts, wars with the
Palestinians and Syrians, Nazo saw it all,'' said
\href{https://www.facebookcorewwwi.onion/EliasHankach2018/}{Elias
Hankach}, a Kataeb parliamentarian. ``Our headquarters looks like a bomb
went off inside. The inside is a mess, it's madness.''

He said the party was waiting for clarity on whether the blast was an
attack, the kind of crude tool used for decades to shape Lebanon's
political landscape, or just an accident resulting from mismanagement.
If it turned out to be accidental, he said, then the disaster is not
particularly surprising, the product of ``cumulative nonchalance at all
levels.''

``Whether you talk about the economy, safety standards, the port, the
corruption --- none of the country's issues have had a serious attempt
at resolution,'' Mr. Hankach said. ``We are living in this doomed
management of the country.''

Ben Hubbard reported from Beirut, and Maria Abi-Habib from Los Angeles.
Nada Rashwan contributed reporting from Cairo, Alan Yuhas from
Philadelphia, Adam Rasgon and Ronen Bergman from Tel Aviv, Rick
Gladstone from Eastham, Mass., and Richard Pérez-Peña from New York.

Advertisement

\protect\hyperlink{after-bottom}{Continue reading the main story}

\hypertarget{site-index}{%
\subsection{Site Index}\label{site-index}}

\hypertarget{site-information-navigation}{%
\subsection{Site Information
Navigation}\label{site-information-navigation}}

\begin{itemize}
\tightlist
\item
  \href{https://help.nytimes3xbfgragh.onion/hc/en-us/articles/115014792127-Copyright-notice}{©~2020~The
  New York Times Company}
\end{itemize}

\begin{itemize}
\tightlist
\item
  \href{https://www.nytco.com/}{NYTCo}
\item
  \href{https://help.nytimes3xbfgragh.onion/hc/en-us/articles/115015385887-Contact-Us}{Contact
  Us}
\item
  \href{https://www.nytco.com/careers/}{Work with us}
\item
  \href{https://nytmediakit.com/}{Advertise}
\item
  \href{http://www.tbrandstudio.com/}{T Brand Studio}
\item
  \href{https://www.nytimes3xbfgragh.onion/privacy/cookie-policy\#how-do-i-manage-trackers}{Your
  Ad Choices}
\item
  \href{https://www.nytimes3xbfgragh.onion/privacy}{Privacy}
\item
  \href{https://help.nytimes3xbfgragh.onion/hc/en-us/articles/115014893428-Terms-of-service}{Terms
  of Service}
\item
  \href{https://help.nytimes3xbfgragh.onion/hc/en-us/articles/115014893968-Terms-of-sale}{Terms
  of Sale}
\item
  \href{https://spiderbites.nytimes3xbfgragh.onion}{Site Map}
\item
  \href{https://help.nytimes3xbfgragh.onion/hc/en-us}{Help}
\item
  \href{https://www.nytimes3xbfgragh.onion/subscription?campaignId=37WXW}{Subscriptions}
\end{itemize}
