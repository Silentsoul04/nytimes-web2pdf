Sections

SEARCH

\protect\hyperlink{site-content}{Skip to
content}\protect\hyperlink{site-index}{Skip to site index}

\href{https://www.nytimes3xbfgragh.onion/section/arts/music}{Music}

\href{https://myaccount.nytimes3xbfgragh.onion/auth/login?response_type=cookie\&client_id=vi}{}

\href{https://www.nytimes3xbfgragh.onion/section/todayspaper}{Today's
Paper}

\href{/section/arts/music}{Music}\textbar{}5 Minutes That Will Make You
Love 21st-Century Composers

\url{https://nyti.ms/2BXrhQC}

\begin{itemize}
\item
\item
\item
\item
\item
\item
\end{itemize}

Advertisement

\protect\hyperlink{after-top}{Continue reading the main story}

Supported by

\protect\hyperlink{after-sponsor}{Continue reading the main story}

\hypertarget{5-minutes-that-will-make-you-love-21st-century-composers}{%
\section{5 Minutes That Will Make You Love 21st-Century
Composers}\label{5-minutes-that-will-make-you-love-21st-century-composers}}

We asked Ivo van Hove, Justin Peck, Du Yun and others to pick the music
that moves them. Listen to their choices.

\includegraphics{https://static01.graylady3jvrrxbe.onion/images/2020/08/05/arts/05fiveminutes-composers/05fiveminutes-composers-articleLarge.gif?quality=75\&auto=webp\&disable=upscale}

\begin{itemize}
\item
  Aug. 5, 2020
\item
  \begin{itemize}
  \item
  \item
  \item
  \item
  \item
  \item
  \end{itemize}
\end{itemize}

In the past, we've asked some of our favorite artists to choose the five
minutes or so they would play to make their friends fall in love with
\href{https://www.nytimes3xbfgragh.onion/2018/09/06/arts/music/5-minutes-that-will-make-you-love-classical-music.html}{classical
music},
\href{https://www.nytimes3xbfgragh.onion/2019/04/19/arts/music/classical-music-piano.html}{the
piano},
\href{https://www.nytimes3xbfgragh.onion/2020/04/28/arts/music/classical-music-opera.html}{opera},
\href{https://www.nytimes3xbfgragh.onion/2020/06/03/arts/music/five-minutes-classical-music-cello.html}{the
cello} and
\href{https://www.nytimes3xbfgragh.onion/2020/07/01/arts/music/classical-music-mozart.html}{Mozart}.

Now we want to convince those curious friends to love music written in
the past 20 years --- some of it meditative, some explosive. We hope you
find lots here to discover and enjoy; leave your choices in the
comments.

\hypertarget{--}{%
\subsubsection{◆ ◆ ◆}\label{--}}

\hypertarget{justin-peck-choreographer}{%
\subsection{Justin Peck,
choreographer}\label{justin-peck-choreographer}}

Caroline Shaw's ``Partita'' spun me round and round, turned me inside
out and launched me into a whole new understanding of what music can be.
The piece feels three-dimensional, voluminous, astronomical --- but also
intimate, personal and incremental. It's like someone whispering into
your ear while you're climbing the tallest mountain. It is uniquely
fragrant; it has needlelike precision; it organically spills through
some of the most sophisticated harmonies. In the mouths of Roomful of
Teeth, it is a virtuosic display of the incredible range of the human
voice.

\hypertarget{caroline-shaw-partita-for-eight-voices}{%
\subsubsection{Caroline Shaw, ``Partita for Eight
Voices''}\label{caroline-shaw-partita-for-eight-voices}}

New Amsterdam Records

\hypertarget{---1}{%
\subsubsection{◆ ◆ ◆}\label{---1}}

\hypertarget{ivo-van-hove-director}{%
\subsection{Ivo van Hove, director}\label{ivo-van-hove-director}}

The Dutch composer Michel van der Aa is an omnivore, influenced by
electronic music, pop, soundscapes, movies and installation art. Genres
and their confinements are of no interest to him, as they aren't to a
whole new generation. Listen to this piece, full of brutal poetry and
great rhythms: It will grip you immediately, ignite your imagination and
give you goose bumps.

\hypertarget{michel-van-der-aa-blank-out}{%
\subsubsection{Michel van der Aa, ``Blank
Out''}\label{michel-van-der-aa-blank-out}}

Netherlands Chamber Choir and Klaas Stok

\hypertarget{---2}{%
\subsubsection{◆ ◆ ◆}\label{---2}}

\hypertarget{jeanine-tesori-composer}{%
\subsection{Jeanine Tesori, composer}\label{jeanine-tesori-composer}}

I love Jessie Montgomery's ``Strum'' because I can find myself in it.
The way it searches and shifts, changing colors and textures; the way
the second violin and viola join forces as the cello and first violin do
the same. The way it explores and grooves and celebrates these
instruments, so you feel they can do anything except land a plane. Like
all great chamber groups, the Catalyst Quartet is
\href{https://youtu.be/qnLE9ygdwrU}{beautiful to watch}, like a family
in lively conversation at the dinner table: anticipating, interrupting,
changing subjects.

\hypertarget{jessie-montgomery-strum}{%
\subsubsection{Jessie Montgomery,
``Strum''}\label{jessie-montgomery-strum}}

Azica Records

\hypertarget{---3}{%
\subsubsection{◆ ◆ ◆}\label{---3}}

\hypertarget{tiona-nekkia-mcclodden-artist}{%
\subsection{Tiona Nekkia McClodden,
artist}\label{tiona-nekkia-mcclodden-artist}}

There is something just stunning that happens in Courtney Bryan's
``Shedding Skin,'' inspired by the poem of the same name by Harryette
Mullen. I included ``Shedding Skin'' in
\href{https://thekitchen.org/event/julius-eastman-that-which-is-fundamental}{the
Julius Eastman retrospective I curated} at the Kitchen in 2018 because
it gave me the sensation his works did when I first heard them. There is
a whole history inclusive of many Black radical music traditions present
here, Ms. Bryan's attempt to notate improvisation within the form of
classical composition.

\hypertarget{courtney-bryan-shedding-skin}{%
\subsubsection{Courtney Bryan, ``Shedding
Skin''}\label{courtney-bryan-shedding-skin}}

American Composers Orchestra

\hypertarget{---4}{%
\subsubsection{◆ ◆ ◆}\label{---4}}

\hypertarget{seth-colter-walls-times-writer}{%
\subsection{Seth Colter Walls, Times
writer}\label{seth-colter-walls-times-writer}}

Joseph C. Phillips Jr. works in a style he calls ``mixed music.'' Here,
his ensemble,
\href{https://numinous.bandcamp.com/album/changing-same-2}{Numinous},
nails his hairpin turns --- and his
\href{https://www.numinousmusic.com/uploads/5/0/4/9/50499401/19_joseph_c_phillips_jr_from_thesis__final_.pdf}{references
to Schoenberg and Curtis Mayfield} --- while offering pristine vocal and
string blends, plus guitar work that embraces funk and fusion-jazz.

\hypertarget{joseph-c-phillips-jr-19}{%
\subsubsection{Joseph C. Phillips Jr.,
``19''}\label{joseph-c-phillips-jr-19}}

Joseph C. Phillips Jr. and Numinous

\hypertarget{---5}{%
\subsubsection{◆ ◆ ◆}\label{---5}}

\hypertarget{vikingur-olafsson-pianist}{%
\subsection{Vikingur Olafsson,
pianist}\label{vikingur-olafsson-pianist}}

The music I love most often gives me the feeling of being in transit ---
ideas and sensations like ever-changing landscapes seen through the
window of a train. In ``Stars --- Sun --- Moon,'' the fourth movement of
Thomas Adès's ``In Seven Days,'' the trip becomes a voyage into space;
soundscapes turn into moonscapes. This gorgeously organized chaos has
some of the most imaginative writing I can think of for piano and
orchestra (here, Kirill Gerstein and the Tanglewood Music Center
Orchestra). When I first heard these sounds 10 years ago, I giggled
softly, which is my slightly awkward physical reaction to being amazed.
I still have that reaction when I hear --- or, these days, play --- this
movement.

\hypertarget{thomas-aduxe8s-in-seven-days}{%
\subsubsection{Thomas Adès, ``In Seven
Days''}\label{thomas-aduxe8s-in-seven-days}}

Myrios Classics

\hypertarget{---6}{%
\subsubsection{◆ ◆ ◆}\label{---6}}

\hypertarget{richard-reed-parry-musician}{%
\subsection{Richard Reed Parry,
musician}\label{richard-reed-parry-musician}}

More of a languid walkabout through a slowly changing musical
environment than a composition with a clear beginning, middle and end,
this piece is exactly the type of place where I have wanted to spend
more and more of my time during recent days. While appearing almost
aimless on its surface, it is in fact a deeply satisfying experience to
hear this slow motion form in its entirety. As a listener, I feel as
though I am sitting in a small rowboat adrift on a lake, with the wind
gently pushing me back and forth between small, exquisitely beautiful
coves, while the boat very slowly turns in a circle; by the end I have
seen and heard the entire 360 gorgeous degrees of horizon around me,
from every angle, countless times.

\hypertarget{john-luther-adams-in-the-white-silence}{%
\subsubsection{John Luther Adams, ``In the White
Silence''}\label{john-luther-adams-in-the-white-silence}}

New World Records

\hypertarget{---7}{%
\subsubsection{◆ ◆ ◆}\label{---7}}

\hypertarget{pam-tanowitz-choreographer}{%
\subsection{Pam Tanowitz,
choreographer}\label{pam-tanowitz-choreographer}}

Ted Hearne's music is heart and head, funny and serious and full of
imagination, intelligently rigorous while being so moving I tear up.
Good art is like that. His music lives in the space between the
historical and personal, past and present, and always takes risks in the
way he shapes time. You feel like he is composing his insides, his guts.
It reminds me of this Morton Feldman quote: ``Art is a crucial operation
we perform ourselves. Unless we take chances we die in art.''

\hypertarget{ted-hearne-law-of-mosaics}{%
\subsubsection{Ted Hearne, ``Law of
Mosaics''}\label{ted-hearne-law-of-mosaics}}

A Far Cry (Crier Records)

\hypertarget{---8}{%
\subsubsection{◆ ◆ ◆}\label{---8}}

\hypertarget{garth-greenwell-writer}{%
\subsection{Garth Greenwell, writer}\label{garth-greenwell-writer}}

Donnacha Dennehy's setting of Yeats's tender, macabre love poem ``He
Wishes His Beloved Were Dead'' is haunting and spare, with slow-moving,
eerie dissonances in winds and strings pierced by bell-like notes from
piano and electric guitar. It sets an intimate stage for the soloist,
her long lines ornamented with turns and grace notes. I fell in love
with Dawn Upshaw's voice as a teenager, when I was first discovering
classical music. In early recordings, her voice is a fountain of gold.
It's a different instrument now: darker, less easy and, like this song,
almost unbearably beautiful.

\hypertarget{donnacha-dennehy-he-wishes-his-beloved-were-dead}{%
\subsubsection{Donnacha Dennehy, ``He Wishes His Beloved Were
Dead''}\label{donnacha-dennehy-he-wishes-his-beloved-were-dead}}

Nonesuch Records

\hypertarget{---9}{%
\subsubsection{◆ ◆ ◆}\label{---9}}

\hypertarget{du-yun-composer}{%
\subsection{Du Yun, composer}\label{du-yun-composer}}

A staple of the New York improvisation scene, the cellist and composer
Okkyung Lee released her latest album two months ago. ``In Stardust'' is
dedicated to the Korean cartoonist Kang Kyung-ok, who created a manhwa
series under that name, a sci-fi story about a normal high school girl
who is later revealed to be the heir to an interstellar kingdom. She was
meant to be sent off to the universe but ended up on earth.

\hypertarget{okkyung-lee-in-stardust-for-kang-kyung-ok}{%
\subsubsection{Okkyung Lee, ``In Stardust (For Kang
Kyung-ok)''}\label{okkyung-lee-in-stardust-for-kang-kyung-ok}}

Shelter Press

\hypertarget{---10}{%
\subsubsection{◆ ◆ ◆}\label{---10}}

\hypertarget{joshua-barone-times-editor-and-writer}{%
\subsection{Joshua Barone, Times editor and
writer}\label{joshua-barone-times-editor-and-writer}}

When contemporary composers engage with traditional forms --- the
symphony, the concerto --- the results can be fascinating. Like the
string quartet, which is nearly 250 years old yet is kept fresh by
artists like Gabriella Smith, whose ``Carrot Revolution'' (played here
by the Aizuri Quartet) dashes from its percussive opening through
stylistic juxtapositions as unruly as an English garden. Both an homage
to classical music's past and a folk jam session, it's a testament to
the history of the string quartet, its possibilities and its vitality.

\hypertarget{gabriella-smith-carrot-revolution}{%
\subsubsection{Gabriella Smith, ``Carrot
Revolution''}\label{gabriella-smith-carrot-revolution}}

New Amsterdam Records

\hypertarget{---11}{%
\subsubsection{◆ ◆ ◆}\label{---11}}

\hypertarget{seth-parker-woods-cellist}{%
\subsection{Seth Parker Woods,
cellist}\label{seth-parker-woods-cellist}}

Du Yun's ``San'' for cello and electronics is a modern-day twisted
vocalise, reaching back in time to honor the guqin, an ancient Chinese
string instrument. The piece seems to transport the listener to a
long-ago era, and the cellist Matt Haimovitz draws out the complex
conversation and storytelling buried within this work through high,
soaring melodies, unmetric rhythmic patterns, lyrical scratches and
scrapes.

\hypertarget{du-yun-san}{%
\subsubsection{Du Yun, ``San''}\label{du-yun-san}}

Pentatone Oxingale Series

\hypertarget{---12}{%
\subsubsection{◆ ◆ ◆}\label{---12}}

\hypertarget{max-richter-composer}{%
\subsection{Max Richter, composer}\label{max-richter-composer}}

Caleb Burhans's ``Contritus,'' recorded by the JACK Quartet, is a
beautifully made doorway to all kinds of listening and thinking. This
score is very much of our time; it is direct and approachable, but
carries within it other, older ways of experiencing. The glacial opening
material seems to have its roots far back, in the viol music of Purcell,
while the shimmering and pulsating surfaces later on evoke music from
our own moment. You don't need to know any of this to enjoy
``Contritus,'' though, because the harmony is so lovely.

\hypertarget{caleb-burhans-contritus}{%
\subsubsection{Caleb Burhans,
``Contritus''}\label{caleb-burhans-contritus}}

Cantaloupe Music

\hypertarget{---13}{%
\subsubsection{◆ ◆ ◆}\label{---13}}

\hypertarget{anthony-tommasini-times-chief-classical-music-critic}{%
\subsection{Anthony Tommasini, Times chief classical music
critic}\label{anthony-tommasini-times-chief-classical-music-critic}}

In his opera ``Written on Skin,'' set in medieval times, the composer
George Benjamin's music is modernist and flinty yet also rapturously
beautiful. A turning point arrives when the illiterate, inquisitive
Agnès (the soprano Barbara Hannigan, in this premiere recording from the
Aix Festival) watches with awe and suspicion as the Boy (the
countertenor Bejun Mehta) creates an illuminated book. The music tells
all: Erotic yearnings well up between the two characters, even during
mundane exchanges.

\hypertarget{george-benjamin-written-on-skin}{%
\subsubsection{George Benjamin, ``Written on
Skin''}\label{george-benjamin-written-on-skin}}

Nimbus Records

\hypertarget{---14}{%
\subsubsection{◆ ◆ ◆}\label{---14}}

\hypertarget{klaus-makela-conductor}{%
\subsection{Klaus Makela, conductor}\label{klaus-makela-conductor}}

One of the great joys of being a conductor is presenting new works to
the audience. Jimmy López's ``Perú Negro'' is one of the things I bring
with me almost everywhere I go. It's a work of astonishing intensity and
groovy rhythms, inspired by Afro-Peruvian music, and the perfect
introduction to orchestral music for a person who has never been to a
symphony concert. There are a lot of layers; one can follow the
complicated rhythms of the percussion or enjoy the tempting melodies of
the woodwinds and the strings.

\hypertarget{jimmy-luxf3pez-peruxfa-negro}{%
\subsubsection{Jimmy López, ``Perú
Negro''}\label{jimmy-luxf3pez-peruxfa-negro}}

Harmonia Mundi

\hypertarget{---15}{%
\subsubsection{◆ ◆ ◆}\label{---15}}

\hypertarget{barbara-hannigan-singer-and-conductor}{%
\subsection{Barbara Hannigan, singer and
conductor}\label{barbara-hannigan-singer-and-conductor}}

Ophelia reappears, onstage with orchestra, and tells us, in her own
words, how it was. Paul Griffiths wrote a book called ``let me tell
you,'' using only the 482 words Shakespeare gave Ophelia, letting her
retell her story. The composer Hans Abrahamsen found inspiration in
this, and the result is a work for soprano and orchestra which is
perhaps the most beautiful piece of music I have ever had the honor to
sing. It is full of Ophelia's innocence and experience, her heart
breaking in an ``ecstasy of light.''

\hypertarget{hans-abrahamsen-let-me-tell-you}{%
\subsubsection{Hans Abrahamsen, ``let me tell
you''}\label{hans-abrahamsen-let-me-tell-you}}

Winter \& Winter

\hypertarget{---16}{%
\subsubsection{◆ ◆ ◆}\label{---16}}

\hypertarget{corinna-da-fonseca-wollheim-times-writer}{%
\subsection{Corinna da Fonseca-Wollheim, Times
writer}\label{corinna-da-fonseca-wollheim-times-writer}}

Not much happens in John Luther Adams's ``Sky With Four Suns,'' the
first movement in a cycle dedicated to sky, wind and bird song. Yet the
piece exerts a magnetic pull. Pulseless and wordless, this choral
meditation seems to exist outside of time. Performed by the Crossing,
harmonies slowly shift --- and with them vocal colors, moving from
resonant warmth to nasal metallics --- so that the music seems to
capture slight changes of clarity and light. Mr. Adams is a devoted
environmental activist and his music marries a mystic's reverence for
natural phenomena with a scientist's keen observation.

\hypertarget{john-luther-adams-sky-with-four-suns}{%
\subsubsection{John Luther Adams, ``Sky With Four
Suns''}\label{john-luther-adams-sky-with-four-suns}}

Cantaloupe Music

\hypertarget{---17}{%
\subsubsection{◆ ◆ ◆}\label{---17}}

\hypertarget{chris-thile-musician}{%
\subsection{Chris Thile, musician}\label{chris-thile-musician}}

Is whoever you share your living space with asleep yet? Good. Same here.
Whatcha drinking? Nice. I'm in. You know the rules: One of us picks the
drink, one of us picks the jam. So: Andrew Norman's ``Sustain.'' A
gracefully eerie orchestral nocturne for the summer of 2020 if ever
there was one (though premiered by Gustavo Dudamel and the Los Angeles
Philharmonic in the fall of 2018). OK, headphones on, lights off, let's
check in in five.

Oh, damn, that was 33 minutes, wasn't it? Whatcha think?

\hypertarget{andrew-norman-sustain}{%
\subsubsection{Andrew Norman, ``Sustain''}\label{andrew-norman-sustain}}

Deutsche Grammophon

\hypertarget{---18}{%
\subsubsection{◆ ◆ ◆}\label{---18}}

Advertisement

\protect\hyperlink{after-bottom}{Continue reading the main story}

\hypertarget{site-index}{%
\subsection{Site Index}\label{site-index}}

\hypertarget{site-information-navigation}{%
\subsection{Site Information
Navigation}\label{site-information-navigation}}

\begin{itemize}
\tightlist
\item
  \href{https://help.nytimes3xbfgragh.onion/hc/en-us/articles/115014792127-Copyright-notice}{©~2020~The
  New York Times Company}
\end{itemize}

\begin{itemize}
\tightlist
\item
  \href{https://www.nytco.com/}{NYTCo}
\item
  \href{https://help.nytimes3xbfgragh.onion/hc/en-us/articles/115015385887-Contact-Us}{Contact
  Us}
\item
  \href{https://www.nytco.com/careers/}{Work with us}
\item
  \href{https://nytmediakit.com/}{Advertise}
\item
  \href{http://www.tbrandstudio.com/}{T Brand Studio}
\item
  \href{https://www.nytimes3xbfgragh.onion/privacy/cookie-policy\#how-do-i-manage-trackers}{Your
  Ad Choices}
\item
  \href{https://www.nytimes3xbfgragh.onion/privacy}{Privacy}
\item
  \href{https://help.nytimes3xbfgragh.onion/hc/en-us/articles/115014893428-Terms-of-service}{Terms
  of Service}
\item
  \href{https://help.nytimes3xbfgragh.onion/hc/en-us/articles/115014893968-Terms-of-sale}{Terms
  of Sale}
\item
  \href{https://spiderbites.nytimes3xbfgragh.onion}{Site Map}
\item
  \href{https://help.nytimes3xbfgragh.onion/hc/en-us}{Help}
\item
  \href{https://www.nytimes3xbfgragh.onion/subscription?campaignId=37WXW}{Subscriptions}
\end{itemize}
