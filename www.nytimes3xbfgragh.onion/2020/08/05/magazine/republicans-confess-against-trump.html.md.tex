Sections

SEARCH

\protect\hyperlink{site-content}{Skip to
content}\protect\hyperlink{site-index}{Skip to site index}

\href{https://myaccount.nytimes3xbfgragh.onion/auth/login?response_type=cookie\&client_id=vi}{}

\href{https://www.nytimes3xbfgragh.onion/section/todayspaper}{Today's
Paper}

These Republicans Have a Confession: They're Not Voting for Trump Again

\url{https://nyti.ms/3fCGS6d}

\begin{itemize}
\item
\item
\item
\item
\item
\item
\end{itemize}

\begin{itemize}
\item
  \href{https://www.nytimes3xbfgragh.onion/live/2020/09/08/us/trump-vs-biden?action=click\&pgtype=Article\&state=default\&region=TOP_BANNER\&context=storylines_menu}{Election
  Updates}
\item
  \href{https://www.nytimes3xbfgragh.onion/interactive/2020/us/elections/election-states-biden-trump.html?action=click\&pgtype=Article\&state=default\&region=TOP_BANNER\&context=storylines_menu}{Paths
  to 270}
\item
  \href{https://www.nytimes3xbfgragh.onion/interactive/2020/08/31/us/politics/vote-by-mail-deadlines.html?action=click\&pgtype=Article\&state=default\&region=TOP_BANNER\&context=storylines_menu}{Voting
  by Mail}
\item
  \href{https://www.nytimes3xbfgragh.onion/interactive/2019/us/elections/2020-presidential-election-calendar.html?action=click\&pgtype=Article\&state=default\&region=TOP_BANNER\&context=storylines_menu}{Key
  Dates}
\item
  \href{https://www.nytimes3xbfgragh.onion/newsletters/politics?action=click\&pgtype=Article\&state=default\&region=TOP_BANNER\&context=storylines_menu}{Politics
  Newsletter}
\end{itemize}

Advertisement

\protect\hyperlink{after-top}{Continue reading the main story}

Supported by

\protect\hyperlink{after-sponsor}{Continue reading the main story}

\href{/column/screenland}{Screenland}

\hypertarget{these-republicans-have-a-confession-theyre-not-voting-for-trump-again}{%
\section{These Republicans Have a Confession: They're Not Voting for
Trump
Again}\label{these-republicans-have-a-confession-theyre-not-voting-for-trump-again}}

\includegraphics{https://static01.graylady3jvrrxbe.onion/images/2020/08/09/magazine/09mag-screenland/09mag-screenland-articleLarge.png?quality=75\&auto=webp\&disable=upscale}

By Jason Zengerle

\begin{itemize}
\item
  Published Aug. 5, 2020Updated Aug. 25, 2020
\item
  \begin{itemize}
  \item
  \item
  \item
  \item
  \item
  \item
  \end{itemize}
\end{itemize}

The man --- bearded, shirtless, a Marlboro Light clutched between two
fingers as it smolders uncomfortably close to his temple --- looks as if
he has something heavy he wants to get off his chest. Like a person
attending his first Alcoholics Anonymous meeting, he seems at once eager
and apprehensive. ``Hi, my name is Josh. I live in North Carolina, and I
voted for
\href{https://www.nytimes3xbfgragh.onion/2020/08/25/us/politics/trump-reelection-supporters.html}{Donald
Trump},'' he begins, in a tone of abject resignation. He cocks his head
and rolls his eyes. ``My bad, fam,'' he apologizes. ``Not my proudest
moment. I will not be voting for him again.''

The confession comes from Josh Harrison, a 40-year-old exterminator from
the Raleigh area, and it appears on the website and social media
platforms of a group called Republican Voters Against Trump. Created by
the conservative writer Bill Kristol and a handful of his fellow Never
Trump Republicans, RVAT, as its name indicates, is dedicated to
defeating the president this November. Toward that end, the group has
curated an online collection of more than 500 selfie videos from
Republicans, many of whom voted for Trump in 2016 and all of whom plan
to vote against him in 2020.

Credit...CreditVideo by Republican Voters Against Trump

Harrison recorded his confession in June, sitting on his back deck
around 2 in the morning, after consuming some White Claw and red wine.
``It's the first time I've ever voted for a Democrat,'' he says in the
video. ``But if Joe Biden drops out and the D.N.C. runs a tomato can, I
will vote for the tomato can, because I believe the tomato can will do
less harm than our current president.'' When Harrison sent the video,
unsolicited, to RVAT, he felt as if he were shouting into a void. But
since RVAT posted the video online, it has been viewed more than a
million times on the group's Twitter account, seen more than 100,000
times on its YouTube channel and received plenty of media attention.

The Never Trump Republican advertising space is a crowded one this
campaign. The Lincoln Project releases new spots seemingly every day ---
one blaming Trump for the pandemic, another claiming that he's seriously
ill, yet another intimating that his genitalia are small. But while the
slick Lincoln Project ads ``work exclusively on the predispositions of
the faithful,'' as
\href{https://www.theatlantic.com/ideas/archive/2020/06/tactics-lincoln-project/613636/}{Andrew
Ferguson has written in The Atlantic,} the bare-bones RVAT testimonials
are intended to do that rarest of things in politics these days:
persuade. And the method RVAT has chosen to persuade Republicans to vote
against Trump is an interesting one: These videos are the group's
attempt to help create a ``permission structure'' for voters to act in
ways they never expected.

The permission-structure strategy was used to great effect by Barack
Obama's old political strategist, David Axelrod. Before Axelrod went to
work for Obama, he cut his teeth helping to elect Black mayors in cities
like Cleveland, Detroit and Philadelphia. The key to winning those
races, which often featured multiple African-American candidates, was
attracting a sizable percentage of the white vote. To do that, Axelrod
spent a lot of time and effort working to win his Black clients what he
called ``third-party authentication'' --- endorsements from individuals
(like elected officials) and institutions (like newspaper editorial
boards) that white voters trusted to make safe, conventional decisions
about whom to vote for. Once Axelrod's Black candidates had those stamps
of mainstream validation, white voters believed they had permission to
vote for them.

Axelrod's track record of selling Black candidates to white voters is a
big reason Obama hired him to run his 2004 Senate campaign in Illinois.
In that race, Axelrod planned to use Paul Simon, a former Illinois
senator, as a third-party authenticator, but Simon died, suddenly,
before he could make an official endorsement. Instead, Axelrod filmed an
ad featuring Simon's daughter, Sheila, in which she said Obama and her
father were ``cut from the same cloth'' --- a powerful signal to the
rural white Illinoisans who had repeatedly cast votes to send Simon to
the Senate.

RVAT has taken Axelrod's strategy and updated it for our current
political moment --- in large part by inverting \emph{where} voters are
looking for permission. The group isn't seeking third-party
authentication from conservative institutions, or notable politicians,
or decorated military officials, or even former members of Trump's
administration --- Republicans' loss of faith in precisely those people
is why they voted for Trump in the first place. Instead of Mitt Romney
or The Weekly Standard-in-exile or William McRaven or John Bolton
telling Republicans that it's OK to vote against Trump, RVAT has turned
to Tom from Arizona (``I've been a Republican all my life, and this
November I'm voting for Joe Biden for president''), Kelly from Florida
(``Biden has my vote because we need to do whatever we can to get that
monster out of the White House'') and Josh from North Carolina to grant
permission. Scrolling through the testimonials on RVAT's website, the
message to Biden-curious Republicans is clear: You are not alone.

That sense of belonging, after all, was part of what propelled voters
into Trump's corner in 2016. They may not have seen many elected
officials or \emph{éminences grises} getting behind Trump, but they
didn't need to; it was enough to see their friends and neighbors, or
people who looked like their friends and neighbors, packing airplane
hangars or lining up outside arenas. Those crowds signaled to potential
Trump voters that the outré reality-TV star they liked watching in the
debates --- the one all the pundits dismissed as a novelty act --- was,
in fact, a realistic candidate to support.

As Axelrod's career attests, this kind of social permission isn't a rare
thing to try to offer voters. It's fascinating, though, to watch it
happen at a moment like this. Americans find themselves seeking
permission for a lot of actions these days, like abiding by (or
flouting) mask requirements and sending (or not sending) their children
to school. Things once viewed as inconceivable are now unavoidable;
things once taken as givens are now in doubt. The unfamiliarity of the
moment has also made its political possibilities seem endless, ranging
from drastic public-health and economic measures to aggressive changes
in policing.

When everything is abnormal, social guidance becomes all the more
powerful. That reassurance is what RVAT is trying to provide. In an era
of extreme polarization and negative partisanship --- one in which
political allegiances are determined less by affection for one party
than by hatred of the other --- the notion of a Republican voting for
Biden feels aberrant. But there's so much aberrant about America right
now that nothing, presented in the right voice by the right messenger,
seems especially outlandish. Not even voting for a tomato can.

\hypertarget{our-2020-election-guide}{%
\section{Our 2020 Election Guide}\label{our-2020-election-guide}}

Updated ~Sept. 8, 2020

\begin{center}\rule{0.5\linewidth}{\linethickness}\end{center}

\begin{itemize}
\item ~
  \hypertarget{the-latest}{%
  \subsection{The Latest}\label{the-latest}}

  \begin{itemize}
  \item
    The campaign
    \href{https://www.nytimes3xbfgragh.onion/live/2020/09/08/us/trump-vs-biden?action=click\&pgtype=Article\&state=default\&region=BELOW_MAIN_CONTENT\&context=storylines_guide}{shifts
    to a higher gear this week}, with President Trump set to visit
    Florida and North Carolina today and Joseph R. Biden heading to
    Michigan tomorrow.
  \end{itemize}
\item ~
  \hypertarget{how-to-win-270}{%
  \subsection{How to Win 270}\label{how-to-win-270}}

  \begin{itemize}
  \item
    Joe Biden and Donald Trump need 270 electoral votes to reach the
    White House. Try building
    \href{https://www.nytimes3xbfgragh.onion/interactive/2020/us/elections/election-states-biden-trump.html?action=click\&pgtype=Article\&state=default\&region=BELOW_MAIN_CONTENT\&context=storylines_guide}{your
    own coalition of battleground states}~to see potential outcomes.
  \end{itemize}
\item ~
  \hypertarget{voting-by-mail}{%
  \subsection{Voting by Mail}\label{voting-by-mail}}

  \begin{itemize}
  \item
    Will you have enough time to vote by mail in your state? Yes, but
    it's risky to procrastinate.
    \href{https://www.nytimes3xbfgragh.onion/interactive/2020/08/31/us/politics/vote-by-mail-deadlines.html?action=click\&pgtype=Article\&state=default\&region=BELOW_MAIN_CONTENT\&context=storylines_guide}{Check
    your state's deadline.}
  \item
    \href{https://www.nytimes3xbfgragh.onion/interactive/2020/us/elections/joe-biden.html?action=click\&pgtype=Article\&state=default\&region=BELOW_MAIN_CONTENT\&context=storylines_guide}{}

    \hypertarget{joe-biden}{%
    \section{Joe Biden}\label{joe-biden}}

    \hypertarget{democrat}{%
    \subsection{Democrat}\label{democrat}}

    \href{https://www.nytimes3xbfgragh.onion/interactive/2020/us/elections/donald-trump.html?action=click\&pgtype=Article\&state=default\&region=BELOW_MAIN_CONTENT\&context=storylines_guide}{}

    \hypertarget{donald-trump}{%
    \section{Donald Trump}\label{donald-trump}}

    \hypertarget{republican}{%
    \subsection{Republican}\label{republican}}
  \end{itemize}
\item
  \hypertarget{keep-up-with-our-coverage}{%
  \subsection{Keep Up With Our
  Coverage}\label{keep-up-with-our-coverage}}

  \begin{itemize}
  \item
    Get an
    \href{https://www.nytimes3xbfgragh.onion/newsletters/politics?action=click\&pgtype=Article\&state=default\&region=BELOW_MAIN_CONTENT\&context=storylines_guide}{email}~recapping
    the day's news
  \item
    Download our mobile app on
    \href{https://apps.apple.com/us/app/nytimes/id284862083?ls=1\&mat_click_id=5c79ae7455014fd1bd66b5610c05b8f2-20191112-16948\&referrer=mat_click_id\%3D5c79ae7455014fd1bd66b5610c05b8f2-20191112-16948\%26link_click_id\%3D722930677036718082}{iOS}~and
    \href{http://a.localytics.com/android?id=com.nytimes.android\&referrer=utm_source\%3Dother_nyt_mobile_web\%26utm_medium\%3DWeb\%2520page\%26utm_term\%3DGeneral\%2520Mobile\%2520Page\%26utm_campaign\%3DNYT\%2520Mobile\%2520General\%2520Page}{Android}~and
    turn on Breaking News and Politics alerts
  \end{itemize}
\end{itemize}

Advertisement

\protect\hyperlink{after-bottom}{Continue reading the main story}

\hypertarget{site-index}{%
\subsection{Site Index}\label{site-index}}

\hypertarget{site-information-navigation}{%
\subsection{Site Information
Navigation}\label{site-information-navigation}}

\begin{itemize}
\tightlist
\item
  \href{https://help.nytimes3xbfgragh.onion/hc/en-us/articles/115014792127-Copyright-notice}{©~2020~The
  New York Times Company}
\end{itemize}

\begin{itemize}
\tightlist
\item
  \href{https://www.nytco.com/}{NYTCo}
\item
  \href{https://help.nytimes3xbfgragh.onion/hc/en-us/articles/115015385887-Contact-Us}{Contact
  Us}
\item
  \href{https://www.nytco.com/careers/}{Work with us}
\item
  \href{https://nytmediakit.com/}{Advertise}
\item
  \href{http://www.tbrandstudio.com/}{T Brand Studio}
\item
  \href{https://www.nytimes3xbfgragh.onion/privacy/cookie-policy\#how-do-i-manage-trackers}{Your
  Ad Choices}
\item
  \href{https://www.nytimes3xbfgragh.onion/privacy}{Privacy}
\item
  \href{https://help.nytimes3xbfgragh.onion/hc/en-us/articles/115014893428-Terms-of-service}{Terms
  of Service}
\item
  \href{https://help.nytimes3xbfgragh.onion/hc/en-us/articles/115014893968-Terms-of-sale}{Terms
  of Sale}
\item
  \href{https://spiderbites.nytimes3xbfgragh.onion}{Site Map}
\item
  \href{https://help.nytimes3xbfgragh.onion/hc/en-us}{Help}
\item
  \href{https://www.nytimes3xbfgragh.onion/subscription?campaignId=37WXW}{Subscriptions}
\end{itemize}
