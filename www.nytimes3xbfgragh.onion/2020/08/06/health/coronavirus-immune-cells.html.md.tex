Sections

SEARCH

\protect\hyperlink{site-content}{Skip to
content}\protect\hyperlink{site-index}{Skip to site index}

\href{https://www.nytimes3xbfgragh.onion/section/health}{Health}

\href{https://myaccount.nytimes3xbfgragh.onion/auth/login?response_type=cookie\&client_id=vi}{}

\href{https://www.nytimes3xbfgragh.onion/section/todayspaper}{Today's
Paper}

\href{/section/health}{Health}\textbar{}The Coronavirus Is New, but Your
Immune System Might Still Recognize It

\url{https://nyti.ms/2XA5A0G}

\begin{itemize}
\item
\item
\item
\item
\item
\end{itemize}

\hypertarget{the-coronavirus-outbreak}{%
\subsubsection{\texorpdfstring{\href{https://www.nytimes3xbfgragh.onion/news-event/coronavirus?name=styln-coronavirus-national\&region=TOP_BANNER\&block=storyline_menu_recirc\&action=click\&pgtype=Article\&impression_id=a73ad0a0-f4bc-11ea-a147-e379c6aa6545\&variant=undefined}{The
Coronavirus
Outbreak}}{The Coronavirus Outbreak}}\label{the-coronavirus-outbreak}}

\begin{itemize}
\tightlist
\item
  live\href{https://www.nytimes3xbfgragh.onion/2020/09/11/world/covid-19-coronavirus.html?name=styln-coronavirus-national\&region=TOP_BANNER\&block=storyline_menu_recirc\&action=click\&pgtype=Article\&impression_id=a73af7b0-f4bc-11ea-a147-e379c6aa6545\&variant=undefined}{Latest
  Updates}
\item
  \href{https://www.nytimes3xbfgragh.onion/interactive/2020/us/coronavirus-us-cases.html?name=styln-coronavirus-national\&region=TOP_BANNER\&block=storyline_menu_recirc\&action=click\&pgtype=Article\&impression_id=a73af7b1-f4bc-11ea-a147-e379c6aa6545\&variant=undefined}{Maps
  and Cases}
\item
  \href{https://www.nytimes3xbfgragh.onion/interactive/2020/science/coronavirus-vaccine-tracker.html?name=styln-coronavirus-national\&region=TOP_BANNER\&block=storyline_menu_recirc\&action=click\&pgtype=Article\&impression_id=a73af7b2-f4bc-11ea-a147-e379c6aa6545\&variant=undefined}{Vaccine
  Tracker}
\item
  \href{https://www.nytimes3xbfgragh.onion/2020/09/10/us/politics/fda-coronavirus-vaccine.html?name=styln-coronavirus-national\&region=TOP_BANNER\&block=storyline_menu_recirc\&action=click\&pgtype=Article\&impression_id=a73af7b3-f4bc-11ea-a147-e379c6aa6545\&variant=undefined}{F.D.A.
  Regulators' Self-Defense}
\item
  \href{https://www.nytimes3xbfgragh.onion/2020/09/09/upshot/coronavirus-surprise-test-fees.html?name=styln-coronavirus-national\&region=TOP_BANNER\&block=storyline_menu_recirc\&action=click\&pgtype=Article\&impression_id=a73af7b4-f4bc-11ea-a147-e379c6aa6545\&variant=undefined}{Surprise
  Test Fees}
\end{itemize}

Advertisement

\protect\hyperlink{after-top}{Continue reading the main story}

Supported by

\protect\hyperlink{after-sponsor}{Continue reading the main story}

\hypertarget{the-coronavirus-is-new-but-your-immune-system-might-still-recognize-it}{%
\section{The Coronavirus Is New, but Your Immune System Might Still
Recognize
It}\label{the-coronavirus-is-new-but-your-immune-system-might-still-recognize-it}}

Some people carry immune cells called T cells that can capitalize on the
virus's resemblance to other members of its family tree.

\includegraphics{https://static01.graylady3jvrrxbe.onion/images/2020/08/11/science/06VIRUS-CROSS-REACTIVE1/merlin_175379085_78b2b3e8-142e-42c8-a9a1-d1d49b9a539c-articleLarge.jpg?quality=75\&auto=webp\&disable=upscale}

By
\href{https://www.nytimes3xbfgragh.onion/by/katherine-j--wu}{Katherine
J. Wu}

\begin{itemize}
\item
  Aug. 6, 2020
\item
  \begin{itemize}
  \item
  \item
  \item
  \item
  \item
  \end{itemize}
\end{itemize}

Eight months ago, the new coronavirus was unknown. But to some of our
immune cells, the virus was already something of a familiar foe.

A \href{https://immunology.sciencemag.org/content/5/48/eabd2071}{flurry}
of
\href{https://www.cell.com/cell/pdf/S0092-8674(20)30610-3.pdf}{recent}
\href{https://www.nature.com/articles/s41586-020-2550-z}{studies} has
revealed that a
\href{https://www.nature.com/articles/s41586-020-2598-9}{large
proportion} of the population ---
\href{https://www.nature.com/articles/s41577-020-0389-z.pdf}{20 to 50
percent} of people in some places --- might
\href{https://www.biorxiv.org/content/10.1101/2020.06.12.148916v1}{harbor
immunity assassins} called T cells that
\href{https://science.sciencemag.org/content/early/2020/08/04/science.abd3871}{recognize
the new coronavirus} despite having never encountered it before.

These T cells, which lurked in the bloodstreams of people long before
the pandemic began, are most likely stragglers from past scuffles with
other, related coronaviruses, including four that frequently cause
\href{https://www.ncbi.nlm.nih.gov/pmc/articles/PMC3416289/}{common
colds}. It's a case of family resemblance: In the eyes of the immune
system, germs with common roots can look alike, such that when a cousin
comes to call, the body may already have an inkling of its intentions.

The presence of these T cells has intrigued experts, who said it was too
soon to tell whether the cells would play a helpful, harmful or entirely
negligible role in the world's fight against the current coronavirus.
But should these so-called cross-reactive T cells exert even a modest
influence on the body's immune response to the new coronavirus, they
might make the disease milder --- and perhaps partly explain why some
people who
\href{https://www.nytimes3xbfgragh.onion/2020/08/04/health/coronavirus-immune-system.html}{catch
the germ} become very sick, while others are dealt only a glancing blow.

``If you have a population of T cells that are armed and ready to
protect you, you could control the infection better than someone who
doesn't have those cross-reactive cells,'' said Marion Pepper, an
immunologist at the University of Washington who is studying the immune
responses of Covid-19 patients. ``That's what we're all hoping for.''

T cells are an exceptionally picky bunch. Each spends the entirety of
its life waiting for a very specific trigger, like a hunk of a dangerous
virus. Once that switch is flipped, the T cell will clone itself into an
army of specialized soldiers, all with their sights set on the same
target. Some T cells are microscopic assassins, tailor-made to home in
on and destroy infected cells; others coax immune cells called B cells
into producing virus-attacking antibodies.

The first time a virus infects the body, this response is sluggish; it
takes several days for the immune system to sort out which T cells are
best suited for the job at hand. But subsequent encounters typically
prompt a response that is stronger and faster, thanks to a reserve force
of T cells, called memory T cells, that lingers after the initial threat
has passed and can quickly be called into action again.

\hypertarget{latest-updates-the-coronavirus-outbreak}{%
\section{\texorpdfstring{\href{https://www.nytimes3xbfgragh.onion/2020/09/11/world/covid-19-coronavirus.html?action=click\&pgtype=Article\&state=default\&region=MAIN_CONTENT_1\&context=storylines_live_updates}{Latest
Updates: The Coronavirus
Outbreak}}{Latest Updates: The Coronavirus Outbreak}}\label{latest-updates-the-coronavirus-outbreak}}

Updated 2020-09-12T05:29:13.829Z

\begin{itemize}
\tightlist
\item
  \href{https://www.nytimes3xbfgragh.onion/2020/09/11/world/covid-19-coronavirus.html?action=click\&pgtype=Article\&state=default\&region=MAIN_CONTENT_1\&context=storylines_live_updates\#link-dfb8a16}{Fauci
  cautions the virus could disrupt life in the U.S. until `maybe even
  towards the end of 2021.'}
\item
  \href{https://www.nytimes3xbfgragh.onion/2020/09/11/world/covid-19-coronavirus.html?action=click\&pgtype=Article\&state=default\&region=MAIN_CONTENT_1\&context=storylines_live_updates\#link-7104d154}{From
  Asia to Africa, China promotes its vaccine candidates to win friends.}
\item
  \href{https://www.nytimes3xbfgragh.onion/2020/09/11/world/covid-19-coronavirus.html?action=click\&pgtype=Article\&state=default\&region=MAIN_CONTENT_1\&context=storylines_live_updates\#link-393ad215}{The
  other way the virus will kill: hunger.}
\end{itemize}

\href{https://www.nytimes3xbfgragh.onion/2020/09/11/world/covid-19-coronavirus.html?action=click\&pgtype=Article\&state=default\&region=MAIN_CONTENT_1\&context=storylines_live_updates}{See
more updates}

More live coverage:
\href{https://www.nytimes3xbfgragh.onion/live/2020/09/11/business/stock-market-today-coronavirus?action=click\&pgtype=Article\&state=default\&region=MAIN_CONTENT_1\&context=storylines_live_updates}{Markets}

Usually, this process operates best when T cells must battle the same
pathogen again and again. But these recruits are more flexible than they
are often given credit for, said Laura Su, an immunologist and T cell
expert at the University of Pennsylvania. Should these cells chance upon
something that bears a strong resemblance to their germ of choice, they
can still be roused to fight, even if the invader is a total newcomer.

In theory, cross-reactive T cells can ``protect almost like a vaccine,''
said Smita Iyer, an immunologist at the University of California, Davis,
who is studying immune responses to the new coronavirus in primates.
Previous studies have shown that cross-reactive T cells may guard people
against
\href{https://www.ncbi.nlm.nih.gov/pmc/articles/PMC6303473/}{different
strains} of
\href{https://www.ncbi.nlm.nih.gov/pmc/articles/PMC6458262/}{the flu
virus}, and perhaps confer a trace of immunity against
\href{https://www.frontiersin.org/articles/10.3389/fimmu.2020.00517/full}{dengue
and Zika viruses}, which share a family tree.

The case for coronaviruses is less clear cut, said Alessandro Sette, an
immunologist at the La Jolla Institute for Immunology who has led
several studies examining cross-reactive T cells to the new coronavirus.
Researchers have found people in the United States, Germany, the
Netherlands, Singapore and the United Kingdom who have never been
exposed to the new coronavirus but who carry T cells that react to it in
the lab.

Researchers are eager to understand the history of these T cells,
because that might help reveal who is more likely to have them. A
growing body of evidence, including data published
\href{https://science.sciencemag.org/content/early/2020/08/04/science.abd3871}{this
week in Science} by Dr. Sette and his colleagues, points to common-cold
coronaviruses as a potential source. But even unrelated viruses can
share similar features, and researchers may never know for sure what
originally ``drove their development,'' said Avery August, an
immunologist and T cell expert at Cornell University.

\includegraphics{https://static01.graylady3jvrrxbe.onion/images/2020/08/11/science/06VIRUS-CROSS-REACTIVE2/merlin_175379139_2f6c35a1-261a-4dc8-9734-78c029242c1c-articleLarge.jpg?quality=75\&auto=webp\&disable=upscale}

Whatever the origin of T cells, their mere existence could be
encouraging news. There is much more to the immune system than T cells,
but even a semblance of pre-existing immunity could mean that people who
have recently grappled with the common cold may have an easier time
fighting off a nastier member of the coronavirus clan.

Cross-reactive T cells alone probably would not be enough to completely
stave off infection or disease. But they might alleviate symptoms of the
coronavirus in people who happen to carry these cells, or extend the
protection provided by a vaccine.

``That would be awesome,'' Dr. Iyer said.

Children, who share lots of germs with their peers, might be good
candidates for this hypothetical scenario.

But cross-reactive T cells are not necessarily a benevolent force. They
could instead be ineffectual souvenirs of infections past, with
``absolutely no relevance'' to how well people fare against the new
coronavirus, Dr. Sette said.

There is even a small chance that pre-existing T cells could raise the
risk for serious symptoms of Covid-19, although experts consider this
possibility unlikely. T cells that are primed to recognize common-cold
coronaviruses might marshal only a lackluster response to the current
coronavirus, potentially sapping resources from other populations of
immune cells that have a better shot at defeating the new invader. ``Now
you have your immune system distracted,'' Dr. Iyer said.

T cells are also expert orchestrators. Depending on the signals they
send out, they can synchronize cells and molecules from disparate parts
of the immune system into a tag-teamed attack, or quell these assaults
to return the body to baseline. If it turns out that cross-reactive T
cells tend toward quieting the response, they could suppress a person's
immune defense before it has a chance to kick into gear, Dr. August
said.

Then again, many types of T cells exist, and all operate as part of a
complex immune system. ``It's almost like some people are trying to say
this is `good' or `bad,''' Dr. Su said. ``It's probably more nuanced
than that.''

Teasing it all apart will not be easy. Unlike antibodies, which are
inanimate proteins that often circulate in the blood, T cells are living
cells that often hole up in hard-to-reach tissues. That makes them much
more difficult to extract, maintain and analyze, Dr. Pepper said.

Researchers could learn more by testing whether cross-reactive T cells
are more abundant in patients who have had mild or serious cases of
Covid-19, although such studies cannot prove cause and effect. A more
laborious effort might involve measuring cross-reactive T cell levels in
large groups of healthy people, then waiting to see if they became
infected or sick from the current coronavirus, Dr. Sette said.

Strong evidence could also come from an animal model, like the rhesus
macaques that Dr. Iyer studies in her lab. Researchers could dose
primates with common-cold coronaviruses, and then see how their immune
responses stack up against the new coronavirus.

Less than a year into this pandemic, plenty of questions remain
unanswered, Dr. Pepper said. Immunologists cannot fully forecast how the
human immune system will respond to this new virus; even with science at
its speediest, that interaction must be studied in real time.

It's a frustrating reality, Dr. Pepper said: ``Until we see it in real
life, we just don't know.''

Advertisement

\protect\hyperlink{after-bottom}{Continue reading the main story}

\hypertarget{site-index}{%
\subsection{Site Index}\label{site-index}}

\hypertarget{site-information-navigation}{%
\subsection{Site Information
Navigation}\label{site-information-navigation}}

\begin{itemize}
\tightlist
\item
  \href{https://help.nytimes3xbfgragh.onion/hc/en-us/articles/115014792127-Copyright-notice}{©~2020~The
  New York Times Company}
\end{itemize}

\begin{itemize}
\tightlist
\item
  \href{https://www.nytco.com/}{NYTCo}
\item
  \href{https://help.nytimes3xbfgragh.onion/hc/en-us/articles/115015385887-Contact-Us}{Contact
  Us}
\item
  \href{https://www.nytco.com/careers/}{Work with us}
\item
  \href{https://nytmediakit.com/}{Advertise}
\item
  \href{http://www.tbrandstudio.com/}{T Brand Studio}
\item
  \href{https://www.nytimes3xbfgragh.onion/privacy/cookie-policy\#how-do-i-manage-trackers}{Your
  Ad Choices}
\item
  \href{https://www.nytimes3xbfgragh.onion/privacy}{Privacy}
\item
  \href{https://help.nytimes3xbfgragh.onion/hc/en-us/articles/115014893428-Terms-of-service}{Terms
  of Service}
\item
  \href{https://help.nytimes3xbfgragh.onion/hc/en-us/articles/115014893968-Terms-of-sale}{Terms
  of Sale}
\item
  \href{https://spiderbites.nytimes3xbfgragh.onion}{Site Map}
\item
  \href{https://help.nytimes3xbfgragh.onion/hc/en-us}{Help}
\item
  \href{https://www.nytimes3xbfgragh.onion/subscription?campaignId=37WXW}{Subscriptions}
\end{itemize}
