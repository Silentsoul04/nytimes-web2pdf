Sections

SEARCH

\protect\hyperlink{site-content}{Skip to
content}\protect\hyperlink{site-index}{Skip to site index}

\href{https://www.nytimes3xbfgragh.onion/section/health}{Health}

\href{https://myaccount.nytimes3xbfgragh.onion/auth/login?response_type=cookie\&client_id=vi}{}

\href{https://www.nytimes3xbfgragh.onion/section/todayspaper}{Today's
Paper}

\href{/section/health}{Health}\textbar{}`It's Kitchen Sink Time': Fast,
Less-Accurate Coronavirus Tests May Be Good Enough

\url{https://nyti.ms/3fA2tMg}

\begin{itemize}
\item
\item
\item
\item
\item
\item
\end{itemize}

\hypertarget{the-coronavirus-outbreak}{%
\subsubsection{\texorpdfstring{\href{https://www.nytimes3xbfgragh.onion/news-event/coronavirus?name=styln-coronavirus-national\&region=TOP_BANNER\&block=storyline_menu_recirc\&action=click\&pgtype=Article\&impression_id=372474b0-f4b7-11ea-8a5e-83a68a0b9550\&variant=undefined}{The
Coronavirus
Outbreak}}{The Coronavirus Outbreak}}\label{the-coronavirus-outbreak}}

\begin{itemize}
\tightlist
\item
  live\href{https://www.nytimes3xbfgragh.onion/2020/09/11/world/covid-19-coronavirus.html?name=styln-coronavirus-national\&region=TOP_BANNER\&block=storyline_menu_recirc\&action=click\&pgtype=Article\&impression_id=372474b1-f4b7-11ea-8a5e-83a68a0b9550\&variant=undefined}{Latest
  Updates}
\item
  \href{https://www.nytimes3xbfgragh.onion/interactive/2020/us/coronavirus-us-cases.html?name=styln-coronavirus-national\&region=TOP_BANNER\&block=storyline_menu_recirc\&action=click\&pgtype=Article\&impression_id=37249bc0-f4b7-11ea-8a5e-83a68a0b9550\&variant=undefined}{Maps
  and Cases}
\item
  \href{https://www.nytimes3xbfgragh.onion/interactive/2020/science/coronavirus-vaccine-tracker.html?name=styln-coronavirus-national\&region=TOP_BANNER\&block=storyline_menu_recirc\&action=click\&pgtype=Article\&impression_id=37249bc1-f4b7-11ea-8a5e-83a68a0b9550\&variant=undefined}{Vaccine
  Tracker}
\item
  \href{https://www.nytimes3xbfgragh.onion/2020/09/10/us/politics/fda-coronavirus-vaccine.html?name=styln-coronavirus-national\&region=TOP_BANNER\&block=storyline_menu_recirc\&action=click\&pgtype=Article\&impression_id=37249bc2-f4b7-11ea-8a5e-83a68a0b9550\&variant=undefined}{F.D.A.
  Regulators' Self-Defense}
\item
  \href{https://www.nytimes3xbfgragh.onion/2020/09/09/upshot/coronavirus-surprise-test-fees.html?name=styln-coronavirus-national\&region=TOP_BANNER\&block=storyline_menu_recirc\&action=click\&pgtype=Article\&impression_id=37249bc3-f4b7-11ea-8a5e-83a68a0b9550\&variant=undefined}{Surprise
  Test Fees}
\end{itemize}

Advertisement

\protect\hyperlink{after-top}{Continue reading the main story}

Supported by

\protect\hyperlink{after-sponsor}{Continue reading the main story}

\hypertarget{its-kitchen-sink-time-fast-less-accurate-coronavirus-tests-may-be-good-enough}{%
\section{`It's Kitchen Sink Time': Fast, Less-Accurate Coronavirus Tests
May Be Good
Enough}\label{its-kitchen-sink-time-fast-less-accurate-coronavirus-tests-may-be-good-enough}}

Experts are revising their views on the best methods to detect
infections, setting aside long-held standards so that the spread of the
virus can be more quickly tracked and contained.

\includegraphics{https://static01.graylady3jvrrxbe.onion/images/2020/08/05/science/05VIRUS-TESTING1/merlin_175307841_d1dd6361-2c24-4bcb-a0ca-0936975e4412-articleLarge.jpg?quality=75\&auto=webp\&disable=upscale}

\href{https://www.nytimes3xbfgragh.onion/by/katherine-j--wu}{\includegraphics{https://static01.graylady3jvrrxbe.onion/images/2020/08/11/reader-center/author-katherine-j-wu/author-katherine-j-wu-thumbLarge.png}}

By
\href{https://www.nytimes3xbfgragh.onion/by/katherine-j--wu}{Katherine
J. Wu}

\begin{itemize}
\item
  Published Aug. 6, 2020Updated Aug. 7, 2020
\item
  \begin{itemize}
  \item
  \item
  \item
  \item
  \item
  \item
  \end{itemize}
\end{itemize}

For months, the call for
\href{https://www.nytimes3xbfgragh.onion/2020/08/15/us/coronavirus-testing-decrease.html}{coronavirus
testing} has been led by one resounding refrain: To keep outbreaks under
control, doctors and researchers need to deploy the most accurate tests
available --- ones reliable enough to root out as many infections as
possible, even in the absence of symptoms.

That's long been the dogma of infectious disease diagnostics, experts
say, since it helps ensure that cases won't be missed. During this
pandemic, that has meant relying heavily on PCR testing, an extremely
accurate but time- and labor-intensive method that requires samples to
be processed at laboratories.

But as the virus continues its rampage across the country and
\href{https://www.nytimes3xbfgragh.onion/2020/07/19/health/coronavirus-testing-viral-spread.html}{tests
remain in short supply} in many regions, researchers and public health
experts have grown increasingly vocal about revising this long-held
credo. The best chance to rein in the sprawling outbreaks in the United
States now, experts say, requires widespread adoption of less accurate
tests, as long as they're
\href{https://www.nytimes3xbfgragh.onion/2020/07/03/opinion/coronavirus-tests.html}{administered
quickly and often enough}.

``Even if you miss somebody on Day 1,'' said Omai Garner, director of
clinical microbiology in the U.C.L.A. Health System. ``If you test them
repeatedly, the argument is, you'll catch them the next time around.''

This quantity-over-quality strategy has its downsides, and is contingent
on an enormous supply of testing kits. But many experts believe more
rapid, frequent testing would identify those who need immediate medical
care --- and perhaps even pinpoint those at greatest risk of spreading
the disease.

Such a considerable shift would likely be a welcome change for a country
where the status quo of testing was just described as ``unacceptable,
period'' by Dr. Anthony Fauci, the director of the National Institute of
Allergy and Infectious Diseases, in an interview Wednesday on CNN.

``If you had asked me this a couple months ago, I would have said we
just need to be doing the PCR tests,'' said Susan Butler-Wu, a clinical
microbiologist at the University of Southern California. ``But we are so
far gone in this country. It is a catastrophe. It's kitchen sink time,
even if the tests are imperfect.''

Of the dozens of coronavirus tests that have been granted emergency use
authorization by the Food and Drug Administration, most rely on complex
laboratory procedures, like PCR, to detect the coronavirus's genetic
material.

Only a handful are quick and simple enough to be run in what is called a
point-of-care setting, like a doctor's office or urgent care clinic,
without the need for lab equipment. And these tests are still relatively
scarce nationwide. Government officials have pledged to
\href{https://www.nih.gov/news-events/news-releases/nih-delivering-new-covid-19-testing-technologies-meet-us-demand}{astronomically
scale up the number of point-of-care tests} by fall, increasing by
millions the weekly tally of tests conducted.

But many of the rapid tests at the center of this buzz still rely on
specialized machines that are neither cheap nor easy to produce in bulk.
And while some rapid tests, like the
\href{https://www.nytimes3xbfgragh.onion/2020/07/06/health/fast-coronavirus-tests.html}{Abbott
ID Now}, were
\href{https://www.nytimes3xbfgragh.onion/interactive/2020/05/12/us/coronavirus-testing-white-house.html}{quickly
adopted into the White House}, most are unlikely to get much play in the
larger community.

\hypertarget{latest-updates-the-coronavirus-outbreak}{%
\section{\texorpdfstring{\href{https://www.nytimes3xbfgragh.onion/2020/09/11/world/covid-19-coronavirus.html?action=click\&pgtype=Article\&state=default\&region=MAIN_CONTENT_1\&context=storylines_live_updates}{Latest
Updates: The Coronavirus
Outbreak}}{Latest Updates: The Coronavirus Outbreak}}\label{latest-updates-the-coronavirus-outbreak}}

Updated 2020-09-12T04:56:54.924Z

\begin{itemize}
\tightlist
\item
  \href{https://www.nytimes3xbfgragh.onion/2020/09/11/world/covid-19-coronavirus.html?action=click\&pgtype=Article\&state=default\&region=MAIN_CONTENT_1\&context=storylines_live_updates\#link-dfb8a16}{Fauci
  cautions the virus could disrupt life in the U.S. until `maybe even
  towards the end of 2021.'}
\item
  \href{https://www.nytimes3xbfgragh.onion/2020/09/11/world/covid-19-coronavirus.html?action=click\&pgtype=Article\&state=default\&region=MAIN_CONTENT_1\&context=storylines_live_updates\#link-7104d154}{From
  Asia to Africa, China promotes its vaccine candidates to win friends.}
\item
  \href{https://www.nytimes3xbfgragh.onion/2020/09/11/world/covid-19-coronavirus.html?action=click\&pgtype=Article\&state=default\&region=MAIN_CONTENT_1\&context=storylines_live_updates\#link-393ad215}{The
  other way the virus will kill: hunger.}
\end{itemize}

\href{https://www.nytimes3xbfgragh.onion/2020/09/11/world/covid-19-coronavirus.html?action=click\&pgtype=Article\&state=default\&region=MAIN_CONTENT_1\&context=storylines_live_updates}{See
more updates}

More live coverage:
\href{https://www.nytimes3xbfgragh.onion/live/2020/09/11/business/stock-market-today-coronavirus?action=click\&pgtype=Article\&state=default\&region=MAIN_CONTENT_1\&context=storylines_live_updates}{Markets}

``We can't have these in every household,'' said Michael Mina, an
epidemiologist at Harvard University's School of Public Health and a
\href{https://www.nytimes3xbfgragh.onion/2020/07/03/opinion/coronavirus-tests.html}{vocal
proponent of speedy testing}.

\includegraphics{https://static01.graylady3jvrrxbe.onion/images/2020/08/05/science/05VIRUS-TESTING2/05VIRUS-TESTING2-articleLarge.jpg?quality=75\&auto=webp\&disable=upscale}

A better option, Dr. Mina said, might be antigen testing, which
identifies pieces of protein. Two such tests, made by BD and Quidel,
have received emergency authorization from the F.D.A. Both require
instruments to run, but even simpler versions of this technology could
provide a fast answer in the same way as pregnancy tests, allowing users
to swab their mouths or noses or spit into a tube, then read the results
as colored bars on a strip of paper within minutes.

These tests could be done at a doctor's office, or even at home --- no
fancy machines or specially trained personnel required --- and cost just
a few dollars a pop, perhaps even less. And there would be no delays of
a week or longer. Companies like the Massachusetts-based E25Bio are
currently cooking up tests that might fit this need.

Such convenient setups could resolve some of the
\href{https://www.nytimes3xbfgragh.onion/2020/07/23/health/coronavirus-testing-supply-shortage.html}{supply
shortages that have plagued testing laboratories across the nation for
months} and caused a national embarrassment over inadequate,
inaccessible testing. In many cities, people are still experiencing
turnaround times of over a week, sometimes three or more --- as people
did at the beginning of the U.S. epidemics --- for results from
PCR-based tests, effectively rendering them useless for themselves and
the people around them.

Hamstrung by a long delay, even an accurate test can morph into a
useless one.

Natalie Magnus, who got tested in Winnebago County, Ill., on July 14,
still didn't have results 22 days later. Her brother and sister-in-law,
who were each tested twice at separate facilities in Colorado on July 7
and July 8, have so far received only one set of results after a 17-day
wait. One of them was positive for the coronavirus.

Ms. Magnus, who has already completed a two-week quarantine at home, no
longer cares if she gets her results. ``By now, that doesn't tell me
anything,'' she said.

Antigen tests, on the other hand, can be low-tech and easy to
manufacture en masse. Distributed weekly or even daily, they could
painlessly screen people headed back into offices, schools and
universities, delivering peace of mind to parents, teachers and
employers. Everyone --- not just those feeling ill --- would have an
easier way to assess their health status on a regular basis.

``The goal here is to detect as many infections as possible,'' said
Daniel Larremore, an applied mathematician who models infectious
diseases at the University of Colorado, Boulder. ``That means taking as
many shots on goal as we can.''

Broad distribution of antigen tests might also ease the demand for PCR,
which still has an important role in hospitals and vulnerable
communities like nursing homes. As things stand, many doctors still
can't get their patients' results within 24 hours.

In those settings, accuracy is crucial, said Melissa Miller, director of
the Clinical Molecular Microbiology Laboratory for U.N.C. Hospitals.
``You don't want to miss that diagnosis.''

There are drawbacks. Antigen tests will miss some people who would test
positive by PCR, which amplifies coronavirus RNA and picks up even tiny
amounts of it. With antigen tests, proteins can't be copied in the same
way, making it easier to mistake some people who are carrying minute
levels of the coronavirus in their bodies as pathogen-free. Some antigen
tests used in the past
\href{https://slack-redir.net/link?url=https\%3A\%2F\%2Fwww.cdc.gov\%2Fflu\%2Fprofessionals\%2Fdiagnosis\%2Fclinician_guidance_ridt.htm}{miss
up to half the infections they look for}.

Early on, many experts balked because of these shortcomings. ``As
laboratorians, we wanted the most sensitive test, the most specific
test, the most accurate test,'' Dr. Miller said. ``End of story.''

But Dr. Mina argues that false negatives might not be as bad as they
seem.

Virus levels vary from person to person, and can wax and wane in an
individual over the course of an infection, typically peaking around the
time symptoms first appear. People carrying --- and thus probably
shedding --- gobs of germs will most likely turn up positive using every
test on the market, Dr. Mina said. Those harboring less virus in their
bodies might get more mixed results. Many of these individuals, however,
probably aren't the cases of most concern.

It's another way to look at testing accuracy, Dr. Mina said: ``Detecting
the most infectious people.''

Researchers don't yet know how much virus a person needs to carry in
their body to actually transmit it. But the range in which the accuracy
of antigen tests starts to drop off is probably far below that level,
Dr. Mina said.

Testing frequency can also be a formidable foe to disease transmission.
In a recent
\href{https://www.medrxiv.org/content/10.1101/2020.06.22.20136309v2}{paper}
that has yet to be published in a peer-reviewed scientific journal, he
and Dr. Larremore showed through mathematical models that a relatively
insensitive test, rolled out twice a week, vastly outperformed a more
accurate test, administered once every two weeks, in curbing the spread
of disease. Other studies pitting speed against sensitivity have
\href{https://jamanetwork.com/journals/jamanetworkopen/fullarticle/2768923}{come
to similar conclusions}.

The upshot here is a practical one, Dr. Garner said. ``You're not trying
to find every single person who has the virus,'' he said. ``You're
trying to mitigate outbreaks.''

That approach is a substantial departure from the way that many lab
researchers have traditionally tackled infectious disease testing.

``We're in sort of a new era of testing,'' said Esther Babady, a
clinical microbiologist at Memorial Sloan Kettering Cancer Center.
``Usually we use tests to diagnose disease states in patients, not look
for disease states. Now, with
\href{https://www.nytimes3xbfgragh.onion/2020/08/07/us/covid-test-accuracy-governor-dewine-ohio.html}{Covid},
we are starting to look for this virus everywhere.''

Image

A coronavirus drive-through test site in Miami-Dade County last
month.Credit...Joe Raedle/Getty Images

A testing rethink this substantial will inevitably come with snags. The
success of the speedy testing strategy hinges heavily on availability
--- something the United States has utterly failed at since the virus
first made landfall within its borders. Ramping up antigen testing may
only add strain to an already sputtering supply chain, potentially
hampering plans for widespread use.

``If you test people all the time, you can account for lack of
sensitivity,'' Dr. Butler-Wu said. ``But are you really going to test
people all the time? If the answer is no, then that lack of sensitivity
becomes more of a big deal.''

And many experts are still hesitant to trust negative antigen results,
which may need to be backed up with a more sensitive test. That might
not be very burdensome in the midst of an outbreak, when positivity
rates are likely to be high, Dr. Babady said. In spots where infection
rates are especially low, however, ``that is not going to be the best
approach to testing,'' she said.

Cheap tests also aren't the same thing as free tests. Even \$1 tests can
start to rack up quite a bill, especially for large families hoping to
do daily checkups
\href{https://www.nytimes3xbfgragh.onion/2020/06/09/health/testing-coronavirus-nursing-homes-workers.html}{or
nursing home aides required to get tested repeatedly.} Left unregulated,
the testing market could end up exacerbating the socioeconomic split
that's already disproportionately burdened some sectors of the
population with the worst effects of Covid-19.

Concerns over accuracy bogged down the approval process for simple,
speedy tests. F.D.A. guidelines stipulate that any new coronavirus test
vying for emergency clearance from the agency
\href{https://www.fda.gov/news-events/press-announcements/coronavirus-covid-19-update-fda-posts-new-template-home-and-over-counter-diagnostic-tests-use-non}{must
perform nearly as well} as the gold standard of PCR.

The F.D.A.'s rules frustrate Dr. Mina, who noted that several companies
on the verge of debuting antigen tests have found the regulatory hurdles
daunting. ``Many of them are not even bothering,'' he said. ```Our
product might not be good enough. So what's the point?'''

That's left the onus on the few companies who already have the F.D.A.'s
green light. In hopes of encouraging a speedier, ramped up production,
the governors of seven states announced this week a joint bid to
\href{https://governor.maryland.gov/2020/08/04/governors-of-maryland-louisiana-massachusetts-michigan-ohio-and-virginia-announce-major-bipartisan-interstate-compact-for-three-million-rapid-antigen-tests/}{purchase
a total of 3.5 million antigen tests from BD and Quidel}.

There probably isn't one way to grapple with all these issues --- and
certainly not an obvious one, Dr. Butler-Wu said. What's clear to her
and others, though, is that the current situation is untenable.

``Our backs are against the wall, and it's Hail Mary time,'' Dr.
Butler-Wu said. ``We have to try something different.''

Advertisement

\protect\hyperlink{after-bottom}{Continue reading the main story}

\hypertarget{site-index}{%
\subsection{Site Index}\label{site-index}}

\hypertarget{site-information-navigation}{%
\subsection{Site Information
Navigation}\label{site-information-navigation}}

\begin{itemize}
\tightlist
\item
  \href{https://help.nytimes3xbfgragh.onion/hc/en-us/articles/115014792127-Copyright-notice}{©~2020~The
  New York Times Company}
\end{itemize}

\begin{itemize}
\tightlist
\item
  \href{https://www.nytco.com/}{NYTCo}
\item
  \href{https://help.nytimes3xbfgragh.onion/hc/en-us/articles/115015385887-Contact-Us}{Contact
  Us}
\item
  \href{https://www.nytco.com/careers/}{Work with us}
\item
  \href{https://nytmediakit.com/}{Advertise}
\item
  \href{http://www.tbrandstudio.com/}{T Brand Studio}
\item
  \href{https://www.nytimes3xbfgragh.onion/privacy/cookie-policy\#how-do-i-manage-trackers}{Your
  Ad Choices}
\item
  \href{https://www.nytimes3xbfgragh.onion/privacy}{Privacy}
\item
  \href{https://help.nytimes3xbfgragh.onion/hc/en-us/articles/115014893428-Terms-of-service}{Terms
  of Service}
\item
  \href{https://help.nytimes3xbfgragh.onion/hc/en-us/articles/115014893968-Terms-of-sale}{Terms
  of Sale}
\item
  \href{https://spiderbites.nytimes3xbfgragh.onion}{Site Map}
\item
  \href{https://help.nytimes3xbfgragh.onion/hc/en-us}{Help}
\item
  \href{https://www.nytimes3xbfgragh.onion/subscription?campaignId=37WXW}{Subscriptions}
\end{itemize}
