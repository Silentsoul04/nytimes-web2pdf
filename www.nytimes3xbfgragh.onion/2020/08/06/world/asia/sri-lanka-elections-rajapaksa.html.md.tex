Sections

SEARCH

\protect\hyperlink{site-content}{Skip to
content}\protect\hyperlink{site-index}{Skip to site index}

\href{https://www.nytimes3xbfgragh.onion/section/world/asia}{Asia
Pacific}

\href{https://myaccount.nytimes3xbfgragh.onion/auth/login?response_type=cookie\&client_id=vi}{}

\href{https://www.nytimes3xbfgragh.onion/section/todayspaper}{Today's
Paper}

\href{/section/world/asia}{Asia Pacific}\textbar{}Sri Lanka Election
Hands Rajapaksa Family a Bigger Slice of Control

\url{https://nyti.ms/3ihgdxc}

\begin{itemize}
\item
\item
\item
\item
\item
\end{itemize}

Advertisement

\protect\hyperlink{after-top}{Continue reading the main story}

Supported by

\protect\hyperlink{after-sponsor}{Continue reading the main story}

\hypertarget{sri-lanka-election-hands-rajapaksa-family-a-bigger-slice-of-control}{%
\section{Sri Lanka Election Hands Rajapaksa Family a Bigger Slice of
Control}\label{sri-lanka-election-hands-rajapaksa-family-a-bigger-slice-of-control}}

President Gotabaya Rajapaksa's party clinched the majority of seats in
Parliament, aiming to amend the Constitution and expand presidential
powers. Rights groups are concerned.

\includegraphics{https://static01.graylady3jvrrxbe.onion/images/2020/08/07/world/07srilanka-1/merlin_175409814_bd17c6ab-3b72-48c0-9857-758e2318e7b3-articleLarge.jpg?quality=75\&auto=webp\&disable=upscale}

\href{https://www.nytimes3xbfgragh.onion/by/maria-abi-habib}{\includegraphics{https://static01.graylady3jvrrxbe.onion/images/2018/10/08/multimedia/author-maria-abi-habib/author-maria-abi-habib-thumbLarge.png}}

By \href{https://www.nytimes3xbfgragh.onion/by/maria-abi-habib}{Maria
Abi-Habib}

\begin{itemize}
\item
  Aug. 6, 2020
\item
  \begin{itemize}
  \item
  \item
  \item
  \item
  \item
  \end{itemize}
\end{itemize}

President Gotabaya Rajapaksa's governing party won a majority of Sri
Lanka's parliamentary seats, the Election Commission announced Friday,
bringing the president a step closer in his quest to amend the
Constitution and expand his executive power.

The Sri Lanka People's Front expanded its majority, winning 145 of the
150 seats needed to push through expanded powers for the president and
ensure that Mr. Rajapaksa's older brother, Mahinda, will continue as
prime minister. The People's Front is expected to easily form an
alliance with another party to secure the extra five seats they need to
declare a super majority with the power to amend the Constitution.

After Mahinda Rajapaksa lost the presidential election in 2015, the new
government passed a constitutional amendment that imposed a two-term
limit on the presidency, revoked immunity from prosecution and made
presidential appointments subject to parliamentary oversight.

Rights groups now worry that the constitutional amendment Mr. Rajapaksa
seeks would undo those reforms at a time when the country's opposition,
activists and news media
\href{https://www.nytimes3xbfgragh.onion/2019/11/27/world/asia/sri-lanka-rajapaksa-crackdown.html}{accuse
the government of censorship and intensifying intimidation of critics}.

The political dynasty was catapulted back into power after a five-year
stint as the opposition when Mr. Rajapaksa won the presidential
elections last year and promptly appointed his brother as prime
minister. Mahinda Rajapaksa was president of Sri Lanka from 2005 until
\href{https://www.nytimes3xbfgragh.onion/2015/01/09/world/asia/sri-lanka-election-president-mahinda-rajapaksa.html}{losing
elections in 2015}, and Gotabaya Rajapaksa served as defense secretary
through that time, overseeing the last phases of the country's deadly
civil war that ended in 2009.

\includegraphics{https://static01.graylady3jvrrxbe.onion/images/2020/08/06/world/06srilanka-election-3/merlin_175322193_c63b70db-ca8e-4f34-b753-a13d12ccf16e-articleLarge.jpg?quality=75\&auto=webp\&disable=upscale}

During its last stint in power, the Rajapaksa government was both
beloved and reviled, celebrated and feared. To their supporters, the
Rajapaksas are seen as war heroes, orchestrating the end of the conflict
and ushering in a period of peace and economic growth.

To their opponents, the Rajapaksas' last tenure in government was marked
by
\href{https://www.nytimes3xbfgragh.onion/2018/06/25/world/asia/china-sri-lanka-port.html}{deep-seated
corruption} and human rights abuses. Their government was accused of
\href{https://www.nytimes3xbfgragh.onion/2020/02/15/world/asia/sri-lanka-us-sanctions.html}{war
crimes rising out of the final offensives} against Tamil Tiger rebels,
in which the United Nations estimates that perhaps 40,000 civilians were
killed. Their administration was also accused of extrajudicial killings
and heavy persecution to silence dissent and the political opposition.

Still, Gotabaya Rajapaksa won
an\href{https://www.nytimes3xbfgragh.onion/2019/11/17/world/asia/sri-lanka-Gotabaya-Rajapaksa-election.html}{overwhelming
victory in the 2019 presidential elections}, riding a wave of deep
discontent with the government as the economy weakened and after
\href{https://www.nytimes3xbfgragh.onion/2019/04/21/world/asia/sri-lanka-bombings.html}{multiple
terrorist attacks killed more than 260 and injured over 500 on Easter
Sunday last yea}r.

This week, many voters said they were voting for the stability the
Rajapaksas provided rather than the liberties ensured under the chaotic
rule of the last government.

``I am confident in the present government,'' said M. Kalyanawathi, 72,
as she lined up to vote in Piliyandala City. She praised the government
for its handling of the coronavirus pandemic and its clamping down on
crime.

Image

A polling officer in Colombo on Wednesday.~Many voters said they were
voting for the stability the Rajapaksas provided.Credit...Eranga
Jayawardena/Associated Press

``Democracy? Hard to think about it,'' Ms. Kalyanawathi said. But ``I
think democracy should also be there.''

On Election Day, Wednesday, voter turnout defied expectations despite
concerns about the spread of the coronavirus. About 71 percent of
eligible voters cast their ballots, the Election Commission said, down
about 5 percent from the last parliamentary polls, in 2015.

The high turnout was testament to the government's apparent success in
containing the pandemic. Only 11 deaths have been reported from
Covid-19, the disease caused by the coronavirus, in an island nation
with a population of about 21.5 million.

``We have shown through our management of Covid-19 that we are capable
of handling crises,'' said Bandula Gunawardena, a spokesman for the
government's cabinet who was running for Parliament with the Rajapaksas'
party.

Image

Voters in Colombo on Wednesday.~ Turnout defied expectations despite
concerns about the spread of the coronavirus. Credit...Lakruwan
Wanniarachchi/Agence France-Presse --- Getty Images

This election was crucial, he added, as ``we hope to change the
Constitution.''

Infighting and disorganization continued to plague the opposition,
leaving a weakened challenge to the Rajapaksas.

Still, opposition leaders, like Eran Wickramaratne from the Samagi Jana
Balawegaya party, said that despite their poor chances of routing the
governing party at the polls, they would continue to try to provide a
check on the president going forward.

The governing party ``is mainly ruled by a family; it is an
authoritarian power,'' said Mr. Wickramaratne, who served in the
previous government's cabinet. ``This model is dangerous. In world
history, we have seen that it starts well and ends poorly,'' he said,
adding, ``We stand for democracy.''

The polls were supposed to be held in April but were postponed twice as
the government focused on containing the pandemic. Sri Lanka became one
of just a few countries so far this year to hold a nationwide election.

Election results usually trickle in on the same day in Sri Lanka, but
this year's voting took longer to conduct and count than years past,
officials said.

Social distancing was maintained as voters lined up at some polling
stations, and many wore masks despite the sweltering, humid, midsummer
heat in the tropical country. Voters were encouraged to wash their hands
before taking their ballots, and water taps were provided at some
polling booths while in others, officials provided hand sanitizer.

The election was peaceful, and there were fewer violations of election
law than in past votes, and those were considered minor, said Manjula
Gajanayaka, the national coordinator of the Center for Monitoring
Election Violence, an independent monitoring group.

Image

Transferring ballots in Colombo on Wednesday. There were fewer
violations of election law than in past votes.Credit...Lakruwan
Wanniarachchi/Agence France-Presse --- Getty Images

Advertisement

\protect\hyperlink{after-bottom}{Continue reading the main story}

\hypertarget{site-index}{%
\subsection{Site Index}\label{site-index}}

\hypertarget{site-information-navigation}{%
\subsection{Site Information
Navigation}\label{site-information-navigation}}

\begin{itemize}
\tightlist
\item
  \href{https://help.nytimes3xbfgragh.onion/hc/en-us/articles/115014792127-Copyright-notice}{©~2020~The
  New York Times Company}
\end{itemize}

\begin{itemize}
\tightlist
\item
  \href{https://www.nytco.com/}{NYTCo}
\item
  \href{https://help.nytimes3xbfgragh.onion/hc/en-us/articles/115015385887-Contact-Us}{Contact
  Us}
\item
  \href{https://www.nytco.com/careers/}{Work with us}
\item
  \href{https://nytmediakit.com/}{Advertise}
\item
  \href{http://www.tbrandstudio.com/}{T Brand Studio}
\item
  \href{https://www.nytimes3xbfgragh.onion/privacy/cookie-policy\#how-do-i-manage-trackers}{Your
  Ad Choices}
\item
  \href{https://www.nytimes3xbfgragh.onion/privacy}{Privacy}
\item
  \href{https://help.nytimes3xbfgragh.onion/hc/en-us/articles/115014893428-Terms-of-service}{Terms
  of Service}
\item
  \href{https://help.nytimes3xbfgragh.onion/hc/en-us/articles/115014893968-Terms-of-sale}{Terms
  of Sale}
\item
  \href{https://spiderbites.nytimes3xbfgragh.onion}{Site Map}
\item
  \href{https://help.nytimes3xbfgragh.onion/hc/en-us}{Help}
\item
  \href{https://www.nytimes3xbfgragh.onion/subscription?campaignId=37WXW}{Subscriptions}
\end{itemize}
