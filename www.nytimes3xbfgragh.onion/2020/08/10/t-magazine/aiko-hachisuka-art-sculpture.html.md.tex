Sections

SEARCH

\protect\hyperlink{site-content}{Skip to
content}\protect\hyperlink{site-index}{Skip to site index}

\href{https://myaccount.nytimes3xbfgragh.onion/auth/login?response_type=cookie\&client_id=vi}{}

\href{https://www.nytimes3xbfgragh.onion/section/todayspaper}{Today's
Paper}

The Artist Making Bulbous, Colorful Sculptures Out of Thrifted Clothes

\begin{itemize}
\item
\item
\item
\item
\item
\end{itemize}

Advertisement

\protect\hyperlink{after-top}{Continue reading the main story}

Supported by

\protect\hyperlink{after-sponsor}{Continue reading the main story}

T Presents

\hypertarget{the-artist-making-bulbous-colorful-sculptures-out-of-thrifted-clothes}{%
\section{The Artist Making Bulbous, Colorful Sculptures Out of Thrifted
Clothes}\label{the-artist-making-bulbous-colorful-sculptures-out-of-thrifted-clothes}}

Each of Aiko Hachisuka's intricate, hand-stitched pieces is a meditation
on memory, materiality and domestic labor.

\includegraphics{https://static01.graylady3jvrrxbe.onion/images/2020/08/10/t-magazine/art/Tadobe-slide-HVS6/Tadobe-slide-HVS6-articleLarge.jpg?quality=75\&auto=webp\&disable=upscale}

By
\href{https://www.nytimes3xbfgragh.onion/by/merrell-hambleton}{Merrell
Hambleton}

\begin{itemize}
\item
  Aug. 10, 2020
\item
  \begin{itemize}
  \item
  \item
  \item
  \item
  \item
  \end{itemize}
\end{itemize}

When, as a young girl in Nagoya, Japan, Aiko Hachisuka observed her
mother and grandmother sewing, she didn't quite know what they were
doing --- maybe mending a sock or attaching buttons, she thinks now ---
but she was keenly aware that they were not to be disturbed. ``I was
very attracted to that focus, and the creative headspace that they were
in,'' she says. It was the closest thing to a studio practice that she
had been exposed to. ``I knew I needed to have work myself,'' she says.
``My first attempt at that was to go and buy a sketchbook.''

As an artist, Hachisuka, 46 --- who now lives in Los Angeles, where she
shares a home and studio with her partner, the painter John Williams ---
has maintained a practice rooted in the humility of domestic work. The
large and vibrant stuffed fabric sculptures for which she is best known
--- and which have been shown most recently at
\href{https://www.vandorenwaxter.com/}{Van Doren Waxter} in New York ---
are constructed slowly and methodically over a period of three to four
months. Yet even hung on a gallery wall, they retain the energy of the
woven mats her grandmother used to construct from the family's old
clothing. ``I liked the fact that it was something you put on the floor,
that you stepped on,'' she says.

\includegraphics{https://static01.graylady3jvrrxbe.onion/images/2020/08/10/t-magazine/10tmag-hachisuka/10tmag-hachisuka-videoSixteenByNine3000.jpg}

But though Hachisuka identified her interest in art-making, especially
as related to textiles and stitching, early on, her path to her current
practice was not a direct one. As a teenager, she applied to be a
foreign exchange student in the United States and was placed with two
different families in Pensacola, Fla., for her junior and senior years
of high school. After a portfolio review her senior year, she was
accepted to the Pratt Institute in Brooklyn and the Ringling College of
Art and Design, in Sarasota, Fla. At the insistence of her host mother,
who thought New York might be dangerous or overwhelming, she chose
Ringling. ``Some people still think I went to circus school,'' Hachisuka
says with a laugh.

In her third year, during a semester-long independent study in New York,
she began to experiment more, both creatively and conceptually. This was
the mid-1990s, a decade after the end of the Pattern and Decoration
movement, and artists such as
\href{https://www.nytimes3xbfgragh.onion/2012/02/02/arts/design/mike-kelley-influential-american-artist-dies-at-57.html}{Mike
Kelley} and
\href{https://www.nytimes3xbfgragh.onion/2007/06/25/arts/25iht-messager.1.6316141.html}{Annette
Messager} were appropriating the vocabulary of textile work to great
critical acclaim. It was common for Hachisuka's fellow students to have
sewing machines in their studios, and, she says, ``kids were making
drawings out of doilies, or taking cubes from cake decoration sets and
incorporating them into paintings.'' She started using puff paint and
found objects --- cheap ornaments and stuffed animals purchased on Canal
Street --- to create sculptures. Instead of pedestals, she would place
the pieces on top of hand-sewn pillows.

Image

``Each piece of clothing is like a mini-painting,'' says Hachisuka, who
uses a modified screen-printing process on the individual components of
her vibrant sculptures.Credit...Philip Cheung

Image

In her studio in Los Angeles's Lincoln Heights neighborhood, the artist
mixes her paint before applying it to the garments.Credit...Philip
Cheung

After graduate school at CalArts, where she studied under the conceptual
artist
\href{https://www.nytimes3xbfgragh.onion/2019/04/07/arts/charles-gaines-edward-macdowell-medal.html}{Charles
Gaines} and took a brief detour into video (``CalArts was very
post-studio right then,'' she says), Hachisuka was ready to return to
three-dimensional work. Her first piece was a nearly life-size fabric
recreation of the crumpled head of a GMC semi truck with a felt shopping
cart and felt pipes and bricks, all things she'd seen at a junkyard in
East Los Angeles.

Still, she kept thinking back to her grandmother's rag rugs --- their
unpretentiousness and how, studded with bits of Hachisuka's childhood
clothing, they functioned a bit like a family photograph. ``I knew what
I wanted to make,'' she says, ``but I didn't know how to find the door
into it.'' One afternoon in 2003, she was sitting on the sofa in her
apartment, her cat draped over her arm and across the seat cushion ---
stretched like a stitch, connecting Hachisuka to the furniture. ``My
body touching the furniture became a unified thing,'' she says. ``I
thought, `What if I make furniture that \emph{is} a body?'''

\includegraphics{https://static01.graylady3jvrrxbe.onion/images/2020/08/10/t-magazine/10tmag-hachisuka-02/10tmag-hachisuka-02-mediumSquareAt3X.jpg}

She'd found her way in and hasn't closed the door since. Each of
Hachisuka's pieces starts with an armature --- initially they were found
couch frames, then foam cylinders, and most recently wall-mounted
supports --- and a collection of both new and secondhand clothing, some
of it acquired from tag sales or her partner's closet. She stuffs the
garments with a natural fiber called Kapok to create volume and density
(``\href{https://www.newyorker.com/magazine/2009/05/11/the-art-doctor}{Christian
Scheidemann}, the art conservator, told me that
\href{https://www.nytimes3xbfgragh.onion/2017/10/16/t-magazine/claes-oldenburg.html}{Claes
Oldenburg} uses it in his soft sculpture, and I've been using it ever
since''). Finally, she arranges and stitches the individual pieces
together into forms that ripple and bulge, at once solid as a compacted
car and yet pushing outward as though full of air. In Hachisuka's
earlier couch pieces, a pair of legs might emerge from a seat cushion or
the arm of a sweatshirt might wrap around a pillow in a gentle embrace.
Her newer work is increasingly abstract. Up close, one can recognize the
ribbed collar of a child's sweater, perhaps, or the elastic waistband of
a pair of athletic shorts --- the body evident only in its absence.

\includegraphics{https://static01.graylady3jvrrxbe.onion/images/2020/08/10/t-magazine/art/Tadobe-slide-13ES/Tadobe-slide-13ES-articleLarge.jpg?quality=75\&auto=webp\&disable=upscale}

Even as the artist's work moves away from recognizable forms, though, it
retains an intimate connection to family and home. Initially, Hachisuka
made work from clothing she bought at Forever 21, but she felt that
reduced it to a comment on fast fashion. Then she started thrifting
clothes, but the pieces became disjointed. After stopping at a tag sale
one day on a whim, she found she could put together a loose family
structure --- two kids, a mother who worked for a dentist's office, a
father who went to church --- based on a rack of discards, and liked
that the clothes told a story both specific and not. ``It had this
family history that I could never replicate,'' Hachisuka says.

About 10 years ago, Hachisuka started adding paint to the clothing
pieces before she stuffed them. She uses a modified screen-printing
process, laying a mesh screen over the garment and applying ink without
a preset design. She's seeking a very pared-down way of working, and she
lets the folds and wrinkles of the dropped clothing determine the
patterns. ``Each piece of clothing is like a mini-painting,'' she says.
``And then I stitch them together like a puzzle.'' Once she has removed
the screen from the clothing, she creates a monoprint by applying paper
to the back of the screen, capturing a kind of photo negative of the
garment. A collection of these prints is
\href{https://www.vandorenwaxter.com/exhibitions/aiko-hachisuka3}{currently
on view} on Van Doren Waxter's website.

Like mending, Hachisuka's art is nearly without ego, and intentionally
so --- she designed the process so as to remove her own hand and eye
from the finished work as much as possible. ``I don't want my taste or
liking to get in the way of making,'' she says. ``I'm there every step
of the way, but when I'm done, looking at it, it doesn't feel like
mine.'' It's more the concept that is personal, with Hachisuka thinking
back to watching her mother and grandmother at work all the while.
``I've always wanted to return to that space,'' she says, ``even
today.''

\hypertarget{t-presents-15-creative-women-for-our-time}{%
\subsubsection{\texorpdfstring{\href{https://www.nytimes3xbfgragh.onion/interactive/2020/08/10/t-magazine/creative-women-designers-artists-chefs.html}{T
Presents: 15 Creative Women for Our
Time}}{T Presents: 15 Creative Women for Our Time}}\label{t-presents-15-creative-women-for-our-time}}

\href{https://www.nytimes3xbfgragh.onion/section/t-magazine}{}

\href{https://www.nytimes3xbfgragh.onion/2020/08/10/t-magazine/priya-ahluwalia-fashion-menswear.html}{\includegraphics{https://static01.graylady3jvrrxbe.onion/newsgraphics/2020/06/17/tmag-adobe/assets/images/ahluwalia-460.jpg}}

Priya Ahluwalia

Fashion Designer

\href{https://www.nytimes3xbfgragh.onion/2020/08/10/t-magazine/alice-cicolini-jewelry-art.html}{\includegraphics{https://static01.graylady3jvrrxbe.onion/newsgraphics/2020/06/17/tmag-adobe/assets/images/cicolini-460.jpg}}

Alice Cicolini

Jewelry Designer

\href{https://nytimes3xbfgragh.onion/2020/08/10/t-magazine/sonya-clark-flags-art.html}{\includegraphics{https://static01.graylady3jvrrxbe.onion/newsgraphics/2020/06/17/tmag-adobe/assets/images/clark-460.jpg}}

Sonya Clark

Artist

\href{https://www.nytimes3xbfgragh.onion/2020/08/10/t-magazine/pierre-davis-no-sesso.html}{\includegraphics{https://static01.graylady3jvrrxbe.onion/newsgraphics/2020/06/17/tmag-adobe/assets/images/davis-460.jpg}}

Pierre Davis

Fashion Designer

\href{https://www.nytimes3xbfgragh.onion/2020/08/10/t-magazine/paria-farzaneh-fashion-menswear.html}{\includegraphics{https://static01.graylady3jvrrxbe.onion/newsgraphics/2020/06/17/tmag-adobe/assets/images/farzaneh-460.jpg}}

Paria Farzaneh

Fashion Designer

\href{https://www.nytimes3xbfgragh.onion/2020/08/10/t-magazine/elizabeth-garouste-interior-design.html}{\includegraphics{https://static01.graylady3jvrrxbe.onion/newsgraphics/2020/06/17/tmag-adobe/assets/images/garouste-460.jpg}}

Elizabeth Garouste

Furniture Designer and Artist

\href{https://www.nytimes3xbfgragh.onion/2020/08/10/t-magazine/jatovia-gary-film.html}{\includegraphics{https://static01.graylady3jvrrxbe.onion/newsgraphics/2020/06/17/tmag-adobe/assets/images/gary-460.jpg}}

Ja'Tovia Gary

Artist and Filmmaker

\href{https://www.nytimes3xbfgragh.onion/2020/08/10/t-magazine/aiko-hachisuka-art-sculpture.html}{\includegraphics{https://static01.graylady3jvrrxbe.onion/newsgraphics/2020/06/17/tmag-adobe/assets/images/hachisuka-460.jpg}}

Aiko Hachisuka

Artist

\href{https://www.nytimes3xbfgragh.onion/2020/08/10/t-magazine/juliana-huxtable.html}{\includegraphics{https://static01.graylady3jvrrxbe.onion/newsgraphics/2020/06/17/tmag-adobe/assets/images/huxtable-460.jpg}}

Juliana Huxtable

Artist

\href{https://www.nytimes3xbfgragh.onion/2020/08/10/t-magazine/mariam-kamara-architect-design.html}{\includegraphics{https://static01.graylady3jvrrxbe.onion/newsgraphics/2020/06/17/tmag-adobe/assets/images/kamara-460.jpg}}

Mariam Kamara

Architect

\href{https://www.nytimes3xbfgragh.onion/2020/08/10/t-magazine/sophia-moreno-bunge-floral-design.html}{\includegraphics{https://static01.graylady3jvrrxbe.onion/newsgraphics/2020/06/17/tmag-adobe/assets/images/bunge-460.jpg}}

Sophia Moreno-Bunge

Floral Designer

\href{https://www.nytimes3xbfgragh.onion/2020/08/10/t-magazine/marina-moscone-fashion-design.html}{\includegraphics{https://static01.graylady3jvrrxbe.onion/newsgraphics/2020/06/17/tmag-adobe/assets/images/moscone-460.jpg}}

Marina Moscone

Fashion Designer

\href{https://www.nytimes3xbfgragh.onion/2020/08/10/t-magazine/amber-pinkerton-photography.html}{\includegraphics{https://static01.graylady3jvrrxbe.onion/newsgraphics/2020/06/17/tmag-adobe/assets/images/pinkerton-460.jpg}}

Amber Pinkerton

Photographer

\href{https://www.nytimes3xbfgragh.onion/2020/08/10/t-magazine/sonoko-sakai-chef-cooking-soba.html}{\includegraphics{https://static01.graylady3jvrrxbe.onion/newsgraphics/2020/06/17/tmag-adobe/assets/images/sakai-460.jpg}}

Sonoko Sakai

Cookbook Author and Food Activist

\href{https://www.nytimes3xbfgragh.onion/2020/08/10/t-magazine/daniela-soto-innes-cooking-chef.html}{\includegraphics{https://static01.graylady3jvrrxbe.onion/newsgraphics/2020/06/17/tmag-adobe/assets/images/ines-460.jpg}}

Daniela Soto-Innes

Chef

Advertisement

\protect\hyperlink{after-bottom}{Continue reading the main story}

\hypertarget{site-index}{%
\subsection{Site Index}\label{site-index}}

\hypertarget{site-information-navigation}{%
\subsection{Site Information
Navigation}\label{site-information-navigation}}

\begin{itemize}
\tightlist
\item
  \href{https://help.nytimes3xbfgragh.onion/hc/en-us/articles/115014792127-Copyright-notice}{©~2020~The
  New York Times Company}
\end{itemize}

\begin{itemize}
\tightlist
\item
  \href{https://www.nytco.com/}{NYTCo}
\item
  \href{https://help.nytimes3xbfgragh.onion/hc/en-us/articles/115015385887-Contact-Us}{Contact
  Us}
\item
  \href{https://www.nytco.com/careers/}{Work with us}
\item
  \href{https://nytmediakit.com/}{Advertise}
\item
  \href{http://www.tbrandstudio.com/}{T Brand Studio}
\item
  \href{https://www.nytimes3xbfgragh.onion/privacy/cookie-policy\#how-do-i-manage-trackers}{Your
  Ad Choices}
\item
  \href{https://www.nytimes3xbfgragh.onion/privacy}{Privacy}
\item
  \href{https://help.nytimes3xbfgragh.onion/hc/en-us/articles/115014893428-Terms-of-service}{Terms
  of Service}
\item
  \href{https://help.nytimes3xbfgragh.onion/hc/en-us/articles/115014893968-Terms-of-sale}{Terms
  of Sale}
\item
  \href{https://spiderbites.nytimes3xbfgragh.onion}{Site Map}
\item
  \href{https://help.nytimes3xbfgragh.onion/hc/en-us}{Help}
\item
  \href{https://www.nytimes3xbfgragh.onion/subscription?campaignId=37WXW}{Subscriptions}
\end{itemize}
