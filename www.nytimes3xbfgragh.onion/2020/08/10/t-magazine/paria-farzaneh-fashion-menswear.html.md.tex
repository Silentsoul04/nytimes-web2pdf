Sections

SEARCH

\protect\hyperlink{site-content}{Skip to
content}\protect\hyperlink{site-index}{Skip to site index}

\href{https://myaccount.nytimes3xbfgragh.onion/auth/login?response_type=cookie\&client_id=vi}{}

\href{https://www.nytimes3xbfgragh.onion/section/todayspaper}{Today's
Paper}

The Designer Bringing Traditional Persian Fabrics to British Men's Wear

\begin{itemize}
\item
\item
\item
\item
\item
\end{itemize}

Advertisement

\protect\hyperlink{after-top}{Continue reading the main story}

Supported by

\protect\hyperlink{after-sponsor}{Continue reading the main story}

T Presents

\hypertarget{the-designer-bringing-traditional-persian-fabrics-to-british-mens-wear}{%
\section{The Designer Bringing Traditional Persian Fabrics to British
Men's
Wear}\label{the-designer-bringing-traditional-persian-fabrics-to-british-mens-wear}}

With her paisley prints and contemporary silhouettes, Paria Farzaneh
shares her singular perspective with an ever-widening audience.

\includegraphics{https://static01.graylady3jvrrxbe.onion/images/2020/08/10/t-magazine/art/Tadobe-slide-DMTD/Tadobe-slide-DMTD-articleLarge.jpg?quality=75\&auto=webp\&disable=upscale}

By Osman Ahmed

\begin{itemize}
\item
  Aug. 10, 2020
\item
  \begin{itemize}
  \item
  \item
  \item
  \item
  \item
  \end{itemize}
\end{itemize}

This past January, during men's fashion week in London, the designer
Paria Farzaneh invited guests to an East London **** boys' school for an
Iranian wedding. A bride sat onstage in a **** traditional **** white
lace gown before **** a banquet of Persian pastries and urns overflowing
with pastel-colored roses and baby's breath. The groom, seated next to
her, wore a paisley-printed ski jacket, its high, mouth-covering neck
pre-empting the protective masks of the pandemic. An older community
leader in a black suit presided over the faux ceremony in Farsi, as the
viewers sat divided by gender --- men on one side of the aisle, women on
the other, though the men outnumbered the women and ended up infringing
on their section. This was a deliberate move on Farzaneh's part intended
to highlight the power imbalance she feels in her industry --- where a
woman truly at the helm of her own brand is still something of a rarity
--- and was typical of the subtle gestures toward the tensions of our
present moment that are embedded in her line. At the end of the
ceremony, the groom rose and strode down the aisle, followed by a
procession of young men dressed in looks from the
\href{https://www.pariafarzaneh.com/aw20}{fall 2020 collection} of
Farzaneh's namesake line, which combined the swagger and oversize
silhouettes of streetwear with the understated, earthy palette of
hand-printed Iranian textiles.

\includegraphics{https://static01.graylady3jvrrxbe.onion/images/2020/08/10/t-magazine/10tmag-paria-02/10tmag-paria-02-mediumSquareAt3X.jpg}

``For me, fashion is merely a platform for something much bigger,'' says
Farzaneh. Which is not to say she doesn't delight in the design process,
or that design **** isn't a tool in and of itself. She may be one of the
few Londoners who, a few months into the pandemic, decamped to Milan,
where she spent several weeks working on her next collection and
communicating more directly with the Italian mills where she researches
and develops her fabrics. She's particularly interested in fabric and
handwork, and sources her signature patterned cotton **** textiles from
Isfahan, Iran's historic center of textile and rug production. They
feature finely wrought florals, paisleys and other motifs created by
hand according to the ancient tradition of Ghalamkar printing, in which
intricately hand-carved wood blocks are used to stamp patterns onto a
length of cloth one color at a time, with only the artisan's eye as a
guide. ``All the dyes are plant-based, using saffron, turmeric and
pomegranate, and the pieces are washed in the river and dried in the
sun,'' explains Farzaneh, who has an ability to transform these fabrics
--- which Iranians might recognize as the sort more typically used for
blankets and bedspreads --- into deeply modern and covetable garments.
She has a similar knack for refreshing other elements that in different
contexts might feel old-fashioned, such as flared pants or especially
wide lapels. She's put drawstrings at the ankles of Ghalamkar-printed
trousers; made polo shirts from strips of acid-green, waterproof,
nylon-blended cloth, overlapped to resemble the lattice of a pie crust;
and created patchwork suiting from leftover scraps. Along the way, she's
caught the attention of the N.B.A. stars LeBron James and Nick ``Swaggy
P'' Young; the apparel companies **** Gore-Tex and Converse, both of
which she's collaborated with; and the selection committee behind the
\href{https://www.lvmhprize.com/designer/paria-farzaneh/}{LVMH Prize},
for which she was shortlisted last year.

Image

At Majocchi Tech, the designer was able to draw on its archive of
discarded or unwanted fabrics for use in her next collection, which will
be her sixth.Credit...Federico Ciamei

Image

Farzaneh's designs are often spliced with traditional Iranian fabrics
sourced from the city of Isfahan, where they are commonly used as
household blankets or tablecloths.Credit...Federico Ciamei

But contextualizing her clothes is almost as important to Farzaneh as
the fabrics and silhouettes themselves. Since launching her label in
2017, she has staged an immersive fashion show each season:
\href{https://www.pariafarzaneh.com/ss19}{For spring 2019}, models posed
along a truck bed installed with seven detailed sets inspired by Nowruz,
the Persian New Year, in a collection dedicated to her late uncle, whose
presence could be felt in uniform-like khaki pieces.
\href{https://www.pariafarzaneh.com/aw19}{The following season},
Farzaneh mailed out her invitations with plastic bags for each audience
member to stash their phone in during the show; on the day of, few
complied, and onstage, the models, some in tech-y fabrics, others in
jackets that wrapped around the body in a suggestion of passive
captivity, moved along a conveyor belt with their eyes glued to their
own devices. \href{https://www.pariafarzaneh.com/ss20}{For spring 2020},
her models wore matching transparent Halloween masks with garish makeup,
T-shirts puff-painted with Farsi script and, in one instance, an
overcoat that looked as though it had been stamped by the British Border
Force, which read like an eerie comment on the United Kingdom's
surveillance and immigration control policies. ``I try to transport
people to a place they've never been before,'' Farzaneh says. ``It's
important for me to put an audience in that position --- it's not just
about pushing a product or a trend.''

\includegraphics{https://static01.graylady3jvrrxbe.onion/images/2020/08/10/t-magazine/10tmag-paria/10tmag-paria-mediumSquareAt3X.jpg}

Farzaneh credits her desire to bring people into her world to her
experience growing up in what she describes as the ``five percent
minority'' in a rural village in Yorkshire. Her parents, who emigrated
from Iran before she was born, ``were never in a position to make me
feel like I was different,'' Farzaneh says, ``but as I got older, I
started to see more of a disconnect. People didn't understand.'' Her
family's influence can be felt deeply in her work --- faded photos of
her relatives fill the label's social feeds, and her father occasionally
models for the brand's look books. Also, Farzaneh's grandfather was a
tailor in Iran, and her mother made much of Farzaneh's childhood
clothing.

Image

Farzaneh is known for celebrating her Iranian heritage in both the
presentation and design of her clothes. For spring 2020, she showed
pieces printed with calligraphic script.Credit...Federico Ciamei

Image

In a reversible gilet from her fall 2020 collection, Farzaneh juxtaposes
neon orange Gore-Tex and a traditional Iranian fabric produced in
Isfahan.Credit...Federico Ciamei

On the other hand, she appreciates the creative distance afforded by
being a woman exploring masculinity, even if she usually wears men's
clothes herself. ``In women's wear, the details of practicality are
compromised by garments being cropped, tight or short,'' she says. In
2016, she earned her degree in fashion design from London's Ravensbourne
University. Since then, Farzaneh, now 26, has established herself among
a wave of London-based female designers of men's wear --- including
Grace Wales Bonner, Martine Rose,
\href{https://www.nytimes3xbfgragh.onion/2020/08/10/t-magazine/priya-ahluwalia-fashion-menswear.html}{Priya
Ahluwalia}, Bianca Saunders and Mowalola Ogunlesi --- who have
reimagined the city's men's wear scene over the past half-decade. Each
offers her own take, but they are alike in that, just by existing in the
field, they push back against the elitist traditions of Savile Row
tailoring firms --- some of which still only dress male customers ---
and the traditionally fusty world of British craftsmanship. Some of
these women, like Farzaneh, also offer artistic visions informed by the
immigrant experience of their families.

``Eventually,'' Farzaneh says, ``people are going to listen, because
they're tired of seeing and smelling and tasting the same monotonous
world.'' By making clothes with care and imbuing them with meaning, she
offers an antidote to the myriad soulless, throwaway options available
on the market today, and advances the conversation about what, and who,
is part of fashion. ``Success for me is about being honest with myself
and not compromising what I believe in,'' she says. ``It's about
authenticity and realness.''

\hypertarget{t-presents-15-creative-women-for-our-time}{%
\subsubsection{\texorpdfstring{\href{https://www.nytimes3xbfgragh.onion/interactive/2020/08/10/t-magazine/creative-women-designers-artists-chefs.html}{T
Presents: 15 Creative Women for Our
Time}}{T Presents: 15 Creative Women for Our Time}}\label{t-presents-15-creative-women-for-our-time}}

\href{https://www.nytimes3xbfgragh.onion/section/t-magazine}{}

\href{https://www.nytimes3xbfgragh.onion/2020/08/10/t-magazine/priya-ahluwalia-fashion-menswear.html}{\includegraphics{https://static01.graylady3jvrrxbe.onion/newsgraphics/2020/06/17/tmag-adobe/assets/images/ahluwalia-460.jpg}}

Priya Ahluwalia

Fashion Designer

\href{https://www.nytimes3xbfgragh.onion/2020/08/10/t-magazine/alice-cicolini-jewelry-art.html}{\includegraphics{https://static01.graylady3jvrrxbe.onion/newsgraphics/2020/06/17/tmag-adobe/assets/images/cicolini-460.jpg}}

Alice Cicolini

Jewelry Designer

\href{https://nytimes3xbfgragh.onion/2020/08/10/t-magazine/sonya-clark-flags-art.html}{\includegraphics{https://static01.graylady3jvrrxbe.onion/newsgraphics/2020/06/17/tmag-adobe/assets/images/clark-460.jpg}}

Sonya Clark

Artist

\href{https://www.nytimes3xbfgragh.onion/2020/08/10/t-magazine/pierre-davis-no-sesso.html}{\includegraphics{https://static01.graylady3jvrrxbe.onion/newsgraphics/2020/06/17/tmag-adobe/assets/images/davis-460.jpg}}

Pierre Davis

Fashion Designer

\href{https://www.nytimes3xbfgragh.onion/2020/08/10/t-magazine/paria-farzaneh-fashion-menswear.html}{\includegraphics{https://static01.graylady3jvrrxbe.onion/newsgraphics/2020/06/17/tmag-adobe/assets/images/farzaneh-460.jpg}}

Paria Farzaneh

Fashion Designer

\href{https://www.nytimes3xbfgragh.onion/2020/08/10/t-magazine/elizabeth-garouste-interior-design.html}{\includegraphics{https://static01.graylady3jvrrxbe.onion/newsgraphics/2020/06/17/tmag-adobe/assets/images/garouste-460.jpg}}

Elizabeth Garouste

Furniture Designer and Artist

\href{https://www.nytimes3xbfgragh.onion/2020/08/10/t-magazine/jatovia-gary-film.html}{\includegraphics{https://static01.graylady3jvrrxbe.onion/newsgraphics/2020/06/17/tmag-adobe/assets/images/gary-460.jpg}}

Ja'Tovia Gary

Artist and Filmmaker

\href{https://www.nytimes3xbfgragh.onion/2020/08/10/t-magazine/aiko-hachisuka-art-sculpture.html}{\includegraphics{https://static01.graylady3jvrrxbe.onion/newsgraphics/2020/06/17/tmag-adobe/assets/images/hachisuka-460.jpg}}

Aiko Hachisuka

Artist

\href{https://www.nytimes3xbfgragh.onion/2020/08/10/t-magazine/juliana-huxtable.html}{\includegraphics{https://static01.graylady3jvrrxbe.onion/newsgraphics/2020/06/17/tmag-adobe/assets/images/huxtable-460.jpg}}

Juliana Huxtable

Artist

\href{https://www.nytimes3xbfgragh.onion/2020/08/10/t-magazine/mariam-kamara-architect-design.html}{\includegraphics{https://static01.graylady3jvrrxbe.onion/newsgraphics/2020/06/17/tmag-adobe/assets/images/kamara-460.jpg}}

Mariam Kamara

Architect

\href{https://www.nytimes3xbfgragh.onion/2020/08/10/t-magazine/sophia-moreno-bunge-floral-design.html}{\includegraphics{https://static01.graylady3jvrrxbe.onion/newsgraphics/2020/06/17/tmag-adobe/assets/images/bunge-460.jpg}}

Sophia Moreno-Bunge

Floral Designer

\href{https://www.nytimes3xbfgragh.onion/2020/08/10/t-magazine/marina-moscone-fashion-design.html}{\includegraphics{https://static01.graylady3jvrrxbe.onion/newsgraphics/2020/06/17/tmag-adobe/assets/images/moscone-460.jpg}}

Marina Moscone

Fashion Designer

\href{https://www.nytimes3xbfgragh.onion/2020/08/10/t-magazine/amber-pinkerton-photography.html}{\includegraphics{https://static01.graylady3jvrrxbe.onion/newsgraphics/2020/06/17/tmag-adobe/assets/images/pinkerton-460.jpg}}

Amber Pinkerton

Photographer

\href{https://www.nytimes3xbfgragh.onion/2020/08/10/t-magazine/sonoko-sakai-chef-cooking-soba.html}{\includegraphics{https://static01.graylady3jvrrxbe.onion/newsgraphics/2020/06/17/tmag-adobe/assets/images/sakai-460.jpg}}

Sonoko Sakai

Cookbook Author and Food Activist

\href{https://www.nytimes3xbfgragh.onion/2020/08/10/t-magazine/daniela-soto-innes-cooking-chef.html}{\includegraphics{https://static01.graylady3jvrrxbe.onion/newsgraphics/2020/06/17/tmag-adobe/assets/images/ines-460.jpg}}

Daniela Soto-Innes

Chef

Advertisement

\protect\hyperlink{after-bottom}{Continue reading the main story}

\hypertarget{site-index}{%
\subsection{Site Index}\label{site-index}}

\hypertarget{site-information-navigation}{%
\subsection{Site Information
Navigation}\label{site-information-navigation}}

\begin{itemize}
\tightlist
\item
  \href{https://help.nytimes3xbfgragh.onion/hc/en-us/articles/115014792127-Copyright-notice}{©~2020~The
  New York Times Company}
\end{itemize}

\begin{itemize}
\tightlist
\item
  \href{https://www.nytco.com/}{NYTCo}
\item
  \href{https://help.nytimes3xbfgragh.onion/hc/en-us/articles/115015385887-Contact-Us}{Contact
  Us}
\item
  \href{https://www.nytco.com/careers/}{Work with us}
\item
  \href{https://nytmediakit.com/}{Advertise}
\item
  \href{http://www.tbrandstudio.com/}{T Brand Studio}
\item
  \href{https://www.nytimes3xbfgragh.onion/privacy/cookie-policy\#how-do-i-manage-trackers}{Your
  Ad Choices}
\item
  \href{https://www.nytimes3xbfgragh.onion/privacy}{Privacy}
\item
  \href{https://help.nytimes3xbfgragh.onion/hc/en-us/articles/115014893428-Terms-of-service}{Terms
  of Service}
\item
  \href{https://help.nytimes3xbfgragh.onion/hc/en-us/articles/115014893968-Terms-of-sale}{Terms
  of Sale}
\item
  \href{https://spiderbites.nytimes3xbfgragh.onion}{Site Map}
\item
  \href{https://help.nytimes3xbfgragh.onion/hc/en-us}{Help}
\item
  \href{https://www.nytimes3xbfgragh.onion/subscription?campaignId=37WXW}{Subscriptions}
\end{itemize}
