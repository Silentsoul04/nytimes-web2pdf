Sections

SEARCH

\protect\hyperlink{site-content}{Skip to
content}\protect\hyperlink{site-index}{Skip to site index}

\href{https://myaccount.nytimes3xbfgragh.onion/auth/login?response_type=cookie\&client_id=vi}{}

\href{https://www.nytimes3xbfgragh.onion/section/todayspaper}{Today's
Paper}

\href{/section/opinion}{Opinion}\textbar{}In One Corner, Trump. In the
Other, Biden. Let the Debate Begin!

\url{https://nyti.ms/2FlyAmN}

\begin{itemize}
\item
\item
\item
\item
\item
\item
\end{itemize}

Advertisement

\protect\hyperlink{after-top}{Continue reading the main story}

\href{/section/opinion}{Opinion}

Supported by

\protect\hyperlink{after-sponsor}{Continue reading the main story}

\hypertarget{in-one-corner-trump-in-the-other-biden-let-the-debate-begin}{%
\section{In One Corner, Trump. In the Other, Biden. Let the Debate
Begin!}\label{in-one-corner-trump-in-the-other-biden-let-the-debate-begin}}

Presidential debates could use a serious overhaul.

\href{https://www.nytimes3xbfgragh.onion/by/michelle-cottle}{\includegraphics{https://static01.graylady3jvrrxbe.onion/images/2019/06/25/opinion/michelle-cottle-circular/michelle-cottle-circular-thumbLarge-v2.png}}

By \href{https://www.nytimes3xbfgragh.onion/by/michelle-cottle}{Michelle
Cottle}

Ms. Cottle is a member of the editorial board.

\begin{itemize}
\item
  Aug. 12, 2020
\item
  \begin{itemize}
  \item
  \item
  \item
  \item
  \item
  \item
  \end{itemize}
\end{itemize}

\includegraphics{https://static01.graylady3jvrrxbe.onion/images/2020/08/18/opinion/18cottle_print/10cottleWeb-articleLarge.jpg?quality=75\&auto=webp\&disable=upscale}

With many usual fixtures of campaigning upended by the coronavirus
pandemic --- rallies, town halls, fund-raisers, conventions ---
President Trump has been looking to beef up one of the few remaining
pieces: the debates.

The Commission on Presidential Debates has scheduled three matchups
between Mr. Trump and former Vice President Joe Biden, the first set for
Sept. 29. Noting that many states will have already begun early voting
by then, the Trump campaign
\href{https://cdn.donaldjtrump.com/public-files/press_assets/giuliani-debate-letter-8-5-20-1.pdf}{last
week}
\href{https://www.cnn.com/2020/08/05/politics/trump-campaign-four-debates/index.html}{sent
a letter} to the commission asking that a fourth debate be added in
early September --- or, barring that, that the final debate be moved up
from Oct. 22.

``A debate, to me, is a Public Service,'' Mr. Trump
\href{https://twitter.com/realdonaldtrump/status/1291339436727382016}{tweeted}
on Thursday. ``Joe Biden and I owe it to the American People!''

The commission
\href{https://www.nytimes3xbfgragh.onion/2020/08/06/us/politics/presidential-debates-trump-biden.html}{rejected
the request}, insisting such a move was unnecessary.

The truth is that scheduling is way down the list of problems with
presidential debates, in this election cycle or any other. Debates are
indeed a public service, providing voters a rare opportunity to see the
presidential contenders side by side and take their measure for an
extended stretch of time in a high-pressure setting. But in practice,
the events have degenerated into media spectacles, showcasing much that
is wrong with both electoral politics and journalism.

Designed to maximize ratings --- and, increasingly, the number of viral
moments --- the debates are light on meaningful discourse and heavy on
\href{https://variety.com/2020/tv/columns/democratic-debate-cnn-des-moines-register-sanders-warren-moderators-1203467552/}{ginned
up conflict}, regurgitated talking points and cheap zingers. With their
countdown clocks, twitchy graphics and breathless hype, the media hosts
too often package the events like pro wrestling matches. The moderators
often focus more on burnishing their personal brands than on
facilitating discussion.

This year's dynamic is complicated by Mr. Trump, whose relationship with
truth is tenuous at best. With his penchant for prevarication, his
desire to turn every appearance into a carnival, his defensiveness about
his job performance and his growing desperation to improve his poll
numbers, the debates seem bound for a new low.

Much research has been done, and many recommendations made, on how to
improve the debates. One starting point is to rein in the media outlets
that host them. Networks need to tone down the gladiator vibe. The
campaigns are not helpless bystanders. They should have a say in the
basic tone of the proceedings. (Of course, Mr. Trump seems just as
likely to advocate even gaudier showmanship.)

The role of the moderators is a perennial area of concern. ``The single
largest criticism of the debates centers on the inability of moderators
to do their job,'' noted a
\href{http://cdn.annenbergpublicpolicycenter.org/wp-content/uploads/Democratizing-The-Debates.pdf}{2015
report} by a debate-reform working group put together by the Annenberg
Public Policy Center. The most common complaints were that the
moderators play favorites and that they ``either do not have the skills
to control the candidates or to call them on `nonanswers.''' This
critique
\href{https://www.washingtonpost.com/lifestyle/media/the-moderators-let-the-democratic-debate-spiral-into-chaos-and-crosstalk-there-must-be-a-better-way/2020/02/26/e735c61c-5893-11ea-9b35-def5a027d470_story.html}{is
as relevant as ever}.

Moderators are in a tough spot. If they let candidates bluster on or
wander too far afield, they get criticized for losing control of the
debate. If they cut candidates off and strictly enforce time limits,
they get criticized for being too intrusive.

But, in general, moderators need to avoid becoming part of the story.
They should encourage direct interaction between the candidates, even if
that means sitting back and missing out on the occasional follow-up
question. Debates are not meant to be modified news conferences or
interviews. The Annenberg report recommended cutting moderators out of
the action as much as possible. With the particular challenges that Mr.
Trump poses, of course, that may call for some adjustments.

The president has made clear that he will say anything, without regard
to the truth. The debate hosts and moderators need to have multiple
systems in place to deal with this and be willing to call him out.
Real-time fact-checking resources should be beefed up, along with
morning-after analyses. As an additional check, particularly egregious
lies spread in one debate could be revisited in subsequent ones, with
the candidates asked to respond.

The basic debate structure could use some tweaking as well. The common
format of allowing each candidate 60 to 90 seconds to answer, followed
by 30 seconds for rebuttals, is too rigid and provides insufficient time
for thoughtful responses. It pushes participants to give every question
equal time.

One proposed alternative is the chess clock model, in which each
candidate would receive a total of 45 minutes, which would tick down
whenever he or she spoke. Within reasonably broad parameters, a
candidate could devote different amounts of time to different questions,
for instance, spending twice as long on climate change as on
decriminalizing border crossings --- or vice versa.

Another Annenberg suggestion for helping candidates define their
priorities: Give each contender two or three topics in advance, for
which they would prepare meaty four-minute statements, and their
opponent would prepare equivalent rebuttals. Topics could be determined
variously by the candidates, the moderators and the voters.

Organizers really ought to consider losing the live audiences ---~even
after crowding into a college auditorium is no longer a public health
hazard. All the jeering and cheering encourages the candidates, and even
some moderators, to play to the crowd. The crowd reaction, in turn,
influences how the home audience processes the event. The entire set up
lends itself to the kind of stunt Mr. Trump pulled at a 2016 debate with
Hillary Clinton, to which he invited
\href{https://www.nytimes3xbfgragh.onion/2016/10/11/us/politics/bill-clinton-accusers-debate.html}{several
women} who had accused President Bill Clinton of sexual misconduct ---
and tried to seat them in his V.I.P. box next to the former president.

The presidential debates don't have to be such circuses. The public and
the candidates ought to demand better.

\emph{The Times is committed to publishing}
\href{https://www.nytimes3xbfgragh.onion/2019/01/31/opinion/letters/letters-to-editor-new-york-times-women.html}{\emph{a
diversity of letters}} \emph{to the editor. We'd like to hear what you
think about this or any of our articles. Here are some}
\href{https://help.nytimes3xbfgragh.onion/hc/en-us/articles/115014925288-How-to-submit-a-letter-to-the-editor}{\emph{tips}}\emph{.
And here's our email:}
\href{mailto:letters@NYTimes.com}{\emph{letters@NYTimes.com}}\emph{.}

\emph{Follow The New York Times Opinion section on}
\href{https://www.facebookcorewwwi.onion/nytopinion}{\emph{Facebook}}\emph{,}
\href{http://twitter.com/NYTOpinion}{\emph{Twitter (@NYTopinion)}}
\emph{and}
\href{https://www.instagram.com/nytopinion/}{\emph{Instagram}}\emph{.}

Advertisement

\protect\hyperlink{after-bottom}{Continue reading the main story}

\hypertarget{site-index}{%
\subsection{Site Index}\label{site-index}}

\hypertarget{site-information-navigation}{%
\subsection{Site Information
Navigation}\label{site-information-navigation}}

\begin{itemize}
\tightlist
\item
  \href{https://help.nytimes3xbfgragh.onion/hc/en-us/articles/115014792127-Copyright-notice}{©~2020~The
  New York Times Company}
\end{itemize}

\begin{itemize}
\tightlist
\item
  \href{https://www.nytco.com/}{NYTCo}
\item
  \href{https://help.nytimes3xbfgragh.onion/hc/en-us/articles/115015385887-Contact-Us}{Contact
  Us}
\item
  \href{https://www.nytco.com/careers/}{Work with us}
\item
  \href{https://nytmediakit.com/}{Advertise}
\item
  \href{http://www.tbrandstudio.com/}{T Brand Studio}
\item
  \href{https://www.nytimes3xbfgragh.onion/privacy/cookie-policy\#how-do-i-manage-trackers}{Your
  Ad Choices}
\item
  \href{https://www.nytimes3xbfgragh.onion/privacy}{Privacy}
\item
  \href{https://help.nytimes3xbfgragh.onion/hc/en-us/articles/115014893428-Terms-of-service}{Terms
  of Service}
\item
  \href{https://help.nytimes3xbfgragh.onion/hc/en-us/articles/115014893968-Terms-of-sale}{Terms
  of Sale}
\item
  \href{https://spiderbites.nytimes3xbfgragh.onion}{Site Map}
\item
  \href{https://help.nytimes3xbfgragh.onion/hc/en-us}{Help}
\item
  \href{https://www.nytimes3xbfgragh.onion/subscription?campaignId=37WXW}{Subscriptions}
\end{itemize}
