Sections

SEARCH

\protect\hyperlink{site-content}{Skip to
content}\protect\hyperlink{site-index}{Skip to site index}

\href{https://www.nytimes3xbfgragh.onion/section/travel}{Travel}

\href{https://myaccount.nytimes3xbfgragh.onion/auth/login?response_type=cookie\&client_id=vi}{}

\href{https://www.nytimes3xbfgragh.onion/section/todayspaper}{Today's
Paper}

\href{/section/travel}{Travel}\textbar{}Charleston Tourism Is Built on
Southern Charm. Locals Say It's Time to Change.

\url{https://nyti.ms/2XUTDmp}

\begin{itemize}
\item
\item
\item
\item
\item
\item
\end{itemize}

\hypertarget{race-and-america}{%
\subsubsection{\texorpdfstring{\href{https://www.nytimes3xbfgragh.onion/news-event/george-floyd-protests-minneapolis-new-york-los-angeles?name=styln-george-floyd\&region=TOP_BANNER\&block=storyline_menu_recirc\&action=click\&pgtype=Article\&impression_id=99e870a0-f1c5-11ea-b574-81c26c4f5aee\&variant=undefined}{Race
and America}}{Race and America}}\label{race-and-america}}

\begin{itemize}
\tightlist
\item
  \href{https://www.nytimes3xbfgragh.onion/2020/09/04/nyregion/rochester-police-daniel-prude.html?name=styln-george-floyd\&region=TOP_BANNER\&block=storyline_menu_recirc\&action=click\&pgtype=Article\&impression_id=99e870a1-f1c5-11ea-b574-81c26c4f5aee\&variant=undefined}{How
  Police Handled Death of Daniel Prude}
\item
  \href{https://www.nytimes3xbfgragh.onion/2020/09/01/us/politics/trump-fact-check-protests.html?name=styln-george-floyd\&region=TOP_BANNER\&block=storyline_menu_recirc\&action=click\&pgtype=Article\&impression_id=99e897b0-f1c5-11ea-b574-81c26c4f5aee\&variant=undefined}{Trump
  Fact Check}
\item
  \href{https://www.nytimes3xbfgragh.onion/2020/08/30/us/portland-shooting-explained.html?name=styln-george-floyd\&region=TOP_BANNER\&block=storyline_menu_recirc\&action=click\&pgtype=Article\&impression_id=99e897b1-f1c5-11ea-b574-81c26c4f5aee\&variant=undefined}{Portland
  Shooting}
\item
  \href{https://www.nytimes3xbfgragh.onion/2020/08/30/us/breonna-taylor-police-killing.html?name=styln-george-floyd\&region=TOP_BANNER\&block=storyline_menu_recirc\&action=click\&pgtype=Article\&impression_id=99e897b2-f1c5-11ea-b574-81c26c4f5aee\&variant=undefined}{Breonna
  Taylor's Life and Death}
\end{itemize}

Advertisement

\protect\hyperlink{after-top}{Continue reading the main story}

Supported by

\protect\hyperlink{after-sponsor}{Continue reading the main story}

\hypertarget{charleston-tourism-is-built-on-southern-charm-locals-say-its-time-to-change}{%
\section{Charleston Tourism Is Built on Southern Charm. Locals Say It's
Time to
Change.}\label{charleston-tourism-is-built-on-southern-charm-locals-say-its-time-to-change}}

A powerful visitors' bureau has pushed the South Carolina city to the
top of ``best'' lists by selling gentility. Critics say that has come at
the expense of history and the city's Black population.

\includegraphics{https://static01.graylady3jvrrxbe.onion/images/2020/08/17/realestate/17charleston1/merlin_174752205_f5d16456-4aa9-404d-9db5-3c9cb52b34ba-articleLarge.jpg?quality=75\&auto=webp\&disable=upscale}

\href{https://www.nytimes3xbfgragh.onion/by/tariro-mzezewa}{\includegraphics{https://static01.graylady3jvrrxbe.onion/images/2018/08/24/opinion/tariro-headshot/tariro-headshot-thumbLarge-v2.png}}\href{https://www.nytimes3xbfgragh.onion/by/kim-severson}{\includegraphics{https://static01.graylady3jvrrxbe.onion/images/2018/06/13/multimedia/author-kim-severson/author-kim-severson-thumbLarge.jpg}}

By \href{https://www.nytimes3xbfgragh.onion/by/tariro-mzezewa}{Tariro
Mzezewa} and
\href{https://www.nytimes3xbfgragh.onion/by/kim-severson}{Kim Severson}

\begin{itemize}
\item
  Aug. 12, 2020
\item
  \begin{itemize}
  \item
  \item
  \item
  \item
  \item
  \item
  \end{itemize}
\end{itemize}

The week that George Floyd was killed by Minneapolis police officers,
the Charleston, S.C., Convention \& Visitors Bureau introduced a
campaign to assure tourists that despite the coronavirus pandemic,
Charleston --- a city that has topped must-go travel lists for years ---
was ready to welcome them back.

The program asked hotels and restaurants to take a ``White Glove
Pledge,'' which would assure guests a high level of commitment to
hygiene. The campaign's logo was a white-gloved hand holding a tray. The
unwitting reference to the servitude of plantation life came at a moment
when Black Lives Matter protests were beginning to fill streets in
cities across the nation.

Image

``The white glove pledge could not have been any less well-conceived,''
said\href{https://www.nytimes3xbfgragh.onion/2019/10/31/dining/steve-palmer-restaurant-addiction.html}{Steve
Palmer}, the managing partner of the Charleston-based Indigo Road
Hospitality Group, which employs about 1,000 people in 20 restaurants
and bars in four Southern states and Washington, D.C.

Days later, the Black Lives Matter protests reached Charleston and
turned violent. Nearly 125 buildings in the core of the city were
damaged.

The next morning, Helen Hill, the chief executive of the Charleston Area
Convention \& Visitors Bureau, who has been marketing the city for more
than 30 years, sent an email to the bureau's members, praising people
who emerged the next morning to clean up.

``They are sweeping and not weeping!'' she wrote, without acknowledging
the pain that had spurred the protests. ``Please remind your staff who
handles social media to post only uplifting and positive content.
Remember our audience is bigger than local!''

To many who make their living from the 7.4 million people who visit the
Charleston region every year, Ms. Hill's response seemed tone deaf at
best and, at worst, laid bare what has for years been simmering just
below the surface of the city's genteel antebellum image: the delicate
balance between the narrative promoted by the powerful visitors' bureau
and the city's history as the capital of the North American slave trade.
That balance could no longer hold.

The tension between the two story lines is not new. In recent years, the
mostly white leadership of the city and the tourism industry have worked
to highlight the region's African-American heritage. The visitors'
bureau added a deeply reported section on Charleston's African-American
history to its website. And after more than two decades of planning and
fund-raising, the city in 2022 will open the
\href{https://www.nytimes3xbfgragh.onion/2018/03/28/arts/charleston-international-african-american-museum.html}{International
African-American Museum}on Gadsden's Wharf, which had been the first
stop for as many as 100,000 Africans --- an estimated 40 percent of the
people captured and brought to America to be sold into slavery.

But as cultural institutions across the country take a more cleareyed
look at interpreting history in the wake of the Black Lives Matter
movement, the push to change how Charleston tells its own story has
taken on a new urgency.

The bureau, with an annual budget of \$18 million and the ability to
help direct
\href{https://www.postandcourier.com/business/visitors-to-charleston-broke-records-again-exceeding-7-2-million-last-year/article_7e0c7d14-57b0-11e9-a912-671dd61d9f4a.html}{\$8
billion in tourism dollars}to specific businesses, is being asked to do
more to tell a more realistic tale and to support Black-owned business,
many of which have been priced out of the city as its tourism industry
has grown.

``There has been a deliberate effort by very powerful industries and
organizations to sanitize and whitewash Charleston and show a `safe' and
white and palatable Charleston,'' said Mika Gadsden, founder of the
\href{https://www.facebookcorewwwi.onion/CharlestonActivistNetwork/}{Charleston
Activist Network}, a media platform that focuses on Black and Gullah
experiences.

She has become one of the most vocal critics of the C.V.B., as the
visitors' bureau is known, saying that its attempt to soften the city's
history of enslavement with a big serving of genteel Southern charm has
worn thin, particularly during a painful moment for many of the people
who keep the tourism industry moving in Charleston.

\includegraphics{https://static01.graylady3jvrrxbe.onion/images/2020/08/17/realestate/17charleston2/merlin_174752520_5dcdc94f-b751-4220-bc43-c517d3c9bc71-articleLarge.jpg?quality=75\&auto=webp\&disable=upscale}

Ms. Hill, who has worked for the C.V.B. for almost 34 years and has seen
the city become a popular tourist destination, said that her cheery
email after the protests was not unlike what she sends out after a
hurricane. The idea was to show the can-do spirit of the city in the
face of disaster. It was misconstrued to make it seem like she and the
C.V.B. don't care about racial justice, she said.

The agency has been working to leave the ``magnolias and moonlight''
Charleston narrative behind for 15 years, Ms. Hill said. The effort
became more urgent after nine Black people
\href{https://www.nytimes3xbfgragh.onion/2015/06/18/us/church-attacked-in-charleston-south-carolina.html}{were
murdered} by a white supremacist at Mother Emanuel AME Church five years
ago.

The agency worked to promote and educate tourists about the city's
history of slavery, so much so that it has been criticized for
capitalizing on Black people's pain, she said. Finding the right balance
is challenging, with criticism coming from people who think it isn't
focusing enough on the Black experience and those who think it's
overcorrecting, she said.

Still, she said, the C.V.B. can do more.

``We've learned through this period of time that we have to do a better
job of getting the story out to the people that are in Charleston about
what we are doing,'' Ms. Hill said. ``We realize we've got to let our
locals know what we're doing, especially, especially around this
issue.''

The C.V.B. has previously been called out for having few Black members,
a criticism Ms. Hill has responded to by saying that the agency has 31
Black-owned businesses as members out of more than 800.

The agency's budget comes from three sources: Charleston's share of a
state accommodations tax, a state grant that matches industry
contributions and contributions from businesses, which pay \$700 a year
to be part of the C.V.B. For the past two years, Black business owners
have been allowed to join for \$300.

Kwadjo Campbell, president of JC \& Associates, a firm that works on
development for African-Americans in Charleston, and K.J. Kearney, the
founder of \href{https://www.blackfoodfridays.com/}{Black Food Fridays,}
an online campaign that encourages people to patronize Black-owned
restaurants on Fridays, said that the C.V.B. hasn't done enough to
connect with Black Charlestonians.

``We haven't seen a change in dollars going to Black businesses,'' Mr.
Campbell said. ``We haven't seen dollars come through from the C.V.B.
The way this will work is if there are real partnerships and
conversations. Helen's got to listen to Black people in this sector. She
has got to share the wealth.''

Being part of the C.V.B. helps businesses connect more with large tour
groups. Members are promoted on the
\href{https://www.charlestoncvb.com/}{Explore Charleston website} and in
social media channels. When tourists inquire about things to do in the
city or where to eat, they are directed to C.V.B. members. The C.V.B.
spends a third of its budget advertising in magazines like Condé Nast
Traveler, which has named Charleston as its No. 1 destination in the
United States for nine consecutive years.

``The C.V.B. has such power and influence and not just locally,'' said
Allyson Sutton, a co-owner of Sightsee Shop, a store and coffee bar in
the Elliotborough neighborhood in downtown Charleston. She and her
husband, who are both white,
\href{https://www.charlestoncitypaper.com/charleston/why-our-business-resigned-from-the-charleston-visitors-bureau/Content?oid=31714396}{recently
resigned}from the C.V.B. in protest.

``For this organization to have a \$20 million operating budget, a huge
social media following and a website they invest a lot of money into,
and for the bulk of that content to whitewash history, not promote the
incredible Black culture we have now and to not at the very least use
its platforms to say `Black Lives Matter' is incredibly disappointing,''
Ms. Sutton said.

Image

Olivia Williams gives tours of the McLeod Plantation that focus on
enslaved women.~Credit...Hunter McRae for The New York Times

\hypertarget{plantations-and-tours}{%
\subsection{Plantations and tours}\label{plantations-and-tours}}

One point of frustration is the agency's
\href{https://www.charlestonweddingguide.com/weddingplanner/venues~153/plantations-parks~772/}{promotion
of plantation weddings} on its website, where people can take a quiz
that matches them with a venue.

Getting married on the grounds of a plantation has long been sold as a
romantic experience. But critics say that celebrating on plantation
grounds where Black people were tortured, killed and, in many cases,
buried, dishonors their history.

Olivia Williams is a historical interpreter at
\href{https://www.ccprc.com/1447/McLeod-Plantation-Historic-Site}{McLeod
Plantation,} whose tours focus on the quarters where enslaved people
lived rather than the grand home that belonged to the white family. Ms.
Williams's tours focus specifically on enslaved women.

``I'm able to make connections between the history of these women and
treatment of Black women and how that treatment hasn't changed,
especially in the wake of Breonna Taylor and the treatment of Black
trans women that we hear about,'' she said.

She said both the C.V.B. and other historic sites could take a cue from
McLeod and tell the stories of the enslaved more accurately, making
Black experiences more central.

``The narrative many plantations have been telling, that the city has
been telling, is a simple one,'' she said. ``It's not easy transitioning
from this one narrative that seems to have worked in bringing people
here to a difficult one, but it has to happen.''

McLeod stopped allowing people to book weddings on its property in 2019
(weddings that were already scheduled for future dates will still take
place). In December, the Knot Worldwide, one of the biggest
\href{https://www.buzzfeednews.com/article/elisabethdonnelly/zola-the-knot-weddings-millennials}{online
wedding-planning platforms} in the United States, and Pinterest, the
image sharing site, said they would no longer promote images that
romanticize plantations.

Ms. Hill said that many plantations tell the story of slavery well and
shouldn't be excluded from the C.V.B. site.

``There's this whole thought that somehow you shouldn't have celebratory
things happening at this beautiful outdoor venue,'' she said. ``We just
feel really strongly that we want to support our attractions because
they have worked so hard, and if they decide that they want to use their
special facility for weddings, we're going to support them.''

Stephanie Burt, a travel writer and host of
\href{https://www.thesouthernfork.com/}{The Southern Fork podcast,} has
been one of a growing chorus of people lobbying for changes at the
visitors' bureau. In its drive to market Charleston, the agency has
smothered the city's history, she said.

``The focus is tourism at any cost and it doesn't matter if we are
drowning in Covid or are telling the wrong story about slavery,'' she
said. ``The tourism industry is decimating African-American communities
and flattening nuance and narrative.''

Image

The visitor's bureau website offers a quiz that matches brides and
grooms with venues that include plantations. One writer took the quiz
and was directed to Boone Hall Plantation.~Credit...The New York Times

Indeed, the influx of expensive hotels and tourist shops has driven up
the cost of living in Charleston and sent the working class --- many of
whom are African-American --- to less expensive parts of the region.

Since the 1980s, the racial makeup of Charleston has flipped. Once, two
out of every three residents was Black. Now, the city is about 72
percent white.

``Charleston's viability has come at the expense of Black folks,'' Ms.
Gadsden said.

In recent years, the C.V.B. has been unpopular among locals who believe
that it is pushing for tourism at any cost. There have been
yearslong\href{https://www.postandcourier.com/news/an-old-debate-over-cruise-ships-churns-as-a-new-ship-calls-charleston-harbor-home/article_013274d2-7674-11e9-b1c8-7b88d7ccc5c7.html}{battles}
over allowing large cruise ships to dock in the city, complaints about
constant bachelorette and bachelor parties and pushback against the
opening of new hotels in residential areas.

\hypertarget{foods-role}{%
\subsection{Food's role}\label{foods-role}}

Restaurants have played a major part in making Charleston a destination
city. The modern Southern food movement, which blew up the cornpone
national perception of Southern eating popularized by cooks like Paula
Deen, and made stars out of the region's restaurants, was built in large
part in the kitchens of Charleston restaurants like
\href{https://www.nytimes3xbfgragh.onion/2011/02/09/dining/09notebook.html}{Husk},
\href{https://carolinas.eater.com/2018/11/13/18091704/berthas-and-fig-38-essential-restaurants}{FIG}
and
\href{https://www.charlotteobserver.com/living/food-drink/article210780499.html}{Rodney
Scott's BBQ}.

While some in the city's food community worked to better tell the
narrative of the region's Gullah Geechee food traditions and support
Black chefs, many restaurant owners paid their dues to the visitors'
bureau and didn't question how the city was being promoted, Mr. Palmer,
of Indigo Road Hospitality, and others in the industry said.

The latest push for social justice and the Black Lives Matter movement
shifted the tone. The Charleston Wine + Food Festival, which works with
the visitors' bureau and last year attracted nearly 12,000 travelers and
national television coverage,
\href{https://www.instagram.com/p/CBgMUP-D_34/}{announced} in June it
would no longer host events at plantations. In addition, unless the city
removed from Marion Square a 12-foot-tall bronze statue of John C.
Calhoun, a former U.S. vice president who was one of the 19th century's
most prominent defenders of slavery, the festival would no longer stage
its events there.

(Pressure had been mounting for the statue's removal for years, and the
city
\href{https://www.cnn.com/2020/06/24/us/charleston-statue-removal-calhoun-trnd/index.html}{took
it down} on June 24. The food festival organizers recently announced
they would not hold the 2021 festival, usually scheduled for March,
citing the pandemic.)

Image

The statue of John C. Calhoun, a prominent defender of slavery, was
recently taken down from its perch above Charleston's Marion
Square.Credit...Hunter McRae for The New York Times

Festival organizers took criticism from people who thought politics were
outside the event's purview, and others who
\href{https://www.postandcourier.com/blog/raskin_around/charleston-wine-food-festival-blasted-by-activists-for-taking-easiest-stance-on-racism/article_cc7bdd44-b017-11ea-9cd1-230f31744846.html}{called
the move performative} and argued that the organizers should do a better
job in the way they treat people of color they ask to participate, both
as volunteers at the festival and as guest cooks and winemakers.

Gillian Zettler, the executive director of the festival, said the
nonprofit organization had since last fall been examining issues of
diversity and inclusion, including diving deeper into the history of
venues it selects for the festival, creating a more diverse board and
developing deeper relationships with South Carolina's Black hospitality
professionals who have been historically underrepresented at the event.

The organization also has pledged to make the festival more accessible
to African Americans and other people of color.

B.J. Dennis, a chef whose ancestors come from a Gullah Geechee community
in Wando, outside Charleston, has worked with the festival to curate
events that more accurately explore the Gullah Geechee food traditions
developed by West Africans who were enslaved along the southeastern
Atlantic coast.

He has long been an advocate for telling a more complete story about
Charleston, and says he
\href{https://www.nytimes3xbfgragh.onion/2019/05/07/dining/charleston-restaurants.html}{has
watched} with a heavy heart as many Black-owned restaurants have been
priced out of the core of the city. But he remains skeptical that
Charleston is really ready to tell its truth.

``I think people are more aware and have been put on notice with the
movement,'' he said, ``but as far as change, people got to want to
change.''

``To get the plantation narrative to move from the `Gone With the Wind'
narrative to telling the true story of plantations, which is really the
story of concentration camps, is not going to come easy,'' he said.
``But for every two of your blue-blooded faithful customers you may lose
by telling the truth, you may gain 10 to 20 followers who will want to
hear the real story.''

\emph{\textbf{Follow New York Times Travel}}
\emph{on}\href{https://www.instagram.com/nytimestravel/}{\emph{Instagram}}\emph{,}\href{https://twitter.com/nytimestravel}{\emph{Twitter}}
\emph{and}\href{https://www.facebookcorewwwi.onion/nytimestravel/}{\emph{Facebook}}\emph{.
And}\href{https://www.nytimes3xbfgragh.onion/newsletters/traveldispatch}{\emph{sign
up for our weekly Travel Dispatch newsletter}} \emph{to receive expert
tips on traveling smarter and inspiration for your next vacation.
Dreaming up a future getaway or just armchair traveling? Check out
our}\href{https://www.nytimes3xbfgragh.onion/interactive/2020/travel/places-to-visit.html}{\emph{52
Places list}}\emph{.}

Advertisement

\protect\hyperlink{after-bottom}{Continue reading the main story}

\hypertarget{site-index}{%
\subsection{Site Index}\label{site-index}}

\hypertarget{site-information-navigation}{%
\subsection{Site Information
Navigation}\label{site-information-navigation}}

\begin{itemize}
\tightlist
\item
  \href{https://help.nytimes3xbfgragh.onion/hc/en-us/articles/115014792127-Copyright-notice}{©~2020~The
  New York Times Company}
\end{itemize}

\begin{itemize}
\tightlist
\item
  \href{https://www.nytco.com/}{NYTCo}
\item
  \href{https://help.nytimes3xbfgragh.onion/hc/en-us/articles/115015385887-Contact-Us}{Contact
  Us}
\item
  \href{https://www.nytco.com/careers/}{Work with us}
\item
  \href{https://nytmediakit.com/}{Advertise}
\item
  \href{http://www.tbrandstudio.com/}{T Brand Studio}
\item
  \href{https://www.nytimes3xbfgragh.onion/privacy/cookie-policy\#how-do-i-manage-trackers}{Your
  Ad Choices}
\item
  \href{https://www.nytimes3xbfgragh.onion/privacy}{Privacy}
\item
  \href{https://help.nytimes3xbfgragh.onion/hc/en-us/articles/115014893428-Terms-of-service}{Terms
  of Service}
\item
  \href{https://help.nytimes3xbfgragh.onion/hc/en-us/articles/115014893968-Terms-of-sale}{Terms
  of Sale}
\item
  \href{https://spiderbites.nytimes3xbfgragh.onion}{Site Map}
\item
  \href{https://help.nytimes3xbfgragh.onion/hc/en-us}{Help}
\item
  \href{https://www.nytimes3xbfgragh.onion/subscription?campaignId=37WXW}{Subscriptions}
\end{itemize}
