Sections

SEARCH

\protect\hyperlink{site-content}{Skip to
content}\protect\hyperlink{site-index}{Skip to site index}

\href{https://www.nytimes3xbfgragh.onion/section/us}{U.S.}

\href{https://myaccount.nytimes3xbfgragh.onion/auth/login?response_type=cookie\&client_id=vi}{}

\href{https://www.nytimes3xbfgragh.onion/section/todayspaper}{Today's
Paper}

\href{/section/us}{U.S.}\textbar{}1,193 Quarantined for Covid. Is This a
Successful School Reopening?

\url{https://nyti.ms/3kwapSF}

\begin{itemize}
\item
\item
\item
\item
\item
\item
\end{itemize}

\hypertarget{school-reopenings}{%
\subsubsection{\texorpdfstring{\href{https://www.nytimes3xbfgragh.onion/spotlight/schools-reopening?name=styln-coronavirus-schools-reopening\&region=TOP_BANNER\&block=storyline_menu_recirc\&action=click\&pgtype=Article\&impression_id=4ec9b8f0-f2a5-11ea-ac64-8d7d25b062bd\&variant=undefined}{School
Reopenings}}{School Reopenings}}\label{school-reopenings}}

\begin{itemize}
\tightlist
\item
  \href{https://www.nytimes3xbfgragh.onion/2020/09/04/us/bar-exam-coronavirus.html?name=styln-coronavirus-schools-reopening\&region=TOP_BANNER\&block=storyline_menu_recirc\&action=click\&pgtype=Article\&impression_id=4ed13300-f2a5-11ea-ac64-8d7d25b062bd\&variant=undefined}{Delayed
  Licensing Exams}
\item
  \href{https://www.nytimes3xbfgragh.onion/2020/09/08/upshot/children-testing-shortfalls-virus.html?name=styln-coronavirus-schools-reopening\&region=TOP_BANNER\&block=storyline_menu_recirc\&action=click\&pgtype=Article\&impression_id=4ed13301-f2a5-11ea-ac64-8d7d25b062bd\&variant=undefined}{Limited
  Testing for Children}
\item
  \href{https://www.nytimes3xbfgragh.onion/2020/09/01/world/schools-reopen-globe-students.html?name=styln-coronavirus-schools-reopening\&region=TOP_BANNER\&block=storyline_menu_recirc\&action=click\&pgtype=Article\&impression_id=4ed13302-f2a5-11ea-ac64-8d7d25b062bd\&variant=undefined}{School
  Around the World}
\item
  \href{https://www.nytimes3xbfgragh.onion/interactive/2020/us/covid-college-cases-tracker.html?name=styln-coronavirus-schools-reopening\&region=TOP_BANNER\&block=storyline_menu_recirc\&action=click\&pgtype=Article\&impression_id=4ed15a10-f2a5-11ea-ac64-8d7d25b062bd\&variant=undefined}{Tracking
  College Cases}
\end{itemize}

Advertisement

\protect\hyperlink{after-top}{Continue reading the main story}

Supported by

\protect\hyperlink{after-sponsor}{Continue reading the main story}

\hypertarget{1193-quarantined-for-covid-is-this-a-successful-school-reopening}{%
\section{1,193 Quarantined for Covid. Is This a Successful School
Reopening?}\label{1193-quarantined-for-covid-is-this-a-successful-school-reopening}}

A suburban Atlanta county opened its schools amid controversy and a
growing case count, previewing a difficult national back-to-school
season.

\includegraphics{https://static01.graylady3jvrrxbe.onion/images/2020/08/10/us/00VIRUS-GASCHOOLS-hasty/merlin_175508088_61155bfd-1e33-457a-b4cb-a784bd2ff341-articleLarge.jpg?quality=75\&auto=webp\&disable=upscale}

By \href{https://www.nytimes3xbfgragh.onion/by/richard-fausset}{Richard
Fausset}

\begin{itemize}
\item
  Published Aug. 12, 2020Updated Aug. 14, 2020
\item
  \begin{itemize}
  \item
  \item
  \item
  \item
  \item
  \item
  \end{itemize}
\end{itemize}

CANTON, Ga. --- The first letter went out on Aug. 4, one day after
students in the Cherokee County School District returned to their
classrooms for the first time since the eruption of the coronavirus
pandemic.

``Dear Parents,'' wrote Dr. Ashley Kennerly, the principal of Sixes
Elementary School. ``I am writing this letter in order to communicate
that a student in 2nd grade has tested positive for Covid-19.''

By the time the last bell rang on Friday afternoon, principals at 10
other schools had sent similar letters to families in Cherokee County, a
bucolic and politically conservative stretch of suburbs north of
Atlanta. This week, more letters went out.

Altogether, nearly 1,200 students and staff members in the district have
already been ordered to quarantine. On Tuesday, one high school
\href{https://www.cherokeek12.net/News/81120message\#sthash.LNefigAz.dpbs}{closed
its doors} until at least Aug. 31. A
\href{https://www.cherokeek12.net/News/81220message\#sthash.YR2pGNJt.dpbs}{second
high school} followed on Wednesday.

While many of the nation's largest school systems have opted in recent
weeks to start the academic year online, other districts have forged
ahead with reopening. In Georgia, Tennessee, Mississippi, Indiana and
elsewhere, some schools, mainly in suburban and rural areas, have been
open for almost two weeks.

Their experience reveals the perils of returning to classrooms in places
where the coronavirus has hardly been tamed. Students and teachers have
\href{https://www.nytimes3xbfgragh.onion/2020/08/01/us/schools-reopening-indiana-coronavirus.html}{immediately
tested positive}, sending others into two-week quarantines and creating
whiplash for schools that were eager to open, only to have to
\href{https://www.nytimes3xbfgragh.onion/2020/08/03/us/school-closing-coronavirus.html}{consider
closing again right away}.

All of this has only further divided communities where parents and
teachers have passionately disagreed over the safety of reopening.

Depending on whom you ask, the string of positive tests and isolation
orders in Cherokee County either proved the district's folly for opening
schools during the worst American public health crisis in decades, or
demonstrated a courageous effort to return to normal.

``This is exactly what we expected to happen,'' said Allison Webb, 44,
who quit her job as a Spanish and French teacher in the district because
of her concerns about reopening schools, and who put her daughter, a
senior, in the district's remote-learning program. ``It's not safe'' to
return to the classrooms now, Ms. Webb said.

But to Jenny Beth Martin, who wanted schools to reopen --- even
appealing directly to President Trump in a visit to the White House ---
the district's return has been a rousing success.

``I think that the opening plan is working,'' said Ms. Martin, a
district parent and co-founder of the national Tea Party Patriots, a
conservative political group. ``They're checking, they're making sure
when people have tested positive that they're watching the exposure and
spread.''

\includegraphics{https://static01.graylady3jvrrxbe.onion/images/2020/08/10/us/00VIRUS-GASCHOOLS-street/merlin_175508097_7002ce06-2c61-44bd-beec-a78ce4f41afd-articleLarge.jpg?quality=75\&auto=webp\&disable=upscale}

Controversy emerged on Day 1, as schools opened in Cherokee and nearby
Paulding County on Aug. 3. At North Paulding High School, at least one
student was suspended, then unsuspended, for
\href{https://www.nytimes3xbfgragh.onion/2020/08/06/us/north-paulding-high-school-coronavirus-georgia.html}{sharing
photos of crowded hallways} on social media, prompting a national
uproar. Her school
\href{https://www.nytimes3xbfgragh.onion/2020/08/09/world/coronavirus-covid-19.html\#link-4c96afe0}{closed
for at least three days} this week after nine positive cases emerged.

Cherokee County had its own firestorm. A photo taken outside Etowah High
School on the first day back
\href{https://twitter.com/GAFollowers/status/1290437298685968385}{showed
scores of students} crowded shoulder to shoulder, smiling and unmasked.
A similar photo from Sequoyah High School was also posted to social
media. Beneath the photo,
\href{https://twitter.com/iyanilenicetv/status/1290428424482414594}{a
commenter wrote}, ``Most of these kids are gonna be sick in the next few
days \ldots{} was it really worth it to appease the anti-mask parents?
At what cost?''

The county's reopening plan was unanimously approved by the school board
on July 9. Families could choose online or in-person, five-days-a-week
instruction, and masks would be encouraged, but not required, for the
district's 42,500 students.

Opposition began to coalesce almost immediately. Ms. Webb, the foreign
language teacher, organized a group on Facebook called Educators for
Common Sense and Safety. The group started an online petition asking
for, among other things, a mask mandate for students and a delayed start
to allow time to rework schedules, classrooms and the curriculum ``to be
safe and engaging for our students.'' It attracted more than 1,100
signatures.

In mid-July, the group, which Ms. Webb said currently counts hundreds of
members,
\href{https://www.gpb.org/news/2020/07/16/crowd-protests-cherokee-county-school-system-reopening-demands-mask-mandate}{picketed
outside a board meeting}. A former English teacher, Miranda Wicker, 38,
became its spokesperson --- a necessity, she said, because current
teachers lacked union protection and feared retaliation if they spoke
out.

``They're terrified,'' Ms. Wicker said. ``They're being asked,
literally, to risk their lives.''

Image

``They're being asked, literally, to risk their lives,'' said~Miranda
Wicker, a former teacher who has become a spokesperson for the
group~Educators for Common Sense and Safety.Credit...Dustin Chambers for
The New York Times

But proponents of reopening, including Ms. Martin, cheered the district
on. Two days before the school board vote in July, she appeared in a
round-table discussion with Mr. Trump at the White House. **** Parents
needed to go to their jobs, she told him. Students needed to be with
their teachers.

``America is not meant to shut down,'' she said.

Late last month, Ms. Martin was
\href{https://www.nbcnews.com/tech/social-media/dark-money-pac-s-coordinated-reopen-push-are-behind-doctors-n1235100}{an
organizer} of a Washington news conference featuring people who
identified themselves as doctors and who made misleading statements
about the coronavirus, including unsupported claims that the drug
hydroxychloroquine was an effective treatment. Mr. Trump tweeted a video
of the event, which was later
\href{https://www.nytimes3xbfgragh.onion/2020/07/28/technology/virus-video-trump.html}{removed
from major social media platforms} on the grounds that it was spreading
misinformation.

In early July, when the school board approved reopening, case tallies in
Cherokee County, with about 260,000 people, had only begun to rise after
remaining flat and relatively low --- an average of about 10 new
confirmed cases a day --- for most of June. Since then, though, the
numbers have climbed steadily, mirroring the state as a whole, with the
county averaging
\href{https://www.nytimes3xbfgragh.onion/interactive/2020/us/georgia-coronavirus-cases.html\#county}{more
than 90 new confirmed cases} daily over the past week. Sixty-four people
in the county have died of Covid-19, including
\href{https://www.nytimes3xbfgragh.onion/interactive/2020/us/georgia-coronavirus-cases.html\#county}{eight
in the past week}.

As the first day of school approached, Ms. Webb, the foreign language
teacher, asked if she could erect a plexiglass barrier in her classroom.
The district's risk management director, Melissa Whatley, told Ms. Webb
in an email that she could not ``make modifications to a classroom
beyond the scope of the approved C.C.S.D. Reopening of Schools Plan.''
Ms. Webb then resigned from teaching and took a job at a law firm.

\includegraphics{https://static01.graylady3jvrrxbe.onion/images/2017/01/29/podcasts/the-daily-album-art/the-daily-album-art-articleInline-v2.jpg?quality=75\&auto=webp\&disable=upscale}

\hypertarget{listen-to-the-daily-why-teachers-arent-ready-to-reopen-schools}{%
\subsubsection{Listen to `The Daily': Why Teachers Aren't Ready to
Reopen
Schools}\label{listen-to-the-daily-why-teachers-arent-ready-to-reopen-schools}}

The president and some parents are demanding schools reopen for
in-person learning --- but teachers and unions are resisting that call.

transcript

Back to The Daily

bars

0:00/27:53

-27:53

transcript

\hypertarget{listen-to-the-daily-why-teachers-arent-ready-to-reopen-schools-1}{%
\subsection{Listen to `The Daily': Why Teachers Aren't Ready to Reopen
Schools}\label{listen-to-the-daily-why-teachers-arent-ready-to-reopen-schools-1}}

\hypertarget{hosted-by-michael-barbaro-produced-by-sydney-harper-and-annie-brown-with-help-from-rachelle-bonja-and-edited-by-lisa-chow}{%
\subsubsection{Hosted by Michael Barbaro; produced by Sydney Harper and
Annie Brown; with help from Rachelle Bonja; and edited by Lisa
Chow}\label{hosted-by-michael-barbaro-produced-by-sydney-harper-and-annie-brown-with-help-from-rachelle-bonja-and-edited-by-lisa-chow}}

\hypertarget{the-president-and-some-parents-are-demanding-schools-reopen-for-in-person-learning--but-teachers-and-unions-are-resisting-that-call}{%
\paragraph{The president and some parents are demanding schools reopen
for in-person learning --- but teachers and unions are resisting that
call.}\label{the-president-and-some-parents-are-demanding-schools-reopen-for-in-person-learning--but-teachers-and-unions-are-resisting-that-call}}

\begin{itemize}
\item
  michael barbaro\\
  From The New York Times, I'm Michael Barbaro. This is ``The Daily.''
\item
  {[}music{]}\\
  So far, the debate over school reopenings has been dominated by a
  president who is determined to send students back into classrooms ---
\item
  archived recording (donald trump)\\
  We want to reopen the schools. Everybody wants it. The moms want it.
  The dads want it. The kids want it. It's time to do it.
\end{itemize}

michael barbaro

--- and by local school officials, who are answering that call.

\begin{itemize}
\tightlist
\item
  archived recording (donald trump)\\
  So we're very much going to put pressure on governors and everybody
  else to open the schools.
\end{itemize}

michael barbaro

Today: My colleague Dana Goldstein on why teachers and their unions are
defying those plans.

It's Thursday, August 13.

\begin{itemize}
\tightlist
\item
  archived recording (ron desantis)\\
  Good evening. I stand here tonight not only as governor of Florida,
  but as a husband, a father, a son and a friend to have a conversation
  about how we as Floridians approach these challenging times. As a
  parent of three, I know that my fellow parents here in Florida want
  nothing more than to provide a bright future for their children. And
  here's the hard truth. While the risks to students from in-person
  learning are low, the cost of keeping schools closed are enormous.
\end{itemize}

michael barbaro

Dana, tell me about this situation with schools in Florida.

dana goldstein

In early July, just as the Trump administration from Washington was
pushing schools to reopen their physical campuses across the country,
Florida was the state that really leaned heavily in that same direction
under their Republican Governor Ron DeSantis.

\begin{itemize}
\tightlist
\item
  archived recording (ron desantis)\\
  The important thing is that our parents have a meaningful choice when
  it comes to in-person education. Let's not let fear get the best of us
  and harm our children in the process.
\end{itemize}

dana goldstein

The state issued this executive order.

\begin{itemize}
\tightlist
\item
  archived recording\\
  The state is announcing it's requiring all schools to reopen for
  in-person classes next month, August.
\end{itemize}

dana goldstein

Telling schools that they had to reopen five days a week.

\begin{itemize}
\item
  archived recording 1\\
  So that announcement coming today, given where Florida is. Your
  analysis.
\item
  archived recording 2\\
  I mean my analysis is that that is insane
\end{itemize}

dana goldstein

And this was shocking to superintendents and school boards. You know,
they had spent the months of May, June, into July mostly planning for a
hybrid model of education. Kids would go to school two or three, or
maybe even just one day a week in person, and be home learning online
the rest of the time. School districts all of a sudden were being told
you have to offer parents and families the option of five days a week in
the building.

\begin{itemize}
\tightlist
\item
  archived recording\\
  So we are not ready to open schools in four weeks. We need to slow
  down and take a pause and get this right around the state first.
\end{itemize}

michael barbaro

And what would happen if schools didn't physically reopen five days a
week?

dana goldstein

You know, I think the kind of underlying threat was that you would lose
state dollars if you don't provide families with this option for
in-person learning. And this threat to them was quite scary. Because
state funding for education is the main funding that funds our school
system in the United States.

michael barbaro

And what was the state of the pandemic when the state of Florida makes
this demand?

dana goldstein

So these numbers were so shocking to us when we did reporting on this
that we actually fact checked them many, many times to make sure they
were correct.

\begin{itemize}
\tightlist
\item
  archived recording\\
  Florida shattering its daily record, recording more than 15,000 cases,
  accounting for a quarter of the total new daily cases in the United
  States.
\end{itemize}

dana goldstein

In some south Florida counties in the month of July ---

\begin{itemize}
\tightlist
\item
  archived recording\\
  South Florida's Miami-Dade has seen a staggering daily positivity rate
  of 33 percent.
\end{itemize}

dana goldstein

--- between 20 and 30 percent of coronavirus tests were coming back
positive. And the World Health Organization, the state of California,
the state of New York have tended to use a range of about 5 percent to
10 percent test positivity rates as something to look at when deciding
whether or not to open schools. So here you might see, you know, four
times that number in a city like Miami.

\begin{itemize}
\tightlist
\item
  archived recording\\
  Here in Miami-Dade, according to county data released yesterday, the
  goal for the county is not to exceed 10 percent. They have exceeded
  that for the past 14 days.
\end{itemize}

dana goldstein

A strong indication that the virus is completely unchecked in that
region. In fact, it was one of the most dangerous cities for the virus
in the United States.

michael barbaro

Right. So what was the reaction across Florida to this executive order?

dana goldstein

Anger.

\begin{itemize}
\tightlist
\item
  archived recording\\
  If the governor wants to open schools publicly, how about we invite
  him to come and teach in the classroom? {[}CHEERING{]}
\end{itemize}

dana goldstein

A lot of teachers and educators were angry.

\begin{itemize}
\tightlist
\item
  archived recording\\
  If he wants to open schools, how about he provide teachers with hazard
  pay? Because that's exactly what you're doing. You're on the
  frontlines of a pandemic that you didn't start, you didn't call for
  and we don't have control for. {[}CHEERING{]}
\end{itemize}

dana goldstein

Because they felt that their safety and, in some respects, safety of the
entire community from a public health perspective was nowhere in this
conversation.

\begin{itemize}
\tightlist
\item
  archived recording\\
  I teach my students the history of America, how this government has
  run, how it works. This is a democracy. Our voices need to be heard.
\end{itemize}

dana goldstein

And my inbox and social media were filled with messages from teachers.

\begin{itemize}
\tightlist
\item
  archived recording\\
  So I want everyone to hear my voice that if I die from catching
  Covid-19 from being forced back into Pinellas County Schools, you can
  drop my dead body right here! Leave my body right here! {[}CHEERING{]}
\end{itemize}

{[}music{]}

dana goldstein

And it was just this sense that the question of whether we should go
back did not pay enough attention to teachers' health risks.

\begin{itemize}
\item
  archived recording 1\\
  Do you feel ready to return to your classroom?
\item
  archived recording 2\\
  I do not. I personally have lost sleep over it. I've cried over it. I
  cry over it a lot. It's very, very scary. And the one thing I'm going
  to say, I will say online learning is not ideal. But it will keep our
  children safe.
\item
  archived recording\\
  I'm a teacher. I've been with Duval County for 23 years. I have a
  mother at home that is sick. And if I am to get the coronavirus, I
  don't want to bring it back to her.
\end{itemize}

dana goldstein

Yes, it's really important that kids get educated. It's really important
that parents be able to work during the day and children have the basic
childcare that schools provide. However ---

\begin{itemize}
\item
  archived recording 1\\
  We teachers love our students. And we agree that the best place for
  students is in school. But that's only if they're safe. If going to
  school is more dangerous for students or for their families, then we
  should hold off and do some sort of distance learning or a hybrid
  model until it's safe for them.
\item
  archived recording 2\\
  I think there's no way to social distance in our already crowded
  classrooms. There is not enough money to provide for the extra staff
  that we would need and the extra P.P.E. that we would need. I don't
  think that it's worth the risk.
\end{itemize}

dana goldstein

We are used to going into schools that sometimes don't have soap in the
bathrooms, that sometimes have broken windows that prevent us from
circulating fresh air, that have dated heating and ventilation systems.
And where is our health in this equation?

\begin{itemize}
\tightlist
\item
  archived recording\\
  This is not how I want to go back. And I want to go back so bad.
  Because I love teaching. I miss my classroom. I miss my kids.
\end{itemize}

michael barbaro

So what did teachers in Florida do?

\begin{itemize}
\tightlist
\item
  archived recording\\
  The largest teachers union in Florida is suing the state over its
  executive order mandating that schools reopen next month with
  in-person instruction.
\end{itemize}

dana goldstein

So a bunch of the local and national union groups that represent
teachers came together and they sued the state of Florida.

\begin{itemize}
\tightlist
\item
  archived recording\\
  In the lawsuit, the union says the state is unconstitutionally forcing
  millions of students and teachers into unsafe schools.
\end{itemize}

dana goldstein

Saying that this executive order requiring schools to reopen five days a
week in person actually violated Florida's own state law that also calls
for schools to be safe.

\begin{itemize}
\tightlist
\item
  archived recording\\
  The suit says children are at risk of contracting and spreading the
  virus and of developing severe illness, resulting in death. And the
  state mandate to open schools is impossible to comply with C.D.C.
  guidelines on physical distancing, hygiene and sanitation if schools
  are operating at full capacity.
\end{itemize}

dana goldstein

It's really very simple what they were arguing, that going back five
days a week is not safe and therefore, cannot be legal.

michael barbaro

Huh. I have to think that it's a pretty unusual act, you know, teachers
suing to stop their own schools from reopening.

dana goldstein

Yes. It's definitely unusual and notable. And interestingly, it paved
the way for similar threats to sue across the country, including in
northern cities like Chicago and New York. And shortly after this
Florida suit came down ---

\begin{itemize}
\tightlist
\item
  archived recording\\
  The American Federation of Teachers has told its 1.7 million members
  that if they choose to strike, the union will have their back.
\end{itemize}

dana goldstein

The American Federation of Teachers, which is one of the two national
unions, authorized any of their locals across the country to plan a
strike in the event that safety precautions are not being met to reopen
schools.

michael barbaro

Wow. So a national teachers union is saying, a grounds for striking ---
which traditionally we've always thought of as wages, health care, those
kinds of issues --- they're now saying you may decide to strike over
unsafe school conditions in the middle of this pandemic?

dana goldstein

Exactly. The threat to strike is very powerful and pragmatic. Because
once teachers threaten to strike over the safety measures and questions
of funding, it really puts pressure on the local school districts to
give them a big seat at the table. And just the core decision, which is,
are we even going to try to have in-person school this fall?

michael barbaro

We'll be right back.

So Dana, as teachers are seeking a place at the table and threatening to
strike if they don't feel like schools are safe, what exactly are they
asking for in order to feel ready to return to the classroom?

dana goldstein

We're seeing a very broad range of demands from teachers. And it runs
the spectrum from very specific and achievable requests, to ones that
are hugely ambitious, time consuming, expensive, or maybe even
impossible to achieve while we're still experiencing any transmission of
Covid-19.

michael barbaro

What do you mean?

dana goldstein

So for example in Orlando, when I spoke to teachers there in July, the
requests were really quite reasonable. They wanted face masks to be
required. They wanted temperature checks in all school district
buildings. And then, the American Federation of Teachers, the national
union that authorized strikes, had a very specific set of demands that
they were looking for nationally. They wanted to see test positivity
rates for the virus below 5 percent, transmission rates below 1 percent,
effective contact tracing for the entire region, the school to require
masks, update ventilation systems, and put in place procedures to
maintain six feet of distance.

michael barbaro

Wow.

dana goldstein

So very much sort of in line with C.D.C. guidelines for being as safe as
possible.

michael barbaro

So the union is making demands of an entire community, and level of
infection and transmission and contact tracing beyond the school?

dana goldstein

Exactly. They're expecting those things to work in the whole region
before you sort of even get to the question of what sort of P.P.E. is
available to teachers or something like that.

michael barbaro

What about less practical requests from teachers?

dana goldstein

So there you see this big movement bubbling up on social media under the
hashtag \#14daysnonewcases. And this is really quite a radical demand
for schools not to reopen physically until there are no new cases in a
region for 14 days. Now many nations have been able to reopen their
schools safely without achieving that standard. And when I've spoken to
public health experts about this, what they say is, you know, ``14 days
no new cases'' is not just a controlled pandemic, it's essentially the
end of the pandemic in that region. And it might require a vaccine to
get to that standard. Not just a vaccine that exists and works, but that
has actually been deployed widely. When will that occur? Will that occur
six months from now, 12 months from now, two years from now? We just
don't know the answer to that. And those start to be very big numbers
when you're thinking about children being out of school.

michael barbaro

I wonder what these demands from teachers look like to parents in this
moment. I mean, I'm mindful that many parents want their kids to return
to school for a variety of very understandable reasons.

dana goldstein

That's right. I mean, I think the really hard thing is that there is no
consensus or even strong majority opinion among parents. One recent
national poll found about 60 percent of parents at this moment believe
it's smarter to delay reopening physical schools until the virus
subsides somewhat and there are more safety measures in place. But in
some big cities, where the virus has been relatively well-controlled,
like New York and Chicago, polls have found that a majority of families
do have some willingness to send their kids back to school.

And to add another layer of complication, it tends to be parents of
color and low income parents that are the most scared of the health
threats to their children of congregating in school buildings. But those
families are also the most concerned about their kids falling back
socially and academically because schools are closed. So there is just
no consensus among parents as to what they feel is safe. It would in
some ways be easier if American parents all agreed with each other about
what was right here.

michael barbaro

Mhm. And of course in the absence of physically returning to schools,
we're left with online learning. And we have covered on the show the
problems with how teachers and school districts are approaching that.

dana goldstein

Yeah. So in the spring, only a small segment of American school
districts actually required teachers to teach live lessons over
something like Zoom video. And here I think there is actually more risk
of tension between parents and teachers. Because we're starting to see
from polls what parents are asking for in a situation of continued
remote learning.

They were not happy that in the spring, many of their kids did not see
teachers live over video. Many teachers were interacting with their
students primarily over email at sort of random times per day. And
that's not what parents want.

They want their students to log on at very specific times and be in
something like an online class, where they would have small group
breakout sessions and discussions and have the opportunity to ask the
teacher questions and get individualized feedback. And teachers unions
are still, in some cases, resisting some of these practices, including
even showing their faces on live video.

michael barbaro

And Dana, why would that be? I guess I'm confused. If teachers are
deeply reluctant to return to schools for very understandable reasons
that you just outlined, and they don't feel school districts are meeting
them halfway, why would they simultaneously be resisting a more enriched
online remote teaching experience?

dana goldstein

Well, some of them make the argument that it's not fair to provide too
much live instruction, because students who don't have an adult to
supervise their online learning at home, say, at exactly 10:00 a.m.,
might just miss out on the live lesson. So they think that that mode of
education is not effective.

But I've also heard some arguments much simpler than that, that they
don't want their homes to be shown. They're not comfortable in that
medium. And they believe it's a violation of their own privacy to be
shown from home in that way. So it's a range of different arguments
there.

michael barbaro

That would seem to raise a real crisis. I mean, teachers both not
wanting to be in classrooms, but also not wanting to teach online the
way parents want them to.

dana goldstein

Well, this has been the sort of crux of these very tense latest
negotiations across the country between teachers and school district
leaders.

michael barbaro

Dana, I know a bunch of school districts around the country have
actually started classes in schools. And I wonder how that has played
out.

dana goldstein

Well, there have been some horror stories, unfortunately.

\begin{itemize}
\tightlist
\item
  archived recording\\
  In Georgia, this photo of a crowded hallway, no mask in sight, from
  North Paulding High School went viral after the school opened for
  in-person learning on August 3.
\end{itemize}

dana goldstein

You know, for one of the first school districts to reopen, which was in
Georgia, hundreds of staff were told to stay home because of potential
exposure to the virus.

\begin{itemize}
\tightlist
\item
  archived recording\\
  Today the school remain closed, a week after that reopening.
\end{itemize}

dana goldstein

In Indiana ---

\begin{itemize}
\tightlist
\item
  archived recording\\
  One student at Greenfield Central Junior High tested positive on the
  very first day of school.
\end{itemize}

dana goldstein

--- right away this junior high school was having to call teachers and
call students' families and ask them to stay home for two weeks.

\begin{itemize}
\tightlist
\item
  archived recording\\
  Students at Elwood Junior Senior High now have to go remote after
  staff members there tested positive for Covid-19.
\end{itemize}

dana goldstein

Now that's extremely alarming. But I want to say that nobody who's a
public health or education expert believes that we're going to reopen
schools without students and teachers showing up from time to time
positive for Covid-19. That's not a realistic expectation.

But what we do need is procedures in place to deal with that when it
happens. I mean, it needs to be clear who is getting told to stay home
for two weeks. And, is their access to testing for anyone who came in
contact with that positive individual? So in many ways, I think these
anecdotes that we're hearing of kind of first-day-back crises in towns
and cities that are trying to reopen physically do show that many of the
concerns that teachers have brought to the table here are quite
legitimate.

michael barbaro

So those are a small number of districts that have already reopened. But
of course, many of the nation's largest school districts --- Chicago,
Los Angeles, Washington, D.C., among others, are now firmly saying that
they will not physically reopen schools at least initially. And that
represents millions of students. So do teachers unions and teachers see
that as a kind of victory?

dana goldstein

They do see it as a victory, absolutely. They believe that it's not only
what's necessary to protect their health but to prevent schools emerging
as potential hot spots for spreading Covid-19.

But I think within that victory, there is also a real tragedy for
American children and actually for our country.

Because to be in a place where the needs of public health and safety are
really juxtaposed against our ability to fully educate our kids, is to
be in a place that very few other developed nations are in right now.
And it is because of our failure to control the pandemic itself. We are
looking at the real likelihood that millions or tens of millions of
children do not attend school for an entire year. A full year of no
school.

And we just know that it's going to lead to big problems. It's going to
make kids less likely to learn to read. It's going to probably lead to
higher high school dropout rates. It's going to lead to students who
don't have enough to eat, because school is where they are fed. And to
students that don't have access to the mental health counseling and the
special education services that they get at schools.

So the fact that we're having to choose between everything crucial that
the physical school provides and public health, it's stunning. It's
stunning to me as a 15-year veteran on the education beat and just also
as a parent. You know, my daughter is going to come through this
pandemic just fine. She has access to a great childcare and we have a
lot of resources in our home and family to bring her through this.

But still, it's really sad for our family that she's missing the
preschool experience that we really wanted her to have. It's been months
since she was with teachers and socializing with a group of students.
And she's started even to become more timid around other kids, we've
noticed when we do take those walks out to the playground. And you know,
it's sad for our family. And it's just a tiny microcosm of how sad it is
for our country.

michael barbaro

Dana, thank you very much.

dana goldstein

Thank you so much, Michael.

michael barbaro

Starting this week, several Florida school districts began holding
in-person classes, even as the lawsuit filed by the state's teachers
union moves ahead. A court hearing in that case is scheduled for later
today. Meanwhile, in New York City on Wednesday, the influential unions
representing principals and teachers called on the city to delay
starting in-person instruction by several weeks. In a statement, one of
the union's leaders said that the city had failed to address teachers'
safety concerns and had failed to give them enough time to implement
complicated safety protocols.

We'll be right back. Here's what else you need to know today.

\begin{itemize}
\tightlist
\item
  archived recording (joe biden)\\
  Good afternoon, everyone. To me and to Kamala, this is an exciting
  day. It's a great day for our campaign and it's a great day for
  America, in my view.
\end{itemize}

michael barbaro

During their first joint appearance as a ticket on Wednesday, Joe Biden
praised Kamala Harris for her record as the attorney general of
California and as a United States senator, calling her an unapologetic
advocate for justice.

\begin{itemize}
\tightlist
\item
  archived recording (kamala harris)\\
  Thank you, Joe. Thank you, Joe. As I said, Joe, when you called me, I
  am incredibly honored by this responsibility. And I'm ready to get to
  work. I am ready to get to work.
\end{itemize}

michael barbaro

In her remarks, Harris immediately delivered a stinging indictment of
President Trump as a self-absorbed leader who has repeatedly failed
America, above all, during the pandemic.

\begin{itemize}
\tightlist
\item
  archived recording (kamala harris)\\
  America is crying out for leadership. Yet we have a president who
  cares more about himself than the people who elected him. A president
  who is making every challenge we face even more difficult to solve.
  But here's the good news. We don't have to accept the failed
  government of Donald Trump and Mike Pence. In just 83 days, we have a
  chance to choose a better future.
\end{itemize}

michael barbaro

And ---

\begin{itemize}
\tightlist
\item
  archived recording (dr. anthony fauci)\\
  I hope that the Russians have actually definitively proven that the
  vaccine is safe and effective. I seriously doubt that they've done
  that.
\end{itemize}

michael barbaro

The Trump administration's top adviser on the pandemic, Dr. Anthony
Fauci, expressed deep doubts about Russia's rushed plan to distribute a
vaccine for the coronavirus. The vaccine, called Sputnik V, was approved
by Russia's government without evidence that the largest and most
important phase of human testing had ever occurred.

\begin{itemize}
\tightlist
\item
  archived recording (anthony fauci)\\
  So if we wanted to take the chance of hurting a lot of people or
  giving them something that doesn't work, we could start doing this,
  you know, next week if we wanted to. But that's not the way it works.
\end{itemize}

michael barbaro

That's it for ``The Daily.'' I'm Michael Barbaro. See you tomorrow.

Ms. Wicker pulled her two children out and decided to home-school them.
But the vast majority of families in the county --- 77 percent
---~signed up for in-person learning.

On the first day of classes, a second grader came to school at Sixes
Elementary, then stayed home the next day after testing positive for the
coronavirus. Officials sent the other 20 students in the child's class
home for two weeks, along with their teacher. The class is now meeting
remotely.

As the week wore on, the letters from principals kept coming. An eighth
grader tested positive at Dean Rusk Middle School. A first grader at
William G. Hasty Sr. Elementary Fine Arts Academy. Two students at
Cherokee High School. And more.

Image

At least two students at Cherokee High School in Canton, Ga., tested
positive for the coronavirus last week.Credit...Dustin Chambers for The
New York Times

Among the children sent home to quarantine was Georgia Hancock, 5, from
R.M. Moore Elementary STEM Academy.

Georgia's 68-year-old grandfather, Phillip White, posted the news on
Facebook. In a phone interview, Mr. White said he suffered from heart
and lung ailments, and had strictly isolated himself from March until
June, when he and his wife began caring for Georgia and two other young
grandchildren at their home.

Mr. White said his son-in-law chose to send Georgia to school. When the
school sent her home to quarantine, Mr. White found himself worrying
that Georgia might have infected him. ``Lord, we pray for you to place
your beautiful hedges around us and keep us safe,'' he wrote on
Facebook.

\href{https://www.nytimes3xbfgragh.onion/spotlight/schools-reopening?action=click\&pgtype=Article\&state=default\&region=MAIN_CONTENT_3\&context=storylines_keepup}{}

\hypertarget{school-reopenings-}{%
\subsubsection{School Reopenings ›}\label{school-reopenings-}}

\hypertarget{back-to-school}{%
\paragraph{Back to School}\label{back-to-school}}

Updated Sept. 8, 2020

The latest on how schools are reopening amid the pandemic.

\begin{itemize}
\item
  \begin{itemize}
  \tightlist
  \item
    The first day of school was a rocky one in many places, as districts
    that started classes online dealt with
    \href{https://www.nytimes3xbfgragh.onion/2020/09/08/us/school-districts-cyberattacks-glitches.html?action=click\&pgtype=Article\&state=default\&region=MAIN_CONTENT_3\&context=storylines_keepup}{technical
    glitches, crashing websites and cyberattacks}.
  \item
    It's not easy to get a coronavirus test for a child. As schools
    reopen,
    \href{https://www.nytimes3xbfgragh.onion/2020/09/08/upshot/children-testing-shortfalls-virus.html?action=click\&pgtype=Article\&state=default\&region=MAIN_CONTENT_3\&context=storylines_keepup}{many
    parents still can't find one nearby}, impeding the fight against the
    pandemic.
  \item
    Life in a quarantine dorm: Colleges are trying to
    \href{https://www.nytimes3xbfgragh.onion/2020/09/09/business/colleges-coronavirus-dormitories-quarantine.html?action=click\&pgtype=Article\&state=default\&region=MAIN_CONTENT_3\&context=storylines_keepup}{isolate
    students who have been exposed to the virus}, but they are running
    into a host of problems.
  \item
    Penn State football defines fall in State College, Pa.
    \href{https://www.nytimes3xbfgragh.onion/2020/09/09/sports/penn-state-college-football-canceled.html?action=click\&pgtype=Article\&state=default\&region=MAIN_CONTENT_3\&context=storylines_keepup}{What
    is the town without it}?
  \end{itemize}
\end{itemize}

On Thursday, grandfather and granddaughter both tested negative. But Mr.
White said he was angry with the district, which he believes should have
started the school year online.

``It was a terrible idea'' to reopen, he said. ``The teachers should not
have had to go out and be at risk.''

As the first week drew to a close, Ms. Wicker, with the educators'
group, took to Facebook to update her followers on the number of
illnesses and students quarantined: 260 after five days of classes.

``I hope with everything within my being that no one who gets sick right
now dies,'' she wrote, adding, ``This did not have to happen. This was
entirely avoidable.''

The schools superintendent, Brian V. Hightower, who did not respond to
requests for comment for this article, also posted an update. He said
the district was being transparent about the situation, and defended the
group photos of unmasked high school students on the first day of
school, saying that the district had learned, ``upon investigation,''
that many of those students wore masks routinely.

``Today is the fifth day of school,'' Dr. Hightower wrote, ``and, this
year, that is an amazing milestone.''

By Tuesday, the number of quarantined students and staff members in the
district had more than tripled, to 925. Etowah High School, where nearly
300 people had been ordered to isolate after at least 14 positive cases,
switched to online classes for the rest of the month. Woodstock High
School, which also sent home nearly 300 people, did the same on
Wednesday, when the number of quarantined students and employees in the
district rose to 1,193.

Dr. Hightower
\href{https://www.cherokeek12.net/News/81120message\#sthash.LNefigAz.dpbs}{pleaded
for more routine mask use}. ``We know all parents do not believe the
scientific research that indicates masks are beneficial,'' he wrote,
``but I believe it, and see masks as an important measure to help us
keep schools open.''

His sentiment wasn't shared by some in a group of about 40 parents who
showed up at the district's offices before the school day started on
Tuesday, carrying balloons and signs declaring their support for the
reopening policy. They cheered as officials pulled into the parking lot.

Image

Supporters of the Cherokee County School District's decision to reopen
classrooms rallied outside the district's headquarters, in Canton, Ga.,
on Tuesday morning.Credit...Dustin Chambers/Reuters

``Don't worry about the basement Bobbys or negative Nancys,'' read the
sign held by Morgan Morrison, 28, the mother of a second grader who
``lost her mask on the second day.''

Ms. Morrison said she and her husband do not wear masks either. ``I feel
like before we're even born, God has a plan for when he's going to take
us to heaven,'' she said. ``There's nothing we can do to stop it.''

Erica Reece, 33, has five children between the ages of 7 and 11 back in
school. She and her husband both work full time, she said, so trying to
oversee at-home learning in the spring was a nearly impossible task.

Just one of her children regularly wears a mask in class, she said,
``out of respect'' for a teacher with a health condition.

Leaving such decisions in the hands of families was what she appreciated
about the district's policy.

``Free will,'' she said. ``Choices.''

Leighton Rowell contributed reporting.

Advertisement

\protect\hyperlink{after-bottom}{Continue reading the main story}

\hypertarget{site-index}{%
\subsection{Site Index}\label{site-index}}

\hypertarget{site-information-navigation}{%
\subsection{Site Information
Navigation}\label{site-information-navigation}}

\begin{itemize}
\tightlist
\item
  \href{https://help.nytimes3xbfgragh.onion/hc/en-us/articles/115014792127-Copyright-notice}{©~2020~The
  New York Times Company}
\end{itemize}

\begin{itemize}
\tightlist
\item
  \href{https://www.nytco.com/}{NYTCo}
\item
  \href{https://help.nytimes3xbfgragh.onion/hc/en-us/articles/115015385887-Contact-Us}{Contact
  Us}
\item
  \href{https://www.nytco.com/careers/}{Work with us}
\item
  \href{https://nytmediakit.com/}{Advertise}
\item
  \href{http://www.tbrandstudio.com/}{T Brand Studio}
\item
  \href{https://www.nytimes3xbfgragh.onion/privacy/cookie-policy\#how-do-i-manage-trackers}{Your
  Ad Choices}
\item
  \href{https://www.nytimes3xbfgragh.onion/privacy}{Privacy}
\item
  \href{https://help.nytimes3xbfgragh.onion/hc/en-us/articles/115014893428-Terms-of-service}{Terms
  of Service}
\item
  \href{https://help.nytimes3xbfgragh.onion/hc/en-us/articles/115014893968-Terms-of-sale}{Terms
  of Sale}
\item
  \href{https://spiderbites.nytimes3xbfgragh.onion}{Site Map}
\item
  \href{https://help.nytimes3xbfgragh.onion/hc/en-us}{Help}
\item
  \href{https://www.nytimes3xbfgragh.onion/subscription?campaignId=37WXW}{Subscriptions}
\end{itemize}
