Sections

SEARCH

\protect\hyperlink{site-content}{Skip to
content}\protect\hyperlink{site-index}{Skip to site index}

\href{https://myaccount.nytimes3xbfgragh.onion/auth/login?response_type=cookie\&client_id=vi}{}

\href{https://www.nytimes3xbfgragh.onion/section/todayspaper}{Today's
Paper}

\href{/section/opinion}{Opinion}\textbar{}Is Individualism America's
Religion?

\url{https://nyti.ms/2DGc3k9}

\begin{itemize}
\item
\item
\item
\item
\item
\end{itemize}

Advertisement

\protect\hyperlink{after-top}{Continue reading the main story}

transcript

Back to The Argument

bars

0:00/0:00

-0:00

transcript

\hypertarget{is-individualism-americas-religion}{%
\subsection{Is Individualism America's
Religion?}\label{is-individualism-americas-religion}}

\hypertarget{with-frank-bruni-and-ross-douthat}{%
\subsubsection{With Frank Bruni and Ross
Douthat}\label{with-frank-bruni-and-ross-douthat}}

\hypertarget{opinions-jeneen-interlandi-and-elizabeth-bruenig-join-the-podcast}{%
\paragraph{Opinion's Jeneen Interlandi and Elizabeth Bruenig join the
podcast.}\label{opinions-jeneen-interlandi-and-elizabeth-bruenig-join-the-podcast}}

Thursday, August 13th, 2020

\begin{itemize}
\item
  frank bruni\\
  I'm Frank Bruni.
\item
  ross douthat\\
  I'm Ross Douthat, and this is ``The Argument.'' {[}THEME MUSIC
  PLAYS{]}

  Today, a conversation about where we stand with COVID-19. Can the U.S.
  still contain its spread, or are we already defeated? Then nearly 1/5
  of all Americans identify as Catholic, but can their church say
  ``Black Lives Matter``?

  Frank, welcome back.
\item
  frank bruni\\
  Hey, Ross.
\item
  ross douthat\\
  So I think our listeners may be especially glad to hear your voice,
  after last week's all-conservative spectacular, {[}BRUNI LAUGHS{]} in
  which I was the moderate squish, arguing with more pro-Trump
  conservatives. Did you give it a listen?
\item
  frank bruni\\
  I did, and all-conservative spectacular was not the phrase that came
  immediately to mind. It was an interesting experience, Ross.
\item
  ross douthat\\
  That's all I can ask, Frank.
\item
  frank bruni\\
  I wanted to be there. I wished I could have inserted myself ever so
  briefly, just to kind of ask your guests, who seemed to think Trump
  had done a good job managing the pandemic, what they make of our
  world-leading case counts and death totals and how they explain that.
  But seriously though, I agreed with many listeners who wrote in and
  said that it was a discussion they wouldn't normally get to hear. It's
  the kind of thing that they hoped to hear when they turned to our
  show, something that hadn't crossed their radar before and they might
  not have thought of. And if it's OK, Ross, I want to read what one
  reader named Donna wrote in because I thought it was very interesting.
  She says, ``I found the latest right-wing coup episode one of your
  best ever. I like to hear both sides of an argument. Your show
  provides that, to an extent. But Ross often seems outnumbered and
  ganged up on by Michelle and Frank. I would like to know more about
  far-right views, but it's difficult for me to listen to Fox News
  opinion segments or AM talk radio for all the interruptions,
  over-speaking, and name-calling.'' So that's a pretty great
  endorsement, Ross.
\item
  ross douthat\\
  There's very little name-calling on our show, and we're proud of it.
\item
  frank bruni\\
  No, you and I only call each other names after we turn off the
  microphones.
\item
  ross douthat\\
  Yeah, or our producers---
\item
  frank bruni\\
  I'm kidding. I'm kidding.
\item
  ross douthat\\
  ---call both of us terrible names.
\item
  frank bruni\\
  Well, that's for sure, yeah.
\item
  ross douthat\\
  Well, I appreciate Donna's take. I know not every listener probably
  had the same take, and I suspect Michelle herself probably had a
  slightly different one. But she's off this week, and I will be
  suspiciously and conveniently on vacation next week. So it'll be a
  little while before I get to hear her take. Anyway, let's get to this
  week's episode. For our first segment, we're joined by Jeneen
  Interlandi, who writes about health and science for opinions editorial
  board and for The New York Times Magazine. Listeners may remember her
  voice from way back in March when we were wondering if quarantine
  would last through the end of that month. Back then, Jeneen told us to
  hunker down. She said the virus wasn't likely to go anywhere through
  at least the end of the year. And at the time, that seemed --- well,
  for a lot of people --- tough to imagine. And yet, here we are, still
  in limbo, waiting on a vaccine with more than 5 million cases since
  March, more than 162,000 dead, the federal government gridlocked, and
  a state-by-state patchwork of rules for re-opening businesses,
  schools, and more. I think, back in March, it still seemed possible
  that the American response could be more effective than Western
  Europe's. But now, at least pending a second wave across the Atlantic,
  our response looks definitively worse. So we asked Jeneen to come back
  to talk about what went wrong, what, if anything, went right, and
  where we go from here. So Jeneen, welcome back to ``The Argument'' and
  congratulations, I guess, on being right about everything.
\item
  jeneen interlandi\\
  Ha! Thank you for having me back, happy to be here. Not happy to be
  right about any of that stuff.
\item
  frank bruni\\
  Hey, Jeneen, great to have you on the show.
\item
  jeneen interlandi\\
  It's great to be here.
\item
  ross douthat\\
  So I, actually, was not here for that episode, mostly because I was
  ill with something that might have been COVID-19 myself. So can you
  just start by giving us a sketch of what's happened between March and
  now and also how, if anything, have your views of the underlying
  situation changed?
\item
  jeneen interlandi\\
  What I think happened is that we did not shut down sufficiently
  enough, we were too quick to reopen, and we didn't employ any of the
  very basic measures that public health experts were calling for
  sufficiently enough to get things under control, and I think we're
  still not doing that.
\item
  ross douthat\\
  To the second part of my question, has anything changed in your
  assessment of the disease itself, in terms of its lethality, how it
  spreads, things like that?
\item
  jeneen interlandi\\
  I did not expect that asymptomatic transmission was as much of a
  problem as it has turned out to be, so I think they've changed on
  that, definitely a lot of evolution in mine and everyone else's
  thinking on mask-wearing. Not proud to admit that I think back then, I
  was certainly someone who was saying, the masks don't protect you very
  much. They might stop you from transmitting it. But if you don't think
  asymptomatic transmission is a problem, then why would you advocate
  mask-wearing? And given that we had such low supplies and hospitals
  were running out, my position in the beginning was very much like,
  save the actual masks for people who need them, for frontline workers.
  And you don't actually need to go out and buy those and start hoarding
  them. So I definitely changed how I think about that.
\item
  frank bruni\\
  Jeneen, Ross's introduction made clear that you were prescient in some
  ways here. You thought that this would be with us for a while. You
  thought we were going to have to hunker down. I'm curious, what aspect
  of the country's response to this has most surprised you, and what's
  most infuriated you? And they may be the same thing.
\item
  jeneen interlandi\\
  You know, one of the things that's actually surprised me is how
  quickly Americans have gotten onboard with the idea of mask-wearing
  and even social distancing. We, in the media --- and it's for
  understandable reasons --- make a lot of the anti-mask contingent and
  of all the pushback against basic sound public health advice. And I
  think sometimes we lose sight of the fact that recent surveys have
  shown 60 percent to 70 percent of Americans are totally onboard with
  mask-wearing, actually want stricter control measures in place, and
  still trust public health experts. And that's really heartening, and
  it's been surprising.
\item
  frank bruni\\
  When you say that, I found myself wondering, do you feel that if our
  leaders could all get on the same page, show some consistency that
  Americans are actually ready to tackle this thing, but they lack the
  leadership?
\item
  jeneen interlandi\\
  I think that consistent messaging is hugely, hugely important. The
  C.D.C. has a playbook for how to communicate during a pandemic or
  during any national health crisis. And one of the key principles is
  consistent messaging and consistent communication. Just because it's a
  new era or a different era than the last time we had a global disease
  outbreak --- we have the internet, we have social media --- doesn't
  mean that you throw that principle out the window. There's so many
  people there that are on the fence, and they don't know what
  information to believe. And so when you have lots of conflicting
  messages, not just out in the ether or off in cyberspace but coming
  from officials talking on national television, it makes it really hard
  for people to know what to do. And it gives anybody who's skeptical
  and out to just kind of pick whatever advice best suits their mood or
  their opinion. And so when you have people speaking with one voice, I
  think you cut a lot of that noise out, and it makes it a lot easier
  for people to figure out what the right course of action is. And the
  fact that we have so much mixed messaging and the most confused we've
  gotten to 60 percent to 70 percent of mask-wearing, I feel like,
  imagine how much better we could do.
\item
  ross douthat\\
  But there is also the problem, as you've just suggested, it's hard to
  have a consistent message when your knowledge of the disease is,
  itself, is evolving.
\item
  jeneen interlandi\\
  Yeah, sure.
\item
  ross douthat\\
  So we went from an initial period where public health authorities were
  focused on hand-to-hand surface transmission, where everybody was
  talking about washing their hands and terms like fomites were getting
  thrown around. And now, months in, we're in a situation where airborne
  transmission seems much more important. And there was there was a
  period of time --- it didn't last incredibly long, but there was a
  period of time--- when certain segments of the internet were more
  trustworthy about whether to use a mask than were official public
  health pronouncements. And it just seems like that's a challenge that
  there's no playbook to deal with, right? Your assessments evolve, and
  you have to change your statements. And then you look like you're
  being inconsistent, and it's hard to map out a way to avoid that in
  advance, except being maybe a little more cautious in how much you
  talk down mask wearing.
\item
  jeneen interlandi\\
  Yeah, I think so. But I also don't think that the fog of war, as we
  refer to it, mentality --- or just the idea that in the beginning of
  an outbreak, especially with a completely novel pathogen, the idea
  that we don't know everything and we're going to make missteps is
  inconsistent with the principle of communicating consistently, right?
  Part of consistent communication is being open and honest about what
  you don't know. And then as you learn more, say, OK, turns out we were
  wrong about this. Here's how we're shifting course. That's actually
  part of consistent and open communication, or that's part of how you
  combat misinformation, is to be honest about what you don't know and
  to correct for it as you go along. And you're right. It also does mean
  a bit of humility and a bit of caution when you're issuing guidelines.
\item
  frank bruni\\
  Hey Jeneen, I want to ask you a question because I think there's some
  difference of opinion between you and Ross that I'd like to get to,
  which is if you were writing the script for what this country does
  over the next four to six weeks, what would that script look like? And
  then I really want to hear Ross's response to it.
\item
  jeneen interlandi\\
  OK, so if I'm writing the script from scratch or just from the point
  that we're at now, I think the first thing I want is much better data
  and much better use of data. So I want states and localities to make
  all of the information they have public. I want the C.D.C. to work
  aggressively with them to make that information useful. And then I
  want state and local officials, if not federal officials, to take that
  information, to take that data --- and individual states and
  localities have done this. Harris County, Texas did a great job at
  this --- basically creating a color-coded threat assessment that tells
  officials and citizens exactly what they need to do, based on how
  extensively the virus is spreading in their communities. So one of the
  things that people have argued a lot about is, you can't have a
  nationwide lockdown. That doesn't work because some places you have
  really bad outbreaks. In other places, you don't have anything. And
  all you do is breed resentment in the communities that don't have bad
  outbreaks, right? And then people don't want to follow anything, and
  they get fatigued. Well, if you use data to actually create some sort
  of forecast or assessment of how the virus is actually spreading in
  given communities, it could be a nationwide kind of response system,
  but it's employed locally. So I think that's the first thing I would
  want to see. And then from there, the very next thing is to get that
  data as accurate as possible, you need much more testing. And so I
  want to see much more ramping up of testing. It's kind of tiresome to
  even say that because we've been talking about this since March, but
  it's still a huge problem. It still takes up to two weeks or longer to
  get test results back in some places. And that makes the testing
  itself almost pointless. If it takes that long to find out what the
  results are, why bother getting tested? So I want to see more of that,
  and I think there's a lot the federal government could do that it's
  not doing or it hasn't done to improve the testing situation.
\item
  ross douthat\\
  What could the federal government do more of on testing?
\item
  jeneen interlandi\\
  Again, I feel weird saying these things because we've been seeing them
  for so long, but they remain true. We have the Defense Production Act
  for a reason. We also have provisions that allow for patent overrides,
  and we should be using those things. If not now, then when? Why do
  those things exist? They exist for exactly these situations. There are
  some companies that have developed rapid point-of-care tests that,
  from all of the sourcing I have, pretty accurate, really good tests
  could solve a lot of the problems. But the companies that develop
  those tests can't make enough of them to deploy them across the
  country in a way that would be meaningful. So what the government need
  to do is step in and say, we're going to take this patent because you
  don't have the ability to produce it at the level that we need. We're
  going to compensate you fairly for it, and then we're going to task
  several companies across the country, under the Defense Production
  Act, with making this product at scale and getting it out as quickly
  as possible. Would that solve every problem tomorrow? No. Could you
  produce every test that you possibly need in the country in a week?
  No, but you should be trying to do those things. And if you did, we'd
  be in a much better place than we are right now.
\item
  ross douthat\\
  There was an interview with Bill Gates a few days ago --- I don't know
  if you saw it--- where he was very angry about many different things
  related to testing. But one point he made that he thought was
  interesting was that he's arguing that these testing companies are
  getting paid large sums of money--- public money, in many cases--- for
  test results but not for timeliness of test results, right? And he was
  saying, basically, you should only reimburse companies that are
  delivering test results within 24 to 72 hours.
\item
  jeneen interlandi\\
  That's a fantastic idea. And other people have floated that to me, and
  I think it makes a lot of sense. You start tying reimbursement levels
  to how long it takes to get the testing results, and guess what?
  Companies are going to figure out how to get those results a lot
  faster.
\item
  ross douthat\\
  So I want to go back to the first thing you said about lockdowns,
  right? You said we didn't lock down well enough and long enough. And I
  think those are slightly separate things, so I want to ask you more
  about that. Do you think that the problem was our lockdowns just
  needed to go on an extra month in most states, and reopening should
  have been slower? Or do you think that we didn't achieve the scale of
  lockdown, in terms of people not leaving their homes at all, that some
  countries in Western Europe at least attempted? Or do you think it's
  both?
\item
  jeneen interlandi\\
  Actually, both. I think it's absolutely both of those things. The
  numbers I've seen are, like, if you look at a place where we had
  really good lockdown, it was essentially 50 percent, compared with I
  think it was 80 percent or 90 percent I saw of lockdown in countries
  where they've successfully not only flattened the curve, but actually
  turned it downward. So they had just much more stringent controls. And
  they had things like roadblocks, and you can't leave a community. And
  if you do, they're tracking if you're going from one state to another.
  We didn't have that here. We had people in New York fleeing, leaving,
  going to vacation homes. We had people going down to Florida. We had
  people traveling. We still had public transportation. And I think we
  needed to do a lot more in certain places than we did. And I
  recognize, absolutely, that that's really tricky because every time
  you tighten the lockdown procedures, you hurt the economy that much
  more. But I would argue that by not doing that, the economy is
  suffering probably worse because it's dragged out a lot longer than it
  might have had to otherwise.
\item
  ross douthat\\
  I tend to agree that the initial lockdown could have been firmer in
  certain ways. But I think in terms of their length, we basically hit
  the plausible limit. And I think you could tell that we hit the
  plausible limit because of the kind of reactions that we started
  getting to it, both right-wing coded reactions like the anti mass
  protests, but also I think the surge of political protests around the
  George Floyd shooting were, in part, a reflection of social energy
  that had been tamped down and maybe wouldn't have burst out in that
  particular way without that kind of provocation, but would have burst
  out in some way over the summer. I feel like you couldn't run the
  lockdown for multiple months, especially, and this also goes back to
  the question of messaging, but the initial messaging was, we're going
  to do this for a few weeks, and the goal is to flatten the curve,
  right? And there's always been this conflict and this uncertainty, I
  think, in public messaging about whether the goal is flattening the
  curve and sort of extending the epidemic in a way that doesn't
  overwhelm hospitals and gives us more time to get treatments and
  eventually gives us more time to get a vaccine, versus suppression,
  versus trying to actually stamp out the virus.
\item
  jeneen interlandi\\
  But there's a third thing there. It's not just that. You're not just
  flattening the curve. What you're trying to do is get certain things
  in place so that you can manage the outbreaks more smartly, right? So
  if you do shutdown in a smart way, you're supposed to be building up
  your testing capacity, your contact tracing capacity, your isolation
  and quarantine, which we never did enough of. You're supposed to build
  those things up so that when you do reopen, you can actually monitor
  more effectively and keep flare-ups from becoming outbreaks all over
  again. We didn't do any of those things. So Americans made huge
  sacrifices in abiding by the lockdowns to the extent that they did,
  and the federal government betrayed them by not using that time as
  effectively as it should have.
\item
  frank bruni\\
  I got to tell you. As I listened to the two of you, I feel such a
  sense of surrender and failure. I hear both of you saying in different
  ways, we can't really do aggressive lockdowns going forward because we
  just don't have it in us. But it turned out, we kind of really didn't
  have it in us, if we're talking about the whole country, in those
  first months. And I'm haunted by this sense that all of this has
  revealed some fundamental shortcomings and flaws in the American
  character that are the principal explanation, along with failed
  political leadership, for our world-leading exceptional status and the
  number of infections and deaths.
\item
  ross douthat\\
  So I would speak up for the American character a little bit. I think
  that yes, we did not have the scale of lockdown that some Western
  European countries had. I also think, though, that the scale that they
  had was, in some cases, effectively overly repressive. And you had
  cops chasing people out of parks in rural parts of England and things
  like that I think were never going to fly in the US. And also, the
  country really did shut down its entire economy --- not for a week,
  but for several months, taking on its own set of significant costs of
  kids out of school and people dealing with depression and mental
  illness and all kinds of things. We really did do that. And we did, in
  fact, flatten the curve, right? I mean, that's the odd thing about all
  this.
\item
  frank bruni\\
  In some places and in some places only recently.
\item
  jeneen interlandi\\
  Yeah, and for really short periods of time. The piece I did for the
  magazine was on Harris County, Texas. They had aggressive stay-at-home
  orders. When Governor Abbott allowed the localities to kind of manage
  the response, Harris County was one place that really had pretty
  aggressive stay-at-home orders. And they had really consistent
  messaging, and they actually did flatten the curve and started to get
  a downturn. But what happens is none of the surrounding communities
  did the same thing. And eventually, Governor Abbott stepped in and
  said, no, you can't do this. You can't mandate masks. You can't do X,
  Y, and Z. And then what you saw was they lost all of those gains that
  they had made, and I think that's the story in a lot of places.
\item
  ross douthat\\
  Well, but I guess I have a sort of twofold view of this, right? I
  think the central government completely failed. I think that President
  Trump obviously completely failed. I think there was an opportunity
  for substantial suppression that would have required major contact
  tracing. It would have required, as Jeneen says, testing to be not
  just ramped up --- we did eventually ramp it up--- but to ramp it up
  sooner and get results back faster. And we haven't been able to
  achieve that. And it probably would have required more aggressive
  quarantine measures, state travel restrictions, like she said, people
  in COVID hotels earlier on. All of that we failed at, clearly.
\item
  frank bruni\\
  Well, that's what I'm saying, Ross.
\item
  ross douthat\\
  What I'm saying, though, is that at the start of the epidemic, the
  public rhetoric was, the danger is that every part of the US is going
  to be like New York City ended up being, where you're going to have a
  really, really high case fatality rate because hospitals are going to
  be overwhelmed, doctors aren't going to be able to treat patients, and
  you're going to end up with basically that soaring spike of deaths
  that you've got in New York City everywhere. And that has not
  happened, and it hasn't happened in places like Texas and Florida and
  the Southwest that have had to new outbreaks. Those outbreaks have
  been bad, but they have looked like the curve on the initial charts
  that we had--- not the spike, but the slower curve that rises and
  falls. And it's a tragedy that we didn't achieve more than that, but
  we also have not had a situation where everywhere has turned out like
  New York City, nothing close to that right now.
\item
  jeneen interlandi\\
  So I would argue that at the outset, it wasn't every place is going to
  be New York City. It was going to be, we're going to have lots of
  different outbreaks. And basically what you're talking about is a lot
  of this is preventable if we use just the basic public health measures
  that we have now. So we don't have to wait for technology that doesn't
  exist. If we do these things, we can save a lot of lives. And guess
  what? Most of those lives are going to be in low-income communities,
  communities of color, and really marginalized places where they don't
  have good hospitals in their communities. So that's the issue I would
  take with that.
\item
  frank bruni\\
  All due respect, Ross, when I listen to you talk, I kind of don't hear
  the full recognition that tens of thousands of Americans, who might
  not have had to die, died here.
\item
  jeneen interlandi\\
  Died, yeah.
\item
  frank bruni\\
  I don't think we can kind of get so deep in the weeds in but this, but
  that, but this is the way we are. I don't think we can skate over that
  so easily. But I want to ask you a question that I want to get
  Jeneen's response to. This is about America, how we've responded and
  kind of failed in a way other countries haven't. This is about the
  American character. Ross, do you think we have found the right balance
  between famous American individualism and communal protection,
  communal concern?
\item
  ross douthat\\
  No, I don't. I think that we could have had more communal protection
  and more communal concern. But I do think lots and lots of people
  expected us to have not a New-York-level epidemic everywhere but
  multiple New Yorks, multiple versions of the mountain of deaths that
  Andrew Cuomo bizarrely put on a poster celebrating how New York
  handled things, and we have not had that. New York, and the Northeast
  generally, is a distinct and tragic curve, and the other curves, as
  bad as they've been, have not been that bad. And that is a testament,
  I think, to real public health achievements and real sacrifices that
  Americans made, even in the places that maybe reopened too early and
  so on. You still had social distancing. You still had mask-wearing.
  And the other thing I'm just trying to stress is that viruses are
  really unpredictable and hard to contain. And those challenges are
  real, and they don't go away just by having better leadership in the
  White House than we've unfortunately had. And I guess the final
  thing--- and here I am I'm really curious about Jeneen's take ---
  there are a lot of questions about where you actually reach herd
  immunity. And there's a sort of lively debate about whether, at least
  with the kind of social distancing we still have in place, New York
  City might be at a de facto herd immunity, with around 20 percent or
  25 percent of the country infected.
\item
  janeen interlandi\\
  I don't agree with that at all. I don't agree with that.
\item
  ross douthat\\
  OK, so tell me why that view is wrong.
\item
  jeneen interlandi\\
  Well, OK, so first of all, I think herd immunity is a little bit
  misunderstood as a concept, right? It's the threshold at which enough
  people in a given population become immune to a virus that it can no
  longer cause huge outbreaks. But we normally calculate that based on a
  standardized intervention, like vaccination. When we talk about herd
  immunity as a scientific concept, we're almost always talking about
  vaccination, how many people need to get vaccinated. That is not the
  same thing as herd immunity or as an ongoing, real-time infectious
  disease outbreak. So for the latter, it's actually much tougher to say
  what the threshold would actually be. And you're correct that some
  studies have cited 20 percent, but many more people seem to think it's
  going to be closer to 60 percent, 70 percent, 80 percent, right? So
  it's going to be much higher than 20\%. And the bottom line really is
  that we just don't know. So I think that's one reason it's kind of
  absurd to talk about herd immunity as a strategy. But realistically,
  most people don't think it's going to be 20 percent. They think it's
  going to be significantly higher. So that's the first thing that I
  would say about that.
\item
  ross douthat\\
  I'm not advocating for it as a strategy.
\item
  jeneen interlandi\\
  OK.
\item
  ross douthat\\
  I'm arguing that we should observe it if it seems to be happening as a
  reality. Sweden was held up as the sort of control group for the virus
  early on, the country in the West that was doing the least to contain
  it. And the Swedish curve, without strong interventions, has, itself,
  bent down to the point where Sweden looks like a low-transmission US
  state at this point. And Sweden is a data point for the argument that
  you could get herd immunity at 20 percent or 25 percent.
\item
  jeneen interlandi\\
  I don't think that they've actually done very well, right? They have
  higher infection rates than their neighboring countries. They have a
  pretty high mortality rate. I don't think they're anywhere near herd
  immunity. In fact, there was an open letter written by, I think, 20 or
  30 epidemiologists, including some from inside Sweden, saying they're
  not even close to herd immunity. There's still high susceptibility in
  the larger population, and their economy doesn't appear to be doing
  much better than their neighbors' economies for the strategy that they
  took.
\item
  ross douthat\\
  I want to be clear on what I'm saying. I don't think the Swedish
  strategy has contained the virus more effectively than their
  neighbors. It clearly has not.
\item
  jeneen interlandi\\
  OK.
\item
  frank bruni\\
  So what are you saying?
\item
  ross douthat\\
  I'm saying that if you look at a chart right now of Sweden's total
  cases and daily new cases and total deaths, the death line has
  flattened. The total case line has almost flattened. And daily new
  cases have done a big arc and now are very low. Sweden has death rates
  that are worse than Denmark and Norway that did severe lockdowns, but
  it's death rates are not nearly as bad as some of the worst centers of
  the epidemic. That cries out for an explanation. So I'm just saying, I
  think there are a lot of data points right now that suggests that
  something maybe limits the virus' spread at around that 25 percent
  threshold.
\item
  jeneen interlandi\\
  I take you at your word at what the Swedish curve looks like right
  now. I don't think that that in any way implies that the herd immunity
  threshold there or anywhere else is necessarily 20 percent. I think
  there's about 5,000 different potential explanations for that. I don't
  think it's a leap from they've managed to bend the curve down to they
  have herd immunity. And I think also to your earlier point ---
\item
  ross douthat\\
  Well, I want to be clear, there is herd immunity for the level of
  social distancing we have now.
\item
  jeneen interlandi\\
  Yeah, herd immunity is not a fixed number.
\item
  ross douthat\\
  Exactly.
\item
  jeneen interlandi\\
  Exactly, yeah. So that's fine. But what I was arguing earlier, you
  want to make decisions about what things need to be closed down, what
  things can be reopened, how much social distancing you need, whether
  you want mask-wearing indoors, outdoors, everywhere based on what the
  case positivity rate is in a given area. Now, if you're basing it on
  the data, then you're doing it the right way. Whether you're at a
  thing that you can reasonably define as herd immunity or not is kind
  of incidental. So I feel like you're arguing for herd immunity as a
  strategy, even though I know you say you're not. I'm just not sure how
  to understand that.
\item
  ross douthat\\
  No, I think we agree.
\item
  frank bruni\\
  {[}LAUGHING{]} I don't know about that.
\item
  ross douthat\\
  I think, in part, it's a question about our level of optimism going
  forward and how bad we expect the disease to get in the fall. I think
  that we agree that you want basically a data-driven assessment of the
  disease's prevalence in your community. And I agree, that's more
  important than a theory of herd immunity that might be proven wrong.
\item
  jeneen interlandi\\
  Yeah, again, I don't think we know the number for SARS-CoV-2 at all. I
  just don't think it's going to end up being near 20 percent, based on
  what I've seen. The other thing I would point out is in places like
  New York and I think in most of the Northeastern United States,
  studies have shown, the thing that they seem to have done right is to
  be much slower to open indoor dining and to open bars.
\item
  ross douthat\\
  Yes. And I think New York City is particularly good at that, and I
  think that's why things are much better in the Northeast than they are
  in other parts of the country. But again, that goes back to not
  throwing your hands up or saying, it's great that we're also
  libertarian here. It's basically based on data. So based on data,
  we're not going to open the bars and restaurants, and we're going to
  be able to maybe open the schools in the fall. It's a trade-off. Well,
  and that I guess can be our last question for you, which is, should
  the schools open in the fall?
\item
  jeneen interlandi\\
  Again, I think it should be based on data. And I think if you do it
  based on the data and you say, OK, we have to have a certain
  positivity rate to safely open the schools, what you're going to find
  quickly from there is, OK, how do we get to that positivity rate?
  Well, the first thing you have to do is think about whether you want
  to keep your indoor dining open, whether you want to keep your bars
  open. And then I think the conclusion is that you probably don't.
\item
  ross douthat\\
  But in practice, it seems like a lot of places have chosen indoor
  dining and bars over schools. And you also have a dynamic where you
  have certain states that maybe have a transmission spread rate that is
  low enough to reopen schools aren't reopening them. This seems to be
  happening in the mid-Atlantic. And then states that have a higher
  transmission rate --- Georgia, Texas, and so on --- are opening them.
  So I think there is a wariness about just following the data in
  liberal states, as well as conservative states. And the sort of
  polarization that once Trump said, let's open the schools, suddenly
  there was ---
\item
  jeneen interlandi\\
  ``That's a terrible idea!''
\item
  ross douthat\\
  --- political pressure not to open them in places that could.
\item
  jeneen interlandi\\
  You have these two kind of conflicting things in play at the same
  time. On one hand, it's extreme fatigue on the part of everybody. I
  don't care what your political persuasion is. You're fatigued right
  now from all of the closures and just the abnormal existence we've
  had. And then on the other hand, if you have gains and you've managed
  to get to a place --- if you live in Manhattan like I do --- where
  you're not hearing ambulances go down your street literally every five
  minutes and you're terrified to even just go outside, you're terrified
  of losing those gains. So you have this kind of terror and also this
  fatigue happening at the same time, and I think that's part of where
  we're at.
\item
  ross douthat\\
  Well, I think terror and fatigue is a good place to close this
  discussion.
\item
  frank bruni\\
  Jeneen, thank you so much.
\item
  jeneen interlandi\\
  Thanks, this was fun.
\item
  ross douthat\\
  Jeneen, we'll have you back at Christmas when we'll all have herd
  immunity, and we'll look forward to that.
\item
  jeneen interlandi\\
  Our holiday special, OK. Thanks, guys.
\item
  ross douthat\\
  We'll be right back.
\item
  frank bruni\\
  Welcome back. We're going to talk about American Catholicism, which,
  in an age of polarization, is a rare institution that sprawls across
  our political divide. It's a church with a progressive-leaning pope
  and a conservative American hierarchy. It's the church of Attorney
  General William Barr and of Joe Biden, the Democratic nominee for
  president. It cuts across class. It cuts across ethnicity, which means
  that it's uniquely cross-pressured. To talk about some of those
  pressures, we've invited our colleague Liz Bruenig. Liz, thanks so
  much for joining us on ``The Argument'' today.
\item
  liz bruenig\\
  Thanks for having me on.
\item
  frank bruni\\
  Liz, you recently wrote an op-ed about American Catholicism's unease
  with the Black Lives Matter movement. ``Racism Makes a Liar Out of
  God'' was the terrific headline on your extremely interesting piece.
  Tell us about the argument you made.
\item
  liz bruenig\\
  Well, thanks very much. I was interviewing Gloria Purvis, who's a
  Black Catholic journalist, in a sense, and that she has run a
  broadcast radio show on EWTN, which is the largest Catholic media
  distributor in the world. She also had a television show. And one of
  EWTN's local affiliates, the Guadalupe Radio Network in Texas, which
  is, I believe, one of their largest affiliates, if not the largest,
  dropped her show after the murder of George Floyd because she began to
  speak out quite vociferously about racism. And my response to that was
  just that it wasn't particularly Catholic of them, that she wasn't
  saying anything unorthodox, that everything she was saying is
  completely evinced and held up by church teaching. And so I just don't
  think it should be that hard for Catholics to look at the claims being
  made, put aside various organizations, whether you like certain
  figureheads and so on and so forth, and just focus on the arguments
  emerging from this movement. And I think they're pretty easy to
  endorse for a Catholic. I don't see the hang-up.
\item
  frank bruni\\
  So why, then, do you think she was dropped? How do you explain? Is
  that about racism? Is it about particular and peculiar discomfort with
  Black Lives Matter? What's going on?
\item
  liz bruenig\\
  I think it's probably somewhat to do with particular discomfort with
  Black Lives Matter, and I think quite a lot of it has to do with the
  fact that having uncomfortable conversations doesn't necessarily make
  for good radio. People don't like to sit around and listen to people
  really hash out issues of life and death import --- and she has
  co-hosts who sometimes have friendly disagreements with her ---
  because that can be very distressing to hear, the same reason people
  don't want to have those conversations interpersonally. And so I think
  it was a combination of that and widespread conservative mistrust of
  the Black Lives Global Network, this organization that has laid claim
  to the phrase ``Black Lives Matter,'' which, of course, preceded it.
\item
  frank bruni\\
  One thing I wanted to ask you before we went on any further is when
  we're talking about the Catholic Church, what are we really talking
  about, and how much meaning does that phrase have? And what I mean by
  that is I would probably be in a pollster's box as a Catholic because
  I was raised in the Catholic Church, although I drifted far, far away.
  I believe you are a convert to Catholicism. Ross has his own
  relationship to Catholicism. So there are Catholics who care about
  what the pope says, and there are Catholics who never look in this
  direction. So what do we mean here by the Catholic Church?
\item
  liz bruenig\\
  Well, it's a global institution, and so every Catholic's response to
  various world goings-on is going to be different. And I think when we
  we're talking about the Catholic Church in this context, we refer to a
  number of things. The Catholic Church refers to, at times, the
  hierarchy, the church authorities. It, at times, refers to the laity,
  everyone who is baptized Catholic and receives the sacraments and is
  confirmed to the church. And then at times, it refers to the internal
  logic, the tradition, the rules of the Catholic Church. It takes all
  of those things to constitute it. So I think we're always referring to
  a sort of mixture of things there.
\item
  ross douthat\\
  In many ways, the church feels incredibly weak at this moment, and its
  sort of sprawl seems like part of its weakness, that it's sort of been
  through a period of decline. It had the sex abuse crisis. It's had
  this sort of internal argument over its own teachings, that, for my
  sins I've been involved in. And so things like the Black Lives Matter
  movement come along, and there is no Catholic authority figure or
  authority figures capable of saying, OK, we're going to harness the
  church's sprawl and spread to support this movement or to critique
  this movement or somewhere in between. And instead, you get this sort
  of very American, I guess, entrepreneurial thing where certain people
  appoint themselves a spokesman for Catholicism, and you have these
  particular clashes, like the one around Gloria Purvis. But there's no
  sense of Catholicism as something that's capable of coming together
  and playing a dynamic role at this moment. Although, but Liz, you sort
  of end your piece by suggesting that the bishops could do that to some
  extent?
\item
  liz bruenig\\
  Well, the bishops could put the hierarchy behind the March for Black
  Life in Washington D.C., which is what the open letter calls for, is
  for the bishops to join that march. What the bishops can't do it they
  can't make anyone like it, and the church teaching already supports.
  So that's two out of three. It's not bad. But American Catholics ---
  and at this point, Catholics all over the world, especially in North
  America and Europe --- these are liberal subjects. So their approach
  to Catholicism is not the approach to Catholicism that someone alive
  during the counter-reformation would have taken. It's an approach to
  Catholicism that's informed by the general ideological atmosphere
  around here, which is I decide what sounds good to me, and I discard
  what doesn't. And I am the sovereign of myself, and I operate within
  certain parameters. But very little is binding on me. And so you have
  some American Catholics who say, OK, if the hierarchy says we do it
  and it matches the church teaching, then we do it. It doesn't really
  matter if it matches our other intuitions.
\item
  ross douthat\\
  So let me push you a little bit on the dilemma here. I'm going to
  quote a not African-American, but African cardinal, Wilfrid Napier,
  who is from South Africa ---
\item
  liz bruenig\\
  That's not the one I was expecting.
\item
  ross douthat\\
  That's not the one you're expecting --- who is, himself, Black. This
  is how the hierarchy of the church now manifests itself, in tweets.
  But he tweeted, Black Lives Matter creates real conflict of
  conscience. The mission statement of BLM-GNF --- that's the
  organization you were talking about earlier --- advocates views on
  marriage and family that are totally repugnant to Catholic teaching,
  as are the actions of its activists. Yet there is an urgent need to
  expose all injustice against people of color. So this, I think, coming
  from outside the U.S., but it's still, I think, distills the dilemma
  for Catholics who are maybe more sympathetic to the cause of criminal
  justice reform or police reform than some of the people who wanted
  Gloria Purvis' show canceled, but I'd say the sort of current protest
  politics in its organizational and institutional forms and see a
  movement that is much more secular and left wing than was, for
  instance, the Civil Rights Movement at its peak, right? It's not
  emerging out of the African-American church. Its rhetoric emerges out
  of an academic milieu that is, I think it's fair to say, is pretty
  hostile to traditional Christianity and Catholicism. And as you just
  mentioned, in the penumbra of these riots, you get a lot of attacks on
  churches --- not tons and tons, but there's been a real uptick in
  vandalism of Catholic churches, some of it focused on figures like
  Saint Junipero Serra, who is a controversial figure for reasons
  related to the treatment of Native Americans under the mission system
  in California. But some of it is just sort of more all-purpose, right?
  Acts of vandalism that just knock the head off a statue of the Virgin
  Mary or something. And so it seems like there is actually some dilemma
  here where you're trying to figure out, how do you lend your support
  to the cause of racial justice without ending up just as sort of a
  religious ally of a movement that is pretty hostile to Catholicism?
\item
  liz bruenig\\
  I don't think that the attacks on churches and statues have been
  anti-Catholic in particular. They tend to be people who are upset
  about, as you said, Spanish colonialism, the treatment of Native
  Americans under the Spanish colonial regime. So there have been
  attacks on missions. There have been attacks on statues of Saint
  Junipero Serra. With any protest, especially of this size, you're
  going to get some delinquent sort of vandalism that has nothing to do
  with the protest at all. It's just people doing things they cannot
  normally do and kind of having a libidinal moment there where they go
  a little crazy. I think quarantine has compounded this
  already-existing tendency. But I wouldn't look at it as these people
  are anti-Catholic, they want the Catholic Church destroyed. That may
  well be in the case of some people. If you're on the internet, you
  hear from those people every day.
\item
  ross douthat\\
  I've never heard from those people. I don't know what you mean.
\item
  liz bruenig\\
  {[}LAUGHS{]} Yeah, who could they possibly be? We could probably
  rattle off some handles. But in reality, it seems like, you look at
  polling, the majority of Black people are Christians. And so it's not
  the case that the Black people who are calling for an end to police
  violence are dead set on the destruction of the nuclear family or
  whatever. That might be the case with these activists who are writing
  the copy, but the Black Lives Matter Global Network is not a
  membership organization. You don't have to pay dues and if you don't
  pay your dues, you can't show up to the protest. The organizing
  efforts against anti-racism don't require you to participate in
  whatever efforts there are or may be to destroy statues or undermine
  the church. On the other hand, the church is pretty good at
  undermining itself, so it doesn't really need any help.
\item
  frank bruni\\
  So when someone like Andrew Sullivan, whom you mentioned in your
  column, says Black Lives Matter and Catholicism are incompatible, is
  he kind of focusing on a detail that is less meaningful than he's
  making it out to be?
\item
  liz bruenig\\
  I think his complaint was that the Black Lives Matter Global Network
  is Marxist and that you can't be a Marxist and a Catholic. You can,
  actually. I have this on good authority.
\item
  ross douthat\\
  In a very complicated sort of way.
\item
  liz bruenig\\
  {[}LAUGHS{]} Marxism can be said in many ways, as I imagine Aquinas
  would have put it. And so I tend to think that people like Andrew
  would probably not be in favor of this movement in any universe,
  right? And so the Catholicism issue is an opportunity to make a shell
  argument that's not substantive. It's about something else. It's a
  procedural objection. Oh, well, I would, but I have all these other
  things that I have to subscribe to ahead of this, so I can't.
\item
  ross douthat\\
  Well, OK, but people are not wrong to look at the current moment of
  protest and say, well, this seems to be about not just racism and
  police brutality, but also a kind of substantial reordering of elite
  American institutions, for instance, in ways that are sort of much
  more secular and left wing than movements that the Catholic Church
  usually associates itself with and more hostile to not just Catholic
  beliefs on one particular issue, but the wider panoply of beliefs,
  right?
\item
  liz bruenig\\
  Right. No, I think that's true. I think that no institution coming out
  of American soil shares the aims of the Catholic Church. These are all
  liberal institutions. Their aims are circumscribed by the event
  horizon of temporality, right? So this is to say, the goals of all of
  the institutions that come out of American thought and American life,
  these are liberal institutions that have goals that are sort of
  utilitarian, or they have to do with sort of freedom and equality. And
  those are aspects of Catholic teaching. They're certainly not the end
  goal of Catholic teaching. So then the question is, how distant is
  this particular movement from Catholic Church teaching, and how
  distant can a movement be before we have to completely abandon it? And
  I would say that all that's required is a little of that good old
  discernment that we hear about so often, especially from our Jesuit
  friends, which would be paying attention to what aspects of the
  movement are being foregrounded and in what way you're being asked to
  support it. So if there is, say, a referendum in your city on police
  funding or the ways in which police can or should try to de-escalate,
  as opposed to defaulting to violence, or body cams or something like
  that, you can go ahead and think, what is the best way to cast a vote
  here that is going to hopefully create a more equitable landscape in
  this city for the people who live here with me? You don't have to
  affirm in the strongest terms even the entirety of the movement. But
  to the degree that you have power as a citizen, I think it's probably
  pretty easy to exercise that in a way that comports with any racist
  goals.
\item
  frank bruni\\
  Liz, what do you make of Trump's attacks on Biden's Catholicism or on
  Biden's religion, which is thus implicitly Catholicism? And do you
  think Biden's Catholicism factors into, colors this coming election in
  any interesting ways?
\item
  liz bruenig\\
  Trump is an interesting figure in that he sort of doesn't even really
  put up the bare pretense of being all that interested in religion.
\item
  frank bruni\\
  {[}LAUGHING{]} He does hold a Bible upside down very photogenically.
\item
  liz bruenig\\
  Right, he'll do stuff like that, which is vaguely insulting, and go
  stand out in front of a church. And the tone of that is always like,
  you piggies sure love this slop, don't you? Lap it up. {[}BRUNI
  LAUGHS{]} And it's a little insulting. I don't like it when any
  politicians talk about their religion. I've never heard one of them do
  it in a way that was remotely convincing, and Trump, in particular,
  just doesn't even really bother with it, which I kind of grudgingly
  respect because it's so clearly a charade. The ones that try to be
  earnest about it are, in some ways, a little more disturbing to me.
  But he has no room to attack Joe Biden's Catholicism, the nature of
  which I know nothing about. And to the degree that it will even matter
  at all in the election, which I doubt it will, it could be good for
  Biden. Catholics are classic swing voters. There are lots of Catholics
  in swing states. Catholics typically split right down the middle,
  50-50, Republican-Democrat. And if he can pull any Catholics who
  wouldn't have otherwise given him a shot because he is Catholic,
  that's possible, and that's interesting, I suppose. But my feeling is
  that American Catholicism has been so thoroughly Protestantized that
  it doesn't really matter. And moreover, very few Protestants could
  tell you in what ways Catholics are morally different than they are in
  the United States, right? So back in the day, you would have been
  called a crypto Catholic if you said something like, well, whether or
  not you go to heaven depends a lot on you. But this is also just sort
  of common American Protestant thought at this point. It's rather
  unusual to find very vocal Calvinists. And so because the lines
  between Catholicism and Protestantism the United States are sort of
  blurred at this point--- the evangelicalized Catholics on the right
  and then the sort of liberalized Protestants who have kind of lost a
  lot of those puritanical elements of Protestantism on the left--- I
  just don't really think it matters. It's definitely not going to be a
  JFK-type situation.
\item
  frank bruni\\
  Well, and what you're saying makes me think that's one of the reasons
  why Trump instinctively attacked Biden's religion in the larger sense
  and not necessarily his Catholicism because I think he, at some level,
  understands what you just said, which is that people don't necessarily
  see these bold lines where they once saw them between Protestantism
  and Catholicism.
\item
  ross douthat\\
  Trump is going for the idea, which is a true idea that among white
  Americans, there is a real pew gap between the parties, in the sense
  that Republicans are more likely to be church-goers, and Democrats are
  not. And there is an odd reversal of that in the actual leaders of the
  parties, in that Trump is not a church-goer and Joe Biden is. So
  Trump, in his way, is sort of going right at that and saying, oh, you
  think Biden's a church-goer, well, he's against God, right? I mean,
  that's sort of the classic Trump.
\item
  frank bruni\\
  What does that even mean, Ross?
\item
  ross douthat\\
  Well, I may not go to church, but he's against God.
\item
  frank bruni\\
  What does that even mean when he said he's against God? What does that
  even mean?
\item
  ross douthat\\
  Well, it's like when Trump was asked, back when he was first trying to
  figure out how to talk like a pro-lifer, and he said something about
  how, well, we really need to punish women who have abortions, which is
  not and has never been the main pro-life position on the issue. But to
  Trump, it seemed like something that he imagined a pro-life person
  would say. And I think you get some of that in the against god
  rhetoric, too.
\item
  liz bruenig\\
  Yeah, Trump was just doing logic in that case.
\item
  frank bruni\\
  {[}LAUGHING{]} Wait, wait, wait, wait, Liz, did you really just use
  Trump and logic in the same phrase?
\item
  liz bruenig\\
  I'm not kidding. He was sitting there. He knows nothing about
  Christianity in America. He knows nothing about the pro-life movement.
  He's not remotely interested in any of these issues. And so someone
  asks him, do you think that women who have abortions should be
  punished? So he has to come up with an answer on the spot because he
  has no relationship to this movement. And he's like, well, yeah, I
  mean, I guess. They say it's murder so yeah. {[}BRUNI LAUGHS{]} And
  that's like very basic {[}LAUGHS{]} logic he was performing there. And
  again, he did the same thing with Biden where he's like, well, damn,
  if you're not with him, you're against him. So the question is, are
  the people that Biden is competing for with Trump --- the swing people
  that Biden needs to grab --- are they really interested in hearing a
  bunch of legitimate-sounding Christian rhetoric, or are these people
  who are interested in something else? And will the swing vote come
  down to that? I am not actually convinced that religion will play that
  big of a role in the swing vote.
\item
  ross douthat\\
  I think Trump would have been better served just attacking Biden on
  the issue of abortion, where Biden actually used to be closer to the
  center and has moved more to the left in his campaign, and just made
  it an issues-based appeal because I think there are a lot of Americans
  who are conflicted about abortion who are swing voters.

  Put it this way --- I think the swing voters, Liz, think the basic
  idea of Joe Biden as a guy who goes to mass on Sunday and doesn't
  always agree with his church, that's them, too, right? So saying that
  those kind of people are somehow against God, it's just an attempt to
  mobilize people already voting for Trump, not to win over the
  undecided.
\item
  frank bruni\\
  We have to let Liz go in a moment. But before we go, Liz, I understand
  you may have a recommendation for our listeners.
\item
  liz bruenig\\
  So I would recommend logging off, which is not to say not using your
  computer or whatever. That's just a tool. But get off social media.
  You know what I mean. I think most of the people who say log off point
  to the fact that there's so much misinformation and lying on the
  internet that it can really cloud your judgment, and that's true. But
  worse is that there is so much true information on the internet. And I
  don't think we're actually particularly well-designed to cope with
  getting news of the entire world updated every 10 minutes on a live
  stream. I think it's extremely stressful. It's too much true
  information. And I think that it can lead to a real sort of spiritual
  darkness. It used to be the sole province of God to know a full
  accounting of all the human evil that was happening at any given time.
  Now it's the province of any Twitter user, and it's a heavy burden. So
  log off. I close Twitter for six hours a day at a time and have
  nothing to do with it, and I'm much happier knowing less.
\item
  frank bruni\\
  Thank you for that, Liz, and thank you so much for coming on the show.
\item
  ross douthat\\
  Yes, thank you, Liz.
\item
  liz bruenig\\
  Thanks so much for having me, guys.
\item
  frank bruni\\
  So that's our show this week. Thank you all for listening. If you have
  a question you want to hear us debate, share it with us in a voicemail
  by calling 347-915-4324. You can also email us at
  \href{mailto:argument@NYTimes.com}{\nolinkurl{argument@NYTimes.com}}.
  ``The Argument'' is a production of The New York Times opinion
  section. The team includes Vishakha Darbha, Phoebe Lett, Paula
  Szuchman, and Pedro Rafael Rosado. Special thanks to Brad Fisher and
  Kristin Lin. We'll see you next week.

  Hey Jeneen, welcome to the technical wonder of ``The Argument.''
\item
  jeneen interlandi\\
  I'm having so much fun right now. This is the most fun meeting I've
  been in all week.
\item
  frank bruni\\
  Wow.
\item
  jeneen interlandi\\
  Don't you feel sad for me?
\item
  ross douthat\\
  That's brutal. That is a brutal assessment of your week.
\end{itemize}

\href{https://www.nytimes3xbfgragh.onion/column/the-argument}{\includegraphics{https://static01.graylady3jvrrxbe.onion/images/2018/10/03/opinion/the-argument-album-art/the-argument-album-art-square320-v3.png}The
Argument}Subscribe:

\begin{itemize}
\tightlist
\item
  \href{https://itunes.apple.com/us/podcast/id1438024613}{Apple
  Podcasts}
\item
  \href{https://www.google.com/podcasts?feed=aHR0cHM6Ly9yc3MuYXJ0MTkuY29tL3RoZS1hcmd1bWVudA\%3D\%3D}{Google
  Podcasts}
\end{itemize}

\hypertarget{is-individualism-americas-religion-1}{%
\section{Is Individualism America's
Religion?}\label{is-individualism-americas-religion-1}}

\hypertarget{opinions-jeneen-interlandi-and-elizabeth-bruenig-join-the-podcast-1}{%
\subsection{Opinion's Jeneen Interlandi and Elizabeth Bruenig join the
podcast.}\label{opinions-jeneen-interlandi-and-elizabeth-bruenig-join-the-podcast-1}}

With Frank Bruni and Ross Douthat

Transcript

transcript

Back to The Argument

bars

0:00/0:00

-0:00

transcript

\hypertarget{is-individualism-americas-religion-2}{%
\subsection{Is Individualism America's
Religion?}\label{is-individualism-americas-religion-2}}

\hypertarget{with-frank-bruni-and-ross-douthat-1}{%
\subsubsection{With Frank Bruni and Ross
Douthat}\label{with-frank-bruni-and-ross-douthat-1}}

\hypertarget{opinions-jeneen-interlandi-and-elizabeth-bruenig-join-the-podcast-2}{%
\paragraph{Opinion's Jeneen Interlandi and Elizabeth Bruenig join the
podcast.}\label{opinions-jeneen-interlandi-and-elizabeth-bruenig-join-the-podcast-2}}

Thursday, August 13th, 2020

\begin{itemize}
\item
  frank bruni\\
  I'm Frank Bruni.
\item
  ross douthat\\
  I'm Ross Douthat, and this is ``The Argument.'' {[}THEME MUSIC
  PLAYS{]}

  Today, a conversation about where we stand with COVID-19. Can the U.S.
  still contain its spread, or are we already defeated? Then nearly 1/5
  of all Americans identify as Catholic, but can their church say
  ``Black Lives Matter``?

  Frank, welcome back.
\item
  frank bruni\\
  Hey, Ross.
\item
  ross douthat\\
  So I think our listeners may be especially glad to hear your voice,
  after last week's all-conservative spectacular, {[}BRUNI LAUGHS{]} in
  which I was the moderate squish, arguing with more pro-Trump
  conservatives. Did you give it a listen?
\item
  frank bruni\\
  I did, and all-conservative spectacular was not the phrase that came
  immediately to mind. It was an interesting experience, Ross.
\item
  ross douthat\\
  That's all I can ask, Frank.
\item
  frank bruni\\
  I wanted to be there. I wished I could have inserted myself ever so
  briefly, just to kind of ask your guests, who seemed to think Trump
  had done a good job managing the pandemic, what they make of our
  world-leading case counts and death totals and how they explain that.
  But seriously though, I agreed with many listeners who wrote in and
  said that it was a discussion they wouldn't normally get to hear. It's
  the kind of thing that they hoped to hear when they turned to our
  show, something that hadn't crossed their radar before and they might
  not have thought of. And if it's OK, Ross, I want to read what one
  reader named Donna wrote in because I thought it was very interesting.
  She says, ``I found the latest right-wing coup episode one of your
  best ever. I like to hear both sides of an argument. Your show
  provides that, to an extent. But Ross often seems outnumbered and
  ganged up on by Michelle and Frank. I would like to know more about
  far-right views, but it's difficult for me to listen to Fox News
  opinion segments or AM talk radio for all the interruptions,
  over-speaking, and name-calling.'' So that's a pretty great
  endorsement, Ross.
\item
  ross douthat\\
  There's very little name-calling on our show, and we're proud of it.
\item
  frank bruni\\
  No, you and I only call each other names after we turn off the
  microphones.
\item
  ross douthat\\
  Yeah, or our producers---
\item
  frank bruni\\
  I'm kidding. I'm kidding.
\item
  ross douthat\\
  ---call both of us terrible names.
\item
  frank bruni\\
  Well, that's for sure, yeah.
\item
  ross douthat\\
  Well, I appreciate Donna's take. I know not every listener probably
  had the same take, and I suspect Michelle herself probably had a
  slightly different one. But she's off this week, and I will be
  suspiciously and conveniently on vacation next week. So it'll be a
  little while before I get to hear her take. Anyway, let's get to this
  week's episode. For our first segment, we're joined by Jeneen
  Interlandi, who writes about health and science for opinions editorial
  board and for The New York Times Magazine. Listeners may remember her
  voice from way back in March when we were wondering if quarantine
  would last through the end of that month. Back then, Jeneen told us to
  hunker down. She said the virus wasn't likely to go anywhere through
  at least the end of the year. And at the time, that seemed --- well,
  for a lot of people --- tough to imagine. And yet, here we are, still
  in limbo, waiting on a vaccine with more than 5 million cases since
  March, more than 162,000 dead, the federal government gridlocked, and
  a state-by-state patchwork of rules for re-opening businesses,
  schools, and more. I think, back in March, it still seemed possible
  that the American response could be more effective than Western
  Europe's. But now, at least pending a second wave across the Atlantic,
  our response looks definitively worse. So we asked Jeneen to come back
  to talk about what went wrong, what, if anything, went right, and
  where we go from here. So Jeneen, welcome back to ``The Argument'' and
  congratulations, I guess, on being right about everything.
\item
  jeneen interlandi\\
  Ha! Thank you for having me back, happy to be here. Not happy to be
  right about any of that stuff.
\item
  frank bruni\\
  Hey, Jeneen, great to have you on the show.
\item
  jeneen interlandi\\
  It's great to be here.
\item
  ross douthat\\
  So I, actually, was not here for that episode, mostly because I was
  ill with something that might have been COVID-19 myself. So can you
  just start by giving us a sketch of what's happened between March and
  now and also how, if anything, have your views of the underlying
  situation changed?
\item
  jeneen interlandi\\
  What I think happened is that we did not shut down sufficiently
  enough, we were too quick to reopen, and we didn't employ any of the
  very basic measures that public health experts were calling for
  sufficiently enough to get things under control, and I think we're
  still not doing that.
\item
  ross douthat\\
  To the second part of my question, has anything changed in your
  assessment of the disease itself, in terms of its lethality, how it
  spreads, things like that?
\item
  jeneen interlandi\\
  I did not expect that asymptomatic transmission was as much of a
  problem as it has turned out to be, so I think they've changed on
  that, definitely a lot of evolution in mine and everyone else's
  thinking on mask-wearing. Not proud to admit that I think back then, I
  was certainly someone who was saying, the masks don't protect you very
  much. They might stop you from transmitting it. But if you don't think
  asymptomatic transmission is a problem, then why would you advocate
  mask-wearing? And given that we had such low supplies and hospitals
  were running out, my position in the beginning was very much like,
  save the actual masks for people who need them, for frontline workers.
  And you don't actually need to go out and buy those and start hoarding
  them. So I definitely changed how I think about that.
\item
  frank bruni\\
  Jeneen, Ross's introduction made clear that you were prescient in some
  ways here. You thought that this would be with us for a while. You
  thought we were going to have to hunker down. I'm curious, what aspect
  of the country's response to this has most surprised you, and what's
  most infuriated you? And they may be the same thing.
\item
  jeneen interlandi\\
  You know, one of the things that's actually surprised me is how
  quickly Americans have gotten onboard with the idea of mask-wearing
  and even social distancing. We, in the media --- and it's for
  understandable reasons --- make a lot of the anti-mask contingent and
  of all the pushback against basic sound public health advice. And I
  think sometimes we lose sight of the fact that recent surveys have
  shown 60 percent to 70 percent of Americans are totally onboard with
  mask-wearing, actually want stricter control measures in place, and
  still trust public health experts. And that's really heartening, and
  it's been surprising.
\item
  frank bruni\\
  When you say that, I found myself wondering, do you feel that if our
  leaders could all get on the same page, show some consistency that
  Americans are actually ready to tackle this thing, but they lack the
  leadership?
\item
  jeneen interlandi\\
  I think that consistent messaging is hugely, hugely important. The
  C.D.C. has a playbook for how to communicate during a pandemic or
  during any national health crisis. And one of the key principles is
  consistent messaging and consistent communication. Just because it's a
  new era or a different era than the last time we had a global disease
  outbreak --- we have the internet, we have social media --- doesn't
  mean that you throw that principle out the window. There's so many
  people there that are on the fence, and they don't know what
  information to believe. And so when you have lots of conflicting
  messages, not just out in the ether or off in cyberspace but coming
  from officials talking on national television, it makes it really hard
  for people to know what to do. And it gives anybody who's skeptical
  and out to just kind of pick whatever advice best suits their mood or
  their opinion. And so when you have people speaking with one voice, I
  think you cut a lot of that noise out, and it makes it a lot easier
  for people to figure out what the right course of action is. And the
  fact that we have so much mixed messaging and the most confused we've
  gotten to 60 percent to 70 percent of mask-wearing, I feel like,
  imagine how much better we could do.
\item
  ross douthat\\
  But there is also the problem, as you've just suggested, it's hard to
  have a consistent message when your knowledge of the disease is,
  itself, is evolving.
\item
  jeneen interlandi\\
  Yeah, sure.
\item
  ross douthat\\
  So we went from an initial period where public health authorities were
  focused on hand-to-hand surface transmission, where everybody was
  talking about washing their hands and terms like fomites were getting
  thrown around. And now, months in, we're in a situation where airborne
  transmission seems much more important. And there was there was a
  period of time --- it didn't last incredibly long, but there was a
  period of time--- when certain segments of the internet were more
  trustworthy about whether to use a mask than were official public
  health pronouncements. And it just seems like that's a challenge that
  there's no playbook to deal with, right? Your assessments evolve, and
  you have to change your statements. And then you look like you're
  being inconsistent, and it's hard to map out a way to avoid that in
  advance, except being maybe a little more cautious in how much you
  talk down mask wearing.
\item
  jeneen interlandi\\
  Yeah, I think so. But I also don't think that the fog of war, as we
  refer to it, mentality --- or just the idea that in the beginning of
  an outbreak, especially with a completely novel pathogen, the idea
  that we don't know everything and we're going to make missteps is
  inconsistent with the principle of communicating consistently, right?
  Part of consistent communication is being open and honest about what
  you don't know. And then as you learn more, say, OK, turns out we were
  wrong about this. Here's how we're shifting course. That's actually
  part of consistent and open communication, or that's part of how you
  combat misinformation, is to be honest about what you don't know and
  to correct for it as you go along. And you're right. It also does mean
  a bit of humility and a bit of caution when you're issuing guidelines.
\item
  frank bruni\\
  Hey Jeneen, I want to ask you a question because I think there's some
  difference of opinion between you and Ross that I'd like to get to,
  which is if you were writing the script for what this country does
  over the next four to six weeks, what would that script look like? And
  then I really want to hear Ross's response to it.
\item
  jeneen interlandi\\
  OK, so if I'm writing the script from scratch or just from the point
  that we're at now, I think the first thing I want is much better data
  and much better use of data. So I want states and localities to make
  all of the information they have public. I want the C.D.C. to work
  aggressively with them to make that information useful. And then I
  want state and local officials, if not federal officials, to take that
  information, to take that data --- and individual states and
  localities have done this. Harris County, Texas did a great job at
  this --- basically creating a color-coded threat assessment that tells
  officials and citizens exactly what they need to do, based on how
  extensively the virus is spreading in their communities. So one of the
  things that people have argued a lot about is, you can't have a
  nationwide lockdown. That doesn't work because some places you have
  really bad outbreaks. In other places, you don't have anything. And
  all you do is breed resentment in the communities that don't have bad
  outbreaks, right? And then people don't want to follow anything, and
  they get fatigued. Well, if you use data to actually create some sort
  of forecast or assessment of how the virus is actually spreading in
  given communities, it could be a nationwide kind of response system,
  but it's employed locally. So I think that's the first thing I would
  want to see. And then from there, the very next thing is to get that
  data as accurate as possible, you need much more testing. And so I
  want to see much more ramping up of testing. It's kind of tiresome to
  even say that because we've been talking about this since March, but
  it's still a huge problem. It still takes up to two weeks or longer to
  get test results back in some places. And that makes the testing
  itself almost pointless. If it takes that long to find out what the
  results are, why bother getting tested? So I want to see more of that,
  and I think there's a lot the federal government could do that it's
  not doing or it hasn't done to improve the testing situation.
\item
  ross douthat\\
  What could the federal government do more of on testing?
\item
  jeneen interlandi\\
  Again, I feel weird saying these things because we've been seeing them
  for so long, but they remain true. We have the Defense Production Act
  for a reason. We also have provisions that allow for patent overrides,
  and we should be using those things. If not now, then when? Why do
  those things exist? They exist for exactly these situations. There are
  some companies that have developed rapid point-of-care tests that,
  from all of the sourcing I have, pretty accurate, really good tests
  could solve a lot of the problems. But the companies that develop
  those tests can't make enough of them to deploy them across the
  country in a way that would be meaningful. So what the government need
  to do is step in and say, we're going to take this patent because you
  don't have the ability to produce it at the level that we need. We're
  going to compensate you fairly for it, and then we're going to task
  several companies across the country, under the Defense Production
  Act, with making this product at scale and getting it out as quickly
  as possible. Would that solve every problem tomorrow? No. Could you
  produce every test that you possibly need in the country in a week?
  No, but you should be trying to do those things. And if you did, we'd
  be in a much better place than we are right now.
\item
  ross douthat\\
  There was an interview with Bill Gates a few days ago --- I don't know
  if you saw it--- where he was very angry about many different things
  related to testing. But one point he made that he thought was
  interesting was that he's arguing that these testing companies are
  getting paid large sums of money--- public money, in many cases--- for
  test results but not for timeliness of test results, right? And he was
  saying, basically, you should only reimburse companies that are
  delivering test results within 24 to 72 hours.
\item
  jeneen interlandi\\
  That's a fantastic idea. And other people have floated that to me, and
  I think it makes a lot of sense. You start tying reimbursement levels
  to how long it takes to get the testing results, and guess what?
  Companies are going to figure out how to get those results a lot
  faster.
\item
  ross douthat\\
  So I want to go back to the first thing you said about lockdowns,
  right? You said we didn't lock down well enough and long enough. And I
  think those are slightly separate things, so I want to ask you more
  about that. Do you think that the problem was our lockdowns just
  needed to go on an extra month in most states, and reopening should
  have been slower? Or do you think that we didn't achieve the scale of
  lockdown, in terms of people not leaving their homes at all, that some
  countries in Western Europe at least attempted? Or do you think it's
  both?
\item
  jeneen interlandi\\
  Actually, both. I think it's absolutely both of those things. The
  numbers I've seen are, like, if you look at a place where we had
  really good lockdown, it was essentially 50 percent, compared with I
  think it was 80 percent or 90 percent I saw of lockdown in countries
  where they've successfully not only flattened the curve, but actually
  turned it downward. So they had just much more stringent controls. And
  they had things like roadblocks, and you can't leave a community. And
  if you do, they're tracking if you're going from one state to another.
  We didn't have that here. We had people in New York fleeing, leaving,
  going to vacation homes. We had people going down to Florida. We had
  people traveling. We still had public transportation. And I think we
  needed to do a lot more in certain places than we did. And I
  recognize, absolutely, that that's really tricky because every time
  you tighten the lockdown procedures, you hurt the economy that much
  more. But I would argue that by not doing that, the economy is
  suffering probably worse because it's dragged out a lot longer than it
  might have had to otherwise.
\item
  ross douthat\\
  I tend to agree that the initial lockdown could have been firmer in
  certain ways. But I think in terms of their length, we basically hit
  the plausible limit. And I think you could tell that we hit the
  plausible limit because of the kind of reactions that we started
  getting to it, both right-wing coded reactions like the anti mass
  protests, but also I think the surge of political protests around the
  George Floyd shooting were, in part, a reflection of social energy
  that had been tamped down and maybe wouldn't have burst out in that
  particular way without that kind of provocation, but would have burst
  out in some way over the summer. I feel like you couldn't run the
  lockdown for multiple months, especially, and this also goes back to
  the question of messaging, but the initial messaging was, we're going
  to do this for a few weeks, and the goal is to flatten the curve,
  right? And there's always been this conflict and this uncertainty, I
  think, in public messaging about whether the goal is flattening the
  curve and sort of extending the epidemic in a way that doesn't
  overwhelm hospitals and gives us more time to get treatments and
  eventually gives us more time to get a vaccine, versus suppression,
  versus trying to actually stamp out the virus.
\item
  jeneen interlandi\\
  But there's a third thing there. It's not just that. You're not just
  flattening the curve. What you're trying to do is get certain things
  in place so that you can manage the outbreaks more smartly, right? So
  if you do shutdown in a smart way, you're supposed to be building up
  your testing capacity, your contact tracing capacity, your isolation
  and quarantine, which we never did enough of. You're supposed to build
  those things up so that when you do reopen, you can actually monitor
  more effectively and keep flare-ups from becoming outbreaks all over
  again. We didn't do any of those things. So Americans made huge
  sacrifices in abiding by the lockdowns to the extent that they did,
  and the federal government betrayed them by not using that time as
  effectively as it should have.
\item
  frank bruni\\
  I got to tell you. As I listened to the two of you, I feel such a
  sense of surrender and failure. I hear both of you saying in different
  ways, we can't really do aggressive lockdowns going forward because we
  just don't have it in us. But it turned out, we kind of really didn't
  have it in us, if we're talking about the whole country, in those
  first months. And I'm haunted by this sense that all of this has
  revealed some fundamental shortcomings and flaws in the American
  character that are the principal explanation, along with failed
  political leadership, for our world-leading exceptional status and the
  number of infections and deaths.
\item
  ross douthat\\
  So I would speak up for the American character a little bit. I think
  that yes, we did not have the scale of lockdown that some Western
  European countries had. I also think, though, that the scale that they
  had was, in some cases, effectively overly repressive. And you had
  cops chasing people out of parks in rural parts of England and things
  like that I think were never going to fly in the US. And also, the
  country really did shut down its entire economy --- not for a week,
  but for several months, taking on its own set of significant costs of
  kids out of school and people dealing with depression and mental
  illness and all kinds of things. We really did do that. And we did, in
  fact, flatten the curve, right? I mean, that's the odd thing about all
  this.
\item
  frank bruni\\
  In some places and in some places only recently.
\item
  jeneen interlandi\\
  Yeah, and for really short periods of time. The piece I did for the
  magazine was on Harris County, Texas. They had aggressive stay-at-home
  orders. When Governor Abbott allowed the localities to kind of manage
  the response, Harris County was one place that really had pretty
  aggressive stay-at-home orders. And they had really consistent
  messaging, and they actually did flatten the curve and started to get
  a downturn. But what happens is none of the surrounding communities
  did the same thing. And eventually, Governor Abbott stepped in and
  said, no, you can't do this. You can't mandate masks. You can't do X,
  Y, and Z. And then what you saw was they lost all of those gains that
  they had made, and I think that's the story in a lot of places.
\item
  ross douthat\\
  Well, but I guess I have a sort of twofold view of this, right? I
  think the central government completely failed. I think that President
  Trump obviously completely failed. I think there was an opportunity
  for substantial suppression that would have required major contact
  tracing. It would have required, as Jeneen says, testing to be not
  just ramped up --- we did eventually ramp it up--- but to ramp it up
  sooner and get results back faster. And we haven't been able to
  achieve that. And it probably would have required more aggressive
  quarantine measures, state travel restrictions, like she said, people
  in COVID hotels earlier on. All of that we failed at, clearly.
\item
  frank bruni\\
  Well, that's what I'm saying, Ross.
\item
  ross douthat\\
  What I'm saying, though, is that at the start of the epidemic, the
  public rhetoric was, the danger is that every part of the US is going
  to be like New York City ended up being, where you're going to have a
  really, really high case fatality rate because hospitals are going to
  be overwhelmed, doctors aren't going to be able to treat patients, and
  you're going to end up with basically that soaring spike of deaths
  that you've got in New York City everywhere. And that has not
  happened, and it hasn't happened in places like Texas and Florida and
  the Southwest that have had to new outbreaks. Those outbreaks have
  been bad, but they have looked like the curve on the initial charts
  that we had--- not the spike, but the slower curve that rises and
  falls. And it's a tragedy that we didn't achieve more than that, but
  we also have not had a situation where everywhere has turned out like
  New York City, nothing close to that right now.
\item
  jeneen interlandi\\
  So I would argue that at the outset, it wasn't every place is going to
  be New York City. It was going to be, we're going to have lots of
  different outbreaks. And basically what you're talking about is a lot
  of this is preventable if we use just the basic public health measures
  that we have now. So we don't have to wait for technology that doesn't
  exist. If we do these things, we can save a lot of lives. And guess
  what? Most of those lives are going to be in low-income communities,
  communities of color, and really marginalized places where they don't
  have good hospitals in their communities. So that's the issue I would
  take with that.
\item
  frank bruni\\
  All due respect, Ross, when I listen to you talk, I kind of don't hear
  the full recognition that tens of thousands of Americans, who might
  not have had to die, died here.
\item
  jeneen interlandi\\
  Died, yeah.
\item
  frank bruni\\
  I don't think we can kind of get so deep in the weeds in but this, but
  that, but this is the way we are. I don't think we can skate over that
  so easily. But I want to ask you a question that I want to get
  Jeneen's response to. This is about America, how we've responded and
  kind of failed in a way other countries haven't. This is about the
  American character. Ross, do you think we have found the right balance
  between famous American individualism and communal protection,
  communal concern?
\item
  ross douthat\\
  No, I don't. I think that we could have had more communal protection
  and more communal concern. But I do think lots and lots of people
  expected us to have not a New-York-level epidemic everywhere but
  multiple New Yorks, multiple versions of the mountain of deaths that
  Andrew Cuomo bizarrely put on a poster celebrating how New York
  handled things, and we have not had that. New York, and the Northeast
  generally, is a distinct and tragic curve, and the other curves, as
  bad as they've been, have not been that bad. And that is a testament,
  I think, to real public health achievements and real sacrifices that
  Americans made, even in the places that maybe reopened too early and
  so on. You still had social distancing. You still had mask-wearing.
  And the other thing I'm just trying to stress is that viruses are
  really unpredictable and hard to contain. And those challenges are
  real, and they don't go away just by having better leadership in the
  White House than we've unfortunately had. And I guess the final
  thing--- and here I am I'm really curious about Jeneen's take ---
  there are a lot of questions about where you actually reach herd
  immunity. And there's a sort of lively debate about whether, at least
  with the kind of social distancing we still have in place, New York
  City might be at a de facto herd immunity, with around 20 percent or
  25 percent of the country infected.
\item
  janeen interlandi\\
  I don't agree with that at all. I don't agree with that.
\item
  ross douthat\\
  OK, so tell me why that view is wrong.
\item
  jeneen interlandi\\
  Well, OK, so first of all, I think herd immunity is a little bit
  misunderstood as a concept, right? It's the threshold at which enough
  people in a given population become immune to a virus that it can no
  longer cause huge outbreaks. But we normally calculate that based on a
  standardized intervention, like vaccination. When we talk about herd
  immunity as a scientific concept, we're almost always talking about
  vaccination, how many people need to get vaccinated. That is not the
  same thing as herd immunity or as an ongoing, real-time infectious
  disease outbreak. So for the latter, it's actually much tougher to say
  what the threshold would actually be. And you're correct that some
  studies have cited 20 percent, but many more people seem to think it's
  going to be closer to 60 percent, 70 percent, 80 percent, right? So
  it's going to be much higher than 20\%. And the bottom line really is
  that we just don't know. So I think that's one reason it's kind of
  absurd to talk about herd immunity as a strategy. But realistically,
  most people don't think it's going to be 20 percent. They think it's
  going to be significantly higher. So that's the first thing that I
  would say about that.
\item
  ross douthat\\
  I'm not advocating for it as a strategy.
\item
  jeneen interlandi\\
  OK.
\item
  ross douthat\\
  I'm arguing that we should observe it if it seems to be happening as a
  reality. Sweden was held up as the sort of control group for the virus
  early on, the country in the West that was doing the least to contain
  it. And the Swedish curve, without strong interventions, has, itself,
  bent down to the point where Sweden looks like a low-transmission US
  state at this point. And Sweden is a data point for the argument that
  you could get herd immunity at 20 percent or 25 percent.
\item
  jeneen interlandi\\
  I don't think that they've actually done very well, right? They have
  higher infection rates than their neighboring countries. They have a
  pretty high mortality rate. I don't think they're anywhere near herd
  immunity. In fact, there was an open letter written by, I think, 20 or
  30 epidemiologists, including some from inside Sweden, saying they're
  not even close to herd immunity. There's still high susceptibility in
  the larger population, and their economy doesn't appear to be doing
  much better than their neighbors' economies for the strategy that they
  took.
\item
  ross douthat\\
  I want to be clear on what I'm saying. I don't think the Swedish
  strategy has contained the virus more effectively than their
  neighbors. It clearly has not.
\item
  jeneen interlandi\\
  OK.
\item
  frank bruni\\
  So what are you saying?
\item
  ross douthat\\
  I'm saying that if you look at a chart right now of Sweden's total
  cases and daily new cases and total deaths, the death line has
  flattened. The total case line has almost flattened. And daily new
  cases have done a big arc and now are very low. Sweden has death rates
  that are worse than Denmark and Norway that did severe lockdowns, but
  it's death rates are not nearly as bad as some of the worst centers of
  the epidemic. That cries out for an explanation. So I'm just saying, I
  think there are a lot of data points right now that suggests that
  something maybe limits the virus' spread at around that 25 percent
  threshold.
\item
  jeneen interlandi\\
  I take you at your word at what the Swedish curve looks like right
  now. I don't think that that in any way implies that the herd immunity
  threshold there or anywhere else is necessarily 20 percent. I think
  there's about 5,000 different potential explanations for that. I don't
  think it's a leap from they've managed to bend the curve down to they
  have herd immunity. And I think also to your earlier point ---
\item
  ross douthat\\
  Well, I want to be clear, there is herd immunity for the level of
  social distancing we have now.
\item
  jeneen interlandi\\
  Yeah, herd immunity is not a fixed number.
\item
  ross douthat\\
  Exactly.
\item
  jeneen interlandi\\
  Exactly, yeah. So that's fine. But what I was arguing earlier, you
  want to make decisions about what things need to be closed down, what
  things can be reopened, how much social distancing you need, whether
  you want mask-wearing indoors, outdoors, everywhere based on what the
  case positivity rate is in a given area. Now, if you're basing it on
  the data, then you're doing it the right way. Whether you're at a
  thing that you can reasonably define as herd immunity or not is kind
  of incidental. So I feel like you're arguing for herd immunity as a
  strategy, even though I know you say you're not. I'm just not sure how
  to understand that.
\item
  ross douthat\\
  No, I think we agree.
\item
  frank bruni\\
  {[}LAUGHING{]} I don't know about that.
\item
  ross douthat\\
  I think, in part, it's a question about our level of optimism going
  forward and how bad we expect the disease to get in the fall. I think
  that we agree that you want basically a data-driven assessment of the
  disease's prevalence in your community. And I agree, that's more
  important than a theory of herd immunity that might be proven wrong.
\item
  jeneen interlandi\\
  Yeah, again, I don't think we know the number for SARS-CoV-2 at all. I
  just don't think it's going to end up being near 20 percent, based on
  what I've seen. The other thing I would point out is in places like
  New York and I think in most of the Northeastern United States,
  studies have shown, the thing that they seem to have done right is to
  be much slower to open indoor dining and to open bars.
\item
  ross douthat\\
  Yes. And I think New York City is particularly good at that, and I
  think that's why things are much better in the Northeast than they are
  in other parts of the country. But again, that goes back to not
  throwing your hands up or saying, it's great that we're also
  libertarian here. It's basically based on data. So based on data,
  we're not going to open the bars and restaurants, and we're going to
  be able to maybe open the schools in the fall. It's a trade-off. Well,
  and that I guess can be our last question for you, which is, should
  the schools open in the fall?
\item
  jeneen interlandi\\
  Again, I think it should be based on data. And I think if you do it
  based on the data and you say, OK, we have to have a certain
  positivity rate to safely open the schools, what you're going to find
  quickly from there is, OK, how do we get to that positivity rate?
  Well, the first thing you have to do is think about whether you want
  to keep your indoor dining open, whether you want to keep your bars
  open. And then I think the conclusion is that you probably don't.
\item
  ross douthat\\
  But in practice, it seems like a lot of places have chosen indoor
  dining and bars over schools. And you also have a dynamic where you
  have certain states that maybe have a transmission spread rate that is
  low enough to reopen schools aren't reopening them. This seems to be
  happening in the mid-Atlantic. And then states that have a higher
  transmission rate --- Georgia, Texas, and so on --- are opening them.
  So I think there is a wariness about just following the data in
  liberal states, as well as conservative states. And the sort of
  polarization that once Trump said, let's open the schools, suddenly
  there was ---
\item
  jeneen interlandi\\
  ``That's a terrible idea!''
\item
  ross douthat\\
  --- political pressure not to open them in places that could.
\item
  jeneen interlandi\\
  You have these two kind of conflicting things in play at the same
  time. On one hand, it's extreme fatigue on the part of everybody. I
  don't care what your political persuasion is. You're fatigued right
  now from all of the closures and just the abnormal existence we've
  had. And then on the other hand, if you have gains and you've managed
  to get to a place --- if you live in Manhattan like I do --- where
  you're not hearing ambulances go down your street literally every five
  minutes and you're terrified to even just go outside, you're terrified
  of losing those gains. So you have this kind of terror and also this
  fatigue happening at the same time, and I think that's part of where
  we're at.
\item
  ross douthat\\
  Well, I think terror and fatigue is a good place to close this
  discussion.
\item
  frank bruni\\
  Jeneen, thank you so much.
\item
  jeneen interlandi\\
  Thanks, this was fun.
\item
  ross douthat\\
  Jeneen, we'll have you back at Christmas when we'll all have herd
  immunity, and we'll look forward to that.
\item
  jeneen interlandi\\
  Our holiday special, OK. Thanks, guys.
\item
  ross douthat\\
  We'll be right back.
\item
  frank bruni\\
  Welcome back. We're going to talk about American Catholicism, which,
  in an age of polarization, is a rare institution that sprawls across
  our political divide. It's a church with a progressive-leaning pope
  and a conservative American hierarchy. It's the church of Attorney
  General William Barr and of Joe Biden, the Democratic nominee for
  president. It cuts across class. It cuts across ethnicity, which means
  that it's uniquely cross-pressured. To talk about some of those
  pressures, we've invited our colleague Liz Bruenig. Liz, thanks so
  much for joining us on ``The Argument'' today.
\item
  liz bruenig\\
  Thanks for having me on.
\item
  frank bruni\\
  Liz, you recently wrote an op-ed about American Catholicism's unease
  with the Black Lives Matter movement. ``Racism Makes a Liar Out of
  God'' was the terrific headline on your extremely interesting piece.
  Tell us about the argument you made.
\item
  liz bruenig\\
  Well, thanks very much. I was interviewing Gloria Purvis, who's a
  Black Catholic journalist, in a sense, and that she has run a
  broadcast radio show on EWTN, which is the largest Catholic media
  distributor in the world. She also had a television show. And one of
  EWTN's local affiliates, the Guadalupe Radio Network in Texas, which
  is, I believe, one of their largest affiliates, if not the largest,
  dropped her show after the murder of George Floyd because she began to
  speak out quite vociferously about racism. And my response to that was
  just that it wasn't particularly Catholic of them, that she wasn't
  saying anything unorthodox, that everything she was saying is
  completely evinced and held up by church teaching. And so I just don't
  think it should be that hard for Catholics to look at the claims being
  made, put aside various organizations, whether you like certain
  figureheads and so on and so forth, and just focus on the arguments
  emerging from this movement. And I think they're pretty easy to
  endorse for a Catholic. I don't see the hang-up.
\item
  frank bruni\\
  So why, then, do you think she was dropped? How do you explain? Is
  that about racism? Is it about particular and peculiar discomfort with
  Black Lives Matter? What's going on?
\item
  liz bruenig\\
  I think it's probably somewhat to do with particular discomfort with
  Black Lives Matter, and I think quite a lot of it has to do with the
  fact that having uncomfortable conversations doesn't necessarily make
  for good radio. People don't like to sit around and listen to people
  really hash out issues of life and death import --- and she has
  co-hosts who sometimes have friendly disagreements with her ---
  because that can be very distressing to hear, the same reason people
  don't want to have those conversations interpersonally. And so I think
  it was a combination of that and widespread conservative mistrust of
  the Black Lives Global Network, this organization that has laid claim
  to the phrase ``Black Lives Matter,'' which, of course, preceded it.
\item
  frank bruni\\
  One thing I wanted to ask you before we went on any further is when
  we're talking about the Catholic Church, what are we really talking
  about, and how much meaning does that phrase have? And what I mean by
  that is I would probably be in a pollster's box as a Catholic because
  I was raised in the Catholic Church, although I drifted far, far away.
  I believe you are a convert to Catholicism. Ross has his own
  relationship to Catholicism. So there are Catholics who care about
  what the pope says, and there are Catholics who never look in this
  direction. So what do we mean here by the Catholic Church?
\item
  liz bruenig\\
  Well, it's a global institution, and so every Catholic's response to
  various world goings-on is going to be different. And I think when we
  we're talking about the Catholic Church in this context, we refer to a
  number of things. The Catholic Church refers to, at times, the
  hierarchy, the church authorities. It, at times, refers to the laity,
  everyone who is baptized Catholic and receives the sacraments and is
  confirmed to the church. And then at times, it refers to the internal
  logic, the tradition, the rules of the Catholic Church. It takes all
  of those things to constitute it. So I think we're always referring to
  a sort of mixture of things there.
\item
  ross douthat\\
  In many ways, the church feels incredibly weak at this moment, and its
  sort of sprawl seems like part of its weakness, that it's sort of been
  through a period of decline. It had the sex abuse crisis. It's had
  this sort of internal argument over its own teachings, that, for my
  sins I've been involved in. And so things like the Black Lives Matter
  movement come along, and there is no Catholic authority figure or
  authority figures capable of saying, OK, we're going to harness the
  church's sprawl and spread to support this movement or to critique
  this movement or somewhere in between. And instead, you get this sort
  of very American, I guess, entrepreneurial thing where certain people
  appoint themselves a spokesman for Catholicism, and you have these
  particular clashes, like the one around Gloria Purvis. But there's no
  sense of Catholicism as something that's capable of coming together
  and playing a dynamic role at this moment. Although, but Liz, you sort
  of end your piece by suggesting that the bishops could do that to some
  extent?
\item
  liz bruenig\\
  Well, the bishops could put the hierarchy behind the March for Black
  Life in Washington D.C., which is what the open letter calls for, is
  for the bishops to join that march. What the bishops can't do it they
  can't make anyone like it, and the church teaching already supports.
  So that's two out of three. It's not bad. But American Catholics ---
  and at this point, Catholics all over the world, especially in North
  America and Europe --- these are liberal subjects. So their approach
  to Catholicism is not the approach to Catholicism that someone alive
  during the counter-reformation would have taken. It's an approach to
  Catholicism that's informed by the general ideological atmosphere
  around here, which is I decide what sounds good to me, and I discard
  what doesn't. And I am the sovereign of myself, and I operate within
  certain parameters. But very little is binding on me. And so you have
  some American Catholics who say, OK, if the hierarchy says we do it
  and it matches the church teaching, then we do it. It doesn't really
  matter if it matches our other intuitions.
\item
  ross douthat\\
  So let me push you a little bit on the dilemma here. I'm going to
  quote a not African-American, but African cardinal, Wilfrid Napier,
  who is from South Africa ---
\item
  liz bruenig\\
  That's not the one I was expecting.
\item
  ross douthat\\
  That's not the one you're expecting --- who is, himself, Black. This
  is how the hierarchy of the church now manifests itself, in tweets.
  But he tweeted, Black Lives Matter creates real conflict of
  conscience. The mission statement of BLM-GNF --- that's the
  organization you were talking about earlier --- advocates views on
  marriage and family that are totally repugnant to Catholic teaching,
  as are the actions of its activists. Yet there is an urgent need to
  expose all injustice against people of color. So this, I think, coming
  from outside the U.S., but it's still, I think, distills the dilemma
  for Catholics who are maybe more sympathetic to the cause of criminal
  justice reform or police reform than some of the people who wanted
  Gloria Purvis' show canceled, but I'd say the sort of current protest
  politics in its organizational and institutional forms and see a
  movement that is much more secular and left wing than was, for
  instance, the Civil Rights Movement at its peak, right? It's not
  emerging out of the African-American church. Its rhetoric emerges out
  of an academic milieu that is, I think it's fair to say, is pretty
  hostile to traditional Christianity and Catholicism. And as you just
  mentioned, in the penumbra of these riots, you get a lot of attacks on
  churches --- not tons and tons, but there's been a real uptick in
  vandalism of Catholic churches, some of it focused on figures like
  Saint Junipero Serra, who is a controversial figure for reasons
  related to the treatment of Native Americans under the mission system
  in California. But some of it is just sort of more all-purpose, right?
  Acts of vandalism that just knock the head off a statue of the Virgin
  Mary or something. And so it seems like there is actually some dilemma
  here where you're trying to figure out, how do you lend your support
  to the cause of racial justice without ending up just as sort of a
  religious ally of a movement that is pretty hostile to Catholicism?
\item
  liz bruenig\\
  I don't think that the attacks on churches and statues have been
  anti-Catholic in particular. They tend to be people who are upset
  about, as you said, Spanish colonialism, the treatment of Native
  Americans under the Spanish colonial regime. So there have been
  attacks on missions. There have been attacks on statues of Saint
  Junipero Serra. With any protest, especially of this size, you're
  going to get some delinquent sort of vandalism that has nothing to do
  with the protest at all. It's just people doing things they cannot
  normally do and kind of having a libidinal moment there where they go
  a little crazy. I think quarantine has compounded this
  already-existing tendency. But I wouldn't look at it as these people
  are anti-Catholic, they want the Catholic Church destroyed. That may
  well be in the case of some people. If you're on the internet, you
  hear from those people every day.
\item
  ross douthat\\
  I've never heard from those people. I don't know what you mean.
\item
  liz bruenig\\
  {[}LAUGHS{]} Yeah, who could they possibly be? We could probably
  rattle off some handles. But in reality, it seems like, you look at
  polling, the majority of Black people are Christians. And so it's not
  the case that the Black people who are calling for an end to police
  violence are dead set on the destruction of the nuclear family or
  whatever. That might be the case with these activists who are writing
  the copy, but the Black Lives Matter Global Network is not a
  membership organization. You don't have to pay dues and if you don't
  pay your dues, you can't show up to the protest. The organizing
  efforts against anti-racism don't require you to participate in
  whatever efforts there are or may be to destroy statues or undermine
  the church. On the other hand, the church is pretty good at
  undermining itself, so it doesn't really need any help.
\item
  frank bruni\\
  So when someone like Andrew Sullivan, whom you mentioned in your
  column, says Black Lives Matter and Catholicism are incompatible, is
  he kind of focusing on a detail that is less meaningful than he's
  making it out to be?
\item
  liz bruenig\\
  I think his complaint was that the Black Lives Matter Global Network
  is Marxist and that you can't be a Marxist and a Catholic. You can,
  actually. I have this on good authority.
\item
  ross douthat\\
  In a very complicated sort of way.
\item
  liz bruenig\\
  {[}LAUGHS{]} Marxism can be said in many ways, as I imagine Aquinas
  would have put it. And so I tend to think that people like Andrew
  would probably not be in favor of this movement in any universe,
  right? And so the Catholicism issue is an opportunity to make a shell
  argument that's not substantive. It's about something else. It's a
  procedural objection. Oh, well, I would, but I have all these other
  things that I have to subscribe to ahead of this, so I can't.
\item
  ross douthat\\
  Well, OK, but people are not wrong to look at the current moment of
  protest and say, well, this seems to be about not just racism and
  police brutality, but also a kind of substantial reordering of elite
  American institutions, for instance, in ways that are sort of much
  more secular and left wing than movements that the Catholic Church
  usually associates itself with and more hostile to not just Catholic
  beliefs on one particular issue, but the wider panoply of beliefs,
  right?
\item
  liz bruenig\\
  Right. No, I think that's true. I think that no institution coming out
  of American soil shares the aims of the Catholic Church. These are all
  liberal institutions. Their aims are circumscribed by the event
  horizon of temporality, right? So this is to say, the goals of all of
  the institutions that come out of American thought and American life,
  these are liberal institutions that have goals that are sort of
  utilitarian, or they have to do with sort of freedom and equality. And
  those are aspects of Catholic teaching. They're certainly not the end
  goal of Catholic teaching. So then the question is, how distant is
  this particular movement from Catholic Church teaching, and how
  distant can a movement be before we have to completely abandon it? And
  I would say that all that's required is a little of that good old
  discernment that we hear about so often, especially from our Jesuit
  friends, which would be paying attention to what aspects of the
  movement are being foregrounded and in what way you're being asked to
  support it. So if there is, say, a referendum in your city on police
  funding or the ways in which police can or should try to de-escalate,
  as opposed to defaulting to violence, or body cams or something like
  that, you can go ahead and think, what is the best way to cast a vote
  here that is going to hopefully create a more equitable landscape in
  this city for the people who live here with me? You don't have to
  affirm in the strongest terms even the entirety of the movement. But
  to the degree that you have power as a citizen, I think it's probably
  pretty easy to exercise that in a way that comports with any racist
  goals.
\item
  frank bruni\\
  Liz, what do you make of Trump's attacks on Biden's Catholicism or on
  Biden's religion, which is thus implicitly Catholicism? And do you
  think Biden's Catholicism factors into, colors this coming election in
  any interesting ways?
\item
  liz bruenig\\
  Trump is an interesting figure in that he sort of doesn't even really
  put up the bare pretense of being all that interested in religion.
\item
  frank bruni\\
  {[}LAUGHING{]} He does hold a Bible upside down very photogenically.
\item
  liz bruenig\\
  Right, he'll do stuff like that, which is vaguely insulting, and go
  stand out in front of a church. And the tone of that is always like,
  you piggies sure love this slop, don't you? Lap it up. {[}BRUNI
  LAUGHS{]} And it's a little insulting. I don't like it when any
  politicians talk about their religion. I've never heard one of them do
  it in a way that was remotely convincing, and Trump, in particular,
  just doesn't even really bother with it, which I kind of grudgingly
  respect because it's so clearly a charade. The ones that try to be
  earnest about it are, in some ways, a little more disturbing to me.
  But he has no room to attack Joe Biden's Catholicism, the nature of
  which I know nothing about. And to the degree that it will even matter
  at all in the election, which I doubt it will, it could be good for
  Biden. Catholics are classic swing voters. There are lots of Catholics
  in swing states. Catholics typically split right down the middle,
  50-50, Republican-Democrat. And if he can pull any Catholics who
  wouldn't have otherwise given him a shot because he is Catholic,
  that's possible, and that's interesting, I suppose. But my feeling is
  that American Catholicism has been so thoroughly Protestantized that
  it doesn't really matter. And moreover, very few Protestants could
  tell you in what ways Catholics are morally different than they are in
  the United States, right? So back in the day, you would have been
  called a crypto Catholic if you said something like, well, whether or
  not you go to heaven depends a lot on you. But this is also just sort
  of common American Protestant thought at this point. It's rather
  unusual to find very vocal Calvinists. And so because the lines
  between Catholicism and Protestantism the United States are sort of
  blurred at this point--- the evangelicalized Catholics on the right
  and then the sort of liberalized Protestants who have kind of lost a
  lot of those puritanical elements of Protestantism on the left--- I
  just don't really think it matters. It's definitely not going to be a
  JFK-type situation.
\item
  frank bruni\\
  Well, and what you're saying makes me think that's one of the reasons
  why Trump instinctively attacked Biden's religion in the larger sense
  and not necessarily his Catholicism because I think he, at some level,
  understands what you just said, which is that people don't necessarily
  see these bold lines where they once saw them between Protestantism
  and Catholicism.
\item
  ross douthat\\
  Trump is going for the idea, which is a true idea that among white
  Americans, there is a real pew gap between the parties, in the sense
  that Republicans are more likely to be church-goers, and Democrats are
  not. And there is an odd reversal of that in the actual leaders of the
  parties, in that Trump is not a church-goer and Joe Biden is. So
  Trump, in his way, is sort of going right at that and saying, oh, you
  think Biden's a church-goer, well, he's against God, right? I mean,
  that's sort of the classic Trump.
\item
  frank bruni\\
  What does that even mean, Ross?
\item
  ross douthat\\
  Well, I may not go to church, but he's against God.
\item
  frank bruni\\
  What does that even mean when he said he's against God? What does that
  even mean?
\item
  ross douthat\\
  Well, it's like when Trump was asked, back when he was first trying to
  figure out how to talk like a pro-lifer, and he said something about
  how, well, we really need to punish women who have abortions, which is
  not and has never been the main pro-life position on the issue. But to
  Trump, it seemed like something that he imagined a pro-life person
  would say. And I think you get some of that in the against god
  rhetoric, too.
\item
  liz bruenig\\
  Yeah, Trump was just doing logic in that case.
\item
  frank bruni\\
  {[}LAUGHING{]} Wait, wait, wait, wait, Liz, did you really just use
  Trump and logic in the same phrase?
\item
  liz bruenig\\
  I'm not kidding. He was sitting there. He knows nothing about
  Christianity in America. He knows nothing about the pro-life movement.
  He's not remotely interested in any of these issues. And so someone
  asks him, do you think that women who have abortions should be
  punished? So he has to come up with an answer on the spot because he
  has no relationship to this movement. And he's like, well, yeah, I
  mean, I guess. They say it's murder so yeah. {[}BRUNI LAUGHS{]} And
  that's like very basic {[}LAUGHS{]} logic he was performing there. And
  again, he did the same thing with Biden where he's like, well, damn,
  if you're not with him, you're against him. So the question is, are
  the people that Biden is competing for with Trump --- the swing people
  that Biden needs to grab --- are they really interested in hearing a
  bunch of legitimate-sounding Christian rhetoric, or are these people
  who are interested in something else? And will the swing vote come
  down to that? I am not actually convinced that religion will play that
  big of a role in the swing vote.
\item
  ross douthat\\
  I think Trump would have been better served just attacking Biden on
  the issue of abortion, where Biden actually used to be closer to the
  center and has moved more to the left in his campaign, and just made
  it an issues-based appeal because I think there are a lot of Americans
  who are conflicted about abortion who are swing voters.

  Put it this way --- I think the swing voters, Liz, think the basic
  idea of Joe Biden as a guy who goes to mass on Sunday and doesn't
  always agree with his church, that's them, too, right? So saying that
  those kind of people are somehow against God, it's just an attempt to
  mobilize people already voting for Trump, not to win over the
  undecided.
\item
  frank bruni\\
  We have to let Liz go in a moment. But before we go, Liz, I understand
  you may have a recommendation for our listeners.
\item
  liz bruenig\\
  So I would recommend logging off, which is not to say not using your
  computer or whatever. That's just a tool. But get off social media.
  You know what I mean. I think most of the people who say log off point
  to the fact that there's so much misinformation and lying on the
  internet that it can really cloud your judgment, and that's true. But
  worse is that there is so much true information on the internet. And I
  don't think we're actually particularly well-designed to cope with
  getting news of the entire world updated every 10 minutes on a live
  stream. I think it's extremely stressful. It's too much true
  information. And I think that it can lead to a real sort of spiritual
  darkness. It used to be the sole province of God to know a full
  accounting of all the human evil that was happening at any given time.
  Now it's the province of any Twitter user, and it's a heavy burden. So
  log off. I close Twitter for six hours a day at a time and have
  nothing to do with it, and I'm much happier knowing less.
\item
  frank bruni\\
  Thank you for that, Liz, and thank you so much for coming on the show.
\item
  ross douthat\\
  Yes, thank you, Liz.
\item
  liz bruenig\\
  Thanks so much for having me, guys.
\item
  frank bruni\\
  So that's our show this week. Thank you all for listening. If you have
  a question you want to hear us debate, share it with us in a voicemail
  by calling 347-915-4324. You can also email us at
  \href{mailto:argument@NYTimes.com}{\nolinkurl{argument@NYTimes.com}}.
  ``The Argument'' is a production of The New York Times opinion
  section. The team includes Vishakha Darbha, Phoebe Lett, Paula
  Szuchman, and Pedro Rafael Rosado. Special thanks to Brad Fisher and
  Kristin Lin. We'll see you next week.

  Hey Jeneen, welcome to the technical wonder of ``The Argument.''
\item
  jeneen interlandi\\
  I'm having so much fun right now. This is the most fun meeting I've
  been in all week.
\item
  frank bruni\\
  Wow.
\item
  jeneen interlandi\\
  Don't you feel sad for me?
\item
  ross douthat\\
  That's brutal. That is a brutal assessment of your week.
\end{itemize}

Previous

More episodes ofThe Argument

\href{https://www.nytimes3xbfgragh.onion/2020/09/11/opinion/the-argument-latino-2020-vote.html?action=click\&module=audio-series-bar\&region=header\&pgtype=Article}{\includegraphics{https://static01.graylady3jvrrxbe.onion/images/2020/09/12/opinion/10argumentWeb/10argumentWeb-thumbLarge-v2.jpg}}

September 11, 2020How to Win the Latino Vote

\href{https://www.nytimes3xbfgragh.onion/2020/09/03/opinion/the-argument-trump-biden-kenosha-portland.html?action=click\&module=audio-series-bar\&region=header\&pgtype=Article}{\includegraphics{https://static01.graylady3jvrrxbe.onion/images/2020/09/05/opinion/03argumentWeb/03argumentWeb-thumbLarge.jpg}}

September 3, 2020Is `American Carnage' Campaign Gold?

\href{https://www.nytimes3xbfgragh.onion/2020/08/27/opinion/the-argument-republican-convention-trump.html?action=click\&module=audio-series-bar\&region=header\&pgtype=Article}{\includegraphics{https://static01.graylady3jvrrxbe.onion/images/2020/08/28/opinion/27argument-ninetytwo1-print/27argument-ninetytwo1-thumbLarge.jpg}}

August 27, 2020Can the Republicans Sell a Whole New Trump?

\href{https://www.nytimes3xbfgragh.onion/2020/08/20/opinion/the-argument-democratic-convention-biden.html?action=click\&module=audio-series-bar\&region=header\&pgtype=Article}{\includegraphics{https://static01.graylady3jvrrxbe.onion/images/2020/08/20/opinion/20argument-ninetyone1/20argument-ninetyone1-thumbLarge.jpg}}

August 20, 2020What Biden Must Do

\href{https://www.nytimes3xbfgragh.onion/2020/08/13/opinion/the-argument-coronavirus-catholic-covid.html?action=click\&module=audio-series-bar\&region=header\&pgtype=Article}{\includegraphics{https://static01.graylady3jvrrxbe.onion/images/2020/08/13/opinion/13argument1/merlin_173532477_02e02102-92e6-4f5a-82bf-5394265f898b-thumbLarge.jpg}}

August 13, 2020Is Individualism America's Religion?

\href{https://www.nytimes3xbfgragh.onion/2020/08/06/opinion/the-argument-trump-coronavirus-election.html?action=click\&module=audio-series-bar\&region=header\&pgtype=Article}{\includegraphics{https://static01.graylady3jvrrxbe.onion/images/2020/08/06/opinion/06argSub/06argSub-thumbLarge.jpg}}

August 6, 2020Trump Supporters Make Their Case for 2020

\href{https://www.nytimes3xbfgragh.onion/2020/07/30/opinion/the-argument-authoritarianism-anne-applebaum.html?action=click\&module=audio-series-bar\&region=header\&pgtype=Article}{\includegraphics{https://static01.graylady3jvrrxbe.onion/images/2020/07/31/opinion/30argumentWeb-print/30argumentWeb-thumbLarge.jpg}}

July 30, 2020When Conservatives Fall for Demagogues

\href{https://www.nytimes3xbfgragh.onion/2020/07/23/opinion/the-argument-israel-palestinian.html?action=click\&module=audio-series-bar\&region=header\&pgtype=Article}{\includegraphics{https://static01.graylady3jvrrxbe.onion/images/2020/07/25/opinion/25audio/21argumentWeb-thumbLarge.jpg}}

July 23, 2020The Case for a One-State Solution

\href{https://www.nytimes3xbfgragh.onion/2020/07/16/opinion/the-argument-tammy-duckworth.html?action=click\&module=audio-series-bar\&region=header\&pgtype=Article}{\includegraphics{https://static01.graylady3jvrrxbe.onion/images/2020/07/17/opinion/16argumentWeb-print/16argumentWeb-thumbLarge.jpg}}

July 16, 2020A Conversation With Tammy Duckworth

\href{https://www.nytimes3xbfgragh.onion/2020/07/09/opinion/is-trumps-fate-sealed.html?action=click\&module=audio-series-bar\&region=header\&pgtype=Article}{\includegraphics{https://static01.graylady3jvrrxbe.onion/images/2020/07/10/opinion/10a2_audio/09argument1-thumbLarge.jpg}}

July 9, 2020Is Trump's Fate Sealed?

\href{https://www.nytimes3xbfgragh.onion/2020/07/02/opinion/the-argument-protest-statue-revolution.html?action=click\&module=audio-series-bar\&region=header\&pgtype=Article}{\includegraphics{https://static01.graylady3jvrrxbe.onion/images/2020/07/05/opinion/02argument-eightyfive1/02argument-eightyfive1-thumbLarge.jpg}}

July 2, 2020Whose Statue Must Fall?

\href{https://www.nytimes3xbfgragh.onion/2020/06/25/opinion/the-argument-biden-vice-president-supreme-court.html?action=click\&module=audio-series-bar\&region=header\&pgtype=Article}{\includegraphics{https://static01.graylady3jvrrxbe.onion/images/2020/06/28/opinion/25argument-eightyfour1/25argument-eightyfour1-thumbLarge.jpg}}

June 25, 2020Place Your Bets on Biden's V.P.

\href{https://www.nytimes3xbfgragh.onion/column/the-argument}{See All
Episodes ofThe Argument}

Next

Aug. 13, 2020

\begin{itemize}
\item
\item
\item
\item
\item
\end{itemize}

\emph{\textbf{Listen and subscribe to ``The Argument'' from your mobile
device:}}

******
\textbf{\href{https://itunes.apple.com/us/podcast/the-argument/id1438024613?mt=2}{\emph{Apple
Podcasts}}} \emph{\textbf{\textbar{}}}
\textbf{\href{https://open.spotify.com/show/6bmhSFLKtApYClEuSH8q42}{\emph{Spotify}}}
\emph{\textbf{\textbar{}}}
\textbf{\href{https://play.google.com/music/m/Idxib4hsg3yviao4gtym76knjjy?t=The_Argument}{\emph{Google
Play}}} \emph{\textbf{\textbar{}}}
\textbf{\href{https://radiopublic.com/the-argument-Wdbepr}{\emph{RadioPublic}}}
\emph{\textbf{\textbar{}}}
\textbf{\href{https://www.stitcher.com/podcast/the-new-york-times/the-argument}{\emph{Stitcher}}}
\emph{\textbf{\textbar{}}}
\textbf{\href{https://rss.art19.com/the-argument}{\emph{RSS Feed}}}

Five months after the editorial board's science writer Jeneen Interlandi
warned the hosts of ``The Argument'' that they should get comfortable in
quarantine, she makes her return to the podcast to talk what comes next.
Ross and Frank press Jeneen on herd immunity possibilities, how to fix
the testing lags in the U.S., and the question on every parent and
teacher's mind: How can we open schools safely?

Then, Opinion writer Elizabeth Bruenig joins Frank and Ross for a debate
on the moral obligations of the Roman Catholic Church in 2020. If the
Movement for Black Lives is promulgating Catholic beliefs, why won't the
church say ``Black Lives Matter''? And how will Joe Biden's Catholicism
play a role in the election? Finally, Elizabeth recommends a break from
omniscience.

\includegraphics{https://static01.graylady3jvrrxbe.onion/images/2020/08/13/opinion/13argument1/merlin_173532477_02e02102-92e6-4f5a-82bf-5394265f898b-articleLarge.jpg?quality=75\&auto=webp\&disable=upscale}

\begin{center}\rule{0.5\linewidth}{\linethickness}\end{center}

\textbf{Background Reading:}

\begin{itemize}
\item
  Jeneen's reporting on Harris County, Tex. for The New York Times
  Magazine,
  ``\href{https://www.nytimes3xbfgragh.onion/2020/07/14/magazine/covid-19-public-health-texas.html}{Why
  We're Losing the Battle With Covid-19}''
\item
  Elizabeth's Op-Ed,
  ``\href{https://www.nytimes3xbfgragh.onion/2020/08/06/opinion/sunday/gloria-purvis-george-floyd-blm.html}{`Racism
  Makes A Liar of God'}''
\item
  From the Editorial Board:
  ``\href{https://www.nytimes3xbfgragh.onion/2020/08/08/opinion/sunday/coronavirus-response-testing-lockdown.html}{America
  Could Control the Pandemic by October. Let's Get to It.}''
\item
  Ross's columns:
  ``\href{https://www.nytimes3xbfgragh.onion/2020/07/07/opinion/protestant-progressive-reformation.html}{The
  Religious Roots of a New Progressive Era}'' and
  ``\href{https://www.nytimes3xbfgragh.onion/2020/08/08/opinion/sunday/covid-lyme-treatment-medicine.html}{What
  to Do When Covid Doesn't Go Away}''
\end{itemize}

\begin{center}\rule{0.5\linewidth}{\linethickness}\end{center}

\textbf{How to listen to ``The Argument'':}

\emph{Press play or read the transcript (found above the center teal
eye) at the top of this page, or tune in on}
\href{https://itunes.apple.com/us/podcast/the-argument/id1438024613?mt=2}{\emph{iTunes}}\emph{,}
\href{https://play.google.com/music/listen?u=0\#/ps/Idxib4hsg3yviao4gtym76knjjy}{\emph{Google
Play}}\emph{,}
\href{https://open.spotify.com/episode/5fIsHqqunLBwoxPSUUSGre?si=Rz5D9VnlRFKdGMu8ixzBOw}{\emph{Spotify}}\emph{,}
\href{https://www.stitcher.com/podcast/the-new-york-times/the-argument}{\emph{Stitcher}}
\emph{or your preferred podcast listening app. Tell us what you think
at} \href{mailto:argument@NYTimes.com}{\emph{argument@NYTimes.com.}}

\begin{center}\rule{0.5\linewidth}{\linethickness}\end{center}

\hypertarget{meet-the-hosts}{%
\section{Meet the Hosts}\label{meet-the-hosts}}

\hypertarget{frank-bruni}{%
\subsection{Frank Bruni}\label{frank-bruni}}

Image

I've been an Op-Ed columnist for The Times since 2011, but my career
with the newspaper stretches back to 1995 and includes many twists and
turns that reflect my embarrassingly scattered interests. I covered
Congress, the White House and several political campaigns; I also spent
five years in the role of chief restaurant critic. As the Rome bureau
chief, I reported on the Vatican; as a staff writer for The Times's
Sunday magazine, I wrote many celebrity profiles. That jumble has
informed my various books, which focus on the Roman Catholic Church,
George W. Bush, my strange eating life, the college admissions process
and meatloaf. Politically, I'm grief-stricken over the way President
Trump has governed and I'm left of center, but I don't think that the
center is a bad place or ``compromise'' a dirty word. I'm
Italian-American, I'm gay and I write a
\href{https://www.nytimes3xbfgragh.onion/newsletters/frank-bruni}{weekly
Times newsletter} in which you'll occasionally encounter my dog, Regan,
who has the run of our Manhattan apartment.
\href{https://twitter.com/FrankBruni}{\emph{@FrankBruni}}

\hypertarget{ross-douthat}{%
\subsection{Ross Douthat}\label{ross-douthat}}

Image

I've been an Op-Ed columnist since 2009, and I write about politics,
religion, pop culture, sociology and the places where they intersect.
I'm a Catholic and a conservative, in that order, which means that I'm
against abortion and critical of the sexual revolution, but I tend to
agree with liberals that the Republican Party is too friendly to the
rich. I was against Donald Trump in 2016 for reasons specific to Donald
Trump, but in general I think the populist movements in Europe and
America have legitimate grievances and I often prefer the populists to
the ``reasonable'' elites. I've written books about Harvard, the G.O.P.,
American Christianity and Pope Francis, and decadence. Benedict XVI was
my favorite pope. I review movies for National Review and have strong
opinions about many prestige television shows. I have four small
children, three girls and a boy, and live in New Haven with my wife.
\href{https://twitter.com/DouthatNYT}{\emph{@DouthatNYT}}

\begin{center}\rule{0.5\linewidth}{\linethickness}\end{center}

``The Argument'' is a production of The New York Times Opinion section.
The team includes Phoebe Lett, Paula Szuchman, Pedro Rafael Rosado and
Vishakha Darbha. Special thanks to Brad Fisher and Kristin Lin. Theme by
Allison Leyton-Brown.

Advertisement

\protect\hyperlink{after-bottom}{Continue reading the main story}

\hypertarget{site-index}{%
\subsection{Site Index}\label{site-index}}

\hypertarget{site-information-navigation}{%
\subsection{Site Information
Navigation}\label{site-information-navigation}}

\begin{itemize}
\tightlist
\item
  \href{https://help.nytimes3xbfgragh.onion/hc/en-us/articles/115014792127-Copyright-notice}{©~2020~The
  New York Times Company}
\end{itemize}

\begin{itemize}
\tightlist
\item
  \href{https://www.nytco.com/}{NYTCo}
\item
  \href{https://help.nytimes3xbfgragh.onion/hc/en-us/articles/115015385887-Contact-Us}{Contact
  Us}
\item
  \href{https://www.nytco.com/careers/}{Work with us}
\item
  \href{https://nytmediakit.com/}{Advertise}
\item
  \href{http://www.tbrandstudio.com/}{T Brand Studio}
\item
  \href{https://www.nytimes3xbfgragh.onion/privacy/cookie-policy\#how-do-i-manage-trackers}{Your
  Ad Choices}
\item
  \href{https://www.nytimes3xbfgragh.onion/privacy}{Privacy}
\item
  \href{https://help.nytimes3xbfgragh.onion/hc/en-us/articles/115014893428-Terms-of-service}{Terms
  of Service}
\item
  \href{https://help.nytimes3xbfgragh.onion/hc/en-us/articles/115014893968-Terms-of-sale}{Terms
  of Sale}
\item
  \href{https://spiderbites.nytimes3xbfgragh.onion}{Site Map}
\item
  \href{https://help.nytimes3xbfgragh.onion/hc/en-us}{Help}
\item
  \href{https://www.nytimes3xbfgragh.onion/subscription?campaignId=37WXW}{Subscriptions}
\end{itemize}
