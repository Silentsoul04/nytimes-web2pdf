Sections

SEARCH

\protect\hyperlink{site-content}{Skip to
content}\protect\hyperlink{site-index}{Skip to site index}

\href{https://www.nytimes3xbfgragh.onion/section/sports}{Sports}

\href{https://myaccount.nytimes3xbfgragh.onion/auth/login?response_type=cookie\&client_id=vi}{}

\href{https://www.nytimes3xbfgragh.onion/section/todayspaper}{Today's
Paper}

\href{/section/sports}{Sports}\textbar{}Lacrosse Plays On in the
Pandemic, Creating Tough Calls for Families

\url{https://nyti.ms/2Fbl0lB}

\begin{itemize}
\item
\item
\item
\item
\item
\item
\end{itemize}

Advertisement

\protect\hyperlink{after-top}{Continue reading the main story}

Supported by

\protect\hyperlink{after-sponsor}{Continue reading the main story}

\hypertarget{lacrosse-plays-on-in-the-pandemic-creating-tough-calls-for-families}{%
\section{Lacrosse Plays On in the Pandemic, Creating Tough Calls for
Families}\label{lacrosse-plays-on-in-the-pandemic-creating-tough-calls-for-families}}

The contentious debate about whether to play or cancel summer recruiting
showcases has resembled the tense and divided national response to the
coronavirus itself.

\includegraphics{https://static01.graylady3jvrrxbe.onion/images/2020/08/16/sports/00virus-lacrosse-1/merlin_174998901_3b14362f-65a4-4662-93f1-8f1aad7ad3ae-articleLarge.jpg?quality=75\&auto=webp\&disable=upscale}

By \href{https://www.nytimes3xbfgragh.onion/by/joe-drape}{Joe Drape} and
\href{https://www.nytimes3xbfgragh.onion/by/jere-longman}{Jeré Longman}

\begin{itemize}
\item
  Aug. 13, 2020
\item
  \begin{itemize}
  \item
  \item
  \item
  \item
  \item
  \item
  \end{itemize}
\end{itemize}

FARMINGTON, Conn. --- The Middlesex Bears had waited for months to
compete in the New England Cup, an elite recruiting showcase for high
school girls' lacrosse players. Already, the coronavirus pandemic had
forced a postponement to late July from June and a change of venue. Then
a legal battle between the women's collegiate lacrosse coaching
association and the cup's organizers had threatened to cancel the
showcase at the last minute.

But here were the Bears, outside the Farmington Sports Arena, streaking
down the field, sticks high. The masks they wore between games were
wrapped around their biceps, hanging on their necks or stuffed into
their pockets. For many of the girls, this was the first game they had
played since the fall.

``When they canceled our town league in the spring, I was so
depressed,'' Grace Reilly, 16, said.

As it has with almost every sport, the pandemic has wreaked havoc on
lacrosse. Major League Lacrosse condensed its season into an eight-day
tournament in Annapolis, Md., in late July, only to lose two semifinal
teams when they withdrew after players tested positive. The Boston
Cannons won the championship, but were short five players who had
prematurely left the so-called bubble meant to reduce risks to
participants, according to
\href{https://www.capitalgazette.com/sports/ac-cs-major-league-lacrosse-bubble-20200726-3le2uhmqejfv5hqoza4hqjrcaq-story.html}{The
Capital Gazette}.

\includegraphics{https://static01.graylady3jvrrxbe.onion/images/2020/08/06/sports/00virus-lacrosse-2/merlin_174998877_ee5101a4-0c7d-4215-bd97-0e011e91bcf8-articleLarge.jpg?quality=75\&auto=webp\&disable=upscale}

But the most visible impact of the coronavirus has been on girls'
lacrosse during what is usually a fevered recruiting period. Players
have tested positive. A number of showcases and tournaments have been
canceled, postponed or reduced in size. And a national coaches
association that runs several major recruiting showcases filed a lawsuit
against the company that organizes them, arguing it was still unsafe to
play.

At the same time, the politics of the sport have come to resemble the
tense and divided national response to the pandemic itself, said
Danielle Gallagher, a member of the
\href{https://www.uslacrosse.org/about-us-lacrosse/hall-of-fame}{National
Lacrosse Hall of Fame} and the founder and director of a prominent
travel team, Long Island Liberty Lacrosse.

``Some wear masks, some don't; some think it's a farce, some don't,''
Gallagher said, speaking in general of those who participate, organize
and follow the sport.

While pro leagues like the N.B.A. and W.N.B.A. are attempting to
complete their seasons in a restricted environment, and universities and
high schools are deciding whether it is safe to play at all, club teams
and travel teams in sports like lacrosse, soccer, baseball and softball
have operated during this chaotic summer in a gray area with little
formal regulation.

In interviews, some girls' lacrosse parents said they felt pressure for
their daughters to attend showcases in order to remain visible, either
in person or on video, to college recruiters. Some coaches and parents
complained that they had received confusing, or conflicting, information
from government officials about rules regarding permissible travel and
play. But some club and travel teams also seem to have openly flouted
restrictions, and at tournaments, parents and coaches reported, there
has been uneven adherence to safety precautions and recommended social
distancing.

Image

Some lacrosse parents said they felt pressure to allow their daughters
to play.Credit...Christopher Capozziello for The New York Times

Alyssa Murray, a former all-American at Syracuse who is a director of
the Iron Horse club in Austin, Texas, recently wrote an anguished essay
on
\href{https://www.insidelacrosse.com/article/murray-a-message-from-a-confused-club-director-/56633}{Inside
Lacrosse} in which she said that ``so many youth tournaments are
pressing forward holding their events of several hundred people without
much thought of the potential risks and pressure that it will put on
players to attend.''

Twenty of the 100 summer events for girls and boys sanctioned by U.S.
Lacrosse, the sport's national governing body, were canceled. U.S.
Lacrosse also withdrew its endorsement of tournaments in Florida, Texas,
California and other states where coronavirus cases were increasing, and
\href{https://www.uslacrosse.org/return-to-play}{issued recommendations
about return-to-play protocols}. In them, officials called for masks,
social distancing and adherence to local rules, even as they
acknowledged those rules --- and public support for them --- varied
widely.

``It's all over the place because the return guidance and what you are
allowed to do in each state is so different,'' said Ann Kitt Carpenetti,
vice president for lacrosse operations at U.S. Lacrosse. ``We're trying
to balance the desires of families to go back to play with what's safe
for the kids and the community alike.''

In April, the Intercollegiate Women's Lacrosse Coaches Association,
known as the I.W.L.C.A., canceled its six recruiting tournaments for
2020, including the New England Cup. The association said it did not
feel it could hold events and sufficiently protect the health of 3,000
to 14,000 players, parents and coaches who were expected to attend each
showcase.

After a coronavirus outbreak in Louisiana had been linked to Mardi Gras
celebrations, Liz Robertshaw, the executive director of the coaches
association, said the organization felt ``we can't be the next New
Orleans.''

\hypertarget{sports-and-the-virus}{%
\subsubsection{Sports and the Virus}\label{sports-and-the-virus}}

\paragraph{}

Updated Sept. 8, 2020

Here's what's happening as the world of sports slowly comes back to
life:

\begin{itemize}
\item
  \begin{itemize}
  \tightlist
  \item
    As the United States Open enters its second week without fans, an
    Italian restaurateur stands outside the gates and
    \href{https://www.nytimes3xbfgragh.onion/2020/09/06/sports/tennis/US-Open-Matteo-Berrettini-fan.html?action=click\&pgtype=Article\&state=default\&region=MAIN_CONTENT_2\&context=storylines_keepup}{bellows
    his support}~for his favorite player.
  \item
    The coronavirus pandemic has had an
    \href{https://www.nytimes3xbfgragh.onion/2020/09/03/sports/ncaafootball/high-school-football-coronavirus-pandemic.html?action=click\&pgtype=Article\&state=default\&region=MAIN_CONTENT_2\&context=storylines_keepup}{uneven
    impact on high school football}~across the United States.
  \item
    The
    \href{https://www.nytimes3xbfgragh.onion/2020/09/02/sports/ncaafootball/coronavirus-cal-athletics-season.html?action=click\&pgtype=Article\&state=default\&region=MAIN_CONTENT_2\&context=storylines_keepup}{most
    complicated puzzle in sports is the return of college
    athletics}~during a pandemic. The University of California, Berkeley
    is allowing The Times an inside look at their journey's ups and
    downs.
  \end{itemize}
\end{itemize}

The association directed Corrigan Sports Enterprises, the company that
organized its showcases, to refund \$1,700 of the \$1,800 entry fee that
each team had paid to participate. Instead, Corrigan decided to proceed
with the showcases on its own. The I.W.L.C.A. sued, but even as the case
is being contested in federal court in North Carolina, some games,
including the ones in Connecticut, have gone ahead.

Image

Parents were asked to adhere to social distancing rules at the New
England Cup. For players, that was impossible.Credit...Christopher
Capozziello for The New York Times

``What went wrong was similar to hiring a contractor to build a
beautiful house and they want to stay in it and say it's their house,''
said Kathy Taylor, the women's lacrosse coach at Colgate, who until July
1 was the president of the I.W.L.C.A.

Lee Corrigan, the president of Corrigan Sports Enterprises, disagreed.
He said that over a decade-long partnership the coaches association had
received about \$6 million from his company, compared with the \$10,000
it took in annually before the partnership began. The coaches
association did not dispute those figures.

``We're partners,'' Corrigan said. ``They say they own everything
outright, and we're working for them. I don't think that's fair, given
that we do the majority of the work.''

When Corrigan Sports decided to proceed with the canceled showcases,
including the New England Cup, the coaches association sought a
temporary restraining order. It was denied.

As a safety precaution, Corrigan Sports requested that only one adult
per player attend its event, and temperature checks of players were
mandatory when teams arrived at the fields. Social distancing was mostly
observed on the sideline, but players routinely tangled in close
quarters on the field.

Coronavirus tests were not required; instead, each coach was responsible
for monitoring the health of a team's players. ``I feel like it's fine
with the proper precautions,'' Corrigan said.

In a normal year, the object of the summer showcases is to draw the
attention of college coaches, the first step in securing a spot on a
college team and, more important, a scholarship. But the N.C.A.A. has
forbidden coaches in Divisions I and II from making in-person contact
with potential recruits at least until Sept. 1. Thirty-one Division III
coaches had registered for the New England Cup, but only 12 checked in
shortly after the event began. To soften the blow to the participants'
investment of time and money, Corrigan provided game film free to all
players.

The New England Cup usually draws as many as 150 teams on the first
weekend in June. This year's delayed event, in late July, with the
coronavirus raging, attracted 32 teams.

Image

Summer showcases are often the best chance for high school players to
draw the eye of college coaches.Credit...Christopher Capozziello for The
New York Times

Kasie Paton, whose daughter A.J. has Type 1 diabetes, said it had been a
difficult decision to allow her to participate. In the end, she said she
felt comfortable traveling from home in Exeter, N.H., given that new
Covid-19 case reports in New England remained low and because cup
organizers seemed to have made a good-faith effort to hold a safe event.

``Our son plays baseball, and that environment is really different,''
Kasie Paton said.

Not all of the summer tournaments escaped the coronavirus. On July 12, a
girls' team from
\href{https://www.timesunion.com/news/article/Queensbury-superintendent-traveled-to-NJ-with-15421769.php}{ADK
Lacrosse} in Queensbury, N.Y., north of Albany, appeared to violate
state guidelines by traveling to play a day of scrimmages in Mount
Olive, N.J. Afterward, a 15-year-old player tested positive for Covid-19
and the rest of the team was quarantined for 14 days. The affected
player was exposed to the virus before traveling and has returned to
practice, according to her mother. Warren County officials said they
were unaware of any other positive cases.

Julia Hotmer-Drao, whose daughter tested positive after playing five
games for ADK in New Jersey, said that her daughter had been exposed to
the virus before traveling and that parents at the club did not
intentionally violate New York State restrictions.

``We were trying to get back to normal,'' Hotmer-Drao said. ``It never
crossed my mind there was anything unusual about it.''

Another 15-year-old girl, who plays for Long Island Liberty, tested
positive after attending a lacrosse camp in mid-July in Seaford, N.Y.
The player's mother, who spoke on condition of anonymity to guard her
daughter's privacy, said she was seeking to get video of her daughter at
the camp to send to college coaches.

Image

Families attending the New England Cup were asked to send only one
parent with a player.Credit...Christopher Capozziello for The New York
Times

In retrospect, the mother said that it was naïve to think the
coronavirus --- which surged in the New York metropolitan region in
March and April --- was now only a problem for other parts of the
country.

``It's still here,'' she said.

As a precaution, Gallagher, the Liberty director, said that training for
about 100 players was shut down for 10 days, and that it was recommended
that those players be tested. No additional positive tests were
reported, Gallagher said, but she said her teams would not be attending
any more showcases or tournaments this summer.

``We're trying to tell parents there's no reason to go to these events
under the current circumstances,'' Gallagher said, especially during a
so-called recruiting dead period. ``It's putting your kids in a position
that's not safe. Some people don't see it as that.''

Joe Drape reported from Farmington, Conn., and Jeré Longman from
Philadelphia. Sheelagh McNeill contributed research.

Advertisement

\protect\hyperlink{after-bottom}{Continue reading the main story}

\hypertarget{site-index}{%
\subsection{Site Index}\label{site-index}}

\hypertarget{site-information-navigation}{%
\subsection{Site Information
Navigation}\label{site-information-navigation}}

\begin{itemize}
\tightlist
\item
  \href{https://help.nytimes3xbfgragh.onion/hc/en-us/articles/115014792127-Copyright-notice}{©~2020~The
  New York Times Company}
\end{itemize}

\begin{itemize}
\tightlist
\item
  \href{https://www.nytco.com/}{NYTCo}
\item
  \href{https://help.nytimes3xbfgragh.onion/hc/en-us/articles/115015385887-Contact-Us}{Contact
  Us}
\item
  \href{https://www.nytco.com/careers/}{Work with us}
\item
  \href{https://nytmediakit.com/}{Advertise}
\item
  \href{http://www.tbrandstudio.com/}{T Brand Studio}
\item
  \href{https://www.nytimes3xbfgragh.onion/privacy/cookie-policy\#how-do-i-manage-trackers}{Your
  Ad Choices}
\item
  \href{https://www.nytimes3xbfgragh.onion/privacy}{Privacy}
\item
  \href{https://help.nytimes3xbfgragh.onion/hc/en-us/articles/115014893428-Terms-of-service}{Terms
  of Service}
\item
  \href{https://help.nytimes3xbfgragh.onion/hc/en-us/articles/115014893968-Terms-of-sale}{Terms
  of Sale}
\item
  \href{https://spiderbites.nytimes3xbfgragh.onion}{Site Map}
\item
  \href{https://help.nytimes3xbfgragh.onion/hc/en-us}{Help}
\item
  \href{https://www.nytimes3xbfgragh.onion/subscription?campaignId=37WXW}{Subscriptions}
\end{itemize}
