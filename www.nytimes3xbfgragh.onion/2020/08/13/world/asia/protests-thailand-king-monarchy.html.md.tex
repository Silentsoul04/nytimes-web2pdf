Sections

SEARCH

\protect\hyperlink{site-content}{Skip to
content}\protect\hyperlink{site-index}{Skip to site index}

\href{https://www.nytimes3xbfgragh.onion/section/world/asia}{Asia
Pacific}

\href{https://myaccount.nytimes3xbfgragh.onion/auth/login?response_type=cookie\&client_id=vi}{}

\href{https://www.nytimes3xbfgragh.onion/section/todayspaper}{Today's
Paper}

\href{/section/world/asia}{Asia Pacific}\textbar{}Protests Take On Thai
Monarchy, Despite Laws Banning Such Criticism

\url{https://nyti.ms/3iElkI5}

\begin{itemize}
\item
\item
\item
\item
\item
\end{itemize}

Advertisement

\protect\hyperlink{after-top}{Continue reading the main story}

Supported by

\protect\hyperlink{after-sponsor}{Continue reading the main story}

\hypertarget{protests-take-on-thai-monarchy-despite-laws-banning-such-criticism}{%
\section{Protests Take On Thai Monarchy, Despite Laws Banning Such
Criticism}\label{protests-take-on-thai-monarchy-despite-laws-banning-such-criticism}}

The rare questioning of Thailand's royal family comes as student-led
demonstrations have gained momentum, testing the government and the
boundaries of strict lèse-majesté laws.

\includegraphics{https://static01.graylady3jvrrxbe.onion/images/2020/08/13/world/13thailand01/merlin_175647204_5c1278c4-ca15-4b92-8239-dfcf0dd604ae-articleLarge.jpg?quality=75\&auto=webp\&disable=upscale}

\href{https://www.nytimes3xbfgragh.onion/by/hannah-beech}{\includegraphics{https://static01.graylady3jvrrxbe.onion/images/2018/10/08/multimedia/author-hannah-beech/author-hannah-beech-thumbLarge.png}}

By \href{https://www.nytimes3xbfgragh.onion/by/hannah-beech}{Hannah
Beech}

\begin{itemize}
\item
  Published Aug. 13, 2020Updated Aug. 20, 2020
\item
  \begin{itemize}
  \item
  \item
  \item
  \item
  \item
  \end{itemize}
\end{itemize}

BANGKOK --- The plainclothes men showed up late at night near Thammasat
University in Bangkok, casing out the residence where the student
activist slept. On Thursday morning, Panusaya Sithijirawattanakul went
to her sociology class, called her mother and waited for her arrest. She
was sure it was to come.

Earlier this week, Ms. Panusaya, 21, stood on a stage during an
anti-government protest at Thammasat and addressed, head on, the role of
the monarchy in a country where criticism of the institution has been
limited by strict lèse-majesté laws.

``In the past, there have been statements fooling us by saying that
people born into the royal family are incarnations of gods and angels,''
she said at the protest on Monday. ``With all due respect, please ask
yourselves, are you sure that angels or gods have this kind of
personality?''

\href{https://www.nytimes3xbfgragh.onion/2019/11/06/world/asia/thailand-king-consort-wives.html}{King
Maha Vajiralongkorn Bodindradebayavarangkun}, who spends most of his
time in Europe, returned to Thailand for his mother's birthday on
Wednesday. By Thursday, the nation's head of state was gone again, with
his fourth wife, the queen.

On the Thammasat campus, like everywhere in Thailand, stands a giant
portrait of the king, the 10th monarch of the Chakri Dynasty, dressed in
golden brocade with a somber expression on his face.

While the country's absolute monarchy was toppled by a bloodless
revolution in 1932, Thailand remains bound by royal traditions. The
father of King Maha Vajiralongkorn reigned for 70 years and was the
world's longest-serving monarch at the time of his death in 2016.

\href{https://www.nytimes3xbfgragh.onion/2020/08/20/world/asia/thailand-arrests-protests.html}{Thailand's
student-led anti-government protests}, which have gained momentum this
summer, have addressed everything from the disappearance of activists
critical of the military and monarchy, to the enduring power of a 2014
coup leader who
\href{https://www.nytimes3xbfgragh.onion/2019/06/05/world/asia/thailand-prayuth-prime-minister.html}{now
serves as prime minister}.

Over the last few days, however, they have added a new element to the
mix: direct criticism of a royal institution that, through decades of
street and student protest, tried to position itself as floating above
politics.

In an interview on Thursday, Ms. Panusaya said that Thailand's problems
were rooted in its monarchical traditions.

``I know I am taking a very high risk that I could go to jail or be
tortured or die,'' she said, ``but I don't think it's the time to be
afraid anymore.''

\includegraphics{https://static01.graylady3jvrrxbe.onion/images/2020/08/13/world/13thailand02/merlin_175647165_178a0ba2-b87f-4996-8ad2-3d2eb3517ef2-articleLarge.jpg?quality=75\&auto=webp\&disable=upscale}

At least nine activists who fled overseas since the 2014 coup have
disappeared after criticizing Thailand's most hallowed institutions. The
bodies of two of them were later found on the banks of a river, their
bellies stuffed with concrete.

Another two critics who over the past week have called for reforms to
the monarchy have been the subject of lèse-majesté complaints. The crime
can carry a jail term of up to 15 years.

Although Thailand has escaped the coronavirus pandemic largely
unscathed, its tourism-dependent economy has been battered. Protesters
have contrasted the economic hardship of millions of Thais with the
wealth of the royal family, which is one of the richest in the world. In
2017,
\href{https://www.nytimes3xbfgragh.onion/2018/06/17/world/asia/thailand-king-assets.html}{the
king took personal control} of the royal coffers, rather than let its
finances be overseen by state agencies.

``While people are starving, the monarchy is spending lavishly,'' Ms.
Panusaya said on Thursday.

Thammasat University, like Tiananmen or Tahrir, is a byword for a
democracy movement violently thwarted. In 1976, security forces and
paramilitaries stormed the university area, shooting, hanging and
beating students and other protesters. Dozens, at least, were killed.

An iconic Associated Press photograph of the massacre captured a
lifeless body hanging from a tree as a man swung a chair at the corpse.
Crowds, including children, appeared to cheer on the attacker.

The protest on Monday at Thammasat was at another campus, located on the
outskirts of Bangkok. Arnon Nampa, a human rights lawyer who was charged
with sedition last week, repeated calls he made for an open discussion
about the monarchy.

The king's powers, he said, should be limited to those of a
constitutional monarchy. Such calls, he stressed, were not meant to
overthrow the institution.

Toward the end of Monday night, Ms. Panusaya read a 10-point statement
from a student group urging reforms to the royal institution. Among the
demands were a call to end the punitive lèse-majesté law and a proposal
to trim the royal budget.

The student group also called for the monarchy to refrain from politics.
Thailand has undergone a dozen successful coups since 1932, and the
monarch has formally endorsed those changes of government.

Last year, a party critical of the junta that took power in 2014
nominated the king's elder sister as its candidate for prime minister.

The king quickly
\href{https://www.nytimes3xbfgragh.onion/2019/02/08/world/asia/thailand-prime-minister-princess.html}{quashed
his sister's political foray}. The party that nominated her was later
\href{https://www.nytimes3xbfgragh.onion/2019/03/07/world/asia/thailand-thai-raksa-chart-princess.html}{dissolved}.
Forces associated with the junta leader, Prayuth Chan-ocha, won the
elections, in a vote that international observers said was deeply
flawed. He remains as prime minister and has referred to his government
as one that represents the king.

Image

Large portraits of King Maha Vajiralongkorn, like this one outside
Thammasat University Hospital, are ubiquitous in Thailand.Credit...Adam
Dean for The New York Times

After the Thammasat protest this week, a university official said that
the student organizers had not followed an agreement on what would be
discussed at the rally.

A police spokesman said on Thursday that the student protesters were
testing the limits with their frank speech.

``To whomever is going to the protest, I believe everyone knows what can
and cannot be done,'' Col. Kissana Phathanacharoen said. ``Things that
you say will be tied to you. There will be evidence kept for the
future.''

Earlier this month, Gen. Apirat Kongsompong, the army chief, said that
while the coronavirus was a curable illness, ``hating your own country
is a disease that is not curable.''

``If being unpatriotic cannot be cured, do they deserve a similar ending
to the students at Thammasat in the 1970s?'' said Sunai Phasuk, a senior
researcher on Thailand for Human Rights Watch. ``This link is one
everyone in Thailand will make.''

On Thursday afternoon, amid torrential downpours, a brief rally took
place at Srinakharinwirot University in Bangkok amid a large police
presence. Some of the organizers said on social media that they were
allowed to proceed only if they did not mention the role of the monarchy
in their speeches.

Sirin Mungcharoen, an activist at Chulalongkorn University in Bangkok,
said that the 10-point manifesto laid out by Ms. Panusaya at Thammasat
was important because ``it opened the way for the public to be able to
criticize the monarchy.''

Intimidating those who expressed such opinions was wrong, she said, and
democratic debate was needed in Thailand.

Still, she added, the protest movement's main agenda remained ridding
the country of its military-drafted constitution, dissolving part of
Parliament and ensuring that dissidents didn't disappear.

``These three demands are what we have demanded since the very
beginning,'' she said. ``There has to be respect for human rights.''

Since the coup six years ago, thousands of people who criticized the
government have been forced to undergo sessions at ``attitude
readjustment camps'' in military compounds. A computer crimes act and
other legislation have been used to imprison others. A state of
emergency put in place because of the pandemic is being used to justify
actions against student protesters.

As evening fell on Thursday, Ms. Panusaya said she had not yet been
arrested. Seeking safety in numbers, she had holed up for the night with
other student activists. She was still waiting.

Muktita Suhartono contributed reporting.

Advertisement

\protect\hyperlink{after-bottom}{Continue reading the main story}

\hypertarget{site-index}{%
\subsection{Site Index}\label{site-index}}

\hypertarget{site-information-navigation}{%
\subsection{Site Information
Navigation}\label{site-information-navigation}}

\begin{itemize}
\tightlist
\item
  \href{https://help.nytimes3xbfgragh.onion/hc/en-us/articles/115014792127-Copyright-notice}{©~2020~The
  New York Times Company}
\end{itemize}

\begin{itemize}
\tightlist
\item
  \href{https://www.nytco.com/}{NYTCo}
\item
  \href{https://help.nytimes3xbfgragh.onion/hc/en-us/articles/115015385887-Contact-Us}{Contact
  Us}
\item
  \href{https://www.nytco.com/careers/}{Work with us}
\item
  \href{https://nytmediakit.com/}{Advertise}
\item
  \href{http://www.tbrandstudio.com/}{T Brand Studio}
\item
  \href{https://www.nytimes3xbfgragh.onion/privacy/cookie-policy\#how-do-i-manage-trackers}{Your
  Ad Choices}
\item
  \href{https://www.nytimes3xbfgragh.onion/privacy}{Privacy}
\item
  \href{https://help.nytimes3xbfgragh.onion/hc/en-us/articles/115014893428-Terms-of-service}{Terms
  of Service}
\item
  \href{https://help.nytimes3xbfgragh.onion/hc/en-us/articles/115014893968-Terms-of-sale}{Terms
  of Sale}
\item
  \href{https://spiderbites.nytimes3xbfgragh.onion}{Site Map}
\item
  \href{https://help.nytimes3xbfgragh.onion/hc/en-us}{Help}
\item
  \href{https://www.nytimes3xbfgragh.onion/subscription?campaignId=37WXW}{Subscriptions}
\end{itemize}
