Sections

SEARCH

\protect\hyperlink{site-content}{Skip to
content}\protect\hyperlink{site-index}{Skip to site index}

\href{https://www.nytimes3xbfgragh.onion/section/nyregion}{New York}

\href{https://myaccount.nytimes3xbfgragh.onion/auth/login?response_type=cookie\&client_id=vi}{}

\href{https://www.nytimes3xbfgragh.onion/section/todayspaper}{Today's
Paper}

\href{/section/nyregion}{New York}\textbar{}N.Y.C. Schools Are Set to
Open in a Month. Principals Fear That's Too Soon.

\url{https://nyti.ms/3fXB5YK}

\begin{itemize}
\item
\item
\item
\item
\item
\item
\end{itemize}

\hypertarget{school-reopenings}{%
\subsubsection{\texorpdfstring{\href{https://www.nytimes3xbfgragh.onion/spotlight/schools-reopening?name=styln-coronavirus-schools-reopening\&region=TOP_BANNER\&block=storyline_menu_recirc\&action=click\&pgtype=Article\&impression_id=ce6b7fc0-f292-11ea-8bc2-93f279d4169d\&variant=undefined}{School
Reopenings}}{School Reopenings}}\label{school-reopenings}}

\begin{itemize}
\tightlist
\item
  \href{https://www.nytimes3xbfgragh.onion/2020/09/04/us/bar-exam-coronavirus.html?name=styln-coronavirus-schools-reopening\&region=TOP_BANNER\&block=storyline_menu_recirc\&action=click\&pgtype=Article\&impression_id=ce6b7fc1-f292-11ea-8bc2-93f279d4169d\&variant=undefined}{Delayed
  Licensing Exams}
\item
  \href{https://www.nytimes3xbfgragh.onion/2020/09/08/upshot/children-testing-shortfalls-virus.html?name=styln-coronavirus-schools-reopening\&region=TOP_BANNER\&block=storyline_menu_recirc\&action=click\&pgtype=Article\&impression_id=ce6b7fc2-f292-11ea-8bc2-93f279d4169d\&variant=undefined}{Limited
  Testing for Children}
\item
  \href{https://www.nytimes3xbfgragh.onion/2020/09/01/world/schools-reopen-globe-students.html?name=styln-coronavirus-schools-reopening\&region=TOP_BANNER\&block=storyline_menu_recirc\&action=click\&pgtype=Article\&impression_id=ce6ba6d0-f292-11ea-8bc2-93f279d4169d\&variant=undefined}{School
  Around the World}
\item
  \href{https://www.nytimes3xbfgragh.onion/interactive/2020/us/covid-college-cases-tracker.html?name=styln-coronavirus-schools-reopening\&region=TOP_BANNER\&block=storyline_menu_recirc\&action=click\&pgtype=Article\&impression_id=ce6ba6d1-f292-11ea-8bc2-93f279d4169d\&variant=undefined}{Tracking
  College Cases}
\end{itemize}

Advertisement

\protect\hyperlink{after-top}{Continue reading the main story}

Supported by

\protect\hyperlink{after-sponsor}{Continue reading the main story}

\hypertarget{nyc-schools-are-set-to-open-in-a-month-principals-fear-thats-too-soon}{%
\section{N.Y.C. Schools Are Set to Open in a Month. Principals Fear
That's Too
Soon.}\label{nyc-schools-are-set-to-open-in-a-month-principals-fear-thats-too-soon}}

School leaders are running up against major obstacles and unanswerable
questions ahead of the planned reopening. A large group of principals
has called on the mayor to delay in-person instruction.

\includegraphics{https://static01.graylady3jvrrxbe.onion/images/2020/08/11/nyregion/00nyvirus-principals/00nyvirus-principals-articleLarge.jpg?quality=75\&auto=webp\&disable=upscale}

\href{https://www.nytimes3xbfgragh.onion/by/eliza-shapiro}{\includegraphics{https://static01.graylady3jvrrxbe.onion/images/2018/12/28/multimedia/author-eliza-shapiro/author-eliza-shapiro-thumbLarge.png}}

By \href{https://www.nytimes3xbfgragh.onion/by/eliza-shapiro}{Eliza
Shapiro}

\begin{itemize}
\item
  Published Aug. 14, 2020Updated Sept. 1, 2020
\item
  \begin{itemize}
  \item
  \item
  \item
  \item
  \item
  \item
  \end{itemize}
\end{itemize}

Alexa Sorden, an elementary school principal in the Bronx, has been
pacing through the empty hallways of her school, trying to figure out
how to make
\href{https://www.nytimes3xbfgragh.onion/2020/08/05/nyregion/nyc-schools-reopening.html}{the
most consequential school reopening effort in the country} actually
work.

Along with about 1,700 other school principals in
\href{https://www.nytimes3xbfgragh.onion/2020/09/01/nyregion/schools-open-coronavirus-nyc.html}{New
York City}, Ms. Sorden has spent the summer racing to complete a
dizzying set of tasks: calculating how many students she can safely
allow in her building while adhering to social distancing; creating a
curriculum and schedule to accommodate children in-person and online;
and keeping a steady line of communication with nervous staff and
traumatized parents.

Her efforts will help to determine whether the nation's largest school
district
\href{https://www.nytimes3xbfgragh.onion/2020/08/07/nyregion/cuomo-schools-reopening.html}{can
reopen} this fall. New York City, where the virus is currently under
control, is the only major school district in the U.S. still planning to
welcome children into classrooms part time this fall.

By trying to execute the high-stakes reopening plan, New York City's
principals have been quietly shaping what pandemic-era schooling could
look like, not just for the city's 1.1 million students but for children
nationwide. New York's reopening plans are being closely watched by
politicians and school superintendents around the country.

Now, with just a month left until the city's
\href{https://www.nytimes3xbfgragh.onion/2020/08/18/nyregion/schools-reopen-nyc.html}{schools
are scheduled to reopen}, some principals have begun raising alarms
about the system's readiness. Though they are typically wary of wading
into politics, a large group of principals made an exception this week,
\href{https://twitter.com/MOREcaucusUFT/status/1293258831041110017}{calling
on the mayor to delay in-person instruction} by a few weeks and then
phase students back into buildings throughout the fall.

Their resistance could have significant consequences for New York City's
school reopening proposal; the city's powerful teachers' union has
already said it does not believe it is currently safe to reopen schools.

The principals' union sent a letter to Mayor Bill de Blasio on Wednesday
on behalf of school leaders, urging the mayor to ``heed their dire
warnings'' about opening too soon.

Ms. Sorden, who founded and now leads Concourse Village Elementary
School, said she does not believe her school will be ready to welcome
children on the planned first day of school.

``We know that Sept. 10 is not even close to realistic,'' she said.

Ms. Sorden's concerns signal how difficult it will be to bring children
back into classrooms even in places where the virus is contained.

\includegraphics{https://static01.graylady3jvrrxbe.onion/images/2020/08/16/nyregion/16nyvirus-principals-print/00nyvirus-principals-04-articleLarge.jpg?quality=75\&auto=webp\&disable=upscale}

``I am afraid. I am nervous,'' she said on a recent weekday morning,
after she plucked a few Lysol wipes and disinfected the arms of every
chair in her office. ``I wish I had all the answers,'' she added. ``This
is a new thing for me, operating without having every answer.''

Ms. Sorden is sorting through the weightiest questions she has ever
encountered in her two decades as an educator.

How can she create two complementary versions of school --- one online,
one in-person --- that both prevent her mostly low-income, Black and
Latino students from falling further behind academically, and keep those
children and their families safe? Ms. Sorden estimates that about 60
percent of the school's roughly 330 families will opt for full-time
remote learning, based on survey results and conversations with parents.

It is easy to get overwhelmed, so Ms. Sorden tries to quiet her mind
with daily morning meditation and writing in her journal. She also
confides in her husband after they have put their three children --- two
of whom are students at her school --- to bed. She monitors her heart
rate frequently on her Apple Watch.

On mornings when it isn't raining, Ms. Sorden ties on her sneakers and
walks a mile and a half from her home in Washington Heights to Concourse
Village Elementary in the South Bronx.

When she arrives at her school building one morning, Ms. Sorden uses a
ruler to measure six feet of distance between every desk and chair. She
stands in each corner of a classroom that once held 30 students and will
now accommodate nine, and wonders if she can position each child so that
they are breathing in different directions.

So far, that has been the easy part.

Ms. Sorden's teachers have asked her what to do if children in pre-K
switch masks, and whether it is safer for them to wear masks, face
shields, or both. They do not know if it will be safe to take off their
masks to eat lunch.

A handful of teachers have left the school over the last few weeks ---
either because they left the city altogether, were concerned about
returning to classrooms, or wanted to work closer to home. Some teachers
have asked whether they should write their wills.

Ms. Sorden barely has time to think about who will watch her own
children on the days when she's at school, but they are learning from
home.

Image

``I do cry a lot, I have cried a lot,'' Ms. Sorden said of this summer,
as she has scrambled to prepare her school for students.Credit...Chang
W. Lee/The New York Times

Ms. Sorden's determination to make schools safe, and her anguish over
how difficult that is proving to be, is shared by school leaders
throughout New York.

Moses Ojeda is the principal of Thomas A. Edison Career and Technical
Education High School in Jamaica, Queens, which prepares students for
careers as electricians, medical assistants and automotive technicians.
Mr. Ojeda is spending his days puzzling through the extraordinary
challenge of how to provide hands-on instruction online.

\href{https://www.nytimes3xbfgragh.onion/spotlight/schools-reopening?action=click\&pgtype=Article\&state=default\&region=MAIN_CONTENT_3\&context=storylines_keepup}{}

\hypertarget{school-reopenings-}{%
\subsubsection{School Reopenings ›}\label{school-reopenings-}}

\hypertarget{back-to-school}{%
\paragraph{Back to School}\label{back-to-school}}

Updated Sept. 8, 2020

The latest on how schools are reopening amid the pandemic.

\begin{itemize}
\item
  \begin{itemize}
  \tightlist
  \item
    The first day of school is an annual rite of passage. But this year,
    it looks very different for tens of millions of students.
    \href{https://www.nytimes3xbfgragh.onion/2020/09/05/us/virtual-return-to-school-covid.html?action=click\&pgtype=Article\&state=default\&region=MAIN_CONTENT_3\&context=storylines_keepup}{We
    talked to some about their hopes and fears}.
  \item
    Coronavirus cases
    \href{https://www.nytimes3xbfgragh.onion/2020/09/06/us/colleges-coronavirus-students.html?action=click\&pgtype=Article\&state=default\&region=MAIN_CONTENT_3\&context=storylines_keepup}{are
    spiking in America's college towns}, leading to concern that young
    people who are infected will contribute to a spread of the virus.
  \item
    A growing number of Catholic schools across the country are
    \href{https://www.nytimes3xbfgragh.onion/2020/09/05/us/catholic-school-closings.html?action=click\&pgtype=Article\&state=default\&region=MAIN_CONTENT_3\&context=storylines_keepup}{shutting
    down forever during the coronavirus pandemic}, citing insurmountable
    financial pressure.
  \item
    The magazine's Ethicist columnist answers a question from a
    spokesperson at a major university:
    \href{https://www.nytimes3xbfgragh.onion/2020/09/08/magazine/university-reopening-safety-ethics.html?action=click\&pgtype=Article\&state=default\&region=MAIN_CONTENT_3\&context=storylines_keepup}{Can
    I promote a reopening plan I have doubts about}?
  \end{itemize}
\end{itemize}

``I don't want to see gaps in our work force,'' he said.

José Jiménez, the principal of Public School 290 in Ridgewood, Queens,
has been fielding questions from parents about safety for weeks: What
kinds of air filters will be installed? Who will be completing the
nightly deep cleans? Who will be checking temperatures, and how often?

When he talks to parents and teachers, Mr. Jimenez said, ``I have to
preface everything with, `Whatever I say now, it could change
tomorrow.''' Roughly 25 percent of his families have already opted for
all-remote learning, he said.

And Leander Windley, the leader of Intermediate School 318 in the
Williamsburg section of Brooklyn, is wrestling with how to welcome his
rising sixth graders into a school they have likely never stepped foot
in. He has been holding virtual meetings in the evenings with their
parents, who come from about 20 different elementary schools.

He is preparing his guidance counselor and social worker to brace for
what he believes will be a uniquely traumatized group of students who
will need more support than ever before, ideally delivered in-person.

But he is worried about the city's ability to tie up all the safety
concerns by September. ``I'm more of a realist than an optimist,'' he
said.

Image

``Everything should be normal'' for the school's incoming kindergarten
students, said teacher Lissette Rosario as she finished decorating a
bulletin board.Credit...Chang W. Lee/The New York Times

All four principals' schools are in largely low-income neighborhoods
where most parents have had little choice but to physically return to
work, and would
\href{https://www.nytimes3xbfgragh.onion/2020/07/10/nyregion/nyc-school-daycare-reopening.html}{likely
have to seek child care options outside the home} if classrooms do not
reopen.

The principals all said they were worried both about how their most
vulnerable students would fare without at least some in-person
instruction, and about how to keep their families safe.

``School, for some of our neediest children, is everything,'' Ms. Sorden
said. ``It's the only place where they know what to expect.''

And yet: many of Ms. Sorden's students live with their grandparents or
relatives with pre-existing conditions. And it was impossible for her
not to see safety minefields everywhere she looked as she surveyed the
building.

Some staircases were likely too cramped to accommodate more than a few
students at a time. Ms. Sorden pointed out that her schoolyard was
hidden under construction scaffolding, rendering outdoor learning all
but impossible. And she wondered how many students would be able to line
up in a hallway that is only six feet wide.

But Ms. Sorden found a brief respite from her worries when she turned a
corner and ran into Lissette Rosario, a kindergarten teacher who had
been working for hours on assembling a fresh bulletin board, complete
with pictures of dinosaurs.

Ms. Sorden did not know whether students would ever see the board in
person. But for a moment, it gave her joy to think about children
walking through the hallways sometime soon.

``When they're here we want them to know that every detail matters, and
that they're safe and that we love them,'' she said. ``And the virus is
definitely interfering with what we love.''

Advertisement

\protect\hyperlink{after-bottom}{Continue reading the main story}

\hypertarget{site-index}{%
\subsection{Site Index}\label{site-index}}

\hypertarget{site-information-navigation}{%
\subsection{Site Information
Navigation}\label{site-information-navigation}}

\begin{itemize}
\tightlist
\item
  \href{https://help.nytimes3xbfgragh.onion/hc/en-us/articles/115014792127-Copyright-notice}{©~2020~The
  New York Times Company}
\end{itemize}

\begin{itemize}
\tightlist
\item
  \href{https://www.nytco.com/}{NYTCo}
\item
  \href{https://help.nytimes3xbfgragh.onion/hc/en-us/articles/115015385887-Contact-Us}{Contact
  Us}
\item
  \href{https://www.nytco.com/careers/}{Work with us}
\item
  \href{https://nytmediakit.com/}{Advertise}
\item
  \href{http://www.tbrandstudio.com/}{T Brand Studio}
\item
  \href{https://www.nytimes3xbfgragh.onion/privacy/cookie-policy\#how-do-i-manage-trackers}{Your
  Ad Choices}
\item
  \href{https://www.nytimes3xbfgragh.onion/privacy}{Privacy}
\item
  \href{https://help.nytimes3xbfgragh.onion/hc/en-us/articles/115014893428-Terms-of-service}{Terms
  of Service}
\item
  \href{https://help.nytimes3xbfgragh.onion/hc/en-us/articles/115014893968-Terms-of-sale}{Terms
  of Sale}
\item
  \href{https://spiderbites.nytimes3xbfgragh.onion}{Site Map}
\item
  \href{https://help.nytimes3xbfgragh.onion/hc/en-us}{Help}
\item
  \href{https://www.nytimes3xbfgragh.onion/subscription?campaignId=37WXW}{Subscriptions}
\end{itemize}
