Sections

SEARCH

\protect\hyperlink{site-content}{Skip to
content}\protect\hyperlink{site-index}{Skip to site index}

\href{https://myaccount.nytimes3xbfgragh.onion/auth/login?response_type=cookie\&client_id=vi}{}

\href{https://www.nytimes3xbfgragh.onion/section/todayspaper}{Today's
Paper}

\href{/section/opinion}{Opinion}\textbar{}Treasure and Turmoil in the
Deep Sea

\url{https://nyti.ms/2E2xqff}

\begin{itemize}
\item
\item
\item
\item
\item
\end{itemize}

Advertisement

\protect\hyperlink{after-top}{Continue reading the main story}

\href{/section/opinion}{Opinion}

Supported by

\protect\hyperlink{after-sponsor}{Continue reading the main story}

\hypertarget{treasure-and-turmoil-in-the-deep-sea}{%
\section{Treasure and Turmoil in the Deep
Sea}\label{treasure-and-turmoil-in-the-deep-sea}}

The growing push to mine the seabed threatens the vast and rich
ecosystem between the surface and the seafloor.

By Steven H.D. Haddock and C. Anela Choy

Dr. Haddock is a senior scientist at the Monterey Bay Aquarium Research
Institute. Dr. Choy is an assistant professor at the Scripps Institution
of Oceanography.

\begin{itemize}
\item
  Aug. 14, 2020
\item
  \begin{itemize}
  \item
  \item
  \item
  \item
  \item
  \end{itemize}
\end{itemize}

\includegraphics{https://static01.graylady3jvrrxbe.onion/images/2020/08/14/opinion/14haddock/14haddock-articleLarge.jpg?quality=75\&auto=webp\&disable=upscale}

There is treasure in the sea, and much of it lies in plain view on the
deep ocean floor. Fields of metallic nodules and towering hydrothermal
chimneys accumulate precious and industrially prized metals, estimated
to be worth many billions to even trillions of dollars.

It is there for the taking, and mining operations around the world are
exploring strategies to plunder this treasure, asserting that mining in
the deep sea is more sustainable and less harmful than doing so on land.
Mining explorations covering more than 500,000 square miles have been
approved by the \href{https://www.isa.org.jm/}{International Seabed
Authority}, which regulates mining in international waters. That's
roughly twice the size of Texas.

But the deep sea is not a barren, lifeless wasteland, as once thought.
Exploration in recent decades has revealed thousand-year-old corals,
microbes that can treat cancer and infectious diseases, and hydrothermal
vent fields of monumental proportions, from which living creatures
convert sulfur and methane into energy, offering a glimpse at the
origins of life on earth. So \href{https://rdcu.be/b5r3o}{the challenge}
is not simply finding the X on the treasure map, but bringing the
materials to the surface in a way that inflicts minimal damage to the
ocean environment.

As deep-sea biologists who study the drifting and swimming inhabitants
of the ocean, we originally felt that any resulting harms from deep-sea
mining would primarily be felt at the bottom of the sea. Nearly all of
the environmental impact studies on deep-sea mining have focused on the
seafloor, where the aftermath is visible for decades. Seabeds are still
denuded 30 years after experimental mining machines passed by.

But as we and our colleagues
\href{https://www.pnas.org/content/117/30/17455}{noted recently} in the
journal Proceedings of the National Academy of Sciences, mining will
have pronounced and debilitating impacts that will be felt not just on
the seabed but also throughout the deep water column, which extends from
about 600 feet below the surface to the seafloor, where the extraction
takes place.

Minerals that miners seek form and accumulate extremely slowly in the
deep ocean, with growth rates of only a few millimeters per million
years. A nodule the size of a tennis ball lying on the sea floor and
consisting largely of prized rare-earth metals could be more than 14
million years old. While the ecosystem might recover to some extent
after a hundred years, the mineral resources will never be replaced.
Mining serves present-day consumers but leaves the environmental
consequences for their children and grandchildren.

Critically, what is missed in assessments of mining impacts is the
effect on the ocean itself. The sea is not just the seafloor alone, but
also what lies above it: roughly 13,000 feet of water on average, more
than twice as deep as the deepest point of the Grand Canyon and
including more than 90 percent of the planet's life-sustaining habitats.
This deep mid-water ecosystem --- from microbes and worms to jellies and
giant squid --- is important and is linked to us in many ways.

When a nodule is
\href{https://www.usgs.gov/media/images/deep-sea-mining-machines}{gouged}
and vacuumed from the seafloor, it is pumped to a surface ship through a
pipeline. The minerals are removed, and then the muddy, silty,
toxin-enriched fluid is pumped back into the sea as what is called a
``dewatering plume.'' Heavier particles will sink to the seafloor but
must pass through thousands of feet of intervening water before
settling. Additionally, the fine silt will drift and flow for miles and
months in the ocean currents. It is frightfully clear that the impact of
this drifting plume on open-water ecosystems will be severe, varied and
global in scale.

Decades of deep-sea science have taught us that organisms in the deep
water have adaptations that make them especially susceptible to these
mining impacts. Many of them feed on small particles that flake down
from the surface, as if in a giant snow globe. These filter-feeders are
not limited to worms and snails, but also include the vampire squid and
30-foot-long gelatinous chains called
\href{https://www.whoi.edu/know-your-ocean/ocean-topics/polar-research/polar-life/the-watery-world-of-salps/}{salps}.
This process of consuming particles contributes to the flow of carbon
from the atmosphere to deep sediments in the ocean, helping regulate the
earth's climate.

We've seen that the food web is
\href{https://www.nytimes3xbfgragh.onion/2017/12/19/science/deep-sea-food-web.html}{complex
and interconnected}, linking ultimately to commercial fisheries worth
billions of dollars. Any toxins in the environment or diet of these fish
will end up on our
\href{https://www.theatlantic.com/science/archive/2019/08/why-changing-climate-means-more-mercury-seafood/595663/}{dinner
plates}. Amazingly, about three-quarters of the animals in the water
column can
\href{https://www.nytimes3xbfgragh.onion/interactive/2017/08/21/science/the-deep-seas-are-alive-with-light.html}{make
their own light}, and they use this bioluminescence to find prey and
mates, while avoiding predators by using glowing camouflage as a
cloaking device.

As a result of the mining, animals already living
\href{https://www.scientificamerican.com/article/the-ocean-is-running-out-of-breath-scientists-warn/}{near
their physiological limits} would be eating mouthfuls of poisonous dirt
for breakfast, respiring through clogged gills and squinting through a
muddy haze to communicate.

Based on
\href{https://www.researchgate.net/publication/267704135_Anthropogenic_impacts_on_the_deep_sea}{predicted
discharge rates}, a single mining ship will release between two million
and 3.5 million cubic feet of effluent every day, enough to fill a fleet
of tanker trucks 15 miles long. Now imagine this process running
continuously for 30 years --- the lifetime of a mining lease. Most
important, these sediment plumes will not respect the neat boundaries
defined by a permit. Regulatory buffer zones set up around the Cook
Islands, for example, extend only 50 nautical miles --- insufficient to
protect their reefs, fisheries and tourism from these expanding
sediments, which are projected to travel hundreds of miles.

The companies and governing agencies that stand to profit from mining
activities are based in the United States, Canada, Europe and Asia. They
are geographically, politically and economically removed from the small
island nations that will bear the brunt of the consequences. While
government leaders may welcome mining for economic gain, it is the
Indigenous people and local communities on these islands who are often
\href{https://www.theguardian.com/world/2019/oct/20/cook-islands-manager-of-worlds-biggest-marine-park-says-she-lost-job-for-backing-sea-mining-moratorium}{without
a meaningful voice} in decisions that will weigh heavily on their
future. In the United States, which is not a member of the nearly
170-nation seabed authority, the Trump administration is exploring
whether it can open portions of existing national marine sanctuaries to
\href{https://www.washingtonpost.com/news/energy-environment/wp/2017/04/28/trump-signs-executive-order-to-expand-offshore-drilling-and-analyze-marine-sanctuaries-oil-and-gas-potential/}{mineral
extraction}.

Most deep-sea mining plans predict plume discharges to be located around
3,300 feet down, even when mining operations are taking place on a
seabed more than 16,000 feet deep. This may be out of sight from the
surface, but it is not deep enough to avoid potentially disastrous
effects on deep-ocean ecosystems and food webs. When mining operations
commence, companies must shoulder the additional expense of depositing
their effluent as close to the original seafloor disturbance as
possible. Doing so will minimize harmful effects of both the sinking and
drifting plumes on water-column life and reduce their spread to nearby
ecosystems.

Historically the deep sea has been considered remote and largely devoid
of life, and to have an inexhaustible capacity to absorb our pollution.
In reality, these deep water ecosystems are fragile, diverse and
connected to us. Mining operations must reduce their impact on the whole
of the ocean and not just the seafloor. The dazzling treasure of oceanic
biodiversity has unfathomable value as well.

\begin{center}\rule{0.5\linewidth}{\linethickness}\end{center}

\href{https://www.mbari.org/haddock-steven/}{Steven H.D. Haddock} is a
senior scientist at the Monterey Bay Aquarium Research Institute.
\href{https://choylab.ucsd.edu/\#people}{C. Anela Choy} is an assistant
professor at the Scripps Institution of Oceanography.

\emph{The Times is committed to publishing}
\href{https://www.nytimes3xbfgragh.onion/2019/01/31/opinion/letters/letters-to-editor-new-york-times-women.html}{\emph{a
diversity of letters}} \emph{to the editor. We'd like to hear what you
think about this or any of our articles. Here are some}
\href{https://help.nytimes3xbfgragh.onion/hc/en-us/articles/115014925288-How-to-submit-a-letter-to-the-editor}{\emph{tips}}\emph{.
And here's our email:}
\href{mailto:letters@NYTimes.com}{\emph{letters@NYTimes.com}}\emph{.}

\emph{Follow The New York Times Opinion section on}
\href{https://www.facebookcorewwwi.onion/nytopinion}{\emph{Facebook}}\emph{,}
\href{http://twitter.com/NYTOpinion}{\emph{Twitter (@NYTopinion)}}
\emph{and}
\href{https://www.instagram.com/nytopinion/}{\emph{Instagram}}\emph{.}

Advertisement

\protect\hyperlink{after-bottom}{Continue reading the main story}

\hypertarget{site-index}{%
\subsection{Site Index}\label{site-index}}

\hypertarget{site-information-navigation}{%
\subsection{Site Information
Navigation}\label{site-information-navigation}}

\begin{itemize}
\tightlist
\item
  \href{https://help.nytimes3xbfgragh.onion/hc/en-us/articles/115014792127-Copyright-notice}{©~2020~The
  New York Times Company}
\end{itemize}

\begin{itemize}
\tightlist
\item
  \href{https://www.nytco.com/}{NYTCo}
\item
  \href{https://help.nytimes3xbfgragh.onion/hc/en-us/articles/115015385887-Contact-Us}{Contact
  Us}
\item
  \href{https://www.nytco.com/careers/}{Work with us}
\item
  \href{https://nytmediakit.com/}{Advertise}
\item
  \href{http://www.tbrandstudio.com/}{T Brand Studio}
\item
  \href{https://www.nytimes3xbfgragh.onion/privacy/cookie-policy\#how-do-i-manage-trackers}{Your
  Ad Choices}
\item
  \href{https://www.nytimes3xbfgragh.onion/privacy}{Privacy}
\item
  \href{https://help.nytimes3xbfgragh.onion/hc/en-us/articles/115014893428-Terms-of-service}{Terms
  of Service}
\item
  \href{https://help.nytimes3xbfgragh.onion/hc/en-us/articles/115014893968-Terms-of-sale}{Terms
  of Sale}
\item
  \href{https://spiderbites.nytimes3xbfgragh.onion}{Site Map}
\item
  \href{https://help.nytimes3xbfgragh.onion/hc/en-us}{Help}
\item
  \href{https://www.nytimes3xbfgragh.onion/subscription?campaignId=37WXW}{Subscriptions}
\end{itemize}
