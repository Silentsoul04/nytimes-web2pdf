Sections

SEARCH

\protect\hyperlink{site-content}{Skip to
content}\protect\hyperlink{site-index}{Skip to site index}

\href{https://myaccount.nytimes3xbfgragh.onion/auth/login?response_type=cookie\&client_id=vi}{}

\href{https://www.nytimes3xbfgragh.onion/section/todayspaper}{Today's
Paper}

\href{/section/opinion}{Opinion}\textbar{}The Unlikely Triumph of
Italian Nationhood

\url{https://nyti.ms/2DTp4GZ}

\begin{itemize}
\item
\item
\item
\item
\item
\item
\end{itemize}

Advertisement

\protect\hyperlink{after-top}{Continue reading the main story}

\href{/section/opinion}{Opinion}

Supported by

\protect\hyperlink{after-sponsor}{Continue reading the main story}

\hypertarget{the-unlikely-triumph-of-italian-nationhood}{%
\section{The Unlikely Triumph of Italian
Nationhood}\label{the-unlikely-triumph-of-italian-nationhood}}

Italy coheres as America breaks apart.

\href{https://www.nytimes3xbfgragh.onion/by/roger-cohen}{\includegraphics{https://static01.graylady3jvrrxbe.onion/images/2014/11/01/opinion/cohen-circular/cohen-circular-thumbLarge-v6.png}}

By \href{https://www.nytimes3xbfgragh.onion/by/roger-cohen}{Roger Cohen}

Opinion Columnist

\begin{itemize}
\item
  Aug. 14, 2020
\item
  \begin{itemize}
  \item
  \item
  \item
  \item
  \item
  \item
  \end{itemize}
\end{itemize}

\includegraphics{https://static01.graylady3jvrrxbe.onion/images/2020/08/17/opinion/17cohen1/merlin_175212624_5b01afb5-c6b5-469d-a150-fb4551fb3a62-articleLarge.jpg?quality=75\&auto=webp\&disable=upscale}

SANTA MARGHERITA, Italy --- In Italy there are seasons --- and then
there is \emph{the} season. Summer comes, the country's woes are set
aside, and, to the eternal refrain of ``tutti al mare'' ** (``Everyone
to the sea''), the exodus to the consoling coast begins. The national
debt fades between sea and stars.

This year is a little different. Masks dangle from ears in the new
insouciant look or are tied around elbows then used for a bump-greeting.
Beach chairs are (anti-) socially distanced. With stores operating
two-at-a-time limits, the line for focaccia is so long that people read
an entire newspaper (and Italians \emph{still} read them) as they wait.
Americans have almost vanished, as have the Russians. Children's beach
chatter revolves around the all-canceling virus*,* which gives a new
edge to games of tag.

The difference is the coronavirus, lurking like the knowledge that
summer will end. Contained, almost defeated, yet out there beyond the
drone of the chirping cicadas, leaving Italians in a limbo between
liberation and fear.

Among Western nations, Italy was the first to be hit hard by the
pandemic. The country learned the
\href{https://www.nytimes3xbfgragh.onion/interactive/2020/03/27/world/europe/coronavirus-italy-bergamo.html}{loneliness
of a new form of death}. Its doctors battled in extremis. It watched
army trucks transporting coffins to remote cremation sites from the
overloaded morgues of Bergamo.

Then, strange thing, after some initial missteps, Italy did what it has
had the most difficulty doing since the unification of the peninsula in
1861: It cohered into a nation and brought a fierce national will to
bear on the virus. It went into disciplined lockdown. It set aside,
through a unified front, the old slurs exchanged between northerners and
southerners, the old parochialism of city-states with longer histories
than the nation they find themselves in, the old derision directed at
its politics.

I am tempted to say that 2020 was the year of Italy's emergence, 159
years after the Piedmont statesman Massimo d'Azeglio declared: ``We have
made Italy. Now we have to make Italians.'' Perhaps that's an
exaggeration, but not without its truth.

Italy brought its rate of new infections --- now
\href{https://www.nytimes3xbfgragh.onion/interactive/2020/world/europe/italy-coronavirus-cases.html}{about
eight per 100,000 inhabitants} --- down to one of the lowest in Europe,
lower even than
\href{https://www.nytimes3xbfgragh.onion/interactive/2020/world/europe/germany-coronavirus-cases.html}{Germany}.
It did so as the United States, which spent untold postwar treasure on
keeping Italy stable, threw its doors open to the pandemic through
leaderless fracture. This, in contrast to Italy, has been the season of
American unraveling.

I mentioned the buzzing cicadas and their summer crescendo. In Aesop's
fable generally known in English as ``The Ant and the Grasshopper,'' but
in Italian as ``La Formica e la Cicala'' (``The Ant and the Cicada''),
the industrious ant spends its summer laying in supplies for the winter
while the carefree cicada passes the time singing, or, as Italians
describe improvident laziness, scratching its belly. When winter comes,
as it does, the starving cicada begs the ant for food. The ant,
vindicated, tells it to go dance away the winter.

Writing the other day in Milan's Corriere della Sera,
\href{https://www.corriere.it/editoriali/20_agosto_11/alla-ricerca-normalitatra-pericolosi-cicaleggi-estivi-99a9717c-dc06-11ea-abc9-41b5baff53c0.shtml}{Antonio
Scurati asked}: ``Dear reader, are you a cicada or an ant?'' His fear,
he said, was that Italians were tending cicada. The sun is shining,
\emph{let's live a bit} and believe that the emergency has passed
forever.

In this Phase 2 of the virus, with a rise in cases in countries
including Spain and France, it's an ant-or-cicada moment in many
societies. I confess to being something of a cicada by inclination, not
in slothful tendencies I hope, but in the belief that a life lived in
fear and obsessive prudence is not worth living. How to weigh the
cicada's chirping pleasure against the ant's cautious husbandry, a short
happy life against a long inhibited one?

The answer is not evident. As with most things in life, it lies in
balance. It's equally hard to say at what point reasonable, lifesaving
precaution over the virus becomes unreasonable, job-destroying,
school-closing and life-quenching fear --- harder still because rampant
fear was a striking characteristic of many societies \emph{before} the
virus. For Italy, the overriding question is how not to suffer a chaotic
relapse from the crisis-induced effectiveness of national unity.

There will be renewed division and disappointments, but I don't believe
anything can undo what Italy revealed of itself. Italy had a good war.
To a degree unimaginable in Donald Trump's America, and beyond even that
of many Europeans, Italians showed what long history teaches: civic
wisdom.

A summer fairy tale gripped Italy. It centered on Atalanta, the small
soccer club of Bergamo, the northern town that was the virus's
epicenter. Against all odds, Atalanta reached the quarterfinal of
Europe's premier club competition, the Champions League, where, in an
empty stadium, it played Paris St. Germain, the French capital's rich
Qatari-owned club. When Atalanta took the lead in the first half, a loud
cheer coursed down the Italian peninsula.

I watched the game this week with Antonio Colpani and Laura Vergani,
both from Bergamo. Colpani told me of his mother's near death and his
own battle with the virus. Vergani recalled the constant sirens and how
one day they stopped because the streets were empty anyway and local
authorities had concluded that the sound spread panic.

``We beat it,'' Colpani said. ``Non mollare mai.'' He smiled as he
uttered the phrase --- \emph{Never give up ---} by which Bergamo lives.
``Mola mia'' in Bergamasco dialect.

Atalanta, unyielding,
\href{https://www.espn.com/soccer/report?gameId=573701}{clung to its 1-0
lead until the last minute}. Then Paris St-Germain scored, and a moment
later scored again to win 2-1.

It would have been wonderful for the fairy tale to continue, but perhaps
for Italy the agonizing defeat was a useful reality check in this summer
limbo, a grain of ant in the song of the cicada.

``That's life,'' Colpani said, ``Everything can change in a minute.''

\emph{The Times is committed to publishing}
\href{https://www.nytimes3xbfgragh.onion/2019/01/31/opinion/letters/letters-to-editor-new-york-times-women.html}{\emph{a
diversity of letters}} \emph{to the editor. We'd like to hear what you
think about this or any of our articles. Here are some}
\href{https://help.nytimes3xbfgragh.onion/hc/en-us/articles/115014925288-How-to-submit-a-letter-to-the-editor}{\emph{tips}}\emph{.
And here's our email:}
\href{mailto:letters@NYTimes.com}{\emph{letters@NYTimes.com}}\emph{.}

\emph{Follow The New York Times Opinion section on}
\href{https://www.facebookcorewwwi.onion/nytopinion}{\emph{Facebook}}\emph{,}
\href{http://twitter.com/NYTOpinion}{\emph{Twitter (@NYTopinion)}}
\emph{and}
\href{https://www.instagram.com/nytopinion/}{\emph{Instagram}}\emph{.}

Advertisement

\protect\hyperlink{after-bottom}{Continue reading the main story}

\hypertarget{site-index}{%
\subsection{Site Index}\label{site-index}}

\hypertarget{site-information-navigation}{%
\subsection{Site Information
Navigation}\label{site-information-navigation}}

\begin{itemize}
\tightlist
\item
  \href{https://help.nytimes3xbfgragh.onion/hc/en-us/articles/115014792127-Copyright-notice}{©~2020~The
  New York Times Company}
\end{itemize}

\begin{itemize}
\tightlist
\item
  \href{https://www.nytco.com/}{NYTCo}
\item
  \href{https://help.nytimes3xbfgragh.onion/hc/en-us/articles/115015385887-Contact-Us}{Contact
  Us}
\item
  \href{https://www.nytco.com/careers/}{Work with us}
\item
  \href{https://nytmediakit.com/}{Advertise}
\item
  \href{http://www.tbrandstudio.com/}{T Brand Studio}
\item
  \href{https://www.nytimes3xbfgragh.onion/privacy/cookie-policy\#how-do-i-manage-trackers}{Your
  Ad Choices}
\item
  \href{https://www.nytimes3xbfgragh.onion/privacy}{Privacy}
\item
  \href{https://help.nytimes3xbfgragh.onion/hc/en-us/articles/115014893428-Terms-of-service}{Terms
  of Service}
\item
  \href{https://help.nytimes3xbfgragh.onion/hc/en-us/articles/115014893968-Terms-of-sale}{Terms
  of Sale}
\item
  \href{https://spiderbites.nytimes3xbfgragh.onion}{Site Map}
\item
  \href{https://help.nytimes3xbfgragh.onion/hc/en-us}{Help}
\item
  \href{https://www.nytimes3xbfgragh.onion/subscription?campaignId=37WXW}{Subscriptions}
\end{itemize}
