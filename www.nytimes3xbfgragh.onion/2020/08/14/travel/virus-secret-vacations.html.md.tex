Sections

SEARCH

\protect\hyperlink{site-content}{Skip to
content}\protect\hyperlink{site-index}{Skip to site index}

\href{https://www.nytimes3xbfgragh.onion/section/travel}{Travel}

\href{https://myaccount.nytimes3xbfgragh.onion/auth/login?response_type=cookie\&client_id=vi}{}

\href{https://www.nytimes3xbfgragh.onion/section/todayspaper}{Today's
Paper}

\href{/section/travel}{Travel}\textbar{}Shhh! We're Heading Off on
Vacation

\url{https://nyti.ms/2E1U1Zc}

\begin{itemize}
\item
\item
\item
\item
\item
\item
\end{itemize}

\href{https://www.nytimes3xbfgragh.onion/spotlight/at-home?action=click\&pgtype=Article\&state=default\&region=TOP_BANNER\&context=at_home_menu}{At
Home}

\begin{itemize}
\tightlist
\item
  \href{https://www.nytimes3xbfgragh.onion/2020/09/07/travel/route-66.html?action=click\&pgtype=Article\&state=default\&region=TOP_BANNER\&context=at_home_menu}{Cruise
  Along: Route 66}
\item
  \href{https://www.nytimes3xbfgragh.onion/2020/09/04/dining/sheet-pan-chicken.html?action=click\&pgtype=Article\&state=default\&region=TOP_BANNER\&context=at_home_menu}{Roast:
  Chicken With Plums}
\item
  \href{https://www.nytimes3xbfgragh.onion/2020/09/04/arts/television/dark-shadows-stream.html?action=click\&pgtype=Article\&state=default\&region=TOP_BANNER\&context=at_home_menu}{Watch:
  Dark Shadows}
\item
  \href{https://www.nytimes3xbfgragh.onion/interactive/2020/at-home/even-more-reporters-editors-diaries-lists-recommendations.html?action=click\&pgtype=Article\&state=default\&region=TOP_BANNER\&context=at_home_menu}{Explore:
  Reporters' Google Docs}
\end{itemize}

Advertisement

\protect\hyperlink{after-top}{Continue reading the main story}

Supported by

\protect\hyperlink{after-sponsor}{Continue reading the main story}

\hypertarget{shhh-were-heading-off-on-vacation}{%
\section{Shhh! We're Heading Off on
Vacation}\label{shhh-were-heading-off-on-vacation}}

Sharing vacation adventures used to be fun. But during the pandemic,
some travelers are content to let the tree fall in the forest, so to
speak, without a single soul to hear it.

\includegraphics{https://static01.graylady3jvrrxbe.onion/images/2020/08/20/fashion/13travel-sneakyvacations1/merlin_175631886_1fa8d1fa-56d7-4e24-9286-1689349837b2-articleLarge.jpg?quality=75\&auto=webp\&disable=upscale}

By Sarah Firshein

\begin{itemize}
\item
  Aug. 14, 2020
\item
  \begin{itemize}
  \item
  \item
  \item
  \item
  \item
  \item
  \end{itemize}
\end{itemize}

Next month, Elena Gaudino will fly from New York to Las Vegas, rent an
S.U.V. and drive to the Grand Canyon, Joshua Tree National Park and
other desert destinations. The 10-day road trip stands in for her
favorite annual tradition --- Burning Man,
\href{https://www.nytimes3xbfgragh.onion/2020/04/11/arts/music/burning-man-canceled-coronavirus.html}{the
Nevada arts festival} that was canceled this year because of the
pandemic --- and gives her something to look forward to after a
coronavirus-induced travel dry-spell.

Now she is itching to trade her Brooklyn apartment for the wide-open
spaces of the American Southwest. But unlike in years past, Ms. Gaudino
will post no requests for restaurant recommendations on Facebook, nor
will she swap excited texts with friends detailing her itinerary. Aside
from her husband and their two travel companions --- and, now, readers
of The New York Times --- Ms. Gaudino has no plans to tell anyone about
her trip.

``Some people believe you're selfish for leaving your home unless it's
to get groceries,'' said Ms. Gaudino, 34, a communications consultant.
``I'd rather avoid potential altercations and I can go into this
experience with a clear mind: I'm taking all the mandated precautions, I
know the risk.''

\href{https://www.nytimes3xbfgragh.onion/interactive/2020/05/06/travel/coronavirus-travel-questions.html}{}

\includegraphics{https://static01.graylady3jvrrxbe.onion/images/2020/05/06/multimedia/06FOT-coverairplanes/06FOT-coverairplanes-articleLarge-v2.jpg}

\hypertarget{the-future-of-travel}{%
\subsection{The Future of Travel}\label{the-future-of-travel}}

Perhaps no industry has been as hard hit by the pandemic as tourism. As
restrictions on companies and travelers ease, what will the new world
look like?

Sharing the details about where we've traveled has always been a way to
transmit our values, tastes and means --- look no further than the
postcards of the 19th century or the Kodak carousels of the 1960s and
70s. Then came Instagram, a decade ago, to turbocharge the practice. And
while technology has made it easy to keep up with loved ones during this
period of physical distance, there is one topic being withheld from
conversations and hidden from social media: vacations. For a variety of
reasons related to the pandemic, some travelers are content to let the
tree fall in the forest, so to speak, without a single soul around to
hear it.

``In addition to protecting your self-image and reputation, a main
reason people keep secrets is to protect relationships and avoid
conflicts,'' said
\href{http://www.columbia.edu/~ms4992/index.html}{Michael Slepian}, a
Columbia Business School associate professor who studies secrecy.
``People often think, `You know, life would just be easier if I didn't
have that fight with my parents, so I'm not going to let them know about
my trip.'''

In the last couple of years, the concept of ``flight shaming'' ---
originally coined as ``flygskam'' by the Swedish climate activist Greta
Thunberg --- has
\href{https://www.nytimes3xbfgragh.onion/2019/12/18/travel/Sweden-flight-shaming.html}{gained
momentum} as part of an anti-air-travel environmental movement. Today,
mid-pandemic, general ``travel shaming'' could also take off.

Two-thirds of the nearly 4,000 Americans
\href{https://www.ketchum.com/travel-economic-development/}{surveyed} in
June by Ketchum Travel, a public relations agency, said they would judge
others for traveling before it's considered ``safe.'' Half expected to
censor their social media posts to avoid being ``travel shamed''
themselves. Compare that with last year, when about 80 percent of the
1,300 respondents in a
\href{https://research.skift.com/wp-content/uploads/2019/03/ExperientialTraveler2019.pdf}{Skift
Research survey} said they posted trip photos on social media.

``The pandemic presents a unique case of travel entering the moral
sphere, because there are two things that happen when you travel: The
first is that I put myself at risk, and the second is by virtue of
putting myself at risk, I could be spreading coronavirus to other
people,'' said Jillian Jordan, a Harvard Business School assistant
professor who studies moral psychology.

All it took for Lauren Pearlman, who lives in Gainesville, Fla., to
discover what she called a friend's ``shame-cation'' was some shrewd
digital sleuthing. One hint? A rogue Instagram post --- depicting a lake
cottage in a decidedly vacation-y setting --- by the friend's husband.

``I feel like it compromises our friendship because it exposes very
different philosophical approaches to the pandemic,'' said Ms. Pearlman,
37, a history professor at the University of Florida. ``And if you're
going to go on vacation, then own it and say that you are. If you don't
feel like you can advertise it, then obviously you aren't positive it's
the ethical thing to do.''

Dr. Jordan said the pandemic --- thanks to its unprecedented nature in
modern times and patchwork of geography-based restrictions --- remains a
gray area for ethical norms. Whereas most people would agree that
shoplifting is unacceptable, for example, so far there is no universally
agreed-upon consensus about
\href{https://www.nytimes3xbfgragh.onion/interactive/2020/07/31/travel/coronavirus-travel-risk.html}{whether
or not to travel.}

``Some people think any trips of any kind are bad; others, meanwhile,
are off flying to hot-spots,'' Dr. Jordan said. ``If you think it's fine
to travel and some people don't think it's fine --- but you're not
persuaded by the opposing argument --- you may feel motivated to hide
your behavior.''

That can be true even when travelers feel confident they're taking
proper health precautions. Ms. Gaudino plans to stay in Airbnbs and
campgrounds; except for grocery shopping --- while wearing a face mask
--- she will not participate in any public indoor activities. To prepare
for a 14-day quarantine upon her return,
\href{https://www.nytimes3xbfgragh.onion/2020/06/24/nyregion/ny-coronavirus-states-quarantine.html}{required}
by New York for anyone coming from states like Arizona and California,
she has stocked her fridge and pantry with long-lasting provisions.

Catharine Jones, 39, also prioritized hygiene and safety when in June
she drove with her family from their home in Rochester, Minn., to a lake
about three-and-a-half hours north. They stayed in-state, wore masks and
bunked in a self-contained cabin.

Watching her children --- ages 2, 4 and 7 --- play happily by the lake
at dusk, she did what many parents might do: She took a photograph and
posted it on Instagram.

``Right after I posted it, I thought, `Wait a second,''' said Ms. Jones,
a journalist. ``Am I going to be judged for doing this? Are people going
to say, `Wait, you left your house?' The second thing that ran through
my mind was an awareness of how lucky we are: to travel, to be able to
spend money, to have a leisurely weekend.''

Though she was not chided for that post, Ms. Jones realized that she
wants to keep her next trip --- another private in-state road trip with
little, if any, contact with strangers --- to herself.

``We're living in this moment when longstanding inequities are
particularly stark and the dividing line is between people whose lives
remain relatively normal and people whose lives have been completely
turned upside down by this pandemic,'' she said. ``I feel like vacation
pictures signal to the world, `Hey! This isn't so bad!' And it has been
really that bad for many, many, many people.''

The question of what, if anything, to publish on social media is even
more complex for travel influencers, whose incomes rely on trips. Some
are concerned about backlash from an audience of thousands; others are
mulling over how to depict travel responsibly.

``As travel storytellers, our influence can sometimes be a double-edged
sword, because while we may have influenced someone to travel to a
certain place, we can't control what they do when they get there,'' said
\href{http://www.oneikaraymond.com}{Oneika Raymond}, a New York ---
based TV host and travel expert. ``Keeping trips quieter might just keep
intense wanderlust, and subsequently these transgressions, at bay.''

Although there are obvious benefits to digitally detaching --- Ms.
Gaudino, for one, is looking forward to a trip without a mad dash for
Wi-Fi --- sneaky trips may have other drawbacks.

``Secrecy can still be hard even in the absence of shame and guilt,
because you want to share your experiences with others,'' said Dr.
Slepian. ``Even before the vacation you can get a lot of joy just from
talking about it, and this is the real reason secrecy is so difficult:
It deprives yourself from a way to connect with other people.''

\includegraphics{https://static01.graylady3jvrrxbe.onion/images/2020/08/20/fashion/13travel-sneakyvacations2/merlin_175631952_ddbd59cd-bd53-4c7b-8d04-0dc18d3c31e1-articleLarge.jpg?quality=75\&auto=webp\&disable=upscale}

Yet it was secrecy that allowed Sonia Chopra, a Brooklyn-based food
editor, to find joy in her wedding last month --- a weekend in upstate
New York in lieu of what had been planned as a blowout bash in Atlanta.
Out of the 350 original guests, only her parents and a couple of close
friends knew about the trip.

She didn't want to endure a barrage of questions: Did she go away? (Yes;
to Tarrytown, N.Y.) Did she stay at a hotel? (Yes;
\href{https://www.tarrytownhouseestate.com/}{Tarrytown House Estate},
which has a slew of
\href{https://www.tarrytownhouseestate.com/covid-19-advisory}{Covid-19
measures.}) Did she and her husband dine out? (Yes; at
\href{https://www.nytimes3xbfgragh.onion/2019/10/21/dining/michelin-guide-nyc-2020-blue-hill-at-stone-barns.html}{Blue
Hill at Stone Barns,} an upscale restaurant offering a contactless
outdoor ``picnic'' where everything is pre-ordered online, including
bottled cocktails.)

``Although we were being very safe and very careful, we wanted to make
sure that nothing put a pall on our day,'' said Ms. Chopra, 31. ``We're
taking this very seriously, but people in very well-meaning ways can
sometimes ask questions that can make you feel badly, and we were trying
really hard to make sure the weekend felt special.''

\href{https://twitter.com/sfirshein?lang=en}{Sarah Firshein} is a
Brooklyn-based writer. If you need advice about a best-laid travel plan
that went awry, \textbf{\href{mailto:travel@NYTimes.com}{send an email
to travel@NYTimes.com}.}

\begin{center}\rule{0.5\linewidth}{\linethickness}\end{center}

\emph{\textbf{Follow New York Times Travel}} \emph{on}
\href{https://www.instagram.com/nytimestravel/}{\emph{Instagram}}\emph{,}
\href{https://twitter.com/nytimestravel}{\emph{Twitter}} \emph{and}
\href{https://www.facebookcorewwwi.onion/nytimestravel/}{\emph{Facebook}}\emph{.
You can also}
\href{https://www.nytimes3xbfgragh.onion/newsletters/traveldispatch?action=click\&module=inline\&pgtype=Article}{\emph{sign
up for our}} **
\href{https://www.nytimes3xbfgragh.onion/newsletters/traveldispatch}{\emph{Travel
Dispatch newsletter}}\emph{: Each week you'll receive tips on traveling
smarter, stories on hot destinations and access to photos from all over
the world.}

Advertisement

\protect\hyperlink{after-bottom}{Continue reading the main story}

\hypertarget{site-index}{%
\subsection{Site Index}\label{site-index}}

\hypertarget{site-information-navigation}{%
\subsection{Site Information
Navigation}\label{site-information-navigation}}

\begin{itemize}
\tightlist
\item
  \href{https://help.nytimes3xbfgragh.onion/hc/en-us/articles/115014792127-Copyright-notice}{©~2020~The
  New York Times Company}
\end{itemize}

\begin{itemize}
\tightlist
\item
  \href{https://www.nytco.com/}{NYTCo}
\item
  \href{https://help.nytimes3xbfgragh.onion/hc/en-us/articles/115015385887-Contact-Us}{Contact
  Us}
\item
  \href{https://www.nytco.com/careers/}{Work with us}
\item
  \href{https://nytmediakit.com/}{Advertise}
\item
  \href{http://www.tbrandstudio.com/}{T Brand Studio}
\item
  \href{https://www.nytimes3xbfgragh.onion/privacy/cookie-policy\#how-do-i-manage-trackers}{Your
  Ad Choices}
\item
  \href{https://www.nytimes3xbfgragh.onion/privacy}{Privacy}
\item
  \href{https://help.nytimes3xbfgragh.onion/hc/en-us/articles/115014893428-Terms-of-service}{Terms
  of Service}
\item
  \href{https://help.nytimes3xbfgragh.onion/hc/en-us/articles/115014893968-Terms-of-sale}{Terms
  of Sale}
\item
  \href{https://spiderbites.nytimes3xbfgragh.onion}{Site Map}
\item
  \href{https://help.nytimes3xbfgragh.onion/hc/en-us}{Help}
\item
  \href{https://www.nytimes3xbfgragh.onion/subscription?campaignId=37WXW}{Subscriptions}
\end{itemize}
