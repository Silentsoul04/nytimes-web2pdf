Sections

SEARCH

\protect\hyperlink{site-content}{Skip to
content}\protect\hyperlink{site-index}{Skip to site index}

\href{https://www.nytimes3xbfgragh.onion/section/politics}{Politics}

\href{https://myaccount.nytimes3xbfgragh.onion/auth/login?response_type=cookie\&client_id=vi}{}

\href{https://www.nytimes3xbfgragh.onion/section/todayspaper}{Today's
Paper}

\href{/section/politics}{Politics}\textbar{}McConnell Appears Set to
Quietly Suffocate Long-Debated F.B.I. Surveillance Bill

\begin{itemize}
\item
\item
\item
\item
\item
\end{itemize}

Advertisement

\protect\hyperlink{after-top}{Continue reading the main story}

Supported by

\protect\hyperlink{after-sponsor}{Continue reading the main story}

\hypertarget{mcconnell-appears-set-to-quietly-suffocate-long-debated-fbi-surveillance-bill}{%
\section{McConnell Appears Set to Quietly Suffocate Long-Debated F.B.I.
Surveillance
Bill}\label{mcconnell-appears-set-to-quietly-suffocate-long-debated-fbi-surveillance-bill}}

The Senate majority leader has long been a strong supporter of extending
Patriot Act provisions. This year, he's letting them remain lapsed.

\includegraphics{https://static01.graylady3jvrrxbe.onion/images/2020/08/14/us/politics/14dc-surveillance/merlin_175577148_12166e6c-1ebb-455c-b2ec-21add7ccafbb-articleLarge.jpg?quality=75\&auto=webp\&disable=upscale}

\href{https://www.nytimes3xbfgragh.onion/by/charlie-savage}{\includegraphics{https://static01.graylady3jvrrxbe.onion/images/2018/06/12/multimedia/author-charlie-savage/author-charlie-savage-thumbLarge-v2.png}}

By \href{https://www.nytimes3xbfgragh.onion/by/charlie-savage}{Charlie
Savage}

\begin{itemize}
\item
  Aug. 14, 2020
\item
  \begin{itemize}
  \item
  \item
  \item
  \item
  \item
  \end{itemize}
\end{itemize}

WASHINGTON --- Every few years since Congress passed the Patriot Act to
bolster F.B.I. surveillance powers after the Sept. 11, 2001, terrorist
attacks, parts of it came up for renewal and national security hawks
darkly warned that even briefly letting them lapse would unacceptably
endanger America.

This year is different.

Those provisions expired months ago, but one of the biggest surveillance
supporters in Congress, Senator Mitch McConnell, Republican of Kentucky
and the majority leader, has single-handedly brought the process of
extending them to a halt.

After the House and Senate passed slightly different versions of a bill
in May, Speaker Nancy Pelosi appointed a group of House members to a
conference committee to reconcile them. But Mr. McConnell has not
appointed any senators to it, and the process has been stuck there ever
since.

On Thursday, Mr. McConnell closed the Senate for its annual August
vacation, meaning that the legislation will remain on ice for at least
another month. He has given no sign that he intends to move forward when
Congress briefly reconvenes before adjourning again ahead of the
November election.

Mr. McConnell's office has declined to answer questions about why he is
keeping the measure in indefinite limbo. But others involved in the
process have suggested that Mr. McConnell may be paralyzed by two
factors.

First, President Trump, stoking his grievances over the Russia
investigation --- a small part of which involved wiretapping a former
campaign aide under the Foreign Intelligence Surveillance Act, or FISA
--- has threatened to veto the bill, saying more time should first be
spent scrutinizing the officials who pursued the Russia inquiry.

Mr. Trump's stance has vaguely suggested that the existing bills, which
have some efforts at reform, do not go far enough in curtailing the
F.B.I.'s surveillance powers for national security investigations. But
the White House has offered no specific proposal or guidance about what
he would be willing to sign. Complicating matters, Attorney General
William P. Barr opposes the current versions as going too far in the
opposite direction of tying the hands of F.B.I. agents fighting
terrorism and espionage. Mr. McConnell shares those concerns.

With no clear path to avoiding a veto confrontation with Mr. Trump, and
no great love for either version of the bill anyway, Mr. McConnell
appears to have decided that leaving the provisions lapsed and doing
nothing is the least-bad course of action.

``After they spent years insisting that the Patriot Act was essential to
protecting America, it's striking that Mitch McConnell and Attorney
General Barr are willing to wait for months to address it, and still
appear to have no plan for moving forward,'' Senator Ron Wyden, Democrat
of Oregon, said in a statement to The New York Times.

But there is a catch to the current stalemate: The expired surveillance
powers have not lapsed for existing investigations.

Congress specified that the powers would remain available for
investigations that already existed at the point the laws expired. Among
them are the F.B.I.'s ability to obtain business records deemed relevant
to a national security investigation, and to get special wiretap
authority to rapidly follow a suspect who changes phone lines to evade
monitoring.

The F.B.I. has existing, open-ended umbrella investigations into all the
major national security threats facing the United States, such as
terrorist groups like Al Qaeda and the Islamic State and nation-state
adversaries like Russia and China. As a result, until a new rival arises
who cannot be characterized as part of the old ones, the bureau may be
able to continue business as usual.

But the bill's collapse also means Congress has been unable to enact a
legislative response to an inspector general's finding of serious
problems with the F.B.I.'s use of the main part of FISA for wiretap
orders in counterespionage and counterterrorism investigations.

A report by the Justice Department's independent inspector general found
that the applications used in the Russia investigation to wiretap a
former Trump campaign adviser with ties to Russian officials, Carter
Page, were riddled with errors and omissions.

A follow-up review by the inspector general that looked at the F.B.I.'s
preparations to apply for 29 unrelated FISA wiretaps
\href{https://www.nytimes3xbfgragh.onion/2020/03/31/us/politics/fbi-fisa-wiretap-trump.html}{found
that there were problems with all of them, suggesting systemic
sloppiness}. The Justice Department has since
\href{https://www.justice.gov/opa/pr/statement-assistant-attorney-general-national-security-john-c-demers-public-release}{told
the FISA court} that its own further review of those 29 applications
found one material omission and one material misstatement, but also said
it did not think either made a legal difference.

But the F.B.I. has also conceded that it should not have applied for two
renewals of the Page wiretap in 2017, and it has tightened its own rules
and procedures for drafting FISA applications. A FISA court judge also
imposed additional safeguards. But Congress has not enacted anything.

The expired provisions are unrelated to the type of wiretaps used in the
Russia investigation, but the bill has become a vehicle for addressing
those problems. Both the House and the Senate have attached different
versions of a provision to more frequently appoint an outside critic to
analyze a government's application in certain types of cases, including
those that could touch on a political campaign.

But civil-libertarian-minded lawmakers have also been trying to use the
bill to impose other new privacy safeguards. A majority in the Senate
backed banning the use of one part of FISA --- the partly lapsed
provision that permits the F.B.I. to obtain business records without a
warrant --- for gathering internet search histories and browsing
records.

For procedural reasons, that idea fell short despite majority support.
When the bill returned to the House, Ms. Pelosi came under pressure to
permit a vote on the same idea. But the version that came to the House
floor was watered down, losing the support of civil libertarians.

Mr. Trump then came out against the bill for different reasons, and many
of his Republican allies abandoned it, too, leading the House to instead
send a previously passed version to the conference committee to be
reconciled with the Senate version --- where it has since remained in
limbo, waiting for senators to show up.

``It's clear there is a large bipartisan majority in both the House and
Senate for reforming FISA and the Patriot Act,'' Mr. Wyden said. ``Even
Donald Trump agrees that reform is needed, Leader McConnell and A.G.
Barr should allow a real debate on reforming government surveillance.''

Nicholas Fandos contributed reporting.

Advertisement

\protect\hyperlink{after-bottom}{Continue reading the main story}

\hypertarget{site-index}{%
\subsection{Site Index}\label{site-index}}

\hypertarget{site-information-navigation}{%
\subsection{Site Information
Navigation}\label{site-information-navigation}}

\begin{itemize}
\tightlist
\item
  \href{https://help.nytimes3xbfgragh.onion/hc/en-us/articles/115014792127-Copyright-notice}{©~2020~The
  New York Times Company}
\end{itemize}

\begin{itemize}
\tightlist
\item
  \href{https://www.nytco.com/}{NYTCo}
\item
  \href{https://help.nytimes3xbfgragh.onion/hc/en-us/articles/115015385887-Contact-Us}{Contact
  Us}
\item
  \href{https://www.nytco.com/careers/}{Work with us}
\item
  \href{https://nytmediakit.com/}{Advertise}
\item
  \href{http://www.tbrandstudio.com/}{T Brand Studio}
\item
  \href{https://www.nytimes3xbfgragh.onion/privacy/cookie-policy\#how-do-i-manage-trackers}{Your
  Ad Choices}
\item
  \href{https://www.nytimes3xbfgragh.onion/privacy}{Privacy}
\item
  \href{https://help.nytimes3xbfgragh.onion/hc/en-us/articles/115014893428-Terms-of-service}{Terms
  of Service}
\item
  \href{https://help.nytimes3xbfgragh.onion/hc/en-us/articles/115014893968-Terms-of-sale}{Terms
  of Sale}
\item
  \href{https://spiderbites.nytimes3xbfgragh.onion}{Site Map}
\item
  \href{https://help.nytimes3xbfgragh.onion/hc/en-us}{Help}
\item
  \href{https://www.nytimes3xbfgragh.onion/subscription?campaignId=37WXW}{Subscriptions}
\end{itemize}
