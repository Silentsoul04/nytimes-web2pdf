Sections

SEARCH

\protect\hyperlink{site-content}{Skip to
content}\protect\hyperlink{site-index}{Skip to site index}

\href{https://myaccount.nytimes3xbfgragh.onion/auth/login?response_type=cookie\&client_id=vi}{}

\href{https://www.nytimes3xbfgragh.onion/section/todayspaper}{Today's
Paper}

\href{/section/opinion}{Opinion}\textbar{}Joe Biden's Secret Untapped
Weapon

\url{https://nyti.ms/2Ydcidn}

\begin{itemize}
\item
\item
\item
\item
\item
\end{itemize}

Advertisement

\protect\hyperlink{after-top}{Continue reading the main story}

\href{/section/opinion}{Opinion}

Supported by

\protect\hyperlink{after-sponsor}{Continue reading the main story}

\hypertarget{joe-bidens-secret-untapped-weapon}{%
\section{Joe Biden's Secret Untapped
Weapon}\label{joe-bidens-secret-untapped-weapon}}

How he could win the Latino vote \ldots{} and the presidency.

By Chuck Rocha

Mr. Rocha is the author of the book
``T\href{https://strongarmpress.com/catalog/tio-bernie/}{ío Bernie: The
Inside Story of How Bernie Sanders Brought Latinos Into the Political
Revolution}.''

\begin{itemize}
\item
  Aug. 18, 2020
\item
  \begin{itemize}
  \item
  \item
  \item
  \item
  \item
  \end{itemize}
\end{itemize}

\includegraphics{https://static01.graylady3jvrrxbe.onion/images/2020/08/18/opinion/18rocha1/18rocha1-articleLarge.jpg?quality=75\&auto=webp\&disable=upscale}

In less than four years, the Trump administration has done great damage
to the Latino community. Yet Donald Trump, who won 28 percent of the
Latino vote in 2016, is still the choice of nearly
\href{https://latinodecisions.com/wp-content/uploads/2020/08/National-and-State-Results-Somos-Unidos-Aug-Svy.pdf}{20
percent} --- enough support to make a difference in a close election.
You'd think that Latinos would rally behind Joe Biden, but he has
\href{https://www.politico.com/news/2020/07/13/biden-latino-vote-360033}{struggled}
to connect with them.

By some estimates,
\href{https://www.nytimes3xbfgragh.onion/2020/05/06/opinion/latino-voters-2020-election.html}{nearly
60 percent} of eligible Latino voters in battleground states, who tend
to view Mr. Trump unfavorably, will sit out the 2020 election. Rallying
these voters --- most of them young, most of them hostile to Mr. Trump
--- will be vital to capturing the White House.

As a senior adviser for Senator Bernie Sanders's presidential campaign,
I led a \$15 million Latino outreach program that secured victories in
Nevada and California, and helped us compete hard everywhere Latinos
could hear our message.

To many, Senator Sanders, an older, white man from Vermont, may seem an
unlikely hero for Latinos. But, as the son of an immigrant, Tío Bernie
was relatable to them. He was the first Democratic presidential
candidate to establish a presence in the Latino community and engage
with its members. Mr. Biden can take a page out of our playbook.

Mr. Biden has begun to ramp up investment in communications to Latinos,
but he has to go further. That includes expanding the universe of Latino
voters he's talking to and engaging with them across different
platforms. The Sanders 2020 campaign worked to reach all the Latinos who
were eligible to vote in Nevada ahead of the caucus including more than
100,000 newly registered voters as well as those with no history of
voting.

We integrated Latino organizers into our staff and reached out to Latino
voters early and often. We visited them at their homes and work sites.
We connected with them via phone and bilingual radio ads and mailers. We
organized community-based events like soccer tournaments where we'd
register voters and host caucus trainings. Our strategy worked ---
Senator Sanders won 73 percent of Nevada's Latino vote. He also won big
in California and Colorado.

If Mr. Biden takes a conventional approach and talks to the same little
sliver of older Latinos in Arizona everyone talks to, he could be in
trouble. But if he taps into the culture of activism in Arizona that
registered 138,000 Latinos since Mr. Trump was elected, helped oust the
anti-immigrant sheriff Joe Arpaio and is working to ensure Mr. Trump is
next, Mr. Biden can reshape the electoral map and win.

Right now, the Biden campaign is spending \$800,000 a week on
Spanish-language cable television ads. But in the age of Covid-19, the
old way of campaigning has been thrown out the window, and more can be
done through a multilayered strategy. This includes swapping terrestrial
for digital radio, advertising in the Spanish-language newspapers
abuelas read and sending bilingual mailers that explain his plan to
rebuild an economy in tatters.

Luz Chaparro Hernandez, a bilingual elementary-school teacher and union
member,
\href{https://www.jsonline.com/story/news/politics/elections/2020/08/10/democratic-national-convention-speakers-milwaukee-include-teacher-retiree/3333188001/}{is
part} of the Democratic convention this week, as well as
\href{https://www.news-press.com/story/news/2020/08/10/fort-myers-man-give-virtual-speech-democratic-national-convention/3332617001/}{Aldo
Martinez}, a 26-year-old DACA recipient and paramedic from Fort Myers,
Fla., who is braving the horrors of the coronavirus head on. The next
step for the campaign is to invest in telling stories like these on
every platform Latinos engage with, from Pandora and YouTube to
Telemundo and El Nuevo Herald.

I've been part of focus groups with Latino voters who bring up the
2016-era problem that Democrats still haven't cracked: An anti-Trump
message is useful --- after all, the president is a disaster for Latinos
--- but Latinos don't know what a Biden administration would mean for
their families. We need to tell them how Mr. Biden's recently announced
Latino agenda would inject their small businesses with capital, raise
the minimum wage to \$15 an hour and eliminate the minimum tipped wage,
which would greatly benefit Latino restaurant workers.

Mr. Biden has one more secret weapon to unlocking the Latino vote:
Kamala Harris, a multicultural daughter of immigrants who embodies the
American dream. Senator Harris --- a public servant raised by a single
mother, an advocate for immigrant children separated from their parents
and Latino small businesses devastated by the pandemic shutdown --- can
bridge the gap between Mr. Biden and the Latino community. A recent
\href{https://latinodecisions.com/wp-content/uploads/2020/07/VPC-Latino-Decisions-June-2020-CROSSTABS.pdf?utm_medium=email\&utm_campaign=New\%20From\%20Latino\%20Decisions\%20-\%20title\%202020-08-11\%20181609\&utm_content=New\%20From\%20Latino\%20Decisions\%20-\%20title\%202020-08-11\%20181609+CID_df93fb1bb6deb24f993668c82850b2dd\&utm_source=Email\%20marketing\%20software\&utm_term=poll}{poll}
found that 52 percent of Latino voters in key battleground states said
the selection of Ms. Harris as his running mate would make them more
likely to vote for Mr. Biden.

If Mr. Biden uses the Sanders multilayered approach to connect with
Latinos, maybe he will also become a tío we can be proud of. If he
manages to reach and mobilize them, these voters could ultimately be the
ones who send Mr. Trump packing.

Chuck Rocha
(\href{https://twitter.com/ChuckRocha?ref_src=twsrc\%5Egoogle\%7Ctwcamp\%5Eserp\%7Ctwgr\%5Eauthor}{@ChuckRocha}),
the founder and president of
\href{https://www.solidaritystrategies.com/}{Solidarity Strategies} and
a former senior adviser to Senator Bernie Sanders, is the author of the
book ``\href{https://strongarmpress.com/catalog/tio-bernie/}{Tío Bernie:
The Inside Story of How Bernie Sanders Brought Latinos Into the
Political Revolution}.''

\emph{The Times is committed to publishing}
\href{https://www.nytimes3xbfgragh.onion/2019/01/31/opinion/letters/letters-to-editor-new-york-times-women.html}{\emph{a
diversity of letters}} \emph{to the editor. We'd like to hear what you
think about this or any of our articles. Here are some}
\href{https://help.nytimes3xbfgragh.onion/hc/en-us/articles/115014925288-How-to-submit-a-letter-to-the-editor}{\emph{tips}}\emph{.
And here's our email:}
\href{mailto:letters@NYTimes.com}{\emph{letters@NYTimes.com}}\emph{.}

\emph{Follow The New York Times Opinion section on}
\href{https://www.facebookcorewwwi.onion/nytopinion}{\emph{Facebook}}\emph{,}
\href{http://twitter.com/NYTOpinion}{\emph{Twitter (@NYTopinion)}}
\emph{and}
\href{https://www.instagram.com/nytopinion/}{\emph{Instagram}}\emph{.}

Advertisement

\protect\hyperlink{after-bottom}{Continue reading the main story}

\hypertarget{site-index}{%
\subsection{Site Index}\label{site-index}}

\hypertarget{site-information-navigation}{%
\subsection{Site Information
Navigation}\label{site-information-navigation}}

\begin{itemize}
\tightlist
\item
  \href{https://help.nytimes3xbfgragh.onion/hc/en-us/articles/115014792127-Copyright-notice}{©~2020~The
  New York Times Company}
\end{itemize}

\begin{itemize}
\tightlist
\item
  \href{https://www.nytco.com/}{NYTCo}
\item
  \href{https://help.nytimes3xbfgragh.onion/hc/en-us/articles/115015385887-Contact-Us}{Contact
  Us}
\item
  \href{https://www.nytco.com/careers/}{Work with us}
\item
  \href{https://nytmediakit.com/}{Advertise}
\item
  \href{http://www.tbrandstudio.com/}{T Brand Studio}
\item
  \href{https://www.nytimes3xbfgragh.onion/privacy/cookie-policy\#how-do-i-manage-trackers}{Your
  Ad Choices}
\item
  \href{https://www.nytimes3xbfgragh.onion/privacy}{Privacy}
\item
  \href{https://help.nytimes3xbfgragh.onion/hc/en-us/articles/115014893428-Terms-of-service}{Terms
  of Service}
\item
  \href{https://help.nytimes3xbfgragh.onion/hc/en-us/articles/115014893968-Terms-of-sale}{Terms
  of Sale}
\item
  \href{https://spiderbites.nytimes3xbfgragh.onion}{Site Map}
\item
  \href{https://help.nytimes3xbfgragh.onion/hc/en-us}{Help}
\item
  \href{https://www.nytimes3xbfgragh.onion/subscription?campaignId=37WXW}{Subscriptions}
\end{itemize}
