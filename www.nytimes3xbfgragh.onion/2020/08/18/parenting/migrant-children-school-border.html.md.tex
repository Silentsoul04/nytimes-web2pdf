Sections

SEARCH

\protect\hyperlink{site-content}{Skip to
content}\protect\hyperlink{site-index}{Skip to site index}

\href{https://www.nytimes3xbfgragh.onion/section/parenting}{Parenting}

\href{https://myaccount.nytimes3xbfgragh.onion/auth/login?response_type=cookie\&client_id=vi}{}

\href{https://www.nytimes3xbfgragh.onion/section/todayspaper}{Today's
Paper}

\href{/section/parenting}{Parenting}\textbar{}Kids Seeking Asylum Find
Some Measure of It in One Makeshift School

\url{https://nyti.ms/2Q2TrwW}

\begin{itemize}
\item
\item
\item
\item
\item
\end{itemize}

\hypertarget{school-reopenings}{%
\subsubsection{\texorpdfstring{\href{https://www.nytimes3xbfgragh.onion/spotlight/schools-reopening?name=styln-coronavirus-schools-reopening\&region=TOP_BANNER\&block=storyline_menu_recirc\&action=click\&pgtype=Article\&impression_id=4a530b40-f4d1-11ea-b32a-256dc0696bd9\&variant=undefined}{School
Reopenings}}{School Reopenings}}\label{school-reopenings}}

\begin{itemize}
\tightlist
\item
  \href{https://www.nytimes3xbfgragh.onion/2020/09/08/us/school-districts-cyberattacks-glitches.html?name=styln-coronavirus-schools-reopening\&region=TOP_BANNER\&block=storyline_menu_recirc\&action=click\&pgtype=Article\&impression_id=4a530b41-f4d1-11ea-b32a-256dc0696bd9\&variant=undefined}{Remote
  Learning Glitches}
\item
  \href{https://www.nytimes3xbfgragh.onion/2020/09/08/upshot/children-testing-shortfalls-virus.html?name=styln-coronavirus-schools-reopening\&region=TOP_BANNER\&block=storyline_menu_recirc\&action=click\&pgtype=Article\&impression_id=4a533250-f4d1-11ea-b32a-256dc0696bd9\&variant=undefined}{Limited
  Testing for Children}
\item
  \href{https://www.nytimes3xbfgragh.onion/2020/09/10/us/des-moines-school-opening-coronavirus.html?name=styln-coronavirus-schools-reopening\&region=TOP_BANNER\&block=storyline_menu_recirc\&action=click\&pgtype=Article\&impression_id=4a533251-f4d1-11ea-b32a-256dc0696bd9\&variant=undefined}{District
  Defies Reopening Order}
\item
  \href{https://www.nytimes3xbfgragh.onion/interactive/2020/us/covid-college-cases-tracker.html?name=styln-coronavirus-schools-reopening\&region=TOP_BANNER\&block=storyline_menu_recirc\&action=click\&pgtype=Article\&impression_id=4a533252-f4d1-11ea-b32a-256dc0696bd9\&variant=undefined}{Tracking
  College Cases}
\end{itemize}

Advertisement

\protect\hyperlink{after-top}{Continue reading the main story}

Supported by

\protect\hyperlink{after-sponsor}{Continue reading the main story}

\hypertarget{kids-seeking-asylum-find-some-measure-of-it-in-one-makeshift-school}{%
\section{Kids Seeking Asylum Find Some Measure of It in One Makeshift
School}\label{kids-seeking-asylum-find-some-measure-of-it-in-one-makeshift-school}}

When the pandemic hit, volunteers had to find creative ways to educate
migrant children living in limbo near the U.S.-Mexico border.

\includegraphics{https://static01.graylady3jvrrxbe.onion/images/2020/08/18/multimedia/18par-border-homeschooling7/merlin_175763541_5bd3b445-6b29-48c7-94e0-58947723d57a-articleLarge.jpg?quality=75\&auto=webp\&disable=upscale}

By Myriam Vidal Valero and Rodrigo Pérez Ortega

\begin{itemize}
\item
  Aug. 18, 2020
\item
  \begin{itemize}
  \item
  \item
  \item
  \item
  \item
  \end{itemize}
\end{itemize}

\href{https://www.nytimes3xbfgragh.onion/es/2020/08/18/espanol/america-latina/escuela-frontera-mexico.html}{Leer
en español}

Ana Morales Becerra, a single mom from Michoacán, Mexico, once described
her former home as a calm place in the middle of a cartel war. With so
many narcos in her neighborhood, in the city of Uruapan, she felt sure
that no one would dare to come in to rob her. But still, she always felt
uneasy as her daily routine --- working two jobs and taking care of her
kids --- was punctuated with trucks full of armed people passing by.

The tipping point came, however, when narco trucks started following her
children. ``No more!'' she remembers saying. ``I'm leaving.'' Fleeing
cartel violence, sexual abuse and death threats, she left her home to
seek a fresh start in the United States. Last October, she arrived in
Tijuana, Mexico, with her four children, very little money and no place
to stay.

But seeking asylum, which Morales Becerra thought would be a relatively
fast process, turned out to be an administrative tar pit that would
strand her and her family for months while they waited for a judge to
decide their fate. ``I didn't know we would have to go through all this
process,'' she said. With their lives on hold, and no access to formal
jobs or school, they have been living at the Embajadores de Jesús
shelter, just three miles south of the U.S.-Mexico border, for almost a
year.

Like Morales Becerra's family, thousands of families from Central
America and Mexico have come to the U.S. southern border in recent years
escaping violence. The White House's Migrant Protection Protocols, also
known as the ``Remain in Mexico'' program, have
\href{https://www.nytimes3xbfgragh.onion/interactive/2019/08/18/us/mexico-immigration-asylum.html}{forced
migrants to wait in Mexico for months} with no guarantees of asylum.

During this time, children have little to no access to formal education.
``Jesús, my oldest son, was worried,'' Morales Becerra said. ``He told
me: `I already lost one year of school, Mom, I don't want to lose
another one.'''

\begin{center}\rule{0.5\linewidth}{\linethickness}\end{center}

\textbf{GET MORE NYT PARENTING}
\href{https://www.nytimes3xbfgragh.onion/newsletters/parenting?action=click\&module=RelatedLinks\&pgtype=Article}{\emph{Sign
up for this newsletter}} \emph{and}
\href{https://www.instagram.com/nytparenting/}{\emph{follow
@nytparenting on Instagram}} \emph{and}
\href{https://twitter.com/nytparenting/}{\emph{Twitter}}\emph{.}

\begin{center}\rule{0.5\linewidth}{\linethickness}\end{center}

In the past decade, the U.S.
\href{https://www.unhcr.org/globaltrends2019/}{registered an estimated
1.7 million} asylum requests, according to a United Nations Refugee
Agency. The Trump administration
\href{https://www.migrationpolicy.org/article/refugees-and-asylees-united-states-2018}{reduced
the number of refugees} the United States accepts annually from 110,000
in 2017 to 30,000 in 2019 --- less than ten percent of filed requests in
that year. Among those who seek asylum, ``children are much more
vulnerable,'' said Germán Casas, a Colombia-based child psychiatrist and
president of Doctors Without Borders Latin America.

The trauma that some experience en route --- family separation, physical
violence, kidnapping, sexual abuse and human trafficking --- is
detrimental for their development and mental health, Casas said. Many
migrant children have difficulty regulating their behaviors and
emotions, handling stress and developing
empathy,\href{https://onlinelibrary.wiley.com/doi/book/10.1002/9780470669280}{according
to research}.

With little help from the Tijuana government, volunteers on both sides
of the border have stepped in to offer classes to some children. But
just as one of these projects was gaining steam, the Covid-19 pandemic
struck.

\includegraphics{https://static01.graylady3jvrrxbe.onion/images/2020/08/18/multimedia/18par-border-homeschooling6/merlin_175817601_1c53c837-ccbf-44ea-916f-a4b825c09adb-articleLarge.jpg?quality=75\&auto=webp\&disable=upscale}

\hypertarget{a-rainbow-colored-classroom-rolls-in}{%
\subsection{A rainbow-colored classroom rolls
in}\label{a-rainbow-colored-classroom-rolls-in}}

Andrea Rincón Cortés, 21, feels a deep connection with migrants. Her
father tried to cross the border in 1992, but eventually settled in
Tijuana, where she was born and raised. As she grew up, she saw that
crossing over was a matter of survival to most migrants. As a teenager,
she started visiting shelters and coordinating donations. ``I felt this
closeness with them because I saw myself reflected,'' she said.

In July of 2019, while juggling university coursework and working for an
advocacy group called Border Angels, Rincón Cortés discovered the
\href{https://www.schoolboxproject.org/us-mexican-border}{School Box
Project}, an international organization that brings educational
activities to refugee children in Greece, Bangladesh and Syria. She
quickly proposed bringing these activities to migrant children on the
Mexican border, too.

Over the next few months, she and four other volunteers from both sides
of the border, onboard a rainbow-colored
school-bus-turned-mobile-classroom, visited three Tijuana shelters to
give children two-hour lessons. ``We focused initially on doing art
therapy activities to identify what educational and emotional needs they
were having,'' Rincón Cortés said.

Image

In 2019, Andrea Rincón Cortés, 21, became involved as a volunteer
providing schooling to children in shelters on the Mexican border. When
the pandemic ceased their mobile-classroom program, she founded her own
nonprofit organization to bring virtual classes to migrant
children.Credit...Guillermo Arias for The New York Times

After growing up in dangerous places, and experiencing trauma during
their journey to the border, migrant children often develop permanent
insecurities and have trouble relating to the world, said Dr. Casas.
They are also at a
\href{https://oxfordre.com/publichealth/view/10.1093/acrefore/9780190632366.001.0001/acrefore-9780190632366-e-12}{higher
risk of developing mental health disorders} such as post-traumatic
stress disorder.

Schooling and a sense of routine gains a deeper meaning for them,
according to Dr. Casas, who has treated refugee children for more than
20 years. It
\href{https://onlinelibrary.wiley.com/doi/10.1002/9780470669280.ch12}{diminishes
their anxiety} by providing a safe environment where they can focus on
useful knowledge, instead of the wretched atmosphere that surrounds
them, he said.

On a cold December morning last year, we boarded the classroom bus at El
Chaparral port of entry and rode with two volunteers scheduled to teach
that day. People stared at the rainbow school bus in the middle of a
stream of drab cars on the Tijuana streets. As soon as we arrived at the
shelter, a dozen children came running to greet us, hugging our legs and
jumping around smiling. Then they sat down to paint, everything from
random finger painting to depictions of their travel through the desert.

Zaida Guillén, the director of the Embajadores de Jesús shelter, said
the classes changed the kids' demeanor and allowed them to blossom.
``The children started to integrate more, were more respectful and
started doing teamwork,'' she said.

The mobile school seemed to take their minds away from their ordeals,
said Dulce García, an immigration attorney in San Diego and the
executive director of Border Angels. ``They have the space to be kids
again, to do homework and to talk about their situation with an
expert,'' she said.

Image

Migrant children play at the Embajadores de Jesús Church and shelter,
just three miles south of the U.S.-Mexico border, in
Tijuana.Credit...Guillermo Arias for The New York Times

\hypertarget{an-already-difficult-job-becomes-nearly-impossible}{%
\subsection{An already difficult job becomes nearly
impossible}\label{an-already-difficult-job-becomes-nearly-impossible}}

After eight months, the school bus project was running smoothly. The
children were used to the schedule, trusted the volunteers (who also
taught math and English) and missed them when they couldn't show up.
``They already saw us as part of their lives,'' Rincón Cortés said.

But then, in March, both countries closed their borders and governments
issued stay-at-home orders because of the pandemic. The director of the
School Box Project told Rincón Cortés that they couldn't safely continue
providing classes and ended their programs worldwide. Rincón Cortés took
the children to the movies as a goodbye field trip, and then she and the
volunteers were left without a bus to continue their teaching.

Morales Becerra's children, along with the other 75 kids at the three
shelters, were suddenly adrift, in lockdown, while their parents learned
their court appointments to apply for asylum
\href{https://www.nytimes3xbfgragh.onion/2020/03/17/us/politics/trump-coronavirus-mexican-border.html}{would
be delayed} because of the coronavirus. Or worse: that they could be
forced
to\href{https://www.nytimes3xbfgragh.onion/2020/03/17/world/americas/immigration-guatemala-us-asylum.html}{return
to the violence they were fleeing}.

As contributions and aid dwindled during the next two months, the
children at Embajadores de Jesús shelter were desperate, stressed and
bored without lessons. ``All the aid stopped coming. The doctors,
donations, the psychologist \ldots{} everything,'' Morales Becerra said.

Her eldest son, 12-year-old Jesús, kept a copy of ``Harry Potter and the
Sorcerer's Stone,'' which tells the adventures of a young wizard.
``Since I had nothing to do, I would finish it and read it again, and
again, and again,'' he said.

Image

The virtual classes, set up by the nonprofit organization International
Activist Youth, have brought meaning and structure to the lives of the
migrant children, many of whom have not seen the inside of a classroom
since leaving their homes.~Credit...Guillermo Arias for The New York
Times

\hypertarget{teachers-get-creative}{%
\subsection{Teachers get creative}\label{teachers-get-creative}}

Back in Michoacán, Morales Becerra had been a single mom who worked two
jobs. Now, at the shelter during lockdown, she languished in depression.
``I'm not used to doing nothing, I always have to be active,'' she said.
``I was desperate.''

\href{https://www.nytimes3xbfgragh.onion/spotlight/schools-reopening?action=click\&pgtype=Article\&state=default\&region=MAIN_CONTENT_3\&context=storylines_keepup}{}

\hypertarget{school-reopenings-}{%
\subsubsection{School Reopenings ›}\label{school-reopenings-}}

\hypertarget{back-to-school}{%
\paragraph{Back to School}\label{back-to-school}}

Updated Sept. 11, 2020

The latest on how schools are reopening amid the pandemic.

\begin{itemize}
\item
  \begin{itemize}
  \tightlist
  \item
    School officials in Des Moines are refusing to hold in-person
    classes,
    \href{https://www.nytimes3xbfgragh.onion/2020/09/10/us/des-moines-school-opening-coronavirus.html?action=click\&pgtype=Article\&state=default\&region=MAIN_CONTENT_3\&context=storylines_keepup}{despite
    an order from Iowa's governor and a judge's ruling}, risking school
    funding and their jobs because they think it's unsafe.
  \item
    The University of Illinois at Urbana-Champaign had one of the most
    comprehensive plans by a major college to keep the virus under
    control. But it
    \href{https://www.nytimes3xbfgragh.onion/2020/09/10/health/university-illinois-covid.html?action=click\&pgtype=Article\&state=default\&region=MAIN_CONTENT_3\&context=storylines_keepup}{failed
    to account for students partying}.
  \item
    College students are
    \href{https://www.nytimes3xbfgragh.onion/2020/09/10/technology/coronavirus-quarantines-college.html?action=click\&pgtype=Article\&state=default\&region=MAIN_CONTENT_3\&context=storylines_keepup}{using
    apps to shame their schools}~into better coronavirus plans.
  \item
    For some families, the pandemic
    \href{https://www.nytimes3xbfgragh.onion/2020/09/10/parenting/family-second-language-coronavirus.html?action=click\&pgtype=Article\&state=default\&region=MAIN_CONTENT_3\&context=storylines_keepup}{has
    meant a return to their native languages}.
  \end{itemize}
\end{itemize}

When she realized her youngest son, 5-year-old Axel, couldn't remember
most of what he learned at day care the year before, she asked Guillén
if they could start informal classes for the little ones, and soon she
was teaching math and reading to the shelter's youngest occupants.

Meanwhile, on the other side of town, Rincón Cortés was forming her own
plan to continue teaching. For her, it was more than just offering
classes to kids. She wanted to make them feel that someone was looking
out for them, she said. ``That they mattered.''

She founded her own nonprofit organization, called
\href{https://www.internationalactivistyouth.com/}{International
Activist Youth}, and recruited other college students to help teach. But
it was obvious that distance learning was the only safe way to reach the
kids. Resorting to online methods meant that they had to set up a
reliable internet service and computers at the shelter. A \$500 donation
helped them jumpstart the new project.

By July they had an internet connection at the shelter, and brought
projectors, speakers, chairs and other donated materials for the
lessons. Rincón Cortés also had to coach the volunteer teachers how to
interact with migrant children. Small details, such as learning a
child's name or actively acknowledging their work, gives kids a sense of
self-confidence and dignity.

In mid-July, they started teaching. Rincón Cortés and her team of 14
volunteers now provide more lessons than they could with the school bus.
Online math, English, reading and art classes take up most of the kids'
days. ``My children have already warned me that I will not see them
during the day because they have so many activities,'' said Morales
Becerra, laughing.

Although her son Jesús misses in-person interaction with his teachers,
he enjoys having more lessons. There are
\href{https://www.nytimes3xbfgragh.onion/2020/05/20/nyregion/coronavirus-students-schools.html}{other
silver linings} as well. ``I feel better because if the teachers were
here I would be more embarrassed,'' said Jesús, who has always been shy.
Now that the lessons are online, he participates more.

His younger brother Axel is also busy with classes. ``I'm beginning to
learn how to read,'' he said. ``I can read: `Mamá me ama' (`Mom loves
me').''

Both children continue dreaming about their future. While Jesús wants to
become a marine biologist or an architect, Axel is torn between becoming
a policeman, soldier or pizza-maker.

Image

Two students laugh together during a virtual class at the church and
migrant shelter in Tijuana, Baja California State, in August.
Credit...Guillermo Arias for The New York Times

The virtual classes also include lessons on basic international
\href{https://www.unicef.org/child-rights-convention/convention-text-childrens-version}{children's
rights}, such as the right to have a safe home, to be protected against
violence or to have an education. ``We focus on one right per lesson,''
said Rincón Cortés. This helps prepare both kids and parents to
recognize abuses and violence. The new program will also help families
get in touch with counselors and organizations for legal or
psychological advice.

Even though Mexico and the U.S. have started to open up after lockdown,
Rincón Cortés is planning to continue with the virtual classes. Morales
Becerra said she and many other parents are finding stability and a
sense of hope, although her goal is still to eventually cross the border
after court appointments resume.

``I have many plans,'' she said. ``I want to go back to school, since I
only have a junior high-school diploma, and I hope this will allow me to
give my kids a better chance in life.''

\begin{center}\rule{0.5\linewidth}{\linethickness}\end{center}

Myriam Vidal Valero is Mexican journalist who covers health and science.
She is a member of the Mexican Network of Science Journalists.

Rodrigo Pérez Ortega is a journalist based in Washington D.C. who covers
health and science.

\emph{Reporting for this story was supported by the Rosalynn Carter
Fellowship for Mental Health Journalism.}

Advertisement

\protect\hyperlink{after-bottom}{Continue reading the main story}

\hypertarget{site-index}{%
\subsection{Site Index}\label{site-index}}

\hypertarget{site-information-navigation}{%
\subsection{Site Information
Navigation}\label{site-information-navigation}}

\begin{itemize}
\tightlist
\item
  \href{https://help.nytimes3xbfgragh.onion/hc/en-us/articles/115014792127-Copyright-notice}{©~2020~The
  New York Times Company}
\end{itemize}

\begin{itemize}
\tightlist
\item
  \href{https://www.nytco.com/}{NYTCo}
\item
  \href{https://help.nytimes3xbfgragh.onion/hc/en-us/articles/115015385887-Contact-Us}{Contact
  Us}
\item
  \href{https://www.nytco.com/careers/}{Work with us}
\item
  \href{https://nytmediakit.com/}{Advertise}
\item
  \href{http://www.tbrandstudio.com/}{T Brand Studio}
\item
  \href{https://www.nytimes3xbfgragh.onion/privacy/cookie-policy\#how-do-i-manage-trackers}{Your
  Ad Choices}
\item
  \href{https://www.nytimes3xbfgragh.onion/privacy}{Privacy}
\item
  \href{https://help.nytimes3xbfgragh.onion/hc/en-us/articles/115014893428-Terms-of-service}{Terms
  of Service}
\item
  \href{https://help.nytimes3xbfgragh.onion/hc/en-us/articles/115014893968-Terms-of-sale}{Terms
  of Sale}
\item
  \href{https://spiderbites.nytimes3xbfgragh.onion}{Site Map}
\item
  \href{https://help.nytimes3xbfgragh.onion/hc/en-us}{Help}
\item
  \href{https://www.nytimes3xbfgragh.onion/subscription?campaignId=37WXW}{Subscriptions}
\end{itemize}
