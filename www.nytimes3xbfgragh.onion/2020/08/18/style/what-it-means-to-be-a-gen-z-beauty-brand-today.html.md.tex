Sections

SEARCH

\protect\hyperlink{site-content}{Skip to
content}\protect\hyperlink{site-index}{Skip to site index}

\href{https://www.nytimes3xbfgragh.onion/section/style}{Style}

\href{https://myaccount.nytimes3xbfgragh.onion/auth/login?response_type=cookie\&client_id=vi}{}

\href{https://www.nytimes3xbfgragh.onion/section/todayspaper}{Today's
Paper}

\href{/section/style}{Style}\textbar{}What It Means to Be a Gen Z Beauty
Brand Today

\begin{itemize}
\item
\item
\item
\item
\item
\item
\end{itemize}

Advertisement

\protect\hyperlink{after-top}{Continue reading the main story}

Supported by

\protect\hyperlink{after-sponsor}{Continue reading the main story}

Skin Deep

\hypertarget{what-it-means-to-be-a-gen-z-beauty-brand-today}{%
\section{What It Means to Be a Gen Z Beauty Brand
Today}\label{what-it-means-to-be-a-gen-z-beauty-brand-today}}

A decade ago, your lipstick brand wasn't expected to comment on social
justice. Now, customers demand it, especially the teens and adults under
25 that make up Gen Z.

\includegraphics{https://static01.graylady3jvrrxbe.onion/images/2020/08/20/fashion/NTY_GENZ_LEADART/NTY_GENZ_LEADART-threeByTwoLargeAt2X-v2.jpg}

By Rachel Strugatz

\begin{itemize}
\item
  Published Aug. 18, 2020Updated Aug. 19, 2020
\item
  \begin{itemize}
  \item
  \item
  \item
  \item
  \item
  \item
  \end{itemize}
\end{itemize}

Tiffany Zhong, 23, self-describes as the ``Gen Z Whisperer.'' She names
her companies after emoji, including Zebra IQ, a platform that helps
content creators make money from their audiences, and Pineapple Capital,
an early stage ``mini fund'' that invests in consumer brands.

Typically, Ms. Zhong likes to get in on companies early. At 21, she was
the first investor in Kinship, a skin care label that's about to hit
shelves at hundreds of Ulta Beauty stores.

All six products in the collection, including Insta Swipe alpha hydroxy
acid wipes and Pimple Potion acne treatment, were created for Generation
Z. For more than two years, the Kinship founders Christin Powell and
Alison Haljun worked with a Kinship Circle of 20 Gen Z-ers to develop
products.

Today, the Kinship Circle is made up of 125 young people who give input
on logo, color, typography, packaging design and beta product testing.
Self Reflect sunscreen, with its airy, whipped texture and vanilla
scent, was formerly Golden Milk Latte, a heavier SPF with notes of
turmeric and cardamom. The original version was scrapped, Ms. Powell
said, because the Kinship Circle found it too ``spicy'' and the texture
lacking.

Gen Z-ers ``care a lot about seeing what's happening behind the scenes,
getting a transparent look into things,'' Ms. Zhong said. Equally
critical to her is that a beauty line is eco-friendly, vegan, cruelty
free and ``stands up for what's right.''

So what does it mean to be a beauty brand for Gen Z today?

A company can't just sell skin care, cosmetics, hair care or perfume.
Good product matters, but what matters more is standing for something,
whether it's being cruelty free (E.L.F. Cosmetics) or simply being the
best version of yourself (Glossier). It can't politely sit out seismic
cultural moments. There's an expectation of comment on sustainability,
social justice, police reform and, soon, a presidential election.

Customers demand it, especially those under 25 that make up Gen Z, also
known as the activist generation.

````This is going to be the group that's driving spending and decisions
for many years to come,'' said Mary Dillon, the chief executive of Ulta.

``They're super-influential. All you have to do is look at the racial
injustice discussion and dialogue we've had in the last few months. Gen
Z is leading the way.''

\hypertarget{where-does-gen-z-buy-beauty}{%
\subsection{Where Does Gen Z Buy
Beauty?}\label{where-does-gen-z-buy-beauty}}

Ulta has become the go-to purveyor of teenage beauty, according to data
from the investment bank Piper Sandler. Of the 5,200 teenagers surveyed
by the firm, 39 percent said Ulta was their top destination for beauty
purchases, followed by Sephora with 24 percent. Spring 2019 was the
first time Ulta was crowned the favorite beauty destination for
teenagers.

How did a store once associated with cheap drugstore makeup and suburban
strip malls become so popular with this demographic?

Because it's just about the only place you can find Gen Z staples like
Morphe, Kylie Cosmetics and Sugar Rush, Tarte's little sister line, IRL.
Ulta also sells drugstore staples from E.L.F. Cosmetics and Maybelline
next to more expensive brands like Anastasia Beverly Hills, an alluring
proposition to Gen Z-ers who are value conscious and mix high and low.

Amaya Smith, a founder of \href{https://www.brownbeautyco-op.com/}{Brown
Beauty Co-op}, a store that sells and incubates Black-owned brands in
Washington, D.C., has had success with brands that home in on a trend or
practice that's specific to Gen Z, like Baby Tress, an edge styler.

``My niece is generally not going out of the house without her edges
styled,'' Ms. Smith said. The edge styler, she said, is an item that
resonates with teenagers because it comes in pastel colors, it's
affordable (\$15) and is a multifunctional. (It's a brush and comb in
one with a pointed tip for parting.) ``Everyone was using toothbrushes
before,'' she said.

\includegraphics{https://static01.graylady3jvrrxbe.onion/images/2020/08/20/fashion/19genz-beauty-spot1/NYT_GENZ_BABYTRESS-mediumSquareAt3X.jpg}

\hypertarget{expensive-doesnt-mean-better}{%
\subsection{Expensive Doesn't Mean
Better}\label{expensive-doesnt-mean-better}}

This demographic's prom was canceled, they don't know if they're going
to college this month, and chances are they grew up with at least one
parent out of work during the financial recession of 2008. Gen Z-ers
pore over ingredient labels and reviews online; they want to know what
they're putting on their skin and what it's doing for them.

``If something is cheap, it's not necessarily bad,'' said Rogelio
Munoz-Franco, 15. ``If something is expensive, it's not necessarily
good.''

This sentiment was unanimous among the 13 people ages 12 to 23 who were
interviewed for this story. Many are regular users of the Inkey List and
the Ordinary, which sells serums for about \$7, or follow
recommendations from the Gen Z skin care oracle Hyram Yarbro, the
creator of the \href{https://www.skincarebyhyram.com/}{Skin Care by
Hyram}.

Mr. Yarbro, 24, has had a meteoric rise. He went from having just under
100,000 TikTok followers in March to having more than five million
today. He credits the spike in skin care interest during quarantine and
his focus on reviewing cheaper products, which he said are not much
different, in terms of ingredients, than ``extreme luxury'' items.

``Ingredients don't lie,'' said Mr. Yarbro, who gained the trust of his
young viewers through honest reviews. Last year he told viewers he had
reservations about Kylie Skin's exfoliating scrub because it was made
with walnuts, which can be abrasive. He doesn't shy away from being
critical when reviewing items from sponsored brands.

``There were a few products that I went pretty hard on,'' Mr. Yarbro
said of last year's paid partnership with the Korean skin care label
Purito.

\hypertarget{no-more-full-beats}{%
\subsection{No More Full Beats}\label{no-more-full-beats}}

Gen Z-ers possess a distinct set of beauty and grooming habits,
attitudes and buying patterns. Many gravitate toward an unvarnished
aesthetic from brands that are progressive in the imagery they use.
Teenagers don't want a ``full beat,'' or the overdone Instagram face
that defined a generation of millennial influencers who were one
cosmetic procedure away from looking like a Kardashian.

``Their intellectual take on beauty is divorced from the millennial
idea, which is Instagram filter, having the perfect lip and looking like
an idealized beauty,'' said Lucie Greene, the founder of Light Years, a
consultancy.

She pointed to the singer Billie Eilish, crediting the 18-year-old with
inspiring a ``competitive creativity'' --- the antithesis of airbrushing
or perfection.

``It's not about the male gaze,'' Ms. Greene said. ``It's her expression
of herself.''

This is why the very millennial-pink brand
\href{https://www.skincarebyhyram.com/}{Glossier} is such a hit with
this crowd.

When building her company, Emily Weiss, the Glossier founder and chief
executive, steered clear of projecting a singular beauty ideal. Instead
of coaxing customers to look like a model or celebrity, Ms. Weiss
encouraged them to be ``more like you'' (but maybe a slightly dewier
version).

``One of the appeals of Glossier is that it's all supposed to enhance
your natural beauty,'' said Abby Kwok, 16.

Ms. Zhong, the Gen Z investor, said: ``Gen Z is like, `How can I be the
realest I can be?' In fact, the messier it looks, the more real they
are, the more real they seem.''

\hypertarget{think-about-beauty-differently}{%
\subsection{Think About Beauty
Differently}\label{think-about-beauty-differently}}

Groundbreaking companies are always contrarian, but what's considered
contrarian is ever evolving. Probably no one would consider minimal-pink
branding and fresh-faced ``no-makeup makeup'' a groundbreaking idea, but
what Ms. Weiss, 35, set out to do nearly six years ago --- create a
beauty company sold entirely online --- was unheard-of. So was the way
Glossier talked to and treated its consumers, communicating and
operating as a ``beauty BFF.''

``What's interesting about Glossier is we just hold up a mirror, and
we're like, `We're here to make you look more like you,''' Ms. Weiss
said.

Olamide Olowe, 23, the co-founder and chief executive of
\href{https://www.mytopicals.com/}{Topicals,} a new skin care line
designed to treat conditions like eczema and hyperpigmentation, has a
similar ethos.

Ms. Olowe raised more than \$2 million in venture capital funding, with
investors that include the chief executives of Allbirds, Warby Parker
and Casper; Issa Rae; Hannah Bronfman; and Bozoma Saint John, the chief
marketing officer of Netflix.

``The brands of yesterday focused on aspiration in making you want to be
somebody,'' said Ms. Olowe, who is Black. ``Gen Z brands focus on
celebrating you in the way that you are.''

In the 1990s and early aughts, MAC Cosmetics was centered around
diversity, self-expression and gender and sexual fluidity --- radical
ideas at the time that deviated from the beauty norms promoted by
mainstream brands. The makeup artist brand was thriving with the drag
scene, while millennials were buying
\href{https://www.clinique.com/thewink/cult-classic/3-step-skin-care-system}{Clinique's
classic ``3-Step Skin Care System''} (plus a free gift with purchase) at
department store counters.

But as its consumers aged and department store beauty counters were
replaced by Sephora and Ulta, MAC struggled to be relevant. Last year,
Drew Elliott was brought in as the global creative director to make the
brand cool again. (He is the person responsible for the
break-the-internet issue of Paper ** magazine with Kim Kardashian West's
glossy bare backside on the cover.)

One of his first projects was the
\href{https://www.maccosmetics.com/mac-underground}{MAC Underground}
collection that came out during Pride month, the kickoff of a series of
limited-edition drops. Each of the 1,000 units, which are numbered, sold
out in 55 minutes, according to the company.

``They're moving at the speed of the internet, not at the speed of a
supply chain,'' Mr. Elliott, 39, said of Gen Z. ``To just get something
out into the market within three months --- for a makeup brand, that's
bonkers.''

In its heyday MAC was the embodiment of Gen Z values today. Except to
this generation, there is nothing radical about self-expression or fluid
sexuality, identity or gender. Millennials talk about being gender fluid
and accepting, but Gen Z is the first generation to live the promise of
those values, said Shireen Jiwan, the founder and chief executive of
Sleuth Brand Consulting,

``Genuinely, off the tip of their tongues, roll the pronouns anybody
asks them to use, and there's no agita about it,'' she said. ``They were
raised in a world where everybody is already a mix of 10 different
races. Nobody asks, `What are you?'''

\includegraphics{https://static01.graylady3jvrrxbe.onion/images/2020/08/20/fashion/19skin-genz-beauty-spot2/NYT_GENZ_ELFTIKTOK2-mediumSquareAt3X.jpg}

\hypertarget{invest-in-tiktok-and-tiktok-creators}{%
\subsection{Invest in TikTok and TikTok
Creators}\label{invest-in-tiktok-and-tiktok-creators}}

Kory Marchisotto, the chief marketing officer of E.L.F. Cosmetics, was
hired last year to modernize the struggling (at the time) makeup brand.
\href{https://www.elfcosmetics.com/}{The company} has always been vegan
and cruelty free, with most of its products costing around \$5 ---
important values to Gen Z-ers --- but it wasn't publicized enough. Nor
was the fact that ``E.L.F.'' stood for ``Eyes Lips Face.''

The brand invested heavily in a TikTok campaign to communicate this to
Gen Z users, many of whom were born at or around the same time E.L.F.
Cosmetics was started in 2004. The Eyes Lips Face Challenge was
introduced on TikTok, and to date, the \#eyeslipsface hashtag has six
billion views, and there are 4.5 million pieces of user-generated
content with the hashtag.

\href{https://www.morphe.com/}{Morphe}, too, is turning its attention
from traditional YouTubers to TikTok stars.

The brand had success with influencer collaborations with YouTubers like
James Charles and Jeffree Star --- it has since severed ties with Mr.
Star because of racist comments that surfaced online --- but knew it
needed to do something different to reach Gen Z.

The company just released
\href{https://www.morphe.com/pages/morphe-2-signup}{Morphe 2}, with
Charli D'Amelio, the most followed account on TikTok, and her older
sister, Dixie, as the faces of the new label. Instead of Morphe's boldly
colored signature makeup and black packaging, Morphe 2 is fresher,
lighter and focused on ``natural beauty.'' Its packaging is all white.

``I don't really love the look of beauty influencers like James Charles,
that sort of makeup that's super-heavy with really bright eye shadow and
contouring,'' said Anya Dua, 16, the founder of
\href{https://genzidentitylab.com/}{Gen Z Identity Lab}, an online
platform that fosters discussions about identity.

Cecilia Granda-Scott, 16, a classmate of Ms. Dua's, said the dramatic
makeup popularized by Ms. Charles and Mr. Star clashes with Gen Z's
``minimalist'' views.

``It's very telling that Morphe's sub-brand for Gen Z is all about being
more natural,'' Ms. Granda-Scott said**.**

Advertisement

\protect\hyperlink{after-bottom}{Continue reading the main story}

\hypertarget{site-index}{%
\subsection{Site Index}\label{site-index}}

\hypertarget{site-information-navigation}{%
\subsection{Site Information
Navigation}\label{site-information-navigation}}

\begin{itemize}
\tightlist
\item
  \href{https://help.nytimes3xbfgragh.onion/hc/en-us/articles/115014792127-Copyright-notice}{©~2020~The
  New York Times Company}
\end{itemize}

\begin{itemize}
\tightlist
\item
  \href{https://www.nytco.com/}{NYTCo}
\item
  \href{https://help.nytimes3xbfgragh.onion/hc/en-us/articles/115015385887-Contact-Us}{Contact
  Us}
\item
  \href{https://www.nytco.com/careers/}{Work with us}
\item
  \href{https://nytmediakit.com/}{Advertise}
\item
  \href{http://www.tbrandstudio.com/}{T Brand Studio}
\item
  \href{https://www.nytimes3xbfgragh.onion/privacy/cookie-policy\#how-do-i-manage-trackers}{Your
  Ad Choices}
\item
  \href{https://www.nytimes3xbfgragh.onion/privacy}{Privacy}
\item
  \href{https://help.nytimes3xbfgragh.onion/hc/en-us/articles/115014893428-Terms-of-service}{Terms
  of Service}
\item
  \href{https://help.nytimes3xbfgragh.onion/hc/en-us/articles/115014893968-Terms-of-sale}{Terms
  of Sale}
\item
  \href{https://spiderbites.nytimes3xbfgragh.onion}{Site Map}
\item
  \href{https://help.nytimes3xbfgragh.onion/hc/en-us}{Help}
\item
  \href{https://www.nytimes3xbfgragh.onion/subscription?campaignId=37WXW}{Subscriptions}
\end{itemize}
