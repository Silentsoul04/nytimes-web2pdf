Sections

SEARCH

\protect\hyperlink{site-content}{Skip to
content}\protect\hyperlink{site-index}{Skip to site index}

\href{https://www.nytimes3xbfgragh.onion/section/politics}{Politics}

\href{https://myaccount.nytimes3xbfgragh.onion/auth/login?response_type=cookie\&client_id=vi}{}

\href{https://www.nytimes3xbfgragh.onion/section/todayspaper}{Today's
Paper}

\href{/section/politics}{Politics}\textbar{}Postal Service Suspends
Changes After Outcry Over Delivery Slowdown

\url{https://nyti.ms/3kZQueO}

\begin{itemize}
\item
\item
\item
\item
\item
\end{itemize}

\begin{itemize}
\item
  \href{https://www.nytimes3xbfgragh.onion/2020/09/12/us/politics/biden-trump-poll-wisconsin-minnesota.html?action=click\&pgtype=Article\&state=default\&region=TOP_BANNER\&context=storylines_menu}{New
  York Times Poll}
\item
  \href{https://www.nytimes3xbfgragh.onion/interactive/2020/us/elections/election-states-biden-trump.html?action=click\&pgtype=Article\&state=default\&region=TOP_BANNER\&context=storylines_menu}{Paths
  to 270}
\item
  \href{https://www.nytimes3xbfgragh.onion/interactive/2019/us/elections/2020-presidential-election-calendar.html?action=click\&pgtype=Article\&state=default\&region=TOP_BANNER\&context=storylines_menu}{Voting
  Deadlines}
\item
  \href{https://www.nytimes3xbfgragh.onion/interactive/2020/08/31/us/politics/vote-by-mail-deadlines.html?action=click\&pgtype=Article\&state=default\&region=TOP_BANNER\&context=storylines_menu}{Voting
  by Mail}
\item
  \href{https://www.nytimes3xbfgragh.onion/newsletters/politics?action=click\&pgtype=Article\&state=default\&region=TOP_BANNER\&context=storylines_menu}{Politics
  Newsletter}
\end{itemize}

Advertisement

\protect\hyperlink{after-top}{Continue reading the main story}

Supported by

\protect\hyperlink{after-sponsor}{Continue reading the main story}

\hypertarget{postal-service-suspends-changes-after-outcry-over-delivery-slowdown}{%
\section{Postal Service Suspends Changes After Outcry Over Delivery
Slowdown}\label{postal-service-suspends-changes-after-outcry-over-delivery-slowdown}}

Policy changes by the postmaster general prompted allegations that the
Trump administration was trying to disenfranchise voters before the 2020
election.

\includegraphics{https://static01.graylady3jvrrxbe.onion/images/2020/08/18/us/politics/18dc-postal/merlin_175860084_a0cfe2af-7f6a-4a13-b305-bd7d8473dab0-articleLarge.jpg?quality=75\&auto=webp\&disable=upscale}

By \href{https://www.nytimes3xbfgragh.onion/by/emily-cochrane}{Emily
Cochrane},
\href{https://www.nytimes3xbfgragh.onion/by/hailey-fuchs}{Hailey Fuchs},
\href{https://www.nytimes3xbfgragh.onion/by/kenneth-p-vogel}{Kenneth P.
Vogel} and
\href{https://www.nytimes3xbfgragh.onion/by/jessica-silver-greenberg}{Jessica
Silver-Greenberg}

\begin{itemize}
\item
  Aug. 18, 2020
\item
  \begin{itemize}
  \item
  \item
  \item
  \item
  \item
  \end{itemize}
\end{itemize}

WASHINGTON --- Postmaster General
\href{https://www.nytimes3xbfgragh.onion/article/general-louis-dejoy-postmaster.html}{Louis
DeJoy}, facing intense backlash over cost-cutting moves that Democrats,
state attorneys general and civil rights groups warn could jeopardize
mail-in voting, said on Tuesday that the
\href{https://www.nytimes3xbfgragh.onion/2020/08/22/business/economy/dejoy-postmaster-general-trump-mnuchin.html}{Postal
Service} would suspend those operational changes until after the 2020
election.

The measures, which included eliminating overtime for mail carriers,
reducing post office hours and removing postal boxes, have been faulted
for
\href{https://www.nytimes3xbfgragh.onion/2020/08/21/us/postal-service-mail-rural.html}{slowing
mail delivery} and criticized as an attempt to disenfranchise voters
seeking to vote safely during the coronavirus pandemic.

Mr. DeJoy, a
\href{https://www.nytimes3xbfgragh.onion/2020/05/07/us/politics/postmaster-general-louis-dejoy.html}{major
donor to President Trump} who was tapped in May to run the Postal
Service, said in
\href{https://about.usps.com/newsroom/national-releases/2020/0818-postmaster-general-louis-dejoy-statement.htm}{a
statement} that ``to avoid even the appearance of any impact on election
mail'' he was suspending changes ``that have been raised as areas of
concern as the nation prepares to hold an election in the midst of a
devastating pandemic.''

Mr. DeJoy said retail hours at the post office would not change, no mail
processing facilities would be closed, and overtime would continue to be
approved ``as needed.''

It was unclear, however, whether the agency would reverse measures
already put in place across the country that union officials and workers
say have inflicted deep damage to the Postal Service. That includes the
removal of hundreds of mail-sorting machines, according to a June 17
letter sent from the Postal Service to the American Postal Workers
Union. Some of those machines have already been destroyed, union
officials and workers said.

\includegraphics{https://static01.graylady3jvrrxbe.onion/images/2017/01/29/podcasts/the-daily-album-art/the-daily-album-art-articleInline-v2.jpg?quality=75\&auto=webp\&disable=upscale}

\hypertarget{listen-to-the-daily-the-president-the-postal-service-and-the-election}{%
\subsubsection{Listen to `The Daily': The President, the Postal Service
and the
Election}\label{listen-to-the-daily-the-president-the-postal-service-and-the-election}}

Recent cuts have raised a question: Is President Trump deliberately
slowing the mail to help his chances in the election?

transcript

Back to The Daily

bars

0:00/26:35

-26:35

transcript

\hypertarget{listen-to-the-daily-the-president-the-postal-service-and-the-election-1}{%
\subsection{Listen to `The Daily': The President, the Postal Service and
the
Election}\label{listen-to-the-daily-the-president-the-postal-service-and-the-election-1}}

\hypertarget{hosted-by-michael-barbaro-produced-by-jessica-cheung-alexandra-leigh-young-and-asthaa-chaturvedi-and-edited-by-mj-davis-lin-and-lisa-tobin}{%
\subsubsection{Hosted by Michael Barbaro, produced by Jessica Cheung,
Alexandra Leigh Young and Asthaa Chaturvedi, and edited by M.J. Davis
Lin and Lisa
Tobin}\label{hosted-by-michael-barbaro-produced-by-jessica-cheung-alexandra-leigh-young-and-asthaa-chaturvedi-and-edited-by-mj-davis-lin-and-lisa-tobin}}

\hypertarget{recent-cuts-have-raised-a-question-is-president-trump-deliberately-slowing-the-mail-to-help-his-chances-in-the-election}{%
\paragraph{Recent cuts have raised a question: Is President Trump
deliberately slowing the mail to help his chances in the
election?}\label{recent-cuts-have-raised-a-question-is-president-trump-deliberately-slowing-the-mail-to-help-his-chances-in-the-election}}

\begin{itemize}
\item
  michael barbaro\\
  From The New York Times, I'm Michael Barbaro. This is ``The Daily.

  Today: The president, the postal service and the election. My
  colleague Luke Broadwater on what's actually going on.
\item
  {[}music{]}\\
  It's Wednesday, August 19.

  Luke, there's a theory floating out there about the post office, the
  president and the upcoming election, and I wonder if you can just
  explain it. What is this theory?
\item
  luke broadwater\\
  So the theory, I guess the short version of it is this: President
  Trump has installed a mega donor and close ally as the postmaster
  general and has set him about on a course to cut the post office. And
  in doing so, wreak havoc onto mail-in voting, thereby helping
  President Trump be reelected.
\item
  michael barbaro\\
  And, Luke, where does this theory come from?
\item
  luke broadwater\\
  Well, we can take it all the way back to about the turn of the
  century, when mail use peaks in America right around 2001. And since
  that time, we've seen about a 50 percent reduction in the mailing of
  first class mail. To accommodate this, a series of postmaster generals
  have approved cuts and reductions to things like mailboxes, to things
  like sorting machines in attempt to shrink the agency along with the
  lower volume of mail.
\item
  michael barbaro\\
  Got it.
\item
  luke broadwater\\
  Then you have this pandemic come in, and you have a ton of people now
  and a ton of states looking at mail-in balloting. Something like three
  out of four Americans may be eligible to vote by mail this year. And
  so you have this whole new demand on the post office. Then, on top of
  all these problems, a new postmaster general is installed.
\item
  archived recording\\
  His postmaster general is a Republican mega donor, who 85 days from
  the election by the way, decided the time was right for a chaotic and
  sweeping overhaul.
\end{itemize}

luke broadwater

His name is Louis DeJoy. He has never worked for the post office before.

\begin{itemize}
\tightlist
\item
  archived recording\\
  Critics of the president say he's trying to sabotage the postal
  service ahead of the election. It was only a month and a half ago
  Trump identified mail-in voting as the biggest threat to a second
  term.
\end{itemize}

luke broadwater

He reassigns more than 20 executives in the post office. He immediately
limits overtime for the postal workers. In the union's view, he speeds
up the removal of mail-sorting machines. He puts in stringent rules that
limit how many times a mail carrier can make a run in a day. And the
result of all these changes, where some people were seeing slower mail,
becomes in many people's view a crisis.

People all across the country are calling their senators, calling their
Congress people. They are saying they haven't gotten mail for weeks. I
talked with one congressperson in Philadelphia who said in a normal
July, he gets something like 10 to 20 complaints about the post office.
And this year in July, he got more than 300, almost 400. And so people
who were already worried about DeJoy and what he was doing at the post
office, their fears were exacerbated when last week ---

\begin{itemize}
\tightlist
\item
  archived recording (donald trump)\\
  But two of the items are the post office and the \$3.5 billion for
  mail-in voting. Now, if we don't make a deal, that means they don't
  get the money. That means they can't have universal mail-in voting.
  They just can't have it.
\end{itemize}

luke broadwater

President Trump came out and basically admitted that he doesn't want to
fund certain aspects of the post office, because they might contribute
to mail-in voting.

{[}music{]}

And when he said that, it was like alarm bells rang across the country
for Democrats.

michael barbaro

What is the president objecting to here exactly? What is his problem
with this funding, and what is this funding?

luke broadwater

So this all comes out of the fight over the latest round of stimulus
legislation to help Americans suffering from the coronavirus and to help
the American economy. In the Democrats' proposal, they have \$25 billion
dollars to help the post office, and they have \$3.6 billion to help
states with their elections. There is a provision in the Democrats' bill
for universal mail-in voting, but that is not directly connected to the
money for the post office.

It appears that President Trump has conflated these two issues. So he
thinks that by blocking the money for the post office, he's preventing
universal mail-in voting. The truth is that states have already decided
on their own whether or not they're doing universal mail-in voting, and
this money for the post office would not change that.

{[}music{]}

michael barbaro

So in the process of opposing this thing that's not really even in the
Democrats' proposals, he's nevertheless admitting very explicitly that
he wants to find a way to curtail mail-in voting by depriving the postal
service of funding.

luke broadwater

Yes, he says that out loud, and you could hear jaws hitting the floor
around the country.

michael barbaro

Got it.

luke broadwater

But to make matters worse, right around this time, you start seeing
reports come out from different states across the country that they have
received letters from the postal service saying we might not be able to
accommodate mail-in balloting in the final weeks of the election for
your state. And so with Trump's comments, then these letters, and
everything else we know about Postmaster General DeJoy and his
background and ties to the Republican Party, and the cuts he's putting
in place, people around the country start to suspect that there's
sabotage going on.

michael barbaro

Right, so here you're laying out the kind of elements of this theory
that now, I'm sure for many people, especially Democrats, is starting to
sound not so much like a theory, but a kind of reality. And so I'm
curious how people within the Democratic party are reacting exactly?
What are they saying? What are they doing?

luke broadwater

Well, I mean, there was immediate outrage.

\begin{itemize}
\tightlist
\item
  archived recording (protesters)\\
  {[}CHANTING{]}
\end{itemize}

luke broadwater

People started protesting.

\begin{itemize}
\tightlist
\item
  archived recording (protesters)\\
  (SINGING) Mama, mama, can't you see? Mama, mama, can't you see? What
  DeJoy has done to me? What DeJoy has done to me?
\end{itemize}

luke broadwater

People went to Postmaster General DeJoy's house and protested outside
his house.

\begin{itemize}
\tightlist
\item
  archived recording (protesters)\\
  (SINGING) He wants to sabotage the post. He wants to sabotage the
  post. And throw away all of our votes. And throw away all of the
  votes. Mama, mama, can't you see?
\end{itemize}

luke broadwater

Nancy Pelosi, Speaker Pelosi has called the House to come back and pass
emergency legislation to block what Postmaster General DeJoy and
President Trump are doing to the post office in her view. And you see a
hearing on Monday, an emergency hearing, in which Mr. DeJoy will be
called in front of Congress to take tough questions about his role in
all this, and what his plans and intentions are.

michael barbaro

So the president's comments, essentially confirming many people's fears,
have poured a lot of fuel onto this theory. But as plausible as this
theory may sound, is there actual evidence that connects DeJoy's actions
to Donald Trump and shows that he's actively seeking to undermine the
postal service to strengthen his chances of re-election?

luke broadwater

I mean, he has said that. But in terms of actual actions, I don't think
you can say that. To pull that off, you would have to destroy the post
office in key areas, right? You would have to do it in Democratic
strongholds, but not in rural areas. You would have to be very sort of
selective about how you went about cutting and weakening the post
office. And that's not really what we've seen. What we've seen is across
the country we've seen problems with the post office. In fact, a lot of
the complaints that we get are from rural Americans, and we see a lot of
rural Republicans who are very upset about what's going on at the post
office --- G.O.P. senators and Republican secretaries of state.

So Donald Trump believes that mail-in voting helps Democrats, but almost
all the studies we've seen is that it doesn't really help anybody. It
just makes more people vote. Now, maybe he views it as if fewer people
vote, I can win. But there's no real evidence of that either, that sort
of a smaller electorate would benefit him over Joe Biden.

michael barbaro

Well, now that they've seen this reaction to the president's words, and
this theory has taken hold, how is the Trump administration responding?

luke broadwater

Well, they've walked back a number of things that Trump said.

\begin{itemize}
\item
  archived recording\\
  If the Democrats were to give you some of what you want, which you
  articulated in a series of tweets in the last hour, would you be
  willing to accept the \$25 billion dollars for the postal service
  including the \$3.5 billion?
\item
  archived recording (donald trump)\\
  Sure, if they give us what we want. And it's not what I want. It's
  what the American people want.
\end{itemize}

luke broadwater

The president has now said he's open to funding the post office in a way
that he wasn't only a day earlier. His Chief of Staff Mark Meadows has
gone out there and said ---

\begin{itemize}
\tightlist
\item
  archived recording (mark meadows)\\
  I'm all about piecemeal. If we can agree on postal, let's do it.
\end{itemize}

luke broadwater

--- that they would be open to a standalone bill to fund the post
office.

\begin{itemize}
\tightlist
\item
  archived recording (mark meadows)\\
  I've been the one that's advocating for that. Speaker Pelosi is the
  one who said that she won't do anything unless it's a big deal.
\end{itemize}

luke broadwater

They've made a number of pledges to try to tamp down some of the
accusations.

\begin{itemize}
\tightlist
\item
  archived recording (mark meadows)\\
  There's no sorting machines that are going offline between now and the
  election. That's not happening.
\end{itemize}

luke broadwater

They've said they'll stop removing the post boxes. They said they have
authorized overtime for the election. And the postmaster general has
pledged that every ballot will be treated with respect and counted and
sent to the proper place. They are pledging up and down there won't be
any sabotage.

{[}music{]}

michael barbaro

We'll be right back.

So at this point, the Trump administration has more or less said we will
stop doing the things that are fueling this theory. That's now very much
out there that we are trying to damage the postal service to try to win
a second term in the middle of a pandemic. I'm curious what the actual
capacity of the postal service is with the cuts that are already in
place. Can the postal service handle an election in which 100 to 200
million people may use mail-in ballots?

luke broadwater

The short answer is yes. They have more than enough capacity to handle
that volume of mail. Pretty much everybody agrees on that --- election
experts, the postal union, the postmaster general. If you look at a
traditional Christmastime, you have much more mail moving than you're
going to see during this election, even though it is a very heightened
amount of mail-in ballots.

The issue for the post office is not the volume or the capacity. The
issue is the timing. And so they have a concern about last-minute
requests for mail-in ballots. That's really their issue. There's 45
states across the country that allow people to request ballots within
two weeks before the election. Some allow the requests as short as four
days. There are even five states that will send out a mail-in ballot to
a voter if they receive an application by mail, even the day before the
election.

michael barbaro

Wow, that's just not a lot of time for the postal service to get a
ballot out, get a person to fill it out and get it back in time before
the election.

luke broadwater

Yes, exactly. So the postal service is saying, even in a good year, even
a year where there is not this big rush of mail-in ballots and there's
no pandemic, we would have trouble with some of those deadlines. And so
we're asking you in a pandemic year, in a year where 75 percent of
voters are eligible to vote by mail, please push your deadlines back,
and give us two weeks to process these things.

michael barbaro

OK, they're just basically creating a bigger buffer of time to make sure
they can handle all of this capacity. That seems kind of reasonable.

luke broadwater

Yeah, I mean, and I think that if this request had gone out without
Donald Trump's comments, without Postmaster General DeJoy's cuts, they
would have been seen as sort of a reasonable letter to ensure that
everyone's vote actually counted, that the post office can accommodate
some of these deadlines. Well, let's change them to make sure that no
one is under a false impression that they're filling out a ballot that's
going to count when it's not actually going to get there in time.

michael barbaro

So, look, I want to return to the original theory here about the
president, the postal service and the election. You identify the ways in
which the president could, if he wanted to, weaken the postal service in
ways that would advantage him in the election by, for example, going
into a Democratic community and making cuts to the postal service. Does
the president actually have that kind of power? Could he demand changes
to the postal service in the next 70 or 80 days that would actually make
it easier for him to win re-election?

luke broadwater

Well, so he does control the board of governors now for the postal
service, and he controls the postmaster general. And he could give them
some marching orders theoretically. That said, I do think it would be
difficult to carry out without raising the alarm of a large unionized
workforce. If all of the sudden, the entire branch of the Philadelphia
post office was closed down, and people were laid off, we would know
about that. These are not shrinking violets who would just be pushed
around. So he'd either need buy-in from the unions to sabotage the
election --- a bunch of unionized workers --- which seems highly
unlikely. Or we would hear crying foul from every corner of the country
about what he was doing.

michael barbaro

So you're saying it's not very practical as a electoral strategy for the
president. But I wonder if you're getting the sense that the president's
actions so far, and his words when it comes to the postal service and
this election, are creating a lot of doubt about whether mail-in voting
is going to work? And whether the postal service can make mail-in voting
work, and if that is starting to in its own way undermine faith in the
postal service in the minds of voters?

luke broadwater

Absolutely. It's an interesting thing the president's doing, because the
post office has long been one of the most popular functions of
government. I think 91 percent of Americans, Republicans and Democrats
alike, support the post office. And so what the unions believe he's
doing is they say there are three levels of support that they have. One
is the board of governors. They say the president's taken that over. The
second is the postmaster general. They say the president has taken that
over. And the third is the American public, and they still have the
American public on their side. But if the service of the post office is
so eroded, and the confidence in it has degraded so much that their
polling starts to fall, and people don't have confidence in them
anymore, well, now the theory goes it's open for privatization.

michael barbaro

Oh, interesting. And what could be a bigger stage for people to judge
the postal service than an election?

luke broadwater

That's right.

\begin{itemize}
\tightlist
\item
  archived recording (donald trump)\\
  There is the issue of voter fraud. Isn't it amazing the way they say
  there's no voter fraud?
\end{itemize}

luke broadwater

Donald Trump has long sought to undermine American elections with his
rhetoric.

\begin{itemize}
\tightlist
\item
  archived recording (donald trump)\\
  There are 1.8 million dead people that are registered right now to
  vote. And folks, some of them vote. I wonder why. I wonder how that
  happened. They woke up from the dead, and they went and voted.
\end{itemize}

luke broadwater

But now what we're seeing is on the other side.

\begin{itemize}
\tightlist
\item
  archived recording (nancy pelosi)\\
  Within this administration is an attempt to make sure your vote
  doesn't count and doesn't count as cast.
\end{itemize}

luke broadwater

Democrats are worried about a rigged election.

\begin{itemize}
\item
  archived recording (nancy pelosi)\\
  The actions this administration are taking, vis-á-vis our voting
  system, our sacred right to vote are a domestic assault on our
  constitution.
\item
  archived recording (chuck schumer)\\
  Donald Trump is aimed at hurting the elections. He says he wants to
  slow down the mail to hurt the elections and make people doubt the
  results of the election.
\end{itemize}

luke broadwater

So now we're seeing both sides questioning the legitimacy of the
election. And I don't know how that's a good recipe for America and for
confidence in our electoral system.

\begin{itemize}
\tightlist
\item
  archived recording (barack obama)\\
  What we've never seen before is a president say, I'm going to try to
  actively kneecap the postal service, and I will be explicit about the
  reason I'm doing it. That's sort of unheard of.
\end{itemize}

michael barbaro

You know, I can't help but think about 2016 and Russian interference.
And I'm kind of haunted by the Russian theory of the case, which is that
you don't actually have to do the thing. You don't actually have to
interfere in the election, because the real power is just in calling the
election itself into doubt.

luke broadwater

No, that's an interesting point. Yeah, I mean, the key distinction
between the two is in 2016, a lot of the concern was about outside
interference. This time, the accusation is that the meddling is coming
from the White House itself. That the call's coming from inside the
house, so to speak. That is obviously a huge cause for alarm, and it's a
very different line of concern than we saw in 2016.

Now, you know, despite all this concern and all the heated rhetoric that
we've heard, there is perhaps a silver lining. And that is when you talk
with a lot of get-out-the-vote folks, and you talk with activists about
how their messaging is changing, now they're really pushing this idea
that you need to vote a good two weeks before the election by mail. And
so people that maybe didn't hear that message and would have voted the
week before election, and maybe the post office wouldn't have gotten the
ballot there in time for their vote to count, I think there's a good
chance now that they will get that ballot in ahead of time, and their
vote will count.

{[}music{]}

michael barbaro

Well, Luke, thank you very much. We appreciate it.

luke broadwater

Thank you.

michael barbaro

On Tuesday, under growing pressure from Democrats, activists and voters,
Postmaster General DeJoy said he would formally suspend the operational
changes he's been making to the postal service until after the 2020
election. Among the changes he will suspend are eliminating overtime for
mail carriers, reducing post office hours and removing postal boxes, all
of which have been blamed for slowing mail delivery and could undermine
mail-in voting. But DeJoy has not said that he will permanently reverse
any of the changes that he has already made.

{[}music{]}

We'll be right back.

Here's what else you need to know today.

\begin{itemize}
\item
  archived recording 1\\
  The Commonwealth of Kentucky cast all 60 votes for the next president
  of the United States, Joe Biden.
\item
  archived recording 2\\
  Mississippi cast two votes for Bernie Sanders and 38 votes for our
  next president Joe Biden.
\item
  archived recording 3\\
  Delaware is proud to cast its 32 votes for our favorite son and our
  next president.
\item
  archived recording 4\\
  Our brand, Delaware, Joe Biden.
\end{itemize}

michael barbaro

During the second night of the Democratic National Convention, after a
virtual roll call from 57 states and U.S. territories, Joe Biden was
formally designated as the party's nominee for president.

\begin{itemize}
\item
  archived recording\\
  {[}CHEERS AND APPLAUSE{]}
\item
  archived recording (joe biden)\\
  Well, that you very, very much from the bottom of my heart. Thank you
  all. It means the world to me and my family. And I'll see you on
  Thursday. Thank you, thank you, thank you.
\item
  archived recording\\
  {[}CHEERS AND APPLAUSE{]}
\end{itemize}

michael barbaro

Later in the evening, Biden's wife, Dr. Jill Biden, delivered the
night's keynote speech from a classroom quieted by the pandemic,
recalling her decision to marry Biden not long after the death of his
first wife and daughter in a car crash.

\begin{itemize}
\tightlist
\item
  archived recording (jill biden)\\
  I never imagined at the age of 26 I would be asking myself, how do you
  make a broken family whole? Still, Joe always told the boys mommy sent
  Jill to us, and how could I argue with her.
\end{itemize}

michael barbaro

Recounting that tragedy, Joe Biden described her husband as uniquely
capable of healing the nation in the middle of a deadly pandemic and
economic collapse.

\begin{itemize}
\tightlist
\item
  archived recording (jill biden)\\
  I know that if we entrust this nation to Joe, he will do for your
  family what he did for ours. Bring us together and make us whole.
  Carry us forward in our time of need. Keep the promise of America for
  all of us.
\end{itemize}

michael barbaro

The speech was the latest sign that the Biden campaign will frame the
coming election as a referendum on President Trump and his handling of
the coronavirus.

{[}music{]}

That's it for ``The Daily.'' I'm Michael Barbaro. See you tomorrow.

The announcement came as lawmakers
\href{https://www.nytimes3xbfgragh.onion/2020/08/18/us/elections/louis-dejoy-the-postmaster-general-will-testify-before-the-senate-on-friday-about-service-cuts.html}{summoned
Mr. DeJoy to testify} before the House and the Senate in the coming days
and as two coalitions of at least 20 state attorneys general said they
would file lawsuits against the Trump administration over the postal
changes. Those lawsuits, which are being led by Washington State and
Pennsylvania, seek to reverse Mr. DeJoy's changes, which they called
``unlawful.''

Lawmakers in both parties and voting rights advocates have accused Mr.
DeJoy, a longtime transportation and logistics executive, of making
policy changes that jeopardize not only voters' ability to cast their
ballots safely in the general election, but also the delivery of
important medicines, paychecks and documents. Mr. DeJoy's
\href{https://www.nytimes3xbfgragh.onion/2020/08/17/us/politics/dejoy-postal-service-mail-in-voting.html}{continuing
financial ties} to companies that stand to benefit from his work at the
Postal Service have also prompted concerns from lawmakers.

Concern about Mr. DeJoy's changes was so pronounced that Speaker Nancy
Pelosi of California
\href{https://www.nytimes3xbfgragh.onion/2020/08/16/us/politics/coronavirus-postal-service-stimulus-bill.html}{interrupted
the House's annual summer recess} and scheduled a vote on Saturday on
Democratic legislation that would revoke policy changes until Jan. 1,
2021, or the end of the pandemic. The bill would also include \$25
billion in funding for the beleaguered agency. Mr. DeJoy's announcement
did not change those plans to return, a senior Democratic aide said on
Tuesday, although text of the legislation had not yet been released.

``They felt the heat,'' Ms. Pelosi said at a Politico Playbook event
moments after Mr. DeJoy released his statement. ``And that's what we
were trying to do, make it too hot for them to handle.''

Representative Steny H. Hoyer of Maryland, the majority leader and among
dozens of Democrats who spent Tuesday denouncing the changes at
in-person events outside postal facilities, said that while Mr. DeJoy
had vowed to stop any changes, Democrats were
``\href{https://www.majorityleader.gov/content/hoyer-remarks-press-conference-outside-usps-hq-postal-day-action}{going
to make sure in law that that is the case}.''

\includegraphics{https://static01.graylady3jvrrxbe.onion/images/2020/08/18/us/politics/18dc-postal2/merlin_175866270_71263b72-6a97-49b9-8bc9-a49ac720360b-articleLarge.jpg?quality=75\&auto=webp\&disable=upscale}

While several Senate Republicans have condemned the changes enforced by
Mr. DeJoy and signaled openness for more funding for the agency, top
Senate Republicans have not yet coalesced around legislation to address
the agency's issues.

``I don't think we'll pass, in the Senate, a postal-only bill,'' Senator
Mitch McConnell of Kentucky, the majority leader, told
\href{https://www.courier-journal.com/story/news/politics/2020/08/18/mcconnell-houses-postal-service-bill-may-negotiating-opportunity/3392267001/}{The
Courier Journal} on Tuesday.

Mr. DeJoy is scheduled to appear before the Senate Homeland Security
Committee on Friday and the House Oversight and Reform Committee on
Monday, as lawmakers continue to push for answers. Robert M. Duncan, the
chairman of the Postal Service board of governors, is also expected to
testify before the House committee on Monday.

Senator Chuck Schumer of New York, the minority leader, spoke to Mr.
DeJoy on Tuesday and asked for a written explanation of which policy
changes were rescinded or kept and for an explicit promise that the
agency would treat election mail as first-class priority.

``I told him there's a lot of mistrust because of statements he and the
president have made about cutbacks in mail delivery during Covid and
about mail-in voting through Election Day,'' Mr. Schumer said in a
statement, adding, ``We cannot allow two things as sacred to our country
as the Post Office and our elections to be undermined.''

As criticism and public pressure mounted, the Postal Service's board of
governors held emergency meetings via conference call on Saturday and
Monday, where the outcry was discussed, according to a person familiar
with the meetings who asked for anonymity because they were closed to
the public.

According to
\href{https://s3.amazonaws.com/public-inspection.federalregister.gov/2020-18366.pdf}{a
notice in the Federal Register}, the Postal Service's general counsel
``certified that the meeting may be closed under the Government in the
Sunshine Act.''

While the board oversees the strategic direction of the Postal Service
and selects the postmaster general, who serves at the pleasure of the
board members, it is not clear what role it played in approving the
cost-cutting changes or suspending them until after the election.

Board members declined to comment or did not respond to requests for
comment on Mr. DeJoy's moves and what role, if any, they have played. A
Postal Service spokesman declined to comment on the board's role or the
meetings, instead asking a reporter to file a Freedom of Information Act
request for the agendas of the meetings.

Mr. DeJoy and his administration allies have argued that the removal of
underused mailboxes and replacing sorting equipment were policies first
set into motion by Mr. DeJoy's predecessors. But Mr. DeJoy has carried
them out with a speed and vigor previous postmasters general had not,
according to people familiar with their execution.

``So, Postmaster General DeJoy, two weeks ago, and again, recently, has
reiterated his commitment to pay overtime to postal workers and letter
carriers if there's an increase in volume that demands that in order to
be able to get ballots or any other first-class mail to its destination
as efficiently and as fast as possible,'' Mark Meadows, the White House
chief of staff, told reporters on Tuesday. ``President Trump at no time
has instructed or directed the post office to cut back on overtime or
any other operational decision that would slow things down.''

Union officials said that some very limited overtime requests had been
fulfilled. The bigger concerns, they said, were the routing and
transportation changes detailed in a memo in July that was widely
circulated. Among the changes, mail carriers are supposed to avoid
waiting for delayed trucks or taking multiple trips.

The cost-cutting measures Mr. DeJoy carried out came after years of
public criticism by Mr. Trump, who has accused the Postal Service of
being poorly run and not charging high enough rates to private companies
like Amazon and UPS.

``Why is the United States Post Office, which is losing many billions of
dollars a year, while charging Amazon and others so little to deliver
their packages, making Amazon richer and the Post Office dumber and
poorer? Should be charging MUCH MORE!'' he
\href{https://twitter.com/realdonaldtrump/status/946728546633953285}{tweeted}
in December 2017.

The president reiterated that idea on Tuesday: ``@Amazon and others in
that business, should be charged (by the U.S. Postal System) much more
per package, and the Post Office would be immediately brought back to
`good health', now vibrant, with ALL jobs saved,''
\href{https://twitter.com/realDonaldTrump/status/1295855291523723264}{he
tweeted}. ``No pass on to customers. Get it done!''

In April 2018, Mr. Trump tapped Treasury Secretary Steven Mnuchin to
lead a task force on updating the Postal Service, which culminated
\href{https://home.treasury.gov/system/files/136/USPS_A_Sustainable_Path_Forward_report_12-04-2018.pdf}{in
a report} that offered a broad critique of the agency's business model
and concluded that an antiquated mission --- along with changing market
forces --- left it ripe for financial collapse.

The report suggested that postal workers were overpaid relative to other
government employees and private-sector delivery services, and that
compensation reductions along with other cost-cutting measures were
needed.

``Their wages and benefits should be aligned to comparable U.S. federal
employee groups, including aligning their ability to collectively
bargain for wages and benefits with other federal employees,'' the
report said.

There were also recommendations to cut services, such as scaling back
delivery days and making it easier to close post offices and remove
mailboxes. The task force said that to be more cost efficient, the
Postal Service should exercise ``discretion to lower service
standards.''

But those recommendations came before the pandemic, which has upended
plans to vote in person this November and is likely to cause an influx
of mail-in ballots.

Washington filed a lawsuit on Tuesday accusing the administration of
unlawfully pushing through changes at the Postal Service, and
Pennsylvania intends to file one as well. Letitia James, New York's
attorney general, said on Tuesday that she intended to file a separate
lawsuit.

William Tong, Connecticut's attorney general, said that his office had
received complaints from across the state about slowdowns in mail
delivery. Among them, medicine was not being delivered to sick seniors
and child support payments were arriving late to financially insecure
mothers. In August, some Connecticut voters were denied the right to
vote when their absentee ballots arrived late, he added.

``We will not allow Donald Trump to steal the election by sabotaging the
United States Postal Service,'' Mr. Tong said in a statement. ``The
president greatly misjudged the anger his unlawful policies would
unleash across this country.''

The states plan to pursue their lawsuit despite Mr. DeJoy's announcement
on Tuesday. James E. Tierney, a former attorney general of Maine and a
professor at Harvard Law School, said the lawsuit provided assurance to
the attorneys general that the postmaster general would follow through
on his promise.

``The attorneys general don't trust the word of the Postal Service,''
Mr. Tierney said. ``They feel more comfortable having this done in open
court.''

Alan Rappeport, Luke Broadwater, Catie Edmondson and Michael D. Shear
contributed reporting.

\hypertarget{our-2020-election-guide}{%
\section{Our 2020 Election Guide}\label{our-2020-election-guide}}

Updated ~Sept. 12, 2020

\begin{center}\rule{0.5\linewidth}{\linethickness}\end{center}

\begin{itemize}
\item ~
  \hypertarget{the-latest}{%
  \subsection{The Latest}\label{the-latest}}

  \begin{itemize}
  \item
    President Trump has failed to erase Joseph R. Biden Jr.'s lead
    across a set of key swing states,
    \href{https://www.nytimes3xbfgragh.onion/2020/09/12/us/politics/biden-trump-poll-wisconsin-minnesota.html?action=click\&pgtype=Article\&state=default\&region=BELOW_MAIN_CONTENT\&context=storylines_guide}{according
    to a poll}~conducted by The Times and Siena College.
  \end{itemize}
\item ~
  \hypertarget{paths-to-270}{%
  \subsection{Paths to 270}\label{paths-to-270}}

  \begin{itemize}
  \item
    Joe Biden and Donald Trump need 270 electoral votes to reach the
    White House. Try building
    \href{https://www.nytimes3xbfgragh.onion/interactive/2020/us/elections/election-states-biden-trump.html?action=click\&pgtype=Article\&state=default\&region=BELOW_MAIN_CONTENT\&context=storylines_guide}{your
    own coalition of battleground states}~to see potential outcomes.
  \end{itemize}
\item ~
  \hypertarget{voting-deadlines}{%
  \subsection{Voting Deadlines}\label{voting-deadlines}}

  \begin{itemize}
  \item
    Early voting for the presidential election starts in September~in
    some states. Take a look at
    \href{https://www.nytimes3xbfgragh.onion/interactive/2019/us/elections/2020-presidential-election-calendar.html?action=click\&pgtype=Article\&state=default\&region=BELOW_MAIN_CONTENT\&context=storylines_guide}{key
    dates}\href{https://www.nytimes3xbfgragh.onion/interactive/2019/us/elections/2020-presidential-election-calendar.html?action=click\&pgtype=Article\&state=default\&region=BELOW_MAIN_CONTENT\&context=storylines_guide}{where
    you
    liv}\href{https://www.nytimes3xbfgragh.onion/interactive/2019/us/elections/2020-presidential-election-calendar.html?action=click\&pgtype=Article\&state=default\&region=BELOW_MAIN_CONTENT\&context=storylines_guide}{e}.
    If you're voting by
    mail,~\href{https://www.nytimes3xbfgragh.onion/interactive/2020/08/31/us/politics/vote-by-mail-deadlines.html?action=click\&pgtype=Article\&state=default\&region=BELOW_MAIN_CONTENT\&context=storylines_guide}{it's
    risky to procrastinate}.
  \item
    \href{https://www.nytimes3xbfgragh.onion/interactive/2020/us/elections/joe-biden.html?action=click\&pgtype=Article\&state=default\&region=BELOW_MAIN_CONTENT\&context=storylines_guide}{}

    \hypertarget{joe-biden}{%
    \section{Joe Biden}\label{joe-biden}}

    \hypertarget{democrat}{%
    \subsection{Democrat}\label{democrat}}

    \href{https://www.nytimes3xbfgragh.onion/interactive/2020/us/elections/donald-trump.html?action=click\&pgtype=Article\&state=default\&region=BELOW_MAIN_CONTENT\&context=storylines_guide}{}

    \hypertarget{donald-trump}{%
    \section{Donald Trump}\label{donald-trump}}

    \hypertarget{republican}{%
    \subsection{Republican}\label{republican}}
  \end{itemize}
\item
  \hypertarget{keep-up-with-our-coverage}{%
  \subsection{Keep Up With Our
  Coverage}\label{keep-up-with-our-coverage}}

  \begin{itemize}
  \item
    Get an
    \href{https://www.nytimes3xbfgragh.onion/newsletters/politics?action=click\&pgtype=Article\&state=default\&region=BELOW_MAIN_CONTENT\&context=storylines_guide}{email}~recapping
    the day's news
  \item
    Download our mobile app on
    \href{https://apps.apple.com/us/app/nytimes/id284862083?ls=1\&mat_click_id=5c79ae7455014fd1bd66b5610c05b8f2-20191112-16948\&referrer=mat_click_id\%3D5c79ae7455014fd1bd66b5610c05b8f2-20191112-16948\%26link_click_id\%3D722930677036718082}{iOS}~and
    \href{http://a.localytics.com/android?id=com.nytimes.android\&referrer=utm_source\%3Dother_nyt_mobile_web\%26utm_medium\%3DWeb\%2520page\%26utm_term\%3DGeneral\%2520Mobile\%2520Page\%26utm_campaign\%3DNYT\%2520Mobile\%2520General\%2520Page}{Android}~and
    turn on Breaking News and Politics alerts
  \end{itemize}
\end{itemize}

Advertisement

\protect\hyperlink{after-bottom}{Continue reading the main story}

\hypertarget{site-index}{%
\subsection{Site Index}\label{site-index}}

\hypertarget{site-information-navigation}{%
\subsection{Site Information
Navigation}\label{site-information-navigation}}

\begin{itemize}
\tightlist
\item
  \href{https://help.nytimes3xbfgragh.onion/hc/en-us/articles/115014792127-Copyright-notice}{©~2020~The
  New York Times Company}
\end{itemize}

\begin{itemize}
\tightlist
\item
  \href{https://www.nytco.com/}{NYTCo}
\item
  \href{https://help.nytimes3xbfgragh.onion/hc/en-us/articles/115015385887-Contact-Us}{Contact
  Us}
\item
  \href{https://www.nytco.com/careers/}{Work with us}
\item
  \href{https://nytmediakit.com/}{Advertise}
\item
  \href{http://www.tbrandstudio.com/}{T Brand Studio}
\item
  \href{https://www.nytimes3xbfgragh.onion/privacy/cookie-policy\#how-do-i-manage-trackers}{Your
  Ad Choices}
\item
  \href{https://www.nytimes3xbfgragh.onion/privacy}{Privacy}
\item
  \href{https://help.nytimes3xbfgragh.onion/hc/en-us/articles/115014893428-Terms-of-service}{Terms
  of Service}
\item
  \href{https://help.nytimes3xbfgragh.onion/hc/en-us/articles/115014893968-Terms-of-sale}{Terms
  of Sale}
\item
  \href{https://spiderbites.nytimes3xbfgragh.onion}{Site Map}
\item
  \href{https://help.nytimes3xbfgragh.onion/hc/en-us}{Help}
\item
  \href{https://www.nytimes3xbfgragh.onion/subscription?campaignId=37WXW}{Subscriptions}
\end{itemize}
