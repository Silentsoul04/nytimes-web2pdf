Sections

SEARCH

\protect\hyperlink{site-content}{Skip to
content}\protect\hyperlink{site-index}{Skip to site index}

\href{https://www.nytimes3xbfgragh.onion/section/politics}{Politics}

\href{https://myaccount.nytimes3xbfgragh.onion/auth/login?response_type=cookie\&client_id=vi}{}

\href{https://www.nytimes3xbfgragh.onion/section/todayspaper}{Today's
Paper}

\href{/section/politics}{Politics}\textbar{}G.O.P.-Led Senate Panel
Details Ties Between 2016 Trump Campaign and Russia

\url{https://nyti.ms/2Yd4kRt}

\begin{itemize}
\item
\item
\item
\item
\item
\item
\end{itemize}

Advertisement

\protect\hyperlink{after-top}{Continue reading the main story}

Supported by

\protect\hyperlink{after-sponsor}{Continue reading the main story}

\hypertarget{gop-led-senate-panel-details-ties-between-2016-trump-campaign-and-russia}{%
\section{G.O.P.-Led Senate Panel Details Ties Between 2016 Trump
Campaign and
Russia}\label{gop-led-senate-panel-details-ties-between-2016-trump-campaign-and-russia}}

A nearly 1,000-page report confirmed the special counsel's findings at a
moment when President Trump's allies have sought to undermine that
inquiry.

\includegraphics{https://static01.graylady3jvrrxbe.onion/images/2020/08/17/us/politics/17dc-intel/17dc-intel-articleLarge-v3.jpg?quality=75\&auto=webp\&disable=upscale}

\href{https://www.nytimes3xbfgragh.onion/by/mark-mazzetti}{\includegraphics{https://static01.graylady3jvrrxbe.onion/images/2018/07/12/multimedia/author-Mark-Mazzetti/author-Mark-Mazzetti-thumbLarge-v4.png}}

By \href{https://www.nytimes3xbfgragh.onion/by/mark-mazzetti}{Mark
Mazzetti}

\begin{itemize}
\item
  Published Aug. 18, 2020Updated Aug. 19, 2020
\item
  \begin{itemize}
  \item
  \item
  \item
  \item
  \item
  \item
  \end{itemize}
\end{itemize}

WASHINGTON --- A sprawling report released Tuesday by a
Republican-controlled Senate panel that spent three years investigating
\href{https://www.nytimes3xbfgragh.onion/2020/09/01/us/politics/us-russia-military-tensions.html}{Russia}'s
interference in the 2016 election laid out an extensive web of contacts
between Trump campaign advisers and Kremlin officials and other
Russians, including at least one intelligence officer and others tied to
the country's spy services.

The report by the Senate Intelligence Committee,
\href{https://int.graylady3jvrrxbe.onion/data/documenttools/senate-intelligence-committee-russian-interference/8cf58e574d235164/full.pdf}{totaling
nearly 1,000 pages}, drew to a close one of the highest-profile
congressional investigations in recent memory and could be the last word
from an official government inquiry about the expansive Russian campaign
to sabotage the 2016 election.

It provided a bipartisan Senate imprimatur for an extraordinary set of
facts: The Russian government disrupted an American election to help Mr.
Trump become president, Russian intelligence services viewed members of
the Trump campaign as easily manipulated, and some of Mr. Trump's
advisers were eager for the help from an American adversary.

The report portrayed a Trump campaign that was stocked with businessmen
with no government experience, advisers working at the fringes of the
foreign policy establishment and other friends and associates Mr. Trump
had accumulated over the years. Campaign figures, the report said,
``presented attractive targets for foreign influence, creating notable
counterintelligence vulnerabilities.''

\href{https://www.nytimes3xbfgragh.onion/interactive/2019/01/26/us/politics/trump-contacts-russians-wikileaks.html}{}

\includegraphics{https://static01.graylady3jvrrxbe.onion/images/2019/01/25/us/trump-contacts-russians-wikileaks-promo-1548467557347/trump-contacts-russians-wikileaks-promo-1548467557347-articleLarge.png}

\hypertarget{mueller-report-shows-depth-of-connections-between-trump-campaign-and-russians}{%
\subsection{Mueller Report Shows Depth of Connections Between Trump
Campaign and
Russians}\label{mueller-report-shows-depth-of-connections-between-trump-campaign-and-russians}}

Donald J. Trump and 18 of his associates had at least 140 contacts with
Russian nationals and WikiLeaks, or their intermediaries, during the
2016 campaign and presidential transition.

Like the special counsel, Robert S. Mueller III, who
\href{https://www.nytimes3xbfgragh.onion/2019/04/18/us/politics/mueller-report-russian-interference-donald-trump.html}{released
his findings} in April 2019, the Senate report did not conclude that the
Trump campaign engaged in a coordinated conspiracy with the Russian
government --- a fact that Republicans seized on to argue that there was
``no collusion.''

But the report showed extensive evidence of contacts between Trump
campaign advisers and people tied to the Kremlin --- including a
longstanding associate of the onetime Trump campaign chairman Paul
Manafort, Konstantin V. Kilimnik, whom the report identified as a
``Russian intelligence officer.''

The Senate report was the first time the government has identified Mr.
Kilimnik as an intelligence officer --- Mr. Mueller's report had labeled
him as someone with ties to Russian intelligence. Most of the details
about his intelligence background were blacked out in the Senate report.

\href{https://www.nytimes3xbfgragh.onion/2020/08/18/us/politics/paul-manafort-konstantin-kilimnik.html}{Mr.
Manafort's willingness to share information with Mr. Kilimnik} and
others affiliated with the Russian intelligence services ``represented a
grave counterintelligence threat,'' the report said.

It also included a potentially explosive detail: that investigators had
uncovered information possibly tying Mr. Kilimnik to Russia's major
election interference operations, conducted by the intelligence service
known as the G.R.U.

Democrats highlighted Mr. Kilimnik's potential ties to the interference
operations in their own appendix to the report, noting that Mr. Manafort
discussed campaign strategy and shared internal campaign polling data
with the Russian and later lied to federal investigators about his
actions.

``This is what collusion looks like,'' Democrats wrote.

Their assertion was a sign that even though the investigation was
carried out in bipartisan fashion, and Republican and Democratic
senators reached broad agreement on its most significant conclusions, a
partisan divide remained on some of the most politically delicate
issues.

The report is an exhaustive look at the various ways that the Kremlin's
intelligence services exploited ties to the Trump campaign to help carry
out a stealth attack on American democracy. By focusing on the Russian
actions as a national security threat, the Senate investigation differed
from the Mueller inquiry, which examined whether there was evidence to
charge anyone with specific crimes.

The Senate investigation found that two other Russians who met at Trump
Tower in 2016 with senior members of the Trump campaign --- including
Mr. Manafort; Jared Kushner, the president's son-in-law; and Donald
Trump Jr., the president's eldest son --- had ``significant connections
to Russian government, including the Russian intelligence services.''

Links between the Kremlin and one of the individuals, Natalia V.
Veselnitskaya, ``were far more extensive and concerning than what had
been publicly known,'' the report said.

The report's findings about Mr. Kilimnik and other Russians in touch
with Trump campaign advisers confirmed
\href{https://www.nytimes3xbfgragh.onion/2017/02/14/us/politics/russia-intelligence-communications-trump.html}{an
article} in The New York Times from 2017 that said there had been
numerous interactions between the Trump campaign and Russian
intelligence in the year before the election. F.B.I. officials had
disputed the report.

Though there was no evidence of any agreement between the Russians and
the Trump campaign to work together, there was clear coordination, said
Senator Angus King, a Maine independent who caucuses with the Democrats
and is a member of the Intelligence Committee.

``The Russians were doing things to disrupt American democracy and help
the Trump campaign and the Trump campaign was doing things to amplify
and utilize what the Russians were supplying,'' Mr. King said in an
interview. ``There may not have been an explicit agreement but they were
both consciously pursing the same end, which was the election of Donald
Trump. And for the Russians, the extra benefit was disrupting American
democracy.''

The president and his allies have long tried to discredit the government
investigations into the 2016 election as part of a ``witch hunt''
intended to undermine the legitimacy of Mr. Trump's stunning election.
Since the release of Mr. Mueller's report, Attorney General William P.
Barr and numerous Republican senators have
\href{https://www.nytimes3xbfgragh.onion/2020/06/04/us/politics/republicans-senate-trump-russia.html}{recast
the president as the victim} of politically motivated national security
officials in the Obama administration.

Releasing the report less than 100 days before Election Day, lawmakers
hoped it would refocus attention on the interference by Russia and other
hostile foreign powers in the American political process, which has
continued unabated.

Members of Mr. Trump's own party led the Intelligence Committee's work.
Much of the investigation was overseen by Senator Richard M. Burr,
Republican of North Carolina, but he
\href{https://www.nytimes3xbfgragh.onion/2020/05/14/us/politics/richard-burr-stocks.html}{temporarily
stepped aside} as the chairman of the panel in May because of a federal
investigation into stock sales he made before the coronavirus pandemic
began rattling the United States. He was succeeded by Senator Marco
Rubio, Republican of Florida, though Mr. Burr voted to endorse the
report's conclusions.

The report could have partisan benefits for Democrats, who were using
their convention this week as a platform to portray Mr. Trump as unfit
and incapable of being president. Andrew Bates, a spokesman for former
Vice President Joseph R. Biden Jr., said the report showed ``the Russian
government intervened in 2016 to help Donald Trump get elected and to
undermine our democracy. Donald Trump welcomed it with open arms. They
are working toward the same goals again this year, and Trump refuses to
reject their assistance.''

President Trump called the report ``a hoax,'' but a White House
spokesman said it helped confirm what the president and his allies had
long said --- ``that there was absolutely no collusion between the Trump
campaign and Russia.

``This never-ending, baseless conspiracy theory peddled by radical
liberals and their partners in the media demonstrates how incapable they
are at accepting the will of the American people and the results of the
2016 election,'' said the spokesman, Judd Deere.

The report is the product of one of the few congressional investigations
in recent memory that retained bipartisan support throughout. Lawmakers
and committee aides interviewed more than 200 witnesses and reviewed
hundreds of thousands of documents, including intelligence reports,
internal F.B.I. notes and correspondence among members of the Trump
campaign. The committee convened hearings in 2017 and 2018, but most of
its work took place out of public view.

The report suggested that Mr. Manafort was compromised by his financial
ties with Russian and Ukrainian oligarchs, who themselves were connected
to Mr. Kilimnik, the Russian intelligence operative.

It cited Mr. Manafort's ties to Oleg Deripaska, a Russian oligarch
described as a ``proxy'' for Russian state and intelligence services who
claimed that Mr. Manafort owed him money. And it described at length Mr.
Manafort's relationships with a cluster of pro-Russia oligarchs in
Ukraine, who had paid him tens of millions of dollars as a political
consultant in Ukraine.

``Manafort conducted influence operations that supported and were a part
of Russian active measures campaigns, including those involving
political influence and electoral interference,'' the report said.

Before, during and after he was forced out as Mr. Trump's campaign
chairman, the report said, Mr. Manafort offered inside information and
assistance to these Russian-aligned interests. Mr. Kilimnik was Mr.
Manafort's intermediary with both Mr. Deripaska and the Ukrainian
oligarchs, according to the report. It recounted how he briefed Mr.
Kilimnik at an August 2016 meeting on the Trump campaign's strategy to
defeat Hillary Clinton, describing efforts in the battleground states of
Michigan, Wisconsin, Pennsylvania and Minnesota and the margins by which
Mr. Trump might win.

The report also shed new light on the interaction between Russian
intelligence and WikiLeaks --- and between WikiLeaks and the Trump
campaign. WikiLeaks, which released tranches of stolen Democratic emails
that helped damage Mrs. Clinton's campaign, not only played a clear role
in the election interference but also ``very likely knew it was
assisting a Russian intelligence influence effort,'' the report said.

The Intelligence Committee sought to track calls between Mr. Trump and
Roger J. Stone Jr. --- an adviser to the Trump campaign who was in
contact with Guccifer 2.0, the online pseudonym for Russian intelligence
operatives dumping the Democratic emails --- in an effort to discover
what Mr. Stone might have told Mr. Trump about the hacked emails.

In written answers to Mr. Mueller, Mr. Trump said he could not recall
discussing WikiLeaks with Mr. Stone, a response challenged in the Senate
report. ``The committee assesses that Trump did, in fact, speak with
Stone about WikiLeaks and with members of his Campaign about Stone's
access to WikiLeaks on multiple occasions,'' the report said.

Last month, Mr. Trump commuted a prison sentence Mr. Stone had received
after he was convicted on seven felonies of obstructing a congressional
investigation that threatened the president, his longtime friend.

The committee sent a letter last summer to the U.S. attorney's office in
Washington suggesting that Trump campaign advisers may have illegally
made false or misleading statements to congressional investigators
conducting the panel's inquiry, according to four people with knowledge
of the letter, which
\href{https://www.latimes.com/politics/story/2020-08-14/senate-committee-sought-investigation-of-bannon-raised-concerns-about-trump-family-testimony}{was
first reported by The Los Angeles Times}.

The committee said in the letter that Mr. Trump's onetime chief
strategist Stephen K. Bannon and his former campaign co-chairman Sam
Clovis may have committed a crime by lying under oath, and they cast
doubt on the testimony of Donald J. Trump Jr. and Mr. Kushner.
Prosecutors never filed charges.

Mr. Barr has appointed a criminal prosecutor, John H. Durham, to review
the actions that intelligence and law enforcement officials took in 2016
to better understand the Kremlin's interference campaign and
interactions between Russians and Trump campaign advisers. Last month,
Mr. Barr
\href{https://www.nytimes3xbfgragh.onion/2020/07/27/us/politics/william-barr-house-judiciary-hearing.html}{told
a congressional committee} that he was determined ``to get to the bottom
of the grave abuses involved in the bogus `Russiagate' scandal.''

The Justice Department's independent inspector general
\href{https://www.nytimes3xbfgragh.onion/2019/12/09/us/politics/fbi-ig-report-russia-investigation.html}{has
found} that law enforcement officials had sufficient basis to open the
Russia investigation and acted without political bias.

But the Senate report did criticize the F.B.I., saying the bureau should
have done more to alert higher-level officials at the Democratic
National Committee that their servers may have been infiltrated by
Russian hackers.

It also criticized the bureau's handling of the so-called Steele
dossier, a compendium of rumors about purported Trump-Russia links
compiled by Christopher Steele, a British former intelligence agent. The
bureau used some of Mr. Steele's information in applications to wiretap
Carter Page, a former Trump campaign adviser.

The Senate report nonetheless endorsed the F.B.I.'s decision to
investigate Mr. Page. ``Page's previous ties to Russian intelligence
officers, coupled with his Russian travel, justified the F.B.I.'s
initial concerns about Page,'' it said.

The report portrayed the dossier as shoddy and criticized the F.B.I.'s
vetting of Mr. Steele as ``not sufficiently rigorous or thorough.''

At the same time, it dove into one of the main subjects of the dossier
--- whether the Russian government has compromising material on Mr.
Trump from his past business dealings in Moscow. The report explicitly
said it ``did not establish'' that the Russian government obtained any
compromising material on Mr. Trump or that it tried to use such
materials as leverage against him.

It did, however, spend pages describing Mr. Trump's relationships with
women in Moscow during his trips there starting in the mid-1990s, when
he began looking for sites to build a Trump Tower. Mr. Deere, the White
House spokesman, did not comment on those details in the report.

According to the report, Mr. Trump
\href{https://www.nytimes3xbfgragh.onion/2020/08/18/us/politics/trump-russia-senate-intelligence.html}{met
a former Miss Moscow} at a party during one trip in 1996. After the
party, a Trump associate told others he had seen Mr. Trump with the
woman on multiple occasions and that they ``might have had a brief
romantic relationship.''

The report also raised the possibility that, during that trip, Mr. Trump
spent the night with two young women who joined him the next morning at
a business meeting with the mayor of Moscow.

Reporting was contributed by Charlie Savage, Sharon LaFraniere, Julian
E. Barnes, Michael S. Schmidt, Nicholas Fandos and Katie Benner.

Advertisement

\protect\hyperlink{after-bottom}{Continue reading the main story}

\hypertarget{site-index}{%
\subsection{Site Index}\label{site-index}}

\hypertarget{site-information-navigation}{%
\subsection{Site Information
Navigation}\label{site-information-navigation}}

\begin{itemize}
\tightlist
\item
  \href{https://help.nytimes3xbfgragh.onion/hc/en-us/articles/115014792127-Copyright-notice}{©~2020~The
  New York Times Company}
\end{itemize}

\begin{itemize}
\tightlist
\item
  \href{https://www.nytco.com/}{NYTCo}
\item
  \href{https://help.nytimes3xbfgragh.onion/hc/en-us/articles/115015385887-Contact-Us}{Contact
  Us}
\item
  \href{https://www.nytco.com/careers/}{Work with us}
\item
  \href{https://nytmediakit.com/}{Advertise}
\item
  \href{http://www.tbrandstudio.com/}{T Brand Studio}
\item
  \href{https://www.nytimes3xbfgragh.onion/privacy/cookie-policy\#how-do-i-manage-trackers}{Your
  Ad Choices}
\item
  \href{https://www.nytimes3xbfgragh.onion/privacy}{Privacy}
\item
  \href{https://help.nytimes3xbfgragh.onion/hc/en-us/articles/115014893428-Terms-of-service}{Terms
  of Service}
\item
  \href{https://help.nytimes3xbfgragh.onion/hc/en-us/articles/115014893968-Terms-of-sale}{Terms
  of Sale}
\item
  \href{https://spiderbites.nytimes3xbfgragh.onion}{Site Map}
\item
  \href{https://help.nytimes3xbfgragh.onion/hc/en-us}{Help}
\item
  \href{https://www.nytimes3xbfgragh.onion/subscription?campaignId=37WXW}{Subscriptions}
\end{itemize}
