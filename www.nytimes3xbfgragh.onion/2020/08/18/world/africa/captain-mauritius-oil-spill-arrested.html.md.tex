Sections

SEARCH

\protect\hyperlink{site-content}{Skip to
content}\protect\hyperlink{site-index}{Skip to site index}

\href{https://www.nytimes3xbfgragh.onion/section/world/africa}{Africa}

\href{https://myaccount.nytimes3xbfgragh.onion/auth/login?response_type=cookie\&client_id=vi}{}

\href{https://www.nytimes3xbfgragh.onion/section/todayspaper}{Today's
Paper}

\href{/section/world/africa}{Africa}\textbar{}Captain in Mauritius Oil
Spill Disaster Is Arrested

\url{https://nyti.ms/2Eb7hej}

\begin{itemize}
\item
\item
\item
\item
\item
\end{itemize}

Advertisement

\protect\hyperlink{after-top}{Continue reading the main story}

Supported by

\protect\hyperlink{after-sponsor}{Continue reading the main story}

\hypertarget{captain-in-mauritius-oil-spill-disaster-is-arrested}{%
\section{Captain in Mauritius Oil Spill Disaster Is
Arrested}\label{captain-in-mauritius-oil-spill-disaster-is-arrested}}

The MV Wakashio, which split into two on Saturday, has spilled at least
1,000 tons of oil into the Indian Ocean, endangering world-renowned
coral reefs and lagoons.

\includegraphics{https://static01.graylady3jvrrxbe.onion/images/2020/08/18/world/18mauritius/merlin_175807935_dee5128e-e52b-4089-818e-952a56b4d148-articleLarge.jpg?quality=75\&auto=webp\&disable=upscale}

\href{https://www.nytimes3xbfgragh.onion/by/abdi-latif-dahir}{\includegraphics{https://static01.graylady3jvrrxbe.onion/images/2020/08/14/reader-center/author-abdi-latif-dahir/author-abdi-latif-dahir-thumbLarge.png}}

By \href{https://www.nytimes3xbfgragh.onion/by/abdi-latif-dahir}{Abdi
Latif Dahir}

\begin{itemize}
\item
  Aug. 18, 2020
\item
  \begin{itemize}
  \item
  \item
  \item
  \item
  \item
  \end{itemize}
\end{itemize}

NAIROBI, Kenya --- The captain of the ship that ran aground in Mauritius
and spilled about 1,000 tons of oil into the Indian Ocean has been
arrested, his lawyer said on Tuesday.

The captain, Sunil Kumar Nandeshwar, was arraigned in a district court
in the country's capital, Port Louis, on the charge of endangering the
safe navigation of a vessel, an offense under
\href{http://blueconomy.govmu.org/English/Documents/THE\%20PIRACY\%20AND\%20MARITIME\%20VIOLENCE\%20ACT\%202011.pdf}{Mauritian
maritime laws}.

Mr. Nandeshwar, an Indian national, was arrested alongside the chief
officer of the ship, Tilak Ratna Suboda, a Sri Lankan. The two were
taken into police custody and will appear again in court on Aug. 25,
Ilshad Munsoor, Mr. Nandeshwar's lawyer, said in a phone interview.

The island nation of about 1.3 million people is still grappling with
how to protect its world-renowned coral reefs and crystal-clear lagoons
after the spill.

The MV Wakashio, a Japanese-owned Panama-flagged bulk carrier, went off
course and grounded on a coral reef off the coast of the island nation
on July 25. Less than two weeks later, the ship, carrying 4,000 tons of
fuel oil,
\href{https://www.nytimes3xbfgragh.onion/2020/08/07/world/africa/mauritius-oil-spill.html}{started
leaking}, leading to an ecological disaster that could have far-reaching
consequences for the country's food security, economy and tourism.

Local news media reported that
\href{https://www.lemauricien.com/actualites/faits-divers/enquete-sur-le-wakashio-lequipage-coule-le-capitaine-en-confirmant-son-absence-de-la-cabine/369098/}{a
preliminary examination by the police showed} the crew was having a
party on the night the ship ran off course.

10 miles

Africa

MAURITIUS

Indian

Ocean

MAURITIUS

Curepipe

Detail

area

ÎLE AUX AIGRETTES

MAURITIUS

SHIP RAN AGROUND

blue bay

marine park

1 mile

By The New York Times

By the time authorities stopped the leak, 1,000 tons of oil had spilled
into the sea, with the slick threatening environmentally sensitive zones
like the Ile aux Aigrettes nature reserve and the Blue Bay Marine Park.

The authorities, who had been receiving support from countries including
France and India, said on Saturday that the ship had ruptured in two.
While much of the 3,000 tons of remaining oil had been extracted, there
was still 90 tons of oil on the ship. To prevent any further damage,
almost seven miles of protective booms have been positioned around the
vessel and some parts of the island, the National Crisis Management
Committee said on Monday.

Mauritius, which is off Africa's eastern coast, is known for its
picturesque beaches and coral reefs and mangrove forests that are
teeming with rich biodiversity. Tourism is a key pillar of its economy,
bringing about \$1.6 billion in 2018.

But the pandemic and its resulting lockdown measures have limited that
income this year. Hotels and parks remain empty as the suspension of
international flights has kept foreign visitors out of the country.

\includegraphics{https://static01.graylady3jvrrxbe.onion/images/2020/08/18/world/18mauritius2/merlin_175754784_60adb8aa-3925-4556-958f-7e44e8f8e1db-articleLarge.jpg?quality=75\&auto=webp\&disable=upscale}

Following the leak, thousands of Mauritians both inside and outside the
country responded by donating money, creating awareness on social media
and
\href{https://www.nytimes3xbfgragh.onion/2020/08/14/world/africa/mauritius-oil-spill.html}{participating
in cleanup efforts}. Volunteers created artisanal booms and stuffed them
with donated hair, sugar cane straws, plastic bottles and other items to
stop the spread of the oil and soak it up.

Many Mauritians, however, remain angry at how the government responded
to the wrecked ship and say officials should have acted more urgently
before it started leaking. While welcoming the arrest of the captain on
Tuesday, Reuben Pillay, the director of
\href{http://reubsvision.mu/Mauritius360.html}{Reubs Vision}, a company
that provides virtual tours of Mauritius, said he still wanted more
answers from the government.

``Will we know the truth?'' Mr. Pillay asked of the ship and its running
aground off the island's coast. ``We think that there is more to this
story.''

Advertisement

\protect\hyperlink{after-bottom}{Continue reading the main story}

\hypertarget{site-index}{%
\subsection{Site Index}\label{site-index}}

\hypertarget{site-information-navigation}{%
\subsection{Site Information
Navigation}\label{site-information-navigation}}

\begin{itemize}
\tightlist
\item
  \href{https://help.nytimes3xbfgragh.onion/hc/en-us/articles/115014792127-Copyright-notice}{©~2020~The
  New York Times Company}
\end{itemize}

\begin{itemize}
\tightlist
\item
  \href{https://www.nytco.com/}{NYTCo}
\item
  \href{https://help.nytimes3xbfgragh.onion/hc/en-us/articles/115015385887-Contact-Us}{Contact
  Us}
\item
  \href{https://www.nytco.com/careers/}{Work with us}
\item
  \href{https://nytmediakit.com/}{Advertise}
\item
  \href{http://www.tbrandstudio.com/}{T Brand Studio}
\item
  \href{https://www.nytimes3xbfgragh.onion/privacy/cookie-policy\#how-do-i-manage-trackers}{Your
  Ad Choices}
\item
  \href{https://www.nytimes3xbfgragh.onion/privacy}{Privacy}
\item
  \href{https://help.nytimes3xbfgragh.onion/hc/en-us/articles/115014893428-Terms-of-service}{Terms
  of Service}
\item
  \href{https://help.nytimes3xbfgragh.onion/hc/en-us/articles/115014893968-Terms-of-sale}{Terms
  of Sale}
\item
  \href{https://spiderbites.nytimes3xbfgragh.onion}{Site Map}
\item
  \href{https://help.nytimes3xbfgragh.onion/hc/en-us}{Help}
\item
  \href{https://www.nytimes3xbfgragh.onion/subscription?campaignId=37WXW}{Subscriptions}
\end{itemize}
