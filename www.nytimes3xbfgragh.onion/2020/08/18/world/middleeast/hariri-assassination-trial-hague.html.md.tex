Sections

SEARCH

\protect\hyperlink{site-content}{Skip to
content}\protect\hyperlink{site-index}{Skip to site index}

\href{https://www.nytimes3xbfgragh.onion/section/world/middleeast}{Middle
East}

\href{https://myaccount.nytimes3xbfgragh.onion/auth/login?response_type=cookie\&client_id=vi}{}

\href{https://www.nytimes3xbfgragh.onion/section/todayspaper}{Today's
Paper}

\href{/section/world/middleeast}{Middle East}\textbar{}15 Years After an
Assassination Rocked Lebanon, a Trial Ends on a Muted Note

\url{https://nyti.ms/321n8UT}

\begin{itemize}
\item
\item
\item
\item
\item
\end{itemize}

Advertisement

\protect\hyperlink{after-top}{Continue reading the main story}

Supported by

\protect\hyperlink{after-sponsor}{Continue reading the main story}

\hypertarget{15-years-after-an-assassination-rocked-lebanon-a-trial-ends-on-a-muted-note}{%
\section{15 Years After an Assassination Rocked Lebanon, a Trial Ends on
a Muted
Note}\label{15-years-after-an-assassination-rocked-lebanon-a-trial-ends-on-a-muted-note}}

After a long and involved investigation, a U.N.-backed tribunal emerged
with only a single conviction, of a minor Hezbollah figure, in the 2005
bombing that killed the former prime minister of Lebanon.

\includegraphics{https://static01.graylady3jvrrxbe.onion/images/2020/08/18/world/18lebanon-tribunal/merlin_175808475_3133173f-d6b3-46e3-beb2-bbfd1bae20fd-articleLarge.jpg?quality=75\&auto=webp\&disable=upscale}

By \href{https://www.nytimes3xbfgragh.onion/by/marlise-simons}{Marlise
Simons} and \href{https://www.nytimes3xbfgragh.onion/by/ben-hubbard}{Ben
Hubbard}

\begin{itemize}
\item
  Aug. 18, 2020
\item
  \begin{itemize}
  \item
  \item
  \item
  \item
  \item
  \end{itemize}
\end{itemize}

The case went to trial in a country far from the crime scene with none
of the accused in custody. It cost hundreds of millions of dollars to
prosecute and employed armies of investigators, researchers and lawyers.

But when the verdict on the most consequential political assassination
in Lebanon's recent history arrived on Tuesday, it left the country
without a sense of closure and failed to answer even the most basic
question:
\href{https://www.nytimes3xbfgragh.onion/2020/08/08/world/middleeast/hariri-assassination-trial-hague.html}{Who
ordered the killing}?

For a huge suicide car bomb attack in Beirut in 2005 that rattled the
Middle East and killed former Prime Minister Rafik Hariri and 21 others,
a United Nations-backed tribunal in the Netherlands acquitted three
defendants for lack of evidence.

The fourth man, Salim Ayyash, was convicted of participating in a
conspiracy to carry out the bombing. But if he is ever apprehended, the
court will have to try him all over again since he was tried in
absentia.

The long-awaited verdict from the Special Tribunal for Lebanon, which
was created in 2009 at the behest of the United Nations Security
Council, disappointed many Lebanese and others who had hoped that an
international inquiry would reveal --- and punish --- those responsible
for the crime and break the country's long cycle of impunity for
political killings.

Although the court said that Syria and Hezbollah, the powerful Lebanese
militant group, had motives to ``eliminate'' Mr. Hariri, it said it
lacked direct evidence implicating them in the crime.

``It's like in 9/11 if you name the hijackers and not bin Laden,'' said
Nadim Houry, executive director of the
\href{https://www.arab-reform.net/}{Arab Reform Initiative}, a research
center based in Paris. ``This was way above Ayyash's pay grade.''

It is unlikely that Mr. Ayyash will ever be found, he said, and in any
case, he was ``a cog in the system,'' not the attack's mastermind.

\includegraphics{https://static01.graylady3jvrrxbe.onion/images/2020/08/18/world/18lebanon-tribunal3-sub/merlin_175848459_f55895ff-0107-459b-a7b1-dfbf237970a7-videoSixteenByNine3000.jpg}

Mr. Hariri was a momentous figure in Lebanon's politics, a charismatic
billionaire businessman with extensive relationships in the United
States, Europe and Saudi Arabia who used his wealth and connections to
jump-start growth in Lebanon after its disastrous 15-year civil war
ended in 1990.

But his killing in 2005 ushered in a new, turbulent era in Lebanese
politics during which his Western- and Gulf-aligned political bloc
competed for power with rivals backed by Syria and Iran, including
Hezbollah, the powerful militant group and political party. A string of
assassinations of other prominent figures followed, with none of their
killers ever identified or punished.

Initially, many Lebanese hoped that the creation of the international
tribunal would provide a way for justice to be done. But the
investigation and hearings dragged on as the killing faded into the
past.

In recent months, protests over corruption and poor governance have
flared against the political elite, and the economy and currency have
all but collapsed. The country is also reeling from a massive explosion
in the Beirut port that killed more than 170 people and wounded 6,000.

Maha Yahya, the director of the Carnegie Middle East Center in Beirut,
said it felt as if the tribunal were ``from a different era.''

The verdict came as Lebanon's politicians are wrangling over the
possibility of an international investigation into the Beirut blast; its
limited convictions could undermine hopes that those responsible for the
explosion will be held accountable.

``After 15 years and a Special Tribunal for Lebanon with international
investigators and we end up with this?'' Ms. Yahya asked. ``How is
anyone ever going to be held accountable for the port explosion?''

\includegraphics{https://static01.graylady3jvrrxbe.onion/images/2020/08/07/world/18lebanon-tribunal5/merlin_11564886_7e1d8c7a-6b00-4934-b08e-a4c64211446f-articleLarge.jpg?quality=75\&auto=webp\&disable=upscale}

Saad Hariri, a son of the assassinated politician and himself a former
prime minister of Lebanon, attended Tuesday's session and told reporters
after the verdict that he and his family accepted it.

Writing on Twitter, he called it a ``historic moment" and ``a message to
whoever carried out and planned this terrorist crime that the era of
using crime for politics without punishment and without a price has
ended.''

The court announced the verdicts after hourslong statements from its
judges summarizing the case and the arguments of the prosecution and
defense teams.

The court deemed the killing a politically motivated terrorist act and
described all four defendants --- Mr. Ayyash, Hassan Habib Merhi,
Hussein Hassan Oneissi and Assad Hassan Sabra --- as supporters of
Hezbollah.

Months before he was killed, the elder Hariri had resigned as prime
minister in anger at Syria's continuing interference in the country,
including the presence of Syrian troops.

The judges did not say who had planned the attack, but said it was
``very likely'' that the final decision to kill him was made after a
Feb. 2, 2005, meeting at which Mr. Hariri and other politicians had
agreed to call for the ``immediate and total withdrawal of Syrian forces
from Lebanon.''

Image

A month afer the assassination, protesters took to the streets of Beirut
to denounce Syria.Credit...Patrick Baz/Agence France-Presse via Getty
Images

After he was killed, general suspicion fell on Syria, which denied any
role. The brazen attack, which injured hundreds of people and left a
yawning crater near Beirut's waterfront, brought more than a million
protesters into the streets, and the outcry, combined with international
pressure, forced Syria to withdraw its troops.

In reading a summary of their 2,600-page ruling, the judges said the
murder plan relied on a massive load of high-grade explosives, and was
intended to cause ``fear and panic'' that would resonate throughout
Lebanon and the region.

Hezbollah's leader, Hassan Nasrallah, has repeatedly dismissed the
tribunal as a Western conspiracy and has threatened to go after any
followers who cooperated with it. The group did not immediately comment
on Tuesday's verdicts, but Mr. Nasrallah said recently that it
considered the court's finding irrelevant.

The key figure among the suspects, prosecutors said, was Mustafa Amine
Badreddinne, a veteran of Hezbollah's special operations and close to
its top leaders. But the case against him ended when Mr. Badreddinne was
\href{https://www.nytimes3xbfgragh.onion/2016/05/14/world/middleeast/mustafa-badreddine-hezbollah.html}{killed
in Syria} in 2016.

To critics of the tribunal, the prosecution of a few low-level Hezbollah
operatives is a far cry from the findings of United Nations
investigators who were sent to Beirut soon after the assassination.

In a report, these investigators called the killing an elaborate
professional conspiracy that required ``substantial logistical
support,'' considerable financing and ``military precision in its
execution.''

Detlev Mehlis, a German prosecutor who led a second inquiry,
\href{https://www.nytimes3xbfgragh.onion/2005/12/14/international/middleeast/un-considers-widening-inquiry-into-lebanese-slaying.html}{ended
a six-month investigation} in 2005 with a list of close to 20 suspects,
including several senior Lebanese and top Syrian officials.

Diplomats said at the time that Mr. Mehlis had reluctantly ended his
mission because he had been warned about two assassination plots against
him. At least two Lebanese police officers who assisted the tribunal's
investigations have been killed.

Image

Saad Hariri, a son of the assassinated politician and himself a former
prime minister of Lebanon, leaving the court on Tuesday. Mr. Hariri told
reporters that he and his family had accepted the
verdict.Credit...Pierre Crom/Getty Images

The prosecutors built their case largely on circumstantial evidence,
much of it extensive records of cellphones used as operatives covertly
tracked Mr. Hariri's movements for weeks.

The court-appointed defense lawyers had all asked for acquittals, saying
there was no proof that their clients had used the cellphones in
question. Electronic records of hundreds of calls could reveal location,
date and time, the lawyers argued, but they did not confirm the identity
of the users.

The verdict was originally scheduled for Aug. 7, but was postponed after
the Beirut port explosion.

Questions have been raised about the cost of the court's 400-strong
staff, including a roster of prosecutors and 11 full-time judges who
were involved in the case.

Half of its \$60 million annual budget has been paid by Lebanon, with
help from Saudi Arabia, and half by voluntary contributions from Western
countries and Arab Gulf states. For many critics, this enormous expense
has not justified the symbolism of an absentee trial.

Ehsan Fayed Al Nasser, whose husband Talal Nasser headed Mr. Hariri's
security team and was killed in the blast, said by phone on Tuesday that
the tribunal had gathered evidence, identified suspects and sentenced
one man.

``I am hoping he'll be arrested and lead us to the mastermind behind
this crime,'' she said.

Hwaida Saad and Kareem Chehayeb contributed reporting.

Advertisement

\protect\hyperlink{after-bottom}{Continue reading the main story}

\hypertarget{site-index}{%
\subsection{Site Index}\label{site-index}}

\hypertarget{site-information-navigation}{%
\subsection{Site Information
Navigation}\label{site-information-navigation}}

\begin{itemize}
\tightlist
\item
  \href{https://help.nytimes3xbfgragh.onion/hc/en-us/articles/115014792127-Copyright-notice}{©~2020~The
  New York Times Company}
\end{itemize}

\begin{itemize}
\tightlist
\item
  \href{https://www.nytco.com/}{NYTCo}
\item
  \href{https://help.nytimes3xbfgragh.onion/hc/en-us/articles/115015385887-Contact-Us}{Contact
  Us}
\item
  \href{https://www.nytco.com/careers/}{Work with us}
\item
  \href{https://nytmediakit.com/}{Advertise}
\item
  \href{http://www.tbrandstudio.com/}{T Brand Studio}
\item
  \href{https://www.nytimes3xbfgragh.onion/privacy/cookie-policy\#how-do-i-manage-trackers}{Your
  Ad Choices}
\item
  \href{https://www.nytimes3xbfgragh.onion/privacy}{Privacy}
\item
  \href{https://help.nytimes3xbfgragh.onion/hc/en-us/articles/115014893428-Terms-of-service}{Terms
  of Service}
\item
  \href{https://help.nytimes3xbfgragh.onion/hc/en-us/articles/115014893968-Terms-of-sale}{Terms
  of Sale}
\item
  \href{https://spiderbites.nytimes3xbfgragh.onion}{Site Map}
\item
  \href{https://help.nytimes3xbfgragh.onion/hc/en-us}{Help}
\item
  \href{https://www.nytimes3xbfgragh.onion/subscription?campaignId=37WXW}{Subscriptions}
\end{itemize}
