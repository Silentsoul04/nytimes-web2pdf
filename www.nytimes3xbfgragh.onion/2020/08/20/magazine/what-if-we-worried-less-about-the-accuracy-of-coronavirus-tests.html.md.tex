Sections

SEARCH

\protect\hyperlink{site-content}{Skip to
content}\protect\hyperlink{site-index}{Skip to site index}

\href{https://myaccount.nytimes3xbfgragh.onion/auth/login?response_type=cookie\&client_id=vi}{}

\href{https://www.nytimes3xbfgragh.onion/section/todayspaper}{Today's
Paper}

What if We Worried Less About the Accuracy of Coronavirus Tests?

\url{https://nyti.ms/2QazRyY}

\begin{itemize}
\item
\item
\item
\item
\item
\end{itemize}

\hypertarget{the-coronavirus-outbreak}{%
\subsubsection{\texorpdfstring{\href{https://www.nytimes3xbfgragh.onion/news-event/coronavirus?name=styln-coronavirus-national\&region=TOP_BANNER\&block=storyline_menu_recirc\&action=click\&pgtype=Article\&impression_id=8eb19b30-f1cf-11ea-b346-6f5f5ea5e57a\&variant=undefined}{The
Coronavirus
Outbreak}}{The Coronavirus Outbreak}}\label{the-coronavirus-outbreak}}

\begin{itemize}
\tightlist
\item
  live\href{https://www.nytimes3xbfgragh.onion/2020/09/08/world/covid-19-coronavirus.html?name=styln-coronavirus-national\&region=TOP_BANNER\&block=storyline_menu_recirc\&action=click\&pgtype=Article\&impression_id=8eb19b31-f1cf-11ea-b346-6f5f5ea5e57a\&variant=undefined}{Latest
  Updates}
\item
  \href{https://www.nytimes3xbfgragh.onion/interactive/2020/us/coronavirus-us-cases.html?name=styln-coronavirus-national\&region=TOP_BANNER\&block=storyline_menu_recirc\&action=click\&pgtype=Article\&impression_id=8eb1c240-f1cf-11ea-b346-6f5f5ea5e57a\&variant=undefined}{Maps
  and Cases}
\item
  \href{https://www.nytimes3xbfgragh.onion/interactive/2020/science/coronavirus-vaccine-tracker.html?name=styln-coronavirus-national\&region=TOP_BANNER\&block=storyline_menu_recirc\&action=click\&pgtype=Article\&impression_id=8eb1c241-f1cf-11ea-b346-6f5f5ea5e57a\&variant=undefined}{Vaccine
  Tracker}
\item
  \href{https://www.nytimes3xbfgragh.onion/2020/09/02/your-money/eviction-moratorium-covid.html?name=styln-coronavirus-national\&region=TOP_BANNER\&block=storyline_menu_recirc\&action=click\&pgtype=Article\&impression_id=8eb1c242-f1cf-11ea-b346-6f5f5ea5e57a\&variant=undefined}{Eviction
  Moratorium}
\item
  \href{https://www.nytimes3xbfgragh.onion/interactive/2020/09/02/magazine/food-insecurity-hunger-us.html?name=styln-coronavirus-national\&region=TOP_BANNER\&block=storyline_menu_recirc\&action=click\&pgtype=Article\&impression_id=8eb1c243-f1cf-11ea-b346-6f5f5ea5e57a\&variant=undefined}{American
  Hunger}
\end{itemize}

Advertisement

\protect\hyperlink{after-top}{Continue reading the main story}

Supported by

\protect\hyperlink{after-sponsor}{Continue reading the main story}

\href{/column/studies-show}{Studies Show}

\hypertarget{what-if-we-worried-less-about-the-accuracy-of-coronavirus-tests}{%
\section{What if We Worried Less About the Accuracy of Coronavirus
Tests?}\label{what-if-we-worried-less-about-the-accuracy-of-coronavirus-tests}}

\includegraphics{https://static01.graylady3jvrrxbe.onion/images/2020/08/23/magazine/23mag-studies/23mag-studies-articleLarge.jpg?quality=75\&auto=webp\&disable=upscale}

By Kim Tingley

\begin{itemize}
\item
  Published Aug. 20, 2020Updated Aug. 24, 2020
\item
  \begin{itemize}
  \item
  \item
  \item
  \item
  \item
  \end{itemize}
\end{itemize}

Accuracy is everything**,** typically, when we take a diagnostic test
--- an incorrect result can lead to anguish and erroneous, if not
harmful, treatment. Currently the most reliable way to identify a
coronavirus infection is by a polymerase chain reaction (P.C.R.) test: A
swab, usually taken from the nasal passage, produces a sample that is
then sent to a specialized laboratory. P.C.R. tests, which can detect
minute amounts of genetic material from the virus, cost upward of \$100;
in ideal circumstances, they take just hours to analyze. But because of
high demand, supply shortages and other issues, many commercial labs are
taking more than a week to process them. That means a positive test
often comes back too late to enable contact tracers to notify those who
have been exposed before they might in turn infect others. In these
circumstances, the diagnosis is useful only for making personal health
decisions and providing data on the rate of infection in a community.

In a July 21 report in
\href{https://jamanetwork.com/journals/jamainternalmedicine/fullarticle/2768834}{JAMA
Internal Medicine,} the C.D.C.'s response team for Covid-19 estimated
that nine out of 10 infections are not being identified --- and
obstacles to getting tested are probably major reasons. To capture more
of those cases, many of which may not show obvious symptoms, says Daniel
Larremore, a computational biologist at the University of Colorado,
Boulder, ``we need to shift our thinking.'' Specifically, he says, we
need to go from prioritizing the accuracy of individual test results to
prioritizing the ability of a testing system to reduce the rate of the
virus in a given population --- even if that results in more
misdiagnoses.

To see how this could work in practice, consider one strategy for
increasing testing capacity: pooling samples for analysis. Suppose one
person in 100 has the virus. Testers take and label a nasal swab from
all of them; a portion of each sample is saved, and the rest is grouped
with the samples taken from nine other people. The lab then runs 10
analyses, one for each group of 10 samples. Nine of those will return
negative results, a determination given to all 90 members in those
groups. The lab then retests each individual sample in the positive
group to find the infected member. Over all, the lab has conducted 20
analyses, rather than the 100 needed to test everyone individually.

At a certain threshold, diluting samples by combining them with so many
others might make the virus harder to detect, but the technique has
proved effective in batches of five for P.C.R. testing.
\href{https://academic.oup.com/ajcp/article/153/6/715/5822023}{Nebraska}
was able to stretch its supplies by pooling, except among populations
with high infection rates, which cause more groups to test positive and
thus require more individual assays. ``That can change week to week and
possibly day to day,'' says Jonathan Kolstad, an economist at the
University of California, Berkeley. ``Florida, three months ago, you
could have done pretty big pools. Now you wouldn't want that.'' But, he
and his colleagues note in a working paper published in July in
\href{https://www.nber.org/papers/w27457}{the National Bureau of
Economic Research,} computer modeling could use factors like a person's
age, job, ZIP code and social networks to classify people by their risk
of infection and group their samples accordingly. In theory, as more
people with the virus are removed from circulating among others, the
infection rate will go down and the pools can be expanded, making
testing more efficient. Consequently, the economists' analysis showed,
testing daily would cost only twice as much as testing monthly.

Even when pooled, P.C.R. testing, which detects about 98 percent of
infections and returns very few false positives, is too impractical as a
way to regularly test millions of people. Because it's so precise, it
may identify traces of the virus for weeks after a person has stopped
being contagious, says Michael Mina, a pathologist at the Harvard T.H.
Chan School of Public Health. But significantly less sensitive tests
could still prevent outbreaks --- if the testing was frequent enough and
returned almost instant results. In July, researchers from the Yale
School of Public Health published a study in
\href{https://jamanetwork.com/journals/jamanetworkopen/fullarticle/2768923}{JAMA
Network Open} that models screening scenarios for college campuses. They
found that testing students every two days with a method that detects
just 70 percent of the infections would be able to contain the virus.
This conclusion depended on an important assumption, however: that
students identified as infected are isolated within eight hours and that
all students are taking preventive measures like social distancing and
wearing masks indoors. If students are tested only upon showing
symptoms, outbreaks could not be contained.

\includegraphics{https://static01.graylady3jvrrxbe.onion/images/2020/08/23/magazine/23mag-studies-02/23mag-studies-02-articleLarge.jpg?quality=75\&auto=webp\&disable=upscale}

The same principle also seems to apply in less tightly controlled
environments. A group of researchers including Larremore and Mina
simulated a small city and a large university, in order to model the
effect of tests that varied in their accuracy, frequency and turnaround
time. In a preprint paper
\href{https://www.medrxiv.org/content/10.1101/2020.06.22.20136309v2.full.pdf}{released
at the end of June}, they reported that weekly screening of everyone
with tests that are 100 times less sensitive than P.C.R. tests would
prevent outbreaks if the results were immediate and positive cases
self-isolate. Mina is pushing for rapid federal deployment of at-home
tests, cheap enough to use daily, which several companies are
developing.

But such tests face regulatory hurdles before they can be produced
widely. Other rapid tests that are available now may need to be refined
further before they can be ``operationalized,'' or used effectively in
an actual setting, like a school, according to Dave O'Connor. He and
colleagues in the AIDS Vaccine Reseach Laboratory at the University of
Wisconsin, Madison, have been piloting what is called a loop-mediated
isothermal amplification (LAMP) test, which can be done on saliva, as
part of the N.I.H. Rapid Acceleration of Diagnostics initiative. They're
running their project out of a minivan. ``The first day we tested five
or six people,'' he told me. ``Today we ran 80.''

The question then is whether such screening actually works. A report
published by
\href{https://www.cidrap.umn.edu/sites/default/files/public/downloads/cidrap-covid19-viewpoint-part3.pdf}{the
Center for Infectious Disease Research and Policy at the University of
Minnesota} warns that simply testing a wider swath of the population
will be counterproductive unless the ``the right test is given to the
right person at the right time'' and then that person takes appropriate
action in response. For instance, says Tom Friedrich, who is part of the
Wisconsin effort, do positive cases self-isolate? ``Or do we discover
that it's just really hard for people to modify their behaviors?'' If
overtaxed local health departments are ``making a trade-off to rapidly
test a bunch of asymptomatic people with no known exposure, instead of
testing symptomatic folks'' --- those who are more likely to pose a risk
to others --- ``that's a problem,'' says Angela Ulrich, one of the
Minnesota report's authors.

Recently seven states struck a deal to jointly purchase three million
antigen tests, which look for viral proteins and return results in
minutes. To detect infection, they require that more of the virus be
present than P.C.R. tests do, but, Mina says, they may be able to
``capture the vast majority of people at risk of spreading it at the
time that they're taking it.'' He adds, though: ``It doesn't mean you're
not going to be positive tomorrow or that you weren't positive
yesterday.''

To filter out enough asymptomatic carriers to reduce overall infection
rates, such rapid tests would need to be given every few days, focusing
first, when supplies are limited, on groups with the highest risk of
spreading the virus. ``Imagine that you go to the airport and check your
bags and then you spit in a tube,'' Larremore told me. If your test is
positive, you can't fly and are sent for a diagnostic follow-up. A
system like that, he says, even if set up by individual companies or
schools, would reduce spread and free up more P.C.R. tests for people
with symptoms: ``Every little bit helps here.''

Compared with a diagnostic test, a screening test would give the taker
no definitive health information; being identified as infected would
incur costs like lost wages. But, Larremore says, we shouldn't
underestimate the value of giving information to someone that ``allows
you to protect people around you.'' If we could act on that knowledge
immediately, testing positive, instead of being the frightening news it
often is now, could actually be, Larremore says, ``really empowering.''

Advertisement

\protect\hyperlink{after-bottom}{Continue reading the main story}

\hypertarget{site-index}{%
\subsection{Site Index}\label{site-index}}

\hypertarget{site-information-navigation}{%
\subsection{Site Information
Navigation}\label{site-information-navigation}}

\begin{itemize}
\tightlist
\item
  \href{https://help.nytimes3xbfgragh.onion/hc/en-us/articles/115014792127-Copyright-notice}{©~2020~The
  New York Times Company}
\end{itemize}

\begin{itemize}
\tightlist
\item
  \href{https://www.nytco.com/}{NYTCo}
\item
  \href{https://help.nytimes3xbfgragh.onion/hc/en-us/articles/115015385887-Contact-Us}{Contact
  Us}
\item
  \href{https://www.nytco.com/careers/}{Work with us}
\item
  \href{https://nytmediakit.com/}{Advertise}
\item
  \href{http://www.tbrandstudio.com/}{T Brand Studio}
\item
  \href{https://www.nytimes3xbfgragh.onion/privacy/cookie-policy\#how-do-i-manage-trackers}{Your
  Ad Choices}
\item
  \href{https://www.nytimes3xbfgragh.onion/privacy}{Privacy}
\item
  \href{https://help.nytimes3xbfgragh.onion/hc/en-us/articles/115014893428-Terms-of-service}{Terms
  of Service}
\item
  \href{https://help.nytimes3xbfgragh.onion/hc/en-us/articles/115014893968-Terms-of-sale}{Terms
  of Sale}
\item
  \href{https://spiderbites.nytimes3xbfgragh.onion}{Site Map}
\item
  \href{https://help.nytimes3xbfgragh.onion/hc/en-us}{Help}
\item
  \href{https://www.nytimes3xbfgragh.onion/subscription?campaignId=37WXW}{Subscriptions}
\end{itemize}
