Sections

SEARCH

\protect\hyperlink{site-content}{Skip to
content}\protect\hyperlink{site-index}{Skip to site index}

\href{https://myaccount.nytimes3xbfgragh.onion/auth/login?response_type=cookie\&client_id=vi}{}

\href{https://www.nytimes3xbfgragh.onion/section/todayspaper}{Today's
Paper}

\href{/section/opinion}{Opinion}\textbar{}What Biden Must Do

\url{https://nyti.ms/3aJ9pG0}

\begin{itemize}
\item
\item
\item
\item
\item
\end{itemize}

Advertisement

\protect\hyperlink{after-top}{Continue reading the main story}

transcript

Back to The Argument

bars

0:00/0:00

-0:00

transcript

\hypertarget{what-biden-must-do}{%
\subsection{What Biden Must Do}\label{what-biden-must-do}}

\hypertarget{with-jamelle-bouie-frank-bruni-michelle-cottle-and-michelle-goldberg}{%
\subsubsection{With Jamelle Bouie, Frank Bruni, Michelle Cottle and
Michelle
Goldberg}\label{with-jamelle-bouie-frank-bruni-michelle-cottle-and-michelle-goldberg}}

\hypertarget{a-special-round-table-episode-on-whats-at-stake-during-this-weeks-democratic-convention-plus-a-requiem-for-the-nerd-coachella}{%
\paragraph{A special round-table episode on what's at stake during this
week's Democratic convention. Plus: a requiem for the ``nerd
Coachella.''}\label{a-special-round-table-episode-on-whats-at-stake-during-this-weeks-democratic-convention-plus-a-requiem-for-the-nerd-coachella}}

Thursday, August 20th, 2020

\begin{itemize}
\item
  michelle goldberg\\
  I'm Michelle Goldberg.
\item
  michelle cottle\\
  I'm Michelle Cottle.
\item
  jamelle bouie\\
  I'm Jamelle Bouie.
\item
  frank bruni\\
  And I'm Frank Bruni, and this is ``The Argument.''

  The Democratic convention is finally here, if you can even call it a
  convention. Like all else about a presidential campaign conducted in
  the shadow of a pandemic, it's nothing like what happened in years
  past. Does that negate its usefulness? What can and should Democrats
  accomplish through this socially-distanced virtual extravaganza?

  To answer these questions, we're doing something a little different
  today. Four of us are hosting the show, but none of the four is Ross,
  who's on vacation. And we're gathering on Tuesday morning after the
  first night of the convention. So we're in the middle of a news event,
  which means plenty is going to happen after this recording. But
  whatever transpires, there are questions that will endure between now
  and election day and the morning after. They're questions about the
  Biden-Harris ticket, about the state of the Democratic party, and
  about the threats to the election. So with no further throat-clearing,
  let me welcome our guest co-hosts --- Jamelle Bouie, a columnist on
  the opinion desk, and Michelle Cottle, a member of our editorial
  board. Welcome to you both.
\item
  michelle goldberg\\
  Yeah, welcome, you guys.
\item
  jamelle bouie\\
  Hello.
\item
  michelle cottle\\
  Excellent to be here.
\item
  frank bruni\\
  So let's get right into it, starting with Kamala Harris. She seems to
  have excited people about November, and excitement was something that
  the Biden campaign was sorely lacking up to this point. I want to know
  what you all made of this pick, what its pros and cons are. And so I'm
  going to start with you, Jamelle. Were you happy with this pick, in a
  strategic sense?
\item
  jamelle bouie\\
  Sure. I can't say that I was happy or unhappy about the pick. My range
  of emotions, with regards to politics, don't really go in that
  spectrum. {[}GOLDBERG LAUGHS{]} But as far as thinking through the
  meaning of the pick, I think, as a political pick, it makes a lot of
  sense. And I think among the broader Democratic electorate, Harris is
  a pretty popular figure, a pretty well-liked figure. It may marginally
  increase the number in turnout for African-Americans and other groups.
  I think the weakness of the pick may come if Biden wins and takes
  office and it's time to govern. Because I think one of Harris's
  weaknesses during the primary was it wasn't really quite sure what her
  vision was for governing the country. And given that a Biden
  administration will have to start immediately --- no time for trying
  to figure things out, this is a running start --- I think it matters
  that Harris, as a political figure, doesn't seem to have some sort
  overarching vision for government, especially given the unusually high
  likelihood that she'll end up taking over for Biden at some point,
  given his age.
\item
  michelle cottle\\
  So for me, I was pleased by the pick. But more than that, I was
  relieved. What I wanted was for whoever was chosen to not be
  surprising, at least for people who watch politics even remotely
  closely. I didn't want it to be a story that derailed everything. You
  don't want this campaign, as it's going right now, to get dramatically
  shifted because someone super out of the blue gets put on the ticket.
  And this sounds weird to say because, obviously, she is the first
  African-American woman who is in this position, but she was a
  completely unsurprising pick, almost the safe pick. She's run for high
  office and been publicly vetted at a level that a lot of the people
  who were considered front-runners were not. It also was a nice signal
  that he's not going to hold a grudge. Considering how petty the
  sitting president is, for Biden to have overlooked the early
  squabbling between him and Kamala, it just sent a nice signal. And
  then, of course, she is exciting, on some level. This is going beyond
  the usual ticket of two boring old white guys. It's not exactly what
  the progressive left would have hoped for, but I think it does what it
  needs to do.
\item
  frank bruni\\
  Michelle Goldberg, I want to kind of pivot off of Michelle Cottle's
  mention of the progressive left. Before this choice, there was a lot
  of talk about progressives having issues with Kamala. And it seems to
  me, but tell me if I'm not seeing it correctly, that since the choice,
  there's been real unity. People have fallen in line. Is that your take
  on it, or what do you make of the progressive reaction thus far to
  Kamala's presence on the ticket?
\item
  michelle goldberg\\
  Well, look, I think that there was always going to be sort of hardcore
  Bernie people who were very disappointed by Kamala, although people
  often forget. It's kind of funny that we talk about Kamala as if she
  was some moderate when, by a lot of measures, she's one of the most
  progressive senators. But she has kind of a troubling criminal justice
  record. She's not as progressive as some of the other candidates on
  issues like Israel that matter a lot to a kind of incoming cohort of
  young progressives. She's seen as closer to Wall Street certainly than
  Elizabeth Warren. My sense is that the Biden camp --- I don't know if
  they did this deliberately, but they were very, very smart, I think,
  to make it seem like it was going to be Susan Rice. Because that meant
  that when it was Kamala, almost everybody could feel a sense of
  relief, including people on the left. And I also think Bernie Sanders
  himself has played a really helpful role here, right? Bernie Sanders
  has endorsed this pick and so have other people who have credibility
  on the left. And then finally, I think that just seeing a lot of
  progressive black women who were really, really moved and overjoyed by
  this. And kind of seeing that made it a little bit harder for some
  other people, I think, on the left to quibble overly much because you
  just sort of had to recognize what this meant for a lot of people that
  are admired on the left. The funny thing about this pick is that
  before the entire primary started, if you would asked I think a lot of
  people, what is the most likely Democratic ticket? They would have
  said Biden-Harris. And so we just went through so much drama, the ups
  and downs of the presidential campaign, of the vice presidential
  search to end up pretty much exactly where it always seemed like we
  were going.
\item
  frank bruni\\
  It's like that line in Edward Albee's ``The Zoo Story,'' sometimes you
  have to go a long way out of the way to come a short way back.
  {[}GOLDBERG LAUGHS{]} That's the story of the Democratic primary. We
  could talk for a long time about Kamala and about this ticket. But
  because we've got so much ground to cover, you just mentioned Bernie
  Sanders. He was there on the first night of the Democratic convention
  --- or maybe one of you as a better noun. I'm not sure what we're
  supposed to call this virtual event. But I'm really curious to hear
  what you all made of the event, given the difficulty of doing this
  without an arena, without a crowd that's clapping and screaming,
  whether you thought the choreography of it worked, or whether you
  think this convention is going to end up mattering less than
  conventions ever have because of the coronavirus-mandated adaptations.
\item
  michelle cottle\\
  Well, I had talked to the convention planners about this, and they had
  been thinking of ways to make it less --- not staged, I guess, but
  even before they kind of knew what they were going to be dealing with
  the virus, they wanted to get more regular people in there. And I
  thought that the clips from people talking about their dreams and
  fears and hardships was actually kind of a good addition. These
  conventions tend to be about a lot of schmoozing and stuff behind the
  scenes. And then the public face of them is a long series of really
  boring, for the most part, speeches. And it's like a big advertisement
  for the party. I do think that particularly in this moment, it helps
  to have a little bit of a reminder of what real people are going
  through. And because they couldn't have everybody in Milwaukee and
  they were going to have to do it virtually anyway, you might as well
  take advantage of different people, different places.
\item
  frank bruni\\
  Jamelle, you wrote in the great scorecard that ``The Times'' does with
  various people, grading what's happening at the convention, you
  thought Bernie Sanders was a real highlight and his speech was
  particularly important. Why is that so, not just in terms of the first
  night of the convention, but in terms of now going through Nov. 3?
\item
  jamelle bouie\\
  So I've been listening to Bernie Sanders speeches in one form or
  another for five years, and they typically have kind of the same
  rhythm to them and they hit the same themes. And I thought this one
  was interesting and notable in that it wasn't really a typical Bernie
  Sanders speech at all, and it was very much the sort of plea for
  coalition-building, right? That one at the very beginning that he'll
  work with anyone, even conservative Republicans, to stop Trump I
  thought was very important. Given that the political theme of the
  convention is essentially that Biden is leading a silent majority of
  anti-Trump Americans across kind of a demographic and political
  spectrum, having Sanders as the most prominent left-wing voice in
  American politics say that was really important. I thought it hit the
  exact right note of impressing upon listeners just the stakes of the
  election and the stakes of letting Trump win again. I thought Michelle
  Obama's speech, on a similar point, did the same thing.
\item
  frank bruni\\
  Michelle Obama was, as expected, extraordinary. I was really struck,
  in terms of Sanders, his tone, which you're right to focus on, felt so
  different to me from 2016. I mean, he said vote for Hillary back then.
  The words were correct. The script was what it was supposed to be. But
  I sensed this time around from Bernie Sanders, and really kind of from
  the whole party, a whole new appreciation of the stakes. And Michelle
  Goldberg, it gave me some real hope that between now and Nov. 3 and on
  Nov. 3, we may see no repeat of what happened in 2016. Am I just being
  a ridiculous optimist, or do you have that same hope?
\item
  michelle goldberg\\
  No, I mean, I think that I have that same hope. I think there might be
  a fair number of people who didn't vote in 2016 not even out of some
  sort of deep ideological aversion to Hillary Clinton, but just out of
  a sense that they didn't care that much for her and they thought it
  was in the bag, right? They didn't foresee this utter catastrophe,
  this utter destruction of most of our day-to-day lives. And so yes, I
  think it's very hard for anyone to pull the kind of tweedle-dum,
  tweedle-dee thing that certain people on the left have been pulling
  for most of my adult life. That said, one of the interesting things
  about doing a convention this way is that it gives a lot less power to
  grassroots activists, for better or worse. The kind of people who
  disrupted the convention in 2016 because they were angry about the
  treatment of Bernie Sanders, because they were angry about some of the
  revelations in the emails that were hacked by the Russians at Trump's
  behest, they just didn't have a say in this, right? So there was this
  kind of interesting duality to it. On the one hand, you do hear in
  this convention from a lot more, quote unquote, ``ordinary people,''
  people outside of politics. On the other hand, ordinary activists have
  basically no say in how things proceed because they can't start cheers
  or chants or interrupting. And so it feels more sort of homespun but
  is actually a much more kind of produced and polished affair.
\item
  frank bruni\\
  In terms of Democratic messaging and that which they can control, what
  do you think they need to accomplish in this convention and going
  forward to maximize the chances of the most votes on November 3?
\item
  jamelle bouie\\
  I think maybe the important thing that needs to happen, with regards
  to the entire convention, is the less about what is said and more
  about how what is said is reported. Because what we've sort of learned
  over the past four years is that earned media, how coverage goes out
  to the public is actually really important to shaping voter
  perceptions. You could make a good case that Biden is the nominee
  because the one-two punch of South Carolina and all those endorsements
  created this wave of positive earned media that drove the Democratic
  primary electorate to his side. And so to the extent that this
  convention presents a Democratic party that is unified, that is big
  tent, plenty of room for lots of different people of many different
  beliefs. To the extent that it shows a Democratic party that has a
  strongly contrasting message with the president, then I think it does
  its job. And I think it leaves the kind of impression you want voters
  to have. But beyond that, the specific things said, I don't know. The
  general impression, that's important.
\item
  frank bruni\\
  I think that's a great point about earned media, by which, for the
  listeners who aren't up on the jargon, we mean the reporting that
  happens independent of a political campaign's chosen ads and stuff
  like that. But you're seeing, apart from the programming at the
  convention, a lot of stories right now about how many Republicans have
  turned their backs on Trump and are actually campaigning for Biden.
  And that's the great example of what the party is programming and
  talking about ends up going out into a thousand other media channels
  and reaching the public in different ways.
\item
  michelle goldberg\\
  Also most people are not going to be watching this in a multi-hour
  block. So a lot of what matters is the little clips that get
  repurposed and shown on the news and shown on talk shows. Journalists
  are going to watch the convention in totality, and what's going to
  matter for how it resonates is these kind of little moments that
  become almost memes or become, as you said, earned media. I miss real
  conventions. I see a lot of people, kind of cynical journalists,
  saying, like, oh yeah, conventions suck, and so much nicer to watch
  them from your couch. I actually love political conventions --- maybe
  not Republican ones, certainly not anymore. People used to call the
  White House Correspondents Dinner ``nerd prom.'' The Democratic
  convention is like ``nerd Coachella.'' You see people that you haven't
  seen in forever. There's all these tertiary events. Except for the big
  headline speeches, there's five different places you could be at any
  one time. And so you experience it in this much more fragmentary way.
  I think that it's not ideal to have a convention like this, from the
  party's perspective. But the party does have much, much more control
  in this format of what to highlight. Often, these speeches would
  happen, and the live coverage would break away because you're
  button-holing people in the hallway to interview different people in
  the party. They have much more control over the message in this
  format, and that could end up being good for them.
\item
  frank bruni\\
  Michelle Cottle, the climax, the headline act of this virtual nerd
  Coachella is Joe Biden's speech on Thursday night. You have written
  extensively and wonderfully about the various influences on Joe Biden,
  about his sort of path through the primary to this point. What are you
  expecting, and what are you wanting from it? What can he accomplish
  with this speech that he maybe hasn't accomplished yet but needs to?
\item
  michelle cottle\\
  I don't think he needs to do a lot of complicated things. I think, as
  with the message of the convention in general, he needs to convey the
  basics of compassion and competence. Now, he has a leg up, at this
  point, because the Trump campaign has basically painted him as a
  drooling imbecile sitting in his basement unable to string a sentence
  together. So all he needs to do, on some level, is clear that hurdle.
  He just needs to not pass out during his speech. But he needs to
  convey what I think people look to him for during the primary, which
  is a sense of humanity, a sense of empathy. That is something that,
  obviously, has been sorely lacking with the current administration.
  And the message of all these real people that they're spotlighting at
  the convention talking about what they've been through is to kind of
  send the signal that this is the candidate who cares about this and
  who will actually take steps to improve your lives, that he cares
  about something other than his own ambitions and his own power. So I
  think just of conveying that, I feel your pain, I'm going to fix this.
  And then, of course, being not scary, and this is where the reaching
  out to Republicans, to progressives, the big-tent message will come
  into this. He has stressed that he wants to work with everyone. There
  is a place for everyone in his party. So I think all of those kind of
  basic things need to come through in the speech. He does not have to
  be Barack Obama. He is not a rhetorical god at this point, and he
  doesn't even need to try that. So he just needs to let Joe be Joe, on
  some level.
\item
  frank bruni\\
  Michelle Goldberg, you made the great point that this convention gives
  Democrats more control than ever before because it's not happening at
  an arena, people aren't standing up able to do whatever they want to
  do. Is it going to matter in the end? And in terms of it having to
  have been reinvented and abbreviated and crunched, do you think this
  might be the beginning of conventions mattering less?
\item
  michelle goldberg\\
  I mean, I think it's really hard to say, right? So it's interesting to
  me to hear Jamelle be so sort of Spock-like in his response to this.
  And I'm curious about whether, Jamelle, have you always been like
  this, or was it the trauma of 2016? Because we were together at the
  conventions four years ago, and I feel like I had a lot of feelings.
  Maybe I was just projecting them onto you. But I remember being at the
  Democratic convention, and I wrote about this at the time, it was
  impeccable. Joe Biden gave one of the best speeches I've ever heard
  him give at that convention. Michelle Obama gave an amazing speech.
  Compared to the total shit show of Trump's convention, which was this
  kind of pathetic shambolic thing with Scott Baio talking and a bunch
  of Nazis outside, I don't think even the best-produced convention can
  change the political dynamics in the country. I do think a couple of
  things. I think first of all, conventions, beyond what they do for a
  candidate a couple months before the election, it's good for a party
  to come together. It's good for people to build these relationships.
  That's why you have conventions not just of political parties, but of
  all sorts of organizations. It fosters cohesion. And so for that
  reason, I really hope that they resume them once they can. I also
  think that, again, because Donald Trump has chosen this particular
  attack on Joe Biden --- that he's senile, that he can't talk, that he
  has to be hidden away from view --- if he just gives a kind of mildly
  competent speech, a sort of ordinary, boilerplate political acceptance
  speech, it will do something to puncture that narrative. It seems ---
  knock on wood --- that it was potentially, from Trump's point of view,
  a bad idea to lower expectations about your opponent to such levels.
\item
  frank bruni\\
  I was just going to echo the points that both Michelle Goldberg and
  Michelle Cottle made that it's really strange, just from a political
  strategy standpoint, to attack your opponent as a bumbling idiot and
  someone incapable of stringing together a sentence ahead of a major
  event where people will see that candidate string together a sentence,
  a series of sentences, and appear competent. If there's any political
  advantage in this moment that the Democratic ticket has is that the
  Republican attacks --- the Trump attacks, specifically --- are just so
  removed from reality that once people get any kind of direct contact
  with that ticket, they're going to be like, oh, well obviously the
  Trump campaign is full of crap, right? So Biden is a drooling
  imbecile, not going to be the case. Kamala Harris is an anti-police
  radical, people are going to see her speak and be like, that's
  nonsense. It's not going to be the case. The ticket as a whole is in
  the thrall of leftists, well, we just saw John Kasich give a four
  minute speech, so that's probably not true. And I think the Trump
  campaign, because it cannot actually campaign on its record or on
  current conditions, is throwing everything at the wall and will find
  that that just doesn't work, that a voting public actually wants to
  hear about what each ticket has to say about things happening now and
  that the Democratic ticket is mainstream to a fault, you could say.
  There's nothing there that's going to cause people to think, oh, these
  people are dangerous or radical.
\item
  michelle goldberg\\
  But Jamelle, can I ask you a question? Both of us have seen Joe Biden
  speak a lot of times, and he's pretty hit or miss. Like I said, in
  2016, I thought he gave one of the best speeches at the convention.
  I've seen him give speeches on the campaign trail that left me with my
  heart in my throat, feeling like I was watching an actor forgetting
  his lines. This is such an important moment. Do you have any anxiety
  about how it's going to come off?
\item
  jamelle bouie\\
  He's always done worse in his impromptu moments. But a prepared
  speech, he's perfectly competent of giving. He's been giving them for
  his entire adult life. So I'm not too worried about that. The things
  I'm worried about have really less to do with the Democratic ticket or
  with the Democratic coalition, although there are, of course, tensions
  and divides to talk about within that coalition. But the things I'm
  worried about have just much more to do with election administration
  and the entire process of the election and how that's going to
  interact with the president's, I think, clear efforts to delegitimize
  the outcome and also the Republican Party, how it relates to those
  attempts, right? Because for as much as the conversation has been
  about Trump's attacks on the election, kind of the key actor on Nov. 3
  and Nov. 4, depending on the outcome, will just be Republican-elected
  officials, how they react to what's happened.
\item
  frank bruni\\
  That's a great point, Jamelle. Let's use that as a pivot because this
  convention is four days in time. It will recede in the rearview mirror
  really quickly. But all around it --- and Michelle Goldberg, you just
  wrote about this --- we have this very clear evidence that Trump is
  cheating, is preparing to cheat yet more. We have this huge melodrama
  with the United States Postal Service. We have various other things
  the president has said and done to convince people that if the result
  of this election is not what they want, they can consider it
  illegitimate. How much does each of you, and let's start with Michelle
  Cottle, worry about this truly being anything close to a free and fair
  election?
\item
  michelle cottle\\
  Oh, I'm completely terrified on some level. Cheater's gonna cheat. And
  I think that they've made clear they have no problem with doing
  whatever is necessary to undermine the process. I also worry about
  what happens if it is a remotely close election because all of
  President Trump's efforts to delegitimize this become that much more
  of an issue if they are like, oh, well look, it was so close. They
  cheated. It's all the systemic fraud from the mail-in voting. I think
  that the larger the margin is, the easier it will be for at least
  respectable Republican officials to say, OK, we need to move on, even
  if that infuriates certain segments of the base.
\item
  frank bruni\\
  That's what Michelle Obama was saying on Monday night at the
  convention. I thought one of the most fascinating parts of that
  extraordinary, excellent speech was when she was essentially saying,
  get your ballot tonight. Mail it in now, or prepare to go. Pack dinner
  and breakfast in case you have to wait on line. She was basically
  saying to America, you have got to deliver the Biden-Harris ticket
  such a margin that the cheating is erased and that there's no ability
  for Donald Trump to say that this is a fraudulent result. I mean, that
  was such a kind of appropriately cynical instruction to the American
  public, don't you think?
\item
  michelle goldberg\\
  Well, and it's not just the numbers that matter. It's also the timing.
  Jamelle wrote about this that right now, there's this asymmetry. And
  the Democrats are saying that they're more likely to vote by mail
  because they are more likely to take coronavirus seriously and
  Republicans the opposite. And what that could end up meaning is if you
  look back to 2018, in the midterms, and on election night, it looked
  like Democrats had had a really disappointing night. And it took the
  days and weeks afterwards, as the full vote was counted, to realize
  that there really was a blue wave. If something like that happens with
  Trump, he's extremely likely to just declare victory on election night
  and to try to shut down the vote count afterwards. And so if you think
  about, for those of us who are old enough to remember 2000, what
  happened with the fight over the recount in Florida and you multiply
  that by many states or many jurisdictions, and think back to the fact
  that Roger Stone, Trump's criminal advisor, was involved in basically
  staging what is now known as the Brooks Brothers riot where a bunch of
  rowdy Republican operatives physically interfered with the vote count
  in a way that very likely changed the course of that election, while
  hapless Democrats relied on the courts. I do think that Democrats are
  not going to do that this time. They're not just going to assume that
  the rule of law will prevail and the courts will protect them. And as
  I wrote today, people are already sort of organizing for the mass
  protest that's going to be necessary if there is some sort of
  shenanigans with the vote count or if Trump tries to interfere within
  in some way. In 2016, the night Trump was elected, when I was
  envisioning all sorts of disasters, could I have imagined that he
  would try to destroy the post office in order to stop people from
  casting their ballots? Probably not. And so I don't think we should
  assume that that is the last important American institution that he
  will use his power to try and wreck if he thinks that his holding onto
  power depends on it.
\item
  frank bruni\\
  Right, so there are things that are outside of the Democratic party's
  control. What can and should it do between now and November 3 to try
  to prevent the worst happening, to try to make this election as free
  and fair as it can possibly be? Jamelle, do you have any thoughts on
  that?
\item
  jamelle bouie\\
  Yeah, I mean, I think Michelle Goldberg got to much of what would be
  included in that. It is first, providing for voters detailed
  instructions on how to get a ballot, how to cast a ballot, how to do
  it in person. It's organizing efforts for people who need to go in
  person, to bring them in person. I think it would probably include
  distributing masks and stuff to voters who want them to kind of help
  people get themselves prepared for standing in line. Similar to what
  the R.N.C. and the Trump campaign are doing, kind of a nationwide plan
  to contest vote counts, which they've sort of announced that this is
  part of their election plan. I think the Democratic party and its
  affiliated institutions and organizations need to be prepping
  basically a defense of ballot counting, sort of a nationwide effort to
  stop that tampering. I think there needs to be plans for protests and
  demonstrations. This is going to sound very hyperbolic, but I think
  that we have to think of the task of getting Trump out as less of a
  traditional democratic transition and more of something akin to
  pushing an authoritarian regime out. And that's going to mean in
  addition to at the ballot box on Nov. 3, there being a Biden lead at
  the end of the night. Because I think Trump is going to contest
  regardless. Regardless of the outcome, Trump is going to say, this was
  rigged. This is fake, whatever. But the key thing is, is there space
  for the Republican legislature in Pennsylvania or Michigan or
  Wisconsin to certify election night results and send those in before
  all the votes are counted to force basically a contest between the
  full count that's if a governor certifies and the partial count that
  the legislature does. And demonstrations, a large vote count on
  election night, all those sorts of things send an important signal to
  Republican legislators and officials that if they take this step, they
  will be taking a step that directly affects their ability to hold
  power in the future. That's the message of a demonstration --- not for
  Trump because Trump I don't think gives a shit.
\item
  michelle cottle\\
  Now, there's also something that the media needs to do. They need to
  start setting expectations, just as Trump is very aggressively pushing
  the idea that election night has to be it. You have to call this race
  on election night. I think there needs to be a fair amount of pushback
  where people are told, this very well could go on past the one night
  that everybody thinks of as the deciding moment, especially if it's
  not a huge margin. And people have talked about the blue shift, where
  Democratic votes have been coming in on the later side. So what starts
  out looking like a close call then widens to be a more comfortable
  Democratic margin. Trump's going to go berserk over things like that,
  and people need to be prepared but that is not cheating. That is not
  rigging this vote, that election night is no longer a drop-dead kind
  of moment. Because he's been pushing the idea that if it's not called
  on that night, it's fake. And that's just something that everyone else
  has to say, no, that's nonsense.
\item
  frank bruni\\
  Can I push back on one thing? Everything you say is 100 percent true,
  but I want any one of you who can reassure me to reassure me. We can
  report that, Michelle. We at The Times, our colleagues at the
  Washington Post, people on the responsible cable networks can do
  everything you're saying, and yet we live in an information ecosystem
  where we're never going to reach some of the people who most need to
  hear exactly what you said. What do we do about that?
\item
  michelle cottle\\
  You're never going to get everybody. I mean, you don't worry about
  something just because it's not going to be 100 percent. But you're
  also talking to Republican officials. You need to put the pressure on
  them. It needs to be one of these things where only the fraction of
  the base who is going to believe whatever, Trump shoot somebody on
  Fifth Avenue, they're cool with that. Everybody else needs to know the
  reality of the situation. You can't break into that weird alternative
  bubble with everything. But everybody else who's not drinking the
  Kool-Aid needs to understand this. The more people you can get to
  understand it, the more pressure gets brought to bear on people like
  Susan Collins, assuming she's still in the Senate at the end of this
  race.
\item
  michelle goldberg\\
  I feel like I could see something maybe mildly reassuring, which is
  out of character for me, which is that usually raising the salience of
  an issue like mail-in voting is a pretty difficult thing to do. It's
  hard to get people excited about these procedural issues. The one
  useful thing that Trump has done by being both so blatant about the
  reason that he's sabotaging the post office, and by just sabotaging
  the post office in a way that affects the lives of a lot of people who
  don't follow politics very closely, is that people understand what's
  going on. Even if they don't understand all the nuances of it, they
  understand that they're not getting their mail. They understand that
  they're not getting their prescriptions. They understand that checks
  that they need are coming late. And so it has raised the public
  salience of an issue that otherwise might seem like a bit of
  procedural minutia.
\item
  jamelle bouie\\
  There's a lot that complain about the fact that cable news media is
  responsible for reporting election results. Every state has a
  Secretary of State's office, but people get the results by way of the
  media. But the advantage in this case is that everyone gets the
  results by way of the media. And so if there's just a collective
  agreement among the non-Fox networks to say we're not going to report
  percentage precincts. We're going to report the amount of the vote
  that has come in. We're not going to do projected winners until a
  certain percentage of the vote that we know probably went out has come
  in. Those sorts of things, if there's a general agreement to do them,
  I think will help. Because in 2000, it was the opposite. Networks were
  rushing to get ahead of each other in announcing who won and who
  didn't. And that actually helped create the space for the shenanigans
  in Florida.
\item
  frank bruni\\
  And maybe we all learned a lesson we won't create that space this time
  around. There's been a lot of kind of apocalyptic sentiment during the
  last few minutes of this. So I want to wrap up our discussion by
  leaving our listeners on a slightly maybe more positive, optimistic
  note. I want to ask each of you, and take it in whatever order you
  like, what are you seeing at this convention or what are you seeing in
  terms of the dynamics of this race that does give you great hope that
  the right thing will happen on November 3 or in the days after
  November 3, as the count continues, and that we will get past these
  last four years? Why don't we start with Michelle Cottle.
\item
  michelle cottle\\
  Well, I think there's clearly much less internal squabbling among
  Democrats. We do not have the small differences battles that were such
  a problem in 2016. And I think Hillary Clinton was an incredibly
  controversial candidate. I know lots of Democrats who just couldn't
  even hold their nose and stomach her. So I think that Biden, for
  better and worse, does not incite those kind of passions. And so it's
  been much easier on both levels, both from you know what the stakes
  are and the Biden isn't scary. Much easier to bring the party together
  in this situation.
\item
  michelle goldberg\\
  I just want to say quickly, I certainly know people who were
  enthusiastic about Hillary Clinton and are going to have a hard time
  voting for Joe Biden, even though they're going to do it. So I don't
  think that Joe Biden is completely anodyne as a candidate, although I
  agree, he partly just, as a white man, doesn't inspire the sort of
  kind of panic and hysteria that Hillary Clinton has always inspired.
  So the thing that gives me a little bit of hope --- and I'm not the
  person you turn to for hope --- is as I said before, that I just think
  Democrats have wised up about relying on the courts and sort of
  procedural neutrality to sort this out. If you go back again and read
  about the Brooks Brothers riots and other sort of shenanigans around
  the 2000 election, Republicans had a multi-pronged strategy. They
  fought it in the courts, but they also fought it in the streets. They
  fought it in the court of public opinion. Democrats, at that time,
  really just had faith in the process and faith that the rules,
  neutrally applied, would see justice done. And I think that that is
  gone, and you have a lot of people now on hair trigger who don't trust
  a lot of these institutions. They certainly don't trust the courts.
  And so there's going to be much more of an uproar if you see anything
  unfair, anything sort of un-kosher about the way that the vote is
  administered or the vote is counted. I just think that the Democratic
  base is much more activated and much more attuned and kind of much
  more ready to take to the streets.
\item
  jamelle bouie\\
  Somewhat counterintuitively, my ray of hope is exactly that Trump is
  trying to corrupt and destroy the election. This is not something that
  a confident authoritarian does. It's not something that a confident
  candidate does. It reflects profound fear and insecurity, that Trump
  almost certainly knows that by any conventional measure, he's losing
  the election, that he knows that the public is against him. That's why
  he's been hyping up these boat parades, flotillas of supporters, that
  the real popular vote is the people who own boats and who fly Trump
  flags. But I think he and his team are very aware of what the
  political mood is against him. And so these, to me, kind of desperate
  attempts to shift the ground on which the election is actually being
  fought --- obviously, there should be vigilance and preparation ---
  but it should also I think convince Trump's opponents --- not just
  Democrats, but sort of the broad anti-Trump coalition --- that they're
  moving in the right direction and that they have the initiative here,
  that they're not on the defensive at the end of the day. I would be
  worried if Trump seemed super confident. That would fill me with fear
  because that's sort of the I'm doing something that you people don't
  know about, are not going to know about, may not ever be able to
  prove. But flailing like a child, that actually makes me feel a little
  more confident.
\item
  frank bruni\\
  That gives you your hope. I tell you what gives me hope. This
  convention gives me hope, in the sense that at every turn--- every 15
  minutes, every five minutes --- someone says something that tells me
  that people appreciate the stakes here. Even though everyone in 2016
  who was for Hillary Clinton and who was worried about Donald Trump
  certainly didn't think he was just any Republican who might win, I
  still think the sense of stakes this time around that this is a
  crossroads, that we can't go down the wrong fork in the road, that
  this is a kind of do-or-die moment for America, I'm hearing that loud
  and clear in this convention. And that gives me some hope that voters
  will bring the right urgency to November 3 and do the right thing.
  Let's stop there. We are going to take a short break, and we will be
  back.

  And we're back. And it's that time when someone gives you all a
  recommendation to leave with. This week, I believe Michelle Cottle has
  something for us. Michelle, you have a recommendation?
\item
  michelle cottle\\
  I do, and it might work best for parents. But you know, it could be
  adapted in some way. So obviously, I cover national politics for the
  board, so I just can't live in that 24/7. So I have two teenagers. I
  have a 15-year-old daughter who's a Broadway drama nerd and a
  17-year-old son who plays any instrument and is in his heavy classic
  rock phase. So he knows more about `70s and `80s music than I do. And
  one extremely effective way to waste time and get my mind a wash is we
  have Sonos on our phones that connects to a stereo speaker in the
  house. And you each take turns picking songs and playing them. And so
  you get into this kind of groove where everybody goes and starts
  picking the music, and then you start fighting over whose music is the
  worst or who's the best or who did this guitar riff better than
  whatever. And then if my husband gets involved, he likes to sing. He's
  a frustrated Broadway diva. And so then the children are appalled, and
  they're, like, yelling at me to make dad stop. And so it can go on
  like this for hours. And for those who have teenagers, my teenagers
  spend most of their time sleeping until 2 pm and then staring at their
  screens during this pandemic because they have nowhere to go. So this
  is a fun way to get everybody kind of thinking about something that
  they love. And I don't sing, but that makes it even more fun for them
  because they can abuse me mercilessly when I feel compelled to break
  into song.
\item
  frank bruni\\
  It's sort of like an old-fashioned jukebox.
\item
  michelle cottle\\
  It is a jukebox. And you can cut off each other because you can break
  in with your phones on each other's song. And at some point, it always
  degenerates into someone playing ``What's New Pussycat'' at top
  volume. {[}GOLDBERG LAUGHS{]} I will not go into the whys of that. But
  at that point, you know it's time to go to bed, really. It's over.
\item
  frank bruni\\
  Michelle, Jamelle, you both have children. Does this sound like fun
  for your whole families?
\item
  michelle goldberg\\
  I mean, the time that we listen to music, we panic-bought a car during
  the pandemic. And so now we drive, which my kids aren't used to. And
  they hate it, and they complain the whole time. But especially my
  five-year-old has very pronounced musical tastes. And the other day,
  for the first time, they asked us to play a song that we weren't
  familiar with, ``Exile'' by Taylor Swift, which is actually a great
  song. And I'm not quite sure where they heard it. But it's kind of
  fascinating to watch them develop musical personalities that are not
  just the albums that we've introduced them to.
\item
  frank bruni\\
  Jamelle, you have just been made an honorary member of the Cottle
  family, and you are playing the Sonos challenge. What song are you
  queuing up?
\item
  jamelle bouie\\
  I am cueing up ``Girls On Film'' by Duran Duran.
\item
  frank bruni\\
  Wow.
\item
  michelle goldberg\\
  Wow.
\item
  frank bruni\\
  Duran Duran, I'm back in college. Yeesh. Michelle Cottle, Jamelle,
  thank you both so much for coming on the show this convention week,
  and we look forward to welcoming you back in the future.
\item
  michelle goldberg\\
  Yeah, thank you, guys.
\item
  jamelle bouie\\
  Thank you for having us.
\item
  frank bruni\\
  Meanwhile, Michelle Goldberg and I will be back next week. That's our
  show for this week. Thank you all for listening. If you have a
  question you want to hear us debate, share it with us in a voicemail
  by calling 347-915-4324. You can also email us at
  \href{mailto:argument@NYTimes.com}{\nolinkurl{argument@NYTimes.com}}.
  ``The Argument'' is a production of The New York Times Opinion
  section. The team includes Phoebe Lett, Paula Szuchman, Pedro Rafael
  Rosado, Vishakha Darbha, Kristin Lin, and Isaac Jones. We'll see you
  next week.
\item
  jamelle bouie\\
  I was about to say the two Michelle's. But I wasn't exactly sure, is
  that OK to say?
\item
  michelle cottle\\
  No, that's good.
\item
  jamelle bouie\\
  OK.
\item
  michelle goldberg\\
  Yeah, it's not like it's Karen. {[}ALL LAUGH{]}
\end{itemize}

\href{https://www.nytimes3xbfgragh.onion/column/the-argument}{\includegraphics{https://static01.graylady3jvrrxbe.onion/images/2018/10/03/opinion/the-argument-album-art/the-argument-album-art-square320-v3.png}The
Argument}Subscribe:

\begin{itemize}
\tightlist
\item
  \href{https://itunes.apple.com/us/podcast/id1438024613}{Apple
  Podcasts}
\item
  \href{https://www.google.com/podcasts?feed=aHR0cHM6Ly9yc3MuYXJ0MTkuY29tL3RoZS1hcmd1bWVudA\%3D\%3D}{Google
  Podcasts}
\end{itemize}

\hypertarget{what-biden-must-do-1}{%
\section{What Biden Must Do}\label{what-biden-must-do-1}}

\hypertarget{a-special-round-table-episode-on-whats-at-stake-during-this-weeks-democratic-convention-plus-a-requiem-for-the-nerd-coachella-1}{%
\subsection{A special round-table episode on what's at stake during this
week's Democratic convention. Plus: a requiem for the ``nerd
Coachella.''}\label{a-special-round-table-episode-on-whats-at-stake-during-this-weeks-democratic-convention-plus-a-requiem-for-the-nerd-coachella-1}}

With Jamelle Bouie, Frank Bruni, Michelle Cottle and Michelle Goldberg

Transcript

transcript

Back to The Argument

bars

0:00/0:00

-0:00

transcript

\hypertarget{what-biden-must-do-2}{%
\subsection{What Biden Must Do}\label{what-biden-must-do-2}}

\hypertarget{with-jamelle-bouie-frank-bruni-michelle-cottle-and-michelle-goldberg-1}{%
\subsubsection{With Jamelle Bouie, Frank Bruni, Michelle Cottle and
Michelle
Goldberg}\label{with-jamelle-bouie-frank-bruni-michelle-cottle-and-michelle-goldberg-1}}

\hypertarget{a-special-round-table-episode-on-whats-at-stake-during-this-weeks-democratic-convention-plus-a-requiem-for-the-nerd-coachella-2}{%
\paragraph{A special round-table episode on what's at stake during this
week's Democratic convention. Plus: a requiem for the ``nerd
Coachella.''}\label{a-special-round-table-episode-on-whats-at-stake-during-this-weeks-democratic-convention-plus-a-requiem-for-the-nerd-coachella-2}}

Thursday, August 20th, 2020

\begin{itemize}
\item
  michelle goldberg\\
  I'm Michelle Goldberg.
\item
  michelle cottle\\
  I'm Michelle Cottle.
\item
  jamelle bouie\\
  I'm Jamelle Bouie.
\item
  frank bruni\\
  And I'm Frank Bruni, and this is ``The Argument.''

  The Democratic convention is finally here, if you can even call it a
  convention. Like all else about a presidential campaign conducted in
  the shadow of a pandemic, it's nothing like what happened in years
  past. Does that negate its usefulness? What can and should Democrats
  accomplish through this socially-distanced virtual extravaganza?

  To answer these questions, we're doing something a little different
  today. Four of us are hosting the show, but none of the four is Ross,
  who's on vacation. And we're gathering on Tuesday morning after the
  first night of the convention. So we're in the middle of a news event,
  which means plenty is going to happen after this recording. But
  whatever transpires, there are questions that will endure between now
  and election day and the morning after. They're questions about the
  Biden-Harris ticket, about the state of the Democratic party, and
  about the threats to the election. So with no further throat-clearing,
  let me welcome our guest co-hosts --- Jamelle Bouie, a columnist on
  the opinion desk, and Michelle Cottle, a member of our editorial
  board. Welcome to you both.
\item
  michelle goldberg\\
  Yeah, welcome, you guys.
\item
  jamelle bouie\\
  Hello.
\item
  michelle cottle\\
  Excellent to be here.
\item
  frank bruni\\
  So let's get right into it, starting with Kamala Harris. She seems to
  have excited people about November, and excitement was something that
  the Biden campaign was sorely lacking up to this point. I want to know
  what you all made of this pick, what its pros and cons are. And so I'm
  going to start with you, Jamelle. Were you happy with this pick, in a
  strategic sense?
\item
  jamelle bouie\\
  Sure. I can't say that I was happy or unhappy about the pick. My range
  of emotions, with regards to politics, don't really go in that
  spectrum. {[}GOLDBERG LAUGHS{]} But as far as thinking through the
  meaning of the pick, I think, as a political pick, it makes a lot of
  sense. And I think among the broader Democratic electorate, Harris is
  a pretty popular figure, a pretty well-liked figure. It may marginally
  increase the number in turnout for African-Americans and other groups.
  I think the weakness of the pick may come if Biden wins and takes
  office and it's time to govern. Because I think one of Harris's
  weaknesses during the primary was it wasn't really quite sure what her
  vision was for governing the country. And given that a Biden
  administration will have to start immediately --- no time for trying
  to figure things out, this is a running start --- I think it matters
  that Harris, as a political figure, doesn't seem to have some sort
  overarching vision for government, especially given the unusually high
  likelihood that she'll end up taking over for Biden at some point,
  given his age.
\item
  michelle cottle\\
  So for me, I was pleased by the pick. But more than that, I was
  relieved. What I wanted was for whoever was chosen to not be
  surprising, at least for people who watch politics even remotely
  closely. I didn't want it to be a story that derailed everything. You
  don't want this campaign, as it's going right now, to get dramatically
  shifted because someone super out of the blue gets put on the ticket.
  And this sounds weird to say because, obviously, she is the first
  African-American woman who is in this position, but she was a
  completely unsurprising pick, almost the safe pick. She's run for high
  office and been publicly vetted at a level that a lot of the people
  who were considered front-runners were not. It also was a nice signal
  that he's not going to hold a grudge. Considering how petty the
  sitting president is, for Biden to have overlooked the early
  squabbling between him and Kamala, it just sent a nice signal. And
  then, of course, she is exciting, on some level. This is going beyond
  the usual ticket of two boring old white guys. It's not exactly what
  the progressive left would have hoped for, but I think it does what it
  needs to do.
\item
  frank bruni\\
  Michelle Goldberg, I want to kind of pivot off of Michelle Cottle's
  mention of the progressive left. Before this choice, there was a lot
  of talk about progressives having issues with Kamala. And it seems to
  me, but tell me if I'm not seeing it correctly, that since the choice,
  there's been real unity. People have fallen in line. Is that your take
  on it, or what do you make of the progressive reaction thus far to
  Kamala's presence on the ticket?
\item
  michelle goldberg\\
  Well, look, I think that there was always going to be sort of hardcore
  Bernie people who were very disappointed by Kamala, although people
  often forget. It's kind of funny that we talk about Kamala as if she
  was some moderate when, by a lot of measures, she's one of the most
  progressive senators. But she has kind of a troubling criminal justice
  record. She's not as progressive as some of the other candidates on
  issues like Israel that matter a lot to a kind of incoming cohort of
  young progressives. She's seen as closer to Wall Street certainly than
  Elizabeth Warren. My sense is that the Biden camp --- I don't know if
  they did this deliberately, but they were very, very smart, I think,
  to make it seem like it was going to be Susan Rice. Because that meant
  that when it was Kamala, almost everybody could feel a sense of
  relief, including people on the left. And I also think Bernie Sanders
  himself has played a really helpful role here, right? Bernie Sanders
  has endorsed this pick and so have other people who have credibility
  on the left. And then finally, I think that just seeing a lot of
  progressive black women who were really, really moved and overjoyed by
  this. And kind of seeing that made it a little bit harder for some
  other people, I think, on the left to quibble overly much because you
  just sort of had to recognize what this meant for a lot of people that
  are admired on the left. The funny thing about this pick is that
  before the entire primary started, if you would asked I think a lot of
  people, what is the most likely Democratic ticket? They would have
  said Biden-Harris. And so we just went through so much drama, the ups
  and downs of the presidential campaign, of the vice presidential
  search to end up pretty much exactly where it always seemed like we
  were going.
\item
  frank bruni\\
  It's like that line in Edward Albee's ``The Zoo Story,'' sometimes you
  have to go a long way out of the way to come a short way back.
  {[}GOLDBERG LAUGHS{]} That's the story of the Democratic primary. We
  could talk for a long time about Kamala and about this ticket. But
  because we've got so much ground to cover, you just mentioned Bernie
  Sanders. He was there on the first night of the Democratic convention
  --- or maybe one of you as a better noun. I'm not sure what we're
  supposed to call this virtual event. But I'm really curious to hear
  what you all made of the event, given the difficulty of doing this
  without an arena, without a crowd that's clapping and screaming,
  whether you thought the choreography of it worked, or whether you
  think this convention is going to end up mattering less than
  conventions ever have because of the coronavirus-mandated adaptations.
\item
  michelle cottle\\
  Well, I had talked to the convention planners about this, and they had
  been thinking of ways to make it less --- not staged, I guess, but
  even before they kind of knew what they were going to be dealing with
  the virus, they wanted to get more regular people in there. And I
  thought that the clips from people talking about their dreams and
  fears and hardships was actually kind of a good addition. These
  conventions tend to be about a lot of schmoozing and stuff behind the
  scenes. And then the public face of them is a long series of really
  boring, for the most part, speeches. And it's like a big advertisement
  for the party. I do think that particularly in this moment, it helps
  to have a little bit of a reminder of what real people are going
  through. And because they couldn't have everybody in Milwaukee and
  they were going to have to do it virtually anyway, you might as well
  take advantage of different people, different places.
\item
  frank bruni\\
  Jamelle, you wrote in the great scorecard that ``The Times'' does with
  various people, grading what's happening at the convention, you
  thought Bernie Sanders was a real highlight and his speech was
  particularly important. Why is that so, not just in terms of the first
  night of the convention, but in terms of now going through Nov. 3?
\item
  jamelle bouie\\
  So I've been listening to Bernie Sanders speeches in one form or
  another for five years, and they typically have kind of the same
  rhythm to them and they hit the same themes. And I thought this one
  was interesting and notable in that it wasn't really a typical Bernie
  Sanders speech at all, and it was very much the sort of plea for
  coalition-building, right? That one at the very beginning that he'll
  work with anyone, even conservative Republicans, to stop Trump I
  thought was very important. Given that the political theme of the
  convention is essentially that Biden is leading a silent majority of
  anti-Trump Americans across kind of a demographic and political
  spectrum, having Sanders as the most prominent left-wing voice in
  American politics say that was really important. I thought it hit the
  exact right note of impressing upon listeners just the stakes of the
  election and the stakes of letting Trump win again. I thought Michelle
  Obama's speech, on a similar point, did the same thing.
\item
  frank bruni\\
  Michelle Obama was, as expected, extraordinary. I was really struck,
  in terms of Sanders, his tone, which you're right to focus on, felt so
  different to me from 2016. I mean, he said vote for Hillary back then.
  The words were correct. The script was what it was supposed to be. But
  I sensed this time around from Bernie Sanders, and really kind of from
  the whole party, a whole new appreciation of the stakes. And Michelle
  Goldberg, it gave me some real hope that between now and Nov. 3 and on
  Nov. 3, we may see no repeat of what happened in 2016. Am I just being
  a ridiculous optimist, or do you have that same hope?
\item
  michelle goldberg\\
  No, I mean, I think that I have that same hope. I think there might be
  a fair number of people who didn't vote in 2016 not even out of some
  sort of deep ideological aversion to Hillary Clinton, but just out of
  a sense that they didn't care that much for her and they thought it
  was in the bag, right? They didn't foresee this utter catastrophe,
  this utter destruction of most of our day-to-day lives. And so yes, I
  think it's very hard for anyone to pull the kind of tweedle-dum,
  tweedle-dee thing that certain people on the left have been pulling
  for most of my adult life. That said, one of the interesting things
  about doing a convention this way is that it gives a lot less power to
  grassroots activists, for better or worse. The kind of people who
  disrupted the convention in 2016 because they were angry about the
  treatment of Bernie Sanders, because they were angry about some of the
  revelations in the emails that were hacked by the Russians at Trump's
  behest, they just didn't have a say in this, right? So there was this
  kind of interesting duality to it. On the one hand, you do hear in
  this convention from a lot more, quote unquote, ``ordinary people,''
  people outside of politics. On the other hand, ordinary activists have
  basically no say in how things proceed because they can't start cheers
  or chants or interrupting. And so it feels more sort of homespun but
  is actually a much more kind of produced and polished affair.
\item
  frank bruni\\
  In terms of Democratic messaging and that which they can control, what
  do you think they need to accomplish in this convention and going
  forward to maximize the chances of the most votes on November 3?
\item
  jamelle bouie\\
  I think maybe the important thing that needs to happen, with regards
  to the entire convention, is the less about what is said and more
  about how what is said is reported. Because what we've sort of learned
  over the past four years is that earned media, how coverage goes out
  to the public is actually really important to shaping voter
  perceptions. You could make a good case that Biden is the nominee
  because the one-two punch of South Carolina and all those endorsements
  created this wave of positive earned media that drove the Democratic
  primary electorate to his side. And so to the extent that this
  convention presents a Democratic party that is unified, that is big
  tent, plenty of room for lots of different people of many different
  beliefs. To the extent that it shows a Democratic party that has a
  strongly contrasting message with the president, then I think it does
  its job. And I think it leaves the kind of impression you want voters
  to have. But beyond that, the specific things said, I don't know. The
  general impression, that's important.
\item
  frank bruni\\
  I think that's a great point about earned media, by which, for the
  listeners who aren't up on the jargon, we mean the reporting that
  happens independent of a political campaign's chosen ads and stuff
  like that. But you're seeing, apart from the programming at the
  convention, a lot of stories right now about how many Republicans have
  turned their backs on Trump and are actually campaigning for Biden.
  And that's the great example of what the party is programming and
  talking about ends up going out into a thousand other media channels
  and reaching the public in different ways.
\item
  michelle goldberg\\
  Also most people are not going to be watching this in a multi-hour
  block. So a lot of what matters is the little clips that get
  repurposed and shown on the news and shown on talk shows. Journalists
  are going to watch the convention in totality, and what's going to
  matter for how it resonates is these kind of little moments that
  become almost memes or become, as you said, earned media. I miss real
  conventions. I see a lot of people, kind of cynical journalists,
  saying, like, oh yeah, conventions suck, and so much nicer to watch
  them from your couch. I actually love political conventions --- maybe
  not Republican ones, certainly not anymore. People used to call the
  White House Correspondents Dinner ``nerd prom.'' The Democratic
  convention is like ``nerd Coachella.'' You see people that you haven't
  seen in forever. There's all these tertiary events. Except for the big
  headline speeches, there's five different places you could be at any
  one time. And so you experience it in this much more fragmentary way.
  I think that it's not ideal to have a convention like this, from the
  party's perspective. But the party does have much, much more control
  in this format of what to highlight. Often, these speeches would
  happen, and the live coverage would break away because you're
  button-holing people in the hallway to interview different people in
  the party. They have much more control over the message in this
  format, and that could end up being good for them.
\item
  frank bruni\\
  Michelle Cottle, the climax, the headline act of this virtual nerd
  Coachella is Joe Biden's speech on Thursday night. You have written
  extensively and wonderfully about the various influences on Joe Biden,
  about his sort of path through the primary to this point. What are you
  expecting, and what are you wanting from it? What can he accomplish
  with this speech that he maybe hasn't accomplished yet but needs to?
\item
  michelle cottle\\
  I don't think he needs to do a lot of complicated things. I think, as
  with the message of the convention in general, he needs to convey the
  basics of compassion and competence. Now, he has a leg up, at this
  point, because the Trump campaign has basically painted him as a
  drooling imbecile sitting in his basement unable to string a sentence
  together. So all he needs to do, on some level, is clear that hurdle.
  He just needs to not pass out during his speech. But he needs to
  convey what I think people look to him for during the primary, which
  is a sense of humanity, a sense of empathy. That is something that,
  obviously, has been sorely lacking with the current administration.
  And the message of all these real people that they're spotlighting at
  the convention talking about what they've been through is to kind of
  send the signal that this is the candidate who cares about this and
  who will actually take steps to improve your lives, that he cares
  about something other than his own ambitions and his own power. So I
  think just of conveying that, I feel your pain, I'm going to fix this.
  And then, of course, being not scary, and this is where the reaching
  out to Republicans, to progressives, the big-tent message will come
  into this. He has stressed that he wants to work with everyone. There
  is a place for everyone in his party. So I think all of those kind of
  basic things need to come through in the speech. He does not have to
  be Barack Obama. He is not a rhetorical god at this point, and he
  doesn't even need to try that. So he just needs to let Joe be Joe, on
  some level.
\item
  frank bruni\\
  Michelle Goldberg, you made the great point that this convention gives
  Democrats more control than ever before because it's not happening at
  an arena, people aren't standing up able to do whatever they want to
  do. Is it going to matter in the end? And in terms of it having to
  have been reinvented and abbreviated and crunched, do you think this
  might be the beginning of conventions mattering less?
\item
  michelle goldberg\\
  I mean, I think it's really hard to say, right? So it's interesting to
  me to hear Jamelle be so sort of Spock-like in his response to this.
  And I'm curious about whether, Jamelle, have you always been like
  this, or was it the trauma of 2016? Because we were together at the
  conventions four years ago, and I feel like I had a lot of feelings.
  Maybe I was just projecting them onto you. But I remember being at the
  Democratic convention, and I wrote about this at the time, it was
  impeccable. Joe Biden gave one of the best speeches I've ever heard
  him give at that convention. Michelle Obama gave an amazing speech.
  Compared to the total shit show of Trump's convention, which was this
  kind of pathetic shambolic thing with Scott Baio talking and a bunch
  of Nazis outside, I don't think even the best-produced convention can
  change the political dynamics in the country. I do think a couple of
  things. I think first of all, conventions, beyond what they do for a
  candidate a couple months before the election, it's good for a party
  to come together. It's good for people to build these relationships.
  That's why you have conventions not just of political parties, but of
  all sorts of organizations. It fosters cohesion. And so for that
  reason, I really hope that they resume them once they can. I also
  think that, again, because Donald Trump has chosen this particular
  attack on Joe Biden --- that he's senile, that he can't talk, that he
  has to be hidden away from view --- if he just gives a kind of mildly
  competent speech, a sort of ordinary, boilerplate political acceptance
  speech, it will do something to puncture that narrative. It seems ---
  knock on wood --- that it was potentially, from Trump's point of view,
  a bad idea to lower expectations about your opponent to such levels.
\item
  frank bruni\\
  I was just going to echo the points that both Michelle Goldberg and
  Michelle Cottle made that it's really strange, just from a political
  strategy standpoint, to attack your opponent as a bumbling idiot and
  someone incapable of stringing together a sentence ahead of a major
  event where people will see that candidate string together a sentence,
  a series of sentences, and appear competent. If there's any political
  advantage in this moment that the Democratic ticket has is that the
  Republican attacks --- the Trump attacks, specifically --- are just so
  removed from reality that once people get any kind of direct contact
  with that ticket, they're going to be like, oh, well obviously the
  Trump campaign is full of crap, right? So Biden is a drooling
  imbecile, not going to be the case. Kamala Harris is an anti-police
  radical, people are going to see her speak and be like, that's
  nonsense. It's not going to be the case. The ticket as a whole is in
  the thrall of leftists, well, we just saw John Kasich give a four
  minute speech, so that's probably not true. And I think the Trump
  campaign, because it cannot actually campaign on its record or on
  current conditions, is throwing everything at the wall and will find
  that that just doesn't work, that a voting public actually wants to
  hear about what each ticket has to say about things happening now and
  that the Democratic ticket is mainstream to a fault, you could say.
  There's nothing there that's going to cause people to think, oh, these
  people are dangerous or radical.
\item
  michelle goldberg\\
  But Jamelle, can I ask you a question? Both of us have seen Joe Biden
  speak a lot of times, and he's pretty hit or miss. Like I said, in
  2016, I thought he gave one of the best speeches at the convention.
  I've seen him give speeches on the campaign trail that left me with my
  heart in my throat, feeling like I was watching an actor forgetting
  his lines. This is such an important moment. Do you have any anxiety
  about how it's going to come off?
\item
  jamelle bouie\\
  He's always done worse in his impromptu moments. But a prepared
  speech, he's perfectly competent of giving. He's been giving them for
  his entire adult life. So I'm not too worried about that. The things
  I'm worried about have really less to do with the Democratic ticket or
  with the Democratic coalition, although there are, of course, tensions
  and divides to talk about within that coalition. But the things I'm
  worried about have just much more to do with election administration
  and the entire process of the election and how that's going to
  interact with the president's, I think, clear efforts to delegitimize
  the outcome and also the Republican Party, how it relates to those
  attempts, right? Because for as much as the conversation has been
  about Trump's attacks on the election, kind of the key actor on Nov. 3
  and Nov. 4, depending on the outcome, will just be Republican-elected
  officials, how they react to what's happened.
\item
  frank bruni\\
  That's a great point, Jamelle. Let's use that as a pivot because this
  convention is four days in time. It will recede in the rearview mirror
  really quickly. But all around it --- and Michelle Goldberg, you just
  wrote about this --- we have this very clear evidence that Trump is
  cheating, is preparing to cheat yet more. We have this huge melodrama
  with the United States Postal Service. We have various other things
  the president has said and done to convince people that if the result
  of this election is not what they want, they can consider it
  illegitimate. How much does each of you, and let's start with Michelle
  Cottle, worry about this truly being anything close to a free and fair
  election?
\item
  michelle cottle\\
  Oh, I'm completely terrified on some level. Cheater's gonna cheat. And
  I think that they've made clear they have no problem with doing
  whatever is necessary to undermine the process. I also worry about
  what happens if it is a remotely close election because all of
  President Trump's efforts to delegitimize this become that much more
  of an issue if they are like, oh, well look, it was so close. They
  cheated. It's all the systemic fraud from the mail-in voting. I think
  that the larger the margin is, the easier it will be for at least
  respectable Republican officials to say, OK, we need to move on, even
  if that infuriates certain segments of the base.
\item
  frank bruni\\
  That's what Michelle Obama was saying on Monday night at the
  convention. I thought one of the most fascinating parts of that
  extraordinary, excellent speech was when she was essentially saying,
  get your ballot tonight. Mail it in now, or prepare to go. Pack dinner
  and breakfast in case you have to wait on line. She was basically
  saying to America, you have got to deliver the Biden-Harris ticket
  such a margin that the cheating is erased and that there's no ability
  for Donald Trump to say that this is a fraudulent result. I mean, that
  was such a kind of appropriately cynical instruction to the American
  public, don't you think?
\item
  michelle goldberg\\
  Well, and it's not just the numbers that matter. It's also the timing.
  Jamelle wrote about this that right now, there's this asymmetry. And
  the Democrats are saying that they're more likely to vote by mail
  because they are more likely to take coronavirus seriously and
  Republicans the opposite. And what that could end up meaning is if you
  look back to 2018, in the midterms, and on election night, it looked
  like Democrats had had a really disappointing night. And it took the
  days and weeks afterwards, as the full vote was counted, to realize
  that there really was a blue wave. If something like that happens with
  Trump, he's extremely likely to just declare victory on election night
  and to try to shut down the vote count afterwards. And so if you think
  about, for those of us who are old enough to remember 2000, what
  happened with the fight over the recount in Florida and you multiply
  that by many states or many jurisdictions, and think back to the fact
  that Roger Stone, Trump's criminal advisor, was involved in basically
  staging what is now known as the Brooks Brothers riot where a bunch of
  rowdy Republican operatives physically interfered with the vote count
  in a way that very likely changed the course of that election, while
  hapless Democrats relied on the courts. I do think that Democrats are
  not going to do that this time. They're not just going to assume that
  the rule of law will prevail and the courts will protect them. And as
  I wrote today, people are already sort of organizing for the mass
  protest that's going to be necessary if there is some sort of
  shenanigans with the vote count or if Trump tries to interfere within
  in some way. In 2016, the night Trump was elected, when I was
  envisioning all sorts of disasters, could I have imagined that he
  would try to destroy the post office in order to stop people from
  casting their ballots? Probably not. And so I don't think we should
  assume that that is the last important American institution that he
  will use his power to try and wreck if he thinks that his holding onto
  power depends on it.
\item
  frank bruni\\
  Right, so there are things that are outside of the Democratic party's
  control. What can and should it do between now and November 3 to try
  to prevent the worst happening, to try to make this election as free
  and fair as it can possibly be? Jamelle, do you have any thoughts on
  that?
\item
  jamelle bouie\\
  Yeah, I mean, I think Michelle Goldberg got to much of what would be
  included in that. It is first, providing for voters detailed
  instructions on how to get a ballot, how to cast a ballot, how to do
  it in person. It's organizing efforts for people who need to go in
  person, to bring them in person. I think it would probably include
  distributing masks and stuff to voters who want them to kind of help
  people get themselves prepared for standing in line. Similar to what
  the R.N.C. and the Trump campaign are doing, kind of a nationwide plan
  to contest vote counts, which they've sort of announced that this is
  part of their election plan. I think the Democratic party and its
  affiliated institutions and organizations need to be prepping
  basically a defense of ballot counting, sort of a nationwide effort to
  stop that tampering. I think there needs to be plans for protests and
  demonstrations. This is going to sound very hyperbolic, but I think
  that we have to think of the task of getting Trump out as less of a
  traditional democratic transition and more of something akin to
  pushing an authoritarian regime out. And that's going to mean in
  addition to at the ballot box on Nov. 3, there being a Biden lead at
  the end of the night. Because I think Trump is going to contest
  regardless. Regardless of the outcome, Trump is going to say, this was
  rigged. This is fake, whatever. But the key thing is, is there space
  for the Republican legislature in Pennsylvania or Michigan or
  Wisconsin to certify election night results and send those in before
  all the votes are counted to force basically a contest between the
  full count that's if a governor certifies and the partial count that
  the legislature does. And demonstrations, a large vote count on
  election night, all those sorts of things send an important signal to
  Republican legislators and officials that if they take this step, they
  will be taking a step that directly affects their ability to hold
  power in the future. That's the message of a demonstration --- not for
  Trump because Trump I don't think gives a shit.
\item
  michelle cottle\\
  Now, there's also something that the media needs to do. They need to
  start setting expectations, just as Trump is very aggressively pushing
  the idea that election night has to be it. You have to call this race
  on election night. I think there needs to be a fair amount of pushback
  where people are told, this very well could go on past the one night
  that everybody thinks of as the deciding moment, especially if it's
  not a huge margin. And people have talked about the blue shift, where
  Democratic votes have been coming in on the later side. So what starts
  out looking like a close call then widens to be a more comfortable
  Democratic margin. Trump's going to go berserk over things like that,
  and people need to be prepared but that is not cheating. That is not
  rigging this vote, that election night is no longer a drop-dead kind
  of moment. Because he's been pushing the idea that if it's not called
  on that night, it's fake. And that's just something that everyone else
  has to say, no, that's nonsense.
\item
  frank bruni\\
  Can I push back on one thing? Everything you say is 100 percent true,
  but I want any one of you who can reassure me to reassure me. We can
  report that, Michelle. We at The Times, our colleagues at the
  Washington Post, people on the responsible cable networks can do
  everything you're saying, and yet we live in an information ecosystem
  where we're never going to reach some of the people who most need to
  hear exactly what you said. What do we do about that?
\item
  michelle cottle\\
  You're never going to get everybody. I mean, you don't worry about
  something just because it's not going to be 100 percent. But you're
  also talking to Republican officials. You need to put the pressure on
  them. It needs to be one of these things where only the fraction of
  the base who is going to believe whatever, Trump shoot somebody on
  Fifth Avenue, they're cool with that. Everybody else needs to know the
  reality of the situation. You can't break into that weird alternative
  bubble with everything. But everybody else who's not drinking the
  Kool-Aid needs to understand this. The more people you can get to
  understand it, the more pressure gets brought to bear on people like
  Susan Collins, assuming she's still in the Senate at the end of this
  race.
\item
  michelle goldberg\\
  I feel like I could see something maybe mildly reassuring, which is
  out of character for me, which is that usually raising the salience of
  an issue like mail-in voting is a pretty difficult thing to do. It's
  hard to get people excited about these procedural issues. The one
  useful thing that Trump has done by being both so blatant about the
  reason that he's sabotaging the post office, and by just sabotaging
  the post office in a way that affects the lives of a lot of people who
  don't follow politics very closely, is that people understand what's
  going on. Even if they don't understand all the nuances of it, they
  understand that they're not getting their mail. They understand that
  they're not getting their prescriptions. They understand that checks
  that they need are coming late. And so it has raised the public
  salience of an issue that otherwise might seem like a bit of
  procedural minutia.
\item
  jamelle bouie\\
  There's a lot that complain about the fact that cable news media is
  responsible for reporting election results. Every state has a
  Secretary of State's office, but people get the results by way of the
  media. But the advantage in this case is that everyone gets the
  results by way of the media. And so if there's just a collective
  agreement among the non-Fox networks to say we're not going to report
  percentage precincts. We're going to report the amount of the vote
  that has come in. We're not going to do projected winners until a
  certain percentage of the vote that we know probably went out has come
  in. Those sorts of things, if there's a general agreement to do them,
  I think will help. Because in 2000, it was the opposite. Networks were
  rushing to get ahead of each other in announcing who won and who
  didn't. And that actually helped create the space for the shenanigans
  in Florida.
\item
  frank bruni\\
  And maybe we all learned a lesson we won't create that space this time
  around. There's been a lot of kind of apocalyptic sentiment during the
  last few minutes of this. So I want to wrap up our discussion by
  leaving our listeners on a slightly maybe more positive, optimistic
  note. I want to ask each of you, and take it in whatever order you
  like, what are you seeing at this convention or what are you seeing in
  terms of the dynamics of this race that does give you great hope that
  the right thing will happen on November 3 or in the days after
  November 3, as the count continues, and that we will get past these
  last four years? Why don't we start with Michelle Cottle.
\item
  michelle cottle\\
  Well, I think there's clearly much less internal squabbling among
  Democrats. We do not have the small differences battles that were such
  a problem in 2016. And I think Hillary Clinton was an incredibly
  controversial candidate. I know lots of Democrats who just couldn't
  even hold their nose and stomach her. So I think that Biden, for
  better and worse, does not incite those kind of passions. And so it's
  been much easier on both levels, both from you know what the stakes
  are and the Biden isn't scary. Much easier to bring the party together
  in this situation.
\item
  michelle goldberg\\
  I just want to say quickly, I certainly know people who were
  enthusiastic about Hillary Clinton and are going to have a hard time
  voting for Joe Biden, even though they're going to do it. So I don't
  think that Joe Biden is completely anodyne as a candidate, although I
  agree, he partly just, as a white man, doesn't inspire the sort of
  kind of panic and hysteria that Hillary Clinton has always inspired.
  So the thing that gives me a little bit of hope --- and I'm not the
  person you turn to for hope --- is as I said before, that I just think
  Democrats have wised up about relying on the courts and sort of
  procedural neutrality to sort this out. If you go back again and read
  about the Brooks Brothers riots and other sort of shenanigans around
  the 2000 election, Republicans had a multi-pronged strategy. They
  fought it in the courts, but they also fought it in the streets. They
  fought it in the court of public opinion. Democrats, at that time,
  really just had faith in the process and faith that the rules,
  neutrally applied, would see justice done. And I think that that is
  gone, and you have a lot of people now on hair trigger who don't trust
  a lot of these institutions. They certainly don't trust the courts.
  And so there's going to be much more of an uproar if you see anything
  unfair, anything sort of un-kosher about the way that the vote is
  administered or the vote is counted. I just think that the Democratic
  base is much more activated and much more attuned and kind of much
  more ready to take to the streets.
\item
  jamelle bouie\\
  Somewhat counterintuitively, my ray of hope is exactly that Trump is
  trying to corrupt and destroy the election. This is not something that
  a confident authoritarian does. It's not something that a confident
  candidate does. It reflects profound fear and insecurity, that Trump
  almost certainly knows that by any conventional measure, he's losing
  the election, that he knows that the public is against him. That's why
  he's been hyping up these boat parades, flotillas of supporters, that
  the real popular vote is the people who own boats and who fly Trump
  flags. But I think he and his team are very aware of what the
  political mood is against him. And so these, to me, kind of desperate
  attempts to shift the ground on which the election is actually being
  fought --- obviously, there should be vigilance and preparation ---
  but it should also I think convince Trump's opponents --- not just
  Democrats, but sort of the broad anti-Trump coalition --- that they're
  moving in the right direction and that they have the initiative here,
  that they're not on the defensive at the end of the day. I would be
  worried if Trump seemed super confident. That would fill me with fear
  because that's sort of the I'm doing something that you people don't
  know about, are not going to know about, may not ever be able to
  prove. But flailing like a child, that actually makes me feel a little
  more confident.
\item
  frank bruni\\
  That gives you your hope. I tell you what gives me hope. This
  convention gives me hope, in the sense that at every turn--- every 15
  minutes, every five minutes --- someone says something that tells me
  that people appreciate the stakes here. Even though everyone in 2016
  who was for Hillary Clinton and who was worried about Donald Trump
  certainly didn't think he was just any Republican who might win, I
  still think the sense of stakes this time around that this is a
  crossroads, that we can't go down the wrong fork in the road, that
  this is a kind of do-or-die moment for America, I'm hearing that loud
  and clear in this convention. And that gives me some hope that voters
  will bring the right urgency to November 3 and do the right thing.
  Let's stop there. We are going to take a short break, and we will be
  back.

  And we're back. And it's that time when someone gives you all a
  recommendation to leave with. This week, I believe Michelle Cottle has
  something for us. Michelle, you have a recommendation?
\item
  michelle cottle\\
  I do, and it might work best for parents. But you know, it could be
  adapted in some way. So obviously, I cover national politics for the
  board, so I just can't live in that 24/7. So I have two teenagers. I
  have a 15-year-old daughter who's a Broadway drama nerd and a
  17-year-old son who plays any instrument and is in his heavy classic
  rock phase. So he knows more about `70s and `80s music than I do. And
  one extremely effective way to waste time and get my mind a wash is we
  have Sonos on our phones that connects to a stereo speaker in the
  house. And you each take turns picking songs and playing them. And so
  you get into this kind of groove where everybody goes and starts
  picking the music, and then you start fighting over whose music is the
  worst or who's the best or who did this guitar riff better than
  whatever. And then if my husband gets involved, he likes to sing. He's
  a frustrated Broadway diva. And so then the children are appalled, and
  they're, like, yelling at me to make dad stop. And so it can go on
  like this for hours. And for those who have teenagers, my teenagers
  spend most of their time sleeping until 2 pm and then staring at their
  screens during this pandemic because they have nowhere to go. So this
  is a fun way to get everybody kind of thinking about something that
  they love. And I don't sing, but that makes it even more fun for them
  because they can abuse me mercilessly when I feel compelled to break
  into song.
\item
  frank bruni\\
  It's sort of like an old-fashioned jukebox.
\item
  michelle cottle\\
  It is a jukebox. And you can cut off each other because you can break
  in with your phones on each other's song. And at some point, it always
  degenerates into someone playing ``What's New Pussycat'' at top
  volume. {[}GOLDBERG LAUGHS{]} I will not go into the whys of that. But
  at that point, you know it's time to go to bed, really. It's over.
\item
  frank bruni\\
  Michelle, Jamelle, you both have children. Does this sound like fun
  for your whole families?
\item
  michelle goldberg\\
  I mean, the time that we listen to music, we panic-bought a car during
  the pandemic. And so now we drive, which my kids aren't used to. And
  they hate it, and they complain the whole time. But especially my
  five-year-old has very pronounced musical tastes. And the other day,
  for the first time, they asked us to play a song that we weren't
  familiar with, ``Exile'' by Taylor Swift, which is actually a great
  song. And I'm not quite sure where they heard it. But it's kind of
  fascinating to watch them develop musical personalities that are not
  just the albums that we've introduced them to.
\item
  frank bruni\\
  Jamelle, you have just been made an honorary member of the Cottle
  family, and you are playing the Sonos challenge. What song are you
  queuing up?
\item
  jamelle bouie\\
  I am cueing up ``Girls On Film'' by Duran Duran.
\item
  frank bruni\\
  Wow.
\item
  michelle goldberg\\
  Wow.
\item
  frank bruni\\
  Duran Duran, I'm back in college. Yeesh. Michelle Cottle, Jamelle,
  thank you both so much for coming on the show this convention week,
  and we look forward to welcoming you back in the future.
\item
  michelle goldberg\\
  Yeah, thank you, guys.
\item
  jamelle bouie\\
  Thank you for having us.
\item
  frank bruni\\
  Meanwhile, Michelle Goldberg and I will be back next week. That's our
  show for this week. Thank you all for listening. If you have a
  question you want to hear us debate, share it with us in a voicemail
  by calling 347-915-4324. You can also email us at
  \href{mailto:argument@NYTimes.com}{\nolinkurl{argument@NYTimes.com}}.
  ``The Argument'' is a production of The New York Times Opinion
  section. The team includes Phoebe Lett, Paula Szuchman, Pedro Rafael
  Rosado, Vishakha Darbha, Kristin Lin, and Isaac Jones. We'll see you
  next week.
\item
  jamelle bouie\\
  I was about to say the two Michelle's. But I wasn't exactly sure, is
  that OK to say?
\item
  michelle cottle\\
  No, that's good.
\item
  jamelle bouie\\
  OK.
\item
  michelle goldberg\\
  Yeah, it's not like it's Karen. {[}ALL LAUGH{]}
\end{itemize}

Previous

More episodes ofThe Argument

\href{https://www.nytimes3xbfgragh.onion/2020/09/11/opinion/the-argument-latino-2020-vote.html?action=click\&module=audio-series-bar\&region=header\&pgtype=Article}{\includegraphics{https://static01.graylady3jvrrxbe.onion/images/2020/09/12/opinion/10argumentWeb/10argumentWeb-thumbLarge-v2.jpg}}

September 11, 2020How to Win the Latino Vote

\href{https://www.nytimes3xbfgragh.onion/2020/09/03/opinion/the-argument-trump-biden-kenosha-portland.html?action=click\&module=audio-series-bar\&region=header\&pgtype=Article}{\includegraphics{https://static01.graylady3jvrrxbe.onion/images/2020/09/05/opinion/03argumentWeb/03argumentWeb-thumbLarge.jpg}}

September 3, 2020Is `American Carnage' Campaign Gold?

\href{https://www.nytimes3xbfgragh.onion/2020/08/27/opinion/the-argument-republican-convention-trump.html?action=click\&module=audio-series-bar\&region=header\&pgtype=Article}{\includegraphics{https://static01.graylady3jvrrxbe.onion/images/2020/08/28/opinion/27argument-ninetytwo1-print/27argument-ninetytwo1-thumbLarge.jpg}}

August 27, 2020Can the Republicans Sell a Whole New Trump?

\href{https://www.nytimes3xbfgragh.onion/2020/08/20/opinion/the-argument-democratic-convention-biden.html?action=click\&module=audio-series-bar\&region=header\&pgtype=Article}{\includegraphics{https://static01.graylady3jvrrxbe.onion/images/2020/08/20/opinion/20argument-ninetyone1/20argument-ninetyone1-thumbLarge.jpg}}

August 20, 2020What Biden Must Do

\href{https://www.nytimes3xbfgragh.onion/2020/08/13/opinion/the-argument-coronavirus-catholic-covid.html?action=click\&module=audio-series-bar\&region=header\&pgtype=Article}{\includegraphics{https://static01.graylady3jvrrxbe.onion/images/2020/08/13/opinion/13argument1/merlin_173532477_02e02102-92e6-4f5a-82bf-5394265f898b-thumbLarge.jpg}}

August 13, 2020Is Individualism America's Religion?

\href{https://www.nytimes3xbfgragh.onion/2020/08/06/opinion/the-argument-trump-coronavirus-election.html?action=click\&module=audio-series-bar\&region=header\&pgtype=Article}{\includegraphics{https://static01.graylady3jvrrxbe.onion/images/2020/08/06/opinion/06argSub/06argSub-thumbLarge.jpg}}

August 6, 2020Trump Supporters Make Their Case for 2020

\href{https://www.nytimes3xbfgragh.onion/2020/07/30/opinion/the-argument-authoritarianism-anne-applebaum.html?action=click\&module=audio-series-bar\&region=header\&pgtype=Article}{\includegraphics{https://static01.graylady3jvrrxbe.onion/images/2020/07/31/opinion/30argumentWeb-print/30argumentWeb-thumbLarge.jpg}}

July 30, 2020When Conservatives Fall for Demagogues

\href{https://www.nytimes3xbfgragh.onion/2020/07/23/opinion/the-argument-israel-palestinian.html?action=click\&module=audio-series-bar\&region=header\&pgtype=Article}{\includegraphics{https://static01.graylady3jvrrxbe.onion/images/2020/07/25/opinion/25audio/21argumentWeb-thumbLarge.jpg}}

July 23, 2020The Case for a One-State Solution

\href{https://www.nytimes3xbfgragh.onion/2020/07/16/opinion/the-argument-tammy-duckworth.html?action=click\&module=audio-series-bar\&region=header\&pgtype=Article}{\includegraphics{https://static01.graylady3jvrrxbe.onion/images/2020/07/17/opinion/16argumentWeb-print/16argumentWeb-thumbLarge.jpg}}

July 16, 2020A Conversation With Tammy Duckworth

\href{https://www.nytimes3xbfgragh.onion/2020/07/09/opinion/is-trumps-fate-sealed.html?action=click\&module=audio-series-bar\&region=header\&pgtype=Article}{\includegraphics{https://static01.graylady3jvrrxbe.onion/images/2020/07/10/opinion/10a2_audio/09argument1-thumbLarge.jpg}}

July 9, 2020Is Trump's Fate Sealed?

\href{https://www.nytimes3xbfgragh.onion/2020/07/02/opinion/the-argument-protest-statue-revolution.html?action=click\&module=audio-series-bar\&region=header\&pgtype=Article}{\includegraphics{https://static01.graylady3jvrrxbe.onion/images/2020/07/05/opinion/02argument-eightyfive1/02argument-eightyfive1-thumbLarge.jpg}}

July 2, 2020Whose Statue Must Fall?

\href{https://www.nytimes3xbfgragh.onion/2020/06/25/opinion/the-argument-biden-vice-president-supreme-court.html?action=click\&module=audio-series-bar\&region=header\&pgtype=Article}{\includegraphics{https://static01.graylady3jvrrxbe.onion/images/2020/06/28/opinion/25argument-eightyfour1/25argument-eightyfour1-thumbLarge.jpg}}

June 25, 2020Place Your Bets on Biden's V.P.

\href{https://www.nytimes3xbfgragh.onion/column/the-argument}{See All
Episodes ofThe Argument}

Next

Aug. 20, 2020

\begin{itemize}
\item
\item
\item
\item
\item
\end{itemize}

\emph{\textbf{Listen and subscribe to ``The Argument'' from your mobile
device:}}

\textbf{\href{https://itunes.apple.com/us/podcast/the-argument/id1438024613?mt=2}{\emph{Apple
Podcasts}}} \emph{\textbf{\textbar{}}}
\textbf{\href{https://open.spotify.com/show/6bmhSFLKtApYClEuSH8q42}{\emph{Spotify}}}
\emph{\textbf{\textbar{}}}
\textbf{\href{https://play.google.com/music/m/Idxib4hsg3yviao4gtym76knjjy?t=The_Argument}{\emph{Google
Play}}} \emph{\textbf{\textbar{}}}
\textbf{\href{https://radiopublic.com/the-argument-Wdbepr}{\emph{RadioPublic}}}
\emph{\textbf{\textbar{}}}
\textbf{\href{https://www.stitcher.com/podcast/the-new-york-times/the-argument}{\emph{Stitcher}}}
\emph{\textbf{\textbar{}}}
\textbf{\href{https://rss.art19.com/the-argument}{\emph{RSS Feed}}}

It's a Democratic convention unlike any other. So who is it for? What
does the party, and its presidential candidate, Joe Biden, need to
accomplish? And how should they approach President Trump's threats to a
free and fair election? This week on the podcast, Frank Bruni and
Michelle Goldberg are joined by Opinion columnist Jamelle Bouie and
editorial board member Michelle Cottle for a round-table discussion of
the virtual ``nerd Coachella'' that is the Democratic National
Convention of 2020.

Then, Michelle Cottle offers a homespun jukebox game that can take the
whole family's mind off politics and the pandemic.

\includegraphics{https://static01.graylady3jvrrxbe.onion/images/2020/08/20/opinion/20argument-ninetyone1/merlin_175888815_82ed425a-29da-475d-86d7-c40e3feb2fe5-articleLarge.jpg?quality=75\&auto=webp\&disable=upscale}

\begin{center}\rule{0.5\linewidth}{\linethickness}\end{center}

\textbf{Background Reading:}

\begin{itemize}
\item
  Jamelle Bouie on
  \href{https://www.nytimes3xbfgragh.onion/2020/08/14/opinion/kamala-harris-black-identity.html}{Kamala
  Harris's race} and
  \href{https://www.nytimes3xbfgragh.onion/2020/08/11/opinion/trump-election-day.html}{how
  to foil Trump's election night strategy}
\item
  Michelle Cottle on
  \href{https://www.nytimes3xbfgragh.onion/2020/08/12/opinion/debates-trump-biden.html}{the
  Trump-Biden debate} and the
  \href{https://www.nytimes3xbfgragh.onion/2020/08/03/opinion/senior-voters-biden-trump-2020.html}{disenchanted-senior
  vote}
\item
  Michelle Goldberg on
  \href{https://www.nytimes3xbfgragh.onion/2020/08/17/opinion/trump-contested-election-protests.html}{countermeasures
  to Trump's cheating}
\item
  Frank Bruni on
  \href{https://www.nytimes3xbfgragh.onion/2020/08/18/opinion/michelle-obama-dnc-election-2020.html}{Michelle
  Obama's D.N.C. speech} and
  \href{https://www.nytimes3xbfgragh.onion/2020/08/14/opinion/kamala-harris-biden-2020.html}{Kamala
  Harris's story}
\item
  All four writers (and more)
  \href{https://www.nytimes3xbfgragh.onion/interactive/2020/08/19/opinion/democratic-convention-best-worst-night-2.html}{grade
  the nights of the convention}
\end{itemize}

\begin{center}\rule{0.5\linewidth}{\linethickness}\end{center}

\textbf{How to listen to ``The Argument'':}

\emph{Press play or read the transcript (found by midday Thursday above
the center teal eye) at the top of this page, or tune in on}
\href{https://itunes.apple.com/us/podcast/the-argument/id1438024613?mt=2}{\emph{iTunes}}\emph{,}
\href{https://play.google.com/music/listen?u=0\#/ps/Idxib4hsg3yviao4gtym76knjjy}{\emph{Google
Play}}\emph{,}
\href{https://open.spotify.com/episode/5fIsHqqunLBwoxPSUUSGre?si=Rz5D9VnlRFKdGMu8ixzBOw}{\emph{Spotify}}\emph{,}
\href{https://www.stitcher.com/podcast/the-new-york-times/the-argument}{\emph{Stitcher}}
\emph{or your preferred podcast listening app. Tell us what you think
at} \href{mailto:argument@NYTimes.com}{\emph{argument@NYTimes.com.}}

\begin{center}\rule{0.5\linewidth}{\linethickness}\end{center}

\hypertarget{meet-the-hosts}{%
\section{Meet the Hosts}\label{meet-the-hosts}}

\hypertarget{frank-bruni}{%
\subsection{Frank Bruni}\label{frank-bruni}}

Image

I've been an Op-Ed columnist for The Times since 2011, but my career
with the newspaper stretches back to 1995 and includes many twists and
turns that reflect my embarrassingly scattered interests. I covered
Congress, the White House and several political campaigns; I also spent
five years in the role of chief restaurant critic. As the Rome bureau
chief, I reported on the Vatican; as a staff writer for The Times's
Sunday magazine, I wrote many celebrity profiles. That jumble has
informed my various books, which focus on the Roman Catholic Church,
George W. Bush, my strange eating life, the college admissions process
and meatloaf. Politically, I'm grief-stricken over the way President
Trump has governed and I'm left of center, but I don't think that the
center is a bad place or ``compromise'' a dirty word. I'm
Italian-American, I'm gay and I write a
\href{https://www.nytimes3xbfgragh.onion/newsletters/frank-bruni}{weekly
Times newsletter} in which you'll occasionally encounter my dog, Regan,
who has the run of our Manhattan apartment.
\href{https://twitter.com/FrankBruni}{\emph{@FrankBruni}}

\hypertarget{michelle-goldberg}{%
\subsection{Michelle Goldberg}\label{michelle-goldberg}}

Image

I've been an Op-Ed columnist at The New York Times since 2017, writing
mainly about politics, ideology and gender. These days people on the
right and the left both use ``liberal'' as an epithet, but that's
basically what I am, though the nightmare of Donald Trump's presidency
has radicalized me and pushed me leftward. I've written three books,
including one, in 2006, about the danger of right-wing populism in its
religious fundamentalist guise. (My other two were about the global
battle over reproductive rights and, in a brief detour from politics,
about an adventurous Russian émigré who helped bring yoga to the West.)
I love to travel; a long time ago, after my husband and I eloped, we
spent a year backpacking through Asia. Now we live in Brooklyn with our
son and daughter.
\href{https://twitter.com/michelleinbklyn}{\emph{@michelleinbklyn}}

\begin{center}\rule{0.5\linewidth}{\linethickness}\end{center}

``The Argument'' is a production of The New York Times Opinion section.
The team includes Phoebe Lett, Paula Szuchman, Pedro Rafael Rosado,
Kathy Tu, Vishakha Darbha, Isaac Jones and Kristin Lin. Theme by Allison
Leyton-Brown.

Advertisement

\protect\hyperlink{after-bottom}{Continue reading the main story}

\hypertarget{site-index}{%
\subsection{Site Index}\label{site-index}}

\hypertarget{site-information-navigation}{%
\subsection{Site Information
Navigation}\label{site-information-navigation}}

\begin{itemize}
\tightlist
\item
  \href{https://help.nytimes3xbfgragh.onion/hc/en-us/articles/115014792127-Copyright-notice}{©~2020~The
  New York Times Company}
\end{itemize}

\begin{itemize}
\tightlist
\item
  \href{https://www.nytco.com/}{NYTCo}
\item
  \href{https://help.nytimes3xbfgragh.onion/hc/en-us/articles/115015385887-Contact-Us}{Contact
  Us}
\item
  \href{https://www.nytco.com/careers/}{Work with us}
\item
  \href{https://nytmediakit.com/}{Advertise}
\item
  \href{http://www.tbrandstudio.com/}{T Brand Studio}
\item
  \href{https://www.nytimes3xbfgragh.onion/privacy/cookie-policy\#how-do-i-manage-trackers}{Your
  Ad Choices}
\item
  \href{https://www.nytimes3xbfgragh.onion/privacy}{Privacy}
\item
  \href{https://help.nytimes3xbfgragh.onion/hc/en-us/articles/115014893428-Terms-of-service}{Terms
  of Service}
\item
  \href{https://help.nytimes3xbfgragh.onion/hc/en-us/articles/115014893968-Terms-of-sale}{Terms
  of Sale}
\item
  \href{https://spiderbites.nytimes3xbfgragh.onion}{Site Map}
\item
  \href{https://help.nytimes3xbfgragh.onion/hc/en-us}{Help}
\item
  \href{https://www.nytimes3xbfgragh.onion/subscription?campaignId=37WXW}{Subscriptions}
\end{itemize}
