Sections

SEARCH

\protect\hyperlink{site-content}{Skip to
content}\protect\hyperlink{site-index}{Skip to site index}

\href{https://myaccount.nytimes3xbfgragh.onion/auth/login?response_type=cookie\&client_id=vi}{}

\href{https://www.nytimes3xbfgragh.onion/section/todayspaper}{Today's
Paper}

A Longstanding and Most Iconic Motif Is Reimagined

\url{https://nyti.ms/2QbRkY4}

\begin{itemize}
\item
\item
\item
\item
\item
\end{itemize}

Advertisement

\protect\hyperlink{after-top}{Continue reading the main story}

Supported by

\protect\hyperlink{after-sponsor}{Continue reading the main story}

\hypertarget{a-longstanding-and-most-iconic-motif-is-reimagined}{%
\section{A Longstanding and Most Iconic Motif Is
Reimagined}\label{a-longstanding-and-most-iconic-motif-is-reimagined}}

Nicolas Ghesquière presents a new, if retro-feeling, take on Louis
Vuitton's signature monogram.

\includegraphics{https://static01.graylady3jvrrxbe.onion/images/2020/08/20/t-magazine/20tmag-lv/20tmag-lv-articleLarge.jpg?quality=75\&auto=webp\&disable=upscale}

By Nick Remsen

\begin{itemize}
\item
  Aug. 20, 2020
\item
  \begin{itemize}
  \item
  \item
  \item
  \item
  \item
  \end{itemize}
\end{itemize}

Some logos move in and out of favor with the seasons. Then there's Louis
Vuitton's signature monogram canvas, with its interlaced L and V flanked
by various geometric flowers, which seems to be another thing entirely,
one impervious to trends and maybe even time itself. In fact, the
monogram hasn't always been around, nor does it date to the house's
founding in 1854, when Louis Vuitton, who had previously been a packer
for Empress Eugénie, the wife of Napoleon III, began selling travel
trunks out of a store on Paris's Rue Neuve-des-Capucines, just off Place
Vendôme. His go-to motif was a waterproof gray canvas. The monogram came
into existence just over four decades later, when Louis's son, Georges,
took inspiration from the kitchen tiles of the family's home in Asnières
in north-central France. The house, built in 1859, was decorated in the
Art Nouveau style, and the tile pattern featured a trio of flowers
outlined by a clover that was in turn set within a circle, as well as
diamonds with petal-like points. Another reference may have been the
Gothic Revival then sweeping through France --- when seen through that
lens, the monogram can't help but recall architectural quatrefoils
carved out of stone or glass in the era's great cathedrals. What is
certain is that, in designing it, Georges was acting in honor of his
father, who had died four years prior, leaving the brand's legacy in his
son's hands.

Today, that legacy has largely been conferred to
\href{https://www.nytimes3xbfgragh.onion/interactive/2019/10/15/t-magazine/nicolas-ghesquiere-louis-vuitton.html}{Nicolas
Ghesquière}, who, since becoming the artistic director in 2013, has
proved himself especially adept at honoring the house's storied past
while making designs that are resoundingly current, if not
futuristic-feeling. His Petit Malle bag shrinks a traditional steamer
trunk to a size not much larger than a cellphone, and his fall 2020
collection features hybridized jackets made of leather and knitted wool
that nod to high-tech protective athletic gear but are embroidered with
baroque threading reminiscent of that found on 17th-century fashions.
Ghesquière's latest tribute to the Vuitton men is called Since 1854.
Launching next month, the 56-piece capsule collection includes a
sleeveless minidress, a combat boot, a bucket bag and a Speedy 25 purse
--- a miniature version of the house's Keepall duffel that was
reportedly created at the request of Audrey Hepburn --- all covered in
an updated jacquard version of the monogram: Here, ``1854'' is set with
a repeating series of concave diamonds, the symmetrical loops of the
``8'' becoming their own sort of petals. Like Georges before him,
Ghesquière looked to interior design, namely to wallpapers from the
1960s. Though, as ever with the designer, what is old is appealingly new
again.

How much should a legacy brand change and how much should it stay the
same? It's a question that Ghesquière must consider often, and he's
hardly the first. Stephen Sprouse covered the Vuitton design in neon
graffiti, Takashi Murakami with smiling cherries. Ghesquière's take is
perhaps more subtle, but no less ambitious. It's also further evidence
that while innovation endures at the house, Georges indeed succeeded in
creating something essentially timeless.

Advertisement

\protect\hyperlink{after-bottom}{Continue reading the main story}

\hypertarget{site-index}{%
\subsection{Site Index}\label{site-index}}

\hypertarget{site-information-navigation}{%
\subsection{Site Information
Navigation}\label{site-information-navigation}}

\begin{itemize}
\tightlist
\item
  \href{https://help.nytimes3xbfgragh.onion/hc/en-us/articles/115014792127-Copyright-notice}{©~2020~The
  New York Times Company}
\end{itemize}

\begin{itemize}
\tightlist
\item
  \href{https://www.nytco.com/}{NYTCo}
\item
  \href{https://help.nytimes3xbfgragh.onion/hc/en-us/articles/115015385887-Contact-Us}{Contact
  Us}
\item
  \href{https://www.nytco.com/careers/}{Work with us}
\item
  \href{https://nytmediakit.com/}{Advertise}
\item
  \href{http://www.tbrandstudio.com/}{T Brand Studio}
\item
  \href{https://www.nytimes3xbfgragh.onion/privacy/cookie-policy\#how-do-i-manage-trackers}{Your
  Ad Choices}
\item
  \href{https://www.nytimes3xbfgragh.onion/privacy}{Privacy}
\item
  \href{https://help.nytimes3xbfgragh.onion/hc/en-us/articles/115014893428-Terms-of-service}{Terms
  of Service}
\item
  \href{https://help.nytimes3xbfgragh.onion/hc/en-us/articles/115014893968-Terms-of-sale}{Terms
  of Sale}
\item
  \href{https://spiderbites.nytimes3xbfgragh.onion}{Site Map}
\item
  \href{https://help.nytimes3xbfgragh.onion/hc/en-us}{Help}
\item
  \href{https://www.nytimes3xbfgragh.onion/subscription?campaignId=37WXW}{Subscriptions}
\end{itemize}
