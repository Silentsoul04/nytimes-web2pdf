\href{/section/business/economy}{Economy}\textbar{}They're Making the
Rent. Is It Costing Their Future?

\url{https://nyti.ms/3hjhOmg}

\begin{itemize}
\item
\item
\item
\item
\item
\item
\end{itemize}

\hypertarget{the-coronavirus-outbreak}{%
\subsubsection{\texorpdfstring{\href{https://www.nytimes3xbfgragh.onion/news-event/coronavirus?name=styln-coronavirus-markets\&region=TOP_BANNER\&block=storyline_menu_recirc\&action=click\&pgtype=Article\&impression_id=f8ec5c60-f52b-11ea-bcaf-0f2585cccc67\&variant=undefined}{The
Coronavirus
Outbreak}}{The Coronavirus Outbreak}}\label{the-coronavirus-outbreak}}

\begin{itemize}
\tightlist
\item
  live\href{https://www.nytimes3xbfgragh.onion/2020/09/12/world/covid-19-coronavirus.html?name=styln-coronavirus-markets\&region=TOP_BANNER\&block=storyline_menu_recirc\&action=click\&pgtype=Article\&impression_id=f8ec8370-f52b-11ea-bcaf-0f2585cccc67\&variant=undefined}{Latest
  Updates}
\item
  \href{https://www.nytimes3xbfgragh.onion/interactive/2020/us/coronavirus-us-cases.html?name=styln-coronavirus-markets\&region=TOP_BANNER\&block=storyline_menu_recirc\&action=click\&pgtype=Article\&impression_id=f8ec8371-f52b-11ea-bcaf-0f2585cccc67\&variant=undefined}{Maps
  and Cases}
\item
  \href{https://www.nytimes3xbfgragh.onion/interactive/2020/science/coronavirus-vaccine-tracker.html?name=styln-coronavirus-markets\&region=TOP_BANNER\&block=storyline_menu_recirc\&action=click\&pgtype=Article\&impression_id=f8ec8372-f52b-11ea-bcaf-0f2585cccc67\&variant=undefined}{Vaccine
  Tracker}
\item
  \href{https://www.nytimes3xbfgragh.onion/2020/09/10/us/politics/fda-coronavirus-vaccine.html?name=styln-coronavirus-markets\&region=TOP_BANNER\&block=storyline_menu_recirc\&action=click\&pgtype=Article\&impression_id=f8ec8373-f52b-11ea-bcaf-0f2585cccc67\&variant=undefined}{F.D.A.
  Regulators' Self-Defense}
\item
  \href{https://www.nytimes3xbfgragh.onion/2020/09/09/upshot/coronavirus-surprise-test-fees.html?name=styln-coronavirus-markets\&region=TOP_BANNER\&block=storyline_menu_recirc\&action=click\&pgtype=Article\&impression_id=f8ec8374-f52b-11ea-bcaf-0f2585cccc67\&variant=undefined}{Surprise
  Test Fees}
\end{itemize}

\includegraphics{https://static01.graylady3jvrrxbe.onion/images/2020/08/23/business/21Virus-Housing-Moshtael/21Virus-Housing-Moshtael-articleLarge.jpg?quality=75\&auto=webp\&disable=upscale}

Sections

\protect\hyperlink{site-content}{Skip to
content}\protect\hyperlink{site-index}{Skip to site index}

\hypertarget{theyre-making-the-rent-is-it-costing-their-future}{%
\section{They're Making the Rent. Is It Costing Their
Future?}\label{theyre-making-the-rent-is-it-costing-their-future}}

Federal aid and eviction bans have kept many tenants in their homes.
With that support ebbing, it can take charity and sacrifice to avoid
dislocation.

After losing her two jobs and then her unemployment pay, Nura Moshtael
felt she had no choice but to pack up her apartment and move in with her
mother in her childhood home.Credit...Audra Melton for The New York
Times

Supported by

\protect\hyperlink{after-sponsor}{Continue reading the main story}

By \href{https://www.nytimes3xbfgragh.onion/by/conor-dougherty}{Conor
Dougherty} and Gillian Friedman

\begin{itemize}
\item
  Aug. 21, 2020
\item
  \begin{itemize}
  \item
  \item
  \item
  \item
  \item
  \item
  \end{itemize}
\end{itemize}

They've made it with government checks and family help. They've made it
with savings and odd jobs. They've made it with church charity,
nonprofit rescue funds, GoFundMe campaigns. One way or another, through
five months of economic dislocation, the nation's tenants have for the
most part made their rent.

Now the question is how much longer these patchwork maneuvers will work
--- and what will happen to the economy if they suddenly don't.

Almost from the moment the coronavirus upended the economy in March,
there has been a persistent fear that the loss of wages and employment,
concentrated among lower-income service workers, would lead to
widespread evictions. According to one study,
\href{https://nlihc.org/sites/default/files/The_Eviction_Crisis_080720.pdf}{as
many as 40 million people in 17 million households} risk eviction by the
end of the year --- an astounding figure.

Yet interviews with dozens of landlords across the country returned
comments like ``no difference,'' ``pleasantly surprised'' and ``seems
like normal.'' That view is reinforced by the corporate earnings reports
of housing providers and a weekly survey of big landlords by the
National Multifamily Housing Council, which for several months has shown
little difference from rent collections a year ago.

On its face, the disconnect between upbeat landlords and anxious tenants
seems to expose a glitch in the data or an example of the growing
economic dissonance --- like the stock market's
\href{https://www.nytimes3xbfgragh.onion/2020/08/18/business/stock-market-record.html}{rise
to new heights} despite a
\href{https://www.bls.gov/news.release/pdf/empsit.pdf}{10.2 percent
unemployment rate}. What it actually shows is that for all of the
government's problems in containing the virus, financial rescue efforts
were largely effective in keeping tenants in their homes.

The \$2 trillion CARES Act, with its \$1,200 stimulus payments and \$600
a week in extended unemployment benefits, helped laid-off renters stay
current, while federal, state and local eviction moratoriums guaranteed
stability for those who could not. But those efforts have largely
lapsed: The \$600 payments ended in July, and about
\href{https://docs.google.com/spreadsheets/u/1/d/e/2PACX-1vTH8dUIbfnt3X52TrY3dEHQCAm60e5nqo0Rn1rNCf15dPGeXxM9QN9UdxUfEjxwvfTKzbCbZxJMdR7X/pubhtml?urp=gmail_link}{20
states} have eviction moratoriums, down from 43 in May.

President Trump signed an
\href{https://www.nytimes3xbfgragh.onion/2020/08/09/business/trump-executive-orders-unemployment.html}{executive
order} telling federal agencies to help avoid evictions, but the
provisions were vague. Congress has been at an impasse over new aid, and
a stopgap
\href{https://www.nytimes3xbfgragh.onion/2020/08/13/business/economy/unemployment-benefits-coronavirus.html}{\$300
weekly unemployment supplement announced by Mr. Trump} has reached few
workers so far and will provide only a few weeks of relief.

In the meantime, mounting bills are prompting tenants to take ever more
desperate measures, with potentially devastating long-term effects.

Lindsey Henderson, a laid-off retail bagging assistant from Round Rock,
Texas, has been paying her rent with a Chase Freedom credit card so that
she and her husband can preserve cash and accrue points that help save
on food and gas. Olivia Meaders, a 24-year-old woman in Beaverton, Ore.,
was laid off twice --- once in March, and again in July --- from her job
as a retail manager at a men's apparel store. To make enough money to
pay rent, she began making deliveries for Postmates. Randy Ping, a
49-year-old street performer in Manchester Township, N.J., received
\$3,000 in donations from friends and has paid his rent through
September, but he expects to miss his payment for October and move out
shortly after.

``I don't want to ask people to donate again,'' Mr. Ping said. ``I
really hate borrowing money.''

For others, efforts to conserve money and avoid missing rent have caused
them to retrench on investments like education. That balancing act, even
if it keeps tenants in their homes for now, won't just affect the
near-term economic recovery ---~it could also cause damage that
permanently alters the trajectories of their lives.

\hypertarget{latest-updates-the-coronavirus-outbreak-and-the-economy}{%
\section{\texorpdfstring{\href{https://www.nytimes3xbfgragh.onion/live/2020/09/11/business/stock-market-today-coronavirus?action=click\&pgtype=Article\&state=default\&region=MAIN_CONTENT_1\&context=storylines_live_updates}{Latest
Updates: The Coronavirus Outbreak and the
Economy}}{Latest Updates: The Coronavirus Outbreak and the Economy}}\label{latest-updates-the-coronavirus-outbreak-and-the-economy}}

\href{https://www.nytimes3xbfgragh.onion/live/2020/09/11/business/stock-market-today-coronavirus?action=click\&pgtype=Article\&state=default\&region=MAIN_CONTENT_1\&context=storylines_live_updates\#the-nyse-may-move-its-data-center-out-of-new-jersey-in-response-to-a-proposed-tax}{23h
ago}

\href{https://www.nytimes3xbfgragh.onion/live/2020/09/11/business/stock-market-today-coronavirus?action=click\&pgtype=Article\&state=default\&region=MAIN_CONTENT_1\&context=storylines_live_updates\#the-nyse-may-move-its-data-center-out-of-new-jersey-in-response-to-a-proposed-tax}{The
N.Y.S.E. may move its data center out of New Jersey in response to a
proposed tax.}

\href{https://www.nytimes3xbfgragh.onion/live/2020/09/11/business/stock-market-today-coronavirus?action=click\&pgtype=Article\&state=default\&region=MAIN_CONTENT_1\&context=storylines_live_updates\#the-federal-budget-deficit-hit-3-trillion-as-of-august}{25h
ago}

\href{https://www.nytimes3xbfgragh.onion/live/2020/09/11/business/stock-market-today-coronavirus?action=click\&pgtype=Article\&state=default\&region=MAIN_CONTENT_1\&context=storylines_live_updates\#the-federal-budget-deficit-hit-3-trillion-as-of-august}{The
federal budget deficit hit \$3 trillion as of August.}

\href{https://www.nytimes3xbfgragh.onion/live/2020/09/11/business/stock-market-today-coronavirus?action=click\&pgtype=Article\&state=default\&region=MAIN_CONTENT_1\&context=storylines_live_updates\#warner-bros-pushes-the-release-of-wonder-woman-1984-to-christmas}{25h
ago}

\href{https://www.nytimes3xbfgragh.onion/live/2020/09/11/business/stock-market-today-coronavirus?action=click\&pgtype=Article\&state=default\&region=MAIN_CONTENT_1\&context=storylines_live_updates\#warner-bros-pushes-the-release-of-wonder-woman-1984-to-christmas}{Warner
Bros. pushes the release of `Wonder Woman 1984' to Christmas.}

\href{https://www.nytimes3xbfgragh.onion/live/2020/09/11/business/stock-market-today-coronavirus?action=click\&pgtype=Article\&state=default\&region=MAIN_CONTENT_1\&context=storylines_live_updates}{See
more updates}

More live coverage:
\href{https://www.nytimes3xbfgragh.onion/2020/09/11/world/covid-19-coronavirus.html?action=click\&pgtype=Article\&state=default\&region=MAIN_CONTENT_1\&context=storylines_live_updates}{Global}

\hypertarget{a-would-be-pilot-grounded}{%
\subsection{A Would-Be Pilot,
Grounded}\label{a-would-be-pilot-grounded}}

\includegraphics{https://static01.graylady3jvrrxbe.onion/images/2020/08/21/business/21Virus-Housing-moshtael-02/21Virus-Housing-moshtael-02-articleLarge.jpg?quality=75\&auto=webp\&disable=upscale}

For as long as she can remember, Nura Moshtael has dreamed of being a
pilot for a major airline. She studied aviation in college, but after
giving birth to a son with Down syndrome, she spent a dozen years
raising him on her own, without the time or the money to finish her
training.

Now 45, Ms. Moshtael decided in January that it was time to pursue
becoming a commercial pilot or let her dream slip away. She started two
jobs, one at a local restaurant and another at a cafe inside a Neiman
Marcus. It was exhausting, yet energizing: Each successive paycheck was
a down payment on hopes for a new career and a better life.

Two months later, after the coronavirus struck, she was furloughed from
Neiman Marcus and laid off from the restaurant. With the \$600 a week in
supplementary benefits, she was able to keep paying the rent and
utilities for her apartment in Atlanta. But in July, just as the \$600
payments ended, her lease came up for renewal.

The \$1,460 a month she was receiving in state unemployment benefits ---
her only income --- would not cover her expenses, which included \$1,950
in rent, a \$430 car payment, plus food, utilities and her student loan.
Ms. Moshtael felt she had no choice but to pack up her apartment and
move in with her mother in her childhood home more than 90 miles away in
Macon, Ga.

``I would not have made this decision if the \$600 had still been
intact, but since it was still up in the air and my lease was expiring,
I just had to move out,'' she said. ``I'm just lucky I have somewhere
else to go.''

Though she is relieved to not have to worry about rent, the move comes
at a sacrifice. Her son, who struggles to adapt to new situations, will
have to make new friends. She will miss the daily companionship of her
boyfriend. But perhaps the most poignant and permanent loss for Ms.
Moshtael is the prospect of being a pilot.

``That's dead in the water now,'' she said. ``I can't afford to chase
that dream anymore.''

\hypertarget{from-a-tripled-income-to-the-food-pantry}{%
\subsection{From a Tripled Income to the Food
Pantry}\label{from-a-tripled-income-to-the-food-pantry}}

Image

Jared Strickland and Karla Dennington, with their children, Gracey, 1,
Serenity, 12, and Riley, 14, at his parents' home in St. Augustine,
Fla., where they live.Credit...Charlotte Kesl for The New York Times

Surveys by the Census Bureau and private apartment companies have shown
tenants to be increasingly skeptical that they will be able to make the
next month's rent. So far, the worst predictions have failed to
materialize.

``There has been lots of noise in the media and among policymakers
regarding the potential avalanche of evictions,'' said Daryl J. Carter,
chief executive of Avanath Capital Management, an investment firm in
Irvine, Calif., with 10,000 housing units across the country. ``This is
simply not the case.''

But a strict focus on evictions can create a misleadingly optimistic
picture. The forcible removal of people from their homes is a
complicated, drawn-out process that varies by state because it is
subject to local regulations and courts. While there are close to four
million eviction filings each year, about a million result in tenants'
being removed from their homes, a number that seems to operate
independent of the economic cycle, according to
\href{https://evictionlab.org/}{Princeton University's Eviction Lab}.

Even if that number never budges in the current crisis, the damage is
already here. And as the effects of congressional spending fade, the
pain tenants feel is likely to intensify. Many urban markets are already
seeing rents fall and vacancies rise. ``People that can't pay may have
already moved out of their apartment and in with their family or
friends, thus no longer in the denominator of the collection rates,''
said John Pawlowski, an analyst with Green Street Advisors, a real
estate research firm in Newport Beach, Calif.

Instead of an avalanche, the appropriate metaphor might be a receding
tide that is exposing layers of financial insecurity.

Even before the pandemic,
\href{https://www.jchs.harvard.edu/sites/default/files/Harvard_JCHS_Americas_Rental_Housing_2020.pdf}{about
25 percent of tenants} were paying at least half their pretax income for
housing. Without a paycheck or enough unemployment pay, many may move
out before they actually face eviction. Those who hang on may have to
cut spending on essentials like food and medicine. And for some who have
longed to make the transition from tenant to homeowner, the outlook has
grown more daunting.

Jared Strickland, 36, and his fiancée, Karla Dennington, 34, have been
living with Mr. Strickland's parents for 16 years to save money while he
worked in low-paying retail jobs and she stayed home with their three
children, ages 17 months to 14 years old. In January, their life
changed: Mr. Strickland found a sales job at a time-share company in St.
Augustine, Fla., and his income more than tripled. Full of hope, they
made plans for things that had been financially out of reach: They set a
wedding date of Sept. 13 and started looking for a house to buy.

When the pandemic hit, Mr. Strickland was furloughed and then laid off.
His state unemployment payments ran out over a month ago, so their
extended family is relying on \$150 a week Mr. Strickland gets from the
federal Pandemic Emergency Unemployment Compensation program, Mr.
Strickland's parents' Social Security payments and his 66-year-old
father's part-time job as a greeter at Walmart. To keep food on the
table, they make monthly trips to their local food pantry.

``The plan was to save up for six months and then move into our own
place,'' he said. ``But then all of a sudden, this pandemic happened, I
lost my job, and we're back to square one. We can't even afford food
now. There's no way we'll be able to move out any time soon.''

\hypertarget{bracing-for-nonpayment}{%
\subsection{Bracing for Nonpayment}\label{bracing-for-nonpayment}}

Image

Joseph Razavian, 34, is a landlord in Atlanta who caters to lower-income
tenants.Credit...Audra Melton for The New York Times

Rental housing is a highly fragmented market of 48 million units across
the country, and while collections have been steady at the generally
more expensive properties whose corporate owners are represented by the
National Multifamily Housing Council, strain is emerging in other tiers.
Even as corporate landlords report little change,
\href{https://nahrep.org/landlord-survey/}{smaller landlords are
reporting} declining collections and in many cases expect to use loans
and personal savings to cover shortfalls.

Partly this is because these landlords have less access to capital than
large corporations, but buildings like duplexes and triplexes --- the
kinds of properties that many small landlords own --- tend to be more
affordable, so they attract lower-income tenants, who have been hit the
hardest by the pandemic. A recent survey by
\href{https://leaselock.com/blog/class-c-rent-payments-covid-19/}{LeaseLock},
a company that sells lease insurance to replace rental deposits, showed
that collections in lower-end properties had declined throughout the
pandemic, with only one-quarter of tenants making full rent payment
through the first 15 days of July, compared with 55 percent over the
same period the first three months of the year.

In the early weeks of the pandemic, Joseph Razavian was expecting a
disaster that never arose. Mr. Razavian is a 34-year-old landlord who
owns a duplex and triplex in the Atlanta area. Charging \$1,400 for a
four-bedroom apartment with no security deposit required, Mr. Razavian
caters to lower-income tenants who can't afford to pay a standard rental
arrangement in which at least two months of rent are required up front.

Given the spike in unemployment and the virus's disparate impact on
hourly workers, Mr. Razavian braced for nonpayments and comforted
himself with the reserve fund he'd built up. Then April, May and June
passed with no decline in rent collections, a phenomenon he attributed
to the \$600 unemployment supplement.

Without it, rents have already started slipping. Several tenants haven't
paid rent. Others are making partial payments and asking for extended
payment plans. ``At the beginning of the pandemic, I expected what I'm
seeing now,'' he said.

Mr. Razavian said a tenant who is two months behind on rent has agreed
to work off the debt by mowing the lawn for the rest of the year. (As a
consequence, Mr. Razavian got rid of his landscaping service.) Another
tenant got a \$50-a-month break in exchange for light management duties
like collecting checks and mail.

Unless Congress acts to augment unemployment pay until the job market
rebounds, Mr. Razavian said, ``there are going to be a lot of folks who
don't have money to pay rent.''

How long can he go before he starts considering evicting people? ``The
next couple of months are going to be very interesting,'' he said,
making it clear that by ``interesting'' he meant dreadful.

Image

A residence in East Point, a suburb of Atlanta, that is owned by Mr.
Razavian.Credit...Audra Melton for The New York Times

Mr. Razavian's lower-rent properties are a national early indicator.
\href{https://www.avail.co/blog/landlords-and-renters-struggling-to-make-ends-meet-during-covid-19-uncertainty}{Avail,}
a platform that helps small landlords manage their properties, recently
surveyed about 5,000 tenants and landlords and found that 42 percent of
tenants and 35 percent of landlords were pulling money from savings and
emergency funds to make it through the pandemic.

Katie Bakken, an unemployed medical biologist in Kansas City, Mo., is
months away from anything dire. She has enough savings to pay her rent
through the end of the year and is confident that she will find a job as
soon as the economy recovers. Unsure when that will be or what further
measures, if any, will emerge to help the jobless, she is paring her
spending by cutting out takeout meals and canceling her Costco card, and
accruing debt by paying only the minimum on her credit card bill.

``I can go maybe five or six months paying rent,'' she said. ``But after
that, what do I do?''

Advertisement

\protect\hyperlink{after-bottom}{Continue reading the main story}

\hypertarget{site-index}{%
\subsection{Site Index}\label{site-index}}

\hypertarget{site-information-navigation}{%
\subsection{Site Information
Navigation}\label{site-information-navigation}}

\begin{itemize}
\tightlist
\item
  \href{https://help.nytimes3xbfgragh.onion/hc/en-us/articles/115014792127-Copyright-notice}{©~2020~The
  New York Times Company}
\end{itemize}

\begin{itemize}
\tightlist
\item
  \href{https://www.nytco.com/}{NYTCo}
\item
  \href{https://help.nytimes3xbfgragh.onion/hc/en-us/articles/115015385887-Contact-Us}{Contact
  Us}
\item
  \href{https://www.nytco.com/careers/}{Work with us}
\item
  \href{https://nytmediakit.com/}{Advertise}
\item
  \href{http://www.tbrandstudio.com/}{T Brand Studio}
\item
  \href{https://www.nytimes3xbfgragh.onion/privacy/cookie-policy\#how-do-i-manage-trackers}{Your
  Ad Choices}
\item
  \href{https://www.nytimes3xbfgragh.onion/privacy}{Privacy}
\item
  \href{https://help.nytimes3xbfgragh.onion/hc/en-us/articles/115014893428-Terms-of-service}{Terms
  of Service}
\item
  \href{https://help.nytimes3xbfgragh.onion/hc/en-us/articles/115014893968-Terms-of-sale}{Terms
  of Sale}
\item
  \href{https://spiderbites.nytimes3xbfgragh.onion}{Site Map}
\item
  \href{https://help.nytimes3xbfgragh.onion/hc/en-us}{Help}
\item
  \href{https://www.nytimes3xbfgragh.onion/subscription?campaignId=37WXW}{Subscriptions}
\end{itemize}
