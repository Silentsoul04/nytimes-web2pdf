Sections

SEARCH

\protect\hyperlink{site-content}{Skip to
content}\protect\hyperlink{site-index}{Skip to site index}

\href{https://myaccount.nytimes3xbfgragh.onion/auth/login?response_type=cookie\&client_id=vi}{}

\href{https://www.nytimes3xbfgragh.onion/section/todayspaper}{Today's
Paper}

\href{/section/opinion}{Opinion}\textbar{}Where Hope and History Rhyme

\url{https://nyti.ms/3l5pQ4r}

\begin{itemize}
\item
\item
\item
\item
\item
\item
\end{itemize}

Advertisement

\protect\hyperlink{after-top}{Continue reading the main story}

\href{/section/opinion}{Opinion}

Supported by

\protect\hyperlink{after-sponsor}{Continue reading the main story}

\hypertarget{where-hope-and-history-rhyme}{%
\section{Where Hope and History
Rhyme}\label{where-hope-and-history-rhyme}}

Joe Biden's speech did not soar. That's good. America is looking for
decency and competence.

\href{https://www.nytimes3xbfgragh.onion/by/roger-cohen}{\includegraphics{https://static01.graylady3jvrrxbe.onion/images/2014/11/01/opinion/cohen-circular/cohen-circular-thumbLarge-v6.png}}

By \href{https://www.nytimes3xbfgragh.onion/by/roger-cohen}{Roger Cohen}

Opinion Columnist

\begin{itemize}
\item
  Aug. 21, 2020
\item
  \begin{itemize}
  \item
  \item
  \item
  \item
  \item
  \item
  \end{itemize}
\end{itemize}

\includegraphics{https://static01.graylady3jvrrxbe.onion/images/2020/08/21/opinion/21cohen1/merlin_175968585_5be84602-386a-4437-a16b-820ab6b96155-articleLarge.jpg?quality=75\&auto=webp\&disable=upscale}

Accepting the Democratic presidential nomination, Joe Biden said:
``Character is on the ballot. Compassion is on the ballot. Decency,
science, democracy.''

The first, character, is the most important for without it the rest are
mere words. Donald Trump has given America the definitive lesson in the
scourge of indecent narcissism. Russia was about him. Race was about
him. The virus was about him. You can't sculpt in rotten wood and you
can't rule from a rotten core.

\href{https://www.nytimes3xbfgragh.onion/2020/08/20/us/politics/biden-presidential-nomination-dnc.html}{Biden's
speech} did not soar. That was good. America, after its season of lies,
is ready for simple declarative sentences. Truth is on the ballot, too.

The Democratic nominee conveyed who he is: a plain-spoken American
schooled from his Scranton youth in the nation's can-do spirit, deepened
by suffering, consoled by faith, driven to end the carnage of the
president's self-obsession. Biden left no doubt that he has lived this
presidency as an offense to America --- and to himself.

``We will choose hope over fear, facts over fiction, fairness over
privilege,'' Biden said.

Each of those goals is important. Truth is fundamental to democracy.
Lies are the stuff of authoritarian regimes, which is why Trump likes
them. But fairness is the most relevant compass for Biden because the
Democratic Party has failed in the 21st century to place fairness before
privilege --- and in 2016 Americans made clear they were done with that
game.

Biden came across as genuine. He spoke not from calculation but his
core. This was his task; he succeeded. On character, he delivered. The
nomination came his way because Americans were not ready for a
revolution. They wanted an anchor in a world upended.

Trump triumphed in 2016 as an impostor. He won as the voice of the
dispossessed, the mouthpiece of the unsayable. Exploiting fear, he
restored violence to a wan political stage of PowerPoint slides. He cut
through the anesthetized language of globalized elites. He attacked
money-wooed Democrats estranged from their white, blue-collar
constituencies. He aimed a howitzer at what the Clintons' Democratic
Party had become.

Because Biden was so much a part of what the Democratic Party had
become,
\href{https://www.nytimes3xbfgragh.onion/2019/05/03/opinion/joe-biden.html}{I
wrote in May last year} that he was not the candidate to beat Trump.
Three months later,
\href{https://www.nytimes3xbfgragh.onion/2019/08/02/opinion/trump-2020-kamala-harris.html}{I
wrote} that the who-can-beat-Trump test led to Kamala Harris, because
``she's tough, broadly of the center, has a great American story, is
passionate on issues including immigrants, African-Americans and
women.'' I quoted her calling Trump a ``predator'' and calling predators
``cowards.''

In the end, a Biden-Harris ticket is the best pick for the Democratic
Party, its best hope to fire Trump. Because the pandemic prioritized a
safe pair of hands; because Biden, prodded by Bernie Sanders and
Elizabeth Warren, has adjusted leftward without losing centrist
Democrats; because Biden no longer looks like the tired restoration of
an old order but an essential pivot to sanity, decency and competence,
and because the ticket embodies ideas of racial justice, generational
balance and reconciliation (between the two candidates and all
Americans).

``We don't need a tax code that rewards wealth more than it rewards
work,'' Biden said. That's scarcely rocket science. If elected, Biden
will be measured on whether he can change the tax code, a foundation of
the growing inequality and injustice in a fractured America whose
ability to cohere has been lost.

This is the America of five million infected with the Covid-19 virus, of
170,000 dead in the pandemic, of
\href{https://www.sfchronicle.com/business/article/More-than-50-million-have-filed-for-unemployment-15412510.php}{over
50 million} unemployment claims, of, as Biden said, ``by far the worst
performance of any nation on earth.''

And, as Biden failed to say, of major United States stock indexes
\href{https://www.nytimes3xbfgragh.onion/2020/08/18/business/stock-market-record.html}{at
or close to record highs}. The virus, destroying small businesses, has
completed the financial money game's takeover of the economy. Trump is
counting on this, and on blather about Democrats' ``socialism,'' to win.
His chances should not be discounted. The state of a 401(k) is a
significant vote indicator. But less so, I think, in this desperate
America of Trump's making.

In recent weeks, I have been watching the United States from a Europe
orphaned of its American ally. ``It's not this bad in Canada. Or
Europe,'' Biden said. It's not. Confronting a crisis with a plan does
help. Under Trump, an American passport in Europe has become
\href{https://www.nytimes3xbfgragh.onion/2020/07/07/world/europe/american-passport-privilege-coronavirus.html}{a
good thing not to have}.

Biden understands an alliance undergirded by values. The passion in his
voice rose as he said: ``I will be a president who will stand with our
allies and friends. I will make it clear to our adversaries the days of
cozying up to dictators are over.''

Hope is often unfounded, but it is not an irrational response to human
experience. The miracle of a peaceful Europe today is built on joint
American-European defiance, in freedom's cause, of fascism and
totalitarianism.

Biden quoted the Irish poet Seamus Heaney on how, every now and again,
``hope and history rhyme.'' Gazing this week at the beauty of Florence,
where
\href{https://www.nytimes3xbfgragh.onion/2014/04/08/opinion/cohen-from-death-into-life.html}{my
uncle Bert Cohen}, of the Sixth South African Armored Division, 19th
Field Ambulance, came from Johannesburg in 1944 to join the fight for
freedom, I thought, yes, they do.

After Biden's sober speech the chances they will again in November have
risen.

\emph{The Times is committed to publishing}
\href{https://www.nytimes3xbfgragh.onion/2019/01/31/opinion/letters/letters-to-editor-new-york-times-women.html}{\emph{a
diversity of letters}} \emph{to the editor. We'd like to hear what you
think about this or any of our articles. Here are some}
\href{https://help.nytimes3xbfgragh.onion/hc/en-us/articles/115014925288-How-to-submit-a-letter-to-the-editor}{\emph{tips}}\emph{.
And here's our email:}
\href{mailto:letters@NYTimes.com}{\emph{letters@NYTimes.com}}\emph{.}

\emph{Follow The New York Times Opinion section on}
\href{https://www.facebookcorewwwi.onion/nytopinion}{\emph{Facebook}}\emph{,}
\href{http://twitter.com/NYTOpinion}{\emph{Twitter (@NYTopinion)}}
\emph{and}
\href{https://www.instagram.com/nytopinion/}{\emph{Instagram}}\emph{.}

Advertisement

\protect\hyperlink{after-bottom}{Continue reading the main story}

\hypertarget{site-index}{%
\subsection{Site Index}\label{site-index}}

\hypertarget{site-information-navigation}{%
\subsection{Site Information
Navigation}\label{site-information-navigation}}

\begin{itemize}
\tightlist
\item
  \href{https://help.nytimes3xbfgragh.onion/hc/en-us/articles/115014792127-Copyright-notice}{©~2020~The
  New York Times Company}
\end{itemize}

\begin{itemize}
\tightlist
\item
  \href{https://www.nytco.com/}{NYTCo}
\item
  \href{https://help.nytimes3xbfgragh.onion/hc/en-us/articles/115015385887-Contact-Us}{Contact
  Us}
\item
  \href{https://www.nytco.com/careers/}{Work with us}
\item
  \href{https://nytmediakit.com/}{Advertise}
\item
  \href{http://www.tbrandstudio.com/}{T Brand Studio}
\item
  \href{https://www.nytimes3xbfgragh.onion/privacy/cookie-policy\#how-do-i-manage-trackers}{Your
  Ad Choices}
\item
  \href{https://www.nytimes3xbfgragh.onion/privacy}{Privacy}
\item
  \href{https://help.nytimes3xbfgragh.onion/hc/en-us/articles/115014893428-Terms-of-service}{Terms
  of Service}
\item
  \href{https://help.nytimes3xbfgragh.onion/hc/en-us/articles/115014893968-Terms-of-sale}{Terms
  of Sale}
\item
  \href{https://spiderbites.nytimes3xbfgragh.onion}{Site Map}
\item
  \href{https://help.nytimes3xbfgragh.onion/hc/en-us}{Help}
\item
  \href{https://www.nytimes3xbfgragh.onion/subscription?campaignId=37WXW}{Subscriptions}
\end{itemize}
