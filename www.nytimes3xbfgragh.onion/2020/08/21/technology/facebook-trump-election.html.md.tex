Sections

SEARCH

\protect\hyperlink{site-content}{Skip to
content}\protect\hyperlink{site-index}{Skip to site index}

\href{https://www.nytimes3xbfgragh.onion/section/technology}{Technology}

\href{https://myaccount.nytimes3xbfgragh.onion/auth/login?response_type=cookie\&client_id=vi}{}

\href{https://www.nytimes3xbfgragh.onion/section/todayspaper}{Today's
Paper}

\href{/section/technology}{Technology}\textbar{}Facebook Braces Itself
for Trump to Cast Doubt on Election Results

\url{https://nyti.ms/3iZTjLg}

\begin{itemize}
\item
\item
\item
\item
\item
\item
\end{itemize}

\begin{itemize}
\item
  \href{https://www.nytimes3xbfgragh.onion/interactive/2020/09/08/us/elections/results-new-hampshire-primary-elections.html?action=click\&pgtype=Article\&state=default\&region=TOP_BANNER\&context=storylines_menu}{New
  Hampshire Results}
\item
  \href{https://www.nytimes3xbfgragh.onion/live/2020/09/08/us/trump-vs-biden?action=click\&pgtype=Article\&state=default\&region=TOP_BANNER\&context=storylines_menu}{Election
  Updates}
\item
  \href{https://www.nytimes3xbfgragh.onion/interactive/2020/us/elections/election-states-biden-trump.html?action=click\&pgtype=Article\&state=default\&region=TOP_BANNER\&context=storylines_menu}{Paths
  to 270}
\item
  \href{https://www.nytimes3xbfgragh.onion/interactive/2020/08/31/us/politics/vote-by-mail-deadlines.html?action=click\&pgtype=Article\&state=default\&region=TOP_BANNER\&context=storylines_menu}{Voting
  by Mail}
\item
  \href{https://www.nytimes3xbfgragh.onion/interactive/2019/us/elections/2020-presidential-election-calendar.html?action=click\&pgtype=Article\&state=default\&region=TOP_BANNER\&context=storylines_menu}{Key
  Dates}
\item
  \href{https://www.nytimes3xbfgragh.onion/newsletters/politics?action=click\&pgtype=Article\&state=default\&region=TOP_BANNER\&context=storylines_menu}{Politics
  Newsletter}
\end{itemize}

Advertisement

\protect\hyperlink{after-top}{Continue reading the main story}

Supported by

\protect\hyperlink{after-sponsor}{Continue reading the main story}

\hypertarget{facebook-braces-itself-for-trump-to-cast-doubt-on-election-results}{%
\section{Facebook Braces Itself for Trump to Cast Doubt on Election
Results}\label{facebook-braces-itself-for-trump-to-cast-doubt-on-election-results}}

The world's biggest social network is working out what steps to take
should President Trump use its platform to dispute the vote.

\includegraphics{https://static01.graylady3jvrrxbe.onion/images/2020/08/20/business/00facebookelection/00facebookelection-articleLarge.gif?quality=75\&auto=webp\&disable=upscale}

\href{https://www.nytimes3xbfgragh.onion/by/mike-isaac}{\includegraphics{https://static01.graylady3jvrrxbe.onion/images/2018/02/16/multimedia/author-mike-isaac/author-mike-isaac-thumbLarge.jpg}}\href{https://www.nytimes3xbfgragh.onion/by/sheera-frenkel}{\includegraphics{https://static01.graylady3jvrrxbe.onion/images/2018/06/14/multimedia/author-sheera-frenkel/author-sheera-frenkel-thumbLarge.png}}

By \href{https://www.nytimes3xbfgragh.onion/by/mike-isaac}{Mike Isaac}
and \href{https://www.nytimes3xbfgragh.onion/by/sheera-frenkel}{Sheera
Frenkel}

\begin{itemize}
\item
  Published Aug. 21, 2020Updated Aug. 24, 2020
\item
  \begin{itemize}
  \item
  \item
  \item
  \item
  \item
  \item
  \end{itemize}
\end{itemize}

SAN FRANCISCO --- Facebook spent years preparing to
\href{https://www.nytimes3xbfgragh.onion/2020/03/29/technology/facebook-google-twitter-november-election.html}{ward
off any tampering} on its site ahead of November's presidential
election. Now the social network is getting ready in case
\href{https://www.nytimes3xbfgragh.onion/2020/09/03/us/politics/trump-2020-election.html}{President
Trump} interferes once the vote is over.

Employees at the Silicon Valley company are laying out contingency plans
and walking through postelection scenarios that include attempts by Mr.
Trump or his campaign to use the platform to delegitimize the results,
people with knowledge of Facebook's plans said.

\href{https://www.nytimes3xbfgragh.onion/2020/09/03/technology/facebook-election-chaos-november.html}{Facebook}
is preparing steps to take should Mr. Trump wrongly claim on the site
that he won another four-year term, said the people, who spoke on the
condition of anonymity. Facebook is also working through how it might
act if Mr. Trump tries to invalidate the results by declaring that the
Postal Service lost mail-in ballots or that other groups meddled with
the vote, the people said.

Mark Zuckerberg, Facebook's chief executive, and some of his lieutenants
have started holding daily meetings about minimizing how the platform
can be used to dispute the election, the people said. They have
discussed a ``kill switch'' to shut off political advertising after
Election Day since the ads, which
\href{https://www.nytimes3xbfgragh.onion/2019/10/17/business/zuckerberg-facebook-free-speech.html}{Facebook
does not police for truthfulness}, could be used to spread
misinformation, the people said.

The preparations underscore how
\href{https://www.nytimes3xbfgragh.onion/2020/08/15/us/post-office-vote-by-mail.html}{rising
concerns}over the integrity of the November election have reached social
media companies, whose sites can be used to amplify lies, conspiracy
theories and inflammatory messages. YouTube and Twitter have also
discussed plans for action if the postelection period becomes
complicated, according to disinformation and political researchers who
have advised the firms.

The tech companies have spent the past few years working to avoid a
repeat of the 2016 election, when
\href{https://www.nytimes3xbfgragh.onion/2018/02/17/technology/indictment-russian-tech-facebook.html}{Russian
operatives used Facebook}, Twitter and YouTube to inflame the American
electorate with divisive messages. While the firms have since clamped
down on foreign meddling, they are reckoning with a surge of domestic
interference, such as from the
\href{https://www.nytimes3xbfgragh.onion/article/what-is-qanon.html}{right-wing
conspiracy group QAnon} and Mr. Trump himself.

In recent weeks, Mr. Trump, who uses social media as a megaphone, has
sharpened his comments about the election. He has questioned the
legitimacy of mail-in voting, suggested that people's mail-in ballots
would not be counted and avoided answering whether he would step down if
he lost.

Alex Stamos, director of Stanford University's Internet Observatory and
\href{https://www.nytimes3xbfgragh.onion/2018/03/19/technology/facebook-alex-stamos.html}{a
former Facebook executive}, said Facebook, Twitter and YouTube faced a
singular situation where they ``have to potentially treat the president
as a bad actor'' who could undermine the democratic process.

``We don't have experience with that in the United States,'' Mr. Stamos
added.

Facebook may be in an especially difficult position because Mr.
Zuckerberg has said the social network
\href{https://www.nytimes3xbfgragh.onion/2020/06/02/technology/zuckerberg-defends-facebook-trump-posts.html}{stands
for free speech}. Unlike Twitter, which has
\href{https://www.nytimes3xbfgragh.onion/2020/05/30/technology/twitter-trump-dorsey.html}{flagged
Mr. Trump's tweets} for being factually inaccurate and glorifying
violence, Facebook has said that politicians' posts are newsworthy and
that the public has the right to see them. Taking any action on posts
from Mr. Trump or his campaign after the vote could open Facebook up to
accusations of censorship and anticonservative bias.

\includegraphics{https://static01.graylady3jvrrxbe.onion/images/2020/08/19/business/00facebookelection2/merlin_143971272_f809ac54-d2cf-4e2a-b08b-fc483f4f592e-articleLarge.jpg?quality=75\&auto=webp\&disable=upscale}

In
\href{https://www.nytimes3xbfgragh.onion/2020/08/02/business/media/election-coverage.html}{an
interview with The New York Times} this month, Mr. Zuckerberg said of
the election that people should be ``ready for the fact that there's a
high likelihood that it takes days or weeks to count this --- and
there's nothing wrong or illegitimate about that.''

A spokesman for Facebook declined to comment on its postelection
strategy. ``We continue to plan for a range of scenarios to make sure we
are prepared for the upcoming election,'' he said.

Judd Deere, a White House spokesman, said, ``President Trump will
continue to work to ensure the security and integrity of our
elections.''

Google, which owns YouTube, confirmed that it was holding conversations
on postelection strategy but declined to elaborate. Jessica
Herrera-Flanigan, Twitter's vice president of public policy, said the
company was evolving its policies to ``better identify, understand and
mitigate threats to the public conversation, both before or after an
election.''

Facebook had initially focused on the run-up to the election --- the
period when, in 2016, most of the Russian meddling took place on its
site. The company mapped out almost 80 scenarios, many of which looked
at what might go wrong on its platform before Americans voted, the
people with knowledge of the discussions said.

Facebook examined what it would do, for instance, if hackers backed by a
nation-state leaked documents online, or if a nation-state unleashed a
widespread disinformation campaign at the last minute to dissuade
Americans from going to the polls, one employee said.

To bolster the effort, Facebook invited those in government, think tanks
and academia to participate and conduct exercises around the
hypothetical election situations.

An idea that came up during one exercise --- that Facebook label posts
from state media so users know they are reading government-sponsored
content --- was put into effect in June, said Graham Brookie, director
of the Atlantic Council's Digital Forensic Research Lab, who joined the
session.

``We can see that their policy decisions are being affected by these
exercises,'' he said.

But Facebook was less decisive on other issues. If a post suggested that
mail-in voting was broken, or encouraged people to send in multiple
copies of their mail-in ballots, the company would not remove the
messages if they were framed as a suggestion or a question, one person
who advised the company said. Under Facebook's rules, it takes down only
voting-related posts that are statements with obviously false and
misleading information.

In recent months, Facebook turned more to postelection planning. That
shift accelerated this month when Mr. Trump said more on the issue, two
Facebook employees said.

On Aug. 3,
\href{https://www.nytimes3xbfgragh.onion/2020/08/03/us/politics/trump-mail-in-voting.html}{Mr.
Trump questioned} whether the Democratic primary in New York's 12th
Congressional District should be rerun because of long delays in
counting
\href{https://www.nytimes3xbfgragh.onion/interactive/2020/08/11/us/politics/vote-by-mail-us-states.html}{mail-in
ballots}.

``Nobody knows what's happening with the ballots and the lost ballots
and the fraudulent ballots, I guess,'' he said.

The next day, Mr. Trump broadened his attack,
\href{https://www.nytimes3xbfgragh.onion/article/mail-in-voting-explained.html}{falsely
stating} that mail-in ballots lead to more voter fraud nationwide.
``Mail ballots are very dangerous for this country because of
cheaters,'' he said. ``They go collect them. They are fraudulent in many
cases.''

Mr. Trump's comments alarmed Facebook employees who work on protecting
its site in the U.S. election. On the group's internal chat channels,
many wondered whether Mr. Trump would launch even more attacks against
mail-in voting, one employee who saw the messages said. Some asked
whether the president was violating Facebook's rules against
disenfranchising voters.

Those questions were ultimately sent to Mr. Zuckerberg, as well as top
executives including
\href{https://www.nytimes3xbfgragh.onion/2016/05/18/technology/facebook-moves-to-repair-its-fractured-relationship-with-the-right.html}{Joel
Kaplan}, the global head of public policy, the employee said.

Image

Questions about postelection issues were ultimately sent to top Facebook
executives, including Mark Zuckerberg, left, and Joel Kaplan,
right.Credit...Samuel Corum/Getty Images

In a staff meeting later that week, Mr. Zuckerberg told employees that
if political figures or commentators tried declaring victory in an
election early, Facebook would consider adding a label to their posts
explaining that the results were not final. Of Mr. Trump, Mr. Zuckerberg
said the company was ``in unprecedented territory with the president
saying some of the things that he's saying that I find quite
troubling.'' The meeting was
\href{https://www.buzzfeednews.com/article/craigsilverman/facebook-zuckerberg-what-if-trump-disputes-election-results}{reported
earlier by BuzzFeed News}.

Since then, executives have discussed the ``kill switch'' for political
advertising, according to two employees, which would turn off political
ads after Nov. 3 if the election's outcome was not immediately clear or
if Mr. Trump disputed the results.

The discussions remain fluid, and it is unclear if Facebook will follow
through with the plan, three people close to the talks said.

In a call with reporters this month, Facebook executives said they had
removed more than 110,000 pieces of content between March and July that
violated the company's election-related policies. They also said there
was a lot about the election that they didn't know.

``In this fast-changing environment, we are always sort of `red teaming'
and working with partners to understand what are the next risks?'' said
Guy Rosen, vice president of integrity at Facebook. ``What are the
different kinds of things that may go wrong?''

Mike Isaac reported from San Francisco, and Sheera Frenkel from Oakland,
Calif.

\hypertarget{our-2020-election-guide}{%
\section{Our 2020 Election Guide}\label{our-2020-election-guide}}

Updated ~Sept. 8, 2020

\begin{center}\rule{0.5\linewidth}{\linethickness}\end{center}

\begin{itemize}
\item ~
  \hypertarget{the-latest}{%
  \subsection{The Latest}\label{the-latest}}

  \begin{itemize}
  \item
    President Trump and his party are using a playbook that aims to
    alarm people about crime in their backyards. It didn't work in 2018,
    but
    \href{https://www.nytimes3xbfgragh.onion/2020/09/08/us/politics/trump-republicans-fear-strategy.html?action=click\&pgtype=Article\&state=default\&region=BELOW_MAIN_CONTENT\&context=storylines_guide}{both
    parties think it could resonate more this year}.
  \end{itemize}
\item ~
  \hypertarget{how-to-win-270}{%
  \subsection{How to Win 270}\label{how-to-win-270}}

  \begin{itemize}
  \item
    Joe Biden and Donald Trump need 270 electoral votes to reach the
    White House. Try building
    \href{https://www.nytimes3xbfgragh.onion/interactive/2020/us/elections/election-states-biden-trump.html?action=click\&pgtype=Article\&state=default\&region=BELOW_MAIN_CONTENT\&context=storylines_guide}{your
    own coalition of battleground states}~to see potential outcomes.
  \end{itemize}
\item ~
  \hypertarget{voting-by-mail}{%
  \subsection{Voting by Mail}\label{voting-by-mail}}

  \begin{itemize}
  \item
    Will you have enough time to vote by mail in your state? Yes, but
    it's risky to procrastinate.
    \href{https://www.nytimes3xbfgragh.onion/interactive/2020/08/31/us/politics/vote-by-mail-deadlines.html?action=click\&pgtype=Article\&state=default\&region=BELOW_MAIN_CONTENT\&context=storylines_guide}{Check
    your state's deadline.}
  \item
    \href{https://www.nytimes3xbfgragh.onion/interactive/2020/us/elections/joe-biden.html?action=click\&pgtype=Article\&state=default\&region=BELOW_MAIN_CONTENT\&context=storylines_guide}{}

    \hypertarget{joe-biden}{%
    \section{Joe Biden}\label{joe-biden}}

    \hypertarget{democrat}{%
    \subsection{Democrat}\label{democrat}}

    \href{https://www.nytimes3xbfgragh.onion/interactive/2020/us/elections/donald-trump.html?action=click\&pgtype=Article\&state=default\&region=BELOW_MAIN_CONTENT\&context=storylines_guide}{}

    \hypertarget{donald-trump}{%
    \section{Donald Trump}\label{donald-trump}}

    \hypertarget{republican}{%
    \subsection{Republican}\label{republican}}
  \end{itemize}
\item
  \hypertarget{keep-up-with-our-coverage}{%
  \subsection{Keep Up With Our
  Coverage}\label{keep-up-with-our-coverage}}

  \begin{itemize}
  \item
    Get an
    \href{https://www.nytimes3xbfgragh.onion/newsletters/politics?action=click\&pgtype=Article\&state=default\&region=BELOW_MAIN_CONTENT\&context=storylines_guide}{email}~recapping
    the day's news
  \item
    Download our mobile app on
    \href{https://apps.apple.com/us/app/nytimes/id284862083?ls=1\&mat_click_id=5c79ae7455014fd1bd66b5610c05b8f2-20191112-16948\&referrer=mat_click_id\%3D5c79ae7455014fd1bd66b5610c05b8f2-20191112-16948\%26link_click_id\%3D722930677036718082}{iOS}~and
    \href{http://a.localytics.com/android?id=com.nytimes.android\&referrer=utm_source\%3Dother_nyt_mobile_web\%26utm_medium\%3DWeb\%2520page\%26utm_term\%3DGeneral\%2520Mobile\%2520Page\%26utm_campaign\%3DNYT\%2520Mobile\%2520General\%2520Page}{Android}~and
    turn on Breaking News and Politics alerts
  \end{itemize}
\end{itemize}

Advertisement

\protect\hyperlink{after-bottom}{Continue reading the main story}

\hypertarget{site-index}{%
\subsection{Site Index}\label{site-index}}

\hypertarget{site-information-navigation}{%
\subsection{Site Information
Navigation}\label{site-information-navigation}}

\begin{itemize}
\tightlist
\item
  \href{https://help.nytimes3xbfgragh.onion/hc/en-us/articles/115014792127-Copyright-notice}{©~2020~The
  New York Times Company}
\end{itemize}

\begin{itemize}
\tightlist
\item
  \href{https://www.nytco.com/}{NYTCo}
\item
  \href{https://help.nytimes3xbfgragh.onion/hc/en-us/articles/115015385887-Contact-Us}{Contact
  Us}
\item
  \href{https://www.nytco.com/careers/}{Work with us}
\item
  \href{https://nytmediakit.com/}{Advertise}
\item
  \href{http://www.tbrandstudio.com/}{T Brand Studio}
\item
  \href{https://www.nytimes3xbfgragh.onion/privacy/cookie-policy\#how-do-i-manage-trackers}{Your
  Ad Choices}
\item
  \href{https://www.nytimes3xbfgragh.onion/privacy}{Privacy}
\item
  \href{https://help.nytimes3xbfgragh.onion/hc/en-us/articles/115014893428-Terms-of-service}{Terms
  of Service}
\item
  \href{https://help.nytimes3xbfgragh.onion/hc/en-us/articles/115014893968-Terms-of-sale}{Terms
  of Sale}
\item
  \href{https://spiderbites.nytimes3xbfgragh.onion}{Site Map}
\item
  \href{https://help.nytimes3xbfgragh.onion/hc/en-us}{Help}
\item
  \href{https://www.nytimes3xbfgragh.onion/subscription?campaignId=37WXW}{Subscriptions}
\end{itemize}
