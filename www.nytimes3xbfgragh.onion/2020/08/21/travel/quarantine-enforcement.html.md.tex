Sections

SEARCH

\protect\hyperlink{site-content}{Skip to
content}\protect\hyperlink{site-index}{Skip to site index}

\href{https://www.nytimes3xbfgragh.onion/section/travel}{Travel}

\href{https://myaccount.nytimes3xbfgragh.onion/auth/login?response_type=cookie\&client_id=vi}{}

\href{https://www.nytimes3xbfgragh.onion/section/todayspaper}{Today's
Paper}

\href{/section/travel}{Travel}\textbar{}Quarantine and Travel: Strict
Penalties, Rare Enforcement

\url{https://nyti.ms/2Qefh0M}

\begin{itemize}
\item
\item
\item
\item
\item
\end{itemize}

\href{https://www.nytimes3xbfgragh.onion/spotlight/at-home?action=click\&pgtype=Article\&state=default\&region=TOP_BANNER\&context=at_home_menu}{At
Home}

\begin{itemize}
\tightlist
\item
  \href{https://www.nytimes3xbfgragh.onion/2020/09/07/travel/route-66.html?action=click\&pgtype=Article\&state=default\&region=TOP_BANNER\&context=at_home_menu}{Cruise
  Along: Route 66}
\item
  \href{https://www.nytimes3xbfgragh.onion/2020/09/04/dining/sheet-pan-chicken.html?action=click\&pgtype=Article\&state=default\&region=TOP_BANNER\&context=at_home_menu}{Roast:
  Chicken With Plums}
\item
  \href{https://www.nytimes3xbfgragh.onion/2020/09/04/arts/television/dark-shadows-stream.html?action=click\&pgtype=Article\&state=default\&region=TOP_BANNER\&context=at_home_menu}{Watch:
  Dark Shadows}
\item
  \href{https://www.nytimes3xbfgragh.onion/interactive/2020/at-home/even-more-reporters-editors-diaries-lists-recommendations.html?action=click\&pgtype=Article\&state=default\&region=TOP_BANNER\&context=at_home_menu}{Explore:
  Reporters' Google Docs}
\end{itemize}

Advertisement

\protect\hyperlink{after-top}{Continue reading the main story}

Supported by

\protect\hyperlink{after-sponsor}{Continue reading the main story}

\hypertarget{quarantine-and-travel-strict-penalties-rare-enforcement}{%
\section{Quarantine and Travel: Strict Penalties, Rare
Enforcement}\label{quarantine-and-travel-strict-penalties-rare-enforcement}}

As quarantine requirements for travelers have become increasingly common
throughout the United States, the prevailing theme is the difficulty of
effective enforcement.

\includegraphics{https://static01.graylady3jvrrxbe.onion/images/2020/08/21/travel/21travel-enforcement/21Quarantine-Enforcement-articleLarge.jpg?quality=75\&auto=webp\&disable=upscale}

By Lauren Sloss

\begin{itemize}
\item
  Published Aug. 21, 2020Updated Aug. 31, 2020
\item
  \begin{itemize}
  \item
  \item
  \item
  \item
  \item
  \end{itemize}
\end{itemize}

Mika Salamanca's mug shot is a departure from most of the images you'll
find of her online, particularly the ones she posts on Instagram to her
1 million-plus followers --- winsome selfies and playful portraits with
her head cocked to the side. But in the photograph released by the
Honolulu Police Department, Ms. Salamanca, a social media influencer
from the Philippines, stares directly into the camera, her expression
inscrutable.

Ms. Salamanca
\href{https://www.kitv.com/story/42406880/social-media-influencer-apologizes-for-breaking-hawaiis-mandated-14day-quarantine}{was
arrested in Honolulu, Hawaii on July 24} for having broken the state's
\href{https://hidot.hawaii.gov/coronavirus/}{mandatory 14-day
quarantine}. She was apprehended after posting images and videos out
with friends within days of her arrival, leading a group of locals to
report her to authorities.

``One day it will all make sense,'' reads an
\href{https://www.instagram.com/p/CDbn_vinVjS/}{Instagram post} from
Aug. 4, which racked up more than 33,000 likes.

Whether it makes sense or not, Ms. Salamanca's transgressions,
subsequent arrest and
\href{https://www.youtube.com/watch?v=1DA9436drzo}{public apology} is
one of the more titillating, and public, examples of quarantine
enforcement in the United States. But as quarantine requirements for
travelers have become
\href{https://www.nytimes3xbfgragh.onion/2020/07/10/travel/state-travel-restrictions.html}{increasingly
common throughout the United States}, in states including New York,
Vermont and Kentucky, plus the city of Chicago, the prevailing theme is
the difficulty of effective enforcement.

Image

A photograph of Anne Salamanca, who goes by Mika Salamanca, released by
the~Hawaii Department of Public Safety.Credit...Hawaii Department of
Public Safety, via Associated Press

Roughly six months into the coronavirus pandemic, local and state
governments have struggled to contain the spread of the coronavirus with
limited manpower and dwindling resources, and following quarantine rules
is more
\href{https://www.nytimes3xbfgragh.onion/2020/08/16/nyregion/coronavirus-quarantine-nyc.html}{often
than not left to the traveler's discretion}. Officials hope that threats
of fines or imprisonment and, more crucially, an honor code state of
mind, will prove effective. But as issues of right and wrong become
increasingly muddied in our ever-shifting pandemic world, can travelers
be counted on to do the right thing?

Hawaii's quarantine order for trans-Pacific travelers went into effect
on March 26, and currently applies to everyone who enters the state
(Gov. David Ige further instituted a partial quarantine requirement
\href{https://www.khon2.com/coronavirus/reminder-partial-interisland-travel-quarantine-goes-into-effect-aug-11/}{for
inter-island travel}, beginning Aug. 11). Those caught breaking
quarantine can be fined up to \$5,000, or imprisoned for up to one year.

In contrast, most other states' quarantine orders are only required for
travelers coming from places experiencing high infection rates.
\href{https://coronavirus.health.ny.gov/covid-19-travel-advisory}{In New
York}, that currently includes more than 30 states, plus Puerto Rico and
the U.S. Virgin Islands, according to a list that is updated weekly.
Those who violate the quarantine order are subject to fines up to
\$10,000. Connecticut
\href{https://ctmirror.org/2020/08/12/lamont-administration-fines-5-more-ct-residents-for-violating-covid-travel-rules/}{has
started issuing \$1,000 fines} to visitors who fail to submit a travel
advisory form to its public health department or violate quarantine
orders. Other states, including
\href{https://coronavirus.ohio.gov/wps/portal/gov/covid-19/families-and-individuals/COVID-19-Travel-Advisory/COVID-19-Travel-Advisory}{Ohio}
and \href{https://www.covidguidance.nh.gov/out-state-visitors}{New
Hampshire}, frame quarantine requirements as recommendations or requests
without the threat of fines.

Hawaii's geographic remoteness, and the fact that it is, well, a group
of islands, puts it at an advantage when it comes to monitoring who is
coming or going. All travelers are required to fill out a form provided
by the state's transportation department upon arrival, and quarantine
enforcement is handled by individual counties. As of Aug. 11, the state
reported 27 arrests on Oahu, 22 in Maui County, 29 on the Big Island and
60 on Kauai according to Krishna Jayaram, special assistant to the state
attorney general.

``This program, like everything with Covid, is something of a learning
process,'' said Mr. Jayaram. ``If we've become aware of someone who may
be violating the order, we conduct a pretty thorough investigation.''

Many investigations are spurred by concerned citizens, some of whom are
banding together on social media to root out quarantine violators.
\href{https://www.facebookcorewwwi.onion/groups/KapuBreakers}{Hawai'i
Quarantine Kapu Breakers}, who alerted the authorities to Ms.
Salamanca's transgressions, write that their purpose is, ``to bring
awareness to issues surrounding tourists and locals not adhering to
public safety standards during the pandemic.'' (Ms. Salamanca did not
respond to requests for comment.)
\href{https://www.washingtonpost.com/nation/2020/07/31/arrest-breaking-quarantine-covid-florida/}{Incidents
of community policing} have also been reported in Florida, South
Carolina and Kentucky, plus examples of
\href{https://www.nytimes3xbfgragh.onion/2020/08/07/travel/Canada-border-crossings-coronavirus.html}{particularly
vigilant enforcement} against Americans attempting to sneak into Canada.

Quarantine violations seem particularly egregious in Hawaii, where,
comparatively, the rules are unambiguous. Brad Johnson, who lives in
Maui, traveled to Kentucky in early June to be with his ailing father.
Mr. Johnson noted that, upon returning to Hawaii weeks later, it was
virtually impossible to exit the airport without filling out a form and
interacting with government officials, including the Hawaii National
Guard. He was required to provide his phone number and told to turn on
his location services. Still, he noticed that a number of people who had
been on his flight were pushing back both in transit and upon arrival.

``There were maybe a dozen people who were not wearing masks, and kind
of creating a ruckus,'' he said. ``I saw one of these women arguing with
officials, and refusing to turn on her phone's tracking. She was
visiting family; it was her right, they were violating it. The officials
made clear that her other option was to get back on the plane and go
back.''

Mr. Johnson was contacted twice by government officials to make sure
that he was abiding by the rules. Other states, in contrast, are making
enforcement a much lower priority. Furthermore, the rules of what is and
isn't allowed are less clear.

New York also requires visitors from its changing list of hot-spot
states to fill out a form sharing their location and contact
information. But the threat of enforcement remains lower than that of
Hawaii, not least because of the sheer number of visitors and the
disparate entry points. Since the first quarantine order was issued on
June 25, the state's health department has worked in partnership with
the Port Authority of New York and New Jersey to ensure that health
officials are stationed at airports and positioned by flights arriving
from applicable states to hand out and collect forms (those who refuse
are subject to a \$2,000 fine).

On Aug. 5, Mayor Bill de Blasio announced plans to set up checkpoints on
metro-area bridges and tunnels, plus similar setups at Penn Station and
the Port Authority Bus Terminal. But details of the plan
\href{https://www.nytimes3xbfgragh.onion/2020/08/05/nyregion/nyc-coronavirus-quarantine-checkpoints.html}{were
scarce, and somewhat confusing}. According to Amanda Kwan, a senior
public information officer at the Port Authority, ``The Port Authority
has been in contact with the city, and their effort will have no
operational impact to any of our facilities. All vehicle checkpoints are
being set up off of Port Authority property without Port Authority
personnel.''
\href{https://www.nbcnewyork.com/news/local/most-of-nycs-coronavirus-checkpoints-stops-have-taken-place-in-one-borough/2567143/}{As
of Aug. 14}, the majority of stops made were on Staten Island, while
only 36 vehicles entering Manhattan were stopped.

Bethany Mitchell returned home to New York on July 14 from San
Francisco, having left the city in March to stay with her boyfriend when
her work as a performer with the
\href{https://batterydance.org/}{Battery Dance} company dried up. She
wanted to try her best to follow the quarantine rules, but found some of
the asks unclear, or unfeasible.

``I hadn't been in my apartment for four months. I had to go to the
grocery store and pharmacy; to do things to function,'' she said. She
looked into delivery options for her groceries, but found that it was
prohibitively expensive. She also got tested shortly after arriving;
when she got her results a week later, she began seeing friends from a
distance in parks. ``My mind-set was, `At least I didn't bring it from
California!'''

According to Gov. Andrew M. Cuomo's office, a negative coronavirus test
is not a pass from quarantine, but this information can be hard to find
(it's on page six of
\href{https://coronavirus.health.ny.gov/system/files/documents/2020/07/nys-covid-travel-advisory-faq_0.pdf}{an
online F.A.Q.}). And for a number of states, including Vermont and
Alaska, a negative test can be a quarantine pass. In lieu of clear
guidelines and regular check-ins, Ms. Mitchell said that she made
choices that felt safe, financially viable and rational to her (she was
contacted by the state health department once, on day 13 of her
quarantine period). The governor's office could not share numbers of
those caught violating quarantine in New York.

The desire to do what's right is common, but defining that is difficult
as people receive mixed messages. \href{http://erezyoeli.com/}{Dr. Erez
Yoeli}, a research scientist at M.I.T.'s Sloan School of Management and
director of the Applied Cooperation Team, said that the lack of
consistency in the rules on the national, state and even city level
leads to ambiguity, which leaves room for plausible deniability.

``That makes it hard to make these rules social norms,'' he said.
``Without those social norms, there's no way that society can enforce
the rules informally.''

Still, ``soft'' enforcement remains central to many of these quarantine
plans. Dr. Allison Arwady, commissioner of Chicago's public health
department, said at
\href{https://www.facebookcorewwwi.onion/ChicagoMayorsOffice/videos/292903845473289}{a
recent}
\href{https://www.facebookcorewwwi.onion/ChicagoMayorsOffice/videos/292903845473289}{news
conference}, ``Our primary objective is about educating and informing
the public.'' Chicago is currently requiring a two-week quarantine from
19 states and Puerto Rico. Dr. Arwady noted that, while the city does
have the ability to issue fines of \$100 to \$500 per day, enforcement
is not her main goal. ``I have absolutely no intention of pulling cars
over that have out-of-state license plates, {[}or{]} developing watch
lists of people flying through the airports.''

This leaves some businesses in a challenging position. Hotels, for
instance, can receive guest reservations from a variety of sources ---
online platforms like Expedia, travel agents, direct booking on their
websites --- that provide varying degrees of data and information about
where a guest is coming from and if they would be required to
quarantine.

``When it comes from our website, we can get more information from
people,'' said Robert Baum, a partner at
\href{https://staybedderman.com/}{Bedderman Lodging}, a hotel group
based in Chicago. ``But even then we have a better shot of knowing where
they live, rather than where they were most recently.''

Mr. Baum said he has reached out to the city for advice on how to deal
with guests. In the meantime, Bedderman is emailing all guests coming to
its three hotels to inform them of the city's rules, and further
reminding them during the check-in process. Again, the hotels aren't
necessarily able to track whether or not guests are coming from hot spot
states; they aren't checking passports or parsing credit card
statements.

``Unlike other retail businesses, it's pretty much a guarantee that
people are coming in from other places,'' Mr. Baum said. ``It puts our
staff in a challenging position. When there's no real way for us to
police these quarantine rules, it's hard for us to reassure them.''

Chicago city officials did not share numbers of those who had been
caught violating quarantine orders.

The patchwork of quarantine rules, not dissimilar from differing norms
regarding mask usage and social distancing, seeds confusion. As Dr.
Yoeli said, ``confusion introduces this ability to act in your own
self-interest.'' But he still thinks it's worth trying to establish more
effective social norms.

``Everyone wants guidance; everyone wants to be OK,'' he said. ``I think
that everyone would find it something of a relief to have a clearer idea
of what to do.''

\emph{\textbf{Follow New York Times Travel}}
\emph{on}\href{https://www.instagram.com/nytimestravel/}{\emph{Instagram}}\emph{,}\href{https://twitter.com/nytimestravel}{\emph{Twitter}}
\emph{and}\href{https://www.facebookcorewwwi.onion/nytimestravel/}{\emph{Facebook}}\emph{.
And}\href{https://www.nytimes3xbfgragh.onion/newsletters/traveldispatch}{\emph{sign
up for our weekly Travel Dispatch newsletter}} \emph{to receive expert
tips on traveling smarter and inspiration for your next vacation.}

Advertisement

\protect\hyperlink{after-bottom}{Continue reading the main story}

\hypertarget{site-index}{%
\subsection{Site Index}\label{site-index}}

\hypertarget{site-information-navigation}{%
\subsection{Site Information
Navigation}\label{site-information-navigation}}

\begin{itemize}
\tightlist
\item
  \href{https://help.nytimes3xbfgragh.onion/hc/en-us/articles/115014792127-Copyright-notice}{©~2020~The
  New York Times Company}
\end{itemize}

\begin{itemize}
\tightlist
\item
  \href{https://www.nytco.com/}{NYTCo}
\item
  \href{https://help.nytimes3xbfgragh.onion/hc/en-us/articles/115015385887-Contact-Us}{Contact
  Us}
\item
  \href{https://www.nytco.com/careers/}{Work with us}
\item
  \href{https://nytmediakit.com/}{Advertise}
\item
  \href{http://www.tbrandstudio.com/}{T Brand Studio}
\item
  \href{https://www.nytimes3xbfgragh.onion/privacy/cookie-policy\#how-do-i-manage-trackers}{Your
  Ad Choices}
\item
  \href{https://www.nytimes3xbfgragh.onion/privacy}{Privacy}
\item
  \href{https://help.nytimes3xbfgragh.onion/hc/en-us/articles/115014893428-Terms-of-service}{Terms
  of Service}
\item
  \href{https://help.nytimes3xbfgragh.onion/hc/en-us/articles/115014893968-Terms-of-sale}{Terms
  of Sale}
\item
  \href{https://spiderbites.nytimes3xbfgragh.onion}{Site Map}
\item
  \href{https://help.nytimes3xbfgragh.onion/hc/en-us}{Help}
\item
  \href{https://www.nytimes3xbfgragh.onion/subscription?campaignId=37WXW}{Subscriptions}
\end{itemize}
