Sections

SEARCH

\protect\hyperlink{site-content}{Skip to
content}\protect\hyperlink{site-index}{Skip to site index}

\href{https://www.nytimes3xbfgragh.onion/section/world/canada}{Canada}

\href{https://myaccount.nytimes3xbfgragh.onion/auth/login?response_type=cookie\&client_id=vi}{}

\href{https://www.nytimes3xbfgragh.onion/section/todayspaper}{Today's
Paper}

\href{/section/world/canada}{Canada}\textbar{}Will It Be Safe to Return
to School?

\url{https://nyti.ms/34nurZF}

\begin{itemize}
\item
\item
\item
\item
\item
\end{itemize}

Advertisement

\protect\hyperlink{after-top}{Continue reading the main story}

Supported by

\protect\hyperlink{after-sponsor}{Continue reading the main story}

Canada Letter

\hypertarget{will-it-be-safe-to-return-to-school}{%
\section{Will It Be Safe to Return to
School?}\label{will-it-be-safe-to-return-to-school}}

Plans to ensure safe classes vary by province and school board or are
still in the works. Two experts offer their views on what's ahead.

\href{https://www.nytimes3xbfgragh.onion/by/ian-austen}{\includegraphics{https://static01.graylady3jvrrxbe.onion/images/2019/07/18/reader-center/author-ian-austen/author-ian-austen-thumbLarge.png}}

By \href{https://www.nytimes3xbfgragh.onion/by/ian-austen}{Ian Austen}

\begin{itemize}
\item
  Aug. 21, 2020
\item
  \begin{itemize}
  \item
  \item
  \item
  \item
  \item
  \end{itemize}
\end{itemize}

For many families, back-to-school season can be a fraught time. But this
year, the pandemic may take that tension to a whole new height.

\includegraphics{https://static01.graylady3jvrrxbe.onion/images/2020/08/21/world/21canadaletter-school/merlin_173451885_fad203a6-eda7-44e3-aee5-7454cd984866-articleLarge.jpg?quality=75\&auto=webp\&disable=upscale}

Among Canadian educators, public health officials and parents who
learned during lockdown that they aren't substitute teachers, there is a
consensus that students should return. But all of the provinces are
proposing varying approaches to ensure safety. Furthermore, within most
provinces, many school boards have their own ideas. And the plans in
many places are still works-in-progress, leaving parents unsure of what
to expect.

For a peek at what lies ahead, I spoke with education researchers this
week. Their broad conclusion: balancing education and infection controls
is likely to require constant adjustments.

``It's a fantasy to think this is going to go smoothly,'' said
\href{https://research.uottawa.ca/people/westheimer-joel}{Joel
Westheimer,} a professor of education at the University of Ottawa. ``All
in all, I think that there are compelling reasons for kids to go back.
But it needs to be done in the right way.''

Jason Ellis, a professor of education at the University of British
Columbia, said everyone in education has been focused on the return
since March. The ever shifting and incomplete plans, he said, are the
result of a lack of ideal options.

``You can't space out the kids dramatically in schools because you would
need to hire thousands of teachers, and they don't exist,'' Professor
Ellis said. ``You also would need much different school buildings than
you have. So they're betting on low community transmission. If it
remains low, as it is in most parts of Canada, then there shouldn't be
too many cases in the schools. We'll see.''

Professor Westheimer said parents should brace themselves for a cycle of
closings and reopenings until class size reduction becomes possible. And
among the provinces, he said Ontario's preparation is one of the most
wanting.

``Ontario has been a standout in providing almost no resources to get
this done,'' he said.

Image

Window repairs at a school in Harrow, Ontario, that will
reopen.Credit...Ian Austen/The New York Times

Both professors agree that teachers face a significant challenge.

``It's an enormous burden on them,'' Professor Ellis said. ``In
provinces where high schools are doing a blend of online and
face-to-face, the preparation will be just a nightmare.''

Professor Westheimer, however, did have some encouraging news about one
frequently raised concern.

``From the very beginning, people were kind of obsessed with whether
kids are going to fall behind or if they could catch up to their
peers,'' he said. ``Stop worrying. This is an unprecedented disaster and
it's worldwide. Who are they falling behind? Everyone's out of school.''

The Times is making education during the pandemic a special focus of its
reporting. That effort includes a new newsletter on education and the
coronavirus for which you can
\href{https://www.nytimes3xbfgragh.onion/newsletters/coronavirus-briefing}{sign
up here}.

In the coming weeks, I'll be contributing from Canada to that effort. So
if you have any particular concerns and observations about the return to
school or if there's something you think is going particularly well in
your province or community, \href{mailto:austen@NYTimes.com}{please drop
me an email.} I may use some of your comments in a future Canada Letter
as well, so please include your full name and where you live.

\begin{center}\rule{0.5\linewidth}{\linethickness}\end{center}

\hypertarget{the-drink}{%
\subsection{The Drink}\label{the-drink}}

A few weeks ago, the Canada Letter spoke with Paul Fairie, a researcher
at the University of Calgary who, improbably, revived that city's
interest in Cronk. (Those of you who missed the newsletter about the
long-forgotten 19th-century beverage, and its cryptic approach to
advertising,
\href{https://www.nytimes3xbfgragh.onion/2020/07/24/world/canada/cronk.html}{can
catch up here.})

Using a somewhat imprecise, industrial scale Cronk recipe, the Cold
Garden Beverage Company in Calgary brewed a batch. Two batches to be
precise. The first one, made
with\href{https://www.seriouseats.com/2017/02/what-is-blackstrap-molasses.html}{blackstrap}rather
than \href{https://www.crosbys.com/fancy-molasses/}{fancy-grade
molasses,} proved undrinkable.

After requests from several of you, I called Dr. Fairie this week, the
day after he knocked back his first glass of Cronk.

``It's amazing to drink it,'' Dr. Fairie said. Over all, Dr. he said,
Cronk has an herbal or spice flavor with noticeable flavors of molasses,
tea and root beer ``but not the kind of root beer you buy today.'' He
compared its aftertaste to that of prune juice.

In terms of style, he said that Cronk is more like an aperitif than
something to be consumed quickly and in large quantity.

Anyone else curious to try Cronk will have to wait --- or may be
disappointed. Blake Belding, the head brewer at the Cold Garden Beverage
Company, told me that 1,500
\href{https://www.canadianliving.com/food/entertaining/article/canadian-beer-bottles-do-you-know-your-history}{``stubbie''-style
glass bottles} sold out in 11 hours. A remaining 200 will be offered for
sale online shortly while the brewery debates whether to produce more.

But Dr. Fairie isn't sure that initial stampede means that Cronk will
again rise to its advertising claim of being ``The Drink.''

``It's very old-fashioned,'' he said.

\begin{center}\rule{0.5\linewidth}{\linethickness}\end{center}

\hypertarget{trans-canada}{%
\subsection{Trans Canada}\label{trans-canada}}

Image

Bill Morneau resigned as Canada's finance minister on
Monday.Credit...Justin Tang/The Canadian Press, via Associated Press

Perhaps it was inevitable in this year of disruptions, but Canadian
politics did not take its usual summer vacation this week. First, Bill
Morneau stepped down as finance minister amid speculation about the
reason for his departure.

Prime Minister Justin Trudeau replaced with him with Chrystia Freeland,
making her Canada's first female finance minister. Then, to round things
out, the prime minister shut down the Parliamentary session.

He explained the move was necessary to reset the government's agenda as
its focus turns to economic recovery. The opposition said that it was to
end their hearings into Mr. Trudeau's role in the WE Charity affair.

During the recent Black Lives Matter protests in the United States, many
police forces have been criticized for ``kettling,'' the practice of
encircling demonstrators and passers-by alike before making often
violent arrests. This week, the Toronto police force agreed to pay 16.5
million Canadian dollars to people it arrested, in some cases using
kettling, during the 2010 G20 meeting.

The Canadian Football League canceled its 2020 season this week after
several efforts to play an abbreviated schedule fell apart.

Carol Schram reports from Edmonton on efforts by the N.H.L. to give a
home game feel to matches played at its playoff hubs in that city and
Toronto.

\begin{center}\rule{0.5\linewidth}{\linethickness}\end{center}

\emph{A native of Windsor, Ontario, Ian Austen was educated in Toronto,
lives in Ottawa and has reported about Canada for The New York Times for
the past 16 years. Follow him on Twitter at @ianrausten.}

\begin{center}\rule{0.5\linewidth}{\linethickness}\end{center}

\hypertarget{how-are-we-doing}{%
\subsubsection{\texorpdfstring{\textbf{How are we
doing?}}{How are we doing?}}\label{how-are-we-doing}}

We're eager to have your thoughts about this newsletter and events in
Canada in general. Please send them to
\href{mailto:nytcanada@NYTimes.com?\%20subject=Canada\%20Letter\%20Newsletter\%20Feedback}{nytcanada@NYTimes.com}.

\hypertarget{like-this-email}{%
\subsubsection{\texorpdfstring{\textbf{Like this
email?}}{Like this email?}}\label{like-this-email}}

Forward it to your friends, and let them know they can sign up
\href{https://www.nytimes3xbfgragh.onion/newsletters/canada-letter?smid=nytemail\&smvar=canadaletter\&te=1\&nl=canada-today\&emc=edit_cnda_20190622}{here}.

Advertisement

\protect\hyperlink{after-bottom}{Continue reading the main story}

\hypertarget{site-index}{%
\subsection{Site Index}\label{site-index}}

\hypertarget{site-information-navigation}{%
\subsection{Site Information
Navigation}\label{site-information-navigation}}

\begin{itemize}
\tightlist
\item
  \href{https://help.nytimes3xbfgragh.onion/hc/en-us/articles/115014792127-Copyright-notice}{©~2020~The
  New York Times Company}
\end{itemize}

\begin{itemize}
\tightlist
\item
  \href{https://www.nytco.com/}{NYTCo}
\item
  \href{https://help.nytimes3xbfgragh.onion/hc/en-us/articles/115015385887-Contact-Us}{Contact
  Us}
\item
  \href{https://www.nytco.com/careers/}{Work with us}
\item
  \href{https://nytmediakit.com/}{Advertise}
\item
  \href{http://www.tbrandstudio.com/}{T Brand Studio}
\item
  \href{https://www.nytimes3xbfgragh.onion/privacy/cookie-policy\#how-do-i-manage-trackers}{Your
  Ad Choices}
\item
  \href{https://www.nytimes3xbfgragh.onion/privacy}{Privacy}
\item
  \href{https://help.nytimes3xbfgragh.onion/hc/en-us/articles/115014893428-Terms-of-service}{Terms
  of Service}
\item
  \href{https://help.nytimes3xbfgragh.onion/hc/en-us/articles/115014893968-Terms-of-sale}{Terms
  of Sale}
\item
  \href{https://spiderbites.nytimes3xbfgragh.onion}{Site Map}
\item
  \href{https://help.nytimes3xbfgragh.onion/hc/en-us}{Help}
\item
  \href{https://www.nytimes3xbfgragh.onion/subscription?campaignId=37WXW}{Subscriptions}
\end{itemize}
