Sections

SEARCH

\protect\hyperlink{site-content}{Skip to
content}\protect\hyperlink{site-index}{Skip to site index}

\href{https://www.nytimes3xbfgragh.onion/section/business/economy}{Economy}

\href{https://myaccount.nytimes3xbfgragh.onion/auth/login?response_type=cookie\&client_id=vi}{}

\href{https://www.nytimes3xbfgragh.onion/section/todayspaper}{Today's
Paper}

\href{/section/business/economy}{Economy}\textbar{}Mnuchin Paved Way for
Postal Service Shake-Up

\url{https://nyti.ms/2QfR7Tt}

\begin{itemize}
\item
\item
\item
\item
\item
\item
\end{itemize}

\begin{itemize}
\item
  \href{https://www.nytimes3xbfgragh.onion/interactive/2020/09/08/us/elections/results-new-hampshire-primary-elections.html?action=click\&pgtype=Article\&state=default\&region=TOP_BANNER\&context=storylines_menu}{New
  Hampshire Results}
\item
  \href{https://www.nytimes3xbfgragh.onion/live/2020/09/08/us/trump-vs-biden?action=click\&pgtype=Article\&state=default\&region=TOP_BANNER\&context=storylines_menu}{Election
  Updates}
\item
  \href{https://www.nytimes3xbfgragh.onion/interactive/2020/us/elections/election-states-biden-trump.html?action=click\&pgtype=Article\&state=default\&region=TOP_BANNER\&context=storylines_menu}{Paths
  to 270}
\item
  \href{https://www.nytimes3xbfgragh.onion/interactive/2020/08/31/us/politics/vote-by-mail-deadlines.html?action=click\&pgtype=Article\&state=default\&region=TOP_BANNER\&context=storylines_menu}{Voting
  by Mail}
\item
  \href{https://www.nytimes3xbfgragh.onion/interactive/2019/us/elections/2020-presidential-election-calendar.html?action=click\&pgtype=Article\&state=default\&region=TOP_BANNER\&context=storylines_menu}{Key
  Dates}
\item
  \href{https://www.nytimes3xbfgragh.onion/newsletters/politics?action=click\&pgtype=Article\&state=default\&region=TOP_BANNER\&context=storylines_menu}{Politics
  Newsletter}
\end{itemize}

Advertisement

\protect\hyperlink{after-top}{Continue reading the main story}

Supported by

\protect\hyperlink{after-sponsor}{Continue reading the main story}

\hypertarget{mnuchin-paved-way-for-postal-service-shake-up}{%
\section{Mnuchin Paved Way for Postal Service
Shake-Up}\label{mnuchin-paved-way-for-postal-service-shake-up}}

At President Trump's behest, the Treasury Secretary sought out
appointees who would restructure the United States Postal Service.

\includegraphics{https://static01.graylady3jvrrxbe.onion/images/2020/08/23/us/politics/23dc-postal-print1/00dc-postal-articleLarge.jpg?quality=75\&auto=webp\&disable=upscale}

By \href{https://www.nytimes3xbfgragh.onion/by/kenneth-p-vogel}{Kenneth
P. Vogel},
\href{https://www.nytimes3xbfgragh.onion/by/jessica-silver-greenberg}{Jessica
Silver-Greenberg},
\href{https://www.nytimes3xbfgragh.onion/by/alan-rappeport}{Alan
Rappeport} and
\href{https://www.nytimes3xbfgragh.onion/by/hailey-fuchs}{Hailey Fuchs}

\begin{itemize}
\item
  Aug. 22, 2020
\item
  \begin{itemize}
  \item
  \item
  \item
  \item
  \item
  \item
  \end{itemize}
\end{itemize}

WASHINGTON --- In early February, Treasury Secretary Steven Mnuchin
invited two Republican members of the Postal Service's board of
governors to his office to update him on a matter in which he had taken
a particular interest --- the search for a new
\href{https://www.nytimes3xbfgragh.onion/2020/08/24/us/politics/postal-service-dejoy-testimony.html}{postmaster
general}.

Mr. Mnuchin had made clear before the meeting that he wanted the
governors to find someone who would push through the kind of
cost-cutting and price increases that President Trump had publicly
called for and that Treasury had recommended in a
\href{https://home.treasury.gov/news/press-releases/sm566}{December 2018
report} as a way to stem years of multibillion-dollar losses.

It was an unusual meeting at an unusual moment.

Since 1970, the Postal Service had been an independent agency, walled
off from political influence. The postmaster general is not appointed by
the president and is not a cabinet member. Instead, the postal chief is
picked by a board of governors, with seats reserved for members of both
parties, who are nominated by the president and confirmed by the Senate
for seven-year terms.

Now, not only was the Trump administration, through Mr. Mnuchin,
involving itself in the process for selecting the next postmaster
general, but the two Democratic governors who were then serving on the
board were not invited to the Treasury meeting. Since the meeting did
not include a quorum of board members, it was not subject to sunshine
laws that apply to official board meetings and there is no formal Postal
Service record or minutes of what was discussed.

Nearly six months later, that meeting, along with other interactions
between Mr. Mnuchin and the postal board, has taken on heightened
significance as the Trump administration confronts allegations it sought
to politicize the Postal Service and hinder its ability to handle a
surge in mail-in ballots in November's election. In interviews,
documents and congressional testimony, Mr. Mnuchin emerges as a key
player in selecting the board members who hired the Trump megadonor now
leading the Postal Service and in pushing the agenda that he has
pursued.

Mr. Trump's animus toward the agency dates to
\href{https://twitter.com/realdonaldtrump/status/299212012121112578}{at
least 2013}, but his criticism of its finances escalated once he took
office and found new focus in late 2017, when he
\href{https://twitter.com/realDonaldTrump/status/946728546633953285}{first
bashed it} for essentially subsidizing Amazon,
\href{https://www.nytimes3xbfgragh.onion/2018/03/29/us/politics/trump-amazon-post-office-fact-check.html}{another
target of his ire}. Amazon's founder and chief executive, Jeff Bezos,
owns The Washington Post, whose coverage has often angered Mr. Trump.

``This Post Office scam must stop. Amazon must pay real costs (and
taxes) now!'' the president
\href{https://twitter.com/realdonaldtrump/status/980065419632566272}{wrote
on Twitter} on
\href{https://twitter.com/realDonaldTrump/status/980063581592047617}{March
31, 2018}, one of several such attacks over the years.

Twelve days later, he issued an
\href{https://www.whitehouse.gov/presidential-actions/executive-order-task-force-united-states-postal-system/}{executive
order} putting Mr. Mnuchin in charge of a postal reform task force. But
it was not until earlier this year that the administration found a way
to enforce its postal agenda --- one that has now collided with the
pandemic and the approaching election.

A few weeks after the February meeting with Mr. Mnuchin, one of the
attendees, Robert M. Duncan, the chairman of the board of governors, who
was appointed by Mr. Trump in 2017, threw a new name for postmaster
general into the mix: Louis DeJoy.

Mr. DeJoy, a longtime logistics executive, was known for his
hard-charging leadership style and his ability to convert
disorganization into efficiency, as well his generous donations to the
Republican Party, including to Mr. Trump. In October 2017, Mr. DeJoy had
\href{https://www.newsobserver.com/article176825156.html}{hosted a
fund-raiser} for the president's re-election campaign at his North
Carolina home.

His résumé was far different than recent postmasters general, most of
whom had risen through the Postal Service ranks. Megan J. Brennan, who
had announced in October 2019 her intention to retire as postmaster
general at the end of January, began her career as a letter carrier in
Pennsylvania.

Mr. DeJoy, who ran New Breed Logistics before selling it to XPO
Logistics in 2014, would be coming from the private sector to assume
control of a highly unionized, sprawling bureaucracy with more than half
a million employees. His companies had experience working with the
Postal Service, moving bulk shipments of packages from fulfillment
centers and ferrying them to local Postal Service centers. But both
companies had fewer than 10,000 employees, none of them unionized, and
he had never worked in the public sector.

The companies were also the subject of a litany of complaints from
workers, including more than a dozen lawsuits accusing managers --- but
not Mr. DeJoy personally --- of presiding over a hostile environment
rife with sexual harassment and racial discrimination and where workers
were fired for getting sick or injured.

\includegraphics{https://static01.graylady3jvrrxbe.onion/images/2020/08/21/us/politics/22dc-postal2/merlin_175343967_ce1ac883-7214-4cf7-a5c3-3d56c2852160-articleLarge.jpg?quality=75\&auto=webp\&disable=upscale}

The board's vice chairman at the time, David C. Williams,
\href{https://www.nytimes3xbfgragh.onion/2020/08/20/us/politics/former-postal-governor-tells-congress-mnuchin-politicized-postal-service.html}{raised
concerns} about Mr. DeJoy's candidacy and Mr. Mnuchin's involvement,
telling lawmakers during sworn testimony this week that he ``didn't
strike me as a serious candidate.'' Mr. Williams, a Democratic
appointee, resigned before the vote as it became clear that Mr. DeJoy
would be the pick.

Three months after the meeting in Mr. Mnuchin's office, the board of
governors announced Mr. DeJoy's selection as the nation's 75th
postmaster general. Within weeks, he began carrying out changes,
including cuts to overtime and limiting mail delivery trips. He
curtailed postal hours and mandated that carriers must adhere to a rigid
schedule. A July memo from the Postal Service warned that the changes
might temporarily result in ``mail left behind or mail on the workroom
floor or docks.''

The measures matched up with recommendations in the task force report,
which blamed the Postal Service for losing billions because of waste,
inefficiency and a failure to respond to declining mail volumes.

But the rapid-fire moves just months before the November election
concerned Postal Service insiders, who said that, since at least the
Obama administration, the agency had generally sought to avoid
significant changes within two or three months of a general election.

Soon, mail was piling up at post offices, veterans were not receiving
their medications, bills were arriving late and questions began
surfacing about the ability of the Postal Service to handle what is
expected to be a record number of mail-in ballots this November because
of the pandemic.

Amid an outcry from lawmakers, civil rights groups and state officials,
Mr. DeJoy
\href{https://www.nytimes3xbfgragh.onion/2020/08/18/us/politics/postal-service-suspends-changes.html}{suspended
many of the changes} on Tuesday, including some that had been underway
before he took the helm of the Postal Service. Yet he made clear during
a Senate hearing on Friday that he planned to move ahead with
``dramatic'' measures after the election, including raising prices and
limiting overtime.

Postal Service employees and union officials say
\href{https://www.nytimes3xbfgragh.onion/2020/08/19/business/economy/postal-service-changes-dejoy.html}{significant
damage has already been done}. Hundreds of mail-sorting machines have
been removed, and the day-to-day changes have caused confusion and
delays among drivers, carriers and other workers.

In
\href{https://www.nytimes3xbfgragh.onion/2020/08/21/us/politics/dejoy-postal-service-senate-hearing.html}{his
Senate testimony on Friday}, Mr. DeJoy chalked that up to growing pains
as the organization tries to get leaner. ``We all feel, you know, bad''
he told lawmakers upset about mail delays affecting their constituents.

Image

A mail carrier in Fairfax, Va. Mr. DeJoy carried out cuts to overtime
and limited mail delivery tripsCredit...Andrew Harrer/Bloomberg

Mr. DeJoy stressed that the changes had nothing to do with the election
and noted that the removal of mailboxes and sorting machines had begun
before his tenure. He said that he had been unaware of the equipment
removal until it became a source of scrutiny. ``This has been going on
in every election year, and every year, for that matter,'' he said,
adding that he had ``no idea'' that it was happening.

Kevin Tabarus, a local president of the mail handlers' union, questioned
Mr. DeJoy's qualifications and the selection process. ``The guy has been
to way more Republican fund-raisers than post offices,'' he said,
emphasizing just how delayed some of the mail has been under Mr. DeJoy's
watch.

After the task force issued its report, Mr. Mnuchin sought to ensure
that the president nominated postal governors who would enact Treasury's
recommendations and would pick a like-minded postmaster general to carry
them out. Mr. Mnuchin referred prospective candidates to the White
House, according to a Treasury spokeswoman, and then regularly asked his
staff for updates, a former Treasury official involved in the process
said.

Over the last two years, Mr. Mnuchin met privately on multiple occasions
about postal matters with Mr. Duncan, a former chairman of the
Republican National Committee who was confirmed by the Senate as a
postal board member in August 2018, according to people familiar with
the meetings.

Mr. Mnuchin also arranged a meeting with John M. Barger, a California
lawyer and financial investment adviser who was recommended to the
Treasury secretary by a mutual associate who knew of Mr. Barger's work
as chairman of the board of the Los Angeles County pension fund. After a
meeting in Washington, Mr. Mnuchin recommended that Mr. Trump appoint
Mr. Barger to the board of governors.

Mr. Barger was confirmed by the Senate last summer, and was tapped to
lead the committee to select a new postmaster general. He attended the
February meeting in Mr. Mnuchin's office with Mr. Duncan.

S. David Fineman, a former member and chairman of the Postal Service's
board, called Mr. Mnuchin's close involvement in the affairs of the
Postal Service ``absolutely unprecedented.''

During his tenure in the Clinton and George W. Bush administrations, he
said the board had minimal interaction with the administrations, and
``certainly no communication regarding the hiring of the postmaster
general.''

The board hired two search firms to assist in the selection process by
conducting a nationwide search. One of them, Russell Reynolds
Associates, compiled a database of prospective candidates and provided a
subset of dozens of names it deemed most promising to the board.

Mr. DeJoy's name was not among those initially provided, according to
people familiar with the process. But Mr. Duncan raised Mr. DeJoy's name
during a discussion among board members about other prospective
candidates. Mr. Duncan, who has been involved in a super PAC that
supports Mr. Trump's re-election, had met Mr. DeJoy through Mr. DeJoy's
wife, Aldona Wos. Both had been appointed by Mr. Trump to help lead the
President's Commission on White House Fellowships.

After the board received a readout from Russell Reynolds, which
indicated that Mr. DeJoy was already in the firm's database and was
qualified, Mr. Barger went to lunch with Mr. DeJoy to assess his
interest in, and suitability for, the post.

Mr. Barger was impressed, and reported his impressions to the full board
the same day. ``It's uncommon to find somebody from outside the Postal
Service who also has a history of success working with the Postal
Service,'' Mr. Barger said in an interview on Friday.

Nearly two months after the meeting in Mr. Mnuchin's office about the
search process, Mr. Duncan wrote
\href{https://s3.amazonaws.com/storage.citizensforethics.org/wp-content/uploads/2020/08/19160346/COVID-19-VIP-Duncan-USPS-Board-Chairman-on.pdf}{a
follow-up letter} to the Treasury secretary indicating that the board
had ``narrowed the search to a small number of finalists, each of whom
would serve the country well.''

Mr. Barger rejected suggestions that Mr. Mnuchin was playing politics
with the Postal Service, noting that Democratic governors had met with
top Treasury officials under Mr. Mnuchin. Mr. Mnuchin was not involved
in the board's decision to select Mr. DeJoy, Mr. Barger said, adding
that the board ``viewed Secretary Mnuchin as a stakeholder who was doing
his job in having an interest in how our process was moving forward, but
certainly nothing more than that.''

A Treasury spokeswoman acknowledged that Mr. Mnuchin urged the board to
act expeditiously to find a new postmaster general when Ms. Brennan
announced her retirement. The spokeswoman noted that Postal Service
board members are part of the executive branch and that, as the head of
the postal reform task force, Mr. Mnuchin had a right to be involved in
the selection process and interview board candidates.

On Friday, the Treasury Department released a
``\href{https://home.treasury.gov/system/files/136/US-Treasury-Role-as-Lender-to-US-Postal-Service.pdf}{fact
sheet}'' to rebut allegations that have been leveled at Mr. Mnuchin in
recent days, including that he
\href{https://www.nytimes3xbfgragh.onion/2020/08/20/us/politics/former-postal-governor-tells-congress-mnuchin-politicized-postal-service.html}{politicized
the Postal Service} and used the department's role as its lender to
leverage influence over postal policies.

The department denied that Mr. Mnuchin had a role in selecting Mr. DeJoy
and said his contact with board members was part of his ``normal course
of fulfilling his obligations'' as chairman of the Federal Financing
Bank, which lends money to the Postal Service and other federal
agencies, and the presidential postal task force that produced the 2018
report.

Image

Treasury Secretary Steven Mnuchin had made clear to board members that
he wanted a postmaster general who would push through the kind of
measures that President Trump had publicly called for.Credit...Anna
Moneymaker for The New York Times

Mr. DeJoy, 63, had transformed his father's Long Island trucking company
from a small shop with 10 employees into a national logistics and
supply-chain provider that won lucrative contracts with Boeing, Verizon
and the Postal Service. By 2014, around the time that he sold it XPO for
\$615 million, the company had about 7,000 employees.

That kind of growth came at a cost. In the logistics industry, speed is
supreme. New Breed Logistics competed with Amazon in the hustle to
deliver products to people's homes as fast as possible. In pursuit of
that goal, New Breed Logistics pushed their workers to extremes,
\href{https://www.nytimes3xbfgragh.onion/interactive/2018/10/21/business/pregnancy-discrimination-miscarriages.html}{according
to a New York Times investigation} published in 2018.

The company's warehouse in Memphis offered a glimpse into the grueling
culture that played out under Mr. DeJoy's leadership. Inside the
warehouse, hundreds of workers, many of them women, lifted and dragged
boxes that could weigh up to 45 pounds. To save money, there was no
air-conditioning, even in the middle of southern summers, causing
temperatures to rise past 100 degrees.

Employees at the warehouse were disciplined using a ``point system,'' in
which they could be fired once they racked up 10 points. Asking for a
break to go to the doctor could earn you a point, as could taking too
long on a break.

In 2013, New Breed was ordered to pay \$1.5 million after the Equal
Employment Opportunity Commission sued, accusing the company of
retaliating against three female employees who said they had been
sexually harassed.

There was no reprieve for women who were pregnant or sick, according to
interviews and lawsuits. Those who asked for lighter work loads were
often denied.

Image

Chasisty Bee suffered a miscarriage while working at the warehouse in
Memphis.Credit...Miranda Barnes for The New York Times

Image

Tasha Murell also lost a pregnancy while working at the warehouse in
Memphis. ``It was like a sweatshop,'' she said.Credit...Miranda Barnes
for The New York Times

That included women like Tasha Murrell, who pleaded with her supervisor
in 2014 to leave early because she was pregnant and lifting had become
unbearable. Instead of a reprieve, her supervisor told her to get an
abortion, according to a discrimination complaint that she later filed
with the employment commission. Shortly after asking for a break, Ms.
Murrell woke up in a pool of her own blood. She rushed to the hospital,
where she learned that she had miscarried.

``It was like a sweatshop,'' Ms. Murrell said. ``All they cared about
was their profits.''

XPO declined to comment for this article. A spokeswoman previously told
The Times the company was ``surprised by the allegations of conduct that
either predate XPO's acquisition of the Memphis facility or weren't
reported to management after we acquired it in 2014.''

During his testimony on Friday, Mr. DeJoy said he spoke to Mr. Mnuchin
during discussions about the terms of a loan that the Postal Service was
receiving as part of the virus economic relief legislation. He said that
they did not discuss his operational plans in ``grave detail.''

``I told him I was working on a plan,'' Mr. DeJoy said, explaining that
he had mentioned improving service and gaining ``cost efficiencies.''

Mr. DeJoy and Mr. Duncan are scheduled to testify on Monday before the
Democratic-controlled House Oversight and Reform Committee, whose
members
\href{https://www.documentcloud.org/documents/7042332-Concerns-About-DeJoy-Were-Ignored-Democrats.html}{have
signaled interest in Mr. DeJoy's hiring}, the changes he enacted and Mr.
Mnuchin's involvement in the Postal Service.

Image

Letters of protest outside Mr. DeJoy's home in
Washington.~Credit...Michael A. Mccoy/Getty Images

Kenneth Vogel, Alan Rappeport and Hailey Fuchs reported from Washington,
and Jessica Silver-Greenberg from New York. Kitty Bennett contributed
research.

\hypertarget{our-2020-election-guide}{%
\section{Our 2020 Election Guide}\label{our-2020-election-guide}}

Updated ~Sept. 8, 2020

\begin{center}\rule{0.5\linewidth}{\linethickness}\end{center}

\begin{itemize}
\item ~
  \hypertarget{the-latest}{%
  \subsection{The Latest}\label{the-latest}}

  \begin{itemize}
  \item
    President Trump and his party are using a playbook that aims to
    alarm people about crime in their backyards. It didn't work in 2018,
    but
    \href{https://www.nytimes3xbfgragh.onion/2020/09/08/us/politics/trump-republicans-fear-strategy.html?action=click\&pgtype=Article\&state=default\&region=BELOW_MAIN_CONTENT\&context=storylines_guide}{both
    parties think it could resonate more this year}.
  \end{itemize}
\item ~
  \hypertarget{how-to-win-270}{%
  \subsection{How to Win 270}\label{how-to-win-270}}

  \begin{itemize}
  \item
    Joe Biden and Donald Trump need 270 electoral votes to reach the
    White House. Try building
    \href{https://www.nytimes3xbfgragh.onion/interactive/2020/us/elections/election-states-biden-trump.html?action=click\&pgtype=Article\&state=default\&region=BELOW_MAIN_CONTENT\&context=storylines_guide}{your
    own coalition of battleground states}~to see potential outcomes.
  \end{itemize}
\item ~
  \hypertarget{voting-by-mail}{%
  \subsection{Voting by Mail}\label{voting-by-mail}}

  \begin{itemize}
  \item
    Will you have enough time to vote by mail in your state? Yes, but
    it's risky to procrastinate.
    \href{https://www.nytimes3xbfgragh.onion/interactive/2020/08/31/us/politics/vote-by-mail-deadlines.html?action=click\&pgtype=Article\&state=default\&region=BELOW_MAIN_CONTENT\&context=storylines_guide}{Check
    your state's deadline.}
  \item
    \href{https://www.nytimes3xbfgragh.onion/interactive/2020/us/elections/joe-biden.html?action=click\&pgtype=Article\&state=default\&region=BELOW_MAIN_CONTENT\&context=storylines_guide}{}

    \hypertarget{joe-biden}{%
    \section{Joe Biden}\label{joe-biden}}

    \hypertarget{democrat}{%
    \subsection{Democrat}\label{democrat}}

    \href{https://www.nytimes3xbfgragh.onion/interactive/2020/us/elections/donald-trump.html?action=click\&pgtype=Article\&state=default\&region=BELOW_MAIN_CONTENT\&context=storylines_guide}{}

    \hypertarget{donald-trump}{%
    \section{Donald Trump}\label{donald-trump}}

    \hypertarget{republican}{%
    \subsection{Republican}\label{republican}}
  \end{itemize}
\item
  \hypertarget{keep-up-with-our-coverage}{%
  \subsection{Keep Up With Our
  Coverage}\label{keep-up-with-our-coverage}}

  \begin{itemize}
  \item
    Get an
    \href{https://www.nytimes3xbfgragh.onion/newsletters/politics?action=click\&pgtype=Article\&state=default\&region=BELOW_MAIN_CONTENT\&context=storylines_guide}{email}~recapping
    the day's news
  \item
    Download our mobile app on
    \href{https://apps.apple.com/us/app/nytimes/id284862083?ls=1\&mat_click_id=5c79ae7455014fd1bd66b5610c05b8f2-20191112-16948\&referrer=mat_click_id\%3D5c79ae7455014fd1bd66b5610c05b8f2-20191112-16948\%26link_click_id\%3D722930677036718082}{iOS}~and
    \href{http://a.localytics.com/android?id=com.nytimes.android\&referrer=utm_source\%3Dother_nyt_mobile_web\%26utm_medium\%3DWeb\%2520page\%26utm_term\%3DGeneral\%2520Mobile\%2520Page\%26utm_campaign\%3DNYT\%2520Mobile\%2520General\%2520Page}{Android}~and
    turn on Breaking News and Politics alerts
  \end{itemize}
\end{itemize}

Advertisement

\protect\hyperlink{after-bottom}{Continue reading the main story}

\hypertarget{site-index}{%
\subsection{Site Index}\label{site-index}}

\hypertarget{site-information-navigation}{%
\subsection{Site Information
Navigation}\label{site-information-navigation}}

\begin{itemize}
\tightlist
\item
  \href{https://help.nytimes3xbfgragh.onion/hc/en-us/articles/115014792127-Copyright-notice}{©~2020~The
  New York Times Company}
\end{itemize}

\begin{itemize}
\tightlist
\item
  \href{https://www.nytco.com/}{NYTCo}
\item
  \href{https://help.nytimes3xbfgragh.onion/hc/en-us/articles/115015385887-Contact-Us}{Contact
  Us}
\item
  \href{https://www.nytco.com/careers/}{Work with us}
\item
  \href{https://nytmediakit.com/}{Advertise}
\item
  \href{http://www.tbrandstudio.com/}{T Brand Studio}
\item
  \href{https://www.nytimes3xbfgragh.onion/privacy/cookie-policy\#how-do-i-manage-trackers}{Your
  Ad Choices}
\item
  \href{https://www.nytimes3xbfgragh.onion/privacy}{Privacy}
\item
  \href{https://help.nytimes3xbfgragh.onion/hc/en-us/articles/115014893428-Terms-of-service}{Terms
  of Service}
\item
  \href{https://help.nytimes3xbfgragh.onion/hc/en-us/articles/115014893968-Terms-of-sale}{Terms
  of Sale}
\item
  \href{https://spiderbites.nytimes3xbfgragh.onion}{Site Map}
\item
  \href{https://help.nytimes3xbfgragh.onion/hc/en-us}{Help}
\item
  \href{https://www.nytimes3xbfgragh.onion/subscription?campaignId=37WXW}{Subscriptions}
\end{itemize}
