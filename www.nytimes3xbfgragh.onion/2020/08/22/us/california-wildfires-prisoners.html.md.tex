Sections

SEARCH

\protect\hyperlink{site-content}{Skip to
content}\protect\hyperlink{site-index}{Skip to site index}

\href{/section/us}{U.S.}\textbar{}Coronavirus Limits California's
Efforts to Fight Fires With Prison Labor

\url{https://nyti.ms/3glT9wb}

\begin{itemize}
\item
\item
\item
\item
\item
\item
\end{itemize}

\hypertarget{wildfires-in-the-west}{%
\subsubsection{\texorpdfstring{\href{https://www.nytimes3xbfgragh.onion/spotlight/california-wildfires?name=styln-california-wildfires\&region=TOP_BANNER\&block=storyline_menu_recirc\&action=click\&pgtype=Article\&impression_id=bf7937e0-f52c-11ea-84f0-cdc2bcc2aabe\&variant=undefined}{Wildfires
in the West}}{Wildfires in the West}}\label{wildfires-in-the-west}}

\begin{itemize}
\tightlist
\item
  live\href{https://www.nytimes3xbfgragh.onion/2020/09/12/us/wildfires-live-updates.html?name=styln-california-wildfires\&region=TOP_BANNER\&block=storyline_menu_recirc\&action=click\&pgtype=Article\&impression_id=bf7937e1-f52c-11ea-84f0-cdc2bcc2aabe\&variant=undefined}{Fires
  Updates}
\item
  \href{https://www.nytimes3xbfgragh.onion/interactive/2020/us/fires-map-tracker.html?name=styln-california-wildfires\&region=TOP_BANNER\&block=storyline_menu_recirc\&action=click\&pgtype=Article\&impression_id=bf7937e2-f52c-11ea-84f0-cdc2bcc2aabe\&variant=undefined}{Maps
  of the Fires}
\item
  \href{https://www.nytimes3xbfgragh.onion/article/wildfires-photos-california-oregon-washington-state.html?name=styln-california-wildfires\&region=TOP_BANNER\&block=storyline_menu_recirc\&action=click\&pgtype=Article\&impression_id=bf795ef0-f52c-11ea-84f0-cdc2bcc2aabe\&variant=undefined}{Photos}
\item
  \href{https://www.nytimes3xbfgragh.onion/2020/09/10/us/climate-change-california-wildfires.html?name=styln-california-wildfires\&region=TOP_BANNER\&block=storyline_menu_recirc\&action=click\&pgtype=Article\&impression_id=bf795ef1-f52c-11ea-84f0-cdc2bcc2aabe\&variant=undefined}{A
  Climate Reckoning}
\item
  \href{https://www.nytimes3xbfgragh.onion/article/wildfires-california-oregon-washington.html?name=styln-california-wildfires\&region=TOP_BANNER\&block=storyline_menu_recirc\&action=click\&pgtype=Article\&impression_id=bf795ef2-f52c-11ea-84f0-cdc2bcc2aabe\&variant=undefined}{Answers
  to Your Questions}
\item
  \href{https://www.nytimes3xbfgragh.onion/2020/09/09/us/california-wildfires.html?name=styln-california-wildfires\&region=TOP_BANNER\&block=storyline_menu_recirc\&action=click\&pgtype=Article\&impression_id=bf795ef3-f52c-11ea-84f0-cdc2bcc2aabe\&variant=undefined}{Newsletter}
\end{itemize}

\includegraphics{https://static01.graylady3jvrrxbe.onion/images/2020/09/22/us/22fire-prisoners02alt/merlin_176092530_cbca8771-ef50-4054-a7f9-e6dd1df72481-articleLarge.jpg?quality=75\&auto=webp\&disable=upscale}

\hypertarget{coronavirus-limits-californias-efforts-to-fight-fires-with-prison-labor}{%
\section{Coronavirus Limits California's Efforts to Fight Fires With
Prison
Labor}\label{coronavirus-limits-californias-efforts-to-fight-fires-with-prison-labor}}

Early releases of prisoners to protect them from the virus have depleted
the ranks of an inmate firefighting program that some say should be
abolished anyway.

Inmate firefighters worked to contain a fire in Healdsburg, Calif., on
Saturday.Credit...Max Whittaker for The New York Times

Supported by

\protect\hyperlink{after-sponsor}{Continue reading the main story}

\href{https://www.nytimes3xbfgragh.onion/by/thomas-fuller}{\includegraphics{https://static01.graylady3jvrrxbe.onion/images/2018/06/12/multimedia/author-thomas-fuller/author-thomas-fuller-thumbLarge.png}}

By \href{https://www.nytimes3xbfgragh.onion/by/thomas-fuller}{Thomas
Fuller}

\begin{itemize}
\item
  Published Aug. 22, 2020Updated Aug. 24, 2020
\item
  \begin{itemize}
  \item
  \item
  \item
  \item
  \item
  \item
  \end{itemize}
\end{itemize}

VACAVILLE, Calif. --- They charge into fire zones with 60-pound packs
and three-foot chain saws, felling trees and hacking through brush to
make wide paths of dirt around anything worth protecting. Bright orange
uniforms set them apart from other firefighters --- and identify them as
inmates of California's state prisons.

``It's the hardest thing I've ever done in my life,'' said Ricardo
Martin, who became an inmate firefighter while serving a seven-year
sentence for driving while intoxicated and injuring another motorist in
a crash. ``But we took special pride in being able to actually save
people's homes,'' Mr. Martin said. ``Everybody talked about that and how
good they felt about it.''

Prisoners have helped California fight fires for decades, playing a
crucial role in containing the blazes striking the state with more
frequency and ferocity in recent years.

\emph{{[}Sign up}
\href{https://www.nytimes3xbfgragh.onion/newsletters/california-today}{\emph{for
California Today}}\emph{, our daily newsletter from the Golden
State.{]}}

This past week, though, Mr. Martin and hundreds of other inmate
firefighters were absent from the fire lines. They had already gone
home, part of an early release program initiated by Gov. Gavin Newsom to
protect them from the coronavirus.

That has highlighted the state's dependence on prisoners in its
firefighting force and complicated its battle against almost 600 fires,
many which continued burning across Northern California on Saturday.
Experts worry that dry thunderstorms forecast to begin on Sunday could
wreak more havoc, further stretching the resources needed to fight what
are now the second- and third-largest fires in modern state history.

To critics the prison program is a cheap and exploitative salve, one
that should be replaced with proper public investment in firefighting;
to others it is an essential part of the state's response to what has
become an annual wildfire crisis. Some have complained that participants
were released just when the state needed them most.

``The inmates should have been put on the fire lines, fighting fires,''
said Mike Hampton, a former corrections officer who worked for decades
at an inmate fire camp. ``How do you justify releasing all these inmates
in prime fire season with all these fires going on?''

Mr. Newsom's answer is that prisoners faced another threat. Across the
United States there have been 112,436 infections of inmates and
correctional officers and 825 have been killed by the virus, according
to a New York Times database. In four of the six prisons that train
incarcerated firefighters, there have been more than 200 infections each
among inmates and staff members, according to The Times.

The virus has also affected non-inmate firefighters. About 80 are
currently in quarantine because of potential exposure to the
coronavirus, according to the union representing firefighters.

At Delta Camp, an inmate firefighter facility outside Vacaville, an
hour's drive northeast of San Francisco, the number of incarcerated
firefighters is down to 55, well below the camp's capacity of 132. Over
all, the state has the capacity to train and house about 3,400 inmate
firefighters. Only 1,306 inmates are currently deployed.

Men like Mr. Martin, who was released on Aug. 11, say they are grateful
to be back home.

The state's main firefighting agency, Cal Fire, says it is overwhelmed
by the size and complexity of the fires in Northern California, which by
Saturday night had burned through nearly one million acres, forcing more
than 119,000 people to evacuate and leaving at least five people dead.

Cal Fire, which has deployed 13,700 firefighters, is pleading for more
personnel, especially the crews that create the so-called hand lines,
the clearings crucial to stopping and slowing down wildfires. Mr. Newsom
has requested more firefighters from as far away as the East Coast and
Australia.

``Inmate fire crews are absolutely imperative to our ability to create
hand line and do arduous work on our fires,'' Brice Bennett, a spokesman
for Cal Fire, said. ``They are a tremendous resource.''

The coronavirus has exposed countless examples of inequality across the
nation, has devastated state budgets, and has left tens of thousands of
families bereft. The debate over California's inmate firefighters shows
how the pandemic's consequences have reached deep into unexpected
corners of society. In California it has been the difference between
having the manpower to save homes from wildfires --- or not.

\includegraphics{https://static01.graylady3jvrrxbe.onion/images/2020/09/22/us/22fire-prisoners01alt/merlin_176092656_f9e854da-e3ca-4a23-8f18-3a5a48b6b17d-articleLarge.jpg?quality=75\&auto=webp\&disable=upscale}

The California prisons department estimates that its Conservation Camp
Program, which includes the inmate firefighters, saves California
taxpayers tens of millions of dollars a year. Hiring firefighters to
replace them, especially given the difficult work involved, would
challenge a state already strapped for cash.

The larger debate in California is whether the state, which has the
largest inmate firefighter program in the country, should be employing
prisoners to fight fires in the first place. Incarcerated firefighters
in California are paid \$1 an hour when they are on the front lines,
leading some to describe it as slave labor. They work in treacherous
conditions, with six inmate firefighters dying over the past three and a
half decades, including one from the state's
\href{https://www.nytimes3xbfgragh.onion/2017/08/31/magazine/the-incarcerated-women-who-fight-californias-wildfires.html}{female
contingent of incarcerated firefighters}.

Already there are plans to shrink the program. Mr. Newsom's budget,
passed over the summer, calls for closing eight inmate fire camps, which
the governor's office estimates will save \$7.4 million.

\href{https://www.nytimes3xbfgragh.onion/spotlight/california-wildfires}{Wildfires
in the West ›}

\hypertarget{live-updates}{%
\subsection{\texorpdfstring{\href{https://www.nytimes3xbfgragh.onion/2020/09/12/us/wildfires-live-updates.html}{Live
Updates}}{Live Updates}}\label{live-updates}}

Updated~

Sept. 12, 2020, 2:53 p.m. ET

\begin{itemize}
\tightlist
\item
  \href{https://www.nytimes3xbfgragh.onion/2020/09/12/us/wildfires-live-updates.html\#link-f3961ff}{President
  Trump will visit California on Monday after destructive fires.}
\item
  \href{https://www.nytimes3xbfgragh.onion/2020/09/12/us/wildfires-live-updates.html\#link-7e503ae9}{Shifting
  weather may improve firefighting conditions on the West Coast.}
\item
  \href{https://www.nytimes3xbfgragh.onion/2020/09/12/us/wildfires-live-updates.html\#link-5e4c548d}{Oregon's
  fire marshal is temporarily replaced as firefighters battle blazes.}
\end{itemize}

The union that represents Cal Fire employees has been urging the
governor and the Legislature to cease relying on inmate firefighters.
Tim Edwards, the president of the union, said the California prisons
department had been lowering the bar for inmates who qualify for fire
camp.

``They are trying to add people who would have never made it into the
camps before either because of multiple offenses or the types of
offense,'' Mr. Edwards said.

The department of corrections says inmates must have less than five
years left on their sentences and are disqualified if they have a
history of escape with force or violence or if they have been convicted
of sexual offenses or arson.

The system of inmate firefighters was born of necessity during World War
II, when many of the state's firefighters were shipped off as soldiers
to Europe and the Pacific. Inmates were deployed to fill their places.
Several states, including Arizona, Georgia, Nevada and Wyoming, employ
prisoners to fight fires, but none have as many as California.

Some Californians, including former inmate firefighters, say the program
provides a sense of purpose, offering prisoners a chance to prove
themselves and the satisfaction of helping others.

``It gave me a sense of direction and a sense of worth,'' said Francis
Lopez, who spent a year as an inmate firefighter. ``There are people
high-fiving you, there are big signs saying, `Thank you to the inmates
for fighting our fires, for saving our homes.' You see that and you
think, `Wow, I can do good. I can be a person who is being respected.'''

Mr. Lopez, who was released three years ago and now works as a bartender
in Fresno, said the incarcerated fire crews were one of the few parts of
the prison system where inmates of different racial backgrounds
fraternized. The food, which is prepared by the inmates, was better than
in prisons, and they could spend large amounts of time outside. In the
winter they worked on flood control projects.

Image

The shortage of inmate firefighters on the front lines in California has
prompted criticism that some were released from prison when the state
needed them most.Credit...Max Whittaker for The New York Times

But it is the firefighting work that was most harrowing. The scene he
witnessed stepping out of the truck at his first fire is indelibly
marked into his memory.

``That door pops, you get out, and there are hills all around you and
everything is on fire,'' he said. ``There's helicopters flying by,
dropping pink retardant. There are fire trucks, hoses everywhere, and
you're hearing radio communication. It's a very, very intense scene.''

Like the non-inmate firefighters, they work 24 hours straight, sometimes
as long as 48 hours, hiking into remote, inaccessible canyons, charging
up steep ridges, all the while carrying gallons of water, survival gear
and their tools.

``We are the guys they send for the most dangerous missions,'' Mr. Lopez
said. ``We are given the jobs that the machines can't do.''

His one complaint: Inmates should be given a direct path to a
firefighting job once they are released. ``At least give him an
interview,'' he said.

Mr. Martin, the inmate firefighter released this month, said that even
before the coronavirus he chose the program as a way to get an earlier
parole and be reunited with his teenage son.

Finding a job with a felony conviction on his record
\href{https://www.nytimes3xbfgragh.onion/2018/11/15/us/california-paying-inmates-fight-fires.html}{will
be challenging}, said Mr. Martin, who was a police officer in Sacramento
for 12 years before he was sent to prison. He is now looking into work
with private fire contractors.

Mr. Martin said inmates would appreciate higher pay; when they are not
fighting fires they earn between \$2.90 and \$5.12 per day, according to
the prisons department. But what many inmates want most is freedom ---
an expedited release date.

``It's dirty, hard work and after a 24-hour shift we sleep on the
mountain with rattlesnakes and scorpions,'' Mr. Martin said. ``I don't
think anyone is there for the pay.''

Nicholas Bogel-Burroughs, Rebecca Griesbach and Maura Turcotte
contributed reporting.

Advertisement

\protect\hyperlink{after-bottom}{Continue reading the main story}

\hypertarget{site-index}{%
\subsection{Site Index}\label{site-index}}

\hypertarget{site-information-navigation}{%
\subsection{Site Information
Navigation}\label{site-information-navigation}}

\begin{itemize}
\tightlist
\item
  \href{https://help.nytimes3xbfgragh.onion/hc/en-us/articles/115014792127-Copyright-notice}{©~2020~The
  New York Times Company}
\end{itemize}

\begin{itemize}
\tightlist
\item
  \href{https://www.nytco.com/}{NYTCo}
\item
  \href{https://help.nytimes3xbfgragh.onion/hc/en-us/articles/115015385887-Contact-Us}{Contact
  Us}
\item
  \href{https://www.nytco.com/careers/}{Work with us}
\item
  \href{https://nytmediakit.com/}{Advertise}
\item
  \href{http://www.tbrandstudio.com/}{T Brand Studio}
\item
  \href{https://www.nytimes3xbfgragh.onion/privacy/cookie-policy\#how-do-i-manage-trackers}{Your
  Ad Choices}
\item
  \href{https://www.nytimes3xbfgragh.onion/privacy}{Privacy}
\item
  \href{https://help.nytimes3xbfgragh.onion/hc/en-us/articles/115014893428-Terms-of-service}{Terms
  of Service}
\item
  \href{https://help.nytimes3xbfgragh.onion/hc/en-us/articles/115014893968-Terms-of-sale}{Terms
  of Sale}
\item
  \href{https://spiderbites.nytimes3xbfgragh.onion}{Site Map}
\item
  \href{https://help.nytimes3xbfgragh.onion/hc/en-us}{Help}
\item
  \href{https://www.nytimes3xbfgragh.onion/subscription?campaignId=37WXW}{Subscriptions}
\end{itemize}
