Sections

SEARCH

\protect\hyperlink{site-content}{Skip to
content}\protect\hyperlink{site-index}{Skip to site index}

\href{https://www.nytimes3xbfgragh.onion/section/politics}{Politics}

\href{https://myaccount.nytimes3xbfgragh.onion/auth/login?response_type=cookie\&client_id=vi}{}

\href{https://www.nytimes3xbfgragh.onion/section/todayspaper}{Today's
Paper}

\href{/section/politics}{Politics}\textbar{}House Votes to Block Postal
Changes and Allocate Funds for Mail

\url{https://nyti.ms/3gpxenv}

\begin{itemize}
\item
\item
\item
\item
\item
\end{itemize}

\begin{itemize}
\item
  \href{https://www.nytimes3xbfgragh.onion/2020/09/12/us/politics/biden-trump-poll-wisconsin-minnesota.html?action=click\&pgtype=Article\&state=default\&region=TOP_BANNER\&context=storylines_menu}{New
  York Times Poll}
\item
  \href{https://www.nytimes3xbfgragh.onion/interactive/2020/us/elections/election-states-biden-trump.html?action=click\&pgtype=Article\&state=default\&region=TOP_BANNER\&context=storylines_menu}{Paths
  to 270}
\item
  \href{https://www.nytimes3xbfgragh.onion/interactive/2019/us/elections/2020-presidential-election-calendar.html?action=click\&pgtype=Article\&state=default\&region=TOP_BANNER\&context=storylines_menu}{Voting
  Deadlines}
\item
  \href{https://www.nytimes3xbfgragh.onion/interactive/2020/08/31/us/politics/vote-by-mail-deadlines.html?action=click\&pgtype=Article\&state=default\&region=TOP_BANNER\&context=storylines_menu}{Voting
  by Mail}
\item
  \href{https://www.nytimes3xbfgragh.onion/newsletters/politics?action=click\&pgtype=Article\&state=default\&region=TOP_BANNER\&context=storylines_menu}{Politics
  Newsletter}
\end{itemize}

Advertisement

\protect\hyperlink{after-top}{Continue reading the main story}

Supported by

\protect\hyperlink{after-sponsor}{Continue reading the main story}

\hypertarget{house-votes-to-block-postal-changes-and-allocate-funds-for-mail}{%
\section{House Votes to Block Postal Changes and Allocate Funds for
Mail}\label{house-votes-to-block-postal-changes-and-allocate-funds-for-mail}}

The Democratic bill would send \$25 billion to the Postal Service and
reverse changes that have slowed service until after November's
election.

\includegraphics{https://static01.graylady3jvrrxbe.onion/images/2020/08/22/us/politics/22dc-cong-update/22dc-cong-update-articleLarge-v2.jpg?quality=75\&auto=webp\&disable=upscale}

\href{https://www.nytimes3xbfgragh.onion/by/nicholas-fandos}{\includegraphics{https://static01.graylady3jvrrxbe.onion/images/2018/11/06/multimedia/author-nicholas-fandos/author-nicholas-fandos-thumbLarge-v2.png}}\href{https://www.nytimes3xbfgragh.onion/by/emily-cochrane}{\includegraphics{https://static01.graylady3jvrrxbe.onion/images/2018/11/28/multimedia/author-emily-cochrane/author-emily-cochrane-thumbLarge-v3.png}}

By \href{https://www.nytimes3xbfgragh.onion/by/nicholas-fandos}{Nicholas
Fandos} and
\href{https://www.nytimes3xbfgragh.onion/by/emily-cochrane}{Emily
Cochrane}

\begin{itemize}
\item
  Aug. 22, 2020
\item
  \begin{itemize}
  \item
  \item
  \item
  \item
  \item
  \end{itemize}
\end{itemize}

WASHINGTON --- The House interrupted its summer recess on Saturday for a
rare weekend session to approve legislation blocking cost-cutting and
operational changes at the Postal Service that Democrats, civil rights
advocates and some Republicans fear could jeopardize mail-in ballots
this fall.

\href{https://rules.house.gov/sites/democrats.rules.house.gov/files/BILLS-116HR8015-RCP116-61.pdf}{The
measure}, put forward by Democratic leaders, would also require the
Postal Service to prioritize the delivery of all election-related mail
and grant the beleaguered agency a rare \$25 billion infusion to cover
revenue lost because of the coronavirus pandemic and ensure it has the
resources to address what is expected to be the largest vote-by-mail
operation in the nation's history.

Democrats were joined by 26 Republicans in voting yes, passing the
legislation 257 to 150, with more than 20 Republicans not voting. But
the bill, as written, appeared unlikely to move through the
Republican-controlled Senate. President Trump opposed the measure
\href{https://twitter.com/realDonaldTrump/status/1297275235432005632}{in
last-minute tweets}, calling it a ``money wasting HOAX'' by Democrats.

Democrats framed Saturday's action as an emergency intervention into the
affairs of an independent agency to protect vital mail and package
services that have seen significant delays this summer as the new
postmaster general, Louis DeJoy,
\href{https://www.nytimes3xbfgragh.onion/2020/08/15/us/post-office-vote-by-mail.html}{moved
swiftly to cut costs} to close a yawning budget gap. They said it was
also necessary to instill confidence in American voters that the agency
would safeguard their ballots despite
\href{https://www.nytimes3xbfgragh.onion/2020/08/03/us/politics/trump-mail-in-voting.html}{near
daily attacks by Mr. Trump on mail-in voting}.

``This is not a partisan issue,'' Representative Carolyn B. Maloney,
Democrat of New York and the lead author of the bill, said, as she
\href{https://oversight.house.gov/sites/democrats.oversight.house.gov/files/PMG\%20Briefing\%20\%E2\%80\%93\%20Service\%20Performance\%20Management\%2008-12-20.pdf}{released
Postal Service statistics documenting the slowdown in delivery since
early July.} ``It makes absolutely no sense to impose these kinds of
dangerous cuts in the middle of a pandemic and just months before the
elections in November.''

Most Republicans in the House opposed it after Mr. DeJoy, facing intense
backlash and with the vote looming,
\href{https://www.nytimes3xbfgragh.onion/2020/08/18/us/politics/postal-service-suspends-changes.html}{announced
this week that he would temporarily halt} the removal of blue mailboxes
and sorting machines, as well as changes to post office hours and to
mail delivery operations until after Nov. 3 out of an abundance of
caution.

\href{https://www.nytimes3xbfgragh.onion/2020/08/21/us/politics/dejoy-postal-service-senate-hearing.html}{In
testimony before the Senate on Friday}, Mr. DeJoy reiterated that pledge
and said ensuring successful mail-in voting would be the agency's ``No.
1 priority.'' He called Democrats' assertion that he was working with
Mr. Trump to hinder the program ``outrageous'' and testified that he
planned to continue the agency's practice of prioritizing election mail.

\includegraphics{https://static01.graylady3jvrrxbe.onion/images/2020/08/23/us/politics/23dc-cong-print1/merlin_175817424_0cd2bf0c-0a0c-41db-bd37-21e1ea232dbf-articleLarge.jpg?quality=75\&auto=webp\&disable=upscale}

He is scheduled to testify again on Monday before the House Oversight
and Reform Committee.

The decision to recall lawmakers back to the Capitol underscored just
how high the political and electoral stakes have become around the
operations of a usually humdrum federal service, especially in the eyes
of Democrats. Even if it does not become law, they reason, the vote will
help elevate the issue in the eyes of regular Americans and further
tarnish Mr. Trump.

``Don't pay any attention to what the president is saying because it's
all designed to suppress the vote,'' Speaker Nancy Pelosi of California
said before the vote.

Postal leaders have been warning for months that the sharp decline in
mail caused by the pandemic could jeopardize the solvency of an agency
that has struggled to turn a profit. But Mr. DeJoy, a Trump donor and
former logistics executive, introduced measures to cut down on
transportation costs and overtime this summer, leading to
\href{https://www.nytimes3xbfgragh.onion/2020/08/21/us/postal-service-mail-rural.html}{substantial
delivery delays of vital items like medicines, checks and even chicks}.
Democrats and postal unions began to caution that the Trump
administration may be moving to destabilize the Postal Service during an
election year and aid its private competitors because of the president's
animus against mail-in voting.

In his tweets on Saturday, he said the Postal Service did not need the
money, despite its outstanding request to Congress for the funds, and
repeated false claims that voting by mail is fraudulent.

He accused Democrats of backing a universal vote-by-mail ``scam'' in
``violation of everything that our Country stands for.''

Voting by mail is neither new, a scam, nor at risk of widespread fraud,
as Mr. Trump insists. Millions of Americans in conservative and liberal
states alike cast their ballots through the Postal Service in 2016 and
2018. Mr. Trump plans to do so this year, but makes a distinction
between absentee voting through the mail and programs overseen by
Republicans and Democrats that proactively send ballots to all voters.

Adding to some lawmakers' worries, Mr. DeJoy's changes coincided with
the long-planned removal of hundreds of blue postal collection boxes
across the country and the decommissioning of mail-sorting machines,
part of a regular practice to adjust to the steady decline of mail.

Many Republican lawmakers have joined Democrats in voicing concern over
the slowdowns and demanding assurances from Mr. DeJoy and others that
the Postal Service will be able to carry out the vote-by-mail
initiatives.

Among the Republicans bucking their leadership on Saturday were
moderates, representatives of heavily rural districts that rely on the
mail for basic services and several lawmakers fighting for re-election
this fall.

``We should be preserving and enhancing U.S.P.S. delivery standards and
services, not implementing operational changes that could delay delivery
times and undermine quality services that every American depends on,''
said one of them, Representative Brian Fitzpatrick, Republican of
Pennsylvania.

More conservative Republicans and allies of Mr. Trump accused Democrats
of continuing to fan hyperbolic and unsupported theories of a conspiracy
overseen by Mr. Trump to sabotage the election for their own political
gain.

``Like the Russia hoax and the impeachment sham, the Democrats have
manufactured another scandal for political purposes,'' said
Representative James Comer of Kentucky, the top Republican on the
oversight panel.

Many Senate Republicans --- including moderate senators facing tough
races in November --- are supportive of granting the agency a direct
appropriation, albeit with some policy stipulations to address its
long-term business model. Congress provided \$10 billion in loan
authority for the agency this spring, as well, that it has yet to use.

But Senator Mitch McConnell of Kentucky, the majority leader, said
plainly on Saturday that he did not plan to bring up a stand-alone bill
in the Senate when lawmakers are at a stalemate over broader coronavirus
relief legislation.

``The facts show the U.S.P.S. is equipped to handle this election, and
if a real need arises, Congress will meet it,'' he said in a statement.
``The Senate will absolutely not pass stand-alone legislation for the
Postal Service while American families continue to go without more
relief.''

The Democratic legislation would amount to an extraordinary
intervention. Though it is a government entity explicitly mentioned in
the Constitution, the modern Postal Service functions as a
self-sustaining business that raises funds through postal products, not
from taxpayers.

Democrats' \$25 billion comes with no strings attached, but their bill
would effectively bar Postal Service leaders from making any changes
that ``impede prompt, reliable and efficient service'' through at least
January. It would reverse changes already put in place that Mr. DeJoy
had declined to.

Image

The postmaster general, Louis DeJoy, facing intense backlash, said this
week that he would temporarily halt the removal of blue mailboxes and
sorting machines.Credit...Mandel Ngan/Agence France-Presse --- Getty
Images

By imposing strict requirements until the end of the pandemic, it would,
if approved, also effectively block the postmaster general from making
more sweeping changes he has planned after Election Day that Democrats
generally oppose.

The \$25 billion in direct funds match
\href{https://www.nytimes3xbfgragh.onion/2020/04/09/us/politics/coronavirus-is-threatening-one-of-governments-steadiest-services-the-mail.html?action=click\&module=RelatedLinks\&pgtype=Article}{a
request the Postal Service made of Congress} this spring to cover lost
revenue.

Mr. DeJoy and the Postal Service remain supportive of a direct grant
from Congress, though they have more cash on hand than previously
anticipated. Still, he opposed the restrictions that Democrats' bill
would impose. On Friday, he urged lawmakers to pass bipartisan
legislation to address the agency's longer-term financial woes by
unburdening it from a requirement to prefund retiree benefits that has
put it deep in the red.

Even a short-term infusion of cash has become tied up in larger
political and policy fights over how to respond to the pandemic.

Lawmakers had hoped to reach bipartisan consensus on a huge relief bill
--- including around \$10 billion for the Postal Service --- before they
left town for their recess, but talks stalled.

Mr. McConnell expressed some optimism in recent days that the intense
interest in the postal issue could present a new negotiating opportunity
to draft a smaller, short-term bill to circumvent the impasse.

Democratic leaders had also been under pressure from more than 100
rank-and-file lawmakers to use Saturday's session to address other
elements of the coronavirus relief negotiations, as well. And Mark
Meadows, the White House chief of staff, spent part of the day buzzing
around the chamber to try to reignite talks.

``If you really want to help Americans, how about passing relief for
small businesses and unemployment assistance ALONG with postal funding?"
he wrote in a
\href{https://twitter.com/MarkMeadows/status/1297196107982942209}{tweet}.
``We agree on these. There's NO reason not to deliver relief for
Americans right now.''

Ms. Pelosi insisted she was unwilling to break up Democrats' \$3.4
trillion relief bill into parts, arguing that the postal bill was
necessary to handle separately because it included policy provisions
that the larger measure did not.

\hypertarget{our-2020-election-guide}{%
\section{Our 2020 Election Guide}\label{our-2020-election-guide}}

Updated ~Sept. 12, 2020

\begin{center}\rule{0.5\linewidth}{\linethickness}\end{center}

\begin{itemize}
\item ~
  \hypertarget{the-latest}{%
  \subsection{The Latest}\label{the-latest}}

  \begin{itemize}
  \item
    President Trump has failed to erase Joseph R. Biden Jr.'s lead
    across a set of key swing states,
    \href{https://www.nytimes3xbfgragh.onion/2020/09/12/us/politics/biden-trump-poll-wisconsin-minnesota.html?action=click\&pgtype=Article\&state=default\&region=BELOW_MAIN_CONTENT\&context=storylines_guide}{according
    to a poll}~conducted by The Times and Siena College.
  \end{itemize}
\item ~
  \hypertarget{paths-to-270}{%
  \subsection{Paths to 270}\label{paths-to-270}}

  \begin{itemize}
  \item
    Joe Biden and Donald Trump need 270 electoral votes to reach the
    White House. Try building
    \href{https://www.nytimes3xbfgragh.onion/interactive/2020/us/elections/election-states-biden-trump.html?action=click\&pgtype=Article\&state=default\&region=BELOW_MAIN_CONTENT\&context=storylines_guide}{your
    own coalition of battleground states}~to see potential outcomes.
  \end{itemize}
\item ~
  \hypertarget{voting-deadlines}{%
  \subsection{Voting Deadlines}\label{voting-deadlines}}

  \begin{itemize}
  \item
    Early voting for the presidential election starts in September~in
    some states. Take a look at
    \href{https://www.nytimes3xbfgragh.onion/interactive/2019/us/elections/2020-presidential-election-calendar.html?action=click\&pgtype=Article\&state=default\&region=BELOW_MAIN_CONTENT\&context=storylines_guide}{key
    dates}\href{https://www.nytimes3xbfgragh.onion/interactive/2019/us/elections/2020-presidential-election-calendar.html?action=click\&pgtype=Article\&state=default\&region=BELOW_MAIN_CONTENT\&context=storylines_guide}{where
    you
    liv}\href{https://www.nytimes3xbfgragh.onion/interactive/2019/us/elections/2020-presidential-election-calendar.html?action=click\&pgtype=Article\&state=default\&region=BELOW_MAIN_CONTENT\&context=storylines_guide}{e}.
    If you're voting by
    mail,~\href{https://www.nytimes3xbfgragh.onion/interactive/2020/08/31/us/politics/vote-by-mail-deadlines.html?action=click\&pgtype=Article\&state=default\&region=BELOW_MAIN_CONTENT\&context=storylines_guide}{it's
    risky to procrastinate}.
  \item
    \href{https://www.nytimes3xbfgragh.onion/interactive/2020/us/elections/joe-biden.html?action=click\&pgtype=Article\&state=default\&region=BELOW_MAIN_CONTENT\&context=storylines_guide}{}

    \hypertarget{joe-biden}{%
    \section{Joe Biden}\label{joe-biden}}

    \hypertarget{democrat}{%
    \subsection{Democrat}\label{democrat}}

    \href{https://www.nytimes3xbfgragh.onion/interactive/2020/us/elections/donald-trump.html?action=click\&pgtype=Article\&state=default\&region=BELOW_MAIN_CONTENT\&context=storylines_guide}{}

    \hypertarget{donald-trump}{%
    \section{Donald Trump}\label{donald-trump}}

    \hypertarget{republican}{%
    \subsection{Republican}\label{republican}}
  \end{itemize}
\item
  \hypertarget{keep-up-with-our-coverage}{%
  \subsection{Keep Up With Our
  Coverage}\label{keep-up-with-our-coverage}}

  \begin{itemize}
  \item
    Get an
    \href{https://www.nytimes3xbfgragh.onion/newsletters/politics?action=click\&pgtype=Article\&state=default\&region=BELOW_MAIN_CONTENT\&context=storylines_guide}{email}~recapping
    the day's news
  \item
    Download our mobile app on
    \href{https://apps.apple.com/us/app/nytimes/id284862083?ls=1\&mat_click_id=5c79ae7455014fd1bd66b5610c05b8f2-20191112-16948\&referrer=mat_click_id\%3D5c79ae7455014fd1bd66b5610c05b8f2-20191112-16948\%26link_click_id\%3D722930677036718082}{iOS}~and
    \href{http://a.localytics.com/android?id=com.nytimes.android\&referrer=utm_source\%3Dother_nyt_mobile_web\%26utm_medium\%3DWeb\%2520page\%26utm_term\%3DGeneral\%2520Mobile\%2520Page\%26utm_campaign\%3DNYT\%2520Mobile\%2520General\%2520Page}{Android}~and
    turn on Breaking News and Politics alerts
  \end{itemize}
\end{itemize}

Advertisement

\protect\hyperlink{after-bottom}{Continue reading the main story}

\hypertarget{site-index}{%
\subsection{Site Index}\label{site-index}}

\hypertarget{site-information-navigation}{%
\subsection{Site Information
Navigation}\label{site-information-navigation}}

\begin{itemize}
\tightlist
\item
  \href{https://help.nytimes3xbfgragh.onion/hc/en-us/articles/115014792127-Copyright-notice}{©~2020~The
  New York Times Company}
\end{itemize}

\begin{itemize}
\tightlist
\item
  \href{https://www.nytco.com/}{NYTCo}
\item
  \href{https://help.nytimes3xbfgragh.onion/hc/en-us/articles/115015385887-Contact-Us}{Contact
  Us}
\item
  \href{https://www.nytco.com/careers/}{Work with us}
\item
  \href{https://nytmediakit.com/}{Advertise}
\item
  \href{http://www.tbrandstudio.com/}{T Brand Studio}
\item
  \href{https://www.nytimes3xbfgragh.onion/privacy/cookie-policy\#how-do-i-manage-trackers}{Your
  Ad Choices}
\item
  \href{https://www.nytimes3xbfgragh.onion/privacy}{Privacy}
\item
  \href{https://help.nytimes3xbfgragh.onion/hc/en-us/articles/115014893428-Terms-of-service}{Terms
  of Service}
\item
  \href{https://help.nytimes3xbfgragh.onion/hc/en-us/articles/115014893968-Terms-of-sale}{Terms
  of Sale}
\item
  \href{https://spiderbites.nytimes3xbfgragh.onion}{Site Map}
\item
  \href{https://help.nytimes3xbfgragh.onion/hc/en-us}{Help}
\item
  \href{https://www.nytimes3xbfgragh.onion/subscription?campaignId=37WXW}{Subscriptions}
\end{itemize}
