Sections

SEARCH

\protect\hyperlink{site-content}{Skip to
content}\protect\hyperlink{site-index}{Skip to site index}

\href{https://www.nytimes3xbfgragh.onion/section/us}{U.S.}

\href{https://myaccount.nytimes3xbfgragh.onion/auth/login?response_type=cookie\&client_id=vi}{}

\href{https://www.nytimes3xbfgragh.onion/section/todayspaper}{Today's
Paper}

\href{/section/us}{U.S.}\textbar{}Covid in the Classroom? Some Schools
Are Keeping It Quiet

\url{https://nyti.ms/329CWFa}

\begin{itemize}
\item
\item
\item
\item
\item
\item
\end{itemize}

\hypertarget{school-reopenings}{%
\subsubsection{\texorpdfstring{\href{https://www.nytimes3xbfgragh.onion/spotlight/schools-reopening?name=styln-coronavirus-schools-reopening\&region=TOP_BANNER\&block=storyline_menu_recirc\&action=click\&pgtype=Article\&impression_id=78b271e0-f299-11ea-a3f6-692f6f3812c0\&variant=undefined}{School
Reopenings}}{School Reopenings}}\label{school-reopenings}}

\begin{itemize}
\tightlist
\item
  \href{https://www.nytimes3xbfgragh.onion/2020/09/04/us/bar-exam-coronavirus.html?name=styln-coronavirus-schools-reopening\&region=TOP_BANNER\&block=storyline_menu_recirc\&action=click\&pgtype=Article\&impression_id=78b271e1-f299-11ea-a3f6-692f6f3812c0\&variant=undefined}{Delayed
  Licensing Exams}
\item
  \href{https://www.nytimes3xbfgragh.onion/2020/09/08/upshot/children-testing-shortfalls-virus.html?name=styln-coronavirus-schools-reopening\&region=TOP_BANNER\&block=storyline_menu_recirc\&action=click\&pgtype=Article\&impression_id=78b298f0-f299-11ea-a3f6-692f6f3812c0\&variant=undefined}{Limited
  Testing for Children}
\item
  \href{https://www.nytimes3xbfgragh.onion/2020/09/01/world/schools-reopen-globe-students.html?name=styln-coronavirus-schools-reopening\&region=TOP_BANNER\&block=storyline_menu_recirc\&action=click\&pgtype=Article\&impression_id=78b298f1-f299-11ea-a3f6-692f6f3812c0\&variant=undefined}{School
  Around the World}
\item
  \href{https://www.nytimes3xbfgragh.onion/interactive/2020/us/covid-college-cases-tracker.html?name=styln-coronavirus-schools-reopening\&region=TOP_BANNER\&block=storyline_menu_recirc\&action=click\&pgtype=Article\&impression_id=78b298f2-f299-11ea-a3f6-692f6f3812c0\&variant=undefined}{Tracking
  College Cases}
\end{itemize}

Advertisement

\protect\hyperlink{after-top}{Continue reading the main story}

Supported by

\protect\hyperlink{after-sponsor}{Continue reading the main story}

\hypertarget{covid-in-the-classroom-some-schools-are-keeping-it-quiet}{%
\section{Covid in the Classroom? Some Schools Are Keeping It
Quiet}\label{covid-in-the-classroom-some-schools-are-keeping-it-quiet}}

Some states and school districts provide detailed data on school
outbreaks. Others choose to keep such information under wraps.

\includegraphics{https://static01.graylady3jvrrxbe.onion/images/2020/08/21/us/21virus-schooldata-1/merlin_175950414_96ba531d-c3b7-4991-902f-4d2c21bdfa8a-articleLarge.jpg?quality=75\&auto=webp\&disable=upscale}

By \href{https://www.nytimes3xbfgragh.onion/by/dan-levin}{Dan Levin}

\begin{itemize}
\item
  Aug. 22, 2020
\item
  \begin{itemize}
  \item
  \item
  \item
  \item
  \item
  \item
  \end{itemize}
\end{itemize}

On the first day of school in Camden County, Ga., local Facebook groups
were already buzzing with rumors that a teacher had tested positive for
the coronavirus. The next day, a warning went out to
\href{https://www.nytimes3xbfgragh.onion/2020/09/08/nyregion/hartford-schools-ransomware.html}{school}
administrators: Keep teachers quiet.

``Staff who test positive are not to notify any other staff members,
parents of their students or any other person/entity that they may have
exposed them,'' Jon Miller, the district's deputy superintendent, wrote
in a confidential email on Aug. 5.

In the weeks since, parents, students and teachers in the coastal
community on the Florida border have heard by word of mouth of more
positive cases linked to district schools. Some parents said they had
been called by local officials and told that their children should
quarantine.

But even as fears of an outbreak have grown, the district has refused to
publicly confirm a single case, either to the local community or The New
York Times.

``This is a danger to our community,'' said Cheryl Honeycutt, the mother
of an 8-year-old Camden student. ``We're safer if we know what's going
on, but their pan answer is, `We can neither confirm or deny.'''

\includegraphics{https://static01.graylady3jvrrxbe.onion/images/2020/08/21/us/21virus-schooldata-2/21virus-schooldata-2-articleLarge.jpg?quality=75\&auto=webp\&disable=upscale}

As schools in parts of the country have reopened classrooms amid a
still-raging pandemic, some districts have been open about coronavirus
cases in their buildings. They send weekly --- and
\href{https://www.cherokeek12.net/Content2/covid-letters-archive}{in
some cases, daily} --- reports to families and
\href{https://www.cherokeek12.net/Content2/casestatusreport}{updating
online dashboards} with the latest positive test results and quarantine
counts.

But other districts have been silent, sometimes citing privacy concerns
to withhold information, to the dismay of some anxious parents,
concerned educators and public health experts trying to combat the
pandemic.

``If schools don't notify, it actually can make disease control more
difficult,'' said Dr. Ashish Jha, director of the Harvard Global Health
Institute. ``And it's not like no one will know. Word will get out
through a rumor mill. You don't scare people by telling them what's
going on. You scare them by hiding information.''

In many places, reopening schools has taken on a distinctly partisan
bent, with President Trump and Republican governors such as Ron DeSantis
of Florida
\href{https://www.nytimes3xbfgragh.onion/2020/08/19/us/coronavirus-schools-florida-local-control.html}{urging
in-person instruction}. A constant flow of information about positive
cases in classrooms and quarantined students could hinder those efforts,
experts said.

``When schools have to shut down after students test positive, that
doesn't look good politically on governors and lawmakers who have
advocated for opening up,'' said Clay Calvert, director of the Marion B.
Brechner First Amendment Project at the University of Florida. ``So the
potential is there to hide behind privacy laws.

``There are definitely battle lines drawn, and the release of
information can sway public opinion.''

Indeed, some states have seen growing concern after school doors opened
and infections were immediately reported. In Georgia, nearly 2,500
students and 62 staff members in the
\href{https://www.nytimes3xbfgragh.onion/2020/08/12/us/georgia-school-coronavirus.html}{Cherokee
County School District} have been ordered to quarantine, while 71 out of
82 counties in Mississippi have reported cases in schools.

State notification policies vary widely across the country. Officials in
Colorado and North Carolina are reporting which schools have had
positive cases, while Louisiana, which had not previously identified
specific schools with outbreaks,
\href{https://www.theadvocate.com/baton_rouge/news/coronavirus/article_07f9a972-e25a-11ea-b188-778b27f6ef01.html}{said
this week that it was creating} a new system to ``efficiently report
relevant Covid-19 data in schools for greater public visibility.''

On the other end of the spectrum, Oklahoma does not require school
districts to report Covid-19 cases to health departments. And some
states that do, including Maine, say that privacy concerns prevent
officials from sharing those details with the public. Tennessee this
week
\href{https://www.tennessean.com/story/news/education/2020/08/18/tennessee-refuses-release-information-covid-19-cases-schools/3394684001/}{backed
away from a previous commitment} by the governor to report the number of
cases linked to schools, and is providing information only by county.

In Virginia, state law prohibits the health department from disclosing
cases at specific facilities, including schools, said Tammie Smith, a
spokeswoman for the state health commissioner. The commissioner had
\href{https://www.whsv.com/content/news/After-months-Va-health-department-releases-detailed-info-on-long-term-care-facility-outbreaks-571373611.html}{originally
said the same thing about nursing homes}, but was later
\href{https://www.washingtonpost.com/local/virginia-nursing-homes-walk-tightrope/2020/06/27/7a14c2da-b64f-11ea-a510-55bf26485c93_story.html}{ordered
to release the data} by Gov. Ralph Northam after a public outcry.

Image

New York City's education department said~officials would tell families
and students about each confirmed virus case.Credit...Hiroko Masuike/The
New York Times

Disclosure plans at the district level reveal a similar patchwork. In
New York City, the nation's largest school system with 1.1 million
students --- and one of the few remaining big districts
\href{https://www.nytimes3xbfgragh.onion/2020/08/18/nyregion/schools-reopen-nyc.html}{planning
to open with in-person instruction} --- officials will tell families and
students about each confirmed case, according to the city's education
department.

In Florida, the Pasco County school district outside Tampa will inform
students and teachers who have been in close contact with someone who
has tested positive for the virus, a district spokesman said, and will
alert the rest of the school and the news media about confirmed cases.

But in nearby Orange County, which includes Orlando, the teachers' union
filed a lawsuit against the district in July after it refused to
disclose the names of schools and workplaces where employees had tested
positive for the virus, citing privacy laws.

``The district is totally nontransparent,'' said Wendy Doromal,
president of the Orange County Classroom Teachers Association, which
represents the district's 14,000 educators. ``Of course we never asked
for the individual's name or any confidential information.''

Ms. Doromal said the union sued after teachers, including some with
health issues that could make them more vulnerable to the virus, went to
schools during the summer to retrieve belongings or volunteer, only to
discover that some buildings were closed for deep cleaning because of a
positive case.

Other employees were told after they were already inside, she said. ``We
felt that was very irresponsible.''

A district spokesman declined to comment, citing the litigation.

The silence worries Kila Murphey, a nurse practitioner with two children
in the district. ``Simply saying that the health department is going to
do contact tracing doesn't really reassure me of anything,'' she said.
``We need to know that there was a Covid case and what steps the school
is taking to ensure they don't have an outbreak.''

Some administrators who have chosen not to publicly disclose infections
at a particular school say they are concerned about the privacy of
individual students or staff members.

\href{https://www.nytimes3xbfgragh.onion/spotlight/schools-reopening?action=click\&pgtype=Article\&state=default\&region=MAIN_CONTENT_3\&context=storylines_keepup}{}

\hypertarget{school-reopenings-}{%
\subsubsection{School Reopenings ›}\label{school-reopenings-}}

\hypertarget{back-to-school}{%
\paragraph{Back to School}\label{back-to-school}}

Updated Sept. 8, 2020

The latest on how schools are reopening amid the pandemic.

\begin{itemize}
\item
  \begin{itemize}
  \tightlist
  \item
    The first day of school is an annual rite of passage. But this year,
    it looks very different for tens of millions of students.
    \href{https://www.nytimes3xbfgragh.onion/2020/09/05/us/virtual-return-to-school-covid.html?action=click\&pgtype=Article\&state=default\&region=MAIN_CONTENT_3\&context=storylines_keepup}{We
    talked to some about their hopes and fears}.
  \item
    Coronavirus cases
    \href{https://www.nytimes3xbfgragh.onion/2020/09/06/us/colleges-coronavirus-students.html?action=click\&pgtype=Article\&state=default\&region=MAIN_CONTENT_3\&context=storylines_keepup}{are
    spiking in America's college towns}, leading to concern that young
    people who are infected will contribute to a spread of the virus.
  \item
    A growing number of Catholic schools across the country are
    \href{https://www.nytimes3xbfgragh.onion/2020/09/05/us/catholic-school-closings.html?action=click\&pgtype=Article\&state=default\&region=MAIN_CONTENT_3\&context=storylines_keepup}{shutting
    down forever during the coronavirus pandemic}, citing insurmountable
    financial pressure.
  \item
    The magazine's Ethicist columnist answers a question from a
    spokesperson at a major university:
    \href{https://www.nytimes3xbfgragh.onion/2020/09/08/magazine/university-reopening-safety-ethics.html?action=click\&pgtype=Article\&state=default\&region=MAIN_CONTENT_3\&context=storylines_keepup}{Can
    I promote a reopening plan I have doubts about}?
  \end{itemize}
\end{itemize}

In the North Kansas City school district in Missouri, which has about
20,000 students, the superintendent, Dan Clemens,
\href{https://www.kcur.org/education/2020-07-30/metro-kansas-city-schools-wont-be-able-to-open-in-september-if-cases-dont-start-going-down}{told
the school board} at a meeting last month that local health officials
had advised him to be cautious about sharing information; if only one or
two people at a school test positive, he said, others might be able to
figure out who it was.

``If I report any type of Covid cases, because the numbers are so low in
Clay County and particularly within this school district, I would be
imposing on the privacy rights of individuals within our community,''
Dr. Clemens said at the meeting. ``So I just want to be very careful.''

Officials often cite privacy laws such as the federal Family Educational
Rights and Privacy Act and the Health Insurance Portability and
Accountability Act when arguing against disclosure. Yet neither law bars
public schools from releasing information about cases as long as they do
not provide personal details about those who are infected, the federal
education and
\href{https://www.hhs.gov/hipaa/for-professionals/faq/513/does-hipaa-apply-to-an-elementary-school/index.html}{health
departments} have said --- and in some situations, even that might be
allowed.

``School notification is an effective method of informing parents and
eligible students of an illness in the school,'' the Education
Department
\href{https://studentprivacy.ed.gov/sites/default/files/resource_document/file/FERPA\%20and\%20Coronavirus\%20Frequently\%20Asked\%20Questions_0.pdf}{wrote
in March}.

Schools have often abused privacy laws to hide damaging information that
could expose them to lawsuits or negative media coverage, said Mr.
Calvert at the University of Florida. ``In the name of protecting
personal privacy, many of those districts are really sacrificing public
health concerns,'' he said.

Such is the fear in Camden County, where in recent weeks the 40-bed
hospital in St. Marys, the county seat,
\href{https://thebrunswicknews.com/news/local_news/hospital-using-ambulances-as-extra-emergency-space/article_2af5be1e-364d-57d6-aded-4ff80b415708.html}{had
to divert ambulances elsewhere} because of a crush of coronavirus
patients.

Image

Kiisa Kennedy let her 11th-grade son attend classes because the district
was not providing certain advanced courses to remote learners, or
letting them play sports.Credit...Charlotte Kesl for The New York Times

Although the district offered a choice between remote learning and
in-person instruction, some families said they felt pressure to return
when school started this month. Kiisa Kennedy, whose two children go to
Camden County High School, said she agreed to let her 11th-grade son, a
football player, attend classes because the district was not providing
certain advanced courses to remote learners or letting them participate
in sports.

Like other parents, she said she had heard of at least nine positive
cases in the schools and entire classes that had to be quarantined, but
the district refused to answer questions.

``They've been very hush-hush,'' she said. ``We as parents cannot make
informed decisions because they're withholding the information.''

The school district did not respond to several requests for comment.

Ginger Heidel, a spokeswoman for the Coastal Health District, the agency
that covers several coastal counties in Georgia including Camden,
declined to answer questions about virus cases in schools, citing
privacy laws. She said schools were only required to alert people who
have been in close contact with someone who had tested positive, but
were allowed to notify the community ``if they want.''

Last week, the Camden County school district reversed course on its
mask-optional policy, \href{http://www.camden.k12.ga.us/}{announcing
that they would now be required on school grounds}. But the announcement
came with no information about cases linked to schools.

``That just made me more scared,'' Ms. Kennedy said.

Barbara Harvey contributed reporting.

Advertisement

\protect\hyperlink{after-bottom}{Continue reading the main story}

\hypertarget{site-index}{%
\subsection{Site Index}\label{site-index}}

\hypertarget{site-information-navigation}{%
\subsection{Site Information
Navigation}\label{site-information-navigation}}

\begin{itemize}
\tightlist
\item
  \href{https://help.nytimes3xbfgragh.onion/hc/en-us/articles/115014792127-Copyright-notice}{©~2020~The
  New York Times Company}
\end{itemize}

\begin{itemize}
\tightlist
\item
  \href{https://www.nytco.com/}{NYTCo}
\item
  \href{https://help.nytimes3xbfgragh.onion/hc/en-us/articles/115015385887-Contact-Us}{Contact
  Us}
\item
  \href{https://www.nytco.com/careers/}{Work with us}
\item
  \href{https://nytmediakit.com/}{Advertise}
\item
  \href{http://www.tbrandstudio.com/}{T Brand Studio}
\item
  \href{https://www.nytimes3xbfgragh.onion/privacy/cookie-policy\#how-do-i-manage-trackers}{Your
  Ad Choices}
\item
  \href{https://www.nytimes3xbfgragh.onion/privacy}{Privacy}
\item
  \href{https://help.nytimes3xbfgragh.onion/hc/en-us/articles/115014893428-Terms-of-service}{Terms
  of Service}
\item
  \href{https://help.nytimes3xbfgragh.onion/hc/en-us/articles/115014893968-Terms-of-sale}{Terms
  of Sale}
\item
  \href{https://spiderbites.nytimes3xbfgragh.onion}{Site Map}
\item
  \href{https://help.nytimes3xbfgragh.onion/hc/en-us}{Help}
\item
  \href{https://www.nytimes3xbfgragh.onion/subscription?campaignId=37WXW}{Subscriptions}
\end{itemize}
