Sections

SEARCH

\protect\hyperlink{site-content}{Skip to
content}\protect\hyperlink{site-index}{Skip to site index}

\href{https://www.nytimes3xbfgragh.onion/section/arts}{Arts}

\href{https://myaccount.nytimes3xbfgragh.onion/auth/login?response_type=cookie\&client_id=vi}{}

\href{https://www.nytimes3xbfgragh.onion/section/todayspaper}{Today's
Paper}

\href{/section/arts}{Arts}\textbar{}Dan Budnik, Who Photographed
History, Is Dead at 87

\url{https://nyti.ms/2Ekz3oP}

\begin{itemize}
\item
\item
\item
\item
\item
\end{itemize}

Advertisement

\protect\hyperlink{after-top}{Continue reading the main story}

Supported by

\protect\hyperlink{after-sponsor}{Continue reading the main story}

\hypertarget{dan-budnik-who-photographed-history-is-dead-at-87}{%
\section{Dan Budnik, Who Photographed History, Is Dead at
87}\label{dan-budnik-who-photographed-history-is-dead-at-87}}

His assignments for leading magazines took him to pivotal events of the
civil rights era. He was also known for his photographs of artists.

\includegraphics{https://static01.graylady3jvrrxbe.onion/images/2020/08/21/obituaries/21budnik4/merlin_175896849_8c86f943-cbf8-46ab-996d-01a8c7f00b25-articleLarge.jpg?quality=75\&auto=webp\&disable=upscale}

\href{https://www.nytimes3xbfgragh.onion/by/neil-genzlinger}{\includegraphics{https://static01.graylady3jvrrxbe.onion/images/2018/06/13/multimedia/author-neil-genzlinger/author-neil-genzlinger-thumbLarge.jpg}}

By \href{https://www.nytimes3xbfgragh.onion/by/neil-genzlinger}{Neil
Genzlinger}

\begin{itemize}
\item
  Aug. 23, 2020
\item
  \begin{itemize}
  \item
  \item
  \item
  \item
  \item
  \end{itemize}
\end{itemize}

Dan Budnik, a photographer who captured abiding images of 1950s artists
at work, key events of the civil rights movement, the Hudson River
restoration effort, Native Americans in the Southwest and more, died on
Aug. 14 at an assisted living residence in Tucson, Ariz. He was 87.

His nephew, Kim Newton, said the causes were metabolic encephalopathy
and dementia.

Mr. Budnik shot assignments for Life, Look and numerous other leading
magazines, and his work was collected in several books, including
``Marching to the Freedom Dream'' (2014), which featured his pictures
from three significant civil rights moments: the 1958 Youth March for
Integrated Schools, the 1963 March on Washington for Jobs and Freedom
and the protest march from Selma to Montgomery, Ala., in 1965.

\includegraphics{https://static01.graylady3jvrrxbe.onion/images/2020/08/21/obituaries/21Budnik15/21Budnik15-articleLarge.jpg?quality=75\&auto=webp\&disable=upscale}

There were many photographers at those events, but Mr. Budnik had a
knack for the unexpected yet telling moment. At the March on Washington,
where the Rev. Dr. Martin Luther King Jr. delivered his ``I Have a
Dream'' speech, Mr. Budnik took a position behind Dr. King as he
addressed the enormous crowd from the steps of the Lincoln Memorial.

``I knew everyone else would be photographing him from the front and the
sides,'' he said in an interview in ``Marching to the Freedom Dream.''
``But I'm about six steps above him, knowing he had to exit in reverse
and go up the steps to where I was.''

He was rewarded with memorable images of Dr. King being swarmed by well
wishers. One showed a white man eager to shake Dr. King's hand.

Image

``King is a sardine, looking at all these bodies between them,'' Mr.
Budnik said of this photograph of Dr. King and an enthusiastic admirer
from the 1963 march.Credit...Dan Budnik/Contact Press Images

``But King is a sardine, looking at all these bodies between them,'' Mr.
Budnik said. ``The guy is leaning on the tops of heads of people. He's
so adamant about shaking Dr. King's hand. And so King squirrels around,
corkscrews, frees up an arm. And then they do the brotherhood clasp.''

At the Selma march, almost two years later, Mr. Budnik again looked for
evocative moments. One came when a Black teenager unfurled an American
flag and started to march. Mr. Budnik captured the image of an Army
National Guard sergeant saluting --- since guardsmen were taught to
salute the flag.

But, he recalled in an interview with The Austin American-Statesman in
2000, ``the next seven guardsmen turned their backs to the boy with the
flag.''

Image

At the Selma-to-Montgomery march in 1965, Mr. Budnik looked for
evocative moments and found one when a Black teenager unfurled an
American flag, and an Army National Guard sergeant saluted.Credit...Dan
Budnik Estate, via Associated Press

Mr. Budnik's nephew, who teaches at the University of Arizona's School
of Journalism and is a noted photographer himself, learned his craft in
part by accompanying his uncle on photo shoots.

``Walking down the street or on a trail, he would point out objects or
subjects that most people would just pass by in their normal rush
through life,'' Professor Newton said by email. ``It was his awareness
and unconventional approach to life that helped bring about the intimacy
one sees in his work.''

Daniel Budnik was born on May 20, 1933, in Mineola, N.Y., on Long
Island. His father, Maxim, was a butcher, and his mother, Tessie
(Lesniak) Budnik, was a bookkeeper.

He grew up on Long Island and often told the story of an incident when
he was 5 that first made him aware of racial prejudice. Playing marbles
with a kindergarten classmate who had recently moved north from Alabama,
he was shocked when the boy saw a local Black man, who was well known in
the village, and started throwing rocks at him and describing him with a
racial epithet.

Image

Mr. Budnik in 2016. His ``unconventional approach to life,'' said his
nephew, who is also a photographer, ``helped bring about the intimacy
one sees in his work.''Credit...Rixt Bosma

``I could not come up with a logical explanation for his behavior,'' Mr.
Budnik
\href{https://www.independent.co.uk/news/world/americas/on-the-road-to-civil-rights-extraordinary-images-of-the-selma-march-seen-for-the-first-time-10057218.html}{told
the British newspaper The Independent} in 2015. ``When I went to Selma
in '65, I thought about that boy.''

At 17, Mr. Budnik moved to the Los Angeles area to live with a sister.
He graduated from high school there and returned east to study at the
Art Students League of New York, thinking he might become a painter. But
in 1952 the artist Charles Alston, one of his teachers, showed him a
book by the photographer Henri Cartier-Bresson.

``Talk about epiphanies,'' Mr. Budnik said. ``It changed my life.''

First, though, came military service: He was drafted into the Army the
next year, serving until 1955. He used his mustering-out pay to buy a
Leica IIIf camera at a pawnshop.

Image

Mr. Budnik's pictures of artists were featured in several gallery shows.
He frequently photographed Georgia O'Keeffe, seen here in 1975, in her
later years.Credit...Dan Budnik/Contact Press Images

Image

Mr. Budnik documented his friend Pete Seeger's efforts to clean up the
Hudson River and became involved in those efforts himself.Credit...Dan
Budnik/Contact Press Images

Among his first subjects were the Abstract Expressionists and other
artists he had come to know in New York. He began photographing them as
they worked, continuing to do so into the mid-1960s, capturing Willem de
Kooning, Lee Bontecou, David Smith and many others.

Decades later, living in Arizona, he frequently photographed another
artist, Georgia O'Keeffe, in her later years. His pictures of artists
were compiled into several gallery shows.

Mr. Budnik took a desk job with the agency
\href{https://www.magnumphotos.com/}{Magnum Photos} in 1957, learning
from some of its top photographers. By the end of the decade he was a
professional photographer himself, generally working as a freelancer.
His work in Alabama in 1965, for instance, started out as an assignment
for Life: He had pitched the idea of photographing the segregationist
side of the civil rights movement.

He had arranged meetings with Gov. George C. Wallace and
\href{https://www.nytimes3xbfgragh.onion/2007/06/07/us/07clark.html}{Jim
Clark}, the notorious racist sheriff, when the clash known as
\href{https://www.history.com/news/selma-bloody-sunday-attack-civil-rights-movement}{Bloody
Sunday} took place in Selma, followed by a Selma-to-Montgomery march,
disrupting his original plan. His images of the march never made any
magazine, but decades later, at the urging of his son, Aaron, he turned
them into a gallery show and then a book.

In the meantime he had also become involved with his friend Pete
Seeger's efforts to clean up the Hudson River. He shot photos for a Look
article about the waterway in 1969. He had also taken an interest in the
Native American life and culture in Arizona, where he eventually
settled.

Image

When Mr. Budnik photographed a Navajo protest in New Mexico in 1974, he
was more than just an observer. ``I was, in effect, an unpaid lobbyist
for Native American causes,'' he once said.Credit...Dan Budnik/Contact
Press Images

As with his involvement in the Hudson River campaign, he did more than
take pictures; for example, he helped the Navajo and Hopi peoples resist
strip mining.

``I was, in effect, an unpaid lobbyist for Native American causes,'' he
told Arizona Highways magazine in 2014.

His photographs illustrate ``The Book of Elders: The Life Stories and
Wisdom of Great American Indians'' (1994), a project he created with
Sandy Johnson.

Mr. Budnik's marriage to Toby Gemperle in 1959 ended in divorce in 1962.
His marriage to Kirsten Williams in 1988 ended in divorce in 1990. In
addition to his nephew and his son, from his first marriage, he is
survived by a grandson.

Advertisement

\protect\hyperlink{after-bottom}{Continue reading the main story}

\hypertarget{site-index}{%
\subsection{Site Index}\label{site-index}}

\hypertarget{site-information-navigation}{%
\subsection{Site Information
Navigation}\label{site-information-navigation}}

\begin{itemize}
\tightlist
\item
  \href{https://help.nytimes3xbfgragh.onion/hc/en-us/articles/115014792127-Copyright-notice}{©~2020~The
  New York Times Company}
\end{itemize}

\begin{itemize}
\tightlist
\item
  \href{https://www.nytco.com/}{NYTCo}
\item
  \href{https://help.nytimes3xbfgragh.onion/hc/en-us/articles/115015385887-Contact-Us}{Contact
  Us}
\item
  \href{https://www.nytco.com/careers/}{Work with us}
\item
  \href{https://nytmediakit.com/}{Advertise}
\item
  \href{http://www.tbrandstudio.com/}{T Brand Studio}
\item
  \href{https://www.nytimes3xbfgragh.onion/privacy/cookie-policy\#how-do-i-manage-trackers}{Your
  Ad Choices}
\item
  \href{https://www.nytimes3xbfgragh.onion/privacy}{Privacy}
\item
  \href{https://help.nytimes3xbfgragh.onion/hc/en-us/articles/115014893428-Terms-of-service}{Terms
  of Service}
\item
  \href{https://help.nytimes3xbfgragh.onion/hc/en-us/articles/115014893968-Terms-of-sale}{Terms
  of Sale}
\item
  \href{https://spiderbites.nytimes3xbfgragh.onion}{Site Map}
\item
  \href{https://help.nytimes3xbfgragh.onion/hc/en-us}{Help}
\item
  \href{https://www.nytimes3xbfgragh.onion/subscription?campaignId=37WXW}{Subscriptions}
\end{itemize}
