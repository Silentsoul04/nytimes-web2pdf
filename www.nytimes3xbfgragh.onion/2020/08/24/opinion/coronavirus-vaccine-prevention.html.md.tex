Sections

SEARCH

\protect\hyperlink{site-content}{Skip to
content}\protect\hyperlink{site-index}{Skip to site index}

\href{https://myaccount.nytimes3xbfgragh.onion/auth/login?response_type=cookie\&client_id=vi}{}

\href{https://www.nytimes3xbfgragh.onion/section/todayspaper}{Today's
Paper}

\href{/section/opinion}{Opinion}\textbar{}A Vaccine That Stops Covid-19
Won't Be Enough

\url{https://nyti.ms/2EzpTVa}

\begin{itemize}
\item
\item
\item
\item
\item
\end{itemize}

Advertisement

\protect\hyperlink{after-top}{Continue reading the main story}

\href{/section/opinion}{Opinion}

Supported by

\protect\hyperlink{after-sponsor}{Continue reading the main story}

\hypertarget{a-vaccine-that-stops-covid-19-wont-be-enough}{%
\section{A Vaccine That Stops Covid-19 Won't Be
Enough}\label{a-vaccine-that-stops-covid-19-wont-be-enough}}

The best vaccines don't just prevent a disease; they also prevent the
pathogen causing the disease from being transmitted. So why aren't we
focusing more on those?

By Adam Finn and Richard Malley

Dr. Finn and Dr. Malley are physicians specializing in infectious
diseases and vaccinology.

\begin{itemize}
\item
  Aug. 24, 2020
\item
  \begin{itemize}
  \item
  \item
  \item
  \item
  \item
  \end{itemize}
\end{itemize}

\includegraphics{https://static01.graylady3jvrrxbe.onion/images/2020/08/24/opinion/24malley/24malley-articleLarge.jpg?quality=75\&auto=webp\&disable=upscale}

Not long after the new coronavirus first surfaced last December, an
ambitious prediction was made: A vaccine would be available
\href{https://www.sciencemag.org/news/2020/01/scientists-are-moving-record-speed-create-new-coronavirus-vaccines-they-may-come-too}{within
12 to 18 months}, and it would stop the pandemic.

Despite serious challenges --- how to mass manufacture, supply and
deliver a vaccine worldwide --- the first prong of that wish could well
be fulfilled.
\href{https://www.nytimes3xbfgragh.onion/interactive/2020/science/coronavirus-vaccine-tracker.html}{Eight
vaccine candidates} are undergoing large-scale efficacy tests, so-called
Phase 3 trials, and results are expected by the end of this year or
early 2021.

But even if one, or more, of those efforts succeeds, a vaccine might not
end the pandemic. This is partly because we seem to be focused at the
moment on developing the kind of vaccine that may well prevent Covid-19,
the disease, but that wouldn't do enough to stop the transmission of
SARS-CoV-2, the virus that causes Covid-19.

Doctors \href{https://vk.ovg.ox.ac.uk/vk/how-do-vaccines-work}{usually
explain vaccines} to patients and the parents of young children by
describing how those protect us from a particular disease: An attenuated
form of a pathogen, or just a bit of it, is inoculated into the human
body in order to trigger its immune response; having learned to fight
off that pathogen once, the body will remember how to fend off the
disease should it be exposed to the same pathogen later.

A vaccine's ability to forestall a disease is also how vaccine
developers typically design --- and how regulators typically evaluate
--- Phase 3 clinical trials for vaccine candidates.

Yet the best vaccines also serve another, critical, function: They block
a pathogen's transmission from one person to another. And this result,
often called an
\href{https://www.youtube.com/watch?v=55wOg9fe_Ms}{``indirect'' effect
of vaccination}, is no less important than the direct effect of
preventing the disease caused by that pathogen. In fact, during a
pandemic, it probably is even more important.

That's what we should be focusing on right now. And yet we are not.

Stopping a virus's transmission reduces the entire population's overall
exposure to the virus. It protects people who may be too frail to
respond to a vaccine, who do not have access to the vaccine, who refuse
to be immunized and whose immune response might wane over time.

The benefits of this approach have been demonstrated with other
pathogens and other diseases.

The Haemophilus influenzae type B (Hib) conjugate vaccines were
designed, and licensed in the early 1990s, to prevent young children
from developing
\href{https://journals.lww.com/pidj/Abstract/1988/03000/Efficacy_of_Haemophilus_influenzae_type_b_capsular.3.aspx}{serious
infections} such as meningitis. Soon enough an unexpected and welcome
side benefit became clear: The vaccine
\href{https://academic.oup.com/jid/article-abstract/171/1/93/904530?redirectedFrom=fulltext}{interrupted
the bacterium's transmission}; after its introduction, occurrences of
the disease dropped
also\href{https://www.sciencedirect.com/science/article/pii/S0264410X10004755}{in
groups that had not been vaccinated}.

The human papillomavirus (HPV) vaccines were developed to prevent
cervical cancer and genital warts in women. They have proved
\href{https://www.cdc.gov/vaccines/vpd/hpv/hcp/vaccines.html}{immensely
effective} among the women to whom they are administered --- and up to
50 percent effective at preventing genital warts
\href{https://bmcinfectdis.biomedcentral.com/articles/10.1186/s12879-017-2663-7}{among
unvaccinated men}, according to a 2017 study of the health insurance
records for 2005-10 of some nine million people in Germany.

To understand why this is the case, remember what it takes for you to
become ill from a pathogen, be it a virus or a bacterium.

First, you are exposed to it. Then it infects you. While you are
infected, you may infect others. In some cases, the infection develops
into a disease. In other cases, it doesn't: Though infected, you remain
asymptomatic.

One way that vaccines can interrupt a pathogen's transmission cycle is
by preventing the pathogen from causing an infection in the first place.
This is how many common vaccines --- against
\href{https://www.who.int/immunization/sage/Module_on_Measles_Immunology_26Aug08.pdf?ua=1}{measles},
\href{https://apps.who.int/iris/bitstream/handle/10665/97885/9789241500661_eng.pdf;jsessionid=06BE7879E677A7A70606302A705BA98E?sequence=1}{mumps},
\href{https://apps.who.int/iris/bitstream/handle/10665/43922/9789241596848_eng.pdf?sequence=1}{rubella}
and
\href{https://apps.who.int/iris/bitstream/handle/10665/43906/9789241596770_eng.pdf?sequence=1}{chickenpox}
--- operate.

Other vaccines --- like the ones against
\href{https://academic.oup.com/cid/advance-article-abstract/doi/10.1093/cid/ciaa610/5843568?redirectedFrom=fulltext}{meningococcal
meningitis} or
\href{https://www.ncbi.nlm.nih.gov/pmc/articles/PMC6611220/pdf/main.pdf}{pneumonia
brought on by the pneumococcus bacterium} --- can block the transmission
of the pathogen by interfering with the infection or by decreasing
either the quantity of pathogen that the infected patient sheds or the
duration of the shedding period.

Some recipients of the pneumococcal pneumonia vaccine simply don't get
infected with the bacterium; others do get infected and carry the
bacterium in their nose, but
\href{https://www.atsjournals.org/doi/full/10.1164/rccm.201503-0542OC}{in
smaller amounts and for shorter periods of time} than if they had not
been vaccinated.

Much still needs to be learned about precisely how such mechanisms work
--- what part do
\href{https://www.britannica.com/science/antibody}{antibodies} play?
\href{https://www.britannica.com/science/T-cell}{T cells}? --- but the
upshot from these examples is this: Vaccines can block the transmission
of viruses or bacteria, and they can do so in several ways.

Given the communitywide benefits of accomplishing that, especially in a
pandemic, current vaccine-development efforts should prioritize finding
vaccines that limit the transmission of SARS-CoV-2.

The \href{https://www.fda.gov/media/139638/download}{U.S. Food and Drug
Administration} has stated that preventing a SARS-CoV-2 infection is in
itself a sufficient endpoint for the Phase 3 trials of vaccine
candidates --- that it is an acceptable alternative goal to preventing
the development of Covid-19. The
\href{https://www.who.int/docs/default-source/blue-print/who-target-product-profiles-for-covid-19-vaccines.pdf?sfvrsn=1d5da7ca_5\&download=true}{World
Health Organization} has said that ``shedding/transmission'' is as well.

These guidelines are an important signal, especially considering that
the F.D.A. has never approved a vaccine based on its effects on
infection alone; instead, the agency has focused exclusively on the
vaccine's effectiveness at disease prevention.

And yet vaccine developers do not seem to be heeding this new call.

Based on our review of the Phase 3 tests listed at
\href{https://clinicaltrials.gov/ct2/results?term=vaccine\%2C+phase+3\%2C+efficacy\&recrs=ab\&type=Intr\&cond=covid-19\&fund=2}{ClinicalTrials.gov},
a database of trials conducted around the world, the primary goal in
each of these studies is to reduce the occurrence of Covid-19.

Four of the six Covid-19 vaccine trials for which information is
available say they will also evaluate the incidence of SARS-CoV-2
infections among subjects --- but only as an ancillary outcome.

This approach is shortsighted: One cannot assume that a vaccine that
prevents the development of Covid-19 in a patient will necessarily also
limit the risk that the patient will transmit SARS-CoV-2 to other
people.

For example, a study of young Australian teenagers published in the New
England Journal of Medicine early this year found that the vaccine used
to prevent the diseases caused by the B strain of meningococcus in
children and teenagers
``\href{https://www.nejm.org/doi/full/10.1056/NEJMoa1900236}{had no
discernible effect}'' on the presence of the relevant bacterium in the
throats of vaccinated subjects displaying no symptoms.

The inactivated polio vaccine prevalent in many developed countries
today, known as IPV, is
\href{https://www.cdc.gov/vaccines/vpd/polio/hcp/effectiveness-duration-protection.html\#:~:text=Two\%20doses\%20of\%20inactivated\%20polio,of\%20IPV\%20and\%20tOPV\%2C\%20or}{highly
effective at protecting individuals against polio}. But it is
\href{https://www.who.int/immunization/diseases/poliomyelitis/endgame_objective2/rationale/en/index3.html}{far
less effective at reducing viral shedding}, at least in fecal
excretions, than the oral vaccine, known as OPV, used more widely in
other parts of the world.

In the late 1990s, the United States, like other wealthy countries,
replaced with an acellular vaccine the killed-whole-cell pertussis
vaccine it had previously used against whooping cough. A resurgence of
whooping cough already was underway, but it
\href{https://www.ncbi.nlm.nih.gov/pmc/articles/PMC5580413/pdf/f1000research-6-12588.pdf}{accelerated}
then: Although the new vaccine was better than the previous one at
protecting the inoculated from the disease, it was
\href{https://www.pnas.org/content/111/2/787.long}{less good at blocking
transmission} of the bacterium that causes the cough.

Conversely, a vaccine that, let's say, offers older adults only modest
protection against developing a disease might nonetheless be very
effective, when administered to healthy adults or children, at curbing a
pathogen's transmission in a population overall.

This is the case with the pneumococcal conjugate vaccine.
\href{https://www.nejm.org/doi/full/10.1056/nejmoa1408544}{A 2015 study}
published in the New England Journal of Medicine found that the vaccine
reduced the occurrence of pneumonia in inoculated adults age 65 or older
by only about 45 percent. Yet, according to
\href{https://www.cdc.gov/mmwr/volumes/68/wr/mm6846a5.htm}{a study last
year} by researchers at the Centers for Disease Control and Prevention
and Stanford University, the immunization of infants and toddlers
reduced ninefold the incidence of pneumococcal disease in the elderly.

With some vaccines, for some diseases, the indirect benefits of
vaccination can be greater than the direct effects.

Based on these precedents, it could be a grave mistake for vaccine
developers now to hew only, or too closely, to the single-minded goal of
preventing Covid-19, the disease.

Doing so could mean privileging vaccines that don't block the
transmission of SARS-CoV-2 at all, or abandoning vaccines that block
transmission well enough but that, by prevailing standards, are deemed
to not forestall enough the development of Covid-19.

That, in turn, would essentially mean that the only way to ever get rid
of SARS-CoV-2 would be near-universal immunization --- a herculean task.

Focusing on how to block the coronavirus's transmission is a much more
efficient approach.

This is why randomized controlled trials of the vaccines currently under
consideration should include regular monitoring for the presence of
SARS-CoV-2 in study subjects. The goal should be to evaluate whether the
subjects acquire the infection at all, and for how long, as well as how
abundantly they shed and spread the virus, when and how.

Studying these issues could also help cast a light on the role of
so-called superspreading events in this pandemic.

\href{https://www.scientificamerican.com/article/how-superspreading-events-drive-most-covid-19-spread1/}{More}
and
\href{https://www.nytimes3xbfgragh.onion/2020/08/07/health/coronavirus-superspreading-contagion.html}{more}
research suggests that a very small number of instances --- gatherings
at restaurants or bars,
\href{https://www.nytimes3xbfgragh.onion/2020/05/12/health/coronavirus-choir.html}{choir
rehearsal}, funerals,
\href{https://www.bbc.com/news/world-asia-53620633}{church services} ---
might account for a vast majority of the cases of infection overall.

But the discussion about those instances has tended to focus on their
settings and circumstances, such as the presence of crowds in confined
spaces for extended periods of time.

Yet the question of whether some infected individuals, perhaps
especially at certain stages of infection, are particularly infectious
--- whether they, themselves, are superspreaders --- also needs to be
studied head-on: When does contagiousness peak in whom and why? And can
vaccines modify any of that?

The best vaccines don't just protect the inoculated from getting sick
from a disease. They also protect everyone else from even contracting
the pathogen that causes that disease.

Preventing the very transmission of SARS-CoV-2, no less than stopping it
from turning into Covid-19, should be a main priority of current efforts
to develop the vaccines to end this pandemic.

Adam Finn (\href{https://twitter.com/adamhfinn}{@adamhfinn}) is a senior
clinician in the pediatric immunology and infectious diseases clinical
service at Bristol Royal Hospital for Children and a professor of
pediatrics at the University of Bristol. Richard Malley
(\href{https://twitter.com/rickmalley}{@rickmalley}) is a physician
specializing in infectious diseases at Boston Children's Hospital and a
professor of pediatrics at Harvard Medical School.

\emph{The Times is committed to publishing}
\href{https://www.nytimes3xbfgragh.onion/2019/01/31/opinion/letters/letters-to-editor-new-york-times-women.html}{\emph{a
diversity of letters}} \emph{to the editor. We'd like to hear what you
think about this or any of our articles. Here are some}
\href{https://help.nytimes3xbfgragh.onion/hc/en-us/articles/115014925288-How-to-submit-a-letter-to-the-editor}{\emph{tips}}\emph{.
And here's our email:}
\href{mailto:letters@NYTimes.com}{\emph{letters@NYTimes.com}}\emph{.}

\emph{Follow The New York Times Opinion section on}
\href{https://www.facebookcorewwwi.onion/nytopinion}{\emph{Facebook}}\emph{,}
\href{http://twitter.com/NYTOpinion}{\emph{Twitter (@NYTopinion)}}
\emph{and}
\href{https://www.instagram.com/nytopinion/}{\emph{Instagram}}\emph{.}

Advertisement

\protect\hyperlink{after-bottom}{Continue reading the main story}

\hypertarget{site-index}{%
\subsection{Site Index}\label{site-index}}

\hypertarget{site-information-navigation}{%
\subsection{Site Information
Navigation}\label{site-information-navigation}}

\begin{itemize}
\tightlist
\item
  \href{https://help.nytimes3xbfgragh.onion/hc/en-us/articles/115014792127-Copyright-notice}{©~2020~The
  New York Times Company}
\end{itemize}

\begin{itemize}
\tightlist
\item
  \href{https://www.nytco.com/}{NYTCo}
\item
  \href{https://help.nytimes3xbfgragh.onion/hc/en-us/articles/115015385887-Contact-Us}{Contact
  Us}
\item
  \href{https://www.nytco.com/careers/}{Work with us}
\item
  \href{https://nytmediakit.com/}{Advertise}
\item
  \href{http://www.tbrandstudio.com/}{T Brand Studio}
\item
  \href{https://www.nytimes3xbfgragh.onion/privacy/cookie-policy\#how-do-i-manage-trackers}{Your
  Ad Choices}
\item
  \href{https://www.nytimes3xbfgragh.onion/privacy}{Privacy}
\item
  \href{https://help.nytimes3xbfgragh.onion/hc/en-us/articles/115014893428-Terms-of-service}{Terms
  of Service}
\item
  \href{https://help.nytimes3xbfgragh.onion/hc/en-us/articles/115014893968-Terms-of-sale}{Terms
  of Sale}
\item
  \href{https://spiderbites.nytimes3xbfgragh.onion}{Site Map}
\item
  \href{https://help.nytimes3xbfgragh.onion/hc/en-us}{Help}
\item
  \href{https://www.nytimes3xbfgragh.onion/subscription?campaignId=37WXW}{Subscriptions}
\end{itemize}
