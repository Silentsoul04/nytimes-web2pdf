Sections

SEARCH

\protect\hyperlink{site-content}{Skip to
content}\protect\hyperlink{site-index}{Skip to site index}

\href{https://myaccount.nytimes3xbfgragh.onion/auth/login?response_type=cookie\&client_id=vi}{}

\href{https://www.nytimes3xbfgragh.onion/section/todayspaper}{Today's
Paper}

Rethinking Who and What Get Memorialized

\url{https://nyti.ms/3goPaPn}

\begin{itemize}
\item
\item
\item
\item
\item
\end{itemize}

Advertisement

\protect\hyperlink{after-top}{Continue reading the main story}

Supported by

\protect\hyperlink{after-sponsor}{Continue reading the main story}

\hypertarget{rethinking-who-and-what-get-memorialized}{%
\section{Rethinking Who and What Get
Memorialized}\label{rethinking-who-and-what-get-memorialized}}

The notion that history can be rewritten is a powerful one. It starts by
taking the pen from the authors we've always had --- and giving it to
someone else.

\includegraphics{https://static01.graylady3jvrrxbe.onion/images/2020/08/30/t-magazine/30tmag-monuments-slide-89UE/30tmag-monuments-slide-89UE-articleLarge.jpg?quality=75\&auto=webp\&disable=upscale}

By \href{https://www.nytimes3xbfgragh.onion/by/hanya-yanagihara}{Hanya
Yanagihara}

Aug. 24, 2020

This is a magazine devoted to documenting visual culture, whether those
visuals come from the worlds of fashion, theater, film, architecture or
art. Some of the most striking visuals from the past half-year in
America --- and there have been many --- have concerned the removal or
defacement of monuments dedicated to figures from the Confederacy, whose
placement in public spaces mostly in the South are, for many of us, an
insult, if not an assault. I am not Black, but I keenly remember the
discomfort and inexplicable shame I often felt while attending my
freshman-year public high school, Robert E. Lee, in Tyler, Texas. Lee
was a Confederate general, a slave owner and a commander in the
Confederate Army during the American Civil War. (The middle school I
attended in Tyler, Hubbard, was also named for a former Confederate
officer, who later became the state's governor.) A large portrait of Lee
hung in the school gym, and whenever I saw it, I felt as if my right to
this country, to Americanness, was being challenged. Why, more than a
century later, was someone who had fought for something this country now
purported to be against still being honored?

I was too cowardly to ever consider articulating my anger, but this past
June, a track star at the school, a young Black woman named
\href{https://www.cnn.com/2020/06/24/us/robert-e-lee-high-school-tyler-texas-trnd/index.html}{Trude
Lamb}, announced that she refused to participate in athletic events as
long as she had to wear a jersey printed with Lee's name; last month,
the town's school board voted to change the name of both that high
school as well as another secondary school, John Tyler, named for the
United States' 10th president --- also a slave owner and a member of the
Confederate Congress. (The name of the town itself, which also
memorializes Tyler, thus far remains unchanged.)

Lamb is just one reminder of how everyday courage can change the things
we have grown to, if not accept, then tolerate. She is also a reminder
of what we should not be asking our citizens to tolerate in the first
place: that cognitive dissonance we live with daily, in which some of us
are given the privilege of knowing that this country will treat us
fairly and honor our past, and others are not. For our story
``\href{https://www.nytimes3xbfgragh.onion/2020/08/24/t-magazine/confederate-monuments-reimagined-racism.html}{America's
Monuments, Reimagined for a More Just Future},'' we asked five
contemporary artists to create a monument for America today: for an
America as it was, as it should be and as it could be. One of the
artists chose to imagine a statue, to celebrate a person whose
significance and accomplishments were ignored and neglected. Others
conceived of America itself as a monument, assigning the very land for
different uses. Such a notion suggests that any nation is a palimpsest,
one whose history can be rewritten, again and again. It's a powerful
idea. But it starts by first taking the pen from the authors we've
always had --- and giving it to someone else.

Advertisement

\protect\hyperlink{after-bottom}{Continue reading the main story}

\hypertarget{site-index}{%
\subsection{Site Index}\label{site-index}}

\hypertarget{site-information-navigation}{%
\subsection{Site Information
Navigation}\label{site-information-navigation}}

\begin{itemize}
\tightlist
\item
  \href{https://help.nytimes3xbfgragh.onion/hc/en-us/articles/115014792127-Copyright-notice}{©~2020~The
  New York Times Company}
\end{itemize}

\begin{itemize}
\tightlist
\item
  \href{https://www.nytco.com/}{NYTCo}
\item
  \href{https://help.nytimes3xbfgragh.onion/hc/en-us/articles/115015385887-Contact-Us}{Contact
  Us}
\item
  \href{https://www.nytco.com/careers/}{Work with us}
\item
  \href{https://nytmediakit.com/}{Advertise}
\item
  \href{http://www.tbrandstudio.com/}{T Brand Studio}
\item
  \href{https://www.nytimes3xbfgragh.onion/privacy/cookie-policy\#how-do-i-manage-trackers}{Your
  Ad Choices}
\item
  \href{https://www.nytimes3xbfgragh.onion/privacy}{Privacy}
\item
  \href{https://help.nytimes3xbfgragh.onion/hc/en-us/articles/115014893428-Terms-of-service}{Terms
  of Service}
\item
  \href{https://help.nytimes3xbfgragh.onion/hc/en-us/articles/115014893968-Terms-of-sale}{Terms
  of Sale}
\item
  \href{https://spiderbites.nytimes3xbfgragh.onion}{Site Map}
\item
  \href{https://help.nytimes3xbfgragh.onion/hc/en-us}{Help}
\item
  \href{https://www.nytimes3xbfgragh.onion/subscription?campaignId=37WXW}{Subscriptions}
\end{itemize}
