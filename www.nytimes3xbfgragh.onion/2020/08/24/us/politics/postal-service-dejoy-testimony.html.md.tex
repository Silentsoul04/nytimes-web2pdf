Sections

SEARCH

\protect\hyperlink{site-content}{Skip to
content}\protect\hyperlink{site-index}{Skip to site index}

\href{https://www.nytimes3xbfgragh.onion/section/politics}{Politics}

\href{https://myaccount.nytimes3xbfgragh.onion/auth/login?response_type=cookie\&client_id=vi}{}

\href{https://www.nytimes3xbfgragh.onion/section/todayspaper}{Today's
Paper}

\href{/section/politics}{Politics}\textbar{}Key Highlights of Louis
DeJoy's Testimony to a House Panel

\url{https://nyti.ms/3j49ZRG}

\begin{itemize}
\item
\item
\item
\item
\item
\item
\end{itemize}

\begin{itemize}
\item
  \href{https://www.nytimes3xbfgragh.onion/interactive/2020/09/08/us/elections/results-new-hampshire-primary-elections.html?action=click\&pgtype=Article\&state=default\&region=TOP_BANNER\&context=storylines_menu}{New
  Hampshire Results}
\item
  \href{https://www.nytimes3xbfgragh.onion/live/2020/09/08/us/trump-vs-biden?action=click\&pgtype=Article\&state=default\&region=TOP_BANNER\&context=storylines_menu}{Election
  Updates}
\item
  \href{https://www.nytimes3xbfgragh.onion/interactive/2020/us/elections/election-states-biden-trump.html?action=click\&pgtype=Article\&state=default\&region=TOP_BANNER\&context=storylines_menu}{Paths
  to 270}
\item
  \href{https://www.nytimes3xbfgragh.onion/interactive/2020/08/31/us/politics/vote-by-mail-deadlines.html?action=click\&pgtype=Article\&state=default\&region=TOP_BANNER\&context=storylines_menu}{Voting
  by Mail}
\item
  \href{https://www.nytimes3xbfgragh.onion/interactive/2019/us/elections/2020-presidential-election-calendar.html?action=click\&pgtype=Article\&state=default\&region=TOP_BANNER\&context=storylines_menu}{Key
  Dates}
\item
  \href{https://www.nytimes3xbfgragh.onion/newsletters/politics?action=click\&pgtype=Article\&state=default\&region=TOP_BANNER\&context=storylines_menu}{Politics
  Newsletter}
\end{itemize}

Advertisement

\protect\hyperlink{after-top}{Continue reading the main story}

Supported by

\protect\hyperlink{after-sponsor}{Continue reading the main story}

\hypertarget{key-highlights-of-louis-dejoys-testimony-to-a-house-panel}{%
\section{Key Highlights of Louis DeJoy's Testimony to a House
Panel}\label{key-highlights-of-louis-dejoys-testimony-to-a-house-panel}}

Louis DeJoy testified before the House Oversight Committee on Monday,
where lawmakers continued to express concern about postal changes that
could complicate mail-in voting.

By \href{https://www.nytimes3xbfgragh.onion/by/catie-edmondson}{Catie
Edmondson},
\href{https://www.nytimes3xbfgragh.onion/by/nicholas-fandos}{Nicholas
Fandos},
\href{https://www.nytimes3xbfgragh.onion/by/alan-rappeport}{Alan
Rappeport},
\href{https://www.nytimes3xbfgragh.onion/by/luke-broadwater}{Luke
Broadwater} and
\href{https://www.nytimes3xbfgragh.onion/by/emily-cochrane}{Emily
Cochrane}

\begin{itemize}
\item
  Aug. 24, 2020
\item
  \begin{itemize}
  \item
  \item
  \item
  \item
  \item
  \item
  \end{itemize}
\end{itemize}

\hypertarget{heres-what-you-need-to-know}{%
\subsubsection{Here's what you need to
know:}\label{heres-what-you-need-to-know}}

\begin{itemize}
\tightlist
\item
  \protect\hyperlink{link-25c465bd}{DeJoy said he was not trying to
  sabotage the election.}
\item
  \protect\hyperlink{link-674e5e34}{Democrats sparred with DeJoy over
  the removal of sorting machines and other changes.}
\item
  \protect\hyperlink{link-1b3ec52f}{Maloney accused DeJoy of carrying
  out Trump's efforts to hobble mail-in voting.}
\item
  \protect\hyperlink{link-3c1f694a}{Republicans framed DeJoy as a victim
  of `cancel culture.'}
\item
  \protect\hyperlink{link-582bf645}{Trump continues to suggest `fraud'
  in 2020 mail-in voting.}
\item
  \protect\hyperlink{link-36448afd}{Democrats grill DeJoy over his
  qualifications.}
\item
  \protect\hyperlink{link-460d7a51}{DeJoy gets a pop quiz.}
\end{itemize}

\hypertarget{dejoy-said-he-was-not-trying-to-sabotage-the-election}{%
\subsection{DeJoy said he was not trying to sabotage the
election.}\label{dejoy-said-he-was-not-trying-to-sabotage-the-election}}

A defiant
\href{https://www.nytimes3xbfgragh.onion/2020/08/24/us/politics/louis-dejoy-post-office-hearing.html}{Louis
DeJoy}, under tough questioning from Democrats, defended the
\href{https://www.nytimes3xbfgragh.onion/2020/08/15/us/post-office-vote-by-mail.html}{cost-cutting
measures} he put in place as postmaster general and rejected suggestions
that the changes were intended to influence the 2020 election by making
mail-in voting less reliable.

``I am not engaged in sabotaging the election,'' an increasingly
exasperated Mr. DeJoy told Democrats. ``We will do everything in our
power and structure to deliver the ballots on time.''

Mr. DeJoy, a megadonor to President Trump, is embroiled in a political
firestorm as recent changes aimed at reducing the Postal Service's costs
--- including cutting overtime and limiting trips --- have led to delays
in mail delivery, including medicine, pension checks and bills. That has
fueled concerns about whether the service will be able to handle what is
expected to be a record number of mail-in ballots for the 2020 election.

Mr. DeJoy, at times shouting over his Democratic questioners, criticized
the ``false narrative'' that he said was being promoted about both his
intentions and the changes, which he described as necessary to address
the Postal Service's financial woes but that civil rights groups, state
attorneys general and Democrats have derided as an attempt to
disenfranchise voters.

Mr. DeJoy told House lawmakers that he did not put in place many of the
changes that had caused concern --- including the removal of blue
mailboxes and mail-sorting machines --- but said he did not know who
ordered those operational changes.

He acknowledged that some of the changes he had implemented, such as
reducing overtime and limiting trips, had caused delays but said those
were necessary and issues that arose from them were being rectified.

``Transitions don't always go smoothly,'' he said, adding that while
``we are very concerned with the deterioration in service, we're seeing
a big recovery this week.''

As the hearing wore on, and Mr. DeJoy found himself being asked to tread
the same ground again and again, he appeared to grow weary of trying to
prove that his intentions were good.

His decision to cut down on late and extra shipments of mail was based
on a ``fundamental, basic principle: Run your trucks on time.''

``I would not know how to reverse that now,'' he said. ``Am I to say do
not run the trucks on time? Is that the answer that we are trying to get
me to say here today?''

Pressed by Representative John Sarbanes, Democrat of Maryland, to put in
writing his assurances to get mail service, including mail-in ballots,
moving on time, Mr. DeJoy said he would commit to giving the Oversight
committee ``an update on the improvement of the service'' by next
Monday.

He had previously demurred or declined to share other analyses or
internal documents with Congress.

\hypertarget{democrats-sparred-with-dejoy-over-the-removal-of-sorting-machines-and-other-changes}{%
\subsection{Democrats sparred with DeJoy over the removal of sorting
machines and other
changes.}\label{democrats-sparred-with-dejoy-over-the-removal-of-sorting-machines-and-other-changes}}

Democrats attacked the postal head for making changes in the midst of a
pandemic and so close to the election but Mr. DeJoy insisted the timing
was appropriate.

``It was summertime, mail volume was down significantly, we're getting
ready for the peak season and the election is three months away,'' he
said. ``It was a good time to try and start to roll this out.''

Many Democrats were not satisfied with that explanation and the hearing
turned testy when Representative Stephen F. Lynch, Democrat of
Massachusetts, accused Mr. DeJoy of ``incompetence'' and asked ``how can
one person screw this up in just a few weeks?''

After repeatedly asking whether Mr. DeJoy would return the mail sorting
machines that have already been removed from post offices, the postal
leader barked ``I will not.'' He then added that Mr. Lynch had spread
``misinformation'' during his furious monologue.

The postal head grew increasingly agitated under additional questioning
from Representative Ro Khanna, Democrat of California, about putting
decommissioned sorting machines back in service.

After Mr. DeJoy acknowledged that replacing those decommissioned sorting
machines would likely cost less than \$1 billion, Mr. Khanna pressed him
on why the agency wouldn't consider replacing those machines before the
general election in November to help instill confidence in mail-in
voting and the election results.

``If it would cost less than a billion dollars regardless of efficient
or not, what is the harm in putting those machines back?'' Mr. Khanna
asked. ``Just for the peace of mind and the confidence of the American
people.''

Mr. DeJoy described the question of whether the Postal Service would
restore machines if the agency was appropriated \$1 billion ``a
hypothetical'' and criticized the ability of Congress to legislate.

``You're not going to give us \$1 billion,'' Mr. DeJoy said. ``You have
no way of getting us \$1 billion. We haven't been funded in 10 years.
You can't pass any legislation.''

When Mr. Khanna did not relent, a visibly exasperated Mr. DeJoy declared
``get me the billion and I'll put the machines in.'' Mr. Khanna took
that as a ``commitment'' to replace those machines.

\hypertarget{maloney-accused-dejoy-of-carrying-out-trumps-efforts-to-hobble-mail-in-voting}{%
\subsection{Maloney accused DeJoy of carrying out Trump's efforts to
hobble mail-in
voting.}\label{maloney-accused-dejoy-of-carrying-out-trumps-efforts-to-hobble-mail-in-voting}}

\includegraphics{https://static01.graylady3jvrrxbe.onion/images/2020/08/24/us/politics/24dc-postal5/merlin_176110779_0cda3df2-2c8d-43d0-93d0-246ddb08ae69-articleLarge.jpg?quality=75\&auto=webp\&disable=upscale}

In searing opening remarks, Representative Carolyn B. Maloney, Democrat
of New York and the chairwoman of the House Oversight Committee, took
aim at Mr. DeJoy, calling the notion that the changes he implemented at
the agency would not cause mail delays ``incompetence at best.''

``Maybe Mr. DeJoy was warned that his changes would cause delays, but he
disregarded those warnings,'' Ms. Maloney said. ``Or, perhaps there is a
far simpler explanation. Perhaps Mr. DeJoy is just doing exactly what
President Trump said he wanted on national television --- using the
blocking of funds to justify sweeping changes to hobble mail-in
voting.''

``All of these options are bad,'' she concluded.

\hypertarget{republicans-framed-dejoy-as-a-victim-of-cancel-culture}{%
\subsection{Republicans framed DeJoy as a victim of `cancel
culture.'}\label{republicans-framed-dejoy-as-a-victim-of-cancel-culture}}

In his introduction of Mr. DeJoy, Representative Mark Walker, Republican
of North Carolina, accused Democrats of trying to ``cancel'' Mr. DeJoy
for purely partisan reasons.

``How sad is it when the cancel culture has reached the halls of
congress,'' Mr. Walker said. ``The man sitting before this committee
today is not who the Democrats have villainized him to be. He's here
today because he supported President Trump.''

Other Republicans on the committee struck a similar tone. Representative
James R. Comer of Kentucky, the top Republican on the committee, used
his opening remarks to chastise Democrats for spreading a ``baseless
conspiracy theory about the Postal Service'' and hastily moving over the
weekend to
\href{https://www.nytimes3xbfgragh.onion/2020/08/22/us/politics/usps-bill-congress-vote.html}{pass
legislation} that would block some service changes and send the
beleaguered agency \$25 billion.

He said Mr. DeJoy was taking good-faith steps to cut costs and noted
that he took responsibility for the delays --- what Mr. Comer called
``temporary growing pains.''

He pointed out that the recent removal of blue mailboxes and mail
sorting machines were long planned and consistent with actions taken
under Mr. DeJoy's predecessor.

``I am disappointed in the hysterical frenzy being whipped up by the
Democrats and their friends in the media,'' Mr. Comer said, adding that
the Postal Service ``has more than adequate capacity to handle the vote
by mail.''

Representative Jody Hice, Republican of Georgia, blamed the mail delays
on the pandemic, saying ``there are thousands of U.S.P.S. workers who
are not showing up for work due to Covid-19'' adding that ``the
postmaster general has nothing to do with Covid-19.''

Mr. DeJoy, under questioning from Representative Ayanna Pressley,
Democrat of Massachusetts, said that 83 Postal Service employees had
died from the coronavirus. He agreed to provide a breakdown of the
number of Postal Service employees impacted by the pandemic to Congress
by Friday.

Image

Postal service employees preparing delivery trucks at a post office in
Los Angeles last week.Credit...Kyle Grillot/Agence France-Presse ---
Getty Images

\hypertarget{trump-continues-to-suggest-fraud-in-2020-mail-in-voting}{%
\subsection{Trump continues to suggest `fraud' in 2020 mail-in
voting.}\label{trump-continues-to-suggest-fraud-in-2020-mail-in-voting}}

Mr. Trump weighed in during the hearing, reiterating claims he has
previously made --- without evidence --- that mail-in voting will lead
to fraud.

``All the Radical Left Democrats are trying to do with the Post Office
hearings is blame the Republicans for the FRAUD that will occur because
of the 51 Million Ballots that are being sent to people who have not
even requested them,'' he wrote on Twitter. ``They are setting the table
for a BIG MESS!''

Mr. Trump has stoked concerns about the validity of mail-in voting ahead
of the 2020 election, saying last week that allowing universal voting by
mail would lead to people fraudulently casting multiple ballots --- a
practice that experts say has been exceedingly rare in places where
mail-in voting has been common for many years.

Mr. DeJoy acknowledged on Monday that he did not find those comments
helpful.

``I have put word around to different people that this is not helpful,''
he said, in response to a question from Mr. Connolly about whether he
had been in contact with the administration or Mr. Trump about his
attacks.

\hypertarget{democrats-grill-dejoy-over-his-qualifications}{%
\subsection{Democrats grill DeJoy over his
qualifications.}\label{democrats-grill-dejoy-over-his-qualifications}}

Democrats pressed Mr. DeJoy, a logistics executive whose name was not on
an initial list of candidates provided to the Postal Service's board of
governors, about how he was selected to run the Postal Service.

As The Times reported on Saturday, Treasury Secretary Steven Mnuchin was
a key player in selecting the board members who hired Mr. DeJoy and in
pushing the agenda that he has pursued.

\emph{{[}Read more about Treasury Secretary Steven Mnuchin's}
\href{https://www.nytimes3xbfgragh.onion/2020/08/22/business/economy/dejoy-postmaster-general-trump-mnuchin.html}{\emph{role
in the selection process}}\emph{.{]}}

In a fiery exchange with Representative Jamie Raskin, a Maryland
Democrat, Mr. DeJoy insisted that Mr. Mnuchin ``had nothing to do with
my selection'' and that he only met with the Treasury secretary after he
was offered the position.

Mr. DeJoy also defended his background as a logistics executive and
insisted that he was properly vetted for the position of postmaster
general. However, when asked if he would share the file from his
background check, he rebuffed the notion.

``No, why would I release a background check,'' Mr. DeJoy said
incredulously.

Late last week, David C. Williams, the former vice chairman of the board
of governors, who was appointed by Mr. Trump as a Democratic member of
the panel,
\href{https://www.nytimes3xbfgragh.onion/2020/08/20/us/politics/former-postal-governor-tells-congress-mnuchin-politicized-postal-service.html}{told
House Democrats in scathing testimony} that Mr. DeJoy was the least
qualified candidate the board interviewed for the job, and that Robert
M. Duncan, the chairman of the Postal Service board of governors, had
suggested him to the panel.

Mr. Williams also accused Mr. Mnuchin of politicizing the Postal
Service, an independent agency whose leader has been walled off from the
White House since 1970.

When he resigned from the board in protest on the eve of Mr. DeJoy's
selection, Mr. Williams said that no serious background investigation
into the candidate had been conducted --- despite his request for one
--- and that a brief review by the agency's inspector general had
surfaced potential concerns about
\href{https://www.nytimes3xbfgragh.onion/2020/08/17/us/politics/dejoy-postal-service-mail-in-voting.html}{contract
work Mr. DeJoy's logistics firm} had done for the Postal Service.

\hypertarget{dejoy-gets-a-pop-quiz}{%
\subsection{DeJoy gets a pop quiz.}\label{dejoy-gets-a-pop-quiz}}

As the hearing wound toward a close, Democrats gave Mr. DeJoy a pop quiz
about the agency he leads.

Representative Katie Porter, Democrat of California, asked the
postmaster general the price of a first-class stamp.

``Fifty-five cents,'' he said correctly. ``Just wanted to check,'' Ms.
Porter replied.

``What about to mail a postcard?'' Mr. DeJoy was stumped.

``You don't know the cost to mail a postcard,'' Ms. Porter said. (A post
card costs 35 cents to mail.)

``I don't,'' he said, adding, after another question about a different
type of mail, ``I'll submit that I know very little about a postage
stamp.''

Mr. DeJoy also could not say how many Americans had voted by mail in the
2016 election.

``I'm glad you know the price of a stamp, but I'm concerned about your
understanding of this agency,'' Mr. Porter said. ``And I am particularly
concerned about it because you started taking very decisive action when
you became postmaster general.''

\hypertarget{duncan-acknowledged-dejoy-was-not-on-the-initial-candidate-list}{%
\subsection{Duncan acknowledged DeJoy was not on the initial candidate
list.}\label{duncan-acknowledged-dejoy-was-not-on-the-initial-candidate-list}}

Image

Robert M. Duncan, the chairman of the Postal Service board of governors,
testifying remotely.Credit...Pool photo by Tom Brenner

Mr. Duncan, the chairman of the Postal Service's board of governors who
also appeared before the committee, acknowledged that Mr. DeJoy was not
included in a list of candidates provided by a search firm.

Mr. Duncan said he raised Mr. DeJoy's name as a candidate after the
search firm, Russell Reynolds Associates, provided the board with an
initial list of 53 names that did not include Mr. DeJoy, confirming a
\href{https://slack-redir.net/link?url=https\%3A\%2F\%2Fwww.nytimes3xbfgragh.onion\%2F2020\%2F08\%2F19\%2Fbusiness\%2Feconomy\%2Fpostal-service-changes-dejoy.html}{report}
in The New York Times. He said he offered Mr. DeJoy's name and others in
an effort to expand the search pool in order to arrive at the best
candidate.

``It was during that period of time that Mr. DeJoy's interest or
availability became known to me,'' Mr. Duncan said under questioning
from Raja Krishnamoorthi, an Illinois Democrat. ``I submitted that name
as I had many others.''

Under questioning from Ms. Pressley, Mr. Duncan said he had no knowledge
of issues surrounding Mr. DeJoy's former companies, including that New
Breed, which he ran for 30 years before selling it to XPO Logistics in
2014, was criticized by the National Labor Relations Board for being
anti-union. He also said he was not that aware that New Breed had to pay
\$1.5 million after losing a lawsuit about sexual harassment in the
workplace in 2013.

Offering little explanation, Mr. Duncan said that ``various background
checks'' were done on Mr. DeJoy.

Mr. Duncan defended selecting Mr. DeJoy to helm the agency, citing his
``deep knowledge'' of the Postal Service from his work in the private
sector and casting the changes he had implemented as crucial to
transforming a flagging institution.

Mr. Krishnamoorthi also asked Mr. Duncan, a former chairman of the
Republican National Committee and a board member for Republican super
PACs, to account for past statements in which he accused Democrats of
trying to steal elections.

Mr. Krishnamoorthi quoted a
\href{https://slack-redir.net/link?url=https\%3A\%2F\%2Fwww.cnn.com\%2F2008\%2FPOLITICS\%2F11\%2F14\%2Frnc.chair\%2Findex.html}{2008
fund-raising mailer signed} by Mr. Duncan, then the R.N.C. chairman,
asserting ``the Obama-Biden Democrats and their liberal special interest
allies are trying to steal these election victories from Republicans.''

Mr. Duncan said he did not recall that letter. But he distanced himself
from its message and from Mr. Trump's claims --- and perhaps implicitly
from Democratic claims about Mr. Trump's intervention in postal affairs
--- explaining, ``I don't believe anyone at this point who is a nominee
of the major parties is trying to steal an election.''

\hypertarget{dejoys-ties-to-trump-provide-ammunition-for-democrats}{%
\subsection{DeJoy's ties to Trump provide ammunition for
Democrats.}\label{dejoys-ties-to-trump-provide-ammunition-for-democrats}}

Mr. DeJoy's political donations to Trump and his ties to the president
became a prominent line of attack for Democrats during the hearing.

Representative Peter Welch, Democrat of Vermont, detailed Mr. DeJoy's
campaign contributions to Republicans, including Mr. Trump, and noted he
gave no money to Democrats.

``Yes, I am a Republican, sir,'' Mr. DeJoy said. ``I give a lot of money
to Republicans.''

Representative Jim Cooper, Democrat of Tennessee, pointedly --- and at
times without evidence or explanation --- insinuated Mr. DeJoy may be
responsible for a pattern of illegality in service of Mr. Trump's
attacks on mail-in voting.

He suggested that Mr. DeJoy's orders for mail trucks to run on time,
even if they were empty, could be considered a felony for delaying the
mail. He asserted that
\href{https://slack-redir.net/link?url=https\%3A\%2F\%2Fwww.nytimes3xbfgragh.onion\%2F2020\%2F08\%2F17\%2Fus\%2Fpolitics\%2Fdejoy-postal-service-mail-in-voting.html}{Mr.
DeJoy's continued financial stake in a company that does business with
the Postal Service} was a conflict of interest. He asked, without
providing any evidence or background for his claim, whether Mr. DeJoy
had facilitated campaign contributions to the Trump campaign in a manner
that would violate so-called straw donation prohibitions and then
asserted that his operational changes at the Postal Service amounted to
an in-kind contribution to the Trump campaign because of the delays they
had caused in the run-up to a major election.

``Do your mail delays help Trump campaign goals of hurting the post
office as stated in his tweets,'' Mr. Cooper asked. ``Are your mail
delays implicit contributions?''

Mr. DeJoy was furious. He said he was not above the law, maintained he
was fully complaint with ethics requirements, and called questions about
illegal campaign donations ``outrageous'' and false.

``I'm not answering these types of questions,'' he said near the end of
the exchange. ``I'm here to represent the Postal Service. All my actions
have to do with improvements to the Postal Service. Am I the only one in
this room who understand that we have a \$10 billion a year loss? Am I
the only one in this room who has looked at the O.I.G. reports that have
stacked up?''

Mr. Cooper said he was only asking questions about potential campaign
finance violations. But he pressed on, drawing moans from Republicans in
the hearing room: ``Is your backup plan to be pardoned like Roger
Stone?''

Mr. DeJoy laughed. ``I have no comment on that,'' he said. ``It's not
worth the comment.''

\hypertarget{lawmakers-question-dejoy-over-conflicts-of-interest}{%
\subsection{Lawmakers question DeJoy over conflicts of
interest}\label{lawmakers-question-dejoy-over-conflicts-of-interest}}

Mr. DeJoy's ongoing financial ties to companies that contract with the
Postal Service also came under scrutiny from Democrats.

Mr. DeJoy continues to hold \$25 million to \$50 million in XPO
Logistics, a \$16 billion transportation company where he served as the
chief executive of its supply chain business until 2015 and was a board
member until 2018. And he continues to
\href{https://www.nytimes3xbfgragh.onion/2020/08/17/us/politics/dejoy-postal-service-mail-in-voting.html}{earn
millions of dollars more in rental payments} from XPO through leasing
agreements at buildings that he owns, according to his financial
disclosure forms.

XPO assists the Postal Service during busy shipping periods, such as
around the holidays, moving bulk shipments of packages from fulfillment
centers and taking them to local Postal Service centers so mail carriers
can deliver them to residences.

Mr. DeJoy defended his continued investment in XPO Logistics.

``I have a significant investment in XPO logistics, which I vetted
before with the ethics department of the Postal Service and I was given
specific types of guidelines that I needed to adhere to,'' he told Mr.
Raskin. ``It's a very, very small part of the Postal Service business I
have nothing to do with. I comply with all ethical requirements and we
have an OIG investigation. I guess they'll get to everything that you're
interested in and we will see what will happen.''

\hypertarget{our-2020-election-guide}{%
\section{Our 2020 Election Guide}\label{our-2020-election-guide}}

Updated ~Sept. 8, 2020

\begin{center}\rule{0.5\linewidth}{\linethickness}\end{center}

\begin{itemize}
\item ~
  \hypertarget{the-latest}{%
  \subsection{The Latest}\label{the-latest}}

  \begin{itemize}
  \item
    President Trump and his party are using a playbook that aims to
    alarm people about crime in their backyards. It didn't work in 2018,
    but
    \href{https://www.nytimes3xbfgragh.onion/2020/09/08/us/politics/trump-republicans-fear-strategy.html?action=click\&pgtype=Article\&state=default\&region=BELOW_MAIN_CONTENT\&context=storylines_guide}{both
    parties think it could resonate more this year}.
  \end{itemize}
\item ~
  \hypertarget{how-to-win-270}{%
  \subsection{How to Win 270}\label{how-to-win-270}}

  \begin{itemize}
  \item
    Joe Biden and Donald Trump need 270 electoral votes to reach the
    White House. Try building
    \href{https://www.nytimes3xbfgragh.onion/interactive/2020/us/elections/election-states-biden-trump.html?action=click\&pgtype=Article\&state=default\&region=BELOW_MAIN_CONTENT\&context=storylines_guide}{your
    own coalition of battleground states}~to see potential outcomes.
  \end{itemize}
\item ~
  \hypertarget{voting-by-mail}{%
  \subsection{Voting by Mail}\label{voting-by-mail}}

  \begin{itemize}
  \item
    Will you have enough time to vote by mail in your state? Yes, but
    it's risky to procrastinate.
    \href{https://www.nytimes3xbfgragh.onion/interactive/2020/08/31/us/politics/vote-by-mail-deadlines.html?action=click\&pgtype=Article\&state=default\&region=BELOW_MAIN_CONTENT\&context=storylines_guide}{Check
    your state's deadline.}
  \item
    \href{https://www.nytimes3xbfgragh.onion/interactive/2020/us/elections/joe-biden.html?action=click\&pgtype=Article\&state=default\&region=BELOW_MAIN_CONTENT\&context=storylines_guide}{}

    \hypertarget{joe-biden}{%
    \section{Joe Biden}\label{joe-biden}}

    \hypertarget{democrat}{%
    \subsection{Democrat}\label{democrat}}

    \href{https://www.nytimes3xbfgragh.onion/interactive/2020/us/elections/donald-trump.html?action=click\&pgtype=Article\&state=default\&region=BELOW_MAIN_CONTENT\&context=storylines_guide}{}

    \hypertarget{donald-trump}{%
    \section{Donald Trump}\label{donald-trump}}

    \hypertarget{republican}{%
    \subsection{Republican}\label{republican}}
  \end{itemize}
\item
  \hypertarget{keep-up-with-our-coverage}{%
  \subsection{Keep Up With Our
  Coverage}\label{keep-up-with-our-coverage}}

  \begin{itemize}
  \item
    Get an
    \href{https://www.nytimes3xbfgragh.onion/newsletters/politics?action=click\&pgtype=Article\&state=default\&region=BELOW_MAIN_CONTENT\&context=storylines_guide}{email}~recapping
    the day's news
  \item
    Download our mobile app on
    \href{https://apps.apple.com/us/app/nytimes/id284862083?ls=1\&mat_click_id=5c79ae7455014fd1bd66b5610c05b8f2-20191112-16948\&referrer=mat_click_id\%3D5c79ae7455014fd1bd66b5610c05b8f2-20191112-16948\%26link_click_id\%3D722930677036718082}{iOS}~and
    \href{http://a.localytics.com/android?id=com.nytimes.android\&referrer=utm_source\%3Dother_nyt_mobile_web\%26utm_medium\%3DWeb\%2520page\%26utm_term\%3DGeneral\%2520Mobile\%2520Page\%26utm_campaign\%3DNYT\%2520Mobile\%2520General\%2520Page}{Android}~and
    turn on Breaking News and Politics alerts
  \end{itemize}
\end{itemize}

Advertisement

\protect\hyperlink{after-bottom}{Continue reading the main story}

\hypertarget{site-index}{%
\subsection{Site Index}\label{site-index}}

\hypertarget{site-information-navigation}{%
\subsection{Site Information
Navigation}\label{site-information-navigation}}

\begin{itemize}
\tightlist
\item
  \href{https://help.nytimes3xbfgragh.onion/hc/en-us/articles/115014792127-Copyright-notice}{©~2020~The
  New York Times Company}
\end{itemize}

\begin{itemize}
\tightlist
\item
  \href{https://www.nytco.com/}{NYTCo}
\item
  \href{https://help.nytimes3xbfgragh.onion/hc/en-us/articles/115015385887-Contact-Us}{Contact
  Us}
\item
  \href{https://www.nytco.com/careers/}{Work with us}
\item
  \href{https://nytmediakit.com/}{Advertise}
\item
  \href{http://www.tbrandstudio.com/}{T Brand Studio}
\item
  \href{https://www.nytimes3xbfgragh.onion/privacy/cookie-policy\#how-do-i-manage-trackers}{Your
  Ad Choices}
\item
  \href{https://www.nytimes3xbfgragh.onion/privacy}{Privacy}
\item
  \href{https://help.nytimes3xbfgragh.onion/hc/en-us/articles/115014893428-Terms-of-service}{Terms
  of Service}
\item
  \href{https://help.nytimes3xbfgragh.onion/hc/en-us/articles/115014893968-Terms-of-sale}{Terms
  of Sale}
\item
  \href{https://spiderbites.nytimes3xbfgragh.onion}{Site Map}
\item
  \href{https://help.nytimes3xbfgragh.onion/hc/en-us}{Help}
\item
  \href{https://www.nytimes3xbfgragh.onion/subscription?campaignId=37WXW}{Subscriptions}
\end{itemize}
