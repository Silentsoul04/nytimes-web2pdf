Sections

SEARCH

\protect\hyperlink{site-content}{Skip to
content}\protect\hyperlink{site-index}{Skip to site index}

\href{https://www.nytimes3xbfgragh.onion/section/politics}{Politics}

\href{https://myaccount.nytimes3xbfgragh.onion/auth/login?response_type=cookie\&client_id=vi}{}

\href{https://www.nytimes3xbfgragh.onion/section/todayspaper}{Today's
Paper}

\href{/section/politics}{Politics}\textbar{}Why Trump's Approval Ratings
on the Economy Remain Durable

\url{https://nyti.ms/2ElUQwv}

\begin{itemize}
\item
\item
\item
\item
\item
\item
\end{itemize}

\begin{itemize}
\item
  \href{https://www.nytimes3xbfgragh.onion/live/2020/09/11/us/trump-vs-biden?action=click\&pgtype=Article\&state=default\&region=TOP_BANNER\&context=storylines_menu}{Election
  Updates}
\item
  \href{https://www.nytimes3xbfgragh.onion/interactive/2020/us/elections/election-states-biden-trump.html?action=click\&pgtype=Article\&state=default\&region=TOP_BANNER\&context=storylines_menu}{Paths
  to 270}
\item
  \href{https://www.nytimes3xbfgragh.onion/interactive/2019/us/elections/2020-presidential-election-calendar.html?action=click\&pgtype=Article\&state=default\&region=TOP_BANNER\&context=storylines_menu}{Key
  Dates}
\item
  \href{https://www.nytimes3xbfgragh.onion/interactive/2020/08/31/us/politics/vote-by-mail-deadlines.html?action=click\&pgtype=Article\&state=default\&region=TOP_BANNER\&context=storylines_menu}{Voting
  by Mail}
\item
  \href{https://www.nytimes3xbfgragh.onion/newsletters/politics?action=click\&pgtype=Article\&state=default\&region=TOP_BANNER\&context=storylines_menu}{Politics
  Newsletter}
\end{itemize}

Advertisement

\protect\hyperlink{after-top}{Continue reading the main story}

Supported by

\protect\hyperlink{after-sponsor}{Continue reading the main story}

\hypertarget{why-trumps-approval-ratings-on-the-economy-remain-durable}{%
\section{Why Trump's Approval Ratings on the Economy Remain
Durable}\label{why-trumps-approval-ratings-on-the-economy-remain-durable}}

Despite the recession, polling data and interviews with voters and
political analysts suggest that a confluence of factors is raising the
president's standing on the economy issue.

\includegraphics{https://static01.graylady3jvrrxbe.onion/images/2020/08/22/us/politics/00trump-economy1/00trump-economy1-articleLarge.jpg?quality=75\&auto=webp\&disable=upscale}

\href{https://www.nytimes3xbfgragh.onion/by/jim-tankersley}{\includegraphics{https://static01.graylady3jvrrxbe.onion/images/2018/10/19/multimedia/author-jim-tankersley/author-jim-tankersley-thumbLarge.png}}

By \href{https://www.nytimes3xbfgragh.onion/by/jim-tankersley}{Jim
Tankersley}

\begin{itemize}
\item
  Published Aug. 24, 2020Updated Aug. 25, 2020
\item
  \begin{itemize}
  \item
  \item
  \item
  \item
  \item
  \item
  \end{itemize}
\end{itemize}

\href{https://www.nytimes3xbfgragh.onion/es/2020/08/25/espanol/estados-unidos/trump-economia.html}{Leer
en español}

It is an enduring political question amid a pandemic recession,
\href{https://www.nytimes3xbfgragh.onion/2020/08/20/business/economy/unemployment-claims.html}{double-digit
unemployment} and a recovery that
\href{https://www.nytimes3xbfgragh.onion/2020/08/21/business/economy/coronavirus-economic-recovery.html?action=click\&module=Top\%20Stories\&pgtype=Homepage}{appears
to be slowing}: Why does
\href{https://www.nytimes3xbfgragh.onion/2020/08/26/us/politics/trump-convention-night-2.html}{President
Trump} continue to get higher marks on economic issues in polls
\href{https://news.gallup.com/poll/313070/trump-economic-ratings-no-longer-best-class.aspx}{than
his predecessors} Barack Obama, George W. Bush and George H.W. Bush
enjoyed when they stood for re-election?

Mr. Trump's relative strength on the economy, and whether Joseph R.
Biden Jr. can cut into it over the next 10 weeks, are among the crucial
dynamics in battleground states in the Midwest and the Sun Belt that are
expected to decide the election. Many of these states have struggled
this summer with rising coronavirus infection and death rates as well as
job losses and vanishing wages and savings --- hard times that, history
suggests, will pose a threat to an incumbent president seeking
re-election.

Yet polling data and interviews with voters and political analysts
suggest that a confluence of factors is raising Mr. Trump's standing on
the economy issue, which remains a centerpiece of his pitch for a second
term and is expected to be a major theme of the Republican National
Convention this week.

The president has built an enduring brand with conservative voters, in
particular, who continue to see him as a successful businessman and
tough negotiator. Many of those voters praise his economic stewardship
before the pandemic hit, and they do not blame him for the damage it has
caused. In interviews, some of those voters cited record stock market
gains --- although \href{https://www.nber.org/papers/w24085}{only about
half of Americans} own any stock at all --- as evidence of a rebound
under the president.

``He's had failures --- so have I --- in business,'' said Dale Georgeff,
58, of Cedarburg, Wis., a Trump supporter who owns parts of a brewery
and a vehicle paint shop and also sells insurance. ``But I think the
biggest thing is that --- and I think this is how it rubs certain people
the wrong way --- he's treating this like a business, and he's running
it like a business.''

David Winton, a Republican strategist and pollster, said that Mr.
Trump's ratings had been bolstered by the addition of nine million jobs
in May, June and July, after the nation lost more than 20 million in
March and April. Mr. Trump's approval on the economy ``has still
generally remained positive, and better than his overall job approval,''
he said. ``This has certainly been helped by the last three good monthly
jobs reports that occurred despite the continuing restrictions on many
businesses to operate.''

Polling suggests that Americans who form Mr. Trump's voter base are less
likely to have lost a job or income than Democratic or independent
voters. That divergence is partially driven by race --- the coronavirus
crisis has disproportionately harmed Black and Latino workers, who lean
heavily Democratic --- but may also reflect regional divides. Small
business owners in small, more rural states that backed Mr. Trump in the
2016 election report less economic damage from the crisis than those in
larger blue states,
\href{https://eig.org/news/taking-the-pulse-of-americas-small-business-sector}{according
to an analysis} of census survey data by the Economic Innovation Group
in Washington.

Perhaps most notably, Mr. Trump is reaping the benefits of
\href{https://www.nytimes3xbfgragh.onion/2019/10/29/business/economy/survey-politics-economy.html}{extreme
polarization of the American electorate}, a divide so intense that it
has overpowered long-running connections between economic performance
and presidential approval ratings. For many Republican voters and
conservatives, optimism about the economy and approval of the president
have become deeply entwined --- and for Democrats, disfavor for Mr.
Trump brought deep pessimism over the economy even in the years of
growth and low unemployment before the crisis.

\includegraphics{https://static01.graylady3jvrrxbe.onion/images/2020/08/23/us/politics/00trump-economy3/merlin_175951377_e0698498-0967-4cd6-9725-edf2cc7c7fbc-articleLarge.jpg?quality=75\&auto=webp\&disable=upscale}

Polls conducted in June, July and August for The New York Times by the
\href{https://www.surveymonkey.com/curiosity/nyt-august-2020-cci/}{online
research firm SurveyMonkey} underscore the degree to which even
Republicans hit hard by the crisis continue to give Mr. Trump and his
economy high marks. Eight in 10 Republican respondents who lost a job in
the recession and have yet to return to work approve of Mr. Trump's
handling of the pandemic. Nearly three in 10 Republicans who lost jobs
say they are better off economically than they were a year ago, a
sentiment that is shared by barely one in 10 Democrats who have kept
their jobs throughout the crisis.

``For so many of these voters, opinions of Trump are basically baked
in,'' said Amy Walter, national editor for the Cook Political Report in
Washington, who
\href{https://cookpolitical.com/analysis/national/national-politics/can-biden-undercut-trumps-continued-advantage-economy}{has
written extensively} on the economy and Mr. Trump's electoral fortunes.
``And what the actual economic situation is in November is less
important to them than it would be in a different time with different
candidates.''

Mr. Trump's
\href{https://projects.fivethirtyeight.com/trump-approval-ratings/}{overall
approval ratings} have never cracked a majority throughout his
presidency. Voters have given him higher approval ratings on his
handling of the economy --- he topped 60 percent in one survey this year
before the pandemic hit --- even as some of his signature economic
initiatives, like the 2017 tax cut package he signed into law,
\href{https://news.gallup.com/opinion/polling-matters/249161/public-opinion-2017-tax-law.aspx}{remain
relatively unpopular}.

But the plunge in economic activity since the coronavirus began to
spread rapidly in the United States late this past winter has hurt Mr.
Trump's standing on economic issues as well as his overall approval.
Most polls now find Americans are evenly split on whether they approve
of his handling of the issue.

Gallup, for example, found Mr. Trump enjoyed a 48 percent approval
rating on the economy this month, down from 63 percent in January. The
decline was particularly acute among moderates, independents and voters
who attended at least some college.

In a recent
\href{https://abcnews.go.com/PollingUnit/election-advantage-stays-biden-enthusiasm-deficit-eases-remains/story?id=72356854\&mc_cid=42713dcf7a\&mc_eid=\%5BUNIQID\%5D}{ABC
News/Washington Post} poll, two-thirds of Americans said the economy was
in bad shape --- the most since 2014, and a 20-percentage-point increase
in negative ratings of the economy since Mr. Trump took office.

The decline in sentiment is hurting Mr. Trump in his campaign against
Mr. Biden, the Democratic nominee. Among registered voters who said they
thought the economy was doing badly, 70 percent planned to support Mr.
Biden and his running mate, Senator Kamala Harris of California, in
November, according to the ABC/Post poll.

But Mr. Biden, the former vice president, is far from commanding on the
issue: Voters were split almost evenly into thirds on the question of
whether the economy would be in better, worse or about the same shape
now, if he were president. And while some polls this summer showed the
candidates deadlocked on the question of who would best handle the
economy, Mr. Trump led Mr. Biden on handling the economy in an
\href{https://www.nbcnews.com/politics/meet-the-press/biden-remains-ahead-trump-nationally-eve-conventions-nbc-news-wsj-n1236873}{NBC
News/Wall Street Journal poll} released this week. A Reuters poll
\href{https://www.ipsos.com/en-us/reutersipsos-core-political-survey-presidential-approval-tracker-08192020}{had
the men tied}.

Mr. Biden emphasized his plans to create jobs and to bring the virus
under control in his acceptance speech at the Democratic National
Convention last week, and he criticized Mr. Trump's handling of the
pandemic. ``I understand something this president doesn't,'' Mr. Biden
said. ``We will never get our economy back on track, we will never get
our kids safely back to school, we will never have our lives back ---
until we deal with this virus.''

The Biden campaign has sought to link Mr. Trump to the recession in
television advertisements,
\href{https://www.youtube.com/watch?v=GSqZUfaxhvA}{including one that}
proclaims that ``Trump's botched handling of the coronavirus pandemic
cost jobs.'' Campaign officials say Mr. Biden and his surrogates will
increase those attacks in the weeks to come.

Mr. Trump ``still has no plan to bring the pandemic under control or end
the recession he catastrophically and needlessly worsened,'' Andrew
Bates, a Biden spokesman, said on Saturday.

The president continues to express confidence that economic issues favor
him in the race, even as he overstates his mixed position in polls.
``We're building up the economy,'' Mr. Trump said on Friday in
Arlington, Va. ``And we're way ahead, by every poll --- even the fake
polls --- we're way ahead on the economy, which is very important.''

Image

Joseph R. Biden Jr. emphasized his plans to create jobs and to bring the
coronavirus under control in his acceptance speech at the Democratic
National Convention last week.Credit...Erin Schaff/The New York Times

Partisan politics --- and divergent experiences with the virus ---
factor heavily into the remaining divide. The SurveyMonkey polling shows
Republicans are less likely to have lost a job in the crisis than
Democrats or independents, though the gap shrinks when comparing only
white voters. In the recovery from the depths of recession, the
unemployment rate has remained higher for Black and Latino workers than
for whites.

``Republicans are putting more importance on the economic issues of the
pandemic,'' said Laura Wronski, a research scientist for SurveyMonkey,
``and Democrats are putting more importance on the health issues.''

Fewer than one in five conservative Republicans worries about losing a
job in the crisis, far less than any other ideological group, the
SurveyMonkey polling shows. (In perhaps a troubling sign for Mr. Trump,
the group that worries most about job loss is independent voters.)
Nearly two in five conservative Republicans say that by late October
``the virus will be under control, and the economy will be strong or
steadily improving,'' which is more than double the rate of Americans
overall. Only 3 percent of Democrats agree with that statement.

``I've seen a steady growth since he's been in office,'' said Rick
Slowicki, president of Nonstop Couriers, a delivery service in
Philadelphia that employs 11 people, runs 14 vehicles and expects
revenue of \$1.3 million this year. ``I just bought three new vehicles
with the confidence that we're going to grow, even during Covid. I'm
doubling down.''

Others praise Mr. Trump's populist trade policies, including tariffs on
imports from China that Mr. Trump claims have returned manufacturing
jobs to America. ``He is the only individual who has actually brought
jobs back to the U.S.A. and put the country first,'' said Dale Palmer,
63, a Republican who supports Mr. Trump and owns a boiler service
business in Byron Center, Mich.

Image

People shop at the Galleria mall in Houston last month. The plunge in
economic activity has hurt Mr. Trump's standing on economic
issues.Credit...Erin Schaff/The New York Times

Democrats predict that if the recovery stalls in the fall and economic
damage mounts anew, Mr. Trump's economic ratings will plunge.

``Trump is a master at convincing people of his alternative reality,''
said Jared Bernstein, an economist at the Center on Budget and Policy
Priorities who is an outside adviser to Mr. Biden. ``But he will be
unable to do so as people face evictions, job losses, falling incomes
and tremendous difficulties meeting their basic needs. At some point,
reality TV collides with reality.''

Reporting was contributed by Ben Casselman, Kathleen Grey, Jon Hurdle,
Tom Kertscher, Alan Rappeport and Giovanni Russonello.

\hypertarget{our-2020-election-guide}{%
\section{Our 2020 Election Guide}\label{our-2020-election-guide}}

Updated ~Sept. 11, 2020

\begin{center}\rule{0.5\linewidth}{\linethickness}\end{center}

\begin{itemize}
\item ~
  \hypertarget{the-latest}{%
  \subsection{The Latest}\label{the-latest}}

  \begin{itemize}
  \item
    Joe Biden and President Trump put
    \href{https://www.nytimes3xbfgragh.onion/2020/09/11/us/politics/shanksville-trump-biden.html?action=click\&pgtype=Article\&state=default\&region=BELOW_MAIN_CONTENT\&context=storylines_guide}{hostilities
    on hold today to travel to ground zero and then to Shanksville, Pa.,
    where they separately honored 9/11 victims}.
  \end{itemize}
\item ~
  \hypertarget{how-to-win-270}{%
  \subsection{How to Win 270}\label{how-to-win-270}}

  \begin{itemize}
  \item
    Joe Biden and Donald Trump need 270 electoral votes to reach the
    White House. Try building
    \href{https://www.nytimes3xbfgragh.onion/interactive/2020/us/elections/election-states-biden-trump.html?action=click\&pgtype=Article\&state=default\&region=BELOW_MAIN_CONTENT\&context=storylines_guide}{your
    own coalition of battleground states}~to see potential outcomes.
  \end{itemize}
\item ~
  \hypertarget{voting-by-mail}{%
  \subsection{Voting by Mail}\label{voting-by-mail}}

  \begin{itemize}
  \item
    Will you have enough time to vote by mail in your state? Yes, but
    it's risky to procrastinate.
    \href{https://www.nytimes3xbfgragh.onion/interactive/2020/08/31/us/politics/vote-by-mail-deadlines.html?action=click\&pgtype=Article\&state=default\&region=BELOW_MAIN_CONTENT\&context=storylines_guide}{Check
    your state's deadline.}
  \item
    \href{https://www.nytimes3xbfgragh.onion/interactive/2020/us/elections/joe-biden.html?action=click\&pgtype=Article\&state=default\&region=BELOW_MAIN_CONTENT\&context=storylines_guide}{}

    \hypertarget{joe-biden}{%
    \section{Joe Biden}\label{joe-biden}}

    \hypertarget{democrat}{%
    \subsection{Democrat}\label{democrat}}

    \href{https://www.nytimes3xbfgragh.onion/interactive/2020/us/elections/donald-trump.html?action=click\&pgtype=Article\&state=default\&region=BELOW_MAIN_CONTENT\&context=storylines_guide}{}

    \hypertarget{donald-trump}{%
    \section{Donald Trump}\label{donald-trump}}

    \hypertarget{republican}{%
    \subsection{Republican}\label{republican}}
  \end{itemize}
\item
  \hypertarget{keep-up-with-our-coverage}{%
  \subsection{Keep Up With Our
  Coverage}\label{keep-up-with-our-coverage}}

  \begin{itemize}
  \item
    Get an
    \href{https://www.nytimes3xbfgragh.onion/newsletters/politics?action=click\&pgtype=Article\&state=default\&region=BELOW_MAIN_CONTENT\&context=storylines_guide}{email}~recapping
    the day's news
  \item
    Download our mobile app on
    \href{https://apps.apple.com/us/app/nytimes/id284862083?ls=1\&mat_click_id=5c79ae7455014fd1bd66b5610c05b8f2-20191112-16948\&referrer=mat_click_id\%3D5c79ae7455014fd1bd66b5610c05b8f2-20191112-16948\%26link_click_id\%3D722930677036718082}{iOS}~and
    \href{http://a.localytics.com/android?id=com.nytimes.android\&referrer=utm_source\%3Dother_nyt_mobile_web\%26utm_medium\%3DWeb\%2520page\%26utm_term\%3DGeneral\%2520Mobile\%2520Page\%26utm_campaign\%3DNYT\%2520Mobile\%2520General\%2520Page}{Android}~and
    turn on Breaking News and Politics alerts
  \end{itemize}
\end{itemize}

Advertisement

\protect\hyperlink{after-bottom}{Continue reading the main story}

\hypertarget{site-index}{%
\subsection{Site Index}\label{site-index}}

\hypertarget{site-information-navigation}{%
\subsection{Site Information
Navigation}\label{site-information-navigation}}

\begin{itemize}
\tightlist
\item
  \href{https://help.nytimes3xbfgragh.onion/hc/en-us/articles/115014792127-Copyright-notice}{©~2020~The
  New York Times Company}
\end{itemize}

\begin{itemize}
\tightlist
\item
  \href{https://www.nytco.com/}{NYTCo}
\item
  \href{https://help.nytimes3xbfgragh.onion/hc/en-us/articles/115015385887-Contact-Us}{Contact
  Us}
\item
  \href{https://www.nytco.com/careers/}{Work with us}
\item
  \href{https://nytmediakit.com/}{Advertise}
\item
  \href{http://www.tbrandstudio.com/}{T Brand Studio}
\item
  \href{https://www.nytimes3xbfgragh.onion/privacy/cookie-policy\#how-do-i-manage-trackers}{Your
  Ad Choices}
\item
  \href{https://www.nytimes3xbfgragh.onion/privacy}{Privacy}
\item
  \href{https://help.nytimes3xbfgragh.onion/hc/en-us/articles/115014893428-Terms-of-service}{Terms
  of Service}
\item
  \href{https://help.nytimes3xbfgragh.onion/hc/en-us/articles/115014893968-Terms-of-sale}{Terms
  of Sale}
\item
  \href{https://spiderbites.nytimes3xbfgragh.onion}{Site Map}
\item
  \href{https://help.nytimes3xbfgragh.onion/hc/en-us}{Help}
\item
  \href{https://www.nytimes3xbfgragh.onion/subscription?campaignId=37WXW}{Subscriptions}
\end{itemize}
