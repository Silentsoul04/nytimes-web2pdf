Sections

SEARCH

\protect\hyperlink{site-content}{Skip to
content}\protect\hyperlink{site-index}{Skip to site index}

\href{https://www.nytimes3xbfgragh.onion/section/books}{Books}

\href{https://myaccount.nytimes3xbfgragh.onion/auth/login?response_type=cookie\&client_id=vi}{}

\href{https://www.nytimes3xbfgragh.onion/section/todayspaper}{Today's
Paper}

\href{/section/books}{Books}\textbar{}Elena Ferrante's New Novel Is a
Suspenseful Story About the Sins of Parents

\url{https://nyti.ms/31p6KOW}

\begin{itemize}
\item
\item
\item
\item
\item
\item
\end{itemize}

Advertisement

\protect\hyperlink{after-top}{Continue reading the main story}

Supported by

\protect\hyperlink{after-sponsor}{Continue reading the main story}

\href{/column/books-of-the-times}{Books of The Times}

\hypertarget{elena-ferrantes-new-novel-is-a-suspenseful-story-about-the-sins-of-parents}{%
\section{Elena Ferrante's New Novel Is a Suspenseful Story About the
Sins of
Parents}\label{elena-ferrantes-new-novel-is-a-suspenseful-story-about-the-sins-of-parents}}

By \href{https://www.nytimes3xbfgragh.onion/by/parul-sehgal}{Parul
Sehgal}

\begin{itemize}
\item
  Published Aug. 25, 2020Updated Sept. 1, 2020
\item
  \begin{itemize}
  \item
  \item
  \item
  \item
  \item
  \item
  \end{itemize}
\end{itemize}

\includegraphics{https://static01.graylady3jvrrxbe.onion/images/2020/08/26/books/25BOOKFERRANTE1/25BOOKFERRANTE1-articleLarge.png?quality=75\&auto=webp\&disable=upscale}

Buy Book ▾

\begin{itemize}
\tightlist
\item
  \href{https://www.amazon.com/gp/search?index=books\&tag=NYTBSREV-20\&field-keywords=The+Lying+Life+of+Adults+Elena+Ferrante}{Amazon}
\item
  \href{https://du-gae-books-dot-nyt-du-prd.appspot.com/buy?title=The+Lying+Life+of+Adults\&author=Elena+Ferrante}{Apple
  Books}
\item
  \href{https://www.anrdoezrs.net/click-7990613-11819508?url=https\%3A\%2F\%2Fwww.barnesandnoble.com\%2Fw\%2F\%3Fean\%3D9781609455910}{Barnes
  and Noble}
\item
  \href{https://www.anrdoezrs.net/click-7990613-35140?url=https\%3A\%2F\%2Fwww.booksamillion.com\%2Fp\%2FThe\%2BLying\%2BLife\%2Bof\%2BAdults\%2FElena\%2BFerrante\%2F9781609455910}{Books-A-Million}
\item
  \href{https://bookshop.org/a/3546/9781609455910}{Bookshop}
\item
  \href{https://www.indiebound.org/book/9781609455910?aff=NYT}{Indiebound}
\end{itemize}

When you purchase an independently reviewed book through our site, we
earn an affiliate commission.

Why is it, wonders a character in Elena Ferrante's new novel, ``The
Lying Life of Adults,'' that when talking about sex, one adjective will
never suffice? *``*Why does it take many --- embarrassing, bland,
tragic, happy, pleasant, repulsive --- and never one at a time but all
together?''

I can think of one word copious enough. The name Ferrante --- the
pseudonym of the Italian novelist --- evokes for me all the ordinary,
warring paradoxes of intimate life. It is shorthand for the tangle of
impulses that drive her heroines, mothers and daughters torn between
mutual dependence and contempt, their desires to devour and abandon each
other, their instincts to nourish and betray.

Ferrante's fiction has become a global phenomenon. ``A cold surface and,
visible underneath it, a magma of unbearable heat,'' she has described
her style, brought smoothly into English by
\href{https://www.nytimes3xbfgragh.onion/2020/08/21/books/elena-ferrante-lying-life-of-adults-ann-goldstein-translator.html}{her
translator Ann Goldstein}. Her quartet of Neapolitan novels, following a
pair of rivalrous friends in postwar Italy, has sold more than 11
million copies worldwide, and
\href{https://www.nytimes3xbfgragh.onion/2018/11/14/arts/television/my-brilliant-friend-review-hbo.html}{was
made into an HBO series}. ``The Lying Life of Adults'' will be adapted
by Netflix.

The new novel is suspenseful and propulsive; in style and theme, a
sibling to her previous books. But it's also a more vulnerable
performance, less tightly woven and deliberately plotted, even turning
uncharacteristically jagged at points as it explores some of the
writer's touchiest preoccupations.

The story begins in typical Ferrante fashion. A woman sits at her desk
recalling a moment of painful disillusionment in her youth. Giovanna
seems to combine the personalities of the two friends in the Neapolitan
novels --- Lila's fire along with milder Lenù's deliberation. But she
has grown up middle-class and in the present day; the world has been
gentler to her. Still, the idyll of her childhood was shattered at age
12, when she overheard her father calling her ugly.

His remark unleashed a wave of shame and self-loathing in the girl,
almost too big for her body to hold. Her father said she was beginning
to resemble his loathed, long-estranged sister Vittoria. ``I slipped
away,'' Giovanna says, ``and am still slipping away.'' Something in her
became permanently untethered.

This information is delivered swiftly in the opening pages. I read them
also flushed with shame, feeling implicated, monstrous, apologetic ---
in short, that horrid sensation: 12 again. I also felt prickly
recognition. This moment recreates a famous scene from ``Madame
Bovary'': Emma, beholding her small daughter, exclaiming at her
ugliness. It's a scene that has long obsessed Ferrante. In essays and
interviews, she has wondered if her own mother ever expressed such a
sentiment. She has envied Flaubert his shocking bluntness. She once
wrote: ``I've believed, angrily, bitterly, that men who are masters of
writing are able to have their female characters say what women truly
think and say and live but do not dare write.''

\emph{{[} This book was one of our most anticipated titles of
September.}
\href{https://www.nytimes3xbfgragh.onion/2020/08/27/books/new-september-books.html}{\emph{See
the full list}}\emph{. {]}}

It's true that Ferrante's women never utter such a phrase. They never
declare their children ugly or unlovable. They run away instead, or
destroy themselves. In this novel, however, Ferrante lifts the line and
twists it, putting it in the mouth of a man.

What does it mean to be ugly to your father? If your mother declares you
ugly, the impulse, as in ``Madame Bovary,'' would be to find fault with
\emph{her ---} the unnatural mother. There's a feeling in Ferrante's
novel that had Giovanna overheard such a remark from her mother, there
would be an immediate confrontation, and, perhaps, no book. But to be
declared coarse and wanting by the father, by the family's voice of
``dazzling authority,'' as Ferrante writes? Giovanna loses her moorings.
She believes him. She begins to court his disapproval and, later, the
disapproval of the world. She becomes consumed with befriending Aunt
Vittoria. Her rebellion tips into an odd kind of freedom. She spies on
her parents for Vittoria and reports on her aunt to her parents,
liberally embellishing her stories. In the course of her double-agent
dealings, she unearths the deep mendacity of the adults around her.

In a sense, Ferrante's fiction has long been preoccupied with the notion
of being ugly in the eyes of the father --- of writing against the
grain, against authority and convention. At the center of her work are
not just women's lives --- but femininity itself. Dolls are important
totems. A bracelet in ``The Lying Life of Adults'' takes on almost
supernatural significance. Reviews of Ferrante's work note how liberally
her work borrows elements of the romance and the potboiler: the
unapologetic melodrama, the cliffhangers. She pays serious attention to
pregnancy, and to little girls and old women --- rarely the subjects of
serious fiction.

It was a form born of initial resistance. Ferrante grew up in the
ordinary way; that is, believing that if she didn't win male approval,
``it would have been tantamount to not existing at all,'' she has
written. Only later did she discover the feminist literature that
reoriented her thinking. ``I realized that I had to do exactly the
opposite: I had to start with myself and with my relationships with
other women --- this is another essential formula --- if I really wanted
to give myself a shape.'' Her work began to draw on the classics as well
as the stories in women's magazines --- ``a fund of pleasure that for
years I repressed in the name of Literature.''

It is the same trajectory she gives Giovanna. The father is dethroned;
who will take his place? For a time, the girl finds a substitute in the
chaotic allure of her aunt. Then, in another man --- the charismatic
Roberto. ``I now felt him as an authority,'' she thinks to herself. He
declares her beautiful, and her self-image, blotted away by one man, is
restored by another. But in Giovanna reigns a streak of stubborn
independence. She reads what Roberto wants her to, but comes to her own
conclusions (she finds the Gospels nonsensical and a bore).

She returns to her childhood friends, and crucially, she finds a freedom
and privacy in deception, in authoring her own reality --- an old theme
in Ferrante. As a young woman the writer kept a diary, striving to
record her life with absolute honesty. When she became terrified it
would be discovered, she planted her ``most unutterable truths'' in
fiction. It's a move that seems to presage the adoption of her pseudonym
and the artistic freedom afforded by anonymity.

Ferrante's women go so spectacularly to pieces that it is easy to forget
that the vast majority of her novels have, if not happy endings, then
notes of reconciliation. Her women come through the fire because they
are writers; the act of narration becomes an act of mending. Not of
truth necessarily; as Lila says in ``My Brilliant Friend'': ``Each of us
narrates our life as it suits us.''

The pleasure for the reader is often in spotting those moments of
disjuncture that Ferrante flags for us, where the narrative is partial
or incomplete. But here is where some wobbliness presents itself in the
new novel. The mournful opening paragraph --- with its caveat that this
tale might only be ``a snarled confusion of suffering, without
redemption'' --- doesn't square with the story in our hands, of the
evolution of a young woman, so brash and sensibly secretive, allergic to
banality, prone to fabrication but honest with herself about her
desires. Ferrante leaves many threads dangling; we're left to wonder at
the initial forecast and the novel's enigmatic, oddly heroic conclusion:
What is this progress that seems to contain the seeds of regression?
When is a revolt indistinguishable from a retreat? There might very well
be a word for it.

Advertisement

\protect\hyperlink{after-bottom}{Continue reading the main story}

\hypertarget{site-index}{%
\subsection{Site Index}\label{site-index}}

\hypertarget{site-information-navigation}{%
\subsection{Site Information
Navigation}\label{site-information-navigation}}

\begin{itemize}
\tightlist
\item
  \href{https://help.nytimes3xbfgragh.onion/hc/en-us/articles/115014792127-Copyright-notice}{©~2020~The
  New York Times Company}
\end{itemize}

\begin{itemize}
\tightlist
\item
  \href{https://www.nytco.com/}{NYTCo}
\item
  \href{https://help.nytimes3xbfgragh.onion/hc/en-us/articles/115015385887-Contact-Us}{Contact
  Us}
\item
  \href{https://www.nytco.com/careers/}{Work with us}
\item
  \href{https://nytmediakit.com/}{Advertise}
\item
  \href{http://www.tbrandstudio.com/}{T Brand Studio}
\item
  \href{https://www.nytimes3xbfgragh.onion/privacy/cookie-policy\#how-do-i-manage-trackers}{Your
  Ad Choices}
\item
  \href{https://www.nytimes3xbfgragh.onion/privacy}{Privacy}
\item
  \href{https://help.nytimes3xbfgragh.onion/hc/en-us/articles/115014893428-Terms-of-service}{Terms
  of Service}
\item
  \href{https://help.nytimes3xbfgragh.onion/hc/en-us/articles/115014893968-Terms-of-sale}{Terms
  of Sale}
\item
  \href{https://spiderbites.nytimes3xbfgragh.onion}{Site Map}
\item
  \href{https://help.nytimes3xbfgragh.onion/hc/en-us}{Help}
\item
  \href{https://www.nytimes3xbfgragh.onion/subscription?campaignId=37WXW}{Subscriptions}
\end{itemize}
