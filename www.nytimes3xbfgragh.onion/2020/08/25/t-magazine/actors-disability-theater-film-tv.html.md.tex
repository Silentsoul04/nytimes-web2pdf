Sections

SEARCH

\protect\hyperlink{site-content}{Skip to
content}\protect\hyperlink{site-index}{Skip to site index}

\href{https://myaccount.nytimes3xbfgragh.onion/auth/login?response_type=cookie\&client_id=vi}{}

\href{https://www.nytimes3xbfgragh.onion/section/todayspaper}{Today's
Paper}

The Actors With Disabilities Redefining Representation

\url{https://nyti.ms/2Qq8AZD}

\begin{itemize}
\item
\item
\item
\item
\item
\end{itemize}

Advertisement

\protect\hyperlink{after-top}{Continue reading the main story}

Supported by

\protect\hyperlink{after-sponsor}{Continue reading the main story}

\hypertarget{the-actors-with-disabilities-redefining-representation}{%
\section{The Actors With Disabilities Redefining
Representation}\label{the-actors-with-disabilities-redefining-representation}}

These performers are creating a new template for the artist-as-activist,
challenging their industry --- and their audiences --- to reconsider
what inclusion really means.

\includegraphics{https://static01.graylady3jvrrxbe.onion/images/2020/08/30/t-magazine/30tmag-disabled-actors-slide-29TO/30tmag-disabled-actors-slide-29TO-articleLarge.jpg?quality=75\&auto=webp\&disable=upscale}

By Mark Harris

\begin{itemize}
\item
  Aug. 25, 2020
\item
  \begin{itemize}
  \item
  \item
  \item
  \item
  \item
  \end{itemize}
\end{itemize}

RYAN O'CONNELL would like you to know that he is tired and pissed off
and horny. He is tired of waiting for what he calls ``our
`\href{https://www.nytimes3xbfgragh.onion/watching/recommendations/watching-tv-transparent}{Transparent}'
moment'' (some of his fellow actors call it, instead, ``our
`\href{https://www.nytimes3xbfgragh.onion/2018/06/01/arts/television/pose-review-fx-ryan-murphy.html}{Pose}'
moment''), by which he means a single piece of breakthrough pop culture
that makes people aware of a heretofore ignored and stereotyped
minority, a prizewinning, noisemaking event that kicks open the door to
mainstream omnipresence and ultimately to normalization. He is pissed
off that it hasn't happened yet. ``I think about this a lot,'' he says.
``Why, in this woke-ass culture that we live in, where so much attention
is given to marginalized populations, do people with disabilities still
largely go ignored?'' The actor, who has cerebral palsy, is also, he
says, ``horny for representation that comes from actual disabled people,
because we live in a dark hellscape of a capitalist country. Real power
can only be accrued through opportunities, and you need to be given the
keys to tell your own story.''

O'Connell is more cheerful --- well, somewhat more cheerful --- than
that makes him sound. At 33, he is the star of a short-form Netflix
series called
``\href{https://www.nytimes3xbfgragh.onion/2019/04/11/arts/television/special-netflix-ryan-oconnell-disabled.html}{Special}''
that, when I spoke to him in February, was filming its second season
(now halted because of the Covid-19 pandemic). The first garnered four
Emmy nominations, including two for O'Connell, who is the show's creator
as well as its leading man. ``Special'' is an alternately gentle,
introspective and raunchy comedy series about the personal, social and
sexual emergence of a young Angeleno who, like O'Connell, is both gay
and disabled. When he first started working in writers' rooms on other
television shows, O'Connell, whose visible symptoms are mild, kept his
condition quiet; he only ``came out of the disabled closet'' five years
ago, when he published a memoir, ``I'm Special: And Other Lies We Tell
Ourselves,'' which inspired his series. Episodes of ``Special'' run only
about 15 minutes; it's hard not to think of Chris Rock's Afro-radical
'90s-era ``Saturday Night Live'' character Nat X, who complained that
his late-night series had to be short form `` 'cause if the man gave me
any more, he would consider that welfare!''

O'Connell, who talks in exuberant bursts punctuated by droll profanity,
doesn't want anyone's charity. He wants action. ``Hollywood is super
horny for profiting off the stories of marginalized people without
giving them actual opportunities,'' he says. Like all of the performers
with disabilities I interviewed for this story, he keeps track, almost
without thinking about it, of what he sees: the successes, the
milestones --- and also the blunders to which the abled world is prone.
When he watched this year's Oscars, he noted the audience's reaction
when the 35-year-old actor Zack Gottsagen, who starred in the 2019 indie
hit
``\href{https://www.nytimes3xbfgragh.onion/2019/08/08/movies/the-peanut-butter-falcon-review.html}{The
Peanut Butter Falcon}'' and who has Down syndrome, presented an award.
``He says a sentence, and people clap as if that should be something
revolutionary, like, `Yes, good for you, you said \emph{four words}!'
He's a \emph{{[}droll profanity{]}} adult! He can talk! Stop
infantilizing him!'' O'Connell takes a breath, slows his tempo and
slides one octave down. ``I don't think that Hollywood is some malicious
devil being like, `Keep them out! Keep them out!' But I want to see the
needle move.''

\includegraphics{https://static01.graylady3jvrrxbe.onion/images/2020/08/30/t-magazine/30tmag-disabled-actors-slide-4RBU/30tmag-disabled-actors-slide-4RBU-articleLarge.jpg?quality=75\&auto=webp\&disable=upscale}

COULD IT BE that, at long last, that movement is starting to become real
for actors with disabilities, as it did for African-Americans in
1970s-era entertainment, for gay people in the early aughts and for
Asian-Americans in the last few years? If a successful cultural
transformation can be defined as the moment when you can finally stop
counting heads, the first sign of that may be when you realize that at
least there are heads to count. As always, television --- because there
is just so much of it --- has led the way: Suddenly, there's a lot to
notice, and not just once a year on awards shows, when
\href{https://www.nytimes3xbfgragh.onion/2012/04/01/magazine/peter-dinklage-was-smart-to-say-no.html}{Peter
Dinklage} wins another Emmy or someone makes an acceptance speech in
American Sign Language. A study of networks, cable and streaming
services in 2018 by the Ruderman Family Foundation
\href{https://rudermanfoundation.org/white_papers/the-ruderman-white-paper-on-authentic-representation-in-tv/}{revealed}
that 56 characters with disabilities were being portrayed by actors with
disabilities. One asterisk: In the study, ``disability'' included
characters who have struggled with addiction, a definition supported by
the Americans With Disabilities Act but one that bothers many who feel
the term should be reserved for those with a visible or apparent
physical, cognitive or neurological impairment. Another, larger issue:
The same study showed that the vast majority of disabled characters are
still played by nondisabled actors.

Nonetheless, the sense of momentum is real.
\href{https://www.nytimes3xbfgragh.onion/2011/04/24/movies/mark-ruffalo-and-christopher-thornton-in-sympathy-for-delicious.html}{Christopher
Thornton}, an actor in a wheelchair, shows up in the rebooted ``Will \&
Grace'' as a romantic interest for Karen. A tetraplegic actor,
\href{https://www.teenvogue.com/story/sex-education-george-robinson-isaac}{George
Robinson}, appears on season two of Netflix's ``Sex Education,'' and
\href{https://www.nytimes3xbfgragh.onion/2019/04/18/arts/television/ramy-review.html}{Steve
Way}, a comedian and actor who has muscular dystrophy and uses a
wheelchair, becomes a scene-stealer on Hulu's ``Ramy.'' Netflix's
``\href{https://www.nytimes3xbfgragh.onion/2020/03/24/movies/crip-camp-review.html}{Crip
Camp},'' a documentary about a summer getaway for disabled kids in the
1960s and 1970s, premieres at the Sundance Film Festival and wins the
audience award for U.S. documentaries. NBC's
``\href{https://www.nytimes3xbfgragh.onion/2020/01/07/arts/television/zoeys-extraordinary-playlist-high-school-musical.html}{Zoey's
Extraordinary Playlist}'' features a musical performance of Rachel
Platten's ``Fight Song'' by the Los Angeles-based company
\href{https://www.nytimes3xbfgragh.onion/2015/10/04/theater/lights-gestures-action-how-to-stage-a-broadway-musical-with-deaf-actors.html}{Deaf
West Theater}.
\href{https://www.nytimes3xbfgragh.onion/2020/05/08/style/modern-love-coronavirus-deaf-motherhood-in-a-quiet-world.html}{Shoshannah
Stern} becomes the first deaf actress to play a doctor on ``Grey's
Anatomy.''

Image

Ryan O'Connell in ``Special.''Credit...Courtesy of Netflix

With every new appearance, word gets out on social media and excitement
builds, but among disabled actors (and viewers), skepticism about
whether it's all mounting to something is not easily vanquished. That's
historically justified: For decades, people with disabilities have seen
themselves incorporated into pop culture not as fully realized human
beings but as teachable moments for nondisabled people; they're gazed at
patronizingly, applauded --- and then forgotten. In television, they
have long been part of what's been called ``very special episode''
syndrome, in which the series' recurring characters, who always fit
whatever is defined as the social, sexual or physical norm, are
fleetingly confronted by the reality that people who are not like them
exist, but only so that, having enlightened the ``regular'' people,
those othered characters --- who are often children, so their
``innocence'' can be emphasized --- can then gently but firmly be placed
back where they have always been, just outside the margins of the show's
visible world. It's infuriating for people with disabilities, who
(shockingly) do not see themselves as walking, talking life lessons.

``We are symbols of inspiration, symbols of overcoming adversity or
symbols of death with dignity,'' says Gregg Mozgala, an actor and
artistic director. ``We have been force-fed these as a culture, right?
But people don't necessarily know how to metabolize stories that
challenge those three narratives.'' Mozgala won the 2018 Lucille Lortel
Award for best Off Broadway actor for his starring role in Martyna
Majok's
``\href{https://www.nytimes3xbfgragh.onion/2017/06/07/theater/cost-of-living-review.html}{Cost
of Living},'' then moved on to play a high-school version of Richard III
in Mike Lew's
``\href{https://www.nytimes3xbfgragh.onion/2018/06/20/theater/teenage-dick-review.html}{Teenage
Dick}'' at the Public Theater. The home page of his website reads,
``Actor, Writer, Cripple.'' He is 42, has cerebral palsy and has been
working as both an actor and an advocate long enough to both believe in
one-step-at-a-time change and to want things to move faster. Like many
in the community, he has studied the progress made by people of color
and L.G.B.T.Q. performers, but he also knows that the parallels go only
so far.

Image

Gregg Mozgala (left) and Jolly Abraham in a 2017 performance of ``Cost
of Living,'' written by Martyna Majok.Credit...Joan
Marcus/Boneau/Bryan-Brown, via Associated Press

``Disability's not gender or ethnicity or geography,'' he says. ``The
context around disability is often trauma and pain. That can be scary
for people to deal with, and there's something inherently
anxiety-inducing about otherness, especially visible physical
difference.'' The 45-year-old playwright Christopher Shinn, whose left
leg was amputated several years ago after he developed a rare form of
cancer,
\href{https://www.theatlantic.com/entertainment/archive/2014/07/why-disabled-characters-are-never-played-by-disabled-actors/374822/}{has
written} that when abled audience members see a disabled actor onstage,
the reality in front of them makes it harder to maintain the distancing
safety of treating disability as a metaphor. ``You have a sort of double
knowledge of the character and the actor,'' Shinn says. ``It's hard to
put into words exactly the kind of difference and tension that
creates.'' After all, the line between ability and disability is often
permeable; circumstances can erase it with no notice. As Susan Sontag
wrote in
``\href{https://www.nytimes3xbfgragh.onion/2005/12/04/magazine/illness-as-more-than-metaphor.html}{Illness
as Metaphor}'' (1978), ``Everyone who is born holds dual citizenship, in
the kingdom of the well and in the kingdom of the sick \ldots{} Sooner
or later each of us is obliged, at least for a spell, to identify
ourselves as citizens of that other place.'' There is no such thing as
an inviolable position of privilege from which to regard a disabled
person; they are us and we are them. Disability resists analogy, and it
defies detachment. Which is to say, it scares people.

Mozgala encourages other disabled actors to lean into that anxiety if
necessary. Performance, he says, has always been a part of life for
people with physical differences. A century ago --- and, in some places,
much more recently than that --- some had to survive by making their
living in carnivals; others were treated as clinical curiosities by
doctors and scientists who barely saw them as human --- an attitude that
was recreated in the play
``\href{https://www.nytimes3xbfgragh.onion/1979/01/15/archives/stage-the-elephant-man-a-parable-of-man.html}{The
Elephant Man}'' (1977) so that audiences could pay to see an actor
pretend to be disabled. Today, disabled people finally have the chance
to insist, even demand, that they present themselves on their own terms.
``Even Broadway is referred to as `the fabulous invalid,' always limping
along,'' Mozgala says wryly. ``So if we are at an inflection point, why
not take ownership of that?''

ALEXANDRIA WAILES is in the middle of making a point when all the lights
around us go dim. We're having a late-afternoon coffee in an Upper West
Side restaurant, and the management has suddenly decided it's time for
the atmosphere to change from daytime to evening.
\href{https://www.nytimes3xbfgragh.onion/2019/11/19/arts/dance/alexandria-wailes-deaf-dancer-for-colored-girls.html}{Wailes},
44, who has been deaf since she was an infant, has been sitting across
from me talking in American Sign Language as an interpreter by my side
translates. For an actress and dancer like Wailes, A.S.L. becomes even
more infused with meaning and nuance in concert with her movement, her
body language and her facial expressions. Lower light isn't just a
mood-changer, it's an impediment. She stops midsentence, and there's no
need for the translator to explain why. What Wailes had been discussing
is how the job of incorporating disabled performers into theatrical
spaces doesn't end with a self-congratulatory moment of casting. ``My
big question,'' she says, ``would be: Do they have control over the
architecture itself, the actual space where things are happening? Are
there level floors? Is it well-lit? There are so many different aspects
that affect our interactions, and that's a huge part of the equation.''

Image

Alexandria Wailes (center) in a scene from ``For Colored Girls Who Have
Considered Suicide/When the Rainbow Is Enuf'' at the Public Theater in
2019.Credit...Sara Krulwich/The New York Times

Wailes co-starred in the Public's 2019 revival of Ntozake Shange's
``\href{https://www.nytimes3xbfgragh.onion/2019/10/22/theater/for-colored-girls-review-ntozake-shange.html}{For
Colored Girls Who Have Considered Suicide/When the Rainbow Is Enuf},''
signing a part originally conceived for a hearing performer. She is also
a choreographer and director. When the director Sam Gold, 42, staged
``\href{https://www.nytimes3xbfgragh.onion/2019/04/04/theater/king-lear-review-glenda-jackson.html}{King
Lear}'' on Broadway with Glenda Jackson in 2019, he hired Wailes to
serve as the production's director of artistic sign language and work
alongside him, two deaf actors and the rest of the cast to help
integrate the ensemble and make sure the language felt right, both for
the production and for Shakespeare. ``One of the challenges was trying
to encourage people not to sign who weren't supposed to, because
everybody wanted to learn,'' she says. ``That was endearing, even
inspiring.''

Gold, one of whose young daughters uses a wheelchair, has been a
trailblazer in casting actors with disabilities in his stage
productions. ``I think A.S.L. is a very good language for interpreting
the imagery contained in Shakespeare,'' he says. ``And that's not about
disability --- it's the opposite. It's about this community having
talent and access to a language that I think brings the theater alive.
I'm not doing them a favor by casting them.''

For Wailes, ``Lear'' was close to an ideal collaboration, but like many
actors with disabilities, she inevitably finds herself doing an unpaid
second job as a teacher and advocate. ``If I'm in the room, I'm going to
do everything I can to influence the conversation and to try to make a
change,'' she says. ``But what comes up often with my colleagues is how
tired we are,'' or, as she described it in a later email, ``the mental
space it takes to educate when we are wanting to simply be creative in a
supportive environment.'' For the 42-year-old actress
\href{https://www.nytimes3xbfgragh.onion/2018/05/11/theater/lauren-ridloff-children-of-a-lesser-god.html}{Lauren
Ridloff}, who received a best actress Tony nomination two years ago for
``\href{https://www.nytimes3xbfgragh.onion/2018/04/11/theater/review-children-of-a-lesser-god-broadway.html}{Children
of a Lesser God}'' and subsequently joined the cast of
``\href{https://www.nytimes3xbfgragh.onion/watching/recommendations/watching-tv-the-walking-dead}{The
Walking Dead},'' it's a familiar bind: Once she has a role, she says,
``I don't want them to start thinking more about what my needs are as a
deaf person than as an actor.'' But, she says, only once those needs are
met can she ``actually provide them with the quality of performance that
they want.'' When Ridloff first arrived on ``The Walking Dead'' set in
Georgia, there was only one locally hired interpreter, whose skills had
not been evaluated by any deaf person on the production. Now there are
four interpreters --- and another deaf actor in the cast.

Sometimes, what's required doesn't become apparent until the problem
announces itself: Earlier this year, O'Connell was preparing to shoot an
episode of ``Special'' featuring several actors with disabilities when
he discovered how difficult it is to find hair and makeup trailers that
are wheelchair-accessible. ``I'm sure the people who make those trailers
never expected a disabled person to \emph{need} hair and makeup,'' he
says. But it can also be true that disabled performers don't necessarily
know what they need from day one. During ``Children of a Lesser God,''
for instance, Ridloff didn't realize until she was well into the run how
much of a strain on her neck, back and arms signing eight performances a
week would cause. At first, she didn't ask for physical-therapy
sessions, although that would have been completely reasonable. But
Wailes notes that even asking can be stressful. Getting hired is hard
enough; will making requests mean that the next production will be less
likely to hire a disabled person? What isn't said --- what doesn't feel
sayable --- creates extra anxiety.

Almost every professional of a certain age who is a member of a minority
community knows this: It's exhausting to be the only one in the room. To
have eyes looking into yours in search of validation of the
enlightenment they showed in hiring you; to wonder if you were employed
on a kind of barter system that suggests a return payment in gratitude
or compliance. And more than exhausting, it's distracting; sometimes ---
\emph{most} of the time --- you just want to do the job. For an actor,
that means burrowing deep into a character, which is virtually the
opposite of representing the imagined, generic totality of a shared
condition.

``I just want to live my experience through a character, and that alone
is all people need,'' says the actor
\href{https://www.nytimes3xbfgragh.onion/2017/03/24/theater/a-wheelchair-on-broadway-isnt-exploitation-its-progress.html}{Madison
Ferris}. ``What's frustrating about new work right now is that some
writers are like, `Well, this character is in a wheelchair, so we have
to tell their whole wheelchair story.' And in life, that's not what
happens at all.'' Ferris, 28, has muscular dystrophy and has used a
wheelchair since she was 19. In 2017, Gold cast her as Laura Wingfield
in his Broadway revival of
``\href{https://www.nytimes3xbfgragh.onion/2017/03/09/theater/the-glass-menagerie-review.html?_r=0}{The
Glass Menagerie}.'' The critical reception was divided, with
\href{https://www.newyorker.com/culture/culture-desk/a-cold-and-hip-glass-menagerie}{a
review in The New Yorker} singling out a scene in which Gold had Laura's
brother, Tom, help her stretch her legs. The critic asked, ``Why is
Ferris's disease called upon to generate a spectacle?''

Gold remains furious about this. ``Multiple reviews got very hung up on
a line in the play where Laura says to Amanda, `I've just been going out
walking.' How could I put an actor onstage who uses a wheelchair when
the line says `walking?''' he recalls. ``My daughter and I say, `We're
going for a walk,' all the time. It was just a ridiculous complaint, as
if that word should be owned only by people who are ambulatory.''

Image

Madison Ferris (left) and Sally Field in the 2017 production of ``The
Glass Menagerie'' at the Belasco Theater.Credit...Sara Krulwich/The New
York Times

``It shook me how the only thing some people wanted to talk about was my
physicality and not my actual performance,'' adds Ferris. ``I think one
person used the word `distracting.''' She tried to ignore that kind of
commentary; in fact, she especially came to love a piece of staging that
developed after Gold asked her if she would be willing to get out of her
chair and pull herself up a short flight of stairs that led from just in
front of the first row of seats to the stage, an action that is second
nature to Ferris but can read to others as a nerve-rattling
demonstration of both upper-body strength and sheer will --- something
that most people might ask, while witnessing it, if they could manage
themselves. ``I just looked at him like, `This is my everyday life,'''
she says. ``If I go to a bar or a restaurant with my friends, this is
what I have to do.'' The first time she did it in performance, she could
feel the silence of the audience, its held breath, its collective
anxiety, the tension not just in her own body but in the bodies of rows
upon rows of theatergoers simultaneously leaning forward and beginning
to comprehend, perhaps for the first time, the sheer ordinariness of
physical challenge. ``It's juicy to have that much power,'' she says,
``to know that you are showing a thousand people something they've never
seen.'' When I watched the scene, I was among those who were struck by
Ferris's courage. Only later did I realize that reading the moment as
``brave'' is simply the mirror image of reading it as distracting, of
watching \emph{her} rather than her performance.

ALI STROKER SPRINTS up the ramp behind the stage at Lincoln Center's
Appel Room, takes the mic and, before a full house with the evening
lights of Central Park South glittering behind her, starts to sing. Over
the next hour of her concert, presented as part of the American Songbook
series, she brings her big, supple voice to more than a dozen tunes,
from Stephen Sondheim's ``Everybody Says Don't'' to ``Never Never Land''
to the Dolly Parton hit ``Here You Come Again'' to the song that she has
come to own: ``\href{https://www.youtube.com/watch?v=FxFS8okUqKk}{I
Cain't Say No},'' from Rodgers and Hammerstein's
``\href{https://www.nytimes3xbfgragh.onion/2019/04/07/theater/oklahoma-review.html}{Oklahoma!}''

Last year, Stroker, who uses a wheelchair, became a sensation as Ado
Annie in the show's Broadway revival; audiences were stunned and
delighted not just by her range but by the apparently effortless energy
with which, from her first moments, she tore into the role. Here, for
the first time, was a musical in which the disabled person was presented
as the most confident presence onstage; we could not watch or listen to
Stroker without realizing how thoroughly the rest of popular culture has
trained us away from seeing energy and sexuality and brio and cockiness
when we see a disabled person. Her performance --- not a triumph over
odds, just a triumph --- was an education, the same way Shannon DeVido's
leading role in the upcoming indie musical ``Best Summer Ever'' or Way's
caustic contempt in ``Ramy'' are: an education that doesn't require the
performers to serve as educators --- these actors assume that their
reality is our job to understand, not theirs to explain.

Image

Ali Stroker (center) as Ado Annie in the 2019 production of the musical
``Oklahoma'' at the Circle in the Square Theater.Credit...Sara
Krulwich/The New York Times

Last year, Stroker's performance
\href{https://www.nytimes3xbfgragh.onion/2019/06/12/theater/ali-stroker-oklahoma-tony.html}{won
her a Tony}. To many disabled theater fans, it was a galvanizing moment,
a giant honor for a disabled performer who was fully integrated into a
show that treated her means of personal conveyance as nothing more or
less than a fact of life. But some noted the irony of the fact that at
Radio City Music Hall, Stroker had to wait backstage rather than in the
audience when her category was up to see if she won; there was no ramp
from the orchestra. With only the slightest note of weariness, Stroker,
33, tells me that's not how it actually was. There were, she says, hours
of discussion about how to build a ramp from the orchestra aisle to the
stage but no way to do it without obstructing several rows of seats. ``I
think people wanted a certain kind of accessibility to be provided, and
in a dream world, it would have been,'' she says. ``There are
opportunities to be creative to solve problems, and I would rather put
my energy in that than in feeling that we've been betrayed.'' Stroker is
an advocate as well as an actress, and complaining about the night where
she got to sing on television and win her field's greatest prize is not
how she's constructed. Still, she gets it: There is a split among
disabled actors, as there is in most advocacy communities, between ``We
are making things better'' and ``What's taking so long?'' (There is a
split over many things, even the word ``disability.'') But these aren't
internal rifts so much as reflections of varied tastes, backgrounds,
politics and ideological approaches.

Shinn says it's ``decades'' to the finish line, which sounds bleak until
you ask yourself what minority group has ever gotten to what it would
define as a finish line. Stroker is sunnier; she's looking ahead to a
career path that will allow her to ``tell authentic stories about people
who have not been represented in movies or on television or on the
stage.'' But there's not necessarily a consensus on what the next
breakthrough should be: More casting of actors with disabilities in
roles that aren't defined by the disability? More stories about disabled
people that present their lives with richness and dimension? More
disabled producers, directors, writers and showrunners to create and
support that work?

All of the above, perhaps, along with an understanding that all
disability is not interchangeable. In 2018, the
\href{https://www.nationaldisabilitytheatre.org/}{National Disability
Theater} was founded, with an advisory board that includes performers
across the full range of physical and developmental disabilities. The
N.D.T., which plans to partner on co-productions with several regional
American theaters, has already commissioned plays that tell stories
``through a lens of disability culture,'' and Shinn is writing one with
a main character whose disability isn't specified in the script but, he
admits, will probably have to be physical rather than intellectual,
since the character is very verbal. Dilemmas like that underscore some
of the challenges of creating such a big tent, but everyone involved
seems to be aware that, in activism, ``stronger together'' doesn't elide
individual differences. They're also familiar with the fact that in
casting, ``diversity'' can too easily translate into ``any disability
will do.'' Mozgala has attended casting calls for disabled actors where
``you will get amputees, you will get people who are deaf, people who
are blind, people who have C.P. or other issues. That's a huge spectrum,
and what it says to me is that those people don't really know what
they're asking for and don't understand the community.''

Image

Lauren Ridloff (left) and Joshua Jackson in the 2018 production of the
play ``Children of a Lesser God'' at Studio 54.Credit...Sara
Krulwich/The New York Times

If a consensus is forming around one principle, it may be, as Wailes
puts it, ``Nothing about us without us.'' When Mozgala was 10 or 11, he
remembers being taken by his mother to see Daniel Day-Lewis play the
Irish writer and painter Christy Brown (whose severe case of cerebral
palsy impaired his speech and almost completely paralyzed him) in
``\href{https://www.nytimes3xbfgragh.onion/watching/titles/my-left-foot}{My
Left Foot}'' (1989) and feeling ``that was the first time I'd ever
really seen anything close to my experience reflected.'' Day-Lewis won
his first Academy Award for the role, but would it still be OK with
Mozgala to see the part of a disabled man cast that way? ``I don't think
so,'' he says. ``Maybe a thousand days ago,'' which is not long after
Eddie Redmayne won an Oscar for playing Stephen Hawking, whose mobility
and speech were limited because of amyotrophic lateral sclerosis, in
``\href{https://www.nytimes3xbfgragh.onion/2014/11/07/movies/in-the-theory-of-everything-stephen-hawkings-home-life.html}{The
Theory of Everything}'' (2014). This year is, he points out,
\href{https://www.nytimes3xbfgragh.onion/interactive/2020/us/disability-ADA-30-anniversary.html}{the
30th anniversary of the passage of the Americans With Disabilities Act
(ADA)}, the landmark civil rights law that mandates equal access to
employment, transportation, education and public and private spaces.
``Enough is enough,'' Mozgala says. ``Do the work. Find the people.
They're out there.'' And yet, there remains a catch-22: Films that
showcase characters with disabilities need financing, financiers want
stars --- and stars don't have disabilities because people with
disabilities aren't given the chance to become stars. That can change
only by a meaningful partnership between advocates, allies and the
industry that involves one of the least popular words in the activist
vocabulary: ``incrementalism,'' the tedious, patience-taxing,
water-dripping-on-a-rock process of reaching out to one more casting
director, of importuning one more producer to have a second thought.

It's a job that Mozgala, who is often used as a kind of informational
clearing house by casting directors and producers, has been doing,
largely pro bono, for years. So has Nic Novicki, a 38-year-old actor,
comedian and producer with pseudoachondroplasia dwarfism best known from
``\href{https://www.nytimes3xbfgragh.onion/watching/recommendations/watching-tv-boardwalk-empire2}{Boardwalk
Empire}.'' Seven years ago, Novicki founded the
\href{https://disabilityfilmchallenge.com/}{Easterseals Disability Film
Challenge}, in which aspiring moviemakers are invited to shoot a
three-to-five-minute film over a weekend with a volunteer crew and at
least one person with a physical or cognitive disability in a principal
acting or creative role. The shorts --- last year there were 71 of them
--- are then posted on Facebook and YouTube for public viewing, and
eventually the best of them are given awards. (This year's challenge was
changed to a make-a-documentary-at-home contest because of the
pandemic.) ``I got pretty lucky, but a lot of the opportunities I've had
stem from projects that I created myself,'' Novicki says. ``Work leads
to work, but if you don't have that first opportunity, networks and
studios get nervous about giving people that shot.''

It's a fight on two fronts --- ensuring that, whenever possible,
characters with disabilities are played by actors with disabilities, but
also that disabled actors are always considered for the vast range of
roles for which ability or disability is not an issue. The first battle
is somewhat akin to an ongoing argument in the L.G.B.T.Q. community
about whether only queer actors should play queer roles. The comparison
doesn't entirely hold up: Sexual self-definition is more fluid, and
there are serious ethical and legal issues that arise when asking job
candidates to reveal their orientation. But the same counterarguments
--- What if the character is an abled person who becomes disabled? What
if the production requires a famous person? What if no disabled actor is
right for the role? --- are often used as a cover for a still-pervasive
attitude: \emph{Why do we have to think about this?} And the most common
response of all --- \emph{Shouldn't any actor be able to play any role?}
--- leads directly to the second battlefront: If that's the case, why
does it simply never occur to many networks, studios, producers or
casting directors to cast, or even consider, actors with disabilities in
roles that don't specify whether a character is disabled or not?

Image

Linda Bove (left) and friends on ``Sesame Street,'' circa
1987.Credit...© Sesame Workshop/Everett Collection

Image

Warwick Davis in the 1988 film ``Willow.''Credit...© MGM/Everett
Collection

SOME OF THE actors have been waiting decades for that to change. Ridloff
remembers watching the performer
\href{https://www.youtube.com/watch?v=GbfMzPuqSro}{Linda Bove on
``Sesame Street,''} who was one of the few deaf people children growing
up in the 1980s could find on television. And when Novicki saw the Ron
Howard fantasy film
``\href{https://www.nytimes3xbfgragh.onion/1988/05/20/movies/review-film-willow-a-george-lucas-production.html}{Willow}''
(1988), which stars the 3-foot-6 actor Warwick Davis, ``it made a huge
impact,'' he says. ``For me, as a little kid, to be able to see myself
as a hero'' was significant enough to nudge him away from sports and
toward theater. Now, he hopes to star as the pioneering actor with
dwarfism
\href{https://www.nytimes3xbfgragh.onion/2000/12/27/us/billy-barty-76-diminutive-actor-and-an-advocate-for-dwarfs.html}{Billy
Barty} in a script he's written that made the Disability List, an
offshoot of the annual Black List, an industry roundup of hot
screenplays. He believes progress is accelerating. So does Gold, who, at
the time we spoke, was getting ready to rehearse a now-unscheduled Off
Broadway production of ``Three Sisters,'' in which he had cast two
actors with disabilities. And ever since a new version of Charles
Dickens's ``A Christmas Carol'' debuted in London in 2017,
\href{https://www.nytimes3xbfgragh.onion/2019/11/13/theater/tiny-tim-a-christmas-carol-disabled-actors.html}{a
series of disabled actors has played Tiny Tim}, a character who has
almost never been authentically cast. When the production came to New
York last year, two young boys with cerebral palsy shared the part.
``The time between Madison playing the first wheelchair user in a
leading role on Broadway to Ali winning the Tony was only a couple of
seasons,'' Gold says. ``Things are moving faster now.'' At a moment when
all of theater is shut down and its failures of inclusion and
representation have become the subject of heated debate, perhaps it's
not too much to hope that any vision for a better future will be
broadened to include actors with disabilities.

Ferris, for her part, spent the beginning of 2020 in Austin, Texas,
filming the first season of a new Y.A. series for Amazon called
``\href{https://deadline.com/2019/05/amazon-studios-picks-up-lauren-oliver-ya-drama-panic-series-1202616616/}{Panic},''
about a group of young people desperate to win a competition that will
get them out of their small town. ``Is it getting better?'' she asks.
``Yeah, but what's the goal? If we want to make art that fully
represents everyone, then we're really far behind.''

``Sometimes, I think people look at entertainment as escapism,'' she
says. ``Like, they want to watch a heist movie or they want to watch hot
people have sex, and that's cool. But people with disabilities are also
hot, and they also have sex, and they can probably steal things way
better. Honestly, the things I've smuggled into a prison, it's crazy.''
She's not kidding. She has a friend who's behind bars, and you can stow
stuff in a wheelchair's seat-back pocket, and you'd be surprised at the
number of people who see a wheelchair and instantly decide they have you
all figured out, and \ldots{} Anyway, it's a good story. Perhaps one day
she'll get to tell it onscreen.

Advertisement

\protect\hyperlink{after-bottom}{Continue reading the main story}

\hypertarget{site-index}{%
\subsection{Site Index}\label{site-index}}

\hypertarget{site-information-navigation}{%
\subsection{Site Information
Navigation}\label{site-information-navigation}}

\begin{itemize}
\tightlist
\item
  \href{https://help.nytimes3xbfgragh.onion/hc/en-us/articles/115014792127-Copyright-notice}{©~2020~The
  New York Times Company}
\end{itemize}

\begin{itemize}
\tightlist
\item
  \href{https://www.nytco.com/}{NYTCo}
\item
  \href{https://help.nytimes3xbfgragh.onion/hc/en-us/articles/115015385887-Contact-Us}{Contact
  Us}
\item
  \href{https://www.nytco.com/careers/}{Work with us}
\item
  \href{https://nytmediakit.com/}{Advertise}
\item
  \href{http://www.tbrandstudio.com/}{T Brand Studio}
\item
  \href{https://www.nytimes3xbfgragh.onion/privacy/cookie-policy\#how-do-i-manage-trackers}{Your
  Ad Choices}
\item
  \href{https://www.nytimes3xbfgragh.onion/privacy}{Privacy}
\item
  \href{https://help.nytimes3xbfgragh.onion/hc/en-us/articles/115014893428-Terms-of-service}{Terms
  of Service}
\item
  \href{https://help.nytimes3xbfgragh.onion/hc/en-us/articles/115014893968-Terms-of-sale}{Terms
  of Sale}
\item
  \href{https://spiderbites.nytimes3xbfgragh.onion}{Site Map}
\item
  \href{https://help.nytimes3xbfgragh.onion/hc/en-us}{Help}
\item
  \href{https://www.nytimes3xbfgragh.onion/subscription?campaignId=37WXW}{Subscriptions}
\end{itemize}
