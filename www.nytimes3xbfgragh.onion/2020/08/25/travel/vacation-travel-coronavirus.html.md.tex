Sections

SEARCH

\protect\hyperlink{site-content}{Skip to
content}\protect\hyperlink{site-index}{Skip to site index}

\href{https://www.nytimes3xbfgragh.onion/section/travel}{Travel}

\href{https://myaccount.nytimes3xbfgragh.onion/auth/login?response_type=cookie\&client_id=vi}{}

\href{https://www.nytimes3xbfgragh.onion/section/todayspaper}{Today's
Paper}

\href{/section/travel}{Travel}\textbar{}The New Pandemic Flash Point:
Your Vacation

\url{https://nyti.ms/32nmhhl}

\begin{itemize}
\item
\item
\item
\item
\item
\item
\end{itemize}

\href{https://www.nytimes3xbfgragh.onion/spotlight/at-home?action=click\&pgtype=Article\&state=default\&region=TOP_BANNER\&context=at_home_menu}{At
Home}

\begin{itemize}
\tightlist
\item
  \href{https://www.nytimes3xbfgragh.onion/2020/09/07/travel/route-66.html?action=click\&pgtype=Article\&state=default\&region=TOP_BANNER\&context=at_home_menu}{Cruise
  Along: Route 66}
\item
  \href{https://www.nytimes3xbfgragh.onion/2020/09/04/dining/sheet-pan-chicken.html?action=click\&pgtype=Article\&state=default\&region=TOP_BANNER\&context=at_home_menu}{Roast:
  Chicken With Plums}
\item
  \href{https://www.nytimes3xbfgragh.onion/2020/09/04/arts/television/dark-shadows-stream.html?action=click\&pgtype=Article\&state=default\&region=TOP_BANNER\&context=at_home_menu}{Watch:
  Dark Shadows}
\item
  \href{https://www.nytimes3xbfgragh.onion/interactive/2020/at-home/even-more-reporters-editors-diaries-lists-recommendations.html?action=click\&pgtype=Article\&state=default\&region=TOP_BANNER\&context=at_home_menu}{Explore:
  Reporters' Google Docs}
\end{itemize}

Advertisement

\protect\hyperlink{after-top}{Continue reading the main story}

Supported by

\protect\hyperlink{after-sponsor}{Continue reading the main story}

\hypertarget{the-new-pandemic-flash-point-your-vacation}{%
\section{The New Pandemic Flash Point: Your
Vacation}\label{the-new-pandemic-flash-point-your-vacation}}

What kind of travel, if any, is acceptable? Opinions on what's OK and
what's not are causing fights between family members and fissures among
friends.

\includegraphics{https://static01.graylady3jvrrxbe.onion/images/2020/08/21/travel/21travelfights/21travelfights-articleLarge.jpg?quality=75\&auto=webp\&disable=upscale}

\href{https://www.nytimes3xbfgragh.onion/by/tariro-mzezewa}{\includegraphics{https://static01.graylady3jvrrxbe.onion/images/2018/08/24/opinion/tariro-headshot/tariro-headshot-thumbLarge-v2.png}}

By \href{https://www.nytimes3xbfgragh.onion/by/tariro-mzezewa}{Tariro
Mzezewa}

\begin{itemize}
\item
  Aug. 25, 2020
\item
  \begin{itemize}
  \item
  \item
  \item
  \item
  \item
  \item
  \end{itemize}
\end{itemize}

Michael Huxley has been getting called out a lot lately. His sin?
Traveling during the coronavirus pandemic. Mr. Huxley flew to Spain from
Liverpool a few weeks ago and has been on a handful of trips within
Britain since the onset of the pandemic, upsetting friends, family and
strangers, who say he should stay home in order to lessen the risk of
contracting or spreading the virus.

``I've been getting criticism in my professional life and from people in
my personal life,'' said Mr. Huxley, who runs the blog
\href{https://bemusedbackpacker.com/}{Bemused Backpacker.} ``Some come
at it from an ethical point of view and think I shouldn't be traveling
and spreading disease anywhere, and then others come from the emotional
`you shouldn't be traveling because you'll kill my grandma' point of
view.''

The decision to travel or stay home has become a flash point this
summer, with people defining what kind of travel, if any, is acceptable
in different ways.

Some people say that people should only go on essential trips. Others
say pleasure trips within driving distance are acceptable. Others, like
Mr. Huxley, who is from Liverpool, say traveling is fine, as long as
travelers follow rules like washing hands and maintaining a clean
environment and keeping distance between themselves and others. The
various delineations of what's right and what's not are causing fights
between family members and creating fissures among friends.

``It was easier to ease my family, who know that I'm a qualified nurse,
that I've traveled the world for 20 years and can look after myself,''
Mr. Huxley said. ``But communicating to acquaintances and people who
don't know me that I have weighed the risks, that I have worked the
various ways I can reduce the risk for myself, and I am still choosing
to travel was impossible.''

Mr. Huxley said that he traveled during other crises, including the SARS
and MERS outbreaks, as well as in the period following the 9/11
terrorist attacks, and he was in Egypt during the 2011 revolution.

``I don't see this as any different from those events,'' he said. ``You
do get outbreaks, pandemics, terrorist attacks, but life goes on. Travel
still goes on.''

Erin Niimi Longhurst, a half-British, half-Japanese
\href{https://www.erinniimilonghurst.com/}{author and director} at a
digital agency in New York, received the silent treatment from her
mother for weeks after she traveled to London from New York this spring
--- a rare thing for the mother and daughter, who are close and
typically talk multiple times a day. Ms. Niimi Longhurst went to London
to be with her partner and relatives, upsetting her mother, who lives in
Hawaii and is not traveling. She stayed there for three months before
returning to New York. Ms. Niimi Longhurst's sister lives in New York
and just had a child.

``My mother really wanted to go and be with my sister, but had made the
decision not to,'' Ms. Niimi Longhurst said. ``Her mentality was, `why
is it OK for you to go back? If everyone acted like you, we'd be in a
worse situation.' She was incredibly worried for me and she was pretty
furious with me.''

Ms. Niimi Longhurst's mother isn't alone in her frustration. On Twitter
and Instagram, people have been venting about family, friends and
co-workers going on nonessential trips and causing friction in their
relationships.

\href{https://twitter.com/cw_janene/status/1281093891727347712}{One
woman wrote} on Twitter that her mother was insisting on flying from
Oakland, Calif., to Portland, Ore., to visit her, but would not be
allowed into the house if she did so. ``I told her, `absolutely not,'''
she wrote. ``We will not see you! I don't care if she comes knocking on
my door. She will not be allowed in.''

Another tweeted that she told her sister that she could not visit for
vacation: ``I said NO WAY! I told her I don't want her to board a plane,
pick up Covid and deposit it in my house. She mad, but I don't care. I
am HIGH risk and will not relent.''

And one woman wrote on Twitter that her stepson was coming to town with
his girlfriend. ``If it was MY kid I would say no but there would be
drama if I suggested the step son delay the trip,'' she wrote.

Jill Locke, a professor of political science at a college in Minnesota,
and her younger sister, Jennifer, who lives in California and is the
chief executive of a wine company, initially didn't see eye to eye about
visiting their parents, who are in their 80s, in Seattle this summer.
The sisters exchanged text messages and phone calls, with the younger
Ms. Locke pushing for the trip while her older sister couldn't justify
the prospect of traveling.

``We were coming at it from such different places,'' the older Ms. Locke
said. ``For many reasons, for me, it felt like it was the wrong thing to
do, even though I really wanted to see our parents, but she didn't feel
the same way.''

Before the pandemic, Ms. Locke planned to fly to Seattle from Minnesota
with her husband and children, but as the coronavirus spread across the
United States, she decided that she would rent an R.V. and drive there.
She soon realized that the cost of the R.V. would be prohibitive, and
felt that some states between Minnesota and Washington weren't taking
the virus seriously enough. In the end, both sisters decided to stay
home.

``Weighing all these contingencies made me wonder what I would be
bringing to my parents even if I traveled as responsibly as possible,''
the older Ms. Locke said. ``There have been a lot of texts between us,
and we both got so worked up and frustrated.''

Ms. Locke's sister said that she didn't take the prospect of traveling
lightly and has been following guidance to not travel during the
pandemic. Nonetheless, she felt that it was important that she see her
aging parents sooner rather than later.

``At the time, I felt like `if we don't go see our parents now, then
when will we?''' the younger Ms. Locke said. ``That's been the gutting
thing: Not knowing the answer to that. It feels like time is being
stolen from us.''

Lindsay Chambers, a \href{https://www.editingskills.com/}{writer and
editor} who lives in Nashville, said that she has been surprised by the
ways people are justifying going on vacation this year, including saying
that they can't pass up cheap flights and those who would not reschedule
bachelor and bachelorette parties. Ms. Chambers said she has barely left
her home since February, but she has been following local news and seen
images of people gathering at bars and popular tourist spots in downtown
Nashville. These tourists, she said, are not being considerate of
others. She was stunned to learn that her own friends were going on a
beach trip this summer.

``I had to stop myself from shouting at friends who told us they'd be
`quarantining at the beach,''' she said. ``Traveling to another state
and staying in a rented condo in the middle of a raging pandemic is not
how quarantine works. At all.''

Ms. Chambers, 41, also described being confounded and upset by how some
people manage to make her feel, like she's overreacting by following the
recommendations from doctors on health and safety. Other people have
also said they experienced this when they stay home while their friends
and family interpret the rules more loosely.

``Maybe there's a degree of paranoia for me, and you couldn't pay me to
get on a plane right now, but is there really such a thing as being too
careful in a pandemic?'' Ms. Chambers asked. ``I feel like it's way,
way, way better to come down on the side of caution.''

\begin{center}\rule{0.5\linewidth}{\linethickness}\end{center}

\emph{\textbf{Follow New York Times Travel}}
\emph{on}\href{https://www.instagram.com/nytimestravel/}{\emph{Instagram}}\emph{,}\href{https://twitter.com/nytimestravel}{\emph{Twitter}}
\emph{and}\href{https://www.facebookcorewwwi.onion/nytimestravel/}{\emph{Facebook}}\emph{.
And}\href{https://www.nytimes3xbfgragh.onion/newsletters/traveldispatch}{\emph{sign
up for our weekly Travel Dispatch newsletter}} \emph{to receive expert
tips on traveling smarter and inspiration for your next vacation.}

Advertisement

\protect\hyperlink{after-bottom}{Continue reading the main story}

\hypertarget{site-index}{%
\subsection{Site Index}\label{site-index}}

\hypertarget{site-information-navigation}{%
\subsection{Site Information
Navigation}\label{site-information-navigation}}

\begin{itemize}
\tightlist
\item
  \href{https://help.nytimes3xbfgragh.onion/hc/en-us/articles/115014792127-Copyright-notice}{©~2020~The
  New York Times Company}
\end{itemize}

\begin{itemize}
\tightlist
\item
  \href{https://www.nytco.com/}{NYTCo}
\item
  \href{https://help.nytimes3xbfgragh.onion/hc/en-us/articles/115015385887-Contact-Us}{Contact
  Us}
\item
  \href{https://www.nytco.com/careers/}{Work with us}
\item
  \href{https://nytmediakit.com/}{Advertise}
\item
  \href{http://www.tbrandstudio.com/}{T Brand Studio}
\item
  \href{https://www.nytimes3xbfgragh.onion/privacy/cookie-policy\#how-do-i-manage-trackers}{Your
  Ad Choices}
\item
  \href{https://www.nytimes3xbfgragh.onion/privacy}{Privacy}
\item
  \href{https://help.nytimes3xbfgragh.onion/hc/en-us/articles/115014893428-Terms-of-service}{Terms
  of Service}
\item
  \href{https://help.nytimes3xbfgragh.onion/hc/en-us/articles/115014893968-Terms-of-sale}{Terms
  of Sale}
\item
  \href{https://spiderbites.nytimes3xbfgragh.onion}{Site Map}
\item
  \href{https://help.nytimes3xbfgragh.onion/hc/en-us}{Help}
\item
  \href{https://www.nytimes3xbfgragh.onion/subscription?campaignId=37WXW}{Subscriptions}
\end{itemize}
