Sections

SEARCH

\protect\hyperlink{site-content}{Skip to
content}\protect\hyperlink{site-index}{Skip to site index}

\href{https://www.nytimes3xbfgragh.onion/section/books}{Books}

\href{https://myaccount.nytimes3xbfgragh.onion/auth/login?response_type=cookie\&client_id=vi}{}

\href{https://www.nytimes3xbfgragh.onion/section/todayspaper}{Today's
Paper}

\href{/section/books}{Books}\textbar{}In the Second Volume of `Hitler,'
How a Dictator Invited His Own Downfall

\url{https://nyti.ms/2EwlK4X}

\begin{itemize}
\item
\item
\item
\item
\item
\end{itemize}

Advertisement

\protect\hyperlink{after-top}{Continue reading the main story}

Supported by

\protect\hyperlink{after-sponsor}{Continue reading the main story}

\href{/column/books-of-the-times}{Books of The Times}

\hypertarget{in-the-second-volume-of-hitler-how-a-dictator-invited-his-own-downfall}{%
\section{In the Second Volume of `Hitler,' How a Dictator Invited His
Own
Downfall}\label{in-the-second-volume-of-hitler-how-a-dictator-invited-his-own-downfall}}

By \href{https://www.nytimes3xbfgragh.onion/by/jennifer-szalai}{Jennifer
Szalai}

\begin{itemize}
\item
  Aug. 26, 2020
\item
  \begin{itemize}
  \item
  \item
  \item
  \item
  \item
  \end{itemize}
\end{itemize}

\includegraphics{https://static01.graylady3jvrrxbe.onion/images/2020/08/27/books/bookullrich1/bookullrich1-articleLarge.png?quality=75\&auto=webp\&disable=upscale}

Buy Book ▾

\begin{itemize}
\tightlist
\item
  \href{https://www.amazon.com/gp/search?index=books\&tag=NYTBSREV-20\&field-keywords=Hitler\%3A+Downfall\%2C+1939-1945+Volker+Ullrich}{Amazon}
\item
  \href{https://du-gae-books-dot-nyt-du-prd.appspot.com/buy?title=Hitler\%3A+Downfall\%2C+1939-1945\&author=Volker+Ullrich}{Apple
  Books}
\item
  \href{https://www.anrdoezrs.net/click-7990613-11819508?url=https\%3A\%2F\%2Fwww.barnesandnoble.com\%2Fw\%2F\%3Fean\%3D9781101874004}{Barnes
  and Noble}
\item
  \href{https://www.anrdoezrs.net/click-7990613-35140?url=https\%3A\%2F\%2Fwww.booksamillion.com\%2Fp\%2FHitler\%253A\%2BDownfall\%252C\%2B1939-1945\%2FVolker\%2BUllrich\%2F9781101874004}{Books-A-Million}
\item
  \href{https://bookshop.org/a/3546/9781101874004}{Bookshop}
\item
  \href{https://www.indiebound.org/book/9781101874004?aff=NYT}{Indiebound}
\end{itemize}

When you purchase an independently reviewed book through our site, we
earn an affiliate commission.

The impulsiveness and grandiosity, the bullying and vulgarity, were
obvious from the beginning; if anything, they accounted for Adolf
Hitler's anti-establishment appeal. For Germany's unpopular conservative
elites, Hitler's energy and theatrics made him an enticing partner when
they appointed him chancellor on Jan. 30, 1933.

But anyone who thought the Nazis would be content with their share ---
that Hitler would rise to the occasion or be hemmed in by it, becoming a
dignified statesman who sought compromise --- was summarily purged from
the system that conservatives assumed they controlled. An utter
impossibility had become the indomitable reality. The Weimar Republic
had become the Third Reich. It would take another world war, a genocide
and millions of dead before the dictatorship finally collapsed in 1945,
a full 12 years after Hitler was invited into power.

In the second and final volume of his biography of Hitler, Volker
Ullrich argues that the very qualities that accounted for the dictator's
astonishing rise were also what brought about his ultimate ruin.
``Hitler: Downfall, 1939-1945'' arrives in English four years after the
publication of
\href{https://www.nytimes3xbfgragh.onion/2016/09/28/books/hitler-ascent-volker-ullrich.html}{``Hitler:
Ascent, 1889-1939.''} It's a biographical project that consumed eight
years of Ullrich's life and ``took a definite psychological toll,'' he
writes in his introduction to the second volume. Like the British
historian Ian Kershaw, who divided his own two-volume biography of
Hitler into
\href{https://archive.nytimes3xbfgragh.onion/www.nytimes3xbfgragh.onion/books/99/01/31/reviews/990131.31reich3t.html}{``Hubris''}
and
\href{https://archive.nytimes3xbfgragh.onion/www.nytimes3xbfgragh.onion/books/00/12/10/reviews/001210.10bur.html}{``Nemesis,''}
Ullrich suggests that the Hitlerian regime was capable of only two
registers: euphoria and despair. Hitler was shrewd about seizing power,
but he was too restless and reckless to govern. A Third Reich that
cultivated peaceful stability was simply unfathomable.

``Downfall'' begins just after Hitler's 50th birthday, with the Führer
entertaining thoughts of invading Poland as if it were a present to
himself. ``I have overcome the chaos in Germany, restored order and
hugely increased productivity in all areas of our national economy,'' he
bragged to the Reichstag, even if the actual situation was considerably
less stellar than he proclaimed. Years of enormous military expenditures
had pushed Germany to the brink of economic collapse. Hitler had made a
mess, and a war would clean it up. The idea, Ullrich writes, was to
``transfer the costs of this financial crisis to the peoples that
Germany was going to subjugate.''

At first, Hitler's standard approach --- lying, blaming others and
launching surprise attacks --- made for a successful wartime strategy.
Nobody seemed willing to believe that he would be so greedy and foolish
as to start an expansionist conflagration until he did. His propaganda
minister Joseph Goebbels instructed journalists to avoid the word
``war'' and make the invasion sound as if Germany were repulsing a
Polish attack, while Hitler was telling his minions that ``the Poles
need to get socked in the face.''

Hitler was who he was --- the question became what the people around him
were willing to do about it. The military commanders who voiced no
objections to the Polish invasion balked when Hitler decided to go to
war with the West, reassuring one another that they were determined to
``put the brakes'' on any disaster that was unfolding. But they were all
intention and no action. ``The final hope is that perhaps reason might
prevail in the end,'' one general confided to his wife.

Image

Volker Ullrich, whose new book completes his two-volume biography of
Hitler.Credit...Roswitha Hecke

Ullrich goes into detail when recounting the military history, depicting
war as the inevitable expression of Germany's fascist regime. In
Jefferson Chase's translation, the narrative moves swiftly, and it will
absorb even those who are familiar with the vast library of Hitler
books. To read ``Downfall'' is to see up close how Hitler lashed out ---
compulsively, destructively --- whenever he felt boxed in. He had the
instinct of a crude social-Darwinist who also liked to gamble,
experiencing the world only in terms of winning and losing. As he told
one of his skittish field marshals, ``I have gone for broke all my
life.''

And he felt boxed in all the time --- in peace but especially in war,
sending his troops to invade the Soviet Union in 1941, less than two
years after signing a nonaggression pact with Stalin. The reason Hitler
gave was couched in euphemisms like ``living space,'' but Ullrich
prefers to define Operation Barbarossa in terms of what it actually
started: ``A racist war of conquest and annihilation unparalleled in
human history.''

Hitler sometimes suggested he would be sated by exclusion and
exploitation. ``We will construct a gigantic wall separating Asia from
Europe,'' he promised. He declared that Slavs in occupied territories
would be used for slave labor, and that their children would be educated
only to the point where they could distinguish between German traffic
signs. But his ambitions, as always, became ever more extreme and
murderous, even if local authorities in the Third Reich had already been
competing among themselves to make themselves ``free of Jews.''

Hitler was a scattershot, undisciplined leader, prone to tardiness and
meandering monologues, but the one unwavering constant was his virulent,
fanatical anti-Semitism. He was continually railing against ``Jewish
Bolshevism'' or ``Jewish plutocracy,'' depending on whether he wanted to
emphasize the enemy to the East or the enemy to the West. As the war
dragged on, he started painting himself as the savior of Europe,
fulminating nonsensically but lethally against ``the
Jewish-capitalist-Bolshevik plot.''

As Ullrich points out, Hitler never issued a written order to
exterminate the Jews, because he didn't need to: He preferred to traffic
in generalities instead of specifics, verbally making his wishes known
so that his careerist minions could figure out the rest. ``Part of his
style of rule was to blur areas of responsibility and encourage
rivalries to remind everyone concerned of his position as the sole
ultimate arbiter,'' Ullrich writes. Kershaw called this ``working
towards the Führer.'' It was a method that allowed Hitler to feed his
vanity while also preserving the option to deflect any blame onto
others.

By 1941, Ullrich writes, Germany's defeat was already assured, but
Hitler would have none of it, getting rid of any military experts who
challenged him. He doubled down on his own pitilessness, even toward his
own people, saying that if they didn't fight ``they deserve to die
out.'' Following Hitler's lead, Goebbels treated the Germans like chumps
to be duped. ``There are so many lies that truth and swindle can
scarcely be distinguished,'' he noted with satisfaction in his diary
during the early stages of Barbarossa. ``That is best for us at the
moment.''

The truth did emerge in the end, but only after years of mass death and
cataclysmic destruction. Hitler had peddled so many lies that the
fantasy he created was stretched impossibly thin. For all his
pretensions to invincibility, he ended up a broken, sickly man, who
confronted the reality bearing down on him by killing himself in his
bunker. He had ordered his people to burn his body, so that only a few
charred bits of bone and pieces of dental work remained. As Ullrich puts
it, ``There was hardly anything else left of the man who at the height
of his career had fancied himself the ruler of the world.''

Advertisement

\protect\hyperlink{after-bottom}{Continue reading the main story}

\hypertarget{site-index}{%
\subsection{Site Index}\label{site-index}}

\hypertarget{site-information-navigation}{%
\subsection{Site Information
Navigation}\label{site-information-navigation}}

\begin{itemize}
\tightlist
\item
  \href{https://help.nytimes3xbfgragh.onion/hc/en-us/articles/115014792127-Copyright-notice}{©~2020~The
  New York Times Company}
\end{itemize}

\begin{itemize}
\tightlist
\item
  \href{https://www.nytco.com/}{NYTCo}
\item
  \href{https://help.nytimes3xbfgragh.onion/hc/en-us/articles/115015385887-Contact-Us}{Contact
  Us}
\item
  \href{https://www.nytco.com/careers/}{Work with us}
\item
  \href{https://nytmediakit.com/}{Advertise}
\item
  \href{http://www.tbrandstudio.com/}{T Brand Studio}
\item
  \href{https://www.nytimes3xbfgragh.onion/privacy/cookie-policy\#how-do-i-manage-trackers}{Your
  Ad Choices}
\item
  \href{https://www.nytimes3xbfgragh.onion/privacy}{Privacy}
\item
  \href{https://help.nytimes3xbfgragh.onion/hc/en-us/articles/115014893428-Terms-of-service}{Terms
  of Service}
\item
  \href{https://help.nytimes3xbfgragh.onion/hc/en-us/articles/115014893968-Terms-of-sale}{Terms
  of Sale}
\item
  \href{https://spiderbites.nytimes3xbfgragh.onion}{Site Map}
\item
  \href{https://help.nytimes3xbfgragh.onion/hc/en-us}{Help}
\item
  \href{https://www.nytimes3xbfgragh.onion/subscription?campaignId=37WXW}{Subscriptions}
\end{itemize}
