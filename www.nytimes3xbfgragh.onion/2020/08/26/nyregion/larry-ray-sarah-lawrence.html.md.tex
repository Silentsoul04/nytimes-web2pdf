Sections

SEARCH

\protect\hyperlink{site-content}{Skip to
content}\protect\hyperlink{site-index}{Skip to site index}

\href{https://www.nytimes3xbfgragh.onion/section/nyregion}{New York}

\href{https://myaccount.nytimes3xbfgragh.onion/auth/login?response_type=cookie\&client_id=vi}{}

\href{https://www.nytimes3xbfgragh.onion/section/todayspaper}{Today's
Paper}

\href{/section/nyregion}{New York}\textbar{}Suspect in Sarah Lawrence
Cult Case Is Accused of Witness Tampering

\url{https://nyti.ms/2YAaH1y}

\begin{itemize}
\item
\item
\item
\item
\item
\end{itemize}

Advertisement

\protect\hyperlink{after-top}{Continue reading the main story}

Supported by

\protect\hyperlink{after-sponsor}{Continue reading the main story}

\hypertarget{suspect-in-sarah-lawrence-cult-case-is-accused-of-witness-tampering}{%
\section{Suspect in Sarah Lawrence Cult Case Is Accused of Witness
Tampering}\label{suspect-in-sarah-lawrence-cult-case-is-accused-of-witness-tampering}}

Lawrence V. Ray, who is charged with sex trafficking and extortion,
tried to intimidate his daughter's former classmates from his jail cell,
prosecutors say.

\href{https://www.nytimes3xbfgragh.onion/by/benjamin-weiser}{\includegraphics{https://static01.graylady3jvrrxbe.onion/images/2018/07/16/multimedia/author-benjamin-weiser/author-benjamin-weiser-thumbLarge.png}}

By \href{https://www.nytimes3xbfgragh.onion/by/benjamin-weiser}{Benjamin
Weiser}

\begin{itemize}
\item
  Aug. 26, 2020
\item
  \begin{itemize}
  \item
  \item
  \item
  \item
  \item
  \end{itemize}
\end{itemize}

Prosecutors say Lawrence V. Ray was
\href{https://www.nytimes3xbfgragh.onion/2020/02/12/nyregion/larry-ray-sarah-lawrence.html?searchResultPosition=3}{a
Svengali-like figure}who moved into his daughter's dormitory at Sarah
Lawrence College and recruited some of her classmates into a bizarre,
cultlike group.

Over the next decade, Mr. Ray, 60, subjected the students to
psychological and physical abuse and even forced one into prostitution,
they have said.
\href{https://www.nytimes3xbfgragh.onion/2020/02/11/nyregion/larry-ray-sarah-lawrence-sex.html?searchResultPosition=1}{Charged
in February with sex trafficking and extortion}, Mr. Ray has been
awaiting trial in a Manhattan jail cell.

Now, though, prosecutors say, his case has taken a disturbing new turn.
The government says Mr. Ray has continued to try to control at least two
of his victims from inside his cell, speaking with his father on the
telephone, in coded language, and asking him to send the women messages,
trying to ensure their loyalty and even promising to marry one of them.

``These messages are plainly designed to tamper with witnesses and deter
these women from cooperating in the government's investigation,'' the
prosecutors wrote on Friday to the judge overseeing Mr. Ray's case.

Image

Lawrence RayCredit...U.S. Attorney's Office, via Associated Press

In one phone conversation in June, which prosecutors quoted in their
court papers, Mr. Ray told his father that the government wanted the
women to testify against him.

``That's never going to happen,'' Mr. Ray declared, adding that ``what
they're going on is not the truth.''

Mr. Ray, who has pleaded not guilty in the sex-trafficking case, has not
been charged with jury tampering, and his lawyers, who declined to
comment, are expected to respond to the new allegations in a court
filing late on Wednesday.

Even before he was charged in February, Mr. Ray was a figure of
intrigue, someone who was in and out of legal trouble and who knew
mobsters, politicians and high-ranking military officials.

In 1998, he was the best man at the wedding of Bernard B. Kerik, the
former New York police commissioner. In a New York magazine article last
year, Mr. Kerik was quoted, calling Mr. Ray ``a psychotic con man who
has victimized every friend he's ever had.''

In announcing charges against Mr. Ray in February, prosecutors said that
he had exploited his victims, initially at Sarah Lawrence, in Yonkers,
N.Y., and later at residences in Manhattan and Pinehurst, North
Carolina.

The authorities said he used psychological manipulation, offering
``therapy'' sessions in which he learned intimate details of their
private lives and mental health struggles under the pretense of helping
them.

According to the indictment, he later extorted hundreds of thousands of
dollars from his victims, relying on tactics like sleep deprivation,
sexual humiliation, verbal abuse and physical violence as he persuaded
them to make false confessions to damaging property or even trying to
kill him.

The investigation that resulted in the charges against him was prompted
by the article in New York magazine, titled
\href{https://www.thecut.com/2020/02/larry-ray-sarah-lawrence-students.html\#_ga=2.39120099.1540278468.1598467520-655201318.1598467520}{``The
Stolen Kids of Sarah Lawrence,}'' officials have said.

Mr. Ray's lawyers recently asked the judge, Lewis J. Liman of Federal
District Court, to temporarily release him into home incarceration,
citing what they called his ``near-total inability'' to meet with them
and review discovery materials because of strict jail conditions related
to the coronavirus pandemic.

The office of Audrey Strauss, the acting U.S. attorney in Manhattan,
objected to the request, writing to the judge that Mr. Ray's recent
calls to his father demonstrated ``ongoing efforts'' to influence and
tamper with victims, co-conspirators and potential witnesses.

Prosecutors said the calls, placed from the Metropolitan Correctional
Center in Manhattan, revealed that Mr. Ray was continuing to communicate
with the two young women who had been living with him before his arrest
and who had been witnesses to his ``pattern of victimization'' of others
over the years.

Ms. Strauss's office said it has evidence that Mr. Ray had physically
and verbally abused the two women, who were not identified. He also had
amassed videos that contained ``graphic sexual content of these two
women, including sexual acts performed at Ray's direction that appear
designed to debase and control them,'' the government said.

Since his arrest, the prosecutors said, Mr. Ray had kept the two women
``in his orbit,'' using his father as a conduit.

In their phone calls, Mr. Ray's father provided his son with ``frequent
reports about the women's whereabouts'' and often used ``coded
references to `company' or `friends' to indicate when the two women are
physically present during a call,'' the government wrote. In one call,
Mr. Ray's father told his son that the two women had moved to a location
near his home.

``Most troubling,'' the prosecutors wrote, ``the calls suggest that
Ray's ongoing communications are designed to ensure the ongoing loyalty
of these women and to inhibit their ability to detach from the influence
he commanded over them for nearly a decade.''

In one call in May, for example, Mr. Ray asked his father to tell the
women that they had ``signed on forever.'' In June, the government said,
Mr. Ray ``made clear his intent to keep the women isolated from others''
and to keep them under his control.

``The defendant admonished his father, seemingly in the presence of the
women, `No new friends. There should be no one in anybody's life except
each other,''' the prosecutors wrote.

Mr. Ray also tried to ensure the women's continued loyalty through
romantic commitments, the government wrote. Before his arrest, he had
described one of the women as his common-law wife and the other as akin
to a daughter. But in one call, prosecutors said, Mr. Ray instructed his
father to tell the one he considered a daughter that he would ``marry''
her and to ``make sure she knows that.''

The prosecutors asked the judge to keep Mr. Ray in jail and to order he
be prohibited from contacting potential witnesses.

Advertisement

\protect\hyperlink{after-bottom}{Continue reading the main story}

\hypertarget{site-index}{%
\subsection{Site Index}\label{site-index}}

\hypertarget{site-information-navigation}{%
\subsection{Site Information
Navigation}\label{site-information-navigation}}

\begin{itemize}
\tightlist
\item
  \href{https://help.nytimes3xbfgragh.onion/hc/en-us/articles/115014792127-Copyright-notice}{©~2020~The
  New York Times Company}
\end{itemize}

\begin{itemize}
\tightlist
\item
  \href{https://www.nytco.com/}{NYTCo}
\item
  \href{https://help.nytimes3xbfgragh.onion/hc/en-us/articles/115015385887-Contact-Us}{Contact
  Us}
\item
  \href{https://www.nytco.com/careers/}{Work with us}
\item
  \href{https://nytmediakit.com/}{Advertise}
\item
  \href{http://www.tbrandstudio.com/}{T Brand Studio}
\item
  \href{https://www.nytimes3xbfgragh.onion/privacy/cookie-policy\#how-do-i-manage-trackers}{Your
  Ad Choices}
\item
  \href{https://www.nytimes3xbfgragh.onion/privacy}{Privacy}
\item
  \href{https://help.nytimes3xbfgragh.onion/hc/en-us/articles/115014893428-Terms-of-service}{Terms
  of Service}
\item
  \href{https://help.nytimes3xbfgragh.onion/hc/en-us/articles/115014893968-Terms-of-sale}{Terms
  of Sale}
\item
  \href{https://spiderbites.nytimes3xbfgragh.onion}{Site Map}
\item
  \href{https://help.nytimes3xbfgragh.onion/hc/en-us}{Help}
\item
  \href{https://www.nytimes3xbfgragh.onion/subscription?campaignId=37WXW}{Subscriptions}
\end{itemize}
