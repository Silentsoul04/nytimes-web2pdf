A Cozy Cottage on an Island Northeast of Stockholm

\url{https://nyti.ms/2D12eNh}

\begin{itemize}
\item
\item
\item
\item
\item
\end{itemize}

\includegraphics{https://static01.graylady3jvrrxbe.onion/images/2020/08/30/t-magazine/design/30tmag-Halleroed-slide-BOQ8/30tmag-Halleroed-slide-BOQ8-articleLarge.jpg?quality=75\&auto=webp\&disable=upscale}

Sections

\protect\hyperlink{site-content}{Skip to
content}\protect\hyperlink{site-index}{Skip to site index}

\hypertarget{a-cozy-cottage-on-an-island-northeast-of-stockholm}{%
\section{A Cozy Cottage on an Island Northeast of
Stockholm}\label{a-cozy-cottage-on-an-island-northeast-of-stockholm}}

After years of creating avant-garde retail spaces for Scandinavian
fashion brands, a pair of designers has built a deceptively simple home
for themselves.

The untreated cedar-clad house sits in a landscape of pine trees and an
undergrowth of blueberry, lingonberry and heather.Credit...Nin Solis

Supported by

\protect\hyperlink{after-sponsor}{Continue reading the main story}

By
\href{https://www.nytimes3xbfgragh.onion/by/alice-newell-hanson}{Alice
Newell-Hanson}

Photographs by Nin Solis

\begin{itemize}
\item
  Aug. 26, 2020
\item
  \begin{itemize}
  \item
  \item
  \item
  \item
  \item
  \end{itemize}
\end{itemize}

AMONG THE MANY fashion boutiques that Christian and Ruxandra Halleroed
have designed is a series of nine stores for the Swedish clothing brand
\href{https://halleroed.com/project-index/acne-studios/}{Acne Studios},
some of which feel like industrial-size meat lockers, with soaring
monolithic brushed stainless-steel walls and floors of mottled poured
concrete, terrazzo or concrete-print carpet. The effect is not unfeeling
but bracing, in the vein of an old-fashioned Austrian spa: The severe
gray backdrop makes the clothes, which often come in pleasantly murky
colors, look more vibrant by comparison. This rigorously modern approach
to design --- often characterized by unbroken expanses of single
materials (a wall of burled elm in one store, diamond-embossed aluminum
in another) combined with inviting planes of color (a peony-pink
wall-to-wall carpet, for example, or a standing shelving unit in copper
sulfate blue) --- has made \href{https://halleroed.com/}{Halleroed}, the
design studio that Christian, 46, founded in Stockholm in 1998, and
which Ruxandra, 39, joined in 2015, sought after by fashion brands,
particularly young Scandinavian labels whose gameness for
experimentation resonates with the firm's own.

While the couple's stores often have an extraterrestrial feel, they are
equally informed by a love of raw materials and traditional Swedish
craft: For the fragrance and leather goods brand
\href{https://halleroed.com/projects/byredo-wooster-street-new-york/}{Byredo's
store in New York}, they installed glass brick walls and angular alder
shelves; for the women's line
\href{https://halleroed.com/project-index/toteme/}{Totême}, they skinned
a Stockholm townhouse with pale lime-wood walls. Christian originally
studied cabinetmaking and furniture design at Carl Malmstensskolan (now
part of Linkoping University), the school founded in the Swedish capital
in 1930 by the influential designer Carl Malmsten, who helped define
what's known today as Scandinavian style. After graduating in 1998, he
started creating his own furniture, mainly for Swedish office décor
companies, which eventually led to architectural and interior
commissions.

\includegraphics{https://static01.graylady3jvrrxbe.onion/images/2020/08/30/t-magazine/design/30tmag-Halleroed-slide-UCK0/30tmag-Halleroed-slide-UCK0-articleLarge.jpg?quality=75\&auto=webp\&disable=upscale}

Considering the meticulously controlled and distinctly urban look of
those projects, ``some people might be surprised by this house,'' says
Ruxandra of the cozy 1,100-square-foot country home that the couple
built for themselves and their 5-year-old daughter, Iolanda, on the
Swedish island of Blido in 2017. The couple chose this area because it's
close enough to their '60s-era apartment in central Stockholm that they
could drive up in just under two hours --- but far enough away to allow
them to unwind. ``There are fancier places closer to the city,'' says
Christian, ``but we wanted to have our own plot, rather than
neighbors.'' Though it's only 35 miles northeast of the capital, Blido
is among the most remote inhabited islands of the Stockholm archipelago,
the swirl of 30,000 or so specks that marble the surrounding Baltic Sea.
Its location makes it an ideal settlement for fishermen, who have lived
here since at least the 16th century. And while vacationers arrive in
the summer months, the island still feels like an intimate community.
There's a grocery store and a farm that sells sheepskin rugs; the houses
are mostly traditional Swedish cottages.

But the Halleroeds' home, on the less populous southern shore, appears
as an unvarnished cube raised slightly above the mossy forest floor,
surrounded by spare, lichen-speckled pines. Indeed, the structure is
not, technically, of this place: Since the couple's work schedule
wouldn't allow them to closely supervise the construction, they had the
house prefabricated to their specifications by a factory in Slovenia and
shipped to the island in giant numbered chunks. With its honey-color
cedar-plank exterior and a standing-seam aluminum hip roof, it still
feels more organic than their firm's work --- and yet it has the same
emphasis on craft and natural materials, most prominently wood. The cozy
interior, clad in raw knotted spruce, is united by a glossy
oxblood-red-painted spruce-board floor (a nod to
\href{https://www.nytimes3xbfgragh.onion/2015/09/06/travel/sweden-islands-runmaro.html}{Falu}
red, the hematite-rich pigment that's been used to paint Swedish houses
since the 18th century) that runs throughout the building, which is
divided down the middle by a 30-foot-long wall. Twenty-three feet high
at its tallest point, it was constructed almost entirely from two vast
four-inch-thick sheets of cross-laminated spruce planks that meet in the
middle above a doorway. On the eastern side are two compact bedrooms, a
bathroom and an approximately 215-square-foot sleeping loft with
mattresses for guests; to the west is a double-height open-plan living
space in which various zones flow into one another across a gentle split
level: A small step leads up from a reading area centered around a
wood-burning cast-iron fireplace to a dining area, kitchen and two
seating nooks with built-in sheepskin- and linen-topped Swedish pine
benches.

Image

In the living room, a linen-covered sofa by~Halleroed, vintage Finnish
stools and a vintage Swedish rug.Credit...Nin Solis

As Ruxandra rolls out dough for a blueberry pie in the kitchen ---
comprising a bank of walnut cabinets and appliances built into the
central partition --- she nods toward a six-foot-wide bean-shaped cutout
in the wall above her to illustrate the more improvisational approach
the couple took to designing a space for themselves. Both Christian and
Ruxandra, who trained as an architect at the KTH Royal Institute of
Technology in Stockholm, typically gravitate toward straight lines and
symmetry, ``but we started with a square window for the sleeping loft
and it was just too boring,'' she says. Midway through the design
process, she sketched a kidneylike shape on the plan as a placeholder,
and neither of them ever revised it. ``Usually,'' she says, ``we are a
little more strict.''

FOR ALL ITS otherness, however, the home ultimately yields to the
surrounding forest. The couple selected the piece of land, just over
half an acre, because of its proximity to the Baltic, then positioned
the house so it would look out over a mossy outcropping of granite. Each
side of the building is punctuated with varying styles and sizes of
plate-glass windows --- nine in total --- so that even in the gloom of
midwinter, the spruces beyond are framed like pictures on the walls.
Spanning the entire southern side of the main room is a 10-foot-wide
pane that provides glimpses of the sea; flanked by a 15-foot-wide
sliding glass door to the west and a hinged glass door to the east (both
of which lead outside), it creates the impression that this part of the
cabin --- where the family often enjoys a midafternoon fika --- is a
pergola, open to the woods. In the bathroom, where glossy maroon wall
tiles and a burgundy red jasper marble floor mimic the painted wooden
floors throughout the rest of the house, a glass door allows guests to
walk straight into the shower from the outside when they return from
swimming in the sea in the summer or foraging for mushrooms in the fall.

Image

In the bathroom, a sink by Jasper Morrison, English ceramic tiles, a
stool by Carl Malmsten and red jasper marble from Italy for the
floor.Credit...Nin Solis

Image

The master bed is covered in a Celine blanket and a Swedish sheepskin
from Svensk Hemslojd, and the ceiling light is by Vanja Sorbon Malmsten,
great-granddaughter of Carl Malmsten.Credit...Nin Solis

To further blur the distinction between inside and outside, square
chunks have been cut from three of the home's four corners to create
small porches sheltered beneath the eaves of the roof. The southeast
corner is arranged with sturdy square-sided Swedish pine armchairs (to
be draped with local reindeer pelts in winter) and a small round pine
table --- all pieces the couple originally designed for a lounge at the
Nordic Museum in Stockholm in 2018. On the southwest porch is a long
pine table --- built by the couple according to a design from the
Italian Modernist artist Enzo Mari's 1974 book ``Autoprogettazione?''
--- where the family often eats meals in the warmer months. What little
furniture the Halleroeds didn't make themselves came from local antique
dealers, another way the house pays tribute to Scandinavian midcentury
design. The home's irregular notched floor plan, in particular, was
influenced by one of the forefathers of Swedish Modernism, the
Austrian-born architect and designer Josef Frank, who made much of his
most important work in Sweden starting in the 1930s --- and who
conceived of a series of houses in 1947, some of which featured
asymmetrical volumes beneath rectangular roofs. ``It's important to know
your history,'' says Ruxandra. ``It might not be directly reflected in
your work, but it influences your mind-set.''

Image

Halleroed's pine furniture, produced by the Swedish manufacturer Tre
Sekel, and a tray Christian Halleroed made when he was 8 years
old.Credit...Nin Solis

Despite the couple's experimental approach, though, the house is above
all a homage to the traditional dwellings of Christian's youth: His
parents owned a small pine-walled cabin --- the kind that Swedes have
built for centuries --- in Salen, about six hours northwest of
Stockholm, where the family would vacation each winter. The angles of
the Halleroeds' home on Blido might be sharper, the palette and
furnishings more austere, but its materials and purpose are the same; it
is an ode to Sweden's woodworking tradition and a refuge from where they
can enjoy the forested landscapes from which that legacy derives. ``The
walls may look raw now,'' Ruxandra says, as she serves her finished pie,
heavy with wild berries, ``but in a few years, when the wood ages, it
will have just the same look.'' Just like the trees that surround them:
older and grander every year, but always recognizably themselves.

Advertisement

\protect\hyperlink{after-bottom}{Continue reading the main story}

\hypertarget{site-index}{%
\subsection{Site Index}\label{site-index}}

\hypertarget{site-information-navigation}{%
\subsection{Site Information
Navigation}\label{site-information-navigation}}

\begin{itemize}
\tightlist
\item
  \href{https://help.nytimes3xbfgragh.onion/hc/en-us/articles/115014792127-Copyright-notice}{©~2020~The
  New York Times Company}
\end{itemize}

\begin{itemize}
\tightlist
\item
  \href{https://www.nytco.com/}{NYTCo}
\item
  \href{https://help.nytimes3xbfgragh.onion/hc/en-us/articles/115015385887-Contact-Us}{Contact
  Us}
\item
  \href{https://www.nytco.com/careers/}{Work with us}
\item
  \href{https://nytmediakit.com/}{Advertise}
\item
  \href{http://www.tbrandstudio.com/}{T Brand Studio}
\item
  \href{https://www.nytimes3xbfgragh.onion/privacy/cookie-policy\#how-do-i-manage-trackers}{Your
  Ad Choices}
\item
  \href{https://www.nytimes3xbfgragh.onion/privacy}{Privacy}
\item
  \href{https://help.nytimes3xbfgragh.onion/hc/en-us/articles/115014893428-Terms-of-service}{Terms
  of Service}
\item
  \href{https://help.nytimes3xbfgragh.onion/hc/en-us/articles/115014893968-Terms-of-sale}{Terms
  of Sale}
\item
  \href{https://spiderbites.nytimes3xbfgragh.onion}{Site Map}
\item
  \href{https://help.nytimes3xbfgragh.onion/hc/en-us}{Help}
\item
  \href{https://www.nytimes3xbfgragh.onion/subscription?campaignId=37WXW}{Subscriptions}
\end{itemize}
