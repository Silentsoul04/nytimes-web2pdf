Sections

SEARCH

\protect\hyperlink{site-content}{Skip to
content}\protect\hyperlink{site-index}{Skip to site index}

\href{https://www.nytimes3xbfgragh.onion/section/world/europe}{Europe}

\href{https://myaccount.nytimes3xbfgragh.onion/auth/login?response_type=cookie\&client_id=vi}{}

\href{https://www.nytimes3xbfgragh.onion/section/todayspaper}{Today's
Paper}

\href{/section/world/europe}{Europe}\textbar{}Schools Can Reopen,
Germany Finds, but Expect a `Roller Coaster'

\url{https://nyti.ms/32ooFEE}

\begin{itemize}
\item
\item
\item
\item
\item
\item
\end{itemize}

\hypertarget{school-reopenings}{%
\subsubsection{\texorpdfstring{\href{https://www.nytimes3xbfgragh.onion/spotlight/schools-reopening?name=styln-coronavirus-schools-reopening\&region=TOP_BANNER\&block=storyline_menu_recirc\&action=click\&pgtype=Article\&impression_id=1579b560-f2bd-11ea-8255-47bcf7935a1c\&variant=undefined}{School
Reopenings}}{School Reopenings}}\label{school-reopenings}}

\begin{itemize}
\tightlist
\item
  \href{https://www.nytimes3xbfgragh.onion/2020/09/08/us/school-districts-cyberattacks-glitches.html?name=styln-coronavirus-schools-reopening\&region=TOP_BANNER\&block=storyline_menu_recirc\&action=click\&pgtype=Article\&impression_id=1579b561-f2bd-11ea-8255-47bcf7935a1c\&variant=undefined}{Remote
  Learning Glitches}
\item
  \href{https://www.nytimes3xbfgragh.onion/2020/09/08/upshot/children-testing-shortfalls-virus.html?name=styln-coronavirus-schools-reopening\&region=TOP_BANNER\&block=storyline_menu_recirc\&action=click\&pgtype=Article\&impression_id=1579b562-f2bd-11ea-8255-47bcf7935a1c\&variant=undefined}{Limited
  Testing for Children}
\item
  \href{https://www.nytimes3xbfgragh.onion/2020/09/01/world/schools-reopen-globe-students.html?name=styln-coronavirus-schools-reopening\&region=TOP_BANNER\&block=storyline_menu_recirc\&action=click\&pgtype=Article\&impression_id=1579b563-f2bd-11ea-8255-47bcf7935a1c\&variant=undefined}{School
  Around the World}
\item
  \href{https://www.nytimes3xbfgragh.onion/interactive/2020/us/covid-college-cases-tracker.html?name=styln-coronavirus-schools-reopening\&region=TOP_BANNER\&block=storyline_menu_recirc\&action=click\&pgtype=Article\&impression_id=1579b564-f2bd-11ea-8255-47bcf7935a1c\&variant=undefined}{Tracking
  College Cases}
\end{itemize}

Advertisement

\protect\hyperlink{after-top}{Continue reading the main story}

Supported by

\protect\hyperlink{after-sponsor}{Continue reading the main story}

\hypertarget{schools-can-reopen-germany-finds-but-expect-a-roller-coaster}{%
\section{Schools Can Reopen, Germany Finds, but Expect a `Roller
Coaster'}\label{schools-can-reopen-germany-finds-but-expect-a-roller-coaster}}

With nations determined to return to in-person learning, many will have
trouble matching Germany's formula: fast and free testing, robust
contact tracing and low community spread.

\includegraphics{https://static01.graylady3jvrrxbe.onion/images/2020/08/27/world/00germany-schools-TOP/merlin_175850829_21f15e38-087a-43ae-98c8-2f004b6c0496-articleLarge.jpg?quality=75\&auto=webp\&disable=upscale}

\href{https://www.nytimes3xbfgragh.onion/by/katrin-bennhold}{\includegraphics{https://static01.graylady3jvrrxbe.onion/images/2018/07/13/multimedia/author-katrin-bennhold/author-katrin-bennhold-thumbLarge.png}}

By \href{https://www.nytimes3xbfgragh.onion/by/katrin-bennhold}{Katrin
Bennhold}

\begin{itemize}
\item
  Published Aug. 26, 2020Updated Aug. 31, 2020
\item
  \begin{itemize}
  \item
  \item
  \item
  \item
  \item
  \item
  \end{itemize}
\end{itemize}

BERLIN --- On the Monday after summer vacation, Dirk Kwee was as nervous
as he had ever been in 31 years of teaching. For the first time since
the pandemic hit, all 900 students at his Berlin school were back,
bursting with excitement.

The dreaded call came just two days later: A girl in sixth grade had the
coronavirus. Mr. Kwee hurried over to the gym where the other 31
students in her class were enjoying their first physical education
session in five months. They were sent home --- immediately.

On Thursday, the whole class got tested. On Friday, all the tests came
back negative. And on Monday, half the children were back in class. But
just as Mr. Kwee allowed himself a sigh of relief, a seventh grader
tested positive.

``It's been a total roller coaster,'' said Mr. Kwee, headmaster of the
Heinz-Berggruen secondary school.

That may be what
\href{https://www.nytimes3xbfgragh.onion/2020/08/26/upshot/school-reopening-partisan-divide.html}{returning
to school} looks like for the foreseeable future.

\includegraphics{https://static01.graylady3jvrrxbe.onion/images/2020/08/22/world/00germany-schools001/00germany-schools-articleLarge.jpg?quality=75\&auto=webp\&disable=upscale}

\href{https://www.nytimes3xbfgragh.onion/2020/08/31/world/europe/reichstag-germany-neonazi-coronavirus.html}{Germany},
like other countries that have managed the pandemic fairly well, was
quick to deploy widespread testing, effective contact tracing and tests
with rapid results. Crucially, that has helped keep the rate of
community transmission low.

So far, the lesson from Germany, Denmark and Norway, among the first
countries to start the new school year, is that schools can reopen and
remain open --- if they build on that kind of foundation. But most
countries, and most parts of the United States, simply can't match those
conditions.

As Americans anxiously debate how to reopen schools, and more campuses
cancel in-person lessons, Europe is a living laboratory. Despite a sharp
increase in coronavirus cases in recent weeks, even countries that were
badly hit last spring, like Italy, Spain, Britain and France, are
determined to return to regular classes this fall.

Germany, which was far less affected at the peak of the pandemic,
shuttered schools early on, then moved to a hybrid model of remote and
in-classroom learning. Class sizes were smaller, and strict
social-distancing rules helped keep infection numbers in check.

But now a new experiment is underway: Teachers and students have been
summoned back to classes, testing whether the new vigilance is enough.

Social distancing and face masks are mandatory on most school grounds,
but rarely inside classrooms,
\href{https://www.bbc.com/news/world-53877292}{despite recent advice
from the World Health Organization} that children 12 and over wear masks
when distancing is impossible. If students had to wear masks for several
hours a day, the argument in Germany goes, their ability to learn would
suffer.

Instead, schools aim to better ventilate classrooms and keep classes
separate so that each student has contact with only a few dozen others,
and outbreaks can be contained.

Germany's departure from the more cautious, part-time reopening strategy
is rooted partly in resource constraints: Like most countries, it has
too few teachers to split students into smaller classes and allow for
social distancing.

But several weeks into returning to school, educators and even
virologists who were skeptical about reopening say that early results
look hopeful. Despite individual infections popping up in dozens of
schools, there have been no serious outbreaks --- and no lasting
closures.

Berlin is a case in point: By the end of last week, 49 infections had
been recorded among teachers and students across the city. But thanks to
fast testing and targeted quarantines, no more than 600 students out of
some 366,000 have had to stay home on any given day. Of 803 schools,
only 39 have been affected.

``It's messy and imperfect and I would have liked to see more
precautions, but the main takeaway so far is: It's working,'' said
Sandra Ciesek, a virologist at the University Hospital of Frankfurt who
signed a statement by leading German virologists supporting the
reopenings.

``Every school that stays open is worth a lot,'' said Professor Ciesek,
whose own daughter started first grade this month.

Image

A mobile testing unit tests and interviews students at the school after
a classmate contracted the coronavirus.Credit...Lena Mucha for The New
York Times

In the United States, some policymakers have focused on the rate of
positive coronavirus tests among the general population, with some
saying it must be below 3 percent to safely reopen. The figure is under
1 percent in Germany, as it is in a handful of other nations and in New
York State.

But most places have far higher positive rates ---
\href{https://www.nytimes3xbfgragh.onion/interactive/2020/us/coronavirus-testing.html}{7
percent for the entire United States}, 8 percent in Spain and more than
40 percent in some Latin American countries.

Among the largest U.S. school systems, only New York City's plans to
reopen next month --- and even there, students will alternate in-person
and online classes. Masks will be required, and Mayor Bill de Blasio has
said the schools will stay closed if the positive test rate reaches 3
percent.

Mass testing has been crucial for countries like Germany, which has led
on many fronts in the pandemic, keeping the number of
\href{https://www.nytimes3xbfgragh.onion/2020/04/04/world/europe/germany-coronavirus-death-rate.html}{deaths
relatively low.}

Hospital and care home staff are tested regularly, people returning from
vacation in ``hot spots'' can get free tests and a positive result is
generally followed by quick contact tracing. Now that regular classes
have resumed, teachers are also offered free tests, even if they have no
symptoms.

Such practices, though imperfect, have helped reassure teachers, some of
whom were reluctant to return.

At the Heinz-Berggruen school in Berlin, the system proved effective in
preventing a wider outbreak. The infected sixth grader had no symptoms
but was tested because someone in her family had tested positive. That
relative was tested after tracing the contacts of someone else, who had
brought the virus home from vacation.

Image

Math class, with windows wide open, but few masks.Credit...Lena Mucha
for The New York Times

The morning after the girl's parents notified the school, a mobile
testing unit from the local health authority arrived at the school and
tested and interviewed all the children and teachers who had been near
her. After every test came back negative, half the class was allowed to
return to school. Those who had sat near the girl were told to
quarantine at home for 14 days.

Other European countries have taken notice.

In Italy, the government is making more than 2 million tests available
to teachers before the school year begins in September. The president of
the Lazio region, Nicola Zingaretti, recently
\href{https://www.ansa.it/sito/videogallery/italia/2020/08/20/scuola-zingaretti-al-governo-non-unora-vada-perduta_51c086d8-caeb-4e31-bd70-218045277d3b.html}{joined
teachers} being tested in Rome. ``Safe schools means testing, and we
have started,'' he said.

``This is how you shut down infection chains and prevent outbreaks,''
said Professor Ciesek, the German virologist. But she cautioned: ``It
only works if community transmission rates in society overall are
manageable.''

For now that is still the case, officials say, but infections have been
rising again across Europe. Germany is averaging more than 1,300 new
cases daily, up from about 300 in early July, but far below the peak of
more than 5,500 in April.

Europe as a whole is averaging over 23,000 new cases a day, more than
double the number in early July, driven primarily by an enormous
resurgence in Spain. The United States has averaged more than 42,000 a
day over the past week.

Image

Students and teachers are required to wear masks on school grounds in
Germany, but not when they are in the classroom.Credit...Lena Mucha for
The New York Times

As infections rise, so do concerns about schools becoming hot spots.

The teachers' unions in the Madrid region have called for a strike to
protest the lack of safety measures in place ahead of the start of the
new school year. The Canary Islands government has postponed the end of
the summer break by two weeks to have more time to prepare.

In Italy, some headmasters threatened to shift to online classes if the
local government did not provide extra classrooms, desks and teaching
staff before the first day of school on Sept. 14.

\href{https://www.nytimes3xbfgragh.onion/spotlight/schools-reopening?action=click\&pgtype=Article\&state=default\&region=MAIN_CONTENT_3\&context=storylines_keepup}{}

\hypertarget{school-reopenings-}{%
\subsubsection{School Reopenings ›}\label{school-reopenings-}}

\hypertarget{back-to-school}{%
\paragraph{Back to School}\label{back-to-school}}

Updated Sept. 8, 2020

The latest on how schools are reopening amid the pandemic.

\begin{itemize}
\item
  \begin{itemize}
  \tightlist
  \item
    The first day of school was a rocky one in many places, as districts
    that started classes online dealt with
    \href{https://www.nytimes3xbfgragh.onion/2020/09/08/us/school-districts-cyberattacks-glitches.html?action=click\&pgtype=Article\&state=default\&region=MAIN_CONTENT_3\&context=storylines_keepup}{technical
    glitches, crashing websites and cyberattacks}.
  \item
    It's not easy to get a coronavirus test for a child. As schools
    reopen,
    \href{https://www.nytimes3xbfgragh.onion/2020/09/08/upshot/children-testing-shortfalls-virus.html?action=click\&pgtype=Article\&state=default\&region=MAIN_CONTENT_3\&context=storylines_keepup}{many
    parents still can't find one nearby}, impeding the fight against the
    pandemic.
  \item
    Life in a quarantine dorm: Colleges are trying to
    \href{https://www.nytimes3xbfgragh.onion/2020/09/09/business/colleges-coronavirus-dormitories-quarantine.html?action=click\&pgtype=Article\&state=default\&region=MAIN_CONTENT_3\&context=storylines_keepup}{isolate
    students who have been exposed to the virus}, but they are running
    into a host of problems.
  \item
    Penn State football defines fall in State College, Pa.
    \href{https://www.nytimes3xbfgragh.onion/2020/09/09/sports/penn-state-college-football-canceled.html?action=click\&pgtype=Article\&state=default\&region=MAIN_CONTENT_3\&context=storylines_keepup}{What
    is the town without it}?
  \end{itemize}
\end{itemize}

Some German teachers point to Israel, where infections at a
\href{https://www.nytimes3xbfgragh.onion/2020/08/04/world/middleeast/coronavirus-israel-schools-reopen.html}{Jerusalem
high school quickly mushroomed} into the largest outbreak in a single
school in the country, ultimately infecting hundreds of students,
teachers and relatives.

``Israel scares me,'' said Doreen Siebernik, president of the Berlin
branch of the GEW, Germany's largest teachers' union. ``We're
undertaking this huge experiment in schools. But many colleagues don't
want to be part of that experiment.''

Image

A young student using a hand sanitizer dispenser at a school in
Berlin.Credit...Lena Mucha for The New York Times

Some parents, too, are uneasy.

``There will be schools where it will work, and others that will shut
down,'' said Stephan Wassmuth, head of Germany's parent association.
``It will become a gamble. But education shouldn't be a gamble.''

Image

The teachers' room at the school.Credit...Lena Mucha for The New York
Times

Stefanie Hubig, education minister of the southwestern state of
Rhineland-Palatinate and the president of a group representing all 16
state education ministers, said the most effective way to protect
schools has to do with the behavior of parents, teachers and students
outside the classroom.

``Before we think about closing schools again we should perhaps think
about closing bars or other large events,'' she said.

``The goal has to be that schools remain open,'' Ms. Hubig added. ``We
are learning every day. We need to be creative.''

That creativity is on full display in different corners of Europe.

In Italy, some students returning to school might be directed to
classrooms in local cinemas, church halls or even tents set up in school
parking lots, venues co-opted to ensure that a one-meter distance is
maintained.

Norway's government is using a traffic-light color code to indicate the
level of danger from the virus, with each color attached to a set of
guidelines for schools.

In England, the government has
\href{https://www.gov.uk/government/publications/actions-for-schools-during-the-coronavirus-outbreak/guidance-for-full-opening-schools\#section-1-public-health-advice-to-minimise-coronavirus-covid-19-risks}{asked
schools} to consider staggering schedules to give students space as they
come and go, a practice gaining ground in Germany, where teachers and
administrators are still working to find the best practices to keep
fully reopened schools safe.

Mr. Kwee, the headmaster at the Heinz-Berggruen school in Berlin,
received guidelines from the local government only three days before
classes started. When the first coronavirus case was discovered, no
contingency plans had been drawn up, let alone an email to parents with
instructions on how to respond to an infection.

Image

Students in the schoolyard at Heinz-Berggruen, a secondary school in
Berlin.Credit...Lena Mucha for The New York Times

Clara Felsenberg, 11, was in the gym with her classmates when Mr. Kwee
abruptly sent them home.

``I was really disappointed,'' she recalled. ``We had only been back in
class for a couple of days.''

The children filed into changing rooms one by one to get dressed and
call home. Soon a WhatsApp group among the parents lit up with anxiety
and confusion. Would the whole family have to quarantine? When and where
would they get tested?

Clara and others took the bus home, which was against protocol, but the
parents said they had not been told.

``It's work in progress,'' said Mr. Kwee, the headmaster. But he noted
one unexpected outcome of the crisis: ``I have students come up to me
and thanking me for a lesson,'' he said. ``That never ever used to
happen.''

Christopher F. Schuetze contributed reporting from Berlin, Raphael
Minder from Madrid, Elisabetta Povoledo from Rome, Eliza Shapiro from
New York, Henrik Pryser Libell from Oslo, Thomas Erdbrink from Amsterdam
and Ben Mueller from Britain.

Advertisement

\protect\hyperlink{after-bottom}{Continue reading the main story}

\hypertarget{site-index}{%
\subsection{Site Index}\label{site-index}}

\hypertarget{site-information-navigation}{%
\subsection{Site Information
Navigation}\label{site-information-navigation}}

\begin{itemize}
\tightlist
\item
  \href{https://help.nytimes3xbfgragh.onion/hc/en-us/articles/115014792127-Copyright-notice}{©~2020~The
  New York Times Company}
\end{itemize}

\begin{itemize}
\tightlist
\item
  \href{https://www.nytco.com/}{NYTCo}
\item
  \href{https://help.nytimes3xbfgragh.onion/hc/en-us/articles/115015385887-Contact-Us}{Contact
  Us}
\item
  \href{https://www.nytco.com/careers/}{Work with us}
\item
  \href{https://nytmediakit.com/}{Advertise}
\item
  \href{http://www.tbrandstudio.com/}{T Brand Studio}
\item
  \href{https://www.nytimes3xbfgragh.onion/privacy/cookie-policy\#how-do-i-manage-trackers}{Your
  Ad Choices}
\item
  \href{https://www.nytimes3xbfgragh.onion/privacy}{Privacy}
\item
  \href{https://help.nytimes3xbfgragh.onion/hc/en-us/articles/115014893428-Terms-of-service}{Terms
  of Service}
\item
  \href{https://help.nytimes3xbfgragh.onion/hc/en-us/articles/115014893968-Terms-of-sale}{Terms
  of Sale}
\item
  \href{https://spiderbites.nytimes3xbfgragh.onion}{Site Map}
\item
  \href{https://help.nytimes3xbfgragh.onion/hc/en-us}{Help}
\item
  \href{https://www.nytimes3xbfgragh.onion/subscription?campaignId=37WXW}{Subscriptions}
\end{itemize}
