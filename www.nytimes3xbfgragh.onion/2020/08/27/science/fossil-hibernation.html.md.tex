Sections

SEARCH

\protect\hyperlink{site-content}{Skip to
content}\protect\hyperlink{site-index}{Skip to site index}

\href{https://www.nytimes3xbfgragh.onion/section/science}{Science}

\href{https://myaccount.nytimes3xbfgragh.onion/auth/login?response_type=cookie\&client_id=vi}{}

\href{https://www.nytimes3xbfgragh.onion/section/todayspaper}{Today's
Paper}

\href{/section/science}{Science}\textbar{}250 Million Years Ago, They
Hibernated at the Bottom of the World

\url{https://nyti.ms/3b0T5Ri}

\begin{itemize}
\item
\item
\item
\item
\item
\end{itemize}

Advertisement

\protect\hyperlink{after-top}{Continue reading the main story}

Supported by

\protect\hyperlink{after-sponsor}{Continue reading the main story}

Trilobites

\hypertarget{250-million-years-ago-they-hibernated-at-the-bottom-of-the-world}{%
\section{250 Million Years Ago, They Hibernated at the Bottom of the
World}\label{250-million-years-ago-they-hibernated-at-the-bottom-of-the-world}}

In the tusks of creatures that lived before dinosaurs, paleontologists
found signs of hibernation-like metabolism.

\includegraphics{https://static01.graylady3jvrrxbe.onion/images/2020/09/01/science/27TB-HIBERNATION1/merlin_176236869_85bf48cc-0c7a-4cff-9bb7-aaea916f8897-articleLarge.jpg?quality=75\&auto=webp\&disable=upscale}

\href{https://www.nytimes3xbfgragh.onion/by/kenneth-chang}{\includegraphics{https://static01.graylady3jvrrxbe.onion/images/2018/02/16/multimedia/author-kenneth-chang/author-kenneth-chang-thumbLarge.jpg}}

By \href{https://www.nytimes3xbfgragh.onion/by/kenneth-chang}{Kenneth
Chang}

\begin{itemize}
\item
  Aug. 27, 2020
\item
  \begin{itemize}
  \item
  \item
  \item
  \item
  \item
  \end{itemize}
\end{itemize}

How to tell if something that died 250 million years ago hibernated when
it was alive?

After all, hibernation --- a state of reduced metabolism --- is a good
strategy for making it through long, harsh winters when food can be
scarce. Biologists would not be surprised that evolution figured this
out early in the history of life. But uncovering convincing evidence of
that is hard.

``As a paleontologist, what you're presented with is a pile of bones,''
said Christian A. Sidor, a professor of biology at the University of
Washington and curator of vertebrate paleontology at the Burke Museum in
Seattle. ``And that just tells you where the animal died. It doesn't
even tell you where the animal lived.''

But Dr. Sidor and Megan R. Whitney, a former graduate student who is now
a postdoctoral researcher at Harvard, believe they have good evidence of
hibernation behavior in an animal that lived in Antarctica a quarter of
a billion years ago --- before the age of dinosaurs.

This was a tumultuous time for life all around the planet, which was
recovering from
\href{https://www.nytimes3xbfgragh.onion/2018/12/07/science/climate-change-mass-extinction.html}{the
largest mass extinction ever on Earth}, marking the end of the Permian
geologic period and the beginning of the Triassic. Antarctica, then as
now, was near the South Pole, and might have provided something of a
haven from the cataclysm, often called the Great Dying. (The
\href{https://www.nytimes3xbfgragh.onion/2014/02/21/science/earth/Mass-Extinction-Permian-Period.html}{cause
of this extinction} is still being debated.)

Dr. Whitney said this animal, Lystrosaurus, was about the size of a
medium dog with a beak like a turtle and two small tusks, and it was one
of the species to make it through the mass extinction.

``It's an odd animal,'' she said. ``It's kind of a sausage shape. And it
had no teeth except for the two tusks that came out from the face.''

\includegraphics{https://static01.graylady3jvrrxbe.onion/images/2020/08/27/science/27TB-HIBERNATION2/27TB-HIBERNATION2-articleLarge.jpg?quality=75\&auto=webp\&disable=upscale}

Despite its dinosaur-sounding name --- it means ``shovel lizard'' in
Greek --- this creature was more closely related to mammals.

The tusks --- just a few inches long, probably used to dig up roots and
tubers to eat --- provided the telltale signs that the metabolism of
Lystrosaurus periodically slowed down.

As with modern-day elephants, the Lystrosaurus tusks grew continuously.
Thus, cutting a thin cross-section of a tusk provided a record of the
animal's life, much like tree rings, with alternating dark and light
circles. Dr. Whitney and Dr. Sidor compared the patterns in the tusks of
six Lystrosaurus that lived in Antarctica with four from South Africa.

The Antarctic tusks included closely spaced, thick rings --- likely
periods where growth of the tusks slowed, maybe stopped, because of
stress --- while the South African ones did not.

Although all of Earth's land at the time was combined into the
supercontinent Pangea, the part that is now Antarctica was still near
the South Pole and the part that is now South Africa was still hundreds
of miles to the north.

Temperatures were warmer then, so Antarctica was not draped with ice
sheets. But
\href{https://www.nytimes3xbfgragh.onion/2017/12/20/science/winter-solstice-december-21.html}{Earth
was tilted about the same} as it is now, which would have led to short
days during winter. The dark days would have slowed the growth of
plants, leaving little in the way of food for herbivores such as
Lystrosaurus to eat.

Image

A cross-section of the fossilized tusk from Lystrosaurus showing layers
of dentine deposited in rings of growth, the oldest at the edge and the
youngest at the center. At top right, a close-up view of the layers,
with the white bar indicating a hibernation-like state.~Credit...Megan
Whitney/Christian Sidor

Thus the researchers interpreted the thick, dark rings as a result of
hibernation-like metabolism. The patterns are similar to what is seen in
the teeth of modern-day mammals that hibernate in winter.

The findings also suggest that Lystrosaurus was warm-blooded. While the
metabolism of cold-blooded reptiles can often shut down entirely,
hibernating mammals periodically rouse themselves.

The findings were published on Thursday in
\href{https://www.nature.com/articles/s42003-020-01207-6}{the journal
Communications Biology.}

``The idea that they're saying, OK, there's actually some interesting
variation in the size of these features that tells us about the life
history of the animals,'' said Kenneth Angielczyk, curator of
paleomammalogy at the Field Museum of Natural History in Chicago.
``That's something kind of new and quite interesting.''

Dr. Angielczyk was not involved with the Lystrosaurus research, although
he is collaborating on other projects with Dr. Sidor and Dr. Whitney.

Whether Lystrosaurus actually hibernated or otherwise slowed its
metabolism --- biologists refer to the strategies as torpor --- may
never be known.

``This is a first study of its kind,'' Dr. Whitney said, ``so it's going
to be preliminary.''

Advertisement

\protect\hyperlink{after-bottom}{Continue reading the main story}

\hypertarget{site-index}{%
\subsection{Site Index}\label{site-index}}

\hypertarget{site-information-navigation}{%
\subsection{Site Information
Navigation}\label{site-information-navigation}}

\begin{itemize}
\tightlist
\item
  \href{https://help.nytimes3xbfgragh.onion/hc/en-us/articles/115014792127-Copyright-notice}{©~2020~The
  New York Times Company}
\end{itemize}

\begin{itemize}
\tightlist
\item
  \href{https://www.nytco.com/}{NYTCo}
\item
  \href{https://help.nytimes3xbfgragh.onion/hc/en-us/articles/115015385887-Contact-Us}{Contact
  Us}
\item
  \href{https://www.nytco.com/careers/}{Work with us}
\item
  \href{https://nytmediakit.com/}{Advertise}
\item
  \href{http://www.tbrandstudio.com/}{T Brand Studio}
\item
  \href{https://www.nytimes3xbfgragh.onion/privacy/cookie-policy\#how-do-i-manage-trackers}{Your
  Ad Choices}
\item
  \href{https://www.nytimes3xbfgragh.onion/privacy}{Privacy}
\item
  \href{https://help.nytimes3xbfgragh.onion/hc/en-us/articles/115014893428-Terms-of-service}{Terms
  of Service}
\item
  \href{https://help.nytimes3xbfgragh.onion/hc/en-us/articles/115014893968-Terms-of-sale}{Terms
  of Sale}
\item
  \href{https://spiderbites.nytimes3xbfgragh.onion}{Site Map}
\item
  \href{https://help.nytimes3xbfgragh.onion/hc/en-us}{Help}
\item
  \href{https://www.nytimes3xbfgragh.onion/subscription?campaignId=37WXW}{Subscriptions}
\end{itemize}
