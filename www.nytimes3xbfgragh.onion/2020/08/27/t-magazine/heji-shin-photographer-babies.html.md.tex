Sections

SEARCH

\protect\hyperlink{site-content}{Skip to
content}\protect\hyperlink{site-index}{Skip to site index}

\href{https://myaccount.nytimes3xbfgragh.onion/auth/login?response_type=cookie\&client_id=vi}{}

\href{https://www.nytimes3xbfgragh.onion/section/todayspaper}{Today's
Paper}

The Photographer Capturing Unvarnished Truths

\url{https://nyti.ms/32zekpq}

\begin{itemize}
\item
\item
\item
\item
\item
\end{itemize}

Advertisement

\protect\hyperlink{after-top}{Continue reading the main story}

Supported by

\protect\hyperlink{after-sponsor}{Continue reading the main story}

\hypertarget{the-photographer-capturing-unvarnished-truths}{%
\section{The Photographer Capturing Unvarnished
Truths}\label{the-photographer-capturing-unvarnished-truths}}

Heji Shin's striking, discomfiting work poses an important question for
the contemporary age: What do we expect art to do, and does the artist
have a responsibility to do it?

\includegraphics{https://static01.graylady3jvrrxbe.onion/images/2020/08/30/t-magazine/30tmag-heji-shin-slide-TYK4/30tmag-heji-shin-slide-TYK4-articleLarge.jpg?quality=75\&auto=webp\&disable=upscale}

By \href{https://www.nytimes3xbfgragh.onion/by/megan-o-grady}{Megan
O'Grady}

\begin{itemize}
\item
  Published Aug. 27, 2020Updated Sept. 3, 2020
\item
  \begin{itemize}
  \item
  \item
  \item
  \item
  \item
  \end{itemize}
\end{itemize}

``WE PUT A kind of call out for cocks. And they brought them. They're
beautiful. I mean, they're extremely, extremely beautiful,'' says the
photographer Heji Shin in the midst of her latest shoot, her first since
the Covid-19 lockdown: large-format studio portraits --- ``very sharp,
very shiny'' --- of bellicose roosters. Two prints from the series,
titled
``\href{http://www.artnet.com/artists/heji-shin/big-cock-2-a-1yMY-WIAcK_wvpQ6kVrIHQ2}{Big
Cocks},'' recently appeared in a group show at Galerie Buchholz in
Berlin. Roosters have long been associated with masculinity (though, in
fact, they don't have penises, Shin informs me); in an era in which
violence tends to be systemic or ``tactical,'' she writes to me later,
``the short-lived outbursts of angry cock energy look Hellenistic and
virile.''

Shin is accustomed to unwieldy subjects, having shot everything from
models having sex (for a discreetly pixelated spring 2017 campaign for
the fashion label Eckhaus Latta) to newborns emerging bloodily into the
world (``Baby,'' 2016-17). Her trickiest subject to date was probably
\href{https://www.galeriebuchholz.de/exhibitions/heji-shin-berlin-2019/}{Kanye
West}, whom Shin captured on a trip with his family and entourage to
Uganda and --- briefly --- in a Los Angeles studio (``Kanye,'' 2018).
Uncomfortable taking direction from her, he posed looking directly into
the camera. Portrait photography always involves things outside the
artist's control --- ruffled feathers, awkward angles, megalomaniacal
personalities --- but the South Korea-born German photographer is
unflustered by difficulty. In fact, ideas that seem improbable or even
unachievable excite her, and this is surely what makes her work stand
out at a time in which images have never felt more fleetingly disposable
and yet freighted with meaning. As eye-catching as they are unsettling,
Shin's photos offer a dose of style, levity and provocation at a
particularly earnest moment in the arts, in which conversations about
the ethics of representation are dominant and the social utility of art
is emphasized.

Ever since the fall of 2016, there's been an expectation, unspoken and
not, that art comments on the political moment --- an expectation that
applies particularly to photography. Cameras, usually camera phones,
have become a citizen's best weapon, a powerful and essential form of
defense; they've become necessary in the fight for equality, a way of
documenting everything from murderous police to bigoted dog walkers ---
a way of holding individuals, and a racist society, accountable. But as
Shin sees it, art-making has always been a kind of power grab. ``You are
making a claim that something is art, and this differentiation is, of
course, like creating a sort of hierarchy. You cannot circumvent that.
It's not a democratic process,'' she says. Case in point: the
pearl-clutching with which her ``Baby'' series was initially greeted.
``Are you kidding?'' one (female) art adviser asked Shin when she saw
them. Was she seriously putting this out into the world as art?
Interestingly, Shin says that men tended to respond more positively to
the images, seeing in them a kind of analogy for the creative process.

It's impossible to look at the series without wondering how she managed
to pull it off. Shin worked with a midwife to secure permission from the
mothers, who received a set of more conventional photos documenting the
birth in exchange for allowing Shin in the room. The focus is on the
babies' aggrieved and rumpled faces; the idea, Shin tells me, came from
the German painter \href{https://www.frieze.com/article/still-lives}{Ull
Hohn}'s unsettling, blurred paintings of creature-like infant bodies
from the early 1990s. At first, Shin wasn't sure she could show them.
``I looked at them and I was like, `This is literally ``The
Exorcist.''''' Her retoucher started to cry. (Newborns, Shin notes, look
completely different only seconds later, after they've drawn a few
breaths.) She seems relieved when I tell her that, as someone with a
little bit of experience --- having given birth (traumatically, after
being overdosed with Pitocin) --- I find the pictures extraordinary, and
not a little refreshing. Outside of medical contexts, babies tend to be
photographed sentimentally, but Shin spares us none of the violent
intensity of bringing about life and of being alive. To have a child is
a risk, to make art is a risk --- and Shin's images simultaneously
showcase the fragility of the human animal and the ferocity of its will.
``I really do believe that we all have a cherry-picking view on what
nature is, like it's a beautiful landscape that we experience on a road
trip,'' she says. Her work is a repudiation of that romance, a reminder
that nature can be brutal, even monstrous, and that we are ultimately at
its mercy, much as we tend to believe the reverse to be true.

WE'RE SPEAKING --- UNNATURALLY --- via Skype, my hopes of meeting her in
a more organic context dashed by quarantine; she's in her apartment in
New York's Chinatown while I'm 700 miles away, in Chicago. It is May.
But any framing device has a way of making certain details pre-eminent:
Shin is wearing a zebra-patterned intarsia sweater that I immediately
covet; the tail of a vintage Felix the Cat clock behind her swings in
and out of the frame metronomically; her sleek hair has a
cheekbone-grazing layer, which is striking at a moment in which most of
us are still at home in athleisure wear with straggles and visible
roots. Unable to travel, she's spent a great deal of her time in
lockdown perusing the news, as we all have, although Shin reads not only
things like The Intercept and The Guardian but also Breitbart and RT,
the Russian state-controlled network, as well as various Hong Kong news
outlets. ``Art seems a little bit impotent now,'' she concedes.

I get the impression that Shin is motivated to read these ideologically
divergent publications not only for a fuller picture of global politics
but out of a very specific kind of curiosity, one felt by a very
specific kind of artist. The assumptions we bring to seeing the world
--- our bubblelike frames of reference --- interest Shin, the way we
seem to form our spheres of belief to protect ourselves from things we
might not want to see or know. Hence the surprise, of some, at the
results of the 2016 election. Hence the astonishment, of others, at the
chillingly offhand extinguishing of
\href{https://www.nytimes3xbfgragh.onion/2020/05/31/us/george-floyd-investigation.html}{George
Floyd}'s life under the knee of a Minneapolis policeman, captured on a
cellphone by a teenage girl. Our understanding of how power works is
shaped by a checklist of demographic factors, from where and when we
were born to what skin we were born into.

\includegraphics{https://static01.graylady3jvrrxbe.onion/images/2020/08/30/t-magazine/30tmag-heji-shin-slide-BWSR/30tmag-heji-shin-slide-BWSR-articleLarge.jpg?quality=75\&auto=webp\&disable=upscale}

Similarly, we each bring a set of assumptions, a personal frame of
reference, to how we perceive art. At one point, Shin turns the question
on her interlocutor: ``Do you think artists should be aware of the
political circumstances they make art in and have to actually make it
part of their art practice?'' she asks, an onion of a question that
points to what Shin sees as a kind of dogmatic turn in art, an
expectation to convey appropriate liberal politics. Sensitivities are
running high in the art world: Think of the uproar that met Jordan
Wolfson's ultraviolent virtual-reality film from 2017,
``\href{https://www.nytimes3xbfgragh.onion/2017/04/28/arts/design/art-goes-political-but-will-that-fly-on-the-london-market.html}{Real
Violence},'' in which the artist, with the help of an animatronic dummy
and C.G.I., appears to beat a passive victim to death with a baseball
bat --- all too nauseatingly on the nose for some, given the excess of
white-male rage in Trump's America --- or the widespread controversy
surrounding Dana Schutz's 2016 painting of Emmett Till,
``\href{https://www.nytimes3xbfgragh.onion/2017/03/21/arts/design/painting-of-emmett-till-at-whitney-biennial-draws-protests.html}{Open
Casket}.'' The ambiguity of perspective and lack of context were, in
both cases, seen to be problematic by many critics. (Schutz's painting
in particular has been seen as an unthought-through act of appropriation
by a white artist of a painful, historic image in a time of ongoing
brutality against Black bodies.) I think that these discussions are not
only important but essential to have, and Shin does not disagree. But
she also sees the way in which the fear of causing offense could lead to
work that is politically and aesthetically overdetermined. This isn't a
matter of outright censorship but rather of public opinion, she says,
something that's potentially internalized.

In talking to her, I keep thinking of something Janet Malcolm wrote in
her 1994 book on Sylvia Plath,
``\href{https://www.nytimes3xbfgragh.onion/1994/03/27/books/the-importance-of-being-biased.html}{The
Silent Woman}'': ``Art is theft, art is armed robbery, art is not
pleasing your mother.'' Art, in other words, isn't about being nice.
Shin isn't willing to cater to public notions of acceptability or spare
us our own discomfort. The reactions to her work tend to be very strong
and not always straightforward. And so it makes a certain sense that her
``Kanye'' portraits are perhaps her most controversial works to date;
those bright, oversize images, evocative of Warhol silk screens, have a
sly grandiosity. The series of 11 photos includes a single candid of him
on a safari with his young daughter North on his shoulders: West the
vacationing dad versus West the cultural projection. When Shin submitted
the work to the
\href{https://www.nytimes3xbfgragh.onion/2019/05/16/arts/design/whitney-biennial-review.html}{2019
Whitney Biennial}, the curators were displeased. At the time, West was
an outspoken Donald Trump supporter; he had made an incendiary statement
about slavery being a choice for African-Americans, for which he later
apologized. ``They said he would be too dominant,'' Shin explains, as a
big American entertainment-industry figure. ``I mean, it was like, yeah,
but an exhibition is not democracy. One even told me that it would
exploit his mental illness. There were these absurd discussions, and it
went on for weeks.'' After some back and forth, they made a compromise
--- they would show two of the ``Kanye'' portraits if Shin would also
include five images from the ``Baby'' series. (A Whitney spokesperson
said that the curators had invited Shin to participate based on the
``Baby'' series and had selected those images from the beginning.) But
in an 11th-hour twist, the curators displayed the ``Kanye'' prints in
the basement, between the restrooms and the coat check. (The babies were
displayed prominently on the fifth floor.) Shin was unperturbed, even
delighted, by the Kanye portraits' unusual placement. ``I thought it was
funny because everybody would see it and also wonder why, and then it's
like the basement is the area for repressed content.''

Art has always served as a playground for the cultural subconscious;
it's what keeps us coming back to look again and again. Just wait, I
think, till Shin gets ahold of
\href{https://www.nytimes3xbfgragh.onion/2020/08/20/nyregion/steve-bannon-arrested-indicted.html}{Steve
Bannon}, our American grotesque, who is one of the figures she names
when I ask her whom she'd like to photograph next. Jane Goodall, so
often pictured with a maternal, almost Madonna-like incandescence,
holding a chimp, is another. (Shin has a thing for animals as amusing
objects of human psychological projection; in 2016, she made
``\href{https://www.galeriebernhard.com/exhibitions/heji-shin-lonelygirl}{\#LonelyGirl,}''
a series of portraits using a monkey to mock the selfie vernacular; one
of them, in which the monkey is nibbling on a dildo, landed on the cover
of Artforum in May 2016.) ``To create is always a step into the
unknown,'' she says. ``Everything has to submit to art. Everything. When
you're an artist, you have to submit politics to your art. Not the other
way around. You have to submit your own emotions, your anxieties, your
ideologies. That's why art is this really kind of sacred thing. And to
do it the other way around always compromises certain things, right?''

At the end of the day, I'm skeptical that politics can ever entirely
submit to art, and I'm not convinced that it ever did; perhaps we're
just more conscious of differentials in power than we once were. We can
all point to moments in the history of photography in which
transgression became exploitation: Think of Robert Mapplethorpe's
'80s-era nudes of Black men, who were often depersonalized or posed to
embody (and perhaps mock) racial stereotypes. Shin mentions Irina
Ionesco, the French photographer who built her career in the 1970s by
taking erotic portraits of her prepubescent daughter, Eva; in 2015, Eva
successfully sued her mother for damages and prevented her from further
disseminating the images without Eva's consent. ``Doing something simply
out of a desire to be transgressive is very stupid,'' says Shin, ``you
know, just to break certain taboos. It has to have more substance than
that, I think.''

If Shin is here to remind us of the sense of adventure to be found in
art, she also reminds us of the risk --- a moral risk, but most of all,
the risk of artistic failure. ``People react very intuitively to what
risk is, and then you get excited when they react either in a negative
way or in a positive way, it doesn't matter, with their amygdala,'' she
says. ``You know, you don't react with your rational side. That's where
it should go.'' That Shin's work is drawing attention in a larger moment
of outrage against a host of social ills doesn't feel coincidental;
looking at it tends to provoke reconsideration of the relevancy of art
in this charged moment. Where is the risk in politically didactic art
when it is posted on Instagram like a kind of virtuous trading card? How
high are the stakes, really, when everyone who follows you agrees? Who
and what, exactly, constitutes ``public opinion,'' and at what point
does sensitivity tip over into pandering?

IN CONTEMPORARY PORTRAIT photography, it has often fallen to women to
show us the potential of the form to subvert conventions and redirect
the gaze, from Cindy Sherman's stereotype-mocking self-portraits to
\href{https://www.nytimes3xbfgragh.onion/2020/07/21/t-magazine/carrie-mae-weems-moma-garden.html}{Carrie
Mae Weems} staring down the camera in her ``Kitchen Table Series,''
filling decades of absence in representation with a demand to be seen.
It's difficult now to imagine the world would see AIDS in quite the same
way without Nan Goldin, or the queer community without Catherine Opie.
Such photography implicitly critiques absences in art history while
imbuing the camera's subject with dignity, as in Deana Lawson's
goddess-like nudes in their living rooms, or (arguably) Diane Arbus's
sideshow performers and sex workers. (Arbus, unsurprisingly, is a
touchstone for Shin.) Women have dominated contemporary portraiture for
good reason: There's tremendous power to be found in moving the margins
to the center. ``Portraiture is always very psychological, and it has a
huge palette of references,'' says Shin, who began her career as a
photographer shooting portraits for a German economics magazine in the
early aughts. ``And it always works. I'm always surprised that it works,
but it does. There's an impact we inherently recognize and react to.''

Shin came to photography late, she says, after receiving a camera for
her 20th birthday. She attended art school in Hamburg before dropping
out and moving to Berlin. It was the late 1990s, and the art scene
hadn't really arrived in Germany's newly reunified capital just yet ---
the city was, at that point, a symbol as much as it was a geographic
place --- but there was a robust music scene, largely techno, and there
were a lot of clubs and house parties. ``Back then, Berlin was so
cool,'' she recalls. ``But people weren't even aware that it was cool at
that time. We lived in very chic apartments that were huge and cost
nothing. They all still had a coal oven.'' She aspired to have an
editorial career in fashion, but her art career took off first (she has
also published work in Pop Magazine, CR Fashion Book and Purple, with
shoots that blur the line between the artistic and commercial), and for
a few years, she was back and forth between Berlin and New York with her
husband, the Canadian artist Mathieu Malouf, whom she met in Berlin in
2008, before settling more permanently in New York.

Shin can be seen as an heir of the German photographers who also made
their name shooting editorial work for magazines, including Juergen
Teller, Wolfgang Tillmans and Helmut Newton (she especially admires
Newton); she seems unconcerned that anyone might find this blurring of
the line between commerce and art glib. I reminisce with Shin about
being a teenager in the late 1980s and early 1990s --- we're both
Generation X'ers in our 40s --- reading magazines like The Face, i-D and
Visionaire, and it occurs to me that she's never lost the barbed
irreverence of that time. It was then that a sense of culture
alternative to the popular and mainstream took hold, and Shin brings
something of that era's ironic sensibility and critique of the status
quo to this one. The early 1990s --- 1993, to be precise --- also
witnessed one of the most controversial
\href{https://www.nytimes3xbfgragh.onion/interactive/2019/07/05/arts/design/whitney-biennial-maps.html}{Whitney
Biennials}, when queer, feminist and nonwhite artists who had been shut
out from the institution finally began making their mark in the
mainstream with work that was provocative and discomfiting. ``I Can't
Imagine Ever Wanting to Be White,'' read one of the admission buttons
designed by Daniel Joseph Martinez. George Holliday's video of the
Rodney King beating was screened on the museum walls. Janine Antoni
nibbled on 600-pound cubes of chocolate and lard. The purpose of art and
who gets to make it was, quite explicitly, a theme. Surely one reason
that art subsequently became so much about self-assertion and identity
was in response to what had for so long been a white monopoly on the
culture.

Image

An image from Shin's series ``Men Photographing Men'' (2018), which was
a staged cop-themed gay porn shoot set in an art gallery.Credit...Heji
Shin

We're nostalgic for the early 1990s, Shin says, for good reason. ``It
was the last time we really demonstrated an identity of what {[}the{]}
times were,'' she says. It was before fashion became a riff or remix of
everything that had come before. Since the internet age, with its new
digital platforms, visual culture has accelerated to a degree that makes
it difficult to pin down; the curation of taste once found in print has
gone online, becoming more individualized; anyone can cultivate a
following on TikTok or Instagram. But Shin believes there is something
more regressive afoot and wants to ``dig deeper why things cannot
culturally be just expressed. Like why this freedom has disappeared. For
example, why millennials are a certain way, or they don't like certain
kinds of liberties anymore.''

Shin's teen years may have been a heyday for fashion photography, but
this period was also notorious for its embarrassing cultural
appropriations --- think of any number of fashion layouts from the 1980s
and early 1990s starring a lanky white, blond model in Asia or Africa,
dressed in designer fashion featuring motifs swiped from traditional
clothing. To contemporary eyes, this sort of thing provokes a tremendous
cringe, but it wasn't long ago that this was a fashion magazine
standard. These days, crimes of appropriation tend to be called out, but
they're not always as straightforwardly offensive or obvious as a
Victoria's Secret model in a feather headdress, and they extend well
beyond the realms of fashion and photography to white chefs writing
about healthy soul food, a novelist unconvincingly ventriloquizing a
Mexican migration experience or a company or arts institution with nary
a person of color in a position of power on staff celebrating Black
creativity.

One side effect of this long history of white plundering of other
cultures is a kind of defensive politics in which everyone is expected
to speak only from their personal ethnic experiences, as though cultural
identity is a form of intellectual property. While understandable, this
too can be problematic in the assumptions it makes: Our identities
aren't always seamlessly pure or reducible. Shin clearly wants to stand
apart from this kind of essentializing. A double immigrant, she has
entirely resisted making art about self-definition in a foreign land, or
that overtly addresses the condition of being an Asian woman first in
Germany and then in the United States. Given the explicit nature of much
of her work, it follows that making the pictures she \emph{does} make
surely requires a certain amount of trust from her subjects. And yet,
intimacy is not what her images evoke; in fact, the opposite is true.
She finds it liberating, she says, to be an outsider, to exist outside
any given cultural framework.

IN ONE WAY or another, Shin has always been something of an initiated
outsider, someone who knows all the rules but feels little need to heed
them. She was born in Seoul, to Korean parents, but moved at 4 to
Hamburg with her mother, a nurse who emigrated during a nursing shortage
in Germany (her parents are divorced, and her father remained in South
Korea; she had a German stepfather). It occurs to me that Shin has now
been witnessing a culture --- first in Germany, then in the United
States --- in the process of painfully contending with its own
historical conscience for much of her life. When I ask her about the
earliest visual memory she can recall, she tells me that she spent much
of her early years, until she was about 10, in a residence for immigrant
children run by the Catholic church. ``All of the imagery is so
fantastic,'' she says. ``I think children are always interested in the
same things --- Jesus's body, and of course Mother Maria, because she's
so beautiful. People don't understand anymore how influential and great
it is to be surrounded by that kind of imagery.''

All portraiture, of course, is a form of icon-making, whether it's the
Virgin Mary or a pink-haired Kate Moss: a public face of our culture at
a moment in time. In 2018, Shin made a witty series of X-rays of herself
while holding a pug or a Chihuahua --- one creepy skeleton holding
another, stripped of cuteness and intimacy: the ``look at me''
transparency of the selfie taken to its logical (and absurdist)
consequence, the artist exposing herself down to the bones while
revealing nothing. But I also can't help but think of the series as a
commentary on the expectation that female artists put themselves into
their work, baring their personal narratives, or even their own faces
and bodies, as indeed many photographers of her generation have done,
like Laurel Nakadate, who first became known for her eerie short films
in which she threw herself birthday parties or danced to Britney Spears
in the homes of strange men, or Elinor Carucci, who has documented
herself throughout different phases of her life, including pregnancy and
motherhood, her marital crisis, even her back pain. These skilled
practitioners of first-person photography are inviting us to look at
them, directing the gaze back onto themselves to elicit a certain potent
intimacy. Shin seems to want us not to look at her but at ourselves.

The question of exactly whose gaze is on display is at the center of her
2018 series ``Men Photographing Men,'' for which Shin staged a
cop-themed gay porn shoot in an art gallery and exhibited the resulting
images in the same space, making visitors feel as if they were wandering
onto a set. Immediately, we're implicated in the looking: The absurdly
good-looking male models in police uniforms seem to be monitoring the
images of men having sex, and here we are as well, looking at all of it.
The pictures themselves, all cheekbones and bare bottoms and holsters,
are amusing, and the models --- white, exuding vanity --- seem to be in
on the joke. At face value, it's transgressive in a playful way. Did she
do it just to prove that she, an Asian woman, another marginalized
class, could objectify white men as a man might, and poke fun at
masculine archetypes? Maybe. But what might be most radical about Shin's
work is the way she puts us front and center in her project, making us
aware of ourselves as uneasy spectators, uncertain of our point of view.
At the opening, she tells me, one attendee, a Black acquaintance who
didn't realize she had taken the pictures, told her that he thought the
show was awful. ``Why do you think that?'' she asked. ``Who do you think
took them?'' Some white dude, he replied. This doesn't feel all that
surprising to me, given the show's title, though I get her point. As we
question the flow of power (not to mention the symbolism of a police
uniform), we are all uncomfortably destabilized right now; those who
aren't probably should be.

It seems right, then, that the art that is meaningful now isn't the work
that makes us feel secure or elated in our righteousness but that makes
us question where we are in the grand scheme of things: art that dares
us to question our intentions with imagery that feels raw and vital and
sensibility-challenging. As Americans see their repressed content
emerging at street level, in the things we call out on Twitter and in
person en masse, across the country, the subtext of power has become
text at a remarkable speed. Everything is under scrutiny, including the
violence of silent acquiescence; no one can really be a bystander
anymore. What we call outsiderness, implying a dispassionate, unbiased
observer, was perhaps always more of a stance than an achievable
reality. But maybe, Shin would have us believe, there's a way to get
past the limitations of our perspectives, to subvert our own framing
devices via art's ability to estrange and transubstantiate. What we see
unfolding on the street will translate into disruptions and revolutions
in galleries, too, in art that is skeptical of a heartwarming or cleanly
unifying story. As we turn our camera on ourselves and each other, the
visuals on the street have outpaced the ones on the walls, leaving us to
ask how far it all will go and where we stand in the fray.

Advertisement

\protect\hyperlink{after-bottom}{Continue reading the main story}

\hypertarget{site-index}{%
\subsection{Site Index}\label{site-index}}

\hypertarget{site-information-navigation}{%
\subsection{Site Information
Navigation}\label{site-information-navigation}}

\begin{itemize}
\tightlist
\item
  \href{https://help.nytimes3xbfgragh.onion/hc/en-us/articles/115014792127-Copyright-notice}{©~2020~The
  New York Times Company}
\end{itemize}

\begin{itemize}
\tightlist
\item
  \href{https://www.nytco.com/}{NYTCo}
\item
  \href{https://help.nytimes3xbfgragh.onion/hc/en-us/articles/115015385887-Contact-Us}{Contact
  Us}
\item
  \href{https://www.nytco.com/careers/}{Work with us}
\item
  \href{https://nytmediakit.com/}{Advertise}
\item
  \href{http://www.tbrandstudio.com/}{T Brand Studio}
\item
  \href{https://www.nytimes3xbfgragh.onion/privacy/cookie-policy\#how-do-i-manage-trackers}{Your
  Ad Choices}
\item
  \href{https://www.nytimes3xbfgragh.onion/privacy}{Privacy}
\item
  \href{https://help.nytimes3xbfgragh.onion/hc/en-us/articles/115014893428-Terms-of-service}{Terms
  of Service}
\item
  \href{https://help.nytimes3xbfgragh.onion/hc/en-us/articles/115014893968-Terms-of-sale}{Terms
  of Sale}
\item
  \href{https://spiderbites.nytimes3xbfgragh.onion}{Site Map}
\item
  \href{https://help.nytimes3xbfgragh.onion/hc/en-us}{Help}
\item
  \href{https://www.nytimes3xbfgragh.onion/subscription?campaignId=37WXW}{Subscriptions}
\end{itemize}
