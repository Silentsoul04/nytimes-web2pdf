Sections

SEARCH

\protect\hyperlink{site-content}{Skip to
content}\protect\hyperlink{site-index}{Skip to site index}

\href{https://www.nytimes3xbfgragh.onion/section/travel}{Travel}

\href{https://myaccount.nytimes3xbfgragh.onion/auth/login?response_type=cookie\&client_id=vi}{}

\href{https://www.nytimes3xbfgragh.onion/section/todayspaper}{Today's
Paper}

\href{/section/travel}{Travel}\textbar{}Move Over, Sustainable Travel.
Regenerative Travel Has Arrived.

\url{https://nyti.ms/2D2nuSG}

\begin{itemize}
\item
\item
\item
\item
\item
\item
\end{itemize}

\hypertarget{the-coronavirus-outbreak}{%
\subsubsection{\texorpdfstring{\href{https://www.nytimes3xbfgragh.onion/news-event/coronavirus?name=styln-coronavirus-national\&region=TOP_BANNER\&block=storyline_menu_recirc\&action=click\&pgtype=Article\&impression_id=4ffd9730-f1b7-11ea-920e-9d20be26e6a7\&variant=undefined}{The
Coronavirus
Outbreak}}{The Coronavirus Outbreak}}\label{the-coronavirus-outbreak}}

\begin{itemize}
\tightlist
\item
  live\href{https://www.nytimes3xbfgragh.onion/2020/09/08/world/covid-19-coronavirus.html?name=styln-coronavirus-national\&region=TOP_BANNER\&block=storyline_menu_recirc\&action=click\&pgtype=Article\&impression_id=4ffd9731-f1b7-11ea-920e-9d20be26e6a7\&variant=undefined}{Latest
  Updates}
\item
  \href{https://www.nytimes3xbfgragh.onion/interactive/2020/us/coronavirus-us-cases.html?name=styln-coronavirus-national\&region=TOP_BANNER\&block=storyline_menu_recirc\&action=click\&pgtype=Article\&impression_id=4ffd9732-f1b7-11ea-920e-9d20be26e6a7\&variant=undefined}{Maps
  and Cases}
\item
  \href{https://www.nytimes3xbfgragh.onion/interactive/2020/science/coronavirus-vaccine-tracker.html?name=styln-coronavirus-national\&region=TOP_BANNER\&block=storyline_menu_recirc\&action=click\&pgtype=Article\&impression_id=4ffd9733-f1b7-11ea-920e-9d20be26e6a7\&variant=undefined}{Vaccine
  Tracker}
\item
  \href{https://www.nytimes3xbfgragh.onion/2020/09/02/your-money/eviction-moratorium-covid.html?name=styln-coronavirus-national\&region=TOP_BANNER\&block=storyline_menu_recirc\&action=click\&pgtype=Article\&impression_id=4ffd9734-f1b7-11ea-920e-9d20be26e6a7\&variant=undefined}{Eviction
  Moratorium}
\item
  \href{https://www.nytimes3xbfgragh.onion/interactive/2020/09/02/magazine/food-insecurity-hunger-us.html?name=styln-coronavirus-national\&region=TOP_BANNER\&block=storyline_menu_recirc\&action=click\&pgtype=Article\&impression_id=4ffdbe40-f1b7-11ea-920e-9d20be26e6a7\&variant=undefined}{American
  Hunger}
\end{itemize}

Advertisement

\protect\hyperlink{after-top}{Continue reading the main story}

Supported by

\protect\hyperlink{after-sponsor}{Continue reading the main story}

\hypertarget{move-over-sustainable-travel-regenerative-travel-has-arrived}{%
\section{Move Over, Sustainable Travel. Regenerative Travel Has
Arrived.}\label{move-over-sustainable-travel-regenerative-travel-has-arrived}}

Can a post-vaccine return to travel be smarter and greener than it was
before March 2020? Some in the tourism industry are betting on it.

\includegraphics{https://static01.graylady3jvrrxbe.onion/images/2020/08/28/travel/28RegenerativeTravel2/merlin_172447659_8bf4071f-e0af-4399-bde4-1f973331804f-articleLarge.jpg?quality=75\&auto=webp\&disable=upscale}

By Elaine Glusac

\begin{itemize}
\item
  Aug. 27, 2020
\item
  \begin{itemize}
  \item
  \item
  \item
  \item
  \item
  \item
  \end{itemize}
\end{itemize}

Tourism, which \href{https://wttc.org/Research/Economic-Impact}{grew
faster} than the global gross domestic product for the past nine years,
has been decimated by the pandemic. Once accounting for 10 percent of
employment worldwide, the sector is poised to shed 121 million jobs,
with losses projected at a minimum of \$3.4 trillion, according to the
World Travel \& Tourism Council.

But in the lull, some in the tourism industry are planning for a
post-vaccine return to travel that's better than it was before March
2020 --- greener, smarter and less crowded. If sustainable tourism,
which aims to counterbalance the social and environmental impacts
associated with travel, was the aspirational outer limit of ecotourism
before the pandemic, the new frontier is ``regenerative travel,'' or
leaving a place better than you found it.

``Sustainable tourism is sort of a low bar. At the end of the day, it's
just not making a mess of the place,'' said Jonathon Day, an associate
professor focused on sustainable tourism at Purdue University.
``Regenerative tourism says, let's make it better for future
generations.''

\hypertarget{defining-regeneration}{%
\subsection{Defining regeneration}\label{defining-regeneration}}

Regenerative travel has its roots in regenerative development and
design, which includes buildings that meet the U.S. Green Building
Council's Leadership in Energy and Environmental Design or
\href{https://www.usgbc.org/help/what-leed}{LEED} standards. The concept
has applications across many fields, including regenerative agriculture,
which aims to restore soils and sequester carbon.

``Generally, sustainability, as practiced today, is about slowing down
the degradation,'' said Bill Reed, an architect and principal of
Regenesis Group, a design firm based in Massachusetts and New Mexico
that has been practicing regenerative design, including tourism
projects, since 1995. He described efforts like fuel efficiency and
reduced energy use as ``a slower way to die.''

``Regeneration is about restoring and then regenerating the capability
to live in a new relationship in an ongoing way,'' he added.

With most travel suspended during the pandemic, regenerative travel
remains at the starting gate. But in the lull, it's the new buzz. Six
nonprofit organizations, including the
\href{https://www.responsibletravel.org/}{Center for Responsible Travel}
and \href{https://sustainabletravel.org/}{Sustainable Travel
International}, have joined together as the
\href{https://www.futureoftourism.org/}{Future of Tourism} coalition,
which aims to ``build a better tomorrow.''

Twenty-two travel groups, including tour operators like
\href{https://www.gadventures.com/}{G Adventures}, destination marketers
such as the Slovenian Tourist Board, and organizations like the
\href{https://www.adventuretravel.biz/}{Adventure Travel Trade
Association}, have signed on to the coalition's 13 guiding principles,
including ``demand fair income distribution'' and ``choose quality over
quantity.''

\href{https://www.tourismnewzealand.com/}{Tourism New Zealand,} the
country's tourism organization, is talking about measuring its success
not solely in economic terms, but against the well-being of the country,
considering nature, human health and community identities. And travel
leaders in Hawaii are discussing repositioning the state as a cultural
destination in hopes of re-engaging islanders, many of whom are fed up
with overtourism, in the vitality of tourism.

To flesh out these broad strokes, Mr. Day, the associate professor,
points to the concept of a circular economy, which aims to design waste
out of the system, keep materials in use through reuse, repair and
upcycling, and regenerate natural systems.

``Tourism is just at the beginning of this process of how we can apply
circular economy ideas to the system,'' he said.

\includegraphics{https://static01.graylady3jvrrxbe.onion/images/2020/08/26/travel/27regenerative-travel/oakImage-1598454446779-articleLarge.jpg?quality=75\&auto=webp\&disable=upscale}

\hypertarget{regeneration-in-action}{%
\subsection{Regeneration in action}\label{regeneration-in-action}}

Having a truly regenerative travel experience may be a unicorn, but a
few operators are pointing the way.

\hypertarget{latest-updates-the-coronavirus-outbreak}{%
\section{\texorpdfstring{\href{https://www.nytimes3xbfgragh.onion/2020/09/08/world/covid-19-coronavirus.html?action=click\&pgtype=Article\&state=default\&region=MAIN_CONTENT_1\&context=storylines_live_updates}{Latest
Updates: The Coronavirus
Outbreak}}{Latest Updates: The Coronavirus Outbreak}}\label{latest-updates-the-coronavirus-outbreak}}

Updated 2020-09-08T09:32:17.217Z

\begin{itemize}
\tightlist
\item
  \href{https://www.nytimes3xbfgragh.onion/2020/09/08/world/covid-19-coronavirus.html?action=click\&pgtype=Article\&state=default\&region=MAIN_CONTENT_1\&context=storylines_live_updates\#link-4a77847f}{As
  senators return to Washington, an impasse over a virus relief package
  looms.}
\item
  \href{https://www.nytimes3xbfgragh.onion/2020/09/08/world/covid-19-coronavirus.html?action=click\&pgtype=Article\&state=default\&region=MAIN_CONTENT_1\&context=storylines_live_updates\#link-1c973131}{`The
  lockdown killed my father': Farmer suicides add to India's virus
  misery.}
\item
  \href{https://www.nytimes3xbfgragh.onion/2020/09/08/world/covid-19-coronavirus.html?action=click\&pgtype=Article\&state=default\&region=MAIN_CONTENT_1\&context=storylines_live_updates\#link-adc17f7}{China's
  leader declares success in suppressing the country's outbreak.}
\end{itemize}

\href{https://www.nytimes3xbfgragh.onion/2020/09/08/world/covid-19-coronavirus.html?action=click\&pgtype=Article\&state=default\&region=MAIN_CONTENT_1\&context=storylines_live_updates}{See
more updates}

More live coverage:

Regenesis worked on the development of
\href{https://www.playaviva.com/}{Playa Viva}, a small resort south of
Zihuatanejo, Mexico, on the Pacific Coast, which opened in 2009. The
firm's assessment of the more than 200-acre property took in the
beaches, the bird-filled estuary and ancient ruins as well as the
problems of turtle poaching and poor schools in the village. Ultimately,
the small town of Juluchuca became the gateway to the property; an
organic agricultural system benefited both the property and local
residents; and a 2 percent fee added to any stay funds a trust that
invests in community development.

``Rather than a resort helicoptering in and taking up land, they said,
`We are the village,''' Mr. Reed said. ``It's a paradigm shift.''

Playa Viva is one of 45 resorts belonging to
\href{https://www.regenerativetravel.com/about/}{Regenerative Travel}, a
booking agency that vets members based on metrics such as carbon usage,
employee well-being, immersive guest activities and sourcing local food.
To date, qualifications for membership have been handled internally, but
in September the company plans to launch a benchmarking system to
demonstrate their regenerative progress.

\href{https://www.oneseedexpeditions.com/}{OneSeed Expeditions}, an
adventure tour operator based in Denver, aims to couple travel with
economic development. It uses 10 percent of its proceeds to provide
zero-interest loans to local nongovernmental organizations where it
operates in places like Nepal and Peru. The local groups then issue
microloans to community entrepreneurs in businesses such as farming and
retail.

``The areas of greatest need are not necessarily in areas of the
greatest tourism attractions,'' said Chris Baker, the founder of OneSeed
Expeditions. ``We want to use tourism to be able to benefit people
outside of those areas.''

Regenerative tourism addresses impacts holistically, from destination
and community perspectives as well as environmental.
\href{https://www.intrepidtravel.com/us}{Intrepid Travel}, the
small-group tour company that, until the pandemic, ran more than 1,000
itineraries globally, has been carbon neutral since 2010. This year it
extended its pledge to cover 125 percent of its carbon emissions.

``There's this notion that business success means you have to do harm to
the world,'' said James Thornton, the chief executive of Intrepid
Travel, which became a B Corporation, an entity dedicated to benefiting
workers, customers, the community and environment, as well as
shareholders, in 2018. ``When the new normality returns, it shouldn't
come at the expense of sustainability.''

Image

Dubrovnik, Croatia, is one of many European cities that has struggled
with overtourism.Credit...Dmitry Kostyukov for The New York Times

\hypertarget{correcting-overtourism}{%
\subsection{Correcting overtourism}\label{correcting-overtourism}}

Implicit in many discussions about regenerative tourism is the threat of
returning to overtourism, which accounted for excessive numbers of
visitors in places like
\href{https://www.nytimes3xbfgragh.onion/2018/08/19/world/europe/dubrovnik-croatia-game-of-thrones.html}{Dubrovnik}
that ultimately had to cap the number of cruise ships allowed to dock
daily in high season.

``For so long, tourism success was defined by growing the numbers ---
numbers of visitors, numbers of cruise passengers,'' said Gregory
Miller, the executive director of the Center for Responsible Travel, a
nonprofit group that advocates for sustainable travel. ``Even before the
pandemic, there was a need for rebalancing.''

Image

In a 2018 survey by the Hawaiian Tourism Authority, two-thirds of
respondents agreed that ``this island is being run for tourists at the
expense of local people.'' Above, a socially distanced scene in
Honolulu.Credit...Michelle Mishina-Kunz for The New York Times

For example, the current recession may have bought Hawaii a few years
before its tourism figures return to what they were in 2019, when 10
million travelers visited the islands --- and that was up from 6.5
million a decade earlier --- resulting in painfully long queues to climb
Diamond Head at sunrise. In a 2018
\href{https://www.hawaiitourismauthority.org/media/2984/resident-sentiment-presentation-to-hta-board-01-31-2019.pdf}{survey}
by the Hawaiian Tourism Authority, two-thirds of respondents agreed that
``This island is being run for tourists at the expense of local
people.''

\href{https://www.nytimes3xbfgragh.onion/news-event/coronavirus?action=click\&pgtype=Article\&state=default\&region=MAIN_CONTENT_3\&context=storylines_faq}{}

\hypertarget{the-coronavirus-outbreak-}{%
\subsubsection{The Coronavirus Outbreak
›}\label{the-coronavirus-outbreak-}}

\hypertarget{frequently-asked-questions}{%
\paragraph{Frequently Asked
Questions}\label{frequently-asked-questions}}

Updated September 4, 2020

\begin{itemize}
\item ~
  \hypertarget{what-are-the-symptoms-of-coronavirus}{%
  \paragraph{What are the symptoms of
  coronavirus?}\label{what-are-the-symptoms-of-coronavirus}}

  \begin{itemize}
  \tightlist
  \item
    In the beginning, the coronavirus
    \href{https://www.nytimes3xbfgragh.onion/article/coronavirus-facts-history.html?action=click\&pgtype=Article\&state=default\&region=MAIN_CONTENT_3\&context=storylines_faq\#link-6817bab5}{seemed
    like it was primarily a respiratory illness}~--- many patients had
    fever and chills, were weak and tired, and coughed a lot, though
    some people don't show many symptoms at all. Those who seemed
    sickest had pneumonia or acute respiratory distress syndrome and
    received supplemental oxygen. By now, doctors have identified many
    more symptoms and syndromes. In April,
    \href{https://www.nytimes3xbfgragh.onion/2020/04/27/health/coronavirus-symptoms-cdc.html?action=click\&pgtype=Article\&state=default\&region=MAIN_CONTENT_3\&context=storylines_faq}{the
    C.D.C. added to the list of early signs}~sore throat, fever, chills
    and muscle aches. Gastrointestinal upset, such as diarrhea and
    nausea, has also been observed. Another telltale sign of infection
    may be a sudden, profound diminution of one's
    \href{https://www.nytimes3xbfgragh.onion/2020/03/22/health/coronavirus-symptoms-smell-taste.html?action=click\&pgtype=Article\&state=default\&region=MAIN_CONTENT_3\&context=storylines_faq}{sense
    of smell and taste.}~Teenagers and young adults in some cases have
    developed painful red and purple lesions on their fingers and toes
    --- nicknamed ``Covid toe'' --- but few other serious symptoms.
  \end{itemize}
\item ~
  \hypertarget{why-is-it-safer-to-spend-time-together-outside}{%
  \paragraph{Why is it safer to spend time together
  outside?}\label{why-is-it-safer-to-spend-time-together-outside}}

  \begin{itemize}
  \tightlist
  \item
    \href{https://www.nytimes3xbfgragh.onion/2020/05/15/us/coronavirus-what-to-do-outside.html?action=click\&pgtype=Article\&state=default\&region=MAIN_CONTENT_3\&context=storylines_faq}{Outdoor
    gatherings}~lower risk because wind disperses viral droplets, and
    sunlight can kill some of the virus. Open spaces prevent the virus
    from building up in concentrated amounts and being inhaled, which
    can happen when infected people exhale in a confined space for long
    stretches of time, said Dr. Julian W. Tang, a virologist at the
    University of Leicester.
  \end{itemize}
\item ~
  \hypertarget{why-does-standing-six-feet-away-from-others-help}{%
  \paragraph{Why does standing six feet away from others
  help?}\label{why-does-standing-six-feet-away-from-others-help}}

  \begin{itemize}
  \tightlist
  \item
    The coronavirus spreads primarily through droplets from your mouth
    and nose, especially when you cough or sneeze. The C.D.C., one of
    the organizations using that measure,
    \href{https://www.nytimes3xbfgragh.onion/2020/04/14/health/coronavirus-six-feet.html?action=click\&pgtype=Article\&state=default\&region=MAIN_CONTENT_3\&context=storylines_faq}{bases
    its recommendation of six feet}~on the idea that most large droplets
    that people expel when they cough or sneeze will fall to the ground
    within six feet. But six feet has never been a magic number that
    guarantees complete protection. Sneezes, for instance, can launch
    droplets a lot farther than six feet,
    \href{https://jamanetwork.com/journals/jama/fullarticle/2763852}{according
    to a recent study}. It's a rule of thumb: You should be safest
    standing six feet apart outside, especially when it's windy. But
    keep a mask on at all times, even when you think you're far enough
    apart.
  \end{itemize}
\item ~
  \hypertarget{i-have-antibodies-am-i-now-immune}{%
  \paragraph{I have antibodies. Am I now
  immune?}\label{i-have-antibodies-am-i-now-immune}}

  \begin{itemize}
  \tightlist
  \item
    As of right
    now,\href{https://www.nytimes3xbfgragh.onion/2020/07/22/health/covid-antibodies-herd-immunity.html?action=click\&pgtype=Article\&state=default\&region=MAIN_CONTENT_3\&context=storylines_faq}{~that
    seems likely, for at least several months.}~There have been
    frightening accounts of people suffering what seems to be a second
    bout of Covid-19. But experts say these patients may have a
    drawn-out course of infection, with the virus taking a slow toll
    weeks to months after initial exposure.~People infected with the
    coronavirus typically
    \href{https://www.nature.com/articles/s41586-020-2456-9}{produce}~immune
    molecules called antibodies, which are
    \href{https://www.nytimes3xbfgragh.onion/2020/05/07/health/coronavirus-antibody-prevalence.html?action=click\&pgtype=Article\&state=default\&region=MAIN_CONTENT_3\&context=storylines_faq}{protective
    proteins made in response to an
    infection}\href{https://www.nytimes3xbfgragh.onion/2020/05/07/health/coronavirus-antibody-prevalence.html?action=click\&pgtype=Article\&state=default\&region=MAIN_CONTENT_3\&context=storylines_faq}{.
    These antibodies may}~last in the body
    \href{https://www.nature.com/articles/s41591-020-0965-6}{only two to
    three months}, which may seem worrisome, but that's~perfectly normal
    after an acute infection subsides, said Dr. Michael Mina, an
    immunologist at Harvard University. It may be possible to get the
    coronavirus again, but it's highly unlikely that it would be
    possible in a short window of time from initial infection or make
    people sicker the second time.
  \end{itemize}
\item ~
  \hypertarget{what-are-my-rights-if-i-am-worried-about-going-back-to-work}{%
  \paragraph{What are my rights if I am worried about going back to
  work?}\label{what-are-my-rights-if-i-am-worried-about-going-back-to-work}}

  \begin{itemize}
  \tightlist
  \item
    Employers have to provide
    \href{https://www.osha.gov/SLTC/covid-19/standards.html}{a safe
    workplace}~with policies that protect everyone equally.
    \href{https://www.nytimes3xbfgragh.onion/article/coronavirus-money-unemployment.html?action=click\&pgtype=Article\&state=default\&region=MAIN_CONTENT_3\&context=storylines_faq}{And
    if one of your co-workers tests positive for the coronavirus, the
    C.D.C.}~has said that
    \href{https://www.cdc.gov/coronavirus/2019-ncov/community/guidance-business-response.html}{employers
    should tell their employees}~-\/- without giving you the sick
    employee's name -\/- that they may have been exposed to the virus.
  \end{itemize}
\end{itemize}

``We have the curse of a strong brand,'' said Frank Haas, a former vice
president with the Hawaiian Tourism Authority and an independent tourism
consultant. ``We're so well known as a sun destination that people
overlook the other aspects, the Hawaiian culture, the royal past, the
interesting geological and natural attractions.''

He thinks it will require more coordinated management --- currently, a
variety of federal, state and local authorities regulate parks and
facilities like airports --- as well as creative entrepreneurs to expand
cultural tourism by appealing to travelers interested in food, art,
history or music.

\hypertarget{who-defines-better-tourism}{%
\subsection{Who defines `better'
tourism?}\label{who-defines-better-tourism}}

Determining what makes a place better and who makes that decision
requires local involvement, according to regenerative tourism
proponents.

\href{https://www.visitflanders.com/en/?country=en_US}{VisitFlanders,}
the tourism organization representing the Northern Belgium region, used
local input to rethink its mission, repositioning its stance from
growing travel for the sake of the economy to creating an ``economy of
meaning,'' according to its master
\href{https://www.reizennaarmorgen.be/wp-content/uploads/2020/02/memorandum-eng.pdf}{plan}.
That includes, among other initiatives, linking visitors with locals who
share their passions for things like history or food and making
storytelling central to sites like its World War I battlefields.

``We've managed to shift the thinking from having their primary
objective be about growing the numbers, to creating flourishing
destinations, flourishing communities and having them say what kind of
tourism they want,'' said Anna Pollock, the founder of
\href{http://www.conscious.travel/}{Conscious Travel}, an education and
consulting enterprise devoted to positioning travel as a force for good,
who worked with VisitFlanders.

\hypertarget{a-travelers-role-in-regeneration}{%
\subsection{A traveler's role in
regeneration}\label{a-travelers-role-in-regeneration}}

Ms. Pollock believes regenerative travel is a supply-side concept that
asks operators to do more for the environment and community than they
take from them. But travelers play a key role in demand.

``Become mindful of the fact that your trip is going to have a set of
costs associated with it, which needs to be paid by somebody,'' she
said. ``In the same way you think, `Should I buy that cheap T-shirt from
the dime store down the road?,' knowing it's created by semi-slave
labor. Now you're thinking consciously about who do I buy it from and is
it quality.''

The experience of the pandemic --- when many are discovering the power
of their pocketbooks in supporting local businesses like bookstores and
restaurants --- is, perhaps, the most instructive in demonstrating
sustainability, even if the travel involved is within a few blocks of
home.

``Travel is an important vote of your principles,'' said Mr. Baker of
OneSeed. ``When you decide to put your time and resources into a trip,
you're affirming that's the type of business you want out there.''

Sustainable travel, let alone regenerative travel, will still have to
find solutions to the carbon emissions produced by air travel. Until the
economy recovers, there's likely to be less travel, more local travel,
or slower travel by car, train, bike or foot. This moment of reflection,
say proponents, is where regeneration begins.

``It's about how to regenerate our relationship with life,'' said Mr.
Reed, the architect. ``That's a continual process. Our children will
need that taught to them. Regeneration is a continual cycle of rebirth.
That's how we sustain the planet. You cannot have a sustainable planet
without regeneration.''

\begin{center}\rule{0.5\linewidth}{\linethickness}\end{center}

\emph{\textbf{Follow New York Times Travel}}
\emph{on}\href{https://www.instagram.com/nytimestravel/}{\emph{Instagram}}\emph{,}\href{https://twitter.com/nytimestravel}{\emph{Twitter}}
\emph{and}\href{https://www.facebookcorewwwi.onion/nytimestravel/}{\emph{Facebook}}\emph{.
And}\href{https://www.nytimes3xbfgragh.onion/newsletters/traveldispatch}{\emph{sign
up for our weekly Travel Dispatch newsletter}} \emph{to receive expert
tips on traveling smarter and inspiration for your next vacation.}

Advertisement

\protect\hyperlink{after-bottom}{Continue reading the main story}

\hypertarget{site-index}{%
\subsection{Site Index}\label{site-index}}

\hypertarget{site-information-navigation}{%
\subsection{Site Information
Navigation}\label{site-information-navigation}}

\begin{itemize}
\tightlist
\item
  \href{https://help.nytimes3xbfgragh.onion/hc/en-us/articles/115014792127-Copyright-notice}{©~2020~The
  New York Times Company}
\end{itemize}

\begin{itemize}
\tightlist
\item
  \href{https://www.nytco.com/}{NYTCo}
\item
  \href{https://help.nytimes3xbfgragh.onion/hc/en-us/articles/115015385887-Contact-Us}{Contact
  Us}
\item
  \href{https://www.nytco.com/careers/}{Work with us}
\item
  \href{https://nytmediakit.com/}{Advertise}
\item
  \href{http://www.tbrandstudio.com/}{T Brand Studio}
\item
  \href{https://www.nytimes3xbfgragh.onion/privacy/cookie-policy\#how-do-i-manage-trackers}{Your
  Ad Choices}
\item
  \href{https://www.nytimes3xbfgragh.onion/privacy}{Privacy}
\item
  \href{https://help.nytimes3xbfgragh.onion/hc/en-us/articles/115014893428-Terms-of-service}{Terms
  of Service}
\item
  \href{https://help.nytimes3xbfgragh.onion/hc/en-us/articles/115014893968-Terms-of-sale}{Terms
  of Sale}
\item
  \href{https://spiderbites.nytimes3xbfgragh.onion}{Site Map}
\item
  \href{https://help.nytimes3xbfgragh.onion/hc/en-us}{Help}
\item
  \href{https://www.nytimes3xbfgragh.onion/subscription?campaignId=37WXW}{Subscriptions}
\end{itemize}
