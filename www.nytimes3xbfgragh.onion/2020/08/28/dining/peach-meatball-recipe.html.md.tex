Sections

SEARCH

\protect\hyperlink{site-content}{Skip to
content}\protect\hyperlink{site-index}{Skip to site index}

\href{https://www.nytimes3xbfgragh.onion/section/food}{Food}

\href{https://myaccount.nytimes3xbfgragh.onion/auth/login?response_type=cookie\&client_id=vi}{}

\href{https://www.nytimes3xbfgragh.onion/section/todayspaper}{Today's
Paper}

\href{/section/food}{Food}\textbar{}A Lighter, Brighter Meatball

\url{https://nyti.ms/3gE9VXc}

\begin{itemize}
\item
\item
\item
\item
\item
\end{itemize}

\href{https://www.nytimes3xbfgragh.onion/spotlight/at-home?action=click\&pgtype=Article\&state=default\&region=TOP_BANNER\&context=at_home_menu}{At
Home}

\begin{itemize}
\tightlist
\item
  \href{https://www.nytimes3xbfgragh.onion/2020/09/07/travel/route-66.html?action=click\&pgtype=Article\&state=default\&region=TOP_BANNER\&context=at_home_menu}{Cruise
  Along: Route 66}
\item
  \href{https://www.nytimes3xbfgragh.onion/2020/09/04/dining/sheet-pan-chicken.html?action=click\&pgtype=Article\&state=default\&region=TOP_BANNER\&context=at_home_menu}{Roast:
  Chicken With Plums}
\item
  \href{https://www.nytimes3xbfgragh.onion/2020/09/04/arts/television/dark-shadows-stream.html?action=click\&pgtype=Article\&state=default\&region=TOP_BANNER\&context=at_home_menu}{Watch:
  Dark Shadows}
\item
  \href{https://www.nytimes3xbfgragh.onion/interactive/2020/at-home/even-more-reporters-editors-diaries-lists-recommendations.html?action=click\&pgtype=Article\&state=default\&region=TOP_BANNER\&context=at_home_menu}{Explore:
  Reporters' Google Docs}
\end{itemize}

Advertisement

\protect\hyperlink{after-top}{Continue reading the main story}

Supported by

\protect\hyperlink{after-sponsor}{Continue reading the main story}

A Good Appetite

\hypertarget{a-lighter-brighter-meatball}{%
\section{A Lighter, Brighter
Meatball}\label{a-lighter-brighter-meatball}}

Quickly made in a skillet with ripe peaches, basil and lime juice, these
meatballs are perfect for weeknights.

\includegraphics{https://static01.graylady3jvrrxbe.onion/images/2020/09/13/dining/28appetite-meatballs2/merlin_176001795_081fd9a5-e477-4647-8726-6b8d769da811-articleLarge.jpg?quality=75\&auto=webp\&disable=upscale}

\href{https://www.nytimes3xbfgragh.onion/by/melissa-clark}{\includegraphics{https://static01.graylady3jvrrxbe.onion/images/2018/06/21/multimedia/author-melissa-clark/author-melissa-clark-thumbLarge.png}}

By \href{https://www.nytimes3xbfgragh.onion/by/melissa-clark}{Melissa
Clark}

\begin{itemize}
\item
  Aug. 28, 2020
\item
  \begin{itemize}
  \item
  \item
  \item
  \item
  \item
  \end{itemize}
\end{itemize}

Meatballs are savory, versatile and easy, but they're definitely not
what comes to mind when I think of ``seasonal cooking.''

Maybe it's their long-term relationship with marinara sauce and
spaghetti. Hearty and crowd-pleasing, yes. Buoyant and summery, not so
much.

But these meatballs turn everything on its head.

With fresh basil, ground cumin and ginger, they're heady and complex.
But it's the quick pan sauce that really sets them apart. The
combination of ripe peaches and plenty of lime juice gives them a tangy
brightness that's refreshing enough for even the sultriest late-summer
nights, along with all the seasonal credibility they could ever need.

\includegraphics{https://static01.graylady3jvrrxbe.onion/images/2020/08/28/dining/28appetite-meatballs/merlin_176001855_01e8bab5-e72e-4bbf-8001-19ea69dd2d2a-articleLarge.jpg?quality=75\&auto=webp\&disable=upscale}

You can make the meatballs with any kind of ground meat. Even vegan meat
will work quite well. Pork, with its brawny, rich flavor, is my
favorite, with dark meat turkey or chicken as close runners-up.

Really, it's the panful of peach drippings that makes this dish shine.
So, it's worth buying the fruit ahead of time, and letting it soften and
sweeten for a few days. Or, if you already have a surfeit of bruised,
overripe fruit leaking nectar all over your counter, this is the recipe
for you.

Just cut out any obviously browned spots before throwing the rest of the
peach flesh into the skillet, where it will dissolve amid a bath of wine
and the aromatic drippings from the meatballs to create the sauce.

As a rule, I don't peel peaches because the fuzz doesn't bother me. Fuzz
haters can peel as they like. A sharp paring knife generally gets the
job done more quickly and efficiently than a vegetable peeler. (No need
to blanch them here.) Or substitute fuzz-free nectarines, which are
arguably superior to peaches anyway.

Image

Credit...Bryan Gardner for The New York Times. Food Stylist: Barrett
Washburne.

When peach season wanes and plum season kicks into high gear, you can
substitute diced plums, holding back slightly on the lime juice to make
up for their tannic, puckery skins.

Then, as the cold sets in and fresh stone fruit disappears, you'll still
be able to throw this together using frozen peaches. Make sure to thaw
and drain them before dicing and adding to the pan.

Of course, using frozen peaches does mean these meatballs would no
longer be considered strictly seasonal fare. But no one will be sad to
gobble them up when winter eventually arrives.

Recipe:
\textbf{\href{https://cooking.nytimes3xbfgragh.onion/recipes/1021402-skillet-meatballs-with-peaches-basil-and-lime}{Skillet
Meatballs With Peaches, Basil and Lime}}

\emph{Follow} \href{https://twitter.com/nytfood}{\emph{NYT Food on
Twitter}} \emph{and}
\href{https://www.instagram.com/nytcooking/}{\emph{NYT Cooking on
Instagram}}\emph{,}
\href{https://www.facebookcorewwwi.onion/nytcooking/}{\emph{Facebook}}\emph{,}
\href{https://www.youtube.com/nytcooking}{\emph{YouTube}} \emph{and}
\href{https://www.pinterest.com/nytcooking/}{\emph{Pinterest}}\emph{.}
\href{https://www.nytimes3xbfgragh.onion/newsletters/cooking}{\emph{Get
regular updates from NYT Cooking, with recipe suggestions, cooking tips
and shopping advice}}\emph{.}

Advertisement

\protect\hyperlink{after-bottom}{Continue reading the main story}

\hypertarget{site-index}{%
\subsection{Site Index}\label{site-index}}

\hypertarget{site-information-navigation}{%
\subsection{Site Information
Navigation}\label{site-information-navigation}}

\begin{itemize}
\tightlist
\item
  \href{https://help.nytimes3xbfgragh.onion/hc/en-us/articles/115014792127-Copyright-notice}{©~2020~The
  New York Times Company}
\end{itemize}

\begin{itemize}
\tightlist
\item
  \href{https://www.nytco.com/}{NYTCo}
\item
  \href{https://help.nytimes3xbfgragh.onion/hc/en-us/articles/115015385887-Contact-Us}{Contact
  Us}
\item
  \href{https://www.nytco.com/careers/}{Work with us}
\item
  \href{https://nytmediakit.com/}{Advertise}
\item
  \href{http://www.tbrandstudio.com/}{T Brand Studio}
\item
  \href{https://www.nytimes3xbfgragh.onion/privacy/cookie-policy\#how-do-i-manage-trackers}{Your
  Ad Choices}
\item
  \href{https://www.nytimes3xbfgragh.onion/privacy}{Privacy}
\item
  \href{https://help.nytimes3xbfgragh.onion/hc/en-us/articles/115014893428-Terms-of-service}{Terms
  of Service}
\item
  \href{https://help.nytimes3xbfgragh.onion/hc/en-us/articles/115014893968-Terms-of-sale}{Terms
  of Sale}
\item
  \href{https://spiderbites.nytimes3xbfgragh.onion}{Site Map}
\item
  \href{https://help.nytimes3xbfgragh.onion/hc/en-us}{Help}
\item
  \href{https://www.nytimes3xbfgragh.onion/subscription?campaignId=37WXW}{Subscriptions}
\end{itemize}
