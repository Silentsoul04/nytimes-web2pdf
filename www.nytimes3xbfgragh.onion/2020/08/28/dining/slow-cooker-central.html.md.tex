Sections

SEARCH

\protect\hyperlink{site-content}{Skip to
content}\protect\hyperlink{site-index}{Skip to site index}

\href{https://www.nytimes3xbfgragh.onion/section/food}{Food}

\href{https://myaccount.nytimes3xbfgragh.onion/auth/login?response_type=cookie\&client_id=vi}{}

\href{https://www.nytimes3xbfgragh.onion/section/todayspaper}{Today's
Paper}

\href{/section/food}{Food}\textbar{}Slow Cooker Central

\url{https://nyti.ms/3hymmoS}

\begin{itemize}
\item
\item
\item
\item
\item
\end{itemize}

\href{https://www.nytimes3xbfgragh.onion/spotlight/at-home?action=click\&pgtype=Article\&state=default\&region=TOP_BANNER\&context=at_home_menu}{At
Home}

\begin{itemize}
\tightlist
\item
  \href{https://www.nytimes3xbfgragh.onion/2020/09/07/travel/route-66.html?action=click\&pgtype=Article\&state=default\&region=TOP_BANNER\&context=at_home_menu}{Cruise
  Along: Route 66}
\item
  \href{https://www.nytimes3xbfgragh.onion/2020/09/04/dining/sheet-pan-chicken.html?action=click\&pgtype=Article\&state=default\&region=TOP_BANNER\&context=at_home_menu}{Roast:
  Chicken With Plums}
\item
  \href{https://www.nytimes3xbfgragh.onion/2020/09/04/arts/television/dark-shadows-stream.html?action=click\&pgtype=Article\&state=default\&region=TOP_BANNER\&context=at_home_menu}{Watch:
  Dark Shadows}
\item
  \href{https://www.nytimes3xbfgragh.onion/interactive/2020/at-home/even-more-reporters-editors-diaries-lists-recommendations.html?action=click\&pgtype=Article\&state=default\&region=TOP_BANNER\&context=at_home_menu}{Explore:
  Reporters' Google Docs}
\end{itemize}

Advertisement

\protect\hyperlink{after-top}{Continue reading the main story}

Supported by

\protect\hyperlink{after-sponsor}{Continue reading the main story}

\href{/column/five-weeknight-dishes}{Five Weeknight Dishes}

\hypertarget{slow-cooker-central}{%
\section{Slow Cooker Central}\label{slow-cooker-central}}

By Margaux Laskey

\begin{itemize}
\item
  Aug. 28, 2020
\item
  \begin{itemize}
  \item
  \item
  \item
  \item
  \item
  \end{itemize}
\end{itemize}

Hello, and welcome to
\href{https://www.nytimes3xbfgragh.onion/column/five-weeknight-dishes}{Five
Weeknight Dishes}. It's no secret to anyone who knows me that I love my
slow cooker. (I've used it so much that the temperature markings have
worn off, and I've had to write over them with a Sharpie.) Slow cookers
get a bad rap for many reasons, but, for a lot of busy home cooks, they
are a lifesaver. The key is choosing your recipes wisely.

The complaint I hear most often is that they're a ``myth'': Many recipes
require several steps, so they're not convenient. My tip? Don't make
those recipes! Or do as I do and skip the searing and sautéing, and use
jarred chopped garlic and frozen chopped onion. Take a cue from our
recipe developer Sarah DiGregorio, and amplify flavors with a 1/2
teaspoon or so of garlic or onion powder.

If you're still not convinced, that's OK! The good news is that many of
this week's dishes can be made on the stovetop or in an electric
pressure cooker. You do you.

Do you love the slow cooker? Do you hate it? Do you want more, or less?
Let me know at
\href{mailto:margaux@NYTimes.com}{\nolinkurl{margaux@NYTimes.com}}.
Emily is back next week. It's been a pleasure!

\emph{{[}\href{https://www.nytimes3xbfgragh.onion/newsletters/five-weeknight-dishes?module=inline}{Sign
up here}} \emph{to receive the Five Weeknight Dishes newsletter in your
inbox every Friday.{]}}

Here are five dishes for the week.

\includegraphics{https://static01.graylady3jvrrxbe.onion/images/2020/08/30/dining/tk-butternut-squash-soup-with-brown-butter/tk-butternut-squash-soup-with-brown-butter-articleLarge-v3.jpg?quality=75\&auto=webp\&disable=upscale}

\textbf{1.}
\href{https://cooking.nytimes3xbfgragh.onion/recipes/1019695-slow-cooker-coconut-curry-soup-with-sweet-potato-and-kale}{\textbf{Slow-Cooker
Curried Sweet Potato Soup With Coconut and Kale}}

This creamy Thai-influenced soup from Sarah DiGregorio is lovely served
on its own or over a pile of rice. You can make it vegan, if your chile
paste doesn't have fish or shrimp in it, and adjust the thickness with
some water. Since the spice level of chile paste can vary so much, start
with a little and add more to taste. One reader used leftovers as a
sauce for chicken the next day, which is a very smart move.

\href{https://cooking.nytimes3xbfgragh.onion/recipes/1019695-slow-cooker-coconut-curry-soup-with-sweet-potato-and-kale}{\emph{View
this recipe.}}

\emph{\_\_\_\_\_}

Image

Sarah DiGregorio's chicken with 20 cloves of garlic.Credit...Linda Xiao
for The New York Times. Food Stylist: Monica Pierini.

\textbf{2.}
\href{https://cooking.nytimes3xbfgragh.onion/recipes/1020840-slow-cooker-chicken-with-20-cloves-of-garlic}{\textbf{Slow-Cooker
Chicken With 20 Cloves of Garlic}}

Sarah's riff on the French classic ---
\href{https://cooking.nytimes3xbfgragh.onion/recipes/3764-chicken-with-40-cloves-of-garlic}{chicken
with 40 cloves of garlic} --- uses only 20 cloves because a slow cooker
doesn't get hot enough to mellow out the traditional 40. If you have a
few extra minutes, slide the chicken under the broiler before serving to
get a nice golden-brown color. Serve with chunks of good bread.

\emph{\href{https://cooking.nytimes3xbfgragh.onion/recipes/1020840-slow-cooker-chicken-with-20-cloves-of-garlic}{View
this recipe.}}

\emph{\_\_\_\_\_}

Image

Credit...Ryan Liebe for The New York Times. Food Stylist: Simon Andrews.

\textbf{3.}
\href{https://cooking.nytimes3xbfgragh.onion/recipes/1020497-slow-cooker-bbq-pulled-pork}{\textbf{Slow-Cooker
BBQ Pulled Pork}}

This pulled pork is based on a three-ingredient recipe --- pork
shoulder, Dr Pepper or cola, and barbecue sauce --- that can be found
all over Pinterest and food blogs. I asked the members of the
\href{https://www.facebookcorewwwi.onion/groups/nytcooks/}{NYT Cooking
Facebook Community} if they'd tried the recipe and liked it. There were
strong opinions! I made it, tweaked it a little, and landed here. There
are a couple more steps and ingredients, but it's still a breeze to
make. Sub coffee or beer for the soda, if you like.

\emph{\href{https://cooking.nytimes3xbfgragh.onion/recipes/1020497-slow-cooker-bbq-pulled-pork}{View
this recipe.}}

\emph{\_\_\_\_\_}

Image

Credit...Julia Gartland for The New York Times. Food Stylist: Barrett
Washburne.

\textbf{4.}
\href{https://cooking.nytimes3xbfgragh.onion/recipes/1019696-slow-cooker-white-bean-parmesan-soup}{\textbf{Slow-Cooker
White Bean Parmesan Soup}}

This hearty meatless stew-soup from Sarah DiGregorio made with creamy
white beans, chewy wheat berries, fennel and Parmesan is an absolute
delight. Wheat berries take a long time to cook, so they're ideal in the
slow cooker, but whole-grain (not pearled) farro or spelt will work,
too. Add them a bit later because they don't require as much cooking and
will get too soft. If you, like me, can never remember to soak your
beans the night before, employ the
\href{https://cooking.nytimes3xbfgragh.onion/guides/21-how-to-cook-beans}{quick-soak
method} or add four cans of cooked white beans about halfway through
cooking time.

\emph{\href{https://cooking.nytimes3xbfgragh.onion/recipes/1019696-slow-cooker-white-bean-parmesan-soup}{View
this recipe.}}

\emph{\_\_\_\_\_}

Image

Mississippi RoastCredit...Melina Hammer for The New York Times

\textbf{5.}
\href{https://cooking.nytimes3xbfgragh.onion/recipes/1017937-mississippi-roast}{\textbf{Mississippi
Roast}}

This is the recipe that made Sam Sifton fall in love with, um... let's
say \emph{reconsider}, slow cookers. It's inspired by another
Pinterest-favorite, which is made with a three-pound beef chuck roast, a
packet of powdered Ranch dressing mix, a packet of powdered au jus gravy
mix, a stick of butter and a handful of pepperoncini peppers. Sam
figured out a from-scratch approach, and readers went wild. It was our
most popular recipe of 2016. (It also works with pork shoulder.)

\emph{\href{https://cooking.nytimes3xbfgragh.onion/recipes/1017937-mississippi-roast}{View
this recipe.}}

\emph{Support our work at NYT Cooking --- and get full access to 20,000
recipes --- by}
\href{https://www.nytimes3xbfgragh.onion/subscription/cooking.html?campaignId=788FJ}{\emph{becoming
a subscriber}}\emph{. (Or}
\href{https://www.nytimes3xbfgragh.onion/subscriptions/Multiproduct/cooking_gift.html?campaignId=78X7R}{\emph{give
a subscription as a gift}}\emph{!) You can also follow NYT Cooking on}
\href{https://www.instagram.com/nytcooking}{\emph{Instagram}}\emph{,}
\href{https://www.facebookcorewwwi.onion/nytcooking/}{\emph{Facebook}}
\emph{and}
\href{https://www.pinterest.com/nytcooking/}{\emph{Pinterest}}\emph{.
Previous newsletters}
\href{https://www.nytimes3xbfgragh.onion/column/five-weeknight-dishes}{\emph{are
archived here}}\emph{. If you have any problems with your account,
email}
\href{mailto:cookingcare@NYTimes.com}{\emph{cookingcare@NYTimes.com}}\emph{.}

Advertisement

\protect\hyperlink{after-bottom}{Continue reading the main story}

\hypertarget{site-index}{%
\subsection{Site Index}\label{site-index}}

\hypertarget{site-information-navigation}{%
\subsection{Site Information
Navigation}\label{site-information-navigation}}

\begin{itemize}
\tightlist
\item
  \href{https://help.nytimes3xbfgragh.onion/hc/en-us/articles/115014792127-Copyright-notice}{©~2020~The
  New York Times Company}
\end{itemize}

\begin{itemize}
\tightlist
\item
  \href{https://www.nytco.com/}{NYTCo}
\item
  \href{https://help.nytimes3xbfgragh.onion/hc/en-us/articles/115015385887-Contact-Us}{Contact
  Us}
\item
  \href{https://www.nytco.com/careers/}{Work with us}
\item
  \href{https://nytmediakit.com/}{Advertise}
\item
  \href{http://www.tbrandstudio.com/}{T Brand Studio}
\item
  \href{https://www.nytimes3xbfgragh.onion/privacy/cookie-policy\#how-do-i-manage-trackers}{Your
  Ad Choices}
\item
  \href{https://www.nytimes3xbfgragh.onion/privacy}{Privacy}
\item
  \href{https://help.nytimes3xbfgragh.onion/hc/en-us/articles/115014893428-Terms-of-service}{Terms
  of Service}
\item
  \href{https://help.nytimes3xbfgragh.onion/hc/en-us/articles/115014893968-Terms-of-sale}{Terms
  of Sale}
\item
  \href{https://spiderbites.nytimes3xbfgragh.onion}{Site Map}
\item
  \href{https://help.nytimes3xbfgragh.onion/hc/en-us}{Help}
\item
  \href{https://www.nytimes3xbfgragh.onion/subscription?campaignId=37WXW}{Subscriptions}
\end{itemize}
