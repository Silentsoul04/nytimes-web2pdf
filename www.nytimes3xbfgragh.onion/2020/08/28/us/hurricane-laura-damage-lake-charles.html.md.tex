Sections

SEARCH

\protect\hyperlink{site-content}{Skip to
content}\protect\hyperlink{site-index}{Skip to site index}

\href{https://www.nytimes3xbfgragh.onion/section/us}{U.S.}

\href{https://myaccount.nytimes3xbfgragh.onion/auth/login?response_type=cookie\&client_id=vi}{}

\href{https://www.nytimes3xbfgragh.onion/section/todayspaper}{Today's
Paper}

\href{/section/us}{U.S.}\textbar{}Hurricane Laura Was Powerful, but
Louisiana Was Prepared

\url{https://nyti.ms/31zYNqe}

\begin{itemize}
\item
\item
\item
\item
\item
\end{itemize}

\hypertarget{hurricane-laura}{%
\subsubsection{Hurricane Laura}\label{hurricane-laura}}

\begin{itemize}
\tightlist
\item
  \href{https://www.nytimes3xbfgragh.onion/2020/08/27/us/hurricane-laura-update.html?name=styln-hurricanes\&region=TOP_BANNER\&block=storyline_menu_recirc\&action=click\&pgtype=Article\&impression_id=825d0c90-f290-11ea-846f-7542dfffcf40\&variant=undefined}{Aftermath
  of the Hurricane}
\item
  \href{https://www.nytimes3xbfgragh.onion/2020/08/28/us/hurricane-laura-damage-lake-charles.html?name=styln-hurricanes\&region=TOP_BANNER\&block=storyline_menu_recirc\&action=click\&pgtype=Article\&impression_id=825d0c91-f290-11ea-846f-7542dfffcf40\&variant=undefined}{How
  Louisiana Fared}
\item
  \href{https://www.nytimes3xbfgragh.onion/2020/08/27/us/hurricane-laura-photos.html?name=styln-hurricanes\&region=TOP_BANNER\&block=storyline_menu_recirc\&action=click\&pgtype=Article\&impression_id=825d0c92-f290-11ea-846f-7542dfffcf40\&variant=undefined}{Photos}
\item
  \href{https://www.nytimes3xbfgragh.onion/2020/05/18/climate/climate-changes-hurricane-intensity.html?name=styln-hurricanes\&region=TOP_BANNER\&block=storyline_menu_recirc\&action=click\&pgtype=Article\&impression_id=825d33a0-f290-11ea-846f-7542dfffcf40\&variant=undefined}{Why
  Storms Are Stronger}
\end{itemize}

Advertisement

\protect\hyperlink{after-top}{Continue reading the main story}

Supported by

\protect\hyperlink{after-sponsor}{Continue reading the main story}

\hypertarget{hurricane-laura-was-powerful-but-louisiana-was-prepared}{%
\section{Hurricane Laura Was Powerful, but Louisiana Was
Prepared}\label{hurricane-laura-was-powerful-but-louisiana-was-prepared}}

In 2005, Hurricane Rita leveled some coastal communities in southwest
Louisiana, forcing changes to building codes and attitudes. As Laura
approached, the region was ready.

\includegraphics{https://static01.graylady3jvrrxbe.onion/images/2020/08/28/us/28hurricane-1/merlin_176256363_092b7fe2-db41-4532-92f1-71d359378aad-articleLarge.jpg?quality=75\&auto=webp\&disable=upscale}

By \href{https://www.nytimes3xbfgragh.onion/by/manny-fernandez}{Manny
Fernandez} and
\href{https://www.nytimes3xbfgragh.onion/by/richard-fausset}{Richard
Fausset}

\begin{itemize}
\item
  Aug. 28, 2020
\item
  \begin{itemize}
  \item
  \item
  \item
  \item
  \item
  \end{itemize}
\end{itemize}

CARLYSS, La. --- Karla Broussard was standing under the awning of her
liquor store on Friday, to stay dry in the rain. All around her the town
of Carlyss was a mess. Hurricane Laura had not been kind.

Power lines, and the poles attached to them, draped the Louisiana roads.
The steeple at the First Baptist Church of Carlyss lay in a pile of
debris by the front doors. Some homes and businesses were see-through,
with missing roofs, walls, doorways. The skies were gray. Thunder
rumbled in the distance.

``This is way worse than Rita,'' she said, echoing the sentiment of many
in the region this week.

Ms. Broussard was referring to Hurricane Rita, the massive 2005 storm
that tore ferociously through southwest Louisiana and East Texas a few
weeks after its more infamous cousin, Hurricane Katrina. It was Katrina,
which crippled New Orleans and killed more than 1,500 people, that got
the world's attention. But out in the wide-open, working-class,
oil-patch country to the west of New Orleans, it was Rita that became
the benchmark storm for a generation.

\includegraphics{https://static01.graylady3jvrrxbe.onion/images/2020/08/28/us/28hurricane-2/merlin_176303019_fe2a4c24-379f-48b3-b801-46f2c6c10d6c-articleLarge.jpg?quality=75\&auto=webp\&disable=upscale}

Image

The roof of a Domino's was blown off in Moss Bluff, La.Credit...Matthew
Busch for The New York Times

In a place like Carlyss, a town of 6,000 people just west of Lake
Charles, Rita is the storm all others are judged against. It leveled
some small coastal communities almost completely and was the first to
seriously rattle the area since Hurricane Audrey, which killed more than
400 people in 1957.

But perhaps most significantly, Rita forced changes in regulations and
attitudes that may have saved lives when Laura roared ashore early
Thursday morning with 150-mile-per-hour winds --- one of the most
powerful hurricanes to hit the U.S. mainland.

Jim Beam, a longtime columnist for the American Press newspaper in Lake
Charles, said that the memory of Rita probably prompted people to take
evacuation orders this week more seriously than they had before. Roughly
half a million people evacuated this week in advance of the storm, which
may have contributed to its relatively low death toll of 14 people.

``My personal experience from Rita convinced me that this time, when
they tell you to leave, you'd better leave,'' Mr. Beam said.

Katrina and Rita also led to changes in construction codes to make
buildings less vulnerable to the violent winds and ruinous floodwaters
of major storms. Some homes close to the coast were elevated as high as
14 feet or more.

**``**While this is not the case universally, you can tell that some of
the buildings built post-Rita held up better than buildings built
pre-Rita,'' Mayor Nic Hunter of Lake Charles said on Friday. ``I am
sitting right now on the sixth floor of historic City Hall and there are
no broken windows because all of them were replaced after Rita.''

Image

Water collected in the front yards of stilt homes in Sulphur,
La.~Credit...Emily Kask for The New York Times

Image

Marshlands Tobacco and Beer in Carlyss, La. The town was left a mess by
Hurricane Laura.Credit...Matthew Busch for The New York Times

That was not the case in much of Lake Charles, one of the hardest-hit
communities. Mr. Hunter said every neighborhood in his city of 78,000
people had seen damage in the storm. No one had electrical service, he
said, but the city was banding together, clearing roads of debris and
welcoming the help of the National Guard and a small army of government
workers.

``We are picking up the pieces,'' he said. ``It's going to be quite a
task.''

Of the 10 known deaths in Louisiana, four people were killed when trees
fell on their homes, one person drowned in a boat, and five people died
from carbon monoxide poisoning related to the use of generators, Gov.
John Bel Edwards said at a news conference on Friday. East Texas, which
braced for the worst but was not as badly hit, had four **** reported
deaths as of Friday, according to The Associated Press, including three
people who died from carbon monoxide poisoning in Port Arthur.

Mr. Edwards said that Louisiana requested a major federal disaster
declaration from the White House around noon on Friday, and he noted
that there were about 6,200 members of the National Guard working on
recovery efforts.

At least 82 water systems were inoperable on Friday, he said, causing a
rolling series of evacuations from nursing homes and hospitals. He also
reminded residents that the coronavirus pandemic had not disappeared and
that community testing sites --- which closed as Laura approached ---
would reopen by Monday.

``This is a very bad week for us not to be doing robust testing,'' he
said.

Image

Hotels near Interstate 210 were badly damaged by the
hurricane.Credit...William Widmer for The New York Times

Image

Generators were being sold from the back of a truck in Lake
Charles.~Credit...William Widmer for The New York Times

As the storm loomed offshore and grew to Category 4 strength, Mr.
Edwards repeatedly made comparisons to Hurricane Rita, saying Laura
resembled Rita in its intensity and path, hoping to telegraph a sense of
the gravity of what might be coming.

When it finally made landfall near Cameron, La., in the dark
early-morning hours of Thursday, Laura's anticipated ``unsurvivable''
storm surge of 15 to 20 feet was more like 11 feet, at most. Still,
Laura unleashed misery and ruin. And most everyone agreed that the wind
damage was as severe as --- or worse than --- Rita, which landed as a
Category 3 storm with 120 m.p.h. winds.

In the aftermath of Laura, residents across southwest Louisiana took
part in a stressful and familiar post-storm dance, returning home to
survey the destruction.

Some residents whose homes were inaccessible because of flooding could
only squint at overhead photos. In a Facebook group, people swapped
numbers for contractors, roofers and lawyers --- and photos of crushed
trailer homes, demolished stores, and fields of storm-tossed debris.

Many residents wondered whether the post-Rita building codes had put
them in a better place.

``I had my home rebuilt after Rita. I remember the inspection process
being a bit of a nightmare,'' Taryn Scarborough, a Lake Charles
resident, wrote in an online message. ``It took a couple inspections and
modifications before the roof inspection passed. Needed hurricane ties
and additional bracing and soffits with air flow.''

Image

In Lake Charles, the entire roof of a three-story Motel 6 was ripped
off.~Credit...William Widmer for The New York Times

Image

Farrah Thiel and Chris Behr waited for family to pick them up from the
motel.Credit...William Widmer for The New York Times

``We're headed to Lake Charles now to see if all the work we put into
rebuilding 15 years ago made a difference,'' she added. ``Fingers
Crossed!''

If Rita taught the people of the region to be prepared for storms that
seemed more intense and frequent, that lesson could only do so much.
Each storm, it seems, delivers its own quirks, cruelties and
vicissitudes.

On Friday afternoon, Marlon Gardner, 42, was busy sweeping up the
branches and detritus on Monroe Street. In a city that is a shambles, at
least one small patch was smooth and clean.

Mr. Gardner, a Lake Charles native, has lived in a house on Monroe
Street off and on since he was a child. It survived Rita with a few
scratches and a few missing shingles. This time, though, it took a
beating.

Large sections of the roof lost shingles. An addition he had built at
the back of the house was ruined. The rain that fell on Lake Charles on
Friday fell inside Mr. Gardner's home. All along his neighborhood, trees
sliced into homes, cracked in two or took down power lines.

As Laura had approached, Mr. Gardner initially decided to ride it out
and stay in Lake Charles. But as the storm got closer, his relatives
kept calling him, pleading for him to flee. He finally did, traveling to
Dallas.

``I think the lesson that was ultimately learned is --- get out while
you still can,'' Mr. Gardner said of Rita and Laura. ``This time, more
people took heed and got out.''

Now, he added: ``Everybody's trying to get home. Some people are on
their way now.''

Next is the familiar drudgery. The weeks of it. The months.

``In my mind, I think about the recovery, like how long is it going to
take to get back to normal,'' he said. ``How is Lake Charles going to be
after this? How long is it going to take for us to recover and clean up
--- the lights, the water, the stores, everything? If I need some milk,
I'm going to have to go to Texas, get me some milk and come back.''

Manny Fernandez reported from Carlyss, and Richard Fausset from Atlanta.
Rick Rojas contributed reporting from Beaumont, Texas, and Giulia
McDonnell Nieto del Rio from Boston.

Advertisement

\protect\hyperlink{after-bottom}{Continue reading the main story}

\hypertarget{site-index}{%
\subsection{Site Index}\label{site-index}}

\hypertarget{site-information-navigation}{%
\subsection{Site Information
Navigation}\label{site-information-navigation}}

\begin{itemize}
\tightlist
\item
  \href{https://help.nytimes3xbfgragh.onion/hc/en-us/articles/115014792127-Copyright-notice}{©~2020~The
  New York Times Company}
\end{itemize}

\begin{itemize}
\tightlist
\item
  \href{https://www.nytco.com/}{NYTCo}
\item
  \href{https://help.nytimes3xbfgragh.onion/hc/en-us/articles/115015385887-Contact-Us}{Contact
  Us}
\item
  \href{https://www.nytco.com/careers/}{Work with us}
\item
  \href{https://nytmediakit.com/}{Advertise}
\item
  \href{http://www.tbrandstudio.com/}{T Brand Studio}
\item
  \href{https://www.nytimes3xbfgragh.onion/privacy/cookie-policy\#how-do-i-manage-trackers}{Your
  Ad Choices}
\item
  \href{https://www.nytimes3xbfgragh.onion/privacy}{Privacy}
\item
  \href{https://help.nytimes3xbfgragh.onion/hc/en-us/articles/115014893428-Terms-of-service}{Terms
  of Service}
\item
  \href{https://help.nytimes3xbfgragh.onion/hc/en-us/articles/115014893968-Terms-of-sale}{Terms
  of Sale}
\item
  \href{https://spiderbites.nytimes3xbfgragh.onion}{Site Map}
\item
  \href{https://help.nytimes3xbfgragh.onion/hc/en-us}{Help}
\item
  \href{https://www.nytimes3xbfgragh.onion/subscription?campaignId=37WXW}{Subscriptions}
\end{itemize}
