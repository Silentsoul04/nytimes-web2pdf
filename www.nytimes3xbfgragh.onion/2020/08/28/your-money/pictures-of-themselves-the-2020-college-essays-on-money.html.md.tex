Sections

SEARCH

\protect\hyperlink{site-content}{Skip to
content}\protect\hyperlink{site-index}{Skip to site index}

\href{https://www.nytimes3xbfgragh.onion/section/your-money}{Your Money}

\href{https://myaccount.nytimes3xbfgragh.onion/auth/login?response_type=cookie\&client_id=vi}{}

\href{https://www.nytimes3xbfgragh.onion/section/todayspaper}{Today's
Paper}

\href{/section/your-money}{Your Money}\textbar{}Pictures of Themselves:
The 2020 College Essays on Money

\url{https://nyti.ms/2FYKSS8}

\begin{itemize}
\item
\item
\item
\item
\item
\end{itemize}

Advertisement

\protect\hyperlink{after-top}{Continue reading the main story}

Supported by

\protect\hyperlink{after-sponsor}{Continue reading the main story}

Your Money

\hypertarget{pictures-of-themselves-the-2020-college-essays-on-money}{%
\section{Pictures of Themselves: The 2020 College Essays on
Money}\label{pictures-of-themselves-the-2020-college-essays-on-money}}

Each year, we ask high school seniors to send us college application
essays that touch on money, work or social class. Here are four from
this year's incoming college freshmen.

\includegraphics{https://static01.graylady3jvrrxbe.onion/images/2020/08/27/business/27money/27money-articleLarge.jpg?quality=75\&auto=webp\&disable=upscale}

\href{https://www.nytimes3xbfgragh.onion/by/ron-lieber}{\includegraphics{https://static01.graylady3jvrrxbe.onion/images/2018/10/22/multimedia/author-ron-lieber/author-ron-lieber-thumbLarge.png}}

By \href{https://www.nytimes3xbfgragh.onion/by/ron-lieber}{Ron Lieber}

\begin{itemize}
\item
  Aug. 28, 2020
\item
  \begin{itemize}
  \item
  \item
  \item
  \item
  \item
  \end{itemize}
\end{itemize}

Consider these dispatches from the time before.

Before the coronavirus, before college students went home and stayed
there, before protests amplified calls for racial justice, a whole bunch
of teenagers did a normal thing at a normal time: They tried to say
something meaningful about who they were to a collection of strangers
who could give them access to a great education.

Reading their college application essays now, it's hard not to feel at
least a little bit optimistic for the future. They hustle. They make do.
They reckon with themselves, how they see the world and how they are
seen in it, too.

Each year, I ask graduating seniors to send me application essays about
work, money, social class or related topics. We adults don't talk about
money and our feelings about it often enough, so it only seems right to
try to learn from the teenagers who have figured out how to do it well.

And so here, we meet the artist who works amid her immigrant parents'
construction projects, a Black man who uses poetry to provide a dose of
perspective, an unlikely conveyance in upscale Connecticut and the maker
of a blanket that stands for so much more.

\begin{center}\rule{0.5\linewidth}{\linethickness}\end{center}

\hypertarget{vienna-va}{%
\subsubsection{Vienna, Va.}\label{vienna-va}}

\includegraphics{https://static01.graylady3jvrrxbe.onion/images/2020/08/28/business/28money-maria/28money1b-articleLarge.jpg?quality=75\&auto=webp\&disable=upscale}

\hypertarget{when-we-arrived-my-parents-caught-the-american-dream-like-tuxedo-conejo-used-his-rabbit-tricksiness-to-outwit-tuxedo-tigre-in-the-fables-so-artfully-that-they-themselves-hardly-believed-theyd-pulled-it-off}{%
\subsection{`When we arrived, my parents caught the American dream like
Tío Conejo used his rabbit tricksiness to outwit Tío Tigre in the
fables: so artfully that they themselves hardly believed they'd pulled
it
off.'}\label{when-we-arrived-my-parents-caught-the-american-dream-like-tuxedo-conejo-used-his-rabbit-tricksiness-to-outwit-tuxedo-tigre-in-the-fables-so-artfully-that-they-themselves-hardly-believed-theyd-pulled-it-off}}

\hypertarget{maria-mendoza-blanco}{%
\subsubsection{Maria Mendoza Blanco}\label{maria-mendoza-blanco}}

\subsubsection{***}

If I had it my way, I'd never set foot in a Home Depot ever again.

Every Ace Hardware, every Lowe's, every boutique tile place, every
obscure little hardware store that only sells Phillips-head screwdrivers
smells the same way: dusty. Sawdust, catdust, paint-flake-dust,
laminate-dust, ancient-grumpy-cashier-dust. It's post-apocalyptic, the
shuffling shoppers dead-eyed from looking at a thousand identical
refrigerators, fluorescent tube lights casting ultramarine pallor over
their faces.

We kill tigers, you see.

``Where are we going?'' I'll ask, and my father will say, ``Lowe's. Hay
que matar tigres.'' Gotta kill tigers, gotta take side jobs to fill in
the gaps where the money doesn't quite reach. Where others might have
taken up Uber, my family started building houses, with me and my brother
in tow.

When we arrived, my parents caught the American dream like Tío Conejo
used his rabbit tricksiness to outwit Tío Tigre in the fables: so
artfully that they themselves hardly believed they'd pulled it off. We
killed tigers in Georgetown and Langley, diplomat townhomes and
tasteless McMansions alike. We moved seven times within the same ZIP
code, as my parents bought ugly houses and sold them beautiful.

It isn't much like HGTV. I spent countless hours searching for
nonexistent cans of Spackle in the back shelves of Home Depot. My mother
laid out carpet samples on the floor and paced around them, forever
deliberating between ivory and cream. She'd be on the phone with some
hung over subcontractor when she picked me up from art club. I'd sit in
an abandoned corner and sketch as they haggled eternally over hardwood
pallets at auction. I wouldn't be surprised if I spent more time under
the watchful eye of an orange-aproned paint mixer than a babysitter.

All this is to say that construction runs in the veins of the Blancos.
My grandfather, after all, came out of nowhere to build a concrete
empire on the baked dirt of Maracaibo. The mixers molder now in that
hinterland, but the force of his success was what fueled our escape from
Venezuela before things got bad.

What use would my grandfather have had for all the sketches I've
sketched, all the paintings I've painted, I wonder? Could my parents
paper their clients' walls with pages from my sketchbook, could they
tear up the canvas and use it for insulation? Probably not.

In art, there's this fantasy of The Muse reaching down and the lucky
artist's paintbrush dancing with a press of her rosy fingertip. The
truth is that I can have the most perfect concept handed to me by the
ghost of Gentileschi herself, and I'll still get in my own way.
Perfectionism won't let me pick a background color for weeks, envy will
distract me with foolhardy attempts at others' success, simple laziness
will keep me in bed watching episodes of ``Chopped'' 15 times.

Whenever my still-white canvas stretched blankly into the infinite, I
thought about that, about the long nights my parents must have spent
thinking about their own parents. About the three hours daily my mother
spends commuting to her day job. About my father's lost stories, the
jokes he doesn't tell because English warps his humor. About the life
they left behind in Maracaibo, all so that we could live here. All so
that I could come here and be an artist, of all things.

So it's not easy moving from concrete to canvas. But I must do it
anyway, because the force of my ambition and, well, my talent demand it.
Because my family's risk deserves a risk of my own. A risk that I must
fight my indolence and ennui for. A risk that will honor our sacrifice
of all these years between two lands. I can't let all those dusty hours
at Home Depot go to waste. Hay que matar tigres.

\begin{center}\rule{0.5\linewidth}{\linethickness}\end{center}

\hypertarget{bethlehem-pa}{%
\subsubsection{Bethlehem, Pa.}\label{bethlehem-pa}}

Image

Julius Ewungkem Jr. is attending Harvard University.Credit...Raymond
Holman for The New York Times

\hypertarget{she-might-be-free-but-the-world-still-doesnt-have-to-treat-her-equally-no-one-is-obligated-to-give-her-a-job-she-is-the-same-person-that-she-was-the-week-before}{%
\subsection{`She might be free, but the world still doesn't have to
treat her equally. No one is obligated to give her a job. She is the
same person that she was the week
before.'}\label{she-might-be-free-but-the-world-still-doesnt-have-to-treat-her-equally-no-one-is-obligated-to-give-her-a-job-she-is-the-same-person-that-she-was-the-week-before}}

\hypertarget{julius-ewungkem-jr}{%
\subsubsection{Julius Ewungkem Jr.}\label{julius-ewungkem-jr}}

\hypertarget{-1}{%
\subsubsection{***}\label{-1}}

\emph{Some challenges transcend time, constantly popping up in different
forms}

\emph{As a society, we strive to quantify success}

\emph{``If you work hard, you'll see results''}

\emph{This phrase is constantly used to blame others}

\emph{And a specific group of people have felt the brunt of this attack}

\emph{Yes, slavery in the U.S. was abolished over 150 years ago}

\emph{But let me paint a picture}

\emph{Let's say your great-great-great-grandmother was a newly liberated
slave in 1863.}

\emph{How free is she truly?}

\emph{She might be free, but the world still doesn't have to treat her
equally. No one is obligated to give her a job. She is the same person
that she was the week before.}

\emph{And her kids}

\emph{They are now growing up with a mother who can't read or write
while at the same time struggling to live in a society evolving to house
a new race}

\emph{Are her children supposed to immediately succeed?}

\emph{And what about their children?}

\emph{And their children?}

\emph{We are so quick to look at issues}

\emph{High rates of crime, poverty, and unemployment}

\emph{And begin to point fingers}

\emph{Yes, the civil rights movement won equal rights for
African-Americans sixty years ago}

\emph{But segregation is as prevalent as it's ever been}

\emph{So, who is really to blame?}

\emph{It's easy for me to look at some of my best friends from my middle
school and blame them}

\emph{``They chose to skip class'' ``They chose to fight in the
hallways''}

\emph{But did they choose to grow up in an environment that doesn't
value education?}

\emph{Did they choose to grow up with one parent who is working two
jobs?}

\emph{Is something wrong with them, or am I just lucky?}

\emph{Lucky to have two parents who've put education before anything}

\emph{Lucky to attend a high school with plentiful resources}

\emph{We no longer have laws in place that are made solely to hold back
those of certain groups}

\emph{But that doesn't mean the effects aren't the same}

\emph{And as we continue along our journey}

\emph{We must ask ourselves}

\emph{``Whose choices truly created this outcome?'' and ``How do we fix
this mentality and issue?''}

\emph{Many solutions have been proposed}

\emph{But one seems to be truly effective in both the short and long
term}

\emph{Education}

\emph{Not just of the perpetrators, the ignorant, but as well as those
who suffer from this society}

\emph{We all need to learn more, not just the students but the teachers
as well}

\emph{Even as I spread awareness, I know there are so many who know more
and there is so much to learn}

\emph{But is he truly racist or did he grow up in a family that
perpetrated those views?}

\emph{An attack should not be our first response, rather, we need to
teach}

\emph{Show the history, the ups, the downs}

\emph{The accomplishments, the breakthroughs, the struggles}

\emph{Show why we are in the state we are in}

\emph{And for me personally?}

\emph{I'll try}

\emph{To break into homogenous communities and try to teach}

\emph{To show my dreads, curls, and naps}

\emph{To not wait for that next person to say something, but rather be
that next person}

\emph{To always be proud of who I am}

\begin{center}\rule{0.5\linewidth}{\linethickness}\end{center}

\hypertarget{westport-conn}{%
\subsubsection{Westport, Conn.}\label{westport-conn}}

Image

Tadeo Messenger is attending the University of Michigan.Credit...Ike
Abakah for The New York Times

\hypertarget{in-the-blistering-summer-heat-she-would-wait-patiently-for-me-while-i-pulled-weeds-for-hours-on-end-with-sweat-trickling-down-my-face-i-would-take-shelter-from-the-sun-in-her-soft-embrace}{%
\subsection{`In the blistering summer heat she would wait patiently for
me while I pulled weeds for hours on end. With sweat trickling down my
face, I would take shelter from the sun in her soft
embrace.'}\label{in-the-blistering-summer-heat-she-would-wait-patiently-for-me-while-i-pulled-weeds-for-hours-on-end-with-sweat-trickling-down-my-face-i-would-take-shelter-from-the-sun-in-her-soft-embrace}}

\hypertarget{tadeo-messenger}{%
\subsubsection{Tadeo Messenger}\label{tadeo-messenger}}

\hypertarget{-2}{%
\subsubsection{***}\label{-2}}

My friends and peers don't understand my relationship with Big Betsy.
This is mainly due to the fact that Big Betsy is far older, louder, and
larger than what is considered ``normal'' at my school. She is
constantly surrounded by others who serve the same exact purpose, but
are more elegant.

Big Betsy was always different. Every time I went out with her I could
feel judgmental eyes wondering why a kid like me would even want
anything to do with her. Despite this, I was always proud of her and
what we accomplished together. She was made fun of relentlessly, but I
always knew deep down that we had something special together.

It was like we had known each other for years when I first laid eyes on
her. I was sure that we would stay together for a long time. Since the
day I bought Big Betsy on Craigslist, I have loved her unconditionally.
I still remember driving down the winding country road to the seller's
sprawling ranch and instantly falling for her. The way that she
glistened in the sunlight beckoned me to her. I had no problem spending
the money for her that I had accumulated over years of saving birthday
gifts, doing undesirable odd jobs and babysitting unruly children. To
me, she was worth more than my entire bank account.

Big Betsy has been loyal to me throughout the past couple of years. She
even provided me with the opportunity to set up my own business, The
Westport Workers. My friend and I realized that all the dump-run
services in our town were grossly overcharging their customers, so we
decided to provide an inexpensive alternative. We have worked countless
jobs together, including transporting an antique bar counter 50 miles
away for a Gilmore Girls fan club meeting and hauling a battered boat
motor through knee-deep sludge to dispose of it at the dump.

Big Betsy and I are constantly relying on each other to get things done.
In the blistering summer heat she would wait patiently for me while I
pulled weeds for hours on end. With sweat trickling down my face, I
would take shelter from the sun in her soft embrace. She and I made a
respectable living through our business, and I would always make sure to
buy her the things that she required to keep her going.

In case it isn't obvious, Big Betsy is my beloved truck, a 1998 Ford
F-150 with over 230,000 miles. The first months I had her, I spent all
my time between early morning football and work fixing her up, and it
was worth it.

Not only has she been a great truck, she also helped me to realize how
little other people's judgments of me matter. I used to be shy and
avoided differentiating myself from my classmates because I was very
concerned about what others would think about me. In a school almost
entirely minority-free, I was always uncomfortable with my ethnicity,
and even my name. I felt extremely self-conscious every time that I
pulled into the high school parking lot filled with Mercedes, Jeep
Wranglers, and BMWs.

However, as time went on, Big Betsy became a bit of a local celebrity
and I became more confident, and not only while driving. I found myself
less anxious when voicing my opinions, applying for leadership
positions, and challenging myself to do better in all aspects of my
life. Big Betsy made me realize how damaging it can be to my potential
when I become unwilling to stand out or take the risks required to
achieve my goals. If it wasn't for her teaching me how to be confident
in myself and that it is good to be pushed out of my comfort zone, I
would not be nearly as happy as I am today.

\begin{center}\rule{0.5\linewidth}{\linethickness}\end{center}

\hypertarget{ashland-ore}{%
\subsubsection{Ashland, Ore.}\label{ashland-ore}}

Image

Kaya Cerecedes-Crosby is heading to Wellesley College.Credit...Chris
Pietsch for The New York Times

\hypertarget{mother-up-at-twilight-to-start-her-day-breath-released-in-freezing-clouds-as-she-milks-the-goats-and-feeds-the-chickens-never-disappointing-the-hungry-mouths-that-depend-on-her}{%
\subsection{`Mother up at twilight to start her day, breath released in
freezing clouds as she milks the goats and feeds the chickens, never
disappointing the hungry mouths that depend on
her.'}\label{mother-up-at-twilight-to-start-her-day-breath-released-in-freezing-clouds-as-she-milks-the-goats-and-feeds-the-chickens-never-disappointing-the-hungry-mouths-that-depend-on-her}}

\hypertarget{kaya-cerecedes-crosby}{%
\subsubsection{Kaya Cerecedes-Crosby}\label{kaya-cerecedes-crosby}}

\hypertarget{-3}{%
\subsubsection{***}\label{-3}}

Twist, bend, through the loop. Repeat.

It took me a month to crochet my first blanket. One month of twisting,
bending, sending my hook through the loop, and repeating. It was an
almost meditative pastime. I spent bus rides and evenings working on my
blanket, determined to finish.

I learned to crochet so that I could feel closer to my mother. I poured
my heart into every stitch. Each square of the blanket meant something
different; the colors represented memories. It was a summary of my life.

Green double treble crochet stitches take me back to the smell of wet
pine needles in the spring, laughter from my sisters climbing high on
tree limbs, the curve of mountain roads. Green is the forest of my
childhood, sheltering my first home. I taste the smoke from our old wood
stove and see the oil lanterns flickering in and out. The cabin in the
woods where my sister was born, water from the river that she took her
first bath in.

Green fades into blue as squares meet, treetops brush the sky. I see
myself, young and spinning across a playground with my classmates. I am
at my one-room schoolhouse, holding hands with the two other children in
my grade and lying with our backs on grass, looking up at the
never-ending sky. We whisper dreams of becoming doctors, actors,
artists.

I see the blue of California oceans as I leave for high school, finding
my home away from home. Pine trees replaced by palm trees and sand
between my toes. I recall beach cleanups and surfing trips, touching shy
sea anemones in tide pools. Blue paint on signs for women's marches and
the sound of people beside me who want to be heard. We demand equality.

Purple is for my mother. It's her favorite color. It reminds me of her
strength and determination. I feel her calloused hands from work on the
farm, work in the field, and chemical burns from cleaning jobs. I smell
her earthy clothes as she studies at the kitchen table, determined to
finish her homework so that she can finally graduate college after
decades of trying. I see the violet sky at dawn; when the sun rises so
does she. Mother up at twilight to start her day, breath released in
freezing clouds as she milks the goats and feeds the chickens, never
disappointing the hungry mouths that depend on her. Each day, I recall
the things she has given up for my sake. Her sacrifice and desire for me
to succeed encourage me to be better and work harder. Yet, I desire
more. I do not want to live like her, I want better.

Red stitches are passionate outbursts. Angry shouts from Dad as he
returns in the middle of the night, breath sour from drinking. Tears of
happiness after receiving his first chip for a year of sobriety. Screams
echoing from my biological father's mouth as he hurls threats that sting
like arrows as his disease makes him chase his family away. Scarlet
stitches of fear during our six months without a roof over our heads
after he forced us from our home. Pain in my sister's eyes after she
begged for help from friends with deaf ears. Promises that we will keep
her safe, and check-in calls after I leave home.

Twist, bend, through the loop. Repeat.

Each stitch is a part of me. I rarely relive these aspects of my
upbringing, but I call on them when I need to be reminded of my
strength. When I completed the blanket, I cried. I was proud. I made
this. This is me.

Advertisement

\protect\hyperlink{after-bottom}{Continue reading the main story}

\hypertarget{site-index}{%
\subsection{Site Index}\label{site-index}}

\hypertarget{site-information-navigation}{%
\subsection{Site Information
Navigation}\label{site-information-navigation}}

\begin{itemize}
\tightlist
\item
  \href{https://help.nytimes3xbfgragh.onion/hc/en-us/articles/115014792127-Copyright-notice}{©~2020~The
  New York Times Company}
\end{itemize}

\begin{itemize}
\tightlist
\item
  \href{https://www.nytco.com/}{NYTCo}
\item
  \href{https://help.nytimes3xbfgragh.onion/hc/en-us/articles/115015385887-Contact-Us}{Contact
  Us}
\item
  \href{https://www.nytco.com/careers/}{Work with us}
\item
  \href{https://nytmediakit.com/}{Advertise}
\item
  \href{http://www.tbrandstudio.com/}{T Brand Studio}
\item
  \href{https://www.nytimes3xbfgragh.onion/privacy/cookie-policy\#how-do-i-manage-trackers}{Your
  Ad Choices}
\item
  \href{https://www.nytimes3xbfgragh.onion/privacy}{Privacy}
\item
  \href{https://help.nytimes3xbfgragh.onion/hc/en-us/articles/115014893428-Terms-of-service}{Terms
  of Service}
\item
  \href{https://help.nytimes3xbfgragh.onion/hc/en-us/articles/115014893968-Terms-of-sale}{Terms
  of Sale}
\item
  \href{https://spiderbites.nytimes3xbfgragh.onion}{Site Map}
\item
  \href{https://help.nytimes3xbfgragh.onion/hc/en-us}{Help}
\item
  \href{https://www.nytimes3xbfgragh.onion/subscription?campaignId=37WXW}{Subscriptions}
\end{itemize}
