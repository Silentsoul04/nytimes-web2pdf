Sections

SEARCH

\protect\hyperlink{site-content}{Skip to
content}\protect\hyperlink{site-index}{Skip to site index}

\href{https://www.nytimes3xbfgragh.onion/section/health}{Health}

\href{https://myaccount.nytimes3xbfgragh.onion/auth/login?response_type=cookie\&client_id=vi}{}

\href{https://www.nytimes3xbfgragh.onion/section/todayspaper}{Today's
Paper}

\href{/section/health}{Health}\textbar{}What to Know About Colon Cancer

\url{https://nyti.ms/3lylgM6}

\begin{itemize}
\item
\item
\item
\item
\item
\item
\end{itemize}

\href{https://www.nytimes3xbfgragh.onion/spotlight/at-home?action=click\&pgtype=Article\&state=default\&region=TOP_BANNER\&context=at_home_menu}{At
Home}

\begin{itemize}
\tightlist
\item
  \href{https://www.nytimes3xbfgragh.onion/2020/09/07/travel/route-66.html?action=click\&pgtype=Article\&state=default\&region=TOP_BANNER\&context=at_home_menu}{Cruise
  Along: Route 66}
\item
  \href{https://www.nytimes3xbfgragh.onion/2020/09/04/dining/sheet-pan-chicken.html?action=click\&pgtype=Article\&state=default\&region=TOP_BANNER\&context=at_home_menu}{Roast:
  Chicken With Plums}
\item
  \href{https://www.nytimes3xbfgragh.onion/2020/09/04/arts/television/dark-shadows-stream.html?action=click\&pgtype=Article\&state=default\&region=TOP_BANNER\&context=at_home_menu}{Watch:
  Dark Shadows}
\item
  \href{https://www.nytimes3xbfgragh.onion/interactive/2020/at-home/even-more-reporters-editors-diaries-lists-recommendations.html?action=click\&pgtype=Article\&state=default\&region=TOP_BANNER\&context=at_home_menu}{Explore:
  Reporters' Google Docs}
\end{itemize}

Advertisement

\protect\hyperlink{after-top}{Continue reading the main story}

Supported by

\protect\hyperlink{after-sponsor}{Continue reading the main story}

\hypertarget{what-to-know-about-colon-cancer}{%
\section{What to Know About Colon
Cancer}\label{what-to-know-about-colon-cancer}}

The cancer that killed Chadwick Boseman is the second-leading cause of
cancer deaths in the United States, and rates are rising among younger
people.

\includegraphics{https://static01.graylady3jvrrxbe.onion/images/2020/08/29/science/29colon/29colon-articleLarge.jpg?quality=75\&auto=webp\&disable=upscale}

\href{https://www.nytimes3xbfgragh.onion/by/pam-belluck}{\includegraphics{https://static01.graylady3jvrrxbe.onion/images/2018/02/16/multimedia/author-pam-belluck/author-pam-belluck-thumbLarge-v2.png}}

By \href{https://www.nytimes3xbfgragh.onion/by/pam-belluck}{Pam Belluck}

\begin{itemize}
\item
  Aug. 29, 2020
\item
  \begin{itemize}
  \item
  \item
  \item
  \item
  \item
  \item
  \end{itemize}
\end{itemize}

\href{https://cn.nytimes3xbfgragh.onion/health/20200831/colon-cancer-chadwick-boseman/}{阅读简体中文版}\href{https://cn.nytimes3xbfgragh.onion/health/20200831/colon-cancer-chadwick-boseman/zh-hant/}{閱讀繁體中文版}

In the wake of
\href{https://www.nytimes3xbfgragh.onion/2020/08/28/movies/chadwick-boseman-dead.html}{Chadwick
Boseman's death} from colon cancer at age 43, many people have questions
about the disease, especially about the risk of colon cancer in younger
people. Here's what is known and what experts recommend.

\hypertarget{doesnt-colon-cancer-mostly-affect-older-people}{%
\subsection{Doesn't colon cancer mostly affect older
people?}\label{doesnt-colon-cancer-mostly-affect-older-people}}

Although the majority of cases are found in older people, there has been
an increase in cases in younger people in recent years.

Among people over 65, rates of colorectal cancer, which includes tumors
in the rectum or the colon, have actually been declining, probably
because of more regular screening. Nonetheless, it is the second-leading
cause of cancer deaths in the United States for men and women combined,
and cases have been rising by
\href{https://acsjournals.onlinelibrary.wiley.com/doi/full/10.3322/caac.21601}{about
2 percent annually in recent years in people under 50}, according to a
recent report by the American Cancer Society.

Experts aren't sure exactly why. For some patients, obesity, diabetes,
smoking or a family history of cancer may play a role, but not all
people who develop colorectal cancer have these risk factors.

``The bottom line is we just don't know,'' said Dr. Robin B. Mendelsohn,
co-director of the Center for Young Onset Colorectal Cancer at Memorial
Sloan Kettering Cancer Center, which was opened in 2018 to treat younger
patients and study the reasons for their diagnoses. She and her
colleagues are exploring whether diet, medications like antibiotics, and
the microbiome --- which have all changed significantly for generations
born in the 1960s and later --- might be contributing to the cancer in
younger people.

\hypertarget{when-is-screening-recommended}{%
\subsection{When is screening
recommended?}\label{when-is-screening-recommended}}

Everybody should begin getting screenings at age 45, the American Cancer
Society and other expert groups recommend. But people with a family
history of colon cancer should start getting tested at age 40, or at 10
years younger than the age at which their family member was diagnosed,
whichever is sooner.

Dr. Mendelsohn recommends early screening also for people with a history
of inflammatory bowel disease, like ulcerative colitis or Crohn's
disease, and for people who have previously received radiation in their
abdomen or pelvis.

Screenings can be done with various tests on stool samples or with
imaging-based tests like colonoscopies. The risks from these tests are
relatively small. The prep for a colonoscopy, drinking liquid to cleanse
the colon the day before, can be uncomfortable. But the advantage of a
colonoscopy is that if it reveals polyps that might be precancerous,
they can be removed during the test, said Dr. Mohamed E. Salem, an
associate professor of medicine at the Levine Cancer Institute at Atrium
Health in Charlotte, N.C.

``It makes a huge difference when you detect cancer early versus late,''
he said.

``The five-year survival rate for young people for early-stage disease
is 94 percent,'' said Rebecca L. Siegel, the scientific director of
surveillance research at the American Cancer Society. For people with
late stages of the disease, the survival rate can be as low as 20
percent, she said. Early diagnosis, Ms. Siegel said, is ``the difference
between life and death.''

Mr. Boseman learned in 2016 that he had Stage 3 colon cancer, according
to an Instagram post announcing his death. Dr. Mendelsohn said that
patients with Stage 3 ``have an approximate 60 percent to 80 percent
chance of cure,'' depending on a number of factors, including whether
the cancer is responsive to chemotherapy.

\hypertarget{are-there-racial-disparities-in-the-risk-of-colon-cancer}{%
\subsection{Are there racial disparities in the risk of colon
cancer?}\label{are-there-racial-disparities-in-the-risk-of-colon-cancer}}

Yes. According to the recent American Cancer Society report, rates of
colorectal cancer are higher among Black people. From 2012 to 2016, the
rate of new cases in non-Hispanic Black people was 45.7 per 100,000,
about 20 percent higher than the rate among non-Hispanic white people
and 50 percent higher than the rate among Asian-Americans and Pacific
Islanders. Alaska Natives had the highest rate: 89 per 100,000.

Ms. Siegel also said that at any age, ``African-Americans are 40 percent
more likely to die from colorectal cancer. It's because of later-stage
diagnosis, it's because of systemic racism and all that this population
has been dealing with for hundreds of years.''

\hypertarget{what-symptoms-should-prompt-someone-to-see-a-doctor-for-possible-colon-cancer}{%
\subsection{What symptoms should prompt someone to see a doctor for
possible colon
cancer?}\label{what-symptoms-should-prompt-someone-to-see-a-doctor-for-possible-colon-cancer}}

Common symptoms include bloody stool or bleeding from the rectum,
doctors say. Other symptoms can include constipation or diarrhea, a
change in bowel habits, dark sticky feces, feeling anemic, abdominal
pain or cramps, nausea, vomiting or unexplained weight loss.

``If you feel something, you have to say something,'' Dr. Salem said.
``Don't put it off because you're busy or because you're a young person
or because you have too much on your plate.''

\hypertarget{are-younger-people-less-likely-to-receive-a-diagnosis-early-and-are-they-less-likely-to-discuss-their-disease}{%
\subsection{Are younger people less likely to receive a diagnosis early,
and are they less likely to discuss their
disease?}\label{are-younger-people-less-likely-to-receive-a-diagnosis-early-and-are-they-less-likely-to-discuss-their-disease}}

Unfortunately, yes. The average time from symptoms to diagnosis for
people under 50 is 271 days, Dr. Siegel said, compared with 29 days for
people 50 and older.

``Both doctors and these young folks are not thinking they have
cancer,'' she said. ``Part of that is screening, but it's not all
screening. Young patients have symptoms, sometimes for years. For one
thing, they're much less likely to have health insurance than older
people, and so they have less money. And they're thinking, `I'm a
30-year-old, what could be wrong with me --- it's going to go away.'''

Also, she said, ``There's the embarrassment factor. `I'm bleeding from
the rectum.'''

Dr. Salem said that ``there is a lot of shame somehow. Nobody likes to
have bleeding, especially from their butt. Especially young people; they
don't like to discuss this or disclose this information. That's
understandable. But it's our obligation to change that culture. It's OK
to talk about your pain in that area, or your bleeding, or your
constipation, or your diarrhea.''

Doctors also need to get better at flagging a younger person's symptoms
as possible colorectal cancer, experts said.

``Anytime patients are 75 years old and have rectal bleeding, we say
`Make sure and get checked out for colon cancer,''' Dr. Salem said.
``When younger people have rectal bleeding, sometimes we say `Oh, that's
hemorrhoids or stress from working too much.' Those symptoms go on for
many, many months or years, and now it's not Stage 1 anymore, it's Stage
3 or 4.''

Once they receive a diagnosis, doctors said, younger people should not
feel ashamed.

``Increasing awareness and reducing stigma, all of this information
could be saving lives now,'' Dr. Siegel said. ``Keeping a secret is not
the way to go.''

\textbf{\emph{{[}}\href{http://on.fb.me/1paTQ1h}{\emph{Like the Science
Times page on Facebook.}}} ****** \emph{\textbar{} Sign up for the}
\textbf{\href{http://nyti.ms/1MbHaRU}{\emph{Science Times
newsletter.}}\emph{{]}}}

Advertisement

\protect\hyperlink{after-bottom}{Continue reading the main story}

\hypertarget{site-index}{%
\subsection{Site Index}\label{site-index}}

\hypertarget{site-information-navigation}{%
\subsection{Site Information
Navigation}\label{site-information-navigation}}

\begin{itemize}
\tightlist
\item
  \href{https://help.nytimes3xbfgragh.onion/hc/en-us/articles/115014792127-Copyright-notice}{©~2020~The
  New York Times Company}
\end{itemize}

\begin{itemize}
\tightlist
\item
  \href{https://www.nytco.com/}{NYTCo}
\item
  \href{https://help.nytimes3xbfgragh.onion/hc/en-us/articles/115015385887-Contact-Us}{Contact
  Us}
\item
  \href{https://www.nytco.com/careers/}{Work with us}
\item
  \href{https://nytmediakit.com/}{Advertise}
\item
  \href{http://www.tbrandstudio.com/}{T Brand Studio}
\item
  \href{https://www.nytimes3xbfgragh.onion/privacy/cookie-policy\#how-do-i-manage-trackers}{Your
  Ad Choices}
\item
  \href{https://www.nytimes3xbfgragh.onion/privacy}{Privacy}
\item
  \href{https://help.nytimes3xbfgragh.onion/hc/en-us/articles/115014893428-Terms-of-service}{Terms
  of Service}
\item
  \href{https://help.nytimes3xbfgragh.onion/hc/en-us/articles/115014893968-Terms-of-sale}{Terms
  of Sale}
\item
  \href{https://spiderbites.nytimes3xbfgragh.onion}{Site Map}
\item
  \href{https://help.nytimes3xbfgragh.onion/hc/en-us}{Help}
\item
  \href{https://www.nytimes3xbfgragh.onion/subscription?campaignId=37WXW}{Subscriptions}
\end{itemize}
