Sections

SEARCH

\protect\hyperlink{site-content}{Skip to
content}\protect\hyperlink{site-index}{Skip to site index}

\href{https://www.nytimes3xbfgragh.onion/section/health}{Health}

\href{https://myaccount.nytimes3xbfgragh.onion/auth/login?response_type=cookie\&client_id=vi}{}

\href{https://www.nytimes3xbfgragh.onion/section/todayspaper}{Today's
Paper}

\href{/section/health}{Health}\textbar{}Your Coronavirus Test Is
Positive. Maybe It Shouldn't Be.

\url{https://nyti.ms/31DJixp}

\begin{itemize}
\item
\item
\item
\item
\item
\end{itemize}

\hypertarget{the-coronavirus-outbreak}{%
\subsubsection{\texorpdfstring{\href{https://www.nytimes3xbfgragh.onion/news-event/coronavirus?name=styln-coronavirus-national\&region=TOP_BANNER\&block=storyline_menu_recirc\&action=click\&pgtype=Article\&impression_id=5b626190-f4ba-11ea-8f75-57b78e238649\&variant=undefined}{The
Coronavirus
Outbreak}}{The Coronavirus Outbreak}}\label{the-coronavirus-outbreak}}

\begin{itemize}
\tightlist
\item
  live\href{https://www.nytimes3xbfgragh.onion/2020/09/11/world/covid-19-coronavirus.html?name=styln-coronavirus-national\&region=TOP_BANNER\&block=storyline_menu_recirc\&action=click\&pgtype=Article\&impression_id=5b6288a0-f4ba-11ea-8f75-57b78e238649\&variant=undefined}{Latest
  Updates}
\item
  \href{https://www.nytimes3xbfgragh.onion/interactive/2020/us/coronavirus-us-cases.html?name=styln-coronavirus-national\&region=TOP_BANNER\&block=storyline_menu_recirc\&action=click\&pgtype=Article\&impression_id=5b6288a1-f4ba-11ea-8f75-57b78e238649\&variant=undefined}{Maps
  and Cases}
\item
  \href{https://www.nytimes3xbfgragh.onion/interactive/2020/science/coronavirus-vaccine-tracker.html?name=styln-coronavirus-national\&region=TOP_BANNER\&block=storyline_menu_recirc\&action=click\&pgtype=Article\&impression_id=5b6288a2-f4ba-11ea-8f75-57b78e238649\&variant=undefined}{Vaccine
  Tracker}
\item
  \href{https://www.nytimes3xbfgragh.onion/2020/09/10/us/politics/fda-coronavirus-vaccine.html?name=styln-coronavirus-national\&region=TOP_BANNER\&block=storyline_menu_recirc\&action=click\&pgtype=Article\&impression_id=5b6288a3-f4ba-11ea-8f75-57b78e238649\&variant=undefined}{F.D.A.
  Regulators' Self-Defense}
\item
  \href{https://www.nytimes3xbfgragh.onion/2020/09/09/upshot/coronavirus-surprise-test-fees.html?name=styln-coronavirus-national\&region=TOP_BANNER\&block=storyline_menu_recirc\&action=click\&pgtype=Article\&impression_id=5b6288a4-f4ba-11ea-8f75-57b78e238649\&variant=undefined}{Surprise
  Test Fees}
\end{itemize}

Advertisement

\protect\hyperlink{after-top}{Continue reading the main story}

Supported by

\protect\hyperlink{after-sponsor}{Continue reading the main story}

\hypertarget{your-coronavirus-test-is-positive-maybe-it-shouldnt-be}{%
\section{Your Coronavirus Test Is Positive. Maybe It Shouldn't
Be.}\label{your-coronavirus-test-is-positive-maybe-it-shouldnt-be}}

The usual diagnostic tests may simply be too sensitive and too slow to
contain the spread of the virus.

\includegraphics{https://static01.graylady3jvrrxbe.onion/images/2020/08/30/science/30virus-test-print/merlin_170363544_1c6703b0-a265-441d-9277-5431b5236763-articleLarge.jpg?quality=75\&auto=webp\&disable=upscale}

By
\href{https://www.nytimes3xbfgragh.onion/by/apoorva-mandavilli}{Apoorva
Mandavilli}

\begin{itemize}
\item
  Published Aug. 29, 2020Updated Sept. 9, 2020
\item
  \begin{itemize}
  \item
  \item
  \item
  \item
  \item
  \end{itemize}
\end{itemize}

Some of the nation's leading public health experts are raising a new
concern in the endless debate over
\href{https://www.nytimes3xbfgragh.onion/2020/09/09/upshot/coronavirus-surprise-test-fees.html}{coronavirus
testing} in the United States: The standard tests are diagnosing huge
numbers of people who may be carrying relatively insignificant amounts
of the virus.

Most of these people are not likely to be contagious, and identifying
them may contribute to bottlenecks that prevent those who are contagious
from being found in time. But researchers say the solution is not to
test less, or to skip testing people without symptoms, as recently
\href{https://www.nytimes3xbfgragh.onion/2020/08/27/us/politics/trump-coronavirus-testing.html}{suggested
by the Centers for Disease Control and Prevention}.

Instead, new data underscore
\href{https://www.nytimes3xbfgragh.onion/2020/08/06/health/rapid-Covid-tests.html}{the
need for more widespread use of rapid tests}, even if they are less
sensitive.

``The decision not to test asymptomatic people is just really
backward,'' said Dr. Michael Mina, an epidemiologist at the Harvard T.H.
Chan School of Public Health, referring to the C.D.C. recommendation.

``In fact, we should be ramping up testing of all different people,'' he
said, ``but we have to do it through whole different mechanisms.''

In what may be a step in this direction, the Trump administration
announced on Thursday that it would purchase 150 million rapid tests.

The most widely used diagnostic test for the new coronavirus, called a
PCR test, provides a simple yes-no answer to the question of whether a
patient is infected.

But similar PCR tests for other viruses do offer some sense of how
contagious an infected patient may be: The results may include a rough
estimate of the amount of virus in the patient's body.

``We've been using one type of data for everything, and that is just
plus or minus --- that's all,'' Dr. Mina said. ``We're using that for
clinical diagnostics, for public health, for policy decision-making.''

But yes-no isn't good enough, he added. It's the amount of virus that
should dictate the infected patient's next steps. ``It's really
irresponsible, I think, to forgo the recognition that this is a
quantitative issue,'' Dr. Mina said.

\hypertarget{latest-updates-the-coronavirus-outbreak}{%
\section{\texorpdfstring{\href{https://www.nytimes3xbfgragh.onion/2020/09/11/world/covid-19-coronavirus.html?action=click\&pgtype=Article\&state=default\&region=MAIN_CONTENT_1\&context=storylines_live_updates}{Latest
Updates: The Coronavirus
Outbreak}}{Latest Updates: The Coronavirus Outbreak}}\label{latest-updates-the-coronavirus-outbreak}}

Updated 2020-09-12T05:29:13.829Z

\begin{itemize}
\tightlist
\item
  \href{https://www.nytimes3xbfgragh.onion/2020/09/11/world/covid-19-coronavirus.html?action=click\&pgtype=Article\&state=default\&region=MAIN_CONTENT_1\&context=storylines_live_updates\#link-dfb8a16}{Fauci
  cautions the virus could disrupt life in the U.S. until `maybe even
  towards the end of 2021.'}
\item
  \href{https://www.nytimes3xbfgragh.onion/2020/09/11/world/covid-19-coronavirus.html?action=click\&pgtype=Article\&state=default\&region=MAIN_CONTENT_1\&context=storylines_live_updates\#link-7104d154}{From
  Asia to Africa, China promotes its vaccine candidates to win friends.}
\item
  \href{https://www.nytimes3xbfgragh.onion/2020/09/11/world/covid-19-coronavirus.html?action=click\&pgtype=Article\&state=default\&region=MAIN_CONTENT_1\&context=storylines_live_updates\#link-393ad215}{The
  other way the virus will kill: hunger.}
\end{itemize}

\href{https://www.nytimes3xbfgragh.onion/2020/09/11/world/covid-19-coronavirus.html?action=click\&pgtype=Article\&state=default\&region=MAIN_CONTENT_1\&context=storylines_live_updates}{See
more updates}

More live coverage:
\href{https://www.nytimes3xbfgragh.onion/live/2020/09/11/business/stock-market-today-coronavirus?action=click\&pgtype=Article\&state=default\&region=MAIN_CONTENT_1\&context=storylines_live_updates}{Markets}

The PCR test amplifies genetic matter from the virus in cycles; the
fewer cycles required, the greater the amount of virus, or viral load,
in the sample. The greater the viral load, the more likely the patient
is to be contagious.

This number of amplification cycles needed to find the virus, called the
cycle threshold, is never included in the results sent to doctors and
coronavirus patients, although it could tell them how infectious the
patients are.

In three sets of testing data that include cycle thresholds, compiled by
officials in Massachusetts, New York and Nevada, up to 90 percent of
people testing positive carried barely any virus, a review by The Times
found.

On Thursday, the United States recorded 45,604 new coronavirus cases,
according to a database maintained by The Times. If the rates of
contagiousness in Massachusetts and New York were to apply nationwide,
then perhaps only 4,500 of those people may actually need to isolate and
submit to contact tracing.

One solution would be to adjust the cycle threshold used now to decide
that a patient is infected. Most tests set the limit at 40, a few at 37.
This means that you are positive for the coronavirus if the test process
required up to 40 cycles, or 37, to detect the virus.

Tests with thresholds so high may detect not just live virus but also
genetic fragments, leftovers from infection that pose no particular risk
--- akin to finding a hair in a room long after a person has left, Dr.
Mina said.

Any test with a cycle threshold above 35 is too sensitive, agreed Juliet
Morrison, a virologist at the University of California, Riverside. ``I'm
shocked that people would think that 40 could represent a positive,''
she said.

A more reasonable cutoff would be 30 to 35, she added. Dr. Mina said he
would set the figure at 30, or even less. Those changes would mean the
amount of genetic material in a patient's sample would have to be
100-fold to 1,000-fold that of the current standard for the test to
return a positive result --- at least, one worth acting on.

\includegraphics{https://static01.graylady3jvrrxbe.onion/images/2020/09/22/science/22VIRUS-TEST2/merlin_171395976_32309997-11ea-4f14-95c6-0c79da5c12fd-articleLarge.jpg?quality=75\&auto=webp\&disable=upscale}

The Food and Drug Administration said in an emailed statement that it
does not specify the cycle threshold ranges used to determine who is
positive, and that
``\href{https://www.fda.gov/media/135900/download}{commercial
manufacturers} and
\href{https://www.fda.gov/media/135658/download}{laboratories} set their
own.''

The Centers for Disease Control and Prevention said it is examining the
use of cycle threshold measures ``for policy decisions.'' The agency
said it would need to collaborate with the F.D.A. and with device
manufacturers to ensure the measures ``can be used properly and with
assurance that we know what they mean.''

The C.D.C.'s own calculations suggest that it is extremely difficult to
detect any live virus in a sample
\href{https://www.cdc.gov/coronavirus/2019-ncov/hcp/duration-isolation.html}{above
a threshold of 33 cycles}. Officials at some state labs said the C.D.C.
had not asked them to note threshold values or to share them with
contact-tracing organizations.

For example, North Carolina's state lab uses the Thermo Fisher
coronavirus test, which automatically classifies results based on a
cutoff of 37 cycles. A spokeswoman for the lab said testers did not have
access to the precise numbers.

This amounts to an enormous missed opportunity to learn more about the
disease, some experts said.

``It's just kind of mind-blowing to me that people are not recording the
C.T. values from all these tests --- that they're just returning a
positive or a negative,'' said Angela Rasmussen, a virologist at
Columbia University in New York.

``It would be useful information to know if somebody's positive, whether
they have a high viral load or a low viral load,'' she added.

Officials at the Wadsworth Center, New York's state lab, have access to
C.T. values from tests they have processed, and analyzed their numbers
at The Times's request. In July, the lab identified 872 positive tests,
based on a threshold of 40 cycles.

With a cutoff of 35, about 43 percent of those tests would no longer
qualify as positive. About 63 percent would no longer be judged positive
if the cycles were limited to 30.

In Massachusetts, from 85 to 90 percent of people who tested positive in
July with a cycle threshold of 40 would have been deemed negative if the
threshold were 30 cycles, Dr. Mina said. ``I would say that none of
those people should be contact-traced, not one,'' he said.

Other experts informed of these numbers were stunned.

``I'm really shocked that it could be that high --- the proportion of
people with high C.T. value results,'' said Dr. Ashish Jha, director of
the Harvard Global Health Institute. ``Boy, does it really change the
way we need to be thinking about testing.''

Dr. Jha said he had thought of the PCR test as a problem because it
cannot scale to the volume, frequency or speed of tests needed. ``But
what I am realizing is that a really substantial part of the problem is
that we're not even testing the people who we need to be testing,'' he
said.

The number of people with positive results who aren't infectious is
particularly concerning, said Scott Becker, executive director of the
Association of Public Health Laboratories. ``That worries me a lot, just
because it's so high,'' he said, adding that the organization intended
to meet with Dr. Mina to discuss the issue.

The F.D.A. noted that people may have a low viral load when they are
newly infected. A test with less sensitivity would miss these
infections.

But that problem is easily solved, Dr. Mina said: ``Test them again, six
hours later or 15 hours later or whatever,'' he said. A rapid test would
find these patients quickly, even if it were less sensitive, because
their viral loads would quickly rise.

PCR tests still have a role, he and other experts said. For example,
their sensitivity is an asset when identifying newly infected people to
enroll in clinical trials of drugs.

But with 20 percent or more of people testing positive for the virus in
some parts of the country, Dr. Mina and other researchers are
questioning the use of PCR tests as a frontline diagnostic tool.

People infected with the virus are most infectious from a day or two
before symptoms appear till about five days after. But at the current
testing rates, ``you're not going to be doing it frequently enough to
have any chance of really capturing somebody in that window,'' Dr. Mina
added.

Highly sensitive PCR tests seemed like the best option for tracking the
coronavirus at the start of the pandemic. But for the outbreaks raging
now, he said, what's needed are coronavirus tests that are fast, cheap
and abundant enough to frequently test everyone who needs it --- even if
the tests are less sensitive.

``It might not catch every last one of the transmitting people, but it
sure will catch the most transmissible people, including the
superspreaders,'' Dr. Mina said. ``That alone would drive epidemics
practically to zero.''

Advertisement

\protect\hyperlink{after-bottom}{Continue reading the main story}

\hypertarget{site-index}{%
\subsection{Site Index}\label{site-index}}

\hypertarget{site-information-navigation}{%
\subsection{Site Information
Navigation}\label{site-information-navigation}}

\begin{itemize}
\tightlist
\item
  \href{https://help.nytimes3xbfgragh.onion/hc/en-us/articles/115014792127-Copyright-notice}{©~2020~The
  New York Times Company}
\end{itemize}

\begin{itemize}
\tightlist
\item
  \href{https://www.nytco.com/}{NYTCo}
\item
  \href{https://help.nytimes3xbfgragh.onion/hc/en-us/articles/115015385887-Contact-Us}{Contact
  Us}
\item
  \href{https://www.nytco.com/careers/}{Work with us}
\item
  \href{https://nytmediakit.com/}{Advertise}
\item
  \href{http://www.tbrandstudio.com/}{T Brand Studio}
\item
  \href{https://www.nytimes3xbfgragh.onion/privacy/cookie-policy\#how-do-i-manage-trackers}{Your
  Ad Choices}
\item
  \href{https://www.nytimes3xbfgragh.onion/privacy}{Privacy}
\item
  \href{https://help.nytimes3xbfgragh.onion/hc/en-us/articles/115014893428-Terms-of-service}{Terms
  of Service}
\item
  \href{https://help.nytimes3xbfgragh.onion/hc/en-us/articles/115014893968-Terms-of-sale}{Terms
  of Sale}
\item
  \href{https://spiderbites.nytimes3xbfgragh.onion}{Site Map}
\item
  \href{https://help.nytimes3xbfgragh.onion/hc/en-us}{Help}
\item
  \href{https://www.nytimes3xbfgragh.onion/subscription?campaignId=37WXW}{Subscriptions}
\end{itemize}
