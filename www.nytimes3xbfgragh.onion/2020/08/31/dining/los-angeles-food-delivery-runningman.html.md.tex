Sections

SEARCH

\protect\hyperlink{site-content}{Skip to
content}\protect\hyperlink{site-index}{Skip to site index}

\href{https://www.nytimes3xbfgragh.onion/section/food}{Food}

\href{https://myaccount.nytimes3xbfgragh.onion/auth/login?response_type=cookie\&client_id=vi}{}

\href{https://www.nytimes3xbfgragh.onion/section/todayspaper}{Today's
Paper}

\href{/section/food}{Food}\textbar{}A Local Alternative to the Big
Delivery Apps, in Los Angeles

\url{https://nyti.ms/2YPkGA5}

\begin{itemize}
\item
\item
\item
\item
\item
\item
\end{itemize}

\href{https://www.nytimes3xbfgragh.onion/spotlight/at-home?action=click\&pgtype=Article\&state=default\&region=TOP_BANNER\&context=at_home_menu}{At
Home}

\begin{itemize}
\tightlist
\item
  \href{https://www.nytimes3xbfgragh.onion/2020/09/07/travel/route-66.html?action=click\&pgtype=Article\&state=default\&region=TOP_BANNER\&context=at_home_menu}{Cruise
  Along: Route 66}
\item
  \href{https://www.nytimes3xbfgragh.onion/2020/09/04/dining/sheet-pan-chicken.html?action=click\&pgtype=Article\&state=default\&region=TOP_BANNER\&context=at_home_menu}{Roast:
  Chicken With Plums}
\item
  \href{https://www.nytimes3xbfgragh.onion/2020/09/04/arts/television/dark-shadows-stream.html?action=click\&pgtype=Article\&state=default\&region=TOP_BANNER\&context=at_home_menu}{Watch:
  Dark Shadows}
\item
  \href{https://www.nytimes3xbfgragh.onion/interactive/2020/at-home/even-more-reporters-editors-diaries-lists-recommendations.html?action=click\&pgtype=Article\&state=default\&region=TOP_BANNER\&context=at_home_menu}{Explore:
  Reporters' Google Docs}
\end{itemize}

Advertisement

\protect\hyperlink{after-top}{Continue reading the main story}

Supported by

\protect\hyperlink{after-sponsor}{Continue reading the main story}

critic's notebook

\hypertarget{a-local-alternative-to-the-big-delivery-apps-in-los-angeles}{%
\section{A Local Alternative to the Big Delivery Apps, in Los
Angeles}\label{a-local-alternative-to-the-big-delivery-apps-in-los-angeles}}

Modeled after food delivery services in Seoul, a tiny Koreatown business
keeps neighborhood restaurants running through the pandemic.

\includegraphics{https://static01.graylady3jvrrxbe.onion/images/2020/09/25/dining/25koreandelivery5/25koreandelivery5-articleLarge.jpg?quality=75\&auto=webp\&disable=upscale}

By \href{https://www.nytimes3xbfgragh.onion/by/tejal-rao}{Tejal Rao}

\begin{itemize}
\item
  Aug. 31, 2020
\item
  \begin{itemize}
  \item
  \item
  \item
  \item
  \item
  \item
  \end{itemize}
\end{itemize}

LOS ANGELES --- The busiest time of the day for Vivian Jung is that blur
between 11 a.m. and 1 p.m., when the lunch orders just won't stop.

Paper boxes of craggy fried chicken. Baby octopus rice bowls. Spam
musubi and grilled mandu and creamy seafood spaghetti. Hot samgyetang
steaming up the lid, the chicken practically glowing in a
ginseng-charged broth.

Ms. Jung, 35, is the manager of
\href{https://www.facebookcorewwwi.onion/RUNNINGMAN213/}{Runningman}, a
three-year-old food-delivery business with headquarters at Mariposa
Avenue and Eighth Street in Koreatown. With a team of two receptionists
and 30 drivers, she coordinates pickups and drop offs with more than 100
restaurants in a two-mile radius, switching constantly between Korean
and English, sending the orders zigzagging across town.

\includegraphics{https://static01.graylady3jvrrxbe.onion/images/2020/09/25/dining/25koreandelivery4/25koreandelivery4-articleLarge.jpg?quality=75\&auto=webp\&disable=upscale}

Compared with major app-based delivery companies --- DoorDash, Grubhub,
Postmates and Uber Eats --- that compete to control the national
delivery market, Runningman is absolutely tiny. But since indoor dining
rooms have been identified as
\href{https://www.nytimes3xbfgragh.onion/2020/08/12/health/Covid-restaurants-bars.html}{hot
spots for coronavirus infection}, and restaurants increasingly rely on
delivery, this small business attuned to local needs has become vital to
the neighborhood.

Ms. Jung, who moved to the United States in 1997 from Seoul, South
Korea, used to run her own restaurant in Koreatown, where she served
little snacks like tteokbokki, the spicy, chewy rice cakes. She worked
with a number of delivery apps at the time, and had the same grievance
as many restaurant owners: The 20- to 30-percent commission on every
order was a punishing revenue loss.

It was also standard. To compete with one another's frequent deals and
discounts, most delivery apps charge diners a minimal fee to use the
service. But they charge restaurants a high commission on the food,
usually in addition to a delivery fee, a marketing fee and other costs.

Since March, as delivery and takeout have grown to represent the bulk of
dining revenue, these fees have become unmanageable for many
restaurants. Some owners learned that they are an even
\href{https://www.nytimes3xbfgragh.onion/2020/06/09/technology/delivery-apps-restaurants-fees-virus.html}{bigger
expense than the costs of food or labor}, and are considering cutting
more staff to stay afloat, or closing for good.

Image

Jeremy Cho, a driver, picking up food from Buil Samgye Tang, in
Koreatown.Credit...Rozette Rago for The New York Times

A few major cities, including Los Angeles and New York, instituted caps
to temporarily restrict the fees to 15 percent. And though apps have
since positioned themselves in their own marketing materials as being in
solidarity with restaurants, they are widely considered shortsighted and
flat-out predatory within the industry.

In April, the New Yorker writer Helen Rosner suggested that diners
should
\href{https://www.newyorker.com/culture/annals-of-gastronomy/pick-up-the-damn-phone-and-other-thoughts-on-ordering-restaurant-delivery}{pick
up the phone} and call in their dinner orders directly. The same month,
Khushbu Shah, the restaurant editor at Food \& Wine magazine, urged
diners to go ahead and
\href{https://www.foodandwine.com/fwpro/delete-your-delivery-apps}{delete
all their delivery apps entirely}.

But they haven't gone away. Maybe because the apps are tantalizingly
efficient --- or at least maintain an illusion of efficiency. In fact,
users and workers are frequently disgruntled; menus, hours and pricing
are often incorrect. About half the time I've ordered through an app,
I've found dishes missing, or notes on orders ignored, and wondered if
the whole thing was worth it.

As a restaurant owner, Ms. Jung didn't think it was. She joined Jacob
Nam, who founded Runningman, inspired by the delivery services back in
Seoul, many of which charged a fee according to the distance the food
traveled to the diner, rather than a commission on the food itself. It
seemed like a more sustainable model to them, and one that the
mom-and-pop, immigrant-owned restaurants of Koreatown needed.

Mr. Nam partnered with a well-established Korean delivery business ---
Hello World --- to use its tech in Los Angeles.

Image

At lunchtime in the summer, the ginseng-chicken soup samgyetang is a
popular order.Credit...Rozette Rago for The New York Times

Runningman doesn't charge a commission on food, or a marketing fee. But
Ms. Jung said it has seen a 20 percent increase in business since March,
enough success to plan a second location nearby in Buena Park, Calif.,
in Orange County's Koreatown.

Restaurants that weren't used to delivery had to adapt, and quickly
---~how does food meant to be cooked at the table, for a group, travel?
What containers work best for fried foods, and for noodles, so they
don't get soggy on the way? Ms. Jung noted that many of the restaurants
she works with have rethought packaging, and invested in heftier to-go
boxes.

``So many restaurants that didn't even used to do delivery, they do it
now --- everything's changed,'' Ms. Jung said. Though many still work
with the major services, they rely on Runningman to deliver to regulars,
and those who live in the neighborhood.

An app connects Runningman's delivery drivers with restaurant kitchens,
mapping out the routes, but Ms. Jung says many owners in Koreatown still
call right at the moment they need a pickup, to share the address where
the food needs to go on the phone, the old-fashioned way.

``I can get a driver there in two minutes,'' Ms. Jung said, ``maybe
three.''

Image

The tiny delivery company was founded three years ago, and business has
picked up by 20 percent since the onset of the pandemic.Credit...Rozette
Rago for The New York Times

Drivers carry copies of current menus in Korean, with some English
translation, printed together in a dreamy catalog of dishes --- another
popular marketing tool taken from delivery companies in Seoul --- and
pass them out on their routes. Some restaurants keep stacks of these
catalogs by their doors, hoping customers will grab one on the way out.

The most recent issue, ``Just Do Eat, Volume 11,'' runs about 45 pages,
and includes menus from the Korean fusion restaurant
\href{http://www.kongjinedonkatsu.com/}{Kong Ji Ne Donkatsu} and
\href{http://www.theyellowhousecafe.com/}{Yellow House Cafe}. I could
pore over the choices for hours, thinking about what to get for lunch,
and I did. But Ms. Jung's core clientele doesn't even need to look.

``They already know exactly what they want,'' she said. ``They order so
regularly.''

\emph{Follow} \href{https://twitter.com/nytfood}{\emph{NYT Food on
Twitter}} \emph{and}
\href{https://www.instagram.com/nytcooking/}{\emph{NYT Cooking on
Instagram}}\emph{,}
\href{https://www.facebookcorewwwi.onion/nytcooking/}{\emph{Facebook}}\emph{,}
\href{https://www.youtube.com/nytcooking}{\emph{YouTube}} \emph{and}
\href{https://www.pinterest.com/nytcooking/}{\emph{Pinterest}}\emph{.}
\href{https://www.nytimes3xbfgragh.onion/newsletters/cooking}{\emph{Get
regular updates from NYT Cooking, with recipe suggestions, cooking tips
and shopping advice}}\emph{.}

Advertisement

\protect\hyperlink{after-bottom}{Continue reading the main story}

\hypertarget{site-index}{%
\subsection{Site Index}\label{site-index}}

\hypertarget{site-information-navigation}{%
\subsection{Site Information
Navigation}\label{site-information-navigation}}

\begin{itemize}
\tightlist
\item
  \href{https://help.nytimes3xbfgragh.onion/hc/en-us/articles/115014792127-Copyright-notice}{©~2020~The
  New York Times Company}
\end{itemize}

\begin{itemize}
\tightlist
\item
  \href{https://www.nytco.com/}{NYTCo}
\item
  \href{https://help.nytimes3xbfgragh.onion/hc/en-us/articles/115015385887-Contact-Us}{Contact
  Us}
\item
  \href{https://www.nytco.com/careers/}{Work with us}
\item
  \href{https://nytmediakit.com/}{Advertise}
\item
  \href{http://www.tbrandstudio.com/}{T Brand Studio}
\item
  \href{https://www.nytimes3xbfgragh.onion/privacy/cookie-policy\#how-do-i-manage-trackers}{Your
  Ad Choices}
\item
  \href{https://www.nytimes3xbfgragh.onion/privacy}{Privacy}
\item
  \href{https://help.nytimes3xbfgragh.onion/hc/en-us/articles/115014893428-Terms-of-service}{Terms
  of Service}
\item
  \href{https://help.nytimes3xbfgragh.onion/hc/en-us/articles/115014893968-Terms-of-sale}{Terms
  of Sale}
\item
  \href{https://spiderbites.nytimes3xbfgragh.onion}{Site Map}
\item
  \href{https://help.nytimes3xbfgragh.onion/hc/en-us}{Help}
\item
  \href{https://www.nytimes3xbfgragh.onion/subscription?campaignId=37WXW}{Subscriptions}
\end{itemize}
