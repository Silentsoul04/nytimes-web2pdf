Sections

SEARCH

\protect\hyperlink{site-content}{Skip to
content}\protect\hyperlink{site-index}{Skip to site index}

\href{https://www.nytimes3xbfgragh.onion/section/reader-center}{Times
Insider}

\href{https://myaccount.nytimes3xbfgragh.onion/auth/login?response_type=cookie\&client_id=vi}{}

\href{https://www.nytimes3xbfgragh.onion/section/todayspaper}{Today's
Paper}

\href{/section/reader-center}{Times Insider}\textbar{}What We Learned
From the D.N.C.

\url{https://nyti.ms/32JEypz}

\begin{itemize}
\item
\item
\item
\item
\item
\item
\end{itemize}

Advertisement

\protect\hyperlink{after-top}{Continue reading the main story}

Supported by

\protect\hyperlink{after-sponsor}{Continue reading the main story}

Times Insider

\hypertarget{what-we-learned-from-the-dnc}{%
\section{What We Learned From the
D.N.C.}\label{what-we-learned-from-the-dnc}}

After the convention ended, a panel of Times political journalists who
analyzed the Democrats' big event discussed the ticket, Joe Biden's
speech and the party's priorities.

\includegraphics{https://static01.graylady3jvrrxbe.onion/images/2020/08/31/us/politics/20dems-ledeall-top1/20dems-ledeall-top1-articleLarge-v4.jpg?quality=75\&auto=webp\&disable=upscale}

By The New York Times

\begin{itemize}
\item
  Aug. 31, 2020
\item
  \begin{itemize}
  \item
  \item
  \item
  \item
  \item
  \item
  \end{itemize}
\end{itemize}

\href{https://www.nytimes3xbfgragh.onion/series/times-insider}{\emph{Times
Insider}} \emph{explains who we are and what we do, and delivers
behind-the-scenes insights into how our journalism comes together.}

After both the Democratic National Convention and the Republican
National Convention, Rachel Dry, deputy politics editor, held live panel
discussions with Times political reporters as part of The Times's
Election 2020 event series. Here are edited excerpts from the
conversation about the D.N.C., featuring Astead W. Herndon, Katie Glueck
and Matt Flegenheimer. The discussion about the R.N.C. will run on
Tuesday.

\textbf{It seemed like the Democratic National Convention tried to reach
a lot of different demographics. What kind of party invites both John
Kasich {[}the Republican former governor of Ohio{]} and Riley Curry
{[}the 8-year-old daughter of N.B.A. star Stephen Curry{]}?}

\textbf{ASTEAD W. HERNDON} This convention and the Biden campaign have
basically made the choice that they could be for everyone, that they
don't have to pick a segment of the population that they're going to
zero in on. That's partly because they have a candidate who can appeal
to different groups through the personal or the political.

I think that so much of the primary was casting Joe Biden as this strict
ideological moderate, and that's not necessarily how the electorate sees
him. They see him as a nice guy. They see him as someone who they know
has served publicly and is a kind of loyal servant of the Democratic
Party. That can be bridged in many different directions.

That's what saw from this convention. You can have John Kasich and
Bernie Sanders on the same night, and they don't see a dissonance there.
I think what unity really means to the Democratic Party right now in
this campaign is not necessarily about merging people on a shared vision
of how to move the country forward, but a shared agreement that to move
the country forward, you have to remove Donald Trump. And they're
basically asking people to prioritize that view of electability over
their individual policy concerns. It is a campaign strategy. It is not a
governing strategy. That could set up tensions if Biden were to win.

\textbf{How are progressives viewing Kamala Harris as a running mate?}

\textbf{HERNDON} The progressive wing of the party, who you would define
as supporting more Sanders or Elizabeth Warren in the primary, is
basically counting its wins from down ballot races and has, frankly,
looked past November.

They are focusing their efforts on policy concerns because they think
this is a fluid ticket, that Biden and Harris are not deeply rooted in
an ideology enough where they could be pulled in one direction or the
other. They are very worried about being seen as a force that is helping
Donald Trump win re-election.

And so I doubt you'll see a lot of criticism in the lead-up to November.
I also think that you don't want to be on the opposite side of someone
who is trying to make history in Kamala Harris. And so I think that
there is a recognition of the power of representation, and the emotion
that some voters are bringing to this ticket will cause folks to tone
down some of the criticisms we saw of her during the primary. But if Joe
Biden and Kamala Harris were to get into office, that kind of détente
would be over.

\textbf{Katie, was there one moment you were thinking about as you
watched Biden's speech, after covering him during the primary?}

\textbf{KATIE GLUECK} I had this moment last fall where I'd been with
him for quite a long time in Iowa, and the speeches were a little low on
energy. He is someone who tries so hard to connect with audiences, and
for a variety of reasons, it was not working for him there. And then I
went with him to North Carolina. He was speaking to a crowd there, and
he was a different candidate. He was so energetic. The crowd was
responding to him, and it was a reminder that, depending on the
environment, he absolutely can bring that energy.

And so heading into his speech, I was curious whether he would be able
to connect without that audience. But it was really striking. He did
seem to bring that energy and vigor.

The other moment I would point to is a speech he gave in Philadelphia
last year as he got his campaign off the ground. The message there was
essentially: We can talk about all kinds of different policy ideas, but
none of this matters until we defeat Donald Trump. That was a message
that faced a lot of skepticism during the primary. But I think this week
we saw everyone onboard with that message. So it seemed to come full
circle.

\textbf{Matt, you wrote about why this campaign worked for Biden --- it
was his third run for president over more than 30 years. Why was this
the moment?}

\textbf{MATT FLEGENHEIMER} In '88, he ran on a personal integrity
message, in large measure, despite his relative youth. In '08, as the
kind of elder statesman senator, he ran on experience and foreign policy
knowledge. That didn't work either.

In some ways he's marrying those two in this campaign. There is the
steady hand statesman who has seen it and done it and knows all the
players. And there's this dominant frame around his own integrity, all
the losses he has suffered, all the resiliency he's demonstrated.

The context of Trump, frankly, is what made that argument resonate in a
way that it hadn't before. The idea that Trump was a national emergency
unto himself is something that Biden has talked about from the
beginning.That frame flows intuitively as the actual national
coronavirus emergency is playing out in real time. So that moment
presented itself to him; in some ways he met it, and in some ways, the
moment met him.

\emph{The next Election 2020 live event is Sept. 15 at 6 p.m. E.T. Find
more details}
\href{https://election2020.splashthat.com/}{\emph{here}}\emph{.}

Advertisement

\protect\hyperlink{after-bottom}{Continue reading the main story}

\hypertarget{site-index}{%
\subsection{Site Index}\label{site-index}}

\hypertarget{site-information-navigation}{%
\subsection{Site Information
Navigation}\label{site-information-navigation}}

\begin{itemize}
\tightlist
\item
  \href{https://help.nytimes3xbfgragh.onion/hc/en-us/articles/115014792127-Copyright-notice}{©~2020~The
  New York Times Company}
\end{itemize}

\begin{itemize}
\tightlist
\item
  \href{https://www.nytco.com/}{NYTCo}
\item
  \href{https://help.nytimes3xbfgragh.onion/hc/en-us/articles/115015385887-Contact-Us}{Contact
  Us}
\item
  \href{https://www.nytco.com/careers/}{Work with us}
\item
  \href{https://nytmediakit.com/}{Advertise}
\item
  \href{http://www.tbrandstudio.com/}{T Brand Studio}
\item
  \href{https://www.nytimes3xbfgragh.onion/privacy/cookie-policy\#how-do-i-manage-trackers}{Your
  Ad Choices}
\item
  \href{https://www.nytimes3xbfgragh.onion/privacy}{Privacy}
\item
  \href{https://help.nytimes3xbfgragh.onion/hc/en-us/articles/115014893428-Terms-of-service}{Terms
  of Service}
\item
  \href{https://help.nytimes3xbfgragh.onion/hc/en-us/articles/115014893968-Terms-of-sale}{Terms
  of Sale}
\item
  \href{https://spiderbites.nytimes3xbfgragh.onion}{Site Map}
\item
  \href{https://help.nytimes3xbfgragh.onion/hc/en-us}{Help}
\item
  \href{https://www.nytimes3xbfgragh.onion/subscription?campaignId=37WXW}{Subscriptions}
\end{itemize}
