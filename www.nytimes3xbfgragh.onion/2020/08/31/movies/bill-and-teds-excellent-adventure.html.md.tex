Sections

SEARCH

\protect\hyperlink{site-content}{Skip to
content}\protect\hyperlink{site-index}{Skip to site index}

\href{https://www.nytimes3xbfgragh.onion/section/movies}{Movies}

\href{https://myaccount.nytimes3xbfgragh.onion/auth/login?response_type=cookie\&client_id=vi}{}

\href{https://www.nytimes3xbfgragh.onion/section/todayspaper}{Today's
Paper}

\href{/section/movies}{Movies}\textbar{}`Bill \& Ted' Explained by Gen X
to Gen Z

\url{https://nyti.ms/2YSISl3}

\begin{itemize}
\item
\item
\item
\item
\item
\end{itemize}

Advertisement

\protect\hyperlink{after-top}{Continue reading the main story}

Supported by

\protect\hyperlink{after-sponsor}{Continue reading the main story}

\hypertarget{bill--ted-explained-by-gen-x-to-gen-z}{%
\section{`Bill \& Ted' Explained by Gen X to Gen
Z}\label{bill--ted-explained-by-gen-x-to-gen-z}}

For many members of a certain generation, ``Bill \& Ted's Excellent
Adventure'' and its sequel were excellent, seminal works. For members of
a younger generation \ldots{}~what?

\includegraphics{https://static01.graylady3jvrrxbe.onion/images/2020/08/31/us/31xpbillted/31xpbillted-articleLarge-v2.jpg?quality=75\&auto=webp\&disable=upscale}

By \href{https://www.nytimes3xbfgragh.onion/by/azi-paybarah}{Azi
Paybarah},
\href{https://www.nytimes3xbfgragh.onion/by/johnny-diaz}{Johnny Diaz},
\href{https://www.nytimes3xbfgragh.onion/by/christina-morales}{Christina
Morales} and
\href{https://www.nytimes3xbfgragh.onion/by/bryan-pietsch}{Bryan
Pietsch}

\begin{itemize}
\item
  Published Aug. 31, 2020Updated Sept. 6, 2020
\item
  \begin{itemize}
  \item
  \item
  \item
  \item
  \item
  \end{itemize}
\end{itemize}

In 1989, two teenagers used a magic telephone booth to travel through
time to finish a school project. For many people around their age, the
movie about those teenagers, Bill S. Preston Esq. and Ted ``Theodore''
Logan, enshrined them in pop culture history: excellent avatars for a
generation of affable, shaggy slackers.

For people around that age now --- 31 years later, in an era without
phone booths --- very little of
\href{https://www.nytimes3xbfgragh.onion/2020/08/27/movies/bill-and-ted-face-the-music-review.html}{a
new movie} released on Aug. 28 about Bill and Ted in middle age makes
any sense.

So two representatives of Gen X, Johnny Diaz and Azi Paybarah,
volunteered to explain the appeal of ``Bill \& Ted's Excellent
Adventure'' to two members of Gen Z, Christina Morales and Bryan
Pietsch. The following is an edited and condensed version of that
attempt to take Christina and Bryan through time.

\hypertarget{before-the-movie-}{%
\subsection{Before the movie \ldots{}}\label{before-the-movie-}}

\textbf{Azi and Johnny, set the scene. What was life like in the
sepia-toned days of the late '80s and early '90s?}

\textbf{Johnny}: Ahh*,* the early '90s. Michael Keaton was Batman.
Mustangs, Miatas and Camaros were popular. So were ``Saved by the Bell''
and ``90210.'' Trips to Blockbuster were routine. MTV was in vogue.
Thanks to Madonna, just about everyone was trying to figure out how to
vogue. Tie-dye shirts, jean shorts and Keds sneakers (at least in South
Florida). Mullets.

\textbf{Azi}: At home, you shared a landline. If you weren't home and
someone called, it would just ring and ring.

\textbf{So how do Bill and Ted enter into this?}

\textbf{Johnny}: Bill and Ted are two California slackers with very good
heads of hair and big heavy metal dreams. They have to pass history
class in order to save the future by time-traveling in a phone booth ---
their version of the DeLorean from ``Back to the Future'' --- to collect
a handful of famous historical figures for their class project.
Amazingly, everyone is able to squeeze into the booth.

\textbf{Azi:} Abraham Lincoln, Napoleon and Socrates walk into a mall.
Two kids will write a song that brings peace to the galaxy, but they
can't get through history class. It's ridiculously aspirational (world
peace!) and absurdly low risk.

\textbf{Christina and Bryan, Does} \emph{\textbf{any}} \textbf{of this
seem familiar? Phone booths? Heavy metal? Keanu Reeves?}

\textbf{Bryan:} My third grade classroom had an out-of-order telephone
booth that we would hang out in for silent reading time, but I've never
used one to make a call. People still did that in the early '90s?

\textbf{Christina:} I have heard of phone booths, George Carlin and
Keanu Reeves. But the only memory I have of a phone booth is seeing one
of those classic red ones in London \ldots{} at Epcot. My familiarity
with Keanu Reeves is through YouTube pop culture. I first Googled him
when \href{https://www.youtube.com/watch?v=T1hT1ANEGC0}{Trisha Paytas}
made videos with a cardboard cutout of him.

\textbf{What do you think this movie is about?}

\textbf{Bryan:} If you just told me the title, I would think it was
about two lame bachelors who were having some sort of life struggle.

\textbf{Christina:} I don't even know where to begin, but it sounds like
a mix of ``Weird Science,'' speckled maybe with some ``Dr. Who,'' ``The
Magic School Bus'' and two nerdy best friends.

\textbf{Does any of this seem appealing? Or dusty and old?}

\textbf{Bryan:} Time-travel movies are always fun. And nothing's too
dusty and old for me! Except maybe silent films.

\textbf{Christina:} I second Bryan. I especially like the essence of
``Back to the Future'' with historical figures. That being said:
Anything that's from the '90s is dusty and old. But it doesn't mean it's
lost its worth.

\textbf{Had you heard of Bill and Ted? Do you think they've influenced
other movies?}

\textbf{Bryan:}
\href{https://www.wired.com/story/sorry-to-this-man-meme/}{In the words
of Keke Palmer}, ``Sorry to this man,'' but I hadn't heard of Bill and
Ted until we started talking about them on Slack. Maybe they've had an
impact that I didn't know about.

\textbf{Christina:} When I asked my younger brother about them, we both
assumed this was related to
\href{https://www.nytimes3xbfgragh.onion/2012/06/29/movies/ted-by-seth-macfarlane-with-mark-wahlberg-and-mila-kunis.html}{the
movie ``Ted,''} with the crude teddy bear and Mark Wahlberg.

\textbf{Bryan:} The title also conjured images of a teddy bear in my
head.

\includegraphics{https://static01.graylady3jvrrxbe.onion/images/2020/08/31/us/31xp-billted2/31xp-billted2-articleLarge-v2.jpg?quality=75\&auto=webp\&disable=upscale}

\hypertarget{after-the-movie--}{%
\subsection{After the movie \ldots{}~}\label{after-the-movie--}}

\textbf{In 1989, a reviewer for The New York Times}
\textbf{\href{https://www.nytimes3xbfgragh.onion/1989/02/17/movies/reviews-film-teen-agers-on-a-tour-of-history.html}{called
the first adventure ``a painfully inept comedy''}} \textbf{whose heroes
were ``inconsistent ciphers'' and ``fond of odd words, such as
bodacious.'' (You can read}
\textbf{\href{https://www.nytimes3xbfgragh.onion/2020/08/27/movies/bill-and-ted-face-the-music-review.html}{A.O.
Scott's review of the new movie, ``Bill \& Ted Face the Music,'' here}.)
Where do your reviews land?}

\textbf{Bryan:} My first laugh-out-loud moment in the movie was not from
intentional comedy but from a sad irony. In the opening scene, Rufus,
the time-travel sherpa, says that in the year 2688, ``The air is clean,
the water is clean, even the dirt is clean.'' I'm anticipating that by
the end of the century, people will be breathing out of tanks and there
will be no water.

In a more clearly fun note: Bill's crop top is totally back in style
now, thanks to TikTok teens.

A not fun note: I thought it was cute when Bill and Ted hug each other
after Ted survives a close call with a knight. But then they use a
homophobic slur, a sign of how times have changed.

\textbf{Christina:} I also laughed at the line about the clean Earth ---
it's sad that it was funny, but the irony lands the same. The movie was
honestly the comedic relief I needed in 2020, even if some of the comedy
doesn't hit the most favorable note today, as Bryan said. (Like the
weird sexualization of Bill and Ted's classmate who becomes Bill's
stepmother.)

But like Bryan, there were tons of funny quotes and scenes that made me
laugh, like ``Caesar is the salad dressing dude.'' It was also funny to
see the special effects and what was thought modern in the late 1980s,
like the idea that the future would be so influenced by rock 'n' roll
greats.

And some of the movie is just perennial teenage humor, like when Bill
and Ted pick the number 69 or when Ted says, ``Strange things are afoot
at the Circle K.'' Like, duh, but it's still hilarious.

\textbf{Johnny and Azi}, \textbf{do you feel vindicated? Have you
already seen the new movie?}

\textbf{Azi:} I haven't seen the movie, but I heard an NPR podcast
review it, which feels like the perfect encapsulation of how I've
outgrown the franchise --- and which is what the Bill and Ted trilogy
has tried to do, pivoting the story to their daughters and wives. What
made us laugh back then would make us cringe today.

So it feels like a victory, dusting off these characters and rewriting
the story to appeal more to a new generation, rather than recycle jokes
for an older one.

\textbf{Johnny:} I couldn't resist. I coughed up \$20 to watch it. I
wanted to see how their wacky humor held up (it did) and how their
charming friendship endured (it's everlasting). This was a fun nostalgia
trip with two middle-aged guys who are only slightly more mature. It was
great to see the next generation, their daughters, carry on the family
tradition of bringing humanity into rhythm and harmony. And even 30
years later, the guys still rock great heads of hair.

Advertisement

\protect\hyperlink{after-bottom}{Continue reading the main story}

\hypertarget{site-index}{%
\subsection{Site Index}\label{site-index}}

\hypertarget{site-information-navigation}{%
\subsection{Site Information
Navigation}\label{site-information-navigation}}

\begin{itemize}
\tightlist
\item
  \href{https://help.nytimes3xbfgragh.onion/hc/en-us/articles/115014792127-Copyright-notice}{©~2020~The
  New York Times Company}
\end{itemize}

\begin{itemize}
\tightlist
\item
  \href{https://www.nytco.com/}{NYTCo}
\item
  \href{https://help.nytimes3xbfgragh.onion/hc/en-us/articles/115015385887-Contact-Us}{Contact
  Us}
\item
  \href{https://www.nytco.com/careers/}{Work with us}
\item
  \href{https://nytmediakit.com/}{Advertise}
\item
  \href{http://www.tbrandstudio.com/}{T Brand Studio}
\item
  \href{https://www.nytimes3xbfgragh.onion/privacy/cookie-policy\#how-do-i-manage-trackers}{Your
  Ad Choices}
\item
  \href{https://www.nytimes3xbfgragh.onion/privacy}{Privacy}
\item
  \href{https://help.nytimes3xbfgragh.onion/hc/en-us/articles/115014893428-Terms-of-service}{Terms
  of Service}
\item
  \href{https://help.nytimes3xbfgragh.onion/hc/en-us/articles/115014893968-Terms-of-sale}{Terms
  of Sale}
\item
  \href{https://spiderbites.nytimes3xbfgragh.onion}{Site Map}
\item
  \href{https://help.nytimes3xbfgragh.onion/hc/en-us}{Help}
\item
  \href{https://www.nytimes3xbfgragh.onion/subscription?campaignId=37WXW}{Subscriptions}
\end{itemize}
