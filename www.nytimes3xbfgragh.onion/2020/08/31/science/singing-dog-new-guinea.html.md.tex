Sections

SEARCH

\protect\hyperlink{site-content}{Skip to
content}\protect\hyperlink{site-index}{Skip to site index}

\href{https://www.nytimes3xbfgragh.onion/section/science}{Science}

\href{https://myaccount.nytimes3xbfgragh.onion/auth/login?response_type=cookie\&client_id=vi}{}

\href{https://www.nytimes3xbfgragh.onion/section/todayspaper}{Today's
Paper}

\href{/section/science}{Science}\textbar{}Singing Dogs Re-emerge From
Extinction for Another Tune

\url{https://nyti.ms/3hOkZmi}

\begin{itemize}
\item
\item
\item
\item
\item
\item
\end{itemize}

Advertisement

\protect\hyperlink{after-top}{Continue reading the main story}

Supported by

\protect\hyperlink{after-sponsor}{Continue reading the main story}

Trilobites

\hypertarget{singing-dogs-re-emerge-from-extinction-for-another-tune}{%
\section{Singing Dogs Re-emerge From Extinction for Another
Tune}\label{singing-dogs-re-emerge-from-extinction-for-another-tune}}

The animal was believed to have disappeared from the highlands of New
Guinea, but was found on the island's Indonesian side.

\includegraphics{https://static01.graylady3jvrrxbe.onion/images/2020/09/08/science/31TB-SINGINGDOGS/31TB-SINGINGDOGS-articleLarge.jpg?quality=75\&auto=webp\&disable=upscale}

\href{https://www.nytimes3xbfgragh.onion/by/james-gorman}{\includegraphics{https://static01.graylady3jvrrxbe.onion/images/2018/02/16/multimedia/author-james-gorman/author-james-gorman-thumbLarge.jpg}}

By \href{https://www.nytimes3xbfgragh.onion/by/james-gorman}{James
Gorman}

\begin{itemize}
\item
  Published Aug. 31, 2020Updated Sept. 3, 2020
\item
  \begin{itemize}
  \item
  \item
  \item
  \item
  \item
  \item
  \end{itemize}
\end{itemize}

The New Guinea Singing Dog, a dingo-like animal with
\href{https://www.facebookcorewwwi.onion/704579679701546/videos/707737052719142}{a
unique howling style}, was considered extinct in the wild. But
scientists reported Monday that the dogs live on, based on DNA collected
by an intrepid and indefatigable field researcher.

Their analysis, published in
\href{https://www.pnas.org/cgi/doi/10.1073/pnas.2007242117}{Proceedings
of the National Academy of
Sciences}\href{https://www.pnas.org/content/early/2020/08/26/2007242117}{,}
showed that the dogs are not simply common village dogs that decided to
try their chances in the wild. The findings not only solve a persistent,
though obscure puzzle, they may shed light on the complicated and still
emerging picture of dog domestication in Asia and Oceania.

Claudio Sillero, a conservation biologist at Oxford University and the
chair of the \href{https://www.canids.org/}{canid specialist group at
the International Union for the Conservation of Nature}, said that the
study confirms the close relatedness between Australian and New Guinea
dogs, ``the most ancient `domestic' dogs on earth.''

James McIntyre, president of the New Guinea Highland Wild Dog Foundation
and the researcher whose forays in the field were central to the
discovery, first searched for New Guinea Singing Dogs in the
forbiddingly rugged highlands of the island, which is split between
Indonesia and Papua New Guinea, in 1996. He was taking a break from
studying intersex pigs in Vanuatu, but that's another story. Mr.
McIntyre has degrees in zoology and education, and has worked at the
Bronx Zoo and other zoos, private conservation organizations and as a
high school biology teacher.

There are highly inbred populations of the dogs in zoos, and some are
kept as exotic pets. But for more than a half-century they remained
elusive in the wild until 2012 when an ecotourism guide snapped a photo
of a wild dog in the highlands of Indonesia's Papua province. It was the
first seen since the 1950s, and Mr. McIntyre set to work. He received
some funding from a mining company, PT Freeport Indonesia. The company,
which has a history of conflict with the local population over
\href{https://www.nytimes3xbfgragh.onion/2005/12/27/world/asia/below-a-mountain-of-wealth-a-river-of-waste.html}{environmental
and safety issues and murky connections to the Indonesian military},
operates a gold mine in the highlands near the wild dog sightings. In
2016 he spent about a month searching and captured 149 photos of 15
individual dogs.

``The locals called them the Highland wild dog,'' he said. ``The New
Guinea Singing Dog was the name developed by Caucasians. Because I
didn't know what they were, I just called them the Highland wild dogs.''

But whether they were really the wild singing dogs that had been
considered extinct was the big question. Even the singing dogs kept in
captivity were a conundrum to scientists who couldn't decide whether
they were a breed, a species or a subspecies. Were these wild dogs the
same as the captive population? Or were they village dogs gone feral
recently?

In 2018, Mr. McIntyre went back to Papua and managed to get DNA from two
trapped wild dogs, quickly released after biological samples were taken,
as well as one other dog that was found dead. He brought the DNA to
researchers who concluded that the highland dogs Mr. McIntyre found are
not village dogs, but appear to belong to the ancestral line from which
the singing dogs descended.

``For decades we've thought that the New Guinea singing dog is extinct
in the wild,'' said Heidi G. Parker of the National Institutes of
Health, who worked with Suriani Surbakti and other researchers from
Indonesia and other countries on analyzing the DNA samples that Mr.
McIntyre returned.

``They are not extinct,'' Dr. Parker said. ``They actually do still
exist in the wild.''

The highland dogs had about 72 percent of their genes in common with
their captive singing cousins. The highland dogs had much more genetic
variation, which would be expected for a wild population. The captive
dogs in conservation centers all descend from seven or eight wild
ancestors.

The 28 percent difference between the wild and captive varieties may
come from some interbreeding with village dogs or from the common
ancestor of all the dogs brought to Oceania. The captive, inbred dogs
may simply have lost a lot of the variation that the wild dogs have.

Their genes could help reinvigorate the captive population of a few
hundred animals in conservation centers, which are very inbred.

Elaine A. Ostrander of the N.I.H., a co-author of the report, says the
finding is also significant for understanding more about dog
domestication. The New Guinea Singing Dogs are closely related to
Australian dingoes and are also related to the Asian dogs that migrated
with humans to Oceania 3,500 years ago or more. It may be that the
singing dogs split off around then from a common ancestor that later
gave rise to breeds like the Akita and Shiba Inu.

``They provide this missing piece that we didn't really have before,''
Dr. Ostrander said.

Laurent Frantz, an evolutionary geneticist at Queen Mary University of
London who studies the domestication and evolution of dogs and was not
involved in the research, said the paper makes clear ``that these
populations have been continuous for a long time.''

But exactly when and where the dogs became feral and ``what is wild,
what is domestic'' are still thorny questions, which the new data will
help to address.

Mr. McIntyre did finish his work on the intersex pigs of Vanuatu, by the
way, and you can find out more at the website of the
\href{http://swprp.org}{Southwest Pacific Research Project}. They are
bred on purpose because they are highly valued by islanders.

Advertisement

\protect\hyperlink{after-bottom}{Continue reading the main story}

\hypertarget{site-index}{%
\subsection{Site Index}\label{site-index}}

\hypertarget{site-information-navigation}{%
\subsection{Site Information
Navigation}\label{site-information-navigation}}

\begin{itemize}
\tightlist
\item
  \href{https://help.nytimes3xbfgragh.onion/hc/en-us/articles/115014792127-Copyright-notice}{©~2020~The
  New York Times Company}
\end{itemize}

\begin{itemize}
\tightlist
\item
  \href{https://www.nytco.com/}{NYTCo}
\item
  \href{https://help.nytimes3xbfgragh.onion/hc/en-us/articles/115015385887-Contact-Us}{Contact
  Us}
\item
  \href{https://www.nytco.com/careers/}{Work with us}
\item
  \href{https://nytmediakit.com/}{Advertise}
\item
  \href{http://www.tbrandstudio.com/}{T Brand Studio}
\item
  \href{https://www.nytimes3xbfgragh.onion/privacy/cookie-policy\#how-do-i-manage-trackers}{Your
  Ad Choices}
\item
  \href{https://www.nytimes3xbfgragh.onion/privacy}{Privacy}
\item
  \href{https://help.nytimes3xbfgragh.onion/hc/en-us/articles/115014893428-Terms-of-service}{Terms
  of Service}
\item
  \href{https://help.nytimes3xbfgragh.onion/hc/en-us/articles/115014893968-Terms-of-sale}{Terms
  of Sale}
\item
  \href{https://spiderbites.nytimes3xbfgragh.onion}{Site Map}
\item
  \href{https://help.nytimes3xbfgragh.onion/hc/en-us}{Help}
\item
  \href{https://www.nytimes3xbfgragh.onion/subscription?campaignId=37WXW}{Subscriptions}
\end{itemize}
