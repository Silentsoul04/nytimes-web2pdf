Sections

SEARCH

\protect\hyperlink{site-content}{Skip to
content}\protect\hyperlink{site-index}{Skip to site index}

\href{https://www.nytimes3xbfgragh.onion/section/business/economy}{Economy}

\href{https://myaccount.nytimes3xbfgragh.onion/auth/login?response_type=cookie\&client_id=vi}{}

\href{https://www.nytimes3xbfgragh.onion/section/todayspaper}{Today's
Paper}

\href{/section/business/economy}{Economy}\textbar{}A Top Fed Official
Warns That Economic Risks Aren't Over

\url{https://nyti.ms/3hUfz9e}

\begin{itemize}
\item
\item
\item
\item
\item
\end{itemize}

\hypertarget{the-coronavirus-outbreak}{%
\subsubsection{\texorpdfstring{\href{https://www.nytimes3xbfgragh.onion/news-event/coronavirus?name=styln-coronavirus-markets\&region=TOP_BANNER\&block=storyline_menu_recirc\&action=click\&pgtype=Article\&impression_id=08a3a190-f4b9-11ea-a720-afe8ca437054\&variant=undefined}{The
Coronavirus
Outbreak}}{The Coronavirus Outbreak}}\label{the-coronavirus-outbreak}}

\begin{itemize}
\tightlist
\item
  live\href{https://www.nytimes3xbfgragh.onion/2020/09/11/world/covid-19-coronavirus.html?name=styln-coronavirus-markets\&region=TOP_BANNER\&block=storyline_menu_recirc\&action=click\&pgtype=Article\&impression_id=08a3a191-f4b9-11ea-a720-afe8ca437054\&variant=undefined}{Latest
  Updates}
\item
  \href{https://www.nytimes3xbfgragh.onion/interactive/2020/us/coronavirus-us-cases.html?name=styln-coronavirus-markets\&region=TOP_BANNER\&block=storyline_menu_recirc\&action=click\&pgtype=Article\&impression_id=08a3a192-f4b9-11ea-a720-afe8ca437054\&variant=undefined}{Maps
  and Cases}
\item
  \href{https://www.nytimes3xbfgragh.onion/interactive/2020/science/coronavirus-vaccine-tracker.html?name=styln-coronavirus-markets\&region=TOP_BANNER\&block=storyline_menu_recirc\&action=click\&pgtype=Article\&impression_id=08a3a193-f4b9-11ea-a720-afe8ca437054\&variant=undefined}{Vaccine
  Tracker}
\item
  \href{https://www.nytimes3xbfgragh.onion/2020/09/10/us/politics/fda-coronavirus-vaccine.html?name=styln-coronavirus-markets\&region=TOP_BANNER\&block=storyline_menu_recirc\&action=click\&pgtype=Article\&impression_id=08a3a194-f4b9-11ea-a720-afe8ca437054\&variant=undefined}{F.D.A.
  Regulators' Self-Defense}
\item
  \href{https://www.nytimes3xbfgragh.onion/2020/09/09/upshot/coronavirus-surprise-test-fees.html?name=styln-coronavirus-markets\&region=TOP_BANNER\&block=storyline_menu_recirc\&action=click\&pgtype=Article\&impression_id=08a3a195-f4b9-11ea-a720-afe8ca437054\&variant=undefined}{Surprise
  Test Fees}
\end{itemize}

Advertisement

\protect\hyperlink{after-top}{Continue reading the main story}

Supported by

\protect\hyperlink{after-sponsor}{Continue reading the main story}

\hypertarget{a-top-fed-official-warns-that-economic-risks-arent-over}{%
\section{A Top Fed Official Warns That Economic Risks Aren't
Over}\label{a-top-fed-official-warns-that-economic-risks-arent-over}}

Lael Brainard, a Federal Reserve governor, said government support would
be needed to support households and businesses as pandemic-induced pain
continued.

\includegraphics{https://static01.graylady3jvrrxbe.onion/images/2020/09/01/business/01DC-Fed-brainard/01DC-Fed-brainard-articleLarge.jpg?quality=75\&auto=webp\&disable=upscale}

\href{https://www.nytimes3xbfgragh.onion/by/jeanna-smialek}{\includegraphics{https://static01.graylady3jvrrxbe.onion/images/2020/07/03/reader-center/author-jeanna-smialek/author-jeanna-smialek-thumbLarge.png}}

By \href{https://www.nytimes3xbfgragh.onion/by/jeanna-smialek}{Jeanna
Smialek}

\begin{itemize}
\item
  Sept. 1, 2020
\item
  \begin{itemize}
  \item
  \item
  \item
  \item
  \item
  \end{itemize}
\end{itemize}

WASHINGTON --- Lael Brainard, a Federal Reserve governor, said on
Tuesday that the U.S. economy remained at risk as the coronavirus
pandemic wore on --- and support from Congress and the White House would
be crucial to cushioning the blow.

``There's a lot of uncertainty that continues to cloud the outlook;
downside risks continue to be important,'' Ms. Brainard said at a
Brookings Institution event. ``It is very important to many households
and businesses to have continued fiscal support --- just as it was
important to them in the early phase of this crisis.''

Her comments came as the future of another government support package
remained unclear. Ms. Brainard, who was appointed during the Obama
administration, said that monetary policy would also play a key role as
pandemic uncertainty persisted, and that central bankers would need to
pivot from stabilizing markets to supporting economic growth in the
coming months.

``It will be important to provide the requisite accommodation to achieve
maximum employment and average inflation of 2 percent over time,'' she
said.

The Fed unveiled a
\href{https://www.nytimes3xbfgragh.onion/2020/08/27/business/economy/federal-reserve-inflation-jerome-powell.html}{new
long-run policy statement} last week, making critical updates to its
strategy for achieving its goals of full employment and stable
inflation. Ms. Brainard said the tweaks, which together laid the
groundwork for long periods of very low interest rates, would help to
guide the central bank's policies coming out of the pandemic.

One key change --- the rate-setting Federal Open Market Committee will
now aim for 2 percent inflation on average over time, instead of as a
more or less absolute goal --- will allow for low rates even as prices
climb slightly, she said.

``I would expect the committee to accommodate rather than offset
inflationary pressures moderately above 2 percent, in a process of
opportunistic reflation,'' she said.

\hypertarget{latest-updates-the-coronavirus-outbreak-and-the-economy}{%
\section{\texorpdfstring{\href{https://www.nytimes3xbfgragh.onion/live/2020/09/11/business/stock-market-today-coronavirus?action=click\&pgtype=Article\&state=default\&region=MAIN_CONTENT_1\&context=storylines_live_updates}{Latest
Updates: The Coronavirus Outbreak and the
Economy}}{Latest Updates: The Coronavirus Outbreak and the Economy}}\label{latest-updates-the-coronavirus-outbreak-and-the-economy}}

\href{https://www.nytimes3xbfgragh.onion/live/2020/09/11/business/stock-market-today-coronavirus?action=click\&pgtype=Article\&state=default\&region=MAIN_CONTENT_1\&context=storylines_live_updates\#the-nyse-may-move-its-data-center-out-of-new-jersey-in-response-to-a-proposed-tax}{9h
ago}

\href{https://www.nytimes3xbfgragh.onion/live/2020/09/11/business/stock-market-today-coronavirus?action=click\&pgtype=Article\&state=default\&region=MAIN_CONTENT_1\&context=storylines_live_updates\#the-nyse-may-move-its-data-center-out-of-new-jersey-in-response-to-a-proposed-tax}{The
N.Y.S.E. may move its data center out of New Jersey in response to a
proposed tax.}

\href{https://www.nytimes3xbfgragh.onion/live/2020/09/11/business/stock-market-today-coronavirus?action=click\&pgtype=Article\&state=default\&region=MAIN_CONTENT_1\&context=storylines_live_updates\#the-federal-budget-deficit-hit-3-trillion-as-of-august}{11h
ago}

\href{https://www.nytimes3xbfgragh.onion/live/2020/09/11/business/stock-market-today-coronavirus?action=click\&pgtype=Article\&state=default\&region=MAIN_CONTENT_1\&context=storylines_live_updates\#the-federal-budget-deficit-hit-3-trillion-as-of-august}{The
federal budget deficit hit \$3 trillion as of August.}

\href{https://www.nytimes3xbfgragh.onion/live/2020/09/11/business/stock-market-today-coronavirus?action=click\&pgtype=Article\&state=default\&region=MAIN_CONTENT_1\&context=storylines_live_updates\#warner-bros-pushes-the-release-of-wonder-woman-1984-to-christmas}{12h
ago}

\href{https://www.nytimes3xbfgragh.onion/live/2020/09/11/business/stock-market-today-coronavirus?action=click\&pgtype=Article\&state=default\&region=MAIN_CONTENT_1\&context=storylines_live_updates\#warner-bros-pushes-the-release-of-wonder-woman-1984-to-christmas}{Warner
Bros. pushes the release of `Wonder Woman 1984' to Christmas.}

\href{https://www.nytimes3xbfgragh.onion/live/2020/09/11/business/stock-market-today-coronavirus?action=click\&pgtype=Article\&state=default\&region=MAIN_CONTENT_1\&context=storylines_live_updates}{See
more updates}

More live coverage:
\href{https://www.nytimes3xbfgragh.onion/2020/09/11/world/covid-19-coronavirus.html?action=click\&pgtype=Article\&state=default\&region=MAIN_CONTENT_1\&context=storylines_live_updates}{Global}

The Fed's rework included a major change to the way it views low
unemployment. Officials tweaked their long-run strategy statement to say
that they would worry about ``shortfalls'' from full employment, rather
than ``deviations.'' In plain terms, that means that the Fed will no
longer raise interest rates simply because joblessness has fallen to low
levels.

That's a big shift from the recent past: The Fed lifted interest rates
nine times between 2015 and the end of 2018 partly out of concern that
the job market would overheat and spur inflation. Instead, price
increases stagnated.

Ms. Brainard said that if the Fed had been using the new strategy
several years ago, ``it is likely that accommodation would have been
withdrawn later, and the gains would have been greater.''

Richard H. Clarida, the Fed's vice chair and the head of the
year-and-a-half review of the Fed's policy framework that resulted in
the updates, also highlighted the central bank's new approach to full
employment
\href{https://www.nytimes3xbfgragh.onion/live/2020/08/31/business/stock-market-today-coronavirus\#a-top-fed-official-says-low-unemployment-alone-wont-trigger-higher-rates}{during
a speech} this week.

Low joblessness ``will not, under our new framework, be a sufficient
trigger for policy action,'' Mr. Clarida said, absent pressing financial
stability concerns or evidence that inflation is overheating or is
likely to run hot.

The Fed's revised statement emphasized that financial stability concerns
will be an important consideration in a world where interest rates are
likely to stay low for extended periods, driving investors to make
bigger bets in hopes of richer payouts. But Ms. Brainard said monetary
policy should not be the primary tool for fighting such bubbles.

Instead, she said, the Fed should use regulation and other forms of
oversight to tamp down risks. As part of that tool kit, she said, banks
should be retaining their capital --- money they can readily tap --- to
make sure they remain healthy amid the pandemic stress.

``I don't think they should be paying out dividends,'' Ms. Brainard said
of commercial banks. ``I think they should be hanging onto their
buffers.''

The Fed oversees large bank holding companies, and has not yet stopped
them from paying dividends to their shareholders.
\href{https://www.federalreserve.gov/newsevents/pressreleases/brainard-statement-20200625c.htm}{Ms.
Brainard objected} to that decision.

The Fed has been grappling with slow-burn changes to the United States
and global economy. The level of interest rates that the economy can
handle without slowing down has fallen, inflation has slipped lower and
the relationship between tight labor markets and price gains seems to
have broken down. Together, the changes have sapped monetary policy of
its strength by leaving officials less room to cut borrowing costs to
stimulate growth during downturns.

The Fed announced that it would revisit its policy approach in late 2018
in light of those shifts, and months of conferences and public forums
led to the strategy update.

In a panel after Ms. Brainard's remarks on Tuesday, former Fed officials
suggested that the framework Chair Jerome H. Powell detailed last week
was a step in the right direction but not a panacea.

``They need to work harder to think about how the different tools
coordinate with each other --- how they're going to think about
financial stability, how they're going to add firepower,'' said Ben S.
Bernanke, who was Fed chair during the 2008 financial crisis. ``There's
a lot to be done still. This is just an aspirational, constitutional
kind of statement.''

And Janet L. Yellen, who was Mr. Bernanke's successor, said the central
bank could not go it alone as it tried to help the United States to
recover from the pandemic-spurred economic crisis.

``In a situation like we're in now, fiscal policy is necessary and plays
an important role,'' Ms. Yellen said. ``There's a little bit more the
Fed can do --- I would look at exploring the tool kit --- but I think we
also need fiscal policy in a situation like this.''

Advertisement

\protect\hyperlink{after-bottom}{Continue reading the main story}

\hypertarget{site-index}{%
\subsection{Site Index}\label{site-index}}

\hypertarget{site-information-navigation}{%
\subsection{Site Information
Navigation}\label{site-information-navigation}}

\begin{itemize}
\tightlist
\item
  \href{https://help.nytimes3xbfgragh.onion/hc/en-us/articles/115014792127-Copyright-notice}{©~2020~The
  New York Times Company}
\end{itemize}

\begin{itemize}
\tightlist
\item
  \href{https://www.nytco.com/}{NYTCo}
\item
  \href{https://help.nytimes3xbfgragh.onion/hc/en-us/articles/115015385887-Contact-Us}{Contact
  Us}
\item
  \href{https://www.nytco.com/careers/}{Work with us}
\item
  \href{https://nytmediakit.com/}{Advertise}
\item
  \href{http://www.tbrandstudio.com/}{T Brand Studio}
\item
  \href{https://www.nytimes3xbfgragh.onion/privacy/cookie-policy\#how-do-i-manage-trackers}{Your
  Ad Choices}
\item
  \href{https://www.nytimes3xbfgragh.onion/privacy}{Privacy}
\item
  \href{https://help.nytimes3xbfgragh.onion/hc/en-us/articles/115014893428-Terms-of-service}{Terms
  of Service}
\item
  \href{https://help.nytimes3xbfgragh.onion/hc/en-us/articles/115014893968-Terms-of-sale}{Terms
  of Sale}
\item
  \href{https://spiderbites.nytimes3xbfgragh.onion}{Site Map}
\item
  \href{https://help.nytimes3xbfgragh.onion/hc/en-us}{Help}
\item
  \href{https://www.nytimes3xbfgragh.onion/subscription?campaignId=37WXW}{Subscriptions}
\end{itemize}
