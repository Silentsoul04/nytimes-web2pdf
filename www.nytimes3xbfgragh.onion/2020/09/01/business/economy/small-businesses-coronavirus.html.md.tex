Sections

SEARCH

\protect\hyperlink{site-content}{Skip to
content}\protect\hyperlink{site-index}{Skip to site index}

\href{https://www.nytimes3xbfgragh.onion/section/business/economy}{Economy}

\href{https://myaccount.nytimes3xbfgragh.onion/auth/login?response_type=cookie\&client_id=vi}{}

\href{https://www.nytimes3xbfgragh.onion/section/todayspaper}{Today's
Paper}

\href{/section/business/economy}{Economy}\textbar{}Small-Business
Failures Loom as Federal Aid Dries Up

\url{https://nyti.ms/2YSssJy}

\begin{itemize}
\item
\item
\item
\item
\item
\item
\end{itemize}

\hypertarget{the-coronavirus-outbreak}{%
\subsubsection{\texorpdfstring{\href{https://www.nytimes3xbfgragh.onion/news-event/coronavirus?name=styln-coronavirus-markets\&region=TOP_BANNER\&block=storyline_menu_recirc\&action=click\&pgtype=Article\&impression_id=bad28bd0-f52f-11ea-ae45-8f05135d08e7\&variant=undefined}{The
Coronavirus
Outbreak}}{The Coronavirus Outbreak}}\label{the-coronavirus-outbreak}}

\begin{itemize}
\tightlist
\item
  live\href{https://www.nytimes3xbfgragh.onion/2020/09/12/world/covid-19-coronavirus.html?name=styln-coronavirus-markets\&region=TOP_BANNER\&block=storyline_menu_recirc\&action=click\&pgtype=Article\&impression_id=bad28bd1-f52f-11ea-ae45-8f05135d08e7\&variant=undefined}{Latest
  Updates}
\item
  \href{https://www.nytimes3xbfgragh.onion/interactive/2020/us/coronavirus-us-cases.html?name=styln-coronavirus-markets\&region=TOP_BANNER\&block=storyline_menu_recirc\&action=click\&pgtype=Article\&impression_id=bad2b2e0-f52f-11ea-ae45-8f05135d08e7\&variant=undefined}{Maps
  and Cases}
\item
  \href{https://www.nytimes3xbfgragh.onion/interactive/2020/science/coronavirus-vaccine-tracker.html?name=styln-coronavirus-markets\&region=TOP_BANNER\&block=storyline_menu_recirc\&action=click\&pgtype=Article\&impression_id=bad2b2e1-f52f-11ea-ae45-8f05135d08e7\&variant=undefined}{Vaccine
  Tracker}
\item
  \href{https://www.nytimes3xbfgragh.onion/2020/09/10/us/politics/fda-coronavirus-vaccine.html?name=styln-coronavirus-markets\&region=TOP_BANNER\&block=storyline_menu_recirc\&action=click\&pgtype=Article\&impression_id=bad2b2e2-f52f-11ea-ae45-8f05135d08e7\&variant=undefined}{F.D.A.
  Regulators' Self-Defense}
\item
  \href{https://www.nytimes3xbfgragh.onion/2020/09/09/upshot/coronavirus-surprise-test-fees.html?name=styln-coronavirus-markets\&region=TOP_BANNER\&block=storyline_menu_recirc\&action=click\&pgtype=Article\&impression_id=bad2b2e3-f52f-11ea-ae45-8f05135d08e7\&variant=undefined}{Surprise
  Test Fees}
\end{itemize}

Advertisement

\protect\hyperlink{after-top}{Continue reading the main story}

Supported by

\protect\hyperlink{after-sponsor}{Continue reading the main story}

\hypertarget{small-business-failures-loom-as-federal-aid-dries-up}{%
\section{Small-Business Failures Loom as Federal Aid Dries
Up}\label{small-business-failures-loom-as-federal-aid-dries-up}}

Many owners face tough choices after a federal loan program and other
government moves to bolster the economy have run their course.

\includegraphics{https://static01.graylady3jvrrxbe.onion/images/2020/08/31/business/00virus-smallbiz-01/merlin_176284503_f8118d6e-5257-4b3a-87f2-aff29f6398fa-articleLarge.jpg?quality=75\&auto=webp\&disable=upscale}

\href{https://www.nytimes3xbfgragh.onion/by/ben-casselman}{\includegraphics{https://static01.graylady3jvrrxbe.onion/images/2018/11/09/multimedia/author-ben-casselman/author-ben-casselman-thumbLarge.png}}

By \href{https://www.nytimes3xbfgragh.onion/by/ben-casselman}{Ben
Casselman}

\begin{itemize}
\item
  Sept. 1, 2020
\item
  \begin{itemize}
  \item
  \item
  \item
  \item
  \item
  \item
  \end{itemize}
\end{itemize}

The United States faces a wave of small-business failures this fall if
the federal government does not provide a new round of financial
assistance --- a prospect that economists warn would prolong the
recession, slow the recovery and perhaps enduringly reshape the American
business landscape.

As the pandemic drags on, it is threatening even well-established
businesses that were financially healthy before the crisis. If they shut
down or are severely weakened, it could accelerate corporate
consolidation and the dominance of the biggest companies.

Tens of thousands of restaurants, bars, retailers and other small
businesses have
\href{https://www.nytimes3xbfgragh.onion/2020/07/13/business/small-businesses-coronavirus.html}{already
closed}. But many more have survived, buoyed in part by billions of
dollars in government assistance to both businesses and their customers.

The
\href{https://www.nytimes3xbfgragh.onion/2020/08/28/business/ppp-small-business-fraud-coronavirus.html}{Paycheck
Protection Program} provided hundreds of billions in loans and grants to
help businesses retain employees and meet other obligations. Billions
more went to the unemployed, in a \$600 weekly supplement to state
jobless benefits, and to many households, through a \$1,200 tax rebate
--- money available to spend at local stores and restaurants.

Now that aid is
\href{https://www.nytimes3xbfgragh.onion/2020/08/06/business/small-businesses-relief-program-ending.html}{largely
gone}, even as the economic recovery that took hold in the spring is
\href{https://www.nytimes3xbfgragh.onion/2020/08/25/business/american-airline-furlough-19000.html}{losing
momentum}. The fall will bring new challenges: Colder weather will
curtail outdoor dining and other weather-dependent adaptations that
helped businesses hang on in much of the country, and epidemiologists
warn that the winter could bring a surge in coronavirus cases.

As a result, many businesses face a stark choice: Do they try to hold on
through a winter that could bring new shutdowns and restrictions, with
no guarantee that sales will bounce back in the spring? Or do they
\href{https://www.nytimes3xbfgragh.onion/2020/08/03/nyregion/nyc-small-businesses-closing-coronavirus.html}{cut
their losses} while they have something to salvage?

For the Cheers Replica bar in Faneuil Hall in Boston, the answer was to
throw in the towel after nearly two decades in business.

``We just came to the conclusion, if we're losing that much money in the
summertime, what's the winter going to look like?'' said Markus
Ripperger, president and chief executive of Hampshire House, the bar's
parent company.

Many businesses that failed in the early weeks of the pandemic were
already struggling, had owners nearing retirement or were otherwise
likely to shut down in the next couple of years. Those closing down now
look different.

Cheers was a longstanding, successful business with access to capital
and owners willing to invest to keep it going. But the bar, built to
resemble the one on the 1980s sitcom, depended heavily on tourist
traffic that collapsed during the pandemic.

\hypertarget{latest-updates-the-coronavirus-outbreak-and-the-economy}{%
\section{\texorpdfstring{\href{https://www.nytimes3xbfgragh.onion/live/2020/09/11/business/stock-market-today-coronavirus?action=click\&pgtype=Article\&state=default\&region=MAIN_CONTENT_1\&context=storylines_live_updates}{Latest
Updates: The Coronavirus Outbreak and the
Economy}}{Latest Updates: The Coronavirus Outbreak and the Economy}}\label{latest-updates-the-coronavirus-outbreak-and-the-economy}}

\href{https://www.nytimes3xbfgragh.onion/live/2020/09/11/business/stock-market-today-coronavirus?action=click\&pgtype=Article\&state=default\&region=MAIN_CONTENT_1\&context=storylines_live_updates\#the-nyse-may-move-its-data-center-out-of-new-jersey-in-response-to-a-proposed-tax}{23h
ago}

\href{https://www.nytimes3xbfgragh.onion/live/2020/09/11/business/stock-market-today-coronavirus?action=click\&pgtype=Article\&state=default\&region=MAIN_CONTENT_1\&context=storylines_live_updates\#the-nyse-may-move-its-data-center-out-of-new-jersey-in-response-to-a-proposed-tax}{The
N.Y.S.E. may move its data center out of New Jersey in response to a
proposed tax.}

\href{https://www.nytimes3xbfgragh.onion/live/2020/09/11/business/stock-market-today-coronavirus?action=click\&pgtype=Article\&state=default\&region=MAIN_CONTENT_1\&context=storylines_live_updates\#the-federal-budget-deficit-hit-3-trillion-as-of-august}{26h
ago}

\href{https://www.nytimes3xbfgragh.onion/live/2020/09/11/business/stock-market-today-coronavirus?action=click\&pgtype=Article\&state=default\&region=MAIN_CONTENT_1\&context=storylines_live_updates\#the-federal-budget-deficit-hit-3-trillion-as-of-august}{The
federal budget deficit hit \$3 trillion as of August.}

\href{https://www.nytimes3xbfgragh.onion/live/2020/09/11/business/stock-market-today-coronavirus?action=click\&pgtype=Article\&state=default\&region=MAIN_CONTENT_1\&context=storylines_live_updates\#warner-bros-pushes-the-release-of-wonder-woman-1984-to-christmas}{26h
ago}

\href{https://www.nytimes3xbfgragh.onion/live/2020/09/11/business/stock-market-today-coronavirus?action=click\&pgtype=Article\&state=default\&region=MAIN_CONTENT_1\&context=storylines_live_updates\#warner-bros-pushes-the-release-of-wonder-woman-1984-to-christmas}{Warner
Bros. pushes the release of `Wonder Woman 1984' to Christmas.}

\href{https://www.nytimes3xbfgragh.onion/live/2020/09/11/business/stock-market-today-coronavirus?action=click\&pgtype=Article\&state=default\&region=MAIN_CONTENT_1\&context=storylines_live_updates}{See
more updates}

More live coverage:
\href{https://www.nytimes3xbfgragh.onion/2020/09/11/world/covid-19-coronavirus.html?action=click\&pgtype=Article\&state=default\&region=MAIN_CONTENT_1\&context=storylines_live_updates}{Global}

The company's three other restaurants, which include the original Cheers
bar on Beacon Hill that was the inspiration for the show, remain in
business. But Mr. Ripperger said he was worried about what a winter
resurgence of the virus might mean.

\includegraphics{https://static01.graylady3jvrrxbe.onion/images/2020/08/31/business/00virus-smallbiz-02/merlin_176333601_0e64007c-0c9f-44f5-aab7-666c97dbb455-articleLarge.jpg?quality=75\&auto=webp\&disable=upscale}

``We're on life support now, and if we have to go through another
shutdown or more restrictions, it's going to be even worse for a lot
more restaurants that are just barely scraping by,'' he said.

On Friday, the Commerce Department reported that consumer spending
\href{https://www.bea.gov/news/2020/personal-income-and-outlays-july-2020}{rose
only modestly in July} after two months of resurgence and remained below
pre-pandemic levels. Economists warn that without the \$600 a week in
extra unemployment insurance, spending is likely to slow further this
fall.

Data from \href{https://joinhomebase.com/data}{Homebase}, which provides
time-management software to small businesses, shows that roughly 20
percent of businesses that were open in January are closed either
temporarily or permanently. The number of hours worked --- a rough proxy
for revenues --- is down by even more during what should be the year's
busiest period. Both figures have stalled or turned down in recent
weeks.

Small businesses have grown more pessimistic as the pandemic has dragged
on. In late April, about a third of small businesses
\href{https://portal.census.gov/pulse/data/}{surveyed by the Census
Bureau} said they expected it to take more than six months for business
to return to normal. Four months later, nearly half say so, and a
further 7.5 percent say they do not expect business ever to bounce back
fully. About 5 percent say they expect to close permanently in the next
six months.

The ultimate damage could be much greater. In a recent survey by the
National Federation of Independent Businesses, a small-business lobbying
group, 21 percent of small businesses said they would have to close if
conditions did not improve in the next six months. Other private-sector
surveys have found similar results.

Widespread business failures could cause lasting economic damage. Nearly
half of American employees work for businesses with staffs under 500,
meaning millions of jobs are at stake. And while new businesses would
inevitably spring up to replace those that close, that process will take
far longer than simply reopening existing businesses.

``The consequences to allowing a tidal wave of closures is we will make
every aspect of the recovery harder,'' said John Lettieri, president and
chief executive of the Economic Innovation Group, a Washington research
organization.

There could also be longer-run implications. Despite high-profile
bankruptcies in the retail industry and other sectors, many large
corporations have been able to solidify their position during the
pandemic: demanding concessions from landlords, borrowing billions of
dollars at low interest rates and leveraging sophisticated supply chains
and distribution systems to reach suddenly homebound customers. Small
businesses, which usually have less access to credit and rely more
heavily on foot traffic, have been struggling to survive.

Image

``I can survive because I'm betting on another stimulus package,'' said
Candace Combs, who runs the In-Symmetry Spa in San Francisco with her
brother. ``But without that, we start to really teeter.''Credit...Jamie
Cotten for The New York Times

The challenge has been particularly acute for Black-owned businesses,
which were more than twice as likely to close down in the early months
of the pandemic than small businesses over all, according to
\href{https://www.newyorkfed.org/medialibrary/media/smallbusiness/DoubleJeopardy_COVID19andBlackOwnedBusinesses}{research}
from the Federal Reserve Bank of New York. Black-owned businesses were
more likely to be in areas hit hard by the virus, had less of a
financial cushion and were less likely to have established banking
relationships, which put them at a disadvantage in seeking loans under
the emergency Paycheck Protection Program in the critical first weeks
that the aid was available.

By the time they got access to the federal money, ``many Black-owned
businesses were already out of business,'' said Ron Busby, president and
chief executive of the U.S. Black Chambers. ``We just couldn't make it
that long.''

Maurice Brewster is hanging on. He runs Mosaic Global Transportation, a
California company that was growing quickly before the pandemic running
the private buses that shuttled tech workers between their San Francisco
homes and their suburban office campuses.

Those campuses have been all but empty since March, and many companies
aren't planning to bring workers back until next year. Other parts of
Mr. Brewster's business --- providing transportation for conventions,
wine tours and other events --- are also suffering.

To survive, Mr. Brewster, who is Black, has slashed costs and sought new
lines of business, including delivering packages for Amazon ---
``anything to get the vehicles moving and get some revenue coming in the
door,'' he said.

Mr. Brewster says he is confident he can make it through the end of the
year. After that, he doesn't know.

``You just can't go a year unless you have just an endless pool of money
to sustain you until March or April of 2021,'' he said. ``A lot of us
are going to go out of business.''

Economists say there is time to limit the damage. Despite a rocky start,
the Paycheck Protection Program eventually paid out more than half a
trillion dollars in loans and probably saved many businesses from
failure, according to
\href{https://www.nber.org/papers/w27623}{research} from economists at
the University of Illinois and Harvard. But the program lapsed in
August, and if Congress doesn't move soon to replace it, the earlier
effort could end up delaying failures rather than preventing them.

Many experts still expect Democratic and Republican leaders to reach a
deal on an aid package that includes support for small businesses, but a
new, large-scale program seems increasingly unlikely.

``Why didn't we use the time that P.P.P. bought us to design the kind of
program that would be commensurate with the national challenge that
we're facing?'' Mr. Lettieri, of the Economic Innovation Group, asked.
``That's all P.P.P. was. It was a mechanism to buy time. It was never
the long-term solution.''

A paycheck protection loan helped keep In-Symmetry Spa afloat early in
the pandemic. But the money is long gone, and the San Francisco spa
hasn't been allowed to reopen. Nearby storefronts are boarded up, and
Candace Combs, who has run the spa with her brother for two decades,
said she doubted that many of those businesses were coming back.

``I can survive because I'm betting on another stimulus package,'' Ms.
Combs said. ``But without that, we start to really teeter.''

Jim Tankersley contributed reporting.

Advertisement

\protect\hyperlink{after-bottom}{Continue reading the main story}

\hypertarget{site-index}{%
\subsection{Site Index}\label{site-index}}

\hypertarget{site-information-navigation}{%
\subsection{Site Information
Navigation}\label{site-information-navigation}}

\begin{itemize}
\tightlist
\item
  \href{https://help.nytimes3xbfgragh.onion/hc/en-us/articles/115014792127-Copyright-notice}{©~2020~The
  New York Times Company}
\end{itemize}

\begin{itemize}
\tightlist
\item
  \href{https://www.nytco.com/}{NYTCo}
\item
  \href{https://help.nytimes3xbfgragh.onion/hc/en-us/articles/115015385887-Contact-Us}{Contact
  Us}
\item
  \href{https://www.nytco.com/careers/}{Work with us}
\item
  \href{https://nytmediakit.com/}{Advertise}
\item
  \href{http://www.tbrandstudio.com/}{T Brand Studio}
\item
  \href{https://www.nytimes3xbfgragh.onion/privacy/cookie-policy\#how-do-i-manage-trackers}{Your
  Ad Choices}
\item
  \href{https://www.nytimes3xbfgragh.onion/privacy}{Privacy}
\item
  \href{https://help.nytimes3xbfgragh.onion/hc/en-us/articles/115014893428-Terms-of-service}{Terms
  of Service}
\item
  \href{https://help.nytimes3xbfgragh.onion/hc/en-us/articles/115014893968-Terms-of-sale}{Terms
  of Sale}
\item
  \href{https://spiderbites.nytimes3xbfgragh.onion}{Site Map}
\item
  \href{https://help.nytimes3xbfgragh.onion/hc/en-us}{Help}
\item
  \href{https://www.nytimes3xbfgragh.onion/subscription?campaignId=37WXW}{Subscriptions}
\end{itemize}
