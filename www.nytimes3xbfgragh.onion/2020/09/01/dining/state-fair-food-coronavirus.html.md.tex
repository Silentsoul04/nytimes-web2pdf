Sections

SEARCH

\protect\hyperlink{site-content}{Skip to
content}\protect\hyperlink{site-index}{Skip to site index}

\href{https://www.nytimes3xbfgragh.onion/section/food}{Food}

\href{https://myaccount.nytimes3xbfgragh.onion/auth/login?response_type=cookie\&client_id=vi}{}

\href{https://www.nytimes3xbfgragh.onion/section/todayspaper}{Today's
Paper}

\href{/section/food}{Food}\textbar{}The State Fair Is Canceled.
Deep-Fried Oreos Are Not.

\begin{itemize}
\item
\item
\item
\item
\item
\item
\end{itemize}

\href{https://www.nytimes3xbfgragh.onion/spotlight/at-home?action=click\&pgtype=Article\&state=default\&region=TOP_BANNER\&context=at_home_menu}{At
Home}

\begin{itemize}
\tightlist
\item
  \href{https://www.nytimes3xbfgragh.onion/2020/09/07/travel/route-66.html?action=click\&pgtype=Article\&state=default\&region=TOP_BANNER\&context=at_home_menu}{Cruise
  Along: Route 66}
\item
  \href{https://www.nytimes3xbfgragh.onion/2020/09/04/dining/sheet-pan-chicken.html?action=click\&pgtype=Article\&state=default\&region=TOP_BANNER\&context=at_home_menu}{Roast:
  Chicken With Plums}
\item
  \href{https://www.nytimes3xbfgragh.onion/2020/09/04/arts/television/dark-shadows-stream.html?action=click\&pgtype=Article\&state=default\&region=TOP_BANNER\&context=at_home_menu}{Watch:
  Dark Shadows}
\item
  \href{https://www.nytimes3xbfgragh.onion/interactive/2020/at-home/even-more-reporters-editors-diaries-lists-recommendations.html?action=click\&pgtype=Article\&state=default\&region=TOP_BANNER\&context=at_home_menu}{Explore:
  Reporters' Google Docs}
\end{itemize}

Advertisement

\protect\hyperlink{after-top}{Continue reading the main story}

Supported by

\protect\hyperlink{after-sponsor}{Continue reading the main story}

The Great Read

\hypertarget{the-state-fair-is-canceled-deep-fried-oreos-are-not}{%
\section{The State Fair Is Canceled. Deep-Fried Oreos Are
Not.}\label{the-state-fair-is-canceled-deep-fried-oreos-are-not}}

Food vendors and their devoted fans are going to great lengths, from
drive-throughs to phone apps, to keep the corn dogs and
chickens-on-a-stick flowing.

\includegraphics{https://static01.graylady3jvrrxbe.onion/images/2020/09/02/dining/01statefair9/01statefair9-articleLarge.jpg?quality=75\&auto=webp\&disable=upscale}

By Marissa Conrad

\begin{itemize}
\item
  Sept. 1, 2020
\item
  \begin{itemize}
  \item
  \item
  \item
  \item
  \item
  \item
  \end{itemize}
\end{itemize}

Last October, to kick off her 20th year as a vendor at the
\href{http://www.ncstatefair.org/index.htm}{North Carolina State Fair},
Felicia Turrentine-Daniel unveiled the Chickenator: a cinnamon roll
sliced like a hamburger bun to hold a deep-fried chicken breast, bacon,
pepper Jack cheese and a drizzle of honeyed Sriracha.

``We make sure that we bring in something new and big every year,'' said
Ms. Turrentine-Daniel, who runs the booth Chef's D'Lites. She met her
husband, Jason Daniel, while they both worked in a grocery store, ``and
we would literally walk up and down the aisles and find different things
to put in the fryer, to see what came out.''

\includegraphics{https://static01.graylady3jvrrxbe.onion/images/2020/09/02/dining/01statefair2/01statefair2-articleLarge.jpg?quality=75\&auto=webp\&disable=upscale}

On July 29, when North Carolina's became the 35th state fair to be
canceled or severely curtailed, her friends, family and regular
customers called and emailed: Was she OK? And oh, by the way, was there
any chance they could still order a deep-fried Cuban roll, or those
fried banana pudding bites?

Image

Ms. Turrentine-Daniel at her home in Greensboro, N.C. This would have
been her 21st year as a vendor at that state's fair.Credit...Mike
Belleme for The New York Times

On the \href{https://www.facebookcorewwwi.onion/chefsdlites/}{Facebook
page} for Chef's D'Lites, Ms. Turrentine-Daniel, 42, now accepts
direct-message orders for dishes she can fry, freeze and ship, including
the Cuban rolls and her deep-fried macaroni and cheese, with
instructions on how to finish the dish in the oven, microwave or deep
fryer. Chickenators don't ship well, but she has delivered them fresh to
customers who live within a 30-minute drive of her home in Greensboro.

``Just meet me halfway, and we'll work it out,'' she said.

Across the country, concessionaires are going to great lengths, from
organizing drive-throughs to buying delivery trucks, to keep the
fair-food pipeline intact as state fairs continue to be called off ---
so far, in 36 states and the District of Columbia --- many for the first
time since World War II. And fair regulars, the people who can tell you
definitively which of eight near-identical cheese curd trailers is the
best, or who keep annually updated checklists of foods on a stick, are
coming out to support their favorite vendors.

Lori Lexvold has attended the Minnesota State Fair, in a Twin Cities
suburb, every summer for 53 of her 58 years. When it was called off in
late May, ``I thought, what in the world is going to happen to all these
vendors?'' she said. ``This is their livelihood.''

She heard that some were setting up where they could find space: church
lawns, mall parking lots, outside a Harley-Davidson dealership.

``I got on Facebook one morning and I created a group,'' said Ms.
Lexvold, who lives in Forest Lake, about a half-hour drive from the
fairgrounds. ``I invited about 100 of my friends. I just said, `Hey, if
you see any food stands around, post it to this page, so we can all
go.'''

The group,
\href{https://www.facebookcorewwwi.onion/groups/245420266734458}{Fair
Food Finder}, now has nearly 179,000 members, a Google map of 139
Minnesota vendors and a phone app created by an enthusiastic fan.

``It was crazy,'' Ms. Lexvold said of a time this summer when she was
getting 10,000 requests a day from strangers wanting to join, and phone
calls from vendors asking how to post on the page. ``I thought, how in
the dickens are you finding my phone number?''

Ms. Lexvold found her fair-food fix at the Anoka County Fairgrounds
outside Minneapolis, where three vendors had parked. They were in the
big, traditional state-fair trailers, Ms. Lexvold said, with Hollywood
bulbs and colorful flags, selling not just fries, but also ice cream,
cheese curds and mini-doughnuts, fresh from the fryer.

``People pulled chairs out of the back of their cars, and sat there and
had a little picnic,'' she said. ``I thought, this is what it's all
about.''

(\href{https://www.statefairtogo.com/}{State Fair To Go}, a new business
based in Minnetonka, Minn., ships boxes of Minnesota State Fair food
within the continental United States. Each box costs \$59.95 --- with
free shipping within the state --- and contains six fair staples,
including Elliott's Up North corn dogs, Rosie's French fries and Sweet
Martha's cookies. This week, the company began selling a \$64.95 box of
foods from the State Fair of Texas.)

Image

After the Minnesota State Fair was canceled, Stephanie Shimp bought a
food truck to take her wares on the road.Credit...Jenn Ackerman for The
New York Times

Stephanie Shimp couldn't exactly tow around
\href{http://bluebarnmn.com/}{the Blue Barn}, a fixture of the Minnesota
State Fair --- it's a full-size barn that stands on the grounds
year-round. In June, she took out a loan to buy and revamp a
decommissioned grocery delivery truck. She held her first event, in a
brewery parking lot, just one day after taking possession of the truck,
and sold out of food in four hours. ``It was kind of chaos,'' said Ms.
Shimp, 49.

The Little Blue Food Truck now travels Minnesota selling the Blue Barn's
greatest hits. To make Nashville-style hot chicken on a stick, Ms. Shimp
brines tenders in buttermilk spiced with cayenne, paprika and cumin,
rolls the meat in cornflakes and deep-fries it, then serves it with
pickles and a slice of white bread. Chicken in a waffle cone is
Instagram catnip: a quilted cone layered with tenders and sausage gravy,
sprinkled with chopped parsley and stabbed with a fork.

Image

At Minnesota's 2019 fair, Ms. Shimp sold nearly 35,000 orders of
Nashville-style hot chicken on a stick. Now, it's the best-selling item
on her food truck's menu.Credit...Jenn Ackerman for The New York Times

Minnesotans who miss the fair have messaged Ms. Shimp to ask if she'll
park in their driveways. Corey Mathisen organized a party with the truck
in early August. Dozens of neighbors came to his front yard in Rosemount
to buy Blue Barn food, with the request that ``people remember social
distancing and wear masks,'' said Mr. Mathisen, 35.

(He also invited a group of children to
\href{https://www.gofundme.com/f/kamryn-amp-friends-bracelets-for-unity-amp-justice}{raise
money} at the event: Neighbors could buy friendship bracelets from
\href{https://www.cnn.com/2020/06/20/us/kamryn-friends-sell-bracelets-businesses-minneapolis-trnd/index.html}{Kamryn
\& Friends: Bracelets for Unity and Justice,} which has collected more
than \$70,000 for food-relief drives and Minneapolis-area businesses,
many of them Black-owned, that were damaged during the unrest after the
killing of
\href{https://www.nytimes3xbfgragh.onion/news-event/george-floyd-protests-minneapolis-new-york-los-angeles}{George
Floyd}.)

In Des Moines, Brenda Smith Parish of Brenda Smith Concessions arrives
at her Crazy Taters trailer at 8 a.m. every Friday to chop tomatoes for
gyros and make sugar water for her state-fair lemonade. By 10:30, cars
are lined up in the drive-through she has created behind her parents'
catering business.

In April, faced with the prospect of a fair-less summer, Ms. Smith
Parish set up with just corn dogs and lemonade, sent out a social-media
blast and hoped people would come. ``And, I mean, they came,'' she said.
She expanded the menu to 20 or so items and created a website,
FairFoodFridays.com, to take preorders through Shopify.

Now, 500 to 600 cars show up each week for food from Crazy Taters and
her parents' stands, All American Grill and Turkey Time. An acoustic
guitarist plays on a flatbed trailer as drivers roll down their windows
to accept handoffs of pickle dogs, turkey legs and deep-fried Oreos.

``I started this thinking, if I could just make my rent and my car
payment, I'm good with that,'' said Ms. Smith Parish, 46. ``I didn't
expect this.'' She has made nearly as much money as she would have in
her typical season at the \href{https://www.iowastatefair.org/}{Iowa
State Fair}, \href{https://www.tulsastatefair.com/}{Tulsa State Fair}
and \href{https://www.dsmpartnership.com/desmoinesfarmersmarket/}{Des
Moines Downtown Farmers' Market}, all of which were canceled, though the
Farmers' Market started its own drive-through in early August.

Many vendors aren't doing as well. Russell Goetze and his two brothers
usually tow all five soft-serve trailers for Goertze's Dairy Kone (their
father added the ``r'' when he founded the business in 1967, to help
customers pronounce the name) to six state fairs a year, mostly along
the East Coast.

This year, Russell Goetze, 58, has been able to park one Dairy Kone
trailer and his sausage stand, Lenny's, at the
\href{https://howardcountyfairmd.com/}{Howard County Fairgrounds} in
West Friendship, Md., selling cones, shakes, corn dogs and sausage
sandwiches, but is making barely enough to cover basic bills. ``It's a
very trying time,'' he said.

The Dallas chef Abel Gonzales said he usually earns 80 percent of his
annual revenue in just 24 days in September and October, selling
deep-fried foods at the \href{https://bigtex.com/}{State Fair of Texas}.
His signature is deep-fried butter: Wrap bread dough around a slab of
butter, freeze it and fry it. The dough crisps and the butter liquefies.

``You bite into it, and the butter gets all over the place,'' said Mr.
Gonzales, 50. ``It's fun.''

Mr. Gonzales is offering a few fair-food items, including fried peanut
butter, jelly and banana sandwiches, at
\href{https://cocinaitaliano.com/}{Cocina Italiano}, his restaurant in
the city, but he knows the math won't add up. Last year, the State Fair
of Texas pulled in 2.5 million visitors.

Image

A new ``Fair Stuff'' menu at Abel Gonzales' Dallas restaurant, Cocina
Italiano, includes deep-fried cookie dough and deep-fried peanut butter,
jelly and banana sandwiches.Credit...Allison V. Smith for The New York
Times

Image

Mr. Gonzales has been deep-frying food at the State Fair of Texas since
2002. Among his most inventive dishes are fried butter and fried
soda.Credit...Allison V. Smith for The New York Times

``I'm not going to make those numbers,'' he said. ``But I have a very
strong support system, and I have a restaurant. I'm thinking about the
people that depend on the fair, everybody from my crew to the guys who
run the rides. Just the ripple effect of the fair not happening. It's
heartbreaking.''

Some fairs are organizing events for vendors. The Wisconsin State Fair's
Fair Food Drive-Thru, which ran weekends through Aug. 16, invited
customers to drive around the Milwaukee Mile, the city's 144-year-old
racetrack, to order from 14 vendors, like Mille's Italian Sausage,
Kora's Cookie Dough and Original Cream Puffs. The towering cream puffs
have been a fixture at the fair since 1924.

Amalia Hetzer, of Cudahy, Wis., usually runs the Cream Puff 5K race, 3.1
miles around the fairgrounds with the promise of a cream puff at the
finish line. ``The cream is that buttery, not super-sweet cream,'' she
said. ``It's just so good.''

This year, Ms. Hetzer, 35, completed the virtual race: She paid \$33 to
run, then received three cream puffs at a curbside pickup location.
Because her mother also ran the race, they got six, which they ate with
family at a park.

Sharing, after all, is part of the fair spirit --- as an industry group
called \href{http://oregondairywomen.com/}{Oregon Dairy Women} can
attest.

Image

Samantha Arnold serving ice cream from a trailer parked outside a Wilco
Farm Store in Canby, Ore. The soft-serve is popular at that state's
fair, which would have taken place from Aug. 28 through Sept.
7.Credit...Cole Wilson for The New York Times

When the organization took its state-fair soft-serve on the road in July
and August, ``one woman ordered 17 milkshakes'' for her office, said
Becky Heimerl, 39, the group's president. ``She had a cardboard box in
her car, ready to go.''

The word that customers kept bringing up, Ms. Heimerl said, was
``normal'': ``Oh, finally something that feels normal about the
summer.''

\emph{Follow} \href{https://twitter.com/nytfood}{\emph{NYT Food on
Twitter}} \emph{and}
\href{https://www.instagram.com/nytcooking/}{\emph{NYT Cooking on
Instagram}}\emph{,}
\href{https://www.facebookcorewwwi.onion/nytcooking/}{\emph{Facebook}}\emph{,}
\href{https://www.youtube.com/nytcooking}{\emph{YouTube}} \emph{and}
\href{https://www.pinterest.com/nytcooking/}{\emph{Pinterest}}\emph{.}
\href{https://www.nytimes3xbfgragh.onion/newsletters/cooking}{\emph{Get
regular updates from NYT Cooking, with recipe suggestions, cooking tips
and shopping advice}}\emph{.}

Advertisement

\protect\hyperlink{after-bottom}{Continue reading the main story}

\hypertarget{site-index}{%
\subsection{Site Index}\label{site-index}}

\hypertarget{site-information-navigation}{%
\subsection{Site Information
Navigation}\label{site-information-navigation}}

\begin{itemize}
\tightlist
\item
  \href{https://help.nytimes3xbfgragh.onion/hc/en-us/articles/115014792127-Copyright-notice}{©~2020~The
  New York Times Company}
\end{itemize}

\begin{itemize}
\tightlist
\item
  \href{https://www.nytco.com/}{NYTCo}
\item
  \href{https://help.nytimes3xbfgragh.onion/hc/en-us/articles/115015385887-Contact-Us}{Contact
  Us}
\item
  \href{https://www.nytco.com/careers/}{Work with us}
\item
  \href{https://nytmediakit.com/}{Advertise}
\item
  \href{http://www.tbrandstudio.com/}{T Brand Studio}
\item
  \href{https://www.nytimes3xbfgragh.onion/privacy/cookie-policy\#how-do-i-manage-trackers}{Your
  Ad Choices}
\item
  \href{https://www.nytimes3xbfgragh.onion/privacy}{Privacy}
\item
  \href{https://help.nytimes3xbfgragh.onion/hc/en-us/articles/115014893428-Terms-of-service}{Terms
  of Service}
\item
  \href{https://help.nytimes3xbfgragh.onion/hc/en-us/articles/115014893968-Terms-of-sale}{Terms
  of Sale}
\item
  \href{https://spiderbites.nytimes3xbfgragh.onion}{Site Map}
\item
  \href{https://help.nytimes3xbfgragh.onion/hc/en-us}{Help}
\item
  \href{https://www.nytimes3xbfgragh.onion/subscription?campaignId=37WXW}{Subscriptions}
\end{itemize}
