Sections

SEARCH

\protect\hyperlink{site-content}{Skip to
content}\protect\hyperlink{site-index}{Skip to site index}

\href{https://www.nytimes3xbfgragh.onion/section/health}{Health}

\href{https://myaccount.nytimes3xbfgragh.onion/auth/login?response_type=cookie\&client_id=vi}{}

\href{https://www.nytimes3xbfgragh.onion/section/todayspaper}{Today's
Paper}

\href{/section/health}{Health}\textbar{}A Parent's Toughest Call:
In-Person Schooling or Not?

\url{https://nyti.ms/2YU6plP}

\begin{itemize}
\item
\item
\item
\item
\item
\item
\end{itemize}

\hypertarget{school-reopenings}{%
\subsubsection{\texorpdfstring{\href{https://www.nytimes3xbfgragh.onion/spotlight/schools-reopening?name=styln-coronavirus-schools-reopening\&region=TOP_BANNER\&block=storyline_menu_recirc\&action=click\&pgtype=Article\&impression_id=d5d64730-f1c6-11ea-8605-93fb3e12e0cf\&variant=undefined}{School
Reopenings}}{School Reopenings}}\label{school-reopenings}}

\begin{itemize}
\tightlist
\item
  \href{https://www.nytimes3xbfgragh.onion/2020/09/04/us/bar-exam-coronavirus.html?name=styln-coronavirus-schools-reopening\&region=TOP_BANNER\&block=storyline_menu_recirc\&action=click\&pgtype=Article\&impression_id=d5d66e40-f1c6-11ea-8605-93fb3e12e0cf\&variant=undefined}{Delayed
  Licensing Exams}
\item
  \href{https://www.nytimes3xbfgragh.onion/interactive/2020/08/31/us/coronavirus-cases-children.html?name=styln-coronavirus-schools-reopening\&region=TOP_BANNER\&block=storyline_menu_recirc\&action=click\&pgtype=Article\&impression_id=d5d66e41-f1c6-11ea-8605-93fb3e12e0cf\&variant=undefined}{Rising
  Cases Among Children}
\item
  \href{https://www.nytimes3xbfgragh.onion/2020/09/01/world/schools-reopen-globe-students.html?name=styln-coronavirus-schools-reopening\&region=TOP_BANNER\&block=storyline_menu_recirc\&action=click\&pgtype=Article\&impression_id=d5d66e42-f1c6-11ea-8605-93fb3e12e0cf\&variant=undefined}{School
  Around the World}
\item
  \href{https://www.nytimes3xbfgragh.onion/interactive/2020/us/covid-college-cases-tracker.html?name=styln-coronavirus-schools-reopening\&region=TOP_BANNER\&block=storyline_menu_recirc\&action=click\&pgtype=Article\&impression_id=d5d66e43-f1c6-11ea-8605-93fb3e12e0cf\&variant=undefined}{Tracking
  College Cases}
\end{itemize}

Advertisement

\protect\hyperlink{after-top}{Continue reading the main story}

Supported by

\protect\hyperlink{after-sponsor}{Continue reading the main story}

\hypertarget{a-parents-toughest-call-in-person-schooling-or-not}{%
\section{A Parent's Toughest Call: In-Person Schooling or
Not?}\label{a-parents-toughest-call-in-person-schooling-or-not}}

Parents are wrestling with difficult choices over sending their children
to school. Here's how one science reporter made the decision.

\includegraphics{https://static01.graylady3jvrrxbe.onion/images/2020/08/28/science/00VIRUS-KIDS-EXPLAINER1/merlin_175828209_6a23c616-39a2-4586-9bf6-ae96f438665d-articleLarge.jpg?quality=75\&auto=webp\&disable=upscale}

By
\href{https://www.nytimes3xbfgragh.onion/by/apoorva-mandavilli}{Apoorva
Mandavilli}

\begin{itemize}
\item
  Sept. 1, 2020
\item
  \begin{itemize}
  \item
  \item
  \item
  \item
  \item
  \item
  \end{itemize}
\end{itemize}

All summer, as information about how the coronavirus affects children
has trickled in, I've been updating a balance sheet in my head. Every
study I read, every expert I talked to, was filling in columns on this
sheet: reasons for and against sending my children back to school come
September.

Into the con column went a study from Chicago that found children carry
\href{https://www.nytimes3xbfgragh.onion/2020/07/30/health/coronavirus-children.html}{large
amounts of virus} in their noses and throats, maybe even more than
adults do. Also in the con column: two South Korean studies, flawed as
they were, which suggested
\href{https://www.nytimes3xbfgragh.onion/2020/08/14/health/older-children-and-the-coronavirus-a-new-wrinkle-in-the-debate.html}{children
can spread the virus} to others --- and made me wonder whether my
sixth-grader, at least, should stay home.

Reports from Europe hinting that it was
\href{https://www.nytimes3xbfgragh.onion/2020/07/11/health/coronavirus-schools-reopen.html}{possible
to reopen schools safely} dribbled onto the pro side of my ledger. But
could we match those countries' careful precautions, or their low
community levels of virus?

I live in Brooklyn, where schools open after Labor Day (if they open
this year at all), so my husband and I have had more time than most
parents in the nation to make up our minds. We're also privileged enough
to have computers and reliable Wi-Fi for my children to learn remotely.

But as other parents called and texted to ask what I was planning to do,
I turned to the real experts: What do we know about the coronavirus and
children? And what should parents like me do?

The virus is so new that there are no definitive answers as yet, the
experts told me. Dozens of coronavirus studies emerge every day, ``but
it is not all good literature, and sorting out the wheat from the chaff
is challenging,'' said Dr. Megan Ranney, an expert in adolescent health
at Brown University.

But she and other experts were clear on one thing: Schools should only
reopen if the level of virus circulating in the community is low ---
that is, if less than 5 percent of people tested have a positive result.
By that measure,
\href{https://www.nytimes3xbfgragh.onion/2020/07/14/us/coronavirus-schools-fall.html}{most
school districts} in the nation cannot reopen
\href{https://www.nytimes3xbfgragh.onion/2020/08/12/us/georgia-school-coronavirus.html}{without
problems}.

``The No. 1 factor is what your local transmission is like,'' said Helen
Jenkins, an expert in infectious diseases and statistics at Boston
University. ``If you're in a really hard-hit part of the country, it's
highly likely that somebody coming into the school will be infected at
some point.''

On the questions of how often children become infected, how sick they
get and how much they contribute to community spread, the answers were
far more nuanced.

\hypertarget{fewer-children-than-adults-become-infected-but-childhood-infection-is-not-uncommon}{%
\subsection{Fewer children than adults become infected. But childhood
infection is not
uncommon.}\label{fewer-children-than-adults-become-infected-but-childhood-infection-is-not-uncommon}}

In the early days of the pandemic, there were so few reports of sick
children that it was unclear whether they could be infected at all.
Researchers guessed even then that younger children could probably catch
the coronavirus, but were mostly
\href{https://www.nytimes3xbfgragh.onion/2020/02/05/health/coronavirus-children.html}{spared
severe symptoms}.

That conjecture has proved correct. ``There is very clear evidence at
this point that kids can get infected,'' Dr. Ranney said.

As the pandemic unfolded, it also appeared that younger children were
less likely --- perhaps only half as likely --- to become infected,
compared with adults, whereas older children had about the same risk as
adults.

But it's impossible to be sure. In most countries hit hard by the
coronavirus, lockdowns and school shutdowns kept young children
cloistered at home and away from sources of infection. And when most of
those countries opened up, they did so with careful adherence to masks
and physical distancing.

\includegraphics{https://static01.graylady3jvrrxbe.onion/images/2020/08/28/science/00VIRUS-KIDS-EXPLAINER3/00VIRUS-KIDS-EXPLAINER3-articleLarge.jpg?quality=75\&auto=webp\&disable=upscale}

Children may turn out to be less at risk of becoming infected, ``but not
meaningfully different enough that I would take solace in it or use it
for decision making,'' said Dr. Ashish Jha, dean of the Brown University
School of Public Health.

In the United States, children under age 19 still represent just over 9
percent of all coronavirus cases. But the number of children infected
rose sharply this summer to nearly half a million, and the incidence
among children has risen much faster than it had been earlier this year.

``And those are just the kids that have been tested,'' said Dr. Leana
Wen, a former health commissioner of Baltimore. ``It's quite possible
that we're missing many cases of asymptomatic or mildly symptomatic
children.''

In the two-week period between Aug. 6 and Aug. 20, for example, the
number of children diagnosed in the United States jumped by 74,160, a 21
percent increase.

``Now that we're doing more community testing, we're seeing higher
proportions of children who are infected,'' Dr. Ranney said. ``I think
that our scientific knowledge on this is going to continue to shift.''

\hypertarget{children-do-become-sick-with-the-virus-but-deaths-are-very-rare}{%
\subsection{Children do become sick with the virus, but deaths are very
rare.}\label{children-do-become-sick-with-the-virus-but-deaths-are-very-rare}}

Even with the rising number of infections, the possibility that panics
parents the most --- that their children could become seriously ill or
even die from the virus --- is still reassuringly slim.

Children and adolescents up to age 20 (definitions and statistics vary
by state) represent
\href{https://services.aap.org/en/pages/2019-novel-coronavirus-covid-19-infections/children-and-covid-19-state-level-data-report/}{less
than 0.3 percent of deaths} related to the coronavirus, and 21 states
have reported no deaths at all among children.

``That remains the silver lining of this pandemic,'' Dr. Jha said.

But reports in adults increasingly suggest that death is not the only
severe outcome. Many adults seem to have debilitating symptoms for weeks
or months after they first fall ill.

``What percentage of kids who are infected have those long-term
consequences that we're increasingly worried about with adults?'' Dr.
Ranney wondered.

\href{https://www.nytimes3xbfgragh.onion/2020/05/19/parenting/pmis-coronavirus-children.html}{Multisystem
inflammatory syndrome}, a mysterious condition that has been linked to
the coronavirus, has also been reported in
\href{https://www.cdc.gov/mis-c/cases/index.html}{about 700 children}
and has caused 11 deaths as of Aug. 20. ``That's a very small percentage
of children,'' Dr. Ranney said. ``But growing numbers of kids are
getting hospitalized, period.''

\hypertarget{children-can-spread-the-virus-to-others-how-often-is-still-unknown}{%
\subsection{Children can spread the virus to others. How often is still
unknown.}\label{children-can-spread-the-virus-to-others-how-often-is-still-unknown}}

Transmission has been the most challenging aspect of the coronavirus to
discern in children, made even more difficult by the lockdowns that kept
them at home.

Because most children are asymptomatic, for example, household surveys
and studies that test people with symptoms often miss children who might
have seeded infections. And when schools are closed, young children
don't venture out; they tend to catch the virus from adults, rather than
the other way around.

To confirm the direction of spread, scientists ideally would genetically
sequence viral samples obtained from children to understand where and
when they were infected, and whether they passed it on.

\href{https://www.nytimes3xbfgragh.onion/spotlight/schools-reopening?action=click\&pgtype=Article\&state=default\&region=MAIN_CONTENT_3\&context=storylines_keepup}{}

\hypertarget{school-reopenings-}{%
\subsubsection{School Reopenings ›}\label{school-reopenings-}}

\hypertarget{back-to-school}{%
\paragraph{Back to School}\label{back-to-school}}

Updated Sept. 4, 2020

The latest on how schools are reopening amid the pandemic.

\begin{itemize}
\item
  \begin{itemize}
  \tightlist
  \item
    There have been at least
    \href{https://www.nytimes3xbfgragh.onion/interactive/2020/us/covid-college-cases-tracker.html?name=styln-coronavirus-schools-reopening\&action=click\&pgtype=Article\&state=default\&region=MAIN_CONTENT_3\&context=storylines_keepup\&region=TOP_BANNER█=storyline_menu_recirc\&action=click\&pgtype=Article\&impression_id=149dfe80-eea3-11ea-aea8-57f827c5e458\&variant=1_Show}{51,000
    coronavirus cases}~at more than 1,000 American college campuses
    since the pandemic began, the latest New York Times's survey shows.
  \item
    \href{https://www.nytimes3xbfgragh.onion/2020/09/03/nyregion/new-york-suny-oneonta-coronavirus.html?action=click\&pgtype=Article\&state=default\&region=MAIN_CONTENT_3\&context=storylines_keepup}{SUNY
    Oneonta}~canceled in-person classes and sent students home because
    of a coronavirus outbreak.
  \item
    \href{https://www.nytimes3xbfgragh.onion/2020/09/04/world/americas/latin-america-education.html?\&action=click\&pgtype=Article\&state=default\&region=MAIN_CONTENT_3\&context=storylines_keepup}{Millions
    of college students}~in Latin America are leaving their studies
    because of the pandemic.
  \item
    Professional licensing exams have been severely disrupted by the
    coronavirus, making it difficult for
    \href{https://www.nytimes3xbfgragh.onion/2020/09/04/us/bar-exam-coronavirus.html?action=click\&pgtype=Article\&state=default\&region=MAIN_CONTENT_3\&context=storylines_keepup}{newly
    trained lawyers, doctors}~and others to start their careers.
  \end{itemize}
\end{itemize}

``I keep saying to people, `It's so hard to study transmission --- it's
just really, really hard,''' Dr. Jenkins said.

Still, based on studies so far, ``I think it still appears that the
younger children might be less likely to transmit than older ones, and
older ones are probably more similar to adults in that regard,'' she
said.

Sadly, the high numbers of infected children in the United States may
actually provide some real data on this question as schools reopen.

Image

Children discussed their summer vacations on the first day of school in
Russia, Ohio, earlier this month.Credit...Luke Gronneberg/The Daily
News, via Associated Press

\hypertarget{so-whats-a-parent-to-do}{%
\subsection{So what's a parent to do?}\label{so-whats-a-parent-to-do}}

That's a tough one to answer, as parents everywhere now know. So much
depends on the particular circumstances of your school district, your
immediate community, your family and your child.

``I think it's a really complex decision, and we need to do everything
we can as a society to enable parents to make this type of decision,''
Dr. Wen said.

There are some precautions everyone can take --- beginning with doing as
much outdoors as possible, maintaining physical distance and wearing
masks.

``I will not send my children to school or to an indoor activity where
the children are not all masked,'' Dr. Ranney said.

Even if there is uncertainty about how often children become infected or
spread the virus, ``when you consider the risk versus benefit, the
balance lies in assuming that kids can both get infected and can spread
it,'' Dr. Ranney said.

For schools, the decision will also come down to having good ventilation
--- even if that's just windows that open --- small pods that can limit
how widely the virus might spread from an infected child, and frequent
testing to cut transmission chains.

Teachers and school nurses will also need protective equipment, Dr.
Jenkins said: ``Good P.P.E. makes all the difference, and school
districts must provide that for the teachers at an absolute minimum.''

As long as these right precautions are in place, ``it's better for kids
to be in school than outside of school,'' Dr. Jha said. ``Teachers are
reasonably safe in those environments, as well.''

But community transmission is the most important factor in deciding
whether children should go back to school, researchers agreed. ``We just
can't keep a school free from the coronavirus if the community is a
hotbed of infection,'' Dr. Wen said.

In New York, the numbers are low enough that my husband and I have a
real choice to make. And \href{https://www.communityroots.org/}{our
children's school}, with a focus on social equity, has said children of
frontline workers and those with disabilities will get the first spots
for in-person learning.

We qualified for two days a week of schooling in person. My
mother-in-law lives downstairs in a separate unit and may be more
vulnerable to the virus. But my children, who are 11 and 8, need to
learn in person and are desperate to see their friends. We've decided to
send our children back to school.

Advertisement

\protect\hyperlink{after-bottom}{Continue reading the main story}

\hypertarget{site-index}{%
\subsection{Site Index}\label{site-index}}

\hypertarget{site-information-navigation}{%
\subsection{Site Information
Navigation}\label{site-information-navigation}}

\begin{itemize}
\tightlist
\item
  \href{https://help.nytimes3xbfgragh.onion/hc/en-us/articles/115014792127-Copyright-notice}{©~2020~The
  New York Times Company}
\end{itemize}

\begin{itemize}
\tightlist
\item
  \href{https://www.nytco.com/}{NYTCo}
\item
  \href{https://help.nytimes3xbfgragh.onion/hc/en-us/articles/115015385887-Contact-Us}{Contact
  Us}
\item
  \href{https://www.nytco.com/careers/}{Work with us}
\item
  \href{https://nytmediakit.com/}{Advertise}
\item
  \href{http://www.tbrandstudio.com/}{T Brand Studio}
\item
  \href{https://www.nytimes3xbfgragh.onion/privacy/cookie-policy\#how-do-i-manage-trackers}{Your
  Ad Choices}
\item
  \href{https://www.nytimes3xbfgragh.onion/privacy}{Privacy}
\item
  \href{https://help.nytimes3xbfgragh.onion/hc/en-us/articles/115014893428-Terms-of-service}{Terms
  of Service}
\item
  \href{https://help.nytimes3xbfgragh.onion/hc/en-us/articles/115014893968-Terms-of-sale}{Terms
  of Sale}
\item
  \href{https://spiderbites.nytimes3xbfgragh.onion}{Site Map}
\item
  \href{https://help.nytimes3xbfgragh.onion/hc/en-us}{Help}
\item
  \href{https://www.nytimes3xbfgragh.onion/subscription?campaignId=37WXW}{Subscriptions}
\end{itemize}
