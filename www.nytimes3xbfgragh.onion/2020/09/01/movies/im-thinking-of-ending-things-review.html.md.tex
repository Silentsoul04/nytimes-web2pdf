Sections

SEARCH

\protect\hyperlink{site-content}{Skip to
content}\protect\hyperlink{site-index}{Skip to site index}

\href{https://www.nytimes3xbfgragh.onion/section/movies}{Movies}

\href{https://myaccount.nytimes3xbfgragh.onion/auth/login?response_type=cookie\&client_id=vi}{}

\href{https://www.nytimes3xbfgragh.onion/section/todayspaper}{Today's
Paper}

\href{/section/movies}{Movies}\textbar{}`I'm Thinking of Ending Things'
Review: Where to Begin?

\url{https://nyti.ms/2QK4kUV}

\begin{itemize}
\item
\item
\item
\item
\item
\end{itemize}

Advertisement

\protect\hyperlink{after-top}{Continue reading the main story}

Supported by

\protect\hyperlink{after-sponsor}{Continue reading the main story}

critic's pick

\hypertarget{im-thinking-of-ending-things-review-where-to-begin}{%
\section{`I'm Thinking of Ending Things' Review: Where to
Begin?}\label{im-thinking-of-ending-things-review-where-to-begin}}

Jessie Buckley and Jesse Plemons play a couple on a trip to some very
odd places in Charlie Kaufman's latest film.

\includegraphics{https://static01.graylady3jvrrxbe.onion/images/2020/09/01/arts/00thinking/merlin_176250954_761469ea-43e1-4f80-b578-30f8c56e0fc8-articleLarge.jpg?quality=75\&auto=webp\&disable=upscale}

\href{https://www.nytimes3xbfgragh.onion/by/a-o--scott}{\includegraphics{https://static01.graylady3jvrrxbe.onion/images/2018/02/20/multimedia/author-a-o-scott/author-a-o-scott-thumbLarge.jpg}}

By \href{https://www.nytimes3xbfgragh.onion/by/a-o--scott}{A.O. Scott}

\begin{itemize}
\item
  Published Sept. 1, 2020Updated Sept. 4, 2020
\item
  \begin{itemize}
  \item
  \item
  \item
  \item
  \item
  \end{itemize}
\end{itemize}

\begin{itemize}
\tightlist
\item
  I'm Thinking of Ending Things\\
  **NYT Critic's Pick Directed by Charlie Kaufman Drama, Horror,
  Thriller R 2h 14m
\end{itemize}

\href{https://www.imdb.com/showtimes/title/tt7939766?ref_=ref_ext_NYT}{Find
Tickets}

When you purchase a ticket for an independently reviewed film through
our site, we earn an affiliate commission.

When I accepted the assignment to review Charlie Kaufman's new movie,
``I'm Thinking of Ending Things,'' I vowed that I would avoid the
recursive, self-conscious, Kaufmanesque flourishes that afflict so much
writing about this screenwriter and filmmaker. What follows is the
record of my abject failure to live up to that promise.

In my defense: He made me do it. Exercising professional due diligence
--- in other words, seizing an opportunity to procrastinate on deadline
--- I acquired a copy of
\href{https://www.nytimes3xbfgragh.onion/2020/07/08/books/review/antkind-charlie-kaufman.html}{``Antkind,''}Kaufman's
recently published novel, only to discover that I'm a minor character in
it. A few hundred pages after faintly praising me as ``a nice enough
fellow and I'm sure a very smart guy for a hack,'' the book's narrator
(a quondam critic with nothing nice to say about Charlie Kaufman)
challenges me to a barroom argument about cinema. I barely get a word in
edgewise, and in the wake of his ``vanquishment of A.O. Scott,'' my
fictional nemesis makes a bold prediction: ``Never will he write again.
Of that I am certain.''

I would like to think I am right this minute proving him wrong, but I'm
not so sure. What is certain is that Kaufman (whom I've met a couple of
times at film festivals) is living in my head, as I seem to be living in
his. And so, whether I like it or not --- and to be honest, I don't
really mind --- I find myself ensnared in a low-key version of one of
his favorite predicaments.

At least since ``Being John Malkovich,'' in which various schemers,
dreamers and paying customers literally inhabit the consciousness of
Malkovich, Kaufman has explored the philosophical vertigo and emotional
upset caused by the inconvenient fact that other people exist. Again and
again, his movies ask: Are we even real to one another, or does each of
us project inner desires and anxieties outward, turning the faces and
feelings of lovers, colleagues and family members into mirrors of our
own narcissism?

Often --- notably in ``Eternal Sunshine of the Spotless Mind,'' in
\href{https://www.nytimes3xbfgragh.onion/2015/12/30/movies/review-anomalisa-pairs-charlie-kaufman-and-lonely-puppets.html}{``Anomalisa''}and
now in ``I'm Thinking of Ending Things'' --- that question arises in,
and threatens to spoil, a heterosexual romance. Men, in particular, have
a habit of confusing the objects of their fantasies with the real women
in front of them. This can be funny, creepy, sad, toxic or sweet,
sometimes all at once.

In ``I'm Thinking of Ending Things,'' the effects arrive before our
understanding of their causes. We know what we're feeling, but we don't
know why. As far as we can guess, we are in the head of a woman named
Lucy (Jessie Buckley), who is taking a car trip with her boyfriend, Jake
(Jesse Plemons). ``You can't fake a thought,'' Lucy muses to herself,
and one of her thoughts is summed up in the movie's title. She and Jake
haven't been dating that long, and she doesn't see much of a future for
them. Does Jake somehow know what she's thinking? He startles her from
time to time by seeming to read her mind, which seems to keep changing.

It's not the only thing that does. As the couple makes their way through
a snowstorm toward the farm where Jake's parents live, little
inconsistencies pop up, mostly about Lucy's interests and background.
One minute, she says she has no interest in poetry and the next she is
reciting a heart-rending lyric she claims to have written herself. She
is variously said to be studying physics, or painting, or gerontology.
Her peacoat is pink, until it is blue. Her name might not even be Lucy.

Once she and Jake are out of the car, the weirdness accelerates. Jake's
mother and father (Toni Collette and David Thewlis) grow older and
younger each time they leave the room. Their awkward, high-strung table
talk is interrupted from time to time by scenes of an old, lonely school
custodian making his rounds, a character whose connection to Jake and
his family is implied but not spelled out.

Until the end, that is, but even then maybe not quite. ``I'm Thinking of
Ending Things'' is based
on\href{https://www.nytimes3xbfgragh.onion/2016/09/04/books/review/im-thinking-of-ending-things-iain-reid.html}{a
novel by the Canadian writer Iain Reid}, a spare and elusive story that
provides Kaufman with a stable enough trellis for his own florid
preoccupations. The film is suspenseful because it generates uncertainty
about its own premises, and because the movements of the camera, the
strangeness of Molly Hughes's production design and the tremors of Jay
Wadley's musical score guide the viewer toward dread. Lucy is often
puzzled, sometimes curious, but maybe not as afraid as she should be.
Unless, that is, her perspective isn't one we should trust. Maybe she
\emph{is} faking her thoughts.

Or at least borrowing them. Kaufman's dialogue is larded with passages
that sound like quotations, only a few of them attributed. Jake
helpfully --- or pompously --- informs Lucy when he's quoting Oscar
Wilde or David Foster Wallace. But at other moments, you may find
yourself tempted to pause the movie
(\href{https://www.netflix.com/title/80211559}{which is streaming on
Netflix}) so you can Google what you just heard, thus discovering (for
example) that Lucy's lengthy, wised-up critique of John Cassavetes's ``A
Woman Under the Influence'' is lifted verbatim from
\href{https://archives.newyorker.com/newyorker/1974-12-09/flipbook/178/}{Pauline
Kael's review of that movie}. A visual clue of sorts has been provided
by the appearance in an earlier scene of a copy of Kael's collection
``For Keeps.'' The weird thing is that the ``Woman Under the Influence''
review doesn't appear in the book.

An annotated version of ``I'm Thinking of Ending Things'' might be nice
to have, though it might also undermine the sense of knowingness that is
both one of the film's minor pleasures and one of its major
provocations. Jake, who is defensive about David Foster Wallace and
oblivious to the rapeyness of the
song\href{https://www.vox.com/identities/2016/12/19/13885552/baby-its-cold-outside-feminist-date-rape-romantic}{``Baby
It's Cold Outside,''} is a guy with a clear need to know, explain and
control things.

He's proud of how smart he is, though also a little ashamed that he won
a medal in school for ``diligence'' rather than ``acumen.'' (His mother
couldn't be prouder.) When Lucy makes an offhand reference to Mussolini
making the trains run on time, Jake is quick to point out that
improvements in Italian rail service actually predated the fascist
dictatorship. His behavior toward her --- his moodiness, his evasive
answers to her questions, his passive-aggressive efforts to shut her
down --- is increasingly alarming, even as it is also the most
consistently realistic aspect of the film.

Much of the second half takes place against the backdrop of a howling
nighttime blizzard, an almost too-perfect metaphor. ``Anomalisa'' partly
camouflaged its melancholy cynicism in the absurdist whimsy of R-rated
stop-motion animation. ``I'm Thinking of Ending Things'' has some of its
own flights of inventiveness and fantasy --- a ballet sequence, a
satirical poke at Robert Zemeckis, a couple of songs from ``Oklahoma,''
a curious homage to
\href{https://www.youtube.com/watch?v=H7Y_mHNgnpA}{``A Beautiful
Mind}''--- but they always land in the same dark and lonely place.

That place is at once vividly cinematic --- this is Kaufman's most
assured and daring work so far as a director --- and deeply suspicious
of the power of movies to infect our minds with meretricious and
misleading ideas. Both Jake and Lucy at times share this suspicion, and
both of them can be seen as victims of the art form that has summoned
them into being. Plemons and especially Buckley play this somewhat
abstract conundrum for real existential stakes, either tricking you into
caring about them or sincerely expressing the need to be cared about.

I was sometimes puzzled and sometimes annoyed by their story, and by the
other possible stories in which they are embedded, but I was also moved.
More evidence that I'm a hack, for sure, but who am I to argue?

\textbf{I'm Thinking of Ending Things}\\
Rated R. Baby, you'll freeze out there. Running time: 2 hours 14
minutes. \href{https://www.netflix.com/title/80211559}{Watch on
Netflix.}

Advertisement

\protect\hyperlink{after-bottom}{Continue reading the main story}

\hypertarget{site-index}{%
\subsection{Site Index}\label{site-index}}

\hypertarget{site-information-navigation}{%
\subsection{Site Information
Navigation}\label{site-information-navigation}}

\begin{itemize}
\tightlist
\item
  \href{https://help.nytimes3xbfgragh.onion/hc/en-us/articles/115014792127-Copyright-notice}{©~2020~The
  New York Times Company}
\end{itemize}

\begin{itemize}
\tightlist
\item
  \href{https://www.nytco.com/}{NYTCo}
\item
  \href{https://help.nytimes3xbfgragh.onion/hc/en-us/articles/115015385887-Contact-Us}{Contact
  Us}
\item
  \href{https://www.nytco.com/careers/}{Work with us}
\item
  \href{https://nytmediakit.com/}{Advertise}
\item
  \href{http://www.tbrandstudio.com/}{T Brand Studio}
\item
  \href{https://www.nytimes3xbfgragh.onion/privacy/cookie-policy\#how-do-i-manage-trackers}{Your
  Ad Choices}
\item
  \href{https://www.nytimes3xbfgragh.onion/privacy}{Privacy}
\item
  \href{https://help.nytimes3xbfgragh.onion/hc/en-us/articles/115014893428-Terms-of-service}{Terms
  of Service}
\item
  \href{https://help.nytimes3xbfgragh.onion/hc/en-us/articles/115014893968-Terms-of-sale}{Terms
  of Sale}
\item
  \href{https://spiderbites.nytimes3xbfgragh.onion}{Site Map}
\item
  \href{https://help.nytimes3xbfgragh.onion/hc/en-us}{Help}
\item
  \href{https://www.nytimes3xbfgragh.onion/subscription?campaignId=37WXW}{Subscriptions}
\end{itemize}
