Sections

SEARCH

\protect\hyperlink{site-content}{Skip to
content}\protect\hyperlink{site-index}{Skip to site index}

\href{https://www.nytimes3xbfgragh.onion/section/nyregion}{New York}

\href{https://myaccount.nytimes3xbfgragh.onion/auth/login?response_type=cookie\&client_id=vi}{}

\href{https://www.nytimes3xbfgragh.onion/section/todayspaper}{Today's
Paper}

\href{/section/nyregion}{New York}\textbar{}9 Major N.Y.C. Roads Get
Lower Speed Limits as Traffic Deaths Surge

\begin{itemize}
\item
\item
\item
\item
\item
\end{itemize}

\hypertarget{the-coronavirus-outbreak}{%
\subsubsection{\texorpdfstring{\href{https://www.nytimes3xbfgragh.onion/news-event/coronavirus?name=styln-coronavirus-national\&region=TOP_BANNER\&block=storyline_menu_recirc\&action=click\&pgtype=Article\&impression_id=2879ff80-f4c5-11ea-96bf-bf14b5b9665f\&variant=undefined}{The
Coronavirus
Outbreak}}{The Coronavirus Outbreak}}\label{the-coronavirus-outbreak}}

\begin{itemize}
\tightlist
\item
  live\href{https://www.nytimes3xbfgragh.onion/2020/09/11/world/covid-19-coronavirus.html?name=styln-coronavirus-national\&region=TOP_BANNER\&block=storyline_menu_recirc\&action=click\&pgtype=Article\&impression_id=287a2690-f4c5-11ea-96bf-bf14b5b9665f\&variant=undefined}{Latest
  Updates}
\item
  \href{https://www.nytimes3xbfgragh.onion/interactive/2020/us/coronavirus-us-cases.html?name=styln-coronavirus-national\&region=TOP_BANNER\&block=storyline_menu_recirc\&action=click\&pgtype=Article\&impression_id=287a2691-f4c5-11ea-96bf-bf14b5b9665f\&variant=undefined}{Maps
  and Cases}
\item
  \href{https://www.nytimes3xbfgragh.onion/interactive/2020/science/coronavirus-vaccine-tracker.html?name=styln-coronavirus-national\&region=TOP_BANNER\&block=storyline_menu_recirc\&action=click\&pgtype=Article\&impression_id=287a2692-f4c5-11ea-96bf-bf14b5b9665f\&variant=undefined}{Vaccine
  Tracker}
\item
  \href{https://www.nytimes3xbfgragh.onion/2020/09/10/us/politics/fda-coronavirus-vaccine.html?name=styln-coronavirus-national\&region=TOP_BANNER\&block=storyline_menu_recirc\&action=click\&pgtype=Article\&impression_id=287a2693-f4c5-11ea-96bf-bf14b5b9665f\&variant=undefined}{F.D.A.
  Regulators' Self-Defense}
\item
  \href{https://www.nytimes3xbfgragh.onion/2020/09/09/upshot/coronavirus-surprise-test-fees.html?name=styln-coronavirus-national\&region=TOP_BANNER\&block=storyline_menu_recirc\&action=click\&pgtype=Article\&impression_id=287a2694-f4c5-11ea-96bf-bf14b5b9665f\&variant=undefined}{Surprise
  Test Fees}
\end{itemize}

Advertisement

\protect\hyperlink{after-top}{Continue reading the main story}

Supported by

\protect\hyperlink{after-sponsor}{Continue reading the main story}

\hypertarget{9-major-nyc-roads-get-lower-speed-limits-as-traffic-deaths-surge}{%
\section{9 Major N.Y.C. Roads Get Lower Speed Limits as Traffic Deaths
Surge}\label{9-major-nyc-roads-get-lower-speed-limits-as-traffic-deaths-surge}}

More drivers, passengers and motorcyclists have been killed so far this
year than all of last year. Reckless driving is a major factor,
officials say.

\includegraphics{https://static01.graylady3jvrrxbe.onion/images/2020/09/01/nyregion/01nycarcrashes/merlin_176420667_369d9a5e-0573-44d3-8d78-338343bf79a9-articleLarge.jpg?quality=75\&auto=webp\&disable=upscale}

\href{https://www.nytimes3xbfgragh.onion/by/christina-goldbaum}{\includegraphics{https://static01.graylady3jvrrxbe.onion/images/2019/11/22/reader-center/author-christina-goldbaum/author-christina-goldbaum-thumbLarge.png}}

By
\href{https://www.nytimes3xbfgragh.onion/by/christina-goldbaum}{Christina
Goldbaum}

\begin{itemize}
\item
  Sept. 1, 2020
\item
  \begin{itemize}
  \item
  \item
  \item
  \item
  \item
  \end{itemize}
\end{itemize}

When the pandemic emptied New York of its usual traffic, the city's
streets transformed into an open speedway where drivers drag raced down
major roads, racked up thousands of tickets and in some cases left fatal
wreckage in their wake.

At the time, city officials saw the rash of reckless driving as an
aberration that would vanish when the city's usual traffic reappeared.

But as restrictions lifted this summer and traffic crept back toward
pre-pandemic levels, the spate of speeding --- and fatal collisions ---
did not end.

Now, alarmed by the sustained rise in fatalities and bracing for the
possibility of
\href{https://www.nytimes3xbfgragh.onion/2020/08/17/nyregion/coronavirus-second-wave-nyc.html}{a
second lockdown} that could worsen the current speeding crisis, city
officials are reducing speed limits by five miles per hour on nine of
the most dangerous streets across the five boroughs.

On Tuesday, officials will announce the speed limit will be lowered to
25 miles per hour --- the standard limit on most the city's roadways ---
on eight of those streets, including parts of Riverside Drive in
Manhattan, Flatbush Avenue in Brooklyn, Northern Boulevard in Queens and
Bruckner Boulevard in the Bronx.

The limit will also drop to 25 miles per hour on Shore Parkway Service
Road and Dahlgren Place in Brooklyn, Webster Avenue in the Bronx and
Targee Street in Staten Island.

On the ninth, Rockaway Boulevard in Queens, the limit will drop from 40
miles per hour to 35.

``People got in the habit of driving too fast and too recklessly when
roads were more open, and unfortunately, we're still seeing that
behavior,'' Polly Trottenberg, the city's transportation commissioner,
said in an interview. ``We're starting to get almost back to normal, but
there are still times and places in the city where traffic levels are
lower and drivers are able to get up to higher speeds.''

\hypertarget{latest-updates-the-coronavirus-outbreak}{%
\section{\texorpdfstring{\href{https://www.nytimes3xbfgragh.onion/2020/09/11/world/covid-19-coronavirus.html?action=click\&pgtype=Article\&state=default\&region=MAIN_CONTENT_1\&context=storylines_live_updates}{Latest
Updates: The Coronavirus
Outbreak}}{Latest Updates: The Coronavirus Outbreak}}\label{latest-updates-the-coronavirus-outbreak}}

Updated 2020-09-12T06:16:33.399Z

\begin{itemize}
\tightlist
\item
  \href{https://www.nytimes3xbfgragh.onion/2020/09/11/world/covid-19-coronavirus.html?action=click\&pgtype=Article\&state=default\&region=MAIN_CONTENT_1\&context=storylines_live_updates\#link-dfb8a16}{Fauci
  cautions the virus could disrupt life in the U.S. until `maybe even
  towards the end of 2021.'}
\item
  \href{https://www.nytimes3xbfgragh.onion/2020/09/11/world/covid-19-coronavirus.html?action=click\&pgtype=Article\&state=default\&region=MAIN_CONTENT_1\&context=storylines_live_updates\#link-7104d154}{From
  Asia to Africa, China promotes its vaccine candidates to win friends.}
\item
  \href{https://www.nytimes3xbfgragh.onion/2020/09/11/world/covid-19-coronavirus.html?action=click\&pgtype=Article\&state=default\&region=MAIN_CONTENT_1\&context=storylines_live_updates\#link-393ad215}{The
  other way the virus will kill: hunger.}
\end{itemize}

\href{https://www.nytimes3xbfgragh.onion/2020/09/11/world/covid-19-coronavirus.html?action=click\&pgtype=Article\&state=default\&region=MAIN_CONTENT_1\&context=storylines_live_updates}{See
more updates}

More live coverage:
\href{https://www.nytimes3xbfgragh.onion/live/2020/09/11/business/stock-market-today-coronavirus?action=click\&pgtype=Article\&state=default\&region=MAIN_CONTENT_1\&context=storylines_live_updates}{Markets}

Already this year, more passengers, drivers and motorcyclists have been
killed in car crashes than all of last year: 28 drivers, 16 passengers
and 26 motorcyclists have died, according to city data.

In June, when traffic in New York City returned to around 80 percent of
pre-pandemic levels, the number of passengers and drivers killed in
collisions jumped 22 percent compared with the same month last year,
according to data from the city and INRIX, a data collection firm. In
July, things got much worse: Those deaths spiked 300 percent compared
with last year.

Since Mayor Bill de Blasio vowed to eliminate all traffic deaths six
years ago, his administration has lowered speed limits and enforced them
with automated speed cameras, bringing traffic deaths to their lowest
level in a century in 2018.

City officials have tried to crack down on reckless driving by
installing 60 new automated speed cameras every month since the
beginning of the year, bringing the total to nearly 1,000. The police
department also
\href{https://www.nytimes3xbfgragh.onion/2020/04/16/nyregion/coronavirus-nyc-speeding.html}{increased
speed-radar enforcement} along some highways and deployed hundreds of
officers to locations with many speeding drivers at the height of the
lockdown.

But with the city's sprawling subway system
\href{https://www.nytimes3xbfgragh.onion/2020/08/26/nyregion/nyc-subway-bus-service-cuts.html}{facing
looming cuts} and New Yorkers buying bicycles, scooters and cars in
record numbers, many transit experts say that Mr. de Blasio needs to
take more drastic action --- like accelerating the creation of new
busways and protected bike lanes, and restricting traffic into Manhattan
during rush hours --- to ensure streets are safe and functional.

``New York City is facing four existential challenges: the death spiral
of public transit, ballooning car ownership, an increase in traffic
deaths and serious injuries and the lack of a plan for addressing these
from the mayor,'' said Danny Harris, executive director of
Transportation Alternatives, an advocacy group.

``The decisions we're making now about street infrastructure will affect
us for decades,'' Mr. Harris added. ``We should be taking decisive
action now.''

Across the country, after the pandemic hit and traffic levels dropped
more than 90 percent in some large cities, speeding --- and the death
rate from car crashes --- surged.

In March, the rate of fatalities nationwide from crashes rose 12
percent, in May it jumped 34 percent and in June --- the latest month
when statistics are available --- it rose 23 percent compared with the
same months last year, according to the National Safety Council, an
advocacy group.

``I think folks started to feel like the roads are emptier and it's an
open speedway for them,'' said Lorraine M. Martin, president of the
council. The empty roads may have also lured some drivers into a false
sense of security, she added, leading them to ignore laws that mandate
wearing a seatbelt and not driving impaired.

But in New York, the sustained rise in fatalities suggests that drivers
who picked up reckless behavior during the lockdown have maintained it
since --- and may continue to imperil street safety.

In March, when the city entered lockdown, the number of automated speed
camera violations nearly doubled compared to the previous month, from
12,672 tickets issued one day at the end of February to 24,765 tickets
issued daily at the end of March.

When the city emerged from lockdown in June, the violations continued:
There were 23,951 issued on the final Friday in July.

Many of the fatal crashes have happened on highways, where drivers can
speed more easily when there is little traffic, city officials said. The
motorcycles involved in some crashes were often unregistered or had
expired registration, and many drivers did not have a
motorcycle-specific license.

At the same time, fewer pedestrians and cyclists have been killed in
collisions this year compared with the same period last year. So far,
there have been 19 fewer pedestrian fatalities and 10 fewer cyclists
killed compared with last year, which was
\href{https://www.nytimes3xbfgragh.onion/2020/01/01/nyregion/nyc-biking-deaths.html}{particularly
deadly for cyclists}.

Still, many transit advocates say that the city needs to take more
aggressive steps to prevent future gridlock and restore recent gains to
street safety as the city braces for uncertain travel patterns in the
wake of the pandemic.

``Without putting in more street-level infrastructure for buses, bikes
and micromobility, we are looking at a confluence of more people using
scooters and bikes and a lot more people driving on streets,'' said Nick
Sifuentes, the executive director of Tri-State Transportation Campaign,
an advocacy group. ``That's a recipe for even more crashes and even more
fatalities than we are seeing right now.''

In June, Mr. de Blasio convened a panel of transportation experts to
make recommendations for avoiding gridlock and maintaining street
safety.

But those experts say that the mayor has yet to review or act on those
recommendations, stoking concerns that without a comprehensive plan to
accommodate shifting commuting patterns, New York's recovery could be
hamstrung by traffic deaths and gridlock.

``Recovering from Covid-19 requires us to reimagine our city, especially
our streets, for the better,'' half of the experts on the panel
\href{https://medium.com/@TransAlt/an-open-letter-to-mayor-de-blasio-from-the-surface-transportation-advisory-council-5a75ae2559ce?source=messageShare-e4ed91e3b0ac-1598967953\&_branch_match_id=618921677228910997}{wrote
in a letter to Mr. de Blasio} last week. ``Without your decisive and
immediate action, we may lose New York City's future to growing
congestion, pollution, inequality, and traffic violence.''

City officials say that, despite budget cuts, they have made strides to
address these issues: This summer, the city broke ground on 20 miles of
new busways, rolled out tens of miles of open streets and opened roads
to several thousands of restaurants.

``The city has had a lot on its plate,'' said Ms. Trottenberg, the
transportation commissioner. ``I hear frustrations from people who say
we could be doing more, but I think we've done a lot to respond to this
urgency.''

Advertisement

\protect\hyperlink{after-bottom}{Continue reading the main story}

\hypertarget{site-index}{%
\subsection{Site Index}\label{site-index}}

\hypertarget{site-information-navigation}{%
\subsection{Site Information
Navigation}\label{site-information-navigation}}

\begin{itemize}
\tightlist
\item
  \href{https://help.nytimes3xbfgragh.onion/hc/en-us/articles/115014792127-Copyright-notice}{©~2020~The
  New York Times Company}
\end{itemize}

\begin{itemize}
\tightlist
\item
  \href{https://www.nytco.com/}{NYTCo}
\item
  \href{https://help.nytimes3xbfgragh.onion/hc/en-us/articles/115015385887-Contact-Us}{Contact
  Us}
\item
  \href{https://www.nytco.com/careers/}{Work with us}
\item
  \href{https://nytmediakit.com/}{Advertise}
\item
  \href{http://www.tbrandstudio.com/}{T Brand Studio}
\item
  \href{https://www.nytimes3xbfgragh.onion/privacy/cookie-policy\#how-do-i-manage-trackers}{Your
  Ad Choices}
\item
  \href{https://www.nytimes3xbfgragh.onion/privacy}{Privacy}
\item
  \href{https://help.nytimes3xbfgragh.onion/hc/en-us/articles/115014893428-Terms-of-service}{Terms
  of Service}
\item
  \href{https://help.nytimes3xbfgragh.onion/hc/en-us/articles/115014893968-Terms-of-sale}{Terms
  of Sale}
\item
  \href{https://spiderbites.nytimes3xbfgragh.onion}{Site Map}
\item
  \href{https://help.nytimes3xbfgragh.onion/hc/en-us}{Help}
\item
  \href{https://www.nytimes3xbfgragh.onion/subscription?campaignId=37WXW}{Subscriptions}
\end{itemize}
