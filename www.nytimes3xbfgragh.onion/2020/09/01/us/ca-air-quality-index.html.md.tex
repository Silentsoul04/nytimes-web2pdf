Sections

SEARCH

\protect\hyperlink{site-content}{Skip to
content}\protect\hyperlink{site-index}{Skip to site index}

\href{https://www.nytimes3xbfgragh.onion/section/us}{U.S.}

\href{https://myaccount.nytimes3xbfgragh.onion/auth/login?response_type=cookie\&client_id=vi}{}

\href{https://www.nytimes3xbfgragh.onion/section/todayspaper}{Today's
Paper}

\href{/section/us}{U.S.}\textbar{}How to Read the Air Quality Index

\url{https://nyti.ms/3lCJ9Cu}

\begin{itemize}
\item
\item
\item
\item
\item
\end{itemize}

\hypertarget{wildfires-in-the-west}{%
\subsubsection{\texorpdfstring{\href{https://www.nytimes3xbfgragh.onion/spotlight/california-wildfires?name=styln-california-wildfires\&region=TOP_BANNER\&block=storyline_menu_recirc\&action=click\&pgtype=Article\&impression_id=6dc09d40-f52b-11ea-9f60-c1748b5f940c\&variant=undefined}{Wildfires
in the West}}{Wildfires in the West}}\label{wildfires-in-the-west}}

\begin{itemize}
\tightlist
\item
  live\href{https://www.nytimes3xbfgragh.onion/2020/09/12/us/wildfires-live-updates.html?name=styln-california-wildfires\&region=TOP_BANNER\&block=storyline_menu_recirc\&action=click\&pgtype=Article\&impression_id=6dc09d41-f52b-11ea-9f60-c1748b5f940c\&variant=undefined}{Fires
  Updates}
\item
  \href{https://www.nytimes3xbfgragh.onion/interactive/2020/us/fires-map-tracker.html?name=styln-california-wildfires\&region=TOP_BANNER\&block=storyline_menu_recirc\&action=click\&pgtype=Article\&impression_id=6dc09d42-f52b-11ea-9f60-c1748b5f940c\&variant=undefined}{Maps
  of the Fires}
\item
  \href{https://www.nytimes3xbfgragh.onion/article/wildfires-photos-california-oregon-washington-state.html?name=styln-california-wildfires\&region=TOP_BANNER\&block=storyline_menu_recirc\&action=click\&pgtype=Article\&impression_id=6dc09d43-f52b-11ea-9f60-c1748b5f940c\&variant=undefined}{Photos}
\item
  \href{https://www.nytimes3xbfgragh.onion/2020/09/10/us/climate-change-california-wildfires.html?name=styln-california-wildfires\&region=TOP_BANNER\&block=storyline_menu_recirc\&action=click\&pgtype=Article\&impression_id=6dc09d44-f52b-11ea-9f60-c1748b5f940c\&variant=undefined}{A
  Climate Reckoning}
\item
  \href{https://www.nytimes3xbfgragh.onion/article/wildfires-california-oregon-washington.html?name=styln-california-wildfires\&region=TOP_BANNER\&block=storyline_menu_recirc\&action=click\&pgtype=Article\&impression_id=6dc09d45-f52b-11ea-9f60-c1748b5f940c\&variant=undefined}{Answers
  to Your Questions}
\item
  \href{https://www.nytimes3xbfgragh.onion/2020/09/09/us/california-wildfires.html?name=styln-california-wildfires\&region=TOP_BANNER\&block=storyline_menu_recirc\&action=click\&pgtype=Article\&impression_id=6dc09d46-f52b-11ea-9f60-c1748b5f940c\&variant=undefined}{Newsletter}
\end{itemize}

Advertisement

\protect\hyperlink{after-top}{Continue reading the main story}

Supported by

\protect\hyperlink{after-sponsor}{Continue reading the main story}

California Today

\hypertarget{how-to-read-the-air-quality-index}{%
\section{How to Read the Air Quality
Index}\label{how-to-read-the-air-quality-index}}

Here's what you need to know about reading daily pollution levels.

\href{https://www.nytimes3xbfgragh.onion/by/marie-tae-mcdermott}{\includegraphics{https://static01.graylady3jvrrxbe.onion/images/2018/11/26/multimedia/author-marie-tae-mcdermott/author-marie-tae-mcdermott-thumbLarge.png}}

By
\href{https://www.nytimes3xbfgragh.onion/by/marie-tae-mcdermott}{Marie
Tae McDermott}

\begin{itemize}
\item
  Published Sept. 1, 2020Updated Sept. 9, 2020
\item
  \begin{itemize}
  \item
  \item
  \item
  \item
  \item
  \end{itemize}
\end{itemize}

\includegraphics{https://static01.graylady3jvrrxbe.onion/images/2020/09/06/us/01californiatoday-1/merlin_176041638_f124532b-64c5-440a-8878-39e3050b095e-articleLarge.jpg?quality=75\&auto=webp\&disable=upscale}

\emph{Good morning.}

Over the past few weeks, hundreds of fires in California have
incinerated more than \href{https://www.fire.ca.gov/incidents/}{a
million and a half acres}, destroyed over 3,000 structures and taken
seven lives.

And it's still early in the wildfire season.

My colleague Thomas Fuller
\href{https://www.nytimes3xbfgragh.onion/2020/08/24/us/california-fires-wildfires.html}{recently
captured the mood in the Bay Area}, a region engulfed in fire and choked
by smokey haze four years in a row. ``First there was the coronavirus,
then the threat of fires and power outages, and now the smoke,'' he
wrote.

According to the \href{https://ww2.arb.ca.gov/}{California Air Resources
Board}, the state agency tasked with tracking and maintaining healthy
air quality throughout the state, one quarter of California is
classified as under very high or extreme fire threat and more than 25
percent of the state's population lives in these high fire-risk areas.
Even more people are vulnerable to plumes of smoke that can travel long
distances and hover over populated areas.

Understanding the quality of the air we breathe is vital and the
information is readily available at our fingertips, according to a group
of experts I spoke to from the agency.

Here are four takeaways from our conversation.

\textbf{We are collecting more information on air quality than ever
before.} According to CARB, for 50 years California has maintained one
of the most extensive air-monitoring networks in the world. There are
over 250 monitoring sites in the state alone.

To get the latest pollution readings, the board recommends using
\href{https://www.airnow.gov/}{AirNow}, a website and app run by the
Environmental Protection Agency. AirNow has
\href{https://fire.airnow.gov/?lat=38.5712128\&lng=-121.4021632\&zoom=10}{a
separate fire and smoke map} that uses portable sensors to track smoke
plumes as they travel around the state. If you don't have access to the
internet, local newspapers also publish general air quality levels for
that day. Since this article was published, several readers also
recommended \href{https://www.purpleair.com/map\#1/14.8/-30}{PurpleAir},
a worldwide map showing live air quality readings. PurpleAir uses a
network of inexpensive sensors that are purchased and installed by
members of the community.

AirNow and PurpleAir use the Air Quality Index, or A.Q.I., a measure of
air quality from 0 to 500 that is composed of six categories, from
``good'' to ``hazardous.'' An A.Q.I. of 50 or below represents good air
quality, while an A.Q.I. of over 300 represents hazardous air quality
that is unhealthy for any person to breathe.

\textbf{Check the index daily, the way you would check the weather.}
It's generally a good idea to check the index before engaging in outdoor
activities, said Webster Tasat, a manager from CARB's air quality
planning and science division.

People should also think about whether they fall into a sensitive group.
``People who are in sensitive groups are especially vulnerable to air
pollution,'' said Bonnie Holmes-Gen, a health expert at CARB. This
includes pregnant women, children, seniors and people that have existing
respiratory conditions such as asthma or heart conditions and
pre-existing respiratory illnesses.

When the A.Q.I. reaches 100 or more, outdoor air is no longer safe for
sensitive groups.

\textbf{Air pollution affects people all year, not just during wildfire
season.} On Aug. 24, smoke from wildfires raging in the Bay Area made
the air quality
\href{https://www.nytimes3xbfgragh.onion/2020/08/24/us/california-fires-wildfires.html}{four
times worse than in Beijing or New Delhi}. Three days later, the
E.P.A'.s smoke map showed much of Northern California buried under a
dark gray haze.

Mr. Tasat said that from Aug. 18 to 24, the daily average particle
pollution level in California was about 500 percent higher than it was
the same week in 2019, largely because of the wildfires.

\href{https://www.nytimes3xbfgragh.onion/spotlight/california-wildfires}{Wildfires
in the West ›}

\hypertarget{live-updates}{%
\subsection{\texorpdfstring{\href{https://www.nytimes3xbfgragh.onion/2020/09/12/us/wildfires-live-updates.html}{Live
Updates}}{Live Updates}}\label{live-updates}}

Updated~

Sept. 12, 2020, 2:53 p.m. ET

\begin{itemize}
\tightlist
\item
  \href{https://www.nytimes3xbfgragh.onion/2020/09/12/us/wildfires-live-updates.html\#link-f3961ff}{President
  Trump will visit California on Monday after destructive fires.}
\item
  \href{https://www.nytimes3xbfgragh.onion/2020/09/12/us/wildfires-live-updates.html\#link-7e503ae9}{Shifting
  weather may improve firefighting conditions on the West Coast.}
\item
  \href{https://www.nytimes3xbfgragh.onion/2020/09/12/us/wildfires-live-updates.html\#link-5e4c548d}{Oregon's
  fire marshal is temporarily replaced as firefighters battle blazes.}
\end{itemize}

However, California is subject to air pollution throughout the year.
Sometimes there are high levels of both ozone and particle pollution
during a particular season. Ozone pollution tends to be worse in the
summer, Mr. Tasat said.

Fine particles in the soot, ash and dust of wildfire smoke make up
particle pollutants, which can be inhaled by the lungs. During summer,
the mixture of smoke pollutants and hotter temperatures generate what
Ms. Holmes-Gen calls a ``double whammy of ozone pollution and particle
pollution.''

``One unique aspect of this time period, unfortunately, is that people
can be affected by multiple types of air pollution,'' she said.

\textbf{You can reduce your exposure to air pollution.} The effects of
air pollution can be mild, like eye and throat irritation, or serious,
requiring hospitalization for heart or respiratory issues. Smoke and
pollution can cause inflammation of the lung tissue and increase the
vulnerability of infections, Ms. Holmes-Gen said.

The most effective way to reduce your exposure to air pollution is to
stay indoors with windows and doors closed, said Melanie Turner, a
public information officer for the California Air Resources Board.
Air-conditioning should remain on continuously, not the auto cycle,
which cycles on and off, Ms. Turner said. It's also helpful to close the
fresh air intake so that smoke doesn't get inside the house. If your
system allows for it, it's recommended to install a high efficiency air
filter, classified as MERV 13 or higher. Portable air cleaners can also
reduce indoor particulate matter in smaller spaces, like individual
rooms.
\href{https://ww2.arb.ca.gov/our-work/programs/air-cleaners-ozone-products/california-certified-air-cleaning-devices}{CARB
certifies all air cleaners}that are currently sold in California to
ensure they meet its regulations for ozone emissions.

\begin{center}\rule{0.5\linewidth}{\linethickness}\end{center}

\hypertarget{heres-what-else-to-read}{%
\subsection{Here's what else to read}\label{heres-what-else-to-read}}

Image

A bill passed on Monday gives tenants a temporary reprieve from missed
rent payments.Credit...Rich Pedroncelli/Associated Press

\begin{itemize}
\tightlist
\item
  On Monday, \textbf{California lawmakers passed a bill to extend
  protections to renters who have lost income because of the pandemic}.
  The bill grants tenants a temporary reprieve from missed payments and
  gives them until January to pay in full. Some tenant advocates were
  still worried.
  {[}\href{https://www.sfchronicle.com/politics/article/Eviction-moratorium-moves-forward-in-California-15528629.php}{San
  Francisco Chronicle}{]}
\end{itemize}

\emph{CalMatters is compiling}
\href{https://calmatters.org/explainers/california-final-bill-tracker-2020-legislature/?_gl=1*qxjqja*_ga*MjU4NzQwMjcxLjE1NTMyODIyNTA.}{\emph{a
list of newly passed bills from this year's legislative session}}
\emph{and tracking their fate through the bill-signing period.}

\begin{itemize}
\tightlist
\item
  One decade and several lawsuits later, \textbf{the Terraces of
  Lafayette, a proposal for 315 apartments that has become emblematic of
  California's housing crisis}, was finally approved.
  {[}\href{https://www.sfgate.com/news/bayarea/article/Controversial-Terraces-Apartments-Approved-15512253.php}{SFGate}{]}
\end{itemize}

\emph{The saga was chronicled by my colleague Conor Dougherty in his
book ``Golden Gates: Fighting for Housing in America,'' an excerpt from
which you can read}
\href{https://www.nytimes3xbfgragh.onion/2020/02/13/business/economy/housing-crisis-conor-dougherty-golden-gates.html}{\emph{here}}\emph{.}

\begin{itemize}
\item
  Because California lacks a uniform policy for reporting workplace
  coronavirus outbreaks, the public, and oftentimes essential workers,
  \textbf{are left in the dark about how many workers have been sickened
  with the virus}.
  {[}\href{https://calmatters.org/california-divide/2020/08/california-counties-wont-report-covid-essential-workplace-outbreaks/}{CalMatters}{]}
\item
  \textbf{The California Public Utilities Commission} \textbf{fired its
  executive director, Alice Stebbins, on Monday} in response to
  allegations of unethical hiring decisions discussed in a recent
  personal audit.
  {[}\href{https://www.sfchronicle.com/business/article/California-Public-Utilities-Commission-fires-15528599.php}{San
  Francisco Chronicle}{]}
\item
  \textbf{San Francisco's Chinatown has been hit hard by the pandemic,}
  but there's hope in seeing how it has weathered a shrinking economy
  and xenophobia in the past.
  {[}\href{https://www.sfchronicle.com/culture/article/SF-Chinatown-has-always-adapted-To-survive-the-15520348.php}{San
  Francisco Chronicle}{]}
\item
  \textbf{Facebook employees are angry with Mark Zuckerberg} because
  they said the company did not do enough to take down pages promoting
  misinformation and conspiracy theories before the shooting in Kenosha,
  Wis.
  {[}\href{https://www.buzzfeednews.com/article/ryanmac/facebook-employees-slam-zuckerberg-kenosha-militia-shooting}{BuzzFeed
  News}{]}
\item
  For decades, Tad Jones lived alone in a California forest. \textbf{His
  friends thought that if anyone could survive a wildfire it would be
  him,} more a friend of nature than of man.
  {[}\href{https://www.nytimes3xbfgragh.onion/2020/08/31/us/fires-california-monk-tad-jones-last-chance-santa-cruz.html}{The
  New York Times}{]}
\item
  ``We all want new experiences, but that's been hard to come by.''
  \textbf{College students in the Covid-19 era are starting ``collab
  houses''} --- but to go to school without staying at home with their
  parents.
  {[}\href{https://www.nytimes3xbfgragh.onion/2020/08/28/style/college-collab-houses-coronavirus.html}{The
  New York Times}{]}
\item
  Modeled after food-delivery services in Seoul, \textbf{a tiny Los
  Angeles Koreatown business keeps neighborhood restaurants running
  through the pandemic}.
  {[}\href{https://www.nytimes3xbfgragh.onion/2020/08/31/dining/los-angeles-food-delivery-runningman.html}{The
  New York Times}{]}
\end{itemize}

\begin{center}\rule{0.5\linewidth}{\linethickness}\end{center}

\emph{California Today goes live at 6:30 a.m. Pacific time weekdays.
Tell us what you want to see:}
\href{mailto:CAtoday@NYTimes.com}{\emph{CAtoday@NYTimes.com}}\emph{.
Were you forwarded this email?}
\href{https://www.nytimes3xbfgragh.onion/newsletters/california-today?module=inline}{\emph{Sign
up for California Today here.}}

\emph{Jill Cowan grew up in Orange County, went to school at U.C.
Berkeley and has reported all over the state, including the Bay Area,
Bakersfield and Los Angeles --- but she always wants to see more. Follow
along here or on Twitter,}
\href{https://twitter.com/JillCowan}{\emph{@jillcowan}}\emph{.}

\emph{California Today is edited by Julie Bloom, who grew up in Los
Angeles and graduated from U.C. Berkeley.}

Advertisement

\protect\hyperlink{after-bottom}{Continue reading the main story}

\hypertarget{site-index}{%
\subsection{Site Index}\label{site-index}}

\hypertarget{site-information-navigation}{%
\subsection{Site Information
Navigation}\label{site-information-navigation}}

\begin{itemize}
\tightlist
\item
  \href{https://help.nytimes3xbfgragh.onion/hc/en-us/articles/115014792127-Copyright-notice}{©~2020~The
  New York Times Company}
\end{itemize}

\begin{itemize}
\tightlist
\item
  \href{https://www.nytco.com/}{NYTCo}
\item
  \href{https://help.nytimes3xbfgragh.onion/hc/en-us/articles/115015385887-Contact-Us}{Contact
  Us}
\item
  \href{https://www.nytco.com/careers/}{Work with us}
\item
  \href{https://nytmediakit.com/}{Advertise}
\item
  \href{http://www.tbrandstudio.com/}{T Brand Studio}
\item
  \href{https://www.nytimes3xbfgragh.onion/privacy/cookie-policy\#how-do-i-manage-trackers}{Your
  Ad Choices}
\item
  \href{https://www.nytimes3xbfgragh.onion/privacy}{Privacy}
\item
  \href{https://help.nytimes3xbfgragh.onion/hc/en-us/articles/115014893428-Terms-of-service}{Terms
  of Service}
\item
  \href{https://help.nytimes3xbfgragh.onion/hc/en-us/articles/115014893968-Terms-of-sale}{Terms
  of Sale}
\item
  \href{https://spiderbites.nytimes3xbfgragh.onion}{Site Map}
\item
  \href{https://help.nytimes3xbfgragh.onion/hc/en-us}{Help}
\item
  \href{https://www.nytimes3xbfgragh.onion/subscription?campaignId=37WXW}{Subscriptions}
\end{itemize}
