Sections

SEARCH

\protect\hyperlink{site-content}{Skip to
content}\protect\hyperlink{site-index}{Skip to site index}

\href{https://www.nytimes3xbfgragh.onion/section/arts/music}{Music}

\href{https://myaccount.nytimes3xbfgragh.onion/auth/login?response_type=cookie\&client_id=vi}{}

\href{https://www.nytimes3xbfgragh.onion/section/todayspaper}{Today's
Paper}

\href{/section/arts/music}{Music}\textbar{}5 Minutes That Will Make You
Love the Violin

\url{https://nyti.ms/31OeiLe}

\begin{itemize}
\item
\item
\item
\item
\item
\item
\end{itemize}

Advertisement

\protect\hyperlink{after-top}{Continue reading the main story}

Supported by

\protect\hyperlink{after-sponsor}{Continue reading the main story}

\hypertarget{5-minutes-that-will-make-you-love-the-violin}{%
\section{5 Minutes That Will Make You Love the
Violin}\label{5-minutes-that-will-make-you-love-the-violin}}

We asked Hilary Hahn, John Adams, André Rieu and others to pick the
music that moves them. Listen to their choices.

\includegraphics{https://static01.graylady3jvrrxbe.onion/images/2020/09/04/arts/02fiveminutes-violin-gif/02fiveminutes-violin-gif-articleLarge-v2.gif?quality=75\&auto=webp\&disable=upscale}

\begin{itemize}
\item
  Sept. 2, 2020
\item
  \begin{itemize}
  \item
  \item
  \item
  \item
  \item
  \item
  \end{itemize}
\end{itemize}

In the past, we've asked some of our favorite artists to choose the five
minutes or so they would play to make their friends fall in love with
\href{https://www.nytimes3xbfgragh.onion/2018/09/06/arts/music/5-minutes-that-will-make-you-love-classical-music.html}{classical
music},
\href{https://www.nytimes3xbfgragh.onion/2019/04/19/arts/music/classical-music-piano.html}{the
piano},
\href{https://www.nytimes3xbfgragh.onion/2020/04/28/arts/music/classical-music-opera.html}{opera},
\href{https://www.nytimes3xbfgragh.onion/2020/06/03/arts/music/five-minutes-classical-music-cello.html}{the
cello},
\href{https://www.nytimes3xbfgragh.onion/2020/07/01/arts/music/classical-music-mozart.html}{Mozart}
and
\href{https://www.nytimes3xbfgragh.onion/2020/08/05/arts/music/five-minutes-classical-music.html}{21st-century
composers}.

Now we want to convince those curious friends to love the sweet, songful
violin. We hope you find lots here to discover and enjoy; leave your
choices in the comments.

\hypertarget{--}{%
\subsubsection{◆ ◆ ◆}\label{--}}

\hypertarget{andrew-norman-composer}{%
\subsection{Andrew Norman, composer}\label{andrew-norman-composer}}

This solo violin piece by Reena Esmail really blew me away when I first
heard it. Like much of her work, it inhabits an intensely lyrical space
informed by both Indian and Western classical musics. In Vijay Gupta's
gripping performance, I hear sounds, colors and expressions
simultaneously familiar and fresh, intimate and epic, grounded and
aloft.

\hypertarget{reena-esmails-darshan}{%
\subsubsection{Reena Esmail's ``Darshan''}\label{reena-esmails-darshan}}

Vijay Gupta, violin

\hypertarget{---1}{%
\subsubsection{◆ ◆ ◆}\label{---1}}

\hypertarget{zachary-woolfe-times-classical-music-editor}{%
\subsection{Zachary Woolfe, Times classical music
editor}\label{zachary-woolfe-times-classical-music-editor}}

Who was ever really happy with one scoop of ice cream rather than two?
Bach's Double Concerto is dessert doubled --- especially in this
recording, featuring a pair of the 20th century's most honeyed tones.
The violins' interplay is playfully fiery in the work's outer movements.
But here, in the central Largo, the mood is shared, serene, blossoming
longing.

\hypertarget{bachs-double-violin-concerto}{%
\subsubsection{Bach's Double Violin
Concerto}\label{bachs-double-violin-concerto}}

Itzhak Perlman and Isaac Stern, violins; Zubin Mehta conducting New York
Philharmonic (Sony Classical)

\hypertarget{---2}{%
\subsubsection{◆ ◆ ◆}\label{---2}}

\hypertarget{mazz-swift-violinist}{%
\subsection{Mazz Swift, violinist}\label{mazz-swift-violinist}}

With Eddie South's performance of this piece, all you need is
\emph{three} minutes to fall in love with the violin. Any violinists who
listen to this recording will surely identify at least one reason they
chose to play the instrument, though it doesn't take a seasoned listener
to be completely delighted. The piece has a dazzling array of
challenging techniques, Romantic lyricism and various fiddling styles,
including jazz, Gypsy jazz and old time; the spontaneous nature of
South's performance brings me so much joy.

\hypertarget{black-gypsy}{%
\subsubsection{``Black Gypsy''}\label{black-gypsy}}

Eddie South

\hypertarget{---3}{%
\subsubsection{◆ ◆ ◆}\label{---3}}

\hypertarget{david-allen-times-writer}{%
\subsection{David Allen, Times writer}\label{david-allen-times-writer}}

You look at the title of the fifth movement of Beethoven's Op. 130
string quartet, ``Cavatina,'' and think of an aria, simple and short.
And the piece is both. But what makes its simplicity so special is not
just the way the first violin arcs its line --- how it traces out its
song --- but also how its partner, the second violin, seems to echo it,
to join it on its path and embrace it, as if in sympathy. This is the
most poignant, tender few minutes that Beethoven ever wrote for violins.

\hypertarget{beethovens-op-130-string-quartet-cavatina}{%
\subsubsection{Beethoven's Op. 130 String Quartet,
``Cavatina''}\label{beethovens-op-130-string-quartet-cavatina}}

Danish String Quartet (ECM)

\hypertarget{---4}{%
\subsubsection{◆ ◆ ◆}\label{---4}}

\hypertarget{marcos-balter-composer}{%
\subsection{Marcos Balter, composer}\label{marcos-balter-composer}}

When composers are their own performers, as in the violin works of
Paganini, Laurie Anderson and Leroy Jenkins, music becomes a
self-portrait in motion. Secluded in his Brooklyn apartment since March
with his instrument and effect pedals, Darian Thomas has been writing an
intimate and vulnerable sonic diary about our times. In ``Darkness Runs
From Light,'' he weaves --- by himself --- a lush string orchestra while
breathily singing of angst and optimism: ``I was up last night
dreaming/About a new day/I was dreaming. Soaring. Hoping.'' His violin
hugs us, and we could all use a hug these days.

\hypertarget{darian-thomass-darkness-runs-from-light}{%
\subsubsection{Darian Thomas's ``Darkness Runs From
Light''}\label{darian-thomass-darkness-runs-from-light}}

Darian Thomas

\hypertarget{---5}{%
\subsubsection{◆ ◆ ◆}\label{---5}}

\hypertarget{corinna-da-fonseca-wollheim-times-writer}{%
\subsection{Corinna da Fonseca-Wollheim, Times
writer}\label{corinna-da-fonseca-wollheim-times-writer}}

Biber's transcendent solo-violin Passacaglia, from his Rosary Sonatas,
precedes Bach's monumental Chaconne by almost 50 years. Yet it already
inhabits the same architectural grandeur, built by a single player and
just four strings. The music unfolds like a dialogue between a solemn,
dependable bass line and filigree variations full of fancy, yearning and
quiet contemplation.

\hypertarget{bibers-rosary-sonata-no-16}{%
\subsubsection{Biber's Rosary Sonata No.
16}\label{bibers-rosary-sonata-no-16}}

Rachel Podger (Channel Classics)

\hypertarget{---6}{%
\subsubsection{◆ ◆ ◆}\label{---6}}

\hypertarget{pekka-kuusisto-violinist}{%
\subsection{Pekka Kuusisto, violinist}\label{pekka-kuusisto-violinist}}

Jörg Widmanns's book of 24 duos for violin and cello is basically a
breathtaking 24-course meal at Noma featuring caribou sperm, spider
eggs, fermented kangaroo sweat and popcorn. The one about the road home
always gives me a most satisfying fright. It sounds as if a person with
advanced memory loss forgets how a Brahms piece unfolds, but keeps
trying before finally getting so profoundly sidetracked that it becomes
a new language --- and then vanishes. It tickles my fears both personal
and global. A gesture both devastating and detached is a tricky thing to
compose, but I think this two-minute cycle of sighs nails it.

\hypertarget{juxf6rg-widmanns-vier-strophen-vom-heimweh}{%
\subsubsection{Jörg Widmann's ``Vier Strophen vom
Heimweh''}\label{juxf6rg-widmanns-vier-strophen-vom-heimweh}}

Ilya Gringolts, violin; Dmitry Kouzov, cello (Delos)

\hypertarget{---7}{%
\subsubsection{◆ ◆ ◆}\label{---7}}

\hypertarget{joshua-barone-times-writer}{%
\subsection{Joshua Barone, Times
writer}\label{joshua-barone-times-writer}}

I was an impatient violin student who often got in trouble for reading
ahead and creating, as my teacher said, bad habits. (She was right.)
When I should have been focusing on concertos by Bruch and Mendelssohn,
I was more interested in works beyond my ability --- like Sibelius's
Violin Concerto, with its lyrical warmth and lush textures. It's
alluring from the start: The soloist enters over frosty, barely audible
violins, with a mysteriously inviting melody that gives way to what
feels like a series of dark tales, shared late at night by the
flickering glow of a dying fire.

\hypertarget{sibeliuss-violin-concerto}{%
\subsubsection{Sibelius's Violin
Concerto}\label{sibeliuss-violin-concerto}}

Lisa Batiashvili, violin; Daniel Barenboim conducting Staatskapelle
Berlin (Deutsche Grammophon)

\hypertarget{---8}{%
\subsubsection{◆ ◆ ◆}\label{---8}}

\hypertarget{jessie-montgomery-violinist-and-composer}{%
\subsection{Jessie Montgomery, violinist and
composer}\label{jessie-montgomery-violinist-and-composer}}

``Mother and Child,'' the second movement of William Grant Still's Suite
for Violin and Piano, is filled with the tenderness you'd imagine from
its title. It hearkens to the storytelling and lyricism of 1950s
Hollywood scores, taking you on a dreamy journey. Even though he wrote
the work in response to a sculpture by Sargent Johnson, it is known that
Still had very close relationship with, and reverence for, his mother,
who was a great supporter of his ambitions and a leader in their
community. I hear in this soulful and robust performance by Rachel
Barton Pine a musical tribute to motherly figures. This piece warms my
heart.

\hypertarget{william-grant-stills-mother-and-child}{%
\subsubsection{William Grant Still's ``Mother and
Child''}\label{william-grant-stills-mother-and-child}}

Rachel Barton Pine, violin; Matthew Hagle, piano (Cedille)

\hypertarget{---9}{%
\subsubsection{◆ ◆ ◆}\label{---9}}

\hypertarget{imani-danielle-mosley-musicologist}{%
\subsection{Imani Danielle Mosley,
musicologist}\label{imani-danielle-mosley-musicologist}}

Choosing ``The Lark Ascending'' to showcase the violin might seem
saccharine or passé; it is commonly voted Britain's favorite piece on
polls each year. But when you strip away its associations with an
imagined pastoral England, what you're left with is an incredibly joyful
flight of fancy. In good hands, the opening violin passages sound
improvised, beginning in the instrument's mellower range. Its
full-throatedness, rich tones and upward ascent mimic a lark so
wonderfully, and Vaughan Williams writes so that the violin blends
seamlessly with solo winds while also performing virtuosic runs --- a
bird floating and diving.

\hypertarget{vaughan-williamss-the-lark-ascending}{%
\subsubsection{Vaughan Williams's ``The Lark
Ascending''}\label{vaughan-williamss-the-lark-ascending}}

David Nolan, violin; Vernon Handley conducting London Philharmonic
Orchestra (Warner Classics)

\hypertarget{---10}{%
\subsubsection{◆ ◆ ◆}\label{---10}}

\hypertarget{hilary-hahn-violinist}{%
\subsection{Hilary Hahn, violinist}\label{hilary-hahn-violinist}}

This is one of the most heart-stopping pieces in the classical
literature. I hold my breath every time I listen to it, or play it. It's
an incredibly special and personal experience. ``The Lark Ascending'' is
all of art in one place: nature, music, poetry, imagery and imagination.
It lifts you immediately out of your seat, out of the space you're in,
and carries you through the ether, through intense emotions, through
joyful, sunny countryside revelry and through sheer orchestral lushness.
The final note returns you to your own soul, yet still you are soaring.

\hypertarget{vaughan-williamss-the-lark-ascending-1}{%
\subsubsection{Vaughan Williams's ``The Lark
Ascending''}\label{vaughan-williamss-the-lark-ascending-1}}

Iona Brown, violin; Neville Marriner conducting Academy of St. Martin in
the Fields (Decca)

\hypertarget{---11}{%
\subsubsection{◆ ◆ ◆}\label{---11}}

\hypertarget{anthony-tommasini-times-chief-classical-music-critic}{%
\subsection{Anthony Tommasini, Times chief classical music
critic}\label{anthony-tommasini-times-chief-classical-music-critic}}

The first movement of Samuel Barber's 1939 Violin Concerto has no
introduction or suspenseful tease. It starts right off with a surging
violin melody, touched with a bit of wistful nostalgia. When you have a
tune that good, why wait? Things turns pensive and darker, but
eventually the melody returns, in full orchestral splendor. This excerpt
will make you want to hear the complete concerto, which ends with a
virtuosic perpetual-motion finale.

\hypertarget{barbers-violin-concerto}{%
\subsubsection{Barber's Violin Concerto}\label{barbers-violin-concerto}}

Isaac Stern, violin; Leonard Bernstein conducting New York Philharmonic
(Sony Classical)

\hypertarget{---12}{%
\subsubsection{◆ ◆ ◆}\label{---12}}

\hypertarget{ray-chen-violinist}{%
\subsection{Ray Chen, violinist}\label{ray-chen-violinist}}

Since it was invented 400 years ago, the violin has been cast in many
different lights, from an angelic voice celebrating God's glory to the
devil's instrument; it has an extremely wide range of colors and
intention. While many of its famous works display some sort of virtuosic
showmanship, I've recently found an ease and a comfort in the Largo from
Bach's Third Sonata for Solo Violin. It's a personal favorite that I had
to include on my new album, ``Solace.''

\hypertarget{bachs-violin-sonata-no-3}{%
\subsubsection{Bach's Violin Sonata No.
3}\label{bachs-violin-sonata-no-3}}

Ray Chen, violin (Decca)

\hypertarget{---13}{%
\subsubsection{◆ ◆ ◆}\label{---13}}

\hypertarget{john-adams-composer}{%
\subsection{John Adams, composer}\label{john-adams-composer}}

With recordings I've become something of an archaeologist ---
fascinated, and often deeply affected, by how the emotional content of a
piece changes as performing traditions evolve. This passage from the
Elgar Violin Concerto, recorded in 1932 by Yehudi Menuhin --- just 16 at
the time --- reveals the violin as the most vocal of instruments.
Menuhin's is a way with the instrument that seems to have vanished. Give
yourself a moment to get beyond the initial blushing connection with
corny old Hollywood romances. Then hear how the elasticity of phrasing
and the expressive slides between notes have the same power to touch you
as a great jazz singer.

\hypertarget{elgars-violin-concerto}{%
\subsubsection{Elgar's Violin Concerto}\label{elgars-violin-concerto}}

Yehudi Menuhin, violin; Edward Elgar conducting London Symphony
Orchestra (Warner Classics)

\hypertarget{---14}{%
\subsubsection{◆ ◆ ◆}\label{---14}}

\hypertarget{seth-colter-walls-times-writer}{%
\subsection{Seth Colter Walls, Times
writer}\label{seth-colter-walls-times-writer}}

Bartok indulged some of his regular obsessions here, including folk-like
melody and modernist patterns. He also thought carefully about the
violin: In the first movement, the soloist and the orchestral strings
engage in some deft handoffs. Before the exuberant cadenza, the violin
plays some quarter-tones (starting at 49 seconds, in the clip below).
Once the orchestra rejoins, you can find some capital-r Romantic
yearning --- yet another facet of this composer's expressive vitality.

\hypertarget{bartoks-violin-concerto-no-2}{%
\subsubsection{Bartok's Violin Concerto No.
2}\label{bartoks-violin-concerto-no-2}}

Tibor Varga, violin; Ferenc Fricsay conducting Berlin Philharmonic
(Deutsche Grammophon)

\hypertarget{---15}{%
\subsubsection{◆ ◆ ◆}\label{---15}}

\hypertarget{andruxe9-rieu-violinist}{%
\subsection{André Rieu, violinist}\label{andruxe9-rieu-violinist}}

When I think of playing the violin, the first thing that comes to mind
is love. Initially my love for my first violin teacher. I was captivated
by her vibrato and longed to imitate the technique which produced such
an amazing sound. I practiced and practiced until one day I succeeded in
playing vibrato myself. I was the happiest boy on earth because I felt
this was the sound that made the violin so beautiful. From that moment
on I practiced day after day, year after year, always searching for
romantic melodies which filled my heart with joy and which I discovered
made other people happy, too. I still practice every day because this is
what makes it possible for me to do the most beautiful job in the world:
making people happy by playing my violin.

\hypertarget{my-way}{%
\subsubsection{``My Way''}\label{my-way}}

André Rieu

\hypertarget{---16}{%
\subsubsection{◆ ◆ ◆}\label{---16}}

\hypertarget{deborah-borda-president-new-york-philharmonic}{%
\subsection{Deborah Borda, president, New York
Philharmonic}\label{deborah-borda-president-new-york-philharmonic}}

It is said that the violin comes closest to expressing the qualities of
the human voice. I experience it as the most human and humane of all
instruments. In ``Erbarme dich, mein Gott,'' from the ``St. Matthew
Passion,'' the violin entwines the voice as a full partner. The text is
a plea for mercy, but the violin, too, speaks, its plaintive grace
moving us to a place of empathy and forgiveness. I chose this, rather
than the bravura of Paganini or the heights of the canon of concertos,
as an expression of the purest yet most ravishing sound of the
instrument. It carries a message of special resonance in these troubling
times.

\hypertarget{bachs-erbarme-dich}{%
\subsubsection{Bach's ``Erbarme dich''}\label{bachs-erbarme-dich}}

John Eliot Gardiner conducting English Baroque Soloists (Archiv)

\hypertarget{---17}{%
\subsubsection{◆ ◆ ◆}\label{---17}}

Advertisement

\protect\hyperlink{after-bottom}{Continue reading the main story}

\hypertarget{site-index}{%
\subsection{Site Index}\label{site-index}}

\hypertarget{site-information-navigation}{%
\subsection{Site Information
Navigation}\label{site-information-navigation}}

\begin{itemize}
\tightlist
\item
  \href{https://help.nytimes3xbfgragh.onion/hc/en-us/articles/115014792127-Copyright-notice}{©~2020~The
  New York Times Company}
\end{itemize}

\begin{itemize}
\tightlist
\item
  \href{https://www.nytco.com/}{NYTCo}
\item
  \href{https://help.nytimes3xbfgragh.onion/hc/en-us/articles/115015385887-Contact-Us}{Contact
  Us}
\item
  \href{https://www.nytco.com/careers/}{Work with us}
\item
  \href{https://nytmediakit.com/}{Advertise}
\item
  \href{http://www.tbrandstudio.com/}{T Brand Studio}
\item
  \href{https://www.nytimes3xbfgragh.onion/privacy/cookie-policy\#how-do-i-manage-trackers}{Your
  Ad Choices}
\item
  \href{https://www.nytimes3xbfgragh.onion/privacy}{Privacy}
\item
  \href{https://help.nytimes3xbfgragh.onion/hc/en-us/articles/115014893428-Terms-of-service}{Terms
  of Service}
\item
  \href{https://help.nytimes3xbfgragh.onion/hc/en-us/articles/115014893968-Terms-of-sale}{Terms
  of Sale}
\item
  \href{https://spiderbites.nytimes3xbfgragh.onion}{Site Map}
\item
  \href{https://help.nytimes3xbfgragh.onion/hc/en-us}{Help}
\item
  \href{https://www.nytimes3xbfgragh.onion/subscription?campaignId=37WXW}{Subscriptions}
\end{itemize}
