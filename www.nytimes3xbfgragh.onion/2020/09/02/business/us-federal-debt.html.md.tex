Sections

SEARCH

\protect\hyperlink{site-content}{Skip to
content}\protect\hyperlink{site-index}{Skip to site index}

\href{https://www.nytimes3xbfgragh.onion/section/business}{Business}

\href{https://myaccount.nytimes3xbfgragh.onion/auth/login?response_type=cookie\&client_id=vi}{}

\href{https://www.nytimes3xbfgragh.onion/section/todayspaper}{Today's
Paper}

\href{/section/business}{Business}\textbar{}Federal Borrowing Amid
Pandemic Puts U.S. Debt on Path to Exceed World War II

\url{https://nyti.ms/32TKkEO}

\begin{itemize}
\item
\item
\item
\item
\item
\end{itemize}

\begin{itemize}
\item
  \href{https://www.nytimes3xbfgragh.onion/interactive/2020/09/08/us/elections/results-new-hampshire-primary-elections.html?action=click\&pgtype=Article\&state=default\&region=TOP_BANNER\&context=storylines_menu}{New
  Hampshire Results}
\item
  \href{https://www.nytimes3xbfgragh.onion/live/2020/09/08/us/trump-vs-biden?action=click\&pgtype=Article\&state=default\&region=TOP_BANNER\&context=storylines_menu}{Election
  Updates}
\item
  \href{https://www.nytimes3xbfgragh.onion/interactive/2020/us/elections/election-states-biden-trump.html?action=click\&pgtype=Article\&state=default\&region=TOP_BANNER\&context=storylines_menu}{Paths
  to 270}
\item
  \href{https://www.nytimes3xbfgragh.onion/interactive/2020/08/31/us/politics/vote-by-mail-deadlines.html?action=click\&pgtype=Article\&state=default\&region=TOP_BANNER\&context=storylines_menu}{Voting
  by Mail}
\item
  \href{https://www.nytimes3xbfgragh.onion/interactive/2019/us/elections/2020-presidential-election-calendar.html?action=click\&pgtype=Article\&state=default\&region=TOP_BANNER\&context=storylines_menu}{Key
  Dates}
\item
  \href{https://www.nytimes3xbfgragh.onion/newsletters/politics?action=click\&pgtype=Article\&state=default\&region=TOP_BANNER\&context=storylines_menu}{Politics
  Newsletter}
\end{itemize}

Advertisement

\protect\hyperlink{after-top}{Continue reading the main story}

Supported by

\protect\hyperlink{after-sponsor}{Continue reading the main story}

\hypertarget{federal-borrowing-amid-pandemic-puts-us-debt-on-path-to-exceed-world-war-ii}{%
\section{Federal Borrowing Amid Pandemic Puts U.S. Debt on Path to
Exceed World War
II}\label{federal-borrowing-amid-pandemic-puts-us-debt-on-path-to-exceed-world-war-ii}}

Federal debt, as a share of the economy, hit 98 percent in the 2020
fiscal year. Many economists are pushing lawmakers to add even more to
it.

\includegraphics{https://static01.graylady3jvrrxbe.onion/images/2020/09/02/business/02DC-CBO-01/merlin_176135361_2fd3d3fe-1e24-42fd-917b-d9b955fa774d-articleLarge.jpg?quality=75\&auto=webp\&disable=upscale}

\href{https://www.nytimes3xbfgragh.onion/by/jim-tankersley}{\includegraphics{https://static01.graylady3jvrrxbe.onion/images/2018/10/19/multimedia/author-jim-tankersley/author-jim-tankersley-thumbLarge.png}}

By \href{https://www.nytimes3xbfgragh.onion/by/jim-tankersley}{Jim
Tankersley}

\begin{itemize}
\item
  Sept. 2, 2020
\item
  \begin{itemize}
  \item
  \item
  \item
  \item
  \item
  \end{itemize}
\end{itemize}

WASHINGTON --- A surge in government borrowing in the face of the
pandemic recession has put the United States in a position it has not
seen since World War II: In order to pay off its national debt this
year, the country would need to spend an amount nearly as large as its
entire annual economy.

And still, economists and many fiscal hawks are urging lawmakers to
borrow even more to fuel the nation's economic recovery.

The amount of U.S. government debt has grown to nearly outpace the size
of the nation's economy in the 2020 fiscal year and is set to exceed it
next year, as the virus downturn saps tax revenues, spurs government
spending and necessitates record amounts of federal borrowing, the
Congressional Budget Office
\href{https://www.cbo.gov/system/files/2020-09/56517-Budget-Outlook.pdf}{said
on Wednesday}. Federal debt, as a share of the economy, is now on track
to smash America's World War II-era record by 2023.

The budget office report underscored the scrambled politics of deficits
in 2020: It showed debt held by the public climbing to 98 percent of the
size of the economy for the fiscal year ending Sept. 30. Forecasters had
previously expected the nation to reach those levels at the end of the
decade, a time frame that had already alarmed fiscal hawks in
Washington, who warned ballooning deficits would consume federal budgets
and chill private investment.

But the virus has upended those predictions, prompting even
\href{https://www.nytimes3xbfgragh.onion/2020/05/16/business/deficits-virus-economists-trump.html}{longtime
champions} of fiscal prudence to urge lawmakers on Wednesday to keep
borrowing more for the time being, in order to help people and
businesses survive the lingering pain of a sharp recession and
\href{https://www.nytimes3xbfgragh.onion/2020/08/21/business/economy/coronavirus-economic-recovery.html}{now-slowing
recovery}.

``We should think and worry about the deficit an awful lot, and we
should proceed to make it larger,'' said Maya MacGuineas, the president
of the Committee for a Responsible Federal Budget in Washington, which
has for years pushed lawmakers to take steps to reduce deficits and
debt.

The turnabout on deficit fears caps several years of declining concern
over Washington spending more than it takes in,
\href{https://www.nytimes3xbfgragh.onion/2018/02/13/us/politics/republicans-have-forgotten-they-hate-deficits.html}{particularly
among Republicans}. Lawmakers voted along party lines in 2017 to pass a
\$1.5 trillion tax cut that President Trump and Republican leaders
insisted would pay for itself but has
\href{https://www.nytimes3xbfgragh.onion/2019/01/11/business/trump-tax-cuts-revenue.html}{instead
added to the deficit.} The budget deficit surpassed \$1 trillion in 2019
--- before the coronavirus pandemic hit --- a jump of 17 percent from
2018 as tax cuts and spending increases continued to force heavy
government borrowing.

The pandemic has plunged the economy into its sharpest quarterly
contraction in growth in nearly 75 years, ballooning the deficit in the
process. With millions out of work and businesses shuttered, tax
revenues have fallen for the federal government, along with states and
municipalities.

Congress and Mr. Trump moved quickly to approve more than \$3 trillion
in new federal spending to help businesses and individuals stay afloat
through the
\href{https://www.nytimes3xbfgragh.onion/2020/03/22/us/politics/coronavirus-economy-shutdown.html}{abrupt
slowdown in economic activity}. All of those factors necessitated large
sums of government borrowing, sending deficits --- which had grown
steadily
\href{https://www.nytimes3xbfgragh.onion/2020/01/13/business/budget-deficit-1-trillion-trump.html}{even
in the middle} of a record economic expansion --- skyward.

The deficit --- the difference between what the United States spends and
what it earns through taxes and other revenue --- is expected to reach
\$3.3 trillion for fiscal year 2020, the budget office said on
Wednesday. That is more than triple the level it reached in the 2019
fiscal year.

That financial gap is exacerbated by additional borrowing over the past
decades. At the end of the fiscal year, the budget office predicts,
total debt held by the public will be about \$20.3 trillion. By
comparison, the total output of the American economy --- its gross
domestic product --- is projected to be just over \$20.6 trillion for
the fiscal year.

Economic theory has long held that rising debt as a share of the economy
would drive up the amount of money governments must pay in interest to
borrowers. Like a household with a lot of loans, the theory went,
creditors would demand higher interest rates to hand cash to a heavily
indebted borrower. With its debt payments more expensive, the household
--- or government --- would have to borrow even more to stay current on
its obligations.

That would result in a debt spiral in which the government was not able
to do anything but fund its debt, the economists said, though such a
spiral did not materialize over the past decade, as debt climbed and
interest rates stayed low.

Because the pandemic hit the economy so quickly and painfully this year,
lawmakers raced to borrow money much faster than they did during the
last recession, when it took two years for the debt ratio to climb by a
similar amount, in percentage-point terms: Debt as a percent of gross
domestic product grew from 39 percent at the end of the 2008 fiscal year
to nearly 61 percent at the end of 2010.

But it has been decades since the amount of federal debt was larger than
the sum of the nation's annual economic output. That came in 1946,
shortly after the war ended.

The fiscal woes are not just confined to the United States' need to
borrow. In a separate report released on Wednesday afternoon, the budget
office updated its forecasts for the solvency of the Social Security
Trust Fund, showing it will run out of money faster than the office
previously forecast in June.

The new estimates imply the fund will be exhausted by 2031, a year
earlier than previously projected, forcing immediate benefit cuts,
unless lawmakers intervene. Medicare's hospital insurance trust fund is
now on track to run out of money in 2024, instead of 2026.

The aggressive federal response to the pandemic in March resulted in
trillions of dollars in additional government spending, as Washington
looked to provide tax breaks, assistance for small and large businesses,
direct checks for low- and middle-income individuals and supplemental
benefits for the unemployed.

Those measures were widely supported, as millions of workers were
suddenly unemployed and businesses were forced to close their doors.
Most economists have continued to call for additional spending, as the
pandemic shows no sign of abating.

Loretta Mester, the president of the Federal Reserve Bank of Cleveland,
who has warned about previous deficits, told reporters on Wednesday that
her own forecasts for the economic recovery hinge in part on continued
fiscal support, and that without it, the United States might struggle to
make it through shutdowns and onto a sustained growth path.

While Ms. Mester said that she was ``not one of those people who think
that deficits don't matter,'' the United States cannot worry about
loading up on debt in the middle of a nascent recovery.

``This isn't the right time to have that conversation,'' she said.

In a sign of how unconcerned investors are about the deficit, stocks
\href{https://www.wsj.com/articles/global-stock-markets-dow-update-9-02-2020-11599039549}{rose
on} Wednesday, with the S\&P 500 rising 1.5 percent to set another
record. It was the index's best day since July 6.

Republican lawmakers who were little troubled by the increase have since
cited debt concerns as a reason to move slowly on a new package of
economic assistance amid the pandemic. Democratic leaders in the House
drafted and passed
\href{https://www.nytimes3xbfgragh.onion/2020/05/15/us/politics/house-simulus-vote.html}{a
\$3 trillion opening bid} for a new rescue package this week, but they
pared it back and dropped some members' top priorities from the bill out
of deficit concerns.

Yet while Mr. Trump, as a candidate in 2016, famously pledged to pay off
the entire national debt in eight years, he and his fellow speakers
during this year's Republican National Convention did not raise the
deficit issue at all. Mr. Trump's most recent budget proposal, offered
before the pandemic spread rapidly in the United States,
\href{https://www.nytimes3xbfgragh.onion/2020/02/10/business/economy/trump-budget.html}{did
not include a balanced budget} even if he were to win re-election.

For decades, analysts argued that an explosion of government borrowing
risked devouring a large part of the nation's savings, leaving less cash
available for private businesses to use for investment.

Those companies would then be forced to pay higher interest rates to
gain access to that smaller pool of funds. And those higher borrowing
costs, it was argued, would curtail investment and hurt economic growth.
The process is known as ``crowding out,'' and there is no sign that it
is happening now. Interest rates remain low and inflation is muted.

``We're in an era where more government debt is not doing so much
crowding out,'' said Douglas Elmendorf, a former director of the
Congressional Budget Office and the current dean of Harvard's John F.
Kennedy School of Government.

``I think the idea that we should not let the debt constrain our
response to the pandemic is exactly right,'' he said. ``But I think the
idea that it never matters how much debt you have, because there's
always some way around that, is wrong.''

Even some fiscal hawks, like Ms. MacGuineas and Michael A. Peterson, the
chief executive of the debt-focused Peterson Foundation, say lawmakers
should continue to spend for now, while targeting their efforts more
effectively to help the economy recover. Eventually, they say, that
spending will need to yield to debt reduction.

``When this devastating pandemic is behind us,'' Mr. Peterson said,
``our leaders must come together to address our growing debt so the next
generation can have better preparedness and greater prosperity.''

Matt Phillips and Jeanna Smialek contributed reporting.

\hypertarget{our-2020-election-guide}{%
\section{Our 2020 Election Guide}\label{our-2020-election-guide}}

Updated ~Sept. 8, 2020

\begin{center}\rule{0.5\linewidth}{\linethickness}\end{center}

\begin{itemize}
\item ~
  \hypertarget{the-latest}{%
  \subsection{The Latest}\label{the-latest}}

  \begin{itemize}
  \item
    President Trump and his party are using a playbook that aims to
    alarm people about crime in their backyards. It didn't work in 2018,
    but
    \href{https://www.nytimes3xbfgragh.onion/2020/09/08/us/politics/trump-republicans-fear-strategy.html?action=click\&pgtype=Article\&state=default\&region=BELOW_MAIN_CONTENT\&context=storylines_guide}{both
    parties think it could resonate more this year}.
  \end{itemize}
\item ~
  \hypertarget{how-to-win-270}{%
  \subsection{How to Win 270}\label{how-to-win-270}}

  \begin{itemize}
  \item
    Joe Biden and Donald Trump need 270 electoral votes to reach the
    White House. Try building
    \href{https://www.nytimes3xbfgragh.onion/interactive/2020/us/elections/election-states-biden-trump.html?action=click\&pgtype=Article\&state=default\&region=BELOW_MAIN_CONTENT\&context=storylines_guide}{your
    own coalition of battleground states}~to see potential outcomes.
  \end{itemize}
\item ~
  \hypertarget{voting-by-mail}{%
  \subsection{Voting by Mail}\label{voting-by-mail}}

  \begin{itemize}
  \item
    Will you have enough time to vote by mail in your state? Yes, but
    it's risky to procrastinate.
    \href{https://www.nytimes3xbfgragh.onion/interactive/2020/08/31/us/politics/vote-by-mail-deadlines.html?action=click\&pgtype=Article\&state=default\&region=BELOW_MAIN_CONTENT\&context=storylines_guide}{Check
    your state's deadline.}
  \item
    \href{https://www.nytimes3xbfgragh.onion/interactive/2020/us/elections/joe-biden.html?action=click\&pgtype=Article\&state=default\&region=BELOW_MAIN_CONTENT\&context=storylines_guide}{}

    \hypertarget{joe-biden}{%
    \section{Joe Biden}\label{joe-biden}}

    \hypertarget{democrat}{%
    \subsection{Democrat}\label{democrat}}

    \href{https://www.nytimes3xbfgragh.onion/interactive/2020/us/elections/donald-trump.html?action=click\&pgtype=Article\&state=default\&region=BELOW_MAIN_CONTENT\&context=storylines_guide}{}

    \hypertarget{donald-trump}{%
    \section{Donald Trump}\label{donald-trump}}

    \hypertarget{republican}{%
    \subsection{Republican}\label{republican}}
  \end{itemize}
\item
  \hypertarget{keep-up-with-our-coverage}{%
  \subsection{Keep Up With Our
  Coverage}\label{keep-up-with-our-coverage}}

  \begin{itemize}
  \item
    Get an
    \href{https://www.nytimes3xbfgragh.onion/newsletters/politics?action=click\&pgtype=Article\&state=default\&region=BELOW_MAIN_CONTENT\&context=storylines_guide}{email}~recapping
    the day's news
  \item
    Download our mobile app on
    \href{https://apps.apple.com/us/app/nytimes/id284862083?ls=1\&mat_click_id=5c79ae7455014fd1bd66b5610c05b8f2-20191112-16948\&referrer=mat_click_id\%3D5c79ae7455014fd1bd66b5610c05b8f2-20191112-16948\%26link_click_id\%3D722930677036718082}{iOS}~and
    \href{http://a.localytics.com/android?id=com.nytimes.android\&referrer=utm_source\%3Dother_nyt_mobile_web\%26utm_medium\%3DWeb\%2520page\%26utm_term\%3DGeneral\%2520Mobile\%2520Page\%26utm_campaign\%3DNYT\%2520Mobile\%2520General\%2520Page}{Android}~and
    turn on Breaking News and Politics alerts
  \end{itemize}
\end{itemize}

Advertisement

\protect\hyperlink{after-bottom}{Continue reading the main story}

\hypertarget{site-index}{%
\subsection{Site Index}\label{site-index}}

\hypertarget{site-information-navigation}{%
\subsection{Site Information
Navigation}\label{site-information-navigation}}

\begin{itemize}
\tightlist
\item
  \href{https://help.nytimes3xbfgragh.onion/hc/en-us/articles/115014792127-Copyright-notice}{©~2020~The
  New York Times Company}
\end{itemize}

\begin{itemize}
\tightlist
\item
  \href{https://www.nytco.com/}{NYTCo}
\item
  \href{https://help.nytimes3xbfgragh.onion/hc/en-us/articles/115015385887-Contact-Us}{Contact
  Us}
\item
  \href{https://www.nytco.com/careers/}{Work with us}
\item
  \href{https://nytmediakit.com/}{Advertise}
\item
  \href{http://www.tbrandstudio.com/}{T Brand Studio}
\item
  \href{https://www.nytimes3xbfgragh.onion/privacy/cookie-policy\#how-do-i-manage-trackers}{Your
  Ad Choices}
\item
  \href{https://www.nytimes3xbfgragh.onion/privacy}{Privacy}
\item
  \href{https://help.nytimes3xbfgragh.onion/hc/en-us/articles/115014893428-Terms-of-service}{Terms
  of Service}
\item
  \href{https://help.nytimes3xbfgragh.onion/hc/en-us/articles/115014893968-Terms-of-sale}{Terms
  of Sale}
\item
  \href{https://spiderbites.nytimes3xbfgragh.onion}{Site Map}
\item
  \href{https://help.nytimes3xbfgragh.onion/hc/en-us}{Help}
\item
  \href{https://www.nytimes3xbfgragh.onion/subscription?campaignId=37WXW}{Subscriptions}
\end{itemize}
