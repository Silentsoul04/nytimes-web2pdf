Can `Athletic Intelligence' Be Measured?

\url{https://nyti.ms/3gT3GPz}

\begin{itemize}
\item
\item
\item
\item
\item
\item
\end{itemize}

\includegraphics{https://static01.graylady3jvrrxbe.onion/images/2020/09/30/magazine/30mag-aiq/30mag-aiq-mediumSquareAt3X.jpg}

Credit...Illustrations by Karan Singh

Sections

\protect\hyperlink{site-content}{Skip to
content}\protect\hyperlink{site-index}{Skip to site index}

The Great ReadFeature

\hypertarget{can-athletic-intelligence-be-measured}{%
\section{Can `Athletic Intelligence' Be
Measured?}\label{can-athletic-intelligence-be-measured}}

Teams in the N.F.L. and other leagues believe performance on a tablet
can predict success in a real game.

Credit...Illustrations by Karan Singh

Supported by

\protect\hyperlink{after-sponsor}{Continue reading the main story}

By Devin Gordon

\begin{itemize}
\item
  Published Sept. 2, 2020Updated Sept. 4, 2020
\item
  \begin{itemize}
  \item
  \item
  \item
  \item
  \item
  \item
  \end{itemize}
\end{itemize}

\hypertarget{listen-to-this-article}{%
\subsubsection{Listen to This Article}\label{listen-to-this-article}}

Audio Recording by Audm

\emph{To hear more audio stories from publishers like The New York
Times,
download}\href{https://www.audm.com/?utm_source=nytmag\&utm_medium=embed\&utm_campaign=left_behind_draper}{**}\href{https://www.audm.com/?utm_source=nytmag\&utm_medium=embed\&utm_campaign=athletic_intelligence_measured_gordon}{\emph{Audm
for iPhone or Android}}\emph{.}

Two N.F.L. **** draft prospects sat on a couch on the shaded patio
outside a hotel room in St. Petersburg, Fla. They were waiting to take
another test. They had only been in town a couple of days, and already
this was their sixth test. Each time it was the same routine: Go to this
room at this time, ask for this person, do whatever he tells you. No one
said what the tests were for, and so far the prospects had never been
told how they did.

The day was a snowbird special for midwinter Florida: hot but not too
hot, humid but not too humid. It was January 2019, and the athletes were
in Florida for the 94th annual East-West Shrine Bowl, a college All-Star
game of sorts whose relevance has been ebbing for years, to the point of
becoming more of a weeklong scouting session with a football game tacked
onto the end. By Saturday's kickoff, at Tropicana Field, most of the
agents and N.F.L. scouts would be gone.

The real action happened during the week, at the TradeWinds Island Grand
hotel, a sprawling, labyrinthine family resort built on St. Pete Beach
in 1985, seemingly by a pirate with vertigo. Eight prospects were
already stuffed inside the hotel room, pre-pandemic style, taking the
test. Wearing earbuds, their college-issue gear and athlete-casual socks
and slides, they all quietly tap-tapped away on iPads. The test proctors
had dragged the couch outside the previous afternoon, when it got
superhot and the bouquet inside the room got too ripe. The two players
out on the couch sat slumped, checking their phones, waiting in that
unhurried, half-awake way of exceptional athletes in recharging mode.

Player 1 was a wide receiver from Princeton named Jesper Horsted, a
classic possession receiver who's gritty, a little too big for the
position, a little too slow, but someone who just knows how to get open.
Player 2 was a taciturn 325-pound offensive lineman the size of a
telecom satellite named Olisaemeka (Oli) Udoh. He came from Elon
College, a small school in North Carolina. Neither Horsted nor Udoh had
ever played on national TV or in a bowl game. They were just hoping to
be drafted in a few months' time. Only one of them would be.

The door to the hotel room opened a crack, and Scott Goldman, co-creator
of the Athletic Intelligence Quotient, or A.I.Q., slipped out onto the
patio. ``Gentlemen,'' he said cheerfully, ``how are you?'' He asked
where they were from, what positions they played, how many tests they
had taken so far that week and apologized for putting them through yet
another one. Then he unspooled his standard introduction to the A.I.Q.
``The test measures sport-specific cognitive abilities,'' he began,
``like how you see the playing field, reaction time. If you think of
sports as an unsolvable puzzle, this is just how you go about solving
it.''

This was Day 2 of data capture for the A.I.Q. team --- another eight
hours of sitting out here on the couch or inside the sweat lodge. But
Goldman, who is 46, with a square head and a voice like a chipper Joe
Pesci, was as bright as the St. Pete sun. Data capture! He loves this
part. Every so often someone will try a strategy he never imagined.
Brains are fascinating that way. Goldman and his business partner, Jim
Bowman, a cognitive psychologist who specializes in test construction
--- a self-described ``assessment nerd,'' with a vast library of old
diagnostic tests stashed in his attic --- have been administering their
test since 2012. More than 6,000 pro athletes across various sports have
taken the A.I.Q. so far, including a vast majority of players currently
in the N.F.L., whether they know it or not.

According to Goldman, their A.I.Q. business topped seven figures in
earnings for the first time in 2017. And yet Bowman, who built the test,
still works full time as a district psychologist in the public-school
system in Great Neck, N.Y., on Long Island, and he has no intentions of
quitting. For Bowman, this started as a night job, and now it's a
considerably more lucrative one. For Goldman, it's pretty much all
A.I.Q. He and his deputy, Alex Auerbach, the company's first and only
employee, spend the months following the college-football season hopping
from one showcase game to another, culminating with a trip to the annual
N.F.L. Draft Combine in April in Indianapolis, where they test the
remaining 400 or so prospects in each class.

Horsted and Udoh were up next, but first Goldman asked if they had any
questions. Udoh did: ``Do teams really use this test?''

Goldman explained that he was here on behalf of the Miami Dolphins, who
had dispatched him to survey this draft class, but that several more
N.F.L. franchises are A.I.Q. clients, who buy tests results à la carte,
cherry-picking the prospects who interest them. (A single report costs
\$300, and unlimited access costs \$100,000 a year for the N.F.L. and
\$25,000 for the N.B.A., because it has a much smaller pool of players.)
Goldman didn't tell Udoh and Horsted who his other clients were, but
they included the Seattle Seahawks, the Atlanta Falcons, the New York
Jets and the Detroit Lions. (The Lions have also hired him separately as
the team's sports psychologist for the 2020 season.)

In a few months, he would give the same spiel to N.B.A. prospects on
behalf of the Golden State Warriors, the Minnesota Timberwolves and the
Washington Wizards, and a few months after that to the Toronto Blue
Jays, in Major League Baseball. The University of Michigan uses the
A.I.Q. a lot --- Goldman lives in Ann Arbor --- as does the University
of Arizona, where he used to work in the sports-psychology department.
``It's fascinating --- it's a component of our process that we truly
believe in,'' says Larry Harris, the Warriors' assistant general
manager. ``I can tell you that a lot of teams are looking at Scott and
what they're doing. We're not the first, and we're not going to be the
last.''

Adam Gase first learned about the A.I.Q. in 2016, during his three-year
stint coaching the Miami Dolphins, and he describes it as love at first
sight, like getting a football version of Lasik surgery. Gase, who now
coaches the New York Jets, is intense and excitable and, like most
N.F.L. coaches, impatient. According to the American Psychological
Association's guideline for test instruments, the A.I.Q.'s data set is
sufficiently robust to draw valid conclusions about whether it works.
But there's a difference between strong correlations and established
science, and football coaches can't wait five years for an advantage to
be empirically proven and then peer-reviewed. There's little academic
research to support the read-option offense, either.

So Gase put Goldman through a test of his own, grilling him about fringe
Dolphins who took the A.I.Q. but hadn't even made the team yet. ``Guys
there was no way he knew anything about,'' Gase told me. Goldman had
never seen them play. All he had were their A.I.Q. scores. His capacity
to turn raw data into the vernacular of pro football, though, startled
the head coach.

``He'd say, `I bet you this guy struggles to learn''' the playbook, Gase
said. ``Or, `This guy takes a second --- like, if he's a punt returner,
he probably runs back and forth because he can't figure out where to go,
but when he does go, it's impressive.' Just all these things about
players that he should not have known. I mean, he was nailing them.''

Before Goldman led Udoh and Horsted inside, he emphasized that the
A.I.Q. was not built to be a crystal ball. What it purports to offer
instead is more like a comprehensive user's manual to an elite athlete's
brain. It's a 10-part test, and scoring ``well'' on any single part
doesn't preordain greatness, nor does scoring poorly mean failure. Some
players score highly across the board; some have peaks and valleys. Your
scores might show you're an athletic savant in several cognitive areas
but not the ones your position demands. The A.I.Q. might expose your
weaknesses, but it might also give your team a head start on how to
overcome them.

Goldman's business style is to underpromise and overdeliver, a tactic
that matches what seems to be his scientific cautiousness, and he takes
care to remind his subjects that the A.I.Q. is still in the
data-gathering stage. But so far, he says, there is a strong
relationship between high test scores across the board --- a full-scale
score --- and the thing athletes care about most: playing time. ``People
with higher A.I.Q. scores tend to get on the field sooner,'' Goldman
told them. ``They tend to start more in their rookie years, and then
they also tend to have longer careers.''

Any more questions? No? He clapped his hands and led them inside.

\includegraphics{https://static01.graylady3jvrrxbe.onion/images/2020/09/30/magazine/30mag-aiq-02/30mag-aiq-02-mediumSquareAt3X.jpg}

\textbf{Evaluating athletic talent} is like playing football in a thick
fog. The object is to isolate raw ability from the countless variables
that impact performance --- to separate the ``smart'' ones from the
``dumb'' ones, the team leaders from the time bombs, and maybe even find
a hidden jewel whose ``intangibles'' can't be measured with a stopwatch.
But in football especially, this is next to impossible, because in
football nothing happens in isolation.

To illustrate the challenge, Goldman brought up a conundrum in the 2019
N.F.L. draft class: a wide receiver from a major conference and a
physical paragon, Goldman said, who missed a significant portion of his
college career because of injuries and was now leaving early to enter
the draft. His statistics were gaudy, but his game tape was limited, and
when he did play, he seemed to run only a few basic routes. Was that
because he couldn't handle more complexity? Or because his quarterback
couldn't? Maybe his coach preferred to run the ball. Maybe the receiver
was a football virtuoso, but his coaches had no idea how to reach him or
had never really tried. Maybe it was the injury he was playing through.
Maybe, maybe, maybe.

Each of the various tests administered to draft prospects measure
different things, but their core purpose is the same: eliminating the
maybes or at least reducing them to a tolerable minimum. There is the
``work habits'' test, full of questions like ``Do you write a paper the
night before it's due, or do you plan ahead?'' There are a pair of
personality tests --- highly subjective, situation-dependent. You might
be calm as a pond under test conditions, but what about when your coach
screams at you or after you get rich? There is the N.F.L.'s own in-house
I.Q. test, a broad, catchall exam called the Player Assessment Test, or
P.A.T., which is a dubious name for a football test, considering that
acronym is already taken for this sport.

The test was designed in 2013 by a psychology professor from Baruch
College and a human-resources management professor from Marymount
University, and now their firm, Siena Consulting, has a deal with the
N.F.L. to administer it during the draft process. According to Michael
Signora, the N.F.L.'s senior vice president of football and
international communications, every team receives the full batch of
P.A.T. scores for every draft class. Whether they look at them or not is
a different matter. None of the N.F.L. sources I spoke with for this
article expressed any use for them. Leonard Zaichkowsky, a retired
Boston University professor of sports psychology and psychometrics and
co-author of ``The Playmaker's Advantage: How to Raise Your Mental Game
to the Next Level,'' says that he has been asking the N.F.L. about the
P.A.T. for years without success. ``It's highly guarded,'' he says.
``And it's unfortunate, because there could be useful information that
they're gathering.''

And then, of course, everyone takes the Wonderlic. If you follow sports,
you've heard of the Wonderlic: an intelligence test graded on a scale
from 1 to 50. It was created in 1937 by a Northwestern University
psychology graduate named Eldon (Al) Wonderlic and based on the brain
science of 1937. Wonderlic Inc., now under the third generation of
family leadership, offers a broad assortment of workplace-related
diagnostics that are still widely administered across corporate America.
The test that N.F.L. players take has been updated over the years, but
it remains basically the same test --- pencil and paper, 50 questions in
12 minutes. An entire brain reduced to a single number.
\href{https://beatthewonderlic.com/take-a-free-wonderlic-test-online/}{You
can take a version of the Wonderlic online right now}. In 2016, Sports
Illustrated published a sample question: ``Are the following two words
similar, contradictory, or not related? Aghast/Unsurprised.''

The N.F.L.'s history with the Wonderlic began in a vegetable garden
outside Chicago in 1975. The Dallas Cowboys' head coach, Tom Landry,
just happened to be driving by when he spotted Wonderlic's headquarters;
recognizing the name, he pulled into the parking lot. Charlie Wonderlic,
who runs the company today, has no idea what Landry was doing in
suburban Chicago. ``It's the story that's been passed down for 45
years,'' he told me. ``I don't know all the details. I just know he
wasn't headed directly there.'' Landry was a Texas type --- lean,
laconic, coached games in a gray suit and a fedora --- and he was
already a legend by this point. So when Landry walked into the lobby and
asked to speak to the man in charge, Al led him to Wonderlic's vegetable
garden for a quick psychology seminar.

From Landry and the Cowboys, the Wonderlic spread to the N.F.L. Combine.
A single flat score is functionally useless for N.F.L. front offices,
but for decades that one score was all they had. The scores were
supposed to be confidential, but they leaked out all the time, and an
unsettling pattern began to emerge: White quarterbacks drew headlines
for sky-high scores, and Black quarterbacks who scored in the single
digits received similar attention. Arguably the player most associated
with the Wonderlic is the former University of Texas quarterback Vince
Young, who reportedly scored a 6 ahead of the 2006 draft --- a clear
harbinger of a future N.F.L. flop, according to the Wonderlic. The
Tennessee Titans still drafted him third over all, though, and he was
named A.F.C. Rookie of the Year in 2006, before soon losing his place in
the league. The Wonderlic is far from the only reason that the
quarterback position (supposedly the ``hardest'' to master) was almost
exclusively white for decades or why skill positions like running back,
wide receiver and cornerback (the ``easiest'') were even more
exclusively Black. For much of the N.F.L.'s recent history, though, it
has supplied bigoted coaches and G.M.s all the rationale they needed to
reinforce their biases.

The league is changing fast. The N.F.L.'s best quarterback right now and
the reigning Super Bowl M.V.P., the Kansas City Chiefs' Patrick Mahomes,
is Black, and the Carolina Panthers' Christian McCaffrey, who may be the
N.F.L.'s best running back, is white. Thanks in large part to the
embrace of data and improvements in understanding how the brain works,
the number of people in the N.F.L. who accept dumb stuff is now in
gradual decline. ``We are evolving so much, I believe, as a league in
understanding that there are racial biases, and it's imperative that we
limit them as much as possible,'' says Thomas Dimitroff, the general
manager of the Atlanta Falcons, who is in the midst of rebuilding his
roster with increasing help from the A.I.Q. ``It's important to us that
we're approaching it in the right way, so we are always trying to find
the test that has no bias.''

Goldman and Bowman say they constantly recalibrate the A.I.Q. so that it
generates useful results from any athlete, in any sport, no matter what
language they speak, whether they grew up wealthy in Alabama or poor in
Switzerland. ``If we do a race comparison, it's not like white people
score higher than African-Americans on our test,'' Goldman says. ``They
don't. In fact, they're equivalent. It's nullified.'' Historically,
intelligence tests have been riddled with all kinds of biases, some of
them pernicious, some that would never cross your mind. ``I grew up in
New Mexico,'' Goldman told me, ``and I remember on one test they would
say, `What is a schooner?' Which is a boat. I lived in New Mexico. I
didn't know what a schooner was.''

The A.I.Q.'s efficacy depends on its capacity to isolate cognitive
ability from ``acquired knowledge.'' Many of the abilities typically
regarded as ``intelligence'' are actually acquired knowledge. Mnemonic
devices are acquired knowledge. Schooner awareness is acquired
knowledge. In the field of intelligence testing for athletics, acquired
knowledge is the enemy, and it must be eliminated, or at least
minimized, for the test to be reliable. Otherwise, according to Goldman,
``it's just testing your ability to take a test.'' I.Q. tests have often
measured privilege as much as intelligence. If you grew up in poverty or
an immigrant community, the tests might as well have been written in
another language.

Bowman designed each of the A.I.Q.'s 10 sections to have the vaguest
feel of sportiness without drawing on acquired knowledge. One section,
for instance, measures navigational and spatial awareness; the task is
to navigate a grid dotted with obstacles in as few moves (taps) as
possible. ``This is the GPS,'' Goldman says. ``Your ability to find the
most efficient route from Point A to Point B.'' For a running back, it
might evoke searching for holes in the offensive and defensive lines. A
linebacker might have the sense of hunting a path to the quarterback.
For basketball players, it's a path to the basket; for soccer players, a
route to the goal. Another section consists of a square grid filled with
patterned glyphs; the goal is to spot three particular glyphs as quickly
and accurately as possible. ``A `Where's Waldo' task,'' Goldman calls
it. ``It's really about locating minute details in a crowded field.'' In
cognitive terms, it measures pattern recognition, rapid decision-making
and long-term information retrieval --- crucial abilities for
quarterbacks who must survey the field and spot receivers in time to get
them the ball. If you're always the last one to find Waldo, chances are
you would make a lousy N.F.L. quarterback.

An athlete's A.I.Q. report consists of a ``full scale'' score, which
tends to fall between 70 at the low end and 130 at the high end, and 10
subscores, each of which measures a particular cognitive ability. A
full-scale score, Goldman told me, is the closest to a conventional I.Q.
score, or a Wonderlic score --- pretty useless, in other words. ``I like
to think of this as country, states, cities,'' he said: The information
becomes richer as you get closer to ground level. A key appeal of the
A.I.Q. rests in its promise to expose bias, force you to face the facts
about yourself, something I experienced firsthand when I took the test.
I've never played in the N.F.L., so for the purposes of the exercise, we
pretended I was a quarterback. Easy enough: I grew up pretending to be a
quarterback, specifically
\href{https://www.nfl.com/videos/miami-dolphins-quarterback-dan-marino-career-highlights-nfl-legends-436975}{the
Miami Dolphins' Hall of Famer Dan Marino.} Stoic in the pocket,
computer-brained, decisive, a human laser

\begin{quote}
show.
\end{quote}

\begin{quote}
\hypertarget{a-key-appeal-of-the-aiq-rests-in-its-promise-to-expose-bias-forcing-you-to-face-the-facts-about-yourself}{%
\subsection{A key appeal of the A.I.Q. rests in its promise to expose
bias, forcing you to face the facts about
yourself.}\label{a-key-appeal-of-the-aiq-rests-in-its-promise-to-expose-bias-forcing-you-to-face-the-facts-about-yourself}}
\end{quote}

It turns out I'm the opposite of Dan Marino. My full-scale score was a
needless reminder of my cognitive mediocrity, and my
information-retrieval scores were so bad that Goldman struggled to put a
sunny spin on them. If my coach rewrites the playbook on Thursday, he
predicted, I will not remember it on Sunday. If you give me four passing
options on a play, I will forget two. What I have instead is the
textbook brain of a scrambling QB. ``Fluid in live time, creative,''
Goldman said. ``You can improvise.'' A smart coach, he explained, would
create offensive schemes to take advantage of what I can do (``put this
guy in open space and let him just go create chaos'') rather than force
me to do what I cannot (remember stuff).

\textbf{Unlike me,} every single player in the N.F.L. is a genius-level
athlete in some respect, and every position requires some form of
superior cognitive ability. So maybe it's not so surprising that Saquon
Barkley and Daniel Jones, the New York Giants' two young emerging stars,
both crushed the A.I.Q. The gifted wideout with all the maybes,
\href{https://www.seahawks.com/video/seahawks-wide-receiver-d-k-metcalf-college-highlights}{D.K.
Metcalf,} went in the second round to the Seahawks, an A.I.Q. client.
(Goldman permitted me to name specific players only if I received
permission from their teams.) He blossomed right away, finishing among
the top three rookie wide receivers in catches, yards and touchdowns.
Udoh, the lineman from Elon, went in the sixth round to the Minnesota
Vikings, and he made the team, but he played in only one game, the
season finale, in Week 17. Horsted, the receiver from Princeton, wasn't
drafted.

The Giants don't use the A.I.Q., and they didn't need it to conclude
that Saquon Barkley is good at football. Where the test really starts to
pay off, Goldman believes, is late in the draft and during the
post-draft free-agent free-for-all, which is when teams are panning for
flakes of gold. And if you'd asked the A.I.Q. after the 2019 draft to
spit back the name of the undrafted rookie most likely to start a N.F.L.
game in his first season, it would beep and whir and give you your
answer: Jesper Horsted. All the way back in January, in that hotel room
in St. Pete, the A.I.Q. spotted him, like a needle in a stack of
needles. This one. This kid.

Aside from quarterback, Goldman told me, tight end is football's most
cognitively demanding position, because tight ends are both blockers and
pass-catchers --- they must remember complex protection schemes as well
as receiving routes. And according to his A.I.Q. results, Jesper
Horsted's brain is built for the job. His full-scale score was 111,
which is very strong, and he scored at least 100 on all 10 sections,
which is very rare, and he tested off the charts on the sections most
relevant to tight ends: ``navigation'' (``the ability to scan a visual
field quickly and effectively and determine the shortest route to the
destination'') plus a pair of sections that measure learning efficiency,
``acquisition'' (how quickly you download information) and, my nemesis,
``recall.'' Horsted scored 121 on the navigation test, 124 on
acquisition and 122 on recall.

Goldman gives players their own A.I.Q. results free, if they ask for
them, which they almost never do. Horsted did, though, right after he
finished the test. Sure, Goldman told him: ``It's your brain.'' The
Chicago Bears are not A.I.Q. clients, but two weeks after the 2019
draft, the team signed Horsted as a tight end. He made a strong
impression during training camp, scoring touchdowns in each of Chicago's
final two preseason games. Then, on the final day of camp, the Bears cut
him. Undrafted rookies pretty much always get cut. A few days later,
though, the Bears signed him to the practice squad --- cannon fodder,
basically, but still. He was on the team. On Nov. 19, he was promoted to
the Bears' active roster. Four days later, in his first N.F.L. game, he
made his first N.F.L. catch. Four days after that, the guy ahead of him
on the depth chart was injured, and Horsted was named the Bears'
starting tight end.

People with higher A.I.Q. scores tend to get on the field sooner,
Goldman had said outside that stuffy hotel room in Florida. They tend to
start more in their rookie years, and then they also tend to have longer
careers.

\textbf{The A.I.Q. got} its start nearly 20 years ago as a two-man ``pop
and pop shop,'' as Goldman calls it. He and Bowman hatched the idea at
the Albert Ellis Institute, a psychotherapy training center in New York,
and during long commutes on the Long Island Railroad. When they first
met in 2004, Goldman and Bowman seemed headed down different career
paths. Goldman wanted to work with pro athletes. Bowman wanted to work
with children. What they had in common, though, was a love for sports.
One night, Goldman shared an idea he was turning over in his head about
building a new kind of sports I.Q. test, one rooted in cognitive theory
and developed with the care and rigor of an academic endeavor.

The idea intrigued Bowman, but he cautioned Goldman that a test of this
sort would need ``a solid theoretical base,'' otherwise, Bowman
explained to me, ``you can just pick and choose factors to measure.''
The logical choice, he told Goldman, was
\href{https://onlinelibrary.wiley.com/doi/full/10.1002/9781118660584.ese0431}{the
Cattell-Horn-Carroll Theory of Cognitive Abilities,} a composite of
multiple peer-reviewed advances in the field of brain science that
represents the most current understanding of how intelligence actually
works.

According to Dawn P. Flanagan, a professor of psychology at St. John's
University and a leading expert in the field of intelligence assessment,
the C.H.C. ``is really a taxonomy of human cognitive abilities,'' and it
covers a number of abilities that ``people don't typically think of as
intelligence.'' The C.H.C. breaks down ``intelligence'' into a minimum
of 10 ``broad'' cognitive abilities, such as reasoning, visual
processing and short-term memory. Each broad ability comprises several
``narrow'' abilities; there are more than 70 in total. The broader
category of reasoning ability includes narrower categories such as
inductive reasoning, deductive reasoning and quantitative reasoning.
Something like visual-spatial processing may not sound like conventional
intelligence, but it comes in handy if you're a pilot or a surgeon ---
or a wide receiver, whose job depends on sprinting in one direction
while looking over your shoulder in the other, with only a split second
to locate and catch an oblong shape spiraling toward you at 50 m.p.h.
through a thicket of large arms.

For the better part of a decade, Goldman and Bowman chipped away at
their secret project on nights and weekends. They put their own money
into it, each borrowing about \$25,000 from their retirement savings.
Lots of N.F.L. teams have tests that evaluate individual cognitive
abilities --- tap tests for reaction time, that sort of thing. A
comprehensive test like the A.I.Q., though, is ``a very expensive
endeavor,'' Flanagan says, and creating it properly takes a long time.
``There's no Wal-Mart for this,'' Bowman says. A critical breakthrough
came from outside their own efforts: the invention of the tablet. From
the start, Bowman knew that a paper-and-pencil test would cut into the
attention span of some athletes. Laptops, meanwhile, were too expensive
and cumbersome. IPads were a gift from the assessment gods.

There are a number of ways to measure a test's value. One critical
factor is reliability. Are the results stable? For the A.I.Q., the
answer is yes. Another essential factor is accuracy. Does a test show
``concurrent validity'' --- that is, do other similar, reputable tests
show comparable results? Again, Bowman says the answer is yes. According
to his and Goldman's research, the A.I.Q. correlates as expected with
other established measures, like the ImPACT test (Immediate
Post-Concussion Assessment and Cognitive Testing) and the Wonderlic. The
problem is, none of Wonderlic's data shows any specific correlation with
being good or bad at football. ``The abilities it measures are toward
the academic end of the intelligence spectrum, and they can be important
in a lot of jobs,'' Bowman says. ``It really doesn't support its use in
athletics.'' All of the N.F.L. sources I interviewed for this article
told me they pay no attention to it. ``Everybody's just very cautious
of: What does the Wonderlic test really tell me?'' says Gase, the Jets
head coach. ``It doesn't give me anything.'' When I shared with Charlie
Wonderlic the critical remarks from Gase and others about the test,
though, he sounded more perplexed than offended.

``Three hundred million people have taken this test,'' he said. ``It
works.''

The promise of the A.I.Q., meanwhile, is the radical notion that once
you stop looking at a player's brain as a single test score and instead
as a multilayered instrument, it changes your view of the person too. So
many talented athletes have been tossed aside because they can't
memorize a playbook, no matter how hard they try, or they got lost in a
scheme that was all wrong for their talents. The purpose of the A.I.Q.
isn't just to identify overlooked talent; it's to change the perception
of them as distressed assets in the first

\begin{quote}
place.
\end{quote}

\begin{quote}
\hypertarget{the-purpose-of-the-aiq-isnt-just-to-identify-overlooked-talent-its-to-change-the-perception-of-some-athletes-as-distressed-assets-in-the-first-place}{%
\subsection{The purpose of the A.I.Q. isn't just to identify overlooked
talent; it's to change the perception of some athletes as distressed
assets in the first
place.}\label{the-purpose-of-the-aiq-isnt-just-to-identify-overlooked-talent-its-to-change-the-perception-of-some-athletes-as-distressed-assets-in-the-first-place}}
\end{quote}

\textbf{When Adam Gase} was hired by the Jets in early 2019, Goldman's
pre-existing deal with the Dolphins precluded contact with a division
rival, which meant that for his first N.F.L. draft with the Jets that
April, Gase had no access to the A.I.Q. ``I felt naked without it,'' he
told me.

The Miami Dolphins' exclusivity clause ended after the season, though,
so Gase and Goldman were reunited for this April's Covid-era bizarro
draft, completed under quarantine conditions and directed via Microsoft
Teams from the basement den of the N.F.L. commissioner, Roger Goodell.
Goldman was on call for the entire week leading up to the draft. ``At
grocery stores, out with his kids, in his backyard --- wherever it was,
day or night, impromptu, he was very, very accessible,'' Dimitroff, the
Falcons' G.M., says. The virus lopped off the critical final weeks of
the pre-draft period, which, according to Gase and Dimitroff, only
increased the value of the data they got from the A.I.Q. And with the
pandemic threatening to upend, if not entirely wipe out, the fall
college football season, the information vacuum might be even more
pronounced in 2021. What if there's no game tape at all?

On draft night, the Jets used the 11th overall pick in the first round
to select \href{https://www.youtube.com/watch?v=FEGDmXiKRgA}{an
offensive lineman from Louisville named Mekhi Becton}, who is listed at
364 pounds, nearly 30 pounds heavier than William (The Refrigerator)
Perry was back in his mid-1980s prime. (``When I met him at the
combine,'' Gase says of Becton, ``I just remember thinking, This is a
very large person.'') When the Jets' pick came up, Becton's A.I.Q.
results helped reaffirm Gase's confidence. In their one-on-one
interview, Gase noticed that Becton effortlessly recalled specific
pass-blocking decisions from his college career. He suspected Becton
might have strong long-term retrieval skills, and the A.I.Q. said so,
too.

There's an adage in football that Super Bowls are won in the off-season,
which is mostly nonsense --- really, the Super Bowl is won by whichever
team keeps its elite quarterback alive long enough --- but this season
it could be prophetic. N.F.L. teams that draft, teach and communicate
well always have an advantage over the league's laggards; this season,
after a truncated spring and summer, that advantage will only grow.

One week after he recorded his first N.F.L. catch for the Bears, on Nov.
28, Jesper Horsted made his first N.F.L. start against the Detroit
Lions. Thanksgiving Day. National television. With just under five
minutes left in the third quarter, Detroit leading 17-10, the Bears
drove inside the 20-yard line, and Horsted, No. 49, lined up in the
slot. He wasn't the first option on the play, he told me later, but when
he noticed pre-snap that he had a favorable matchup, he knew the ball
might come his way. He also knew which of the three receiving patterns
in his route tree would get him open. In other words, he made the
correct adjustment, just as the A.I.Q. predicted he would.

At the snap, Horsted ran a deep crossing route up the middle for 18
yards, making a beeline for the end zone. The Chicago quarterback,
Mitchell Trubisky, lofted a nice pass, but it was going to be a tough
catch --- over the shoulder, through traffic. Twenty-seven million
people were watching at home as
\href{https://twitter.com/NFL/status/1200142027486310406?s=20}{Fox
announcer Joe Buck made the call.}

``Trubisky on first down. Floats. The pass iiis \ldots{} juggled and
caught for the touchdown! Horsted!''

\begin{center}\rule{0.5\linewidth}{\linethickness}\end{center}

Devin Gordon is a writer based in Brookline, Mass. He is the author of
``So Many Ways To Lose: The Amazin' True Story of the New York Mets, the
Best Worst Team in Sports,'' to be published March 2021.

Advertisement

\protect\hyperlink{after-bottom}{Continue reading the main story}

\hypertarget{site-index}{%
\subsection{Site Index}\label{site-index}}

\hypertarget{site-information-navigation}{%
\subsection{Site Information
Navigation}\label{site-information-navigation}}

\begin{itemize}
\tightlist
\item
  \href{https://help.nytimes3xbfgragh.onion/hc/en-us/articles/115014792127-Copyright-notice}{©~2020~The
  New York Times Company}
\end{itemize}

\begin{itemize}
\tightlist
\item
  \href{https://www.nytco.com/}{NYTCo}
\item
  \href{https://help.nytimes3xbfgragh.onion/hc/en-us/articles/115015385887-Contact-Us}{Contact
  Us}
\item
  \href{https://www.nytco.com/careers/}{Work with us}
\item
  \href{https://nytmediakit.com/}{Advertise}
\item
  \href{http://www.tbrandstudio.com/}{T Brand Studio}
\item
  \href{https://www.nytimes3xbfgragh.onion/privacy/cookie-policy\#how-do-i-manage-trackers}{Your
  Ad Choices}
\item
  \href{https://www.nytimes3xbfgragh.onion/privacy}{Privacy}
\item
  \href{https://help.nytimes3xbfgragh.onion/hc/en-us/articles/115014893428-Terms-of-service}{Terms
  of Service}
\item
  \href{https://help.nytimes3xbfgragh.onion/hc/en-us/articles/115014893968-Terms-of-sale}{Terms
  of Sale}
\item
  \href{https://spiderbites.nytimes3xbfgragh.onion}{Site Map}
\item
  \href{https://help.nytimes3xbfgragh.onion/hc/en-us}{Help}
\item
  \href{https://www.nytimes3xbfgragh.onion/subscription?campaignId=37WXW}{Subscriptions}
\end{itemize}
