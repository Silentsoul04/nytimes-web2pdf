Sections

SEARCH

\protect\hyperlink{site-content}{Skip to
content}\protect\hyperlink{site-index}{Skip to site index}

\href{https://www.nytimes3xbfgragh.onion/section/travel}{Travel}

\href{https://myaccount.nytimes3xbfgragh.onion/auth/login?response_type=cookie\&client_id=vi}{}

\href{https://www.nytimes3xbfgragh.onion/section/todayspaper}{Today's
Paper}

\href{/section/travel}{Travel}\textbar{}Do Safari Companies Really Want
African Travelers?

\url{https://nyti.ms/32S9aoE}

\begin{itemize}
\item
\item
\item
\item
\item
\end{itemize}

\hypertarget{the-coronavirus-outbreak}{%
\subsubsection{\texorpdfstring{\href{https://www.nytimes3xbfgragh.onion/news-event/coronavirus?name=styln-coronavirus-national\&region=TOP_BANNER\&block=storyline_menu_recirc\&action=click\&pgtype=Article\&impression_id=805bfd90-f1ad-11ea-b69e-dd4be0f736de\&variant=undefined}{The
Coronavirus
Outbreak}}{The Coronavirus Outbreak}}\label{the-coronavirus-outbreak}}

\begin{itemize}
\tightlist
\item
  live\href{https://www.nytimes3xbfgragh.onion/2020/09/07/world/covid-19-coronavirus.html?name=styln-coronavirus-national\&region=TOP_BANNER\&block=storyline_menu_recirc\&action=click\&pgtype=Article\&impression_id=805bfd91-f1ad-11ea-b69e-dd4be0f736de\&variant=undefined}{Latest
  Updates}
\item
  \href{https://www.nytimes3xbfgragh.onion/interactive/2020/us/coronavirus-us-cases.html?name=styln-coronavirus-national\&region=TOP_BANNER\&block=storyline_menu_recirc\&action=click\&pgtype=Article\&impression_id=805bfd92-f1ad-11ea-b69e-dd4be0f736de\&variant=undefined}{Maps
  and Cases}
\item
  \href{https://www.nytimes3xbfgragh.onion/interactive/2020/science/coronavirus-vaccine-tracker.html?name=styln-coronavirus-national\&region=TOP_BANNER\&block=storyline_menu_recirc\&action=click\&pgtype=Article\&impression_id=805bfd93-f1ad-11ea-b69e-dd4be0f736de\&variant=undefined}{Vaccine
  Tracker}
\item
  \href{https://www.nytimes3xbfgragh.onion/2020/09/02/your-money/eviction-moratorium-covid.html?name=styln-coronavirus-national\&region=TOP_BANNER\&block=storyline_menu_recirc\&action=click\&pgtype=Article\&impression_id=805bfd94-f1ad-11ea-b69e-dd4be0f736de\&variant=undefined}{Eviction
  Moratorium}
\item
  \href{https://www.nytimes3xbfgragh.onion/interactive/2020/09/02/magazine/food-insecurity-hunger-us.html?name=styln-coronavirus-national\&region=TOP_BANNER\&block=storyline_menu_recirc\&action=click\&pgtype=Article\&impression_id=805c24a0-f1ad-11ea-b69e-dd4be0f736de\&variant=undefined}{American
  Hunger}
\end{itemize}

Advertisement

\protect\hyperlink{after-top}{Continue reading the main story}

Supported by

\protect\hyperlink{after-sponsor}{Continue reading the main story}

\hypertarget{do-safari-companies-really-want-african-travelers}{%
\section{Do Safari Companies Really Want African
Travelers?}\label{do-safari-companies-really-want-african-travelers}}

During these lean times for tourism, travel companies are appealing to
residents with special rates. But locals ask: Why didn't you reach out
before?

\includegraphics{https://static01.graylady3jvrrxbe.onion/images/2020/09/02/travel/02safari/02safari-articleLarge.jpg?quality=75\&auto=webp\&disable=upscale}

\href{https://www.nytimes3xbfgragh.onion/by/tariro-mzezewa}{\includegraphics{https://static01.graylady3jvrrxbe.onion/images/2018/08/24/opinion/tariro-headshot/tariro-headshot-thumbLarge-v2.png}}

By \href{https://www.nytimes3xbfgragh.onion/by/tariro-mzezewa}{Tariro
Mzezewa}

\begin{itemize}
\item
  Sept. 2, 2020
\item
  \begin{itemize}
  \item
  \item
  \item
  \item
  \item
  \end{itemize}
\end{itemize}

When the Safari Collection, which owns the Giraffe Manor, a popular
hotel in Nairobi known for its roaming giraffes, posted a photo on its
Instagram account in June with a caption saying that the hotel would be
``welcoming those who can still travel within Nairobi'' and Kenyans
could pay a special reduced rate to stay at the hotel, it received
backlash from the very people it hoped to entice.

For many Kenyans, the offer was seen as insulting and only extended
because foreign visitors, the lifeblood of the safari industry, were
largely absent because of restrictions on international travel. Offering
local rates seemed like a desperate ploy from a hotel where locals say
they previously never felt welcome.

``My eyes just rolled when I saw that,''
\href{https://www.instagram.com/harrietowalla/}{Harriet Akinyi}, a
Nairobi-based travel writer, said in a phone interview. ``It's
hypocritical that it took a pandemic for them to realize that they have
to cater to the Kenyan market as well, not just the international
market.''

\href{https://www.thesafaricollection.com/properties/giraffe-manor/}{The
Safari Collection} said that the post was published by an independent
agent and the wording was misconstrued. The hotel, the company said, has
always been open to all guests, but this was the first time it ran a
special, because it was the first time it had the empty rooms to do so.

``The situation with Covid-19 has resulted in an opportunity to welcome
local guests who wouldn't normally book in advance --- we are usually
booked up months in advance,'' said Jessica Pattison, head of sales and
marketing for the Safari Collection, in an email, adding that the
company employs more than 240 people and 95 percent of them are Black.
The company would not reveal its local rates, but for international
visitors, rooms range from \$875 per night per person to \$3,000 for a
suite sleeping up to four per night, depending on the season and type of
room.

Still, the incident renewed conversations among Kenyans and other
Africans who live in countries that are home to safari companies, tour
operators and luxury lodgings that focus on attracting American and
European visitors. Who, they ask, is really welcome to take part in
these offerings? People in South Africa, Kenya, Tanzania and Botswana
said that the local rates have allowed them to enjoy their own countries
in recent months, but as borders start to reopen, they remain skeptical
about whether they'll still be invited. Ms. Akinyi shared an email from
\href{https://www.arijiju.com/}{Arijiju}, a private-home retreat in
Laikipia, where she was told that the residence had `a strict target
list of top-tier international media,' so she could not visit.

Border closures, national lockdowns and other measures put in place to
stop the spread of the coronavirus have devastated Africa's
\href{https://www.brookings.edu/wp-content/uploads/2018/12/Africas-tourism-potential_LandrySigne1.pdf}{\$39
billion tourism industry}. The tourism industry funds wildlife
conservation across the continent and the lack of international tourists
has led experts to fear that threatened animals would be poached at
higher rates, further endangering them.

Safari travel as marketed internationally is largely a luxury product,
with beautifully appointed tent camps or lodges appealing to wealthy
travelers. At the high end it can cost thousands of dollars a night,
with guests flown between remote camps on private planes; even at the
more modest end of the spectrum the cost tends to run to hundreds of
dollars a night per person. That puts them far beyond the means of many
Africans: In South Africa, for example, where Kruger National Park is a
major draw for safari vacationers, the average earnings for Black
residents between 2011 and 2015 were 6,899 rand (\$413) per month,
according to
\href{http://www.statssa.gov.za/publications/Report-03-10-19/Report-03-10-192017.pdf}{a
2019 report} by the country's Department of Statistics. The same report
showed that for white South Africans, who tend to go on more safari
vacations, average earnings were 24,646 rand (\$1,400) per month ---
more than three times what Black South Africans make.

Some think it is time to reshape the industry.

``We can't expect that Covid is the last issue that may cause a pause in
travel, and companies have now seen that solely relying on international
travelers is not sustainable,'' said Naledi Khabo, chief executive of
\href{https://africatourismassociation.org/}{Africa Tourism
Association}, a U.S.-based agency that promotes tourism to and within
Africa. ``Having a consistent base of local travelers will be key to any
safari business who wants to survive.''

\hypertarget{latest-updates-the-coronavirus-outbreak}{%
\section{\texorpdfstring{\href{https://www.nytimes3xbfgragh.onion/2020/09/07/world/covid-19-coronavirus.html?action=click\&pgtype=Article\&state=default\&region=MAIN_CONTENT_1\&context=storylines_live_updates}{Latest
Updates: The Coronavirus
Outbreak}}{Latest Updates: The Coronavirus Outbreak}}\label{latest-updates-the-coronavirus-outbreak}}

Updated 2020-09-08T08:03:39.372Z

\begin{itemize}
\tightlist
\item
  \href{https://www.nytimes3xbfgragh.onion/2020/09/07/world/covid-19-coronavirus.html?action=click\&pgtype=Article\&state=default\&region=MAIN_CONTENT_1\&context=storylines_live_updates\#link-71fba04f}{For
  reopened U.S. universities, parties and pushback are upending plans.}
\item
  \href{https://www.nytimes3xbfgragh.onion/2020/09/07/world/covid-19-coronavirus.html?action=click\&pgtype=Article\&state=default\&region=MAIN_CONTENT_1\&context=storylines_live_updates\#link-adc17f7}{China's
  leader declares success in suppressing the country's outbreak.}
\item
  \href{https://www.nytimes3xbfgragh.onion/2020/09/07/world/covid-19-coronavirus.html?action=click\&pgtype=Article\&state=default\&region=MAIN_CONTENT_1\&context=storylines_live_updates\#link-47c32b10}{On
  a holiday weekend in the U.S., the virus continues its grinding hold
  on Americans' lives.}
\end{itemize}

\href{https://www.nytimes3xbfgragh.onion/2020/09/07/world/covid-19-coronavirus.html?action=click\&pgtype=Article\&state=default\&region=MAIN_CONTENT_1\&context=storylines_live_updates}{See
more updates}

More live coverage:

In Tanzania, another country with a fast-growing middle class that is a
popular destination for safaris and luxury stays, local rates have
become the norm. Lodges and safaris are being marketed at half price in
some cases and at even less than that at others.

Andrew Mahiga, a project manager for a development project in Dar es
Salaam, said that when he returned to Tanzania a decade ago, after
living in the United States and Britain, he was committed to
\href{https://www.instagram.com/drudysseus/?hl=en}{traveling the
country} and becoming reacquainted with his home. But he and his fiancée
have noticed that at some hotels and resorts, the services offered to
locals can differ vastly from those offered to international tourists.
That hasn't changed since the start of the pandemic.

``Even though places are becoming more accessible to us right now, the
experiences are still not equal to those offered to foreigners,'' he
said. ``When I stay someplace, they aren't as invested in me as a guest.
They don't give suggestions for places nearby to see or excursions to go
on and things to try, but foreigners --- that's all provided at
length.''

Mr. Mahiga, 35, said that recently he and his fiancée have been focusing
on staying at Airbnbs that are owned by Tanzanians and going on tours
that are run by locals.

Although many safari and lodging companies have Black African guides and
staff, the African Travel and Tourism Association estimates that 15
percent of its 600-plus members are
\href{https://www.cntraveller.com/article/locally-owned-safari-camps}{Black
owners}, something that locals say plays a part in the feeling that they
are not welcome.

``These luxury resorts and companies that focus on foreigners are
finally being friendly to local Tanzanians and that's good,'' Mr. Mahiga
said. ``But since Covid started I've found myself wondering, `Why don't
I support really local business, especially when the foreign ones never
wanted my money before?'''

For their part, companies say that locals tend to plan their trips later
than foreign visitors, so that usually when they inquire, they are
already booked.

\href{https://www.nytimes3xbfgragh.onion/news-event/coronavirus?action=click\&pgtype=Article\&state=default\&region=MAIN_CONTENT_3\&context=storylines_faq}{}

\hypertarget{the-coronavirus-outbreak-}{%
\subsubsection{The Coronavirus Outbreak
›}\label{the-coronavirus-outbreak-}}

\hypertarget{frequently-asked-questions}{%
\paragraph{Frequently Asked
Questions}\label{frequently-asked-questions}}

Updated September 4, 2020

\begin{itemize}
\item ~
  \hypertarget{what-are-the-symptoms-of-coronavirus}{%
  \paragraph{What are the symptoms of
  coronavirus?}\label{what-are-the-symptoms-of-coronavirus}}

  \begin{itemize}
  \tightlist
  \item
    In the beginning, the coronavirus
    \href{https://www.nytimes3xbfgragh.onion/article/coronavirus-facts-history.html?action=click\&pgtype=Article\&state=default\&region=MAIN_CONTENT_3\&context=storylines_faq\#link-6817bab5}{seemed
    like it was primarily a respiratory illness}~--- many patients had
    fever and chills, were weak and tired, and coughed a lot, though
    some people don't show many symptoms at all. Those who seemed
    sickest had pneumonia or acute respiratory distress syndrome and
    received supplemental oxygen. By now, doctors have identified many
    more symptoms and syndromes. In April,
    \href{https://www.nytimes3xbfgragh.onion/2020/04/27/health/coronavirus-symptoms-cdc.html?action=click\&pgtype=Article\&state=default\&region=MAIN_CONTENT_3\&context=storylines_faq}{the
    C.D.C. added to the list of early signs}~sore throat, fever, chills
    and muscle aches. Gastrointestinal upset, such as diarrhea and
    nausea, has also been observed. Another telltale sign of infection
    may be a sudden, profound diminution of one's
    \href{https://www.nytimes3xbfgragh.onion/2020/03/22/health/coronavirus-symptoms-smell-taste.html?action=click\&pgtype=Article\&state=default\&region=MAIN_CONTENT_3\&context=storylines_faq}{sense
    of smell and taste.}~Teenagers and young adults in some cases have
    developed painful red and purple lesions on their fingers and toes
    --- nicknamed ``Covid toe'' --- but few other serious symptoms.
  \end{itemize}
\item ~
  \hypertarget{why-is-it-safer-to-spend-time-together-outside}{%
  \paragraph{Why is it safer to spend time together
  outside?}\label{why-is-it-safer-to-spend-time-together-outside}}

  \begin{itemize}
  \tightlist
  \item
    \href{https://www.nytimes3xbfgragh.onion/2020/05/15/us/coronavirus-what-to-do-outside.html?action=click\&pgtype=Article\&state=default\&region=MAIN_CONTENT_3\&context=storylines_faq}{Outdoor
    gatherings}~lower risk because wind disperses viral droplets, and
    sunlight can kill some of the virus. Open spaces prevent the virus
    from building up in concentrated amounts and being inhaled, which
    can happen when infected people exhale in a confined space for long
    stretches of time, said Dr. Julian W. Tang, a virologist at the
    University of Leicester.
  \end{itemize}
\item ~
  \hypertarget{why-does-standing-six-feet-away-from-others-help}{%
  \paragraph{Why does standing six feet away from others
  help?}\label{why-does-standing-six-feet-away-from-others-help}}

  \begin{itemize}
  \tightlist
  \item
    The coronavirus spreads primarily through droplets from your mouth
    and nose, especially when you cough or sneeze. The C.D.C., one of
    the organizations using that measure,
    \href{https://www.nytimes3xbfgragh.onion/2020/04/14/health/coronavirus-six-feet.html?action=click\&pgtype=Article\&state=default\&region=MAIN_CONTENT_3\&context=storylines_faq}{bases
    its recommendation of six feet}~on the idea that most large droplets
    that people expel when they cough or sneeze will fall to the ground
    within six feet. But six feet has never been a magic number that
    guarantees complete protection. Sneezes, for instance, can launch
    droplets a lot farther than six feet,
    \href{https://jamanetwork.com/journals/jama/fullarticle/2763852}{according
    to a recent study}. It's a rule of thumb: You should be safest
    standing six feet apart outside, especially when it's windy. But
    keep a mask on at all times, even when you think you're far enough
    apart.
  \end{itemize}
\item ~
  \hypertarget{i-have-antibodies-am-i-now-immune}{%
  \paragraph{I have antibodies. Am I now
  immune?}\label{i-have-antibodies-am-i-now-immune}}

  \begin{itemize}
  \tightlist
  \item
    As of right
    now,\href{https://www.nytimes3xbfgragh.onion/2020/07/22/health/covid-antibodies-herd-immunity.html?action=click\&pgtype=Article\&state=default\&region=MAIN_CONTENT_3\&context=storylines_faq}{~that
    seems likely, for at least several months.}~There have been
    frightening accounts of people suffering what seems to be a second
    bout of Covid-19. But experts say these patients may have a
    drawn-out course of infection, with the virus taking a slow toll
    weeks to months after initial exposure.~People infected with the
    coronavirus typically
    \href{https://www.nature.com/articles/s41586-020-2456-9}{produce}~immune
    molecules called antibodies, which are
    \href{https://www.nytimes3xbfgragh.onion/2020/05/07/health/coronavirus-antibody-prevalence.html?action=click\&pgtype=Article\&state=default\&region=MAIN_CONTENT_3\&context=storylines_faq}{protective
    proteins made in response to an
    infection}\href{https://www.nytimes3xbfgragh.onion/2020/05/07/health/coronavirus-antibody-prevalence.html?action=click\&pgtype=Article\&state=default\&region=MAIN_CONTENT_3\&context=storylines_faq}{.
    These antibodies may}~last in the body
    \href{https://www.nature.com/articles/s41591-020-0965-6}{only two to
    three months}, which may seem worrisome, but that's~perfectly normal
    after an acute infection subsides, said Dr. Michael Mina, an
    immunologist at Harvard University. It may be possible to get the
    coronavirus again, but it's highly unlikely that it would be
    possible in a short window of time from initial infection or make
    people sicker the second time.
  \end{itemize}
\item ~
  \hypertarget{what-are-my-rights-if-i-am-worried-about-going-back-to-work}{%
  \paragraph{What are my rights if I am worried about going back to
  work?}\label{what-are-my-rights-if-i-am-worried-about-going-back-to-work}}

  \begin{itemize}
  \tightlist
  \item
    Employers have to provide
    \href{https://www.osha.gov/SLTC/covid-19/standards.html}{a safe
    workplace}~with policies that protect everyone equally.
    \href{https://www.nytimes3xbfgragh.onion/article/coronavirus-money-unemployment.html?action=click\&pgtype=Article\&state=default\&region=MAIN_CONTENT_3\&context=storylines_faq}{And
    if one of your co-workers tests positive for the coronavirus, the
    C.D.C.}~has said that
    \href{https://www.cdc.gov/coronavirus/2019-ncov/community/guidance-business-response.html}{employers
    should tell their employees}~-\/- without giving you the sick
    employee's name -\/- that they may have been exposed to the virus.
  \end{itemize}
\end{itemize}

For Beks Ndlovu, the founder of
\href{https://africanbushcamps.com/}{African Bush Camps}, an
independently owned safari company, promoting local rates has always
been a key part of operating a business in any country. Mr. Ndlovu's
company has 15 luxury tented camps and lodges in Botswana, Zimbabwe and
Zambia, and has, for years, offered advantageous local rates to
residents of countries that are part of the Southern African Development
Community, a regional economic community. For foreign guests, depending
on the season, a stay can run between \$400 and \$950 per person per
night, but for locals and people from the region it is \$250 to \$380
per person per night.

``This is not something that's new to us,'' he said. ``We've actively
promoted our offerings and the rate we offer is very favorable to locals
because we understand the earnings in this part of the world are
different from that of the international traveler.''

Mr. Ndlovu, who is Zimbabwean, said that offering local rates isn't
enough; he believes that his company has been successful among locals in
the countries where it has camps because locals are treated as well as
Europeans and Americans are --- something that goes a long way, he said.

Some people, like Lelo Boyana, who works in finance in Johannesburg and
hosts the \href{https://chicatravel.co.za/}{travel podcast Chica
Travel}, worry that the push for local guests won't last past the
pandemic. Ms. Boyana said that although she has taken advantage of local
rates
\href{https://www.reuters.com/article/us-health-coronavirus-safrica-tourism/with-borders-closed-south-africa-pins-hopes-on-cash-strapped-local-tourists-idUSKBN25R1LI}{throughout
South Africa} this year, she remains skeptical of how much of the money
spent by travelers goes to locals, another common criticism of safari
companies. Travelers, she said, need to ask more questions about where
their money is going and companies need to do more than discount stays.

``I think what's obvious is that these companies are desperate for
business,'' she said. ``If they wanted to have us, they would have made
these rates available to us long ago and they would have publicized them
before the pandemic. We are their last resort and their attempts are
halfhearted. They still have a lot of work to do.''

Ms. Khabo of the African Tourism Association said she thought this might
be a turning point. As leisure travel within Africa becomes more
popular, with new flight routes and the easing of visa policies, it
would be bad business for companies to ignore African buying power. She
expects the shift to a focus on domestic travel to continue after the
pandemic.

``There is now a new opportunity to engage the local market and
cultivate local ambassadors, so we will see which companies keep it
going,'' she said.

In many cases, the decision to cater to local residents more
aggressively is keeping camps busy.
\href{https://www.asiliaafrica.com/}{Asilia Africa}, a company that
operates 20 camps in East Africa, has always had local rates, but this
year marks the first time the company has actively marketed them to
residents. In previous years, people found out about the discounts by
word of mouth and the occasional mention in the local press, Mercedes
Bailey, a press manager at Asilia said.

Staying open and keeping rangers in the bush discouraged poaching, Ms.
Bailey added. Resident sales in June were about 60 percent higher than
during the same period last year. Those sales have also never been
higher than 20 percent, but the company expects them to surpass 50
percent this year.

``We have a social media campaign, targeting Africans,'' Ms. Bailey said
in a phone interview. ``That's something we haven't done before, but
people are coming and we are now asking ourselves: Are people coming now
because they can't go anywhere else or have they always wanted to come,
but didn't feel welcome?''

\begin{center}\rule{0.5\linewidth}{\linethickness}\end{center}

\emph{\textbf{Follow New York Times Travel}}
\emph{on}\href{https://www.instagram.com/nytimestravel/}{\emph{Instagram}}\emph{,}\href{https://twitter.com/nytimestravel}{\emph{Twitter}}
\emph{and}\href{https://www.facebookcorewwwi.onion/nytimestravel/}{\emph{Facebook}}\emph{.
And}\href{https://www.nytimes3xbfgragh.onion/newsletters/traveldispatch}{\emph{sign
up for our weekly Travel Dispatch newsletter}} \emph{to receive expert
tips on traveling smarter and inspiration for your next vacation.}

Advertisement

\protect\hyperlink{after-bottom}{Continue reading the main story}

\hypertarget{site-index}{%
\subsection{Site Index}\label{site-index}}

\hypertarget{site-information-navigation}{%
\subsection{Site Information
Navigation}\label{site-information-navigation}}

\begin{itemize}
\tightlist
\item
  \href{https://help.nytimes3xbfgragh.onion/hc/en-us/articles/115014792127-Copyright-notice}{©~2020~The
  New York Times Company}
\end{itemize}

\begin{itemize}
\tightlist
\item
  \href{https://www.nytco.com/}{NYTCo}
\item
  \href{https://help.nytimes3xbfgragh.onion/hc/en-us/articles/115015385887-Contact-Us}{Contact
  Us}
\item
  \href{https://www.nytco.com/careers/}{Work with us}
\item
  \href{https://nytmediakit.com/}{Advertise}
\item
  \href{http://www.tbrandstudio.com/}{T Brand Studio}
\item
  \href{https://www.nytimes3xbfgragh.onion/privacy/cookie-policy\#how-do-i-manage-trackers}{Your
  Ad Choices}
\item
  \href{https://www.nytimes3xbfgragh.onion/privacy}{Privacy}
\item
  \href{https://help.nytimes3xbfgragh.onion/hc/en-us/articles/115014893428-Terms-of-service}{Terms
  of Service}
\item
  \href{https://help.nytimes3xbfgragh.onion/hc/en-us/articles/115014893968-Terms-of-sale}{Terms
  of Sale}
\item
  \href{https://spiderbites.nytimes3xbfgragh.onion}{Site Map}
\item
  \href{https://help.nytimes3xbfgragh.onion/hc/en-us}{Help}
\item
  \href{https://www.nytimes3xbfgragh.onion/subscription?campaignId=37WXW}{Subscriptions}
\end{itemize}
