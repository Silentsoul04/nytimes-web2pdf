Sections

SEARCH

\protect\hyperlink{site-content}{Skip to
content}\protect\hyperlink{site-index}{Skip to site index}

\href{https://www.nytimes3xbfgragh.onion/section/nyregion}{New York}

\href{https://myaccount.nytimes3xbfgragh.onion/auth/login?response_type=cookie\&client_id=vi}{}

\href{https://www.nytimes3xbfgragh.onion/section/todayspaper}{Today's
Paper}

\href{/section/nyregion}{New York}\textbar{}7 Police Officers Suspended
as a Black Man's Suffocation Roils Rochester

\url{https://nyti.ms/351M0Px}

\begin{itemize}
\item
\item
\item
\item
\item
\item
\end{itemize}

\hypertarget{race-and-america}{%
\subsubsection{\texorpdfstring{\href{https://www.nytimes3xbfgragh.onion/news-event/george-floyd-protests-minneapolis-new-york-los-angeles?name=styln-george-floyd\&region=TOP_BANNER\&block=storyline_menu_recirc\&action=click\&pgtype=Article\&impression_id=c6423970-f291-11ea-8ecb-a3874efea1d0\&variant=undefined}{Race
and America}}{Race and America}}\label{race-and-america}}

\begin{itemize}
\tightlist
\item
  \href{https://www.nytimes3xbfgragh.onion/2020/09/04/nyregion/rochester-police-daniel-prude.html?name=styln-george-floyd\&region=TOP_BANNER\&block=storyline_menu_recirc\&action=click\&pgtype=Article\&impression_id=c6423971-f291-11ea-8ecb-a3874efea1d0\&variant=undefined}{What
  Happened in Rochester, N.Y.}
\item
  \href{https://www.nytimes3xbfgragh.onion/2020/09/01/us/politics/trump-fact-check-protests.html?name=styln-george-floyd\&region=TOP_BANNER\&block=storyline_menu_recirc\&action=click\&pgtype=Article\&impression_id=c6426080-f291-11ea-8ecb-a3874efea1d0\&variant=undefined}{Trump
  Fact Check}
\item
  \href{https://www.nytimes3xbfgragh.onion/2020/08/30/us/portland-shooting-explained.html?name=styln-george-floyd\&region=TOP_BANNER\&block=storyline_menu_recirc\&action=click\&pgtype=Article\&impression_id=c6426081-f291-11ea-8ecb-a3874efea1d0\&variant=undefined}{Portland
  Shooting}
\item
  \href{https://www.nytimes3xbfgragh.onion/2020/08/30/us/breonna-taylor-police-killing.html?name=styln-george-floyd\&region=TOP_BANNER\&block=storyline_menu_recirc\&action=click\&pgtype=Article\&impression_id=c6426082-f291-11ea-8ecb-a3874efea1d0\&variant=undefined}{Breonna
  Taylor's Life and Death}
\end{itemize}

Advertisement

\protect\hyperlink{after-top}{Continue reading the main story}

Supported by

\protect\hyperlink{after-sponsor}{Continue reading the main story}

\hypertarget{7-police-officers-suspended-as-a-black-mans-suffocation-roils-rochester}{%
\section{7 Police Officers Suspended as a Black Man's Suffocation Roils
Rochester}\label{7-police-officers-suspended-as-a-black-mans-suffocation-roils-rochester}}

Daniel Prude, who was having a psychotic episode, died after police
officers placed a mesh hood over his head in March.

\includegraphics{https://static01.graylady3jvrrxbe.onion/images/2020/09/05/nyregion/03rochester/03rochester-articleLarge-v2.jpg?quality=75\&auto=webp\&disable=upscale}

By \href{https://www.nytimes3xbfgragh.onion/by/sarah-maslin-nir}{Sarah
Maslin Nir},
\href{https://www.nytimes3xbfgragh.onion/by/michael-wilson}{Michael
Wilson}, \href{https://www.nytimes3xbfgragh.onion/by/troy-closson}{Troy
Closson} and
\href{https://www.nytimes3xbfgragh.onion/by/jesse-mckinley}{Jesse
McKinley}

\begin{itemize}
\item
  Published Sept. 3, 2020Updated Sept. 8, 2020
\item
  \begin{itemize}
  \item
  \item
  \item
  \item
  \item
  \item
  \end{itemize}
\end{itemize}

\href{https://www.nytimes3xbfgragh.onion/es/2020/09/04/espanol/estados-unidos/daniel-prude-rochester-policia.html}{Leer
en español}

ROCHESTER, N.Y. --- Seven
\href{https://www.nytimes3xbfgragh.onion/2020/09/08/nyregion/rochester-police-chief-resigns-prude.html}{Rochester
police} officers were suspended on Thursday in the suffocation of a
Black man as he was being detained in March, although the mayor and
senior state officials faced escalating questions about why more than
five months passed before action was taken.

The man, Daniel Prude, who was having a psychotic episode,
\href{https://www.nytimes3xbfgragh.onion/2020/09/02/nyregion/daniel-prude-rochester-police.html}{was
handcuffed by officers after he ran into the street} naked in the middle
of the cold night and told at least one passer-by that he had the
coronavirus. Mr. Prude began spitting, and the officers responded by
\href{https://www.nytimes3xbfgragh.onion/2020/09/03/nyregion/spit-hoods-police.html}{pulling
a mesh hood over his head}, according to police body camera footage.

When he tried to rise, the officers forced Mr. Prude face down on the
ground, one of them pushing his head to the pavement, the video footage
showed. Mr. Prude was held down by the police for two minutes, and had
to be resuscitated. He died a week later at the hospital.

His death did not receive widespread attention until Wednesday, when his
family
\href{https://www.democratandchronicle.com/story/news/2020/09/02/daniel-prude-rochester-ny-police-died-march-2020-after-officers-restrained-him/5682948002/}{released
raw police videos} of the encounter, which they just obtained through an
open records request. The scene --- a Black man, handcuffed and sitting
in a street, wearing nothing but a white hood --- seemed a shocking
combination of physical helplessness and racist imagery from another
era.

Image

Daniel Prude, 41, arrived in Rochester shortly before his encounter with
the police, and quickly began acting erratically, his brother
said.~Credit...Roth and Roth LLP, via Associated Press

Rochester, a city of 200,000 in Western New York, became the latest city
to be roiled by the death of a Black person in police custody, with
protesters taking to the streets.

By about 9:45 p.m. on Thursday, a crowd of perhaps 100 demonstrators had
gathered outside Rochester's Public Safety Building on Exchange
Boulevard. People were sitting, singing, chanting and eating pizza.

At around 10:30 p.m., the dozen or so police officers who had been
monitoring the demonstrators from behind a barricade were joined by
around 20 reinforcements in riot gear.

The officers suddenly surged toward the barricade and began firing an
irritant into the crowd. It was unclear what led them to do so.

The protesters pushed into the barricade toward the police, prompting
the officers to fire the irritant again, as protesters yelled, ``Why?
Why?''

The back and forth continued for 45 minutes or so, with the police
repeatedly firing irritant.

The disciplinary action against the seven officers was the first in
response to Mr. Prude's death. In a news conference on Thursday
afternoon, Mayor Lovely Warren apologized to the Prude family, saying
that Mr. Prude had been failed ``by our police department, our mental
health care system, our society. And he was failed by me.''

Ms. Warren did not offer details on why the investigations into the
March 23 encounter had taken so long, but suggested that she had been
misled by the police chief, La'Ron D. Singletary.

``Experiencing and ultimately dying from the drug overdose in police
custody, as I was told by the chief, is entirely different than what I
ultimately witnessed, on the video,'' the mayor said.

Chief Singletary bristled on Wednesday at the suggestion that his
department had been trying to keep Mr. Prude's death away from public
attention.

``This is not a cover-up,'' he said, adding that he ordered criminal and
internal investigations hours after the encounter. He stood by the
officers' response to what had initially been a mental-health related
call: ``Our job is to try to get some sort of medical intervention, and
that's exactly what happened that night.''

On Wednesday, the state attorney general, Letitia James, made her first
statement on the case, offering condolences to Mr. Prude's family and
promising ``a fair and independent investigation.''

``We will work tirelessly to provide the transparency and accountability
that all our communities deserve,'' she said.

\includegraphics{https://static01.graylady3jvrrxbe.onion/images/2020/09/03/nyregion/03rochester-02/03rochester-02-articleLarge.jpg?quality=75\&auto=webp\&disable=upscale}

Investigations into police-related killings of unarmed civilians in New
York are overseen by Ms. James's office, and findings of fact are not
publicized until complete. In Mr. Prude's case, Ms. James's
investigation began in April, and is continuing.

Still, in the wake of high-profile police killings around the country,
including the deaths of George Floyd and Breonna Taylor, and the
shooting of Jacob Blake, the lag between Mr. Prude's death and public
calls for justice by Ms. James and Gov. Andrew M. Cuomo --- both
Democrats who have been outspoken on the issue of police brutality ---
seemed jarring.

The Monroe County medical examiner ruled Mr. Prude's death a homicide
caused by ``complications of asphyxia in the setting of physical
restraint,'' according to an autopsy report.

``Excited delirium'' and acute intoxication by phencyclidine, or the
drug PCP, were contributing factors, the report said.

The city did not identify the officers who were suspended.

Interviews, police records and body camera footage offer an unusually
detailed timeline of what happened the night that Mr. Prude was detained
by officers.

A light snow was falling, the streets empty and dark at 3 a.m. on March
23, when the call came in over the police radio: A naked man, Mr. Prude,
41, was running outside, under the influence of PCP, and shouting that
he had the coronavirus.

The hours leading up to the encounter with the police were troubled ones
for Mr. Prude, who was struggling with some combination of suicidal
fantasy and drug use that an hourslong admittance to a hospital did
nothing to treat.

The day before, Mr. Prude had arrived in Rochester. His brother, Joe
Prude, had picked him up from a shelter in nearby Buffalo after Daniel
Prude had been kicked off a train from Chicago, where he lived, Joe
Prude told the police.

Image

The release of the police video footage of Mr. Prude's arrest led to
protests in Rochester.Credit...Joshua Rashaad McFadden for The New York
Times

But soon after, Mr. Prude began behaving erratically, accusing his
brother of wanting to kill him and even seemingly trying to take his own
life. ``He jumped 21 stairs down to my basement, head first,'' Joe Prude
told the police.

Joe Prude had his brother admitted to Strong Memorial Hospital for an
evaluation. Mr. Prude was released hours later, returning to Joe Prude's
home, where he seemed to have calmed down. But then he asked for a
cigarette, and when his brother rose to get one, he bolted out a back
door, barely dressed.

Joe Prude called the police, giving a description of what his brother
had been wearing --- ``white tank top, black long-johns, no shoes, no
coat'' --- and saying he seemed to be under the influence of PCP. He
told an officer that he feared Daniel may have run toward the sound of
an approaching train, to possibly try again to hurt himself.

As they spoke, a call came over the officer's radio: ``There's a male at
the location with blood all over him telling the complainant he's sick
and not wearing clothes,'' a dispatcher announced.

Joe Prude, hearing the call, said, ``That's my brother.''

Body camera footage shows officers arriving at 3:16 a.m. near downtown
Rochester, their headlights illuminating a naked Mr. Prude in the
roadway. The police believe he had broken a store window with a brick,
and minutes earlier had stopped a passing tow truck driver and told him
he had the coronavirus.

An officer stepped out of his police vehicle, pointed a Taser at Mr.
Prude and ordered him to get on the ground. Mr. Prude immediately
obeyed, lying face down and spread-eagled. He did not resist as officers
handcuffed him behind his back.

He alternately demanded officers ``get off me'' while imploring, ``In
Jesus Christ I pray,'' at one point asking for money, and at others, for
a gun. Officers could be heard chuckling in the background, until Mr.
Prude grew more agitated: ``Give me your gun. I need it.'' All the
while, he remained sitting.

Image

Mr. Prude's death was ruled a homicide caused by ``complications of
asphyxia in the setting of physical restraint.'' Acute intoxication was
a contributing factor.Credit...Rochester Police Department, via
Associated Press

The officers seemed preoccupied with concern that they might catch
something from Mr. Prude. A day earlier, with the coronavirus quickly
spreading, the state ordered all nonessential workers to stay at home.
``Sir, you don't got AIDS, do you?'' one asked.

Mr. Prude spit on the ground multiple times, and while not aiming at the
officers, his action drew their attention. ``Stop spitting,'' one said.
``Anybody got a spit sock?'' another asked, referring to the device
commonly carried by the police and used by corrections officers.

At 3:19 a.m., an officer unfolded a white hood, approached Mr. Prude
from behind and pulled it over his head, where it hung loosely. Mr.
Prude began rolling in the road, pleading for it to be taken off.

A minute later, after spitting repeatedly inside the hood and shouting,
``Give me the gun,'' Mr. Prude seemed to try to rise to his feet. Three
officers who had been keeping a distance hurried forward and pushed him
to the street.

One officer, identified as Mark Vaughn, held Mr. Prude's head facedown,
seeming to push it to the street as he held a fistful of the hood.

Mr. Prude's angry protests turned tearful, then devolved into incoherent
grunts and gurgling sounds, according to the video. An officer asked
him, ``You good, man?'' There was no reply.

``He's puking, just straight water,'' an officer said. ``You see that
water come out of his mouth?''

An ambulance arrived. ``Roll him on his back,'' a paramedic instructed
as officers searched for a handcuff key. A paramedic began performing
CPR as Mr. Prude remained handcuffed.

Finally, the handcuffs were removed, and Mr. Prude was placed on a
stretcher and into the ambulance, where he was given shots of
epinephrine and sodium bicarbonate, and soon after, his heartbeat
returned on its own, according to a police report.

The same officer who had questioned Mr. Prude's brother earlier that
morning returned to say Mr. Prude had been found and hospitalized. Joe
Prude seemed relieved. ``I'm glad he went that way,'' he told the
officer, ``and not the way of that damn train.''

Mr. Prude lived in Chicago with his sister, and had five adult children.
One of his three daughters, Tashyra Prude, said she felt ``instant
rage'' when she saw the video this week.

``The person that everybody sees in the video is totally different from
the person that I knew,'' she said.

She is starting college this fall. ``This is something I wanted to go
through with my father by my side, and I've just been deprived of this
experience because of what happened, and it just breaks my heart,'' she
said.

On Wednesday evening, as outrage over the circumstances of Mr. Prude's
death spread, Mr. Cuomo said he had not seen the body camera footage.

By Thursday, however, the governor was calling for answers, saying the
video was ``deeply disturbing,'' and urging a quickening of the
investigation.

``For the sake of Mr. Prude's family and the greater Rochester
community, I am calling for this case to be concluded ‎as expeditiously
as possible,'' the governor said in a statement. ``For that to occur, we
need the full and timely cooperation of the Rochester Police Department
and I trust it will fully comply‎.''

With the release of the camera footage, Joe Prude's assessment of that
night in March was filled with outrage. ``I placed a phone call to get
my brother help,'' he told reporters on Wednesday, ``not to have my
brother lynched.''

Advertisement

\protect\hyperlink{after-bottom}{Continue reading the main story}

\hypertarget{site-index}{%
\subsection{Site Index}\label{site-index}}

\hypertarget{site-information-navigation}{%
\subsection{Site Information
Navigation}\label{site-information-navigation}}

\begin{itemize}
\tightlist
\item
  \href{https://help.nytimes3xbfgragh.onion/hc/en-us/articles/115014792127-Copyright-notice}{©~2020~The
  New York Times Company}
\end{itemize}

\begin{itemize}
\tightlist
\item
  \href{https://www.nytco.com/}{NYTCo}
\item
  \href{https://help.nytimes3xbfgragh.onion/hc/en-us/articles/115015385887-Contact-Us}{Contact
  Us}
\item
  \href{https://www.nytco.com/careers/}{Work with us}
\item
  \href{https://nytmediakit.com/}{Advertise}
\item
  \href{http://www.tbrandstudio.com/}{T Brand Studio}
\item
  \href{https://www.nytimes3xbfgragh.onion/privacy/cookie-policy\#how-do-i-manage-trackers}{Your
  Ad Choices}
\item
  \href{https://www.nytimes3xbfgragh.onion/privacy}{Privacy}
\item
  \href{https://help.nytimes3xbfgragh.onion/hc/en-us/articles/115014893428-Terms-of-service}{Terms
  of Service}
\item
  \href{https://help.nytimes3xbfgragh.onion/hc/en-us/articles/115014893968-Terms-of-sale}{Terms
  of Sale}
\item
  \href{https://spiderbites.nytimes3xbfgragh.onion}{Site Map}
\item
  \href{https://help.nytimes3xbfgragh.onion/hc/en-us}{Help}
\item
  \href{https://www.nytimes3xbfgragh.onion/subscription?campaignId=37WXW}{Subscriptions}
\end{itemize}
