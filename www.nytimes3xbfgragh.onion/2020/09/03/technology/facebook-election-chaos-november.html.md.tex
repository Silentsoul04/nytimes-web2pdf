Sections

SEARCH

\protect\hyperlink{site-content}{Skip to
content}\protect\hyperlink{site-index}{Skip to site index}

\href{https://www.nytimes3xbfgragh.onion/section/technology}{Technology}

\href{https://myaccount.nytimes3xbfgragh.onion/auth/login?response_type=cookie\&client_id=vi}{}

\href{https://www.nytimes3xbfgragh.onion/section/todayspaper}{Today's
Paper}

\href{/section/technology}{Technology}\textbar{}Facebook Moves to Limit
Election Chaos in November

\url{https://nyti.ms/3hWlg6H}

\begin{itemize}
\item
\item
\item
\item
\item
\item
\end{itemize}

\begin{itemize}
\item
  \href{https://www.nytimes3xbfgragh.onion/live/2020/09/09/us/trump-vs-biden?action=click\&pgtype=Article\&state=default\&region=TOP_BANNER\&context=storylines_menu}{Election
  Updates}
\item
  \href{https://www.nytimes3xbfgragh.onion/interactive/2020/09/08/us/elections/results-new-hampshire-primary-elections.html?action=click\&pgtype=Article\&state=default\&region=TOP_BANNER\&context=storylines_menu}{New
  Hampshire Results}
\item
  \href{https://www.nytimes3xbfgragh.onion/interactive/2020/us/elections/election-states-biden-trump.html?action=click\&pgtype=Article\&state=default\&region=TOP_BANNER\&context=storylines_menu}{Paths
  to 270}
\item
  \href{https://www.nytimes3xbfgragh.onion/interactive/2020/08/31/us/politics/vote-by-mail-deadlines.html?action=click\&pgtype=Article\&state=default\&region=TOP_BANNER\&context=storylines_menu}{Voting
  by Mail}
\item
  \href{https://www.nytimes3xbfgragh.onion/interactive/2019/us/elections/2020-presidential-election-calendar.html?action=click\&pgtype=Article\&state=default\&region=TOP_BANNER\&context=storylines_menu}{Key
  Dates}
\item
  \href{https://www.nytimes3xbfgragh.onion/newsletters/politics?action=click\&pgtype=Article\&state=default\&region=TOP_BANNER\&context=storylines_menu}{Politics
  Newsletter}
\end{itemize}

Advertisement

\protect\hyperlink{after-top}{Continue reading the main story}

Supported by

\protect\hyperlink{after-sponsor}{Continue reading the main story}

\hypertarget{facebook-moves-to-limit-election-chaos-in-november}{%
\section{Facebook Moves to Limit Election Chaos in
November}\label{facebook-moves-to-limit-election-chaos-in-november}}

The social network said it would block new political ads in late
October, among other measures, to reduce misinformation and
interference.

\includegraphics{https://static01.graylady3jvrrxbe.onion/images/2020/09/02/technology/03facebook-election/oakImage-1599090952249-articleLarge.jpg?quality=75\&auto=webp\&disable=upscale}

\href{https://www.nytimes3xbfgragh.onion/by/mike-isaac}{\includegraphics{https://static01.graylady3jvrrxbe.onion/images/2018/02/16/multimedia/author-mike-isaac/author-mike-isaac-thumbLarge.jpg}}

By \href{https://www.nytimes3xbfgragh.onion/by/mike-isaac}{Mike Isaac}

\begin{itemize}
\item
  Sept. 3, 2020
\item
  \begin{itemize}
  \item
  \item
  \item
  \item
  \item
  \item
  \end{itemize}
\end{itemize}

SAN FRANCISCO --- Facebook on Thursday moved to clamp down on any
confusion about the November election on its service, rolling out
\href{https://about.fb.com/news/2020/09/additional-steps-to-protect-the-us-elections/}{a
sweeping set of changes} to try to limit voter misinformation and
prevent interference from President Trump and other politicians.

In an acknowledgment of how powerful its effect on public discourse can
be,
\href{https://www.nytimes3xbfgragh.onion/2020/09/04/technology/facebooks-political-ads-block-election.html}{Facebook
said it planned to bar any new political ads} on its site in the week
before Election Day. The social network said it would also strengthen
measures against posts that tried to dissuade people from voting.
Postelection, it said, it will quash any candidates' attempts at
claiming false victories by redirecting users to accurate information on
the results.

Facebook is bracing for what is set to be a highly contentious
presidential election. With two months to go, President Trump and Joseph
R. Biden Jr. have ratcheted up their attacks against each other,
clashing over issues including the
\href{https://www.nytimes3xbfgragh.onion/news-event/coronavirus?name=styln-coronavirus\&region=TOP_BANNER\&block=storyline_menu_recirc\&action=click\&pgtype=Article\&impression_id=af12e9e0-ed8f-11ea-80b3-d97d3f5df025\&variant=1_Show}{coronavirus
pandemic} and
\href{https://www.nytimes3xbfgragh.onion/2020/09/02/us/politics/biden-ads-trump.html?action=click\&module=Top\%20Stories\&pgtype=Homepage}{racial
unrest}. Mr. Trump, who uses social media as a megaphone, has suggested
that even when the results are in,
\href{https://www.latimes.com/world-nation/story/2020-07-19/trump-wont-commit-to-accepting-election-result}{he
may not accept them}, and he has questioned the legitimacy of mail-in
voting.

Mark Zuckerberg, Facebook's chief executive, wrote
\href{https://www.facebookcorewwwi.onion/zuck/posts/10112270823363411}{in
a post} on Thursday that the divisions in the United States and the
prospect of taking days or weeks to finalize election results could lead
``to an increased risk of civil unrest across the country.''

Facebook's changes indicate how proactive the Silicon Valley company has
become on election interference, especially after it was slow to react
to
\href{https://www.nytimes3xbfgragh.onion/2018/02/17/technology/indictment-russian-tech-facebook.html}{Russians
using the service in 2016 to sway the American electorate} and promote
Mr. Trump. Since then, Mr. Zuckerberg has worked to prevent the social
network from being misused, aiming to turn the tide of negative
perception about his company.

But Facebook's moves may already be too little and too late, critics
said. Some of the measures, such as the blocking of new political ads a
week before Election Day, are temporary. Misinformation and other toxic
content also flows freely on Facebook outside of ads, including in
private Facebook groups and in posts by users, which the company's
changes do not address.

Some of the actions may unintentionally make Facebook even more
politicized before the election, critics said. When political ads are
blocked on the site, right-wing publishers on Facebook, such as
Breitbart and Fox News, could fill the vacuum, said Tara McGowan, the
chief executive of the liberal nonprofit group Acronym.

``By banning new political ads in the final critical days of the 2020
election, Facebook has decided to tip the scales of the election to
those with the greatest followings on Facebook --- and that includes
President Trump and the right-wing media that serves him,'' she said in
a statement.

The Trump campaign disputed that, saying people would instead be
influenced on Facebook by ads from ``biased'' media. It added that the
social network was censoring politicians.

``When millions of voters will be making their decisions, the president
will be silenced by the Silicon Valley mafia, who will at the same time
allow corporate media to run their biased ads to swing voters in key
states,'' said Samantha Zager, a Trump campaign spokeswoman.

Hours after rolling out its changes, Facebook applied its new rules to
one of Mr. Trump's
\href{https://www.facebookcorewwwi.onion/DonaldTrump/posts/10165384569280725?__xts__\%5B0\%5D=68.ARBXuBeKPLt7PFhVsUpmef8NGDwtIYkEF4DyPrhNbUWSuDR3Bhpwz67vHV0OtoWTKE6l3pCk3cQpOdvPjT-BDXU1HvoTvVIlP5l7S4-7C1Lg57sqxCcthaE2RQBjKPHij04vBj6kmGKP3DKvi-pKE8rJmRxtwJcGN6tVaGRABLUoXus844c-AFuq8XBx7QBEd9MygrrEJ_Gq6Ujkx7UqhgZUyO6nD8ej_3_nqyNAyS5bO81NSIkcAt-AISrH1ru6957VfTBMNk9-9qUQsnLkXFKuuytPv29ozrcbYyc4lyRNTs5G3uUTuLBzNoQR8G-nSOjuAvRZ4LXeKvpv\&__tn__=-R}{posts}
on his Facebook page, in which he cast doubt on the vote-by-mail
process. The company added a warning label that read, ``Voting by mail
has a long history of trustworthiness in the U.S. and the same is
predicted this year.''

Mr. Biden's campaign didn't immediately return a request for comment.

Other social media companies, including YouTube and Twitter, have also
moved to minimize political manipulation on their platforms. Twitter
\href{https://www.nytimes3xbfgragh.onion/2019/10/30/technology/twitter-political-ads-ban.html}{banned
political advertising last year} and has added labels to politicians'
tweets. On Thursday, Twitter also added a label to the tweet sent by Mr.
Trump about voting that echoed his language on Facebook. YouTube has
confirmed that it was holding conversations on postelection strategy,
but has declined to elaborate.

Facebook, a key battleground for both presidential campaigns, has been
most in the eye because of its billions of users. It has faced
increasing scrutiny in recent months as domestic misinformation about
this year's election has proliferated. Yet Mr. Zuckerberg has declined
to remove much of that false information, saying that Facebook supports
free speech and that politicians' posts are newsworthy. Many of the
\href{https://www.nytimes3xbfgragh.onion/2020/06/01/technology/facebook-employee-protest-trump.html}{company's
own employees have objected} to that position.

On Tuesday, Facebook said the Kremlin-backed group that interfered in
the 2016 presidential election, the Internet Research Agency, had
\href{https://www.nytimes3xbfgragh.onion/2020/09/01/technology/facebook-russia-disinformation-election.html}{tried
to meddle on its}service again using fake accounts and a website set up
to look like a left-wing news site. Facebook, which was warned by the
Federal Bureau of Investigation about the Russian effort, said it had
removed the fake accounts and news site before they gained much
traction.

In his post, Mr. Zuckerberg said Facebook had removed over 100 networks
worldwide in the last four years that were trying to influence
elections. But increasingly, the threats to undermine the legitimacy of
the November election were coming ``from within our own borders,'' he
said.

\href{https://www.nytimes3xbfgragh.onion/news-event/2020-election}{Election
2020 ›}

\hypertarget{live-updates}{%
\subsection{\texorpdfstring{\href{https://www.nytimes3xbfgragh.onion/live/2020/09/09/us/trump-vs-biden}{Live
Updates}}{Live Updates}}\label{live-updates}}

\href{https://www.nytimes3xbfgragh.onion/live/2020/09/09/us/trump-vs-biden\#biden-maintains-a-narrow-lead-in-wisconsin-and-widens-his-advantage-in-pennsylvania-new-polls-find}{}

Sept. 9, 2020, 2:31 p.m. ET

\href{https://www.nytimes3xbfgragh.onion/live/2020/09/09/us/trump-vs-biden\#biden-maintains-a-narrow-lead-in-wisconsin-and-widens-his-advantage-in-pennsylvania-new-polls-find}{Biden
maintains a narrow lead in Wisconsin and widens his advantage in
Pennsylvania, new polls
find.}\href{https://www.nytimes3xbfgragh.onion/live/2020/09/09/us/trump-vs-biden\#trumps-campaign-was-badly-outraised-by-biden-and-the-democrats-last-month}{}

Sept. 9, 2020, 2:07 p.m. ET

\href{https://www.nytimes3xbfgragh.onion/live/2020/09/09/us/trump-vs-biden\#trumps-campaign-was-badly-outraised-by-biden-and-the-democrats-last-month}{Trump's
campaign was badly outraised by Biden and the Democrats last
month.}\href{https://www.nytimes3xbfgragh.onion/live/2020/09/09/us/trump-vs-biden\#a-new-poll-shows-most-voters-think-trump-did-a-bad-job-on-the-virus-and-doubt-he-can-help-the-country-recover}{}

Sept. 9, 2020, 1:15 p.m. ET

\href{https://www.nytimes3xbfgragh.onion/live/2020/09/09/us/trump-vs-biden\#a-new-poll-shows-most-voters-think-trump-did-a-bad-job-on-the-virus-and-doubt-he-can-help-the-country-recover}{A
new poll shows most voters think Trump did a bad job on the virus and
doubt he can help the country recover.}

As a result, Facebook said, it will begin barring politicians from
placing new ads on Facebook and Instagram, the photo-sharing service it
owns, on Oct. 27. Existing political ads will not be affected. Political
candidates will still be able to adjust both the groups of people their
existing ads are targeting and the amount of money they spend on those
ads. They can resume running new political ads after Election Day, the
company said.

In another change, Facebook said it would place a voting information
center --- a hub for accurate, up-to-date information on how, when and
where to register to vote --- at the top of its News Feed through
Election Day. The company had rolled out the
\href{https://www.nytimes3xbfgragh.onion/2020/08/13/us/politics/facebook-election-hub-vote-by-mail.html}{voter
information center} in June and has continued
\href{https://www.nytimes3xbfgragh.onion/2020/06/16/technology/opt-out-political-ads-facebook.html}{promoting
it}, with a goal of registering four million people and encouraging them
to vote.

To curb voting misinformation, Facebook said, it will remove posts that
tell people they will catch Covid-19 if they vote. For posts that used
the coronavirus to discourage people from voting in other, less obvious
ways, the company said it would attach a label and link to its voter
information center.

Facebook also widened its removal of posts that both explicitly and
implicitly aim to disenfranchise people from voting; previously, the
company took down only posts that actively discouraged people from
voting. Now, a post that causes confusion around who is eligible to vote
or some part of the voting process --- such as a misstatement about what
documentation is needed to receive a ballot --- will also be expunged.

The company added that it would limit the number of people that users
could forward messages to in its Messenger app to no more than five
people, down from more than 150. The move mirrors what WhatsApp, the
messaging app also owned by Facebook, did in 2018 when it limited
message forwarding to 20 people from a previous maximum of 250. WhatsApp
now limits message forwarding to five people maximum.

Misinformation across private communication channels is a much more
difficult problem to tackle than on public social networks because it is
hidden. Limiting message forwarding could slow that spread.

To get accurate information on the election's results, Facebook said, it
plans to work with Reuters, the news organization, to provide verified
results to the voting information center. If any candidate tried
declaring victory falsely or preemptively, Facebook said, it would add a
label to those posts directing users to the official outcome.

\href{https://www.nytimes3xbfgragh.onion/2020/08/21/technology/facebook-trump-election.html}{Facebook
teams have worked for months} to walk through different contingency
plans for how to handle the election. The company has built an arsenal
of tools and products to safeguard elections in the past four years. It
has also invited people in government, think tanks and academia to
participate.

In recent months, Mr. Zuckerberg and some of his lieutenants had started
holding daily meetings about minimizing how the platform could be used
to dispute the election, people with knowledge of the company have said.
Last month, Facebook employees asked how the social network would act if
Mr. Trump tried to cast doubt on the election results, and Mr.
Zuckerberg, at a staff meeting, said he found the president's comment on
mail-in voting ``quite troubling.''

The chief executive helped drive the new election-related changes,
according to two people familiar with the company, who declined to be
identified because the details are confidential. On Tuesday, Mr.
Zuckerberg and his wife, Dr. Priscilla Chan, separately
\href{https://www.axios.com/mark-zuckerberg-priscilla-chan-election-security-a4950a93-2efd-42a6-9d7a-5fcc763f9214.html}{donated
\$300 million} to support voting infrastructure and security efforts.

Mr. Zuckerberg added that Facebook would not make any further changes to
its site between now and when there was an official election result.

``It's going to take a concerted effort by all of us --- political
parties and candidates, election authorities, the media and social
networks, and ultimately voters as well --- to live up to our
responsibilities,'' he said.

Maggie Haberman and Nick Corasaniti contributed reporting.

\hypertarget{our-2020-election-guide}{%
\section{Our 2020 Election Guide}\label{our-2020-election-guide}}

Updated ~Sept. 9, 2020

\begin{center}\rule{0.5\linewidth}{\linethickness}\end{center}

\begin{itemize}
\item ~
  \hypertarget{the-latest}{%
  \subsection{The Latest}\label{the-latest}}

  \begin{itemize}
  \item
    Joe Biden heads today to Michigan, a battleground state where
    President Trump has resumed advertising ahead of a visit there on
    Thursday.
    \href{https://www.nytimes3xbfgragh.onion/live/2020/09/09/us/trump-vs-biden?action=click\&pgtype=Article\&state=default\&region=BELOW_MAIN_CONTENT\&context=storylines_guide}{Read
    live updates}.
  \end{itemize}
\item ~
  \hypertarget{how-to-win-270}{%
  \subsection{How to Win 270}\label{how-to-win-270}}

  \begin{itemize}
  \item
    Joe Biden and Donald Trump need 270 electoral votes to reach the
    White House. Try building
    \href{https://www.nytimes3xbfgragh.onion/interactive/2020/us/elections/election-states-biden-trump.html?action=click\&pgtype=Article\&state=default\&region=BELOW_MAIN_CONTENT\&context=storylines_guide}{your
    own coalition of battleground states}~to see potential outcomes.
  \end{itemize}
\item ~
  \hypertarget{voting-by-mail}{%
  \subsection{Voting by Mail}\label{voting-by-mail}}

  \begin{itemize}
  \item
    Will you have enough time to vote by mail in your state? Yes, but
    it's risky to procrastinate.
    \href{https://www.nytimes3xbfgragh.onion/interactive/2020/08/31/us/politics/vote-by-mail-deadlines.html?action=click\&pgtype=Article\&state=default\&region=BELOW_MAIN_CONTENT\&context=storylines_guide}{Check
    your state's deadline.}
  \item
    \href{https://www.nytimes3xbfgragh.onion/interactive/2020/us/elections/joe-biden.html?action=click\&pgtype=Article\&state=default\&region=BELOW_MAIN_CONTENT\&context=storylines_guide}{}

    \hypertarget{joe-biden}{%
    \section{Joe Biden}\label{joe-biden}}

    \hypertarget{democrat}{%
    \subsection{Democrat}\label{democrat}}

    \href{https://www.nytimes3xbfgragh.onion/interactive/2020/us/elections/donald-trump.html?action=click\&pgtype=Article\&state=default\&region=BELOW_MAIN_CONTENT\&context=storylines_guide}{}

    \hypertarget{donald-trump}{%
    \section{Donald Trump}\label{donald-trump}}

    \hypertarget{republican}{%
    \subsection{Republican}\label{republican}}
  \end{itemize}
\item
  \hypertarget{keep-up-with-our-coverage}{%
  \subsection{Keep Up With Our
  Coverage}\label{keep-up-with-our-coverage}}

  \begin{itemize}
  \item
    Get an
    \href{https://www.nytimes3xbfgragh.onion/newsletters/politics?action=click\&pgtype=Article\&state=default\&region=BELOW_MAIN_CONTENT\&context=storylines_guide}{email}~recapping
    the day's news
  \item
    Download our mobile app on
    \href{https://apps.apple.com/us/app/nytimes/id284862083?ls=1\&mat_click_id=5c79ae7455014fd1bd66b5610c05b8f2-20191112-16948\&referrer=mat_click_id\%3D5c79ae7455014fd1bd66b5610c05b8f2-20191112-16948\%26link_click_id\%3D722930677036718082}{iOS}~and
    \href{http://a.localytics.com/android?id=com.nytimes.android\&referrer=utm_source\%3Dother_nyt_mobile_web\%26utm_medium\%3DWeb\%2520page\%26utm_term\%3DGeneral\%2520Mobile\%2520Page\%26utm_campaign\%3DNYT\%2520Mobile\%2520General\%2520Page}{Android}~and
    turn on Breaking News and Politics alerts
  \end{itemize}
\end{itemize}

Advertisement

\protect\hyperlink{after-bottom}{Continue reading the main story}

\hypertarget{site-index}{%
\subsection{Site Index}\label{site-index}}

\hypertarget{site-information-navigation}{%
\subsection{Site Information
Navigation}\label{site-information-navigation}}

\begin{itemize}
\tightlist
\item
  \href{https://help.nytimes3xbfgragh.onion/hc/en-us/articles/115014792127-Copyright-notice}{©~2020~The
  New York Times Company}
\end{itemize}

\begin{itemize}
\tightlist
\item
  \href{https://www.nytco.com/}{NYTCo}
\item
  \href{https://help.nytimes3xbfgragh.onion/hc/en-us/articles/115015385887-Contact-Us}{Contact
  Us}
\item
  \href{https://www.nytco.com/careers/}{Work with us}
\item
  \href{https://nytmediakit.com/}{Advertise}
\item
  \href{http://www.tbrandstudio.com/}{T Brand Studio}
\item
  \href{https://www.nytimes3xbfgragh.onion/privacy/cookie-policy\#how-do-i-manage-trackers}{Your
  Ad Choices}
\item
  \href{https://www.nytimes3xbfgragh.onion/privacy}{Privacy}
\item
  \href{https://help.nytimes3xbfgragh.onion/hc/en-us/articles/115014893428-Terms-of-service}{Terms
  of Service}
\item
  \href{https://help.nytimes3xbfgragh.onion/hc/en-us/articles/115014893968-Terms-of-sale}{Terms
  of Sale}
\item
  \href{https://spiderbites.nytimes3xbfgragh.onion}{Site Map}
\item
  \href{https://help.nytimes3xbfgragh.onion/hc/en-us}{Help}
\item
  \href{https://www.nytimes3xbfgragh.onion/subscription?campaignId=37WXW}{Subscriptions}
\end{itemize}
