Sections

SEARCH

\protect\hyperlink{site-content}{Skip to
content}\protect\hyperlink{site-index}{Skip to site index}

\href{https://www.nytimes3xbfgragh.onion/section/us}{U.S.}

\href{https://myaccount.nytimes3xbfgragh.onion/auth/login?response_type=cookie\&client_id=vi}{}

\href{https://www.nytimes3xbfgragh.onion/section/todayspaper}{Today's
Paper}

\href{/section/us}{U.S.}\textbar{}Irving Kanarek, Lawyer Who Defended
Charles Manson, Dies at 100

\url{https://nyti.ms/32Wmlox}

\begin{itemize}
\item
\item
\item
\item
\item
\end{itemize}

Advertisement

\protect\hyperlink{after-top}{Continue reading the main story}

Supported by

\protect\hyperlink{after-sponsor}{Continue reading the main story}

\hypertarget{irving-kanarek-lawyer-who-defended-charles-manson-dies-at-100}{%
\section{Irving Kanarek, Lawyer Who Defended Charles Manson, Dies at
100}\label{irving-kanarek-lawyer-who-defended-charles-manson-dies-at-100}}

The national spotlight that focused on Mr. Kanarek made his disruptive
circus of courtroom tactics almost as fascinating as his bizarre
clients.

\includegraphics{https://static01.graylady3jvrrxbe.onion/images/2020/09/04/obituaries/03Kanarek-sub/merlin_176535603_42d9b2db-c066-4425-a4c5-cfe8ae1789c0-articleLarge.jpg?quality=75\&auto=webp\&disable=upscale}

By \href{https://www.nytimes3xbfgragh.onion/by/robert-d-mcfadden}{Robert
D. McFadden}

\begin{itemize}
\item
  Sept. 3, 2020
\item
  \begin{itemize}
  \item
  \item
  \item
  \item
  \item
  \end{itemize}
\end{itemize}

Irving Kanarek, a Los Angeles lawyer who defended
\href{https://www.nytimes3xbfgragh.onion/2017/11/20/obituaries/charles-manson-dead.html}{Charles
Manson} in the cult killings of the actress Sharon Tate and six other
people, and
\href{https://www.nytimes3xbfgragh.onion/2007/04/10/obituaries/10smith.html}{Jimmy
Smith}, whose murder of a police officer was chillingly retold in Joseph
Wambaugh's 1973 best seller ``The Onion Field,'' died on Wednesday in
Garden Grove, Calif. He was 100.

His nephew Kany Levine confirmed the death.

Those killings were among the most notorious crimes of the 1960s, and
the national spotlight that focused on their trials made Mr. Kanarek's
disruptive circus of courtroom tactics almost as fascinating as his
bizarre clients --- Mr. Manson, the cult leader with a ``family'' of
young drifters, and Mr. Smith, a petty thief who did not quite know how
to operate the automatic pistol he carried.

For Mr. Kanarek, the trials were high points in a three-decade practice
given to a more routine caseload of personal injury and damage claims.
The law was not even his first calling. He had been an aerospace
engineer for North American Aviation, but had lost his Air Force
security clearance and his job after being falsely accused of Communist
associations in the 1950s. He cleared his name, but the experience had
soured him on science.

His first major case arose in Los Angeles on a March night in 1963 with
a routine traffic stop for a broken taillight on a car carrying Mr.
Smith and
\href{https://www.nytimes3xbfgragh.onion/2012/08/14/us/gregory-powell-the-onion-field-killer-dies-at-79.html}{Gregory
Powell}. As two officers,
\href{https://latimesblogs.latimes.com/lanow/2013/03/onion-field-police-officer-honored-with-hollywood-sign.html}{Ian
Campbell} and
\href{https://www.nytimes3xbfgragh.onion/1994/05/06/obituaries/karl-hettinger-onion-field-officer-59.html}{Karl
Hettinger}, approached, Mr. Smith and Mr. Powell drew guns, disarmed the
officers and drove them 90 miles north to a remote onion farm near
Bakersfield, Calif.

Mr. Wambaugh's novelistic treatment described Mr. Campbell's killing:

``Gregory Powell raised his arm and shot Ian in the mouth,'' he wrote.
``For a few white-hot seconds the three watched him being lifted up by
the blinding fireball and slammed down on his back, eyes open, watching
the stars. He probably never saw the shadow in the leather jacket
looming over him, and never really felt the four bullets flaming down
into his chest.''

Mr. Hettinger fled into the darkness and escaped. Mr. Powell and Mr.
Smith were caught, tried for murder, convicted and sentenced to death.

But the case became a seven-year marathon of appeals, mistrials,
reversals and reinstatements. Mr. Kanarek won Mr. Smith's first reversal
and defended him in other proceedings, but he was eventually fired by
Mr. Smith, who threw a chair at him.

Those death sentences were commuted to life in prison in 1972 by a
California Supreme Court ruling that temporarily invalided the state's
death penalty. Mr. Smith was paroled in 1982, but
\href{https://www.nytimes3xbfgragh.onion/1987/11/16/us/onion-field-killer-on-parole-is-arrested.html}{was
in and out of prison} for the rest of his life on parole violations. He
and Mr. Powell both died in prison in their late 70s.

\includegraphics{https://static01.graylady3jvrrxbe.onion/images/2020/09/04/obituaries/03Kanarek4/merlin_176533758_b2ab1bfd-ccc2-484d-b672-d349d9fe7a40-articleLarge.jpg?quality=75\&auto=webp\&disable=upscale}

Mr. Kanarek's next --- and last --- famous client was Mr. Manson. On
Aug. 9, 1969, a cleaning lady entering a Benedict Canyon home in North
Beverly Hills, Calif., found the mutilated bodies of Ms. Tate, 26, the
pregnant wife of the director Roman Polanski, as well as three friends
and a chance visitor. All had been stabbed and shot many times, and Ms.
Tate had been hung from a rafter.

A day later, the bodies of a grocery magnate, Leno LaBianca, and his
wife, Rosemary, were found in their Los Angeles home. They had been
killed in ferocious attacks that left little doubt they had been slain
by the same people who killed Ms. Tate and her companions.

Within months, Mr. Manson and four followers were arrested and
implicated by Linda Kasabian, an accomplice who admitted her role in the
crimes. Ms. Kasabian was granted immunity and became the state's star
witness in a trial that began in July 1970 and lasted six months.
(Charles Watson, a cult member who joined in the killings, was committed
to a mental institution and not tried with the others.)

Mr. Kanarek's courtroom tactics --- a Niagara of objections,
interruptions, shouting matches with the judge and witnesses, shoving
incidents with two prosecutors and a scuffle with his client, who
repeatedly tried to fire him --- made him an outcast in some legal
circles, but in others an exemplar of legal tenacity. He was jailed
twice for contempt of court and vilified by much of the press and
public.

The state called 84 witnesses and adduced that Mr. Manson, hoping to
trigger an apocalyptic race war in America, had planned and ordered the
killings, which were executed by his co-defendants, Susan Atkins, Leslie
Van Houten and Patricia Krenwinkel, and by Mr. Watson. The defense
rested without calling a single witness because, Mr. Kanarek said, the
three women wanted to confess on the stand to ``save'' Mr. Manson.

In 1971, all four defendants were convicted of murder and conspiracy and
sentenced to die in the gas chamber. Mr. Kanarek scoffed at the rulings
and the trial.

``It was entertainment for the public,'' he said.

A year later, when California's death penalty was temporarily
invalidated, the sentences were commuted to life in prison. Mr. Manson
was never released. He
\href{https://www.nytimes3xbfgragh.onion/2017/11/20/obituaries/charles-manson-dead.html}{died
in 2017} at 83.

Mr. Manson's crimes generated books, plays, television dramas,
documentaries and feature films --- most recently Quentin Tarantino's
Oscar-nominated ``Once Upon a Time \ldots{} in Hollywood.'' After the
trial, Mr. Kanarek prospered for a few years, but he never again made
national headlines.

In 1989, he was arrested on a charge of disorderly conduct and
hospitalized for a psychiatric evaluation. In 1990, he lost his law
license over unpaid debts. He later lived in motel rooms.

Irving Allen Kanarek was born in Seattle on May 12, 1920, to Meyer and
Beatrice (Prupis) Kanarek. His father was an insurance salesman.

Irving and his sister, Zillah, grew up in Seattle and attended Garfield
High School. Irving graduated from the University of Washington in 1941
with a chemistry degree.

In the 1940s and early '50s, he was an engineer for North American
Aviation, working on aerospace projects in California, and held a patent
for work on rocket fuels. After losing his security clearance and his
job, he won a suit for reinstatement and back pay.

But he had already decided on a new career. He earned a degree in 1956
at Loyola Law School and began his practice in 1957.

His marriage to Sally Nava ended in divorce. He is survived by their two
daughters, Irvina and Walesa Kanarek.

Long retired from law practice, Mr. Kanarek in recent years had resided
at an assisted-living facility in Garden Grove.

Advertisement

\protect\hyperlink{after-bottom}{Continue reading the main story}

\hypertarget{site-index}{%
\subsection{Site Index}\label{site-index}}

\hypertarget{site-information-navigation}{%
\subsection{Site Information
Navigation}\label{site-information-navigation}}

\begin{itemize}
\tightlist
\item
  \href{https://help.nytimes3xbfgragh.onion/hc/en-us/articles/115014792127-Copyright-notice}{©~2020~The
  New York Times Company}
\end{itemize}

\begin{itemize}
\tightlist
\item
  \href{https://www.nytco.com/}{NYTCo}
\item
  \href{https://help.nytimes3xbfgragh.onion/hc/en-us/articles/115015385887-Contact-Us}{Contact
  Us}
\item
  \href{https://www.nytco.com/careers/}{Work with us}
\item
  \href{https://nytmediakit.com/}{Advertise}
\item
  \href{http://www.tbrandstudio.com/}{T Brand Studio}
\item
  \href{https://www.nytimes3xbfgragh.onion/privacy/cookie-policy\#how-do-i-manage-trackers}{Your
  Ad Choices}
\item
  \href{https://www.nytimes3xbfgragh.onion/privacy}{Privacy}
\item
  \href{https://help.nytimes3xbfgragh.onion/hc/en-us/articles/115014893428-Terms-of-service}{Terms
  of Service}
\item
  \href{https://help.nytimes3xbfgragh.onion/hc/en-us/articles/115014893968-Terms-of-sale}{Terms
  of Sale}
\item
  \href{https://spiderbites.nytimes3xbfgragh.onion}{Site Map}
\item
  \href{https://help.nytimes3xbfgragh.onion/hc/en-us}{Help}
\item
  \href{https://www.nytimes3xbfgragh.onion/subscription?campaignId=37WXW}{Subscriptions}
\end{itemize}
