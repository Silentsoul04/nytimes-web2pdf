Sections

SEARCH

\protect\hyperlink{site-content}{Skip to
content}\protect\hyperlink{site-index}{Skip to site index}

\href{https://www.nytimes3xbfgragh.onion/section/books}{Books}

\href{https://myaccount.nytimes3xbfgragh.onion/auth/login?response_type=cookie\&client_id=vi}{}

\href{https://www.nytimes3xbfgragh.onion/section/todayspaper}{Today's
Paper}

\href{/section/books}{Books}\textbar{}David Graeber, Caustic Critic of
Inequality, Is Dead at 59

\url{https://nyti.ms/2Z7qGEw}

\begin{itemize}
\item
\item
\item
\item
\item
\end{itemize}

Advertisement

\protect\hyperlink{after-top}{Continue reading the main story}

Supported by

\protect\hyperlink{after-sponsor}{Continue reading the main story}

\hypertarget{david-graeber-caustic-critic-of-inequality-is-dead-at-59}{%
\section{David Graeber, Caustic Critic of Inequality, Is Dead at
59}\label{david-graeber-caustic-critic-of-inequality-is-dead-at-59}}

He wrote about crushing debt, pointless jobs and the negative effects of
globalization. And he played a leading role in the Occupy Wall Street
movement.

\includegraphics{https://static01.graylady3jvrrxbe.onion/images/2020/09/03/obituaries/03Graeber3/merlin_8834425_af4d29ed-115e-4a05-b9d3-f2de57812ea1-articleLarge.jpg?quality=75\&auto=webp\&disable=upscale}

\href{https://www.nytimes3xbfgragh.onion/by/sam-roberts}{\includegraphics{https://static01.graylady3jvrrxbe.onion/images/2018/02/20/multimedia/author-sam-roberts/author-sam-roberts-thumbLarge.jpg}}

By \href{https://www.nytimes3xbfgragh.onion/by/sam-roberts}{Sam Roberts}

\begin{itemize}
\item
  Sept. 4, 2020
\item
  \begin{itemize}
  \item
  \item
  \item
  \item
  \item
  \end{itemize}
\end{itemize}

David Graeber, the radical anthropologist, provocative critic of
economic and social inequality and self-proclaimed anarchist who was a
coiner of ``We Are the 99 Percent,'' the slogan of the Occupy Wall
Street movement, died on Wednesday at a hospital in Venice. He was 59.

His death was
\href{https://twitter.com/nikadubrovsky/status/1301504647769792512}{announced
on social media}by his wife, Nika Dubrovsky, an artist. She did not
specify the cause, but Dr. Graeber reported on YouTube last week that he
had been feeling ill.

A public intellectual, professor, political activist and author, Dr.
Graeber captivated a cult following that grew globally over the past
decade with each book he published.

In
\href{https://www.nytimes3xbfgragh.onion/2011/12/11/books/review/anarchist-anthropology.html}{``Debt:
The First 5000 Years''} (2011), he explored the changing definitions of
borrowing and who owed what to whom. He advocated a ``jubilee'' of loan
forgiveness. Writing in The New York Times Book Review, Thomas Meaney
called the book ``more than a screed'' and praised its ``brash, engaging
style.'' In ``The Utopia of Rules'' (2015), Dr. Graeber ridiculed the
bureaucracy that is typically associated with government, but that also
permeates the corporate world and everyday business transactions.

Image

In his book ``Debt: The First 5000 Years,'' Dr. Graeber explored the
changing definitions of borrowing and who owed what to whom.

In
\href{https://www.nytimes3xbfgragh.onion/2018/06/26/books/review/david-graeber-bullshit-jobs.html}{``Bullshit
Jobs: A Theory''} (2018), he wondered what happened to the 15-hour week
that the economist John Maynard Keynes, in 1930, had predicted would be
possible by the end of the 20th century. (``This book asks readers
whether there might be a better way to organize the world of work,''
Alana Semuels wrote in her Times review. ``That's a question worth
asking.'')

``In technological terms, we are quite capable of this,'' Mr. Graeber
wrote. ``And yet it didn't happen. Instead, technology has been
marshaled, if anything, to figure out ways to make us all work more.
Huge swaths of people, in Europe and North America in particular, spend
their entire working lives performing tasks they believe to be
unnecessary.

``The moral and spiritual damage that comes from this situation is
profound,'' he added. ``It is a scar across our collective soul.''

Dr. Graeber admitted that imposing an objective measure of social value
would be challenging, and that a world without, say, teachers, wouldn't
work.

``But it's not entirely clear how humanity would suffer,'' he was quoted
as saying in
\href{https://www.theguardian.com/books/2015/mar/21/books-interview-david-graeber-the-utopia-of-rules?paging=off}{The
Guardian} in 2015, ``were all private equity C.E.O.s, lobbyists, P.R.
researchers, actuaries, telemarketers, bailiffs or legal consultants to
similarly vanish.''

He was an associate professor of anthropology at Yale in 2005 when the
university informed him that his contract would not be renewed. He
attributed his termination to his unguarded derogation of capitalism,
and of both the political and academic establishments. Thousands of
supporters signed petitions urging Yale to reverse its decision, in
vain.

He received invitations to deliver prestigious lectures and was
recruited to teaching positions elsewhere. At the time of his death, he
was a professor at the London School of Economics.

\includegraphics{https://static01.graylady3jvrrxbe.onion/images/2020/09/05/obituaries/05graeber-obit4/merlin_49693290_6bbdcc2a-ff8c-45d3-8ab0-6f829a38340a-articleLarge.jpg?quality=75\&auto=webp\&disable=upscale}

Dr. Graeber was regarded as something of a leader --- or at least
someone others in the protest movements for
\href{https://www.nytimes3xbfgragh.onion/2019/05/01/opinion/extinction-rebellion-climate-change.html}{environmental},
social and economic justice and against the drawbacks of globalization
tended to follow.

He played a leading early role in the Occupy Wall Street demonstrations
in Lower Manhattan in 2011. But he insisted, despite repeated accounts
giving him sole credit, that the group's slogan was collaborative.

``No, I didn't personally come up with the slogan `We are the 99
percent,''' he said on his
\href{https://davidgraeber.industries/contact}{website}. ``I did first
suggest that we call ourselves the 99 percent. Then two Spanish
indignados and a Greek anarchist added the `we' and later a
food-not-bombs veteran put the `are' between them. And they say you
can't create something worthwhile by committee!''

As protests raged around the world in 2017 after President Donald J.
Trump's election, Dr. Graeber told
\href{https://www.nytimes3xbfgragh.onion/2017/02/02/us/anarchists-respond-to-trumps-inauguration-by-any-means-necessary.html}{The
New York Times}: ``We tried to warn you, with `Occupy.' We understood
that people were sick of the political system, which is fundamentally
corrupt. People want something radically different.''

David Rolfe Graeber was born on Feb. 12, 1961, in Manhattan to
self-taught leftist intellectuals. His father, Kenneth, who fought with
the Republicans in the Spanish Civil War, had a blue-collar job at an
offset printing plant. His Polish-born mother, Ruth (Rubinstein)
Graeber, was a garment worker who performed in her union's musical,
``Pins and Needles,'' which ran on Broadway in the late 1930s.

Raised in Penn South, a union-sponsored co-op apartment complex in the
Chelsea section of Manhattan, David translated Mayan hieroglyphics while
he was in junior high school and so impressed professional
archaeologists that he won a scholarship to Phillips Academy in Andover,
Mass.

He earned a bachelor's degree in anthropology in 1984 from the State
University of New York, Purchase and, while pursuing his doctorate at
the University of Chicago, won a Fulbright fellowship to conduct
ethnographic fieldwork in Madagascar.

He finished his thesis on magic, slavery and politics and received his
degree in 1998. Two years later, he was hired by Yale.

He was, he said, an anarchist in spirit at 16, but avoided involvement
in politics until 1999, when he participated in protests against the
World Trade Organization in Seattle. He was surprised at how fast and
far he could rise in a leaderless, anarchic movement.

Image

Mr. Graeber in 2012. He said he was an anarchist in spirit at 16, but
avoided involvement in politics until much later.Credit...Pier Marco
Tacca/Getty Images

``If you're really dedicated to this stuff, things can happen very
quickly,'' he told Businessweek in 2011. ``The first action you go to,
you're just a total outsider. You don't know what's going on. The second
one, you know everything. By the third, you're effectively part of the
leadership if you want to be. Anybody can be if you're willing to put in
the time and energy.''

Among his other books was ``The Democracy Project: A History, a Crisis,
a Movement,'' published in 2013. His ``The Dawn of Everything: A New
History of Humanity,'' written with David Wengrow, is scheduled to be
published next year by Farrar, Straus \& Giroux.

Dr. Graeber became involved in British politics last year, supporting
the Labour Party leader Jeremy Corbyn in the general election as ``a
beacon of hope in the struggle against emergent far-right nationalism,
xenophobia and racism in much of the democratic world.''

He remained surprisingly optimistic to the end, despite his sometimes
apocalyptic warnings and the disappointment he expressed at how
different the world he inherited as an adult was from the one he had
envisioned as a child.

``Speaking as someone who was 8 years old at the time of the Apollo moon
landing, I have clear memories of calculating that I would be 39 years
of age in the magic year 2000, and wondering what the world around me
would be like,'' he once said.

``Did I honestly expect I would be living in a world of such wonders? Of
course. Do I feel cheated now? Absolutely.''

Advertisement

\protect\hyperlink{after-bottom}{Continue reading the main story}

\hypertarget{site-index}{%
\subsection{Site Index}\label{site-index}}

\hypertarget{site-information-navigation}{%
\subsection{Site Information
Navigation}\label{site-information-navigation}}

\begin{itemize}
\tightlist
\item
  \href{https://help.nytimes3xbfgragh.onion/hc/en-us/articles/115014792127-Copyright-notice}{©~2020~The
  New York Times Company}
\end{itemize}

\begin{itemize}
\tightlist
\item
  \href{https://www.nytco.com/}{NYTCo}
\item
  \href{https://help.nytimes3xbfgragh.onion/hc/en-us/articles/115015385887-Contact-Us}{Contact
  Us}
\item
  \href{https://www.nytco.com/careers/}{Work with us}
\item
  \href{https://nytmediakit.com/}{Advertise}
\item
  \href{http://www.tbrandstudio.com/}{T Brand Studio}
\item
  \href{https://www.nytimes3xbfgragh.onion/privacy/cookie-policy\#how-do-i-manage-trackers}{Your
  Ad Choices}
\item
  \href{https://www.nytimes3xbfgragh.onion/privacy}{Privacy}
\item
  \href{https://help.nytimes3xbfgragh.onion/hc/en-us/articles/115014893428-Terms-of-service}{Terms
  of Service}
\item
  \href{https://help.nytimes3xbfgragh.onion/hc/en-us/articles/115014893968-Terms-of-sale}{Terms
  of Sale}
\item
  \href{https://spiderbites.nytimes3xbfgragh.onion}{Site Map}
\item
  \href{https://help.nytimes3xbfgragh.onion/hc/en-us}{Help}
\item
  \href{https://www.nytimes3xbfgragh.onion/subscription?campaignId=37WXW}{Subscriptions}
\end{itemize}
