Sections

SEARCH

\protect\hyperlink{site-content}{Skip to
content}\protect\hyperlink{site-index}{Skip to site index}

\href{https://www.nytimes3xbfgragh.onion/section/food}{Food}

\href{https://myaccount.nytimes3xbfgragh.onion/auth/login?response_type=cookie\&client_id=vi}{}

\href{https://www.nytimes3xbfgragh.onion/section/todayspaper}{Today's
Paper}

\href{/section/food}{Food}\textbar{}Julia Reed, Chronicler of Politics,
Food and the South, Dies at 59

\url{https://nyti.ms/31WBuHk}

\begin{itemize}
\item
\item
\item
\item
\item
\end{itemize}

Advertisement

\protect\hyperlink{after-top}{Continue reading the main story}

Supported by

\protect\hyperlink{after-sponsor}{Continue reading the main story}

\hypertarget{julia-reed-chronicler-of-politics-food-and-the-south-dies-at-59}{%
\section{Julia Reed, Chronicler of Politics, Food and the South, Dies at
59}\label{julia-reed-chronicler-of-politics-food-and-the-south-dies-at-59}}

In Ms. Reed's writing for Newsweek, Vogue and other publications, her
canvas included the follies of the powerful and the pleasures of
Southern food.

\includegraphics{https://static01.graylady3jvrrxbe.onion/images/2020/09/06/obituaries/06Reed-obit1/04Reed-articleLarge.jpg?quality=75\&auto=webp\&disable=upscale}

\href{https://www.nytimes3xbfgragh.onion/by/penelope-green}{\includegraphics{https://static01.graylady3jvrrxbe.onion/images/2018/07/18/multimedia/author-penelope-green/author-penelope-green-thumbLarge-v3.png}}

By \href{https://www.nytimes3xbfgragh.onion/by/penelope-green}{Penelope
Green}

\begin{itemize}
\item
  Sept. 4, 2020
\item
  \begin{itemize}
  \item
  \item
  \item
  \item
  \item
  \end{itemize}
\end{itemize}

Julia Reed, an irreverent, expansive chronicler of politics, food and
Southern life, died on Aug. 28 in Newport, R.I. She was 59.

The cause was cancer, said the historian Jon Meacham, a family friend.
She had been on vacation visiting friends.

In reporting for Newsweek, Vogue, The New York Times Magazine and other
publications, Ms. Reed covered presidents and their spouses, notably
both Bushes and the Clintons, along with powerful women, country music,
Southern rogues and Southern food. Her canvas was the foibles of power,
and even though (or perhaps because) she was the daughter of a Nixon-era
Republican grandee, she was cleareyed about the vices and virtues of
both parties.

In her profile of Laura Bush in the run-up to the 2000 election, Ms.
Reed wrote of the night early in her husband's political career when
Mrs. Bush told him a speech he had delivered wasn't very good. ``He
drove into the garage wall,'' Ms. Reed reported. ``They've both grown a
lot since then.''

In a statement to The New York Times, the Bushes wrote: ``Julia was a
longtime friend of ours. We loved to be with her because she was
charming, observant, funny and irreverent. We'll miss listening to her
stories and laughing with her.''

Al Gore, Mr. Bush's opponent in the 2000 election, also sent a
statement. ``She was an original,'' he said, ``one of the last who
combined, among other things that seem to have passed, a deep knowledge
and love of the self-conscious South and a command of the newsrooms in
which she worked.''

Deeply imprinted by the Mississippi Delta traditions she grew up with,
Ms. Reed was as well known for her entertaining as her journalism. In
\href{https://www.nytimes3xbfgragh.onion/2002/11/10/magazine/the-morning-after.html}{one
of her many food columns}for The New York Times Magazine, she described
a New Year's Eve party that had gone off the rails. There was a
fistfight, more than one bathroom dalliance, the unmasking of an arms
dealer, a fainting, a fire and more --- all of which she missed but
heard about secondhand by phone when she awoke with a hangover the next
day.

``I have always said that danger --- or at least the possibility of it
--- is a crucial element of any good party,'' she wrote. ``Parties
thrive on secrets that are made or told, alliances formed, dalliances
done, someone striking a match in someone else's inappropriate heart.''

\includegraphics{https://static01.graylady3jvrrxbe.onion/images/2020/09/02/obituaries/02Reed1/02Reed1-articleLarge.jpg?quality=75\&auto=webp\&disable=upscale}

Ms. Reed earned her first byline at 19, when she was a sophomore at
Georgetown University in Washington and a part-time library assistant
and phone answerer, as she put it, at Newsweek, a job she had held since
she was a student at Madeira, an all-girls boarding school in Virginia.

When the school's headmistress, Jean Harris, murdered her lover, Dr.
Herman Tarnower, the celebrity doctor and creator of the Scarsdale Diet,
Newsweek's Washington bureau chief sent Ms. Reed to get the Madeira
angle.
\href{https://gardenandgun.com/articles/the-high-the-low-good-country-bad-behavior/}{As
Ms. Reed wrote}, he woke her up with an order to head back to her old
school. When she wondered why, he barked, ``You idiot, your headmistress
just shot the diet doctor!''

Ms. Reed liked to say she was sorry the doctor had to give his life in
service to her career as a journalist.

Julia Evans Reed was born on Sept. 11, 1960, in Greenville, Miss. Her
mother, Judy (Brooks) Reed, came from a prominent Nashville family in
the Belle Meade neighborhood there. Her father, Clarke Reed, was a
businessman and Republican power broker who traveled frequently ---
``Saving the free world,''
\href{https://www.nytimes3xbfgragh.onion/2004/08/30/us/republicans-convention-new-york-apple-s-almanac-father-southern-strategy-76-here.html}{he
liked to tell his daughter}when she asked what he was up to --- and
entertained enthusiastically. Ms. Reed grew up cooking for William F.
Buckley Jr. and George and Barbara Bush, among others.

Greenville was a place, as she wrote in ``Ham Biscuits, Hostess Gowns,
and Other Southern Specialties: An Entertaining Life With Recipes,''
``where cooking was of paramount importance. We give food away as
presents and peace offerings, and sometimes because it is just so
incredibly good we have to share it. We tote it to people in times of
grief (when my grandparents were killed in a car wreck, the first thing
my mother told me to do as she ran out the door was to empty the
refrigerator); we use it to say bon voyage or welcome back.''

Image

Ms. Reed was the author of eight books, many of them collections of her
essays on food and the good life.Credit...Sonny Figoeroa/The New York
Times

Ms. Reed was the author of eight books, many of them collections of her
essays on food and the good life.

Mr. Meacham described her as a foreign correspondent in her own land,
``filing dispatches about the sacred and the profane.''

Colleagues at Vogue recalled her as both larger than life and free from
hubris --- a rare combination at 350 Madison Avenue, Condé Nast's old
headquarters, where egos roamed free.

``We were both children of the South, but from opposite ends of the
spectrum,'' said André Leon Talley, the longtime Vogue editor. ``She was
like a brassy marquise at Versailles, and at the same time a big hunky
dose of Babe Paley, Nancy Mitford, Rosalind Russell and Tallulah
Bankhead, with that cognac whiskey voice.''

In the wake of her death, many tried to describe her distinctive
baritone. ``She sounded like Barbara Stanwyck in `Meet John Doe,' if
Stanwyck was from the Mississippi Delta,'' Hilton Als of The New Yorker
wrote \href{https://www.instagram.com/p/CEez249B_9t/}{on Instagram}.

Mr. Talley also recounted the story of Ms. Reed's aborted marriage to a
charming Australian foreign correspondent. She canceled the wedding, a
full-on Southern affair with nearly 1,000 guests, but the couple went on
their honeymoon anyway --- it was paid for, after all --- ending up at
the Ritz in Paris, where they met Mr. Talley, and holding court in the
bar until the early hours of the morning, with characters as various as
Madonna's bodyguards, Kate Moss, Johnny Depp and Arlene Dahl.

``It was couture week, so everybody was there,'' Mr. Talley explained.

Writing about that night six years later
\href{https://archive.vogue.com/article/1996/6/1/dis-engaged}{in Vogue},
Ms. Reed called it ``one of the most memorable evenings of my life.''

Ms. Reed's marriage to John Pearce, a New Orleans lawyer
\href{https://www.elledecor.com/design-decorate/house-interiors/g254/greek-revival-interiors-julia-reed/?slide=1}{with
whom she renovated a house in the garden district} there, ended in
divorce in 2016. She is survived by her parents and a brother, Clarke
Reed Jr. Another brother, Reynolds Crews Reed, died in 2019.

Image

Ms. Reed in New Orleans in 2018.~``I have always said,'' she once wrote,
``that danger --- or at least the possibility of it --- is a crucial
element of any good party.''Credit...Paul Costello

Ms. Reed was a sought-after speaker, by all accounts a born entertainer
--- energetic, savvy and hilarious. In 2013, she took over the
\href{https://www.mainstreetgreenville.com/delta-hot-tamale-festival.html}{Delta
Hot Tamale Festival} in Greenville, turning a small local fair into
three-day national literary and culinary extravaganza to benefit her
hometown. The event concluded with a raucous barbecue on a sandbar.

Last year, Ms. Reed opened an online home goods store,
\href{https://www.reedsmythe.com}{Reed Smythe \& Company}, with her
friend Keith Smythe Meacham. The store sells pieces inspired by Ms.
Reed's own taste --- bronze drawer pulls shaped like the head of her
beagle, Henry; porcelain plates; engraved note cards --- many of which
are made by Southern artisans. Last year Mississippi's arts commission
named Ms. Reed a cultural ambassador to the state. A few months ago she
also opened a bookstore, Brown Water Books, in downtown Greenville.

Since 2011, Ms. Reed
\href{https://gardenandgun.com/feature/the-indelible-voice-of-julia-reed/}{had
been a columnist}at Garden \& Gun.
H\href{https://gardenandgun.com/articles/my-home-is-my-animal-kingdom/}{er
last piece}, posted last week, was about bedbugs and other vermin.

David DiBenedetto, the magazine's editor, explained what it was like
working with Ms. Reed. ``We had standard deadlines and Julia's
deadline,'' he wrote in an email. ``And her excuses for filing late were
often as entertaining as the actual columns or features themselves.''

On one particularly hairy occasion, he said, it was really down to the
wire. The editors were sending pages to the printer when Mr. DiBenedetto
received an email from Ms. Reed. It read: ``Just landed in Dallas this
second. Polished up at least 1,500 words before computer ran out of
juice. Gonna send that when I get to the hotel and will finish the rest
tonight after dinner with Laura Bush. She doesn't drink so it will
happen.''

``The piece was fabulous, as always,'' he said, adding, ``We were all
voyeurs, in a way, her readers and editors, to a life so much larger ---
and more fun --- than our own.''

Advertisement

\protect\hyperlink{after-bottom}{Continue reading the main story}

\hypertarget{site-index}{%
\subsection{Site Index}\label{site-index}}

\hypertarget{site-information-navigation}{%
\subsection{Site Information
Navigation}\label{site-information-navigation}}

\begin{itemize}
\tightlist
\item
  \href{https://help.nytimes3xbfgragh.onion/hc/en-us/articles/115014792127-Copyright-notice}{©~2020~The
  New York Times Company}
\end{itemize}

\begin{itemize}
\tightlist
\item
  \href{https://www.nytco.com/}{NYTCo}
\item
  \href{https://help.nytimes3xbfgragh.onion/hc/en-us/articles/115015385887-Contact-Us}{Contact
  Us}
\item
  \href{https://www.nytco.com/careers/}{Work with us}
\item
  \href{https://nytmediakit.com/}{Advertise}
\item
  \href{http://www.tbrandstudio.com/}{T Brand Studio}
\item
  \href{https://www.nytimes3xbfgragh.onion/privacy/cookie-policy\#how-do-i-manage-trackers}{Your
  Ad Choices}
\item
  \href{https://www.nytimes3xbfgragh.onion/privacy}{Privacy}
\item
  \href{https://help.nytimes3xbfgragh.onion/hc/en-us/articles/115014893428-Terms-of-service}{Terms
  of Service}
\item
  \href{https://help.nytimes3xbfgragh.onion/hc/en-us/articles/115014893968-Terms-of-sale}{Terms
  of Sale}
\item
  \href{https://spiderbites.nytimes3xbfgragh.onion}{Site Map}
\item
  \href{https://help.nytimes3xbfgragh.onion/hc/en-us}{Help}
\item
  \href{https://www.nytimes3xbfgragh.onion/subscription?campaignId=37WXW}{Subscriptions}
\end{itemize}
