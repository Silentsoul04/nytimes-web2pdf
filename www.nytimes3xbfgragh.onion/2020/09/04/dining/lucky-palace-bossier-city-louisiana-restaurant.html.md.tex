\href{/section/food}{Food}\textbar{}In Louisiana, Love for a Chinese
Restaurant and Its Magnetic Owner

\url{https://nyti.ms/3lNN1R8}

\begin{itemize}
\item
\item
\item
\item
\item
\item
\end{itemize}

\includegraphics{https://static01.graylady3jvrrxbe.onion/images/2020/09/09/dining/04lucky-palace8/04lucky-palace8-articleLarge.jpg?quality=75\&auto=webp\&disable=upscale}

Sections

\protect\hyperlink{site-content}{Skip to
content}\protect\hyperlink{site-index}{Skip to site index}

\hypertarget{in-louisiana-love-for-a-chinese-restaurant-and-its-magnetic-owner}{%
\section{In Louisiana, Love for a Chinese Restaurant and Its Magnetic
Owner}\label{in-louisiana-love-for-a-chinese-restaurant-and-its-magnetic-owner}}

For years, Lucky Palace has drawn fans for its intriguing wine list.
Now, they come to help their dear friend Kuan Lim in his time of need.

Kuan Lim started learning about wine in order to attract gamblers from
the casinos that surround Lucky Palace, in Bossier City, La. He became a
passionate expert, and is widely admired for his generosity and
taste.~Credit...Zerb Mellish for The New York Times

Supported by

\protect\hyperlink{after-sponsor}{Continue reading the main story}

By \href{https://www.nytimes3xbfgragh.onion/by/brett-anderson}{Brett
Anderson}

\begin{itemize}
\item
  Sept. 4, 2020
\item
  \begin{itemize}
  \item
  \item
  \item
  \item
  \item
  \item
  \end{itemize}
\end{itemize}

BOSSIER CITY, La. --- In late August, Kuan Lim rolled his wheelchair up
to a table set for 12 at \href{https://lucky-palace.com}{Lucky Palace},
the restaurant he operates inside a budget-priced motel in northwest
Louisiana. Mr. Lim is battling cancer --- he lost the bottom half of his
right leg to the disease --- and hadn't been to the restaurant since a
Chinese New Year celebration in January.

Lucky Palace is treasured in Bossier City and neighboring Shreveport for
its menu of traditional and modern Chinese dishes, and for an
adventurous
\href{https://lucky-palace.com/wine-list-by-the-bottle/}{wine list} that
has made it a
\href{https://gardenandgun.com/articles/lucky-palace-louisiana-gem/}{cult
favorite of oenophiles} from New Orleans to California to France.

The restaurant's fans are intensely loyal, particularly to Mr. Lim. The
network of people who have cared for him during his illness ---
providing food, companionship and rides to medical appointments --- is
made up entirely of Lucky Palace employees and friends he has made
through the restaurant.

Chris Jay, a \href{https://stuffedandbusted.com/}{food-and-wine writer}
in Shreveport and a Lucky Palace regular, said the pandemic has only
strengthened the connection locals feel to the restaurant and its owner,
whom intimates call simply ``Lim.''

``When I see a friend, they always say, `How's Lim?''' Mr. Jay said.

Mr. Lim, 55, was a warm and gregarious fixture at Lucky Palace from the
day it opened, in 1997, until his osteosarcoma was diagnosed in 2016.
This year, the impulse to show him physical affection has set up a
potentially dangerous dynamic: In March, he began a new round of
chemotherapy treatments, which compromise his immunity, just as the
coronavirus started spreading through Louisiana.

All of this is why Mr. Lim's emergence from isolation for a special,
socially distanced dinner in August, a day before Hurricane Laura made
landfall on Louisiana's Gulf Coast, was a closely guarded secret. Holly
Lim, Lucky Palace's manager, is not related to Mr. Lim, but she
considers herself his protector. She hung a dark sheet near the entrance
to keep his table, in the restaurant's bar, hidden from customers.

``If they see him, they'll want to kiss on him,'' said Ms. Lim, 56. ``I
don't want to hurt anybody's feelings, but I will.''

\includegraphics{https://static01.graylady3jvrrxbe.onion/images/2020/09/04/dining/04lucky-palace2/04lucky-palace2-articleLarge.jpg?quality=75\&auto=webp\&disable=upscale}

The friends invited to the dinner were all longtime regulars grateful
for the opportunity to ``share Lim'' and ``enjoy Lim.'' Those were
phrases bandied about as pre-dinner Champagne was poured from a magnum
of Ruinart Brut Rosé, a gift from one of the guests, Dr. Philip
Isherwood, a local physician.

``I love Champagne,'' Mr. Lim said before the meal, ``and also pinot
noir and Bordeaux and all of it.''

Mr. Lim remains engaged with his restaurant's wine service, from a
distance. Customers text him for wine-pairing advice ``all the time,''
he said. In January, he wrote a mission statement for his staff,
detailing the wines he wanted to emphasize with diners in 2020,
including wine from the Jura region of France, German riesling and cru
Beaujolais, all of which he believes offer good value and complement
Lucky Palace's food.

Joe Davis, the winemaker at
\href{http://www.princeofpinot.com/winery/300/}{Arcadian Winer}y in
Santa Barbara County, Calif., said Mr. Lim's unbridled enthusiasm is
unique. ``Lim is generous to a fault. He'll open up anything for you to
try,'' he said. ``He's going to convince you one way or another to love
wine as much as he does.''

Fine wine was not on Mr. Lim's mind when he and his wife, Evelyn, opened
Lucky Palace 23 years ago. They had met in the 1980s as students at
Southern Illinois University. He was born in Batu Pahat, a town in
Peninsular Malaysia; she is from Taiwan.

The couple were en route to San Antonio to look at a restaurant to buy
when they stopped in Shreveport. They decided instead to open their
Chinese-American restaurant in a Ramada Inn in Bossier City.

For the first few years, Mr. Lim woke at 4 a.m. to cook for the
breakfast buffet. He was also the delivery driver. ``I went to the gas
station and bought a map, and just started driving,'' he said. ``The
first delivery took me two hours, so I comped the meal.''

Lucky Palace's evolution --- it no longer serves breakfast or a buffet
--- began about a year after it opened, when an employee from one of the
nearby casinos advised Mr. Lim to carry more expensive wines to sell to
money-flush gamblers.

``I said, `What do you mean? I have a white zin and a few other
things,''' Mr. Lim recalled. His favorite wine at the time, he said, was
Blue Nun.

He embarked on a self-education in wine that coincided with a gradual
transformation of Lucky Palace's menu. The restaurant's original head
chef, James Cheng, was from Taipei, Taiwan, and he was assisted by Mr.
Lim's mother-in-law, Chun Ling Fei, and brother-in-law, Teck Ong.
Gerardo Orta Marcial, who is from San Luis Potosí, Mexico, has headed
the kitchen for the past 15 years.

Image

Lucky Palace's kitchen staff: from left, Herminio Bravo; Gerardo Orta
Marcial, the restaurant's longtime head chef; Orlando Parras; and
Alberto Orta Marcial, the sous-chef (and Gerardo's
brother).Credit...Zerb Mellish for The New York Times

Image

Duck on scallion pancake is Lucky Palace's answer to Peking
duck.Credit...Zerb Mellish for The New York Times

Image

Holly Lim is Lucky Palace's general manager. She is not related to Mr.
Lim, but they regard each other as family.Credit...Zerb Mellish for The
New York Times

Image

McKinley LaVonne Pippenger, Ms. Lim's granddaughter, has been ``raised
in the restaurant,'' Ms. Lim said. She calls Mr. Lim
``uncle.''Credit...Zerb Mellish for The New York Times

David Bridges, a Shreveport chef, remembers being attracted to the
restaurant in the early 2000s by word on the street. ``I heard he had
some dishes you didn't see around here, like shark fin, which wasn't so
frowned upon back then, and jellyfish salad,'' he said.

Mr. Bridges and Mr. Lim became close friends, and developed a routine:
They would spend the week reading about new wines, and then pick bottles
to taste during Sunday-night dinners at Lucky Palace. The dinners grew
larger, drawing in local wine enthusiasts and creating others, as Mr.
Lim fine-tuned his palate.

Mr. Bridges, 48, recalls a blind tasting of chardonnay with Mr. Lim.
``Right off the bat he says, `I think this is from volcanic stone,'''
Mr. Bridges said. ``I do some research and realize he's exactly right.
At that point I was like, `Lim, you've surpassed me.'''

As Lucky Palace went more upscale, the motel that houses it went in the
other direction. Rooms at what is now called the Bossier Inn \& Suites
go for \$195 a week. The motel's disheveled appearance --- coupled with
its location, hours away from the South's established culinary capitals
--- is so discordant with Lucky Palace's ambitions that the disconnect
is part of the restaurant's legend.

Brent Sloan, the owner and vintner of
\href{https://rapportwines.com/}{Rapport Wines}, in Napa Valley, visited
Lucky Palace for the first time last year. ``My Uber driver was like,
`I'm going to warn you, you're going to think I'm taking you to a
sketchy place,''' Mr. Sloan said.

He left as impressed with Mr. Lim as he was with Lucky Palace's
idiosyncratic wine list. It includes about 300 selections from a
collection of roughly 1,200 bottles stored in racks that line the dining
rooms, as well as in a closet near the cash register. Mr. Sloan pointed
to bottles of Chateau Musar, from Lebanon, and some aged chenin blancs
from the Loire Valley, as representative of Mr. Lim's taste.

``You can see his personality in the wines he serves,'' said Mr. Sloan,
48. ``This guy isn't buying off some guide.''

Chris Hunter, 61, a longtime wine wholesaler in New Orleans, said: ``Lim
is happy to turn people on to wine that they can afford, and that isn't
that highly rated. He's an old-style host and restaurateur who has an
ability to treat everybody like they're the only people he's talking
to.''

Lucky Palace is not the secret it used to be. The British wine writer
\href{http://www.clive-coates.com/}{Clive Coates} has been co-host of
several wine dinners at Lucky Palace. Ms. Lim said she has seen an
increase in tourist traffic since the restaurant was featured in an
\href{https://www.shreveporttimes.com/story/money/business/2018/09/14/sec-network-feature-shreveport-bossier-restaurants/37807111/}{episode}
of the TV series ``True South'' in 2018. For the last three years, it
has been a
\href{https://www.shreveporttimes.com/story/entertainment/dining/2020/02/27/lucky-palace-bossier-city-makes-james-beard-awards-semifinals/4883334002/}{semifinalist}
for the James Beard Foundation's Outstanding Wine Program award.

Image

Lucky Palace's 1,200-bottle collection is spread out on racks that line
the dining rooms, and in this wine closet.Credit...Zerb Mellish for The
New York Times

Kevin Hill, a geophysicist in the oil business and a Lucky Palace
regular, is among several friends who have urged Mr. Lim to take better
advantage of his renown by moving his restaurant to a more attractive
location. He even found a new space, and offered to buy it.

``We were going to give it to him rent-free,'' Mr. Hill said. ``He said:
`It's Lucky Palace. I've been lucky here. I want to stay.'''

The dinner honoring Mr. Lim last month began with salt-and-pepper
cuttlefish, paired with Le Mesnil Blanc de Blancs, a moderately priced
nonvintage Champagne. ``I always like to pair this with Champagne,''
said Mr. Lim, who sat at the head of the table.

The restaurant now seats only 60 diners, half of its normal capacity, in
keeping with state rules in the pandemic, and the bar is usually closed.

``I miss this,'' said Lane Pittard, a local district judge, as he
settled in for dinner. ``We miss seeing Lim here.''

Mr. Pittard and his wife, Adelise, have continued to eat at Lucky Palace
once a week during the pandemic, as they have for 20 years, even if only
for takeout. They regularly bring meals to Mr. Lim at his house.

``Lim loves soul food --- black-eyed peas, cornbread, stuff like that,''
Mr. Pittard said.

Mr. Lim has lived alone since he and Evelyn divorced 10 years ago. She
and their son, Joshua, live in New York City. Ms. Lim, who had started
working at the restaurant in the mid-2000s, stepped in to take over
Evelyn's management duties.

``Lim's ex-wife is my best friend,'' she said. ``Kind of weird, I
know.''

The divorce was a difficult transition for both Mr. Lim and Lucky
Palace. He had to become an American citizen to keep the restaurant's
liquor license, which was in Evelyn's name.

Mr. Pittard helped cut through legal red tape. Karen Vanderkuy, a local
chef and restaurateur, drove Mr. Lim to New Orleans to complete the
paperwork. (During the pandemic, she brings him duck-truffle pâté, his
favorite dish from her restaurant,
\href{https://www.themarketshreveport.com/}{the Market}.)

Citizenship saved Lucky Palace, but didn't change Mr. Lim, according to
the chef Mr. Bridges. ``I told Lim, `There are two things you need to
learn how to do, now that you're an American citizen: You need to learn
how to lie, and you need to put yourself first,''' he said. ``As an
American, he's a complete failure.''

Today, Ms. Lim anchors a staff of 14 --- down just one from pre-Covid
levels --- that is unusually close. Dexter Huewitt, 45, worked with Ms.
Lim at another restaurant before joining Lucky Palace's staff 10 years
ago. He is also a minister.

``Dexter preached my father's funeral and my brother's funeral,'' said
Ms. Lim, ``and he assisted with my daughter's funeral.''

The arrival of a signature dish, crisp-skinned duck breast served on a
scallion pancake, inspired a brief round of applause. It is Lucky
Palace's answer to Peking duck. To go with it, Mr. Lim chose an Arcadian
Sleepy Hollow Vineyard pinot noir from 2001, a revered vintage in
California, after consulting with Mr. Davis, whom he called during the
meal to thank.

Image

Mr. Lim in the parking lot outside Lucky Palace, which is located inside
the Bossier Inn \& Suites, a budget-priced motel. He lost his hair to
cancer treatments.Credit...Zerb Mellish for The New York Times

Mr. Lim was not feeling his best. He had just regained his appetite the
day before, following his latest round of cancer treatments. It was only
the second time he had drunk wine since March. ``I don't taste as well
as I used to,'' he said.

He still appeared to enjoy sharing some of his favorite bottles. He
opened a 1994 López de Heredia Viña Tondonia Gran Reserva to drink with
a flat-noodle osso buco, an off-the-menu special usually served in the
colder months. After dinner, he poured a bottle of
\href{https://www.nytimes3xbfgragh.onion/2008/11/05/dining/05pour.html}{Jacques
Selosse} ``Initial'' Blanc de Blancs, ``one of maybe nine bottles in the
state of Louisiana,'' he said. Lucky Palace sells it for \$246, a shade
less than the retail price.

``Lim charges way too little for wine,'' said Barry Regula, 60, the
general manager of two local casinos, Margaritaville and Boomtown. The
annual benefit dinner that Mr. Regula co-founded three years ago to help
pay Mr. Lim's medical bills is on hold this year because of the
coronavirus.

Uncertainty over when friends will be able to dine again with Mr. Lim at
Lucky Palace hung in the air. Mr. Lim said nobody in his family is
interested in taking over the business. But Ms. Lim has vowed to keep it
going should anything happen to its owner. ``Lim is Lim, and I will make
sure that his memory is carried on,'' she said.

His legacy was secured long ago. Mr. Jay, the writer, recalled saving up
his money as a younger man to educate himself on food and wine, inspired
by
\href{https://www.theringer.com/pop-culture/2020/7/14/21323748/anthony-bourdain-kitchen-confidential-book-20-year-anniversary}{Anthony
Bourdain}.

``I grew up in a single-wide trailer in Sarepta, La.,'' he said, choking
back tears. ``I never felt that any of the food-and-drink people wanted
me around. That was the vibe everywhere, except for at Lucky Palace.''

Mr. Jay said he has already imagined its passing.

``I feel like it's Shangri-La,'' he said. ``We're all going to drive up
one day, and it will be gone. The hotel clerk will say, `There was never
a Chinese restaurant here.'''

\emph{Follow} \href{https://twitter.com/nytfood}{\emph{NYT Food on
Twitter}} \emph{and}
\href{https://www.instagram.com/nytcooking/}{\emph{NYT Cooking on
Instagram}}\emph{,}
\href{https://www.facebookcorewwwi.onion/nytcooking/}{\emph{Facebook}}\emph{,}
\href{https://www.youtube.com/nytcooking}{\emph{YouTube}} \emph{and}
\href{https://www.pinterest.com/nytcooking/}{\emph{Pinterest}}\emph{.}
\href{https://www.nytimes3xbfgragh.onion/newsletters/cooking}{\emph{Get
regular updates from NYT Cooking, with recipe suggestions, cooking tips
and shopping advice}}\emph{.}

Advertisement

\protect\hyperlink{after-bottom}{Continue reading the main story}

\hypertarget{site-index}{%
\subsection{Site Index}\label{site-index}}

\hypertarget{site-information-navigation}{%
\subsection{Site Information
Navigation}\label{site-information-navigation}}

\begin{itemize}
\tightlist
\item
  \href{https://help.nytimes3xbfgragh.onion/hc/en-us/articles/115014792127-Copyright-notice}{©~2020~The
  New York Times Company}
\end{itemize}

\begin{itemize}
\tightlist
\item
  \href{https://www.nytco.com/}{NYTCo}
\item
  \href{https://help.nytimes3xbfgragh.onion/hc/en-us/articles/115015385887-Contact-Us}{Contact
  Us}
\item
  \href{https://www.nytco.com/careers/}{Work with us}
\item
  \href{https://nytmediakit.com/}{Advertise}
\item
  \href{http://www.tbrandstudio.com/}{T Brand Studio}
\item
  \href{https://www.nytimes3xbfgragh.onion/privacy/cookie-policy\#how-do-i-manage-trackers}{Your
  Ad Choices}
\item
  \href{https://www.nytimes3xbfgragh.onion/privacy}{Privacy}
\item
  \href{https://help.nytimes3xbfgragh.onion/hc/en-us/articles/115014893428-Terms-of-service}{Terms
  of Service}
\item
  \href{https://help.nytimes3xbfgragh.onion/hc/en-us/articles/115014893968-Terms-of-sale}{Terms
  of Sale}
\item
  \href{https://spiderbites.nytimes3xbfgragh.onion}{Site Map}
\item
  \href{https://help.nytimes3xbfgragh.onion/hc/en-us}{Help}
\item
  \href{https://www.nytimes3xbfgragh.onion/subscription?campaignId=37WXW}{Subscriptions}
\end{itemize}
