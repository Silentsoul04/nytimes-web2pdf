Sections

SEARCH

\protect\hyperlink{site-content}{Skip to
content}\protect\hyperlink{site-index}{Skip to site index}

\href{https://www.nytimes3xbfgragh.onion/section/health}{Health}

\href{https://myaccount.nytimes3xbfgragh.onion/auth/login?response_type=cookie\&client_id=vi}{}

\href{https://www.nytimes3xbfgragh.onion/section/todayspaper}{Today's
Paper}

\href{/section/health}{Health}\textbar{}Vaping Links to Covid Risk Are
Becoming Clear

\url{https://nyti.ms/2GqMTqp}

\begin{itemize}
\item
\item
\item
\item
\item
\end{itemize}

\hypertarget{the-coronavirus-outbreak}{%
\subsubsection{\texorpdfstring{\href{https://www.nytimes3xbfgragh.onion/news-event/coronavirus?name=styln-coronavirus-national\&region=TOP_BANNER\&block=storyline_menu_recirc\&action=click\&pgtype=Article\&impression_id=197cf000-f1be-11ea-82b8-c746ff1e5987\&variant=undefined}{The
Coronavirus
Outbreak}}{The Coronavirus Outbreak}}\label{the-coronavirus-outbreak}}

\begin{itemize}
\tightlist
\item
  live\href{https://www.nytimes3xbfgragh.onion/2020/09/08/world/covid-19-coronavirus.html?name=styln-coronavirus-national\&region=TOP_BANNER\&block=storyline_menu_recirc\&action=click\&pgtype=Article\&impression_id=197cf001-f1be-11ea-82b8-c746ff1e5987\&variant=undefined}{Latest
  Updates}
\item
  \href{https://www.nytimes3xbfgragh.onion/interactive/2020/us/coronavirus-us-cases.html?name=styln-coronavirus-national\&region=TOP_BANNER\&block=storyline_menu_recirc\&action=click\&pgtype=Article\&impression_id=197cf002-f1be-11ea-82b8-c746ff1e5987\&variant=undefined}{Maps
  and Cases}
\item
  \href{https://www.nytimes3xbfgragh.onion/interactive/2020/science/coronavirus-vaccine-tracker.html?name=styln-coronavirus-national\&region=TOP_BANNER\&block=storyline_menu_recirc\&action=click\&pgtype=Article\&impression_id=197cf003-f1be-11ea-82b8-c746ff1e5987\&variant=undefined}{Vaccine
  Tracker}
\item
  \href{https://www.nytimes3xbfgragh.onion/2020/09/02/your-money/eviction-moratorium-covid.html?name=styln-coronavirus-national\&region=TOP_BANNER\&block=storyline_menu_recirc\&action=click\&pgtype=Article\&impression_id=197cf004-f1be-11ea-82b8-c746ff1e5987\&variant=undefined}{Eviction
  Moratorium}
\item
  \href{https://www.nytimes3xbfgragh.onion/interactive/2020/09/02/magazine/food-insecurity-hunger-us.html?name=styln-coronavirus-national\&region=TOP_BANNER\&block=storyline_menu_recirc\&action=click\&pgtype=Article\&impression_id=197cf005-f1be-11ea-82b8-c746ff1e5987\&variant=undefined}{American
  Hunger}
\end{itemize}

Advertisement

\protect\hyperlink{after-top}{Continue reading the main story}

Supported by

\protect\hyperlink{after-sponsor}{Continue reading the main story}

\hypertarget{vaping-links-to-covid-risk-are-becoming-clear}{%
\section{Vaping Links to Covid Risk Are Becoming
Clear}\label{vaping-links-to-covid-risk-are-becoming-clear}}

Researchers are starting to home in on the ways in which the use of
e-cigarettes raises the chances of catching the virus, and suffering its
worst effects.

\includegraphics{https://static01.graylady3jvrrxbe.onion/images/2020/09/08/science/04VIRUS-VAPING1/merlin_176554212_6ac19ffb-37e1-4564-b0b6-adcf8e002d56-articleLarge.jpg?quality=75\&auto=webp\&disable=upscale}

\href{https://www.nytimes3xbfgragh.onion/by/katherine-j--wu}{\includegraphics{https://static01.graylady3jvrrxbe.onion/images/2020/08/11/reader-center/author-katherine-j-wu/author-katherine-j-wu-thumbLarge.png}}

By
\href{https://www.nytimes3xbfgragh.onion/by/katherine-j--wu}{Katherine
J. Wu}

\begin{itemize}
\item
  Sept. 4, 2020
\item
  \begin{itemize}
  \item
  \item
  \item
  \item
  \item
  \end{itemize}
\end{itemize}

Twenty-year-old Janan Moein vaped his first pen a year ago. By late
fall, he was blowing through several THC-laced cartridges a week ---
more, he said, than most people can handle.

Then in early December, he found himself in the emergency room of Sharp
Grossmont Hospital in San Diego with a collapsed lung and a diagnosis of
\href{https://www.nytimes3xbfgragh.onion/2019/09/07/health/vaping-lung-illness.html}{vaping-related
lung illness}. His hospital stay plunged him into a medically induced
coma, forced him onto a breathing machine and stripped nearly 50 pounds
off his 6-foot-1-inch frame in just two weeks.

At one point, Mr. Moein said, his doctors gave him a 5 percent chance of
survival. He resolved that the wax pen he had vaped before his
hospitalization would be his last.

When he contracted a mild case of Covid-19 during a family barbecue
three months ago, he knew he had quit not a moment too soon. ``If I had
caught Covid-19 within the week before I got really ill, I probably
would have died,'' he said.

Since the start of the pandemic, experts have
\href{https://www.who.int/news-room/commentaries/detail/smoking-and-covid-19}{warned}
that the coronavirus --- a respiratory pathogen ---
\href{https://www.nytimes3xbfgragh.onion/2020/04/09/health/coronavirus-smoking-vaping-risks.html}{most
likely capitalizes on the scarred lungs of smokers and vapers}. Doctors
and researchers are now starting to pinpoint the ways in which smoking
and vaping seem to enhance the virus's ability to spread from person to
person, infiltrate the lungs and spark some of Covid-19's worst
symptoms.

``I have no doubt in saying that smoking and vaping could put people at
increased risk of poor outcomes from Covid-19,'' said Dr. Stephanie
Lovinsky-Desir, a pediatric pulmonologist at Columbia University. ``It
is quite clear that smoking and vaping are bad for the lungs, and the
predominant symptoms of Covid are respiratory. Those two things are
going to be bad in combination.''

Last year's vaping crisis, during which thousands of people like Mr.
Moein were sickened and hospitalized with severe lung and respiratory
illnesses, underscored the hazards of many e-cigarette and vaping
products,
\href{https://www.nytimes3xbfgragh.onion/2019/11/08/health/vaping-illness-cdc.html}{especially
illicitly sold marijuana-based vapes}.

\includegraphics{https://static01.graylady3jvrrxbe.onion/images/2020/09/08/science/04VIRUS-VAPING2/04VIRUS-VAPING2-articleLarge.jpg?quality=75\&auto=webp\&disable=upscale}

But while several studies have found that
\href{https://www.nejm.org/doi/full/10.1056/NEJMoa2002032}{smoking} can
\href{https://academic.oup.com/ntr/article/22/9/1653/5835834}{more than
double a person's risk} of
\href{https://www.jahonline.org/article/S1054-139X(20)30338-4/fulltext}{severe
Covid-19 symptoms}, the data on the relationship between vaping and
Covid-19 are only beginning to emerge. A team of researchers recently
reported that young adults who vape are
\href{https://www.sciencedirect.com/science/article/pii/S1054139X20303992}{five
times more likely} to receive a coronavirus diagnosis.

Much of what underlies the relationship between smoking, vaping and the
coronavirus remains unclear. Doctors aren't sure why vaping makes some
people seriously sick, but seems to spare others. And Mr. Moein's
unexpectedly mild encounter with the coronavirus remains mysterious as
well.

These and other lingering questions have made the risks of smoking and
vaping during the pandemic tough to communicate.

\hypertarget{latest-updates-the-coronavirus-outbreak}{%
\section{\texorpdfstring{\href{https://www.nytimes3xbfgragh.onion/2020/09/08/world/covid-19-coronavirus.html?action=click\&pgtype=Article\&state=default\&region=MAIN_CONTENT_1\&context=storylines_live_updates}{Latest
Updates: The Coronavirus
Outbreak}}{Latest Updates: The Coronavirus Outbreak}}\label{latest-updates-the-coronavirus-outbreak}}

Updated 2020-09-08T10:06:50.372Z

\begin{itemize}
\tightlist
\item
  \href{https://www.nytimes3xbfgragh.onion/2020/09/08/world/covid-19-coronavirus.html?action=click\&pgtype=Article\&state=default\&region=MAIN_CONTENT_1\&context=storylines_live_updates\#link-4a77847f}{As
  senators return to Washington, an impasse over a virus relief package
  looms.}
\item
  \href{https://www.nytimes3xbfgragh.onion/2020/09/08/world/covid-19-coronavirus.html?action=click\&pgtype=Article\&state=default\&region=MAIN_CONTENT_1\&context=storylines_live_updates\#link-1c973131}{`The
  lockdown killed my father': Farmer suicides add to India's virus
  misery.}
\item
  \href{https://www.nytimes3xbfgragh.onion/2020/09/08/world/covid-19-coronavirus.html?action=click\&pgtype=Article\&state=default\&region=MAIN_CONTENT_1\&context=storylines_live_updates\#link-adc17f7}{China's
  leader declares success in suppressing the country's outbreak.}
\end{itemize}

\href{https://www.nytimes3xbfgragh.onion/2020/09/08/world/covid-19-coronavirus.html?action=click\&pgtype=Article\&state=default\&region=MAIN_CONTENT_1\&context=storylines_live_updates}{See
more updates}

More live coverage:

James Ippolito, a 26-year-old Army veteran who lives in Hingham, Mass.,
has been hooked on vaping nicotine for about six years. ``I vape every
day, all day long,'' Mr. Ippolito said.

The looming threat of the virus doesn't intimidate him. ``I hate to say
it, but if I got the virus, I would still be vaping --- I wouldn't even
think it was related,'' he said.

Such stubbornness troubles experts, who pointed out that Covid is hardly
the first disease to hit smokers and vapers harder.

``Lungs aren't designed to regularly breathe in smoke and vape,'' said
Dr. Drew Harris, a pulmonologist at UVA Health in Virginia. These
products, he added, ``do just about everything bad you can think of.''

About
\href{https://www.cdc.gov/mmwr/volumes/68/ss/ss6812a1.htm?s_cid=ss6812a1_w}{34
million adults smoke cigarettes} in the United States, many of them from
communities of color and low socioeconomic status --- groups already
known to be more vulnerable to the virus. And more than 5 million middle
and high school students recently
\href{https://www.cdc.gov/mmwr/volumes/68/ss/ss6812a1.htm?s_cid=ss6812a1_w}{reported
using vapes}.

The active contents of cigarettes and vapes vary immensely, ranging from
nicotine to THC, the high-inducing ingredient in marijuana. But many
experts are more concerned about the other ingredients that tend to
accompany them: additives like heavy metals and
\href{https://www.nytimes3xbfgragh.onion/2019/11/08/health/vaping-illness-cdc.html}{vitamin
E acetate}, which bathe the lung in toxins and ultrafine particles that
can poison or pulverize delicate tissues.

Decades of research have unmasked smoking's ability to
\href{https://www.cancer.org/cancer/cancer-causes/tobacco-and-cancer/health-risks-of-smoking-tobacco.html}{put
the immune system on the fritz}. The punch of harmful chemicals packed
into each puff is thought to
\href{https://www.ncbi.nlm.nih.gov/pmc/articles/PMC5352117/}{discombobulate
the system of checks and balances} needed to direct disease-fighting
cells and molecules toward harmful invaders like germs, while waylaying
any misguided attacks on healthy tissues.

A body hamstrung by a smoking habit can
\href{https://pubmed.ncbi.nlm.nih.gov/30789425/}{struggle to rouse a
sufficient defense against viruses} --- but has little trouble turning
its arsenal of weapons inward. Eventually, deteriorating lungs can
become chronically inflamed and awash with mucus, narrowing the airways
and stymieing the flow of oxygen into the blood. Certain patients may
end up with lungs pockmarked by scar tissue, further impeding the
movement of air.

Dr. Lovinsky-Desir describes the internal architecture of these tissues
as bunches of gas-filled grapes, enmeshed in a network of blood vessels.
``Chronic smoking destroys those grapes,'' she said. ``They become saggy
and floppy.''

Smoke can also compromise little hairlike structures known as cilia that
boot toxins and microbes out of the airways, making it easier for
pathogens to set up shop in the lungs.

Image

Dr. Stephanie Lovinsky-Desir, a pediatric pulmonologist at Columbia
University Medical Center, likens the internal structure of lung tissue
to gas-filled grapes. ``Chronic smoking destroys those grapes,'' she
said.Credit...Laylah Amatullah Barrayn for The New York Times

Should a virus then enter the mix, Dr. Lovinsky-Desir said, ``it will
cause more destruction,'' clogging the already damaged grapes with a
glut of cellular debris. Years of data have borne out these
relationships. Smokers who catch the flu, for instance, are
\href{https://pubmed.ncbi.nlm.nih.gov/30789425/}{more likely than
nonsmokers to wind up in the hospital}.

Less is known about vaping, a relative newcomer. But
\href{https://www.bmj.com/content/366/bmj.l5275}{similar trends} have
\href{https://www.jci.org/articles/view/128531}{been noted} for
e-cigarettes and vape pens. Several studies have shown that
\href{https://pubmed.ncbi.nlm.nih.gov/25651083/}{vaping makes mice} more
vulnerable to \href{https://pubmed.ncbi.nlm.nih.gov/26804311/}{bacteria}
and viruses, and sends surges of inflammation throughout the
body,\href{https://pubmed.ncbi.nlm.nih.gov/29384700/}{beyond the
boundaries of the lungs}.

Mr. Moein was one of thousands who last year fell prey to a disease
called e-cigarette or vaping-associated lung injury, or Evali. Many
Evali patients had vaped products containing a sticky substance called
vitamin E acetate, which has been found in the branded Dr. Zodiak
cartridges Mr. Moein preferred.

Mr. Moein still recalls his hospital stay in vivid detail.

``My lips were blue,'' he said. ``They had to tape my eyes shut. I was
hallucinating the entire time that the nurses were trying to kill me,
that the walls were made of human skin. It was a really bad situation.''

Nearly a year later, Mr. Moein, a towering athlete who played
competitive sports in high school, said he was now once again ``very
healthy.''

But Dr. Laura Crotty Alexander, a pulmonologist and vaping expert at
University of California San Diego and one of Mr. Moein's doctors, said
experts were still teasing apart the potential long-term effects of
vaping,~even brushes briefer than his.

``Just because he feels 100 percent recovered doesn't mean his lung
function returned to 100 percent,'' she said.

\href{https://www.nytimes3xbfgragh.onion/news-event/coronavirus?action=click\&pgtype=Article\&state=default\&region=MAIN_CONTENT_3\&context=storylines_faq}{}

\hypertarget{the-coronavirus-outbreak-}{%
\subsubsection{The Coronavirus Outbreak
›}\label{the-coronavirus-outbreak-}}

\hypertarget{frequently-asked-questions}{%
\paragraph{Frequently Asked
Questions}\label{frequently-asked-questions}}

Updated September 4, 2020

\begin{itemize}
\item ~
  \hypertarget{what-are-the-symptoms-of-coronavirus}{%
  \paragraph{What are the symptoms of
  coronavirus?}\label{what-are-the-symptoms-of-coronavirus}}

  \begin{itemize}
  \tightlist
  \item
    In the beginning, the coronavirus
    \href{https://www.nytimes3xbfgragh.onion/article/coronavirus-facts-history.html?action=click\&pgtype=Article\&state=default\&region=MAIN_CONTENT_3\&context=storylines_faq\#link-6817bab5}{seemed
    like it was primarily a respiratory illness}~--- many patients had
    fever and chills, were weak and tired, and coughed a lot, though
    some people don't show many symptoms at all. Those who seemed
    sickest had pneumonia or acute respiratory distress syndrome and
    received supplemental oxygen. By now, doctors have identified many
    more symptoms and syndromes. In April,
    \href{https://www.nytimes3xbfgragh.onion/2020/04/27/health/coronavirus-symptoms-cdc.html?action=click\&pgtype=Article\&state=default\&region=MAIN_CONTENT_3\&context=storylines_faq}{the
    C.D.C. added to the list of early signs}~sore throat, fever, chills
    and muscle aches. Gastrointestinal upset, such as diarrhea and
    nausea, has also been observed. Another telltale sign of infection
    may be a sudden, profound diminution of one's
    \href{https://www.nytimes3xbfgragh.onion/2020/03/22/health/coronavirus-symptoms-smell-taste.html?action=click\&pgtype=Article\&state=default\&region=MAIN_CONTENT_3\&context=storylines_faq}{sense
    of smell and taste.}~Teenagers and young adults in some cases have
    developed painful red and purple lesions on their fingers and toes
    --- nicknamed ``Covid toe'' --- but few other serious symptoms.
  \end{itemize}
\item ~
  \hypertarget{why-is-it-safer-to-spend-time-together-outside}{%
  \paragraph{Why is it safer to spend time together
  outside?}\label{why-is-it-safer-to-spend-time-together-outside}}

  \begin{itemize}
  \tightlist
  \item
    \href{https://www.nytimes3xbfgragh.onion/2020/05/15/us/coronavirus-what-to-do-outside.html?action=click\&pgtype=Article\&state=default\&region=MAIN_CONTENT_3\&context=storylines_faq}{Outdoor
    gatherings}~lower risk because wind disperses viral droplets, and
    sunlight can kill some of the virus. Open spaces prevent the virus
    from building up in concentrated amounts and being inhaled, which
    can happen when infected people exhale in a confined space for long
    stretches of time, said Dr. Julian W. Tang, a virologist at the
    University of Leicester.
  \end{itemize}
\item ~
  \hypertarget{why-does-standing-six-feet-away-from-others-help}{%
  \paragraph{Why does standing six feet away from others
  help?}\label{why-does-standing-six-feet-away-from-others-help}}

  \begin{itemize}
  \tightlist
  \item
    The coronavirus spreads primarily through droplets from your mouth
    and nose, especially when you cough or sneeze. The C.D.C., one of
    the organizations using that measure,
    \href{https://www.nytimes3xbfgragh.onion/2020/04/14/health/coronavirus-six-feet.html?action=click\&pgtype=Article\&state=default\&region=MAIN_CONTENT_3\&context=storylines_faq}{bases
    its recommendation of six feet}~on the idea that most large droplets
    that people expel when they cough or sneeze will fall to the ground
    within six feet. But six feet has never been a magic number that
    guarantees complete protection. Sneezes, for instance, can launch
    droplets a lot farther than six feet,
    \href{https://jamanetwork.com/journals/jama/fullarticle/2763852}{according
    to a recent study}. It's a rule of thumb: You should be safest
    standing six feet apart outside, especially when it's windy. But
    keep a mask on at all times, even when you think you're far enough
    apart.
  \end{itemize}
\item ~
  \hypertarget{i-have-antibodies-am-i-now-immune}{%
  \paragraph{I have antibodies. Am I now
  immune?}\label{i-have-antibodies-am-i-now-immune}}

  \begin{itemize}
  \tightlist
  \item
    As of right
    now,\href{https://www.nytimes3xbfgragh.onion/2020/07/22/health/covid-antibodies-herd-immunity.html?action=click\&pgtype=Article\&state=default\&region=MAIN_CONTENT_3\&context=storylines_faq}{~that
    seems likely, for at least several months.}~There have been
    frightening accounts of people suffering what seems to be a second
    bout of Covid-19. But experts say these patients may have a
    drawn-out course of infection, with the virus taking a slow toll
    weeks to months after initial exposure.~People infected with the
    coronavirus typically
    \href{https://www.nature.com/articles/s41586-020-2456-9}{produce}~immune
    molecules called antibodies, which are
    \href{https://www.nytimes3xbfgragh.onion/2020/05/07/health/coronavirus-antibody-prevalence.html?action=click\&pgtype=Article\&state=default\&region=MAIN_CONTENT_3\&context=storylines_faq}{protective
    proteins made in response to an
    infection}\href{https://www.nytimes3xbfgragh.onion/2020/05/07/health/coronavirus-antibody-prevalence.html?action=click\&pgtype=Article\&state=default\&region=MAIN_CONTENT_3\&context=storylines_faq}{.
    These antibodies may}~last in the body
    \href{https://www.nature.com/articles/s41591-020-0965-6}{only two to
    three months}, which may seem worrisome, but that's~perfectly normal
    after an acute infection subsides, said Dr. Michael Mina, an
    immunologist at Harvard University. It may be possible to get the
    coronavirus again, but it's highly unlikely that it would be
    possible in a short window of time from initial infection or make
    people sicker the second time.
  \end{itemize}
\item ~
  \hypertarget{what-are-my-rights-if-i-am-worried-about-going-back-to-work}{%
  \paragraph{What are my rights if I am worried about going back to
  work?}\label{what-are-my-rights-if-i-am-worried-about-going-back-to-work}}

  \begin{itemize}
  \tightlist
  \item
    Employers have to provide
    \href{https://www.osha.gov/SLTC/covid-19/standards.html}{a safe
    workplace}~with policies that protect everyone equally.
    \href{https://www.nytimes3xbfgragh.onion/article/coronavirus-money-unemployment.html?action=click\&pgtype=Article\&state=default\&region=MAIN_CONTENT_3\&context=storylines_faq}{And
    if one of your co-workers tests positive for the coronavirus, the
    C.D.C.}~has said that
    \href{https://www.cdc.gov/coronavirus/2019-ncov/community/guidance-business-response.html}{employers
    should tell their employees}~-\/- without giving you the sick
    employee's name -\/- that they may have been exposed to the virus.
  \end{itemize}
\end{itemize}

After
\href{https://www.nytimes3xbfgragh.onion/interactive/2019/health/vaping-illness-tracker.html}{peaking
last September}, emergency department visits linked to Evali plummeted.
But the Centers for Disease Control and Prevention
\href{https://www.cdc.gov/tobacco/basic_information/e-cigarettes/severe-lung-disease.html\#latest-outbreak-information}{has
not updated their counts since February}, leaving experts worried that
concerns over vaping have fallen to the wayside. ``This has not gone
away from patients,'' said Michelle Eakin, a pulmonary disease expert at
Johns Hopkins University.

Dr. Crotty Alexander noted that she and other researchers have struggled
to follow up on many of last year's Evali cases, paradoxically thanks to
a pandemic that might hit some of these patients especially hard.

Early evidence hints that the virus may have an easier time breaking
into the bodies of smokers and vapers. Smoking appears to
\href{https://www.medrxiv.org/content/10.1101/2020.02.05.20020107v1?versioned=true}{alter
the surfaces} of
\href{https://www.medrxiv.org/content/10.1101/2020.02.05.20020107v3}{certain
cells}, prompting them to
\href{https://www.ncbi.nlm.nih.gov/pmc/articles/PMC7301735/}{coat
themselves with more of a molecule called ACE-2} --- the protein the
coronavirus uses to break into its targets.

``If you have higher expression, you're going to have more virus
entering cells,'' Dr. Crotty Alexander said. ``I'm now seeing the same
sort of data come out on the vaping side.''

That pattern, layered on top of the ways in which vaping weakens the
lungs, may help explain why a recent survey of more than 4,000 people
ages 13 to 24 found that vaping was
\href{https://www.sciencedirect.com/science/article/pii/S1054139X20303992}{strongly
linked to catching the coronavirus}. But Bonnie Halpern-Felsher, a
pediatrics researcher at Stanford University and an author on the study,
said that there was probably more than biology at play.

People who vape often do it socially, sharing spaces and equipment. And
vaping, like smoking, involves a lot of hand-to-mouth movement,
providing germs an easy path into the airway, Dr. Eakin said. ``And if
you're smoking or vaping,'' she said, ``you're not wearing a mask.''

Image

Arlie Frahmann, a longtime smoker, hesitated to give up cigarettes when
the coronavirus first struck. ``The last thing I wanted was to be
stressed out during quarantine,'' she said.Credit...Greta Rybus for The
New York Times

Still unclear are the long-term consequences of Covid's effects on those
who smoked or vaped. Accumulating evidence suggests that the coronavirus
can wreak havoc on blood vessels, seeding clots that suffocate and warp
tissues, including the lungs --- most likely making any smoking or
vaping after Covid even more dangerous than before.

``Some of these patients will have permanent issues,'' said Dr. Anne
Melzer, a pulmonologist at the University of Minnesota.

Arlie Frahmann, a longtime smoker who picked up her first cigarette at
the age of 9, hesitated to give up cigarettes when the coronavirus first
infiltrated her community in Damariscotta, Maine, this spring. ``The
last thing I wanted was to be stressed out during quarantine,'' she
said.

As of this week, though, Ms. Frahmann is eager to quit. She started a
new job at a bakery, where she will have to interact with strangers.

``It was one thing to explain it away to myself when I wasn't going into
public at all,'' she said of her smoking. ``But now I can't justify
it.''

A few early reports suggest that
\href{https://www.medrxiv.org/content/10.1101/2020.06.29.20142661v1}{some
people} \href{https://www.bbc.com/news/health-53403610}{may be shelving
their cigarettes} or vapes. As schools reopen for in-person learning,
though, it might become easy to relapse.

And Dr. Lovinsky-Desir worries that the stressors brought on by the
pandemic may be pushing some people to smoke or vape even more.

Mr. Moein recalls brushing off warnings from his father, who used to
send him articles about the dangers of vaping.

``I used to tell him, `You're out of touch, vaping is safer,''' he said.
``At one point, I was getting so many articles that I blocked his
number.''

But last year's events flipped Mr. Moein's worldview. The pandemic, he
said, is another reminder that the risks of vaping simply aren't worth
it: ``There's no way in hell that vaping helps Covid-19.''

Advertisement

\protect\hyperlink{after-bottom}{Continue reading the main story}

\hypertarget{site-index}{%
\subsection{Site Index}\label{site-index}}

\hypertarget{site-information-navigation}{%
\subsection{Site Information
Navigation}\label{site-information-navigation}}

\begin{itemize}
\tightlist
\item
  \href{https://help.nytimes3xbfgragh.onion/hc/en-us/articles/115014792127-Copyright-notice}{©~2020~The
  New York Times Company}
\end{itemize}

\begin{itemize}
\tightlist
\item
  \href{https://www.nytco.com/}{NYTCo}
\item
  \href{https://help.nytimes3xbfgragh.onion/hc/en-us/articles/115015385887-Contact-Us}{Contact
  Us}
\item
  \href{https://www.nytco.com/careers/}{Work with us}
\item
  \href{https://nytmediakit.com/}{Advertise}
\item
  \href{http://www.tbrandstudio.com/}{T Brand Studio}
\item
  \href{https://www.nytimes3xbfgragh.onion/privacy/cookie-policy\#how-do-i-manage-trackers}{Your
  Ad Choices}
\item
  \href{https://www.nytimes3xbfgragh.onion/privacy}{Privacy}
\item
  \href{https://help.nytimes3xbfgragh.onion/hc/en-us/articles/115014893428-Terms-of-service}{Terms
  of Service}
\item
  \href{https://help.nytimes3xbfgragh.onion/hc/en-us/articles/115014893968-Terms-of-sale}{Terms
  of Sale}
\item
  \href{https://spiderbites.nytimes3xbfgragh.onion}{Site Map}
\item
  \href{https://help.nytimes3xbfgragh.onion/hc/en-us}{Help}
\item
  \href{https://www.nytimes3xbfgragh.onion/subscription?campaignId=37WXW}{Subscriptions}
\end{itemize}
