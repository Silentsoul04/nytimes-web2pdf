Instagram Is Coming for Your Sock Drawer

\url{https://nyti.ms/2Z6KpnB}

\begin{itemize}
\item
\item
\item
\item
\item
\item
\end{itemize}

\includegraphics{https://static01.graylady3jvrrxbe.onion/images/2020/09/04/magazine/04-mag-home-edit-sock-still-wide/04-mag-home-edit-sock-still-wide-superJumbo.jpg}

Photo illustration by Margeaux Walter

Sections

\protect\hyperlink{site-content}{Skip to
content}\protect\hyperlink{site-index}{Skip to site index}

\hypertarget{instagram-is-coming-for-your-sock-drawer}{%
\section{Instagram Is Coming for Your Sock
Drawer}\label{instagram-is-coming-for-your-sock-drawer}}

The Home Edit won over Khloé Kardashian and Gwyneth Paltrow with
a~particular style of home organization. Now a new Netflix show will
showcase its high aspirations and ``low bar'' ethos.

Photo illustration by Margeaux WalterCredit...

Supported by

\protect\hyperlink{after-sponsor}{Continue reading the main story}

By Amanda FitzSimons

\begin{itemize}
\item
  Sept. 4, 2020
\item
  \begin{itemize}
  \item
  \item
  \item
  \item
  \item
  \item
  \end{itemize}
\end{itemize}

\hypertarget{listen-to-this-article}{%
\subsubsection{Listen to This Article}\label{listen-to-this-article}}

Audio Recording by Audm

\emph{To hear more audio stories from publishers like The New York
Times,}
\href{https://www.audm.com/?utm_source=nytmag\&utm_medium=embed\&utm_campaign=instagram_sock_drawer_fitzsimons}{\emph{download
Audm for iPhone or Android}}\emph{.}

Khloé Kardashian's color-coded hair-extension closet might sound like
some kind of art installation about late-stage capitalism, but it's
actually a thing that exists in Kardashian's Los Angeles home, and it's
exactly what it sounds like: a carpeted room dedicated entirely to 50 or
so clip-in extensions arranged on uniform black hangers in order of
lightest to darkest, beginning with a sheet of bleached-out, nearly
white strands and culminating in a swatch of strawberry-tinged almost
brown.

In January, Kardashian shared a picture of the room, with the caption
``50 shades of blonde,'' with her 120 million Instagram followers. The
closet was an extreme example of a trend that had been apparent in
celebrity and influencer culture for a while: aspirational organization,
tidiness as a sort of luxury item. At the forefront of this work is the
Home Edit, a Nashville-based home-organization company owned by Clea
Shearer and Joanna Teplin, that was responsible for Kardashian's closet.

Even if you're not one of the Home Edit's 1.6 million Instagram
followers, you've almost certainly seen the aesthetic that it has, if
not invented, at the very least codified; a look that can best be summed
up as Pinterest organization porn: rows of pristine white shelves filled
to no more than 75 percent capacity, pantries with paper towels artfully
arranged in the shape of a pyramid and, its hallmark, items organized in
order of Roygbiv (an acronym for the order of the colors as they appear
in a rainbow --- red-orange-yellow, etc.). If you
\href{https://www.instagram.com/explore/tags/pantry/?hl=en}{search for
the hashtag \#pantry on Instagram}, you'll turn up 400,000 images, many
of them kitchen closets organized in the style of the Home Edit.

This aesthetic has earned the Home Edit a roster of A-list celebrity
fans like the Kardashians (Kim, Kourtney, Khloé and their mother, Kris
Jenner), Reese Witherspoon and Gwyneth Paltrow --- all of whom are not
only clients but Home Edit evangelists. (Khloé Kardashian referred to
Shearer and Teplin as her ``soul mates.'') The Home Edit has a line of
organization accessories available at the Container Store, a
best-selling book title and another set to be published later this
month. On Sept. 9, Shearer and Teplin's show, ``Get Organized With the
Home Edit,'' will debut on Netflix.

The Home Edit's success would be singular were it not for the fact that
this decade has already ushered in a home-organization brand with a
Netflix show and best-selling book: Marie Kondo. When Kondo famously
encouraged followers to shed parts of their lives that no longer
``sparked joy,'' she was essentially peddling a form of self-help. It
was a promise laid bare even in the title of Kondo's book: ``The
Life-Changing Magic of Tidying Up.'' The Home Edit, by contrast, isn't
trying to sell you an improved life --- they're selling you permission
not to improve your life (at least outside of your closet). Highly
stylized home-organization content is only half of the Home Edit's
output on social media. The other half is the founders' quotidian ups
and downs, which they document on Instagram Stories. These vignettes
illustrate what the founders describe as their ``low-bar lifestyle'':
their contempt for working out, the trouble they have getting their
children to sleep, their neuroses (Teplin has an exhaustively documented
fear of battery acid). This has earned them such a dedicated audience
that even supporting cast members, like Shearer's mother, Roberta --- an
impossibly regal figure who routinely, to borrow internet parlance,
``throws low-key shade'' at her daughter --- get stopped in airports by
fans asking for selfies. On the brand's website, alongside organization
merchandise, they sell T-shirts with sayings such as ``Surviving Not
Thriving,'' ``Caffeine Until Cocktail Hour'' and ``Champagne Is
Basically Sparkling Water.''

I first met Shearer and Teplin in February, at a cocktail bar in the
Gramercy neighborhood of Manhattan connected to the hotel where they
were staying. They had just come from meetings at Martha Stewart Living
magazine and Instagram. It's not hard to understand why someone would
give them a TV show: Shearer and Teplin have a Penn-and-Teller
chemistry, with Teplin the straight man and Shearer, who speaks about 90
percent more, the jokester. (It's a dynamic exaggerated by their
physicality. Teplin, a 5-foot-1 blonde, is a head and change shorter
than the dark brunette Shearer.) When Shearer began to tell me about how
she would look forward to spring break in middle school as an
opportunity to clean her room, Teplin cut her off. ``She's so chill,''
Teplin said, rolling her eyes. They spoke in an on-brand patois. ``I'm
surviving not thriving by 8 p.m. with a glass of wine,'' Shearer said to
me at one point.

Their dynamic during our drinks matched their Instagram presence so
closely that when Shearer ordered Champagne, I stifled a laugh. Shearer
knew why. ``Some people see us in a restaurant or at an airport or
something like that, they come up to us, `Oh, my God, you really order
Champagne?''' she said. ``It's wild to me that people think I
wouldn't.''

But the fact is they \emph{do} seem like characters. The experience of
watching the Home Edit's Stories is like watching a sitcom --- a network
sitcom. There are catchphrases, recurring bits and issues that get
neatly resolved within the span of a few posts. It's focused and
digestible --- you feel as if you know these women well, even if you
only know a few things about them, like that Shearer's favorite drink is
Champagne. The women's relatability, and its disconnect from their work,
isn't lost on them; in fact, it is part of the pitch for their service:
``If we can figure out how to organize a pantry, we promise any of you
can.''

\textbf{In 2015, the} year the two women met, Teplin and Shearer were in
their early 30s, mothers of young children and recent transplants to
Nashville (Teplin from San Francisco; Shearer from Los Angeles). They
are also both Jewish, which made them something of a novelty in their
new hometown. (``There were, like, 11 of us,'' they joke.) It wasn't
long before they found themselves in each other's orbit. Each had
relocated for her husband's job and was looking for her next act.
Shearer had a background in social media, having worked for brands like
Nickelodeon and Myspace; Teplin had owned a few businesses, including a
greeting-card company. ``It's funny, we never asked the question `Would
you mind going into business together?''' Shearer said. ``We never did;
we just started talking.''

Technically, their business was organizing clients' homes in Nashville.
But even from the start, they were laying the foundation for something
bigger. Part of that meant creating a signature look. Even as the Home
Edit has inspired many knockoffs, some critics might call the brand's
style unoriginal --- certainly the Home Edit aesthetic drew from
different visual references already popular online. The genius was
reimagining it in the context of home organization. Beauty blogs had
kicked off the trend of personal belongings artfully arranged in
medicine cabinets and helped spur coinage of the word ``shelfie.''
Shearer and Teplin saw an opportunity for content in other places no one
ever bothered to make pretty --- linen closets, pantries, junk drawers,
the inside of a fridge, laundry rooms. A.S.M.R., a phenomenon that found
certain simple stimuli (for example, a whisper) can cause mild euphoria,
had fueled the rise of mindless content intended to delight the senses.
Shearer and Teplin designed tableaus that were not just visually
appealing but almost put the viewer into a trancelike calm: pantries
devoid of garish grocery-store packaging in favor of soothingly uniform
containers (tall plastic cylinders, raffia baskets) outfitted with twee
cursive labels to denote their contents: stevia, sugar, brown sugar.
(The Home Edit's distinctive labels are actually Shearer's handwriting,
which they made into a font; writing each label by hand wasn't
scalable.)

Perhaps the most obvious inspiration was the trend of bookshelves
grouped according to the color of book jacket and arranged in Roygbiv
order --- a look that appeared in the pages of glossy magazines like
Domino in the late aughts and spread quickly on Pinterest. ``We didn't
invent the rainbow,'' Shearer said. ``But we leveraged it.''

But it wasn't until Instagram introduced Stories, about a year into
their business, that the Home Edit really gained traction. As with their
aesthetic, Shearer and Teplin played with a subculture that was already
popular on Instagram, one that's come to be known as ``wine mom,'' in
which women broadcast their not having it all together. They remember
the first time they, as Teplin put it, ``pulled back the curtain'': four
years ago, en route to Dallas for business. ``Until that point, all you
saw was picture perfect, picture perfect,'' Teplin said. ``Let's show
people what it's really like to fly with us, and all of our airport
rituals, all of our things, and having to sit at the gate and watch
people and poll people as they get off the plane'' --- they like to ask
passengers getting off the plane they're about to get on what they can
expect --- ``and interview the pilot.'' Immediately, they saw a
``crazy'' response.

Their husbands, children and the aforementioned Roberta make appearances
on their Stories, but it's mostly the women, who spend more time
together than they do with their families. They travel together for work
up to six times a month and prefer to stay in the same hotel room,
``like the grandparents in the Chocolate Factory,'' Shearer told me,
where they text each other from the adjoining beds. Their codependence
is another huge part of their shtick. ``There is no time when there's
separation between us,'' Teplin said at the bar, and Shearer continued:
``Remember when you opened the door and I was in the shower? And I was
like, `Is this an emergency? Can. You. Write. It. Down? I'm literally
showering.'''

Eva Chen, head of fashion partnerships at Instagram and a Home Edit
client, explained to me why Stories was such a boon for them. ``Their
work is all about detail and everything is in its place. Then you watch
their Stories and it's their life and everything is not in its place.
Their flights are delayed, and they have issues just like us. That's the
best combination.''

That combination has turned out to be even more appealing since the
pandemic; in some ways, it's hard to imagine a brand more uniquely
suited to the needs of this new reality than the Home Edit. Lockdowns
changed people's relationship to their spaces, made them more aware of
the flaws in their homes: The fact that the shelf designated for coffee
mugs is nowhere near the dishwasher really grates when people are
unloading that dishwasher multiple times a week. Overnight, homes had to
become an office, a classroom, a restaurant and a Zoom-worthy backdrop.
People have been eager for projects --- and something they can control.
But it's also the Home Edit's ``low bar'' ethos that resonates with the
moment. On its website, slogan T-shirts addressed the pandemic more
directly: ``What Day Is It?'' And ``This Is Schitty'' (a reference to
one of their favorite TV shows, ``Schitt's Creek''). Its Instagram
account gained 100,000 followers this spring. In June, when most
companies were uncertain about their futures, the Home Edit announced it
was starting operations in 11 new cities, in addition to existing ones
in New York, L.A. and Nashville. In July, the women announced a
forthcoming line of products, including soap and hand sanitizer --- an
extension they described as ``organizing adjacent.''

\includegraphics{https://static01.graylady3jvrrxbe.onion/images/2020/09/04/magazine/04-mag-home-edit-spin-wide/04-mag-home-edit-spin-wide-threeByTwoMediumAt2X.jpg}

\textbf{The tension between} the two sides of the Home Edit brand ---
high aspirations combined with low expectations --- is built into the
structure of their Netflix show. Like 2019's ``Tidying Up With Marie
Kondo'' --- which was a hit with audiences --- the show will document
home-reorganization makeovers. Each episode will feature two --- a
regular person's and a celebrity's. Brandon Riegg, the Netflix executive
who greenlit their show, says viewers will get as much practical service
from the celebrity segments as they will from the segments featuring, as
he put it, ``civilians.'' ``I think they have a very accessible but
sophisticated aesthetic. You're not just watching something thinking:
Oh, if only I had that closet. If only I had that room.'' He explained
that the show will ``bookend what was started with Marie Kondo. With
Marie, it was all about decluttering. With the Home Edit, it's about
arranging.'' Guests for the season include Khloé Kardashian and
Witherspoon, whose production company, Hello Sunshine, helped produce
the show and who first discovered the Home Edit on Instagram. (When I
asked what she liked about Shearer and Teplin, Witherspoon emailed me a
picture of her pantry --- her cookbooks color-coded and snacks in glass
containers.)

Shearer, who is married to a photographer who often shoots country-music
stars (it's why they moved to Nashville), had a proximity to celebrity
that has proved useful. Her friends Selma Blair and Christina Applegate
helped spread the word about her business --- in particular to Gwyneth
Paltrow, who was their first big get, in 2017, after she enlisted
Shearer and Teplin to redo the playroom at her house in East Hampton.

Rare is the celebrity whose endorsement can be enough to hang an entire
career on; there are probably only a handful. One is Paltrow. Another is
any Kardashian family member. The Home Edit might be one of the few
brands that has managed to secure the endorsement of both. ``In the
Jewish religion, it's called `\emph{Dayenu},''' Teplin said, referring
to the Hebrew refrain, traditionally invoked during Passover Seder, that
roughly translates to ``it would have been enough.''

Kris Jenner told me she first heard about the brand two years ago, when
she visited her daughter Khloé's house and saw her pantry, which Shearer
and Teplin had redone. ``It was like nothing I'd ever seen before,''
Jenner said. ``Everything from the fruit and vegetables on her
countertops and the cookies and cupcakes and the way they styled
everything in her cabinet. To me, it was nirvana.''

It's easy to understand why the Kardashians, who can easily pay the Home
Edit rate, which ranges from \$185 to \$250 an hour, and then employ
people who can help maintain the order the women establish, would be
drawn to their services. But their aesthetic is so rigorous, it's
sometimes hard to imagine why the average person would even want to
attempt it. Shearer and Teplin have an answer: If something looks nice,
you're more likely to maintain it. (One of their favorite sayings is
``Form plus function equals magic.'')

But elements of the Home Edit's method can seem impractical. For
instance, what happens when you have a pantry stocked with green glass
bottles of Mountain Valley Spring Water --- as Shearer and Teplin
arranged for Mandy Moore --- and you just bought a bottle of Coke Zero?
Must you throw it out in the name of coordination? They have a term for
that: ``pantry paralysis.'' Teplin explained, ``You've got a bin for
pasta, a bin for beans, and now you have a free floater.'' She adopted a
mock-dramatic tone, continuing, ``All of a sudden, Oh, my God, your
world is in anarchy.''

When I pointed out to Teplin that I never saw a free floater in any of
their pictures, she replied, ``Well, that's for Instagram.'' In other
words, not all of the Home Edit's output is meant literally. Teplin went
on to explain that it's enough to put that Coke Zero in the pantry with
your other drinks or the fork in its rightful place in the utensil
drawer; lining everything up so that it's ``picture perfect'' is for
their own satisfaction.

Eva Chen admitted that those picture perfect elements could be difficult
to maintain. When the Home Edit organized her children's play area last
fall, Play-Doh tools were arranged in Roygbiv order, which, she said,
``blew my mind.'' But, she continued, ``Now anytime one of my kids wants
to take a cookie cutter out, I'll be like, `You have to put it back ---
maintain the rainbow.'''

Chen was poking fun at herself, but there was truth to her comment.
Instagram ``has caused us to become slightly cartoonlike,'' the fashion
designer Tom Ford said last year. He was referring to clothes, but it
applies to home décor as well. It's not simply that you should make
everything in your life beautiful so that you can document it on
Instagram; now many people make things beautiful \emph{for} Instagram.

When Netflix's Riegg was describing the Home Edit's show, this was part
of his pitch: ``They're not trying to be anything other than who they
are. There's Instagram and then there's real life.'' Riegg meant that
Shearer and Teplin have an obvious authenticity, which is an
increasingly rare commodity in our Instagram-perfect world. But in the
process, he had stumbled onto an inconsistency in the Home Edit. There's
probably no better illustration of the brand's divergent identities than
on its website, where a school-supply organization kit consisting of
acrylic dividers meant to sort crayons into color order is on offer
alongside a ``Low Bar Lifestyle'' slogan tee.

The whole reason the Home Edit's routine about living a low-bar
lifestyle resonates is because there are people on Instagram who are
setting up unrealistic expectations about how perfect your life should
look --- a trend the Home Edit fuels. We want to buy a Low Bar Lifestyle
T-shirt \emph{because} there are people who keep showing off their
color-coded drawers.

\textbf{In early March,} Shearer and Teplin attended a book signing for
``The Home Edit: A Guide to Organizing and Realizing Your House Goals''
at White's Mercantile in Charleston, S.C., one of those
gilded-age-meets-hipster boutiques, in the same vein of restaurants with
the word ``larder'' in their name. Even though the book came out a year
earlier and it was pouring rain, the turnout was impressive: a crowd of
70 --- almost all women, most under 35, some wearing rainbows in honor
of the authors --- that snaked out the door at points. **** Awareness of
the coronavirus was only starting to take hold, but the guests knew to
bump elbows in lieu of shaking hands and made ample use of a station of
hand sanitizer next to a tray of cocktails. (Still, that 70 unmasked
people were packed into a space the size of a one-bedroom apartment
seemed to faze no one.)

Revelers wound their way past repurposed apothecary tables stocked with
items like a plant-based ``sublingual spray'' for insomnia and plush
LickCroix dog chew toys to the front of the line, where the authors were
standing at attention in front of an Instagram-ready balloon-bouquet
backdrop. There was a group of women in their 20s who had clearly just
come from work (their laptop-size nylon bags giving them away) and a
local professional organizer who admired Shearer and Teplin's work. One
woman in her mid-50s, who had come with her two young-adult daughters,
told a long joke about Moira Rose, the character played by the actress
Catherine O'Hara on ``Schitt's Creek.''

The experience inside the store was not unlike being at a bizarro Star
Trek convention --- Shearer and Teplin had created this weird little
world, and many of the people there seemed fluent in its culture.
(``Surviving not thriving,'' one woman said as she picked up a French
fizz-style cocktail.) When one woman finally got to the front of the
line, she apologized for being wet from the rain and getting the authors
wet in the process. Shearer shrugged it off, ``That's low-bar life!'' In
that moment, it was hard not to think just how nicely that low-bar life
was working out for Shearer and Teplin.

\begin{center}\rule{0.5\linewidth}{\linethickness}\end{center}

Amanda FitzSimons is a writer based in Brooklyn and a former editor at
Elle magazine.
\href{https://www.nytimes3xbfgragh.onion/2019/05/22/magazine/the-view-politics-tv.html}{Her
last feature for this magazine was about the daytime talk show ``The
View.''}

Advertisement

\protect\hyperlink{after-bottom}{Continue reading the main story}

\hypertarget{site-index}{%
\subsection{Site Index}\label{site-index}}

\hypertarget{site-information-navigation}{%
\subsection{Site Information
Navigation}\label{site-information-navigation}}

\begin{itemize}
\tightlist
\item
  \href{https://help.nytimes3xbfgragh.onion/hc/en-us/articles/115014792127-Copyright-notice}{©~2020~The
  New York Times Company}
\end{itemize}

\begin{itemize}
\tightlist
\item
  \href{https://www.nytco.com/}{NYTCo}
\item
  \href{https://help.nytimes3xbfgragh.onion/hc/en-us/articles/115015385887-Contact-Us}{Contact
  Us}
\item
  \href{https://www.nytco.com/careers/}{Work with us}
\item
  \href{https://nytmediakit.com/}{Advertise}
\item
  \href{http://www.tbrandstudio.com/}{T Brand Studio}
\item
  \href{https://www.nytimes3xbfgragh.onion/privacy/cookie-policy\#how-do-i-manage-trackers}{Your
  Ad Choices}
\item
  \href{https://www.nytimes3xbfgragh.onion/privacy}{Privacy}
\item
  \href{https://help.nytimes3xbfgragh.onion/hc/en-us/articles/115014893428-Terms-of-service}{Terms
  of Service}
\item
  \href{https://help.nytimes3xbfgragh.onion/hc/en-us/articles/115014893968-Terms-of-sale}{Terms
  of Sale}
\item
  \href{https://spiderbites.nytimes3xbfgragh.onion}{Site Map}
\item
  \href{https://help.nytimes3xbfgragh.onion/hc/en-us}{Help}
\item
  \href{https://www.nytimes3xbfgragh.onion/subscription?campaignId=37WXW}{Subscriptions}
\end{itemize}
