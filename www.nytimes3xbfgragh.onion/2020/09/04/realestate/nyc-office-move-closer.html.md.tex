Sections

SEARCH

\protect\hyperlink{site-content}{Skip to
content}\protect\hyperlink{site-index}{Skip to site index}

\href{https://www.nytimes3xbfgragh.onion/section/realestate}{Real
Estate}

\href{https://myaccount.nytimes3xbfgragh.onion/auth/login?response_type=cookie\&client_id=vi}{}

\href{https://www.nytimes3xbfgragh.onion/section/todayspaper}{Today's
Paper}

\href{/section/realestate}{Real Estate}\textbar{}As Some New Yorkers
Flee, Others Move Closer to the Office

\url{https://nyti.ms/3bpaLGx}

\begin{itemize}
\item
\item
\item
\item
\item
\end{itemize}

\hypertarget{the-coronavirus-outbreak}{%
\subsubsection{\texorpdfstring{\href{https://www.nytimes3xbfgragh.onion/news-event/coronavirus?name=styln-coronavirus-national\&region=TOP_BANNER\&block=storyline_menu_recirc\&action=click\&pgtype=Article\&impression_id=36ab9870-f1c8-11ea-b5fb-29abc66233f3\&variant=undefined}{The
Coronavirus
Outbreak}}{The Coronavirus Outbreak}}\label{the-coronavirus-outbreak}}

\begin{itemize}
\tightlist
\item
  live\href{https://www.nytimes3xbfgragh.onion/2020/09/08/world/covid-19-coronavirus.html?name=styln-coronavirus-national\&region=TOP_BANNER\&block=storyline_menu_recirc\&action=click\&pgtype=Article\&impression_id=36ab9871-f1c8-11ea-b5fb-29abc66233f3\&variant=undefined}{Latest
  Updates}
\item
  \href{https://www.nytimes3xbfgragh.onion/interactive/2020/us/coronavirus-us-cases.html?name=styln-coronavirus-national\&region=TOP_BANNER\&block=storyline_menu_recirc\&action=click\&pgtype=Article\&impression_id=36abbf80-f1c8-11ea-b5fb-29abc66233f3\&variant=undefined}{Maps
  and Cases}
\item
  \href{https://www.nytimes3xbfgragh.onion/interactive/2020/science/coronavirus-vaccine-tracker.html?name=styln-coronavirus-national\&region=TOP_BANNER\&block=storyline_menu_recirc\&action=click\&pgtype=Article\&impression_id=36abbf81-f1c8-11ea-b5fb-29abc66233f3\&variant=undefined}{Vaccine
  Tracker}
\item
  \href{https://www.nytimes3xbfgragh.onion/2020/09/02/your-money/eviction-moratorium-covid.html?name=styln-coronavirus-national\&region=TOP_BANNER\&block=storyline_menu_recirc\&action=click\&pgtype=Article\&impression_id=36abbf82-f1c8-11ea-b5fb-29abc66233f3\&variant=undefined}{Eviction
  Moratorium}
\item
  \href{https://www.nytimes3xbfgragh.onion/interactive/2020/09/02/magazine/food-insecurity-hunger-us.html?name=styln-coronavirus-national\&region=TOP_BANNER\&block=storyline_menu_recirc\&action=click\&pgtype=Article\&impression_id=36abbf83-f1c8-11ea-b5fb-29abc66233f3\&variant=undefined}{American
  Hunger}
\end{itemize}

Advertisement

\protect\hyperlink{after-top}{Continue reading the main story}

Supported by

\protect\hyperlink{after-sponsor}{Continue reading the main story}

\hypertarget{as-some-new-yorkers-flee-others-move-closer-to-the-office}{%
\section{As Some New Yorkers Flee, Others Move Closer to the
Office}\label{as-some-new-yorkers-flee-others-move-closer-to-the-office}}

For workers who are required to be on-site, fear of public
transportation has spurred a wave of demand for homes within walking
distance of their jobs.

\includegraphics{https://static01.graylady3jvrrxbe.onion/images/2020/09/06/realestate/06moving-coronawalk/06moving-coronawalk-articleLarge-v4.gif?quality=75\&auto=webp\&disable=upscale}

By \href{https://www.nytimes3xbfgragh.onion/by/joyce-cohen}{Joyce Cohen}

\begin{itemize}
\item
  Sept. 4, 2020
\item
  \begin{itemize}
  \item
  \item
  \item
  \item
  \item
  \end{itemize}
\end{itemize}

From her East Village walk-up, Jessica Fine used to take the subway to
her job as a physician assistant on the Upper East Side. When the
pandemic began, she switched to Citi Bike for the three-mile commute.

Now she has plans to move, so she can avoid the subway for good.

In the spring, she and her fiancé started hunting for a co-op to buy.
``Proximity to work is a big factor for me,'' said Ms. Fine, 29. ``We
are looking in the radius where I can bike or walk to work. I work in a
hospital, so I will never be working from home.''

Avoiding public transit has been a ``precautionary measure that is easy
to take,'' she said, ``because even though I don't think I'm sick, I
don't know who's riding that subway car with me, who's sitting next to
me, who sat there an hour ago and coughed on the pole.''

Subway ridership fell by 90 percent in April, after Gov. Andrew Cuomo
signed the New York State on PAUSE executive order, advising New Yorkers
to limit their use of public transportation. With
\href{https://www.nytimes3xbfgragh.onion/2020/07/07/nyregion/nyc-unemployment.html}{unemployment}
reaching record levels, subway ridership has since
\href{https://www.nytimes3xbfgragh.onion/2020/08/26/nyregion/nyc-subway-bus-service-cuts.html}{reached
only a quarter of usual levels} even as more people have returned to
work in recent months.

Ms. Fine and her fiancé, Bryon Shek, are focusing on co-ops close to her
workplace at the Hospital for Special Surgery, on East 70th Street. Mr.
Shek, 33, who works in the field of demand planning, has an office north
of Herald Square, but is working from home indefinitely.

\includegraphics{https://static01.graylady3jvrrxbe.onion/images/2020/09/04/realestate/04Moving3/04Moving3-articleLarge.jpg?quality=75\&auto=webp\&disable=upscale}

While many city dwellers with the wherewithal are
\href{https://www.nytimes3xbfgragh.onion/2020/08/30/nyregion/nyc-suburbs-housing-demand.html}{moving
to the suburbs}, where they can find more space and work more
comfortably from home, real estate agents are reporting a surge of
interest from clients looking to live closer to their city jobs. For
essential workers and those whose jobs require them to be on-site ---
including medical workers like Ms. Fine --- the issue is especially
germane.

Ms. Fine and Mr. Shek, who were already getting tired of living on the
sixth floor of a seven-story walk-up, focused on Midtown East, said
their agent, Harjot Kaur Nayar, of Keller Williams NYC.

``Their priorities were very clear, which made it much easier to look
for homes --- a starter one-bedroom co-op type, easy enough for her to
get to the hospital and for him to work out of,'' Ms. Nayar said.

After touring about two dozen places, the couple set their sights on a
large
\href{https://www.nytimes3xbfgragh.onion/2017/12/20/realestate/living-in-murray-hill-manhattan.html}{Murray
Hill} studio, priced in the high \$400,000s, with a sleeping nook and an
office space carved from the living room.

\hypertarget{latest-updates-the-coronavirus-outbreak}{%
\section{\texorpdfstring{\href{https://www.nytimes3xbfgragh.onion/2020/09/08/world/covid-19-coronavirus.html?action=click\&pgtype=Article\&state=default\&region=MAIN_CONTENT_1\&context=storylines_live_updates}{Latest
Updates: The Coronavirus
Outbreak}}{Latest Updates: The Coronavirus Outbreak}}\label{latest-updates-the-coronavirus-outbreak}}

Updated 2020-09-08T11:04:36.368Z

\begin{itemize}
\tightlist
\item
  \href{https://www.nytimes3xbfgragh.onion/2020/09/08/world/covid-19-coronavirus.html?action=click\&pgtype=Article\&state=default\&region=MAIN_CONTENT_1\&context=storylines_live_updates\#link-4a77847f}{As
  senators return to Washington, an impasse over a virus relief package
  looms.}
\item
  \href{https://www.nytimes3xbfgragh.onion/2020/09/08/world/covid-19-coronavirus.html?action=click\&pgtype=Article\&state=default\&region=MAIN_CONTENT_1\&context=storylines_live_updates\#link-679303d7}{Nine
  drugmakers pledge to thoroughly vet any coronavirus vaccine.}
\item
  \href{https://www.nytimes3xbfgragh.onion/2020/09/08/world/covid-19-coronavirus.html?action=click\&pgtype=Article\&state=default\&region=MAIN_CONTENT_1\&context=storylines_live_updates\#link-1c973131}{`The
  lockdown killed my father': Farmer suicides add to India's virus
  misery.}
\end{itemize}

\href{https://www.nytimes3xbfgragh.onion/2020/09/08/world/covid-19-coronavirus.html?action=click\&pgtype=Article\&state=default\&region=MAIN_CONTENT_1\&context=storylines_live_updates}{See
more updates}

More live coverage:
\href{https://www.nytimes3xbfgragh.onion/live/2020/09/08/business/stock-market-today-coronavirus?action=click\&pgtype=Article\&state=default\&region=MAIN_CONTENT_1\&context=storylines_live_updates}{Markets}

If Mr. Shek returns to his office, he will have a quick walk rather than
a subway trip. A bonus: ``We will be able to avoid a crowded laundromat
because there is a laundry room on each floor,'' he said.

For Ms. Fine, whose workday sometimes starts at 6 a.m., the subway or
bus ``is there if I want it,'' she said. ``I anticipate biking more
often than I did pre-Covid. I think it has become more valuable to
people --- the proximity to work and not having to rely on public
transit. It's a factor in a big decision because we are making a
multiyear, multi-thousand-dollar investment, and want to make sure work
is not going to be compromised.''

Pandemic or not, New York is a walkable city. In recent years, around 10
percent of New Yorkers have reported walking to work, with about 56
percent using public transit, according to census information.

Recent figures from the real estate data site
\href{http://urbandigs.com/}{UrbanDigs}, examining the ratio of new
listings to contracts signed, show that so far this year, pending sales
of homes within a 10-minute walk to ``work'' neighborhoods have fallen
34 percent from the market average. (``Work'' neighborhoods are defined
as those dense with office buildings, such as Midtown and the financial
district. Meanwhile, pending sales in ``neutral'' neighborhoods, which
are more than a 10-minute walk from a concentration of schools or office
buildings, have risen 16 percent.) That's not surprising --- many people
who are working remotely or who have lost their jobs aren't looking to
move into crowded neighborhoods right now. But real estate agents say
that among workers who must travel to work --- as well as some who know
they'll be returning to their offices eventually --- demand in these
areas is robust.

At an available one-bedroom co-op unit on East 57th Street, listed in
the high \$500,000s, many prospective buyers specifically have mentioned
proximity to their
\href{https://www.nytimes3xbfgragh.onion/2014/06/08/realestate/condos-and-restaurants-make-midtown-east-more-desirable.html}{Midtown}
offices, said the listing agent, Ryan Aussem, of Brown Harris Stevens.
``People state it like it's normal now: `I will go back to work at some
point; I am looking where I won't have to rely on public
transportation.'''

His sellers there, Heather and Patrick Hofer, both of whom work in
finance, enjoyed walking to their Midtown offices before the city shut
down, with stops for coffee (her) or the gym (him) along the way. ``When
everyone started getting freaked out and said we have to work at home, I
was not concerned, because I didn't have to take the subway,'' Mr. Hofer
said. ``I could walk to work and keep myself safe. I still wanted to go
to the office because we were blessed to have an apartment so close to
work.''

Mr. Hofer was soon required to work remotely, so his walk became moot.
Now, with an infant son, the Hofers are planning to decamp to a house in
Connecticut.

Image

Alessandra Rago on the Upper East Side, wearing her new commuting
sneakers. ``I am using Google maps to gauge how sustainable a walking
commute would be for any given apartment,'' she said. ``Will I walk an
extra seven minutes for a bigger apartment?''Credit...Courtesy of
Alessandra Rago

Many New Yorkers cannot avoid a lengthy subway or bus ride because they
commute to jobs in Manhattan from other boroughs. But until this year,
Mr. Aussem said, most of his buyers were generally content with a
20-minute subway commute. ``Now, it's: Let's make that a 15-minute
walk,'' he said. ``You have people who are really focusing on a
long-term play in their life, where they are altering their
transportation situation so they can have a safer, or what is perceived
as safer, way to get to work.''

It's not just trips to work. Some people hope to avoid public
transportation to almost everywhere. The pandemic spurred Alessandra
Rago to give up her downtown rental, which required a subway trip to her
office near Rockefeller Center. She has been staying with her parents in
New Jersey while working remotely. It seemed like a good time to hunt
for a one-bedroom to buy.

``I was a bit of a germaphobe before the virus, so I am probably of the
population that is taking the virus more seriously than others,'' said
Ms. Rago, 29, who works at an asset management firm. ``I can't even
imagine being on the subway until everything calms down and we return to
some kind of normalcy. Everything is so uncertain going forward that I
want as much flexibility in my new apartment as possible. I don't want
to depend on the subway or the bus for the places I go every day.''

Those places include a good grocery store and her exercise studio, which
has multiple branches. (Gyms in New York City were
\href{https://www.nytimes3xbfgragh.onion/2020/08/17/nyregion/nyc-gyms-reopening.html}{permitted
to reopen on Sept. 2}, with restrictions.)

``Before we looked at anything, she had to `Google walk' them to find
out if that element would work for her,'' said her agent, Brenda Di
Bari, of Halstead.

``I am using Google Maps to gauge how sustainable a walking commute
would be for any given apartment,'' said Ms. Rago, who is targeting
co-ops in Midtown East in the \$600,000 range. ``Will I walk an extra
seven minutes for a bigger apartment? A lot of these places have
trade-offs.''

\href{https://www.nytimes3xbfgragh.onion/news-event/coronavirus?action=click\&pgtype=Article\&state=default\&region=MAIN_CONTENT_3\&context=storylines_faq}{}

\hypertarget{the-coronavirus-outbreak-}{%
\subsubsection{The Coronavirus Outbreak
›}\label{the-coronavirus-outbreak-}}

\hypertarget{frequently-asked-questions}{%
\paragraph{Frequently Asked
Questions}\label{frequently-asked-questions}}

Updated September 4, 2020

\begin{itemize}
\item ~
  \hypertarget{what-are-the-symptoms-of-coronavirus}{%
  \paragraph{What are the symptoms of
  coronavirus?}\label{what-are-the-symptoms-of-coronavirus}}

  \begin{itemize}
  \tightlist
  \item
    In the beginning, the coronavirus
    \href{https://www.nytimes3xbfgragh.onion/article/coronavirus-facts-history.html?action=click\&pgtype=Article\&state=default\&region=MAIN_CONTENT_3\&context=storylines_faq\#link-6817bab5}{seemed
    like it was primarily a respiratory illness}~--- many patients had
    fever and chills, were weak and tired, and coughed a lot, though
    some people don't show many symptoms at all. Those who seemed
    sickest had pneumonia or acute respiratory distress syndrome and
    received supplemental oxygen. By now, doctors have identified many
    more symptoms and syndromes. In April,
    \href{https://www.nytimes3xbfgragh.onion/2020/04/27/health/coronavirus-symptoms-cdc.html?action=click\&pgtype=Article\&state=default\&region=MAIN_CONTENT_3\&context=storylines_faq}{the
    C.D.C. added to the list of early signs}~sore throat, fever, chills
    and muscle aches. Gastrointestinal upset, such as diarrhea and
    nausea, has also been observed. Another telltale sign of infection
    may be a sudden, profound diminution of one's
    \href{https://www.nytimes3xbfgragh.onion/2020/03/22/health/coronavirus-symptoms-smell-taste.html?action=click\&pgtype=Article\&state=default\&region=MAIN_CONTENT_3\&context=storylines_faq}{sense
    of smell and taste.}~Teenagers and young adults in some cases have
    developed painful red and purple lesions on their fingers and toes
    --- nicknamed ``Covid toe'' --- but few other serious symptoms.
  \end{itemize}
\item ~
  \hypertarget{why-is-it-safer-to-spend-time-together-outside}{%
  \paragraph{Why is it safer to spend time together
  outside?}\label{why-is-it-safer-to-spend-time-together-outside}}

  \begin{itemize}
  \tightlist
  \item
    \href{https://www.nytimes3xbfgragh.onion/2020/05/15/us/coronavirus-what-to-do-outside.html?action=click\&pgtype=Article\&state=default\&region=MAIN_CONTENT_3\&context=storylines_faq}{Outdoor
    gatherings}~lower risk because wind disperses viral droplets, and
    sunlight can kill some of the virus. Open spaces prevent the virus
    from building up in concentrated amounts and being inhaled, which
    can happen when infected people exhale in a confined space for long
    stretches of time, said Dr. Julian W. Tang, a virologist at the
    University of Leicester.
  \end{itemize}
\item ~
  \hypertarget{why-does-standing-six-feet-away-from-others-help}{%
  \paragraph{Why does standing six feet away from others
  help?}\label{why-does-standing-six-feet-away-from-others-help}}

  \begin{itemize}
  \tightlist
  \item
    The coronavirus spreads primarily through droplets from your mouth
    and nose, especially when you cough or sneeze. The C.D.C., one of
    the organizations using that measure,
    \href{https://www.nytimes3xbfgragh.onion/2020/04/14/health/coronavirus-six-feet.html?action=click\&pgtype=Article\&state=default\&region=MAIN_CONTENT_3\&context=storylines_faq}{bases
    its recommendation of six feet}~on the idea that most large droplets
    that people expel when they cough or sneeze will fall to the ground
    within six feet. But six feet has never been a magic number that
    guarantees complete protection. Sneezes, for instance, can launch
    droplets a lot farther than six feet,
    \href{https://jamanetwork.com/journals/jama/fullarticle/2763852}{according
    to a recent study}. It's a rule of thumb: You should be safest
    standing six feet apart outside, especially when it's windy. But
    keep a mask on at all times, even when you think you're far enough
    apart.
  \end{itemize}
\item ~
  \hypertarget{i-have-antibodies-am-i-now-immune}{%
  \paragraph{I have antibodies. Am I now
  immune?}\label{i-have-antibodies-am-i-now-immune}}

  \begin{itemize}
  \tightlist
  \item
    As of right
    now,\href{https://www.nytimes3xbfgragh.onion/2020/07/22/health/covid-antibodies-herd-immunity.html?action=click\&pgtype=Article\&state=default\&region=MAIN_CONTENT_3\&context=storylines_faq}{~that
    seems likely, for at least several months.}~There have been
    frightening accounts of people suffering what seems to be a second
    bout of Covid-19. But experts say these patients may have a
    drawn-out course of infection, with the virus taking a slow toll
    weeks to months after initial exposure.~People infected with the
    coronavirus typically
    \href{https://www.nature.com/articles/s41586-020-2456-9}{produce}~immune
    molecules called antibodies, which are
    \href{https://www.nytimes3xbfgragh.onion/2020/05/07/health/coronavirus-antibody-prevalence.html?action=click\&pgtype=Article\&state=default\&region=MAIN_CONTENT_3\&context=storylines_faq}{protective
    proteins made in response to an
    infection}\href{https://www.nytimes3xbfgragh.onion/2020/05/07/health/coronavirus-antibody-prevalence.html?action=click\&pgtype=Article\&state=default\&region=MAIN_CONTENT_3\&context=storylines_faq}{.
    These antibodies may}~last in the body
    \href{https://www.nature.com/articles/s41591-020-0965-6}{only two to
    three months}, which may seem worrisome, but that's~perfectly normal
    after an acute infection subsides, said Dr. Michael Mina, an
    immunologist at Harvard University. It may be possible to get the
    coronavirus again, but it's highly unlikely that it would be
    possible in a short window of time from initial infection or make
    people sicker the second time.
  \end{itemize}
\item ~
  \hypertarget{what-are-my-rights-if-i-am-worried-about-going-back-to-work}{%
  \paragraph{What are my rights if I am worried about going back to
  work?}\label{what-are-my-rights-if-i-am-worried-about-going-back-to-work}}

  \begin{itemize}
  \tightlist
  \item
    Employers have to provide
    \href{https://www.osha.gov/SLTC/covid-19/standards.html}{a safe
    workplace}~with policies that protect everyone equally.
    \href{https://www.nytimes3xbfgragh.onion/article/coronavirus-money-unemployment.html?action=click\&pgtype=Article\&state=default\&region=MAIN_CONTENT_3\&context=storylines_faq}{And
    if one of your co-workers tests positive for the coronavirus, the
    C.D.C.}~has said that
    \href{https://www.cdc.gov/coronavirus/2019-ncov/community/guidance-business-response.html}{employers
    should tell their employees}~-\/- without giving you the sick
    employee's name -\/- that they may have been exposed to the virus.
  \end{itemize}
\end{itemize}

She has been touring prospective homes masked and gloved, she said:
``Brokers don't even want you to touch anything.''

For Josie O'Toole, a renter, the pandemic became an unexpected reason to
leave her beloved West Village, where she lived for seven years. She is
a pediatric nurse practitioner at Mount Sinai Hospital, north of
\href{https://www.nytimes3xbfgragh.onion/2016/11/13/realestate/carnegie-hill-a-quiet-enclave-bordering-the-park.html}{Carnegie
Hill}. Working from home, she said, ``doesn't apply to health care. I
have to be there.''

Image

Josie O'Toole, a pediatric nurse practitioner at Mount Sinai Hospital,
started commuting by Citi Bike, but the ride was long. ``I decided to
look on the Upper East Side, which I thought I would never do,'' she
said. ``The West Village was where my friends were and my life
was.''Credit...September Dawn Bottoms/The New York Times

Ms. O'Toole, 30, had been commuting via the packed 6 train. ``When
everything started to get scary, I had taken the subway once and decided
that was going to be the last time,'' she said. She switched to Citi
Bike, but the ride was long, so she temporarily stayed at a friend's
place near her hospital. (The friend had skipped town to avoid the
virus.)

``I didn't know how long this was going to go on for,'' she said, ``so I
decided to look on the Upper East Side, which I thought I would never
do. The West Village was where my friends were and my life was.''

No longer. Several friends moved back home with their parents. ``That
made it easier to leave the West Village,'' she said. ``My friends
weren't even there. All of my favorite places were boarded up.''

Some rentals weren't allowing in-person visits, which limited her
options, and she refused to take a place sight unseen. ``I was way too
skeptical because that can be so deceiving,'' Ms. O'Toole said. ``You
can't check stuff like water pressure. You can't see into every
corner.''

She was able to visit one studio in person, about 20 blocks south of the
hospital. That's the one she now calls home. She has a 20-minute walk to
work, or a five-minute Citi Bike ride. ``I will bring a Clorox wipe with
me and wipe down the handlebars,'' she said.

The listing agent for her rental, Peggy Dahan, of Brown Harris Stevens,
is seeing a brisk rental market. ``People want to be no more than 30
blocks from work because they want to get the fresh air,'' she said.
``Once they get to work, they are in their mask, so because they are
going to be in the mask all day, they prefer to walk. They would love to
take the Citi Bike, but sooner or later it is going to start raining and
be cold.''

Ms. Dahan also sees a trend away from shared apartments, even if it
means downsizing to studios. ``People want to separate from their
roommates,'' she said. ``One roommate wants to be in the mask and one
couldn't care less. One is partying and one is scared to go out. People
will spend an extra \$800 or \$1,000 a month to be healthy alone, on
their own.''

Image

Dr. Elie Harouche, a surgeon, was so eager to be close to work that he
bought a one-bedroom Midtown pied-à-terre before setting foot in
it.Credit...Hilary Swift for The New York Times

Despite the risk in buying a home sight unseen, Dr. Elie Harouche, a
surgeon, bought a one-bedroom Midtown pied-à-terre before setting foot
in it. (He and his wife, Rosemary Harouche, live primarily in Suffolk
County.) His main criterion was to be minutes from his workplace, a new
medical spa on East 57th Street, Clinique des Champs-Élysées, where he
is the director. The opening, which was delayed, is scheduled for next
week.

``The idea is to be proactive, so that if this virus is as virulent as
it's explained, then we have to respect that,'' said Dr. Harouche, 71.
``As a physician, I totally agree with all the distancing and
face-covering and hand-washing. There are ways to maintain health and
decrease the collateral damage this thing is doing to us.''

In April, there were plenty of available options for what the Harouches
were seeking --- a one-bedroom in a doorman building in the 50s for
around \$600,000, said their agent, Jed Lewin of Triplemint.

``The trend I've been seeing is an either-or,'' Mr. Lewin said, noting
that some people crave more space, whether indoor or outdoor, while
others who anticipate a return to the office want proximity to work.
``Everyone has been re-evaluating whether being in a certain area has
the same advantages as it used to. For some, they really want to reduce
their risk of exposure by being around as few people as possible. For
people who do need to be somewhere specific, being able to walk is a
driving force.''

A transit-free commute, he said ``is a small measure of control, and
control is what a lot of people have felt has been lacking.''

For weekly email updates on residential real estate news,
\href{http://www.nytimes3xbfgragh.onion/newsletters/realestate/}{sign up
here}. Follow us on Twitter:
\href{https://twitter.com/nytrealestate}{@nytrealestate}.

Advertisement

\protect\hyperlink{after-bottom}{Continue reading the main story}

\hypertarget{site-index}{%
\subsection{Site Index}\label{site-index}}

\hypertarget{site-information-navigation}{%
\subsection{Site Information
Navigation}\label{site-information-navigation}}

\begin{itemize}
\tightlist
\item
  \href{https://help.nytimes3xbfgragh.onion/hc/en-us/articles/115014792127-Copyright-notice}{©~2020~The
  New York Times Company}
\end{itemize}

\begin{itemize}
\tightlist
\item
  \href{https://www.nytco.com/}{NYTCo}
\item
  \href{https://help.nytimes3xbfgragh.onion/hc/en-us/articles/115015385887-Contact-Us}{Contact
  Us}
\item
  \href{https://www.nytco.com/careers/}{Work with us}
\item
  \href{https://nytmediakit.com/}{Advertise}
\item
  \href{http://www.tbrandstudio.com/}{T Brand Studio}
\item
  \href{https://www.nytimes3xbfgragh.onion/privacy/cookie-policy\#how-do-i-manage-trackers}{Your
  Ad Choices}
\item
  \href{https://www.nytimes3xbfgragh.onion/privacy}{Privacy}
\item
  \href{https://help.nytimes3xbfgragh.onion/hc/en-us/articles/115014893428-Terms-of-service}{Terms
  of Service}
\item
  \href{https://help.nytimes3xbfgragh.onion/hc/en-us/articles/115014893968-Terms-of-sale}{Terms
  of Sale}
\item
  \href{https://spiderbites.nytimes3xbfgragh.onion}{Site Map}
\item
  \href{https://help.nytimes3xbfgragh.onion/hc/en-us}{Help}
\item
  \href{https://www.nytimes3xbfgragh.onion/subscription?campaignId=37WXW}{Subscriptions}
\end{itemize}
