Sections

SEARCH

\protect\hyperlink{site-content}{Skip to
content}\protect\hyperlink{site-index}{Skip to site index}

\href{https://www.nytimes3xbfgragh.onion/section/style}{Style}

\href{https://myaccount.nytimes3xbfgragh.onion/auth/login?response_type=cookie\&client_id=vi}{}

\href{https://www.nytimes3xbfgragh.onion/section/todayspaper}{Today's
Paper}

\href{/section/style}{Style}\textbar{}`You Should Break Up With Me'

\url{https://nyti.ms/31UVNEU}

\begin{itemize}
\item
\item
\item
\item
\item
\end{itemize}

Advertisement

\protect\hyperlink{after-top}{Continue reading the main story}

Supported by

\protect\hyperlink{after-sponsor}{Continue reading the main story}

Modern Love

\hypertarget{you-should-break-up-with-me}{%
\section{`You Should Break Up With
Me'}\label{you-should-break-up-with-me}}

Life can be fleeting. She wanted to make sure he knew the risks of
connection.

\includegraphics{https://static01.graylady3jvrrxbe.onion/images/2020/09/06/fashion/06MODERN-EXHALE/06MODERN-EXHALE-articleLarge.jpg?quality=75\&auto=webp\&disable=upscale}

By Marjorie S. Rosenthal

\begin{itemize}
\item
  Sept. 4, 2020
\item
  \begin{itemize}
  \item
  \item
  \item
  \item
  \item
  \end{itemize}
\end{itemize}

My husband, Amal, used to spend a lot of time in helicopters, his bow
tie peeking out of his flight suit. This was in Baltimore, where he and
I were pediatricians. Amal worked in a hospital's intensive care unit,
and part of his job was flying with very sick children who needed to be
transferred by air. When I expressed worry about him doing this, he
would point out that more people die in cars than helicopters.

And then, four years and two children into our marriage, I watched from
a trailing car as the rear left tire exploded on the S.U.V. he was
driving, causing the vehicle to flip over twice, crushing him.

The ensuing grief, for me, felt like being near one of Amal's landing
helicopters --- a swirl of noise and confusion, the earth vibrating. In
the years since his death, my grief subsided --- enough for me to
consider dating, of all people, a helicopter engineer.

Brian emailed me through Match.com about 13 years after Amal died. His
profile name was ``RelaxExhale,'' and he wrote that he believed in the
scientific method. A science-minded guy who was into yoga seemed a great
match for me.

Over the phone, though, I learned that ``RelaxExhale'' was not about
yoga; it was Brian coaching himself to take the leap into online dating.
By then I was living in New Haven, Conn., and our first date was on a
warm Sunday morning at a coffee shop called Cafe Romeo.

\emph{{[}}\href{https://www.nytimes3xbfgragh.onion/newsletters/love-letter}{\emph{Sign
up for Love Letter, our weekly email about Modern Love, weddings and
relationships.}}\emph{{]}}

``I feel like you're not like my kids' pediatrician,'' Brian said. ``You
do other stuff.''

Clearly, he had Googled me.

``I do regular pediatrician stuff,'' I said. ``But I also teach research
skills to doctors and nurses who want to study things like asthma,
homelessness, diabetes.''

``I have diabetes!'' he said, a little too excitedly.

``I have colon cancer,'' I said, which, in hindsight, seems like a
ridiculous response, but in the moment felt like the next step in the
conversation. ``I have a port in my chest, under my skin, where I get
chemotherapy every two weeks.''

I realized I was divulging a lot for a first date. Not only was I a
widow and a single mother, but also a cancer patient. Wearing a blousy
linen white shirt over a pink tank top, I was glad he couldn't see the
raised outline and persistent bruise over my embedded plastic
port-a-cath.

``I have two ports,'' Brian said. ``I mean, not that I want to outdo
you.'' He lifted his shirt to show me two clear tubes running from what
looked like high-tech Band-Aids on his belly to a black plastic box the
size of a beeper.

I did not lift my shirt to show off my port, but we declared ourselves
compatible enough for a second date.

We planned a hike and a swim at Bluff Point State Park, and Brian
offered to drive. I knew it was a little risky to give this relative
stranger my home address and agree to a daylong outing over 4th of July
weekend for our second date. But my adolescent daughters were at camp in
New Hampshire for two weeks, and Brian seemed like a good guy.

The day before, he texted: ``Because I am Irish, I will bring plenty of
sunscreen.''

``Because I am a pediatrician,'' I replied, ``I will already have plenty
of sunscreen on.''

When I saw his car pull up, my heart lurched. I stood up from my porch
steps. Amal had been a part of every relationship I'd had since becoming
a widow. And here he was again: Brian was driving the same make and
model of S.U.V. as the one Amal died in.

I almost went back into my house to avoid the car, Brian, Amal, all of
it. ``Relax, exhale,'' I said to myself. I, too, believed in science.
The matching S.U.V.s were probably nothing but a coincidence.

The traffic from New Haven to the state park was bumper to bumper. A
one-hour trip became two. On the day Amal died, the faulty car and tire
conspired with typical highway speeds to create the fatal crash. Amal
might not have died had traffic like this slowed him down.

As we drove, Brian told long stories of travels and family. I volleyed
back with my own. He was interested in my complicated,
pediatrician-widow-with-two-teenage-daughters-and-cancer life. He was
gracious in his empathy over Amal's death but could not hold back his
engineer self from declaring his disappointment in our malfunctioning
S.U.V.

On a secluded part of the hike, we separated to change into our bathing
suits. Between the bushes, I fumbled with my shoulder strap to make sure
it covered my port. But when we reunited, I realized my concerns were
for naught; Brian blushed at the sight of me in my suit. He wasn't
eyeing me as a person with cancer and wondering about my port. This was
a date.

After the swim, Brian checked one of his ports, ate some Craisins, and
gave himself insulin through the other port. We ate buttery lobster on
picnic tables near the beach, and later, in New Haven, licked ice cream
cones at little metal tables in front of the art museum.

I invited him to my neighbors' 3rd of July potluck the next day. This
time when Brian showed up at my house in his death car and I rose from
my porch steps, he held a pan of potato salad, his deceased mother's
recipe.

Of course, I laughed to myself, it was only fair. Brian needed to bring
a dead person from his side to our third date.

On our first date, we had discussed talking and not talking about our
feelings.

``Like any good Irishman,'' he said, ``I don't like to talk about my
feelings. Maybe I tell long, elliptical stories so I don't have to.''

``Like any good Jewish woman, I talk about my feelings a lot,'' I said.
``Do you want to know how I feel about my hangnail?''

That summer, we talked about Amal, Brian's mother, my grandfather who
died from diabetes, and all the other dead people who were in this
relationship with us. When we walked on the beach and saw helicopters
overhead, Brian told me which ones he had worked on and where they were
likely going.

I didn't talk about how the low survival rate of metastatic colon cancer
connected me to my own mortality. I knew all about being honest about
symptoms and prognosis, but I didn't want to ruin what felt like a
carefree, fun fling. And Brian didn't ask.

Four months later, as we hiked on fallen yellow leaves, I said, ``You
should break up with me.''

My cancer had grown. I was about to go back on intensive chemotherapy
and my surgeon was planning to remove large parts of my colon, pancreas
and abdominal wall.

Brian had just returned from California and brought me earrings with
butterfly parts encased in plastic. He promised no butterflies had died
for my earrings, and I promised him that the right decision would be for
him to get out of this relationship.

``What if I get really sick or die?'' I said.

``But what if you get better?''

Who was this guy? One of the first things I learned about Brian was that
he believed in the scientific method. Every time he marveled at my
well-seeming, cancer-laden self, I was pretty sure he was comparing me
to what he had read and knew. He had the math skills to understand what
a 14 percent five-year survival rate meant.

Maybe ``RelaxExhale'' was more than Brian coaching himself to get online
and date. Maybe he was a yoga guy whose heart had veto power over his
engineer head.

Brian didn't break up with me. And, as predicted, the 14-hour operation
left me sick with pancreatitis and sepsis. I had to go back into the
operating room for a small bowel obstruction, after which I was in and
out of the hospital for four months. Many of those times, Brian met me
in the E.R. and followed me to my room, where he would strum his
ukulele, make lip-syncing videos and sleep on the couch next to my bed.

When Brian had an extended business trip, he gave away his S.U.V*.* We
played that one more Irish than Jewish: We didn't discuss how every time
I entered that car I'd wondered if I was going to die, or how its
absence now gave cancer a larger chance of doing me in.

On our hike in the fall before my surgery, I tried to teach Brian that
love always ends. Car crashes, diabetes, cancer. What happened instead
is he taught me that love is a long, elliptical story. And that, with
and without all of our ghosts, I was lucky to be a part of his.

\href{https://twitter.com/marjoriesue}{Marjorie S. Rosenthal} is a
pediatrician in New Haven, Connecticut.

Modern Love can be reached at
\href{mailto:modernlove@NYTimes.com}{\nolinkurl{modernlove@NYTimes.com}}.

Want more from Modern Love? Watch the
\href{https://www.nytimes3xbfgragh.onion/2019/09/12/style/modern-love-tv-show-trailer.html}{TV
series}; sign up for the
\href{https://www.nytimes3xbfgragh.onion/newsletters/love-letter}{newsletter};
or listen to the
\href{https://www.nytimes3xbfgragh.onion/column/modern-love-podcast}{podcast}
on
\href{https://itunes.apple.com/us/podcast/modern-love/id1065559535?mt=2\&version=meter+at+0\&module=meter-Links\&pgtype=article\&contentId=\&mediaId=\&referrer=\&priority=true\&action=click\&contentCollection=meter-links-click}{iTunes},
\href{https://open.spotify.com/show/03Er7mSPq9IEewOgbPD3vO}{Spotify} or
\href{https://play.google.com/music/listen?u=0\#/ps/Iktqjbkz7bychbnofblw32dik64}{Google
Play}. We also have swag at
\href{https://store.nytimes3xbfgragh.onion/collections/modern-love}{the
NYT Store} and a book,
``\href{https://www.penguinrandomhouse.com/books/623036/modern-love-revised-and-updated-by-edited-by-daniel-jones-with-contributions-by-andrew-rannells-ayelet-waldman-amy-krouse-rosenthal-veronica-chambers-and-more/}{Modern
Love: True Stories of Love, Loss, and Redemption}.''

Advertisement

\protect\hyperlink{after-bottom}{Continue reading the main story}

\hypertarget{site-index}{%
\subsection{Site Index}\label{site-index}}

\hypertarget{site-information-navigation}{%
\subsection{Site Information
Navigation}\label{site-information-navigation}}

\begin{itemize}
\tightlist
\item
  \href{https://help.nytimes3xbfgragh.onion/hc/en-us/articles/115014792127-Copyright-notice}{©~2020~The
  New York Times Company}
\end{itemize}

\begin{itemize}
\tightlist
\item
  \href{https://www.nytco.com/}{NYTCo}
\item
  \href{https://help.nytimes3xbfgragh.onion/hc/en-us/articles/115015385887-Contact-Us}{Contact
  Us}
\item
  \href{https://www.nytco.com/careers/}{Work with us}
\item
  \href{https://nytmediakit.com/}{Advertise}
\item
  \href{http://www.tbrandstudio.com/}{T Brand Studio}
\item
  \href{https://www.nytimes3xbfgragh.onion/privacy/cookie-policy\#how-do-i-manage-trackers}{Your
  Ad Choices}
\item
  \href{https://www.nytimes3xbfgragh.onion/privacy}{Privacy}
\item
  \href{https://help.nytimes3xbfgragh.onion/hc/en-us/articles/115014893428-Terms-of-service}{Terms
  of Service}
\item
  \href{https://help.nytimes3xbfgragh.onion/hc/en-us/articles/115014893968-Terms-of-sale}{Terms
  of Sale}
\item
  \href{https://spiderbites.nytimes3xbfgragh.onion}{Site Map}
\item
  \href{https://help.nytimes3xbfgragh.onion/hc/en-us}{Help}
\item
  \href{https://www.nytimes3xbfgragh.onion/subscription?campaignId=37WXW}{Subscriptions}
\end{itemize}
