\href{/section/style}{Style}\textbar{}Reed Hastings Had Us All Staying
Home Before We Had To

\url{https://nyti.ms/2FaaPOs}

\begin{itemize}
\item
\item
\item
\item
\item
\end{itemize}

\includegraphics{https://static01.graylady3jvrrxbe.onion/images/2020/09/06/fashion/06WITH-REEDHASTINGS01/merlin_175628331_66d08e50-0b83-459b-93eb-76ba3f5aa2b8-articleLarge.jpg?quality=75\&auto=webp\&disable=upscale}

Sections

\protect\hyperlink{site-content}{Skip to
content}\protect\hyperlink{site-index}{Skip to site index}

With \ldots{}

\hypertarget{reed-hastings-had-us-all-staying-home-before-we-had-to}{%
\section{Reed Hastings Had Us All Staying Home Before We Had
To}\label{reed-hastings-had-us-all-staying-home-before-we-had-to}}

Netflix started with sending DVDs --- remember them? --- through the
mail, but now the streaming pioneer sits atop a Hollywood it has
thoroughly upended.

Reed Hastings, the head of Netflix.Credit...Cayce Clifford for The New
York Times

Supported by

\protect\hyperlink{after-sponsor}{Continue reading the main story}

\href{https://www.nytimes3xbfgragh.onion/by/maureen-dowd}{\includegraphics{https://static01.graylady3jvrrxbe.onion/images/2018/04/02/opinion/maureen-dowd/maureen-dowd-thumbLarge.png}}

By \href{https://www.nytimes3xbfgragh.onion/by/maureen-dowd}{Maureen
Dowd}

\begin{itemize}
\item
  Sept. 4, 2020
\item
  \begin{itemize}
  \item
  \item
  \item
  \item
  \item
  \end{itemize}
\end{itemize}

Does it feel good to be the man who killed Hollywood?

``No,'' said Reed Hastings, who nurtured Netflix into the Godzilla of
the entertainment world. ``But, of course, we haven't killed
Hollywood.''

At 59, the slender, gray-haired Mr. Hastings remains a mystery in the
industry he dominates. ``He's a complete cipher here,'' one Hollywood
macher said.

You won't find Mr. Hastings hanging with the stars at the
\href{https://www.nytimes3xbfgragh.onion/2019/02/23/style/san-vicente-bungalows-jeff-klein.html}{San
Vicente Bungalows}. He doesn't bellow at the pool at the Hotel du Cap or
swan around at premieres. He may show up in line at Sundance, but he's
not cutting the line.

He started a delivery system for movies, and now his company is one of
the most powerful forces in movies. In the capital of drama, Mr.
Hastings is, without drama, ripping out the infrastructure and replacing
it with his own.

Studio bosses are toppling, agents are scrambling, golden parachutes are
disappearing,
\href{https://www.nytimes3xbfgragh.onion/2020/08/04/business/media/disney-earnings-coronavirus.html}{Disney
is reeling}, Covid is wreaking havoc on theme parks and movie theaters
and \#MeToo is still reverberating.

Amid these tectonic plate shifts, Netflix has blotted out the sun.
Streaming, resisted for so long by the old clubby powers, is now
absolute king. R.I.P., Louis B. Mayer.

Ben Smith, the New York Times media columnist,
\href{https://www.nytimes3xbfgragh.onion/2020/08/16/business/media/hollywood-studios-firings-streaming.html}{wrote
an obit} recently for old Hollywood. And Janice Min, the former
co-president of The Hollywood Reporter, agrees that Netflix is ``winning
the pandemic,'' siphoning viewers from broadcast and cable.

``They were all asleep to it during the early ascendance of Netflix,"
\href{https://www.nytimes3xbfgragh.onion/2018/03/24/style/barry-diller-iac.html}{Barry
Diller} said of his fellow Hollywood moguls. ``Now they've woken up to
it, and it has slipped away from them and is never to be regained. They
lost hegemony over an entire industry.''

As Mr. Diller notes, businesspeople ordinarily gravitate to Hollywood
for status and glamour, but Mr. Hastings is that rarest of creatures
``who will never be seduced'' even though he is ``playing the game there
like a pitch-perfect violin virtuoso.''

So how did a self-described ``math wonk'' whose favorite pastimes are
walking and thinking, a man who trained for a time with the Marine Corps
before switching to the Peace Corps, teaching math in Swaziland, render
old Hollywood irrelevant?

Mr. Hastings said that his mother was a Boston debutante from a Social
Register family who married a lawyer who later worked in the Nixon
administration. She was repulsed by the world of high society and taught
her children to disdain it. So young Reed grew up thinking that it was a
good thing to distance yourself from elites and avoid pretensions.

The new overlord of the land of artifice and playacting hates artifice
and playacting.

``Probably it all comes down to, you know, your mother or your father,''
he murmured.

The height of his flashiness was posing on a Porsche in 1995 on the
cover of USA Today, when he was a tech executive. He said he put aside
that kind of ``superfun'' immaturity and sold the Porsche in favor of a
Toyota Avalon. (Now he drives a Tesla.)

\includegraphics{https://static01.graylady3jvrrxbe.onion/images/2020/09/06/fashion/06WITH-REEDHASTINGS-ted/merlin_174650643_74553e96-404d-4668-b0f9-300c97a43fe1-articleLarge.jpg?quality=75\&auto=webp\&disable=upscale}

But for all the low-key charm, there's no doubt that Mr. Hastings ---
along with his more wheeling-dealing Hollywood-based partner, Ted
Sarandos --- is running the show.

``The heart and soul of our content,'' is how Mr. Hastings describes Mr.
Sarandos, who grew up glued to the TV and dropped out of community
college in Arizona to work in a video store. Mr. Hastings, who recently
moved over to share his C.E.O. role with Mr. Sarandos, describes their
partnership as ``a positive, low-ego thing.''

Ms. Min notes that ``there are all sorts of ways people have tried to
hate the company,'' for not getting their calls returned or not being
able to schmooze their way into a big production deal with a friend or
not getting cushy back-end deals. People whisper about the Netflix
culture being arrogant and cultlike, a culture of fear.

``But now,'' Ms. Min said, ``they're too big to hate.''

\hypertarget{the-stream-that-became-a-flood}{%
\subsection{The Stream That Became a
Flood}\label{the-stream-that-became-a-flood}}

Netflix is like the British Empire at its height, expanding across the
globe. Indeed, in addition to all the royals in ``The Crown,'' Netflix
now has its very own prince. The company this week
\href{https://www.nytimes3xbfgragh.onion/2020/09/02/business/media/harry-meghan-netflix.html}{signed}
Harry and Meghan to a multiyear deal.

They join the Obamas; Ryan Murphy; Shonda Rhimes; Kenya Barris; Ava
DuVernay, who is teaming up with Colin Kaepernick for a Netflix series;
and the erstwhile lords of HBO, the ``Game of Thrones'' showrunners
David Benioff and D.B. Weiss, who are adapting a Chinese sci-fi epic by
Liu Cixin called ``The Three-Body Problem,'' about humanity's first
contact with alien civilization.

After a long period when the club of mostly white, supposedly liberal
men running Hollywood secured the power in a lockbox, keeping a death
grip on the Academy of Motion Picture Arts and Sciences and acting
shocked anew every time a movie with Asian or Black or female leads did
great box office, Netflix is swiftly democratizing things.

Image

Claire Foy and Matt Smith in a scene from ``The Crown.''Credit...Robert
Viglasky/Netflix, via Associated Press

Its offerings include a show about a Japanese underwear store, a Belgian
crime drama, a Spanish period piece about phone operators, a Portuguese
bull-riding show. Netflix has also
\href{https://www.nytimes3xbfgragh.onion/2020/07/05/business/media/netflix-hollywood-black-culture.html}{invested
heavily in Black programming}.

But operating a global empire is not without its hazards. Mr. Hastings
took heat last year for
\href{https://www.nytimes3xbfgragh.onion/2019/01/06/business/media/netflix-saudi-arabia-censorship-hasan-minhaj.html}{bowing
to Saudi censors} and pulling an episode of the comedy show ``Patriot
Act,'' starring Hasan Minhaj, which was critical of Crown Prince
Mohammed bin Salman. Challenged, the Netflix chief spurred more
criticism when he said, ``We're not trying to do truth to power. We're
trying to entertain.''

He told me that he used ``an awkward phrase'' and that the company
sometimes has to make ``hard choices'' and compromises where ``it
definitely gets squirmy and makes us feel unsettled.'' But he said
Netflix kept that episode on YouTube and that ``Queer Eye'' is available
in Saudi Arabia, so ``real positive stuff comes out of that.''

When I asked him where Hollywood will be in 15 years, Mr. Hastings said:
``I see producing stories and sharing them as bigger than ever. But
those stories will be produced in Atlanta, in Vancouver, in London, all
over the world as opposed to strictly in Hollywood.''

Could the new Hollywood, which often feels ruled by algorithms, not
capricious tastemakers, ever create a star like Grace Kelly?

Yes, he replied, but she would need a social media component in addition
to being a performer.

I told Mr. Hastings that, while some may be weirded out by the Netflix
algorithm that figures out what you want to watch next, I love it.

I simply type in ``betrayal,'' ``revenge,'' ``lives ruined,'' and it
brings up everything I want to see. He said his taste runs to
independent films, ``dark, difficult things.''

Mr. Hastings, who was, he said, ``a pretty average kid with no
particular talent,'' has a master's degree in computer science from
Stanford.

He founded a software company, Pure Software, before pioneering DVDs by
mail with Marc Randolph. (There is a split about the company's origin
story, with Mr. Randolph saying the two founders came up with the idea
while driving, and Mr. Hastings saying it was a light-bulb moment after
he had to pay a \$40 late fee on a VHS rental.)

In our interview, Mr. Hastings was uncommonly self-effacing for a
billionaire.

He told me that
\href{https://www.nytimes3xbfgragh.onion/2020/07/25/style/elon-musk-maureen-dowd.html}{Elon
Musk} is ``100 times more interesting a person'' than he is. ``I'll,
like, do the basic core, traditional stuff very well,'' Mr. Hastings
said. ``And he is a maverick in every dimension. He's just, like,
amazing.''

Mr. Hastings noted: ``I'll never be Steve Jobs, the creative, brilliant
person.'' And he praised
\href{https://www.nytimes3xbfgragh.onion/2019/09/22/style/disney-bob-iger-book.html}{the
Disney chairman of the board}. ``I'm an Iger wannabe. He's such a
statesman.''

I told Mr. Hastings that, given all the poaching that the big-spending
Netflix does, I'm surprised that some Disney executive hasn't thrown a
drink in his face at a chichi restaurant, ``Appointment in Samarra''
style.

``Sounds like a good storytelling device,'' he said dryly, though he
conceded that Disney bosses do get mad when he steals executives and
talent.

\hypertarget{cafeteria-man}{%
\subsection{Cafeteria Man}\label{cafeteria-man}}

For our Zoom interview, the Netflix mogul looked comfy in a checked
shirt, khakis and bare feet in his ``Covid hide-out'': his son's old
bedroom, in the house in Santa Cruz, Calif., he shares with Patty
Quillin, his wife of 29 years.

``It was great sport making fun of this bedroom on our earnings call
four months ago,'' he said, smiling. ``I don't want to really set up a
home office because I want to believe that the pandemic is going to end
soon. So, month by month, I stay here without fixing it up out of kind
of stubborn hope.''

Because he believes ``any locked area is symbolic of hidden things,'' he
does not have an office or even a cubicle with drawers that close, even
at his headquarters. He writes that he might grab a conference room if
he needs it but prefers walking meetings.

``He makes his own cappuccino at machines, and we have no private dining
rooms in our Hollywood office,'' said a Netflix colleague. ``He and Ted
get food in the cafeteria like everyone else.''

Has the pandemic altered Mr. Hastings's perception of the competition?

It's the ``sideways threats'' that bite companies, he said. ``If you
think of Kodak and Fuji, competing in film for 100 years, but then
ultimately it turns out to be Instagram.''

Speaking of which, I wondered if he thinks that Mark Zuckerberg, Sheryl
Sandberg and Jack Dorsey have done enough as far as election meddling
and disinformation threats?

``Every new technology has real issues that have to be thought through
and, you know, we're in that phase for social media,'' he said, adding:
``The car, many people think is a great invention for human freedom, but
it also has killed a lot of people over time. Film got used by Hitler
for terrible purposes.''

He continued: ``So I find Mark and Sheryl to be sincere in trying to
think these things through.''

In 2016, he was vocal about his fear that Donald Trump ``would destroy
much of what is great about America,''
\href{https://www.nytimes3xbfgragh.onion/2017/08/08/technology/the-culture-wars-have-come-to-silicon-valley.html}{even
telling} one of Facebook's original investors,
\href{https://www.nytimes3xbfgragh.onion/2017/01/11/fashion/peter-thiel-donald-trump-silicon-valley-technology-gawker.html}{Peter
Thiel}, that he had to give him a negative evaluation of his performance
on the Facebook board because of his ``bad judgment'' after Mr. Thiel
spoke at the Republican convention.

After the Muslim ban in 2017, Mr. Hastings
\href{https://www.facebookcorewwwi.onion/reed1960/posts/10154654737174584}{called}
President Trump's actions ``un-American'' in a post on Facebook.

Image

Devon Terrell as Barack Obama in a scene from "Barry."Credit...Linda
Kallerus/Netflix, via Associated Press

He thinks if Mr. Trump wins re-election, ``it would not be good but I'm
not worried that it's the end of America. I mean, America is
super-resilient, and I feel great about our civic institutions, whether
that's the military or the Civil Service. It won't be as traumatic as
the Civil War or the Great Depression.''

He is supporting Joe Biden, but he is not as outspoken as he was last
time and didn't watch either convention.

``You know, C.E.O. announcements about politics don't carry much weight
with most people,'' he said.

I asked if he would ever give Mr. Trump a Netflix deal like the Obamas.

``I haven't thought about that,'' he said, noting that he doesn't try to
tailor the company to his own political views.

\hypertarget{farm-for-dissent}{%
\subsection{`Farm for Dissent'}\label{farm-for-dissent}}

The Netflix psyche is dissected in Mr. Hastings's new book, written with
Erin Meyer, ``No Rules Rules: Netflix and the Culture of Reinvention.''

The book was born from the Netflix Culture Deck, a famous --- and
infamous --- show of 127 slides that Mr. Hastings put online in 2009. It
was hailed, in a 2013 GQ article, as possibly ``the most important
document ever to come out of Silicon Valley'' by Ms. Sandberg. (Mr.
Hastings was on the board of Facebook at the time.)

Even Ms. Meyer, a business professor, loathed some of the tenets at
first and compared the company culture to the Hunger Games. But Mr.
Hastings believes it was essential to his revolution.

Netflix pays top dollar and wants what it calls High Talent Density,
which means only stars, no average people. Some of the rules of the
Freedom and Responsibility workplace sound rigid.

``Adequate performance gets a generous severance,'' one rule says.

Managers use The Keeper Test to figure out which employees are merely
average and to weed out complainers and pessimists. How hard would you
fight to keep someone? If the answer is ``not that hard,'' that employee
should go. As one former executive frets in the book, they are more like
penguins, who abandon those in the group that are weak or struggling,
than elephants, who nurture the weak back to life.

Employees are also encouraged to use The Keeper Test Prompt, to ask
bosses if they would fight hard to keep them.

Maxing Up Candor, getting rid of the ``normal polite human protocols,''
is a part of daily life at Netflix with a daily Circle of Feedback and
annual written and live 360 Assessments, in which you meet with the team
to get ripped apart.

Mr. Hastings, who grew up in a house where emotions were never
discussed, said he got the idea for more transparency after going to
marriage counseling.

By making things less hierarchical, Mr. Hastings believes the company
can be more nimble.

Employees are encouraged to critique those above and below them at any
time. (This does not seem to apply to top talent, like Shonda Rhimes or
Ryan Murphy.) Staff members must Farm for Dissent and Socialize new
ideas. Failures should be Sunshined, talked about openly and frequently.

Mr. Hastings does not think of his employees as family, but as a sports
team --- and one that has to win trophies.

``For people who value job security over winning championships, Netflix
is not the right choice, and we try to be clear and non-judgmental about
that,'' he writes.

Mr. Hastings writes of his managers: ``To feel good about cutting
someone they like and respect requires them to desire to help the
organization and to recognize that everyone at Netflix is happier and
more successful when there is a star in every position.''

Holy Ayn Rand!

Mr. Hastings even demoted Mr. Randolph, who
\href{https://www.nytimes3xbfgragh.onion/2019/09/15/business/media/netflix-chief-executive-reed-hastings-marc-randolph.html}{described
his own reaction} to his co-founder's radical candor: ``There is no way
I'm sitting here while you pitch me on why I suck.''

And Mr. Hastings canned one of his best friends and original employees,
Patty McCord, who helped create the Culture Deck and who drove to work
with him, from her H.R. job.

``It's not easy, just like you said,'' he acknowledged. ``There's a
conflict between the head and the heart.'' He added that sometimes you
just have to tell someone ``you're not as engaged, or we needed someone
who's got these additional skill sets as we grow and face new
challenges.'' He said it's ``very much a joint conversation'' and ``it's
not like `The Apprentice' or something.''

He writes in the book: ``We all stay friends and there is no shame.''

One fired Netflix executive told me, ``When Reed views somebody's
contribution as less than the problems they're causing or potential
risk, he gets rid of them. He's an extraordinary guy, but he's coldly
rational and calculating. But the trade-off is, you get to go on this
amazing fun ride, make a lot of dough, and when your number's up, your
number's up.''

Image

Mr. Hastings outside of the Netflix headquarters in Los Gatos,
Calif.~``They were all asleep to it during the early ascendance of
Netflix,'' Barry Diller said of his fellow Hollywood moguls. ``Now
they've woken up to it, and it has slipped away from them and is never
to be regained.''Credit...Cayce Clifford for The New York Times

Ms. Meyer initially wondered whether Netflix's culture represented bad
management --- ``hypermasculine, excessively confrontational and
downright aggressive'' --- and whether it was ``ethical to fire
hard-working employees who don't manage to do extraordinary work.''

How could people feel safe to ``dream, speak up and take risks'' if they
were being injected with fear daily?

But she concludes in the book that Netflix's ``incredible'' success is
hard to argue with, and employee surveys show a high degree of
satisfaction. She said she did not discover the back-stabbing she
expected.

Mr. Hastings writes that all the rules apply to him: ``I tell my bosses,
the board of directors, that I should be treated no differently. They
shouldn't have to wait for me to fail to replace me.''

He adds: ``I find it motivating that I have to play for my position
every quarter, and I try to keep improving myself to stay ahead.''

But, I pressed, the board wouldn't really dismiss him, right? With a
cascade of tears and apologies, he survived the Qwikster debacle --- a
separate company he created in 2011 to handle the DVD market --- after
the Netflix stock dropped more than 75 percent and ``everything we'd
built was crashing down.''

``They really would do it,'' he said of the board, ``if there was a
better leader.'' But he conceded, ``I guess it's unproven, so I'm sure
it doesn't generate a lot of credibility.''

The book describes the problems of imposing ``the Netflix Way'' on other
cultures, especially in Asia and Brazil, where it can be considered rude
or debilitating. (The Dutch seemed fine; they're even more blunt than
Americans.) But Mr. Hastings does not give up. He simply doubles down:
``With less direct cultures, increase formal feedback moments,''
including feedback clinics.

``A high sharing environment,'' as Mr. Hastings calls it, is my idea of
hell. That's why I'm not on Facebook.

I broke the news to Mr. Hastings that I could never work at Netflix
because I am extremely sensitive to criticism. (I know that is ironic,
given my job.) I like to complain and be pessimistic.

``There are a few probably, like you, who don't like the criticism,''
Mr. Hastings said, noting that Netflix is not a good fit for everyone.

With trepidation, I asked Mr. Hastings how I would fare if he gave me
The Keepers Test based on our interview.

``Would you fire me right now?'' I asked.

Mr. Hastings decided to be diplomatic. ``I look forward to having a redo
sometime when we're in person,'' he said, ``which I'm sure is just
richer in every way.''

{[}\emph{How about a Confirm or Deny binge?}{]}

\textbf{Maureen Dowd: Your favorite movie on Netflix is the erotic flick
``365 Days.''}

Reed Hastings: Let's say it's more stimulating than most people realize.

\textbf{You still haven't figured out if you're subscribed to HBO Go or
HBO Max.}

Confirm.

\textbf{You have never felt the need to Netflix and chill.}

Deny.

\textbf{Jeff Bezos is going through a midlife crisis.}

No comment.

\textbf{You hated ``Roma.''}

False. ``Roma'' is incredible.

\textbf{Helen Mirren, who last year told a convention for theater owners
what Netflix}
\textbf{\href{https://www.hollywoodreporter.com/news/helen-mirren-says-f-netflix-thanks-theater-owners-at-cinemacon-1199086\#:~:text=Helen\%20Mirren\%20on\%20Tuesday\%20gave,owners\%20gathered\%20in\%20Las\%20Vegas.}{could
do with itself}, is on your Dead to Me list.}

No. Everyone is traditionally against us.

\textbf{\href{https://www.nytimes3xbfgragh.onion/2019/09/22/style/disney-bob-iger-book.html}{Bob
Iger}} \textbf{should have bought Twitter instead of Fox.}

That's a very playful and interesting one. I'd say false. Remember in
Michael Eisner's days, they bought Go.com, and then it was just too
different and they killed it. Twitter, you've got all that
user-generated content, all that controversy. So I think Iger made the
right set of decisions to go big and buy Fox.

\textbf{You send John Malone and Greg Maffei a thank-you note every year
on the anniversary of the Starz deal.}

I would say that's not literally true.

\textbf{The person you never got involved in Netflix that you wish you
had is John Malone.}

Yeah. He's close to Bill Gates in terms of who I admire.

\textbf{As a kid, when your father worked in the Nixon administration,
you spent a weekend at Camp David and saw Nixon's gold-colored toilet
seat.}

Confirm.

\textbf{In 2010, when he was C.E.O. of Time Warner, Jeff Bewkes scoffed
at the idea of Netflix taking over Hollywood, saying, ``Is the Albanian
Army going to take over the world?'' So now, every two weeks, you text
Bewkes, ``How do you like them apples?''}

Well, I'll firmly deny. He is a great and thoughtful guy.

\textbf{But you do have a tattoo of the Albanian Army logo on your
back.}

I've got my Albanian Army dog tags.

\textbf{The Netflix lobby is the new MGM canteen.}

Confirm.

\textbf{TikTok is your toughest competitor.}

Deny.

\textbf{You sold vacuum cleaners door to door and served coffee at a
computer company in Boston.}

Confirm.

\textbf{Executives at media companies make too much money.}

Confirm.

Advertisement

\protect\hyperlink{after-bottom}{Continue reading the main story}

\hypertarget{site-index}{%
\subsection{Site Index}\label{site-index}}

\hypertarget{site-information-navigation}{%
\subsection{Site Information
Navigation}\label{site-information-navigation}}

\begin{itemize}
\tightlist
\item
  \href{https://help.nytimes3xbfgragh.onion/hc/en-us/articles/115014792127-Copyright-notice}{©~2020~The
  New York Times Company}
\end{itemize}

\begin{itemize}
\tightlist
\item
  \href{https://www.nytco.com/}{NYTCo}
\item
  \href{https://help.nytimes3xbfgragh.onion/hc/en-us/articles/115015385887-Contact-Us}{Contact
  Us}
\item
  \href{https://www.nytco.com/careers/}{Work with us}
\item
  \href{https://nytmediakit.com/}{Advertise}
\item
  \href{http://www.tbrandstudio.com/}{T Brand Studio}
\item
  \href{https://www.nytimes3xbfgragh.onion/privacy/cookie-policy\#how-do-i-manage-trackers}{Your
  Ad Choices}
\item
  \href{https://www.nytimes3xbfgragh.onion/privacy}{Privacy}
\item
  \href{https://help.nytimes3xbfgragh.onion/hc/en-us/articles/115014893428-Terms-of-service}{Terms
  of Service}
\item
  \href{https://help.nytimes3xbfgragh.onion/hc/en-us/articles/115014893968-Terms-of-sale}{Terms
  of Sale}
\item
  \href{https://spiderbites.nytimes3xbfgragh.onion}{Site Map}
\item
  \href{https://help.nytimes3xbfgragh.onion/hc/en-us}{Help}
\item
  \href{https://www.nytimes3xbfgragh.onion/subscription?campaignId=37WXW}{Subscriptions}
\end{itemize}
