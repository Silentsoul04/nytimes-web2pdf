\href{/section/technology}{Technology}\textbar{}Forget TikTok. China's
Powerhouse App Is WeChat, and Its Power Is Sweeping.

\url{https://nyti.ms/3lSDHLZ}

\begin{itemize}
\item
\item
\item
\item
\item
\end{itemize}

\includegraphics{https://static01.graylady3jvrrxbe.onion/images/2020/09/06/business/04Wechat-illo/04Wechat-illo-articleLarge.jpg?quality=75\&auto=webp\&disable=upscale}

Sections

\protect\hyperlink{site-content}{Skip to
content}\protect\hyperlink{site-index}{Skip to site index}

\hypertarget{forget-tiktok-chinas-powerhouse-app-is-wechat-and-its-power-is-sweeping}{%
\section{Forget TikTok. China's Powerhouse App Is WeChat, and Its Power
Is
Sweeping.}\label{forget-tiktok-chinas-powerhouse-app-is-wechat-and-its-power-is-sweeping}}

A vital connection for the Chinese diaspora, the app has also become a
global conduit of Chinese state propaganda, surveillance and
intimidation. The United States has proposed banning it.

Credit...Delcan \& Company

Supported by

\protect\hyperlink{after-sponsor}{Continue reading the main story}

\href{https://www.nytimes3xbfgragh.onion/by/paul-mozur}{\includegraphics{https://static01.graylady3jvrrxbe.onion/images/2018/10/15/multimedia/author-paul-mozur/author-paul-mozur-thumbLarge.png}}

By \href{https://www.nytimes3xbfgragh.onion/by/paul-mozur}{Paul Mozur}

\begin{itemize}
\item
  Sept. 4, 2020
\item
  \begin{itemize}
  \item
  \item
  \item
  \item
  \item
  \end{itemize}
\end{itemize}

\href{https://cn.nytimes3xbfgragh.onion/technology/20200907/wechat-china-united-states/}{阅读简体中文版}\href{https://cn.nytimes3xbfgragh.onion/technology/20200907/wechat-china-united-states/zh-hant/}{閱讀繁體中文版}

Just after the 2016 presidential election in the United States, Joanne
Li realized the app that connected her to fellow Chinese immigrants had
disconnected her from reality.

Everything she saw on the Chinese app, WeChat, indicated Donald J. Trump
was an admired leader and impressive businessman. She believed it was
the unquestioned consensus on the newly elected American president.
``But then I started talking to some foreigners about him,
non-Chinese,'' she said. ``I was totally confused.''

She began to read more widely, and Ms. Li, who lived in Toronto at the
time, increasingly found WeChat filled with gossip, conspiracy theories
and outright lies. One article claimed Prime Minister Justin Trudeau of
Canada
\href{https://thetyee.ca/News/2020/08/26/Conservative-WeChat-Ad-Trudeau/}{planned
to legalize hard drugs}. Another rumor purported that Canada had begun
selling marijuana in grocery stores. A post from a news account in
Shanghai warned Chinese people to take care lest they accidentally bring
the drug back from Canada and get arrested.

She also questioned what was being said about China. When a top Huawei
executive
\href{https://www.nytimes3xbfgragh.onion/2018/12/05/business/huawei-cfo-arrest-canada-extradition.html}{was
arrested in Canada} in 2018, articles from foreign news media were
quickly censored on WeChat. Her Chinese friends both inside and outside
China began
\href{https://www.nytimes3xbfgragh.onion/2018/12/07/world/asia/huawei-arrest-china.html}{to
say that Canada had no justice}, which contradicted her own experience.
``All of a sudden I discovered talking to others about the issue didn't
make sense,'' Ms. Li said. ``It felt like if I only watched Chinese
media, all of my thoughts would be different.''

Ms. Li had little choice but to take the bad with the good. Built to be
everything for everyone, WeChat
\href{https://www.nytimes3xbfgragh.onion/video/technology/100000004574648/china-internet-wechat.html}{is
indispensable}.

For most Chinese people in China, WeChat is a sort of
\href{https://www.nytimes3xbfgragh.onion/2017/07/16/business/china-cash-smartphone-payments.html}{all-in-one
app}: a way to swap stories, talk to old classmates, pay bills,
coordinate with co-workers, post envy-inducing vacation photos, buy
stuff and get news. For the millions of members of China's diaspora, it
is the bridge that links them to the trappings of home, from family
chatter to food photos.

Woven through it all is the
\href{https://www.nytimes3xbfgragh.onion/2018/07/08/business/china-surveillance-technology.html}{ever
more muscular surveillance} and propaganda of the Chinese Communist
Party. As WeChat has become ubiquitous, it has become a
\href{https://www.nytimes3xbfgragh.onion/2020/03/16/business/china-coronavirus-internet-police.html}{powerful
tool of social control}, a way for
\href{https://www.nytimes3xbfgragh.onion/2019/12/17/technology/china-surveillance.html}{Chinese
authorities to guide and police} what people say, whom they talk to and
what they read.

It has even extended Beijing's reach beyond its borders. When secret
police
\href{https://www.nytimes3xbfgragh.onion/2019/08/15/podcasts/the-daily/china-xinjiang-uighur-detention.html}{issue
threats abroad}, they often do so on WeChat. When
\href{https://www.nytimes3xbfgragh.onion/2020/07/22/world/asia/us-china-houston-consulate.html}{military
researchers} working undercover in the United States needed to talk to
China's embassies, they used WeChat, according to court documents. The
party coordinates via WeChat with members studying overseas.

As a cornerstone of China's surveillance state, WeChat is
\href{https://www.nytimes3xbfgragh.onion/2020/08/07/business/trump-china-wechat-tiktok.html}{now
considered} a national security threat in the United States. The Trump
administration has proposed
\href{https://www.nytimes3xbfgragh.onion/2020/08/06/technology/trump-wechat-tiktok-china.html}{banning
WeChat outright}, along with the Chinese short video app TikTok.
Overnight, two of China's biggest internet innovations became a new
front
\href{https://www.nytimes3xbfgragh.onion/2020/08/17/technology/trump-tiktok-wechat-ban.html}{in
the sprawling tech standoff} between China and the United States.

While the two apps are lumped in the same category by the Trump
administration, they represent
\href{https://www.nytimes3xbfgragh.onion/2016/08/10/technology/china-homegrown-internet-companies-rest-of-the-world.html}{two
distinct approaches} to the Great Firewall that blocks Chinese access to
foreign websites.

The hipper, better-known TikTok was designed for the wild world outside
of China's cloistering censorship; it exists only beyond China's
borders. By hiving off an independent app to win over global users,
TikTok's owner, ByteDance, created
\href{https://www.nytimes3xbfgragh.onion/2020/08/03/technology/tiktok-bytedance-us-china.html}{the
best bet any Chinese start-up has had} to compete with the internet
giants in the West. The separation of TikTok from its cousin apps in
China, along with deep popularity, has fed corporate campaigns in the
United States to save it, even as Beijing potentially upended any deals
by
\href{https://www.nytimes3xbfgragh.onion/2020/08/29/technology/china-tiktok-export-controls.html}{labeling
its core technology} a national security priority.

Though WeChat has different rules for users inside and outside of China,
it remains a single, unified social network spanning China's Great
Firewall. In that sense, it has helped bring
\href{https://www.nytimes3xbfgragh.onion/2018/03/02/technology/china-technology-censorship-borders-expansion.html}{Chinese
censorship to the world}. A ban would cut dead millions of conversations
between family and friends, a reason one group has filed a lawsuit to
block the Trump administration's efforts. It would also be an easy
victory for American policymakers seeking to push back against China's
techno-authoritarian overreach.

\includegraphics{https://static01.graylady3jvrrxbe.onion/images/2020/09/06/business/00Wechat-li/merlin_176110002_3e5ad239-3533-4295-bedf-4623663982d8-articleLarge.jpg?quality=75\&auto=webp\&disable=upscale}

Ms. Li felt the whipcrack of China's internet controls firsthand when
she returned to China in 2018 to take a real estate job. After her
experience overseas, she sought to balance her news diet with groups
that shared articles on world events. As the coronavirus spread in early
2020 and China's relations with countries around the world strained, she
posted an article on WeChat from the U.S. government-run Radio Free Asia
about the deterioration of Chinese-Canadian diplomacy, a piece that
would have been censored.

The next day, four police officers showed up at her family's apartment.
They carried guns and riot shields.

``My mother was terrified,'' she said. ``She turned white when she saw
them.''

The police officers took Ms. Li, along with her phone and computer, to
the local police station. She said they manacled her legs to a
restraining device known as a tiger chair for questioning. They asked
repeatedly about the article and her WeChat contacts overseas before
locking her in a barred cell for the night.

Twice she was released, only to be dragged back to the station for fresh
interrogation sessions. Ms. Li said an officer even insisted China had
freedom of speech protections as he questioned her over what she had
said online. ``I didn't say anything,'' she said. ``I just thought, what
is your freedom of speech? Is it the freedom to drag me down to the
police station and keep me night after sleepless night interrogating
me?''

Finally, the police forced her to write out a confession and vow of
support for China, then let her go.

\hypertarget{the-walls-are-getting-higher}{%
\subsection{`The walls are getting
higher'}\label{the-walls-are-getting-higher}}

WeChat started out as a simple copycat. Its parent, the Chinese internet
giant Tencent, had built an enormous user base on a chat app designed
for personal computers. But a new generation of mobile chat apps
threatened to upset its hold over the way young Chinese talked to one
another.

The visionary Tencent engineer Allen Zhang fired off a message to the
company founder, Pony Ma, concerned that they weren't keeping up. The
missive led to a new mandate, and Mr. Zhang fashioned a digital Swiss
Army knife that became a necessity for daily life in China. WeChat
piggybacked on the popularity of the other online platforms run by
Tencent, combining payments, e-commerce and social media into a single
service.

It became a hit, eventually eclipsing the apps that inspired WeChat. And
Tencent, which made billions in profits from the online games piped into
its disparate platforms, now had a way to make money off nearly every
aspect of a person's digital identity --- by serving ads, selling stuff,
processing payments and facilitating services like food delivery.

Image

The Beijing offices of Tencent, the parent company of WeChat.Credit...Wu
Hong/EPA

The tech world inside and outside of China marveled. Tencent rival
Alibaba scrambled to come up with its own product to compete. Silicon
Valley
\href{https://www.nytimes3xbfgragh.onion/2016/08/03/technology/china-mobile-tech-innovation-silicon-valley.html}{studied
the ways} it mixed services and
\href{https://www.nytimes3xbfgragh.onion/2019/03/07/technology/facebook-zuckerberg-wechat.html}{followed
its cues}.

Built for China's closed world of internet services, WeChat's only
failure came outside the Great Firewall. Tencent made a big marketing
push overseas, even hiring the soccer player Lionel Messi as a spokesman
in some markets. For non-China users, it created a separate set of
rules. International accounts would not face direct censorship and data
would be stored on servers overseas.

But WeChat didn't have the same appeal without the many services
available only in China. It looked more prosaic outside the country,
like any other chat app. The main overseas users, in the end, would be
the Chinese diaspora.

Tencent did not respond to a request for comment.

Over time, the distinctions between the Chinese and international app
have mattered less. Chinese people who create accounts within China, but
then leave, carry with them a censored and monitored account. If
international users chat with users inside China, their posts can be
censored.

For news and gossip, most comes from WeChat users inside China and
spreads out to the world. Whereas most social networks have myriad
filter bubbles that reinforce different biases, WeChat is dominated by
one super-filter bubble, and it hews closely to the official propaganda
narratives.

``The filter bubbles on WeChat have nothing to do with algorithms ---
they come from China's closed internet ecosystem and censorship. That
makes them worse than other social media,'' said Fang Kecheng, a
professor in the School of Journalism and Communications at the Chinese
University of Hong Kong.

Mr. Fang first noticed the limitations of WeChat in 2018 as a graduate
student at the University of Pennsylvania, teaching an online course in
media literacy to younger Chinese.

Soft-spoken and steeped in the media echo chambers of the United States
and China, Mr. Fang expected to reach mostly curious Chinese inside
China. An unexpected group dialed into the classes: Chinese immigrants
and expatriates living in the United States, Canada and elsewhere.

``It seemed obvious. Because they were all outside China, it should be
easy for them to gain an understanding of foreign media. In their
day-to-day life they would see it and read it,'' Mr. Fang said. ``I
realized it wasn't the case. They were outside of China, but their media
environment was still entirely inside China, their channel for
information was all from public accounts on WeChat.''

Mr. Fang's six-week online courses were inspired by a WeChat account he
ran called News Lab that sought to teach readers about journalism. With
his courses, he assigned articles from media like Reuters along with
work sheets that taught students to analyze the pieces --- pushing them
to draw distinctions between pundit commentary and primary sourcing.

During one course in 2019, he focused on the fire at Notre-Dame
cathedral in Paris, which inspired many conspiracy theories on WeChat.
One professor at the prestigious Tsinghua University reposted an article
alleging that Muslims were behind the fire, which was untrue.

The classes were a big draw. In 2018, Mr. Fang attracted 500 students.
The next year he got 1,300. In 2020, a year of coronavirus rumors and
censorship, Tencent took down his News Lab account. He decided it was
not safe to teach the class on another platform given the more
``hostile'' climate toward foreign media.

Still, he said that blocking WeChat would be unlikely to help much, as
users could easily switch to other Chinese apps filled with propaganda
and rumors. A better idea would be to create rules that force social
media companies like Tencent to be more transparent, he said.

Creating such internet blocks, he said, rarely improved the quality of
information.

``Information is like water. Water quality can be improved, but without
any flow, water easily grows fetid,'' he said.

In a class in 2019, he warned broadly about barriers to information
flow.

``Now, the walls are getting higher and higher. The ability to see the
outside has become ever harder,'' he said. ``Not just in China, but in
much of the world.''

\hypertarget{what-its-like-to-lose-contact}{%
\subsection{`What it's like to lose
contact'}\label{what-its-like-to-lose-contact}}

When Ferkat Jawdat's mother disappeared into China's sprawling system of
\href{https://www.nytimes3xbfgragh.onion/2018/09/08/world/asia/china-uighur-muslim-detention-camp.html}{re-education
camps} to indoctrinate Uighurs, his WeChat became a kind of memorial.

The app might have been used as evidence against her. But he, like many
Uighurs, found himself opening WeChat again and again. It contained
years of photos and conversations with his mother. It also held a remote
hope he clung to, that one day she would again reach out.

When against all odds she did,
the\href{https://www.nytimes3xbfgragh.onion/interactive/2019/04/04/world/asia/xinjiang-china-surveillance-prison.html}{secret
police followed}.

If propaganda and censorship have found their way to WeChat users
overseas, so too has China's government.

For ethnic minority Uighurs,
\href{https://www.nytimes3xbfgragh.onion/interactive/2019/11/16/world/asia/china-xinjiang-documents.html}{who
have been targeted} by
\href{https://www.nytimes3xbfgragh.onion/2019/05/22/world/asia/china-surveillance-xinjiang.html}{draconian
digital controls} at home in China, the chat app has become a conduit
for threats from Chinese security forces. In court documents, the
Federal Bureau of Investigation said China's embassies communicated on
WeChat with military researchers who had entered the United States to
steal scientific research. The Chinese Communist Party has used it to
\href{http://news.upc.edu.cn/info/1436/56987.htm}{keep up ties} and
organize overseas members, including
\href{http://zhibu.univs.cn/front/article/show/1/dfc97ce2d56d11e79b5e5254004dfc45}{foreign-exchange
students}.

Not all uses are nefarious. During the pandemic, local governments
\href{http://chuxin.people.cn/n1/2020/0401/c428144-31657155.html}{used
the app} to update residents
\href{http://www.zgqt.zj.cn/6904967.html}{traveling and living} abroad
about the virus. China's embassies use it to issue travel warnings.

While the Chinese government could use any chat app, WeChat has
advantages. Police know well its surveillance capabilities. Within China
most accounts are linked to the real identity of users.

Mr. Jawdat's mother, sick and worn, was released from the camps in the
summer of 2019. Chinese police gave her a phone and signed her into
WeChat. At the sound of his mother's voice Mr. Jawdat fought back a
flood of emotions. He hadn't been sure if she was even alive. Despite
the relief, he noticed something was off. She offered stilted words of
praise for the Chinese Communist Party.

Then the police reached out to him. They approached him with an
anonymous friend request over WeChat. When he accepted, a man introduced
himself as a high-ranking officer in China's security forces in the
Xinjiang region, the epicenter of re-education camps. The man had a
proposal. If Mr. Jawdat, an American citizen and Uighur activist, would
quiet his attempts to raise awareness about the camps, then his mother
might be given a passport and allowed to join her family in the United
States.

``It was a kind of threat,'' he said. ``I stayed quiet for two or three
weeks, just to see what he did.''

It all came to nothing. After turning down a media interview and
skipping a speaking event, Mr. Jawdat grew impatient and confronted the
man. ``He started threatening me, saying, `You're only one person going
against the superpower. Compared to China, you are nothing.'''

The experience gave Mr. Jawdat little tolerance for the app that made
the threats possible, even if it had been his only line to his mother.
He said he knew two other Uighur Americans who had similar experiences.
Accounts from others point to similar occurrences around the world.

``I don't know if it's karma or justice served, for the Chinese people
to also feel the pain of what it's like to lose contact with your family
members,'' Mr. Jawdat said of the proposed ban by the Trump
administration. ``There are many Chinese officials who have their kids
in the U.S. WeChat must be one of the tools they use to keep in contact.
If they feel this pain, maybe they can relate better to the Uighurs.''

Image

A WeChat ban would cut dead millions of conversations. It would also be
an easy victory for American policymakers seeking to push back against
China's techno-authoritarian overreach.Credit...Jacquelyn
Martin/Associated Press

\hypertarget{then-you-are-alone}{%
\subsection{`Then you are alone'}\label{then-you-are-alone}}

Ms. Li was late to the WeChat party. Away in Toronto when it exploded in
popularity, she joined only in 2013, after her sister's repeated urging.

It opened up a new world for her. Not in China, but in Canada.

She found people nearby similar to her. Many of her Chinese friends were
on it. They found restaurants nearly as good as those at home and
explored the city together. One public account set up by a Chinese
immigrant organized activities. It kindled more than a few romances.
``It was incredibly fun to be on WeChat,'' she recalled.

Now the app reminds her of jail. During questioning, police told her
that a surveillance system, which they called Skynet, flagged the link
she shared. Sharing a name with the A.I. from the Terminator movies,
Skynet is a
\href{https://www.nytimes3xbfgragh.onion/2019/04/14/technology/china-surveillance-artificial-intelligence-racial-profiling.html}{real-life
techno-policing system}, one of several Beijing has spent billions to
create.

The surveillance push has supported a fast-growing force of internet
police. The group prowls services like WeChat for posts deemed
politically sensitive, anything from a link to a joke mocking leader Xi
Jinping. To handle WeChat's hundreds of millions of users and their
conversations, software analyzes keywords, links and images to generate
leads.

Although Ms. Li registered her account in Canada, she fell under Chinese
rules when she was back in China. Even outside of China, traffic on
WeChat appears to be feeding these automated systems of control. A
\href{https://citizenlab.ca/2020/05/wechat-surveillance-explained/}{report}
from Citizen Lab, a University of Toronto-based research group, showed
that Tencent surveilled images and files sent by WeChat users outside of
China to help train its censorship algorithms within China. In effect,
even when overseas users of WeChat are not being censored, the app
learns from them how to better censor.

Wary of falling into automated traps, Ms. Li now writes with typos.
Instead of referring directly to police, she uses a pun she invented,
calling them golden forks. She no longer shares links from news sites
outside of WeChat and holds back her inclination to talk politics.

Still, to be free she would have to delete WeChat, and she can't do
that. As the coronavirus crisis struck China,
\href{https://www.nytimes3xbfgragh.onion/2020/02/15/business/china-coronavirus-lockdown.html}{her
family used it} to coordinate food orders during lockdowns. She also
needs a local government
\href{https://www.nytimes3xbfgragh.onion/2020/03/01/business/china-coronavirus-surveillance.html}{health
code} featured on the app to use public transport or enter stores.

``I want to switch to other chat apps, but there's no way,'' she said.

``If there were a real alternative I would change, but WeChat is
terrible because there is no alternative. It's too closely tied to life.
For shopping, paying, for work, you have to use it,'' she said. ``If you
jump to another app, then you are alone.''

Lin Qiqing contributed research.

Advertisement

\protect\hyperlink{after-bottom}{Continue reading the main story}

\hypertarget{site-index}{%
\subsection{Site Index}\label{site-index}}

\hypertarget{site-information-navigation}{%
\subsection{Site Information
Navigation}\label{site-information-navigation}}

\begin{itemize}
\tightlist
\item
  \href{https://help.nytimes3xbfgragh.onion/hc/en-us/articles/115014792127-Copyright-notice}{©~2020~The
  New York Times Company}
\end{itemize}

\begin{itemize}
\tightlist
\item
  \href{https://www.nytco.com/}{NYTCo}
\item
  \href{https://help.nytimes3xbfgragh.onion/hc/en-us/articles/115015385887-Contact-Us}{Contact
  Us}
\item
  \href{https://www.nytco.com/careers/}{Work with us}
\item
  \href{https://nytmediakit.com/}{Advertise}
\item
  \href{http://www.tbrandstudio.com/}{T Brand Studio}
\item
  \href{https://www.nytimes3xbfgragh.onion/privacy/cookie-policy\#how-do-i-manage-trackers}{Your
  Ad Choices}
\item
  \href{https://www.nytimes3xbfgragh.onion/privacy}{Privacy}
\item
  \href{https://help.nytimes3xbfgragh.onion/hc/en-us/articles/115014893428-Terms-of-service}{Terms
  of Service}
\item
  \href{https://help.nytimes3xbfgragh.onion/hc/en-us/articles/115014893968-Terms-of-sale}{Terms
  of Sale}
\item
  \href{https://spiderbites.nytimes3xbfgragh.onion}{Site Map}
\item
  \href{https://help.nytimes3xbfgragh.onion/hc/en-us}{Help}
\item
  \href{https://www.nytimes3xbfgragh.onion/subscription?campaignId=37WXW}{Subscriptions}
\end{itemize}
