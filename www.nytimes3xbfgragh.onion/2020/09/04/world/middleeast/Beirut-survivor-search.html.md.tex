Sections

SEARCH

\protect\hyperlink{site-content}{Skip to
content}\protect\hyperlink{site-index}{Skip to site index}

\href{https://www.nytimes3xbfgragh.onion/section/world/middleeast}{Middle
East}

\href{https://myaccount.nytimes3xbfgragh.onion/auth/login?response_type=cookie\&client_id=vi}{}

\href{https://www.nytimes3xbfgragh.onion/section/todayspaper}{Today's
Paper}

\href{/section/world/middleeast}{Middle East}\textbar{}Rescue Dog and
Hopes of a Miracle Captivate Ravaged Beirut

\url{https://nyti.ms/3h3fhLY}

\begin{itemize}
\item
\item
\item
\item
\item
\end{itemize}

\hypertarget{beirut-explosion}{%
\subsubsection{Beirut Explosion}\label{beirut-explosion}}

\begin{itemize}
\tightlist
\item
  \href{https://www.nytimes3xbfgragh.onion/2020/08/11/world/middleeast/lebanon-government-resigns-explainer.html?name=styln-beirut\&region=TOP_BANNER\&block=storyline_menu_recirc\&action=click\&pgtype=Article\&impression_id=fed4d540-f2cd-11ea-8b78-5953de45aae8\&variant=undefined}{Lebanon
  in Political Limbo}
\item
  \href{https://www.nytimes3xbfgragh.onion/2020/08/05/world/middleeast/beirut-explosion-footage.html?name=styln-beirut\&region=TOP_BANNER\&block=storyline_menu_recirc\&action=click\&pgtype=Article\&impression_id=fed4d541-f2cd-11ea-8b78-5953de45aae8\&variant=undefined}{Footage
  of the Blast}
\item
  \href{https://www.nytimes3xbfgragh.onion/2020/08/05/world/middleeast/beirut-explosion-ammonium-nitrate.html?name=styln-beirut\&region=TOP_BANNER\&block=storyline_menu_recirc\&action=click\&pgtype=Article\&impression_id=fed4d542-f2cd-11ea-8b78-5953de45aae8\&variant=undefined}{What
  is Ammonium Nitrate?}
\item
  \href{https://www.nytimes3xbfgragh.onion/interactive/2020/08/04/world/middleeast/beirut-explosion-damage.html?name=styln-beirut\&region=TOP_BANNER\&block=storyline_menu_recirc\&action=click\&pgtype=Article\&impression_id=fed4d543-f2cd-11ea-8b78-5953de45aae8\&variant=undefined}{Mapping
  the Damage}
\end{itemize}

Advertisement

\protect\hyperlink{after-top}{Continue reading the main story}

Supported by

\protect\hyperlink{after-sponsor}{Continue reading the main story}

\hypertarget{rescue-dog-and-hopes-of-a-miracle-captivate-ravaged-beirut}{%
\section{Rescue Dog and Hopes of a Miracle Captivate Ravaged
Beirut}\label{rescue-dog-and-hopes-of-a-miracle-captivate-ravaged-beirut}}

A search squad from Chile and its dog Flash discovered what appeared to
be a heartbeat underneath debris from the Aug. 4 port blast.

\includegraphics{https://static01.graylady3jvrrxbe.onion/images/2020/09/04/world/04Lebanon1/merlin_176555331_cf08d4c7-d7f5-40e6-abe5-35c49cfba52d-videoSixteenByNine3000.jpg}

By \href{https://www.nytimes3xbfgragh.onion/by/ben-hubbard}{Ben Hubbard}
and Kareem Chehayeb

\begin{itemize}
\item
  Sept. 4, 2020
\item
  \begin{itemize}
  \item
  \item
  \item
  \item
  \item
  \end{itemize}
\end{itemize}

\href{https://www.nytimes3xbfgragh.onion/es/2020/09/04/espanol/mundo/libano-rescate-topos.html}{Leer
en español}

BEIRUT, Lebanon --- The sudden glimmer of hope in a devastated Beirut
neighborhood came from a dog named Flash with a shaggy black coat, a
white snout and red bootees to protect his paws from shattered glass.

One month after a massive explosion in Beirut's port killed 190 people
and ravaged the Lebanese capital, the dog smelled something in the
rubble of a destroyed historic building, and a technician working with a
Chilean rescue team deployed a sensor that picked up a slow pulse
underneath that could have been a heartbeat.

In the hours since the dog's discovery Thursday evening, the Lebanese
have been glued to their televisions, watching live coverage of rescue
crews in yellow vests sifting through debris and wondering if, after a
string of bruising traumas, they dared hope for a miracle.

Had someone survived under the rubble all this time?

But by Friday evening, the rescue crews had yet to find anything with a
pulse, and the leader of the Chilean team declined to tell reporters
when it had last picked up any sign of possible life. The Chilean
rescuers suspended their search and said they would resume Saturday
morning.

The Aug. 4 explosion, caused by the combustion of thousands of tons of
hazardous chemicals stored improperly in the Beirut port
\href{https://www.nytimes3xbfgragh.onion/2020/08/05/world/middleeast/beirut-explosion-ship.html}{since
2014}, was the most recent in a series of crises that have fueled deep
anger at the country's political elite over decades of mismanagement and
corruption.

Since last fall, the economy has been in free fall, the currency has
been shedding value and frequent anti-government protests have trashed
much of downtown. Many Lebanese are furious that their leaders let the
country deteriorate to this point, and that the politicians have failed
to take any meaningful steps.

Across Lebanon, people observed a moment of silence at 6:08 p.m. to mark
the one month anniversary of the blast as the time it had shattered the
capital.

A group of firefighters drove the route from their station toward the
port in memory of 10 of their colleagues who had gone to fight the
warehouse fire believed to have caused the blast and were all killed in
the explosion.

Near the port, where a deep crater marks the blast spot next to towering
grain silos shredded by the explosion, soldiers fired a salute and white
roses were laid on a memorial --- one for each of the blast's known
victims.

Anger at the country's politicians coursed through the commemorations,
and among the residents and volunteers who had gathered near the
collapsed historic building on Friday to await updates on the search for
the possible survivor. Many blamed the government not only for having
failed to prevent the blast, but also for having failed to help people
in the aftermath.

\includegraphics{https://static01.graylady3jvrrxbe.onion/images/2020/09/04/world/04Lebanon2/merlin_176557785_e925e3ae-9685-4490-ab53-0bda0adec1bc-articleLarge.jpg?quality=75\&auto=webp\&disable=upscale}

``What can we say other than shame on the government?'' said Nour
Hassan, a university student who came to the site with a volunteer
cleaning crew. ``This is so upsetting.''

She wondered, how could there even be a question of whether anyone
remained under the rubble from the Aug. 4 explosion?

``The state should have verified all this,'' she said. ``Now we don't
know if there are other bodies in other buildings, alive or dead.''

It appeared extremely unlikely that anyone had survived under the rubble
for a month, especially since daily temperatures in Beirut have been
sweltering, with high humidity.

But on Thursday, after Flash drew rescue workers to the destroyed
building, the rescue crew's equipment picked up a pulse of 18 beats per
minute. Suspecting that it could be a heartbeat, the crew started
digging.

The Chilean rescue
team\href{https://www.chicureohoy.cl/actualidad/topos-chile-los-socorristas-que-buscan-al-posible-sobreviviente-en-beirut/}{was
dispatched by Topos Chile}, an organization that also participated in
the rescue of 33 Chilean miners almost exactly a decade ago. The team
arrived in Beirut on Tuesday.

Francesco Lermonda, a Chilean volunteer, told the The Associated Press
that his team's equipment identified breathing and heartbeats from
humans, not animals. He said it was rare, but not unheard-of, for
someone to survive in such conditions for a month.

Tensions flared overnight Thursday when volunteers accused the Lebanese
Army of calling off the search. Hours later, a Civil Defense team
brought heavy machinery to help clear the rubble, and work resumed.

The army released a statement saying it was the search teams who had
stopped working, fearing that walls could collapse on them.

On Friday, teams of workers in hard hats and yellow vests dug carefully
through the rubble with shovels and bare hands, so as not to wound any
possible survivors or damage any remains found underneath.

The Chilean search team occasionally called for silence on the nearby
street to allow the sensors to pick up sounds from under the wreckage,
and rescuers created 3-D images of the ruins to try to identify where
survivors or bodies might be hidden.

On Friday morning, a member of the Chilean search team told a local
television station that the latest test has detected only seven beats
per minute.

After sunset Friday, another member of the Chilean team declined to say
when the team had last picked up any sign of life and insisted they
would keep searching as long as there was even a one percent chance of
saving someone.

An artist, Ivan Debs, created an image of Flash bravely standing on a
pile of rubble, his heart connecting with a heart underground.

``We have lot to learn from him,'' the artist wrote on Twitter.

The area, in the predominately Christian neighborhood of Gemmayze, was
once home to some of the city's most vibrant nightlife, its main street
lined with restaurants and bars where patrons regularly spilled out on
the sidewalk late into the night. The destroyed building where the crews
searched had been part of row containing a Chinese restaurant, a photo
studio and a grocery story called Twenty-Four Seven.

Now, those businesses have been erased, most residents have left their
damaged apartments and nearby shops and eateries are closed. At night,
the area is almost entirely dark.

Advertisement

\protect\hyperlink{after-bottom}{Continue reading the main story}

\hypertarget{site-index}{%
\subsection{Site Index}\label{site-index}}

\hypertarget{site-information-navigation}{%
\subsection{Site Information
Navigation}\label{site-information-navigation}}

\begin{itemize}
\tightlist
\item
  \href{https://help.nytimes3xbfgragh.onion/hc/en-us/articles/115014792127-Copyright-notice}{©~2020~The
  New York Times Company}
\end{itemize}

\begin{itemize}
\tightlist
\item
  \href{https://www.nytco.com/}{NYTCo}
\item
  \href{https://help.nytimes3xbfgragh.onion/hc/en-us/articles/115015385887-Contact-Us}{Contact
  Us}
\item
  \href{https://www.nytco.com/careers/}{Work with us}
\item
  \href{https://nytmediakit.com/}{Advertise}
\item
  \href{http://www.tbrandstudio.com/}{T Brand Studio}
\item
  \href{https://www.nytimes3xbfgragh.onion/privacy/cookie-policy\#how-do-i-manage-trackers}{Your
  Ad Choices}
\item
  \href{https://www.nytimes3xbfgragh.onion/privacy}{Privacy}
\item
  \href{https://help.nytimes3xbfgragh.onion/hc/en-us/articles/115014893428-Terms-of-service}{Terms
  of Service}
\item
  \href{https://help.nytimes3xbfgragh.onion/hc/en-us/articles/115014893968-Terms-of-sale}{Terms
  of Sale}
\item
  \href{https://spiderbites.nytimes3xbfgragh.onion}{Site Map}
\item
  \href{https://help.nytimes3xbfgragh.onion/hc/en-us}{Help}
\item
  \href{https://www.nytimes3xbfgragh.onion/subscription?campaignId=37WXW}{Subscriptions}
\end{itemize}
