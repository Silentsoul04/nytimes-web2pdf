Sections

SEARCH

\protect\hyperlink{site-content}{Skip to
content}\protect\hyperlink{site-index}{Skip to site index}

\href{https://www.nytimes3xbfgragh.onion/section/books/review}{Book
Review}

\href{https://myaccount.nytimes3xbfgragh.onion/auth/login?response_type=cookie\&client_id=vi}{}

\href{https://www.nytimes3xbfgragh.onion/section/todayspaper}{Today's
Paper}

\href{/section/books/review}{Book Review}\textbar{}A Son's Future, a
Father's Final Down

\url{https://nyti.ms/3lM3UeW}

\begin{itemize}
\item
\item
\item
\item
\item
\end{itemize}

Advertisement

\protect\hyperlink{after-top}{Continue reading the main story}

Supported by

\protect\hyperlink{after-sponsor}{Continue reading the main story}

\href{/column/childrens-books}{Children's Books}

\hypertarget{a-sons-future-a-fathers-final-down}{%
\section{A Son's Future, a Father's Final
Down}\label{a-sons-future-a-fathers-final-down}}

\includegraphics{https://static01.graylady3jvrrxbe.onion/images/2020/09/13/books/review/13-BKS-ARCHIBOLD-KIDS/13-BKS-ARCHIBOLD-KIDS-articleLarge.jpg?quality=75\&auto=webp\&disable=upscale}

Buy Book ▾

\begin{itemize}
\tightlist
\item
  \href{https://www.amazon.com/gp/search?index=books\&tag=NYTBSREV-20\&field-keywords=Before+the+Ever+After+Jacqueline+Woodson}{Amazon}
\item
  \href{https://du-gae-books-dot-nyt-du-prd.appspot.com/buy?title=Before+the+Ever+After\&author=Jacqueline+Woodson}{Apple
  Books}
\item
  \href{https://www.anrdoezrs.net/click-7990613-11819508?url=https\%3A\%2F\%2Fwww.barnesandnoble.com\%2Fw\%2F\%3Fean\%3D9780399545436}{Barnes
  and Noble}
\item
  \href{https://www.anrdoezrs.net/click-7990613-35140?url=https\%3A\%2F\%2Fwww.booksamillion.com\%2Fp\%2FBefore\%2Bthe\%2BEver\%2BAfter\%2FJacqueline\%2BWoodson\%2F9780399545436}{Books-A-Million}
\item
  \href{https://bookshop.org/a/3546/9780399545436}{Bookshop}
\item
  \href{https://www.indiebound.org/book/9780399545436?aff=NYT}{Indiebound}
\end{itemize}

When you purchase an independently reviewed book through our site, we
earn an affiliate commission.

By
\href{https://www.nytimes3xbfgragh.onion/by/randal-c-archibold}{Randal
C. Archibold}

\begin{itemize}
\item
  Sept. 5, 2020
\item
  \begin{itemize}
  \item
  \item
  \item
  \item
  \item
  \end{itemize}
\end{itemize}

\textbf{BEFORE THE EVER AFTER}\\
By Jacqueline Woodson

Zachariah Johnson Jr. (ZJ) is living a 12-year-old boy's dream: His
father is a star professional football player, he lives in a comfortable
home in the suburbs with a half basketball court upstairs, he has a trio
of friends who always show up at the right times and his budding
songwriting talent seems destined to take him far.

He is also living a nightmare.

Jacqueline Woodson's new novel, ``Before the Ever After,'' is not a work
of horror (despite the haunting title), but a creeping, invisible force
is upending ZJ's world and slowly stealing away his father --- known as
``Zachariah 44,'' for his jersey number --- before his and his mother's
eyes.

The father's hands have begun to tremble uncontrollably. He stares
vacantly. He forgets basic things, most achingly the name of the son who
bears, and at times is burdened by, his name. He's prone to angry
outbursts, to the point that ZJ's friends no longer want to come by the
house.

He is suffering the effects of a degenerative brain disease that, while
not named, bears a strong resemblance to chronic traumatic
encephalopathy, or C.T.E., which has been found in scores of former
N.F.L. players. Until 2016, the league for years denied any connection
between brain trauma on the field and hundreds of players' crippling
neurological ailments and, in many cases, deaths.

``My dad probably holds the Football Hall of Fame record for the most
concussions,'' ZJ says, relating how his mother has grown bitter about
the game. ``Even with a helmet on.''

Although you can envision fretful parents handing this book to young
boys eager to play, it's not a stern lecture. It's an elegiac meditation
on loss and longing told, like Woodson's seminal memoir,
``\href{https://www.nytimes3xbfgragh.onion/2014/08/24/books/review/jacqueline-woodsons-brown-girl-dreaming.html}{Brown
Girl Dreaming},'' mostly in verse.

This approach, and Woodson's evocative language (``the night is so dark,
it looks like a black wall''), helps pull us through the foreboding and
gives us much to contemplate; leitmotifs such as trees and song deepen
the story and provoke reflection on childhood, change and remembrance.

The story is set in 1999-2000, when the cost of brain injury in the
sport was just starting to come to light. The uncertainty over what has
happened, and what might be coming, bewilders ZJ and his mother.

``Sitting there with my mom and my dad snoring on the couch and the
doctors knowing but not knowing,'' he says, ``I feel like someone's
holding us, keeping us from getting back to where we were before and
keeping us from the next place too.''

This is largely a father-son tale, leaving ZJ's mother in the
background, revealed in the occasional tender scene --- Zachariah 44
drapes his arms around her in a moment of clarity --- but mostly in
quiet anguish.

``I think they're not telling the whole truth,'' ZJ overhears his mother
telling a friend. ``Too many of them ---''

ZJ is so disillusioned that he gives away one of his father's coveted
footballs to his friend Everett, in a scene that reminds us of the
staying power of the sport: ``Everett's eyes get wide. \emph{This is
Zachariah 44's ball?} I nod. \emph{For real?''}

ZJ finds solace in the music, literal and symbolic, that he and his
father have made together. ``Until the doctors figure out what's wrong,
this is what I have for him,'' ZJ says. ``My music, our songs.''

Woodson has said she seeks to instill optimism and hope. ZJ's patient
and supportive mother and his group of friends who are always buoying
him up serve that purpose here. Yet at times this striving for hope
feels strained, given a condition that so often offers no Hail Mary. ZJ
may not fully realize it, but we all know what's coming. The
nightmarish, seemingly irreversible decline of the once mighty and
strong has broken the hearts and wills of football families. A lyrical
portrayal of a player's fade and a boy coming to terms with it doesn't
change that.

Advertisement

\protect\hyperlink{after-bottom}{Continue reading the main story}

\hypertarget{site-index}{%
\subsection{Site Index}\label{site-index}}

\hypertarget{site-information-navigation}{%
\subsection{Site Information
Navigation}\label{site-information-navigation}}

\begin{itemize}
\tightlist
\item
  \href{https://help.nytimes3xbfgragh.onion/hc/en-us/articles/115014792127-Copyright-notice}{©~2020~The
  New York Times Company}
\end{itemize}

\begin{itemize}
\tightlist
\item
  \href{https://www.nytco.com/}{NYTCo}
\item
  \href{https://help.nytimes3xbfgragh.onion/hc/en-us/articles/115015385887-Contact-Us}{Contact
  Us}
\item
  \href{https://www.nytco.com/careers/}{Work with us}
\item
  \href{https://nytmediakit.com/}{Advertise}
\item
  \href{http://www.tbrandstudio.com/}{T Brand Studio}
\item
  \href{https://www.nytimes3xbfgragh.onion/privacy/cookie-policy\#how-do-i-manage-trackers}{Your
  Ad Choices}
\item
  \href{https://www.nytimes3xbfgragh.onion/privacy}{Privacy}
\item
  \href{https://help.nytimes3xbfgragh.onion/hc/en-us/articles/115014893428-Terms-of-service}{Terms
  of Service}
\item
  \href{https://help.nytimes3xbfgragh.onion/hc/en-us/articles/115014893968-Terms-of-sale}{Terms
  of Sale}
\item
  \href{https://spiderbites.nytimes3xbfgragh.onion}{Site Map}
\item
  \href{https://help.nytimes3xbfgragh.onion/hc/en-us}{Help}
\item
  \href{https://www.nytimes3xbfgragh.onion/subscription?campaignId=37WXW}{Subscriptions}
\end{itemize}
