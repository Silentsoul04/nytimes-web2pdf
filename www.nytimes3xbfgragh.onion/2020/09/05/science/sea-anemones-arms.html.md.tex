Sections

SEARCH

\protect\hyperlink{site-content}{Skip to
content}\protect\hyperlink{site-index}{Skip to site index}

\href{https://www.nytimes3xbfgragh.onion/section/science}{Science}

\href{https://myaccount.nytimes3xbfgragh.onion/auth/login?response_type=cookie\&client_id=vi}{}

\href{https://www.nytimes3xbfgragh.onion/section/todayspaper}{Today's
Paper}

\href{/section/science}{Science}\textbar{}When These Sea Anemones Eat,
It Goes Straight to Their Arms

\url{https://nyti.ms/2ZaPxqR}

\begin{itemize}
\item
\item
\item
\item
\item
\end{itemize}

Advertisement

\protect\hyperlink{after-top}{Continue reading the main story}

Supported by

\protect\hyperlink{after-sponsor}{Continue reading the main story}

Trilobites

\hypertarget{when-these-sea-anemones-eat-it-goes-straight-to-their-arms}{%
\section{When These Sea Anemones Eat, It Goes Straight to Their
Arms}\label{when-these-sea-anemones-eat-it-goes-straight-to-their-arms}}

They're the first animals known to turn food into extra limbs.

\includegraphics{https://static01.graylady3jvrrxbe.onion/images/2020/09/08/science/04TB-ANEMONES/04TB-ANEMONES-articleLarge.jpg?quality=75\&auto=webp\&disable=upscale}

By Cara Giaimo

\begin{itemize}
\item
  Sept. 5, 2020
\item
  \begin{itemize}
  \item
  \item
  \item
  \item
  \item
  \end{itemize}
\end{itemize}

People have a lot of strategies for dealing with the effects of large
meals --- constitutionals, antacids, workouts, naps.

Starlet sea anemones have found a better way: After they eat a lot, they
simply sprout some extra arms.

In a paper published Wednesday in Nature Communications, researchers
described how an abundance of food
\href{https://www.nature.com/articles/s41467-020-18133-0}{spurs these
anemones to grow new tentacles}, an ability never before seen in
animals.

Cnidarians --- a group that includes sea anemones, jellyfish and corals
--- diverged evolutionarily from the other animals more than half a
billion years ago. Flies, humans and the rest of the animal kingdom tend
to stick to the same body structure after they mature. But cnidarians
are famously adaptable. Adult anemones switch up their
\href{https://pubmed.ncbi.nlm.nih.gov/29351015/}{body size},
\href{https://journals.plos.org/plosone/article?id=10.1371/journal.pone.0011874}{reproductive
strategy} and even their
\href{https://www.atlasobscura.com/articles/sea-anemone-venom-change}{venom
composition} in response to environmental shifts.

``They never stop developing,'' said Aissam Ikmi, a group leader at the
European Molecular Biology Lab Heidelberg and lead author of the new
paper.

As larvae, starlet sea anemones grow four base tentacles. After they
reach adulthood, they add more until they have as many as 24, although
most max out at 16. (Other sea anemone species can grow hundreds.)
Because anemones are stuck in one spot, ``they have the same challenges
as a plant,'' Dr. Ikmi said. Their tentacles bring the world to them,
helping them capture food and sense the environment.

Dr. Ikmi noticed an association between how much the animals were eating
and how quickly these adult-stage tentacles appeared. Well-fed anemones
grew new pairs within three or four days. But if supplies dropped,
they'd pause, getting by with a limited set of six, eight or 10.

``I realized, oh --- you can control the tentacle addition just by
controlling the amount of food you provide,'' he said.

To test this hypothesis, Dr. Ikmi and his colleagues raised more than
1,000 anemones on a diet of brine shrimp, which is both snackable and,
for laboratory purposes, easily apportioned.

``It's like popcorn for them,'' Dr. Ikmi said.

The researchers fed the anemones set amounts of shrimp for a few days,
stopped for a few more, and then counted how many tentacles they had
sprouted and where. ``We did that for over six months,'' he said. They
created a ``tentacle map'' --- the sequence in which most anemones add
new arms.

The researchers then pinpointed the proteins and molecules through which
food abundance triggers tentacle growth. They found that adult-stage
tentacles develop differently than those first four larval ones do, even
though the resulting structure is the same. ``There is not one recipe to
build a tentacle,'' Dr. Ikmi said.

They also noticed that when particular individuals were kept from
spawning, they grew even more arms --- suggesting that they were
redirecting energy that would have been used for reproduction into a
tentacle bonanza.

So far, the starlet sea anemone is the only species known to approach
tentacle production in this way. But that the strategy occurs in adults
and also involves a trigger as common as food availability suggests that
``it's probably a broad phenomenon,'' at least within cnidarians, ****
said Christian Voolstra, a biology professor at the University of
Konstanz in Germany who was not involved in the study. **** (New anemone
talents are regularly discovered: Last year,
\href{https://link.springer.com/article/10.1007/s13199-019-00603-9}{Dr.
Voolstra's lab found} that when food is scarce, a different anemone,
Aiptasia, produces offspring with tiny tentacles or none at all.)

And molecular pathways like the one uncovered in the starlet anemone may
manifest in other animals as well, although they will likely express
themselves differently in different species. ``So learning something
here means learning something about us,'' Dr. Voolstra said.

Let's check in after Thanksgiving.

Advertisement

\protect\hyperlink{after-bottom}{Continue reading the main story}

\hypertarget{site-index}{%
\subsection{Site Index}\label{site-index}}

\hypertarget{site-information-navigation}{%
\subsection{Site Information
Navigation}\label{site-information-navigation}}

\begin{itemize}
\tightlist
\item
  \href{https://help.nytimes3xbfgragh.onion/hc/en-us/articles/115014792127-Copyright-notice}{©~2020~The
  New York Times Company}
\end{itemize}

\begin{itemize}
\tightlist
\item
  \href{https://www.nytco.com/}{NYTCo}
\item
  \href{https://help.nytimes3xbfgragh.onion/hc/en-us/articles/115015385887-Contact-Us}{Contact
  Us}
\item
  \href{https://www.nytco.com/careers/}{Work with us}
\item
  \href{https://nytmediakit.com/}{Advertise}
\item
  \href{http://www.tbrandstudio.com/}{T Brand Studio}
\item
  \href{https://www.nytimes3xbfgragh.onion/privacy/cookie-policy\#how-do-i-manage-trackers}{Your
  Ad Choices}
\item
  \href{https://www.nytimes3xbfgragh.onion/privacy}{Privacy}
\item
  \href{https://help.nytimes3xbfgragh.onion/hc/en-us/articles/115014893428-Terms-of-service}{Terms
  of Service}
\item
  \href{https://help.nytimes3xbfgragh.onion/hc/en-us/articles/115014893968-Terms-of-sale}{Terms
  of Sale}
\item
  \href{https://spiderbites.nytimes3xbfgragh.onion}{Site Map}
\item
  \href{https://help.nytimes3xbfgragh.onion/hc/en-us}{Help}
\item
  \href{https://www.nytimes3xbfgragh.onion/subscription?campaignId=37WXW}{Subscriptions}
\end{itemize}
