Sections

SEARCH

\protect\hyperlink{site-content}{Skip to
content}\protect\hyperlink{site-index}{Skip to site index}

\href{https://www.nytimes3xbfgragh.onion/section/sports/basketball}{Pro
Basketball}

\href{https://myaccount.nytimes3xbfgragh.onion/auth/login?response_type=cookie\&client_id=vi}{}

\href{https://www.nytimes3xbfgragh.onion/section/todayspaper}{Today's
Paper}

\href{/section/sports/basketball}{Pro Basketball}\textbar{}The Best Part
of Layshia Clarendon's Game? `Fearlessness'

\url{https://nyti.ms/3lTgw44}

\begin{itemize}
\item
\item
\item
\item
\item
\end{itemize}

Advertisement

\protect\hyperlink{after-top}{Continue reading the main story}

Supported by

\protect\hyperlink{after-sponsor}{Continue reading the main story}

Under the Radar

\hypertarget{the-best-part-of-layshia-clarendons-game-fearlessness}{%
\section{The Best Part of Layshia Clarendon's Game?
`Fearlessness'}\label{the-best-part-of-layshia-clarendons-game-fearlessness}}

``It doesn't matter if I get blocked --- I'm going to go right back in
the paint again,'' the Liberty guard said. They bring that same tenacity
to their role on the W.N.B.A. union's executive committee.

\includegraphics{https://static01.graylady3jvrrxbe.onion/images/2020/09/05/sports/05wnba-under-liberty-1/merlin_175891599_c13d1ac7-2a63-4588-a807-3ea1b24a5ed7-articleLarge.jpg?quality=75\&auto=webp\&disable=upscale}

By
\href{https://www.nytimes3xbfgragh.onion/by/danielle-allentuck}{Danielle
Allentuck}

\begin{itemize}
\item
  Sept. 5, 2020
\item
  \begin{itemize}
  \item
  \item
  \item
  \item
  \item
  \end{itemize}
\end{itemize}

On the court, Layshia Clarendon has become the leader of a young Liberty
team coursing through a season with many unexpected challenges. As an
executive committee member of the W.N.B.A.'s players' union, Clarendon
has also become a voice for social justice for the league this year.

Clarendon is averaging a career-high 11.7 points a game, but the Liberty
are 2-15 and in last place in the Eastern Conference. The team is
playing this season without several key players who decided to opt out
of this season because of concerns over the coronavirus. Sabrina
Ionescu, their star No. 1 pick in the 2020 draft, has been out since she
\href{https://www.nytimes3xbfgragh.onion/2020/08/01/sports/basketball/sabrina-ionescu-injury.html}{sprained
her ankle in the third game of the season}.

Clarendon, who is in their eighth season, missed all but nine games last
year, with the Connecticut Sun, because of a right ankle injury. The
ankle ``gets stiff every now and then,'' Clarendon said, but they have
still been able to serve as a critical veteran presence for a young
Liberty team that began the year with seven rookies.

But this year is about more than just statistics for Clarendon. The
W.N.B.A. has dedicated its season to
\href{https://www.nytimes3xbfgragh.onion/2020/07/25/sports/wnba-seattle-storm-new-york-liberty.html?searchResultPosition=3}{Breonna
Taylor}, a 26-year-old Black woman who was shot and killed by the police
in Louisville, Ky., and the Say Her Name campaign, which focuses on
Black women and girls affected by police brutality and violence.
Clarendon is one of the players leading the W.N.B.A.'s social justice
initiatives.

The New York Times talked to Clarendon about playing fearlessly, the
challenges of the tight game schedule this season and how, they said,
social justice movements often overlook Black women.

\textbf{Q: What has life in the bubble been like for you?}

\textbf{Clarendon}: It's up and down, depending on the moment. The
schedule is really hectic. I don't think I expected it to be this
challenging to consistently play three games a week, so that's been
really tough from a purely recovery standpoint.

There's just not a lot of downtime or time off just because we are
playing so often.

\textbf{How do you cope with the busy schedule?}

Normal ice baths and recovery stuff that I do every season. It's
definitely been a more mentally challenging season. Obviously, with
Covid going on, the state of the world and police murdering people left
and right, it's been more emotional and spiritual than physical most of
the time. You can sleep for nine hours and still wake up and feel the
weight of the world on your shoulders.

\textbf{What do you think is the best part of your game?}

I would say tenacity and fearlessness. It doesn't matter if I get
blocked, I'm going to go right back in the paint again against the same
player who blocked me on the previous play.

\textbf{How did you develop that tenacity and fearlessness?}

Practice finishing a lot, trying to get into people's bodies and create
contact. I think it's a mind-set, too. If you are really early on in the
league and you get your stuff thrown into the stands, it's embarrassing.
But if you don't get crossed-up in this league or your shot blocked in a
game or something really bad in game, then you probably aren't playing
hard enough and you really don't have your heart in it.

This league is so good, it's just going to happen. Part of that is a
mind-set of knowing that when you go up against Sue Bird or A'ja Wilson,
they are going to block you. You are also going to get them. It's about
looking at it as a challenge and an opportunity rather than a: `Oh, I
got blocked. That's so embarrassing.' No one wants to be on the wrong
side of ``SportsCenter.''

\textbf{As a member of the executive committee, you helped lead the day
of reflection last week. What went into the decision to call it a day of
reflection instead of a}
\textbf{\href{https://www.nytimes3xbfgragh.onion/2020/08/27/us/difference-boycott-strike-nba.html}{boycott
or strike}?} \emph{(The league missed two days of games last week after
its}
\href{https://www.nytimes3xbfgragh.onion/2020/08/27/sports/basketball/kenosha-nba-protests-players-boycott.html}{\emph{players
joined an N.B.A. work stoppage}} \emph{to protest the}
\href{https://www.nytimes3xbfgragh.onion/2020/08/28/us/kenosha-shooting-protests.html}{\emph{police
shooting of Jacob Blake}}\emph{, a Black man who was shot in the back
multiple times by the police in Kenosha, Wis. )}

After we had a players meeting we realized just how exhausted everyone
was. It was more like we needed that day. I think very much of it was
standing in solidarity with our N.B.A. brothers. You could see it in
people's faces that day on TV how exhausted and heartbroken everyone
was.

Yes, we were striking. Yes, we were fighting injustice, but we are
exhausted and we are tired. We are calling it for a day of reflection
and a day of mourning. We needed the time to take a step back.

\textbf{What would you like to see the W.N.B.A. do next?}

Voting is going to be a big one that we are trying to figure out a
strategy around it right now. We could wear a `vote' mask, which would
be great awareness, but there has to be some longer game strategy behind
how we are going to engage people. We are going to really focus on
voting and the work we have been doing with Say Her Name, which I think
can't be understated.

At a time when Black women continue to be erased from the larger
conversation of police brutality and violence, that's why our work is
particularly important. While, yes, we also stand with Jacob Blake and
his family and all of the men who have been murdered in this movement,
it is a constant reminder of how women have to choose between erasing
themselves to stand up for their race or standing up for women, standing
up for their own women.

That's the constant struggle I feel like we are always in. I don't want
us to get away from Saying Her Name because that's the whole point of
the movement and why we came here. It's sad that we always have to
choose between standing up for our men and standing up for ourselves,
because who stands up for us? No one.

Advertisement

\protect\hyperlink{after-bottom}{Continue reading the main story}

\hypertarget{site-index}{%
\subsection{Site Index}\label{site-index}}

\hypertarget{site-information-navigation}{%
\subsection{Site Information
Navigation}\label{site-information-navigation}}

\begin{itemize}
\tightlist
\item
  \href{https://help.nytimes3xbfgragh.onion/hc/en-us/articles/115014792127-Copyright-notice}{©~2020~The
  New York Times Company}
\end{itemize}

\begin{itemize}
\tightlist
\item
  \href{https://www.nytco.com/}{NYTCo}
\item
  \href{https://help.nytimes3xbfgragh.onion/hc/en-us/articles/115015385887-Contact-Us}{Contact
  Us}
\item
  \href{https://www.nytco.com/careers/}{Work with us}
\item
  \href{https://nytmediakit.com/}{Advertise}
\item
  \href{http://www.tbrandstudio.com/}{T Brand Studio}
\item
  \href{https://www.nytimes3xbfgragh.onion/privacy/cookie-policy\#how-do-i-manage-trackers}{Your
  Ad Choices}
\item
  \href{https://www.nytimes3xbfgragh.onion/privacy}{Privacy}
\item
  \href{https://help.nytimes3xbfgragh.onion/hc/en-us/articles/115014893428-Terms-of-service}{Terms
  of Service}
\item
  \href{https://help.nytimes3xbfgragh.onion/hc/en-us/articles/115014893968-Terms-of-sale}{Terms
  of Sale}
\item
  \href{https://spiderbites.nytimes3xbfgragh.onion}{Site Map}
\item
  \href{https://help.nytimes3xbfgragh.onion/hc/en-us}{Help}
\item
  \href{https://www.nytimes3xbfgragh.onion/subscription?campaignId=37WXW}{Subscriptions}
\end{itemize}
