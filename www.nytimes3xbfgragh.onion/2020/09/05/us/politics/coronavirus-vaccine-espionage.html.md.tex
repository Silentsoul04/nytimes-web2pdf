Sections

SEARCH

\protect\hyperlink{site-content}{Skip to
content}\protect\hyperlink{site-index}{Skip to site index}

\href{https://www.nytimes3xbfgragh.onion/section/politics}{Politics}

\href{https://myaccount.nytimes3xbfgragh.onion/auth/login?response_type=cookie\&client_id=vi}{}

\href{https://www.nytimes3xbfgragh.onion/section/todayspaper}{Today's
Paper}

\href{/section/politics}{Politics}\textbar{}Race for Coronavirus Vaccine
Pits Spy Against Spy

\url{https://nyti.ms/2F2oAPd}

\begin{itemize}
\item
\item
\item
\item
\item
\item
\end{itemize}

\hypertarget{school-reopenings}{%
\subsubsection{\texorpdfstring{\href{https://www.nytimes3xbfgragh.onion/spotlight/schools-reopening?name=styln-coronavirus-schools-reopening\&region=TOP_BANNER\&block=storyline_menu_recirc\&action=click\&pgtype=Article\&impression_id=1791e5a0-f27e-11ea-b383-c73395fd03d7\&variant=undefined}{School
Reopenings}}{School Reopenings}}\label{school-reopenings}}

\begin{itemize}
\tightlist
\item
  \href{https://www.nytimes3xbfgragh.onion/2020/09/04/us/bar-exam-coronavirus.html?name=styln-coronavirus-schools-reopening\&region=TOP_BANNER\&block=storyline_menu_recirc\&action=click\&pgtype=Article\&impression_id=1791e5a1-f27e-11ea-b383-c73395fd03d7\&variant=undefined}{Delayed
  Licensing Exams}
\item
  \href{https://www.nytimes3xbfgragh.onion/2020/09/08/upshot/children-testing-shortfalls-virus.html?name=styln-coronavirus-schools-reopening\&region=TOP_BANNER\&block=storyline_menu_recirc\&action=click\&pgtype=Article\&impression_id=17920cb0-f27e-11ea-b383-c73395fd03d7\&variant=undefined}{Limited
  Testing for Children}
\item
  \href{https://www.nytimes3xbfgragh.onion/2020/09/01/world/schools-reopen-globe-students.html?name=styln-coronavirus-schools-reopening\&region=TOP_BANNER\&block=storyline_menu_recirc\&action=click\&pgtype=Article\&impression_id=17920cb1-f27e-11ea-b383-c73395fd03d7\&variant=undefined}{School
  Around the World}
\item
  \href{https://www.nytimes3xbfgragh.onion/interactive/2020/us/covid-college-cases-tracker.html?name=styln-coronavirus-schools-reopening\&region=TOP_BANNER\&block=storyline_menu_recirc\&action=click\&pgtype=Article\&impression_id=17920cb2-f27e-11ea-b383-c73395fd03d7\&variant=undefined}{Tracking
  College Cases}
\end{itemize}

Advertisement

\protect\hyperlink{after-top}{Continue reading the main story}

Supported by

\protect\hyperlink{after-sponsor}{Continue reading the main story}

\hypertarget{race-for-coronavirus-vaccine-pits-spy-against-spy}{%
\section{Race for Coronavirus Vaccine Pits Spy Against
Spy}\label{race-for-coronavirus-vaccine-pits-spy-against-spy}}

The intelligence wars over vaccine research have intensified as China
and Russia expand their efforts to steal American work at both research
institutes and companies.

\includegraphics{https://static01.graylady3jvrrxbe.onion/images/2020/09/06/us/politics/06dc-virus-intel-print/merlin_172039368_ae954102-4fba-4dd0-8afd-f066baa7b919-articleLarge.jpg?quality=75\&auto=webp\&disable=upscale}

By \href{https://www.nytimes3xbfgragh.onion/by/julian-e-barnes}{Julian
E. Barnes} and Michael Venutolo-Mantovani

\begin{itemize}
\item
  Sept. 5, 2020
\item
  \begin{itemize}
  \item
  \item
  \item
  \item
  \item
  \item
  \end{itemize}
\end{itemize}

\href{https://cn.nytimes3xbfgragh.onion/usa/20200907/coronavirus-vaccine-espionage/}{阅读简体中文版}\href{https://cn.nytimes3xbfgragh.onion/usa/20200907/coronavirus-vaccine-espionage/zh-hant/}{閱讀繁體中文版}

WASHINGTON --- Chinese intelligence hackers were intent on stealing
coronavirus vaccine data, so they looked for what they believed would be
an easy target. Instead of simply going after pharmaceutical companies,
they conducted digital reconnaissance on the University of North
Carolina and other schools doing cutting-edge research.

They were not the only spies at work. Russia's premier intelligence
service, the S.V.R., targeted vaccine research networks in the United
States, Canada and Britain, espionage efforts that were first detected
by a British spy agency monitoring international fiber optic cables.

Iran, too, has drastically stepped up its attempts to steal information
about vaccine research, and the United States has increased its own
efforts to track the espionage of its adversaries and shore up its
defenses.

In short, every major spy service around the globe is trying to find out
what everyone else is up to.

The coronavirus pandemic has prompted one of the fastest peacetime
mission shifts in recent times for the world's intelligence agencies,
pitting them against one another in a new grand game of spy versus spy,
according to interviews with current and former intelligence officials
and others tracking the espionage efforts.

Nearly all of the United States' adversaries intensified their attempts
to steal American research while Washington, in turn, has moved to
protect the universities and corporations doing the most advanced work.
NATO intelligence, normally concerned with the movement of Russian tanks
and terrorist cells, has expanded to scrutinize Kremlin efforts to steal
vaccine research as well, according to a Western official briefed on the
intelligence.

The contest is reminiscent of the space race, where the Soviet Union and
America relied on their spy services to catch up when the other looked
likely to achieve a milestone. But where the Cold War contest to reach
the Earth's orbit and the moon played out over decades, the timeline to
help secure data on coronavirus treatments is sharply compressed as the
need for a vaccine
\href{https://www.nytimes3xbfgragh.onion/2020/05/02/us/politics/vaccines-coronavirus-research.html}{grows
more urgent} each day.

``It would be surprising if they were not trying to steal the most
valuable biomedical research going on right now,'' John C. Demers, a top
Justice Department official,
\href{https://www.csis.org/events/online-event-countering-chinese-espionage}{said
of China last month} during an event held by the Center for Strategic
and International Studies. ``Valuable from a financial point of view and
invaluable from a geopolitical point of view.''

China's push is complex. Its operatives have also surreptitiously used
information from the World Health Organization to guide its vaccine
hacking attempts, both in the United States and Europe, according to a
current and a former official familiar with the intelligence.

It was not clear how exactly China was using its influential position in
the W.H.O. to gather information about vaccine work around the globe.
The organization does collect data about vaccines under development, and
while much of it is eventually made public, Chinese hackers could have
benefited by getting early information on what coronavirus vaccine
research efforts the W.H.O. viewed as most promising, according to a
former intelligence official.

American intelligence officials learned about China's efforts in early
February as the virus was gaining a foothold in the United States,
according to current and former American officials. The C.I.A. and other
agencies closely watch China's moves inside international agencies,
including the W.H.O.

The intelligence conclusion helped push the White House toward
\href{https://www.nytimes3xbfgragh.onion/2020/05/19/us/politics/trump-who-coronavirus.html}{the
tough line it adopted} in May on the W.H.O., according to the former
intelligence official.

Besides the University of North Carolina, Chinese hackers have also
targeted other universities around the country and some may have had
their networks breached, American officials said. Mr. Demers said in his
speech that China had conducted ``multiple intrusions'' beyond what the
Justice Department revealed
\href{https://www.nytimes3xbfgragh.onion/2020/07/21/us/politics/china-hacking-coronavirus-vaccine.html}{in
an indictment} in July, which accused two hackers of working on behalf
of China's Ministry of State Security spy service to pursue vaccine
information and research from American biotechnology companies.

The F.B.I. warned officials at U.N.C. in recent weeks about the hacking
attempts, according to two people familiar with the matter. The Chinese
hacking teams were trying to break into the computer networks of the
school's epidemiology department but did not infiltrate them.

A U.N.C. spokeswoman, Leslie Minton, said that the school ``regularly
receives threat alerts from U.S. security agencies.'' She directed
further questions to the federal government, but said the school had
invested in ``around-the-clock monitoring'' to ``help guard against
advanced persistent threat attacks from state sponsored organizations.''

Besides hacking, China has pushed into universities in other ways. Some
government officials believe it is trying to take advantage of research
partnerships that American universities have forged with Chinese
institutions.

Others have warned that Chinese intelligence agents in the United States
and elsewhere have tried to collect information on researchers
themselves. The Trump administration ordered China on July 22
\href{https://www.nytimes3xbfgragh.onion/2020/07/22/world/asia/us-china-houston-consulate.html}{to
close its consulate in Houston} in part because Chinese operatives had
used it as an outpost to try to make inroads with medical experts in the
city, according to the F.B.I.

Chinese intelligence officials are focused on universities in part
because they view the institutions' data protections as less robust than
those of pharmaceutical companies. But spy work is also intensifying as
researchers share more vaccine candidates and antiviral treatments for
peer review, giving adversaries a better chance of gaining access to
formulations and vaccine development strategies, said an American
government official briefed on the intelligence.

So far, officials believe that foreign spies have taken little
information from the American biotech companies they targeted: Gilead
Sciences, Novavax and Moderna.

\includegraphics{https://static01.graylady3jvrrxbe.onion/images/2020/08/28/us/politics/28dc-virus-intel2/merlin_173530260_88974c65-dccb-4b61-b65b-4151323a5a17-articleLarge.jpg?quality=75\&auto=webp\&disable=upscale}

At the same time the British electronic surveillance agency G.C.H.Q. was
learning about the Russian effort and American intelligence learned of
the Chinese hacking, the Department of Homeland Security and F.B.I.
dispatched teams to work with American biotech teams to bolster their
computer networks' defenses.

The Russian effort,
\href{https://www.nytimes3xbfgragh.onion/2020/07/16/us/politics/vaccine-hacking-russia.html}{announced
by British, American and Canadian intelligence agencies in July}, was
primarily focused on gathering intelligence about research by Oxford
University and its pharmaceutical corporate partner, AstraZeneca.

\href{https://www.nytimes3xbfgragh.onion/spotlight/schools-reopening?action=click\&pgtype=Article\&state=default\&region=MAIN_CONTENT_3\&context=storylines_keepup}{}

\hypertarget{school-reopenings-}{%
\subsubsection{School Reopenings ›}\label{school-reopenings-}}

\hypertarget{back-to-school}{%
\paragraph{Back to School}\label{back-to-school}}

Updated Sept. 8, 2020

The latest on how schools are reopening amid the pandemic.

\begin{itemize}
\item
  \begin{itemize}
  \tightlist
  \item
    The first day of school is an annual rite of passage. But this year,
    it looks very different for tens of millions of students.
    \href{https://www.nytimes3xbfgragh.onion/2020/09/05/us/virtual-return-to-school-covid.html?action=click\&pgtype=Article\&state=default\&region=MAIN_CONTENT_3\&context=storylines_keepup}{We
    talked to some about their hopes and fears}.
  \item
    Coronavirus cases
    \href{https://www.nytimes3xbfgragh.onion/2020/09/06/us/colleges-coronavirus-students.html?action=click\&pgtype=Article\&state=default\&region=MAIN_CONTENT_3\&context=storylines_keepup}{are
    spiking in America's college towns}, leading to concern that young
    people who are infected will contribute to a spread of the virus.
  \item
    A growing number of Catholic schools across the country are
    \href{https://www.nytimes3xbfgragh.onion/2020/09/05/us/catholic-school-closings.html?action=click\&pgtype=Article\&state=default\&region=MAIN_CONTENT_3\&context=storylines_keepup}{shutting
    down forever during the coronavirus pandemic}, citing insurmountable
    financial pressure.
  \item
    The magazine's Ethicist columnist answers a question from a
    spokesperson at a major university:
    \href{https://www.nytimes3xbfgragh.onion/2020/09/08/magazine/university-reopening-safety-ethics.html?action=click\&pgtype=Article\&state=default\&region=MAIN_CONTENT_3\&context=storylines_keepup}{Can
    I promote a reopening plan I have doubts about}?
  \end{itemize}
\end{itemize}

The Russians caught trying to get vaccine information were part of the
group known as Cozy Bear, a collection of hackers affiliated with the
S.V.R. Cozy Bear was one of the hacking groups that in 2016 broke into
Democratic computer servers.

Homeland security officials have warned pharmaceutical companies and
universities about the attacks and helped institutions review their
security. For the most part, officials have observed the would-be
vaccine hackers using known vulnerabilities that have yet to be patched,
not the more exquisite cyberweapons that target unknown gaps in computer
security.

No corporation or university has announced any data thefts resulting
from the publicly identified hacking efforts. But some of the hacking
attempts succeeded in at least penetrating defenses to get inside
computer networks, according to one American government official. And
hackers for China and Russia test weaknesses every day, according to
intelligence officials.

``It is really a race against time for good guys to find the
vulnerabilities and get them patched, get those patches deployed before
the adversary finds them and exploits them,'' said Bryan S. Ware, the
assistant director of cybersecurity for the Homeland Security
Department's Cybersecurity and Infrastructure Security Agency. ``The
race is tighter than ever.''

While only two teams of hackers, one each from Russia and China, have
been publicly identified, multiple hacking teams from nearly all the
intelligence services of those two countries have been trying to steal
vaccine information, according to law enforcement and intelligence
officials.

Russia
\href{https://www.nytimes3xbfgragh.onion/2020/08/11/world/europe/russia-coronavirus-vaccine-approval.html}{announced
on Aug. 11} that it had approved a vaccine, a declaration that
immediately aroused suspicion that its scientists were at least aided by
its spy agencies' work to steal research information from other
countries.

American officials insist their own spy services' efforts are defensive
and that intelligence agencies have not been ordered to steal
coronavirus research. But other current and former intelligence
officials said the reality was not nearly so black and white. As
American intelligence agencies try to find out what Russia, China and
Iran may have stolen, they could encounter information on those
countries' research and collect it.

Officials expressed concerns that further hacking attempts could hurt
vaccine development efforts. Hackers extracting data could inadvertently
--- or purposefully --- damage research systems.

``When an adversary is doing a smash-and-grab, there is even more likely
a chance of not just stealing information but somehow disrupting the
victim's operations networks,'' Mr. Ware said.

While some of Russia's and China's spying may have been aimed at
checking their own research or looking for shortcuts, some current and
former officials raised the possibility that the countries sought
instead to sow distrust in an eventual vaccine from Western countries.

Both Russia and China have already spread disinformation about the
virus, its origins and the
\href{https://www.nytimes3xbfgragh.onion/2020/04/22/us/politics/coronavirus-china-disinformation.html}{American
response}. Russian intelligence services in particular are laying the
groundwork for a more aggressive effort to escalate the anti-vaccine
movement in the West and could use the allegations of spying to give its
narrative greater traction.

Russia has a long record of trying to amplify divisions in American
society. Current and former national security officials said they expect
Russia to eventually spread disinformation about any vaccine approved in
the West.

``This case seems to be a throwback to the old Soviet Union,'' said
Fiona Hill, the former National Security Council official and Russia
expert who
\href{https://www.nytimes3xbfgragh.onion/2019/11/21/us/politics/fiona-hill-impeachment-ukraine.html}{testified
in the impeachment hearings} against President Trump. ``Russia and the
Chinese have been out there on disinformation campaigns. How better to
create confusion and weaken the U.S. further than to whip up the antivax
movement? But you make sure all your guys are vaccinated.''

David E. Sanger and Ronen Bergman contributed reporting.

Advertisement

\protect\hyperlink{after-bottom}{Continue reading the main story}

\hypertarget{site-index}{%
\subsection{Site Index}\label{site-index}}

\hypertarget{site-information-navigation}{%
\subsection{Site Information
Navigation}\label{site-information-navigation}}

\begin{itemize}
\tightlist
\item
  \href{https://help.nytimes3xbfgragh.onion/hc/en-us/articles/115014792127-Copyright-notice}{©~2020~The
  New York Times Company}
\end{itemize}

\begin{itemize}
\tightlist
\item
  \href{https://www.nytco.com/}{NYTCo}
\item
  \href{https://help.nytimes3xbfgragh.onion/hc/en-us/articles/115015385887-Contact-Us}{Contact
  Us}
\item
  \href{https://www.nytco.com/careers/}{Work with us}
\item
  \href{https://nytmediakit.com/}{Advertise}
\item
  \href{http://www.tbrandstudio.com/}{T Brand Studio}
\item
  \href{https://www.nytimes3xbfgragh.onion/privacy/cookie-policy\#how-do-i-manage-trackers}{Your
  Ad Choices}
\item
  \href{https://www.nytimes3xbfgragh.onion/privacy}{Privacy}
\item
  \href{https://help.nytimes3xbfgragh.onion/hc/en-us/articles/115014893428-Terms-of-service}{Terms
  of Service}
\item
  \href{https://help.nytimes3xbfgragh.onion/hc/en-us/articles/115014893968-Terms-of-sale}{Terms
  of Sale}
\item
  \href{https://spiderbites.nytimes3xbfgragh.onion}{Site Map}
\item
  \href{https://help.nytimes3xbfgragh.onion/hc/en-us}{Help}
\item
  \href{https://www.nytimes3xbfgragh.onion/subscription?campaignId=37WXW}{Subscriptions}
\end{itemize}
