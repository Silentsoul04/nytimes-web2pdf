Sections

SEARCH

\protect\hyperlink{site-content}{Skip to
content}\protect\hyperlink{site-index}{Skip to site index}

\href{https://www.nytimes3xbfgragh.onion/section/world}{World}

\href{https://myaccount.nytimes3xbfgragh.onion/auth/login?response_type=cookie\&client_id=vi}{}

\href{https://www.nytimes3xbfgragh.onion/section/todayspaper}{Today's
Paper}

\href{/section/world}{World}\textbar{}`Dwarf Pride' Was Hard Won. Will a
Growth Drug Undermine It?

\url{https://nyti.ms/3h4vyAa}

\begin{itemize}
\item
\item
\item
\item
\item
\end{itemize}

Advertisement

\protect\hyperlink{after-top}{Continue reading the main story}

Supported by

\protect\hyperlink{after-sponsor}{Continue reading the main story}

\hypertarget{dwarf-pride-was-hard-won-will-a-growth-drug-undermine-it}{%
\section{`Dwarf Pride' Was Hard Won. Will a Growth Drug Undermine
It?}\label{dwarf-pride-was-hard-won-will-a-growth-drug-undermine-it}}

An experimental medication that increases height in children with the
most common form of dwarfism has raised hope that it can help them lead
easier lives. But some say the condition is not a problem in need of a
cure.

\includegraphics{https://static01.graylady3jvrrxbe.onion/images/2020/09/05/world/05dwarfism4/merlin_176603772_3f9c3e51-0854-4847-8cfa-a267769b6e50-articleLarge.jpg?quality=75\&auto=webp\&disable=upscale}

By Serena Solomon

\begin{itemize}
\item
  Published Sept. 5, 2020Updated Sept. 6, 2020
\item
  \begin{itemize}
  \item
  \item
  \item
  \item
  \item
  \end{itemize}
\end{itemize}

AUCKLAND, New Zealand --- It's a question many parents of children with
dwarfism have contemplated: If a medication could make them taller,
would they give it to them?

Now, that possibility is becoming less hypothetical. A
\href{https://www.thelancet.com/journals/lancet/article/PIIS0140-6736(20)31541-5/fulltext?rss=yes}{study}
published this weekend in the journal The Lancet found that an
experimental drug called vosoritide increased growth in children with
the most common form of dwarfism to nearly the same rate as in children
without the condition.

The study has raised hope that the drug, if taken over the course of
years, can make life easier for those with the condition, known as
achondroplasia, including the distant prospect of alleviating major
quality-of-life issues such as back pain and breathing difficulties.

But the drug has also ignited a contentious debate in a community that
sees ``dwarf pride'' as a hard-won tenet --- where being a little person
is a unique trait to be celebrated, not a problem in need of a cure.

Weeks before their son Lachlan was born, Dr. Simone Watkins and her
husband learned that he most likely had achondroplasia, which affects
about one in 25,000 infants.

After his birth, Dr. Watkins recalled, she and her husband said over
him: ``We love you. You're perfect. We are so happy you're here. You're
going to have a great life.''

She now feels that vosoritide could compromise that sentiment.

``I want him to have the best life possible with less complications and
not to be bullied and to fit into society,'' Dr. Watkins said as
Lachlan, 2, played next to her in a pile of pillows at their home in
Auckland, New Zealand. ``But also, I don't want to give him the message
that he needs to change.''

Achondroplasia is a genetic disorder that disrupts the transition of
cartilage to bone. Those with the condition have shorter arms and legs
than those found in people of average stature, as well as defining
facial features. Their adult height is typically a little over 4 feet.
More than 80 percent of those with achondroplasia are born to parents of
average stature, and a child with the condition has a 50 percent chance
of passing it on.

The study in The Lancet found that children who took the drug grew an
additional 0.6 inches on average in one year, with minimal side effects.
If taken over many years, vosoritide could produce a significant
increase in adult height, though the study was limited to a year and
does not address this possibility, or resolve whether the medication can
ease the medical complications common to dwarfism.

\includegraphics{https://static01.graylady3jvrrxbe.onion/images/2020/09/05/world/05dwarfism3/merlin_176596242_0e24eabd-c7ab-4267-90dc-831655571c77-articleLarge.jpg?quality=75\&auto=webp\&disable=upscale}

The trial examined 121 children ages 5 to 17 over a 12-month period.
Participants were located in seven countries.

In August, BioMarin, the American pharmaceutical company behind
vosoritide, submitted the study's findings to the Food and Drug
Administration as well as the European Medicines Agency. If approved,
vosoritide could be available within months.

``It doesn't totally restore all of the growth, but it does make a
pretty significant dent in the difference,'' said Dr. Eric Rush, a
clinical geneticist at Children's Mercy Hospital in Kansas City, Mo.,
and an associate professor of pediatrics at the University of Missouri,
Kansas City.

He was not involved in the vosoritide trial, but has consulted for
BioMarin and is involved in trials for a similar drug.

Vosoritide utilizes a synthetic form of a protein that humans produce
naturally. It targets the overactive signal that prevents bone growth in
children with achondroplasia, said Dr. Ravi Savarirayan, a clinical
geneticist at the Murdoch Children's Research Institute in Melbourne,
Australia, who led the trial.

He compared the condition's effects to watering a plant. ``It's not
going to grow if it gets too much water, so we are just regulating the
amount of water,'' Dr. Savarirayan said, calling the drug a ``precision
therapy that actually counteracts the underlying problem.''

Dr. Savarirayan offered a moving example of what longer limbs could
deliver.

Image

Dr. Ravi Savarirayan in his home in Melbourne,
Australia.Credit...Christina Simons for The New York Times

``We've got 12- and 13-year-old girls who now for the first time can do
their own feminine hygiene and don't need to be helped by someone
because their arms are longer,'' he said.

Sarah Cohen, an 11-year-old who lives in Geelong, near Melbourne,
started taking vosoritide at age 7 and continues to use it as part of
another trial.

At 4 feet 1 inch, she has already reached what her full adult height
could have been without vosoritide. In the early stages of her
treatment, she dreaded the daily injections. ``I got used to it,'' she
said, ``and I am growing.''

That has produced some milestones that others might take for granted.
When her family returned to a water park recently, she cleared the
4-foot height requirement to use a water slide for the first time.
``There's a real confidence that goes with those things,'' said her
father, Paul Cohen.

Megan Schimmel attributes much of her strength, compassion and empathy
to living with achondroplasia. She said that she wouldn't want to change
herself, and that she isn't going to change her 2-year-old daughter,
Lily, who also has the condition.

Image

Megan Schimmel with her daughter, Lily, 2, and husband,
Jeremy.Credit...Marissa Palma

``I can do everything that someone a foot taller can do, with minor
accommodations,'' Ms. Schimmel wrote in an email, adding that vosoritide
sent a message that those with achondroplasia ``are broken.''

Melissa Mills, of Jacksonville, Fla., who does not have the condition,
said she had already decided that her 4-year-old daughter, Eden, would
use vosoritide if it is approved by the F.D.A.

Yes, Mrs. Mills could get a \$900 custom bike so her daughter could ride
or teach her to drive a car with pedal extenders, but she will embrace
an alternative. ``With dwarfism, the world wasn't built for my child, so
if there is something I can do to help her navigate the world a little
bit better and on her own, I want to do it,'' she said.

After Eden's diagnosis, Mrs. Mills said, she joined every support group
she could find to learn about her daughter's condition. Her questions
about treatments that increased height whipped up tension. ``The more I
got involved in the groups and the L.P.A.'' --- the organization Little
People of America --- ``the more I pulled away.''

The debate over the drug resembles a decades-long discussion among deaf
people over cochlear implants, with some taking exception to the
suggestion that they should be ``fixed'' with the device.

Vosoritide, said Mark Povinelli, the L.P.A.'s president, ``is one of the
most divisive things that we've come across in our 63-year existence.''

The organization does not endorse specific treatments, but encourages
members to consider
\href{https://www.lpaonline.org/genetic-biotechnology-research-position-statement}{more
than height} in medical decisions. ``We want to show that you can have a
completely fulfilling life without having to worry about growth
velocity,'' said Mr. Povinelli, calling fixations on height a societal
issue.

When the group formed in 1957, there were no treatments in the United
States to increase height. The organization focused on changing how the
outside world saw people with the condition, emphasizing pride and
forming a community that now numbers 8,000.

In 2012, when BioMarin first presented vosoritide to the group, it
received a lackluster response, Mr. Povinelli said. An uneasy truce has
since developed.

``For better or for worse, as uncomfortable as it was, it put these
therapies front and center in everyone's mind,'' he said.

The drug --- whose price has not yet been set, though it is likely to be
costly --- could provide an alternative to arduous limb-lengthening
surgery, a process that involves cutting bone and extending a limb over
several weeks, said Marco Sessa, the president of the Association for
the Information and Study of Achondroplasia in Italy.

Image

Sarah Cohen at her home.Credit...Christina Simons for The New York Times

The surgery has not caught on in the United States as it has in Italy,
where more than 90 percent of people with achondroplasia undergo it,
adding a foot of height in some cases. ``It is a very painful, long
operation, so people think with the vosoritide we will finish the era of
leg-lengthening,'' Mr. Sessa said.

Dr. Watkins, the pediatric trainee in Auckland, said that she and her
husband were leaning toward treating their son with vosoritide. It isn't
so much about the height, she said, but the potential quality-of-life
benefits.

Still, Dr. Watkins wonders about the effects on Lachlan's relationships
with his peers who have dwarfism if he grows taller than they do. She
also worries about the potential for negative side effects that did not
show up in the trials.

For now, she will wait, if vosoritide is approved, to see how it
continues to perform. ``I don't think it is very straightforward,'' she
said.

Advertisement

\protect\hyperlink{after-bottom}{Continue reading the main story}

\hypertarget{site-index}{%
\subsection{Site Index}\label{site-index}}

\hypertarget{site-information-navigation}{%
\subsection{Site Information
Navigation}\label{site-information-navigation}}

\begin{itemize}
\tightlist
\item
  \href{https://help.nytimes3xbfgragh.onion/hc/en-us/articles/115014792127-Copyright-notice}{©~2020~The
  New York Times Company}
\end{itemize}

\begin{itemize}
\tightlist
\item
  \href{https://www.nytco.com/}{NYTCo}
\item
  \href{https://help.nytimes3xbfgragh.onion/hc/en-us/articles/115015385887-Contact-Us}{Contact
  Us}
\item
  \href{https://www.nytco.com/careers/}{Work with us}
\item
  \href{https://nytmediakit.com/}{Advertise}
\item
  \href{http://www.tbrandstudio.com/}{T Brand Studio}
\item
  \href{https://www.nytimes3xbfgragh.onion/privacy/cookie-policy\#how-do-i-manage-trackers}{Your
  Ad Choices}
\item
  \href{https://www.nytimes3xbfgragh.onion/privacy}{Privacy}
\item
  \href{https://help.nytimes3xbfgragh.onion/hc/en-us/articles/115014893428-Terms-of-service}{Terms
  of Service}
\item
  \href{https://help.nytimes3xbfgragh.onion/hc/en-us/articles/115014893968-Terms-of-sale}{Terms
  of Sale}
\item
  \href{https://spiderbites.nytimes3xbfgragh.onion}{Site Map}
\item
  \href{https://help.nytimes3xbfgragh.onion/hc/en-us}{Help}
\item
  \href{https://www.nytimes3xbfgragh.onion/subscription?campaignId=37WXW}{Subscriptions}
\end{itemize}
