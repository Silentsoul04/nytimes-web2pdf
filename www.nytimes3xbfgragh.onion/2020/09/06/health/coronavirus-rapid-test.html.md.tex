Sections

SEARCH

\protect\hyperlink{site-content}{Skip to
content}\protect\hyperlink{site-index}{Skip to site index}

\href{https://www.nytimes3xbfgragh.onion/section/health}{Health}

\href{https://myaccount.nytimes3xbfgragh.onion/auth/login?response_type=cookie\&client_id=vi}{}

\href{https://www.nytimes3xbfgragh.onion/section/todayspaper}{Today's
Paper}

\href{/section/health}{Health}\textbar{}Daily Coronavirus Testing at
Home? Many Experts Are Skeptical

\url{https://nyti.ms/335Tu1l}

\begin{itemize}
\item
\item
\item
\item
\item
\end{itemize}

\hypertarget{the-coronavirus-outbreak}{%
\subsubsection{\texorpdfstring{\href{https://www.nytimes3xbfgragh.onion/news-event/coronavirus?name=styln-coronavirus-national\&region=TOP_BANNER\&block=storyline_menu_recirc\&action=click\&pgtype=Article\&impression_id=acf11f50-f4b6-11ea-bb95-3783730020e3\&variant=undefined}{The
Coronavirus
Outbreak}}{The Coronavirus Outbreak}}\label{the-coronavirus-outbreak}}

\begin{itemize}
\tightlist
\item
  live\href{https://www.nytimes3xbfgragh.onion/2020/09/11/world/covid-19-coronavirus.html?name=styln-coronavirus-national\&region=TOP_BANNER\&block=storyline_menu_recirc\&action=click\&pgtype=Article\&impression_id=acf11f51-f4b6-11ea-bb95-3783730020e3\&variant=undefined}{Latest
  Updates}
\item
  \href{https://www.nytimes3xbfgragh.onion/interactive/2020/us/coronavirus-us-cases.html?name=styln-coronavirus-national\&region=TOP_BANNER\&block=storyline_menu_recirc\&action=click\&pgtype=Article\&impression_id=acf14660-f4b6-11ea-bb95-3783730020e3\&variant=undefined}{Maps
  and Cases}
\item
  \href{https://www.nytimes3xbfgragh.onion/interactive/2020/science/coronavirus-vaccine-tracker.html?name=styln-coronavirus-national\&region=TOP_BANNER\&block=storyline_menu_recirc\&action=click\&pgtype=Article\&impression_id=acf14661-f4b6-11ea-bb95-3783730020e3\&variant=undefined}{Vaccine
  Tracker}
\item
  \href{https://www.nytimes3xbfgragh.onion/2020/09/10/us/politics/fda-coronavirus-vaccine.html?name=styln-coronavirus-national\&region=TOP_BANNER\&block=storyline_menu_recirc\&action=click\&pgtype=Article\&impression_id=acf14662-f4b6-11ea-bb95-3783730020e3\&variant=undefined}{F.D.A.
  Regulators' Self-Defense}
\item
  \href{https://www.nytimes3xbfgragh.onion/2020/09/09/upshot/coronavirus-surprise-test-fees.html?name=styln-coronavirus-national\&region=TOP_BANNER\&block=storyline_menu_recirc\&action=click\&pgtype=Article\&impression_id=acf14663-f4b6-11ea-bb95-3783730020e3\&variant=undefined}{Surprise
  Test Fees}
\end{itemize}

Advertisement

\protect\hyperlink{after-top}{Continue reading the main story}

Supported by

\protect\hyperlink{after-sponsor}{Continue reading the main story}

\hypertarget{daily-coronavirus-testing-at-home-many-experts-are-skeptical}{%
\section{Daily Coronavirus Testing at Home? Many Experts Are
Skeptical}\label{daily-coronavirus-testing-at-home-many-experts-are-skeptical}}

The buzzy idea is impractical, critics said. And there isn't yet
real-world data to show it will work.

\includegraphics{https://static01.graylady3jvrrxbe.onion/images/2020/09/04/science/00VIRUS-TESTING-FAST1/00VIRUS-TESTING-FAST1-articleLarge.jpg?quality=75\&auto=webp\&disable=upscale}

\href{https://www.nytimes3xbfgragh.onion/by/katherine-j--wu}{\includegraphics{https://static01.graylady3jvrrxbe.onion/images/2020/08/11/reader-center/author-katherine-j-wu/author-katherine-j-wu-thumbLarge.png}}

By
\href{https://www.nytimes3xbfgragh.onion/by/katherine-j--wu}{Katherine
J. Wu}

\begin{itemize}
\item
  Published Sept. 6, 2020Updated Sept. 8, 2020
\item
  \begin{itemize}
  \item
  \item
  \item
  \item
  \item
  \end{itemize}
\end{itemize}

Over the past few weeks, a Harvard scientist has made headlines for a
bold idea to curb the spread of the coronavirus: rolling out so-called
antigen tests, a
\href{https://www.ncbi.nlm.nih.gov/pmc/articles/PMC7119943/}{decades-old}
underdog in testing technology, to tens of millions of Americans for
near-daily, at-home use.

These tests
\href{https://www.nytimes3xbfgragh.onion/2020/08/06/health/rapid-Covid-tests.html}{aren't
very good at picking up low-level infections}. But they're cheap,
convenient and fast, returning results in minutes. Real-time
information,
\href{https://www.nytimes3xbfgragh.onion/2020/07/03/opinion/coronavirus-tests.html}{argued}
Dr. Michael Mina, would be a lot better than the
\href{https://www.nytimes3xbfgragh.onion/2020/07/23/health/coronavirus-testing-supply-shortage.html}{long
delays} clogging the testing pipeline.

The fast-and-frequent approach to testing has captured the attention of
\href{https://www.microbe.tv/twiv/twiv-640/}{scientists} and journalists
around the world, as well as
\href{https://twitter.com/HHS_ASH/status/1300146630201610240}{top
officials} at the Department of Health and Human Services.

Deployed often enough and widely enough, speedy tests could
``\href{https://www.nytimes3xbfgragh.onion/2020/07/03/opinion/coronavirus-tests.html}{really
squash the virus},'' Dr. Mina said. ``I think it's crazy not to get
behind this.''

But more than a dozen experts said that near-ubiquitous antigen testing,
while
\href{https://www.medrxiv.org/content/10.1101/2020.06.22.20136309v2}{intriguing
in theory}, might not fly in practice ---~and is unlikely to be a
pandemic panacea. In addition to posing herculean logistical hurdles,
they said, the plan hinges on broad buy-in and compliance from a country
full of people who have grown increasingly disillusioned with testing
for the virus. And that's assuming that rapid tests
\href{https://jcm.asm.org/content/early/2020/08/07/JCM.01695-20}{can
achieve their intended purpose at all}.

``We are open to thinking outside the box and coming up with new ways to
handle this pandemic,'' said Esther Babady, director of the clinical
microbiology service at Memorial Sloan Kettering Cancer Center in New
York. But antigen tests that could work at home have yet to enter the
market, she said.

And no one has yet done a rigorous study to show that fast-and-frequent
trumps sensitive-but-slow in the real world, she said: ``The data for
that is what's missing.''

Although fast-and-frequent testing could work, what's been put forth so
far about the approach has been ``largely aspirational, and we need to
check it against reality,'' said Dr. Alexander McAdam, director of the
infectious diseases diagnostic laboratory at Boston Children's Hospital,
who recently co-authored
\href{https://jcm.asm.org/content/early/2020/08/24/JCM.02225-20}{an
article on pandemic testing strategies} in the Journal of Clinical
Microbiology.

Most of the coronavirus tests run so far rely on a laboratory technique
called PCR, long considered the gold standard of infectious disease
diagnostics because it can pick up even very small amounts of genetic
material from germs like the coronavirus.

But sputtering supply chains have compromised efforts to collect, ship
and process samples for PCR, driving delays in turnaround times. The
longer the wait, the less useful the result. PCR also isn't cheap or
user-friendly, making it an unlikely candidate for widespread home use.

\hypertarget{latest-updates-the-coronavirus-outbreak}{%
\section{\texorpdfstring{\href{https://www.nytimes3xbfgragh.onion/2020/09/11/world/covid-19-coronavirus.html?action=click\&pgtype=Article\&state=default\&region=MAIN_CONTENT_1\&context=storylines_live_updates}{Latest
Updates: The Coronavirus
Outbreak}}{Latest Updates: The Coronavirus Outbreak}}\label{latest-updates-the-coronavirus-outbreak}}

Updated 2020-09-12T04:56:54.924Z

\begin{itemize}
\tightlist
\item
  \href{https://www.nytimes3xbfgragh.onion/2020/09/11/world/covid-19-coronavirus.html?action=click\&pgtype=Article\&state=default\&region=MAIN_CONTENT_1\&context=storylines_live_updates\#link-dfb8a16}{Fauci
  cautions the virus could disrupt life in the U.S. until `maybe even
  towards the end of 2021.'}
\item
  \href{https://www.nytimes3xbfgragh.onion/2020/09/11/world/covid-19-coronavirus.html?action=click\&pgtype=Article\&state=default\&region=MAIN_CONTENT_1\&context=storylines_live_updates\#link-7104d154}{From
  Asia to Africa, China promotes its vaccine candidates to win friends.}
\item
  \href{https://www.nytimes3xbfgragh.onion/2020/09/11/world/covid-19-coronavirus.html?action=click\&pgtype=Article\&state=default\&region=MAIN_CONTENT_1\&context=storylines_live_updates\#link-393ad215}{The
  other way the virus will kill: hunger.}
\end{itemize}

\href{https://www.nytimes3xbfgragh.onion/2020/09/11/world/covid-19-coronavirus.html?action=click\&pgtype=Article\&state=default\&region=MAIN_CONTENT_1\&context=storylines_live_updates}{See
more updates}

More live coverage:
\href{https://www.nytimes3xbfgragh.onion/live/2020/09/11/business/stock-market-today-coronavirus?action=click\&pgtype=Article\&state=default\&region=MAIN_CONTENT_1\&context=storylines_live_updates}{Markets}

The at-home arena is where antigen tests could shine, Dr. Mina said. At
their simplest, they might function much like a pregnancy test,
analyzing bodily fluid and spitting out a result within a few minutes,
no health workers or fancy machines necessary.

As Dr. Mina sees it, these tests could be crafted from materials as
cheap as cardboard and be shipped like rations to communities around the
country. They'd act as bouncers at the entrances to schools or
workplaces, and allow Americans to check themselves at home for the
coronavirus several times a week, perhaps even daily.

But achieving that reality would require an antigen test that is not yet
approved for widespread use, and the infrastructure to manufacture it en
masse. Only four antigen tests so far have received emergency approval
from the Food and Drug Administration, and are intended to be used by
health care workers on people who recently showed symptoms. All of them
also rely on swabs to collect test samples, and three require somewhat
bulky and expensive machines to read out results.

``We just don't have tests ready to occupy this space right now,'' Dr.
McAdam said.

\href{https://www.nytimes3xbfgragh.onion/2020/09/02/us/politics/covid-testing.html}{Several
companies} have other
\href{https://www.nytimes3xbfgragh.onion/2020/07/06/health/fast-coronavirus-tests.html}{rapid
tests in development}. But there's no guarantee newcomers will meet
F.D.A. standards. And the past few months have clearly demonstrated that
no test is impervious to shortages.

``There's no reason to believe that the supply chain issues we've
encountered with all other coronavirus testing will not still be an
issue here, too,'' said April Abbott, microbiology director at Deaconess
Health System in Indiana. ``We can't build new product lines
overnight.''

\includegraphics{https://static01.graylady3jvrrxbe.onion/images/2020/09/04/science/00VIRUS-TESTING-FAST2/merlin_175367826_447e84db-1312-4752-971d-54d781dc53ea-articleLarge.jpg?quality=75\&auto=webp\&disable=upscale}

Experts also noted that antigen tests aren't great at sussing out small
amounts of the coronavirus, which means they're far more likely to miss
a case that a technique like PCR would catch. Some antigen tests
\href{https://www.cdc.gov/flu/professionals/diagnosis/clinician_guidance_ridt.htm\#:~:text=References-,Background,of\%20RIDTs\%20are\%20commercially\%20available.}{catch
only half the infections they look for}. And while some new products
perform better in the lab, advertised accuracy rates will almost
certainly drop when used at home, said Linoj Samuel, a medical
microbiologist at Henry Ford Health System in Michigan.

(Some have argued that PCR
\href{https://www.nytimes3xbfgragh.onion/2020/08/29/health/coronavirus-testing.html}{may
actually be too sensitive in some settings}, picking up on scraps of
innocuous coronavirus genetic material in patients who are no longer
sick; antigen testing could circumvent this.)

Dr. Mina argues that dips in quality could be overcome with quantity:
Near-daily tests would be able to identify infections on the cusp of
contagiousness faster than the backlogged PCR pipeline could, helping
people self-isolate in the nick of time. From a public health
perspective, what matters most is finding people at the peak of
infection --- something that even antigen tests should be able to do
with high levels of accuracy, he said.

But researchers don't yet know how much virus someone has to have in
their body to be contagious --- the amount almost certainly varies from
person to person. And there will inevitably be exceptions to the ``more
virus, more transmission'' trend.

``We just don't have any proof that a negative test result means you're
not infectious,'' said Susan Butler-Wu, a clinical microbiologist at the
University of Southern California's Keck School of Medicine. Some
antigen tests miss
\href{https://www.ncbi.nlm.nih.gov/pmc/articles/PMC7383555/}{up to 18
percent of cases} shown by PCR to involve high levels of the
coronavirus.

The opposite issue, false positives, are rarer with antigen tests, but
they do happen. In July, dozens of positive antigen tests that had
officials in Manchester, Vt., bracing for an outbreak
\href{https://www.burlingtonfreepress.com/story/news/2020/07/23/covid-19-testing-how-versions-differ-speed-use-and-accuracy-coronavirus/5446176002/}{turned
out to be errors}. And in August, Gov. Mike DeWine of Ohio
\href{https://www.nytimes3xbfgragh.onion/2020/08/06/us/mike-dewine-coronavirus.html}{tested
positive for the coronavirus} by an antigen test, only to test negative
\href{https://www.nytimes3xbfgragh.onion/2020/08/09/health/covid-testing.html}{thrice
in a row by PCR}.

In regions where the virus has infected only a few people, the number of
false positives could end up dwarfing the number of true positives.

Dr. McAdam said that widely deploying a test with imperfect specificity
to a region where the virus is scarce ``is a bad idea, and I'll die on
that hill.''

High rates of inaccurate results, coupled with continued confusion about
the deluge of new coronaviruses tests, could fuel public skepticism of
science at a particularly precarious time, said Amanda Harrington,
director of the clinical microbiology laboratory at Loyola University
Medical Center in Illinois.

In the past six months alone, coronavirus tests have been alternately
billed as game-changers and national embarrassments, seeding a sense of
perpetual whiplash among testing experts.

``My own family is telling me they're not sure what to believe,'' Dr.
Harrington said. ``You're eroding confidence to the point where people
don't trust it.''

And a nation of people wary of tests will probably be less likely to
take them regularly, even if they're available at home.

Uma Karmarkar, a human behavior expert at the University of California,
San Diego, said it's possible that compliance would be low for the
fast-and-frequent approach. She pointed to the example of daily
medications, like birth control pills, as well as spotty use of masks.

``Even when there's a vested interest, there's slippage,'' she said.
With near-daily testing, even cheap products could add up to big bills,
further disincentivizing use. (Dr. Mina said the federal government
should foot the bill to avoid that issue.)

A subset of people might still adopt the fast-and-frequent approach with
enthusiasm, Dr. Karmarkar said. But that could be a skewed sector of the
population, such as those who are already more inclined to trust the
medical system, and could exacerbate the
\href{https://www.nytimes3xbfgragh.onion/interactive/2020/07/05/us/coronavirus-latinos-african-americans-cdc-data.html}{pandemic's
health inequities}.

Until more data is gathered to support the fast-and-frequent approach,
Dr. Samuel proposed a tentative middle ground. Schools,
\href{https://jamanetwork.com/journals/jamanetworkopen/fullarticle/2768923}{universities}
and workplaces may be good candidates for regular antigen testing, for
example, whereas hospitals and other medical care settings would still
prioritize PCR.

``The whole idea is to use the right test for the right patient at the
right time,'' Dr. Babady said.

Dr. Mina agreed, noting that PCR remains crucial for diagnosing sick
patients --- a situation that calls for the most sensitive test, so the
right treatments can be administered.

Still, he remains optimistic that the fast-and-frequent strategy could
make a major dent in the nation's coronavirus catastrophe. That should
be incentive enough, he said: ``I truly believe people will want to use
these tests.''

Advertisement

\protect\hyperlink{after-bottom}{Continue reading the main story}

\hypertarget{site-index}{%
\subsection{Site Index}\label{site-index}}

\hypertarget{site-information-navigation}{%
\subsection{Site Information
Navigation}\label{site-information-navigation}}

\begin{itemize}
\tightlist
\item
  \href{https://help.nytimes3xbfgragh.onion/hc/en-us/articles/115014792127-Copyright-notice}{©~2020~The
  New York Times Company}
\end{itemize}

\begin{itemize}
\tightlist
\item
  \href{https://www.nytco.com/}{NYTCo}
\item
  \href{https://help.nytimes3xbfgragh.onion/hc/en-us/articles/115015385887-Contact-Us}{Contact
  Us}
\item
  \href{https://www.nytco.com/careers/}{Work with us}
\item
  \href{https://nytmediakit.com/}{Advertise}
\item
  \href{http://www.tbrandstudio.com/}{T Brand Studio}
\item
  \href{https://www.nytimes3xbfgragh.onion/privacy/cookie-policy\#how-do-i-manage-trackers}{Your
  Ad Choices}
\item
  \href{https://www.nytimes3xbfgragh.onion/privacy}{Privacy}
\item
  \href{https://help.nytimes3xbfgragh.onion/hc/en-us/articles/115014893428-Terms-of-service}{Terms
  of Service}
\item
  \href{https://help.nytimes3xbfgragh.onion/hc/en-us/articles/115014893968-Terms-of-sale}{Terms
  of Sale}
\item
  \href{https://spiderbites.nytimes3xbfgragh.onion}{Site Map}
\item
  \href{https://help.nytimes3xbfgragh.onion/hc/en-us}{Help}
\item
  \href{https://www.nytimes3xbfgragh.onion/subscription?campaignId=37WXW}{Subscriptions}
\end{itemize}
