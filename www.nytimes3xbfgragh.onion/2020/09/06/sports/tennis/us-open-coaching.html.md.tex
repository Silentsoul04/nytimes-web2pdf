Sections

SEARCH

\protect\hyperlink{site-content}{Skip to
content}\protect\hyperlink{site-index}{Skip to site index}

\href{https://www.nytimes3xbfgragh.onion/section/sports/tennis}{Tennis}

\href{https://myaccount.nytimes3xbfgragh.onion/auth/login?response_type=cookie\&client_id=vi}{}

\href{https://www.nytimes3xbfgragh.onion/section/todayspaper}{Today's
Paper}

\href{/section/sports/tennis}{Tennis}\textbar{}Listen Closely, and You
Will Hear Coaching During U.S. Open Matches

\url{https://nyti.ms/3bAShTp}

\begin{itemize}
\item
\item
\item
\item
\item
\end{itemize}

Advertisement

\protect\hyperlink{after-top}{Continue reading the main story}

Supported by

\protect\hyperlink{after-sponsor}{Continue reading the main story}

\hypertarget{listen-closely-and-you-will-hear-coaching-during-us-open-matches}{%
\section{Listen Closely, and You Will Hear Coaching During U.S. Open
Matches}\label{listen-closely-and-you-will-hear-coaching-during-us-open-matches}}

When an umpire called out Serena Williams over coaching in the 2018 U.S.
Open final, it erupted into a scene. This year, with no fans in the
seats, there is chatter galore between coaches and players, even though
it breaks the rules.

\includegraphics{https://static01.graylady3jvrrxbe.onion/images/2020/09/06/sports/06usopen-coaching01/merlin_176530128_8a15e6a5-d5d3-478c-89c4-c08c7db6d356-articleLarge.jpg?quality=75\&auto=webp\&disable=upscale}

\href{https://www.nytimes3xbfgragh.onion/by/matthew-futterman}{\includegraphics{https://static01.graylady3jvrrxbe.onion/images/2020/02/24/reader-center/author-matthew-futterman/author-matthew-futterman-thumbLarge.png}}

By
\href{https://www.nytimes3xbfgragh.onion/by/matthew-futterman}{Matthew
Futterman}

\begin{itemize}
\item
  Sept. 6, 2020
\item
  \begin{itemize}
  \item
  \item
  \item
  \item
  \item
  \end{itemize}
\end{itemize}

The prohibition against coaching during a match at the United States
Open is supposed to be straightforward.

``Coaching is considered to be communication, advice or instruction of
any kind and by any means to a player,'' Article III, Section L states
on page 44 of the 2020 Official Grand Slam Rule Book.

This year, not so much.

At this tournament, the strict prohibition is largely being ignored, two
years after it prompted one of the most contentious moments in recent
U.S. Open history --- a harsh confrontation at the 2018 women's final
when
\href{https://www.nytimes3xbfgragh.onion/2018/09/10/sports/the-coaching-rule-that-upset-serena-williams-explained.html}{the
chair umpire issued a code violation to Serena Williams} for receiving
coaching.

Because of
\href{https://www.nytimes3xbfgragh.onion/2020/08/31/sports/tennis/us-open.html}{the
lack of ticket-buying spectators}, coaches have been among the few
people watching matches, and chatter with their players has become one
of the unforeseen consequences of staging a Grand Slam tournament in the
middle of the coronavirus pandemic.

The stadiums are silent and nearly empty. The players have no one else
to turn to for support during the pressure cooker of one of the world's
biggest tournaments. There are fewer umpires to police violations. And,
in some cases, masks hide the moving lips of each coach. The resulting
combination is plenty of communication --- sometimes in both directions
--- between the players and the people most important to their success.

\includegraphics{https://static01.graylady3jvrrxbe.onion/images/2020/09/06/sports/06usopen-coaching/merlin_176410023_4ec3a5f2-b3bd-4b45-b924-119a53c18a1f-articleLarge.jpg?quality=75\&auto=webp\&disable=upscale}

``What do I do?'' Frances Tiafoe said as he grabbed a towel placed a few
feet from his coach, Wayne Ferreira, in the middle of his five-set
thriller in the second round against John Millman of Australia.

At that moment, Tiafoe was getting drilled. Ferreira's response could
not be heard from six feet behind him. But two sets later, as Tiafoe
chased down two would-be winners in the corners and closed down on the
comeback win, he returned to the towel and Ferreira.

``Is that what you mean by playing some defense?'' Tiafoe said.

Ferreira responded: ``D. D. D.''

Asked later about his running conversation with Ferreira during the
match, Tiafoe said it had been a way to both pass the time and search
for an edge.

``Some of it's boredom; some of it's trying to get information,'' he
said. Tiafoe explained that Ferreira had criticized him for not working
hard to extend points. ``He's like, `You don't play defense.' I was
like, `Man, I'm one of the guys with the quickest wheels here.' He's
like, `Yeah, but you always bail out on defense.' Then I was like, `How
is that for defense, Wayne?' Kind of like throwing some shade at him.''

That all of this is unfolding at the U.S. Open is a bit ironic.

In 2018, the chair umpire punished Serena Williams for being coached by
Patrick Mouratoglou during the final against Naomi Osaka. Williams did
not appreciate being accused of cheating and exploded later in the match
when she was penalized a point for slamming her racket to the ground.
Mouratoglou later admitted he had signaled to Williams during play.

All of this speaks to a larger question for the sport: Where is the line
between encouragement and coaching? Grand Slam singles, after all, is
meant to be a solo performance. Yet other solo athletes have support.

Boxers have corners. Golfers have caddies. In tennis, each player
attempts to solve the puzzle of a match alone.

\href{https://www.nytimes3xbfgragh.onion/2019/08/23/sports/tennis/us-open-coaching-serena-williams.html}{After
the Williams incident} in 2018, the tennis world debated whether the
prohibition on coaching was outdated. This year, the WTA Tour began
allowing coaching during its events because it was too hard to police,
though coaching remains prohibited for women during the Grand Slams.

Image

In 2018, Serena Williams's dispute with the chair umpire Carlos Ramos
began when he punished her for receiving coaching during the U.S. Open
final.Credit...Chang W. Lee/The New York Times

Soeren Friemel, the tournament referee, said the tournament had not
changed its interpretation of the rule. Umpires have issued two coaching
violations so far this tournament, but they allow communication that
falls into the under ``encouragement'' in any language. \emph{Come on!
Vámonos! Allez! Allez!}

``We are not expecting coaches to sit in the seats and not move, because
you can also interpret communication as clapping and nodding, or raising
a fist,'' Friemel said. ``We want to see that they are into the match
and motivating the players --- this kind of communication that would be
acceptable, but influencing the way a player plays is not.''

Whether that is happening can be nearly impossible to know. The Houston
Astros showed that telling a batter what pitch was coming was as simple
as banging on a garbage can. There is no evidence that a coach and a
player here have worked out a tennis version of that (\emph{If I clap
twice, serve to the forehand and go to the net).} But at the empty
venues this year, players say they can hear nearly everything their
coaches are saying, and they often talk back.

Andy Murray was down two sets and fighting for survival during his
first-round match last week when he blasted a return of Yoshihito
Nishioka's serve into the middle of the net. Murray immediately turned
to look his coach, Jamie Delgado, who was sitting just a few rows from
the court in Arthur Ashe Stadium and talking at Murray whenever they
were on the same side of the court.

Murray then winked at Delgado and said, ``Next time.''

Brandon Nakashima, the 19-year-old American who lost his second round
match against Alexander Zverev of Germany in four sets Wednesday, got an
earful from his coaches: Pat Cash, the 1987 Wimbledon champion, and
Dusan Vemic of Serbia, who sat a few rows up from the court at Louis
Armstrong Stadium.

``Good playing, mate,'' Cash, an Australian, told Nakashima after a deft
volley and a few other solid shots. A well-placed serve brought a ``keep
it coming now.''

``Great acceleration there,'' Vemic said after Nakashima crushed a
forehand.

While the urgings could not deliver Nakashima a win --- he lost, 7-5,
6-7 (8), 6-3, 6-1 --- he said hearing his coaches had helped him remain
competitive in his first match against a top-10 player.

``They are both so knowledgeable,'' Nakashima said. ``They try not to
say too much. They know my personality. I could definitely hear them
giving me some confidence and giving me some energy when I needed it.''

The opportunities were especially ripe on the field courts, where many
players had early-round matches. The field courts are surrounded by
bleachers that allow coaches to sit just a few feet from the players.
Also, because the U.S. Open is using the electronic Hawk-Eye system to
call the lines in all but the two stadium courts, there are no line
officials on most courts who might normally pick up on the communication
and report it to the chair umpire.

During a first-round match against Cameron Norrie of Britain, the
Argentine baseliner Diego Schwartzman repeatedly wandered close to his
coach, Juan Chela, who spoke to Schwartzman in Spanish. Chela wore a
mask even though he did not have to do so in his designated coaching
seat. No one could see his lips move. The chair umpire was on the other
side of the court. Schwartzman lost the match in five sets.

Of course there is only so much a coach can convey in quick snippets.
There were countless tips David Kass would have liked to deliver as he
watched his star student, the 16-year-old Katrina Scott, as she battled
through her three-set second-round match against her fellow American
Amanda Anisimova. There were even moments when Kass cupped his hands
around his mouth and almost started to speak, then stopped.

What advice would he have given her?

``Being more aggressive with the forehand, serving to specific places,
looking for serves coming at one side or another, because there were
some tells,'' Kass said after the match.

Instead, he stuck with the standard inspirational messages. ``You got to
go to work,'' he said as she fell behind in the third set.

Scott heard it, even though she came up short.

``I'm definitely listening,'' Scott said. ``I could definitely hear that
support, always, and I really appreciated it, especially with no fans.''

So did Varvara Gracheva. On Wednesday, Gracheva, a Russian 20-year-old,
was down a set and 5-1, 15-0 to Kristina Mladenovic, yet somehow climbed
back to win. It helped that during half of the games, her towel was
about two feet from her coach, Bruce Gorregues. They spoke quietly in
French.

``We are a team, we are working together,'' Gracheva said. ``He was
giving me great support. He just tried his best to make me play.''

``It's my job to push,'' Gorregues explained. ``The rest comes from
her.''

Advertisement

\protect\hyperlink{after-bottom}{Continue reading the main story}

\hypertarget{site-index}{%
\subsection{Site Index}\label{site-index}}

\hypertarget{site-information-navigation}{%
\subsection{Site Information
Navigation}\label{site-information-navigation}}

\begin{itemize}
\tightlist
\item
  \href{https://help.nytimes3xbfgragh.onion/hc/en-us/articles/115014792127-Copyright-notice}{©~2020~The
  New York Times Company}
\end{itemize}

\begin{itemize}
\tightlist
\item
  \href{https://www.nytco.com/}{NYTCo}
\item
  \href{https://help.nytimes3xbfgragh.onion/hc/en-us/articles/115015385887-Contact-Us}{Contact
  Us}
\item
  \href{https://www.nytco.com/careers/}{Work with us}
\item
  \href{https://nytmediakit.com/}{Advertise}
\item
  \href{http://www.tbrandstudio.com/}{T Brand Studio}
\item
  \href{https://www.nytimes3xbfgragh.onion/privacy/cookie-policy\#how-do-i-manage-trackers}{Your
  Ad Choices}
\item
  \href{https://www.nytimes3xbfgragh.onion/privacy}{Privacy}
\item
  \href{https://help.nytimes3xbfgragh.onion/hc/en-us/articles/115014893428-Terms-of-service}{Terms
  of Service}
\item
  \href{https://help.nytimes3xbfgragh.onion/hc/en-us/articles/115014893968-Terms-of-sale}{Terms
  of Sale}
\item
  \href{https://spiderbites.nytimes3xbfgragh.onion}{Site Map}
\item
  \href{https://help.nytimes3xbfgragh.onion/hc/en-us}{Help}
\item
  \href{https://www.nytimes3xbfgragh.onion/subscription?campaignId=37WXW}{Subscriptions}
\end{itemize}
