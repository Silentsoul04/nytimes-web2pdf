Sections

SEARCH

\protect\hyperlink{site-content}{Skip to
content}\protect\hyperlink{site-index}{Skip to site index}

\href{https://www.nytimes3xbfgragh.onion/section/politics}{Politics}

\href{https://myaccount.nytimes3xbfgragh.onion/auth/login?response_type=cookie\&client_id=vi}{}

\href{https://www.nytimes3xbfgragh.onion/section/todayspaper}{Today's
Paper}

\href{/section/politics}{Politics}\textbar{}Joe Biden's China Journey

\url{https://nyti.ms/35fB2pB}

\begin{itemize}
\item
\item
\item
\item
\item
\end{itemize}

\begin{itemize}
\item
  \href{https://www.nytimes3xbfgragh.onion/interactive/2020/09/08/us/elections/results-new-hampshire-primary-elections.html?action=click\&pgtype=Article\&state=default\&region=TOP_BANNER\&context=storylines_menu}{New
  Hampshire Results}
\item
  \href{https://www.nytimes3xbfgragh.onion/live/2020/09/08/us/trump-vs-biden?action=click\&pgtype=Article\&state=default\&region=TOP_BANNER\&context=storylines_menu}{Election
  Updates}
\item
  \href{https://www.nytimes3xbfgragh.onion/interactive/2020/us/elections/election-states-biden-trump.html?action=click\&pgtype=Article\&state=default\&region=TOP_BANNER\&context=storylines_menu}{Paths
  to 270}
\item
  \href{https://www.nytimes3xbfgragh.onion/interactive/2020/08/31/us/politics/vote-by-mail-deadlines.html?action=click\&pgtype=Article\&state=default\&region=TOP_BANNER\&context=storylines_menu}{Voting
  by Mail}
\item
  \href{https://www.nytimes3xbfgragh.onion/interactive/2019/us/elections/2020-presidential-election-calendar.html?action=click\&pgtype=Article\&state=default\&region=TOP_BANNER\&context=storylines_menu}{Key
  Dates}
\item
  \href{https://www.nytimes3xbfgragh.onion/newsletters/politics?action=click\&pgtype=Article\&state=default\&region=TOP_BANNER\&context=storylines_menu}{Politics
  Newsletter}
\end{itemize}

Advertisement

\protect\hyperlink{after-top}{Continue reading the main story}

Supported by

\protect\hyperlink{after-sponsor}{Continue reading the main story}

The Long Run

\hypertarget{joe-bidens-china-journey}{%
\section{Joe Biden's China Journey}\label{joe-bidens-china-journey}}

As a United States senator, he spoke of transforming China through
trade. As a presidential candidate two decades later, he denounces it as
a ``dictatorship.''

\includegraphics{https://static01.graylady3jvrrxbe.onion/images/2020/09/07/us/politics/00dc-biden-china1/merlin_175668705_884f4e78-c870-4464-aa0a-af7bc91cc75d-articleLarge.jpg?quality=75\&auto=webp\&disable=upscale}

\href{https://www.nytimes3xbfgragh.onion/by/edward-wong}{\includegraphics{https://static01.graylady3jvrrxbe.onion/images/2018/09/24/multimedia/author-edward-wong/author-edward-wong-thumbLarge-v5.png}}\href{https://www.nytimes3xbfgragh.onion/by/michael-crowley}{\includegraphics{https://static01.graylady3jvrrxbe.onion/images/2019/10/25/reader-center/author-michael-crowley/author-michael-crowley-thumbLarge-v2.png}}\href{https://www.nytimes3xbfgragh.onion/by/ana-swanson}{\includegraphics{https://static01.graylady3jvrrxbe.onion/images/2018/12/10/multimedia/author-ana-swanson/author-ana-swanson-thumbLarge.png}}

By \href{https://www.nytimes3xbfgragh.onion/by/edward-wong}{Edward
Wong},
\href{https://www.nytimes3xbfgragh.onion/by/michael-crowley}{Michael
Crowley} and
\href{https://www.nytimes3xbfgragh.onion/by/ana-swanson}{Ana Swanson}

\begin{itemize}
\item
  Sept. 6, 2020
\item
  \begin{itemize}
  \item
  \item
  \item
  \item
  \item
  \end{itemize}
\end{itemize}

\href{https://cn.nytimes3xbfgragh.onion/usa/20200908/biden-china/}{阅读简体中文版}\href{https://cn.nytimes3xbfgragh.onion/usa/20200908/biden-china/zh-hant/}{閱讀繁體中文版}

WASHINGTON --- On a steamy August day along the Chinese coast, Senator
Joseph R. Biden Jr. stepped off a minibus at a seaside compound for a
series of unusual meetings with China's Communist Party leaders.

At a lunch banquet, Mr. Biden and three other senators argued with
Chinese officials about what the O.J. Simpson trial had revealed about
the integrity of the U.S. legal system. When the senators met afterward
with the party secretary, Jiang Zemin, they sparred over that and other
\href{https://www.csmonitor.com/2001/0810/p6s1-wosc.html}{thorny
issues}: missile technology proliferation, human rights and Taiwan.

But Mr. Biden, leading his first overseas trip as the chairman of the
Senate Foreign Relations Committee, was also there
\href{https://www.nytimes3xbfgragh.onion/2012/07/22/world/asia/chinas-communist-elders-take-backroom-intrigue-beachside.html}{in
Beidaihe} in 2001 to help usher in an important era in America's
relationship with China --- the building of a commercial link that would
allow the Communist nation entry into the World Trade Organization.

``The United States welcomes the emergence of a prosperous, integrated
China on the global stage, because we expect this is going to be a China
that plays by the rules,'' Mr. Biden told Mr. Jiang, recalled
\href{https://mansfieldfdn.org/about/mansfield-foundation-board/frank-jannuzi/}{Frank
Jannuzi}, the Senate aide who organized the trip and took notes at Mr.
Biden's side.

The senator traveled days later to a dirt-road village near the Great
Wall. Seven thousand miles from Delaware, his adopted home state, Mr.
Biden glad-handed bemused locals like a candidate, even taking holy
communion from a Roman Catholic priest. He returned to Washington seeing
more promise than peril, offering reporters the same message he had
delivered to Chinese leaders: The United States welcomed China's
emergence ``as a great power, because great powers adhere to
international norms in the areas of nonproliferation, human rights and
trade.''

Two decades later, China has emerged as a great power --- and, in the
eyes of many Americans, a dangerous rival. Republicans and Democrats say
it has exploited the global integration that Mr. Biden and many other
officials supported.

The 2020 election has been partly defined by what much of Washington
sees as a kind of new Cold War. And as Mr. Biden faces
\href{https://www.nytimes3xbfgragh.onion/2020/05/15/us/politics/trump-ads-joe-biden.html}{fierce
campaign attacks} from President Trump, his language on China points to
a drastic shift in thinking.

Mr. Biden calls Xi Jinping, the authoritarian Chinese leader, a
``thug.'' He has threatened, if elected, to impose ``swift economic
sanctions'' if China tries to silence American citizens and companies.
``The United States does need to get tough on China,'' he wrote this
winter in an
\href{https://www.foreignaffairs.com/articles/united-states/2020-01-23/why-america-must-lead-again}{essay
in Foreign Affairs}. Mr. Biden now sees the country as a top strategic
challenge, according to interviews with more than a dozen of his
advisers and foreign policy associates, and
\href{https://www.washingtonpost.com/graphics/politics/policy-2020/foreign-policy/}{his
own words}.

\includegraphics{https://static01.graylady3jvrrxbe.onion/images/2020/09/07/us/politics/00dc-biden-china2/merlin_175668603_d5daa283-dba1-4fd6-8c52-a86212df2229-articleLarge.jpg?quality=75\&auto=webp\&disable=upscale}

Mr. Biden's 20-year road from wary optimism to condemnation --- while
still straining for some cooperation --- is emblematic of the arc of
U.S.-China relations, which have deteriorated to an unstable,
potentially explosive state. But as Mr. Trump denounces what he
describes as failures by the Washington establishment on China, Mr.
Biden, an avatar of that establishment, is not recanting his past
enthusiasm for engagement.

In a
\href{https://www.foreignaffairs.com/articles/china/2018-02-13/china-reckoning}{Foreign
Affairs essay} in 2018, two former Obama administration officials who
advise Mr. Biden, Kurt M. Campbell and Ely Ratner, said both Democratic
and Republican administrations had been guilty of fundamental policy
missteps on China.

``Across the ideological spectrum, we in the U.S. foreign policy
community have remained deeply invested in expectations about China ---
about its approach to economics, domestic politics, security and global
order --- even as evidence against them has accumulated,'' they wrote.
``The policies built on such expectations have failed to change China in
the ways we intended or hoped.''

While refraining from admitting fault in his previous views, Mr. Biden
speaks these days not of transforming China but of restoring the United
States, according to his policy statements and interviews with his
aides. They say the United States must regain its role as a leader of
liberal values and economic innovation, and that will give Washington
the standing to rally like-minded nations to constrain China.

Among Mr. Biden's priorities are rebuilding alliances and reasserting a
global defense of democracy, which Mr. Trump has eroded, they say. Mr.
Biden's ``Build Back Better''
\href{https://www.nytimes3xbfgragh.onion/2020/07/09/us/politics/biden-buy-american.html}{economic
plan} promotes investments in U.S. industries and research, partly to
compete with China. And he sees some areas where Washington and Beijing
can
\href{https://www.foreignaffairs.com/articles/united-states/2020-01-23/why-america-must-lead-again}{cooperate}:
climate change, health security and nonproliferation.

But relations are at their lowest point since the re-establishment of
formal ties in 1979. Chinese officials have accelerated their
authoritarian policies, and Beijing's assertions of power in Hong Kong,
the South China Sea and elsewhere are seen in Washington as open
defiance.

While
\href{https://www.nytimes3xbfgragh.onion/2020/07/25/world/asia/us-china-trump-xi.html}{Trump
administration hawks} aim to set the two nations on a long-term course
of confrontation, Mr. Trump himself
\href{https://www.nytimes3xbfgragh.onion/2020/06/18/us/politics/trump-china-bolton.html}{vacillates
wildly on China}. He halted his
\href{https://www.nytimes3xbfgragh.onion/2020/01/22/business/economy/trade-economy.html}{damaging
trade war} this year, called Mr. Xi
``\href{https://www.scmp.com/week-asia/opinion/article/3046586/donald-trump-really-xi-jinpings-friend-phase-one-us-china-trade}{a
very, very good friend}'' and expressed
``\href{https://twitter.com/realDonaldTrump/status/1243407157321560071?s=20}{much
respect!}'' on Twitter. But Mr. Trump now talks angrily of the ``China
virus,'' referring to the coronavirus outbreak.

Mr. Biden calls for a steadier approach, but he has no easy answers for
how two superpowers with intertwined economies can deal with their
ideological differences. In an interview in May with The New York Times,
he said he \href{http://en.people.cn/90883/7573905.html}{met with Mr.
Xi} repeatedly in 2011 and 2012 to try to figure out whether it was
possible to have ``a competitive relationship with China without it
being a bellicose relationship, without it being a relationship based on
force.''

\hypertarget{to-change-china}{%
\subsection{To Change China}\label{to-change-china}}

Image

Mr. Biden giving a passionate speech about nuclear warfare and Chinese
espionage to a group of high school students on the steps of the Capitol
in 1999.Credit...Stephen Crowley/The New York Times

Mr. Biden says he has had a ``long interest in the evolving nature of
the Chinese Communist Party'' from his first visit to the country ``as a
kid in the Senate'' in April 1979, as part of the first U.S.
congressional delegation to the country since China's Communist
revolution in 1949. He met with the country's leader, Deng Xiaoping, who
was then beginning to transform China's command economy with market
reforms.

Hosting Chinese officials as the vice president in May 2011, Mr. Biden
\href{https://obamawhitehouse.archives.gov/the-press-office/2011/05/09/remarks-vice-president-joe-biden-opening-session-us-china-strategic-econ}{recalled
that trip} fondly. While acknowledging a ``debate'' on the question, he
said he ``believed then what I believe now: that a rising China is a
positive, positive development, not only for China but for America and
the world writ large.''

But as the country boomed over the decades, Mr. Biden was also a
frequent critic, especially on human rights.

Outraged by the 1989 crackdown against protesters around Tiananmen
Square, he introduced legislation to create a federally funded news
media network to promote democratic values within the country. Mr. Biden
realized China was a ``brutal system,'' said James P. Rubin, a Senate
Foreign Relations Committee aide who later served as a State Department
spokesman. The station went online in 1996 as Radio Free Asia and
operates to this day.

Mr. Biden also saw limits to what the United States could realistically
demand. In 1991, as Congress debated granting China favored-nation
trading status, he
\href{https://www.nytimes3xbfgragh.onion/1991/11/15/opinion/IHT-its-time-to-rope-inchinas-rogueelephant-act.html}{acknowledged}
the country had a ``reprehensible'' record on human rights and ``unfair
trade practices.'' But he argued that the top priority for the United
States was China's sale of missiles to Iran and Syria, which could
threaten Israel.

By the end of the decade, Republicans and a growing number of moderate
Democrats were extolling the benefits of freer trade with China. When
the Senate debated in September 2000 whether to end 20 years of annual
reviews of China's status and permanently normalize trade, paving the
way for the country's entry into the World Trade Organization, Mr. Biden
was a strong supporter.

Like many others in Congress, he argued that China's global integration
might ``influence the structure of their internal social, economic and
political systems.'' Permanently normalized trade, he said on the Senate
floor, ``continues a process of careful engagement designed to encourage
China's development as a productive, responsible member of the world
community.'' Mr. Biden also predicted that Delaware's chemical and
poultry industries would benefit, as well as General Motors and
Chrysler, both of which operated major plants in the state.

On Sept. 19, 2000, the Senate
\href{https://www.nytimes3xbfgragh.onion/2000/09/20/news/final-passage-of-bill-to-normalize-us-ties-is-approved-83-to-15-senate.html}{approved}
the measure, 83 to 15. As in the House, much of the modest opposition
centered on China's record on human rights and workers' rights.

Image

Construction workers in Wuhan, China, last month. The Chinese economy
has doubled about every eight years.Credit...Agence France-Presse ---
Getty Images

Mr. Trump now calls China's entry into the World Trade Organization
``one of the greatest geopolitical and economic disasters in world
history.''

But support for China's membership was widespread at the time, including
in corporations and the Republican Party. And excluding the world's most
populous nation from the international trade system might have led to
worse outcomes, analysts say.

Chad Bown, a senior fellow at the Peterson Institute for International
Economics, said that if China had not joined the World Trade
Organization, the United States could have still lost manufacturing jobs
to other countries as global trade and automation increased.

``It's not obvious to me that if China hadn't been allowed in, that
things would have developed in the U.S. differently,'' he said.

The United States also exacted a heavy price for China's membership, far
higher than for any other country that had joined the group thus far.
China was forced to lower its high tariffs, alter thousands of laws and
regulations, and adopt policies to open up markets.

But over the decades, China disappointed hopes for a broader
transformation. State-owned enterprises strengthened their control of
strategic industries, officials coerced technology transfer from foreign
companies or outright stole corporate secrets, and the Communist Party
limited the development of an independent judiciary. As its economy
became stronger, China's political system became less free.

Some Democrats say President George W. Bush neglected China during a
crucial period. As Beijing pushed forward with its economic opening, Mr.
Bush --- along with most American policymakers, including Mr. Biden ---
remained consumed with the Middle East and Afghanistan after the
terrorist attacks on Sept. 11, 2001.

Many American companies and consumers did benefit from the trade, but in
parts of the country --- especially in the industrial states that helped
elect Mr. Trump in 2016 --- shuttered factories and exported jobs
produced fury at Beijing and Washington.

From 1999 to 2011, competition from China cost the United States
\href{http://economics.mit.edu/files/11560}{more than two million
factory jobs}, according to academic research. In the midst of that,
flaws in the U.S. financial system set off a global economic crisis. In
2008 and 2009, as Mr. Biden took the reins of the second most powerful
office in the United States, the major G.M. and Chrysler plants in his
state shuttered.

\hypertarget{basketball-and-battleships}{%
\subsection{Basketball and
Battleships}\label{basketball-and-battleships}}

Image

Mr. Biden watching a basketball game in Beijing in 2011.Credit...Pool
photo by Ng Han Guan

At the end of his first term, President Barack Obama rolled out an
\href{https://www.nytimes3xbfgragh.onion/2011/11/16/world/asia/united-states-sees-china-everywhere-as-it-shifts-attention-to-asia.html}{ambitious
shift in U.S. foreign policy}, moving diplomatic and military resources
from the Middle East to Asia, mainly to address the challenge of China.
Secretary of State Hillary Clinton called it a
``\href{https://2009-2017.state.gov/secretary/20092013clinton/rm/2011/11/176999.htm}{pivot},''
and Mr. Obama
\href{https://www.theguardian.com/world/2011/nov/17/obama-asia-pacific-address-australia-parliament}{said}
``the United States is a Pacific power, and we are here to stay.'' He
put Marines in Australia and tried
\href{https://www.nytimes3xbfgragh.onion/2017/01/17/world/asia/china-tpp-ambassadors.html}{forging
a trade pact} among 12 Pacific Rim nations that was implicitly aimed at
countering China.

Mr. Biden
\href{https://www.nytimes3xbfgragh.onion/2012/02/18/world/asia/chinese-vice-president-xi-jinping-tours-los-angeles-port.html}{met
with Mr. Xi} at least eight times in 2011 and 2012 to gauge China's
incoming leader, even
\href{https://www.nytimes3xbfgragh.onion/2011/08/22/world/asia/22china.html}{playing
basketball} with him at a high school in Sichuan Province.

Mr. Campbell, the assistant secretary of state for East Asian and
Pacific Affairs who helped organize the trips, recalled that Mr. Biden
had ended up judging Mr. Xi as tough and unsentimental, someone who
questioned American power and believed in the superiority of the
Communist Party. In a White House meeting, he said, Mr. Biden told
advisers, ``I think we've got our hands full with this guy.''

Mr. Xi and other Chinese officials saw the pivot as Cold War-style
containment. And in 2013, they started bolstering territorial and
maritime claims in the
\href{https://www.nytimes3xbfgragh.onion/2013/11/24/world/asia/china-warns-of-action-against-aircraft-over-disputed-seas.html}{East
China Sea} and
\href{https://www.nytimes3xbfgragh.onion/2014/06/17/world/asia/spratly-archipelago-china-trying-to-bolster-its-claims-plants-islands-in-disputed-waters.html}{South
China Sea}, which the U.S. military dominates. Mr. Biden supported the
administration's decision to fly U.S. bombers and sail warships through
the zones, and he told Mr. Xi of Washington's growing anger. The old
relationship was fading.

``I wanted to make it clear that as long as they played by a set of
basic international rules that were written, and he did not like the
fact he didn't write them --- they didn't write them --- we'd have no
problem,'' Mr. Biden said in his interview with The Times. ``But to the
extent they tried to fundamentally alter the rules of airspace and
seaspace, what constitutes freedom of navigation, et cetera, then we'd
have a problem.''

\hypertarget{a-kettle-of-hawks}{%
\subsection{A Kettle of Hawks}\label{a-kettle-of-hawks}}

Image

Soldiers from the People's Liberation Army at a ceremony in Hong Kong in
June.Credit...Isaac Lawrence/Agence France-Presse --- Getty Images

Hours before Mr. Biden gave his
\href{https://www.youtube.com/watch?v=pnmQr0WfSvo}{Democratic nomination
acceptance speech} in August on a stage in Wilmington, Del., he received
an unexpected boost.

Seventy-five Republican national security specialists, some of whom had
worked for Mr. Trump, released a
\href{https://www.defendingdemocracytogether.org/national-security/}{letter
endorsing Mr. Biden}. They asserted that Mr. Trump ``lacks the character
and competence to lead this nation and has engaged in corrupt
behavior.''

The
\href{https://www.nytimes3xbfgragh.onion/2020/08/20/us/politics/republican-national-security-biden.html}{writers}
mentioned two episodes from Mr. Trump's relationship with Mr. Xi: when
he called on the Chinese leader last year to ``start an investigation''
into Mr. Biden and when he praised Mr. Xi as a ``brilliant leader'' ---
an example of Mr. Trump's cozying up to dictators. The letter echoed
recent devastating accounts, including
\href{https://www.wsj.com/articles/john-bolton-the-scandal-of-trumps-china-policy-11592419564}{from
John R. Bolton}, the former national security adviser, who called Mr.
Trump's approach to China haphazard and based on self-interest rather
than the national interest.

That message dovetailed with Mr. Biden's: that Mr. Trump's supposed
toughness on China was a mirage. The Biden campaign has hammered the
president over his response to the coronavirus, running advertisements
reminding voters that Mr. Trump praised Mr. Xi's handling of the
pandemic. And Mr. Biden has said that Mr. Trump's
\href{https://www.reuters.com/article/us-usa-election-china/biden-says-trumps-china-trade-deal-is-failing-badly-idUSKCN25131X}{trade
deal} with China is ``failing.''

Mr. Biden's attempts to out-hawk Mr. Trump have prompted some blowback:
Some Asian-Americans have
\href{https://www.politico.com/news/2020/04/23/biden-ad-exposes-left-rift-china-202241}{criticized}
his anti-China advertisements as racist. And progressive critics of
American power say Mr. Biden is perpetuating
\href{https://quincyinst.org/2020/03/09/why-joe-bidens-foreign-policy-vision-isnt-so-visionary/}{misguided
ideas} of U.S. superiority.

But Mr. Biden is under political pressure to look tough. A new
\href{https://www.pewresearch.org/global/2020/07/30/americans-fault-china-for-its-role-in-the-spread-of-covid-19/?utm_source=AdaptiveMailer\&utm_medium=email\&utm_campaign=20-08-03\%20US\%20Views\%20of\%20China\%202\%20Gen\%20Distr\&org=982\&lvl=100\&ite=6786\&lea=1495217\&ctr=0\&par=1\&trk=}{poll}
conducted by the Pew Research Center found that 73 percent of Americans
had an unfavorable view of China, the highest in at least 15 years. More
than half see China as a competitor.

With his trade proposals, Mr. Biden has tried to bridge the views
between the Democratic Party's center and its left wing, led by Senator
Bernie Sanders of Vermont. That has sometimes resulted in ambiguity. Mr.
Biden has not committed to removing Mr. Trump's tariffs on China; his
aides say he would first review how they affect the American middle
class.

Mr. Biden has also held back from promising to have the United States
enter the Trans-Pacific Partnership, which, despite Mr. Obama's efforts,
failed to gain enough support among Americans partly because of
opposition from labor unions and progressive Democratic politicians.
Japan helped finalize the agreement.

Some of Mr. Biden's ideas echo those of Trump officials, including
incentives to move important corporate supply chains out of China. He
envisions using the federal government's purchasing power, through ``Buy
American plans,'' to bolster manufacturing of critical goods like
pharmaceuticals at home.

But while Mr. Trump and Mr. Sanders call for punishing China, Mr.
Biden's aides emphasize a restoration of U.S. domestic strength.
Speaking in June at the Carnegie Endowment for International Peace, Jake
Sullivan, one of Mr. Biden's top advisers, said the United States
``should put less focus on trying to slow China down and more emphasis
on trying to run faster ourselves.'' Aides say that includes making
investments in scientific research and emerging U.S. industries, as well
as restoring alliances abroad.

Image

A high-security facility in Hotan, China, that is believed to be a
re-education camp for members of the Uighur ethnic
minority.Credit...Greg Baker/Agence France-Presse --- Getty Images

On human rights, Mr. Biden insists China must pay a price. A campaign
spokesman said in August that Mr. Biden believed the Chinese government
was
\href{https://www.politico.com/news/2020/08/25/trump-administration-china-genocide-uighurs-401581}{committing
``genocide''} against ethnic Uighur Muslims in the Xinjiang region. Mr.
Biden says he will
\href{https://twitter.com/elyratner/status/1278458007811248130}{impose
sanctions and commercial restrictions} on Chinese officials and entities
responsible for repression. While the Trump administration has recently
\href{https://www.nytimes3xbfgragh.onion/2020/07/20/business/economy/china-sanctions-uighurs-labor.html}{sanctioned
companies} and individuals involved in Xinjiang, Mr. Trump had
previously encouraged Mr. Xi to keep building internment camps there,
Mr. Bolton wrote, and to handle pro-democracy protesters in Hong Kong in
his own way.

Mr. Biden plans to try to win China's cooperation on issues like climate
change, Iran and North Korea. But that could be a challenge if Trump
administration hawks succeed in hard-wiring hostility into the
relationship. And regardless, every interaction with China, Mr. Campbell
said, was a negotiation in which Chinese officials tried to find a
source of leverage, ``even when it's something that's in their mutual
interest, like climate change.''

In the past few years, China has lost the benefit of the doubt among Mr.
Campbell and other key Biden advisers, all Obama administration veterans
who are likely to hold important government posts if Mr. Biden wins.

In their 2018 essay, Mr. Campbell and Mr. Ratner urged ``doing away with
the hopeful thinking'' of the past. Mr. Sullivan, Antony J. Blinken and
Jeffrey Prescott, all members of Mr. Biden's inner circle, agree on the
need to confront China on bad behavior. Susan Rice and Samantha Power,
often mentioned as potential candidates for secretary of state,
\href{https://www.npr.org/2020/08/04/898853269/susan-rice-is-on-bidens-short-list-to-be-his-running-mate}{denounce}
Beijing's atrocities on ethnic Uighurs and repression in Hong Kong.

``They'll use carrots and sticks and pressure and reassurance to
negotiate with the Chinese side,'' said
\href{https://gps.ucsd.edu/faculty-directory/susan-shirk.html}{Susan L.
Shirk}, a China scholar at the University of California, San Diego, and
a State Department official under President Bill Clinton. ``I don't
think they'll shy away from imposing costs.''

One thing is clear: If Mr. Biden becomes president, his 40-year
association with China will reach a crescendo. Analysts on both sides of
the Pacific say greater conflict may be inevitable, given the two
nations' ideological systems, nationalist sentiments and trajectories
--- one a superpower on the ascent, the other trying to preserve its
reach. Wang Yi, China's foreign minister,
\href{https://www.fmprc.gov.cn/mfa_eng/zxxx_662805/t1804328.shtml}{said}
his nation rejected a ``new Cold War,'' but he emphasized that ``the
United States must abandon its fantasy of remodeling China to U.S.
needs.''

Mr. Wang's words have added resonance as Mr. Biden and his fellow
policymakers wrestle with their earlier mission of trying to transform
China. Even on his 2001 trip, Mr. Biden heard a similar message about
the limits of American agency when he tried to highlight democratic
ideals in a discussion with about 40 graduate students at Fudan
University in Shanghai.

``There's a question I've been meaning to ask students of China,'' Mr.
Biden said, according to Mr. Jannuzi, who is now the president of the
Maureen and Mike Mansfield Foundation. ``The students of Tiananmen
Square, were they patriots or traitors to the People's Republic of
China?''

There was silence. Then, a physics student, a scholar of Newton and
Einstein, stood up.

``The students of Tiananmen were heroes of the People's Republic of
China,'' he said. ``Senator, change will come to China. But it will be
we, the students of Newton, who determine the pace and the direction of
that change, and not you or anyone else working on the banks of the
Potomac.''

\hypertarget{our-2020-election-guide}{%
\section{Our 2020 Election Guide}\label{our-2020-election-guide}}

Updated ~Sept. 8, 2020

\begin{center}\rule{0.5\linewidth}{\linethickness}\end{center}

\begin{itemize}
\item ~
  \hypertarget{the-latest}{%
  \subsection{The Latest}\label{the-latest}}

  \begin{itemize}
  \item
    President Trump and his party are using a playbook that aims to
    alarm people about crime in their backyards. It didn't work in 2018,
    but
    \href{https://www.nytimes3xbfgragh.onion/2020/09/08/us/politics/trump-republicans-fear-strategy.html?action=click\&pgtype=Article\&state=default\&region=BELOW_MAIN_CONTENT\&context=storylines_guide}{both
    parties think it could resonate more this year}.
  \end{itemize}
\item ~
  \hypertarget{how-to-win-270}{%
  \subsection{How to Win 270}\label{how-to-win-270}}

  \begin{itemize}
  \item
    Joe Biden and Donald Trump need 270 electoral votes to reach the
    White House. Try building
    \href{https://www.nytimes3xbfgragh.onion/interactive/2020/us/elections/election-states-biden-trump.html?action=click\&pgtype=Article\&state=default\&region=BELOW_MAIN_CONTENT\&context=storylines_guide}{your
    own coalition of battleground states}~to see potential outcomes.
  \end{itemize}
\item ~
  \hypertarget{voting-by-mail}{%
  \subsection{Voting by Mail}\label{voting-by-mail}}

  \begin{itemize}
  \item
    Will you have enough time to vote by mail in your state? Yes, but
    it's risky to procrastinate.
    \href{https://www.nytimes3xbfgragh.onion/interactive/2020/08/31/us/politics/vote-by-mail-deadlines.html?action=click\&pgtype=Article\&state=default\&region=BELOW_MAIN_CONTENT\&context=storylines_guide}{Check
    your state's deadline.}
  \item
    \href{https://www.nytimes3xbfgragh.onion/interactive/2020/us/elections/joe-biden.html?action=click\&pgtype=Article\&state=default\&region=BELOW_MAIN_CONTENT\&context=storylines_guide}{}

    \hypertarget{joe-biden}{%
    \section{Joe Biden}\label{joe-biden}}

    \hypertarget{democrat}{%
    \subsection{Democrat}\label{democrat}}

    \href{https://www.nytimes3xbfgragh.onion/interactive/2020/us/elections/donald-trump.html?action=click\&pgtype=Article\&state=default\&region=BELOW_MAIN_CONTENT\&context=storylines_guide}{}

    \hypertarget{donald-trump}{%
    \section{Donald Trump}\label{donald-trump}}

    \hypertarget{republican}{%
    \subsection{Republican}\label{republican}}
  \end{itemize}
\item
  \hypertarget{keep-up-with-our-coverage}{%
  \subsection{Keep Up With Our
  Coverage}\label{keep-up-with-our-coverage}}

  \begin{itemize}
  \item
    Get an
    \href{https://www.nytimes3xbfgragh.onion/newsletters/politics?action=click\&pgtype=Article\&state=default\&region=BELOW_MAIN_CONTENT\&context=storylines_guide}{email}~recapping
    the day's news
  \item
    Download our mobile app on
    \href{https://apps.apple.com/us/app/nytimes/id284862083?ls=1\&mat_click_id=5c79ae7455014fd1bd66b5610c05b8f2-20191112-16948\&referrer=mat_click_id\%3D5c79ae7455014fd1bd66b5610c05b8f2-20191112-16948\%26link_click_id\%3D722930677036718082}{iOS}~and
    \href{http://a.localytics.com/android?id=com.nytimes.android\&referrer=utm_source\%3Dother_nyt_mobile_web\%26utm_medium\%3DWeb\%2520page\%26utm_term\%3DGeneral\%2520Mobile\%2520Page\%26utm_campaign\%3DNYT\%2520Mobile\%2520General\%2520Page}{Android}~and
    turn on Breaking News and Politics alerts
  \end{itemize}
\end{itemize}

Advertisement

\protect\hyperlink{after-bottom}{Continue reading the main story}

\hypertarget{site-index}{%
\subsection{Site Index}\label{site-index}}

\hypertarget{site-information-navigation}{%
\subsection{Site Information
Navigation}\label{site-information-navigation}}

\begin{itemize}
\tightlist
\item
  \href{https://help.nytimes3xbfgragh.onion/hc/en-us/articles/115014792127-Copyright-notice}{©~2020~The
  New York Times Company}
\end{itemize}

\begin{itemize}
\tightlist
\item
  \href{https://www.nytco.com/}{NYTCo}
\item
  \href{https://help.nytimes3xbfgragh.onion/hc/en-us/articles/115015385887-Contact-Us}{Contact
  Us}
\item
  \href{https://www.nytco.com/careers/}{Work with us}
\item
  \href{https://nytmediakit.com/}{Advertise}
\item
  \href{http://www.tbrandstudio.com/}{T Brand Studio}
\item
  \href{https://www.nytimes3xbfgragh.onion/privacy/cookie-policy\#how-do-i-manage-trackers}{Your
  Ad Choices}
\item
  \href{https://www.nytimes3xbfgragh.onion/privacy}{Privacy}
\item
  \href{https://help.nytimes3xbfgragh.onion/hc/en-us/articles/115014893428-Terms-of-service}{Terms
  of Service}
\item
  \href{https://help.nytimes3xbfgragh.onion/hc/en-us/articles/115014893968-Terms-of-sale}{Terms
  of Sale}
\item
  \href{https://spiderbites.nytimes3xbfgragh.onion}{Site Map}
\item
  \href{https://help.nytimes3xbfgragh.onion/hc/en-us}{Help}
\item
  \href{https://www.nytimes3xbfgragh.onion/subscription?campaignId=37WXW}{Subscriptions}
\end{itemize}
