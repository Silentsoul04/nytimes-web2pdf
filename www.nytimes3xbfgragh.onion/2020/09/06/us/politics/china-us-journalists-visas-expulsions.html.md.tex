Sections

SEARCH

\protect\hyperlink{site-content}{Skip to
content}\protect\hyperlink{site-index}{Skip to site index}

\href{https://www.nytimes3xbfgragh.onion/section/politics}{Politics}

\href{https://myaccount.nytimes3xbfgragh.onion/auth/login?response_type=cookie\&client_id=vi}{}

\href{https://www.nytimes3xbfgragh.onion/section/todayspaper}{Today's
Paper}

\href{/section/politics}{Politics}\textbar{}China Freezes Credentials
for Journalists at U.S. Outlets, Hinting at Expulsions

\url{https://nyti.ms/3369UXx}

\begin{itemize}
\item
\item
\item
\item
\item
\end{itemize}

Advertisement

\protect\hyperlink{after-top}{Continue reading the main story}

Supported by

\protect\hyperlink{after-sponsor}{Continue reading the main story}

\hypertarget{china-freezes-credentials-for-journalists-at-us-outlets-hinting-at-expulsions}{%
\section{China Freezes Credentials for Journalists at U.S. Outlets,
Hinting at
Expulsions}\label{china-freezes-credentials-for-journalists-at-us-outlets-hinting-at-expulsions}}

CNN, The Wall Street Journal, Bloomberg News and Getty Images are
affected. Chinese officials told journalists, who can still work, that
their fate depends on what the United States does to Chinese media
employees.

\includegraphics{https://static01.graylady3jvrrxbe.onion/images/2020/09/06/world/06dc-china-journalists2/merlin_170894475_8e629cf2-52fc-4408-a239-32f5b83fed38-articleLarge.jpg?quality=75\&auto=webp\&disable=upscale}

\href{https://www.nytimes3xbfgragh.onion/by/edward-wong}{\includegraphics{https://static01.graylady3jvrrxbe.onion/images/2018/09/24/multimedia/author-edward-wong/author-edward-wong-thumbLarge-v5.png}}

By \href{https://www.nytimes3xbfgragh.onion/by/edward-wong}{Edward Wong}

\begin{itemize}
\item
  Sept. 6, 2020
\item
  \begin{itemize}
  \item
  \item
  \item
  \item
  \item
  \end{itemize}
\end{itemize}

\href{https://cn.nytimes3xbfgragh.onion/china/20200907/china-us-journalists-visas-expulsions/}{阅读简体中文版}\href{https://cn.nytimes3xbfgragh.onion/china/20200907/china-us-journalists-visas-expulsions/zh-hant/}{閱讀繁體中文版}

WASHINGTON --- The Chinese government has stopped renewing press
credentials for
\href{https://www.nytimes3xbfgragh.onion/2018/01/05/sunday-review/china-military-economic-power.html}{foreign
journalists} working for American news organizations in China and has
implied it will proceed with expulsions if the Trump administration
takes further action against Chinese media employees in the United
States, according to six journalists and U.S. officials with knowledge
of the events and statements on Monday by a Chinese Foreign Ministry
spokesman.

The actions and threats raise the stakes in the continuing cycles of
retribution between Washington and Beijing over news media
organizations. Those rounds of retaliation are a prominent element of a
much broader downward spiral in
\href{https://www.nytimes3xbfgragh.onion/2019/06/26/world/asia/united-states-china-conflict.html}{U.S.-China
relations}, one that involves mutually hostile policies and actions over
trade, technology,
\href{https://www.nytimes3xbfgragh.onion/2020/05/28/us/politics/china-hong-kong-trump-student-visas.html}{education},
\href{https://www.nytimes3xbfgragh.onion/2020/07/22/world/asia/us-china-houston-consulate.html}{diplomatic
missions},
\href{https://www.nytimes3xbfgragh.onion/2020/08/17/us/politics/trump-china-taiwan-hong-kong.html}{Taiwan}
and military presence in Asia.

American news organizations immediately affected by China's latest
actions are CNN, The Wall Street Journal, Bloomberg News and Getty
Images. Journalists from all four organizations tried to renew press
cards with the Foreign Ministry in the past week, but were told the
cards, which are usually good for one year, could not be renewed. In
total, at least five journalists in the four organizations have been
affected so far.

One journalist said Foreign Ministry officials told him that his fate
depended on whether the United States decided in the fall to renew the
visas of Chinese journalists working in America who are under
\href{https://www.nytimes3xbfgragh.onion/2020/05/09/us/politics/china-journalists-us-visa-crackdown.html}{new
visa regulations} imposed by the Department of Homeland Security in May.
Other journalists have received similar messages.

The journalist said he was told by Chinese officials that if the Trump
administration decided to expel Chinese journalists, Beijing would take
reciprocal action. Many of the Chinese journalists work for state-run
news organizations.

The \href{https://fccchina.org/about/}{Foreign Correspondents' Club of
China} said in a statement Monday that it was ``very alarmed'' at the
halt in press credential renewals.

``These coercive practices have again turned accredited foreign
journalists in China into pawns in a wider diplomatic conflict,'' it
said. ``The F.C.C.C. calls on the Chinese government to halt this cycle
of tit-for-tat reprisals in what is quickly becoming the darkest year
yet for media freedoms.''

The organization noted that the Chinese government had already expelled
17 foreign journalists in the first half of 2020 --- including seven
from The New York Times --- by canceling their press credentials. The
group said it expected more China-based journalists to have their press
card renewal process halted by the Foreign Ministry this fall.

Foreign journalists working in China must renew their press cards to get
new residence permits from the Public Security Bureau, the main police
organization. The residence permits are the equivalent of visas that
allow foreigners to live in China. The journalists with expired press
cards were told by police officials after their discussions with Foreign
Ministry officials that they would be given residence permits that are
good until November 6.

They were given letters from the Foreign Ministry that said they could
continue to work in China for the time being despite the expired press
cards, according to a copy of one such letter obtained by The New York
Times.

The November 6 end point for the residence permits corresponds to when
the Trump administration might decide not to renew visas for many
Chinese news media employees in the United States, which would result in
their expulsions.

In May, the administration announced that all Chinese journalists would
now have 90-day work visas --- a significant reduction from the
open-ended, single-entry visas they had gotten previously. The
journalists would be allowed to apply for extensions of 90 days each. In
early August, the visas expired, but the Department of Homeland
Security, in consultation with the State Department, did not expel any
of the Chinese journalists or renew their visas, which meant they got a
de facto extension of 90 days, according to the language of the new
regulation.

They can continue to live and work in the United States until early
November --- the same week as the U.S. elections --- as they await word
on their renewal applications. But American officials could also decide
to expel some of them sooner. Officials consider many of the Chinese
employees to be propaganda workers and, in a few cases, maybe even
spies. More than 130 Chinese journalists and news media employees have
been affected.

Chinese diplomats are concerned about the potential expulsions and have
spoken to their American counterparts.

China's recent freezing of press card renewals has affected citizens
from a number of countries working for American news organizations. They
include David Culver, the only CNN correspondent in China, who is
American; Jeremy Page, a British correspondent for The Wall Street
Journal; and Andrea Verdelli, an Italian freelance photographer whose
credential is with Getty.

After the Times first reported on Sunday on the Chinese government's new
restrictions, Bloomberg News posted an
\href{https://www.bloomberg.com/news/articles/2020-09-07/china-delays-credentials-for-journalists-with-u-s-media-outlets}{article}
that said two of their journalists were affected, without stating their
nationalities. The Wall Street Journal also posted a story
\href{https://www.wsj.com/articles/china-delays-approving-press-credentials-for-foreign-reporters-in-media-standoff-11599401714}{confirming
the status} of Mr. Page.

CNN released a statement on Sunday that said: ``One of our Beijing-based
journalists was recently issued a visa valid for two months, instead of
the usual 12. However, our presence on the ground in China remains
unchanged, and we are continuing to work with local authorities to
ensure that continues.''

In Beijing, Zhao Lijian, a spokesman for the Chinese Foreign Ministry,
accused Washington on Monday of being ``arrogant and unreasonable'' in
its negotiations with China over the renewal of visas for Chinese
journalists in the United States.

He said the freezing of press credential renewals was a reciprocal move
on China's part, and warned that ``all cards were on the table'' --- a
hint that Beijing was prepared to expel the journalists if it deemed
necessary.

``If the U.S. government really cares about American journalists, it
should extend visas for all Chinese journalists as soon as possible,
instead of kidnapping journalists from the two countries as hostages,''
Mr. Zhao told reporters in a daily briefing.

Morgan Ortagus, a State Department spokeswoman, said in a statement on
Sunday that the Chinese Foreign Ministry had recently told the U.S.
Embassy in Beijing that it was denying press card renewals to foreign
journalists and refusing to process pending visa applications for
journalists who were expelled earlier this year.

``The United States is of course troubled that these proposed actions by
the P.R.C. will worsen the reporting environment in China, which is
already suffering a dearth of open and independent media reporting,''
she said, referring to the People's Republic of China.

There are American news organizations that have not been affected by
China's new limits yet because their journalists' press cards have not
come up for renewal. This is the case with the one New York Times
foreign correspondent, an American, who remains in mainland China.

The Chinese government put a similar freeze on press card renewals in
late 2013 for journalists from The Times and Bloomberg News
\href{https://www.nytimes3xbfgragh.onion/2013/11/09/world/asia/bloomberg-news-is-said-to-curb-articles-that-might-anger-china.html}{because
of sensitivities involving coverage}. The halt on renewals led to fears
of mass expulsions and efforts by Joseph R. Biden Jr., then vice
president, to stress to Chinese leaders the importance of a free press
during a visit to Beijing. The Chinese government eventually renewed
those press cards and granted new residence permits, but refused to
issue visas for any new incoming journalists from the two organizations
for a couple of years.

This March, China
\href{https://www.nytimes3xbfgragh.onion/2020/03/17/business/media/china-expels-american-journalists.html}{expelled
almost all American journalists} for The Times, The Wall Street Journal
and The Washington Post from mainland bureaus. That was done after the
State Department told five Chinese state-run news organizations they
could employ a total of only 100 non-Americans, which resulted in the
expulsions of 60 Chinese workers from the United States. Earlier, in
February, Washington
\href{https://www.nytimes3xbfgragh.onion/2020/02/18/world/asia/china-media-trump.html}{designated}
those five organizations diplomatic entities, requiring them to turn
over lists of their employees and property assets. (It designated
\href{https://www.nytimes3xbfgragh.onion/2020/06/22/us/politics/us-china-news-organizations.html}{four
more Chinese organizations} in June.)

On Feb. 19, the day after Washington made the first round of
designations, Beijing
\href{https://www.nytimes3xbfgragh.onion/2020/02/19/business/media/china-wall-street-journal.html}{announced
the expulsions} of three Wall Street Journal reporters, the first such
outright evictions of journalists since 1998. Chinese officials said
they took action because of what they called an inflammatory headline
---
\href{https://www.wsj.com/articles/china-is-the-real-sick-man-of-asia-11580773677}{``China
Is the Real Sick Man of Asia''} --- on a Journal opinion column.

Chinese officials have also forced the ouster of
\href{https://www.cjr.org/analysis/chinese-nationals-forced-resignations.php}{some
Chinese employees} from the Beijing bureaus of several American news
organizations, including The Times, The Journal and CNN.

Chinese officials have also said the journalists expelled this year
cannot work in Hong Kong, a former British colony that in theory
operates under a semi-autonomous governance system. But Beijing imposed
a new national security law on Hong Kong in June, one that mainland and
Hong Kong officials have already begun using to
\href{https://www.nytimes3xbfgragh.onion/2020/07/31/world/asia/hong-kong-election-national-security-law.html}{curb
civil liberties} in the territory. And in recent months, Hong Kong
officials have
\href{https://www.nytimes3xbfgragh.onion/2020/07/14/business/media/new-york-times-hong-kong.html}{delayed
or refused to issue work permits} to some foreign journalists, an
occurrence that was once rare. As a result, the Times and other news
organizations are moving some of their journalists based there to other
cities in Asia.

American officials say their actions against Chinese news media
organizations and employees are in reaction to the tightening
restrictions Beijing has imposed on foreign journalists in China in
recent years. Those include limiting the duration of press cards,
residence permits and visas from the standard length of one year to as
short as one month in cases where Chinese officials seek to punish a
reporter or try to get a news organization to self-censor.

Some Trump administration officials are intent on steering the
U.S.-China relationship toward
\href{https://www.nytimes3xbfgragh.onion/2020/07/25/world/asia/us-china-trump-xi.html}{a
course of permanent competition and confrontation}, one they hope will
be difficult to reverse even if President Trump loses to Mr. Biden in
the November election. Mr. Trump cares about maintaining robust trade
with China and has
\href{https://www.nytimes3xbfgragh.onion/2020/06/18/us/politics/trump-china-bolton.html}{praised
Xi Jinping}, the Chinese leader. But he has become more openly critical
of China in the run-up to the election, because his strategists see it
as an issue that can help divert voters away from
\href{https://www.nytimes3xbfgragh.onion/2020/07/18/us/politics/trump-coronavirus-response-failure-leadership.html}{the
president's vast failures} on the coronavirus pandemic.

Advertisement

\protect\hyperlink{after-bottom}{Continue reading the main story}

\hypertarget{site-index}{%
\subsection{Site Index}\label{site-index}}

\hypertarget{site-information-navigation}{%
\subsection{Site Information
Navigation}\label{site-information-navigation}}

\begin{itemize}
\tightlist
\item
  \href{https://help.nytimes3xbfgragh.onion/hc/en-us/articles/115014792127-Copyright-notice}{©~2020~The
  New York Times Company}
\end{itemize}

\begin{itemize}
\tightlist
\item
  \href{https://www.nytco.com/}{NYTCo}
\item
  \href{https://help.nytimes3xbfgragh.onion/hc/en-us/articles/115015385887-Contact-Us}{Contact
  Us}
\item
  \href{https://www.nytco.com/careers/}{Work with us}
\item
  \href{https://nytmediakit.com/}{Advertise}
\item
  \href{http://www.tbrandstudio.com/}{T Brand Studio}
\item
  \href{https://www.nytimes3xbfgragh.onion/privacy/cookie-policy\#how-do-i-manage-trackers}{Your
  Ad Choices}
\item
  \href{https://www.nytimes3xbfgragh.onion/privacy}{Privacy}
\item
  \href{https://help.nytimes3xbfgragh.onion/hc/en-us/articles/115014893428-Terms-of-service}{Terms
  of Service}
\item
  \href{https://help.nytimes3xbfgragh.onion/hc/en-us/articles/115014893968-Terms-of-sale}{Terms
  of Sale}
\item
  \href{https://spiderbites.nytimes3xbfgragh.onion}{Site Map}
\item
  \href{https://help.nytimes3xbfgragh.onion/hc/en-us}{Help}
\item
  \href{https://www.nytimes3xbfgragh.onion/subscription?campaignId=37WXW}{Subscriptions}
\end{itemize}
