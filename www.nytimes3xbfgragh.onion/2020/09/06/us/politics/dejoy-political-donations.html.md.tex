Sections

SEARCH

\protect\hyperlink{site-content}{Skip to
content}\protect\hyperlink{site-index}{Skip to site index}

\href{https://www.nytimes3xbfgragh.onion/section/politics}{Politics}

\href{https://myaccount.nytimes3xbfgragh.onion/auth/login?response_type=cookie\&client_id=vi}{}

\href{https://www.nytimes3xbfgragh.onion/section/todayspaper}{Today's
Paper}

\href{/section/politics}{Politics}\textbar{}DeJoy Pressured Workers to
Donate to G.O.P. Candidates, Former Employees Say

\url{https://nyti.ms/32YeneM}

\begin{itemize}
\item
\item
\item
\item
\item
\end{itemize}

Advertisement

\protect\hyperlink{after-top}{Continue reading the main story}

Supported by

\protect\hyperlink{after-sponsor}{Continue reading the main story}

\hypertarget{dejoy-pressured-workers-to-donate-to-gop-candidates-former-employees-say}{%
\section{DeJoy Pressured Workers to Donate to G.O.P. Candidates, Former
Employees
Say}\label{dejoy-pressured-workers-to-donate-to-gop-candidates-former-employees-say}}

Former employees at New Breed Logistics say they were expected to donate
to candidates whom their executive, Louis DeJoy, was supporting, and
would be rewarded through yearly bonuses.

\includegraphics{https://static01.graylady3jvrrxbe.onion/images/2020/09/06/us/politics/06dc-dejoy1/06dc-dejoy1-articleLarge.jpg?quality=75\&auto=webp\&disable=upscale}

By \href{https://www.nytimes3xbfgragh.onion/by/catie-edmondson}{Catie
Edmondson},
\href{https://www.nytimes3xbfgragh.onion/by/jessica-silver-greenberg}{Jessica
Silver-Greenberg} and
\href{https://www.nytimes3xbfgragh.onion/by/luke-broadwater}{Luke
Broadwater}

\begin{itemize}
\item
  Sept. 6, 2020
\item
  \begin{itemize}
  \item
  \item
  \item
  \item
  \item
  \end{itemize}
\end{itemize}

Postmaster General Louis DeJoy, a major donor to President Trump and
fund-raiser for the Republican Party, cultivated an environment at his
former company that left employees feeling pressured to make donations
to Republican candidates, and rewarded them with bonuses for doing so,
according to former employees.

The arrangement was described by three former employees at New Breed
Logistics, Mr. DeJoy's former company, who said that workers would
receive bonuses if they donated to candidates he supported, and that it
was expected that managers would participate. A fourth employee
confirmed that managers at the company were routinely solicited to make
donations. The four former employees spoke on the condition of anonymity
for fear of professional retaliation.

The former employees did not say how explicit Mr. DeJoy was about
linking the campaign contributions he was encouraging to the extra
compensation, but three of them said it was widely believed that the
bonuses were meant to reimburse the political donations, an allegation
first
\href{https://www.washingtonpost.com/investigations/louis-dejoy-campaign-contributions/2020/09/06/1187bc2c-e3fe-11ea-8181-606e603bb1c4_story.html}{reported
by The Washington Post}. Federal campaign finance law bars straw-donor
schemes, in which an individual reimburses someone else to donate to a
political campaign in order to skirt contribution limits. But it is
legal to encourage employees to make donations, as Mr. DeJoy routinely
did.

A review of campaign finance records shows that over a dozen
management-level employees at New Breed would routinely donate to the
same candidate on the same day, often writing checks for an identical
amount of money. One day in October 2014, for example, 20 midlevel and
senior officials at the company donated a total of \$37,600 to the
campaign of Senator Thom Tillis, Republican of North Carolina, who was
running to unseat a Democratic incumbent. Each official wrote a check
for either \$2,600, the maximum allowable donation, or \$1,000.

Similar patterns of donations --- including to the Republican National
Committee and every Republican presidential nominee from President
George W. Bush to Mitt Romney --- stretch back to 2003, campaign finance
records show. Mr. DeJoy's wife, Dr. Aldona Wos, was the vice chairwoman
of Mr. Bush's North Carolina fund-raising team, and Mr. Bush later
appointed her to serve as the ambassador to Estonia. Mr. DeJoy, a
Republican megadonor, served as the chief executive of New Breed from
1983 to 2014, until the company was sold to XPO Logistics.

Monty Hagler, a spokesman for Mr. DeJoy, said in a lengthy statement
provided to The New York Times that the former New Breed executive
``consistently provided family members and employees with various
volunteer opportunities to get involved in activities that a family
member or employee might feel was important or enjoyable to that
individual.''

Mr. DeJoy ``was never notified'' of any pressure they might have felt to
make a political contribution, Mr. Hagler said, and ``regrets if any
employee felt uncomfortable for any reason.''

Mr. Hagler added that Mr. DeJoy had consulted with the former general
counsel of the Federal Election Commission on election laws ``to ensure
that he, New Breed Logistics and any person affiliated with New Breed
fully complied with any and all laws.''

At a
\href{https://www.nytimes3xbfgragh.onion/2020/08/24/us/politics/postal-service-dejoy-testimony.html}{hearing}
last month, Mr. DeJoy angrily denied a suggestion by Representative Jim
Cooper, Democrat of Tennessee, that he had reimbursed his employees'
political donations.

``That's an outrageous claim, sir, and I resent it,'' Mr. DeJoy
responded. ``What are you accusing me of?''

It is unclear how the arrangement was communicated to employees or how
extensive it was. One former New Breed employee said he donated to a
Republican candidate and never received a bonus, prompting him to never
again make another donation. Two other employees, Dave Bell, a current
vice president at XPO Logistics who started as a vice president at New
Breed in 2010, and Edi Dirkes, a human resources manager at the company
from 2007 to 2010, said they had never heard of the arrangement.

``No one ever approached me'' to make any political contributions, Mr.
Bell said in a brief interview.

Still, the revelations are likely to fuel further scrutiny of Mr. DeJoy,
who is under fire for his continuing
\href{https://www.nytimes3xbfgragh.onion/2020/09/02/us/politics/louis-dejoy-usps-paid.html}{financial
ties} to a company that does business with the Postal Service and his
previous work fund-raising for Republicans.

``These are very serious allegations that must be investigated
immediately, independent of Donald Trump's Justice Department,'' Senator
Chuck Schumer of New York, the Democratic leader, said in a statement.
``The North Carolina attorney general, an elected official who is
independent of Donald Trump, is the right person to start this
investigation.''

Josh Stein, North Carolina's attorney general,
\href{https://twitter.com/JoshStein_/status/1302678272493449216}{said in
a statement} that ``it is against the law to directly or indirectly
reimburse someone for a political contribution'' and that ``any credible
allegations of such actions merit investigation by the appropriate state
and federal authorities.''

Scrutiny of Mr. DeJoy began shortly after he took the helm of the Postal
Service in June and began carrying out a series of cost-cutting steps
that have led to slower and less reliable delivery as Mr. Trump has
stepped up his attacks on voting by mail before the election in
November.

Mr. Trump has assailed the money-losing Postal Service in recent months
while
\href{https://www.nytimes3xbfgragh.onion/article/mail-in-voting-explained.html}{falsely
warning} that voting by mail will lead to fraud and lost or stolen
ballots. He has also cautioned that the practice could lead to long
delays in determining a winner.

Mr. DeJoy has
\href{https://www.nytimes3xbfgragh.onion/2020/08/24/us/politics/louis-dejoy-post-office-hearing.html}{characterized
those comments} as ``not helpful'' and condemned the ``false narrative''
that he said was being promoted about both his intentions and the
changes at the agency, which he has argued are necessary to shore up its
financial health.

``I am not engaged in sabotaging the election,'' Mr. DeJoy told
lawmakers last month, noting that he had committed to
\href{https://www.nytimes3xbfgragh.onion/2020/08/19/business/economy/postal-service-changes-dejoy.html}{reversing
many of the changes} that have drawn the sharpest backlash.

Advertisement

\protect\hyperlink{after-bottom}{Continue reading the main story}

\hypertarget{site-index}{%
\subsection{Site Index}\label{site-index}}

\hypertarget{site-information-navigation}{%
\subsection{Site Information
Navigation}\label{site-information-navigation}}

\begin{itemize}
\tightlist
\item
  \href{https://help.nytimes3xbfgragh.onion/hc/en-us/articles/115014792127-Copyright-notice}{©~2020~The
  New York Times Company}
\end{itemize}

\begin{itemize}
\tightlist
\item
  \href{https://www.nytco.com/}{NYTCo}
\item
  \href{https://help.nytimes3xbfgragh.onion/hc/en-us/articles/115015385887-Contact-Us}{Contact
  Us}
\item
  \href{https://www.nytco.com/careers/}{Work with us}
\item
  \href{https://nytmediakit.com/}{Advertise}
\item
  \href{http://www.tbrandstudio.com/}{T Brand Studio}
\item
  \href{https://www.nytimes3xbfgragh.onion/privacy/cookie-policy\#how-do-i-manage-trackers}{Your
  Ad Choices}
\item
  \href{https://www.nytimes3xbfgragh.onion/privacy}{Privacy}
\item
  \href{https://help.nytimes3xbfgragh.onion/hc/en-us/articles/115014893428-Terms-of-service}{Terms
  of Service}
\item
  \href{https://help.nytimes3xbfgragh.onion/hc/en-us/articles/115014893968-Terms-of-sale}{Terms
  of Sale}
\item
  \href{https://spiderbites.nytimes3xbfgragh.onion}{Site Map}
\item
  \href{https://help.nytimes3xbfgragh.onion/hc/en-us}{Help}
\item
  \href{https://www.nytimes3xbfgragh.onion/subscription?campaignId=37WXW}{Subscriptions}
\end{itemize}
