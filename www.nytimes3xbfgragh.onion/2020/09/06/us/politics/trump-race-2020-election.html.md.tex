Sections

SEARCH

\protect\hyperlink{site-content}{Skip to
content}\protect\hyperlink{site-index}{Skip to site index}

\href{https://www.nytimes3xbfgragh.onion/section/politics}{Politics}

\href{https://myaccount.nytimes3xbfgragh.onion/auth/login?response_type=cookie\&client_id=vi}{}

\href{https://www.nytimes3xbfgragh.onion/section/todayspaper}{Today's
Paper}

\href{/section/politics}{Politics}\textbar{}More Than Ever, Trump Casts
Himself as the Defender of White America

\url{https://nyti.ms/2FbTrsB}

\begin{itemize}
\item
\item
\item
\item
\item
\end{itemize}

\hypertarget{race-and-america}{%
\subsubsection{\texorpdfstring{\href{https://www.nytimes3xbfgragh.onion/news-event/george-floyd-protests-minneapolis-new-york-los-angeles?name=styln-george-floyd\&region=TOP_BANNER\&block=storyline_menu_recirc\&action=click\&pgtype=Article\&impression_id=30917540-f286-11ea-828f-7bf945cdf89b\&variant=undefined}{Race
and America}}{Race and America}}\label{race-and-america}}

\begin{itemize}
\tightlist
\item
  \href{https://www.nytimes3xbfgragh.onion/2020/09/04/nyregion/rochester-police-daniel-prude.html?name=styln-george-floyd\&region=TOP_BANNER\&block=storyline_menu_recirc\&action=click\&pgtype=Article\&impression_id=30919c50-f286-11ea-828f-7bf945cdf89b\&variant=undefined}{What
  Happened in Rochester, N.Y.}
\item
  \href{https://www.nytimes3xbfgragh.onion/2020/09/01/us/politics/trump-fact-check-protests.html?name=styln-george-floyd\&region=TOP_BANNER\&block=storyline_menu_recirc\&action=click\&pgtype=Article\&impression_id=30919c51-f286-11ea-828f-7bf945cdf89b\&variant=undefined}{Trump
  Fact Check}
\item
  \href{https://www.nytimes3xbfgragh.onion/2020/08/30/us/portland-shooting-explained.html?name=styln-george-floyd\&region=TOP_BANNER\&block=storyline_menu_recirc\&action=click\&pgtype=Article\&impression_id=30919c52-f286-11ea-828f-7bf945cdf89b\&variant=undefined}{Portland
  Shooting}
\item
  \href{https://www.nytimes3xbfgragh.onion/2020/08/30/us/breonna-taylor-police-killing.html?name=styln-george-floyd\&region=TOP_BANNER\&block=storyline_menu_recirc\&action=click\&pgtype=Article\&impression_id=30919c53-f286-11ea-828f-7bf945cdf89b\&variant=undefined}{Breonna
  Taylor's Life and Death}
\end{itemize}

Advertisement

\protect\hyperlink{after-top}{Continue reading the main story}

Supported by

\protect\hyperlink{after-sponsor}{Continue reading the main story}

News Analysis

\hypertarget{more-than-ever-trump-casts-himself-as-the-defender-of-white-america}{%
\section{More Than Ever, Trump Casts Himself as the Defender of White
America}\label{more-than-ever-trump-casts-himself-as-the-defender-of-white-america}}

Presenting himself as a warrior against identity politics, the president
has increasingly made appeals to the grievances of white supporters a
centerpiece of his re-election campaign.

\includegraphics{https://static01.graylady3jvrrxbe.onion/images/2020/09/06/us/politics/06trump-race1/06trump-race1-articleLarge.jpg?quality=75\&auto=webp\&disable=upscale}

\href{https://www.nytimes3xbfgragh.onion/by/peter-baker}{\includegraphics{https://static01.graylady3jvrrxbe.onion/images/2018/06/13/multimedia/peter-baker/peter-baker-thumbLarge-v2.png}}

By \href{https://www.nytimes3xbfgragh.onion/by/peter-baker}{Peter Baker}

\begin{itemize}
\item
  Sept. 6, 2020
\item
  \begin{itemize}
  \item
  \item
  \item
  \item
  \item
  \end{itemize}
\end{itemize}

WASHINGTON --- After a summer when hundreds of thousands of people took
to the streets protesting racial injustice against Black Americans,
\href{https://www.nytimes3xbfgragh.onion/interactive/2020/us/elections/donald-trump.html}{President
Trump} has made it clear over the last few days that, in his view, the
country's real race problem is bias against white Americans.

Just days after returning from Kenosha, Wis., where he
\href{https://www.nytimes3xbfgragh.onion/2020/09/01/us/politics/trump-conspiracy-theory-thugs-plane.html}{staunchly
backed law enforcement} and did not mention the name of Jacob Blake, the
Black man shot seven times in the back by the police,
\href{https://www.nytimes3xbfgragh.onion/2020/09/04/us/politics/trump-race-sensitivity-training.html}{Mr.
Trump issued an order} on Friday to purge the federal government of
racial sensitivity training that his White House called ``divisive,
anti-American propaganda.''

The president then spent much of the weekend tweeting about his action,
presenting himself as a warrior against identity politics. ``This is a
sickness that cannot be allowed to continue,''
\href{https://twitter.com/realDonaldTrump/status/1302212909808971776}{he
wrote of such programs}. ``Please report any sightings so we can quickly
extinguish!'' He
\href{https://twitter.com/ColumbiaBugle/status/1302214228527456264}{reposted
a tweet} from a conservative outlet hailing his order: ``Sorry liberals!
How to be Anti-White 101 is permanently cancelled!''

Not in generations has a sitting president so overtly declared himself
the candidate of white America. While Mr. Trump's campaign sought to
temper the culture war messaging at the Republican National Convention
last month by
\href{https://www.nytimes3xbfgragh.onion/2020/08/29/us/politics/rnc-trump-race-gender.html}{showcasing
Black and Hispanic supporters who denied that he is a racist}, the
president himself has increasingly made appeals to the grievances of
white supporters a centerpiece of his campaign to win a second term.

The message appears designed to galvanize supporters who have cheered
what they see as a defiant stand against political correctness since the
days when he kicked off his last presidential campaign in 2015 by
denouncing, without evidence, Mexicans crossing the border as
``rapists.'' While he initially voiced concern over
\href{https://www.nytimes3xbfgragh.onion/2020/05/31/us/george-floyd-investigation.html}{the
killing of George Floyd} under the knee of a white police officer in
Minneapolis this spring, which touched off nationwide protests, he has
focused since then almost entirely on defending the police and
condemning demonstrations during which there have been outbreaks of
looting and violence.

He has described American cities as hotbeds of chaos,
\href{https://www.nytimes3xbfgragh.onion/2020/07/30/upshot/trump-suburban-voters.html}{played
to ``suburban housewives''} he casts as fearful of low-income people
moving into their neighborhoods,
\href{https://www.nytimes3xbfgragh.onion/2020/06/10/us/politics/trump-rejects-renaming-military-bases.html}{sought
to block a move --- backed by the Pentagon and Republican lawmakers} ---
to rename Army bases named for Confederate generals,
\href{https://www.nytimes3xbfgragh.onion/2020/07/06/us/politics/trump-bubba-wallace-nascar.html}{criticized
NASCAR} for banning the Confederate flag, called Black Lives Matter a
``symbol of hate'' and vowed to strip funding from cities that do not
take what he deems tough enough action against protesters.

In effect, he is reaching out to a subset of white voters who think the
news media and political elites see Trump supporters as inherently
racist. Mr. Trump has repeatedly rejected the notion that America has a
problem with systemic racial bias, dismissing instances of police
brutality against Black Americans as the work of a few ``bad apples,''
in his words.

``Trump is the most extreme, and he has done something that is beyond
the bounds of anything we have seen,'' said Sherrilyn Ifill, the
president of the NAACP Legal Defense and Educational Fund. ``Playing
with racism is a dangerous game. It's not that you can do it a little
bit or do it slyly or do it with a dog whistle. It's all dangerous, and
it's all potentially violent.''

Aides said Mr. Trump's actions were aimed at eliminating pernicious
views that actually exacerbate prejudice. ``President Trump believes
that all men and women are created equal, and he will stand against
anti-American philosophies of all kinds that promote racial division,''
Kayleigh McEnany, the White House press secretary, said on Sunday.

Public views of Mr. Trump flow through a racial prism.
\href{https://drive.google.com/file/d/1hmvwY-EJIMWKveUWPTdpx_-JL6VTqCcx/view}{A
poll by CBS News} last week found that 66 percent of registered voters
believed Mr. Trump favored white people, versus 4 percent who said he
worked against their interests. By contrast, 20 percent thought he
favored Black people and 50 percent said he worked against Black people.
Among Black voters, 81 percent said he worked against their interests.

In the poll, Mr. Trump led former Vice President
\href{https://www.nytimes3xbfgragh.onion/interactive/2020/us/elections/joe-biden.html}{Joseph
R. Biden Jr.}, his Democratic challenger, among white voters by 51
percent to 43 percent, but trailed among Black voters with just 9
percent support, compared with 85 percent for Mr. Biden. Among Hispanic
voters, Mr. Biden led by 63 percent to 25 percent.

The president's latest order came as a
\href{https://www.nytimes3xbfgragh.onion/2020/09/06/us/politics/cohen-book-trump.html}{book
to be published on Tuesday by Michael D. Cohen}, his former lawyer and
fixer, describes a dismissive attitude toward nonwhite voters during the
2016 campaign. They were ``not my people,'' Mr. Cohen quotes Mr. Trump
as saying. ``I will never get the Hispanic vote,'' Mr. Trump added,
according to the book. ``Like the Blacks, they're too stupid to vote for
Trump.''

The president's approach in recent days seems to belie the Republican
convention programming that sought to soften his image on race by
featuring validators like Herschel Walker, the onetime football star,
and Vernon Jones, a Black Democratic state legislator from Georgia, who
testified to Mr. Trump's friendship and support for people of all races.

\includegraphics{https://static01.graylady3jvrrxbe.onion/images/2020/09/06/us/politics/06trump-race2/06trump-race2-articleLarge.jpg?quality=75\&auto=webp\&disable=upscale}

The president often makes the unfounded assertion that he has done more
for Black Americans than any president other perhaps than Abraham
Lincoln. He cites his
\href{https://www.nytimes3xbfgragh.onion/2019/09/10/us/politics/trump-black-colleges.html}{support
for funding for historically Black colleges and universities}, his
signature on
\href{https://www.nytimes3xbfgragh.onion/2019/04/01/us/politics/first-step-act-donald-trump.html}{legislation
overhauling criminal justice sentencing} and an unemployment rate for
Black people that dropped to record lows on his watch, continuing a
trend that had begun under his predecessor, until it rose again with the
pandemic-related economic slowdown.

But analysts said the convention had been aimed at making it easier for
white voters uncomfortable with Mr. Trump's history on race to support
him and that it might have appealed to nonwhite voters who bristle at
the so-called cancel culture that has become a favorite target of the
right.

Like other policies put forth with little advance notice, Mr. Trump's
focus on diversity training seems to have originated with something he
saw on Fox News. On Tuesday night, Tucker Carlson interviewed
Christopher F. Rufo, a conservative scholar at the Discovery Institute
who criticized what he called the ``cult indoctrination'' of ``critical
race theory'' programs in the government.

``It's absolutely astonishing how critical race theory has pervaded
every institution in the federal government, and what I've discovered is
that critical race theory has become in essence the default ideology of
the federal bureaucracy and is now being weaponized against the American
people,'' Mr. Rufo said on the program.

On his website, Mr. Rufo identified six agencies that had conducted
training sessions that he said asserted that America is inherently
racist and promoted concepts like unconscious bias, white privilege and
white fragility. At the Treasury Department, for instance, he said
employees had been told that ``virtually all white people contribute to
racism'' and that white staff members should ``struggle to own their
racism.''

Mr. Trump's memo on Friday adopted much of this language, attributing it
to ``press reports.'' The memo, signed by Russell T. Vought, the
director of the Office of Management and Budget, said ``this divisive,
false, and demeaning propaganda of the critical race theory movement is
contrary to all we stand for as Americans and should have no place in
the Federal government.''

Mr. Trump wrote or reposted roughly 20 Twitter messages about the memo
on Saturday and on Sunday said
\href{https://twitter.com/realDonaldTrump/status/1302586046551597061}{the
Education Department would investigate} schools that use curriculum from
the 1619 Project by The New York Times Magazine, an effort to look at
American history through the frame of slavery's consequences and the
contributions of Black Americans.

Critics said the president's move was a transparent play for white votes
with less than two months until the Nov. 3 election.

``To say antiracism is anti-American is to say racism is American, which
is to say Trump wants white Americans to be racist,'' said Ibram X.
Kendi, the author of ``How to Be an Anti-Racist'' and director of the
Boston University Center for Antiracist Research. ``And that's precisely
the point. He's relying on manipulating the racist fears of white voters
to win them over. Once white people lose those fears through
interventions like trainings, Trump loses their votes.''

Mr. Rufo, though, said on Sunday that Mr. Trump was pitting his America
First narrative celebrating the nation's heritage against what he called
the Black Lives Matter narrative that America was founded on racism.
``The president is framing the election for voters in these terms,'' he
said. ``Do they want to preserve the American way of life or do they
want to burn it down?''

Advertisement

\protect\hyperlink{after-bottom}{Continue reading the main story}

\hypertarget{site-index}{%
\subsection{Site Index}\label{site-index}}

\hypertarget{site-information-navigation}{%
\subsection{Site Information
Navigation}\label{site-information-navigation}}

\begin{itemize}
\tightlist
\item
  \href{https://help.nytimes3xbfgragh.onion/hc/en-us/articles/115014792127-Copyright-notice}{©~2020~The
  New York Times Company}
\end{itemize}

\begin{itemize}
\tightlist
\item
  \href{https://www.nytco.com/}{NYTCo}
\item
  \href{https://help.nytimes3xbfgragh.onion/hc/en-us/articles/115015385887-Contact-Us}{Contact
  Us}
\item
  \href{https://www.nytco.com/careers/}{Work with us}
\item
  \href{https://nytmediakit.com/}{Advertise}
\item
  \href{http://www.tbrandstudio.com/}{T Brand Studio}
\item
  \href{https://www.nytimes3xbfgragh.onion/privacy/cookie-policy\#how-do-i-manage-trackers}{Your
  Ad Choices}
\item
  \href{https://www.nytimes3xbfgragh.onion/privacy}{Privacy}
\item
  \href{https://help.nytimes3xbfgragh.onion/hc/en-us/articles/115014893428-Terms-of-service}{Terms
  of Service}
\item
  \href{https://help.nytimes3xbfgragh.onion/hc/en-us/articles/115014893968-Terms-of-sale}{Terms
  of Sale}
\item
  \href{https://spiderbites.nytimes3xbfgragh.onion}{Site Map}
\item
  \href{https://help.nytimes3xbfgragh.onion/hc/en-us}{Help}
\item
  \href{https://www.nytimes3xbfgragh.onion/subscription?campaignId=37WXW}{Subscriptions}
\end{itemize}
