Sections

SEARCH

\protect\hyperlink{site-content}{Skip to
content}\protect\hyperlink{site-index}{Skip to site index}

\href{https://www.nytimes3xbfgragh.onion/section/world}{World}

\href{https://myaccount.nytimes3xbfgragh.onion/auth/login?response_type=cookie\&client_id=vi}{}

\href{https://www.nytimes3xbfgragh.onion/section/todayspaper}{Today's
Paper}

\href{/section/world}{World}\textbar{}In Sign of Progress, Fewer Than
1\% of New York's Virus Tests Are Positive

\url{https://nyti.ms/2F4ldYb}

\begin{itemize}
\item
\item
\item
\item
\item
\end{itemize}

\hypertarget{the-coronavirus-outbreak}{%
\subsubsection{\texorpdfstring{\href{https://www.nytimes3xbfgragh.onion/news-event/coronavirus?name=styln-coronavirus-national\&region=TOP_BANNER\&block=storyline_menu_recirc\&action=click\&pgtype=Article\&impression_id=11146f00-f287-11ea-8c2a-475020e0db7d\&variant=undefined}{The
Coronavirus
Outbreak}}{The Coronavirus Outbreak}}\label{the-coronavirus-outbreak}}

\begin{itemize}
\tightlist
\item
  live\href{https://www.nytimes3xbfgragh.onion/2020/09/09/world/covid-19-coronavirus.html?name=styln-coronavirus-national\&region=TOP_BANNER\&block=storyline_menu_recirc\&action=click\&pgtype=Article\&impression_id=11146f01-f287-11ea-8c2a-475020e0db7d\&variant=undefined}{Latest
  Updates}
\item
  \href{https://www.nytimes3xbfgragh.onion/interactive/2020/us/coronavirus-us-cases.html?name=styln-coronavirus-national\&region=TOP_BANNER\&block=storyline_menu_recirc\&action=click\&pgtype=Article\&impression_id=11146f02-f287-11ea-8c2a-475020e0db7d\&variant=undefined}{Maps
  and Cases}
\item
  \href{https://www.nytimes3xbfgragh.onion/interactive/2020/science/coronavirus-vaccine-tracker.html?name=styln-coronavirus-national\&region=TOP_BANNER\&block=storyline_menu_recirc\&action=click\&pgtype=Article\&impression_id=11146f03-f287-11ea-8c2a-475020e0db7d\&variant=undefined}{Vaccine
  Tracker}
\item
  \href{https://www.nytimes3xbfgragh.onion/2020/09/02/your-money/eviction-moratorium-covid.html?name=styln-coronavirus-national\&region=TOP_BANNER\&block=storyline_menu_recirc\&action=click\&pgtype=Article\&impression_id=11149610-f287-11ea-8c2a-475020e0db7d\&variant=undefined}{Eviction
  Moratorium}
\item
  \href{https://www.nytimes3xbfgragh.onion/interactive/2020/09/02/magazine/food-insecurity-hunger-us.html?name=styln-coronavirus-national\&region=TOP_BANNER\&block=storyline_menu_recirc\&action=click\&pgtype=Article\&impression_id=11149611-f287-11ea-8c2a-475020e0db7d\&variant=undefined}{American
  Hunger}
\end{itemize}

Advertisement

\protect\hyperlink{after-top}{Continue reading the main story}

Supported by

\protect\hyperlink{after-sponsor}{Continue reading the main story}

\hypertarget{in-sign-of-progress-fewer-than-1-of-new-yorks-virus-tests-are-positive}{%
\section{In Sign of Progress, Fewer Than 1\% of New York's Virus Tests
Are
Positive}\label{in-sign-of-progress-fewer-than-1-of-new-yorks-virus-tests-are-positive}}

The state's share of positive tests has stayed below 1 percent for 30
straight days, Gov. Andrew M. Cuomo said, though he urged caution amid
Labor Day celebrations. India surpassed Brazil to become the country
with the second-highest number of cases.

\begin{itemize}
\item
  Published Sept. 6, 2020Updated Sept. 8, 2020
\item
  \begin{itemize}
  \item
  \item
  \item
  \item
  \item
  \end{itemize}
\end{itemize}

This briefing has ended. Read
\href{https://www.nytimes3xbfgragh.onion/2020/09/08/world/covid-19-coronavirus.html}{live
coronavirus updates here}.

\hypertarget{heres-what-you-need-to-know}{%
\subsubsection{Here's what you need to
know:}\label{heres-what-you-need-to-know}}

\begin{itemize}
\tightlist
\item
  \protect\hyperlink{link-32da45cc}{`Be smart': New Yorkers get good
  news on virus numbers, but also a warning.}
\item
  \protect\hyperlink{link-5e4c215d}{Skyrocketing cases push India to No.
  2 on the pandemic list.}
\item
  \protect\hyperlink{link-38fff5b5}{Britain is stunned by a spike in
  daily cases.}
\item
  \protect\hyperlink{link-69b06270}{Kamala Harris expresses distrust of
  any vaccine promoted by President Trump.}
\item
  \protect\hyperlink{link-15f638e1}{As hope builds over possible
  frequent at-home testing, experts call the idea a long shot.}
\item
  \protect\hyperlink{link-7cc375a8}{Chinese firms are testing vaccines
  on their own employees. One executive says it's working.}
\item
  \protect\hyperlink{link-72bd89e3}{A university known for coronavirus
  research warns its scientists to look out for suspicious packages.}
\end{itemize}

\includegraphics{https://static01.graylady3jvrrxbe.onion/images/2020/09/06/business/06virus-briefing-nyc/merlin_175676598_b9bcbbc4-03db-48a0-a577-02fafeed2836-articleLarge.jpg?quality=75\&auto=webp\&disable=upscale}

\hypertarget{be-smart-new-yorkers-get-good-news-on-virus-numbers-but-also-a-warning}{%
\subsection{`Be smart': New Yorkers get good news on virus numbers, but
also a
warning.}\label{be-smart-new-yorkers-get-good-news-on-virus-numbers-but-also-a-warning}}

The share of virus tests coming back positive in New York State has
stayed below 1 percent for 30 straight days, suggesting that the state's
aggressive approach to containing its outbreak --- once the most severe
in the country --- has largely worked.

The state's positivity rate, announced on Sunday, remained below 1
percent even as parts of the economy gradually reopened, the number of
people being tested continued to trend upward, and other states grappled
with sharply rising case counts.

But for all the encouragement offered by the monthlong marker, many New
Yorkers remain anxious heading into the fall and winter, when case
counts could rise as the nation's largest public school district and
more businesses are preparing to reopen.

Even Gov. Andrew M. Cuomo, in announcing the figure, took the
opportunity to urge not celebration but continued restraint, pointing in
a statement to New York's approach to reopening --- slower and more
controlled than in most other states --- as well as its statewide mask
mandate.

``Caution is a virtue, not a vice,'' Mr. Cuomo said.

New York State is now averaging slightly more than 700 cases a day,
\href{https://www.nytimes3xbfgragh.onion/interactive/2020/us/new-york-coronavirus-cases.html}{according
to a New York Times database} --- up a bit from about 600 in late
August, but still a fraction of the 9,000 to 10,000 cases a day it was
reporting at the peak in April. The number of people in hospitals
because of the virus dropped to 410 on Saturday, the lowest figure since
March 16.

The governor's announcement came in the middle of a holiday weekend
that, like others before it, seemed certain to tempt many to gather
socially as the summer wanes. Mr. Cuomo warned that the state's gains
could be imperiled by any backsliding on precautions like mask wearing
and social distancing.

``Our actions today determine the rate of infection tomorrow,'' he said.
``So as the Labor Day weekend continues, I urge everyone to be smart, so
we don't see a spike in the weeks ahead.''

A
\href{https://slack-redir.net/link?url=https\%3A\%2F\%2Fwww.nytimes3xbfgragh.onion\%2F2020\%2F09\%2F03\%2Fnyregion\%2Fnew-york-suny-oneonta-coronavirus.html}{recent
outbreak} at the State University of New York at Oneonta, a public
college in Central New York, showed how quickly new clusters can flare
up.

After some students held large parties, more than 500 students there
tested positive; officials canceled in-person instruction for the
semester less than two weeks after it began, closed the school's
dormitories and sent students home. On Sunday, Mr. Cuomo said a state
rapid testing team sent to the City of Oneonta had found 91 more cases,
largely among college-age adults.

At New York University, more than 20 students have been suspended for
violating virus-related rules, the school
\href{https://twitter.com/nyuniversity/status/1302380699010433026}{announced
on Twitter on Saturday}.

\hypertarget{skyrocketing-cases-push-india-to-no-2-on-the-pandemic-list}{%
\subsection{Skyrocketing cases push India to No. 2 on the pandemic
list.}\label{skyrocketing-cases-push-india-to-no-2-on-the-pandemic-list}}

Image

Testing for the coronavirus in New Delhi on Saturday. India has the
fastest-growing outbreak in the world.Credit...Manish Swarup/Associated
Press

India, home to the world's fastest-growing coronavirus outbreak, has
surpassed Brazil to become the country with the second-highest number of
cases.

On Monday, India reported 90,802 new infections, breaking its own record
from the day before and taking its total to more than 4.2 million,
according to a
\href{https://www.nytimes3xbfgragh.onion/interactive/2020/world/asia/india-coronavirus-cases.html}{New
York Times database}. Brazil is now third with more than 4.1 million
cases.

In early July, India surpassed Russia to become the country with the
third-highest number of cases. By then, the United States was entrenched
at No. 1, where it remains today with more than 6.2 million cases.

``Crowded cities, lockdown fatigue and a lack of contact tracing have
spread Covid-19 to every corner of this country of 1.3 billion people,''
The Times's Jeffrey Gettleman and Sameer Yasir
\href{https://www.nytimes3xbfgragh.onion/2020/08/28/world/asia/india-coronavirus.html}{reported}
in late August.

India has recorded 71,642 deaths from the virus, the world's
third-highest toll after the United States and Brazil, though India has
a relatively low death rate per capita in a youthful nation.

The country's surge in cases comes as the government continues to ease
lockdown measures in an effort to help the economy. On Monday, the
subway system in New Delhi, the capital, began a
\href{http://www.delhimetrorail.com/press_reldetails.aspx?id=YEUIu1E8HMklld}{phased
reopening} after being shut for more than five months.

The pandemic has been economically devastating for India, which not so
long ago
\href{https://www.nytimes3xbfgragh.onion/2020/09/05/world/asia/india-economy-coronavirus.html}{dreamed
of becoming a global powerhouse}. Last week, the government
\href{https://www.nytimes3xbfgragh.onion/2020/08/31/world/asia/india-economy-gdp.html}{reported
a 24 percent contraction} in the second quarter, the worst among the
world's top economies.

\hypertarget{britain-is-stunned-by-a-spike-in-daily-cases}{%
\subsection{Britain is stunned by a spike in daily
cases.}\label{britain-is-stunned-by-a-spike-in-daily-cases}}

Image

London Bridge during rush hour last week.Credit...Henry Nicholls/Reuters

British health officials on Sunday announced a sharp rise in new
infections, prompting warnings that they may need to reconsider the
aggressive reopening of the country.

The British Public Health agency reported
that\href{https://coronavirus.data.gov.uk/?_ga=2.212010196.1361742393.1599439298-600955461.1599439298}{2,998
new cases had been confirmed} --- the highest number since late May,
during the British outbreak's peak.

Amid criticism that the government had once again lost control of an
outbreak that has already killed more than 41,000 in Britain, government
officials signaled that they were prepared to crack down.

``We'll take whatever action is necessary,''
\href{https://www.bbc.com/news/uk-54050342}{said Matt Hancock}, the
health secretary, declaring that ``we can use, and we will use, local
lockdowns if that's what's necessary.''

But noting that as is the case in many parts of the world, the newest
outbreak is hitting mostly younger people, Mr. Hancock implored them to
think of their grandparents and be vigilant.

``The first line of defense is that people should follow social
distancing,'' he said.

There have been
\href{https://www.nytimes3xbfgragh.onion/interactive/2020/world/europe/united-kingdom-coronavirus-cases.html}{almost
350,000 coronavirus cases} in Britain, which was initially reluctant to
acknowledge the threat posed by the outbreak and act decisively to shut
down. It suffered some of the worst losses in Europe in April and May,
but gradually cases began to decline after the government moved to lock
down.

In August, however, cases began rising again.

With
\href{https://www.nytimes3xbfgragh.onion/2020/08/29/world/europe/britain-schools-reopen-cornavirus.html}{schools
newly reopened}, some British experts are sounding the alarm over the
newest infection numbers.

``They've lost control of the virus,'' Gabriel Scally, a former National
Health Service official,
\href{https://www.theguardian.com/world/2020/sep/06/coronavirus-fears-uk-government-lost-control-cases-soar}{told
The Guardian}.

\hypertarget{kamala-harris-expresses-distrust-of-any-vaccine-promoted-by-president-trump}{%
\subsection{Kamala Harris expresses distrust of any vaccine promoted by
President
Trump.}\label{kamala-harris-expresses-distrust-of-any-vaccine-promoted-by-president-trump}}

Image

Senator Kamala Harris of California in August.~Credit...Kenny Holston
for The New York Times

Senator Kamala Harris of California, the Democratic nominee for vice
president, said she would not trust President Trump's assurances that a
coronavirus vaccine was safe, and instead would wait for medical experts
to confirm the vaccine was reliable before she received an inoculation.

``I will not take his word for it,'' Ms. Harris said of Mr. Trump
\href{https://www.cnn.com/2020/09/05/politics/kamala-harris-not-trust-trump-vaccine-cnntv/index.html}{on
the CNN program ``State of the Union.''}(An earlier version of this
article misidentified the program as ``Inside Politics.'')

\hypertarget{latest-updates-the-coronavirus-outbreak}{%
\section{\texorpdfstring{\href{https://www.nytimes3xbfgragh.onion/2020/09/09/world/covid-19-coronavirus.html?action=click\&pgtype=Article\&state=default\&region=MAIN_CONTENT_1\&context=storylines_live_updates}{Latest
Updates: The Coronavirus
Outbreak}}{Latest Updates: The Coronavirus Outbreak}}\label{latest-updates-the-coronavirus-outbreak}}

Updated 2020-09-09T10:05:44.525Z

\begin{itemize}
\tightlist
\item
  \href{https://www.nytimes3xbfgragh.onion/2020/09/09/world/covid-19-coronavirus.html?action=click\&pgtype=Article\&state=default\&region=MAIN_CONTENT_1\&context=storylines_live_updates\#link-70cea8bb}{As
  drugmakers pledge to thoroughly vet a vaccine, one company pauses its
  trials for a safety review.}
\item
  \href{https://www.nytimes3xbfgragh.onion/2020/09/09/world/covid-19-coronavirus.html?action=click\&pgtype=Article\&state=default\&region=MAIN_CONTENT_1\&context=storylines_live_updates\#link-780eaa2f}{Britain
  is expected to ban gatherings of more than six people.}
\item
  \href{https://www.nytimes3xbfgragh.onion/2020/09/09/world/covid-19-coronavirus.html?action=click\&pgtype=Article\&state=default\&region=MAIN_CONTENT_1\&context=storylines_live_updates\#link-11cec4c0}{Quarantine
  breakdowns at colleges in the U.S. are leaving some at risk.}
\end{itemize}

\href{https://www.nytimes3xbfgragh.onion/2020/09/09/world/covid-19-coronavirus.html?action=click\&pgtype=Article\&state=default\&region=MAIN_CONTENT_1\&context=storylines_live_updates}{See
more updates}

More live coverage:
\href{https://www.nytimes3xbfgragh.onion/live/2020/09/08/business/stock-market-today-coronavirus?action=click\&pgtype=Article\&state=default\&region=MAIN_CONTENT_1\&context=storylines_live_updates}{Markets}

``He wants us to inject bleach,'' she added, referring to remarks in
April when the president
\href{https://www.nytimes3xbfgragh.onion/2020/04/26/us/politics/trump-disinfectant-coronavirus.html}{incomprehensibly
suggested a dangerous coronavirus treatment}.

Ms. Harris's remarks came after federal officials alerted state and
major city public health agencies last week to prepare to distribute a
vaccine to health care workers and other high-risk groups
\href{https://www.nytimes3xbfgragh.onion/2020/09/02/health/covid-19-vaccine-cdc-plans.html}{as
soon as late October or early November}. Given that
\href{https://www.nytimes3xbfgragh.onion/interactive/2020/science/coronavirus-vaccine-tracker.html}{no
vaccine candidates have completed the kind of large-scale human trials}
that can prove efficacy and safety, that time frame has heightened
concerns that the Trump administration is seeking to rush a vaccine
rollout ahead of Election Day, Nov. 3.

For months, Ms. Harris and Joseph R. Biden Jr. have assailed Mr. Trump
for his handling of the coronavirus crisis. Ms. Harris's comments on
Sunday questioning a potential vaccine, as scientists racing for a
vaccine
\href{https://www.nytimes3xbfgragh.onion/2020/08/02/us/politics/coronavirus-vaccine.html}{report
constant pressure} from a White House anxious for good news, are likely
to
\href{https://www.nytimes3xbfgragh.onion/2020/07/18/health/coronavirus-anti-vaccine.html}{further
sow skepticism} among Americans considering whether to get the vaccine
when it becomes available.

With concern about the politicization of vaccines and treatments on the
rise, five drug companies are preparing to issue a statement this week
\href{https://www.nytimes3xbfgragh.onion/2020/09/04/science/covid-vaccine-pharma-pledge.html}{pledging
to not release a vaccine} unless it meets rigorous standards for
effectiveness and safety. The companies --- Pfizer, Moderna, Johnson \&
Johnson, GlaxoSmithKline and Sanofi --- are aiming to reassure the
public that they will not seek premature approval under political
pressure.

Ms. Harris on Sunday also said she and Mr. Biden would set a national
``standard'' for mask wearing, stopping short of endorsing a mandate.

``This is not about punishment. It's not about Big Brother,'' Ms. Harris
said, adding that wearing a mask is a ``sacrifice'' in a time of crisis.

Her comments appeared to be a softening of the position she and Mr.
Biden have previously staked out.

Last month, Mr. Biden and Ms. Harris
\href{https://www.nytimes3xbfgragh.onion/2020/08/13/us/politics/wear-masks-mandate-biden.html}{called
for Americans to be required to wear masks}, telling reporters after
receiving a briefing from public health experts that every American
should wear a mask while outside for at least the next three months and
that all governors should mandate mask wearing.

Mr. Biden in July suggested that if he were president, he
\href{https://www.nytimes3xbfgragh.onion/2020/06/25/us/politics/biden-speech-trump-coronavirus.html}{would
require mask wearing in public}, and, asked if he could use ``federal
leverage to mandate that,'' said he could, and ``would from an executive
standpoint.''

\href{https://www.nytimes3xbfgragh.onion/interactive/2020/science/coronavirus-vaccine-tracker.html}{}

\includegraphics{https://static01.graylady3jvrrxbe.onion/images/2020/09/03/us/coronavirus-vaccine-tracker-promo-1599144009786/coronavirus-vaccine-tracker-promo-1599144009786-articleLarge.png}

\hypertarget{coronavirus-vaccine-tracker}{%
\subsection{Coronavirus Vaccine
Tracker}\label{coronavirus-vaccine-tracker}}

A look at all the vaccines that have reached trials in humans.

\hypertarget{as-hope-builds-over-possible-frequent-at-home-testing-experts-call-the-idea-a-long-shot}{%
\subsection{As hope builds over possible frequent at-home testing,
experts call the idea a long
shot.}\label{as-hope-builds-over-possible-frequent-at-home-testing-experts-call-the-idea-a-long-shot}}

Image

Rapid antigen tests are not effective at picking up low-level
infections, but are cheap and convenient.Credit...Dibyangshu
Sarkar/Agence France-Presse --- Getty Images

Over the past few weeks, a Harvard scientist has made headlines for a
bold idea to curb the spread of the virus: rolling out antigen tests, a
\href{https://www.ncbi.nlm.nih.gov/pmc/articles/PMC7119943/}{decades-old}
underdog in testing technology, to tens of millions of Americans for
near-daily, at-home use.

These tests
\href{https://www.nytimes3xbfgragh.onion/2020/08/06/health/rapid-Covid-tests.html}{are
not very good at picking up low-level infections}. But they are cheap
and convenient, and return results in minutes. Real-time information,
\href{https://www.nytimes3xbfgragh.onion/2020/07/03/opinion/coronavirus-tests.html}{argued}
Dr. Michael Mina, the Harvard scientist, would be far better than the
\href{https://www.nytimes3xbfgragh.onion/2020/07/23/health/coronavirus-testing-supply-shortage.html}{long
delays} clogging the testing pipeline.

The fast-and-frequent approach to testing has captured the attention of
\href{https://www.microbe.tv/twiv/twiv-640/}{scientists} and journalists
around the world, and that of
\href{https://twitter.com/HHS_ASH/status/1300146630201610240}{top
officials} at the Department of Health and Human Services.

But more than a dozen experts said that near-ubiquitous antigen testing,
while
\href{https://www.medrxiv.org/content/10.1101/2020.06.22.20136309v2}{intriguing
in theory}, may not be effective in practice. In addition to posing huge
logistical hurdles, they said, the plan hinges on broad buy-in and
compliance from people who have grown increasingly disillusioned with
coronavirus testing. The aim also assumes that rapid tests
\href{https://jcm.asm.org/content/early/2020/08/07/JCM.01695-20}{can
achieve their intended purpose}.

``We are open to thinking outside the box and coming up with new ways to
handle this pandemic,'' said Esther Babady, the director of the clinical
microbiology service at Memorial Sloan Kettering Cancer Center in New
York. But she said antigen tests that could work at home had yet to
enter the market.

Also, no rigorous study has shown that fast and frequent testing is
better than sensitive but slower in the real world, she said. ``The data
for that is what's missing.''

What has been put forth about the approach is ``largely aspirational,
and we need to check it against reality,'' said Dr. Alexander McAdam,
the director of the infectious diseases diagnostic laboratory at Boston
Children's Hospital and an author of
\href{https://jcm.asm.org/content/early/2020/08/24/JCM.02225-20}{a
recent report on pandemic testing strategies} in The Journal of Clinical
Microbiology.

Most of the virus tests to date rely on a laboratory technique called
PCR, long considered the gold standard because it can pick up even small
amounts of genetic material from germs like the coronavirus.

But sputtering supply chains have compromised efforts to collect, ship
and process samples for PCR tests, lengthening turnaround times. And the
longer the wait, the less useful the result.

\hypertarget{tracking-the-coronavirus-}{%
\subsection{\texorpdfstring{\href{https://www.nytimes3xbfgragh.onion/interactive/2020/us/coronavirus-us-cases.html}{Tracking
the Coronavirus
›}}{Tracking the Coronavirus ›}}\label{tracking-the-coronavirus-}}

\href{https://www.nytimes3xbfgragh.onion/interactive/2020/us/coronavirus-us-cases.html}{}

\hypertarget{where-cases-are-highest-per-capita}{%
\subsubsection{\texorpdfstring{Where cases are \textbf{highest} per
capita}{Where cases are highest per capita}}\label{where-cases-are-highest-per-capita}}

\href{https://www.nytimes3xbfgragh.onion/interactive/2020/us/north-dakota-coronavirus-cases.html}{}

N.D.
\href{https://www.nytimes3xbfgragh.onion/interactive/2020/us/south-dakota-coronavirus-cases.html}{}

S.D.
\href{https://www.nytimes3xbfgragh.onion/interactive/2020/us/iowa-coronavirus-cases.html}{}

Iowa
\href{https://www.nytimes3xbfgragh.onion/interactive/2020/us/missouri-coronavirus-cases.html}{}

Mo.
\href{https://www.nytimes3xbfgragh.onion/interactive/2020/us/kansas-coronavirus-cases.html}{}

Kan.
\href{https://www.nytimes3xbfgragh.onion/interactive/2020/us/arkansas-coronavirus-cases.html}{}

Ark.
\href{https://www.nytimes3xbfgragh.onion/interactive/2020/us/tennessee-coronavirus-cases.html}{}

Tenn.
\href{https://www.nytimes3xbfgragh.onion/interactive/2020/us/oklahoma-coronavirus-cases.html}{}

Okla.
\href{https://www.nytimes3xbfgragh.onion/interactive/2020/us/mississippi-coronavirus-cases.html}{}

Miss.
\href{https://www.nytimes3xbfgragh.onion/interactive/2020/us/illinois-coronavirus-cases.html}{}

Ill.
\href{https://www.nytimes3xbfgragh.onion/interactive/2020/us/alabama-coronavirus-cases.html}{}

Ala.
\href{https://www.nytimes3xbfgragh.onion/interactive/2020/us/south-carolina-coronavirus-cases.html}{}

S.C.

\href{https://www.nytimes3xbfgragh.onion/interactive/2020/us/coronavirus-us-cases.html}{}

\hypertarget{us-hot-spots-}{%
\subsubsection{U.S. hot spots ›}\label{us-hot-spots-}}

\includegraphics{https://static01.graylady3jvrrxbe.onion/newsgraphics/2020/03/16/coronavirus-maps/f524c310078e698fa7711474553092d65f5a5d85/images/orphan_usa-threeByTwoSmallAt2X.png}
\href{https://www.nytimes3xbfgragh.onion/interactive/2020/07/28/us/covid-19-colleges-universities.html}{}

\hypertarget{college-cases-}{%
\subsubsection{College cases ›}\label{college-cases-}}

\includegraphics{https://static01.graylady3jvrrxbe.onion/newsgraphics/2020/03/16/coronavirus-maps/f524c310078e698fa7711474553092d65f5a5d85/images/orphan_colleges-threeByTwoSmallAt2X.png}

\hypertarget{chinese-firms-are-testing-vaccines-on-their-own-employees-one-executive-says-its-working}{%
\subsection{Chinese firms are testing vaccines on their own employees.
One executive says it's
working.}\label{chinese-firms-are-testing-vaccines-on-their-own-employees-one-executive-says-its-working}}

Image

The Chinese pharmaceutical company Sinopharm says it has been testing a
coronavirus vaccine on its employees.~Credit...Mark
Schiefelbein/Associated Press

A Chinese pharmaceutical company, which has tested coronavirus vaccines
on its own employees, said the workers traveled to countries with large
outbreaks without becoming infected.

Zhou Song, the general counsel of the vaccine manufacturer Sinopharm,
suggested on Sunday that the vaccines, which are still in the final
stages of testing, might be effective in controlling the virus. But it
will be months before any final conclusions can be drawn, and the
employee data cannot be used to obtain regulatory approvals.

Under an ``emergency use'' program approved by the Chinese government in
July, a broad array of people considered to be at high risk of virus
exposure, including border officials, soldiers, medical personnel and
employees of state-owned companies, are allowed to
\href{https://www.nytimes3xbfgragh.onion/2020/07/16/business/china-vaccine-coronavirus.html}{receive
unapproved coronavirus vaccines} outside of official clinical trials.
Chinese vaccine makers are also conducting clinical trials according to
normal regulatory processes in Brazil and other countries that ---
unlike China --- have large, active outbreaks.

Mr. Zhou did not say to which countries the Sinopharm employees had
traveled, or specify which vaccine they received. The state-owned
company has
\href{https://www.nytimes3xbfgragh.onion/interactive/2020/science/coronavirus-vaccine-tracker.html}{two
vaccines in Phase 3 trials}.

It is also unclear whether the employees who received the vaccine had
mingled with locals on their trips abroad, increasing their chances of
exposure, or had been sequestered to their living quarters. If they
avoided infection by keeping to themselves, that would not prove the
vaccine works.

In an interview with eastday.com, a Shanghai-based news website, Mr.
Zhou said that the absence of infections among inoculated employees was
``a remarkable thing.''

He also said that none of the workers had shown any serious adverse
reactions, and that ``if one is optimistic, the vaccines could be
launched by the end of the year.''

Separately, Sinovac, a Beijing-based company that also has a coronavirus
vaccine in the last stage of testing, said that almost all of its
employees and their family members --- around 3,000 people --- had been
vaccinated on a voluntary basis under the emergency use program, The
South China Morning Post
\href{https://www.scmp.com/news/china/science/article/3100447/chinese-drug-firm-sinovac-says-thousands-employees-and-their}{reported
on Sunday}. Yin Weidong, the chief executive of Sinovac, said he
expected the vaccine to be approved for use as soon as the end of the
year.

\hypertarget{a-university-known-for-coronavirus-research-warns-its-scientists-to-look-out-for-suspicious-packages}{%
\subsection{A university known for coronavirus research warns its
scientists to look out for suspicious
packages.}\label{a-university-known-for-coronavirus-research-warns-its-scientists-to-look-out-for-suspicious-packages}}

Since the coronavirus pandemic started, public health officials in the
United States have faced harassment and death threats, and some have
even been
\href{https://www.nytimes3xbfgragh.onion/2020/06/22/us/coronavirus-health-officials.html}{driven
from office}. Now a university deeply involved in studying the virus has
warned hundreds of its researchers to be on the lookout for dangerous
packages.

Last Monday, the University of Washington, based in Seattle, sent an
email to about 500 of its researchers telling them to be wary of
suspicious packages and saying that virus researchers elsewhere had been
targeted.

``We have received unfortunate reports from our contacts at the Federal
Bureau of Investigations (FBI) that threatening mail has been sent to
COVID-19 researchers on the east coast of the United States,'' said the
email, which was first reported by
\href{https://www.buzzfeednews.com/article/kenbensinger/suspicious-package-covid-19-researchers-fbi}{BuzzFeed
News on Saturday}.

The BuzzFeed News article quoted an F.B.I. spokesman saying that the
bureau, ``along with our local law enforcement partners, responded to a
suspicious package sent to a few university researchers'' and that
``preliminary testing has indicated there is no threat to public safety
in connection with this mailing.''

A University of Washington spokeswoman, Susan Gregg, provided a copy of
the university's email to The New York Times and said no suspicious
packages had been reported so far.

The email warned researchers to be on the lookout for signs of
suspicious mail, including an address with misspelled words, no return
address, oily stains, discoloration or a strange odor. Any mail that
raised concerns, the email said, should be left unopened and reported to
the police by calling 911.

Research at the University of Washington includes
\href{https://www.iths.org/iths-covid-19-research-resources/current-covid-19-research/}{16
clinical studies related to the virus} and a prominent but sometimes
criticized forecasting model. The model estimated last week that
Covid-19 would kill about 410,000 people in the United States by the end
of the year, more than double the current death toll,
\href{https://twitter.com/CT_Bergstrom/status/1301738951078629376}{drawing
skepticism from experts} who said predictions about the course of the
pandemic months into the future are too uncertain to be useful.

The report of threats to researchers follows earlier signs of the risks
faced by public health officials and others involved in the pandemic
response. Dr. Anthony Fauci, a member of President Trump's virus task
force and the nation's leading expert on infectious diseases,
\href{https://www.nytimes3xbfgragh.onion/2020/04/01/us/politics/coronavirus-fauci-security.html}{received
additional security in April} after threats, and he said the security
was also
\href{https://www.cnbc.com/2020/08/05/dr-fauci-says-his-daughters-need-security-as-family-continues-to-get-death-threats.html}{expanded
to his daughters}. Local and state health officials have
\href{https://www.nytimes3xbfgragh.onion/2020/06/22/us/coronavirus-health-officials.html}{also
been targeted} by those challenging public health measures.

\hypertarget{after-earlier-post-holiday-spikes-in-cases-a-warning-for-labor-day-weekend}{%
\subsection{After earlier post-holiday spikes in cases, a warning for
Labor Day
weekend.}\label{after-earlier-post-holiday-spikes-in-cases-a-warning-for-labor-day-weekend}}

Image

Taking samples at a coronavirus testing site in Los Angeles on Friday.
Testing sites in the county were closed over the holiday
weekend.Credit...Frederic J. Brown/Agence France-Presse --- Getty Images

For many Americans, Labor Day is a goodbye to summer before children go
back to school and cold weather arrives. But public health experts worry
that in the midst of a pandemic, this weekend could result in disaster
in the fall.

After the Memorial Day and Fourth of July weekends, cases of Covid-19
surged around the United States after people held family gatherings or
congregated in large groups.

Dr. Anthony S. Fauci, the country's top infectious disease expert, said
he wanted people to enjoy Labor Day weekend, but urged precautions.

``You don't want to tell people on a holiday weekend that even outdoors
is bad --- they will get completely discouraged,'' Dr. Fauci said.
``What we try to say is enjoy outdoors, but you can do it with safe
spacing. You can be on a beach, and you don't have to be falling all
over each other. You can be six, seven, eight, nine or 10 feet apart.
You can go on a hike. You can go on a run. You can go on a picnic with a
few people. You don't have to be in a crowd with 30, 40 or 50 people all
breathing on each other.''

In terms of daily case counts, the United States is in worse shape going
into Labor Day weekend than it was for Memorial Day weekend. The nation
now averages about 40,000 new confirmed cases per day, up from about
22,000 per day ahead of Memorial Day weekend.

Colleges are struggling to keep students from breaking safety protocols,
and many have seen significant outbreaks, as have many college towns.
ABC News posted a video on Twitter showing crowds at a sports bar near
the University of South Carolina. The university, which
\href{https://www.nytimes3xbfgragh.onion/2020/09/01/us/usc-covid-restrictions-parties.html}{disciplined
some of its Greek houses last week}, has reported more than 1,735 cases
since Aug. 1, including 1,461 active cases, according to its
\href{https://www.sc.edu/safety/coronavirus/dashboard/index.php}{Covid-19
dashboard}.

Dr. Fauci said that a spike in infections after Labor Day would make it
far harder to control the virus's spread in the fall, when cooler
temperatures force more people indoors.

Public health experts said it was more challenging to persuade people to
curtail their Labor Day weekend plans compared with past holiday
weekends, because so many people are feeling pandemic fatigue after six
months of restrictions, closures and separation.

``People are getting tired of taking these precautions and of having
their lives upended,'' said
\href{https://www.bu.edu/sph/profile/eleanor-murray/}{Eleanor J.
Murray,} an assistant professor of epidemiology at the Boston University
School of Public Health. ``They're missing their friends and family, and
everyone wishes things were back to normal. That's totally
understandable, but unfortunately we don't get a say, really.''

Even so, there are signs that one pandemic precaution ---~mask wearing
---~has gained increasing acceptance over the summer.
\href{https://www.pewresearch.org/fact-tank/2020/08/27/more-americans-say-they-are-regularly-wearing-masks-in-stores-and-other-businesses/}{A
Pew Research Center survey} found that 85 percent of Americans said they
wore masks all or most of the time when in stores or businesses,
compared with 65 percent in June.

\hypertarget{new-york-plans-more-post-mortem-tests-for-the-coronavirus-and-the-flu-to-help-increase-death-data-accuracy}{%
\subsection{New York plans more post-mortem tests for the coronavirus
and the flu to help increase `death data'
accuracy.}\label{new-york-plans-more-post-mortem-tests-for-the-coronavirus-and-the-flu-to-help-increase-death-data-accuracy}}

Image

Refrigerated trucks used as makeshift morgues for Covid-19 fatalities in
Manhattan in May.Credit...Hiroko Masuike/The New York Times

With fall fast approaching, symptoms alone will not be useful in
distinguishing the coronavirus from similar-looking cases of the flu.
That means routinely testing for both viruses will be crucial --- even,
perhaps, after some patients have died.

In New York, officials
\href{https://health.ny.gov/press/releases/2020/2020-08-31_postmortem_covid19_influenza_testing.htm}{recently
announced} a
\href{https://regs.health.ny.gov/sites/default/files/pdf/emergency_regulations/Confirmatory-testing-regs-FINAL.pdf}{ramp-up
in post-mortem testing} for the coronavirus as well as for the flu.
Deaths linked to respiratory illnesses that were not confirmed before a
person died are to be followed up with tests for both viruses within 48
hours, according to the new regulation.

``These regulations will ensure we have the most accurate death data
possible as we continue to manage Covid-19 while preparing for flu
season,'' Dr. Howard Zucker, the state's health commissioner, said
\href{https://health.ny.gov/press/releases/2020/2020-08-31_postmortem_covid19_influenza_testing.htm}{in
a statement} last week.

\href{https://www.nytimes3xbfgragh.onion/news-event/coronavirus?action=click\&pgtype=Article\&state=default\&region=MAIN_CONTENT_3\&context=storylines_faq}{}

\hypertarget{the-coronavirus-outbreak-}{%
\subsubsection{The Coronavirus Outbreak
›}\label{the-coronavirus-outbreak-}}

\hypertarget{frequently-asked-questions}{%
\paragraph{Frequently Asked
Questions}\label{frequently-asked-questions}}

Updated September 4, 2020

\begin{itemize}
\item ~
  \hypertarget{what-are-the-symptoms-of-coronavirus}{%
  \paragraph{What are the symptoms of
  coronavirus?}\label{what-are-the-symptoms-of-coronavirus}}

  \begin{itemize}
  \tightlist
  \item
    In the beginning, the coronavirus
    \href{https://www.nytimes3xbfgragh.onion/article/coronavirus-facts-history.html?action=click\&pgtype=Article\&state=default\&region=MAIN_CONTENT_3\&context=storylines_faq\#link-6817bab5}{seemed
    like it was primarily a respiratory illness}~--- many patients had
    fever and chills, were weak and tired, and coughed a lot, though
    some people don't show many symptoms at all. Those who seemed
    sickest had pneumonia or acute respiratory distress syndrome and
    received supplemental oxygen. By now, doctors have identified many
    more symptoms and syndromes. In April,
    \href{https://www.nytimes3xbfgragh.onion/2020/04/27/health/coronavirus-symptoms-cdc.html?action=click\&pgtype=Article\&state=default\&region=MAIN_CONTENT_3\&context=storylines_faq}{the
    C.D.C. added to the list of early signs}~sore throat, fever, chills
    and muscle aches. Gastrointestinal upset, such as diarrhea and
    nausea, has also been observed. Another telltale sign of infection
    may be a sudden, profound diminution of one's
    \href{https://www.nytimes3xbfgragh.onion/2020/03/22/health/coronavirus-symptoms-smell-taste.html?action=click\&pgtype=Article\&state=default\&region=MAIN_CONTENT_3\&context=storylines_faq}{sense
    of smell and taste.}~Teenagers and young adults in some cases have
    developed painful red and purple lesions on their fingers and toes
    --- nicknamed ``Covid toe'' --- but few other serious symptoms.
  \end{itemize}
\item ~
  \hypertarget{why-is-it-safer-to-spend-time-together-outside}{%
  \paragraph{Why is it safer to spend time together
  outside?}\label{why-is-it-safer-to-spend-time-together-outside}}

  \begin{itemize}
  \tightlist
  \item
    \href{https://www.nytimes3xbfgragh.onion/2020/05/15/us/coronavirus-what-to-do-outside.html?action=click\&pgtype=Article\&state=default\&region=MAIN_CONTENT_3\&context=storylines_faq}{Outdoor
    gatherings}~lower risk because wind disperses viral droplets, and
    sunlight can kill some of the virus. Open spaces prevent the virus
    from building up in concentrated amounts and being inhaled, which
    can happen when infected people exhale in a confined space for long
    stretches of time, said Dr. Julian W. Tang, a virologist at the
    University of Leicester.
  \end{itemize}
\item ~
  \hypertarget{why-does-standing-six-feet-away-from-others-help}{%
  \paragraph{Why does standing six feet away from others
  help?}\label{why-does-standing-six-feet-away-from-others-help}}

  \begin{itemize}
  \tightlist
  \item
    The coronavirus spreads primarily through droplets from your mouth
    and nose, especially when you cough or sneeze. The C.D.C., one of
    the organizations using that measure,
    \href{https://www.nytimes3xbfgragh.onion/2020/04/14/health/coronavirus-six-feet.html?action=click\&pgtype=Article\&state=default\&region=MAIN_CONTENT_3\&context=storylines_faq}{bases
    its recommendation of six feet}~on the idea that most large droplets
    that people expel when they cough or sneeze will fall to the ground
    within six feet. But six feet has never been a magic number that
    guarantees complete protection. Sneezes, for instance, can launch
    droplets a lot farther than six feet,
    \href{https://jamanetwork.com/journals/jama/fullarticle/2763852}{according
    to a recent study}. It's a rule of thumb: You should be safest
    standing six feet apart outside, especially when it's windy. But
    keep a mask on at all times, even when you think you're far enough
    apart.
  \end{itemize}
\item ~
  \hypertarget{i-have-antibodies-am-i-now-immune}{%
  \paragraph{I have antibodies. Am I now
  immune?}\label{i-have-antibodies-am-i-now-immune}}

  \begin{itemize}
  \tightlist
  \item
    As of right
    now,\href{https://www.nytimes3xbfgragh.onion/2020/07/22/health/covid-antibodies-herd-immunity.html?action=click\&pgtype=Article\&state=default\&region=MAIN_CONTENT_3\&context=storylines_faq}{~that
    seems likely, for at least several months.}~There have been
    frightening accounts of people suffering what seems to be a second
    bout of Covid-19. But experts say these patients may have a
    drawn-out course of infection, with the virus taking a slow toll
    weeks to months after initial exposure.~People infected with the
    coronavirus typically
    \href{https://www.nature.com/articles/s41586-020-2456-9}{produce}~immune
    molecules called antibodies, which are
    \href{https://www.nytimes3xbfgragh.onion/2020/05/07/health/coronavirus-antibody-prevalence.html?action=click\&pgtype=Article\&state=default\&region=MAIN_CONTENT_3\&context=storylines_faq}{protective
    proteins made in response to an
    infection}\href{https://www.nytimes3xbfgragh.onion/2020/05/07/health/coronavirus-antibody-prevalence.html?action=click\&pgtype=Article\&state=default\&region=MAIN_CONTENT_3\&context=storylines_faq}{.
    These antibodies may}~last in the body
    \href{https://www.nature.com/articles/s41591-020-0965-6}{only two to
    three months}, which may seem worrisome, but that's~perfectly normal
    after an acute infection subsides, said Dr. Michael Mina, an
    immunologist at Harvard University. It may be possible to get the
    coronavirus again, but it's highly unlikely that it would be
    possible in a short window of time from initial infection or make
    people sicker the second time.
  \end{itemize}
\item ~
  \hypertarget{what-are-my-rights-if-i-am-worried-about-going-back-to-work}{%
  \paragraph{What are my rights if I am worried about going back to
  work?}\label{what-are-my-rights-if-i-am-worried-about-going-back-to-work}}

  \begin{itemize}
  \tightlist
  \item
    Employers have to provide
    \href{https://www.osha.gov/SLTC/covid-19/standards.html}{a safe
    workplace}~with policies that protect everyone equally.
    \href{https://www.nytimes3xbfgragh.onion/article/coronavirus-money-unemployment.html?action=click\&pgtype=Article\&state=default\&region=MAIN_CONTENT_3\&context=storylines_faq}{And
    if one of your co-workers tests positive for the coronavirus, the
    C.D.C.}~has said that
    \href{https://www.cdc.gov/coronavirus/2019-ncov/community/guidance-business-response.html}{employers
    should tell their employees}~-\/- without giving you the sick
    employee's name -\/- that they may have been exposed to the virus.
  \end{itemize}
\end{itemize}

Deceased hospital patients and nursing home residents, as well as bodies
in the care of funeral directors or medical examiners, will be among
those targeted for follow-up testing.

These tests can help health officials track the prevalence of both types
of infections, as well as indicate whether to warn close contacts of the
deceased that they may need to quarantine.

``People need to know who around them was sick,'' said Dr. Valerie
Fitzhugh, a pathologist at Rutgers New Jersey Medical School. ``If
someone can't be tested in life, why not test them soon after death?''

Putting regulations in place ahead of time will also encourage counties
to bolster their testing readiness ahead of autumn and winter, when
seasonal viruses like flu and respiratory syncytial virus, or R.S.V.,
tend to thrive, said Dr. Mary Fowkes, a pathologist at Mount Sinai
Hospital in New York.

In many parts of the United States, coronavirus cases are
\href{https://www.nytimes3xbfgragh.onion/interactive/2020/us/coronavirus-us-cases.html}{still
ratcheting up every day} --- and will become more difficult to track
when similar sicknesses muddle the picture.

\hypertarget{children-please-put-on-your-rain-suits}{%
\subsection{Children, please put on your rain
suits.}\label{children-please-put-on-your-rain-suits}}

\includegraphics{https://static01.graylady3jvrrxbe.onion/images/2020/09/06/business/06virus-schoolsupplies-1-copy/06virus-schoolsupplies-1-videoSixteenByNineJumbo1600.jpg}

With a number of schools in the United States opting for outdoor
education over the potentially germier confines of their traditional
indoor spaces,
\href{https://www.nytimes3xbfgragh.onion/2020/09/06/business/schools-outdoor-gear.html}{some
outdoor-oriented companies are starting new product lines or repurposing
existing ones} to capitalize on how the pandemic has changed the
education experience.

Demand for waterproof clothing and related gear ``has been
overwhelming,'' said Sam Taylor, the chief executive of Oaki, a maker of
a rain suit based in the Salt Lake City area. Mr. Taylor said demand for
Oaki products had increased 60 percent this year.

``There's been a ton of research that's shown how productive being
outside is,'' Mr. Taylor said. ``There's no reason a little moisture or
rain should stop that. If anything, that should be a positive if you've
got the right gear.''

Those searching for weatherproof supplies have also turned to
\href{https://www.riteintherain.com/}{Rite in the Rain,} a century-old
company based in Tacoma, Wash., that sells waterproof products including
notebooks and printer paper.

Fifty percent of Rite in the Rain's business comes from the government,
mostly the military. But aside from ``pretty decent business with
college bookstores,'' said Ryan McDonald, the company's director of
marketing, it hadn't focused much on students until recently, with an
increase in orders from elementary and high schools.

GLOBAL ROUNDUP

\hypertarget{mexico-citys-mayor-wants-people-to-take-the-virus-seriously-even-when-their-president-doesnt}{%
\subsection{Mexico City's mayor wants people to take the virus
seriously, even when their president
doesn't.}\label{mexico-citys-mayor-wants-people-to-take-the-virus-seriously-even-when-their-president-doesnt}}

Image

``I'm going to abide by what I believe in,'' said Mexico City's mayor,
Claudia Sheinbaum, who is a scientist. Credit...Meghan Dhaliwal for The
New York Times

The coronavirus has thrived in Mexico's dense capital, Mexico City,
which is home to nine million people, half of them poor. But while more
than 11,000 have died, analysts say it could have been worse
\href{https://www.nytimes3xbfgragh.onion/2020/09/05/world/americas/mexico-mayor-amlo-sheinbaum.html}{without
the interventions of Mayor Claudia Sheinbaum}.

Although she is one of President Andrés Manuel López Obrador's most
trusted confidants, she has been careful to distance herself from him
when possible when it comes to the virus. Mr. López Obrador minimized
the pandemic early on, questioning the science behind face masks and
doing little testing. Seeking to avert economic pain, he has barely
restricted travel.

Under his watch, Mexico has the fourth-highest coronavirus death toll
worldwide.

As of Saturday, Mexico had recorded 67,326 coronavirus deaths, according
to a
\href{https://www.nytimes3xbfgragh.onion/interactive/2020/world/americas/mexico-coronavirus-cases.html}{Times
database}. But the health ministry also said that the country had
recorded 122,765 more deaths than usual from the time the pandemic
started until August, suggesting that its true toll could be much higher
than reported.

When Mr. López Obrador was still kissing babies at rallies and comparing
the virus to the flu, Ms. Sheinbaum was planning for a long pandemic.
She pushed an aggressive testing and contact tracing campaign, and set
up testing kiosks where people get swabbed for free.

She also required that everyone in Mexico City use face coverings on
public transit, and wore a mask each time she addressed the news media.
And when doctors told her the N95 masks the federal government had
imported from China were too narrow to fit Mexican faces, she had a
local factory converted into a mask-making operation.

For Ms. Sheinbaum, a scientist with a Ph.D. in energy engineering,
aligning too closely with the president would mean ignoring the
practices she knows are in the best interest of public health. Stray too
far, and she risks losing the support of a political kingmaker who is
said to be considering her --- the first woman and first Jewish person
elected to lead the nation's capital --- as the party's next
presidential candidate.

So far, her strategy has been to follow the science while refusing to
criticize the president.

Other coronavirus news from around the world:

\begin{itemize}
\item
  \textbf{Israel} on Sunday announced overnight curfews on some 40
  cities and towns hit hard by the virus, but backed away from imposing
  full lockdowns after an uproar by politically powerful religious
  politicians, The Associated Press reported. In July, under heavy
  public pressure, Prime Minister Benjamin Netanyahu appointed a
  respected hospital director as the national ``coronavirus project
  manager,'' but when the new official began pushing for full lockdowns
  on areas with the worst outbreaks --- among them ultra-Orthodox Jewish
  communities --- leaders there strongly resisted. In an apparent
  compromise, Mr. Netanyahu opted for curfews. The measures will go into
  effect on Monday night.
\item
  Only a week after opening, most schools in a province in central
  \textbf{Cuba} are shutting their doors because of an outbreak.
  Seventy-five of Ciego de Ávila's 90 schools will now return to
  televised teaching, The A.P. reported. The province has reported 30
  new infections over the past 15 days. Students had returned to
  classrooms on Sept. 1 after a six-month break. Cuba as a whole has
  reported more than 4,300 infections and 100 deaths since March, with
  the biggest problem in Havana, which remains under a nighttime curfew.
\item
  \textbf{Melbourne}, Australia's second-largest city, on Sunday
  extended its lockdown by two weeks until at least Sept. 28. The state
  of Victoria, the center of Australia's worst outbreak, has been under
  lockdown since early August.
\end{itemize}

\hypertarget{the-virus-is-spiking-around-college-campuses-as-students-return}{%
\subsection{The virus is spiking around college campuses as students
return.}\label{the-virus-is-spiking-around-college-campuses-as-students-return}}

\href{https://www.nytimes3xbfgragh.onion/interactive/2020/us/covid-college-cases-tracker.html}{}

\includegraphics{https://static01.graylady3jvrrxbe.onion/images/2020/09/04/us/covid-college-cases-tracker-promo-1599194588528/covid-college-cases-tracker-promo-1599194588528-articleLarge.png}

\hypertarget{tracking-coronavirus-cases-at-us-colleges-and-universities}{%
\subsection{Tracking Coronavirus Cases at U.S. Colleges and
Universities}\label{tracking-coronavirus-cases-at-us-colleges-and-universities}}

Large new outbreaks emerged on campuses as students returned for the
fall semester.

Within days of the University of Iowa's reopening, students were
complaining that they couldn't get coronavirus tests or were bumping
into people who were supposed to be in isolation. Undergraduates
\href{https://twitter.com/VanessaMiller12/status/1297386480164364290/photo/1}{were
jamming sidewalks} and
\href{https://twitter.com/VanessaMiller12/status/1297392675700776960}{downtown
bars}, masks hanging below their chins, never mind the city's mask
mandate.

Now, Iowa City is a full-blown pandemic hot spot --- one of about 100
college communities around the United States where infections have
spiked in recent weeks as students have returned for the fall semester.
Although the rate of infection has bent downward in the Northeast, where
the virus first peaked in the United States, it remains high across many
states in the Midwest and the South, and evidence suggests that students
returning to big campuses are a major factor.

In a New York Times review of 203 U.S. counties where students make up
at least 10 percent of the population, about half have experienced their
worst weeks of the pandemic since Aug. 1. In about half of those,
figures showed that the number of new infections is currently peaking.

Despite the surge in cases, there has been no uptick in deaths in
college communities, data shows. This suggests that most of the
infections are stemming from campuses, since young people who contract
the virus are far less likely to die than older people.

However, leaders fear that young people who are infected will contribute
to the spread of the virus throughout the community.

The surge in infections reported by county health departments comes as
many college administrations are also disclosing clusters on their
campuses, and taking disciplinary actions against students who flout
rules. Northeastern University
\href{https://www.nytimes3xbfgragh.onion/2020/09/05/world/coronavirus-covid.html}{dismissed
11 students} for violations last week, keeping their tuition.

And on Saturday, New York University said it had
\href{https://twitter.com/nyuniversity/status/1302380699010433026}{suspended
20 students} since classes resumed. The virus's potential spread beyond
campus greens has deeply affected the workplaces, schools, governments
and other institutions of local communities.

The result is often an exacerbation of traditional town-and-gown
tensions as college towns have tried to balance
\href{https://www.nytimes3xbfgragh.onion/2020/06/28/us/coronavirus-college-towns.html}{economic
dependence} on universities with visceral public health fears.

\hypertarget{for-people-of-color-the-pandemic-era-virtual-office-may-create-additional-hurdles-to-moving-ahead}{%
\subsection{For people of color, the pandemic-era virtual office may
create additional hurdles to moving
ahead.}\label{for-people-of-color-the-pandemic-era-virtual-office-may-create-additional-hurdles-to-moving-ahead}}

Image

Kimberly Bryant, an engineer who founded a nonprofit group, says office
encounters with colleagues of color ``would lead to connections that
were instrumental in terms of my success.''Credit...Marissa Leshnov for
The New York Times

Chance office encounters that used to allow for networking have been
replaced by the formal geometry of the Zoom screen. And with fewer and
less extensive connections than white colleagues to begin with, Black
and Hispanic workers can find themselves more isolated than ever.

Assignments end up flowing to people who look more like top managers ---
a longstanding issue --- while workers of color hesitate to raise their
voices during online meetings, said Sara Prince, a partner at the
consulting firm McKinsey.

``It's a critical issue, and there is a real risk facing diversity and
inclusion in the current environment,'' Ms. Prince said. ``When the
leader is looking for someone to take up the mantle, most of them go to
the comfort zone of people who remind them of themselves. This is
exacerbated by the virtual office.''

It's harder to tell which employees have shrunk back in their chairs or
otherwise withdrawn in virtual meetings, said Evelyn Carter, managing
director at Paradigm, a consulting firm, but moderators should pay
attention to clues, like people with their cameras off, and try to draw
those participants back into the discussion.

Some experts do see upsides for office workers who might have been
marginalized.

``Most minorities are left out of informal networks and might not have
been invited out for drinks or lunch,'' said Tina Shah Paikeday, who
oversees global diversity and inclusion advisory services at Russell
Reynolds, a recruiting firm.

``The Zoom meeting is intentionally planned, and managers feel very
intentional about inviting everyone.''

``It's a great equalizer, and it creates opportunities for affinity
group within large organizations,'' she said. ``It could end up being a
good thing for minorities.''

\hypertarget{in-indonesia-students-climb-trees-and-travel-miles-in-search-of-a-signal-for-their-remote-classes}{%
\subsection{In Indonesia, students climb trees and travel miles in
search of a signal for their remote
classes.}\label{in-indonesia-students-climb-trees-and-travel-miles-in-search-of-a-signal-for-their-remote-classes}}

Image

Students needed to go to a roadside area to get internet access at a
village in Central Java, Indonesia.Credit...Ulet Ifansasti

Around the globe, including in some of the world's wealthiest countries,
educators are struggling with how to facilitate distance learning during
the pandemic. But in poorer countries like Indonesia, the challenge is
particularly difficult.

In North Sumatra, students climb to the tops of tall trees a mile from
their mountain village. Perched on branches high above the ground, they
hope for a cellphone signal strong enough to complete their assignments.

The travails of these students and others like them have come to
symbolize the hardships faced by millions of schoolchildren across the
Indonesian archipelago. Officials have closed schools and brought in
remote learning, but internet and cellphone service is limited and many
students do not have smartphones and computers.

More than a third of Indonesian students have limited or no internet
access, according to the Education Ministry, and experts fear that many
students will fall far behind, especially in remote areas where online
study remains a novelty.

Indonesia's efforts to slow the spread of the coronavirus have met with
mixed results. As of Saturday, the country had 190,665 cases and 7,940
deaths. But testing has been limited and independent health experts say
the actual number of cases is many times higher.

With the start of a new academic year in July, schools in virus-free
zones were allowed to reopen, but these schools serve only a fraction of
the nation's students. As of August, communities in low-risk areas could
decide whether to reopen schools, but few have done so.

``Students have no idea what to do, and parents think it is just a
holiday,'' said Itje Chodidjah, an educator and teacher trainer in
Jakarta, the capital. ``We still have lots of areas where there is no
internet access. In some areas, there is even difficulty getting
electricity.''

\hypertarget{hong-kongs-police-thwart-a-protest-over-elections-postponed-on-pandemic-grounds}{%
\subsection{Hong Kong's police thwart a protest over elections postponed
on pandemic
grounds.}\label{hong-kongs-police-thwart-a-protest-over-elections-postponed-on-pandemic-grounds}}

\includegraphics{https://static01.graylady3jvrrxbe.onion/images/2020/09/06/world/06hongkong1-sub/06hongkong1-sub-videoSixteenByNine3000-v2.jpg}

Thousands of police officers in riot gear filled the streets of Hong
Kong on Sunday, stifling a protest over the postponement of legislative
elections because of the pandemic and over China's imposition of a
national security law that gives the authorities
\href{https://www.nytimes3xbfgragh.onion/interactive/2020/09/04/world/asia/hong-kong-speech.html}{sweeping
new powers to pursue critics}.

A large police presence was seen across the Kowloon Peninsula, where
some activists had called for a march on the day the elections were
initially scheduled to take place, despite social distancing rules that
prohibit mass gatherings. Although occasional pro-democracy chants broke
out as small groups wound through side streets, the number of
demonstrators remained small compared with the huge crowds that gathered
last year.

While Hong Kong has seen an increase in coronavirus cases over the past
month, a recent wave has largely been brought under control. The city
announced 21 new cases on Sunday, after more than a week of daily totals
in the single or low double digits.

Hong Kong's government, with the aid of a team from mainland China,
began a universal testing program last week that it said was necessary
to break hidden chains of virus transmission. Some activists and health
care workers urged residents to boycott the plan, calling it a waste of
resources motivated by a political desire to burnish the image of
China's central government.

Health officials said on Thursday that six positive cases had been found
in the first batch of 128,000 tested in the program, including four
people with previously confirmed cases who were treated in hospitals.
Five more cases detected through the program were announced on Sunday.
About one million people in the city of 7.5 million have registered for
tests.

Reporting was contributed by Kenneth Chang, Catie Edmondson, Natasha
Frost, Robert Gebeloff, Shawn Hubler, Danielle Ivory, Jennifer Jett,
Natalie Kitroeff, Sarah Kliff, Patrick J. Lyons, Tiffany May, Dera Menra
Sijabat, Eric Nagourney, Richard C. Paddock, Tara Parker-Pope, Austin
Ramzy, Nelson D. Schwartz, Mike Seely, Sarah Watson, Sui-Lee Wee,
Katherine J. Wu and Mihir Zaveri.

Advertisement

\protect\hyperlink{after-bottom}{Continue reading the main story}

\hypertarget{site-index}{%
\subsection{Site Index}\label{site-index}}

\hypertarget{site-information-navigation}{%
\subsection{Site Information
Navigation}\label{site-information-navigation}}

\begin{itemize}
\tightlist
\item
  \href{https://help.nytimes3xbfgragh.onion/hc/en-us/articles/115014792127-Copyright-notice}{©~2020~The
  New York Times Company}
\end{itemize}

\begin{itemize}
\tightlist
\item
  \href{https://www.nytco.com/}{NYTCo}
\item
  \href{https://help.nytimes3xbfgragh.onion/hc/en-us/articles/115015385887-Contact-Us}{Contact
  Us}
\item
  \href{https://www.nytco.com/careers/}{Work with us}
\item
  \href{https://nytmediakit.com/}{Advertise}
\item
  \href{http://www.tbrandstudio.com/}{T Brand Studio}
\item
  \href{https://www.nytimes3xbfgragh.onion/privacy/cookie-policy\#how-do-i-manage-trackers}{Your
  Ad Choices}
\item
  \href{https://www.nytimes3xbfgragh.onion/privacy}{Privacy}
\item
  \href{https://help.nytimes3xbfgragh.onion/hc/en-us/articles/115014893428-Terms-of-service}{Terms
  of Service}
\item
  \href{https://help.nytimes3xbfgragh.onion/hc/en-us/articles/115014893968-Terms-of-sale}{Terms
  of Sale}
\item
  \href{https://spiderbites.nytimes3xbfgragh.onion}{Site Map}
\item
  \href{https://help.nytimes3xbfgragh.onion/hc/en-us}{Help}
\item
  \href{https://www.nytimes3xbfgragh.onion/subscription?campaignId=37WXW}{Subscriptions}
\end{itemize}
