Sections

SEARCH

\protect\hyperlink{site-content}{Skip to
content}\protect\hyperlink{site-index}{Skip to site index}

\href{https://www.nytimes3xbfgragh.onion/section/business}{Business}

\href{https://myaccount.nytimes3xbfgragh.onion/auth/login?response_type=cookie\&client_id=vi}{}

\href{https://www.nytimes3xbfgragh.onion/section/todayspaper}{Today's
Paper}

\href{/section/business}{Business}\textbar{}With Washington Deadlocked
on Aid, States Face Dire Fiscal Crises

\url{https://nyti.ms/3jTT8Bx}

\begin{itemize}
\item
\item
\item
\item
\item
\end{itemize}

\hypertarget{the-coronavirus-outbreak}{%
\subsubsection{\texorpdfstring{\href{https://www.nytimes3xbfgragh.onion/news-event/coronavirus?name=styln-coronavirus-markets\&region=TOP_BANNER\&block=storyline_menu_recirc\&action=click\&pgtype=Article\&impression_id=398e6170-f296-11ea-b172-61b0efec9af7\&variant=undefined}{The
Coronavirus
Outbreak}}{The Coronavirus Outbreak}}\label{the-coronavirus-outbreak}}

\begin{itemize}
\tightlist
\item
  live\href{https://www.nytimes3xbfgragh.onion/2020/09/09/world/covid-19-coronavirus.html?name=styln-coronavirus-markets\&region=TOP_BANNER\&block=storyline_menu_recirc\&action=click\&pgtype=Article\&impression_id=398e6171-f296-11ea-b172-61b0efec9af7\&variant=undefined}{Latest
  Updates}
\item
  \href{https://www.nytimes3xbfgragh.onion/interactive/2020/us/coronavirus-us-cases.html?name=styln-coronavirus-markets\&region=TOP_BANNER\&block=storyline_menu_recirc\&action=click\&pgtype=Article\&impression_id=398e6172-f296-11ea-b172-61b0efec9af7\&variant=undefined}{Maps
  and Cases}
\item
  \href{https://www.nytimes3xbfgragh.onion/interactive/2020/science/coronavirus-vaccine-tracker.html?name=styln-coronavirus-markets\&region=TOP_BANNER\&block=storyline_menu_recirc\&action=click\&pgtype=Article\&impression_id=398e8880-f296-11ea-b172-61b0efec9af7\&variant=undefined}{Vaccine
  Tracker}
\item
  \href{https://www.nytimes3xbfgragh.onion/2020/09/02/your-money/eviction-moratorium-covid.html?name=styln-coronavirus-markets\&region=TOP_BANNER\&block=storyline_menu_recirc\&action=click\&pgtype=Article\&impression_id=398e8881-f296-11ea-b172-61b0efec9af7\&variant=undefined}{Eviction
  Moratorium}
\item
  \href{https://www.nytimes3xbfgragh.onion/interactive/2020/09/02/magazine/food-insecurity-hunger-us.html?name=styln-coronavirus-markets\&region=TOP_BANNER\&block=storyline_menu_recirc\&action=click\&pgtype=Article\&impression_id=398e8882-f296-11ea-b172-61b0efec9af7\&variant=undefined}{American
  Hunger}
\end{itemize}

Advertisement

\protect\hyperlink{after-top}{Continue reading the main story}

Supported by

\protect\hyperlink{after-sponsor}{Continue reading the main story}

\hypertarget{with-washington-deadlocked-on-aid-states-face-dire-fiscal-crises}{%
\section{With Washington Deadlocked on Aid, States Face Dire Fiscal
Crises}\label{with-washington-deadlocked-on-aid-states-face-dire-fiscal-crises}}

Local officials are slashing funding for everything from education and
health care to orchestra subsidies.

\includegraphics{https://static01.graylady3jvrrxbe.onion/images/2020/09/02/business/00virus-statecuts-01/00virus-statecuts-01-articleLarge.jpg?quality=75\&auto=webp\&disable=upscale}

\href{https://www.nytimes3xbfgragh.onion/by/mary-williams-walsh}{\includegraphics{https://static01.graylady3jvrrxbe.onion/images/2017/01/31/multimedia/mary-williams-walsh/mary-williams-walsh-thumbLarge-v2.png}}

By \href{https://www.nytimes3xbfgragh.onion/by/mary-williams-walsh}{Mary
Williams Walsh}

\begin{itemize}
\item
  Sept. 7, 2020
\item
  \begin{itemize}
  \item
  \item
  \item
  \item
  \item
  \end{itemize}
\end{itemize}

Alaska chopped resources for public broadcasting.
\href{https://www1.nyc.gov/assets/dsny/site/services/food-scraps-and-yard-waste-page/overview-residents-organics}{New
York City} gutted a nascent
\href{https://www1.nyc.gov/assets/dsny/site/services/food-scraps-and-yard-waste-page/overview-residents-organics}{composting
program} that could have kept tons of food waste out of landfills. New
Jersey postponed property-tax relief payments.

Prisoners in
\href{https://www.tampabay.com/news/health/2020/07/02/desantis-cuts-money-for-prison-updates-hepatitis-c-treatment/}{Florida}
will continue to swelter in their cells, because plans to air-condition
its prisons are on hold. Many states have already cut planned raises for
teachers.

And that's just the start.

Across the nation, states and cities have made an
\href{https://www.nytimes3xbfgragh.onion/2020/05/14/business/virus-state-budgets.html}{array
of fiscal maneuvers} to stay solvent and are planning more in case
Congress can't agree on a fiscal relief package after the August recess.

House Democrats included nearly \$1 trillion in state and local aid in
the
\href{https://www.congress.gov/116/bills/hr6800/BILLS-116hr6800ih.pdf}{relief
bill} they passed in May, but the Senate majority leader, Mitch
McConnell of Kentucky, has said he doesn't want to hand out a ``blank
check'' to pay for what he considers fiscal mismanagement, including the
enormous public-pension obligations some states have accrued. There has
been little movement in that stalemate lately.

Economists warn that further state spending reductions could prolong the
downturn by shaking the confidence of residents, whose day-to-day lives
depend heavily on state and local services.

``People look to government as their backstop when things are completely
falling apart,'' said Mark Zandi, chief economist at Moody's Analytics.
``If they feel like there's no support there, they lose faith and they
run for the bunker and pull back on everything.''

States and municipalities are also crucial employers and spenders that
keep the economy moving. ``We run the risk of descending into a dark
vicious cycle,'' Mr. Zandi said.

\hypertarget{latest-updates-the-coronavirus-outbreak-and-the-economy}{%
\section{\texorpdfstring{\href{https://www.nytimes3xbfgragh.onion/live/2020/09/09/business/stock-market-today-coronavirus?action=click\&pgtype=Article\&state=default\&region=MAIN_CONTENT_1\&context=storylines_live_updates}{Latest
Updates: The Coronavirus Outbreak and the
Economy}}{Latest Updates: The Coronavirus Outbreak and the Economy}}\label{latest-updates-the-coronavirus-outbreak-and-the-economy}}

\href{https://www.nytimes3xbfgragh.onion/live/2020/09/09/business/stock-market-today-coronavirus?action=click\&pgtype=Article\&state=default\&region=MAIN_CONTENT_1\&context=storylines_live_updates\#why-a-licensing-expert-and-a-mall-operator-bought-brooks-brothers-forever-21-and-others}{1h
ago}

\href{https://www.nytimes3xbfgragh.onion/live/2020/09/09/business/stock-market-today-coronavirus?action=click\&pgtype=Article\&state=default\&region=MAIN_CONTENT_1\&context=storylines_live_updates\#why-a-licensing-expert-and-a-mall-operator-bought-brooks-brothers-forever-21-and-others}{Why
a licensing expert and a mall operator bought Brooks Brothers, Forever
21 and others.}

\href{https://www.nytimes3xbfgragh.onion/live/2020/09/09/business/stock-market-today-coronavirus?action=click\&pgtype=Article\&state=default\&region=MAIN_CONTENT_1\&context=storylines_live_updates\#lvmh-says-it-is-pulling-out-of-its-16-billion-takeover-of-tiffany}{1h
ago}

\href{https://www.nytimes3xbfgragh.onion/live/2020/09/09/business/stock-market-today-coronavirus?action=click\&pgtype=Article\&state=default\&region=MAIN_CONTENT_1\&context=storylines_live_updates\#lvmh-says-it-is-pulling-out-of-its-16-billion-takeover-of-tiffany}{LVMH
says it is pulling out of its \$16 billion takeover of Tiffany.}

\href{https://www.nytimes3xbfgragh.onion/live/2020/09/09/business/stock-market-today-coronavirus?action=click\&pgtype=Article\&state=default\&region=MAIN_CONTENT_1\&context=storylines_live_updates\#six-months-in-seafarers-on-ships-around-the-world-still-have-no-way-home}{1h
ago}

\href{https://www.nytimes3xbfgragh.onion/live/2020/09/09/business/stock-market-today-coronavirus?action=click\&pgtype=Article\&state=default\&region=MAIN_CONTENT_1\&context=storylines_live_updates\#six-months-in-seafarers-on-ships-around-the-world-still-have-no-way-home}{Six
months in, seafarers on ships around the world still have no way home.}

\href{https://www.nytimes3xbfgragh.onion/live/2020/09/09/business/stock-market-today-coronavirus?action=click\&pgtype=Article\&state=default\&region=MAIN_CONTENT_1\&context=storylines_live_updates}{See
more updates}

More live coverage:
\href{https://www.nytimes3xbfgragh.onion/2020/09/09/world/covid-19-coronavirus.html?action=click\&pgtype=Article\&state=default\&region=MAIN_CONTENT_1\&context=storylines_live_updates}{Global}

State and local governments administer most of America's programs for
education, public safety, health care and unemployment insurance. They
also provide a wide variety of smaller services, such as outdoor
recreational facilities or highway rest stops, that improve the quality
of life. The costs of many of these programs have spiraled because of
the pandemic, which has at the same time caused an economic slump that
has driven down tax revenues.

Collectively, state governments will have budget shortfalls of \$312
billion through the summer of 2022, according to a review by Moody's
Analytics. When local governments are factored in, the shortfall rises
to \$500 billion. That estimate assumes the pandemic doesn't get worse.

When the lockdowns started in March, state and local governments quickly
cut 1.3 million jobs. But then they paused, waiting to see if revenue
would continue to fall --- and what Washington might do to replace it.

\includegraphics{https://static01.graylady3jvrrxbe.onion/images/2020/09/02/business/00virus-statecuts-02/merlin_141657045_a36d93c6-7ec2-40ba-9b23-1934a5656b51-articleLarge.jpg?quality=75\&auto=webp\&disable=upscale}

Lawmakers soon passed the \$2 trillion
\href{https://www.nytimes3xbfgragh.onion/article/coronavirus-stimulus-package-questions-answers.html}{CARES
Act}, which authorized one-time stimulus payments and temporary
supplemental unemployment payments, which buoyed consumer spending and
helped states' sales-tax revenues. The law also allocated about \$150
billion to states for expenses directly attributable to the pandemic, in
areas ranging from education and health care to the operation of nearly
empty airports. But the rules for what expenses that money can cover
have kept much of it from being spent, according to the
\href{https://home.treasury.gov/system/files/136/Interim-Report-of-Costs-by-Category-Incurred-by-State-and-Local-Recipients-through-June-30.pdf}{Treasury
Department}. New York State, for example, has been sent about \$2.9
billion that it can't put toward other uses.

Although states' budget challenges would be eased if Congress relaxed
those rules, that still wouldn't be enough to fill the gap.

Gov. Andrew M. Cuomo has warned that without further relief New York
will cut \$8.2 billion in grants to local governments, a blow he said
had ``no precedent in modern times.'' The cuts would hit ``nearly every
activity funded by state government,'' including special education,
pediatric health care, substance abuse programs, property-tax relief and
mass transit, he said.

No two states have tackled the budget crunch the same way. Several have
torn up their annual budgets and are doling out money to programs one or
two months at a time. Some have earmarked cuts but not yet carried them
out.

Delaware has decided to issue less debt, and a bond issue that was
supposed to fund clean-water projects has been shelved. In California,
people who go to court without lawyers --- an estimated 4.3 million a
year --- will continue to deal with confusion because the state has
scrapped plans for ``court navigators'' to shepherd them through.
\href{https://www.reviewjournal.com/news/politics-and-government/nevada/nevada-legislature-makes-small-progress-on-budget-gap-2074754/}{Nevada}
said it would forgo the penalties and interest it normally charged tax
cheats, hoping to coax them and their unpaid millions up from
underground. In Maryland, the
\href{https://www.nytimes3xbfgragh.onion/2019/06/17/arts/music/baltimore-symphony-musicians-lock-out.html}{Baltimore
Symphony Orchestra} will lose a \$1.6 million state subsidy.

Some states are trying to save cash on their pension contributions.
Kentucky has delayed its payments to the state workers' pension fund,
already one of the most poorly funded in the country. Colorado and
Maryland are among the states planning to reduce their contributions.
Some, like California and New Jersey, had recently committed to raising
their contributions to cover past underpayments --- but now can't afford
to do so.

Without further federal aid, some of the biggest cuts will be to
education and health care. California says it will send its school
districts \$12.5 billion in I.O.U.s if Washington doesn't step in, and
it will be on the schools to figure out how to fund themselves in the
meantime. Preschool programs are being cut in many states; so are
free-tuition college programs. State university systems are slated to
lose billions of dollars in state funding, although some states say the
cuts will be quickly reversed if enough federal money arrives.

And many states say they will reduce their outlays for Medicaid. The
health care program for low-income people has been growing rapidly in
the pandemic as millions have lost their jobs along with their employee
health benefits. States are struggling to find a way to pay for all
these additional people. Some, like Colorado, are increasing the
co-payments that their Medicaid patients must pay for doctor visits,
pharmaceuticals and medical transport.

Image

New York City gutted a nascent composting program that could have kept
tons of food waste out of landfills.Credit...Christopher Lee for The New
York Times

State officials say they have little choice but to keep cutting if more
aid doesn't arrive. All but one state, Vermont, are legally bound to
balance their budgets every year, and Vermont does so voluntarily. They
can't borrow their way out of a cash crunch, the way Washington can,
because they have laws limiting how much bond debt they can carry. If
they veer too close to the limit, lenders will start demanding higher
interest rates and the rating agencies will downgrade them.

In May, the
\href{https://www.nytimes3xbfgragh.onion/2020/06/03/business/economy/fed-expands-municipal-bond-program.html}{Federal
Reserve} offered to buy states' bonds if terms in the municipal bond
market become onerous. But most states think the Fed loans cost too much
and have to be paid back too quickly to be of much help. So far only one
state, Illinois, and one state authority, New York's Metropolitan
Transportation Authority, have taken the Fed up on its offer. New Jersey
and Hawaii are exploring deals, according to the National Conference of
State Legislatures, which tracks the states' fiscal plans as they
develop.

Public pensions have been a central point of contention in discussions
over additional federal aid.

In April, with economic activity at low ebb, Illinois lawmakers sent
\href{https://www.nytimes3xbfgragh.onion/2020/04/17/business/dealbook/illinois-pension-coronavirus.html}{a
detailed wish list}to their state's congressional delegation that
included \$10 billion for the coming year's pension contribution. They
also asked for \$9.6 billion for Illinois's cities, which needed the
money to ``fund retirement systems for the police, firefighters and
other first responders providing emergency services during this Covid-19
outbreak.''

The request drew scorn in Washington.

On a syndicated radio show, Mr. McConnell said Senate Republicans would
``certainly insist that anything we'd borrow to send down to the states
is not spent on solving problems that they created for themselves over
the years with their pension programs.''

Glenn Hubbard, an economic conservative who was chairman of the Council
of Economic Advisers under President George W. Bush, said he agreed that
federal money should not be used to prop up failing state pension funds.
But he acknowledged that the states' cash needs were becoming urgent and
said there wasn't time for a complete overhaul of troubled state pension
systems.

For the sake of speed, Mr. Hubbard said in an interview, Congress could
send the states money with a simple, and probably breakable, rule that
it not be used to reduce taxes or bail out pensions. Public pension
reform, which would be grueling, could come later.

Or, as Mr. Hubbard said in an online seminar hosted by the Economic
Policy Institute last month, ``if an overweight person comes to the E.R.
with a heart attack, you treat the heart attack before you lecture him
or her about weight.''

Advertisement

\protect\hyperlink{after-bottom}{Continue reading the main story}

\hypertarget{site-index}{%
\subsection{Site Index}\label{site-index}}

\hypertarget{site-information-navigation}{%
\subsection{Site Information
Navigation}\label{site-information-navigation}}

\begin{itemize}
\tightlist
\item
  \href{https://help.nytimes3xbfgragh.onion/hc/en-us/articles/115014792127-Copyright-notice}{©~2020~The
  New York Times Company}
\end{itemize}

\begin{itemize}
\tightlist
\item
  \href{https://www.nytco.com/}{NYTCo}
\item
  \href{https://help.nytimes3xbfgragh.onion/hc/en-us/articles/115015385887-Contact-Us}{Contact
  Us}
\item
  \href{https://www.nytco.com/careers/}{Work with us}
\item
  \href{https://nytmediakit.com/}{Advertise}
\item
  \href{http://www.tbrandstudio.com/}{T Brand Studio}
\item
  \href{https://www.nytimes3xbfgragh.onion/privacy/cookie-policy\#how-do-i-manage-trackers}{Your
  Ad Choices}
\item
  \href{https://www.nytimes3xbfgragh.onion/privacy}{Privacy}
\item
  \href{https://help.nytimes3xbfgragh.onion/hc/en-us/articles/115014893428-Terms-of-service}{Terms
  of Service}
\item
  \href{https://help.nytimes3xbfgragh.onion/hc/en-us/articles/115014893968-Terms-of-sale}{Terms
  of Sale}
\item
  \href{https://spiderbites.nytimes3xbfgragh.onion}{Site Map}
\item
  \href{https://help.nytimes3xbfgragh.onion/hc/en-us}{Help}
\item
  \href{https://www.nytimes3xbfgragh.onion/subscription?campaignId=37WXW}{Subscriptions}
\end{itemize}
