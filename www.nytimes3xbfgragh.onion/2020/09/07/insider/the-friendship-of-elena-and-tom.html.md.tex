\href{/section/reader-center}{Times Insider}\textbar{}The Friendship of
Elena and Tom

\url{https://nyti.ms/2Fg9zJj}

\begin{itemize}
\item
\item
\item
\item
\item
\item
\end{itemize}

\includegraphics{https://static01.graylady3jvrrxbe.onion/images/2020/09/07/pageoneplus/07insider-elena-and-seaver/07insider-elena-and-seaver-articleLarge.jpg?quality=75\&auto=webp\&disable=upscale}

Sections

\protect\hyperlink{site-content}{Skip to
content}\protect\hyperlink{site-index}{Skip to site index}

Times Insider

\hypertarget{the-friendship}{%
\section{The Friendship}\label{the-friendship}}

of Elena and Tom

It's rare that someone meets her favorite ballplayer. It's even rarer
that he becomes a fan of hers.

The baseball great Tom Seaver and Elena Gustines in 2009.Credit...Joe
Ward

Supported by

\protect\hyperlink{after-sponsor}{Continue reading the main story}

\href{https://www.nytimes3xbfgragh.onion/by/vincent-m-mallozzi}{\includegraphics{https://static01.graylady3jvrrxbe.onion/images/2019/02/14/multimedia/author-vincent-m-mallozzi/author-vincent-m-mallozzi-thumbLarge.png}}

By
\href{https://www.nytimes3xbfgragh.onion/by/vincent-m-mallozzi}{Vincent
M. Mallozzi}

\begin{itemize}
\item
  Sept. 7, 2020
\item
  \begin{itemize}
  \item
  \item
  \item
  \item
  \item
  \item
  \end{itemize}
\end{itemize}

\href{https://www.nytimes3xbfgragh.onion/series/times-insider}{\emph{Times
Insider}} \emph{explains who we are and what we do, and delivers
behind-the-scenes insights into how our journalism comes together.}

Elena Gustines grew up to be the kind of baseball fan who thought the
moon and the sun had seams on them. But in 1994, she walked away from
the sport.

A player strike in the middle of the season wiped out the playoffs,
leaving many people heartbroken and angry. The next spring, when most
fans returned to the national pastime, Ms. Gustines said she would not.
She vowed never to attend another game. Unless.

``The only way I'll go back to baseball is if Tom Seaver himself invites
me back,'' said Ms. Gustines, a devoted Mets fan.

Ten years later she hadn't budged. Even when presented with a
handwritten letter from the Mets center fielder Mike Cameron, who
pleaded with her to come back.

``Nope,'' Ms. Gustines said. ``I told you, it has to be Tom Seaver.''

Ms. Gustines had rooted for and cherished Mr. Seaver, a Hall of Famer
who retired in 1987, since watching him pitch for the Mets against
Oakland in the 1973 World Series.

``Even though the Mets lost, those games sparked my interest in
baseball,'' she said.

She also remembered listening to the radio in amazement in 1974 as Mr.
Seaver, mired in a subpar season with health issues, struck out 14
Philadelphia Phillies in his last appearance to finish the year with 201
strikeouts. He became the first National League pitcher to strike out
200 batters in seven consecutive seasons.

``It seemed that 200 strikeouts would be unreachable,'' Ms. Gustines
said. ``I watched all of his starts that season, and though he finished
11-11, I came to appreciate the kind of intelligent pitcher he was. On
that night in particular, he became my Superman.''

So in 2004, determined to get her back to the game, a few of Ms.
Gustines's friends and colleagues at The New York Times, where she still
works, tried the only thing left to try. They crafted a letter to Jay
Horwitz, the longtime public relations officer for the Mets. They
described Ms. Gustines's situation, and practically begged him to reach
out to Mr. Seaver.

In May, Ms. Gustines received a birthday card.

``Dear Elena, we miss you! Please come back to the baseball family. The
game really needs you and so do the Mets! Your friend, Tom Seaver.''

Image

The card that brought Elena Gustines back to baseball.Credit...via Elena
Gustines

Ms. Gustines recalled opening the card and seeing Mr. Seaver's note.

``My hands started trembling,'' she said. ``I was in complete shock.''

She returned to Shea Stadium that September, and so started an
improbable friendship in which Ms. Gustines, 58, somehow managed to flip
scripts with the dominant right-hander. It was Mr. Seaver who would one
day end a note to Ms. Gustines signed ``Your No. 1 Fan.''

When it was announced last week that Mr. Seaver had died at age 75, Ms.
Gustines had already spilled enough tears to postpone a baseball game
when she began accepting heartfelt condolences from family and friends.

The driving force behind the 1969 Miracle Mets, Mr. Seaver finished his
career with 311 victories, three Cy Young Awards and 3,640 strikeouts.
Ms. Gustines began working for The Times in 1989, where she builds print
pages and researches statistics.

They met in person on July 31, 2009. The Times reporter Richard Sandomir
was hosting Mr. Seaver for an interview and invited colleagues to sit in
on the discussion.

``I finally met my favorite player,'' Ms. Gustines said. ``I was in
awe.''

Ms. Gustines also met Mollie Ann Bracigliano, a marketing agent who
represented Mr. Seaver at autograph signings. They hit it off, and Ms.
Bracigliano would invite Ms. Gustines to events that she knew Mr. Seaver
would attend.

In the ensuing years, Mr. Seaver and Ms. Gustines kept in touch. They
met at autograph shows and a Mets celebration. Mr. Seaver met Ms.
Gustines's father, Jorge (a Mets fan), and Ms. Gustines corresponded
with Nancy Seaver, Mr. Seaver's wife. Ms. Gustines sent the Seavers
customized M\&Ms with his stats on them.

\includegraphics{https://static01.graylady3jvrrxbe.onion/images/2020/09/07/pageoneplus/07insider-elenaseaver-diptych1/07insider-elenaseaver-diptych1-articleLarge.jpg?quality=75\&auto=webp\&disable=upscale}

``Tom loved Elena,'' Ms. Bracigliano said. ``He told me, `You don't know
the joy I get out of somebody like her, she's so real and so genuine.'''

Soon Mr. Seaver was in awe of Ms. Gustines, who made two books for him,
one filled with every box score from Mr. Seaver's 61 career shutouts,
and another filled with his 311 victories.

After receiving his career wins book, Ms. Gustines's No. 1 fan turned
the pages with an incredulous look on his face.

``This was a labor of love for you, wasn't it?'' he asked.

Ms. Gustines nodded.

``As pitching was for me,'' he said.

Image

Tom Seaver died on Aug. 31. He was 75.Credit...Earl Wilson/The New York
Times

Advertisement

\protect\hyperlink{after-bottom}{Continue reading the main story}

\hypertarget{site-index}{%
\subsection{Site Index}\label{site-index}}

\hypertarget{site-information-navigation}{%
\subsection{Site Information
Navigation}\label{site-information-navigation}}

\begin{itemize}
\tightlist
\item
  \href{https://help.nytimes3xbfgragh.onion/hc/en-us/articles/115014792127-Copyright-notice}{©~2020~The
  New York Times Company}
\end{itemize}

\begin{itemize}
\tightlist
\item
  \href{https://www.nytco.com/}{NYTCo}
\item
  \href{https://help.nytimes3xbfgragh.onion/hc/en-us/articles/115015385887-Contact-Us}{Contact
  Us}
\item
  \href{https://www.nytco.com/careers/}{Work with us}
\item
  \href{https://nytmediakit.com/}{Advertise}
\item
  \href{http://www.tbrandstudio.com/}{T Brand Studio}
\item
  \href{https://www.nytimes3xbfgragh.onion/privacy/cookie-policy\#how-do-i-manage-trackers}{Your
  Ad Choices}
\item
  \href{https://www.nytimes3xbfgragh.onion/privacy}{Privacy}
\item
  \href{https://help.nytimes3xbfgragh.onion/hc/en-us/articles/115014893428-Terms-of-service}{Terms
  of Service}
\item
  \href{https://help.nytimes3xbfgragh.onion/hc/en-us/articles/115014893968-Terms-of-sale}{Terms
  of Sale}
\item
  \href{https://spiderbites.nytimes3xbfgragh.onion}{Site Map}
\item
  \href{https://help.nytimes3xbfgragh.onion/hc/en-us}{Help}
\item
  \href{https://www.nytimes3xbfgragh.onion/subscription?campaignId=37WXW}{Subscriptions}
\end{itemize}
