Sections

SEARCH

\protect\hyperlink{site-content}{Skip to
content}\protect\hyperlink{site-index}{Skip to site index}

\href{https://myaccount.nytimes3xbfgragh.onion/auth/login?response_type=cookie\&client_id=vi}{}

\href{https://www.nytimes3xbfgragh.onion/section/todayspaper}{Today's
Paper}

\href{/section/opinion}{Opinion}\textbar{}Our Mental Health Amid a
Pandemic

\url{https://nyti.ms/35e2IeE}

\begin{itemize}
\item
\item
\item
\item
\item
\end{itemize}

Advertisement

\protect\hyperlink{after-top}{Continue reading the main story}

\href{/section/opinion}{Opinion}

Supported by

\protect\hyperlink{after-sponsor}{Continue reading the main story}

letters

\hypertarget{our-mental-health-amid-a-pandemic}{%
\section{Our Mental Health Amid a
Pandemic}\label{our-mental-health-amid-a-pandemic}}

Readers react to a Sunday Review article that asked ``Are You Depressed
or Bored?,'' suggesting that ``we're just bored out of our minds''
rather than clinically ill.

Sept. 7, 2020

\begin{itemize}
\item
\item
\item
\item
\item
\end{itemize}

\includegraphics{https://static01.graylady3jvrrxbe.onion/images/2020/05/19/smarter-living/00well-corona-depression/00well-corona-depression-articleLarge.jpg?quality=75\&auto=webp\&disable=upscale}

\textbf{To the Editor:}

Re
``\href{https://www.nytimes3xbfgragh.onion/2020/08/21/opinion/sunday/covid-depression-boredom.html}{Are
You Depressed or Bored?},'' by Richard A. Friedman (Sunday Review, Aug.
23):

I am quarantined with a 17-year-old young person desperate for peer
interaction and a laid-off spouse consumed by a fruitless job search. I
find myself 100 percent responsible for income and health care, with a
job in education, now forced to launch K-12 schooling in a state led by
Trump groupies, putting my health and that of my colleagues at risk.

Boredom sounds fantastic. Instead, I am plowing my way through
significant anxiety, grateful for help from a therapist and a physician
who (thankfully) are not diminishing the fear this pandemic has loosed
on us all.

Karen Yeager Kimball\\
McKinney, Tex.

\textbf{To the Editor:}

Dr. Richard A. Friedman is right to question the premature conclusion
that the pandemic is causing dramatically increased rates of clinical
depression and anxiety. The claim is based on a handful of mental health
surveys. These self-reported data are notoriously poor predictors of
whether people actually meet diagnostic criteria for mental illness,
which is based on clinical observation.

As a clinical psychologist and college professor, I have been struck by
how my patients and students say that they feel ``depressed,'' but when
asked what they mean describe an altogether different experience. As Dr.
Friedman anticipates, they most commonly describe feelings of boredom
and listlessness. However, people also use the word ``depressed'' to
identify a litany of aversive states: anxiety, restlessness, pent-up
frustration/irritability, inadequacy, guilt and suppressed anger.

In these stressful times, knowing what we feel may not be so
straightforward. If we don't know what we're feeling, we can't
adequately cope with it.

Paul Siegel\\
New York

\textbf{To the Editor:}

While I agree with Dr. Friedman that we shouldn't pathologize normal
everyday stress or boredom, I don't think that the fact that many of his
patients have not experienced depression and anxiety flare-ups during
the pandemic is representative.

So many people have faced tremendous losses, especially my community
health patient base. Those losses include loss of housing, food sources,
jobs, health care, along with family and friends who have passed away.
It is far from ``premature'' to prepare for an uptick in mental health
diagnoses related to the pandemic. That uptick will include
post-traumatic stress, along with many other pathologies. Indeed,
increased preparedness will help ensure that treatment is effective.

Karen Katz\\
Brookline, Mass.\\
\emph{The writer is a community mental health clinician with Advocates
Community Counseling.}

\textbf{To the Editor:}

Out of boredom often springs creativity. This article made a lot of
important points, but what Dr. Friedman didn't say was that when we are
bored, or receive the gift of boredom, we often have our most creative
ideas. Often we have so much going on, so much stimulation that we don't
have space in our brains to be creative. A reason not to overschedule
our children or ourselves is to allow us/them to be bored. Boredom
forces us to search around for something new.

Kathryn Kert Green\\
Santa Monica, Calif.\\
\emph{The writer is an artist and art teacher.}

\textbf{To the Editor:}

Lacking access to excitement and novelty and other distractions may
explain why some people are bored during the pandemic lockdown. But
there are other reasons that while they may not cause depression can
cause something akin to it --- call it sadness or melancholy --- that
are deeply tied to our human nature.

For example, it is part of our human nature to want simple contact with
others. Most of us enjoy friendly conversation with friends. Touching
and hugging fulfill for many a need for human contact that cannot be
replaced by simply finding exciting or distracting things to do.

This is why so many people during the lockdown are texting, making
telephone calls and renewing old friendships --- not because of boredom
but because of their very human need for contact with others.

Maybe some are bored. But many are just yearning for human contact.

Edward Volpintesta\\
Bethel, Conn.

Advertisement

\protect\hyperlink{after-bottom}{Continue reading the main story}

\hypertarget{site-index}{%
\subsection{Site Index}\label{site-index}}

\hypertarget{site-information-navigation}{%
\subsection{Site Information
Navigation}\label{site-information-navigation}}

\begin{itemize}
\tightlist
\item
  \href{https://help.nytimes3xbfgragh.onion/hc/en-us/articles/115014792127-Copyright-notice}{©~2020~The
  New York Times Company}
\end{itemize}

\begin{itemize}
\tightlist
\item
  \href{https://www.nytco.com/}{NYTCo}
\item
  \href{https://help.nytimes3xbfgragh.onion/hc/en-us/articles/115015385887-Contact-Us}{Contact
  Us}
\item
  \href{https://www.nytco.com/careers/}{Work with us}
\item
  \href{https://nytmediakit.com/}{Advertise}
\item
  \href{http://www.tbrandstudio.com/}{T Brand Studio}
\item
  \href{https://www.nytimes3xbfgragh.onion/privacy/cookie-policy\#how-do-i-manage-trackers}{Your
  Ad Choices}
\item
  \href{https://www.nytimes3xbfgragh.onion/privacy}{Privacy}
\item
  \href{https://help.nytimes3xbfgragh.onion/hc/en-us/articles/115014893428-Terms-of-service}{Terms
  of Service}
\item
  \href{https://help.nytimes3xbfgragh.onion/hc/en-us/articles/115014893968-Terms-of-sale}{Terms
  of Sale}
\item
  \href{https://spiderbites.nytimes3xbfgragh.onion}{Site Map}
\item
  \href{https://help.nytimes3xbfgragh.onion/hc/en-us}{Help}
\item
  \href{https://www.nytimes3xbfgragh.onion/subscription?campaignId=37WXW}{Subscriptions}
\end{itemize}
