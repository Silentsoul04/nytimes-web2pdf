Sections

SEARCH

\protect\hyperlink{site-content}{Skip to
content}\protect\hyperlink{site-index}{Skip to site index}

\href{https://myaccount.nytimes3xbfgragh.onion/auth/login?response_type=cookie\&client_id=vi}{}

\href{https://www.nytimes3xbfgragh.onion/section/todayspaper}{Today's
Paper}

\href{/section/opinion}{Opinion}\textbar{}Kids Can Learn to Love
Learning, Even Over Zoom

\url{https://nyti.ms/2Zf2xfe}

\begin{itemize}
\item
\item
\item
\item
\item
\end{itemize}

Advertisement

\protect\hyperlink{after-top}{Continue reading the main story}

\href{/section/opinion}{Opinion}

Supported by

\protect\hyperlink{after-sponsor}{Continue reading the main story}

\hypertarget{kids-can-learn-to-love-learning-even-over-zoom}{%
\section{Kids Can Learn to Love Learning, Even Over
Zoom}\label{kids-can-learn-to-love-learning-even-over-zoom}}

There are ways for teachers to nurture curiosity --- and they're
especially important in online classes.

By \href{https://www.nytimes3xbfgragh.onion/column/adam-grant}{Adam
Grant} and Allison Sweet Grant

Dr. Grant, a contributing opinion writer, and Ms. Grant, a psychiatric
nurse practitioner, are the authors of
``\href{https://www.amazon.com/Leif-Fall-Allison-Sweet-Grant/dp/1984815490}{Leif
and the Fall}.''

\begin{itemize}
\item
  Sept. 7, 2020
\item
  \begin{itemize}
  \item
  \item
  \item
  \item
  \item
  \end{itemize}
\end{itemize}

\includegraphics{https://static01.graylady3jvrrxbe.onion/images/2020/09/07/opinion/07GrantGrant/merlin_176457990_b2a256b0-eb8d-465d-822e-5606a4a58049-articleLarge.jpg?quality=75\&auto=webp\&disable=upscale}

``Can independently mute and unmute himself when requested to do so.''
That's praise we never expected to see a year ago on our son's
kindergarten report card. We're so proud.

As the new school year begins, many students are learning virtually,
either by personal choice or requirement --- and many parents and
teachers are concerned that students will fall behind in their
knowledge. But a greater risk to our students may be that they lose
their curiosity.

Whether students are in kindergarten or college, knowledge is always
attainable. Teachers can and will catch kids up on their multiplication
tables and periodic tables. But in school and in life, success depends
less on how much we know than on how much we want to learn. One of the
highest aims of education is to cultivate and sustain the intrinsic
motivation to learn.

A classic
\href{https://www.amazon.com/Developing-Talent-Young-People-Benjamin/dp/034531509X}{study}
found that world-class artists, athletes, musicians and scientists
typically had an early coach or teacher who made learning fun and
motivated them to hone their skills. An
\href{https://psycnet.apa.org/buy/2014-03897-001}{analysis} of 125
studies of nearly 200,000 students found that the more the students
enjoyed learning, the better they performed from elementary school all
the way to college. Students with high levels of
\href{https://journals.sagepub.com/doi/abs/10.1177/1745691611421204}{intellectual
curiosity} get
\href{https://psycnet.apa.org/record/2017-57172-001}{better grades} than
their peers, even after controlling for their IQ and work ethic.

Unfortunately, remote learning can stifle curiosity. For students, it's
easy to zone out. Staring at a screen all day can be exhausting. For
teachers, transmitting excitement into a webcam is not a simple task: it
can feel like talking into a black hole. Technical difficulties mean
that key points get lost and even brief communication delays can make
students
\href{https://www.sciencedirect.com/science/article/abs/pii/S1071581914000287}{seem}
disengaged, crushing rapport and killing timing.

Still, there are ways for teachers to nurture interest in learning ---
and they're especially important in online classes. Three key principles
are mystery, exploration and meaning.

Curiosity begins with a mystery: a gap between what we understand and
what we want to find out. Behavioral economists
\href{https://books.google.com/books?id=vGmVroMwC0cC\&lpg=PA169\&ots=qwPbpCAlAj\&dq=\%22scientists\%20are\%20relieved\%20of\%20the\%20itch\%20of\%20curiosity\%20that\%20constantly\%22\&pg=PA169\#v=onepage\&q=\%22scientists\%20are\%20relieved\%20of\%20the\%20itch\%20of\%20curiosity\%20that\%20constantly\%22\&f=false}{argue}
that an information gap is like an itch. We can't resist the temptation
to scratch it. Information gaps can motivate us to tear through a
whodunit novel, sit glued to the TV during a quiz show or stare at a
crossword puzzle for hours. Great teachers approach their classes the
same way: They open with a mystery and turn their students into
detectives, sending them off to gather clues.

For example, if you've ever watched dolphins closely, you might have
noticed that they're awake for remarkable stretches of time. A typical
dolphin can stay alert and active 24 hours a day for 15 days straight.
How do they do it?

Given all the challenges of going online, it's natural for teachers to
focus on just getting through the material. But remote learning is
perfectly suited to mystery --- teachers need to find the right puzzles
for students to solve.

If gaps in knowledge are the seeds of curiosity, exploration is the
sunlight. Hundreds of
\href{https://www.pnas.org/content/111/23/8410}{studies} with thousands
of students have shown that when science, technology and math courses
include active learning, students are less likely to fail and more
likely to excel. A key feature of active learning is interaction. But
too many online classes have students listening to one-way monologues
instead of having two-way dialogues. Too many students are sitting in
front of a screen when they could be exploring out in the world.

Leaving a desk isn't just fun; it can promote a lasting desire to learn.
In one
\href{https://www.tandfonline.com/doi/abs/10.1080/19345747.2015.1086915?journalCode=uree20}{experiment},
researchers randomly assigned thousands of students to take a museum
field trip. Three weeks later, when the students wrote essays analyzing
pieces of art, those who had visited the museum scored higher in
critical thinking than those who did not make the trip. The museum-goers
made richer observations and more creative associations. They were also
more curious about views that differed from their own. And the benefits
were even more pronounced for students from rural areas and high-poverty
schools.

When field trips aren't possible, teachers can still take students on
virtual tours and send them off to do hands-on learning projects. In the
past few months, our kids have been lucky to learn from social studies
teachers who challenged them to survey people about their stereotypes of
the elderly, computer science teachers who invited them to design their
own amusement parks, and drama teachers who had them film their own
documentaries.

Meaning is the final piece of the motivation puzzle. Not every lesson
will be riveting; not every class discussion will be electrifying.
However, when students see the real-world consequences of what they are
studying, they're more likely to stay engaged.

Psychologists \href{https://psycnet.apa.org/buy/2014-38071-001}{find}
that when college students have a purpose for learning beyond the self,
they spend more time on tedious math problems and less time playing
video games and watching viral videos. And high schoolers get better
grades in STEM courses after being randomly assigned to reflect on how
the material would help them help others. That's a question every
teacher can ask and answer, even over Zoom: Why does this content
matter? When the answer to this question is clear, students are less
likely to doze through class with one eye open.

Or, in the case of dolphins, with one side of their brains open. They
can put one hemisphere of their brains to sleep and leave the other
alert. That's how they stay active for two weeks straight.

The purpose of school is not just to impart knowledge; it's to instill a
love of learning. In online schools and hybrid classrooms, that love
doesn't have to be lost.

One good thing about virtual school is that children are building skills
that will serve them well throughout their lives. Although learning how
to mute and unmute himself is not something we ever thought our
kindergartner would need to know, it's one of many new skills from
online classes that will continue to come in handy. And for those adults
who are still having trouble with that particular skill (you know who
you are), he's available for online instruction.

Adam Grant is an organizational psychologist at Wharton and contributing
opinion writer. Allison Sweet Grant is a psychiatric nurse practitioner
and writer. They are married and co-authors of the new children's book
``\href{https://www.amazon.com/Leif-Fall-Allison-Sweet-Grant/dp/1984815490}{Leif
and the Fall}.''

\emph{The Times is committed to publishing}
\href{https://www.nytimes3xbfgragh.onion/2019/01/31/opinion/letters/letters-to-editor-new-york-times-women.html}{\emph{a
diversity of letters}} \emph{to the editor. We'd like to hear what you
think about this or any of our articles. Here are some}
\href{https://help.nytimes3xbfgragh.onion/hc/en-us/articles/115014925288-How-to-submit-a-letter-to-the-editor}{\emph{tips}}\emph{.
And here's our email:}
\href{mailto:letters@NYTimes.com}{\emph{letters@NYTimes.com}}\emph{.}

\emph{Follow The New York Times Opinion section on}
\href{https://www.facebookcorewwwi.onion/nytopinion}{\emph{Facebook}}\emph{,}
\href{http://twitter.com/NYTOpinion}{\emph{Twitter (@NYTopinion)}}
\emph{and}
\href{https://www.instagram.com/nytopinion/}{\emph{Instagram}}\emph{.}

Advertisement

\protect\hyperlink{after-bottom}{Continue reading the main story}

\hypertarget{site-index}{%
\subsection{Site Index}\label{site-index}}

\hypertarget{site-information-navigation}{%
\subsection{Site Information
Navigation}\label{site-information-navigation}}

\begin{itemize}
\tightlist
\item
  \href{https://help.nytimes3xbfgragh.onion/hc/en-us/articles/115014792127-Copyright-notice}{©~2020~The
  New York Times Company}
\end{itemize}

\begin{itemize}
\tightlist
\item
  \href{https://www.nytco.com/}{NYTCo}
\item
  \href{https://help.nytimes3xbfgragh.onion/hc/en-us/articles/115015385887-Contact-Us}{Contact
  Us}
\item
  \href{https://www.nytco.com/careers/}{Work with us}
\item
  \href{https://nytmediakit.com/}{Advertise}
\item
  \href{http://www.tbrandstudio.com/}{T Brand Studio}
\item
  \href{https://www.nytimes3xbfgragh.onion/privacy/cookie-policy\#how-do-i-manage-trackers}{Your
  Ad Choices}
\item
  \href{https://www.nytimes3xbfgragh.onion/privacy}{Privacy}
\item
  \href{https://help.nytimes3xbfgragh.onion/hc/en-us/articles/115014893428-Terms-of-service}{Terms
  of Service}
\item
  \href{https://help.nytimes3xbfgragh.onion/hc/en-us/articles/115014893968-Terms-of-sale}{Terms
  of Sale}
\item
  \href{https://spiderbites.nytimes3xbfgragh.onion}{Site Map}
\item
  \href{https://help.nytimes3xbfgragh.onion/hc/en-us}{Help}
\item
  \href{https://www.nytimes3xbfgragh.onion/subscription?campaignId=37WXW}{Subscriptions}
\end{itemize}
