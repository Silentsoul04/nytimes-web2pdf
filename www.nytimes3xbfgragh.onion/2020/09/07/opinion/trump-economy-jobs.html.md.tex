Sections

SEARCH

\protect\hyperlink{site-content}{Skip to
content}\protect\hyperlink{site-index}{Skip to site index}

\href{https://myaccount.nytimes3xbfgragh.onion/auth/login?response_type=cookie\&client_id=vi}{}

\href{https://www.nytimes3xbfgragh.onion/section/todayspaper}{Today's
Paper}

\href{/section/opinion}{Opinion}\textbar{}Gross Domestic Misery Is
Rising

\url{https://nyti.ms/2F5hklO}

\begin{itemize}
\item
\item
\item
\item
\item
\item
\end{itemize}

Advertisement

\protect\hyperlink{after-top}{Continue reading the main story}

\href{/section/opinion}{Opinion}

Supported by

\protect\hyperlink{after-sponsor}{Continue reading the main story}

\hypertarget{gross-domestic-misery-is-rising}{%
\section{Gross Domestic Misery Is
Rising}\label{gross-domestic-misery-is-rising}}

The recovery is bypassing those who need it most.

\href{https://www.nytimes3xbfgragh.onion/by/paul-krugman}{\includegraphics{https://static01.graylady3jvrrxbe.onion/images/2018/04/02/opinion/paul-krugman/paul-krugman-thumbLarge.png}}

By \href{https://www.nytimes3xbfgragh.onion/by/paul-krugman}{Paul
Krugman}

Opinion Columnist

\begin{itemize}
\item
  Sept. 7, 2020
\item
  \begin{itemize}
  \item
  \item
  \item
  \item
  \item
  \item
  \end{itemize}
\end{itemize}

\includegraphics{https://static01.graylady3jvrrxbe.onion/images/2020/09/07/opinion/07Krugman/merlin_176518893_18b2743f-613a-4401-98d5-8cd50f8781eb-articleLarge.jpg?quality=75\&auto=webp\&disable=upscale}

Are you better off now than you were in July?

On the face of it, that shouldn't even be a question. After all, stocks
are up; the economy
\href{https://www.nytimes3xbfgragh.onion/2020/09/04/business/economy/jobs-report.html}{added
more than a million jobs} in ``August'' (I'll explain the scare quotes
in a minute); preliminary estimates suggest that G.D.P.
\href{https://www.calculatedriskblog.com/2020/09/q3-gdp-forecasts.html}{is
growing rapidly} in the third quarter, which ends this month.

But the stock market isn't the economy:
\href{https://fred.stlouisfed.org/series/WFRBST01122}{more than half} of
all stocks are owned by only 1 percent of Americans, while the bottom
half of the population owns only
\href{https://fred.stlouisfed.org/series/WFRBSB50203}{0.7 percent} of
the market.

Jobs and G.D.P., by contrast, sort of are the economy. But they aren't
the economy's point. What some economists and many politicians often
forget is that economics isn't fundamentally about data, it's about
people. I like data as much as, or probably more than, the next guy. But
an economy's success should be judged not by impersonal statistics, but
by whether people's lives are getting better.

And the simple fact is that over the past few weeks the lives of many
Americans have gotten much worse.

Obviously this is true for the roughly
\href{https://ourworldindata.org/coronavirus-data-explorer?zoomToSelection=true\&year=latest\&time=2020-03-01..2020-09-06\&country=USA~EuropeanUnion\&region=World\&deathsMetric=true\&interval=total\&smoothing=0\&pickerMetric=total_deaths\&pickerSort=desc}{30,000
Americans} who died of Covid-19 in August --- for comparison, only 4,000
people died in the European Union, which has a larger population ---
plus the unknown but large number of our citizens who suffered long-term
health damage. And don't look now, but the number of new coronavirus
cases, which had been declining, seems to have
\href{https://covidtracking.com/data/charts/us-daily-positive}{plateaued};
between Labor Day and school re-openings, there's a pretty good chance
that the virus situation is about to take another turn for the worse.

But things have already gotten worse for millions of families that lost
most of their normal income as a result of the pandemic and still
haven't gotten it back. For the first few months of the pandemic
depression many of these Americans were getting by thanks to emergency
federal aid. But much of that aid was cut off at the end of July, and
despite job gains we're in the midst of a huge increase in national
misery.

So let's talk about that employment report.

One important thing to bear in mind about official monthly job
statistics is that they're based on surveys conducted during the second
week of the month. That's why I used scare quotes around ``August'':
What Friday's report actually gave us was a snapshot of the state of the
labor market around Aug. 12.

This may be important.
\href{https://twitter.com/bencasselman/status/1301970073830199298}{Private
data} suggest a slowdown in job growth since late July. So the next
employment report, which will be based on data collected \emph{this
week} --- and will also be the last report before the election --- will
probably (not certainly) be weaker than the last.

In any case, that August report wasn't great considering the context. In
normal times a gain of 1.4 million jobs would be impressive, even if
some of those jobs were a temporary blip associated with the census. But
we're still more than
\href{https://fred.stlouisfed.org/series/PAYEMS}{11 million jobs down}
from where we were in February.

And the situation remains dire for the hardest-hit workers. The pandemic
slump disproportionately hit workers in the leisure and hospitality
sector --- think restaurants --- and employment in that sector is still
down
\href{https://twitter.com/ernietedeschi/status/1302305128863739906}{around
25 percent}, while the unemployment rate for workers in the industry is
still \href{https://www.bls.gov/news.release/pdf/empsit.pdf}{over 20
percent}, more than four times what it was a year ago.

In part because of where the slump was concentrated, the unemployed tend
to be Americans who were earning low wages even before the slump. And
one disturbing fact about the August report was that average wages rose.
No, that's not a misprint: If the low-wage workers hit worst by the
slump were being rehired, we'd expect average wages to fall, as they did
during the
\href{https://www.bls.gov/news.release/pdf/empsit.pdf}{snapback} of May
and June. Rising average wages at this point are a sign that those who
really need jobs aren't getting them.

So the economy is still bypassing those who need a recovery most.

Yet most of the safety net that temporarily sustained the economic
victims of the coronavirus has been torn down.

The CARES Act, enacted in March, gave the unemployed an extra \$600 a
week in benefits. This supplement played a crucial role in limiting
extreme hardship; poverty may even have
\href{https://www.nytimes3xbfgragh.onion/2020/06/21/us/politics/coronavirus-poverty.html}{gone
down}.

But the supplement ended on July 31, and all indications are that
Republicans in the Senate will do nothing to restore aid before the
election. President Trump's attempt to implement a \$300 per week
supplement by executive action will fail to reach many and prove
inadequate even for those who get it. Families may have scraped by for a
few weeks on saved money, but things are about to get very hard for
millions.

The bottom line here is that before you cite economic statistics, you
want to think about what they mean for people and their lives. The data
aren't meaningless: A million jobs gained is better than a million jobs
lost, and growing G.D.P. is better than shrinking G.D.P. But there is
often a disconnect between the headline numbers and the reality of
American life, and that is especially true right now.

The fact is that this economy just isn't working for many Americans, who
are facing hard times that --- thanks to political decisions by Trump
and his allies --- are just getting harder.

\emph{The Times is committed to publishing}
\href{https://www.nytimes3xbfgragh.onion/2019/01/31/opinion/letters/letters-to-editor-new-york-times-women.html}{\emph{a
diversity of letters}} \emph{to the editor. We'd like to hear what you
think about this or any of our articles. Here are some}
\href{https://help.nytimes3xbfgragh.onion/hc/en-us/articles/115014925288-How-to-submit-a-letter-to-the-editor}{\emph{tips}}\emph{.
And here's our email:}
\href{mailto:letters@NYTimes.com}{\emph{letters@NYTimes.com}}\emph{.}

\emph{Follow The New York Times Opinion section on}
\href{https://www.facebookcorewwwi.onion/nytopinion}{\emph{Facebook}}\emph{,}
\href{http://twitter.com/NYTOpinion}{\emph{Twitter (@NYTopinion)}}
\emph{and}
\href{https://www.instagram.com/nytopinion/}{\emph{Instagram}}\emph{.}

Advertisement

\protect\hyperlink{after-bottom}{Continue reading the main story}

\hypertarget{site-index}{%
\subsection{Site Index}\label{site-index}}

\hypertarget{site-information-navigation}{%
\subsection{Site Information
Navigation}\label{site-information-navigation}}

\begin{itemize}
\tightlist
\item
  \href{https://help.nytimes3xbfgragh.onion/hc/en-us/articles/115014792127-Copyright-notice}{©~2020~The
  New York Times Company}
\end{itemize}

\begin{itemize}
\tightlist
\item
  \href{https://www.nytco.com/}{NYTCo}
\item
  \href{https://help.nytimes3xbfgragh.onion/hc/en-us/articles/115015385887-Contact-Us}{Contact
  Us}
\item
  \href{https://www.nytco.com/careers/}{Work with us}
\item
  \href{https://nytmediakit.com/}{Advertise}
\item
  \href{http://www.tbrandstudio.com/}{T Brand Studio}
\item
  \href{https://www.nytimes3xbfgragh.onion/privacy/cookie-policy\#how-do-i-manage-trackers}{Your
  Ad Choices}
\item
  \href{https://www.nytimes3xbfgragh.onion/privacy}{Privacy}
\item
  \href{https://help.nytimes3xbfgragh.onion/hc/en-us/articles/115014893428-Terms-of-service}{Terms
  of Service}
\item
  \href{https://help.nytimes3xbfgragh.onion/hc/en-us/articles/115014893968-Terms-of-sale}{Terms
  of Sale}
\item
  \href{https://spiderbites.nytimes3xbfgragh.onion}{Site Map}
\item
  \href{https://help.nytimes3xbfgragh.onion/hc/en-us}{Help}
\item
  \href{https://www.nytimes3xbfgragh.onion/subscription?campaignId=37WXW}{Subscriptions}
\end{itemize}
