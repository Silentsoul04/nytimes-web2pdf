Sections

SEARCH

\protect\hyperlink{site-content}{Skip to
content}\protect\hyperlink{site-index}{Skip to site index}

\href{https://myaccount.nytimes3xbfgragh.onion/auth/login?response_type=cookie\&client_id=vi}{}

\href{https://www.nytimes3xbfgragh.onion/section/todayspaper}{Today's
Paper}

\href{/section/opinion}{Opinion}\textbar{}What the Oil Spill in
Venezuela Tells Us About Its Politics

\url{https://nyti.ms/2ZbEGx6}

\begin{itemize}
\item
\item
\item
\item
\item
\end{itemize}

Advertisement

\protect\hyperlink{after-top}{Continue reading the main story}

\href{/section/opinion}{Opinion}

Supported by

\protect\hyperlink{after-sponsor}{Continue reading the main story}

\hypertarget{what-the-oil-spill-in-venezuela-tells-us-about-its-politics}{%
\section{What the Oil Spill in Venezuela Tells Us About Its
Politics}\label{what-the-oil-spill-in-venezuela-tells-us-about-its-politics}}

The largest oil reserves and one of the world's most incompetent
governments have brought authoritarianism, economic collapse and
environmental disaster to the country.

By Javier Corrales

Mr. Corrales, an expert on Latin America, is a professor of political
science at Amherst College.

\begin{itemize}
\item
  Sept. 7, 2020
\item
  \begin{itemize}
  \item
  \item
  \item
  \item
  \item
  \end{itemize}
\end{itemize}

\includegraphics{https://static01.graylady3jvrrxbe.onion/images/2020/09/07/opinion/07corrales2/merlin_175615317_d0df9f86-bc5a-4c0e-8328-af79ed977825-articleLarge.jpg?quality=75\&auto=webp\&disable=upscale}

\href{https://www.nytimes3xbfgragh.onion/es/2020/09/07/espanol/opinion/lo-que-los-derrames-petroleros-en-venezuela-revelan.html}{Leer
en español}

AMHERST, Mass. --- It has been a tough summer for Venezuela. The already
ailing country, in the throes of a
\href{https://www.bbc.com/news/world-52103747}{severe lockdown}, is also
experiencing a major environmental disaster. In July, a state-owned
refinery began to spill oil into the
\href{https://www.bbc.com/news/world-latin-america-53767424}{Morrocoy
National Park}, one of the country's most biodiverse areas. Venezuela
also experienced a new political crisis. The government essentially
voided several opposition parties by
\href{https://www.nytimes3xbfgragh.onion/2020/06/16/world/americas/venezuela-maduro-supreme-court.html}{taking
control} of their executive boards.

These catastrophes are two sides of the same coin. Rising
authoritarianism in Venezuela has led to oil mismanagement, which in
turn has led to environmental degradation. And oil mismanagement is now
turning the regime even more autocratic, which in turn is leading to
opposition debasement.

Pundits often debate whether
\href{https://www.vox.com/2014/4/10/5601062/oil-curse-explained}{rising
oil fortunes contribute to the rise of authoritarianism}. Large oil
windfalls, the argument goes, allow states to offer consumption booms to
the public in lieu of political rights and to fund repressive forces.
But the Venezuelan case seems to be showing that declining oil fortunes
can be both a cause and a consequence of hardening authoritarianism.

Venezuela used to be one of the most competitive oil producers in the
world. But its oil industry has been run into the ground over the last
two decades, first under President Hugo Chávez and now under his
successor, Nicolás Maduro. Measured in terms of proven reserves,
Venezuela may have
\href{https://www.opec.org/opec_web/en/data_graphs/330.htm}{more oil
than Saudi Arabia}. But in terms of output, Venezuela's oil industry has
collapsed. The country's production of oil is at a
\href{https://www.reuters.com/article/us-venezuela-oil-exports/venezuelas-oil-exports-sank-in-june-to-77-year-low-data-idUSKBN2427AC}{77-year
low}.

The lesson is clear. Political accountability, human rights and
environmental sustainability constitute a modern-day trifecta. Lose the
former, and the rest disappears as well.

Mr. Chávez eroded the checks and balances inside and outside the state
oil company and turned it into his own A.T.M. Party loyalists replaced
oil engineers. Investment protocols were discontinued. Safety standards
were ignored. All that mattered was for the oil company to channel
dollars to fund the ruling party's elections.

Not surprisingly,
\href{https://www.researchgate.net/publication/341117583_Oil_and_regime_type_in_Latin_America_Reversing_the_line_of_causality}{production
declined} between 2003 and 2014. Oddly, the decline took place at a time
when the price of oil price was booming. No freely trading oil nation
experienced this strange outcome. While many blame U.S. sanctions under
President Trump, the evidence that the collapse was
homemade\href{https://growthlab.cid.harvard.edu/publications/venezuelan-oil-assessment}{and
pre-Trump} is overwhelming.

Venezuela's oil collapse has taken a huge environmental toll.
Corruption, underinvestments and weak controls led to the state-owned
oil company's economic collapse. They are also responsible for an
increase in \href{https://venezuelanalysis.com/analysis/14984}{company
accidents} and oil spills. According to a report, the country
experienced
\href{https://www.derechos.org.ve/actualidad/en-seis-anos-pdvsa-derramo-856-72285-barriles-de-petroleo-al-medio-ambiente}{46,820
toxic spills} from 2010 to 2018, totaling 856,000 barrels of spilled
oil. From July to August of this year, an estimated
\href{https://www.forbes.com/sites/nishandegnarain/2020/08/28/oil-spill-august-what-the-major-oil-spills-in-venezuela-and-mauritius-mean-for-the-world/\#68dd06bd59bd}{26,000
barrels} of oil may have affected more than
\href{http://www.petroleumworld.com/storyt20082001.htm}{210 miles} of
shoreline. The July spill is the second major spill in a year. And a few
days ago, reports surfaced that a huge oil tanker stationed in
Venezuelan waters, the FSO Nabarima, was on the verge of sinking because
of a lack of proper maintenance. If it sinks, the resulting oil spill
could be
\href{https://www.nytimes3xbfgragh.onion/aponline/2020/09/02/world/americas/ap-lt-venezuela-sinking-oil-tanker.html\#:~:text=CARACAS\%2C\%20Venezuela\%20\%E2\%80\%94\%20The\%20sight\%20of,a\%20dangerous\%20state\%20of\%20disrepair.}{five
times larger than the}Exxon Valdez spill of 1989.

The collapse of oil prices from mid-2014 to early 2016 also deepened
Venezuela's economic crisis, overwhelming the administration of Mr.
Maduro. Because production was already so low in 2015, Venezuela's
economy sank more than those of other petrostates. The country's economy
has continued to contract every year since then, leading to a
humanitarian and
\href{https://www.nytimes3xbfgragh.onion/2019/02/20/world/americas/venezuela-refugees-colombia.html}{refugee
crisis} comparable to that experienced by
\href{https://www.devex.com/news/venezuela-crisis-is-on-the-scale-of-syria-unhcr-says-93465}{war-torn
Syria}.

This oil crisis is also producing a hardening of authoritarianism. Under
normal circumstances, an economic crisis such as Venezuela's would have
produced one of two political outcomes: a change in policy or a change
in government. In Venezuela, it is producing more repression.

For ruling parties, policy changes make sense when the ruling party is
interested in staying electorally competitive. But since the mid-2000s,
Venezuela's ruling party has given up on fair elections. It is only
interested in staying in office.

So instead of policy corrections, Mr. Maduro has relied on unregulated
gold mining (whose toll on the environment and human security is also
dismal), crackdowns on citizens' protests and electoral tricks to disarm
the opposition. This culminated in this summer's nationalization of the
opposition parties.

The Constitution of Venezuela mandates that Mr. Maduro schedule a
legislative election this year. The government knows it cannot win such
an election competing freely, so it has
\href{https://www.hrw.org/news/2020/07/07/venezuela-rulings-threaten-free-and-fair-elections}{opted
to change the electoral rules}. The government has expanded the number
of seats in the National Assembly from 167 to 277 with the aim of
diluting the power of the strongest opposition parties now in control.
It has also refused to make electoral authorities impartial and replaced
the leadership of opposition parties with people willing to go along
with the government. Mr. Maduro has pardoned more than 100 political
prisoners, which is a nice concession, but has kept the electoral
irregularities in place. These irregularities have
\href{https://www.washingtonpost.com/world/the_americas/venezuelas-maduro-pardons-more-than-100-political-opponents-ahead-of-elections/2020/08/31/c5770df0-ebbf-11ea-b4bc-3a2098fc73d4_story.html}{split
the opposition into two camps}, with one sector hoping to compete
electorally and another calling for abstention.

The United States is claiming, rightly, that
\href{https://www.reuters.com/article/us-venezuela-politics/us-accuses-venezuelas-maduro-of-seeking-to-rig-upcoming-vote-idUSKBN23M2OJ}{the
election is rigged}. It may even be encouraging the opposition to
abstain rather than unite electorally. The problem is that abstention is
exactly what the Venezuelan government wants. The United States may be
unintentionally helping the government weaken the once electorally
mighty opposition.

The United States has also played a role in the oil spill. While the
spill is the result of industry decay in Venezuela, its continuation is
connected to the U.S. oil embargo. Venezuela is now
\href{https://www.washingtonpost.com/world/the_americas/venezuela-crisis-oil-gas-shortage-maduro-guaido/2020/04/15/19fd9864-7daa-11ea-84c2-0792d8591911_story.html}{precluded}
from using refineries in the United States to process its oil into
gasoline. This is one reason the government has not shut down the
damaged refinery: It is the only one in the country that produces
gasoline. So
\href{https://www.cambio16.com/derrame-de-petroleo-en-venezuela-aun-no-ha-cesado/}{the
leak has continued}, with oil now entering rivers and lakes.

It's easy to blame factors such as poor vision by the opposition and
inconsistent responses by the United States for Venezuela's turn to
incompetent and mean authoritarianism. Those factors are present, but
they are not the main drivers. Venezuela's descent into authoritarianism
has the same source as July's oil spill: Venezuela is a petrostate that
has lost interest in accountability.

Javier Corrales,
(\href{https://twitter.com/jcorrales2011}{@jcorrales2011}) a professor
of political science at Amherst College, is the author, most recently,
of ``Fixing Democracy: Why Constitutional Change Often Fails to Enhance
Democracy in Latin America.''

\emph{The Times is committed to publishing}
\href{https://www.nytimes3xbfgragh.onion/2019/01/31/opinion/letters/letters-to-editor-new-york-times-women.html}{\emph{a
diversity of letters}} \emph{to the editor. We'd like to hear what you
think about this or any of our articles. Here are some}
\href{https://help.nytimes3xbfgragh.onion/hc/en-us/articles/115014925288-How-to-submit-a-letter-to-the-editor}{\emph{tips}}\emph{.
And here's our email:}
\href{mailto:letters@NYTimes.com}{\emph{letters@NYTimes.com}}\emph{.}

\emph{Follow The New York Times Opinion section on}
\href{https://www.facebookcorewwwi.onion/nytopinion}{\emph{Facebook}}\emph{,}
\href{http://twitter.com/NYTOpinion}{\emph{Twitter (@NYTopinion)}}
\emph{and}
\href{https://www.instagram.com/nytopinion/}{\emph{Instagram}}\emph{.}

Advertisement

\protect\hyperlink{after-bottom}{Continue reading the main story}

\hypertarget{site-index}{%
\subsection{Site Index}\label{site-index}}

\hypertarget{site-information-navigation}{%
\subsection{Site Information
Navigation}\label{site-information-navigation}}

\begin{itemize}
\tightlist
\item
  \href{https://help.nytimes3xbfgragh.onion/hc/en-us/articles/115014792127-Copyright-notice}{©~2020~The
  New York Times Company}
\end{itemize}

\begin{itemize}
\tightlist
\item
  \href{https://www.nytco.com/}{NYTCo}
\item
  \href{https://help.nytimes3xbfgragh.onion/hc/en-us/articles/115015385887-Contact-Us}{Contact
  Us}
\item
  \href{https://www.nytco.com/careers/}{Work with us}
\item
  \href{https://nytmediakit.com/}{Advertise}
\item
  \href{http://www.tbrandstudio.com/}{T Brand Studio}
\item
  \href{https://www.nytimes3xbfgragh.onion/privacy/cookie-policy\#how-do-i-manage-trackers}{Your
  Ad Choices}
\item
  \href{https://www.nytimes3xbfgragh.onion/privacy}{Privacy}
\item
  \href{https://help.nytimes3xbfgragh.onion/hc/en-us/articles/115014893428-Terms-of-service}{Terms
  of Service}
\item
  \href{https://help.nytimes3xbfgragh.onion/hc/en-us/articles/115014893968-Terms-of-sale}{Terms
  of Sale}
\item
  \href{https://spiderbites.nytimes3xbfgragh.onion}{Site Map}
\item
  \href{https://help.nytimes3xbfgragh.onion/hc/en-us}{Help}
\item
  \href{https://www.nytimes3xbfgragh.onion/subscription?campaignId=37WXW}{Subscriptions}
\end{itemize}
