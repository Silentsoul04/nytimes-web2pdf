Sections

SEARCH

\protect\hyperlink{site-content}{Skip to
content}\protect\hyperlink{site-index}{Skip to site index}

\href{https://www.nytimes3xbfgragh.onion/section/sports}{Sports}

\href{https://myaccount.nytimes3xbfgragh.onion/auth/login?response_type=cookie\&client_id=vi}{}

\href{https://www.nytimes3xbfgragh.onion/section/todayspaper}{Today's
Paper}

\href{/section/sports}{Sports}\textbar{}2020 U.S. Open: What to Watch on
Monday

\url{https://nyti.ms/3lYS1Cn}

\begin{itemize}
\item
\item
\item
\item
\item
\end{itemize}

Advertisement

\protect\hyperlink{after-top}{Continue reading the main story}

Supported by

\protect\hyperlink{after-sponsor}{Continue reading the main story}

\hypertarget{2020-us-open-what-to-watch-on-monday}{%
\section{2020 U.S. Open: What to Watch on
Monday}\label{2020-us-open-what-to-watch-on-monday}}

The United States Open continues with its Labor Day lineup, as Serena
Williams and Sofia Kenin headline the day.

By Max Gendler

\begin{itemize}
\item
  Sept. 7, 2020
\item
  \begin{itemize}
  \item
  \item
  \item
  \item
  \item
  \end{itemize}
\end{itemize}

\textbf{How to watch:} From 11 a.m. to 4 p.m. Eastern time on ESPN; 4
p.m. to 11 p.m. on ESPN2; and streaming on the ESPN app.

The singles round of 16 and doubles quarterfinal matches continue on
Monday at the United States Open. With the top seeds out in each of the
competitions, it seems as if this year's titles may be up for grabs for
whoever can string together one good run.

\hypertarget{here-are-some-matches-to-keep-an-eye-on}{%
\subsection{Here are some matches to keep an eye
on.}\label{here-are-some-matches-to-keep-an-eye-on}}

\emph{Because of the number of matches cycling through courts, the times
for individual matchups are at best a guess and are certain to fluctuate
based on the times at which earlier play is completed. All times are
Eastern.}

Arthur Ashe Stadium \textbar{} NOON

\hypertarget{serena-williams-vs-maria-sakkari}{%
\subsubsection{\texorpdfstring{\textbf{Serena Williams vs. Maria
Sakkari}}{Serena Williams vs. Maria Sakkari}}\label{serena-williams-vs-maria-sakkari}}

On Saturday, as Williams came to the end of the first set of her matchup
against Sloane Stephens, it seemed for a moment that there was nowhere
for the 23-time major champion to go. Stephens was playing well ---
controlling points, knocking Williams's powerful groundstrokes back with
ease. But Williams drew on her experience and mental fortitude and
simply started to play better. It's a pattern we see all too often with
great champions: There always seems to be an extra gear, a reserve of
talent that comes out just when it needs to. Now that mental toughness
will be tested as Williams plays Sakkari for a spot in the
quarterfinals.

Sakkari, the 15th seed, beat Williams in three sets during their meeting
at the Western \& Southern Open last week. Sakkari has been having a
breakthrough year, reaching her first major round of 16 at the
Australian Open in January and now repeating that feat. In her news
conference, Sakkari acknowledged the tough task in front of her. Asked
what the key had been to beating Serena last week, Sakkari demurred,
saying: ``Serena is Serena. You have to come up with some great tennis.
Otherwise there is no chance against her.''

Williams will need to go into the match the way she finished her third
round --- with energy. If she can make it clear from the onset that
Sakkari will need to play great tennis to scrounge a game, then she can
create unforced errors as Sakkari tries to push her game to its limits.

Louis Armstrong Stadium \textbar{} 4 p.m.

\hypertarget{matteo-berrettini-vs-andrey-rublev}{%
\subsubsection{\texorpdfstring{\textbf{Matteo Berrettini vs. Andrey
Rublev}}{Matteo Berrettini vs. Andrey Rublev}}\label{matteo-berrettini-vs-andrey-rublev}}

At last year's U.S. Open, Berrettini and Rublev were having breakout
tournaments, impressing pundits and fans alike with their explosive
styles of tennis. They met in the fourth round, where Berrettini
overcame Rublev in three sets.

Now they meet once again in the fourth round, no longer as unknowns but
as clear challengers for the title. Rublev has yet to drop a set in this
year's tournament and lost only four games in his runaway victory over
Salvatore Caruso in the third round. Berrettini also has not lost a set
in this tournament, and although his score lines may not seem as
convincing at a glance, the performances have been dominant.

It's a fascinating show of the spectrum of tennis players' physicality
to see this matchup in particular. Rublev and Berrettini both resemble
boxers --- but of entirely different weight classes. Rublev looks like a
lightweight, one who makes it difficult to understand where the immense
strength of his flat, probing shots comes from. Berrettini is the
heavyweight, pure muscle whose power needs to be brought under control
with heavy topspin, lest his ball fly to the top of the stands. But on a
tennis court, there are no weight classes, and these two will be judged
by who can take control of their powerful baseline exchanges.

Arthur Ashe Stadium \textbar{} 7 p.m.

\hypertarget{frances-tiafoe-vs-daniil-medvedev}{%
\subsubsection{\texorpdfstring{\textbf{Frances Tiafoe vs. Daniil
Medvedev}}{Frances Tiafoe vs. Daniil Medvedev}}\label{frances-tiafoe-vs-daniil-medvedev}}

Tiafoe, the last American standing in the men's singles competition,
cruised past Marton Fucsovics in their third-round matchup. Tiafoe
relied heavily on his powerful serving to set up easy points and
executed on a plan of action that was clear and businesslike. He spoke
during his post-match interview on Saturday about his determination to
increase his level of focus, explaining that while he has fun on the
court, he has been working with his coach to improve in this area.
Against Medvedev, that focus will come into question.

Medvedev, last year's runner-up, has not lost a set on his way to the
round of 16, and on average has lost only eight games per match. While
Tiafoe has a powerful serve, Medvedev is an adept returner, his lanky
frame and quick reflexes allowing him to redirect powerful serves deep
into an opponent's court. His excellent court movement allows him to
play defensively, wearing down opponents until they give him a short
ball to attack or commit an unforced error. While Medvedev will be
highly favored in today's match, Tiafoe can give him a strong challenge
if he is able to bring the high level of play that he demonstrated on
Saturday.

Arthur Ashe Stadium \textbar{} 10 p.m.

\hypertarget{sofia-kenin-vs-elise-mertens}{%
\subsubsection{\texorpdfstring{\textbf{Sofia Kenin vs. Elise
Mertens}}{Sofia Kenin vs. Elise Mertens}}\label{sofia-kenin-vs-elise-mertens}}

Kenin, the Australian Open champion, has won twice before in matchups
with Mertens. But although the young American came out on top in both
encounters, neither was a decisive victory. The stage is set for another
tough competition, as both players have been on form in this year's U.S.
Open.

Neither has dropped a set on her road to the round of 16, and each has
done so in the face of dominant players. An all-around player with
plenty of shots to choose from, Kenin has relied on her outstanding
backhand to push past opponents. Even while cramping in the second set
of her matchup against Ons Jabeur, her court movement was unimpeachable,
and it will be interesting to see if she makes any changes to avoid
similar discomfort today.

Mertens, who reached the quarterfinals at last year's U.S. Open, has had
some struggles with her serving form through the first week of the
competition. If she's going to upset the second seed, she'll need to be
more consistent and try to force some returning errors.

\hypertarget{other-important-matches}{%
\subsection{Other important matches:}\label{other-important-matches}}

\begin{itemize}
\item
  Alizé Cornet versus Tsvetana Pironkova, Louis Armstrong Stadium
  \textbar{} 2 p.m.

  The unranked Pironkova's fairy-tale return to the WTA tour continues,
  in her first tournament since Wimbledon in 2017. She has upset two
  seeded players without dropping a set en route to the round of 16.
  She'll be hoping to continue that streak on the strength of her
  powerful serve and backhand.
\item
  Victoria Azarenka versus Karolina Muchova, Louis Armstrong Stadium
  \textbar{} 7 p.m.

  Azarenka, a two time Australian Open champion, looks as dominant as
  she did at the peak of her career. The 24-year-old Muchova has a game
  built in a similar style, and the clash between these two full-court
  players should be enthralling.
\item
  Dominic Thiem versus Felix Auger-Aliassime, Arthur Ashe Stadium
  \textbar{} 2 p.m.

  Thiem, who has been to three Grand Slam finals, is now the top-ranked
  player left in this year's competition, but Auger-Aliassime has a
  style of play that the fast hardcourts of Flushing Meadows favor.
\item
  Alex de Minaur versus Vasek Pospisil, Louis Armstrong Stadium
  \textbar{} 11 a.m.

  Pospisil, a doubles specialist with a big serve, is looking to reach
  the quarterfinals of a major tournament for the second time. De
  Minaur, a fleet-footed defensive specialist, will look to outmaneuver
  him for a chance to go to his first.
\item
  Neal Skupski/Jamie Murray versus Bruno Soares/Mate Pavic, Court 17
  \textbar{} 1 p.m.

  Murray and Soares, opponents today, won two Grand Slam doubles titles
  together in 2016. In the highly tactical and choreographed world of
  doubles, it will be interesting to see how they try to outfox their
  former partners today.
\item
  Nicole Melichar/Xu Yifan versus Hayley Carter/Luisa Stefani, Court 17
  \textbar{} 5 p.m.

  Both of these pairs began playing together within the past year, and
  the quality of the partnerships has been clear. While communication is
  clearly stressed by any doubles coach, both teams seem to move as if
  in lock step with each other, and with the lack of fans, even casual
  viewers will be able to gain insight into their split-second
  decision-making and how to be an effective partner.
\end{itemize}

Advertisement

\protect\hyperlink{after-bottom}{Continue reading the main story}

\hypertarget{site-index}{%
\subsection{Site Index}\label{site-index}}

\hypertarget{site-information-navigation}{%
\subsection{Site Information
Navigation}\label{site-information-navigation}}

\begin{itemize}
\tightlist
\item
  \href{https://help.nytimes3xbfgragh.onion/hc/en-us/articles/115014792127-Copyright-notice}{©~2020~The
  New York Times Company}
\end{itemize}

\begin{itemize}
\tightlist
\item
  \href{https://www.nytco.com/}{NYTCo}
\item
  \href{https://help.nytimes3xbfgragh.onion/hc/en-us/articles/115015385887-Contact-Us}{Contact
  Us}
\item
  \href{https://www.nytco.com/careers/}{Work with us}
\item
  \href{https://nytmediakit.com/}{Advertise}
\item
  \href{http://www.tbrandstudio.com/}{T Brand Studio}
\item
  \href{https://www.nytimes3xbfgragh.onion/privacy/cookie-policy\#how-do-i-manage-trackers}{Your
  Ad Choices}
\item
  \href{https://www.nytimes3xbfgragh.onion/privacy}{Privacy}
\item
  \href{https://help.nytimes3xbfgragh.onion/hc/en-us/articles/115014893428-Terms-of-service}{Terms
  of Service}
\item
  \href{https://help.nytimes3xbfgragh.onion/hc/en-us/articles/115014893968-Terms-of-sale}{Terms
  of Sale}
\item
  \href{https://spiderbites.nytimes3xbfgragh.onion}{Site Map}
\item
  \href{https://help.nytimes3xbfgragh.onion/hc/en-us}{Help}
\item
  \href{https://www.nytimes3xbfgragh.onion/subscription?campaignId=37WXW}{Subscriptions}
\end{itemize}
