Sections

SEARCH

\protect\hyperlink{site-content}{Skip to
content}\protect\hyperlink{site-index}{Skip to site index}

\href{/section/travel}{Travel}\textbar{}A Nostalgic (if Isolating) Road
Trip Along Route 66

\url{https://nyti.ms/2ZdMJsZ}

\begin{itemize}
\item
\item
\item
\item
\item
\item
\end{itemize}

\href{https://www.nytimes3xbfgragh.onion/spotlight/at-home?action=click\&pgtype=Article\&state=default\&region=TOP_BANNER\&context=at_home_menu}{At
Home}

\begin{itemize}
\tightlist
\item
  \href{https://www.nytimes3xbfgragh.onion/2020/09/07/travel/route-66.html?action=click\&pgtype=Article\&state=default\&region=TOP_BANNER\&context=at_home_menu}{Cruise
  Along: Route 66}
\item
  \href{https://www.nytimes3xbfgragh.onion/2020/09/04/dining/sheet-pan-chicken.html?action=click\&pgtype=Article\&state=default\&region=TOP_BANNER\&context=at_home_menu}{Roast:
  Chicken With Plums}
\item
  \href{https://www.nytimes3xbfgragh.onion/2020/09/04/arts/television/dark-shadows-stream.html?action=click\&pgtype=Article\&state=default\&region=TOP_BANNER\&context=at_home_menu}{Watch:
  Dark Shadows}
\item
  \href{https://www.nytimes3xbfgragh.onion/interactive/2020/at-home/even-more-reporters-editors-diaries-lists-recommendations.html?action=click\&pgtype=Article\&state=default\&region=TOP_BANNER\&context=at_home_menu}{Explore:
  Reporters' Google Docs}
\end{itemize}

\includegraphics{https://static01.graylady3jvrrxbe.onion/images/2020/09/08/travel/07travel-highway-12-print/07travel-highway-12-print-articleLarge-v2.jpg?quality=75\&auto=webp\&disable=upscale}

The World Through a Lens

\hypertarget{a-nostalgic-if-isolating-road-trip-along-route-66}{%
\section{A Nostalgic (if Isolating) Road Trip Along Route
66}\label{a-nostalgic-if-isolating-road-trip-along-route-66}}

Driven by an early fascination with the American West, a photographer
set off to travel part of America's most celebrated highway.

The Globetrotter Lodge in Holbrook, Ariz.Credit...

Supported by

\protect\hyperlink{after-sponsor}{Continue reading the main story}

Photographs and Text by Luke Sharrett

\begin{itemize}
\item
  Sept. 7, 2020
\item
  \begin{itemize}
  \item
  \item
  \item
  \item
  \item
  \item
  \end{itemize}
\end{itemize}

\emph{At the onset of the coronavirus pandemic, with travel restrictions
in place worldwide, we launched a new series ---}
\href{https://www.nytimes3xbfgragh.onion/column/the-world-through-a-lens}{\emph{The
World Through a Lens}} \emph{--- in which photojournalists help
transport you, virtually, to some of our planet's most beautiful and
intriguing places. This week, Luke Sharrett shares a collection of
images taken along Route 66.}

\begin{center}\rule{0.5\linewidth}{\linethickness}\end{center}

Growing up in suburban Virginia, I only experienced the romance of the
American West on the occasional family vacation, or on Boy Scout camping
excursions. But what I felt on those trips left long-lasting
impressions. The big sky stretching out over endless prairies made me
feel minuscule, even as a beefy teenager. The enchanting rock formations
and rusty windmills seemed to transport me back in time to the days when
the Western United States was (in my imagination, at least) still wild
and untamed.

In April I embarked on a cross-country train trip to document the Amtrak
passengers who were
\href{https://www.nytimes3xbfgragh.onion/2020/05/06/opinion/amtrak-trains-coronavirus.html}{still
traveling by rail during the pandemic}. But as I zipped through northern
New Mexico and Arizona, I sat in the observation car longing to be
conveyed via a different mode of transportation, one that harkened back
to my childhood: I wanted the freedom to spend a few days cruising along
Route 66.

\includegraphics{https://static01.graylady3jvrrxbe.onion/images/2020/09/07/travel/07travel-highway-07/07travel-highway-07-articleLarge-v2.jpg?quality=75\&auto=webp\&disable=upscale}

Image

In Holbrook, Ariz., a tire shop occupies a building that once housed a
Kentucky Fried Chicken.

A few weeks later, I gave in. I flew to Albuquerque, booked the cheapest
rental vehicle I could find and headed west toward the Mother Road, as
John Steinbeck called it --- or what was left of it, anyway.

Image

Neon signage outside the Historic Route 66 Motel in Seligman, Ariz.

Once stretching more than 2,400 miles from Chicago to Santa Monica,
Calif., Route 66 has long existed as a testament to the American love
affair with the automobile. During the highway's golden era, local
economies --- including gas stations, mom-and-pop cafes, motor lodges,
drive-in restaurants, movie theaters and roadside oddities --- thrived
on the money brought in by a seemingly endless stream of motorists.

Then came the interstate.

Image

Little remains of the Tonto Drive-In Theater in Winslow, Ariz.

The construction of Interstate 40 --- a faster, if less colorful,
highway --- marked the beginning of the end for Route 66, much of whose
western portion was paralleled or overlaid by the new road. Dozens of
once-vibrant communities in northern New Mexico and Arizona were
permanently bypassed in favor of I-40's long, straight path through the
desert.

Yet the memory of Route 66, which was formally decommissioned by the
federal government in 1985, lives on in many of these forgotten
communities.

Image

An eastbound Burlington Northern Santa Fe train rockets along the
Southern Transcon Line in Laguna Pueblo, N.M.

On the Historic Route 66 west of Albuquerque, in Gallup, N.M., vintage
signs advertise an array of car dealerships, and a statue of a Navajo
code talker stands outside the city's train station. The station, near
the Navajo Nation, served as the debarkation point for some 400 Navajo
men who enlisted in the United States Marine Corps as radio operators
during World War II, their language confounding Japanese soldiers who,
up until that point, had successfully intercepted the communications of
U.S. forces in the Pacific.

Image

Though no longer serviced by passenger trains, the Atchison, Topeka and
Santa Fe Railway Depot still stands along the Southern Transcon in
Holbrook, Ariz.

As the highway approaches the Arizona border, signs appear for roadside
Native American gift shops. Jewelry, rugs and buffalo jerky all tempt
passing motorists to pull over and spend their money inside the walls of
the Yellow Horse Trading Post, situated just across the state line in
Lupton, Ariz. A bit farther west stands the remains of Fort Courage. The
once impressive frontier-themed rest area is now home to little more
than an abandoned pancake house and a long-defunct Taco Bell.

Image

The Yellow Horse Trading Post stands along Interstate 40 near the Navajo
Indian Reservation in Lupton, Ariz.

Another hour's drive to the west brings motorists to Holbrook, Ariz.,
where intrepid (and weary) travelers might be enticed by the city's
Wigwam Motel, advertised by a buzzing neon sign. Fifteen 28-foot-tall
concrete teepees encircle the property in a U-shaped formation. Classic
cars in various states of rust and decay sit parked around a gravel
parking lot, their permanent presence lending the motel an almost regal
ambience, even on the most vacant of nights.

Image

The Wigwam Motel in Holbrook, Ariz.

Image

Classic cars lend the motel an almost regal ambience, even when the
motel sits largely vacant.

Thirty miles farther down Route 66 from Holbrook stands Winslow, Ariz.
Banners hang forlornly across the town's main drag, requesting that
residents spend their money in the community's tiny economy. ``Please,''
they bid in stark letters, ``shop local.''

Image

A sign advertises the Navajo Indian Reservation along Interstate 40 in
Mentmore, N.M.

Nearby a pair of retired Santa Fe cabooses sit on display in a small
railroad park. Behind them, trains come and go from the bustling
Burlington Northern Santa Fe rail yard and crew-change point. Not far
from the cabooses stands a formidable wooden totem pole. It towers above
the flat, sandy terrain in recognition of the region's Native American
residents.

Image

Two retired Santa Fe cupola cabooses sit on display in First Street Park
in Winslow, Ariz.

Image

A wooden totem pole in Winslow, Ariz.

Little remains of the original Route 66 between Winslow and Flagstaff.
Instead, the four-lane 75-m.p.h. interstate plows through the desert
with ruthless efficiency. Casinos and souvenir shops dot the sprawling
landscape. Every so often the crumbling shell of an old service station
appears on the horizon.

Image

Signs for car dealerships and tire shops in Gallup, N.M.

At Twin Arrows, the graffiti-covered ruins of a former trading post
still remain. Two earth-struck, larger-than-life arrows beckon motorists
to stop in for a selfie among the cannibalized gasoline pumps and
ever-accumulating mountain of tumbleweeds.

Image

The vandalized and cannibalized remains of the Twin Arrows Trading Post
in Twin Arrows, Ariz.

Image

A pair of earth-struck, larger-than-life arrows beckon motorists to stop
for selfies.

As I cruised down these struggling main streets, I tried to imagine what
they must have looked like during Route 66's heyday, when gleaming
porcelain signs directed American-made sedans toward shiny roadside
motels. The irony of the moment wasn't lost on me: Here I was, obsessing
about the past, when the imaginations of most people in the atomic age
were fixated on the wonders of the future.

Aside from my socially distanced contact with an occasional front-desk
clerk or drive-through cashier, the trip proved to be just as isolating
--- if not more so --- than life at home in Kentucky. Throughout the
spring, I'd become accustomed to reading on my front porch as neighbors
walked by with dogs or strollers. Out here in the desert, there was
little evidence of passers-by other than the distant hum of big rigs on
I-40.

Image

Classic cars sit in the parking lot of the Wigwam Motel in Holbrook,
Ariz.

\href{https://www.lukesphoto.com/}{\emph{Luke Sharrett}} \emph{is a
photographer based in Louisville, Ky. You can follow his work on}
\href{https://www.instagram.com/howdyluke/}{\emph{Instagram}}\emph{.}

\emph{\textbf{Follow New York Times Travel}} \emph{on}
\href{https://www.instagram.com/nytimestravel/}{\emph{Instagram}}\emph{,}
\href{https://twitter.com/nytimestravel}{\emph{Twitter}} \emph{and}
\href{https://www.facebookcorewwwi.onion/nytimestravel/}{\emph{Facebook}}\emph{.
And}
\href{https://www.nytimes3xbfgragh.onion/newsletters/traveldispatch}{\emph{sign
up for our weekly Travel Dispatch newsletter}} \emph{to receive expert
tips on traveling smarter and inspiration for your next vacation.}

Advertisement

\protect\hyperlink{after-bottom}{Continue reading the main story}

\hypertarget{site-index}{%
\subsection{Site Index}\label{site-index}}

\hypertarget{site-information-navigation}{%
\subsection{Site Information
Navigation}\label{site-information-navigation}}

\begin{itemize}
\tightlist
\item
  \href{https://help.nytimes3xbfgragh.onion/hc/en-us/articles/115014792127-Copyright-notice}{©~2020~The
  New York Times Company}
\end{itemize}

\begin{itemize}
\tightlist
\item
  \href{https://www.nytco.com/}{NYTCo}
\item
  \href{https://help.nytimes3xbfgragh.onion/hc/en-us/articles/115015385887-Contact-Us}{Contact
  Us}
\item
  \href{https://www.nytco.com/careers/}{Work with us}
\item
  \href{https://nytmediakit.com/}{Advertise}
\item
  \href{http://www.tbrandstudio.com/}{T Brand Studio}
\item
  \href{https://www.nytimes3xbfgragh.onion/privacy/cookie-policy\#how-do-i-manage-trackers}{Your
  Ad Choices}
\item
  \href{https://www.nytimes3xbfgragh.onion/privacy}{Privacy}
\item
  \href{https://help.nytimes3xbfgragh.onion/hc/en-us/articles/115014893428-Terms-of-service}{Terms
  of Service}
\item
  \href{https://help.nytimes3xbfgragh.onion/hc/en-us/articles/115014893968-Terms-of-sale}{Terms
  of Sale}
\item
  \href{https://spiderbites.nytimes3xbfgragh.onion}{Site Map}
\item
  \href{https://help.nytimes3xbfgragh.onion/hc/en-us}{Help}
\item
  \href{https://www.nytimes3xbfgragh.onion/subscription?campaignId=37WXW}{Subscriptions}
\end{itemize}
