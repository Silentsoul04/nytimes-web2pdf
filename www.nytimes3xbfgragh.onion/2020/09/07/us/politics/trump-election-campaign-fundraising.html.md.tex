Sections

SEARCH

\protect\hyperlink{site-content}{Skip to
content}\protect\hyperlink{site-index}{Skip to site index}

\href{/section/politics}{Politics}\textbar{}How Trump's Billion-Dollar
Campaign Lost Its Cash Advantage

\url{https://nyti.ms/2R06yQ6}

\begin{itemize}
\item
\item
\item
\item
\item
\item
\end{itemize}

\begin{itemize}
\item
  \href{https://www.nytimes3xbfgragh.onion/live/2020/09/11/us/trump-vs-biden?action=click\&pgtype=Article\&state=default\&region=TOP_BANNER\&context=storylines_menu}{Election
  Updates}
\item
  \href{https://www.nytimes3xbfgragh.onion/interactive/2020/us/elections/election-states-biden-trump.html?action=click\&pgtype=Article\&state=default\&region=TOP_BANNER\&context=storylines_menu}{Paths
  to 270}
\item
  \href{https://www.nytimes3xbfgragh.onion/interactive/2019/us/elections/2020-presidential-election-calendar.html?action=click\&pgtype=Article\&state=default\&region=TOP_BANNER\&context=storylines_menu}{Key
  Dates}
\item
  \href{https://www.nytimes3xbfgragh.onion/interactive/2020/08/31/us/politics/vote-by-mail-deadlines.html?action=click\&pgtype=Article\&state=default\&region=TOP_BANNER\&context=storylines_menu}{Voting
  by Mail}
\item
  \href{https://www.nytimes3xbfgragh.onion/newsletters/politics?action=click\&pgtype=Article\&state=default\&region=TOP_BANNER\&context=storylines_menu}{Politics
  Newsletter}
\end{itemize}

\includegraphics{https://static01.graylady3jvrrxbe.onion/images/2020/09/04/us/politics/00trump-spending00/merlin_176546619_b3fcc086-e036-4cda-98de-7ef93a88e0e6-articleLarge.jpg?quality=75\&auto=webp\&disable=upscale}

\hypertarget{how-trumps-billion-dollar-campaign-lost-its-cash-advantage}{%
\section{How Trump's Billion-Dollar Campaign Lost Its Cash
Advantage}\label{how-trumps-billion-dollar-campaign-lost-its-cash-advantage}}

Five months ago, President Trump's re-election campaign had a huge
financial edge over Joseph R. Biden Jr.'s. The Times conducted an
extensive review of how the Trump team spent lavishly to show how that
advantage evaporated.

A rally for President Trump in Latrobe, Pa., last week. Mr. Trump's
campaign has spent \$800 million.Credit...Anna Moneymaker for The New
York Times

Supported by

\protect\hyperlink{after-sponsor}{Continue reading the main story}

\href{https://www.nytimes3xbfgragh.onion/by/shane-goldmacher}{\includegraphics{https://static01.graylady3jvrrxbe.onion/images/2018/07/27/multimedia/author-shane-goldmacher/author-shane-goldmacher-thumbLarge.png}}\href{https://www.nytimes3xbfgragh.onion/by/maggie-haberman}{\includegraphics{https://static01.graylady3jvrrxbe.onion/images/2018/07/12/multimedia/author-maggie-haberman/author-maggie-haberman-thumbLarge.png}}

By \href{https://www.nytimes3xbfgragh.onion/by/shane-goldmacher}{Shane
Goldmacher} and
\href{https://www.nytimes3xbfgragh.onion/by/maggie-haberman}{Maggie
Haberman}

\begin{itemize}
\item
  Sept. 7, 2020
\item
  \begin{itemize}
  \item
  \item
  \item
  \item
  \item
  \item
  \end{itemize}
\end{itemize}

Money was supposed to have been one of the great advantages of
incumbency for
\href{https://www.nytimes3xbfgragh.onion/interactive/2020/us/elections/donald-trump.html}{President
Trump}, much as it was for President Barack Obama in 2012 and George W.
Bush in 2004. After getting outspent in 2016, Mr. Trump filed for
re-election on the day of his inauguration --- earlier than any other
modern president --- betting that the head start would deliver him a
decisive financial advantage this year.

It seemed to have worked. His rival,
\href{https://www.nytimes3xbfgragh.onion/interactive/2020/us/elections/joe-biden.html}{Joseph
R. Biden Jr.}, was
\href{https://www.nytimes3xbfgragh.onion/2020/04/21/us/politics/biden-2020-fundraising.html}{relatively
broke} when he emerged as the presumptive Democratic nominee this
spring, and Mr. Trump and the Republican National Committee had a nearly
\$200 million cash advantage.

Five months later, Mr. Trump's financial supremacy has evaporated. Of
the \$1.1 billon his campaign and the party raised from the beginning of
2019 through July, more than \$800 million has already been spent. Now
some people inside the campaign are forecasting what was once
unthinkable: a cash crunch with less than 60 days until the election,
according to Republican officials briefed on the matter.

Brad Parscale, the former campaign manager, liked to call Mr. Trump's
re-election war machine an ``unstoppable juggernaut.'' But interviews
with more than a dozen current and former campaign aides and Trump
allies, and a review of thousands of items in federal campaign filings,
show that the president's campaign and the R.N.C. developed some
profligate habits as they burned through hundreds of millions of
dollars. Since Bill Stepien replaced Mr. Parscale in July, the campaign
has imposed a series of belt-tightening measures that have reshaped
initiatives, including hiring practices, travel and the advertising
budget.

\includegraphics{https://static01.graylady3jvrrxbe.onion/images/2020/09/04/us/politics/00trump-spending2/merlin_176315865_6ff55683-bcd3-4199-b306-4bd00db039f4-articleLarge.jpg?quality=75\&auto=webp\&disable=upscale}

Under Mr. Parscale, more than \$350 million --- almost half of the \$800
million spent --- went to fund-raising operations, as no expense was
spared in finding new donors online. The campaign assembled a big and
well-paid staff and housed the team at a cavernous, well-appointed
office in the Virginia suburbs; outsize legal bills were treated as
campaign costs; and more than \$100 million was spent on a television
advertising blitz before the party convention, the point when most of
the electorate historically begins to pay close attention to the race.

Among the splashiest and perhaps most questionable purchases was a pair
of Super Bowl ads the campaign reserved for \$11 million, according to
Advertising Analytics --- more than it has spent on TV in some top
battleground states. It was a vanity splurge that allowed Mr. Trump to
match the billionaire Michael R. Bloomberg's buy for the big game.

There was also a cascade of smaller choices that added up: The campaign
hired a coterie of highly paid consultants (Mr. Trump's former bodyguard
and White House aide has been paid more than \$500,000 by the R.N.C.
since late 2017); spent \$156,000 for planes to pull aerial banners in
recent months; and paid nearly \$110,000 to Yondr, a company that makes
magnetic pouches used to store cellphones during fund-raisers so that
donors could not secretly record Mr. Trump and leak his remarks.

Some people familiar with the budget noted that Mr. Parscale had a car
and driver, an unusual expense for a campaign manager. Mr. Trump has
told people gleefully that Mr. Stepien took a pay cut when the president
gave him the job.

Critics of the campaign's management say the lavish spending was
ineffective: Mr. Trump enters the fall trailing in most national and
swing state polls, and Mr. Biden has surpassed him as a fund-raising
powerhouse, after posting a record-setting haul of
\href{https://www.nytimes3xbfgragh.onion/live/2020/09/02/us/trump-vs-biden\#biden-shatters-fund-raising-records-with-a-364-5-million-haul-in-august}{nearly
\$365 million in August}. The
\href{https://www.nytimes3xbfgragh.onion/2020/09/10/us/politics/trump-campaign-virus-woodward.html}{Trump
campaign} has not revealed its August fund-raising figure.

``If you spend \$800 million and you're 10 points behind, I think you've
got to answer the question `What was the game plan?''' said Ed Rollins,
a veteran Republican strategist who runs a small pro-Trump super PAC,
and who accused Mr. Parscale of spending ``like a drunken sailor.''

Image

Brad Parscale, Mr. Trump's former campaign manager, at a fund-raising
event in San Antonio last year.Credit...Erin Schaff/The New York Times

``I think a lot of money was spent when voters weren't paying
attention,'' he added.

Mr. Parscale, who is still a senior adviser on the campaign, said in an
interview that the Trump operation invested heavily in attracting donors
to erase the large advantage that Democrats had built digitally after
the Obama years. ``We closed that gap,'' he said, crediting early
spending as ``the only reason Republicans are even close'' in terms of
online fund-raising.

``I ran the campaign the same way I did in 2016, which also included all
of the marketing, strategy and expenses under the very close eye of the
family,'' said Mr. Parscale, who was the digital director, not the
campaign manager, in 2016. ``No decision was made without their
approval.''

Mr. Trump's son-in-law, Jared Kushner, who has overseen the campaign
from his position as a senior White House aide, had posed
for\href{https://www.forbes.com/sites/stevenbertoni/2016/11/22/exclusive-interview-how-jared-kushner-won-trump-the-white-house/\#2fce288a3af6}{a
Forbes magazine cover} as the person who ran the 2016 campaign soon
after the election.

``Any spending arrangements with the R.N.C. since 2016 were in
partnership with Ronna McDaniel,'' Mr. Parscale said, referring to the
party chairwoman, ``who I consider a strategic partner and friend.'' Mr.
Parscale
\href{https://twitter.com/parscale/status/1303088755943919618}{said on
Twitter}that the campaign spent less than \$11 million on the Super Bowl
ads, after moving one of them to the post-game portion of the telecast.

Nicholas Everhart, a Republican strategist who owns a firm specializing
in placing political ads, said the \$800 million spent so far shows the
``peril of starting a re-election campaign just weeks after winning.''

``A presidential campaign costs a lot of money to run,'' Mr. Everhart
said. ``In essence, the campaign has been spending nonstop for almost
four years straight.''

\hypertarget{reining-in-the-budget}{%
\subsection{Reining in the budget}\label{reining-in-the-budget}}

At the top of the whiteboard in Mr. Stepien's office are the latest
numbers on the campaign budget, and Mr. Stepien has instituted a number
of changes since he was promoted from deputy campaign manager. A
proposal to spend \$50 million in costs related to coalitions groups was
cast aside. An idea to spend \$3 million for a NASCAR car bearing Mr.
Trump's name was discarded.

The number of staff members allowed to travel to events has been pared
back to avoid what one senior campaign official described as
``sponsoring vacations.''

Trips aboard Air Force One, popular because they enable aides to get
face time with the president --- but which have to be compensated by the
campaign --- have been slashed.

``The most important thing I do every day is pay attention to the
budget,'' Mr. Stepien said in a brief interview. He declined to discuss
budget specifics but said the campaign had enough funds to win.

Most visibly, the Trump campaign slashed its August television spending,
mostly abandoning the airwaves during the party conventions. In the last
two weeks of the month, Mr. Biden's campaign spent \$35.9 million on
television, compared with \$4.8 million for Mr. Trump, according to
Advertising Analytics.

Image

Joseph R. Biden Jr., in Pittsburgh last week. His campaign raised nearly
\$365 million in August.Credit...Amr Alfiky/The New York Times

``We held on to cash to make sure that we'll have the firepower that we
need'' for the fall, said Jason Miller, a senior Trump strategist, who
contended that airing ads during the conventions would prove a waste for
Mr. Biden. ``We want to make sure that we're saving it for when it
really matters, when it's going to move the needle.''

Mr. Miller defended spending money on television ads earlier this spring
and summer, calling it a ``tough'' decision necessary to keep Mr. Trump
competitive as the nation suffered through a pandemic and its economic
fallout.

``We had to claw our way back,'' he said.

One of the reasons Mr. Biden was able to wipe away Mr. Trump's early
cash edge was that he sharply contained costs with a minimalistic
campaign during the pandemic's worst months. Trump officials derisively
dismissed it as his ``basement'' strategy, but from that basement Mr.
Biden fully embraced Zoom fund-raisers, with top donors asked to give as
much as \$720,000.

These virtual events typically took less than 90 minutes of the
candidate's time, could raise millions of dollars and cost almost
nothing. Mr. Trump has almost entirely refused to hold such
fund-raisers. Aides say he doesn't like them.

\hypertarget{door-knocking-to-win-over-voters}{%
\subsection{Door-knocking to win over
voters}\label{door-knocking-to-win-over-voters}}

There is some disagreement in the extended Trump operation about the
depth of any potential cash-on-hand shortage. Some officials believe
that plenty more money will come in during the last two months from
online donors and that cutting back on TV advertising in August was
shortsighted. The campaign announced a combined \$76 million haul with
the party during the four days of its convention.

Others said the campaign had expected the low-dollar fund-raising to
continue at the same pace, and were also counting on a significant
number of \$5,600 checks, the limit for direct campaign giving, that
didn't materialize; that was in part because they rely on in-person
events, which was more difficult with the virus.

Some party officials defended the early **** spending as prudent,
including money devoted to the expansive ground operation and an online
network of donors that was setting fund-raising records. The G.O.P. has
more than 2,000 staff members across 100 offices and claims that
volunteers knock on one million doors per week; the Biden campaign has
forgone door-knocking so far during the pandemic.

``The Biden campaign is hoarding money and hoping that fall TV ads help
put them over the edge,'' said Richard Walters, chief of staff for the
R.N.C. ``But when a state comes down to 10,700 votes like Michigan did
in 2016, we think that direct voter contact --- those millions of door
knocks and phone calls we make each week --- is going to be critical.''

The Trump campaign has undertaken its own financial review of spending
under Mr. Parscale. Among the first changes implemented was shutting
down an ad campaign that
\href{https://www.nytimes3xbfgragh.onion/2020/06/30/us/politics/brad-parscale-trump.html}{used
Mr. Parscale's personal social media accounts to deliver pro-Trump ads}.
More than \$800,000 had been poured into boosting Mr. Parscale's
Facebook and Instagram pages; those ads ceased the day after he was
removed as campaign manager.

Mr. Parscale said the Facebook page was ``not my idea'' and the
``family's direct approval'' had been sought on the program.

``I built an unprecedented infrastructure with the Republican Party
under this family's leadership since 2016,'' Mr. Parscale said in a
statement to The Times. ``I am proud of my achievements.''

\hypertarget{some-trump-pleasing-expenditures}{%
\subsection{Some Trump-pleasing
expenditures}\label{some-trump-pleasing-expenditures}}

Some spending choices appear devised, at least in part, to satisfy Mr.
Trump himself, including the Super Bowl ads, which were purchased as
part of an advertising arms race with Mr. Bloomberg. The two ads on game
day cost more than the Trump campaign spent on local television through
the end of July in each of four battleground states: Wisconsin (\$3.9
million), Michigan (\$3.6 million), Iowa (\$2 million) and Minnesota
(\$1.3 million).

Another Trump-pleasing expense: more than \$1 million in ads aired in
the Washington, D.C., media market, a region that is not likely to be
competitive in the fall but where the president, a famously voracious
television consumer, resides.

Image

President Trump hosted Brazilian President Jair Bolsonaro at a dinner at
Mar-a-Lago in March.Credit...T.J. Kirkpatrick for The New York Times

Mr. Trump, who once joked he could be the first candidate to make money
running for president, has steered, along with the Republican Party,
about \$4 million into the Trump family businesses since 2019: hundreds
of thousands of dollars to Mr. Trump's club at Mar-a-Lago in Florida,
lavish donor retreats at Trump hotels, office space in Trump Tower, and
thousands of dollars at the steakhouse in Mr. Trump's Washington, D.C.,
hotel.

Many of the specifics of Mr. Trump's spending are opaque; since 2017,
the campaign and the R.N.C. have routed \$227 million through a single
limited liability company linked to Trump campaign officials. That firm,
American Made Media Consultants, has been used to place television and
digital ads and was the subject of a recent
\href{https://www.courthousenews.com/wp-content/uploads/2020/07/FEC-trump-complaint.pdf}{Federal
Election Commission complaint} arguing it was used to disguise the final
destination of spending, which
\href{https://www.nytimes3xbfgragh.onion/2020/03/09/us/trump-campaign-brad-parscale.html}{has
included paychecks to Lara Trump and Kimberly Guilfoyle}, the partners
of Mr. Trump's two adult sons.

Millions more followed to firms tied to R.N.C. and Trump-linked
officials, including more than \$39 million to two firms, Parscale
Strategy LLC and Giles-Parscale, controlled by Mr. Parscale since the
beginning of 2017. ****

Mr. Parscale said that he had ``no ownership or financial interest in
A.M.M.C.'' and that he had ``negotiated a contract with the family for 1
percent of digital ad spend and after becoming campaign manager took no
percentage.''

\hypertarget{you-have-to-spend-money-to-make-money}{%
\subsection{`You have to spend money to make
money'}\label{you-have-to-spend-money-to-make-money}}

There is little question that Mr. Parscale helped the Trump campaign
construct an unparalleled Republican operation to lure small donors
online. He directed a nine-figure investment in digital ads and
list-building that appears to have largely paid for itself. Some of the
president's advisers believe it will continue to pay great dividends in
the final weeks, pointing to the \$165 million raised by the president
and his party in July --- more than any month in 2016.

``You have to spend money to make money,'' explained Mr. Walters, the
R.N.C. chief of staff. ``We have had a big increase in revenue because
of early investments we made in online fund-raising and direct mail.''

Still, the costs of the G.O.P. money operation have been enormous.

Since 2019, Mr. Trump, the R.N.C. and their shared committees have spent
\$145 million on costs related to direct mail, almost \$42 million on
digital list acquisition and rentals (to expand their list of email
addresses) and tens of millions more in online advertising for new
donors.

Image

Supporters listened to Mr. Trump speak in Phoenix.Credit...Doug
Mills/The New York Times

Just procuring the Trump paraphernalia that supporters buy costs a lot.
Two firms that make campaign swag were paid more than \$30 million
combined since 2019.

At Mr. Trump's direction, the party has taken a spare-no-expense
approach to donor maintenance, with the R.N.C. spending more than \$6
million in ``donor mementos.'' The spending has gone to stationery
shops, the White House Historical Association (\$538,000) and the
Hershey Company, the chocolate-maker (\$337,000), which cover costs for
items such as the White House-branded candies given away by
administrations of both parties.

Mr. Trump has also accumulated many costs that are unusual for a
presidential re-election.

Republicans, for instance, have been saddled with
\href{https://www.nytimes3xbfgragh.onion/2020/09/05/us/politics/trump-campaign-funds-legal-bills.html}{extra
legal costs}, more than \$21 million since 2019, resulting from
investigations into Mr. Trump and, eventually, his impeachment trial.
The R.N.C. also paid a large legal bill of \$666,666.67 to Reuters News
\& Media at the end of June. Both Reuters and the R.N.C. declined to
discuss the payment. It was labeled ``legal proceedings --- IP
resolution,'' suggesting it was related to a potential litigation over
intellectual property.

There have been other squandered costs driven by Mr. Trump's sometimes
mercurial desires. He switched his convention plans twice, incurring
many expenses along the way. In July, for instance, the R.N.C. made a
\$325,000 payment to the Ritz-Carlton Amelia Island near Jacksonville
for the convention that never happened there. The party is not expected
to get that money back.

Rachel Shorey contributed reporting.

\hypertarget{our-2020-election-guide}{%
\section{Our 2020 Election Guide}\label{our-2020-election-guide}}

Updated ~Sept. 11, 2020

\begin{center}\rule{0.5\linewidth}{\linethickness}\end{center}

\begin{itemize}
\item ~
  \hypertarget{the-latest}{%
  \subsection{The Latest}\label{the-latest}}

  \begin{itemize}
  \item
    Joe Biden and President Trump put
    \href{https://www.nytimes3xbfgragh.onion/2020/09/11/us/politics/shanksville-trump-biden.html?action=click\&pgtype=Article\&state=default\&region=BELOW_MAIN_CONTENT\&context=storylines_guide}{hostilities
    on hold today to travel to ground zero and then to Shanksville, Pa.,
    where they separately honored 9/11 victims}.
  \end{itemize}
\item ~
  \hypertarget{how-to-win-270}{%
  \subsection{How to Win 270}\label{how-to-win-270}}

  \begin{itemize}
  \item
    Joe Biden and Donald Trump need 270 electoral votes to reach the
    White House. Try building
    \href{https://www.nytimes3xbfgragh.onion/interactive/2020/us/elections/election-states-biden-trump.html?action=click\&pgtype=Article\&state=default\&region=BELOW_MAIN_CONTENT\&context=storylines_guide}{your
    own coalition of battleground states}~to see potential outcomes.
  \end{itemize}
\item ~
  \hypertarget{voting-by-mail}{%
  \subsection{Voting by Mail}\label{voting-by-mail}}

  \begin{itemize}
  \item
    Will you have enough time to vote by mail in your state? Yes, but
    it's risky to procrastinate.
    \href{https://www.nytimes3xbfgragh.onion/interactive/2020/08/31/us/politics/vote-by-mail-deadlines.html?action=click\&pgtype=Article\&state=default\&region=BELOW_MAIN_CONTENT\&context=storylines_guide}{Check
    your state's deadline.}
  \item
    \href{https://www.nytimes3xbfgragh.onion/interactive/2020/us/elections/joe-biden.html?action=click\&pgtype=Article\&state=default\&region=BELOW_MAIN_CONTENT\&context=storylines_guide}{}

    \hypertarget{joe-biden}{%
    \section{Joe Biden}\label{joe-biden}}

    \hypertarget{democrat}{%
    \subsection{Democrat}\label{democrat}}

    \href{https://www.nytimes3xbfgragh.onion/interactive/2020/us/elections/donald-trump.html?action=click\&pgtype=Article\&state=default\&region=BELOW_MAIN_CONTENT\&context=storylines_guide}{}

    \hypertarget{donald-trump}{%
    \section{Donald Trump}\label{donald-trump}}

    \hypertarget{republican}{%
    \subsection{Republican}\label{republican}}
  \end{itemize}
\item
  \hypertarget{keep-up-with-our-coverage}{%
  \subsection{Keep Up With Our
  Coverage}\label{keep-up-with-our-coverage}}

  \begin{itemize}
  \item
    Get an
    \href{https://www.nytimes3xbfgragh.onion/newsletters/politics?action=click\&pgtype=Article\&state=default\&region=BELOW_MAIN_CONTENT\&context=storylines_guide}{email}~recapping
    the day's news
  \item
    Download our mobile app on
    \href{https://apps.apple.com/us/app/nytimes/id284862083?ls=1\&mat_click_id=5c79ae7455014fd1bd66b5610c05b8f2-20191112-16948\&referrer=mat_click_id\%3D5c79ae7455014fd1bd66b5610c05b8f2-20191112-16948\%26link_click_id\%3D722930677036718082}{iOS}~and
    \href{http://a.localytics.com/android?id=com.nytimes.android\&referrer=utm_source\%3Dother_nyt_mobile_web\%26utm_medium\%3DWeb\%2520page\%26utm_term\%3DGeneral\%2520Mobile\%2520Page\%26utm_campaign\%3DNYT\%2520Mobile\%2520General\%2520Page}{Android}~and
    turn on Breaking News and Politics alerts
  \end{itemize}
\end{itemize}

Advertisement

\protect\hyperlink{after-bottom}{Continue reading the main story}

\hypertarget{site-index}{%
\subsection{Site Index}\label{site-index}}

\hypertarget{site-information-navigation}{%
\subsection{Site Information
Navigation}\label{site-information-navigation}}

\begin{itemize}
\tightlist
\item
  \href{https://help.nytimes3xbfgragh.onion/hc/en-us/articles/115014792127-Copyright-notice}{©~2020~The
  New York Times Company}
\end{itemize}

\begin{itemize}
\tightlist
\item
  \href{https://www.nytco.com/}{NYTCo}
\item
  \href{https://help.nytimes3xbfgragh.onion/hc/en-us/articles/115015385887-Contact-Us}{Contact
  Us}
\item
  \href{https://www.nytco.com/careers/}{Work with us}
\item
  \href{https://nytmediakit.com/}{Advertise}
\item
  \href{http://www.tbrandstudio.com/}{T Brand Studio}
\item
  \href{https://www.nytimes3xbfgragh.onion/privacy/cookie-policy\#how-do-i-manage-trackers}{Your
  Ad Choices}
\item
  \href{https://www.nytimes3xbfgragh.onion/privacy}{Privacy}
\item
  \href{https://help.nytimes3xbfgragh.onion/hc/en-us/articles/115014893428-Terms-of-service}{Terms
  of Service}
\item
  \href{https://help.nytimes3xbfgragh.onion/hc/en-us/articles/115014893968-Terms-of-sale}{Terms
  of Sale}
\item
  \href{https://spiderbites.nytimes3xbfgragh.onion}{Site Map}
\item
  \href{https://help.nytimes3xbfgragh.onion/hc/en-us}{Help}
\item
  \href{https://www.nytimes3xbfgragh.onion/subscription?campaignId=37WXW}{Subscriptions}
\end{itemize}
