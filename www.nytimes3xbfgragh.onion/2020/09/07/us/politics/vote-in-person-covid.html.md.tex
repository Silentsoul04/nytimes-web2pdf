Sections

SEARCH

\protect\hyperlink{site-content}{Skip to
content}\protect\hyperlink{site-index}{Skip to site index}

\href{https://www.nytimes3xbfgragh.onion/section/politics}{Politics}

\href{https://myaccount.nytimes3xbfgragh.onion/auth/login?response_type=cookie\&client_id=vi}{}

\href{https://www.nytimes3xbfgragh.onion/section/todayspaper}{Today's
Paper}

\href{/section/politics}{Politics}\textbar{}In Year of Voting by Mail, a
Scramble to Beef Up In-Person Voting, Too

\url{https://nyti.ms/3i5jBvH}

\begin{itemize}
\item
\item
\item
\item
\item
\item
\end{itemize}

\begin{itemize}
\item
  \href{https://www.nytimes3xbfgragh.onion/live/2020/09/08/us/trump-vs-biden?action=click\&pgtype=Article\&state=default\&region=TOP_BANNER\&context=storylines_menu}{Election
  Updates}
\item
  \href{https://www.nytimes3xbfgragh.onion/interactive/2020/us/elections/election-states-biden-trump.html?action=click\&pgtype=Article\&state=default\&region=TOP_BANNER\&context=storylines_menu}{Paths
  to 270}
\item
  \href{https://www.nytimes3xbfgragh.onion/interactive/2020/08/31/us/politics/vote-by-mail-deadlines.html?action=click\&pgtype=Article\&state=default\&region=TOP_BANNER\&context=storylines_menu}{Voting
  by Mail}
\item
  \href{https://www.nytimes3xbfgragh.onion/interactive/2019/us/elections/2020-presidential-election-calendar.html?action=click\&pgtype=Article\&state=default\&region=TOP_BANNER\&context=storylines_menu}{Key
  Dates}
\item
  \href{https://www.nytimes3xbfgragh.onion/newsletters/politics?action=click\&pgtype=Article\&state=default\&region=TOP_BANNER\&context=storylines_menu}{Politics
  Newsletter}
\end{itemize}

Advertisement

\protect\hyperlink{after-top}{Continue reading the main story}

Supported by

\protect\hyperlink{after-sponsor}{Continue reading the main story}

\hypertarget{in-year-of-voting-by-mail-a-scramble-to-beef-up-in-person-voting-too}{%
\section{In Year of Voting by Mail, a Scramble to Beef Up In-Person
Voting,
Too}\label{in-year-of-voting-by-mail-a-scramble-to-beef-up-in-person-voting-too}}

Communities are running short on time to hire poll workers and
reconfigure in-person voting to make it safe during a pandemic.

\includegraphics{https://static01.graylady3jvrrxbe.onion/images/2020/09/04/us/politics/00inperson1/merlin_176457843_43a6511e-2a57-4d25-b6ff-59956959ca11-articleLarge.jpg?quality=75\&auto=webp\&disable=upscale}

By \href{https://www.nytimes3xbfgragh.onion/by/nick-corasaniti}{Nick
Corasaniti} and
\href{https://www.nytimes3xbfgragh.onion/by/michael-wines}{Michael
Wines}

\begin{itemize}
\item
  Sept. 7, 2020
\item
  \begin{itemize}
  \item
  \item
  \item
  \item
  \item
  \item
  \end{itemize}
\end{itemize}

PHILADELPHIA --- Unnerved by the difficulties of voting amid a pandemic
and faced with both the political static injected by President Trump and
the limits on expanding voting by mail, state and local authorities
across the country are racing to rethink and reinforce the polling sites
where tens of millions of people are still expected to cast their
ballots.

For all of the attention on voting by mail, perhaps four in 10 votes ---
60 million ballots --- are likely to be cast in person this fall, either
early or on Election Day. Overall turnout could well reach 150 million
for the first time, up from 137.5 million in 2016, according to Barry C.
Burden, the director of the
\href{https://www.elections.wisc.edu}{Elections Research Center} at the
University of Wisconsin-Madison.

Against the backdrop of Mr. Trump's relentless criticism of voting by
mail, the breakdowns at the Postal Service and the relatively high rate
of rejections of mailed-in ballots, election officials and activists in
both parties are amping up efforts to hire and train poll workers;
integrate stadiums, arenas and malls into their voting options; and come
up with contingency plans if there's a surge in coronavirus cases in the
fall.

A major area of concern is finding younger people who are able to
replace older ones most susceptible to the ravages of Covid-19 at a time
when 58 percent of the nation's poll workers are 61 or older.

``Everyone's focusing on the rate of voting by mail, which is going to
easily double what it was in 2016 --- somewhere north of 80 million
ballots,'' said \href{https://blogs.reed.edu/paul-gronke/}{Paul Gronke},
an expert on in-person voting at Reed College in Portland, Ore. ``But
people aren't paying attention to what might happen if there's a spike
in the pandemic or a shortage of poll workers and there's a last-minute
reduction in in-person voting.''

``In some of our minds, the nightmare scenario isn't about voting by
mail,'' he said. ``It's a meltdown at the polling places.''

It's not clear whether many voters are rethinking plans to vote by mail,
although early research on the issue confirms that a disproportionate
number of Democrats plans to vote by mail, and many Republicans are
following Mr. Trump's cue and refusing that option. There is a clear
need to ensure that in-person voting works, particularly in many cities
with large Black and brown populations.

In Cuyahoga County in Ohio, which includes Cleveland and is roughly 29
percent Black and 66 percent white, white neighborhoods had a 30 percent
absentee ballot request rate in the 2020 primary, whereas Black
neighborhoods had only a 15 percent request rate, and Hispanic
neighborhoods had a 17 percent request rate,
\href{https://www.demos.org/research/how-ohio-continued-silence-black-and-brown-voters-vote-mail-election}{according
to a recent study by Demos}, a liberal think tank.

Shoring up in-person voting has focused on both poll workers and polling
places.

The stuffy church basements and senior-living centers that were once
reliable voting sites are now unusable. The tolerance for long lines has
shrunk drastically. Where administrators used to fret about the
occasional equipment breakdowns and ballot shortages, they must worry
now about backup plans if a coronavirus outbreak shutters a polling site
or sidelines the poll workers who have staffed it.

\includegraphics{https://static01.graylady3jvrrxbe.onion/images/2020/03/26/us/00inperson5/merlin_170594097_baff4196-6c87-4f71-9b8b-c21778089ae1-articleLarge.jpg?quality=75\&auto=webp\&disable=upscale}

Mr. Gronke recalled a tabletop exercise this summer about possible
election disaster scenarios. One official from a tiny jurisdiction asked
for advice should one of the four election workers in its single office
contract the coronavirus a week before the election, forcing the
remaining three into quarantine.

``That just stopped the conversation,'' he said.

Some locales are making do with relatively minor adjustments. In
Hennepin County, Minn., home to Minneapolis and 800,000 registered
voters, officials have closed a few cramped voting sites and
consolidated them with precincts where the polling place is more
spacious. Recruitment of poll workers has gone well, and a trial run of
the new plans during a local primary last month was judged a success,
said Ginny Gelms, the Hennepin County elections manager.

``This election will be difficult,'' she said, ``but we're feeling
pretty good right now.''

Maricopa County, Ariz., one of the nation's largest voting
jurisdictions, which includes Phoenix, has effectively redrawn its
election system to address health concerns.

A review after the primary election in March concluded that many of the
500 polling places were too small to safely accommodate voters, Megan
Gilbertson, a spokeswoman for the county elections department, said in
an interview. So for the November election, polling has been moved to as
many as 175 voting centers, in places like shopping malls and convention
facilities.

Eighty of the sites will be open for a 27-day early voting period that
will feature expanded evening and weekend hours. Both then and on
November 3, voters will be able to cast a ballot at any of the sites.

The consolidation carries an extra benefit: It allowed the county to
slash its 3,600-person corps of poll workers in half --- to 1,800 ---
even while expanding the number of check-in stations at each polling
place.

Maricopa was able to use federal grant money from the coronavirus
stimulus program for its transformation. But major election changes cost
money that most jurisdictions don't have.

Image

Isaiah Thomas, a Philadelphia council member, speaking as city officials
and attendees gather for National Poll Worker Recruitment Day in
Philadelphia.Credit...Mark Makela for The New York Times

A central goal nationwide is to prevent the drastic consolidation of
polling locations that plagued some of the biggest cities in the country
during the primaries and led to hourslong lines. Milwaukee had just five
polling locations from 180 in April, and Philadelphia consolidated to
just 200 sites from 830 in June. While some locations were moved because
they were near sites with populations vulnerable to the coronavirus,
such as a nursing home, the largest factor in consolidating polling
stations has been a shortage of poll workers.

\href{https://www.nytimes3xbfgragh.onion/news-event/2020-election}{Election
2020 ›}

\hypertarget{live-updates}{%
\subsection{\texorpdfstring{\href{https://www.nytimes3xbfgragh.onion/live/2020/09/08/us/trump-vs-biden}{Live
Updates}}{Live Updates}}\label{live-updates}}

\href{https://www.nytimes3xbfgragh.onion/live/2020/09/08/us/trump-vs-biden\#obama-gives-harris-the-inside-scoop-on-biden-ice-cream-is-big}{}

Sept. 8, 2020, 11:28 a.m. ET

\href{https://www.nytimes3xbfgragh.onion/live/2020/09/08/us/trump-vs-biden\#obama-gives-harris-the-inside-scoop-on-biden-ice-cream-is-big}{Obama
gives Harris the inside scoop on Biden: `Ice cream is
big.'}\href{https://www.nytimes3xbfgragh.onion/live/2020/09/08/us/trump-vs-biden\#kathryn-garcia-nycs-sanitation-commissioner-resigns-to-mull-a-run-for-mayor}{}

Sept. 8, 2020, 11:10 a.m. ET

\href{https://www.nytimes3xbfgragh.onion/live/2020/09/08/us/trump-vs-biden\#kathryn-garcia-nycs-sanitation-commissioner-resigns-to-mull-a-run-for-mayor}{Kathryn
Garcia, N.Y.C.'s sanitation commissioner, resigns to mull a run for
mayor.}\href{https://www.nytimes3xbfgragh.onion/live/2020/09/08/us/trump-vs-biden\#a-top-house-democrat-calls-for-the-suspension-of-postmaster-general-louis-dejoy-over-campaign-finance-allegations}{}

Sept. 8, 2020, 10:00 a.m. ET

\href{https://www.nytimes3xbfgragh.onion/live/2020/09/08/us/trump-vs-biden\#a-top-house-democrat-calls-for-the-suspension-of-postmaster-general-louis-dejoy-over-campaign-finance-allegations}{A
top House Democrat calls for the suspension of Postmaster General Louis
DeJoy over campaign finance allegations.}

Given health fears for older people, government officials and
nonprofits, including
\href{https://www.nytimes3xbfgragh.onion/2020/08/24/us/politics/lebron-james-poll-workers.html?searchResultPosition=9}{More
Than a Vote,}the collective of athletes headlined by LeBron James, have
started national campaigns to recruit younger citizens to be poll
workers.

Philadelphia is mounting a 60-day sprint to find roughly 3,000 more poll
workers between now and Election Day --- on top of the more than 4000
already hired. To sweeten the pot, the city announced last week that it
would be raising the pay to as much as \$250 per poll worker for
Election Day.

At an event outside Philadelphia's City Hall last week, the message was
clear: Save the 2020 election. Become a poll worker.

``So, today, I'm calling on millennials --- 18 to 35, or 40 --- to step
up, take this baton and represent democracy,'' said Abu Edwards, a
co-founder of Millennials in Action, one of roughly a dozen speakers who
called for younger voters to work the polls. ``We are in the fourth
quarter of this election, and poll workers, you are our referees.''

Some saw more practical reasons to help out.

``With the pandemic, we're all unemployed, so, why not?'' said Lee
Jones, 20, who signed up after the speeches.

Other areas have had more luck, thanks to an outpouring of public
support after reports of election debacles this spring. Spurred by the
chaos in Georgia's primary, the Metro Atlanta Chamber
\href{https://gapollworker.com}{started a website} last month to recruit
poll workers, and has netted 1,500 volunteers from 50 of the state's 159
counties. Fulton County, home to Atlanta, needs 2,000 volunteers --- and
now has a list of more than 8,000, said Katie Kirkpatrick, the Chamber's
president.

In Buncombe County, N.C., where early voting started on Friday, the
county has already filled 500 of its 700 poll-worker slots and has a
long list of applicants for the remainder. County government employees
are being allowed to work at the polls instead of their regular jobs,
and the state legislature has voted to exempt pay for poll workers from
figuring in unemployment benefit calculations.

Image

Lee Jones, 20, signed up to be a polling station volunteer for the first
time.Credit...Mark Makela for The New York Times

Businesses are playing an increasing role. The clothing chain Old Navy
said this week that it would give a paid day off to any of its 50,000
employees who work at the polls.
\href{https://apnews.com/cbb2ff495dcf09875ce87b3a53dcda52}{Hundreds of
companies,} from the Coca-Cola Company to Mailchimp to Patagonia, have
made Election Day a paid holiday or given employees time off to vote or
to perform election work, and 23 states require companies to grant time
off to vote.

Professional sports teams in Detroit, Houston, Los Angeles, Pittsburgh
and elsewhere have offered up their arenas for voting and other election
purposes; their size making for a much safer location amid coronavirus
than a middle school basement.

In Detroit, all four major teams offered up their facilities, and each
will be used differently. Comerica Park, the home of the Detroit Tigers,
will serve as a ballot drop off center. Little Ceasars Arena, where the
Pistons and Red Wings play, will be used to safely train poll workers.
Ford Field, home of the Lions, will become a centralized hub where all
precincts will deliver their ballots for processing and auditing.

Election experts have welcomed the addition of arenas, but only if they
serve to increase the options for voters, not replace them.

``What's really important is that the arenas are an addition, not an
instead of,'' said Sylvia Albert, the director of voting and elections
at Common Cause, a nonprofit watchdog group based in Washington. ``We
can't close down local polling locations in neighborhoods, because
people need to be able to vote in their communities.''

Image

Trana Loglisci, 49, signed up to be a polling station volunteer for the
first time.Credit...Mark Makela for The New York Times

At the event in Philadelphia, tables were set up to recruit poll
workers. One woman, Trana Loglisci, 49, approached a table before the
event started and signed up. She said she recalled how difficult it was
in June to vote in the primary and wanted to get involved.

``I like to try to make a difference somehow,'' she said.

\hypertarget{our-2020-election-guide}{%
\section{Our 2020 Election Guide}\label{our-2020-election-guide}}

Updated ~Sept. 8, 2020

\begin{center}\rule{0.5\linewidth}{\linethickness}\end{center}

\begin{itemize}
\item ~
  \hypertarget{the-latest}{%
  \subsection{The Latest}\label{the-latest}}

  \begin{itemize}
  \item
    The campaign
    \href{https://www.nytimes3xbfgragh.onion/live/2020/09/08/us/trump-vs-biden?action=click\&pgtype=Article\&state=default\&region=BELOW_MAIN_CONTENT\&context=storylines_guide}{shifts
    to a higher gear this week}, with President Trump set to visit
    Florida and North Carolina today and Joseph R. Biden heading to
    Michigan tomorrow.
  \end{itemize}
\item ~
  \hypertarget{how-to-win-270}{%
  \subsection{How to Win 270}\label{how-to-win-270}}

  \begin{itemize}
  \item
    Joe Biden and Donald Trump need 270 electoral votes to reach the
    White House. Try building
    \href{https://www.nytimes3xbfgragh.onion/interactive/2020/us/elections/election-states-biden-trump.html?action=click\&pgtype=Article\&state=default\&region=BELOW_MAIN_CONTENT\&context=storylines_guide}{your
    own coalition of battleground states}~to see potential outcomes.
  \end{itemize}
\item ~
  \hypertarget{voting-by-mail}{%
  \subsection{Voting by Mail}\label{voting-by-mail}}

  \begin{itemize}
  \item
    Will you have enough time to vote by mail in your state? Yes, but
    it's risky to procrastinate.
    \href{https://www.nytimes3xbfgragh.onion/interactive/2020/08/31/us/politics/vote-by-mail-deadlines.html?action=click\&pgtype=Article\&state=default\&region=BELOW_MAIN_CONTENT\&context=storylines_guide}{Check
    your state's deadline.}
  \item
    \href{https://www.nytimes3xbfgragh.onion/interactive/2020/us/elections/joe-biden.html?action=click\&pgtype=Article\&state=default\&region=BELOW_MAIN_CONTENT\&context=storylines_guide}{}

    \hypertarget{joe-biden}{%
    \section{Joe Biden}\label{joe-biden}}

    \hypertarget{democrat}{%
    \subsection{Democrat}\label{democrat}}

    \href{https://www.nytimes3xbfgragh.onion/interactive/2020/us/elections/donald-trump.html?action=click\&pgtype=Article\&state=default\&region=BELOW_MAIN_CONTENT\&context=storylines_guide}{}

    \hypertarget{donald-trump}{%
    \section{Donald Trump}\label{donald-trump}}

    \hypertarget{republican}{%
    \subsection{Republican}\label{republican}}
  \end{itemize}
\item
  \hypertarget{keep-up-with-our-coverage}{%
  \subsection{Keep Up With Our
  Coverage}\label{keep-up-with-our-coverage}}

  \begin{itemize}
  \item
    Get an
    \href{https://www.nytimes3xbfgragh.onion/newsletters/politics?action=click\&pgtype=Article\&state=default\&region=BELOW_MAIN_CONTENT\&context=storylines_guide}{email}~recapping
    the day's news
  \item
    Download our mobile app on
    \href{https://apps.apple.com/us/app/nytimes/id284862083?ls=1\&mat_click_id=5c79ae7455014fd1bd66b5610c05b8f2-20191112-16948\&referrer=mat_click_id\%3D5c79ae7455014fd1bd66b5610c05b8f2-20191112-16948\%26link_click_id\%3D722930677036718082}{iOS}~and
    \href{http://a.localytics.com/android?id=com.nytimes.android\&referrer=utm_source\%3Dother_nyt_mobile_web\%26utm_medium\%3DWeb\%2520page\%26utm_term\%3DGeneral\%2520Mobile\%2520Page\%26utm_campaign\%3DNYT\%2520Mobile\%2520General\%2520Page}{Android}~and
    turn on Breaking News and Politics alerts
  \end{itemize}
\end{itemize}

Advertisement

\protect\hyperlink{after-bottom}{Continue reading the main story}

\hypertarget{site-index}{%
\subsection{Site Index}\label{site-index}}

\hypertarget{site-information-navigation}{%
\subsection{Site Information
Navigation}\label{site-information-navigation}}

\begin{itemize}
\tightlist
\item
  \href{https://help.nytimes3xbfgragh.onion/hc/en-us/articles/115014792127-Copyright-notice}{©~2020~The
  New York Times Company}
\end{itemize}

\begin{itemize}
\tightlist
\item
  \href{https://www.nytco.com/}{NYTCo}
\item
  \href{https://help.nytimes3xbfgragh.onion/hc/en-us/articles/115015385887-Contact-Us}{Contact
  Us}
\item
  \href{https://www.nytco.com/careers/}{Work with us}
\item
  \href{https://nytmediakit.com/}{Advertise}
\item
  \href{http://www.tbrandstudio.com/}{T Brand Studio}
\item
  \href{https://www.nytimes3xbfgragh.onion/privacy/cookie-policy\#how-do-i-manage-trackers}{Your
  Ad Choices}
\item
  \href{https://www.nytimes3xbfgragh.onion/privacy}{Privacy}
\item
  \href{https://help.nytimes3xbfgragh.onion/hc/en-us/articles/115014893428-Terms-of-service}{Terms
  of Service}
\item
  \href{https://help.nytimes3xbfgragh.onion/hc/en-us/articles/115014893968-Terms-of-sale}{Terms
  of Sale}
\item
  \href{https://spiderbites.nytimes3xbfgragh.onion}{Site Map}
\item
  \href{https://help.nytimes3xbfgragh.onion/hc/en-us}{Help}
\item
  \href{https://www.nytimes3xbfgragh.onion/subscription?campaignId=37WXW}{Subscriptions}
\end{itemize}
