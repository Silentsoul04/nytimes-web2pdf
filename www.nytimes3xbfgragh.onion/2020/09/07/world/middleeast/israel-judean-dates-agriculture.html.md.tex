Sections

SEARCH

\protect\hyperlink{site-content}{Skip to
content}\protect\hyperlink{site-index}{Skip to site index}

\href{https://www.nytimes3xbfgragh.onion/section/world/middleeast}{Middle
East}

\href{https://myaccount.nytimes3xbfgragh.onion/auth/login?response_type=cookie\&client_id=vi}{}

\href{https://www.nytimes3xbfgragh.onion/section/todayspaper}{Today's
Paper}

\href{/section/world/middleeast}{Middle East}\textbar{}Aided by Modern
Ingenuity, a Taste of Ancient Judean Dates

\url{https://nyti.ms/3i7a1s8}

\begin{itemize}
\item
\item
\item
\item
\item
\end{itemize}

Advertisement

\protect\hyperlink{after-top}{Continue reading the main story}

Supported by

\protect\hyperlink{after-sponsor}{Continue reading the main story}

Israel Dispatch

\hypertarget{aided-by-modern-ingenuity-a-taste-of-ancient-judean-dates}{%
\section{Aided by Modern Ingenuity, a Taste of Ancient Judean
Dates}\label{aided-by-modern-ingenuity-a-taste-of-ancient-judean-dates}}

The harvest of the much-extolled but long-lost Judean dates was
something of a scientific miracle. The fruit sprouted from seeds 2,000
years old.

\includegraphics{https://static01.graylady3jvrrxbe.onion/images/2020/09/06/world/06israel-date-dispatch/merlin_176514033_28188a63-cc1a-4821-a4b9-18fb88a09191-articleLarge.jpg?quality=75\&auto=webp\&disable=upscale}

\href{https://www.nytimes3xbfgragh.onion/by/isabel-kershner}{\includegraphics{https://static01.graylady3jvrrxbe.onion/images/2018/10/12/multimedia/author-isabel-kershner/author-isabel-kershner-thumbLarge.png}}

By \href{https://www.nytimes3xbfgragh.onion/by/isabel-kershner}{Isabel
Kershner}

\begin{itemize}
\item
  Sept. 7, 2020
\item
  \begin{itemize}
  \item
  \item
  \item
  \item
  \item
  \end{itemize}
\end{itemize}

KETURA, Israel --- The plump, golden-brown dates hanging in a bunch just
above the sandy soil were finally ready to pick.

They had been slowly ripening in the desert heat for months. But the
young tree on which they grew had a much more ancient history ---
sprouting from a 2,000-year-old seed retrieved from an archaeological
site in the Judean wilderness.

``They are beautiful!'' exclaimed Dr. Sarah Sallon with the elation of a
new mother, as each date, its skin slightly wrinkled, was plucked gently
off its stem at a sunbaked kibbutz in southern Israel.

They were tasty, too, with a fresh flavor that gave no hint of their
two-millenium incubation period. The honey-blonde, semi-dry flesh had a
fibrous, chewy texture and a subtle sweetness.

These were the much-extolled but long-lost Judean dates, and the harvest
this month was hailed as a modern miracle of science.

Dr. Sallon, who researches natural medicine, had joined up with Elaine
Solowey, an expert on arid agriculture, to find and germinate the
ancient seeds. This harvesting of the fruit, celebrated in a small
ceremony earlier this month at Kibbutz Ketura, was the culmination of
their 15-year quest.

``In these troubled times of climate change, pollution and species dying
out at alarming rates, to bring something back to life from dormancy is
so symbolic,'' Dr. Sallon said. ``To pollinate and produce these
incredible dates is like a beam of light in a dark time.''

\includegraphics{https://static01.graylady3jvrrxbe.onion/images/2020/09/06/world/06israel-date-dispatch2/06israel-date-dispatch2-articleLarge.jpg?quality=75\&auto=webp\&disable=upscale}

Date palms were praised in the Bible and the Quran, and became symbols
of beauty, precious shade and succulent plenty. In antiquity, the Judean
palms, prized for their quality, appeared as motifs in synagogues.

A Roman coin minted around A.D. 70 to celebrate the conquest of Judea
depicted the Jewish defeat as a woman weeping under a date palm.

But by the Middle Ages, the famed Judean plantations had died out. Wars
and upheaval likely made their cultivation impractical, as did their
need for copious amounts of water in summer.

So Dr. Sallon went on a hunt.

A pediatric gastroenterologist who directs the Louis L. Borick Natural
Medicine Research Center at the Hadassah Hospital in Jerusalem, Dr.
Sallon was on a mission to revive old knowledge for use in modern
medicine. She had learned from a dusty archive in Jerusalem that dates
were not only good for digestion but were thought by traditional healers
to improve blood production and memory, and to have aphrodisiac
properties.

She obtained a few of the date seeds that had been found in the 1960s
during an excavation of Masada, the desert fortress by the Dead Sea
where Jewish zealots, besieged by the Romans in A.D. 73, famously died
by their own hand rather than fall into slavery.

She immediately turned to Dr. Solowey, who runs the Center for
Sustainable Agriculture at the Arava Institute for Environmental Studies
in Kibbutz Ketura.

The institute, established in 1996 after the Israeli-Palestinian Oslo
peace accords, is dedicated to advancing cross-border environmental
cooperation in the face of political conflict, and offers academic
programs to Jordanians, Palestinians and Israelis as well as
international students.

Dr. Solowey planted the seeds in quarantined pots in January 2005, not
expecting much, but nevertheless employing a few ``horticultural
tricks,'' she said, to try to coax them out of their long slumber,
involving warming, careful hydration, a plant hormone and enzymatic
fertilizer.

Image

Elaine Solowey, an expert on arid agriculture, worked with Dr. Sallon to
grow the dates.Credit...Dan Balilty for The New York Times

Weeks later, she said, she was ``utterly astonished'' to see
\href{https://www.nytimes3xbfgragh.onion/2005/06/12/world/middleeast/after-2000-years-a-seed-from-ancient-judea-sprouts.html?searchResultPosition=1}{the
earth had cracked and a tiny shoot had emerged}. Named Methuselah after
the biblical patriarch known for his longevity, that shoot has since
grown into a sturdy tree outside her office.

But Methuselah turned out to be a male, and male palm trees are not good
for much on their own. (Gender can be confirmed once the trees flower or
by genetic testing.)

So Dr. Sallon went searching again and chose more than 30 seeds from
another stash from archaeological sites in the Judean desert, including
Qumran, where the Dead Sea Scrolls were found. Planted at Ketura between
2011 and 2014, six of the seeds sprouted.

They were given the names of biblical figures when they germinated, but
as their genders became clear over time, Judah became Judith, Eve became
Adam, and Jeremiah became Hannah.

Hannah's seed, which came from an ancient burial cave in Wadi el-Makkukh
near Jericho, now in the West Bank, was carbon dated to between the
first and fourth centuries B.C.E., becoming one of the oldest known
seeds to have ever been germinated.

The research was peer reviewed and detailed in
\href{https://advances.sciencemag.org/content/6/6/eaax0384}{a paper}
published in February this year in Science Advances, a leading
scientific journal.

A month later, there was another surprise. After growing for six years,
Hannah flowered in a nearby plot. Now, it was time to play matchmaker.
Dr. Solowey painstakingly collected pollen from Methuselah and brushed
it onto Hannah's flowers, ``because I wanted Methuselah to be the
father,'' she said.

Image

The proud father Methuselah, grown from ancient seeds, at Kibbutz Ketura
in the~ Arava desert, Israel.~Credit...Dan Balilty for The New York
Times

The night before the picking of Hannah's dates, there was some
discussion of what the proper Hebrew blessing would be at the ceremony
--- the usual one for the fruit of the tree or the ``shehecheyanu,'' a
blessing of thanks for new and unusual experiences.

The next morning, both were recited, to a resounding Amen.

Hannah's fruit most reminded connoisseurs of the zahidi, an Iraqi
variety known for its mildly sweet and nutty flavor.

Genetic experts from the University of Montpellier in France said the
genotyping for the germinated plants indicated that the older seeds,
including Methuselah and Hannah, were closer to eastern varieties that
flourished from Mesopotamia to Arabia and all the way to Pakistan. Date
palm cultivation is thought to be up to 6,500 years old.

The younger the seeds, the more they resembled the varieties that
flourished west of Egypt, like the moist, treacly sweet Moroccan medjoul
date that is popular today, and is commercially cultivated in
plantations along the Jordan Rift Valley, including at Ketura.

It all made perfect sense to Dr. Sallon.

Image

The dates had a fresh flavor that gave no hint of their extraordinary
incubation period.Credit...Dan Balilty for The New York Times

Ancient Judea was ideally placed between North Africa and Asia, along
major trade routes, and the Romans, who traded all over the
Mediterranean, could have brought western varieties with them to
pollinate the older varieties from the east.

``Putting it simply, what do we find?'' Dr. Sallon said. ``The story of
ancient Israel and the Jewish people, of diasporas, trade routes and
commerce throughout the Middle East.''

After the dates were harvested, there was little chance to savor the
moment in the ensuing flurry of activity. Minutes after the picking and
tasting, the dates were whisked away to be measured and weighed. About a
dozen of the hundred or so from the bunch were individually wrapped in
aluminum foil, packed on ice and sent to the Ministry of Agriculture's
research institute.

Even the pips of those that had been eaten were collected for further
study.

Aside from Dr. Sallon's interest in their medicinal properties, there
was some banter among the institute staff about mass producing the
old-new fruit, with an eye to marketing the fruit as ``the dates that
Jesus ate,'' and using the funds for research.

``Lucky, it tasted good,'' Dr. Solowey said. ``If it had been awful what
would I have said? That in the old days they didn't know what a good
date was? There's a lot of literature about how they were the best dates
in the world.''

Advertisement

\protect\hyperlink{after-bottom}{Continue reading the main story}

\hypertarget{site-index}{%
\subsection{Site Index}\label{site-index}}

\hypertarget{site-information-navigation}{%
\subsection{Site Information
Navigation}\label{site-information-navigation}}

\begin{itemize}
\tightlist
\item
  \href{https://help.nytimes3xbfgragh.onion/hc/en-us/articles/115014792127-Copyright-notice}{©~2020~The
  New York Times Company}
\end{itemize}

\begin{itemize}
\tightlist
\item
  \href{https://www.nytco.com/}{NYTCo}
\item
  \href{https://help.nytimes3xbfgragh.onion/hc/en-us/articles/115015385887-Contact-Us}{Contact
  Us}
\item
  \href{https://www.nytco.com/careers/}{Work with us}
\item
  \href{https://nytmediakit.com/}{Advertise}
\item
  \href{http://www.tbrandstudio.com/}{T Brand Studio}
\item
  \href{https://www.nytimes3xbfgragh.onion/privacy/cookie-policy\#how-do-i-manage-trackers}{Your
  Ad Choices}
\item
  \href{https://www.nytimes3xbfgragh.onion/privacy}{Privacy}
\item
  \href{https://help.nytimes3xbfgragh.onion/hc/en-us/articles/115014893428-Terms-of-service}{Terms
  of Service}
\item
  \href{https://help.nytimes3xbfgragh.onion/hc/en-us/articles/115014893968-Terms-of-sale}{Terms
  of Sale}
\item
  \href{https://spiderbites.nytimes3xbfgragh.onion}{Site Map}
\item
  \href{https://help.nytimes3xbfgragh.onion/hc/en-us}{Help}
\item
  \href{https://www.nytimes3xbfgragh.onion/subscription?campaignId=37WXW}{Subscriptions}
\end{itemize}
