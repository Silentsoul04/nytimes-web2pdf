Sections

SEARCH

\protect\hyperlink{site-content}{Skip to
content}\protect\hyperlink{site-index}{Skip to site index}

\href{https://www.nytimes3xbfgragh.onion/section/books/review}{Book
Review}

\href{https://myaccount.nytimes3xbfgragh.onion/auth/login?response_type=cookie\&client_id=vi}{}

\href{https://www.nytimes3xbfgragh.onion/section/todayspaper}{Today's
Paper}

\href{/section/books/review}{Book Review}\textbar{}Novels From Around
the World Examine Kinship and Conflict

\url{https://nyti.ms/35cpOSV}

\begin{itemize}
\item
\item
\item
\item
\item
\end{itemize}

Advertisement

\protect\hyperlink{after-top}{Continue reading the main story}

Supported by

\protect\hyperlink{after-sponsor}{Continue reading the main story}

The Shortlist

\hypertarget{novels-from-around-the-world-examine-kinship-and-conflict}{%
\section{Novels From Around the World Examine Kinship and
Conflict}\label{novels-from-around-the-world-examine-kinship-and-conflict}}

\includegraphics{https://static01.graylady3jvrrxbe.onion/images/2020/08/11/books/review/Shortlist_Tepper1/Shortlist_Tepper1-articleLarge.jpg?quality=75\&auto=webp\&disable=upscale}

By Anderson Tepper

\begin{itemize}
\item
  Sept. 8, 2020, 5:00 a.m. ET
\item
  \begin{itemize}
  \item
  \item
  \item
  \item
  \item
  \end{itemize}
\end{itemize}

\textbf{\textbf{TWENTY AFTER MIDNIGHT}}\\
By Daniel Galera\\
Translated by Julia Sanches\\
217 pp. Penguin. Paper, \$16.

It's a sweltering summer in Porto Alegre, Brazil, made worse by the
stench of uncollected garbage and the simmering frustrations of a bus
strike. The dashed optimism of last year's street protests, leading up
to the 2014 World Cup, still hangs over the city. For three former
college friends --- Emiliano, Antero and Aurora --- things only get
grimmer when they hear that Andrei Dukelsky has been gunned down in a
robbery. Dukelsky, ``one of the most promising new talents of
contemporary Brazilian literature,'' was the enigmatic ringleader who
brought them all together in the late 1990s to write for his pioneering
webzine, Orangutan*.*

So where does that leave the three of them now, as they reminisce in
middle age about those freewheeling days at the dawn of the internet
age? Emiliano, a hack journalist, wrestles with ``the feeling I'd had
since turning 40 that I was already in the process of decomposing.''
Antero, a trendy marketing guru, clings to appearances of youth, while
cheating on his wife and engaging in epic bouts of masturbation. Aurora,
the most cynical of the group, is an ecological ``doomsdayer'' who plans
to flee to the forest and listen to ``the distant echoes of
civilization's demise'' from there.

Galera is the author of several gritty novels set in contemporary Brazil
and one of Granta's Best Young Brazilian novelists; he writes with a
heady, voracious energy captured in robust prose by Sanches. Masterfully
picking away at the impotence and disappointments of this aging band of
literary provocateurs, Galera presents a vision of failed promise --- of
a generation and country --- that is as sordid and rotting as the
streets of Porto Alegre.

\textbf{\textbf{THAT TIME OF YEAR}}\\
By Marie NDiaye\\
Translated by Jordan Stump\\
136 pp. Two Lines Press. \$19.95.

\includegraphics{https://static01.graylady3jvrrxbe.onion/images/2020/08/11/books/review/Shortlist_Tepper2/Shortlist_Tepper2-articleLarge.jpg?quality=75\&auto=webp\&disable=upscale}

``That Time of Year,'' by the prolific, genre-defying French-Senegalese
author NDiaye, also takes place in summer --- or rather, the days right
after a summer holiday's end. The Parisians Herman and his wife, Rose,
have been vacationing in the same rural town in southern France for the
past 10 years, but when they decide to stay one day longer --- until
Sept. 1 --- everything takes a sinister twist.

What a difference a day makes! Soon, a cold rain and fog descend; then,
Rose and their 8-year-old son go missing. Herman wanders the streets in
search of his family, spied on suspiciously by the townspeople and
gripped by a growing panic. It's as if he's unwittingly stepped onto a
Jordan Peele set*.* ``Everything's turned hostile all of a sudden,'' he
mutters to himself. ``Is it because I've seen the fall, is this the
price you have to pay?''

Herman is given the runaround by vacant-looking, Kafkaesque officials in
matching outfits and unnaturally yellow hair, before moving into a local
hotel to better get the lay of the land. Meanwhile, he feels himself
metamorphosing: ``Like that morning, everything inside him seemed damp
and mortified, shrunken, slowly rotting.'' More than just losing his
mind, he seems to be physically melting away, succumbing to a ``dulled,
larval inertia.''

NDiaye's slim novel, first published in France in 1994 and translated by
Stump in a sly, deceptive monotone, is a worthy addition to her oeuvre.
For all its elements of pyschological horror, there is something
hauntingly real to NDiaye's world, where ``pale, serene, detached,
smiling faces hid an inconsolable sorrow.''

\textbf{\textbf{THE SWORD AND THE SPEAR}}\\
By Mia Couto\\
Translated by David Brookshaw\\
273 pp. Farrar, Straus \& Giroux. \$27.

Image

The Mozambican writer Couto, winner of the 2014 Neustadt International
Prize for Literature, has built up his own extraordinary body of work
exploring the wounds and dreams of his native land, in books such as
``Every Man Is a Race,'' ``A River Called Time,'' ``The Tuner of
Silences'' and ``Confessions of the Lioness.'' His work is every bit as
bewitching as the titles suggest.

``The Sword and the Spear,'' the second volume in a historical trilogy,
continues the saga of the final confrontations between Portuguese
colonial armies and Ngunganyane, king of the Gaza empire, in
late-19th-century Mozambique. Like ``Woman of Ashes,'' the first volume,
the new novel unfolds in a series of letters, mainly between Sgt.
Germano de Melo, a reluctant soldier besotted with a young VaChopi girl
named Imani, and the careerist Lt. Ayres de Ornelas. Their
correspondence is interspersed with the voice of Imani herself, and here
is where Couto's storytelling truly soars (poetically rendered from the
Portuguese by Brookshaw).

As the novel opens, Imani is accompanying the wounded de Melo down the
Inharrime River to a hospital. ``Everything always begins with a
farewell,'' she writes. ``This story begins with an ending: the end of
my adolescence.'' On the way, they hole up at a former Catholic mission,
where she reunites with the Goan priest who trained her ``to be white''
as a young girl at convent school. As in all his books, Couto calls into
question the very essence of race and identity, belief and belonging, in
Mozambique and beyond. ``So you've discovered you want to be African?''
the priest asks Imani. ``I'm curious, my daughter. What is it to be
African?''

\textbf{\textbf{HEAVEN AND EARTH}}\\
By Paolo Giordano\\
Translated by Anne Milano Appel\\
406 pp. Pamela Dorman/Viking. \$28.

Image

Credit...1

Teresa, the narrator of Giordano's bighearted novel ``Heaven and
Earth,'' is barely a teenager when she first encounters the three boys
who will alter the course of her life. Bern, Nicola and Tommaso live in
the \emph{masseria} --- part-farmhouse, part-religious order --- next
door to Teresa's grandmother's home in southern Italy, which Teresa
visits each summer from Turin. Bern, especially, makes a powerful
impression. Passionate and intense, he is eager to ``storm the
heavens.''

Like Galera, Giordano follows the trajectories of his characters over
the course of several decades, beginning in the mid-90s. But his book is
less about a generation's decline than about the tangle of emotions and
rivalries that play out between friends. Teresa and Bern are soon
inseparable, and, in their 20s, become the heart of a homesteading
collective at the \emph{masseria}.

There are obstacles to their dreams of a green utopia --- local
capitalists, for one, and the police, for another. Internal divisions,
too, drive a wedge between them, with tragic results. Ultimately, Bern's
passion is all-consuming. ``At that moment I felt the frightening
immensity of the love he had inside,'' Teresa says. ``It wasn't just
about the trees, it was about everything and everyone, and it didn't let
him breathe, it was suffocating him.''

Giordano is a fluid, expansive writer (smoothly translated by Appel):
The chapters flow effortlessly back and forth in time, pulling us deeper
into the story of Teresa and Bern's great love. The Italian landscape
shimmers with their longing. ``It all belonged to us,'' Bern says. ``The
trees and the stone walls. The heavens. Even the heavens belonged to us,
Teresa.''

Advertisement

\protect\hyperlink{after-bottom}{Continue reading the main story}

\hypertarget{site-index}{%
\subsection{Site Index}\label{site-index}}

\hypertarget{site-information-navigation}{%
\subsection{Site Information
Navigation}\label{site-information-navigation}}

\begin{itemize}
\tightlist
\item
  \href{https://help.nytimes3xbfgragh.onion/hc/en-us/articles/115014792127-Copyright-notice}{©~2020~The
  New York Times Company}
\end{itemize}

\begin{itemize}
\tightlist
\item
  \href{https://www.nytco.com/}{NYTCo}
\item
  \href{https://help.nytimes3xbfgragh.onion/hc/en-us/articles/115015385887-Contact-Us}{Contact
  Us}
\item
  \href{https://www.nytco.com/careers/}{Work with us}
\item
  \href{https://nytmediakit.com/}{Advertise}
\item
  \href{http://www.tbrandstudio.com/}{T Brand Studio}
\item
  \href{https://www.nytimes3xbfgragh.onion/privacy/cookie-policy\#how-do-i-manage-trackers}{Your
  Ad Choices}
\item
  \href{https://www.nytimes3xbfgragh.onion/privacy}{Privacy}
\item
  \href{https://help.nytimes3xbfgragh.onion/hc/en-us/articles/115014893428-Terms-of-service}{Terms
  of Service}
\item
  \href{https://help.nytimes3xbfgragh.onion/hc/en-us/articles/115014893968-Terms-of-sale}{Terms
  of Sale}
\item
  \href{https://spiderbites.nytimes3xbfgragh.onion}{Site Map}
\item
  \href{https://help.nytimes3xbfgragh.onion/hc/en-us}{Help}
\item
  \href{https://www.nytimes3xbfgragh.onion/subscription?campaignId=37WXW}{Subscriptions}
\end{itemize}
