\href{/section/business}{Business}\textbar{}The Two Men Buying Your
Favorite Retailers

\url{https://nyti.ms/3jWVqjr}

\begin{itemize}
\item
\item
\item
\item
\item
\item
\end{itemize}

\includegraphics{https://static01.graylady3jvrrxbe.onion/images/2020/09/01/business/01virus-retailowners-3/merlin_176318523_71f33406-5492-4e7f-9d4c-9938940f7b60-articleLarge.jpg?quality=75\&auto=webp\&disable=upscale}

Sections

\protect\hyperlink{site-content}{Skip to
content}\protect\hyperlink{site-index}{Skip to site index}

\hypertarget{the-two-men-buying-your-favorite-retailers}{%
\section{The Two Men Buying Your Favorite
Retailers}\label{the-two-men-buying-your-favorite-retailers}}

Jamie Salter and David Simon, one a licensing expert and the other a
mall operator, are reshaping the shopping landscape by acquiring
bankrupt brands like Brooks Brothers and Forever 21.

The closed Brooks Brothers store on Madison Avenue in New
York.Credit...Vincent Tullo for The New York Times

Supported by

\protect\hyperlink{after-sponsor}{Continue reading the main story}

By Sapna Maheshwari and Vanessa Friedman

\begin{itemize}
\item
  Sept. 8, 2020
\item
  \begin{itemize}
  \item
  \item
  \item
  \item
  \item
  \item
  \end{itemize}
\end{itemize}

It has been a prolonged period of retail carnage: storied names
\href{https://www.nytimes3xbfgragh.onion/2020/04/21/business/coronavirus-department-stores-neiman-marcus.html}{declaring
bankruptcy}, mass market brands closing thousands of stores, tens of
thousands of shop employees furloughed or laid off, garment workers in
dire straits. More ominous still are the predictions that we will never
shop the same way again.

For Jamie Salter and David Simon, however, it has been a time of great
opportunity.

Mr. Salter is the founder and chief executive of the Authentic Brands
Group, a company known for buying the intellectual property of famous
brands at discount prices and then striking licensing deals with other
companies that want to stick those well-known names on their products.
Mr. Simon is the chief executive of the Simon Property Group, the
largest mall operator in the United States with more than 100
properties. Together, they are reshaping the American retail landscape.

Last week, they closed a deal to buy the
\href{https://www.nytimes3xbfgragh.onion/2020/08/12/business/brooks-brothers-sale-authentic-brands.html?searchResultPosition=1}{bankrupt}
\href{https://www.nytimes3xbfgragh.onion/2020/07/08/business/brooks-brothers-chapter-11-bankruptcy.html?searchResultPosition=3}{Brooks
Brothers}, the 202-year-old American fashion brand and retailer,
\href{https://www.nytimes3xbfgragh.onion/2020/08/12/business/brooks-brothers-sale-authentic-brands.html}{for
\$325 million}. Last month, they
\href{https://abgnewsroom.com/home/authentic-brands-group-and-sparc-group-to-acquire-lucky-brand}{acquired}
Lucky Brand denim, and in February, they bought
\href{https://www.nytimes3xbfgragh.onion/2019/10/23/business/forever-21-bankruptcy-chang-family.html?searchResultPosition=8}{Forever
21}.

Together, the acquisitions will bring the global revenue generated by
the company's brands --- a sprawling mix that includes Sports
Illustrated and rights tied to Marilyn Monroe's likeness --- to \$15
billion annually. And Mr. Salter is hunting for more.

\includegraphics{https://static01.graylady3jvrrxbe.onion/images/2020/08/31/business/00virrus-retailowners-alt/merlin_176409537_e4c0bd94-49d6-43e6-bf4a-5d65175ec12c-articleLarge.jpg?quality=75\&auto=webp\&disable=upscale}

``Look, if the world ends, which I don't think it's going to, then
there's no doubt about it, I'm not so smart,'' Mr. Salter, a 57-year-old
Toronto native, said in a phone interview. ``But I don't believe the
world's going to end.''

``Last year, we said within five years, we want to be at \$20 billion,''
he added, referring to the overall revenue generated from brands owned
or jointly owned by Authentic Brands. ``Another two to three deals could
get us there.''

Many of the acquisitions are being made through a joint venture with Mr.
Simon called SPARC, for Simon Properties Authentic Retail Concepts. Its
roots go back to 2016, but it was created in its present form in January
as a vehicle that turned out to be almost perfectly positioned to take
advantage of the current state of the industry.

By teaming up, Mr. Simon,
\href{https://www.buzzfeednews.com/article/sapna/mall-industry-hires-pr-firm-to-fight-death-of-the-mall-narra}{a
press-averse} Indianapolis real estate scion who declined to comment for
this article, gets assurance that bankrupt chains and other tenants will
remain in his shopping centers, while Mr. Salter gets a friendly
landlord for his brands at a time when rent costs are crushing
retailers, plus the chance to earn money by licensing the well-known
names. Together, they own and operate 1,500 stores through their
deals,~which sometimes include Brookfield Properties, another mall
giant.

The purchase of Brooks Brothers, where layoff notices have already
started going out, has put a spotlight on this arrangement --- and
invited new scrutiny. Supporters say SPARC is saving the businesses it's
buying. Critics say it's simply exploiting their traumas for fast
profits in ways that cheapen the brands' legacies. They say the SPARC
strategy treats brands and stores less like hothouses of creativity that
need careful tending, and more like chess pieces to be moved around for
maximum, if momentary, gain.

That suspicion has been hard to shake for Mr. Salter. Authentic Brands'
purchase of the
\href{https://www.nytimes3xbfgragh.onion/2019/10/03/business/media/sports-illustrated-layoffs.html}{Sports
Illustrated} brand last year is viewed as a prime example of the
company's bottom-line approach to licensing. It sold the rights to
operate the magazine and website to another company, which gutted the
staff, while simultaneously putting the Sports Illustrated name on
\href{https://www.amazon.com/Sports-Illustrated-Nutrition-Protein-Isolate/dp/B087BQSRZK}{protein}
powder, CBD cream and
\href{https://www.venus.com/products.aspx?BRANCH=24~7308~\&sc=FS75\&cm_mmc=SEMNB-_-Google-_-sports\%20illustrated\%20cover-_-Swim-Sports-Illustrated-nc\&gclid=CjwKCAjwyo36BRAXEiwA24CwGQdbioudyHMOgSGBt-IqmeRYAAiUCDWXq5kc6678hfWiG2OmrW07KRoCnXcQAvD_BwE}{swimsuits}.
And Authentic Brands' purchase of
\href{https://www.nytimes3xbfgragh.onion/2019/10/30/business/barneys-bankruptcy.html}{Barneys
New York}'s intellectual property last year was fiercely contested by a
group of investors who waged a ``Save Barneys'' social media campaign to
avert liquidations and the licensing of the name, painting Mr. Salter as
a villain who sought to dismantle a cultural institution.

Image

A group of investors waged a ``Save Barneys'' social media campaign last
fall, painting Mr. Salter as a villain who sought to dismantle a New
York cultural institution.Credit...Haruka Sakaguchi for The New York
Times

``It's not a long-term quality play,'' said one retail executive who
asked not to be identified because the executive had been approached
about the Brooks Brothers deal. ``It's not about a love of the brand or
the goods. It's predatory and opportunistic.''

Understanding Authentic Brands' business is crucial to understanding the
tides of retail today.

The company, founded by Mr. Salter in 2010, bets on famous names in
fashion and entertainment, often buying their intellectual property with
the aim of striking licensing deals with those who want to use the brand
names internationally or on new products. Authentic Brands tends to earn
an estimated 4 to 6 percent in royalties through this model.

``History,'' was one of the answers Mr. Salter gave when asked what he
looks for in a brand. ``Does it have good archives we can bring back,
because the world repeats itself all the time. The longer the history,
the better.'' The potential to cut costs was another.

For years, Mr. Salter led a division of Hilco, a financial firm, as it
snapped up the intellectual property of bankrupt retailers
\href{https://www.nytimes3xbfgragh.onion/2009/01/19/technology/companies/19sharper.html}{like
Sharper Image}. While the retailer's stores closed, Hilco was involved
with deals that put Sharper Image's name on products like garment
steamers that were cheaper than wares at the original retailer and then
sold in chains like Bed Bath \& Beyond.

At Authentic Brands, Mr. Salter pulled off an early coup by acquiring
the exclusive rights tied to Marilyn Monroe, whose likeness drew the
interest of everyone from Dolce \& Gabbana to Walmart. His stable of 50
brands now includes Juicy Couture, Elvis Presley, Muhammad Ali and
Frederick's of Hollywood.

The Juicy acquisition in 2013, where Mr. Salter bought the brand but
couldn't secure its locations, made him realize the value of physical
stores. Losing the stores, he said, hurt Juicy. ``I can tell you
unequivocally it's easier to build brands with a retail footprint ---
touch, feel, try on,'' he said.

Though Authentic Brands does not own the types of luxury retailers and
labels as European conglomerates like Kering and LVMH, Mr. Salter said
that LVMH served as ``inspiration'' and that they shared ``similar
ambitions.'' He thinks of his company, where his four sons are also
among the 200 employees (his eldest, Corey, is chief operating officer)
as a family enterprise despite a roster of investors including
BlackRock, Leonard Green \& Partners and General Atlantic. The biggest
individual investor after Mr. Salter, whose family owns about 20
percent, is Shaquille O'Neal, whose brand is managed by the Authentic
Brands. Mr. Salter said that he has considered an initial public
offering of stock but that the company has plenty of money and he
doesn't want to exit.

``Other people do want in,'' he said. But, he added, ``It's a lot easier
when you have two guys, and if there's a problem, you pick up the phone
and work it out in 10 minutes.''

Simon Property also holds about 7 percent after an investment in
January, when it also increased its interest in SPARC to 50 percent,
\href{https://investors.simon.com/node/24506/html}{according to}
filings.

Image

David Simon, the chief executive of the largest mall operator in the
United States, recently compared critics of his venture with Mr. Salter
to those who told Amazon to remain in the book business.Credit...Patrick
T. Fallon/Bloomberg

Four years ago, Mr. Salter said, ``David came to me and said, `Why do
you always close the stores when you buy the company?''' Mr. Salter
replied that he was too nervous to operate the stores, worrying that the
leases could become too expensive. Mr. Simon proposed teaming up with
Brookfield to buy Aéropostale, which led to the formation of a venture
called Aero OpCo. Mr. Salter owned 20 percent, and Brookfield and Simon
the rest. (Brookfield, which is not part of SPARC, declined to comment.)

The mall operators wanted their tenants to stay and ideally resume
making money. They were also interested in Mr. Salter's marketing
prowess and his brands, which they figured could eventually turn into
stores at their malls.

``At the beginning, Simon just wanted `get my rent,''' Mr. Salter said.
``But we started turning profits very quickly, and it started to be
about building a business.''

Each side benefits. Mr. Salter's brands have ``variable rent'' contracts
with Mr. Simon's malls, meaning their rent goes up and down with their
sales and, in a lucrative arrangement, most don't have minimums. Mr.
Simon also receives a percentage of royalties from sales associated with
the brand names. In January, Mr. Salter bought out Brookfield's interest
and the venture was renamed SPARC.

``Covid is a good lesson for all of us because thank God we had
percentage rent,'' Mr. Salter said. ``We furloughed whatever number we
had to furlough in Forever 21, and you're only paying rent on a
percentage of sales. It hurts a lot less.''

Image

The parking lot of the Simon Property Group's North East Mall in Hurst,
Texas, before it reopened in May. Some analysts say it isn't good to see
mall operators buying their own tenants out of bankruptcy at such a
rapid pace.Credit...Tom Pennington/Getty Images

Still, some analysts say it isn't good to see mall operators buying
their own tenants out of bankruptcy at this pace.

There may be few options. As long as large retailers or hedge funds are
unwilling to buy bankrupt chains like
\href{https://www.nytimes3xbfgragh.onion/live/2020/09/01/business/stock-market-today-coronavirus/jc-penney-has-10-days-to-avoid-liquidation}{J.C.
Penney}, which could ultimately liquidate, ``mall owners are the only
viable acquirers,'' analysts at Coresight Research, an advisory and
research firm, wrote in a recent note. The firm estimated that 20,000 to
25,000 U.S. retail stores would close this year, and at least 50 percent
are mall-based.

``Acquiring retailers raises questions about mall owners' long-term
viability,'' they wrote. ``Mall owners cannot buy every anchor retailer
in their malls, and often they will have to let stores fail instead of
propping them up,'' the analysts wrote.

Mr. Simon bristled on a recent earnings call at the notion that he was
buying retailers for rent. ``We believe in the brand and we think we can
make money,'' he said. He compared critics of the venture to those who
told Amazon to remain in the book business.

Still, rent is no small concern. In filings, Forever 21, a top tenant at
Brookfield and Simon malls in the year
\href{https://www.nytimes3xbfgragh.onion/2019/10/23/business/forever-21-bankruptcy-chang-family.html}{before
its bankruptcy}, said the aggregate occupancy cost for its stores was
\$450 million annually. Lucky listed \$66 million in rent and occupancy
costs last year. Brooks Brothers said its 187 store leases and other
corporate property leases cost about \$86 million a year. On top of
that, there are
\href{https://www.nytimes3xbfgragh.onion/2020/07/05/business/coronavirus-malls-department-stores-bankruptcy.html}{co-tenancy
agreements}, which can allow other tenants to break leases or demand
rent reductions based on vacancy rates or the exit of certain retailers.

``I do believe that the strategy by Simon and Brookfield is to protect
their co-tenancy in a lot of cases, but I think it's a Band-Aid,'' said
Jackie Levy, chief business officer of Caruso, the real estate firm that
owns California open-air shopping centers like the Grove. ``It might
solve the immediate issue of keeping some of their smaller retailers or
shops in the malls, but long-term, those leases are going to expire at
some point and there's going to be a flight to quality.''

For his part, Mr. Salter sees opportunities to meld the brands that go
beyond reducing corporate staff and sharing e-commerce capabilities. He
can imagine, for example, Brooks Brothers teaming up with Spyder to make
performance outerwear, and with Volcom for swim trunks. Saks Fifth
Avenue still plans to introduce Barneys New York shops within its New
York flagship and Connecticut stores.

Image

The brands of Mr. Salter, above, have ``variable rent'' contracts with
Mr. Simon's malls, meaning their rent goes up and down with their
sales.Credit...Mark Sommerfeld for The New York Times

``If I could buy anything, I'd buy Reebok,'' he said. ``Hanna Barbera. I
like the Flintstones, Yogi Bear. Got big ideas for Yogi Bear. **** I
love the Jetsons. They should be the delivery system for Amazon. Just
call the Jetsons, they'll deliver it to you in two seconds!''

Though Mr. Salter said he wasn't joining a bid by Simon and Brookfield
for J.C. Penney, he can envision pursuing a similar chain in the future.

``There's no doubt about it that Jamie Salter's dream is to have an
A.B.G. department store,'' Mr. Salter said. ``And as David Simon says,
maybe one day you'll have your own mall.''

Contact Sapna Maheshwari at
\href{mailto:sapna@NYTimes.com}{\nolinkurl{sapna@NYTimes.com}} or
Vanessa Friedman at
\href{mailto:vanessa.friedman@NYTimes.com}{\nolinkurl{vanessa.friedman@NYTimes.com}}.

Advertisement

\protect\hyperlink{after-bottom}{Continue reading the main story}

\hypertarget{site-index}{%
\subsection{Site Index}\label{site-index}}

\hypertarget{site-information-navigation}{%
\subsection{Site Information
Navigation}\label{site-information-navigation}}

\begin{itemize}
\tightlist
\item
  \href{https://help.nytimes3xbfgragh.onion/hc/en-us/articles/115014792127-Copyright-notice}{©~2020~The
  New York Times Company}
\end{itemize}

\begin{itemize}
\tightlist
\item
  \href{https://www.nytco.com/}{NYTCo}
\item
  \href{https://help.nytimes3xbfgragh.onion/hc/en-us/articles/115015385887-Contact-Us}{Contact
  Us}
\item
  \href{https://www.nytco.com/careers/}{Work with us}
\item
  \href{https://nytmediakit.com/}{Advertise}
\item
  \href{http://www.tbrandstudio.com/}{T Brand Studio}
\item
  \href{https://www.nytimes3xbfgragh.onion/privacy/cookie-policy\#how-do-i-manage-trackers}{Your
  Ad Choices}
\item
  \href{https://www.nytimes3xbfgragh.onion/privacy}{Privacy}
\item
  \href{https://help.nytimes3xbfgragh.onion/hc/en-us/articles/115014893428-Terms-of-service}{Terms
  of Service}
\item
  \href{https://help.nytimes3xbfgragh.onion/hc/en-us/articles/115014893968-Terms-of-sale}{Terms
  of Sale}
\item
  \href{https://spiderbites.nytimes3xbfgragh.onion}{Site Map}
\item
  \href{https://help.nytimes3xbfgragh.onion/hc/en-us}{Help}
\item
  \href{https://www.nytimes3xbfgragh.onion/subscription?campaignId=37WXW}{Subscriptions}
\end{itemize}
