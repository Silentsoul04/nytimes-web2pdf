Sections

SEARCH

\protect\hyperlink{site-content}{Skip to
content}\protect\hyperlink{site-index}{Skip to site index}

\href{https://myaccount.nytimes3xbfgragh.onion/auth/login?response_type=cookie\&client_id=vi}{}

\href{https://www.nytimes3xbfgragh.onion/section/todayspaper}{Today's
Paper}

\href{/section/opinion}{Opinion}\textbar{}The Pandemic Was Supposed to
Be Great for Strongmen. What Happened?

\url{https://nyti.ms/33bAqi4}

\begin{itemize}
\item
\item
\item
\item
\item
\item
\end{itemize}

Advertisement

\protect\hyperlink{after-top}{Continue reading the main story}

\href{/section/opinion}{Opinion}

Supported by

\protect\hyperlink{after-sponsor}{Continue reading the main story}

\hypertarget{the-pandemic-was-supposed-to-be-great-for-strongmen-what-happened}{%
\section{The Pandemic Was Supposed to Be Great for Strongmen. What
Happened?}\label{the-pandemic-was-supposed-to-be-great-for-strongmen-what-happened}}

From Trump to Lukashenko, authoritarians are discovering that this isn't
their kind of crisis.

\href{https://www.nytimes3xbfgragh.onion/by/ivan-krastev}{\includegraphics{https://static01.graylady3jvrrxbe.onion/images/2018/04/02/opinion/ivan-krastev/ivan-krastev-thumbLarge.png}}

By \href{https://www.nytimes3xbfgragh.onion/by/ivan-krastev}{Ivan
Krastev}

Contributing Opinion Writer

\begin{itemize}
\item
  Sept. 8, 2020
\item
  \begin{itemize}
  \item
  \item
  \item
  \item
  \item
  \item
  \end{itemize}
\end{itemize}

\includegraphics{https://static01.graylady3jvrrxbe.onion/images/2020/09/08/opinion/08Krastev/08Krastev-articleLarge.jpg?quality=75\&auto=webp\&disable=upscale}

VIENNA ---~For an East European of my generation, watching the current
protests in Belarus is like going through an old photo album. The scenes
of striking workers call forth the shipyards of Gdansk, Poland, and the
Solidarity movement of the 1980s. Moscow's dilemma whether to offer
President Aleksandr G. Lukashenko's regime ``friendly'' support reminds
me of Czechoslovakia in 1968, when Soviet troops entered the Czech
capital to scotch the popular Prague Spring. And the West's striking
incapacity to support civil society in Belarus screams of 1989 ---
though not in Eastern Europe but in China. The question of the moment is
whether Mr. Lukashenko will repeat the tragedy of Tiananmen.

What I have been thinking most about is not a protest movement of my
youth, but a natural disaster. The uprising in Belarus stands in the
shadow of Chernobyl, the worst nuclear catastrophe in human history,
which took place in the neighboring Soviet republic of Ukraine.
Thirty-four years later, citizens have realized that nothing has changed
in their country, and that they are ruled by a government ready to
sacrifice its people in order to hide the regime's decay.

This spring, when all of Europe was in lockdown to combat the
coronavirus pandemic, Mr. Lukashenko informed Belarusians that there is
nothing to fear. The best thing they could do, he said, was ignore the
global hysteria, head to football stadiums and cheer on their favorite
clubs. Many did so; many also got infected with the virus and died. We
can only speculate how many Belarusians would have taken to the streets
were it not for Covid-19. But it is clear that the government's feckless
response to the pandemic was a turning point.

The protests in Belarus should force us to rethink the relationship
between the pandemic and authoritarianism. Does the virus infect our
societies with authoritarian governance or, alternatively, can it
strengthen democratic immunity?

Some fear that more than any other crisis, a public health emergency
like this one will impel people to accept restrictions on their
liberties in the hope of improving personal security. The pandemic has
increased tolerance of invasive surveillance and bans on freedom of
assembly. In several Western countries --- including the United States
and Germany --- there were public protests against mask mandates and
lockdowns.

At the same time, the pandemic has eroded the power of authoritarians
and the authoritarian-inclined. The instinctive reaction of leaders like
Mr. Lukashenko in Belarus, Vladimir Putin in Russia, Jair Bolsonaro in
Brazil and Donald Trump in the United States was not to take advantage
of the state of emergency to expand their authority --- it was to play
down the seriousness of the pandemic.

Why are authoritarian leaders who thrive on crises and who are fluent in
the politics of fear reluctant to embrace the opportunity? Why do they
seem to hate a crisis that they should love? The answer is
straightforward: Authoritarians only enjoy those crises they have
manufactured themselves. They need enemies to defeat, not problems to
solve. The freedom authoritarian leaders cherish most is the freedom to
choose which crises merit a response. It is this capacity that allows
them to project an image of Godlike power.

In pre-Covid-19 Russia, Mr. Putin could ``solve'' one crisis by ginning
up another. He reversed the decline of his popularity after the protest
movement of 2011-12 by dramatically annexing Crimea. Mr. Trump could
once claim that migrant caravans from Mexico are the greatest threat his
country is facing, and disregard the civilizational threat of climate
change. In the age of coronavirus, this is no longer possible.

There is just this one crisis, here and now: the pandemic. And
governments are being judged by how they manage it. Authoritarian actors
not only loathe crises they have not freely chosen, they also dislike
``exceptional situations'' that force them to respond with standardized
rules and protocols rather than with ad hoc, discretionary moves.
Mundane behaviors like physical distancing, self-isolation and
handwashing are the best ways to halt the spread of the virus. A
leader's bold stroke of genius will be of no help. Following rules is
not the same as obeying orders.

Even more threatening for authoritarian elites in the Covid-19 world is
that they lack the key advantage all democratic leaders enjoy: The
luxury to survive even when appearing weak. Imagine that Mr. Putin
orders all Russian citizens to wear masks and half of the population
elect not to. For a democratic leader, this would be an embarrassment;
for an authoritarian it is a direct challenge to his power.

The ubiquity of the disease also poses challenges for authoritarians.
Because the pandemic affects every country in the world, citizens can
compare the actions of their governments with those of others. Success
or failure at flattening the curve provides a common metric, making
cross-national comparisons possible and putting pressure on governments
that had previously succeeded in insulating themselves from public
criticism.

In this context, Covid-19 has become deadly dangerous for ossifying
authoritarian regimes like Mr. Lukashenko's in Belarus. It is still
possible that the patient will survive if it is put in an artificial
coma in Mr. Putin's emergency room. But it is now clear that the virus
is a curse rather than a blessing for authoritarians like him.

In 1986, the Chernobyl tragedy made the people of the Soviet Union see
the reality of the Communist system hidden behind the state propaganda:
It wasn't all powerful. In fact, it wasn't even competent. The regime
lasted only a few more years.

\emph{The Times is committed to publishing}
\href{https://www.nytimes3xbfgragh.onion/2019/01/31/opinion/letters/letters-to-editor-new-york-times-women.html}{\emph{a
diversity of letters}} \emph{to the editor. We'd like to hear what you
think about this or any of our articles. Here are some}
\href{https://help.nytimes3xbfgragh.onion/hc/en-us/articles/115014925288-How-to-submit-a-letter-to-the-editor}{\emph{tips}}\emph{.
And here's our email:}
\href{mailto:letters@NYTimes.com}{\emph{letters@NYTimes.com}}\emph{.}

\emph{Follow The New York Times Opinion section on}
\href{https://www.facebookcorewwwi.onion/nytopinion}{\emph{Facebook}}\emph{,}
\href{http://twitter.com/NYTOpinion}{\emph{Twitter (@NYTopinion)}}
\emph{and}
\href{https://www.instagram.com/nytopinion/}{\emph{Instagram}}\emph{.}

Advertisement

\protect\hyperlink{after-bottom}{Continue reading the main story}

\hypertarget{site-index}{%
\subsection{Site Index}\label{site-index}}

\hypertarget{site-information-navigation}{%
\subsection{Site Information
Navigation}\label{site-information-navigation}}

\begin{itemize}
\tightlist
\item
  \href{https://help.nytimes3xbfgragh.onion/hc/en-us/articles/115014792127-Copyright-notice}{©~2020~The
  New York Times Company}
\end{itemize}

\begin{itemize}
\tightlist
\item
  \href{https://www.nytco.com/}{NYTCo}
\item
  \href{https://help.nytimes3xbfgragh.onion/hc/en-us/articles/115015385887-Contact-Us}{Contact
  Us}
\item
  \href{https://www.nytco.com/careers/}{Work with us}
\item
  \href{https://nytmediakit.com/}{Advertise}
\item
  \href{http://www.tbrandstudio.com/}{T Brand Studio}
\item
  \href{https://www.nytimes3xbfgragh.onion/privacy/cookie-policy\#how-do-i-manage-trackers}{Your
  Ad Choices}
\item
  \href{https://www.nytimes3xbfgragh.onion/privacy}{Privacy}
\item
  \href{https://help.nytimes3xbfgragh.onion/hc/en-us/articles/115014893428-Terms-of-service}{Terms
  of Service}
\item
  \href{https://help.nytimes3xbfgragh.onion/hc/en-us/articles/115014893968-Terms-of-sale}{Terms
  of Sale}
\item
  \href{https://spiderbites.nytimes3xbfgragh.onion}{Site Map}
\item
  \href{https://help.nytimes3xbfgragh.onion/hc/en-us}{Help}
\item
  \href{https://www.nytimes3xbfgragh.onion/subscription?campaignId=37WXW}{Subscriptions}
\end{itemize}
