Sections

SEARCH

\protect\hyperlink{site-content}{Skip to
content}\protect\hyperlink{site-index}{Skip to site index}

\href{https://www.nytimes3xbfgragh.onion/section/sports/soccer}{Soccer}

\href{https://myaccount.nytimes3xbfgragh.onion/auth/login?response_type=cookie\&client_id=vi}{}

\href{https://www.nytimes3xbfgragh.onion/section/todayspaper}{Today's
Paper}

\href{/section/sports/soccer}{Soccer}\textbar{}Barcelona Members Race
Clock in Effort to Oust Bartomeu

\url{https://nyti.ms/2GILUSM}

\begin{itemize}
\item
\item
\item
\item
\item
\end{itemize}

Advertisement

\protect\hyperlink{after-top}{Continue reading the main story}

Supported by

\protect\hyperlink{after-sponsor}{Continue reading the main story}

\hypertarget{barcelona-members-race-clock-in-effort-to-oust-bartomeu}{%
\section{Barcelona Members Race Clock in Effort to Oust
Bartomeu}\label{barcelona-members-race-clock-in-effort-to-oust-bartomeu}}

A group of Barcelona fans has two weeks to collect thousands of
handwritten signatures to force a vote that could force out the club's
unpopular president.

\includegraphics{https://static01.graylady3jvrrxbe.onion/images/2020/02/25/sports/25onsoccer-bartomeu/merlin_169341525_55bde13f-5bf8-4817-b0a0-ee474cb12d2c-articleLarge.jpg?quality=75\&auto=webp\&disable=upscale}

By \href{https://www.nytimes3xbfgragh.onion/by/tariq-panja}{Tariq Panja}

\begin{itemize}
\item
  Sept. 8, 2020
\item
  \begin{itemize}
  \item
  \item
  \item
  \item
  \item
  \end{itemize}
\end{itemize}

\href{https://www.nytimes3xbfgragh.onion/es/2020/09/10/espanol/deportes/barcelona-bartomeu.html}{Leer
en español}

A group of F.C. Barcelona members that has spent months mobilizing,
planning and plotting to force out the club's unpopular president, Josep
Maria Bartomeu, is using the soccer team's latest crisis to begin an
official campaign to force a change in leadership.

The effort's biggest opponent is not Bartomeu, though, but rather a
ticking clock, restrictions brought on by the coronavirus pandemic and
the club's own byzantine rules: To force a no-confidence vote in the
board, the organizers first must collect the handwritten signatures of
portion of the club's 140,000 members who are eligible to vote --- about
16,500 people. And they must do it in the next nine days.

``We have to go door to door,'' said Marc Duch, one of the Barcelona
members behind the campaign. ``This Covid thing is a massive issue at
the moment.''

Discussions that had been taking place for months, he said, suddenly
gathered pace in recent weeks. Already furious about
\href{https://www.nytimes3xbfgragh.onion/2020/04/16/sports/fc-barcelona-boardroom.html}{months
of boardroom infighting} and angry about
\href{https://www.nytimes3xbfgragh.onion/2020/08/14/sports/soccer/bayern-barcelona-8-2-champions-league.html?action=click\&module=RelatedLinks\&pgtype=Article}{a
humiliating exit} from the Champions League, the organizers of the
campaign said Lionel Messi's announcement
\href{https://www.nytimes3xbfgragh.onion/2020/08/25/sports/soccer/lionel-messi-barcelona.html}{that
he wanted out of Barcelona} --- a plan Messi has since
\href{https://www.nytimes3xbfgragh.onion/2020/09/04/sports/soccer/lionel-messi-barcelona.html}{abandoned}
--- was the last straw.

``That sped things up,'' Duch said.

Removing a board that has been duly elected is no easy task, though.
According to Barcelona's bylaws, a vote of no confidence can only be
called if 15 percent of the club's eligible voters --- people who have
been members for more than a year --- provide handwritten signatures on
official forms provided by the club within two weeks. (The forms must be
accompanied by a photocopy of the front and back of each signer's
national identification card.) And that's just to get the vote. To pass,
the motion requires approval by a two-thirds majority.

Those rules, coupled with coronavirus-related restrictions, have set the
scene for a race against time for Bartomeu's opponents, a disparate
alliance of fan groups that also has the support of three men who have
said they would stand as candidates in the next election for club
president. Bartomeu's current term runs through next spring.

To officially launch their campaign, Duch, a tax adviser by profession,
and a handful of other members arrived at the team's headquarters last
Wednesday. There they were provided with boxes upon boxes of the
required forms --- 32,000 in total --- that they loaded into cars and
vans.

That was the easy part.

In normal circumstances, getting the forms in front of the team's
members would require little more than stationing volunteers outside the
club's Camp Nou stadium on match days and handing them out. But with the
team currently in a brief off-season, and social distancing rules
limiting gatherings to fewer than 10 people, collecting the thousands of
signatures by next week's deadline has been a complex logistical
conundrum from the start.

To spread the word, batches of forms have been left at more than 130
office blocks, restaurants and other businesses around Catalonia. The
group also posted locations where Barcelona members can pick up a form
on \href{https://www.mocio2020.cat/en/where-to-sign/}{a website}, and it
continues to try to spread the word on social media*.* So far, Duch and
others said, they have gathered 7,500 signatures.

Barcelona declined to comment on the effort, but at least one of
Bartomeu's rivals does not appear to oppose it.

``I thought the defeat in Lisbon was the bottom, but the bottom-bottom
was having the best player in the history of the sport, who has been 20
years in the club, wanting to leave after such a defeat and through the
back door,'' said Victor Font, a technology entrepreneur and one of the
front-runners in the race to replace Bartomeu.

Bartomeu can remain in charge until the middle of next year, but,
according to Font, changes needs to come much faster than that, not
least because the risk of losing Messi next year remains a strong
possibility. Messi can speak with potential suitors --- Manchester City
is among those that have reportedly expressed interest --- and even sign
a precontract agreement as soon as Jan. 1.

Should Bartomeu's opponents succeed in ousting the current leadership,
elections would have to take place within three months. Whoever is in
charge, however, will face a bulging inbox of immediate issues beyond
the fate of Messi.

Key sponsorship agreements --- including with the team's principal
sponsor Rakuten --- will be up for renewal; a contentious and hugely
expensive stadium refurbishment will need to be addressed; and, perhaps
most important for the team's fans, the roster will need to be rebuilt.
But so will the club's battered image.

``They have ruined it all, in economic terms, sporting terms and
institutionally we have lost all Europe's respect as a club,'' Duch
said.

Bloodlettings are not rare at Barcelona. Allies of Bartomeu once almost
succeeded in ousting a former president, Joan Laporta, in 2008. Laporta
narrowly survived, and went on to lay the foundations for much of the
team's current success by naming a largely untested former player, Pep
Guardiola, as coach. Under Guardiola, with Messi leading on the field,
Barcelona went on to enjoy a decade of unparalleled success.

Bartomeu took over in 2014, stepping up from a vice president role after
his ally Sandro Rosell was forced to step down amid claims of improper
conduct in the signing of the Brazilian forward Neymar.

But the tide against Bartomeu's control of the 120-year-old club has
been rising for months. Last week, the Spanish newspaper
\href{https://www.elmundo.es/deportes/futbol/primera-division/2020/09/03/5f51354efc6c8361398b46a9.html}{El
Mundo reported} that the police in Catalonia were investigating Bartomeu
for corruption.

Duch said eight groups had come together to push for Bartomeu's ouster.
But with only days to go, the campaign is entering its most difficult
stage: Persuading older Barcelona members --- unaware of an effort that
to date has largely played out on social media platforms like Facebook
and Twitter --- to add their names to the effort.

Advertisement

\protect\hyperlink{after-bottom}{Continue reading the main story}

\hypertarget{site-index}{%
\subsection{Site Index}\label{site-index}}

\hypertarget{site-information-navigation}{%
\subsection{Site Information
Navigation}\label{site-information-navigation}}

\begin{itemize}
\tightlist
\item
  \href{https://help.nytimes3xbfgragh.onion/hc/en-us/articles/115014792127-Copyright-notice}{©~2020~The
  New York Times Company}
\end{itemize}

\begin{itemize}
\tightlist
\item
  \href{https://www.nytco.com/}{NYTCo}
\item
  \href{https://help.nytimes3xbfgragh.onion/hc/en-us/articles/115015385887-Contact-Us}{Contact
  Us}
\item
  \href{https://www.nytco.com/careers/}{Work with us}
\item
  \href{https://nytmediakit.com/}{Advertise}
\item
  \href{http://www.tbrandstudio.com/}{T Brand Studio}
\item
  \href{https://www.nytimes3xbfgragh.onion/privacy/cookie-policy\#how-do-i-manage-trackers}{Your
  Ad Choices}
\item
  \href{https://www.nytimes3xbfgragh.onion/privacy}{Privacy}
\item
  \href{https://help.nytimes3xbfgragh.onion/hc/en-us/articles/115014893428-Terms-of-service}{Terms
  of Service}
\item
  \href{https://help.nytimes3xbfgragh.onion/hc/en-us/articles/115014893968-Terms-of-sale}{Terms
  of Sale}
\item
  \href{https://spiderbites.nytimes3xbfgragh.onion}{Site Map}
\item
  \href{https://help.nytimes3xbfgragh.onion/hc/en-us}{Help}
\item
  \href{https://www.nytimes3xbfgragh.onion/subscription?campaignId=37WXW}{Subscriptions}
\end{itemize}
