Sections

SEARCH

\protect\hyperlink{site-content}{Skip to
content}\protect\hyperlink{site-index}{Skip to site index}

\href{https://www.nytimes3xbfgragh.onion/section/sports/tennis}{Tennis}

\href{https://myaccount.nytimes3xbfgragh.onion/auth/login?response_type=cookie\&client_id=vi}{}

\href{https://www.nytimes3xbfgragh.onion/section/todayspaper}{Today's
Paper}

\href{/section/sports/tennis}{Tennis}\textbar{}Despite Big Hiccups and
No Fans, the U.S. Open Has Had Some Classics

\url{https://nyti.ms/2R3EQlJ}

\begin{itemize}
\item
\item
\item
\item
\item
\end{itemize}

Advertisement

\protect\hyperlink{after-top}{Continue reading the main story}

Supported by

\protect\hyperlink{after-sponsor}{Continue reading the main story}

On Tennis

\hypertarget{despite-big-hiccups-and-no-fans-the-us-open-has-had-some-classics}{%
\section{Despite Big Hiccups and No Fans, the U.S. Open Has Had Some
Classics}\label{despite-big-hiccups-and-no-fans-the-us-open-has-had-some-classics}}

And some of the tennis may have been great precisely because of the lack
of fans during the coronavirus pandemic.

\includegraphics{https://static01.graylady3jvrrxbe.onion/images/2020/09/08/sports/08usopen-ontennis/merlin_176601186_f0c983ba-b0a2-4091-bea1-6f57ae15daff-articleLarge.jpg?quality=75\&auto=webp\&disable=upscale}

\href{https://www.nytimes3xbfgragh.onion/by/christopher-clarey}{\includegraphics{https://static01.graylady3jvrrxbe.onion/images/2018/09/10/multimedia/author-christopher-clarey/author-christopher-clarey-thumbLarge.png}}

By
\href{https://www.nytimes3xbfgragh.onion/by/christopher-clarey}{Christopher
Clarey}

\begin{itemize}
\item
  Sept. 8, 2020, 5:00 a.m. ET
\item
  \begin{itemize}
  \item
  \item
  \item
  \item
  \item
  \end{itemize}
\end{itemize}

Phase 1 of the weirdest United States Open was full of tennis lessons we
never expected we would have to learn.

Don't pull a ball out of your pocket
\href{https://www.nytimes3xbfgragh.onion/2020/09/07/sports/tennis/novak-djokovic-us-open.html}{and
smack it without looking}.

Don't play cards
\href{https://www.nytimes3xbfgragh.onion/2020/08/31/sports/tennis/us-open-coronavirus-contact-tracing.html}{with
Benoît Paire}.

Don't sign a new protocol and stay in a Long Island hotel. You
\href{https://www.nytimes3xbfgragh.onion/2020/09/05/sports/the-us-open-virus-quarantine-chaos.html}{still
might not be allowed} to cross a county line to play your match in
Queens.

Don't argue line calls on the outside courts.
\href{https://www.nytimes3xbfgragh.onion/2020/08/03/sports/tennis/us-open-hawkeye-line-judges.html}{With
automated calls}, there is no one to argue with.

But there was another revelation, too. You don't need a crowd to have a
classic U.S. Open night match.

Until now, the players and the spectators seemed to be essential
ingredients: feeding off one another, inspiring one another.

But Borna Coric and Stefanos Tsitsipas did it on their own in Louis
Armstrong Stadium, forging a mutual masterpiece as they exchanged
shouts, dirty looks and all manner of shots: bold, subtle, cocksure and
humanizingly shaky in the third round.

Tsitsipas, a prodigiously talented Greek full of hunger and swagger,
seemed to have the match under control at 5-1 in the fourth set and
seemed to have it under lock and key serving at 5-4, 40-0. But Coric,
who has a tattoo that reads ``There is nothing worse in life than being
ordinary,'' stayed true to his body art.

One of the best movers in the men's game, the young, bristle-haired
Croatian kept grinding and swinging. He saved six match points and
leveled the match at two sets apiece as Friday night turned into
Saturday.

Tsitsipas could have been excused for curling up into a ball on the
baseline at that stage. But he stayed upright and even went up a break
in the fifth set before Coric leveled.

Tsitsipas had four more break-point chances down the stretch. But Coric
held phenomenally firm and Tsitsipas cracked again, double faulting
twice in the fifth-set tiebreaker as Coric prevailed 6-7 (2), 6-4, 4-6,
7-5, 7-6 (4).

``I have to be honest, and say I was really lucky,'' said Coric, who is
now in the quarterfinals. ``In the third and fourth set, he was playing
unbelievable tennis, and I felt like I had no chance.''

It was not the first tennis pandemic epic (a pandepic, perhaps?): Andy
Murray and his bionic hip won a five-setter of their own in the first
round against Yoshihito Nishioka. Earlier on Friday, Denis Shapovalov
came back from a break down in the fifth to defeat Taylor Fritz.

Although Novak Djokovic's fourth-round default was certainly the most
dramatic moment of the first week, he and Pablo Carreño Busta did not
even finish the first set. For long-form quality, relentless intensity
and midnight madness timing, there was no topping Coric and Tsitsipas.

``This is probably the saddest and funniest at the same time thing that
has ever happened in my career,''
\href{https://twitter.com/StefTsitsipas/status/1302112839906349056}{tweeted
Tsitsipas}, in new-generation fashion, just minutes after it happened.

It would have been the match of just about any tournament --- this one,
coronavirus willing, still has matches through Sunday --- and that it
could happen in a fan-free environment in an individual sport was both
reaffirming and unsettling.

How much do the roars and the jeers really matter?

The thought is, of course, not unique to tennis at the moment. Sport
after sport is discovering what it means to play behind closed doors.

But there were moments on Friday night when the lack of outside buzz and
external distraction actually seemed to elevate the duel, making it
possible to hear every sneaker squeak, every grunt and mutter.

The court-level camera angles helped, too, bringing viewers into the
players' space and avoiding the wider shots that would have made clear
that hardly anyone was watching in person.

It was intimate, even meditative at times, as the two rivals took turns
being brilliant under pressure to the sounds of the passing trains and a
few shouts from their entourages.

``Look, it would have been an amazing atmosphere to have fans in there
--- cheering a guy on as he makes this amazing comeback,'' said Brad
Gilbert, who called the match for ESPN. ``But I do think that the
players start getting locked in, and that it's just about you and the
opponent. I don't think they even were noticing there was no crowd.''

Call it their own bubble within a bubble.

``You could see everything develop with clarity because you had no
distractions,'' Gilbert said. ``But listen, I'm just so grateful we have
a chance to do the tennis and just see the tennis. Obviously, this model
without a crowd is not sustainable for the rest of tennis ever, but for
the moment, it's a lot better than no tennis.''

The problem in New York during Week 1 was that not everyone who crossed
the Atlantic to play tennis was allowed to do so, and that in Djokovic's
case, the biggest star in the men's game essentially eliminated himself.

Staging this tournament at all has been an immense undertaking, and the
U.S.T.A. does not have the same financial means as the N.B.A. with its
locked-down campus at Walt Disney World in Florida. Nor did it have the
wherewithal to quarantine an international field of players for two full
weeks before the first ball was struck.

There were bound to be issues. For now, Paire is the only player known
to have tested positive for the coronavirus in the controlled
environment set up for the Western \& Southern Open and the U.S. Open.
But the devil has been in the details of the contact tracing, which
forced seven players who had been in close contact with Paire to sign a
new, more restrictive agreement in order to keep playing.

When Nassau County health officials learned that those in contact with
Paire were being allowed to compete instead of remaining in full
quarantine, they effectively voided the new agreement. On Saturday, the
French star Kristina Mladenovic, one of those in contact with Paire, was
not permitted to travel to the Billie Jean King National Tennis Center
from the player hotel.

She and her Hungarian doubles partner, Timea Babos, the No. 1 seeds,
were forced to withdraw before their second-round match, after Adrian
Mannarino of France had been allowed to play singles on Friday after
great debate. He ended up losing to Alexander Zverev.

This moving of the goal posts is not the way this situation should have
been handled. Inconsistency undermines the rules, and that Mannarino was
allowed to play because he was not at the hotel in Nassau County when
the new edict was issued is not a good enough excuse.

Every probable scenario should have been talked through and made clear
with all the potentially relevant health authorities before the
tournament began.

Failing to do so undermines the U.S.T.A.'s remarkable efforts and
certainly does not play well internationally.

``US Open 2020: un tournoi amateur'' (an amateur tournament)
\href{https://www.lequipe.fr/Tennis/Article/Us-open-2020-un-tournoi-amateur/1168610}{wrote
L'Équipe in a headline} over the weekend, bemoaning the lack of
consistency and the lack of agreement among health officials within the
same state. ``The show has sadly moved outside the tennis courts,''
L'Équipe wrote. ``Even in the midst of a health crisis, that is not
worthy of a Grand Slam tournament.''

Babos, already back in Europe, echoed those sentiments in an Instagram
post on Sunday.

``I'm sitting in my kitchen crying,'' she said. ``It's terribly unfair.
I see no reasonable reason why it had to be like this.''

Clearly, watching Mannarino play on Friday, it did not have to be like
this. But that does not mean the 2020 U.S. Open, even tarnished and
having lost its biggest men's star, has not had its shining moments.

Most of the players seem to appreciate the opportunity (and the
paycheck), and they have paid it back with tennis worthy of the
occasion, worthy of a Grand Slam tournament.

Coric versus Tsitsipas was only the best of many examples: a late-night
classic no doubt, even without the customary soundtrack.

Advertisement

\protect\hyperlink{after-bottom}{Continue reading the main story}

\hypertarget{site-index}{%
\subsection{Site Index}\label{site-index}}

\hypertarget{site-information-navigation}{%
\subsection{Site Information
Navigation}\label{site-information-navigation}}

\begin{itemize}
\tightlist
\item
  \href{https://help.nytimes3xbfgragh.onion/hc/en-us/articles/115014792127-Copyright-notice}{©~2020~The
  New York Times Company}
\end{itemize}

\begin{itemize}
\tightlist
\item
  \href{https://www.nytco.com/}{NYTCo}
\item
  \href{https://help.nytimes3xbfgragh.onion/hc/en-us/articles/115015385887-Contact-Us}{Contact
  Us}
\item
  \href{https://www.nytco.com/careers/}{Work with us}
\item
  \href{https://nytmediakit.com/}{Advertise}
\item
  \href{http://www.tbrandstudio.com/}{T Brand Studio}
\item
  \href{https://www.nytimes3xbfgragh.onion/privacy/cookie-policy\#how-do-i-manage-trackers}{Your
  Ad Choices}
\item
  \href{https://www.nytimes3xbfgragh.onion/privacy}{Privacy}
\item
  \href{https://help.nytimes3xbfgragh.onion/hc/en-us/articles/115014893428-Terms-of-service}{Terms
  of Service}
\item
  \href{https://help.nytimes3xbfgragh.onion/hc/en-us/articles/115014893968-Terms-of-sale}{Terms
  of Sale}
\item
  \href{https://spiderbites.nytimes3xbfgragh.onion}{Site Map}
\item
  \href{https://help.nytimes3xbfgragh.onion/hc/en-us}{Help}
\item
  \href{https://www.nytimes3xbfgragh.onion/subscription?campaignId=37WXW}{Subscriptions}
\end{itemize}
