\href{/section/style}{Style}\textbar{}Guitars Are Back, Baby!

\url{https://nyti.ms/331Ugw4}

\begin{itemize}
\item
\item
\item
\item
\item
\item
\end{itemize}

\href{https://www.nytimes3xbfgragh.onion/spotlight/at-home?action=click\&pgtype=Article\&state=default\&region=TOP_BANNER\&context=at_home_menu}{At
Home}

\begin{itemize}
\tightlist
\item
  \href{https://www.nytimes3xbfgragh.onion/2020/09/07/travel/route-66.html?action=click\&pgtype=Article\&state=default\&region=TOP_BANNER\&context=at_home_menu}{Cruise
  Along: Route 66}
\item
  \href{https://www.nytimes3xbfgragh.onion/2020/09/04/dining/sheet-pan-chicken.html?action=click\&pgtype=Article\&state=default\&region=TOP_BANNER\&context=at_home_menu}{Roast:
  Chicken With Plums}
\item
  \href{https://www.nytimes3xbfgragh.onion/2020/09/04/arts/television/dark-shadows-stream.html?action=click\&pgtype=Article\&state=default\&region=TOP_BANNER\&context=at_home_menu}{Watch:
  Dark Shadows}
\item
  \href{https://www.nytimes3xbfgragh.onion/interactive/2020/at-home/even-more-reporters-editors-diaries-lists-recommendations.html?action=click\&pgtype=Article\&state=default\&region=TOP_BANNER\&context=at_home_menu}{Explore:
  Reporters' Google Docs}
\end{itemize}

\includegraphics{https://static01.graylady3jvrrxbe.onion/images/2020/09/08/fashion/08GUITAR-CLAPTON-SUB/08GUITAR-CLAPTON-SUB-articleLarge.jpg?quality=75\&auto=webp\&disable=upscale}

Sections

\protect\hyperlink{site-content}{Skip to
content}\protect\hyperlink{site-index}{Skip to site index}

\hypertarget{guitars-are-back-baby}{%
\section{Guitars Are Back, Baby!}\label{guitars-are-back-baby}}

Painted by some as a boomer relic just years ago, the guitar is seeing a
revival that may just extend past the stress-purchase quarantine bounce.

The British guitar player Eric Clapton on stage in Boston, Mass. in
1974.Credit...Fin Costello/Redferns, via Getty Images

Supported by

\protect\hyperlink{after-sponsor}{Continue reading the main story}

\includegraphics{https://static01.graylady3jvrrxbe.onion/images/2020/05/06/reader-center/alex-williams/alex-williams-thumbLarge-v2.jpg}

By Alex Williams

\begin{itemize}
\item
  Sept. 8, 2020
\item
  \begin{itemize}
  \item
  \item
  \item
  \item
  \item
  \item
  \end{itemize}
\end{itemize}

Not so long ago, things didn't look so great for the guitar, that global
symbol of youthful freedom and rebellion for 70 years running.

With hip-hop and Beyoncé-style spectacle pop supposedly owning the
hearts and wallets of millennials and Generation Z --- and so many
20th-century guitar deities either dead (Jimi Hendrix, Kurt Cobain) or
soloing into their 70s (Eric Clapton, Jimmy Page) --- electric guitar
sales had skidded by about one-third in the decade since 2007, according
to \href{https://www.musictrades.com/about.html}{Music Trades}, a
research organization that tracks industry data.

Gibson guitars, whose celebrated
\href{https://www.gibson.com/Guitars/Les-Paul}{Les Paul} line had helped
put the Led in Zeppelin, was
\href{https://www.reuters.com/article/us-gibson-brands-bankruptcy/after-bankruptcy-gibson-plots-return-to-basics-and-possibly-ukuleles-idUSKBN1JN3AL\#:~:text=Nashville\%2Dbased\%20Gibson\%2C\%20the\%20maker,new\%20ownership\%20of\%20its\%20lenders.}{sliding
toward bankruptcy}.

All of this was enough for The Washington Post to declare the
``\href{https://www.washingtonpost.com/graphics/2017/lifestyle/the-slow-secret-death-of-the-electric-guitar/}{slow,
secret death} of the six-string electric'' in 2017. That same year, even
Mr. Clapton himself, known simply as ``God'' to devotees more than half
a century ago, sounded ready to spread the ashes. ``Maybe,'' he mused at
a 2017 news conference for the documentary
\href{https://www.nytimes3xbfgragh.onion/2017/11/23/movies/eric-clapton-a-life-in-12-bars-review.html}{``Eric
Clapton: A Life in 12 Bars,''} ``the
\href{https://www.billboard.com/articles/columns/rock/7957989/eric-clapton-guitar-sales-declining-tiff}{guitar
is over}.''

Hold the obituaries.

A half-year into a pandemic that has threatened to sink entire
industries, people are turning to the guitar as a quarantine companion
and psychological salve, spurring a surge in sales for some of the most
storied companies (Fender, Gibson, Martin, Taylor) that has shocked even
industry veterans.

``I would never have predicted that we would be looking at having a
record year,'' said Andy Mooney, the chief executive of Fender Musical
Instruments Corporation, the Los Angeles-based guitar giant that has
equipped Rock \& Roll Hall of Famers since Buddy Holly strapped on a
\href{https://guitar.com/news/watch-new-documentary-unearths-what-could-be-buddy-hollys-long-lost-strat/}{1954
sunburst} Fender Stratocaster back in the tail-fin 1950s.

``We've broken so many records,'' Mr. Mooney said. ``It will be the
biggest year of sales volume in Fender history, record days of
double-digit growth, e-commerce sales and beginner gear sales. I never
would have thought we would be where we are today if you asked me back
in March.''

It's not just graying baby boomer men looking to live out one last Peter
Frampton fantasy. Young adults and teenagers, many of them female, are
helping to power this guitar revival, manufacturers and retailers said,
putting their own generational stamp on the instrument that rocked their
parents' generation while also discovering the powers of six-string
therapy.

\includegraphics{https://static01.graylady3jvrrxbe.onion/images/2020/09/08/fashion/08GUITAR-FENDER/merlin_175820145_93ed225b-bb90-4195-bc73-1dd94148c75b-articleLarge.jpg?quality=75\&auto=webp\&disable=upscale}

\hypertarget{playing-away-the-blues}{%
\subsection{Playing Away the Blues}\label{playing-away-the-blues}}

It all started with a collective breaking point, according to Jensen
Trani, a guitar instructor in Los Angeles whose thousands of
\href{https://www.youtube.com/channel/UCkD5j0H8JRsVaegeQdkVUaA}{instructional
videos on YouTube}, he estimated, have attracted some 75 million views
over the past 14 years.

``There was this point with my students where I could tell that numbing
out on Netflix and Instagram and Facebook was just not working
anymore,'' Mr. Trani, 38, said. ``People could no longer go to their
usual coping mechanisms. They were saying, `How do I want to spend my
day?'''

For many, apparently, the answer was ``strumming.''

Shortly after stay-at-home orders were announced in the spring, Mr.
Trani saw a surge of traffic for his videos, he said, and quickly
tripled his number of private students taking lessons remotely. Popular
instructional sites like
\href{https://www.justinguitar.com/}{JustinGuitar.com} and
\href{https://www.guitartricks.com/}{GuitarTricks} saw similar spikes
during the spring.

And most of the new students were not looking to rekindle memories of
\href{https://www.youtube.com/watch?v=As5_xq78VRA}{Foghat live}in 1976.
Most of them probably did not know who Foghat was, given that the
majority of Mr. Trani's new students were, as he put it,
female-presenting people in their late 20s or early 30s.

The biggest names in the business of online guitar instruction were
seeing a similar pattern. Fender said that its guitar-instruction app,
\href{https://www.fender.com/play}{Fender Play}, which features Mr.
Trani as an instructor, saw its user base shoot to 930,000 from 150,000
between late March and late June, with a considerable assist from a
three-month promotional giveaway.

Nearly 20 percent of the newcomers were under 24, and 70 percent were
under 45, the company reported. Female users accounted for 45 percent of
the new wave, compared with 30 percent before the pandemic.

In a narrow sense, the surge made sense. Prospective players who had
never quite found the time to take up an instrument suddenly had little
excuse not to. As James Curleigh, the chief executive of Gibson Brands,
put it: ``In a world of digital acceleration, time is always your enemy.
All of a sudden time became your friend.''

Image

Miranda Lambert performs on The Tonight Show, remotely, in
May.~Credit...NBC/NBCU Photo Bank, via Getty Images

But there was more to it, Mr. Trani said. Many newcomers to the
instrument seemed to be looking for an oasis of calm in a turbulent
world. ``There is,'' he said, ``this sense of learning how to sit with
yourself.''

That was the case for one of his new students, Kayla Lucido, 31, of San
Jose, Calif., who decided to make good on her longstanding ambitions to
learn guitar in March, despite a frenzied schedule juggling remote work
as a project coordination manager at a technology company and parenting
duties for her 17-month-old son.

``It's been quite healing for me, learning something new, and being able
to drown everything else out,'' said Ms. Lucido, who has been plucking
out songs like
\href{https://www.youtube.com/watch?v=ONS51QzCh1Y}{``Beautiful
Stranger''} by Halsey or
\href{https://www.youtube.com/watch?v=nUB8ogvze_8}{``Bluebird''} by
Miranda Lambert, even for 10 minutes each day.

``You just really have to focus on your hand placement, the chords
you're playing, then pairing that with the strumming,'' she added. ``If
I'm working out, my mind still wanders, but when I'm playing guitar, I
just get lost in it. It's like meditation.''

No wonder. Learning guitar, or piano, or oboe or bassoon, benefits the
brain on profound levels, according to
\href{https://www.daniellevitin.com/}{Daniel Levitin}, a neuroscientist,
musician and the author of the 2006 New York Times best seller
\href{https://www.nytimes3xbfgragh.onion/2006/12/31/arts/music/31thom.html}{``This
Is Your Brain on Music.''} (Many
\href{https://www.cambridgebrainsciences.com/more/articles/playing-a-musical-instrument-can-have-immediate-brain-benefits}{psychological}
\href{https://journals.sagepub.com/doi/abs/10.1177/0255761411408505}{studies}
have shown the therapeutic benefits of playing an instrument, as well.)

The process, Dr. Levitin wrote in an email, is ``neuroprotective'' in
that it ``requires that you grow new neural pathways --- something you
can do at literally any age.'' He added that ``using your brain for
something that is challenging, but not impossible, tends to be
rewarding, and hence comforting.''

Learning the guitar, he wrote, is also a forward-looking process,
kindling hope and optimism, which helps regulate stable mood chemicals
like serotonin and dopamine.

And ``there is a very real sense of mastery and accomplishment,'' Dr.
Levitin said. ``I'm working on a Chopin piece on the piano right now ---
the Prelude in E minor --- and I keep reminding myself I'm putting my
fingers in the same configurations that Chopin did. For a few minutes, I
can \emph{be} Chopin.''

``The same,'' he added, ``holds true for Clapton when I play guitar.''

Image

No teardrops on Taylor Swift's guitar.Credit...Kevin Mazur/TAS, via
WireImage

\hypertarget{every-day-is-black-friday}{%
\subsection{`Every Day is Black
Friday'}\label{every-day-is-black-friday}}

``I've been in the instrument retail business for 25-plus years and I've
never seen anything like it,'' Brendan Murphy, a senior salesman at
\href{https://www.sweetwater.com/}{Sweetwater}, an online retailer of
guitars and other instruments, wrote in an email in July. ``It feels
like every day is black Friday.''

Other online retailers were reporting the same thing in the spring and
into summer. Despite having to close 293 of its 296 giant retail
showrooms in March and April because of the coronavirus, Guitar Center
was soon seeing triple-digit sales growth for most top guitar brands on
the website, according to Michael Doyle, the company's senior vice
president of guitar merchandising.

Guitars are hardly the only consumer item to experience a quarantine
bounce, of course. Sales have spiked for many items since lockdowns
began --- bicycles,
\href{https://www.businessinsider.com/coronavirus-sales-increases-unlikely-items-selling-well-high-demand-2020-4\#hair-dye-2}{baking
yeast},
\href{https://www.cnbc.com/2020/03/23/what-people-are-buying-during-the-coronavirus-outbreak-and-why.html}{board
games}, yoga mats, beans and even
\href{https://www.nytimes3xbfgragh.onion/2020/07/09/style/how-everclear-became-a-pandemic-favorite.html}{Everclear},
the 190-proof spirit.

But a guitar is not a bag of lentils. A new guitar usually requires an
investment of several hundred dollars, if not several thousand, and new
players and virtuosos alike often live with their trusty ax for years,
bonding with it as a statement of personal taste and style.

It's what economists would call a ``discretionary'' purchase, the sort
of nonessential consumer item that is usually the last thing one might
buy when the economy is plunging and unemployment is skyrocketing. Throw
in monthslong factory closures for manufacturers and a virtual
disappearance of brick-and-mortar retailers, and the situation seemed
nearly apocalyptic.

``I figured that this is one of those business-falls-off a-cliff
situations,'' said Chris Martin, the chief executive of
\href{https://www.martinguitar.com/}{C.F. Martin \& Co.}, the
187-year-old manufacturer of acoustic guitars that has supplied
contemporary stars like John Mayer and Ed Sheeran, as well as legends
like Bob Dylan, Joni Mitchell and some
\href{https://www.martinguitar.com/standard-series-reimagined/}{guy
named Elvis}, over decades. ``We'll pick up the pieces and put the
company back together whenever.''

But after a ``terrible'' March, with revenues 40 percent below normal,
business roared back.

``It's crazy,'' said Mr. Martin, the sixth-generation Martin to run the
company. ``It's unbelievable the demand there is right now for acoustic
guitars. I've been through guitar booms before, but this one caught me
completely by surprise.''

\href{https://www.taylorguitars.com/}{Taylor Guitars}, which equips
Taylor Swift, Justin Bieber and Ben Harper,
\href{https://www.taylorguitars.com/artists}{among others}, with guitars
fashioned from fine tonewoods (including, in a recent eco-minded move,
shamel ash trees salvaged from
\href{https://www.rollingstone.com/pro/features/taylor-guitars-highway-trees-989888/}{Los
Angeles freeway}s)\emph{,} has seen a similar famine-to-feast rebound.

``We just had the biggest June, in terms of orders received, that we've
ever had since we've been in business,'' said Kurt Listug, who founded
the company with Bob Taylor in 1974. June and July alone, he added,
accounted for half the orders that the company had projected,
pre-pandemic, for all of 2020.

``Guitars hit the stores now, they unbox them, and they're gone,'' Mr.
Listug said.

Electric guitars may not have exactly the same
plunk-through-a-few-Neil-Young-tunes-on-the-bed appeal, but sales have
been strong on that front for the electric-guitar giants Fender and
Gibson, too (both companies also make acoustic guitars).

The pandemic hit at a sensitive time for Gibson. The company had
declared bankruptcy in 2018, after previous management had made an
aggressive push to expand into home and commercial audio electronics,
and attempted to jetpack this company founded in 1894 into the future
with 21st-century reinterpretations of classic Gibson stadium shakers
--- some featuring built-in
\href{https://www.youtube.com/watch?v=zb7G6KbwCpE\&t=279s}{electronic
``robot'' tuners}.

A new management team headed by Mr. Curleigh, the former president of
Levi's Brand, ditched the onboard robotics, rebooted the brand's
budget-priced \href{https://www.epiphone.com/}{Epiphone} line and
\href{https://www.gearnews.com/namm-2020-gibson-announces-new-models-for-original-collection-and-modern-collection/}{released
new Original and Modern} **** collections
featuring**\href{https://guitar.com/news/gear-news/gibson-2020-full-line-up-revealed/}{}**\href{https://guitar.com/news/gear-news/gibson-2020-full-line-up-revealed/}{fresh
interpretations} of classic Gibsons from the 1950s and 1960s that today
fetch five- and six-figure prices on the
\href{https://www.antiquesandthearts.com/1959-gibson-les-paul-standard-burst-takes-216000-at-westport-auction/}{vintage
market}.

The company was earning
\href{https://guitarpedaldemos.com/gibson-guitars-quality-control-2019/}{rave
reviews} for its
\href{https://www.forbes.com/sites/stevebaltin/2019/02/26/inside-the-gibson-guitar-comeback-story/\#73871fab2861}{new
product lines and improved quality control} before factories closed in
April.

``When we had no production,'' Mr. Curleigh said, ``we had no sales,
let's face it.''

By late summer, however, ``we literally couldn't deliver enough,'' he
said. ``Everything we were making, we could sell.''

To Mr. Curleigh, the guitar rebound was a signifier of deeper
psychological currents circulating among a traumatized population.
``It's Maslow's hierarchy of needs,'' he said, citing a theory of human
motivation proposed by the psychologist Abraham Maslow in the 1940s.
Maslow's \href{https://www.simplypsychology.org/maslow.html}{five-tier
pyramid of needs} proposed that people first must satisfy fundamental
requirements like sustenance and personal security before they can scale
toward the higher goals of creative fulfillment.

``That's what the world went through,'' Mr. Curleigh said. ``First we
were figuring out the basic essentials --- where to buy toilet paper,
making sure you were isolated in quarantine. Then the psychological
reset hit. People said, `Well, I can still self-actualize, I can still
self-fulfill.'''

Image

The K26ceGS by Taylor Guitars, one of the company's top three sellers
during the pandemic.

\hypertarget{will-it-last}{%
\subsection{Will It Last?}\label{will-it-last}}

It may be easy to guess that a lot of those glossy new guitars may end
up in the closet as soon as people once again whisk off their masks and
pack into crowded restaurants, bars, ballparks and movie theaters.
Indeed, interest in online tutorials has already cooled a bit from the
peaks in the spring, according to several sites.

And the overall retail picture for the industry remains rather fuzzy in
the short-term: Despite the sales bounce for marquee American companies,
overall sales of all fretted instruments --- including banjos, ukuleles
and bass guitars --- dipped 2.4 percent in the second quarter compared
with last year, according to Music Trades.

That dip also reflects a precipitous drop in imports --- nearly 23
percent for acoustics and 44 percent for electrics --- over the same
period, in large part because of factory closures, severed supply lines
and bottlenecks in shipping ports, particularly in Asia, said Paul
Majeski, the publisher of Music Trades.

Even so, electric guitar sales had rebounded to about 1.25 million
instruments by the end of last year after bottoming out around one
million in 2015. And in dollar terms, guitar sales have grown steadily
since the Great Recession of 2009, Music Trades reports. Last year, they
topped \$8 billion.

And that's not accounting for the market for secondhand guitars on eBay,
Craigslist and Etsy, and vintage sellers like
\href{https://reverb.com/}{Reverb}, which dwarfs retail sales at music
shops, and indicates that ``the public's interest in fretted instruments
has never been greater,'' Mr. Majeski said. (It's worth pointing out
that sales of new guitars are inherently dampened by the very durability
of the product. A quality electric guitar can last 50 years or more with
minimal care, and the classics often improve with age, many players
believe. Smartphones these aren't.)

Sure, there's still the issue of the idols. The calendar is not suddenly
running in reverse for Jeff Beck or Pete Townshend.

Maybe the issue isn't too few guitar heroes, but too many of them. As
any 30-minute foray through cover-song videos on YouTube will attest,
there are approximately 1,000,000,007 much-better-than-average
guitarists out there, many of whom are in their teens or early 20s.

A great many of them are tearing through
\href{https://www.youtube.com/watch?v=QUOn6YY6Kyc}{Hendrix},
\href{https://www.youtube.com/watch?v=1t9h-TlKF4k}{Eddie Van Halen} or
\href{https://www.youtube.com/watch?v=1t9h-TlKF4k}{Jimmy Page} licks.
And a great many of them positively shred.

In other words, you could argue that the guitar god is dead. You could
also argue that the guitar gods did their job.

Advertisement

\protect\hyperlink{after-bottom}{Continue reading the main story}

\hypertarget{site-index}{%
\subsection{Site Index}\label{site-index}}

\hypertarget{site-information-navigation}{%
\subsection{Site Information
Navigation}\label{site-information-navigation}}

\begin{itemize}
\tightlist
\item
  \href{https://help.nytimes3xbfgragh.onion/hc/en-us/articles/115014792127-Copyright-notice}{©~2020~The
  New York Times Company}
\end{itemize}

\begin{itemize}
\tightlist
\item
  \href{https://www.nytco.com/}{NYTCo}
\item
  \href{https://help.nytimes3xbfgragh.onion/hc/en-us/articles/115015385887-Contact-Us}{Contact
  Us}
\item
  \href{https://www.nytco.com/careers/}{Work with us}
\item
  \href{https://nytmediakit.com/}{Advertise}
\item
  \href{http://www.tbrandstudio.com/}{T Brand Studio}
\item
  \href{https://www.nytimes3xbfgragh.onion/privacy/cookie-policy\#how-do-i-manage-trackers}{Your
  Ad Choices}
\item
  \href{https://www.nytimes3xbfgragh.onion/privacy}{Privacy}
\item
  \href{https://help.nytimes3xbfgragh.onion/hc/en-us/articles/115014893428-Terms-of-service}{Terms
  of Service}
\item
  \href{https://help.nytimes3xbfgragh.onion/hc/en-us/articles/115014893968-Terms-of-sale}{Terms
  of Sale}
\item
  \href{https://spiderbites.nytimes3xbfgragh.onion}{Site Map}
\item
  \href{https://help.nytimes3xbfgragh.onion/hc/en-us}{Help}
\item
  \href{https://www.nytimes3xbfgragh.onion/subscription?campaignId=37WXW}{Subscriptions}
\end{itemize}
