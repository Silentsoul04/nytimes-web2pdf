Sections

SEARCH

\protect\hyperlink{site-content}{Skip to
content}\protect\hyperlink{site-index}{Skip to site index}

\href{https://myaccount.nytimes3xbfgragh.onion/auth/login?response_type=cookie\&client_id=vi}{}

\href{https://www.nytimes3xbfgragh.onion/section/todayspaper}{Today's
Paper}

\href{/section/upshot}{The Upshot}\textbar{}It's Not Easy to Get a
Coronavirus Test for a Child

\url{https://nyti.ms/35lSVTX}

\begin{itemize}
\item
\item
\item
\item
\item
\item
\end{itemize}

\hypertarget{school-reopenings}{%
\subsubsection{\texorpdfstring{\href{https://www.nytimes3xbfgragh.onion/spotlight/schools-reopening?name=styln-coronavirus-schools-reopening\&region=TOP_BANNER\&block=storyline_menu_recirc\&action=click\&pgtype=Article\&impression_id=a24d4900-f279-11ea-a8e6-7d6a1fa58798\&variant=undefined}{School
Reopenings}}{School Reopenings}}\label{school-reopenings}}

\begin{itemize}
\tightlist
\item
  \href{https://www.nytimes3xbfgragh.onion/2020/09/04/us/bar-exam-coronavirus.html?name=styln-coronavirus-schools-reopening\&region=TOP_BANNER\&block=storyline_menu_recirc\&action=click\&pgtype=Article\&impression_id=a24d7010-f279-11ea-a8e6-7d6a1fa58798\&variant=undefined}{Delayed
  Licensing Exams}
\item
  \href{https://www.nytimes3xbfgragh.onion/2020/09/08/upshot/children-testing-shortfalls-virus.html?name=styln-coronavirus-schools-reopening\&region=TOP_BANNER\&block=storyline_menu_recirc\&action=click\&pgtype=Article\&impression_id=a24d7011-f279-11ea-a8e6-7d6a1fa58798\&variant=undefined}{Limited
  Testing for Children}
\item
  \href{https://www.nytimes3xbfgragh.onion/2020/09/01/world/schools-reopen-globe-students.html?name=styln-coronavirus-schools-reopening\&region=TOP_BANNER\&block=storyline_menu_recirc\&action=click\&pgtype=Article\&impression_id=a24d7012-f279-11ea-a8e6-7d6a1fa58798\&variant=undefined}{School
  Around the World}
\item
  \href{https://www.nytimes3xbfgragh.onion/interactive/2020/us/covid-college-cases-tracker.html?name=styln-coronavirus-schools-reopening\&region=TOP_BANNER\&block=storyline_menu_recirc\&action=click\&pgtype=Article\&impression_id=a24d9720-f279-11ea-a8e6-7d6a1fa58798\&variant=undefined}{Tracking
  College Cases}
\end{itemize}

Advertisement

\protect\hyperlink{after-top}{Continue reading the main story}

Upshot

Supported by

\protect\hyperlink{after-sponsor}{Continue reading the main story}

\hypertarget{its-not-easy-to-get-a-coronavirus-test-for-a-child}{%
\section{It's Not Easy to Get a Coronavirus Test for a
Child}\label{its-not-easy-to-get-a-coronavirus-test-for-a-child}}

As schools reopen, many parents still can't find a test nearby, impeding
the fight against the virus.

\href{https://www.nytimes3xbfgragh.onion/by/sarah-kliff}{\includegraphics{https://static01.graylady3jvrrxbe.onion/images/2020/08/25/reader-center/author-sarah-kliff/author-sarah-kliff-thumbLarge.png}}\href{https://www.nytimes3xbfgragh.onion/by/margot-sanger-katz}{\includegraphics{https://static01.graylady3jvrrxbe.onion/images/2019/12/13/reader-center/author-margot-sanger-katz/author-margot-sanger-katz-thumbLarge.png}}

By \href{https://www.nytimes3xbfgragh.onion/by/sarah-kliff}{Sarah Kliff}
and
\href{https://www.nytimes3xbfgragh.onion/by/margot-sanger-katz}{Margot
Sanger-Katz}

\begin{itemize}
\item
  Sept. 8, 2020
\item
  \begin{itemize}
  \item
  \item
  \item
  \item
  \item
  \item
  \end{itemize}
\end{itemize}

\includegraphics{https://static01.graylady3jvrrxbe.onion/images/2020/09/04/upshot/00up-virus-children-tests/merlin_176309922_70d41a70-e906-46fd-8366-eb96f98e4468-articleLarge.jpg?quality=75\&auto=webp\&disable=upscale}

When Audrey Blute's almost 2-year-old son, George, had a runny nose in
July, she wanted to do what she felt was responsible: get him tested for
coronavirus.

It wasn't easy.

Ms. Blute, 34, planned to walk to one of Washington, D.C.'s free testing
sites --- until she learned
\href{https://coronavirus.dc.gov/testing}{they do not test children
younger than 6}. She called her pediatrician's office, which also
declined to test George.

As child care centers and schools reopen, parents are encountering
another coronavirus testing bottleneck: Few sites will test children.
Even in large cities with dozens of test sites, parents are driving long
distances and calling multiple centers to track down one accepting
children.

The age policies at testing sites reflect a range of concerns, including
differences in health insurance, medical privacy rules, holes in test
approval, and fears of squirmy or shrieking children.

The limited testing hampers schools' ability to quickly isolate and
trace coronavirus cases among students. It could also create a new
burden on working parents, with some schools and child care centers
requiring symptomatic children to test negative for coronavirus before
rejoining class.

``There is no good reason not to do it in kids,'' said Sean O'Leary, a
Colorado pediatrician who sits on the American Academy of Pediatrics'
committee on infectious diseases. ``It's a matter of people not being
comfortable with doing it.''

Many testing sites, including those run by cities and states, do not
test any children, or they set age minimums that exclude young children.
The age limits vary widely from place to place. Los Angeles offers
public testing without any age minimum, while San Francisco, which
initially saw only adults, recently began offering tests to children 13
and older. Dallas sets a cutoff at
\href{https://www.dallascounty.org/covid-19/testing-locations.php}{5
years old.}

The District of Columbia decided not to test young children at its
public sites because children have nearly universal health coverage in
the city, meaning they could be tested at a pediatrician's office.

Parents like Ms. Blute, however, are finding that pediatricians' offices
appear to have limited testing capacities. George never got a test for
his runny nose. Instead, Ms. Blute and her husband kept him isolated at
home while they tried to work their full-time jobs.

``We were told to assume that everyone in the household has it, which
didn't seem like the best information --- we're both big believers in
contributing to the data pool,'' she said. ``We think that's really
important.''

In Florida, the division of emergency management announced last month
that it would ``prioritize'' pediatric testing as students there begin
to return to in-person school. Still, only a quarter of the
\href{https://www.floridadisaster.org/news-media/news/20200803-state-of-florida-prioritizes-symptomatic-vulnerable-and-pediatric-populations-at-state-supported-covid-19-testing-sites/}{60
testing sites} the agency supports will see children of all ages. The
state's 18 drive-through sites are limited to patients 5 and older, but
did recently add priority lanes for symptomatic children.

``When we first started, and there was a lack of access to testing, this
kind of triage might have made sense,'' said Daniella Levine Cava, a
county commissioner in Miami-Dade. ``Clearly it doesn't make sense in
the current environment. We know that children contract the disease, we
know that children spread the disease, and just because they are less
likely to show symptoms, that doesn't mean they pose any less of a risk
to others.''

Pediatricians say the test itself is the same when administered to a
child, although it can sometimes require additional supplies. Not all
coronavirus tests have gone through safety testing in children, and
sometimes providers use smaller swabs on toddlers.

The nonprofit group CORE runs free testing clinics in Atlanta for anyone
2 and older. Olivia Boyd helps run the clinic's testing program and said
that, as camps and day cares began reopening this summer, she began
fielding numerous calls confirming that the clinic could test children.

```I heard you test under 18. Is that true?''' she said, recounting a
typical query. ``And we'd say yes. Then, `Thank goodness.'''

Large pharmacy chains, which have set up thousands of testing sites
across the country, have generally catered to adults. Walgreens does not
see children at its drive-through clinics.

CVS Health has slowly dropped the age minimum at its 1,944 drive-through
testing sites across the country. The pharmacies initially accepted only
adult patients but dropped the age minimum to 16 in August, and are in
the process of lowering it to 12 this month.

``Because we use self-administered swabs, we've been evolving our
testing protocols as we learn more about what's possible,'' said William
Durling, a CVS spokesman. ``Twelve years old is the age that our team
felt a child could likely swab themselves.''

\href{https://www.nytimes3xbfgragh.onion/spotlight/schools-reopening?action=click\&pgtype=Article\&state=default\&region=MAIN_CONTENT_3\&context=storylines_keepup}{}

\hypertarget{school-reopenings-}{%
\subsubsection{School Reopenings ›}\label{school-reopenings-}}

\hypertarget{back-to-school}{%
\paragraph{Back to School}\label{back-to-school}}

Updated Sept. 8, 2020

The latest on how schools are reopening amid the pandemic.

\begin{itemize}
\item
  \begin{itemize}
  \tightlist
  \item
    The first day of school is an annual rite of passage. But this year,
    it looks very different for tens of millions of students.
    \href{https://www.nytimes3xbfgragh.onion/2020/09/05/us/virtual-return-to-school-covid.html?action=click\&pgtype=Article\&state=default\&region=MAIN_CONTENT_3\&context=storylines_keepup}{We
    talked to some about their hopes and fears}.
  \item
    Coronavirus cases
    \href{https://www.nytimes3xbfgragh.onion/2020/09/06/us/colleges-coronavirus-students.html?action=click\&pgtype=Article\&state=default\&region=MAIN_CONTENT_3\&context=storylines_keepup}{are
    spiking in America's college towns}, leading to concern that young
    people who are infected will contribute to a spread of the virus.
  \item
    A growing number of Catholic schools across the country are
    \href{https://www.nytimes3xbfgragh.onion/2020/09/05/us/catholic-school-closings.html?action=click\&pgtype=Article\&state=default\&region=MAIN_CONTENT_3\&context=storylines_keepup}{shutting
    down forever during the coronavirus pandemic}, citing insurmountable
    financial pressure.
  \item
    The magazine's Ethicist columnist answers a question from a
    spokesperson at a major university:
    \href{https://www.nytimes3xbfgragh.onion/2020/09/08/magazine/university-reopening-safety-ethics.html?action=click\&pgtype=Article\&state=default\&region=MAIN_CONTENT_3\&context=storylines_keepup}{Can
    I promote a reopening plan I have doubts about}?
  \end{itemize}
\end{itemize}

Early in the pandemic, public health officials were not focused on
children as an at-risk population, given how few ended up hospitalized
for the virus. Some scientists even thought that children might be safe
from coronavirus infection altogether.

But now, with schools underway, and with evidence of childhood infection
more established, the testing infrastructure for children in many
communities has major holes. Nir Menachemi, a professor of **** health
policy and management at Indiana University, called it a blind spot that
was interfering with school reopening plans and with efforts to
understand how the virus was spreading.

``Having a blind spot makes you not able to respond from a public health
perspective, either with the correct messaging or with the right
policies to put into place to protect the people who are vulnerable,''
he said.

When Christine Carter's 5-year-old son, West, was experiencing a fever
and vomiting, she worried it might be coronavirus. But her
pediatrician's office said it did those tests only on Tuesdays and
Thursdays, and all appointments that week had already been booked.

``By the time I was going to be able to get him tested, he'd already
have been a week into having it,'' said Ms. Carter, who lives outside
Baltimore. ``It turned out to be an allergic reaction, but if I do
really need to get him tested in the future, I fear the process will be
really lengthy.''

In Chicago, Jen Cowhy's pediatrician declined to test her 11-month-old
daughter after a day care classmate tested positive. Ms. Cowhy, 31,
called the city's two pediatric hospitals, and both told her they would
not test a child who had been exposed but was asymptomatic.

The limited testing sites for children reflect broader patterns in
medical care delivery. Even when it comes to more longstanding health
needs --- like flu shots, checkups or an assessment of a sore wrist ---
many clinics, urgent care centers and drugstores that offer services to
adults won't accept children. So even if workers there can technically
swab a child's nose the same way they would swab an adult's, they may
not feel comfortable doing so.

Joe Little, the clinical supervisor for coronavirus testing at the
AllCare Family Medicine and Urgent Care in Washington's Dupont Circle
neighborhood, said health workers without pediatric training sometimes
worry that children will be resistant or emotional. But nurses at his
clinic, one of the few places in the region testing young children, have
had success administering nasal swab tests to people of all ages.

``They generally tolerate it pretty well,'' said Mr. Little, who is
trained as a nurse. ``When we do it, the nurse will say: `We're going to
tickle your nose. Tickle, tickle.' And then you're doing it. And they're
like, `Oh, it didn't hurt.'''

Advertisement

\protect\hyperlink{after-bottom}{Continue reading the main story}

\hypertarget{site-index}{%
\subsection{Site Index}\label{site-index}}

\hypertarget{site-information-navigation}{%
\subsection{Site Information
Navigation}\label{site-information-navigation}}

\begin{itemize}
\tightlist
\item
  \href{https://help.nytimes3xbfgragh.onion/hc/en-us/articles/115014792127-Copyright-notice}{©~2020~The
  New York Times Company}
\end{itemize}

\begin{itemize}
\tightlist
\item
  \href{https://www.nytco.com/}{NYTCo}
\item
  \href{https://help.nytimes3xbfgragh.onion/hc/en-us/articles/115015385887-Contact-Us}{Contact
  Us}
\item
  \href{https://www.nytco.com/careers/}{Work with us}
\item
  \href{https://nytmediakit.com/}{Advertise}
\item
  \href{http://www.tbrandstudio.com/}{T Brand Studio}
\item
  \href{https://www.nytimes3xbfgragh.onion/privacy/cookie-policy\#how-do-i-manage-trackers}{Your
  Ad Choices}
\item
  \href{https://www.nytimes3xbfgragh.onion/privacy}{Privacy}
\item
  \href{https://help.nytimes3xbfgragh.onion/hc/en-us/articles/115014893428-Terms-of-service}{Terms
  of Service}
\item
  \href{https://help.nytimes3xbfgragh.onion/hc/en-us/articles/115014893968-Terms-of-sale}{Terms
  of Sale}
\item
  \href{https://spiderbites.nytimes3xbfgragh.onion}{Site Map}
\item
  \href{https://help.nytimes3xbfgragh.onion/hc/en-us}{Help}
\item
  \href{https://www.nytimes3xbfgragh.onion/subscription?campaignId=37WXW}{Subscriptions}
\end{itemize}
