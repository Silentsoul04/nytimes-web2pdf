Sections

SEARCH

\protect\hyperlink{site-content}{Skip to
content}\protect\hyperlink{site-index}{Skip to site index}

\href{https://www.nytimes3xbfgragh.onion/section/world/asia}{Asia
Pacific}

\href{https://myaccount.nytimes3xbfgragh.onion/auth/login?response_type=cookie\&client_id=vi}{}

\href{https://www.nytimes3xbfgragh.onion/section/todayspaper}{Today's
Paper}

\href{/section/world/asia}{Asia Pacific}\textbar{}`The Lockdown Killed
My Father': Farmer Suicides Add to India's Virus Misery

\url{https://nyti.ms/2ZguQK5}

\begin{itemize}
\item
\item
\item
\item
\item
\item
\end{itemize}

\hypertarget{the-coronavirus-outbreak}{%
\subsubsection{\texorpdfstring{\href{https://www.nytimes3xbfgragh.onion/news-event/coronavirus?name=styln-coronavirus-national\&region=TOP_BANNER\&block=storyline_menu_recirc\&action=click\&pgtype=Article\&impression_id=11499020-f284-11ea-ae55-8bacd3660bac\&variant=undefined}{The
Coronavirus
Outbreak}}{The Coronavirus Outbreak}}\label{the-coronavirus-outbreak}}

\begin{itemize}
\tightlist
\item
  live\href{https://www.nytimes3xbfgragh.onion/2020/09/09/world/covid-19-coronavirus.html?name=styln-coronavirus-national\&region=TOP_BANNER\&block=storyline_menu_recirc\&action=click\&pgtype=Article\&impression_id=11499021-f284-11ea-ae55-8bacd3660bac\&variant=undefined}{Latest
  Updates}
\item
  \href{https://www.nytimes3xbfgragh.onion/interactive/2020/us/coronavirus-us-cases.html?name=styln-coronavirus-national\&region=TOP_BANNER\&block=storyline_menu_recirc\&action=click\&pgtype=Article\&impression_id=11499022-f284-11ea-ae55-8bacd3660bac\&variant=undefined}{Maps
  and Cases}
\item
  \href{https://www.nytimes3xbfgragh.onion/interactive/2020/science/coronavirus-vaccine-tracker.html?name=styln-coronavirus-national\&region=TOP_BANNER\&block=storyline_menu_recirc\&action=click\&pgtype=Article\&impression_id=11499023-f284-11ea-ae55-8bacd3660bac\&variant=undefined}{Vaccine
  Tracker}
\item
  \href{https://www.nytimes3xbfgragh.onion/2020/09/02/your-money/eviction-moratorium-covid.html?name=styln-coronavirus-national\&region=TOP_BANNER\&block=storyline_menu_recirc\&action=click\&pgtype=Article\&impression_id=1149b730-f284-11ea-ae55-8bacd3660bac\&variant=undefined}{Eviction
  Moratorium}
\item
  \href{https://www.nytimes3xbfgragh.onion/interactive/2020/09/02/magazine/food-insecurity-hunger-us.html?name=styln-coronavirus-national\&region=TOP_BANNER\&block=storyline_menu_recirc\&action=click\&pgtype=Article\&impression_id=1149b731-f284-11ea-ae55-8bacd3660bac\&variant=undefined}{American
  Hunger}
\end{itemize}

Advertisement

\protect\hyperlink{after-top}{Continue reading the main story}

Supported by

\protect\hyperlink{after-sponsor}{Continue reading the main story}

\hypertarget{the-lockdown-killed-my-father-farmer-suicides-add-to-indias-virus-misery}{%
\section{`The Lockdown Killed My Father': Farmer Suicides Add to India's
Virus
Misery}\label{the-lockdown-killed-my-father-farmer-suicides-add-to-indias-virus-misery}}

Farm bankruptcies and debts have been the source of misery in India for
decades. But experts say the suffering has reached new levels in the
pandemic.

\includegraphics{https://static01.graylady3jvrrxbe.onion/images/2020/08/18/world/00india-farm-top/00india-farm-top-articleLarge.jpg?quality=75\&auto=webp\&disable=upscale}

\href{https://www.nytimes3xbfgragh.onion/by/karan-deep-singh}{\includegraphics{https://static01.graylady3jvrrxbe.onion/images/2019/12/02/reader-center/author-karan-deep-singh/author-karan-deep-singh-thumbLarge.png}}

Photographs and Text by
\href{https://www.nytimes3xbfgragh.onion/by/karan-deep-singh}{Karan Deep
Singh}

\begin{itemize}
\item
  Sept. 8, 2020
\item
  \begin{itemize}
  \item
  \item
  \item
  \item
  \item
  \item
  \end{itemize}
\end{itemize}

SIRSIWALA, India --- Randhir Singh was already deeply in debt when the
coronavirus pandemic struck. Looking out at his paltry cotton field by
the side of a railway track, he walked in circles, hopeless. In early
May, he killed himself by lying on the same track.

``This is what we feared,'' said Rashpal Singh, Mr. Singh's 22-year-old
son, choking back tears in his family home in Sirsiwala, a small village
in the northern Indian state of Punjab. ``The lockdown killed my
father.''

Months ago, when Prime Minister Narendra Modi imposed one of the world's
strictest lockdowns to prevent the spread of the coronavirus, Mr.
Singh's livelihood came crashing down. His one-acre farm had barely
produced enough cotton to cover the cost of growing it, and the lockdown
even robbed him of his side job as a bus driver.

\href{https://www.nytimes3xbfgragh.onion/2020/08/28/world/asia/india-coronavirus.html}{India
now leads the world in new daily reported coronavirus cases} and has the
second-highest number of cases globally,
\href{https://www.nytimes3xbfgragh.onion/2020/09/07/world/covid-19-coronavirus.html}{surpassing
Brazil on Monday}. In Punjab, where cases have surged,
\href{https://www.nytimes3xbfgragh.onion/2020/07/16/world/asia/coronavirus-india-million-cases.html}{lockdowns
have been imposed all over again}. The measures, economists say, are
\href{https://link.springer.com/content/pdf/10.1007/s40812-020-00170-x.pdf}{forcing
millions of households into poverty} and contributing to a long-running
tragedy: farmer suicides.

Farm bankruptcies and debts like the one that tormented Mr. Singh have
been the source of misery in the country for decades, but experts say
the suffering has reached new levels in the pandemic.

``This crisis is the making of this government,'' said Vikas Rawal, a
professor of economics at the Jawaharlal Nehru University in New Delhi,
the capital. Mr. Rawal, who has spent the last 25 years studying
agrarian distress in India, said that he believes thousands of people
who live and work on farms have most likely killed themselves in the
last few months.

\includegraphics{https://static01.graylady3jvrrxbe.onion/images/2020/08/17/world/00india-farm-02/merlin_175808238_34bbd5f5-794f-4ac3-ac5e-fd56b3ae53ab-articleLarge.jpg?quality=75\&auto=webp\&disable=upscale}

Image

Farm workers tending to rice paddies in Goslan, India. Desperate for
money after months of lockdown, some farm laborers have been demanding
as much as three times the typical wage.~~

After India's lockdown was extended for the third time, Mr. Singh became
convinced he would never pull himself out of debt with the economy shut
down, his family said. ``He kept saying, `It won't open now,''' said
Paramjeet Kaur, his widow, wiping away tears. ``Now, what will happen to
us? Who will feed us?''

India has one of the
\href{https://apps.who.int/iris/bitstream/handle/10665/131056/9789241564878_eng.pdf?sequence=8}{highest
suicide rates in the world}. In 2019, a total of 10,281 farmers and farm
laborers died by suicide across the country, according to statistics
from the
\href{https://ncrb.gov.in/sites/default/files/ADSI-2019-FULL-REPORT.pdf}{National
Crime Records Bureau}. Taking one's own life is still a crime in India,
and experts have said for years that the actual numbers are far higher
because most people fear the stigma of reporting.

\hypertarget{latest-updates-the-coronavirus-outbreak}{%
\section{\texorpdfstring{\href{https://www.nytimes3xbfgragh.onion/2020/09/09/world/covid-19-coronavirus.html?action=click\&pgtype=Article\&state=default\&region=MAIN_CONTENT_1\&context=storylines_live_updates}{Latest
Updates: The Coronavirus
Outbreak}}{Latest Updates: The Coronavirus Outbreak}}\label{latest-updates-the-coronavirus-outbreak}}

Updated 2020-09-09T10:04:09.931Z

\begin{itemize}
\tightlist
\item
  \href{https://www.nytimes3xbfgragh.onion/2020/09/09/world/covid-19-coronavirus.html?action=click\&pgtype=Article\&state=default\&region=MAIN_CONTENT_1\&context=storylines_live_updates\#link-70cea8bb}{As
  drugmakers pledge to thoroughly vet a vaccine, one company pauses its
  trials for a safety review.}
\item
  \href{https://www.nytimes3xbfgragh.onion/2020/09/09/world/covid-19-coronavirus.html?action=click\&pgtype=Article\&state=default\&region=MAIN_CONTENT_1\&context=storylines_live_updates\#link-780eaa2f}{Britain
  is expected to ban gatherings of more than six people.}
\item
  \href{https://www.nytimes3xbfgragh.onion/2020/09/09/world/covid-19-coronavirus.html?action=click\&pgtype=Article\&state=default\&region=MAIN_CONTENT_1\&context=storylines_live_updates\#link-11cec4c0}{Quarantine
  breakdowns at colleges in the U.S. are leaving some at risk.}
\end{itemize}

\href{https://www.nytimes3xbfgragh.onion/2020/09/09/world/covid-19-coronavirus.html?action=click\&pgtype=Article\&state=default\&region=MAIN_CONTENT_1\&context=storylines_live_updates}{See
more updates}

More live coverage:
\href{https://www.nytimes3xbfgragh.onion/live/2020/09/08/business/stock-market-today-coronavirus?action=click\&pgtype=Article\&state=default\&region=MAIN_CONTENT_1\&context=storylines_live_updates}{Markets}

Few of the recent examples among farmers have been
\href{https://www.indiatoday.in/india/story/maharashtra-s-amravati-division-records-206-farmer-suicides-in-march-may-1703036-2020-07-22}{reported
in
the}\href{https://timesofindia.indiatimes.com/city/mumbai/1074-farmers-ended-lives-in-maharashtra-in-6-months/articleshow/77404297.cms}{Indian
news media}, according to Mr. Rawal. ``It's hard to say exactly how many
because there was massive underreporting of deaths, and even the media
could not reach the hinterland because of the lockdown,'' he said.

A spokeswoman from the Ministry of Agriculture in New Delhi declined to
answer questions about farmer suicides. The office of the chief minister
of Punjab also declined to comment, citing the demands of the
coronavirus crisis.

Image

Every season, migrant workers from the northern states of Uttar Pradesh
and Bihar travel hundreds of miles to work at the world's largest grain
market in Khanna, India.

Image

Sacks of shelled corn at the market in Khanna.

Over the last five years, farmer suicides in Punjab increased by more
than 12 times,
\href{https://ncrb.gov.in/sites/default/files/adsi-2014\%20full\%20report.pdf}{according
to government data}. Three to four farm deaths are reported in the local
news almost every day.

The state's lush green fields that stretch all the way into the horizon
mask decades of crippling debt and abuse of land. In the 1960s, the
government introduced the high-yielding varieties of rice and wheat that
eventually made India self-sufficient in grains. But over the years,
groundwater dropped to critical levels.

Farmers, struggling to save their crops, dug their bore wells even
deeper. And to fend off increasing pest attacks, they loaded their
fields with chemicals. The skyrocketing agricultural costs forced many
farmers to take on more debt, and crop failures over the years
eventually destroyed generations of rural families.

Twenty years ago, Nirmal Singh's father drank a bottle of pesticide when
he lost most of the land he owned to a huge debt of nearly two million
rupees, about \$26,700. Then Mr. Singh's sister took her own life
because the family could not afford to bear the expenses for her
wedding.

In 2016, Mr. Singh's son died by putting himself in the path of a train
after their cotton fields were devoured by whiteflies. ``He was just
23,'' said Mr. Singh, pointing to a framed portrait of his son.

Image

Migrant workers from Hardoi, India, rode on the back of a tractor full
of paddy stalks.

Image

Bijay Kumar, a worker from Bihar, used a flashlight to prepare a meal of
potato curry and rice.

Mr. Singh is trapped under a punishing debt of \$20,000 that he
accumulated over the years to keep his farm running. But farming, he
said, is more unprofitable than ever. On a sweltering June afternoon, he
walked gingerly through his parched fields. ``Have you ever heard of a
politician or an industrialist committing suicide?'' he asked. ``It's
always a farmer or a laborer.''

In his village alone, a suicide takes place almost every month, he said.
``We are left with no tears,'' he said. ``It has turned our hearts to
stone.''

Mr. Singh says he is spending even more money to run his farm these days
because Mr. Modi's government raised fuel prices in the middle of the
pandemic, citing the costs of the lockdown. ``Modi promised `better
days,' but he has only brought the worst days so far,'' said Mr. Singh,
adding that fertilizers and pesticides prices have also increased under
Mr. Modi.

Image

``The lockdowns have destroyed us,'' said Nirmal Singh, a farmer.
``Every month someone commits suicide here.''

Image

A young man praying in Goslan. Scarred by death and misery, a young
generation of Punjabis is looking to leave rural life behind in the hope
for a better future.~

When farmers in Punjab began sowing rice in the pandemic, they had no
access to farm labor. They scrambled to arrange and pay for buses,
tractors --- whatever they could find --- to bring in workers who
typically traveled from the northern states of Bihar and Uttar Pradesh
every summer.

\href{https://www.nytimes3xbfgragh.onion/news-event/coronavirus?action=click\&pgtype=Article\&state=default\&region=MAIN_CONTENT_3\&context=storylines_faq}{}

\hypertarget{the-coronavirus-outbreak-}{%
\subsubsection{The Coronavirus Outbreak
›}\label{the-coronavirus-outbreak-}}

\hypertarget{frequently-asked-questions}{%
\paragraph{Frequently Asked
Questions}\label{frequently-asked-questions}}

Updated September 4, 2020

\begin{itemize}
\item ~
  \hypertarget{what-are-the-symptoms-of-coronavirus}{%
  \paragraph{What are the symptoms of
  coronavirus?}\label{what-are-the-symptoms-of-coronavirus}}

  \begin{itemize}
  \tightlist
  \item
    In the beginning, the coronavirus
    \href{https://www.nytimes3xbfgragh.onion/article/coronavirus-facts-history.html?action=click\&pgtype=Article\&state=default\&region=MAIN_CONTENT_3\&context=storylines_faq\#link-6817bab5}{seemed
    like it was primarily a respiratory illness}~--- many patients had
    fever and chills, were weak and tired, and coughed a lot, though
    some people don't show many symptoms at all. Those who seemed
    sickest had pneumonia or acute respiratory distress syndrome and
    received supplemental oxygen. By now, doctors have identified many
    more symptoms and syndromes. In April,
    \href{https://www.nytimes3xbfgragh.onion/2020/04/27/health/coronavirus-symptoms-cdc.html?action=click\&pgtype=Article\&state=default\&region=MAIN_CONTENT_3\&context=storylines_faq}{the
    C.D.C. added to the list of early signs}~sore throat, fever, chills
    and muscle aches. Gastrointestinal upset, such as diarrhea and
    nausea, has also been observed. Another telltale sign of infection
    may be a sudden, profound diminution of one's
    \href{https://www.nytimes3xbfgragh.onion/2020/03/22/health/coronavirus-symptoms-smell-taste.html?action=click\&pgtype=Article\&state=default\&region=MAIN_CONTENT_3\&context=storylines_faq}{sense
    of smell and taste.}~Teenagers and young adults in some cases have
    developed painful red and purple lesions on their fingers and toes
    --- nicknamed ``Covid toe'' --- but few other serious symptoms.
  \end{itemize}
\item ~
  \hypertarget{why-is-it-safer-to-spend-time-together-outside}{%
  \paragraph{Why is it safer to spend time together
  outside?}\label{why-is-it-safer-to-spend-time-together-outside}}

  \begin{itemize}
  \tightlist
  \item
    \href{https://www.nytimes3xbfgragh.onion/2020/05/15/us/coronavirus-what-to-do-outside.html?action=click\&pgtype=Article\&state=default\&region=MAIN_CONTENT_3\&context=storylines_faq}{Outdoor
    gatherings}~lower risk because wind disperses viral droplets, and
    sunlight can kill some of the virus. Open spaces prevent the virus
    from building up in concentrated amounts and being inhaled, which
    can happen when infected people exhale in a confined space for long
    stretches of time, said Dr. Julian W. Tang, a virologist at the
    University of Leicester.
  \end{itemize}
\item ~
  \hypertarget{why-does-standing-six-feet-away-from-others-help}{%
  \paragraph{Why does standing six feet away from others
  help?}\label{why-does-standing-six-feet-away-from-others-help}}

  \begin{itemize}
  \tightlist
  \item
    The coronavirus spreads primarily through droplets from your mouth
    and nose, especially when you cough or sneeze. The C.D.C., one of
    the organizations using that measure,
    \href{https://www.nytimes3xbfgragh.onion/2020/04/14/health/coronavirus-six-feet.html?action=click\&pgtype=Article\&state=default\&region=MAIN_CONTENT_3\&context=storylines_faq}{bases
    its recommendation of six feet}~on the idea that most large droplets
    that people expel when they cough or sneeze will fall to the ground
    within six feet. But six feet has never been a magic number that
    guarantees complete protection. Sneezes, for instance, can launch
    droplets a lot farther than six feet,
    \href{https://jamanetwork.com/journals/jama/fullarticle/2763852}{according
    to a recent study}. It's a rule of thumb: You should be safest
    standing six feet apart outside, especially when it's windy. But
    keep a mask on at all times, even when you think you're far enough
    apart.
  \end{itemize}
\item ~
  \hypertarget{i-have-antibodies-am-i-now-immune}{%
  \paragraph{I have antibodies. Am I now
  immune?}\label{i-have-antibodies-am-i-now-immune}}

  \begin{itemize}
  \tightlist
  \item
    As of right
    now,\href{https://www.nytimes3xbfgragh.onion/2020/07/22/health/covid-antibodies-herd-immunity.html?action=click\&pgtype=Article\&state=default\&region=MAIN_CONTENT_3\&context=storylines_faq}{~that
    seems likely, for at least several months.}~There have been
    frightening accounts of people suffering what seems to be a second
    bout of Covid-19. But experts say these patients may have a
    drawn-out course of infection, with the virus taking a slow toll
    weeks to months after initial exposure.~People infected with the
    coronavirus typically
    \href{https://www.nature.com/articles/s41586-020-2456-9}{produce}~immune
    molecules called antibodies, which are
    \href{https://www.nytimes3xbfgragh.onion/2020/05/07/health/coronavirus-antibody-prevalence.html?action=click\&pgtype=Article\&state=default\&region=MAIN_CONTENT_3\&context=storylines_faq}{protective
    proteins made in response to an
    infection}\href{https://www.nytimes3xbfgragh.onion/2020/05/07/health/coronavirus-antibody-prevalence.html?action=click\&pgtype=Article\&state=default\&region=MAIN_CONTENT_3\&context=storylines_faq}{.
    These antibodies may}~last in the body
    \href{https://www.nature.com/articles/s41591-020-0965-6}{only two to
    three months}, which may seem worrisome, but that's~perfectly normal
    after an acute infection subsides, said Dr. Michael Mina, an
    immunologist at Harvard University. It may be possible to get the
    coronavirus again, but it's highly unlikely that it would be
    possible in a short window of time from initial infection or make
    people sicker the second time.
  \end{itemize}
\item ~
  \hypertarget{what-are-my-rights-if-i-am-worried-about-going-back-to-work}{%
  \paragraph{What are my rights if I am worried about going back to
  work?}\label{what-are-my-rights-if-i-am-worried-about-going-back-to-work}}

  \begin{itemize}
  \tightlist
  \item
    Employers have to provide
    \href{https://www.osha.gov/SLTC/covid-19/standards.html}{a safe
    workplace}~with policies that protect everyone equally.
    \href{https://www.nytimes3xbfgragh.onion/article/coronavirus-money-unemployment.html?action=click\&pgtype=Article\&state=default\&region=MAIN_CONTENT_3\&context=storylines_faq}{And
    if one of your co-workers tests positive for the coronavirus, the
    C.D.C.}~has said that
    \href{https://www.cdc.gov/coronavirus/2019-ncov/community/guidance-business-response.html}{employers
    should tell their employees}~-\/- without giving you the sick
    employee's name -\/- that they may have been exposed to the virus.
  \end{itemize}
\end{itemize}

Desperate and jobless for nearly three months because of the lockdown,
the workers demanded double and triple their usual rates.

In the early days of the lockdown, farmers were so constricted that they
were only able to bring
\href{https://coronapolicyimpact.org/wp-content/uploads/2020/04/sserwp2001-2.pdf}{a
small fraction of their produce} to the market. Unable to sell their
crops, they
\href{https://www.ndtv.com/india-news/coronavirus-no-mill-amid-lockdown-punjab-farmer-burns-sugarcane-worth-rs-5-lakh-2255613}{set
their farms on fire} and dumped millions of dollars worth of fruits and
vegetables on the roads or plowed them back into the fields.

Leela Singh, a farmer in Akanwali village, feared his farm would be
seized and tried to borrow a few thousand rupees, about \$100, to help
him stay afloat. Unable to secure the loan, he hanged himself in June,
said Gurpreet Singh, his 24-year-old son, who dropped out of school so
the family could save on tuition fees. ``We are now having to beg for
money from someone or the other,'' he said.

``He just wanted to save his farm,'' he added.

Image

The pandemic and its economic disruption have put millions of Indians
out of work.

Image

Farmers are turning to chemicals to save their crops, said Gurcharan
Singh, who runs a pesticide store in Khamanon, India.

In early June, Mr. Modi's government used its executive powers to push
through
\href{https://pib.gov.in/PressReleasePage.aspx?PRID=1629033}{sweeping
changes} aimed at privatizing agriculture. It promised farmers greater
freedom to sell their produce outside large agricultural markets taxed
by state governments.

In August, thousands of farmers gathered to protest the new orders,
\href{https://twitter.com/ramanmann1974/status/1294625753363910657?s=20}{burning
their copies in the street} and arguing the orders could expose them to
a monopoly of corporate buyers rather than empowering them.

On a recent afternoon in Nirmal Singh's village, dozens of women and
children led a procession to mark an ancient ritual: the funeral of a
doll made of dry twigs and wrapped in fine silk. It is believed the
funeral forces the gods to unleash rain and ease suffering on earth.

``Look what you have done to our daughter,'' the women sing in unison,
some grieving, beating their breasts and throwing their hands up in the
air. After the ceremony, it began to rain. The ritual worked, said Mr.
Singh. Some of their suffering had been relieved.

``Now, we just hope Modi gets the message.''

Image

Women of Sirsiwala gathered for an ancient ritual believed to force the
gods to unleash rain and ease suffering.~

\emph{If you are having thoughts of suicide, call the U.S. National
Suicide Prevention Lifeline at 800-273-8255 (TALK) or go to}
\href{http://speakingofsuicide.com/resources}{\emph{SpeakingOfSuicide.com/resources}}
\emph{for a list of additional resources.}

Advertisement

\protect\hyperlink{after-bottom}{Continue reading the main story}

\hypertarget{site-index}{%
\subsection{Site Index}\label{site-index}}

\hypertarget{site-information-navigation}{%
\subsection{Site Information
Navigation}\label{site-information-navigation}}

\begin{itemize}
\tightlist
\item
  \href{https://help.nytimes3xbfgragh.onion/hc/en-us/articles/115014792127-Copyright-notice}{©~2020~The
  New York Times Company}
\end{itemize}

\begin{itemize}
\tightlist
\item
  \href{https://www.nytco.com/}{NYTCo}
\item
  \href{https://help.nytimes3xbfgragh.onion/hc/en-us/articles/115015385887-Contact-Us}{Contact
  Us}
\item
  \href{https://www.nytco.com/careers/}{Work with us}
\item
  \href{https://nytmediakit.com/}{Advertise}
\item
  \href{http://www.tbrandstudio.com/}{T Brand Studio}
\item
  \href{https://www.nytimes3xbfgragh.onion/privacy/cookie-policy\#how-do-i-manage-trackers}{Your
  Ad Choices}
\item
  \href{https://www.nytimes3xbfgragh.onion/privacy}{Privacy}
\item
  \href{https://help.nytimes3xbfgragh.onion/hc/en-us/articles/115014893428-Terms-of-service}{Terms
  of Service}
\item
  \href{https://help.nytimes3xbfgragh.onion/hc/en-us/articles/115014893968-Terms-of-sale}{Terms
  of Sale}
\item
  \href{https://spiderbites.nytimes3xbfgragh.onion}{Site Map}
\item
  \href{https://help.nytimes3xbfgragh.onion/hc/en-us}{Help}
\item
  \href{https://www.nytimes3xbfgragh.onion/subscription?campaignId=37WXW}{Subscriptions}
\end{itemize}
