Sections

SEARCH

\protect\hyperlink{site-content}{Skip to
content}\protect\hyperlink{site-index}{Skip to site index}

\href{https://www.nytimes3xbfgragh.onion/section/books/review}{Book
Review}

\href{https://myaccount.nytimes3xbfgragh.onion/auth/login?response_type=cookie\&client_id=vi}{}

\href{https://www.nytimes3xbfgragh.onion/section/todayspaper}{Today's
Paper}

\href{/section/books/review}{Book Review}\textbar{}Doctor Dolittle's
Talking Animals Still Have Much to Say

\url{https://nyti.ms/2F5KrFM}

\begin{itemize}
\item
\item
\item
\item
\item
\end{itemize}

Advertisement

\protect\hyperlink{after-top}{Continue reading the main story}

Supported by

\protect\hyperlink{after-sponsor}{Continue reading the main story}

essay

\hypertarget{doctor-dolittles-talking-animals-still-have-much-to-say}{%
\section{Doctor Dolittle's Talking Animals Still Have Much to
Say}\label{doctor-dolittles-talking-animals-still-have-much-to-say}}

\includegraphics{https://static01.graylady3jvrrxbe.onion/images/2020/09/13/books/13TRAUB-COMBO/15TRAUB-COMBO-articleLarge.jpg?quality=75\&auto=webp\&disable=upscale}

By James Traub

\begin{itemize}
\item
  Sept. 9, 2020
\item
  \begin{itemize}
  \item
  \item
  \item
  \item
  \item
  \end{itemize}
\end{itemize}

Doctor Dolittle, the hero of Hugh Lofting's children's series about an
English country doctor who learned to speak the language of animals,
turns 100 this year. My own acquaintance with the doctor dates back to
1963, when, at age 9, I triumphantly completed ``The Voyages of Doctor
Dolittle,'' my first big book, weighing in at 364 pages --- a jumbo size
for any work for children. I caught up with the doctor again this
spring, when school went virtual. I had been reading high-minded
founding father biographies to third graders at the Academy of the City,
a charter school in Queens, where I serve on the board. I felt that the
kids needed an escape from a reality turned dismal, and I thought of the
good doctor and his talking parrot, dog, pig, monkey and pushmi-pullyu.

We finished the first book in the series, ``The Story of Doctor
Dolittle.'' Say what you will about Zoom, but I can report that the kids
were transfixed. Their questions hinted at their degree of imaginative
immersion: ``How did the monkeys get back to the ground after they made
the bridge with their arms?'' ``Does the pushmi-pullyu actually have two
heads?''

``The Story of Doctor Dolittle'' appeared in 1920, and was republished
almost annually thereafter, as were many of the 11 other books in the
series. In a preface to the 1922 edition, the novelist Hugh Walpole
called the book ``a work of genius'' and ``the first real children's
classic since `Alice.''' Yet almost everyone knows about Alice, and
Pooh, and Peter Rabbit. If it weren't for the movie versions --- first
starring Rex Harrison, then Eddie Murphy and, this past winter,
\href{https://www.nytimes3xbfgragh.onion/2020/01/15/movies/dolittle-review.html?searchResultPosition=2}{Robert
Downey Jr}. --- Doctor Dolittle's name might be remembered no better
than Walpole's own. I didn't read Lofting's books to my son; most of you
probably didn't either. The doctor's centennial has gone unnoticed. What
happened?

No one could say that the books have grown quaint or stale; just ask my
third graders. Nor was Walpole indulging in hyperbole. Doctor Dolittle
is a wonderful creation: a Victorian eccentric from the pages of
Dickens; a perpetual bachelor who drives conventional humans from his
life but is much loved by the poor and the marginal; a gentleman whose
exquisite politesse never falters, even before sharks and pirates; a
peace-loving naturalist prepared to wage war to defend his friends from
evil depredations. Only by the standards of the world of grown-ups does
he ``do little.''

Lofting can hit many registers, but he saves the lyrical for the animals
themselves, who experience life as fully as we do, though you'd never
know it if you can't understand them. Here is Clippa, a fidgit --- a
small fish --- who has been imprisoned in an aquarium along with her
brother, and who mourns her vanished life with a depth of feeling
unknown to the Little Mermaid and her friends: ``To chase the shrimps on
a summer evening, when the sky is red and the light's all pink within
the foam! To lie on the top, in the doldrums' noonday calm, and warm
your tummy in the tropic sun! To wander hand in hand once more through
the giant seaweed forests of the Indian Ocean, seeking the delicious
eggs of the pop-pop!'' And then the poor thing collapses in sobs.

Lofting really was a genius of children's literature. But he was also a
product of the British Empire. When Doctor Dolittle goes to Africa to
cure the monkeys, he stumbles into the Kingdom of Jolliginki. Prince
Bumpo, the heir to the throne, is a mooncalf who mistakes fairy tales
for real life, speaks in Elizabethan periphrasis and murmurs to himself:
``If only I were a \emph{white} prince!'' In the pencil sketches with
which Lofting illustrates his texts, Prince Bumpo looks like the missing
link between man and ape. Lofting's biographer, Gary D. Schmidt,
defensively notes that Doctor Dolittle himself rarely utters a bigoted
word. But the doctor is only a character; the narrator and the
illustrator are none other than our author. While Lofting never fails to
give his Africans a measure of nobility, he is also quite certain of
their savagery.

The edition I read was probably published in 1950, three years after
Lofting's death. By the 1970s, he had gone into eclipse. Over the years,
new editions appeared that attempted to address the racism, including
one in 1988 from which all pictures of Prince Bumpo and his parents had
been removed, along with all references to their skin color, not to
mention their wish to change it. ``If this verbal and visual caution
occasionally seems almost craven,'' a reviewer for The New York Times
Book Review wrote, the blind spots for which it sought to compensate
were real.

Lofting's own story is almost as remarkable as the doctor's. Though we
might imagine a donnish Lewis Carroll or C. S. Lewis as the author of
such twee fables, Lofting was a wanderer and an adventurer, a civil
engineer who prospected for gold in Canada and built railroads in
Nigeria and Cuba before settling in the United States and starting a
family in 1912. When the war broke out he returned to England to enlist,
and was sent to the trenches in France and Flanders. His children begged
for letters, with drawings. Lofting would not relate the unspeakable
truth. He had observed, as he wrote many years later, that the animals
serving alongside the soldiers had, like them, become ``fatalists,''
trudging into the same hail of artillery fire. But when a horse was
wounded, it wasn't sent to the dispensary; it was dispatched with a
bullet. This was cruel. Lofting imagined that we would spare animals if
only we could see inside them, as we can our fellow humans. And so he
wrote letters home about talking animals. These letters formed the basis
of ``The Story of Doctor Dolittle.''

Because he \emph{does} understand animals, Doctor Dolittle comes to
recognize their astonishing gifts of smell, sight, hearing. The animals
are the books' heroes every bit as much as the doctor himself; it is
they who miraculously find lost and starving men or turn back a
marauding tribe. The doctor loves them as they deserve to be loved, and
protects them from abuse, just as his creator dreamed of doing --- for
all that he internalized the racist human hierarchy of his day. In ``The
Voyages of Doctor Dolittle,'' the doctor offers to step into a bullring
and outperform a great matador, on the condition that the local
authorities agree to end bullfighting forever should he win. Of course
they accept the lunatic wager. The good doctor arranges everything with
the bulls beforehand: They charge straight at him before dropping to the
ground in front of him or letting him perform acrobatics on their horns.
The great matador gnashes his teeth while the \emph{señoritas} throw
flowers and jewels at the doctor's feet.

Even the very young reader will not miss the moral anger beneath the
whimsy. Lofting was no Kipling. The experience of the trenches turned
him against war and the glorification of combat, including in children's
books. In 1942 he risked his reputation by publishing ``Victory for the
Slain,'' an epic poem deploring the war in which England was already
enmeshed. He aspired to be a novelist, a journalist, a moralizing
essayist; owing to the peculiar bent of his genius, he was to achieve
all that through the fidgit --- and, of course, the portly gentleman in
the waistcoat and battered top hat.

Unlike his creator, Doctor Dolittle is, in fact, a man for our time.
When he finds the citizens of the Monkey Kingdom suffering from an
infectious virus, he spends three days and three nights vaccinating the
healthy and places the sick in quarantine for 14 days. They all recover.

Advertisement

\protect\hyperlink{after-bottom}{Continue reading the main story}

\hypertarget{site-index}{%
\subsection{Site Index}\label{site-index}}

\hypertarget{site-information-navigation}{%
\subsection{Site Information
Navigation}\label{site-information-navigation}}

\begin{itemize}
\tightlist
\item
  \href{https://help.nytimes3xbfgragh.onion/hc/en-us/articles/115014792127-Copyright-notice}{©~2020~The
  New York Times Company}
\end{itemize}

\begin{itemize}
\tightlist
\item
  \href{https://www.nytco.com/}{NYTCo}
\item
  \href{https://help.nytimes3xbfgragh.onion/hc/en-us/articles/115015385887-Contact-Us}{Contact
  Us}
\item
  \href{https://www.nytco.com/careers/}{Work with us}
\item
  \href{https://nytmediakit.com/}{Advertise}
\item
  \href{http://www.tbrandstudio.com/}{T Brand Studio}
\item
  \href{https://www.nytimes3xbfgragh.onion/privacy/cookie-policy\#how-do-i-manage-trackers}{Your
  Ad Choices}
\item
  \href{https://www.nytimes3xbfgragh.onion/privacy}{Privacy}
\item
  \href{https://help.nytimes3xbfgragh.onion/hc/en-us/articles/115014893428-Terms-of-service}{Terms
  of Service}
\item
  \href{https://help.nytimes3xbfgragh.onion/hc/en-us/articles/115014893968-Terms-of-sale}{Terms
  of Sale}
\item
  \href{https://spiderbites.nytimes3xbfgragh.onion}{Site Map}
\item
  \href{https://help.nytimes3xbfgragh.onion/hc/en-us}{Help}
\item
  \href{https://www.nytimes3xbfgragh.onion/subscription?campaignId=37WXW}{Subscriptions}
\end{itemize}
