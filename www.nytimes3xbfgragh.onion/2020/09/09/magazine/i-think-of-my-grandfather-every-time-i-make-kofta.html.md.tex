Sections

SEARCH

\protect\hyperlink{site-content}{Skip to
content}\protect\hyperlink{site-index}{Skip to site index}

\href{https://myaccount.nytimes3xbfgragh.onion/auth/login?response_type=cookie\&client_id=vi}{}

\href{https://www.nytimes3xbfgragh.onion/section/todayspaper}{Today's
Paper}

I Think of My Grandfather Every Time I Make Kofta

\url{https://nyti.ms/3hljOtz}

\begin{itemize}
\item
\item
\item
\item
\item
\item
\end{itemize}

Advertisement

\protect\hyperlink{after-top}{Continue reading the main story}

Supported by

\protect\hyperlink{after-sponsor}{Continue reading the main story}

Eat

\hypertarget{i-think-of-my-grandfather-every-time-i-make-kofta}{%
\section{I Think of My Grandfather Every Time I Make
Kofta}\label{i-think-of-my-grandfather-every-time-i-make-kofta}}

\includegraphics{https://static01.graylady3jvrrxbe.onion/images/2020/09/13/magazine/13mag-eat/13mag-eat-articleLarge.jpg?quality=75\&auto=webp\&disable=upscale}

By \href{https://www.nytimes3xbfgragh.onion/by/tejal-rao}{Tejal Rao}

\begin{itemize}
\item
  Sept. 9, 2020
\item
  \begin{itemize}
  \item
  \item
  \item
  \item
  \item
  \item
  \end{itemize}
\end{itemize}

Whatever my grandfather did, he devoted himself completely to it. He
took his time. He did it well. It didn't matter how the small the job
--- cutting a melon for dessert; ironing shirts and trousers and, to my
embarrassment, even underpants; wrapping gifts with paper by folding
perfect lines; parallel parking on the curb in front of the fish market.
Everything, to him, was worth doing properly, carefully, thoughtfully,
perfectly.

I lived with him and my grandmother most summers when I was a child, and
the way he took care with the utterly mundane --- his habit of low speed
and high quality --- was infuriating. Mostly, I think, because I
couldn't cultivate the same habit myself. Or I didn't see the point. I
wanted all those tiny, tiny things that he took time to do so well not
to matter at all. If he gave me a task, and I didn't do it correctly,
he'd follow up behind me, resetting the table or refolding the laundry
just the way he showed me to do in the first place.

``No one cares!'' I insisted. ``You're doing all this work for
nothing!'' My misdirected fury made him laugh, warmly, because somehow
he was kind and gentle and patient even while correcting me. It was, I
thought as a child, borderline deranged.

This is the grandfather who gave me my name when I was born. He wrote it
on a piece of paper in his ornate cursive and mailed it to my mother in
London. By the time I knew him, he was running the ice cream company he
founded in Nairobi, and he let me visit him at the office as often as I
liked, which was often. I'd sit by his desk and eat tutti-frutti and
chocolate ice cream and play with his paperweights and do made-up
calculations on his calculator and go home with the smell of industrial
freezer in my hair.

``You were always his favorite,'' my grandmother says, when I bring him
up now, and I hate how good it feels to hear that, because he was my
favorite, too.

When he got sick, really sick, I went to Nairobi and sat by his bedside
in the hospital. He'd say my name cheerfully when I came in, making the
T soft, as it's meant to be, following with a string of nicknames he had
for me like some kind of royal title. But then he'd get quiet. He was
tired, and sometimes confused. I brought him his brown resin comb and
combed his silver hair in a deep side part, the way he combed it his
entire life. I fed him a gelatinous goat-trotter broth, sent over by my
auntie, one spoon at a time, sometimes waiting a minute between bites
for him to signal that he was ready for more. He called me by my
mother's name. He fell asleep.

He died six years ago, but in my earliest memories of my grandfather,
he's drinking whiskey out of a beautiful glass. He's pulling a clean
handkerchief from his pocket and pressing it to my watering eye. He's
laughing from his belly like an evil cartoon character. But mostly he's
cooking --- browning English sausages for us in the morning before a
road trip, taking his time so every single link is evenly browned all
over, with no lines. He's frying lamb \emph{kofta} in a wide, scratched
saucepan --- the meatballs spherical, each one the same size, then
carefully transporting them to a pot of tomato sauce.

He was known, within a wide circle of family and friends, for this dish,
and for making it on request. He emailed me the recipe when I was in my
early 20s --- that was when we emailed each other a lot. I followed the
directions as closely I could, but the dish wasn't as good as his. Not
because of the veneer of nostalgia. Not because he was the kind of cook
guided by instinct, or the kind who withheld his technique --- he did,
in fact, measure things, and when he was asked for a recipe, he gladly
shared those measurements. I think his \emph{kofta} was better because
he was really good, better than most other people, and definitely better
than me, at every step of the dish. He paid attention. He cared. And
that's that.

The only way around this has been changing the dish and changing my
expectations for it. A mash of beans and herbs, held together with egg,
generously seasoned with ginger, garlic and green chile, makes a truly
delicious vegetarian base. Roasting the \emph{kofta} in a sheet pan
means they turn out evenly brown and crisp without your standing over
them, watching them, turning them. I love the version I make, which is
vegetarian and often uses canned tomato, even though it would surely
upset my grandfather to see all those lumpy meatballs ---
\emph{meatballs without any meat in them}? But I also know he would have
never let on.

Recipe:
\href{https://cooking.nytimes3xbfgragh.onion/recipes/1021415-vegetarian-kofta-curry}{Vegetarian
Kofta Curry}

Advertisement

\protect\hyperlink{after-bottom}{Continue reading the main story}

\hypertarget{site-index}{%
\subsection{Site Index}\label{site-index}}

\hypertarget{site-information-navigation}{%
\subsection{Site Information
Navigation}\label{site-information-navigation}}

\begin{itemize}
\tightlist
\item
  \href{https://help.nytimes3xbfgragh.onion/hc/en-us/articles/115014792127-Copyright-notice}{©~2020~The
  New York Times Company}
\end{itemize}

\begin{itemize}
\tightlist
\item
  \href{https://www.nytco.com/}{NYTCo}
\item
  \href{https://help.nytimes3xbfgragh.onion/hc/en-us/articles/115015385887-Contact-Us}{Contact
  Us}
\item
  \href{https://www.nytco.com/careers/}{Work with us}
\item
  \href{https://nytmediakit.com/}{Advertise}
\item
  \href{http://www.tbrandstudio.com/}{T Brand Studio}
\item
  \href{https://www.nytimes3xbfgragh.onion/privacy/cookie-policy\#how-do-i-manage-trackers}{Your
  Ad Choices}
\item
  \href{https://www.nytimes3xbfgragh.onion/privacy}{Privacy}
\item
  \href{https://help.nytimes3xbfgragh.onion/hc/en-us/articles/115014893428-Terms-of-service}{Terms
  of Service}
\item
  \href{https://help.nytimes3xbfgragh.onion/hc/en-us/articles/115014893968-Terms-of-sale}{Terms
  of Sale}
\item
  \href{https://spiderbites.nytimes3xbfgragh.onion}{Site Map}
\item
  \href{https://help.nytimes3xbfgragh.onion/hc/en-us}{Help}
\item
  \href{https://www.nytimes3xbfgragh.onion/subscription?campaignId=37WXW}{Subscriptions}
\end{itemize}
