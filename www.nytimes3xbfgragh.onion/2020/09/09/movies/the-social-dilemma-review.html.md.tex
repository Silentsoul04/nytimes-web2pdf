Sections

SEARCH

\protect\hyperlink{site-content}{Skip to
content}\protect\hyperlink{site-index}{Skip to site index}

\href{https://www.nytimes3xbfgragh.onion/section/movies}{Movies}

\href{https://myaccount.nytimes3xbfgragh.onion/auth/login?response_type=cookie\&client_id=vi}{}

\href{https://www.nytimes3xbfgragh.onion/section/todayspaper}{Today's
Paper}

\href{/section/movies}{Movies}\textbar{}`The Social Dilemma' Review:
Unplug and Run

\url{https://nyti.ms/3hgMunc}

\begin{itemize}
\item
\item
\item
\item
\item
\end{itemize}

Advertisement

\protect\hyperlink{after-top}{Continue reading the main story}

Supported by

\protect\hyperlink{after-sponsor}{Continue reading the main story}

\hypertarget{the-social-dilemma-review-unplug-and-run}{%
\section{`The Social Dilemma' Review: Unplug and
Run}\label{the-social-dilemma-review-unplug-and-run}}

This documentary from Jeff Orlowski explores how addiction and privacy
breaches are features, not bugs, of social media platforms.

\includegraphics{https://static01.graylady3jvrrxbe.onion/images/2020/09/11/arts/09socialdilemma-art/merlin_176722209_7289da9d-ecd3-44cb-b194-01df7bcbba79-articleLarge.jpg?quality=75\&auto=webp\&disable=upscale}

By Devika Girish

\begin{itemize}
\item
  Sept. 9, 2020
\item
  \begin{itemize}
  \item
  \item
  \item
  \item
  \item
  \end{itemize}
\end{itemize}

\begin{itemize}
\tightlist
\item
  The Social Dilemma\\
  Directed by Jeff Orlowski Documentary, Drama PG-13 1h 29m
\end{itemize}

\href{https://www.imdb.com/showtimes/title/tt11464826?ref_=ref_ext_NYT}{Find
Tickets}

When you purchase a ticket for an independently reviewed film through
our site, we earn an affiliate commission.

That social media can be addictive and creepy isn't a revelation to
anyone who uses Facebook, Twitter, Instagram and the like. But in Jeff
Orlowski's documentary ``The Social Dilemma,'' conscientious defectors
from these companies explain that the perniciousness of social
networking platforms is a feature, not a bug.

They claim that the manipulation of human behavior for profit is coded
into these companies with Machiavellian precision: Infinite scrolling
and push notifications keep users constantly engaged; personalized
recommendations use data not just to predict but also to influence our
actions, turning users into easy prey for advertisers and propagandists.

As in his documentaries about climate change,
\href{https://www.nytimes3xbfgragh.onion/2012/11/09/movies/chasing-ice-documents-the-work-of-james-balog.html}{``Chasing
Ice''} and
\href{https://www.nytimes3xbfgragh.onion/2017/07/13/movies/chasing-coral-review.html}{``Chasing
Coral,''} Orlowski takes a reality that can seem too colossal and
abstract for a layperson to grasp, let alone care about, and scales it
down to a human level. In ``The Social Dilemma,'' he recasts one of the
oldest tropes of the horror genre --- Dr. Frankenstein, the scientist
who went too far --- for the digital age.

In briskly edited interviews, Orlowski speaks with men and (a few) women
who helped build social media and now fear the effects of their
creations on users' mental health and the foundations of democracy. They
deliver their cautionary testimonies with the force of a start-up pitch,
employing crisp aphorisms and pithy analogies.

``Never before in history have 50 designers made decisions that would
have an impact on two billion people,'' says Tristan Harris, a former
design ethicist at Google. Anna Lembke, an addiction expert at Stanford
University, explains that these companies exploit the brain's
evolutionary need for interpersonal connection. And Roger McNamee, an
early investor in Facebook, delivers a chilling allegation: Russia
didn't hack Facebook; it simply \emph{used} the platform.

Much of this is familiar, but ``The Social Dilemma'' goes the extra
explainer-mile by interspersing the interviews with P.S.A.-style
fictional scenes of a suburban family suffering the consequences of
social-media addiction. There are silent dinners, a pubescent daughter
(Sophia Hammons) with self-image issues and a teenage son (Skyler
Gisondo) who's radicalized by YouTube recommendations promoting a vague
ideology.

This fictionalized narrative exemplifies the limitations of the
documentary's sometimes hyperbolic emphasis on the medium at the expense
of the message. For instance, the movie's interlocutors pin an increase
in mental illness on social media usage yet don't acknowledge factors
like a rise in
\href{https://www.nytimes3xbfgragh.onion/2015/07/13/business/rising-economic-insecurity-tied-to-decades-long-trend-in-employment-practices.html}{economic
insecurity}. Polarization, riots and protests are presented as
particular symptoms of the social-media era without historical context.

Despite their vehement criticisms, the interviewees in ``The Social
Dilemma'' are not all doomsayers; many suggest that with the right
changes, we can salvage the good of social media without the bad. But
the grab bag of personal and political solutions they present in the
film confuses two distinct targets of critique: the technology that
causes destructive behaviors and the culture of unchecked capitalism
that produces it.

Nevertheless, ``The Social Dilemma'' is remarkably effective in sounding
the alarm about the incursion of data mining and manipulative technology
into our social lives and beyond. Orlowski's film is itself not spared
by the phenomenon it scrutinizes. The movie is
\href{https://www.netflix.com/title/81254224}{streaming on Netflix},
where it'll become another node in the service's data-based algorithm.

\textbf{The Social Dilemma}\\
Rated PG-13 for dystopian speculation and some graphic images of
violence. Running time: 1 hour 34 minutes.
\href{https://www.netflix.com/title/81254224}{Watch on Netflix}.

Advertisement

\protect\hyperlink{after-bottom}{Continue reading the main story}

\hypertarget{site-index}{%
\subsection{Site Index}\label{site-index}}

\hypertarget{site-information-navigation}{%
\subsection{Site Information
Navigation}\label{site-information-navigation}}

\begin{itemize}
\tightlist
\item
  \href{https://help.nytimes3xbfgragh.onion/hc/en-us/articles/115014792127-Copyright-notice}{©~2020~The
  New York Times Company}
\end{itemize}

\begin{itemize}
\tightlist
\item
  \href{https://www.nytco.com/}{NYTCo}
\item
  \href{https://help.nytimes3xbfgragh.onion/hc/en-us/articles/115015385887-Contact-Us}{Contact
  Us}
\item
  \href{https://www.nytco.com/careers/}{Work with us}
\item
  \href{https://nytmediakit.com/}{Advertise}
\item
  \href{http://www.tbrandstudio.com/}{T Brand Studio}
\item
  \href{https://www.nytimes3xbfgragh.onion/privacy/cookie-policy\#how-do-i-manage-trackers}{Your
  Ad Choices}
\item
  \href{https://www.nytimes3xbfgragh.onion/privacy}{Privacy}
\item
  \href{https://help.nytimes3xbfgragh.onion/hc/en-us/articles/115014893428-Terms-of-service}{Terms
  of Service}
\item
  \href{https://help.nytimes3xbfgragh.onion/hc/en-us/articles/115014893968-Terms-of-sale}{Terms
  of Sale}
\item
  \href{https://spiderbites.nytimes3xbfgragh.onion}{Site Map}
\item
  \href{https://help.nytimes3xbfgragh.onion/hc/en-us}{Help}
\item
  \href{https://www.nytimes3xbfgragh.onion/subscription?campaignId=37WXW}{Subscriptions}
\end{itemize}
