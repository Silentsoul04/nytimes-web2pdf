Sections

SEARCH

\protect\hyperlink{site-content}{Skip to
content}\protect\hyperlink{site-index}{Skip to site index}

\href{https://www.nytimes3xbfgragh.onion/section/world/europe}{Europe}

\href{https://myaccount.nytimes3xbfgragh.onion/auth/login?response_type=cookie\&client_id=vi}{}

\href{https://www.nytimes3xbfgragh.onion/section/todayspaper}{Today's
Paper}

\href{/section/world/europe}{Europe}\textbar{}Fire Destroys Most of
Europe's Largest Refugee Camp, on Greek Island of Lesbos

\url{https://nyti.ms/35n2yS6}

\begin{itemize}
\item
\item
\item
\item
\item
\item
\end{itemize}

Advertisement

\protect\hyperlink{after-top}{Continue reading the main story}

Supported by

\protect\hyperlink{after-sponsor}{Continue reading the main story}

\hypertarget{fire-destroys-most-of-europes-largest-refugee-camp-on-greek-island-of-lesbos}{%
\section{Fire Destroys Most of Europe's Largest Refugee Camp, on Greek
Island of
Lesbos}\label{fire-destroys-most-of-europes-largest-refugee-camp-on-greek-island-of-lesbos}}

Campaigners have long warned that the overcrowded conditions at the
impoverished camp might lead to catastrophe.

\includegraphics{https://static01.graylady3jvrrxbe.onion/images/2020/09/10/world/09Greece-Migrants01/09Greece-Migrants01-videoSixteenByNineJumbo1600.jpg}

\href{https://www.nytimes3xbfgragh.onion/by/patrick-kingsley}{\includegraphics{https://static01.graylady3jvrrxbe.onion/images/2018/10/15/multimedia/author-patrick-kingsley/author-patrick-kingsley-thumbLarge.png}}

By \href{https://www.nytimes3xbfgragh.onion/by/patrick-kingsley}{Patrick
Kingsley}

\begin{itemize}
\item
  Sept. 9, 2020
\item
  \begin{itemize}
  \item
  \item
  \item
  \item
  \item
  \item
  \end{itemize}
\end{itemize}

Europe's largest refugee camp, on the Greek island of Lesbos, has long
been a desperate makeshift home for thousands of refugees and migrants
who have risked everything to flee war and economic hardship for a
better life.

They lived in cramped tents with limited access to toilets, showers and
health care. For years, rights groups warned that these squalid
conditions would sooner or later prompt a humanitarian disaster.

On Tuesday night, that disaster came. A fast-moving fire destroyed much
of the camp, leaving most of its 12,000 residents homeless. By
Wednesday, a process of soul-searching had begun among many Europeans,
for whom the Moria camp, and the neglect of its residents, has long been
synonymous with the continent's increasingly unsympathetic approach to
refugees.

No deaths were initially reported. But vast stretches of the camp and an
adjacent spillover site were destroyed in the fire, leaving only a
medical facility and small clusters of tents untouched.

Since 2015, Moria has filled with an influx of migrants --- now mostly
Afghan refugees --- seeking to reach northern Europe. It is a bleak tent
camp designed for 3,000 people that at times has swollen to more than
20,000 after Europe started blocking their paths in 2016.

Ursula von der Leyen, president of the European Union's executive arm,
the European Commission, said she felt ``deep sorrow'' about the fire,
while the governor of a region in western Germany, Armin Laschet, said
he was willing to admit up to 1,000 refugees from the camp as part of a
wider European resettlement program that has yet to be developed.

Some residents of the camp managed to escape to the island's main town
of Mytilene, while others were able to remain in their tents in small
areas of the camp that were unaffected by the blaze. But many were being
held nearby on Wednesday morning while the Greek authorities decided
where to house them.

Aid workers said that the fire at Moria, which is named after a nearby
village, began shortly after 10 p.m. on Tuesday following protests by
residents over recent coronavirus restrictions, and that it spread
quickly because of high winds and the explosion of gas canisters.

\includegraphics{https://static01.graylady3jvrrxbe.onion/images/2020/09/10/world/09Greece-Migrants02/09Greece-Migrants02-videoSixteenByNine3000.jpg}

Aid workers, activists and officials said a series of fires were started
intentionally by a group of camp residents who were furious at being
forced to quarantine after at least 35 people tested positive for
coronavirus at the camp.

A new, smaller fire broke out on Wednesday evening in one of the few
areas that had survived the first blaze, displacing roughly 1,000 more
people, aid workers said.

Notis Mitarachi, the Greek migration minister, said during a Wednesday
evening news conference that those responsible for the fires would not
go unpunished.

The fire quickly destroyed much of the camp's formal enclosure,
including a facility for 400 unaccompanied children and much of its
water infrastructure, before spreading to a spillover site in olive
groves close to the camp's fence. Prime Minister Kyriakos Mitsotakis
said a state of emergency had been declared for all of Lesbos and noted
that all unaccompanied minors would be transferred off the island.

Videos provided to The New York Times by aid workers at the camp showed
residents hurrying from Moria in droves in the early hours of Wednesday
morning. They carried their belongings in bags slung over their
shoulders, some of them pushing infants in strollers, and others draped
in blankets.

``It was absolute chaos,'' said Jonathan Turner, an aid worker who been
building water infrastructure in the camp on behalf of
\href{https://watershed-foundation.de/en/home}{Watershed Foundation} and
\href{https://helprefugees.org/choose-love/}{Choose Love}. ``There were
just so many people trying to move, trying to escape.''

By sunrise, footage showed that much of the camp's formal infrastructure
had collapsed, with many of the tents burned. Several metal portable
cabins were blackened with soot, their walls having buckled in the heat.
Trees on the nearby slopes had been charred.

\includegraphics{https://static01.graylady3jvrrxbe.onion/images/2020/09/10/world/09greece-migrants03/merlin_176761044_9d83655f-3c1c-4e84-8f4b-5bf8e9f17507-videoSixteenByNine3000.jpg}

Thousands of displaced residents were left with nowhere to go, with many
simply sitting down a few hundred meters from the camp.

``There are thousands of people just sitting on the main road,'' said
Nick Powell, an Australian aid worker who witnessed the fire and its
aftermath, and who was helping to provide food to the survivors on
Wednesday.

It is still unclear where they will be taken. George Koumoutsakos,
Greece's deputy migration minister, said during a Wednesday news
conference that efforts were being made to rehouse around 3,000 people
in new tents.

The priority was to rehouse the most vulnerable, with some 400
unaccompanied minors being moved to ``safe zones'' and hotels, he said.

Moria was started in 2015, when more than 850,000 war refugees and
migrants made their way by boat from Turkey to nearby Greek islands like
Lesbos, hoping to travel farther north. A further 300,000 have arrived
in the years since.

At first, when Europe was more tolerant of migrants, people tended to
pass through the camp quickly. But in 2016, Europe changed tack,
blocking the onward movement of migrants to countries like Germany and
leaving thousands stranded in squalid Greek camps like Moria, which soon
became overcrowded.

Since then,
\href{https://www.nytimes3xbfgragh.onion/2018/10/02/world/europe/greece-lesbos-moria-refugees.html}{Moria
has been considered an emblem of Europe's hardening approach} to
migrants in the aftermath of the 2015 crisis.

Through the European Union, other European countries provided Greece
with money to care for its refugee population. But European leaders
refused to allow many of them to leave Greek camps for sanctuary
elsewhere in Europe.

Stuck in Moria, migrants lined up for hours for food that was often
moldy. And they became enmeshed in what for many of them seemed an
interminably complex asylum application process, leading to what some
doctors deemed a mental health crisis at the camp.

\includegraphics{https://static01.graylady3jvrrxbe.onion/images/2020/09/10/world/09greece-migrants04/merlin_176761080_8f1a53d6-806b-45b9-a0f4-74c69dcded99-articleLarge.jpg?quality=75\&auto=webp\&disable=upscale}

The situation has been no better in other camps on nearby Greek islands.
Across the Greek islands before the fire, more than 23,000 people were
crammed into camps built for just 6,000, according to recent statistics
compiled by aid groups.

The dynamic has created
\href{https://www.nytimes3xbfgragh.onion/2020/03/07/world/europe/greece-turkey-migrants.html}{deep
hostility} between migrants and Greek islanders who, once welcoming to
their new neighbors, have grown increasingly resentful. It has also led
the Greek government to immediately expel many new arrivals this year,
\href{https://www.nytimes3xbfgragh.onion/2020/08/14/world/europe/greece-migrants-abandoning-sea.html}{abandoning
more than 1,000 immigrants in rafts at sea}.

Given these conditions, campaigners had long predicted a catastrophe at
the camp.

``This fire was expected,'' said Eva Cossé, who leads research in Greece
for Human Rights Watch, an independent New York-based rights
organization. ``It's not surprising. It's a testament to the European
Union's negligence and Greece's negligence.''

Human Rights Watch has been calling for the camp to be closed or its
number of residents to be significantly reduced for years.

``E.U. member states need to have a serious discussion about reducing
numbers on the island, and alleviate the pressure on Greece, because
Greece cannot deal with this alone,'' Ms. Cossé said.

While Mr. Mitsotakis, the prime minister, condemned those who started
the fire, he said the disaster could ``become an opportunity to deliver
better conditions and a new reality in Lesbos.''

Offers of support began on Wednesday, with the European Commission
saying it would immediately help relocate the 400 unaccompanied minors
to mainland Greece and onward to new homes in E.U. member states. These
children are the last of 1,200 that the bloc has been helping place in
other countries.

Ylva Johansson, the European commissioner for migration, said that the
commission was also paying for a boat that was on its way to Lesbos on
Wednesday afternoon and would serve as a makeshift hotel for the most
vulnerable.

She also said that, despite recent efforts to improve the overwhelmed
camp, conditions had remained very poor. Thousands of people were
transferred off the island as the pandemic began, reducing numbers from
more than 20,000 to 12,000, though it remained vastly overstretched.

``There are still too many people there,'' she said, calling the
conditions in Moria ``unacceptable.''

Reporting was contributed by Niki Kitsantonis and Iliana Magra from
Athens, Matina Stevis-Gridneff from Brussels, and Melissa Eddy from
Berlin.

Advertisement

\protect\hyperlink{after-bottom}{Continue reading the main story}

\hypertarget{site-index}{%
\subsection{Site Index}\label{site-index}}

\hypertarget{site-information-navigation}{%
\subsection{Site Information
Navigation}\label{site-information-navigation}}

\begin{itemize}
\tightlist
\item
  \href{https://help.nytimes3xbfgragh.onion/hc/en-us/articles/115014792127-Copyright-notice}{©~2020~The
  New York Times Company}
\end{itemize}

\begin{itemize}
\tightlist
\item
  \href{https://www.nytco.com/}{NYTCo}
\item
  \href{https://help.nytimes3xbfgragh.onion/hc/en-us/articles/115015385887-Contact-Us}{Contact
  Us}
\item
  \href{https://www.nytco.com/careers/}{Work with us}
\item
  \href{https://nytmediakit.com/}{Advertise}
\item
  \href{http://www.tbrandstudio.com/}{T Brand Studio}
\item
  \href{https://www.nytimes3xbfgragh.onion/privacy/cookie-policy\#how-do-i-manage-trackers}{Your
  Ad Choices}
\item
  \href{https://www.nytimes3xbfgragh.onion/privacy}{Privacy}
\item
  \href{https://help.nytimes3xbfgragh.onion/hc/en-us/articles/115014893428-Terms-of-service}{Terms
  of Service}
\item
  \href{https://help.nytimes3xbfgragh.onion/hc/en-us/articles/115014893968-Terms-of-sale}{Terms
  of Sale}
\item
  \href{https://spiderbites.nytimes3xbfgragh.onion}{Site Map}
\item
  \href{https://help.nytimes3xbfgragh.onion/hc/en-us}{Help}
\item
  \href{https://www.nytimes3xbfgragh.onion/subscription?campaignId=37WXW}{Subscriptions}
\end{itemize}
