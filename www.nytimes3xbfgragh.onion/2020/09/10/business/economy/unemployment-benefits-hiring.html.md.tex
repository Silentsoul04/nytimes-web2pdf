Sections

SEARCH

\protect\hyperlink{site-content}{Skip to
content}\protect\hyperlink{site-index}{Skip to site index}

\href{/section/business/economy}{Economy}\textbar{}Do Jobless Benefits
Deter Workers? Some Employers Say Yes. Studies Don't.

\url{https://nyti.ms/3k3aTyn}

\begin{itemize}
\item
\item
\item
\item
\item
\item
\end{itemize}

\hypertarget{the-coronavirus-outbreak}{%
\subsubsection{\texorpdfstring{\href{https://www.nytimes3xbfgragh.onion/news-event/coronavirus?name=styln-coronavirus-markets\&region=TOP_BANNER\&block=storyline_menu_recirc\&action=click\&pgtype=Article\&impression_id=52916d50-f52c-11ea-a92a-d1075f6219dd\&variant=undefined}{The
Coronavirus
Outbreak}}{The Coronavirus Outbreak}}\label{the-coronavirus-outbreak}}

\begin{itemize}
\tightlist
\item
  live\href{https://www.nytimes3xbfgragh.onion/2020/09/12/world/covid-19-coronavirus.html?name=styln-coronavirus-markets\&region=TOP_BANNER\&block=storyline_menu_recirc\&action=click\&pgtype=Article\&impression_id=52919460-f52c-11ea-a92a-d1075f6219dd\&variant=undefined}{Latest
  Updates}
\item
  \href{https://www.nytimes3xbfgragh.onion/interactive/2020/us/coronavirus-us-cases.html?name=styln-coronavirus-markets\&region=TOP_BANNER\&block=storyline_menu_recirc\&action=click\&pgtype=Article\&impression_id=52919461-f52c-11ea-a92a-d1075f6219dd\&variant=undefined}{Maps
  and Cases}
\item
  \href{https://www.nytimes3xbfgragh.onion/interactive/2020/science/coronavirus-vaccine-tracker.html?name=styln-coronavirus-markets\&region=TOP_BANNER\&block=storyline_menu_recirc\&action=click\&pgtype=Article\&impression_id=52919462-f52c-11ea-a92a-d1075f6219dd\&variant=undefined}{Vaccine
  Tracker}
\item
  \href{https://www.nytimes3xbfgragh.onion/2020/09/10/us/politics/fda-coronavirus-vaccine.html?name=styln-coronavirus-markets\&region=TOP_BANNER\&block=storyline_menu_recirc\&action=click\&pgtype=Article\&impression_id=52919463-f52c-11ea-a92a-d1075f6219dd\&variant=undefined}{F.D.A.
  Regulators' Self-Defense}
\item
  \href{https://www.nytimes3xbfgragh.onion/2020/09/09/upshot/coronavirus-surprise-test-fees.html?name=styln-coronavirus-markets\&region=TOP_BANNER\&block=storyline_menu_recirc\&action=click\&pgtype=Article\&impression_id=52919464-f52c-11ea-a92a-d1075f6219dd\&variant=undefined}{Surprise
  Test Fees}
\end{itemize}

\includegraphics{https://static01.graylady3jvrrxbe.onion/images/2020/09/03/business/03virus-hiring-01/merlin_176487678_4cf5e98d-adbb-4561-936c-38d64333d724-articleLarge.jpg?quality=75\&auto=webp\&disable=upscale}

\hypertarget{do-jobless-benefits-deter-workers-some-employers-say-yes-studies-dont}{%
\section{Do Jobless Benefits Deter Workers? Some Employers Say Yes.
Studies
Don't.}\label{do-jobless-benefits-deter-workers-some-employers-say-yes-studies-dont}}

A \$600-a-week supplement that expired in July has been credited with
bolstering the economy. Its impact on hiring is central to a political
fight.

Lorrie Jackson sorting parts at Clips \& Clamps, a Michigan company that
was flooded with applications for an opening last year.Credit...Sylvia
Jarrus for The New York Times

Supported by

\protect\hyperlink{after-sponsor}{Continue reading the main story}

\href{https://www.nytimes3xbfgragh.onion/by/patricia-cohen}{\includegraphics{https://static01.graylady3jvrrxbe.onion/images/2018/02/16/multimedia/author-patricia-cohen/author-patricia-cohen-thumbLarge.jpg}}

By \href{https://www.nytimes3xbfgragh.onion/by/patricia-cohen}{Patricia
Cohen}

\begin{itemize}
\item
  Sept. 10, 2020
\item
  \begin{itemize}
  \item
  \item
  \item
  \item
  \item
  \item
  \end{itemize}
\end{itemize}

When Clips \& Clamps, a metal forming company in Plymouth, Mich.,
advertised for a die setter and operator last year, more than a hundred
applications came sailing in.

This summer, the company sought to hire another operator, offering \$17
to \$22 an hour and benefits. After three months, not a single person
had responded.

``I received zero applicants,'' said Jeff Aznavorian, the company's
president. ``I've been dumbfounded.''

Mr. Aznavorian, whose grandmother founded the company 66 years ago, has
no clear explanation for his hiring troubles. In the Detroit area, there
should be plenty of qualified candidates, he said. And Michigan's
\href{https://www.michigan.gov/dtmb/0,5552,7-358-82546_9352_99726-537169--,00.html}{unemployment
rate} was 8.7 percent in July, more than double what it was last summer.

``I'm guessing it has something to do with the extra benefits associated
with unemployment,'' he said.

The \$600-a-week jobless
\href{https://www.nytimes3xbfgragh.onion/article/coronavirus-stimulus-package-questions-answers.html?action=click\&module=RelatedLinks\&pgtype=Article}{benefit
supplement} that Congress approved in March as part of the
\href{https://www.nytimes3xbfgragh.onion/2020/03/27/us/politics/coronavirus-house-voting.html}{CARES
Act} has been widely credited by economists with keeping the economy
functioning through the coronavirus pandemic. Households used
the\href{https://www.nytimes3xbfgragh.onion/2020/07/29/business/economy/unemployment-benefits-coronavirus.html}{extra
cash} to pay rent, buy food and cover medical, utility and credit card
bills when many businesses abruptly shut and cars lined up for miles at
\href{https://www.nytimes3xbfgragh.onion/2020/04/08/business/economy/coronavirus-food-banks.html}{food
banks}.

With the supplement, which ended in July,
\href{https://bfi.uchicago.edu/working-paper/2020-62/}{most unemployed
workers} got more than they had earned in wages; without it, they fell
short of their previous income. So did the supplement simply provide a
lifeline, or did it discourage people from taking jobs?

The answer has consequences for
\href{https://www.nytimes3xbfgragh.onion/2020/08/08/business/economy/lost-unemployment-benefits.html}{tens
of millions of Americans}, particularly those on the
\href{https://www.nytimes3xbfgragh.onion/2020/08/07/upshot/unemployment-benefits-racial-disparity.html}{lower
end of the income ladder}; for businesses trying to restore their
operations; and for an economy that largely depends on the lifeblood of
consumer spending.

There has been striking agreement among conservative and liberal
economists who have studied the issue that the \$600 supplement has
deterred few workers from accepting a job. But the relief is not only a
matter of contention among business owners; it is also at the center of
an acrimonious debate in Congress that has held up agreement on a new
aid package.

Image

Clips \& Clamps sought to hire another operator this summer, offering
\$17 to \$22 an hour and benefits.Credit...Sylvia Jarrus for The New
York Times

Image

After three months, not a single person had responded. ``I've been
dumbfounded,'' said Jeff Aznavorian, the company's
president.Credit...Sylvia Jarrus for The New York Times

Democrats insisted on extending the full \$600 payment beyond July,
while Republicans pushed for no more than \$200, arguing that the extra
income deterred people from working.

\hypertarget{latest-updates-the-coronavirus-outbreak-and-the-economy}{%
\section{\texorpdfstring{\href{https://www.nytimes3xbfgragh.onion/live/2020/09/11/business/stock-market-today-coronavirus?action=click\&pgtype=Article\&state=default\&region=MAIN_CONTENT_1\&context=storylines_live_updates}{Latest
Updates: The Coronavirus Outbreak and the
Economy}}{Latest Updates: The Coronavirus Outbreak and the Economy}}\label{latest-updates-the-coronavirus-outbreak-and-the-economy}}

\href{https://www.nytimes3xbfgragh.onion/live/2020/09/11/business/stock-market-today-coronavirus?action=click\&pgtype=Article\&state=default\&region=MAIN_CONTENT_1\&context=storylines_live_updates\#the-nyse-may-move-its-data-center-out-of-new-jersey-in-response-to-a-proposed-tax}{23h
ago}

\href{https://www.nytimes3xbfgragh.onion/live/2020/09/11/business/stock-market-today-coronavirus?action=click\&pgtype=Article\&state=default\&region=MAIN_CONTENT_1\&context=storylines_live_updates\#the-nyse-may-move-its-data-center-out-of-new-jersey-in-response-to-a-proposed-tax}{The
N.Y.S.E. may move its data center out of New Jersey in response to a
proposed tax.}

\href{https://www.nytimes3xbfgragh.onion/live/2020/09/11/business/stock-market-today-coronavirus?action=click\&pgtype=Article\&state=default\&region=MAIN_CONTENT_1\&context=storylines_live_updates\#the-federal-budget-deficit-hit-3-trillion-as-of-august}{25h
ago}

\href{https://www.nytimes3xbfgragh.onion/live/2020/09/11/business/stock-market-today-coronavirus?action=click\&pgtype=Article\&state=default\&region=MAIN_CONTENT_1\&context=storylines_live_updates\#the-federal-budget-deficit-hit-3-trillion-as-of-august}{The
federal budget deficit hit \$3 trillion as of August.}

\href{https://www.nytimes3xbfgragh.onion/live/2020/09/11/business/stock-market-today-coronavirus?action=click\&pgtype=Article\&state=default\&region=MAIN_CONTENT_1\&context=storylines_live_updates\#warner-bros-pushes-the-release-of-wonder-woman-1984-to-christmas}{25h
ago}

\href{https://www.nytimes3xbfgragh.onion/live/2020/09/11/business/stock-market-today-coronavirus?action=click\&pgtype=Article\&state=default\&region=MAIN_CONTENT_1\&context=storylines_live_updates\#warner-bros-pushes-the-release-of-wonder-woman-1984-to-christmas}{Warner
Bros. pushes the release of `Wonder Woman 1984' to Christmas.}

\href{https://www.nytimes3xbfgragh.onion/live/2020/09/11/business/stock-market-today-coronavirus?action=click\&pgtype=Article\&state=default\&region=MAIN_CONTENT_1\&context=storylines_live_updates}{See
more updates}

More live coverage:
\href{https://www.nytimes3xbfgragh.onion/2020/09/11/world/covid-19-coronavirus.html?action=click\&pgtype=Article\&state=default\&region=MAIN_CONTENT_1\&context=storylines_live_updates}{Global}

Faced with the standoff, President Trump decided to use
\href{https://www.nytimes3xbfgragh.onion/2020/08/13/business/economy/unemployment-benefits-coronavirus.html}{federal
disaster relief funds} to give most jobless workers
\href{https://www.nytimes3xbfgragh.onion/article/stimulus-unemployment-payment-benefit.html}{\$300
a week}, but officials said the funds would cover only four or five
weeks of payments. The issue is likely to continue to resonate through
the election campaign.

For most people collecting unemployment benefits, there are simply no
jobs. Roughly half of the 22 million jobs that evaporated with the
coronavirus outbreak have not yet returned. Freelancers, gig workers,
the self-employed and others have also seen their contracts and incomes
shrink.

But what about those who declined to return to a previous job, or take a
new one?

Turning down a job offer to stay on unemployment insurance is considered
fraud and is grounds for losing all jobless benefits. But many states
suspended verification checks, and with the flood of claims, keeping
track of applicants' job searching can be difficult.

So can determining the reason for declining a job. A
\href{https://www.nytimes3xbfgragh.onion/2020/06/03/business/economy/coronavirus-working-women.html}{lack
of child care} or health concerns related to Covid-19 are generally
considered acceptable excuses. Making more money on unemployment
insurance is not.

On a gut level, the Republicans' argument makes sense. With the
supplement,
\href{https://bfi.uchicago.edu/working-paper/2020-62/}{nearly seven in
10 jobless workers} got a bigger payment from the government than from
their previous employer, according to one study. On its face, choosing
to get more money and not work seems more appealing than settling for
less and working.

\includegraphics{https://static01.graylady3jvrrxbe.onion/images/2020/09/02/business/00virus-hiring-04/merlin_176427279_84480643-8f28-45e2-ad5a-1c609b1f407e-articleLarge.jpg?quality=75\&auto=webp\&disable=upscale}

That's the way Carl Livesay, vice president for operations of Maryland
Thermoform in Baltimore, sees it. Before the pandemic, the low
unemployment rate made hiring a struggle, but now, even with high
unemployment rates, he said, ``it's worse than it's ever been.''

He has been trying to hire eight people as entry-level machine operators
or warehouse workers, paying \$12 to \$15 an hour.

``Only about 50 percent show up for the interview,'' Mr. Livesay said.
``Only 50 percent of those that we hire actually show up for work the
first day. And of those, 25 percent don't make it through the first
week.''

When he called his 60 employees back to work in early May, he said, some
were worried about taking public transportation, so he offered to pay
for a round trip by Uber until they felt comfortable. Mr. Livesay, who
is on Gov. Larry Hogan's task force to reopen manufacturing, said he had
instituted a range of safety and sanitation measures to protect his
workers.

As far as he knows, only one employee, a single father, has been unable
to return because of child care responsibilities.

Mr. Livesay is convinced that the \$600 supplement made it harder to
hire.

Image

Carl Livesay of Maryland Thermoform said the tight labor market before
the pandemic had made hiring a struggle. Now, even with high
unemployment rates, ``it's worse than it's ever been,'' he
said.Credit...Andrew Mangum for The New York Times

Within two or three days of the benefit's expiration, he said,
applications tripled. When the government approved the \$300
replacement, he said, the numbers began to dwindle, even though most
states have yet to start making the payments.

``It's free money, so they feel they don't have to work anymore,'' he
said.

Other employers share his sentiment. One-third of small-business owners
surveyed by the National Federation of Independent Business said the
supplement made hiring harder.

There are, of course, examples that tell a different story --- millions
of them. In May, June and July, more than 9.3 million workers returned
to a job, forgoing the generous unemployment benefits.

And that story turns out to be by far the most common.

\href{https://news.yale.edu/2020/07/27/yale-study-finds-expanded-jobless-benefits-did-not-reduce-employment}{Researchers
at Yale University} who reviewed scheduling and time clock data for
small businesses said, ``We find no evidence that more generous benefits
disincentivized work either at the onset of the expansion or as firms
looked to return to business over time.''

Five
\href{https://www.nber.org/papers/w27613.pdf}{other}\href{https://www.jec.senate.gov/public/_cache/files/d995d820-20d6-4a24-b93f-b1a16ce2b0d2/studies-show-600-weekly-enhanced-unemployment-benefit-has-not-slowed-labor-market-recovery-final.pdf}{studies}by
different groups of
\href{https://papers.ssrn.com/sol3/papers.cfm?abstract_id=3664265}{economists}
produced the same results.

And in
\href{https://www.nytimes3xbfgragh.onion/live/2020/09/01/business/stock-market-today-coronavirus/extra-unemployment-pay-deters-few-from-seeking-work-a-survey-finds}{a
survey by Franklin Templeton-Gallup}, conducted in early August, most
people said extra government relief would not keep them from going back
to work.

One reason is that people generally look ahead. ``The latest results
show that Americans rationally understand the greater long-term security
of returning to work rather than relying on ongoing government
assistance,'' said Sonal Desai, chief investment officer of Franklin
Templeton Fixed Income.

\href{https://bfi.uchicago.edu/working-paper/2020-112/}{New research}
from economists at the University of Chicago and New York University
came to the same conclusion. The extra benefits, even if extended, are
fleeting. In a recession, the possibility of not getting another job
offer after refusing one is scary, as is the likelihood that lower wages
and career setbacks could be permanent. Stability is worth a lot.

The desire for security may have something to do with the smaller pool
of applicants. Several business owners noted that before the pandemic,
labor shortages meant that many new hires were already employed
elsewhere. Now, laid-off employees who expect to be rehired may prefer
to wait for a callback than switch to a new job.

What the result showed, said Simon Mongey of the University of Chicago,
one of the paper's co-authors, is that ``it's very hard to rationalize
why a worker would turn down an offer of returning to a previous
employer at a previous wage'' even with the \$600 supplement in place.

That was what Walt Rowen, the owner of Susquehanna Glass in Lancaster,
Pa., saw. He had to furlough most of his 75 employees for 10 weeks, but
afterward most returned to their jobs at \$10 to \$15 an hour. Problems
with child care or family health were the main reasons that some did
not, he said.

``We didn't get the feeling that there were very many people at all that
made that calculation'' about unemployment benefits, Mr. Rowen said.

That doesn't mean there aren't exceptions, as Mr. Livesay at Maryland
Thermoform and some other employers have reported, and they can feed a
perception that the preference for benefits over work is more widespread
than it is.

Research economists noted, for instance, that dental assistants in
Chicago and New York were less likely than coffee shop workers to return
quickly. The reason is that dental assistants, because of their
specialized skills, know they are hard to replace, while food
preparation workers are not.

Some low-paid part-time workers who don't get benefits like health
insurance or retirement savings --- or workers with no long-term
prospects --- may also choose the generous unemployment benefit package.

Image

Maryland Thermoform found that some employees were uneasy about
returning. The company has instituted a range of safety and sanitation
measures.Credit...Andrew Mangum for The New York Times

That's what Bruce Zoldan, chief executive of Phantom Fireworks, based in
Youngstown, Ohio, found when he sought to hire hundreds of workers for a
one- to three-month stint earning \$12 to \$15 an hour during the summer
fireworks season.

People were interested in a full-time job, he said, but he wasn't able
to offer one.

Mr. Zoldan, whose pyrotechnical razzle-dazzle lit up the National Mall
last year as Mr. Trump watched, said, ``I certainly understand why
people would not want to work this particular July Fourth season in my
industry.''

The \$600 supplement caused hiring problems, Mr. Zoldan said, but it was
also responsible for his best sales on record --- because it put money
into customers' hands.

``I do believe that the incentive checks are important right now,'' he
said. ``For the time being, it's something people need. Those people who
bought fireworks for the Fourth of July are also going to restaurants,
and spending money that keeps businesses going.''

That is a point that Wall Street analysts and economists continue to
emphasize: that getting money to consumers will keep businesses afloat
and workers on staff.

Without it, not only will millions of needy Americans suffer, said Janet
L. Yellen, a former chair of the Federal Reserve, but ``the overall
economy could degrade from its current slow rebound in growth to no
growth at all.''

Mr. Aznavorian of Clips \& Clamps agrees that the extra money has been
crucial to millions of families. What he objects to are breakdowns in
enforcement, allowing workers who turn down jobs to keep receiving
benefits.

He also knows from experience, though, that the desire to work is
powerful. His wife, Tara, returned to her part-time job as a medical
assistant, giving up hundreds of dollars a week in unemployment
benefits.

``She chose to go back to work,'' he said. ``She hated taking that
jobless benefit.''

Jim Tankersley contributed reporting.

Advertisement

\protect\hyperlink{after-bottom}{Continue reading the main story}

\hypertarget{site-index}{%
\subsection{Site Index}\label{site-index}}

\hypertarget{site-information-navigation}{%
\subsection{Site Information
Navigation}\label{site-information-navigation}}

\begin{itemize}
\tightlist
\item
  \href{https://help.nytimes3xbfgragh.onion/hc/en-us/articles/115014792127-Copyright-notice}{©~2020~The
  New York Times Company}
\end{itemize}

\begin{itemize}
\tightlist
\item
  \href{https://www.nytco.com/}{NYTCo}
\item
  \href{https://help.nytimes3xbfgragh.onion/hc/en-us/articles/115015385887-Contact-Us}{Contact
  Us}
\item
  \href{https://www.nytco.com/careers/}{Work with us}
\item
  \href{https://nytmediakit.com/}{Advertise}
\item
  \href{http://www.tbrandstudio.com/}{T Brand Studio}
\item
  \href{https://www.nytimes3xbfgragh.onion/privacy/cookie-policy\#how-do-i-manage-trackers}{Your
  Ad Choices}
\item
  \href{https://www.nytimes3xbfgragh.onion/privacy}{Privacy}
\item
  \href{https://help.nytimes3xbfgragh.onion/hc/en-us/articles/115014893428-Terms-of-service}{Terms
  of Service}
\item
  \href{https://help.nytimes3xbfgragh.onion/hc/en-us/articles/115014893968-Terms-of-sale}{Terms
  of Sale}
\item
  \href{https://spiderbites.nytimes3xbfgragh.onion}{Site Map}
\item
  \href{https://help.nytimes3xbfgragh.onion/hc/en-us}{Help}
\item
  \href{https://www.nytimes3xbfgragh.onion/subscription?campaignId=37WXW}{Subscriptions}
\end{itemize}
