Sections

SEARCH

\protect\hyperlink{site-content}{Skip to
content}\protect\hyperlink{site-index}{Skip to site index}

\href{https://www.nytimes3xbfgragh.onion/section/sports}{Sports}

\href{https://myaccount.nytimes3xbfgragh.onion/auth/login?response_type=cookie\&client_id=vi}{}

\href{https://www.nytimes3xbfgragh.onion/section/todayspaper}{Today's
Paper}

\href{/section/sports}{Sports}\textbar{}N.F.L. Season Kicks Off With
Players' Protesting Racism

\url{https://nyti.ms/3bNJGwS}

\begin{itemize}
\item
\item
\item
\item
\item
\end{itemize}

Advertisement

\protect\hyperlink{after-top}{Continue reading the main story}

Supported by

\protect\hyperlink{after-sponsor}{Continue reading the main story}

\hypertarget{nfl-season-kicks-off-with-players-protesting-racism}{%
\section{N.F.L. Season Kicks Off With Players' Protesting
Racism}\label{nfl-season-kicks-off-with-players-protesting-racism}}

After an off-season of social turmoil, the Houston Texans chose not to
be on the field for the playing of ``Lift Every Voice and Sing'' and
``The Star-Spangled Banner'' before the league's season opener in Kansas
City, Mo.

\includegraphics{https://static01.graylady3jvrrxbe.onion/images/2020/09/10/sports/10Anthem-sub/merlin_176831079_78dbb280-1be1-46df-9fd6-67f283f355bf-articleLarge.jpg?quality=75\&auto=webp\&disable=upscale}

By \href{https://www.nytimes3xbfgragh.onion/by/ken-belson}{Ken Belson}

\begin{itemize}
\item
  Sept. 10, 2020
\item
  \begin{itemize}
  \item
  \item
  \item
  \item
  \item
  \end{itemize}
\end{itemize}

After an off-season of social and political turmoil, N.F.L. players made
it clear on the night of the season opener that they will continue to
shine a light on social injustice and police brutality against African
Americans.

The Houston Texans, who were in Kansas City, Mo., on Thursday to face
the Chiefs for the first game of the year, remained in their locker room
during the playing of ``The Star-Spangled Banner'' and
``\href{https://twitter.com/NFL/status/1304212564599955456?s=20}{Lift
Every Voice and Sing},'' which is known as the Black national anthem.
After the protests following the police killing of George Floyd in
Minneapolis in late May, the league said the song would be played before
every game in Week 1 of the season.

The Texans stayed inside to keep the focus on systemic racism and to
avoid a debate over kneeling or standing during either or both songs.

``It's really not about the flag, it's about making sure that people
understand that Black lives do matter,'' Texans Coach Bill O'Brien said
after the game.

The Chiefs lined up along their sideline while ``The Star-Spangled
Banner'' played. One player, defensive end
\href{https://www.chiefs.com/team/players-roster/alex-okafor/}{Alex
Okafor}, knelt and raised an arm. A teammate put his hand on Okafor's
shoulder. Many other players linked arms.

After the anthem was played, the Texans ran on to the field to a
smattering of boos from the Arrowhead Stadium crowd, which had been
reduced to 22 percent capacity because of the coronavirus. Some fans
also booed
\href{https://twitter.com/ComplexSports/status/1304216261904281600}{as
the players linked arms} in the middle of the field for a moment of
silence, which Texans defensive lineman J.J. Watt called
``unfortunate.'' Chiefs quarterback Patrick Mahomes and Texans
quarterback Deshaun Watson, who were both outspoken this summer about
the need for change, were at the center of the line, arms linked.

The protests at the N.F.L. opener were the latest in a wave of
demonstrations by professional athletes that began late last month with
the widespread postponement of games as players in the N.B.A., the
W.N.B.A., Major League Baseball and other leagues chose to
\href{https://www.nytimes3xbfgragh.onion/2020/08/26/sports/basketball/nba-boycott-bucks-magic-blake-shooting.html}{walk
out in response to the police shooting of Jacob Blake in Kenosha, Wis.}

Even though about 70 percent of its players are Black, the N.F.L. has
wrestled for several years with how to react to player protests and
calls to address systematic racism and social injustice. The league
largely ignored quarterback Colin Kaepernick when he knelt during the
national anthem throughout the 2016 season to protest police brutality
against African Americans.

But after President Trump in September 2017 called for teams to fire
players who did not stand for the anthem, the league and its owners
tried to tamp down protests while also pledging tens of millions of
dollars to groups fighting social injustice. The league backed off
trying to ban protests during the anthem after the players' union filed
a grievance.

Kaepernick, once considered one of the sport's rising stars, has not
been on an N.F.L. roster since the 2016 season, and in 2017 he accused
the league of blackballing him because of his protests. The N.F.L.
agreed to a multimillion-dollar settlement of the case. In November, the
league organized a tryout for Kaepernick, but the two sides could not
agree on the terms of the audition.
\href{https://www.nytimes3xbfgragh.onion/2019/11/16/sports/football/colin-kaepernick-nfl-workout.html}{Kaepernick
held his own workout} in front of about a half-dozen scouts, but he
remains unsigned.

Only a handful of players protested the past couple of seasons. But the
issue was reignited this summer with nationwide demonstrations after
Floyd's death. In early June, Commissioner
\href{https://www.nytimes3xbfgragh.onion/2020/06/05/sports/football/trump-anthem-kneeling-kaepernick.html}{Roger
Goodell apologized for not listening sooner}to the concerns of
African-American players. At the same time, President Trump renewed his
attacks on the league.

Broadcasters, who pay the league billions of dollars for the rights to
show games, have largely tiptoed around the protests. But in a sign of
the new attitude, Cris Collinsworth, a former player who was one of the
announcers calling Thursday's game for NBC Sports, lent his support to
the protesters.

``I stand behind these players 100 percent. 100 percent,'' he said
before kickoff. ``What they're trying to do is create positive change in
this country that frankly is long, long overdue.''

The bulk of the N.F.L. games will be on Sunday, and it is already clear
there will be more protests. About an hour before the Chiefs and Texans
kicked off, members of the Miami Dolphins took aim at the league's
efforts to address systematic racism and said they, too, would remain in
the locker room during the playing of ``The Star-Spangled Banner'' and
``Lift Every Voice and Sing.''

\href{https://twitter.com/realjaywilliams/status/1304186433054420992?s=12}{In
a video} they posted on Twitter, and which was first reported by ESPN,
the players said they did not appreciate the league's empty marketing
slogans, which they called ``fluff and empty gestures.''

``We don't need another publicity parade, so we'll just stay inside
until it's time to play the game,'' the players said, referring to their
game against the New England Patriots on Sunday.

Playing ``Lift Every Voice and Sing'' before games, they said, ``is just
a way to save face.''

``We need changed hearts, not just a response to pressure,'' they added.

The video ended with Dolphins Coach Brian Flores, who is Afro-Latino and
one of just three Black head coaches in the league, repeating the
message, ``We'll just stay inside.''

The league this summer
\href{https://www.seahawks.com/news/seahawks-to-wear-helmet-decals-supporting-social-justice-initiatives-honoring-vi}{approved
a plan for players to wear decals} on the backs of their helmets with
the names of victims of racist violence. Teams are stenciling the words
``End Racism'' in the end zones, and the N.F.L. has encouraged teams to
use their stadiums as polling centers on Election Day.

Some of the league's biggest stars are getting messages across in
advertisements. Mahomes, who in July signed a
\href{https://www.nytimes3xbfgragh.onion/2020/07/06/sports/football/pat-mahomes-contract.html}{10-year
contract worth as much as \$500 million},
\href{https://twitter.com/PatrickMahomes/status/1303504412242071552}{appeared
in an Adidas ad} in which he said: ``We're gonna be playing sports. But
at the same time we're gonna be taking action, and we're gonna be making
change in the world.''

Mahomes has been a vocal supporter of voter registration initiatives and
\href{https://chiefswire.usatoday.com/2020/06/23/kansas-city-chiefs-patrick-mahomes-lebron-james-more-than-a-vote/}{fighting
voter suppression}.

Like Goodell, some N.F.L. team owners have said they support the
players' right to protest. Last week, John Mara, co-owner of the Giants,
\href{https://www.giants.com/news/john-mara-joe-judge-daniel-jones-saquon-barkley-jabrill-peppers-training-camp}{said
he preferred that players stand} for the national anthem, but that he
would back those who did not.

``I'm going to support your right to do that because I believe in the
First Amendment, and I believe in the right of people, especially
players, to take a knee in silent protest if that's what they want to
do,'' he said.

Advertisement

\protect\hyperlink{after-bottom}{Continue reading the main story}

\hypertarget{site-index}{%
\subsection{Site Index}\label{site-index}}

\hypertarget{site-information-navigation}{%
\subsection{Site Information
Navigation}\label{site-information-navigation}}

\begin{itemize}
\tightlist
\item
  \href{https://help.nytimes3xbfgragh.onion/hc/en-us/articles/115014792127-Copyright-notice}{©~2020~The
  New York Times Company}
\end{itemize}

\begin{itemize}
\tightlist
\item
  \href{https://www.nytco.com/}{NYTCo}
\item
  \href{https://help.nytimes3xbfgragh.onion/hc/en-us/articles/115015385887-Contact-Us}{Contact
  Us}
\item
  \href{https://www.nytco.com/careers/}{Work with us}
\item
  \href{https://nytmediakit.com/}{Advertise}
\item
  \href{http://www.tbrandstudio.com/}{T Brand Studio}
\item
  \href{https://www.nytimes3xbfgragh.onion/privacy/cookie-policy\#how-do-i-manage-trackers}{Your
  Ad Choices}
\item
  \href{https://www.nytimes3xbfgragh.onion/privacy}{Privacy}
\item
  \href{https://help.nytimes3xbfgragh.onion/hc/en-us/articles/115014893428-Terms-of-service}{Terms
  of Service}
\item
  \href{https://help.nytimes3xbfgragh.onion/hc/en-us/articles/115014893968-Terms-of-sale}{Terms
  of Sale}
\item
  \href{https://spiderbites.nytimes3xbfgragh.onion}{Site Map}
\item
  \href{https://help.nytimes3xbfgragh.onion/hc/en-us}{Help}
\item
  \href{https://www.nytimes3xbfgragh.onion/subscription?campaignId=37WXW}{Subscriptions}
\end{itemize}
