Sections

SEARCH

\protect\hyperlink{site-content}{Skip to
content}\protect\hyperlink{site-index}{Skip to site index}

\href{https://www.nytimes3xbfgragh.onion/section/technology}{Technology}

\href{https://myaccount.nytimes3xbfgragh.onion/auth/login?response_type=cookie\&client_id=vi}{}

\href{https://www.nytimes3xbfgragh.onion/section/todayspaper}{Today's
Paper}

\href{/section/technology}{Technology}\textbar{}Social Media Shaming
Your College

\url{https://nyti.ms/2ZlTwkw}

\begin{itemize}
\item
\item
\item
\item
\item
\end{itemize}

\hypertarget{school-reopenings}{%
\subsubsection{\texorpdfstring{\href{https://www.nytimes3xbfgragh.onion/spotlight/schools-reopening?name=styln-coronavirus-schools-reopening\&region=TOP_BANNER\&block=storyline_menu_recirc\&action=click\&pgtype=Article\&impression_id=bd7fd850-f4bd-11ea-b11e-678a241facd9\&variant=undefined}{School
Reopenings}}{School Reopenings}}\label{school-reopenings}}

\begin{itemize}
\tightlist
\item
  \href{https://www.nytimes3xbfgragh.onion/2020/09/08/us/school-districts-cyberattacks-glitches.html?name=styln-coronavirus-schools-reopening\&region=TOP_BANNER\&block=storyline_menu_recirc\&action=click\&pgtype=Article\&impression_id=bd7fff60-f4bd-11ea-b11e-678a241facd9\&variant=undefined}{Remote
  Learning Glitches}
\item
  \href{https://www.nytimes3xbfgragh.onion/2020/09/08/upshot/children-testing-shortfalls-virus.html?name=styln-coronavirus-schools-reopening\&region=TOP_BANNER\&block=storyline_menu_recirc\&action=click\&pgtype=Article\&impression_id=bd7fff61-f4bd-11ea-b11e-678a241facd9\&variant=undefined}{Limited
  Testing for Children}
\item
  \href{https://www.nytimes3xbfgragh.onion/2020/09/10/us/des-moines-school-opening-coronavirus.html?name=styln-coronavirus-schools-reopening\&region=TOP_BANNER\&block=storyline_menu_recirc\&action=click\&pgtype=Article\&impression_id=bd7fff62-f4bd-11ea-b11e-678a241facd9\&variant=undefined}{District
  Defies Reopening Order}
\item
  \href{https://www.nytimes3xbfgragh.onion/interactive/2020/us/covid-college-cases-tracker.html?name=styln-coronavirus-schools-reopening\&region=TOP_BANNER\&block=storyline_menu_recirc\&action=click\&pgtype=Article\&impression_id=bd7fff63-f4bd-11ea-b11e-678a241facd9\&variant=undefined}{Tracking
  College Cases}
\end{itemize}

Advertisement

\protect\hyperlink{after-top}{Continue reading the main story}

Supported by

\protect\hyperlink{after-sponsor}{Continue reading the main story}

on tech

\hypertarget{social-media-shaming-your-college}{%
\section{Social Media Shaming Your
College}\label{social-media-shaming-your-college}}

Students are using apps to shame their schools into better coronavirus
plans.

\includegraphics{https://static01.graylady3jvrrxbe.onion/images/2020/09/10/business/10ontech-still/10ontech-still-articleLarge-v9.gif?quality=75\&auto=webp\&disable=upscale}

\href{https://www.nytimes3xbfgragh.onion/by/shira-ovide}{\includegraphics{https://static01.graylady3jvrrxbe.onion/images/2020/03/18/reader-center/author-shira-ovide/author-shira-ovide-thumbLarge-v2.png}}

By \href{https://www.nytimes3xbfgragh.onion/by/shira-ovide}{Shira Ovide}

\begin{itemize}
\item
  Sept. 10, 2020
\item
  \begin{itemize}
  \item
  \item
  \item
  \item
  \item
  \end{itemize}
\end{itemize}

\emph{This article is part of the On Tech newsletter. You can}
\href{https://www.nytimes3xbfgragh.onion/newsletters/signup/OT}{\emph{sign
up here}} \emph{to receive it weekdays.}

We've all seen
\href{https://www.nytimes3xbfgragh.onion/2020/05/11/arts/social-distance-shaming.html}{social
media used to shame people we disagree with}. Those milliseconds of
tsk-tsking might feel good, but I doubt they're helpful.

Then my colleague
\href{https://www.nytimes3xbfgragh.onion/by/natasha-singer}{Natasha
Singer} told me about pandemic shaming I can get behind.

College students are using TikTok, Twitter and other apps to embarrass
their
\href{https://www.nytimes3xbfgragh.onion/2020/09/10/health/university-illinois-covid.html}{universities}
when they fail to care for people who have been isolated in special
\href{https://www.nytimes3xbfgragh.onion/2020/09/10/health/university-illinois-covid.html}{Covid-19}
dorms or are in quarantine units because of a possible exposure.

Natasha, who wrote this week about
\href{https://www.nytimes3xbfgragh.onion/2020/09/09/business/colleges-coronavirus-dormitories-quarantine.html}{universities
botching on-campus quarantines}, talked to me about how young people ---
often being shamed for acting irresponsibly in the pandemic --- are now
turning the tables on the grown-ups, and how colleges are sometimes
over-relying on technology that doesn't do much to protect students.

\textbf{Shira: Tell me your tales of students using social media to
shame their schools.}

\textbf{Natasha:} Many people have seen the online videos of students
stuck in quarantine or isolation documenting
\href{https://www.nytimes3xbfgragh.onion/2020/08/22/nyregion/coronavirus-tiktok-college-quarantine-food.html}{crummy
or nonexistent university-provided meals}.

But what I found went deeper: Sick students are making videos about how
they felt universities
\href{https://dailyiowan.com/2020/08/20/i-felt-like-a-guinea-pig-students-awful-quarantine-experience-prompts-university-of-iowa-apology/}{abandoned
them} once they tested positive and moved into special Covid dorms.

And there are a bunch of students who shared online their shock that
virus-infected students or people who were waiting for tests were
\href{https://twitter.com/volkporter/status/1295899925209899009}{assigned
to share a room}, bathroom or
\href{https://twitter.com/sarahortbal/status/1298099716127895553}{dorm}
--- conditions that they worried could foster infections. In some cases,
their colleges then improved services for quarantined students.

\textbf{College students are also being shamed on social media for their
behavior.}

Yes, some kids are
\href{https://www.nbcnews.com/news/us-news/more-20-nyu-students-suspended-breaking-coronavirus-rules-school-says-n1239443}{partying}
or
\href{https://www.al.com/news/2020/08/this-has-to-become-a-cause-careless-partying-at-ua-leads-to-campus-restrictions.html}{going
to bars in large numbers without masks}. But epidemiologists said some
schools also made the risks worse by failing to make systemic changes to
help curtail the virus. They also said some schools have significantly
reduced occupancy in dorms, a change that could help hinder outbreaks.

\textbf{Sending infected students home is dangerous because it}
\textbf{\href{https://www.nytimes3xbfgragh.onion/2020/09/09/briefing/astrazeneca-california-wildfires-justice-department-your-wednesday-briefing.html}{risks
spreading the coronavirus to their families and communities}. What
should colleges do?}

Public health experts say the best practice is for schools to care for
the mental and physical health of students who are quarantined, and not
leave them to fend for themselves.

Many schools didn't seem to have a plan in place to closely monitor and
care for students in isolation dorms, and hadn't envisioned what it's
like for an 18-year-old who gets sick and feels cut off.

\textbf{What are examples of colleges that did make useful changes?}

Tulane University has nurses on staff 24-7 in a dorm for students with
infections. The nurses deliver meals three times a day and check on
students to make sure they're OK.

Tufts University created modular, individual isolation housing units in
a parking lot for students with virus infections. School officials said
they didn't want to put sick students in old dorms that lacked elevators
--- which might be needed to transport a student to a hospital.

\textbf{You previously reported on workplaces}
\textbf{\href{https://www.nytimes3xbfgragh.onion/2020/06/22/business/virus-office-workplace-return.html}{trying
to protect employees}} \textbf{from the coronavirus. How are colleges
acting differently or the same?}

One similarity is that workplaces have used a lot of
\href{https://www.nytimes3xbfgragh.onion/2020/05/11/technology/coronavirus-worker-testing-privacy.html}{unproven
or iffy technology}, like fever screening devices, that make people feel
safer but might not actually do much to mitigate coronavirus risks.
Universities are now
\href{https://www.nytimes3xbfgragh.onion/2020/08/19/business/alabama-uab-coronavirus-tests.html}{going
ahead with some of the same technologies}, when they could be using a
more proven technique: frequent virus testing.

\emph{If you don't already get this newsletter in your inbox,}
\href{https://www.nytimes3xbfgragh.onion/newsletters/signup/OT}{\emph{please
sign up here}}\emph{.}

\begin{center}\rule{0.5\linewidth}{\linethickness}\end{center}

\hypertarget{why-your-wildfire-photos-dont-look-like-real-life}{%
\subsection{Why your wildfire photos don't look like real
life}\label{why-your-wildfire-photos-dont-look-like-real-life}}

The sky in places on the West Coast turned a
\href{https://www.nytimes3xbfgragh.onion/2020/09/09/us/pictures-photos-california-fires.html}{murky
orange this week because of wildfires}. But some people said that photos
they took on their phones made their apocalyptic skies look almost
normal.

What gives? Well, digital cameras try to take snapshots that look better
than reality.

``Any camera doesn't see exactly what the human eye sees; it's not an
exact duplication,''
\href{https://www.nytimes3xbfgragh.onion/by/james-estrin}{James Estrin},
a staff photographer for The New York Times, told me. Most smartphones,
he said, are ``programmed to make the most pleasing photos for people,
and that usually means a bright blue sky.''

Imagine the software in your smartphone camera digesting that
\href{https://twitter.com/AirDistrict/status/1303746736414883840}{eerie
orange hue}, and figuring that is not how the sky is supposed to look.
That's why some people were having trouble capturing how scary it looked
outside their windows.

\href{https://www.nytimes3xbfgragh.onion/spotlight/schools-reopening?action=click\&pgtype=Article\&state=default\&region=MAIN_CONTENT_3\&context=storylines_keepup}{}

\hypertarget{school-reopenings-}{%
\subsubsection{School Reopenings ›}\label{school-reopenings-}}

\hypertarget{back-to-school}{%
\paragraph{Back to School}\label{back-to-school}}

Updated Sept. 11, 2020

The latest on how schools are reopening amid the pandemic.

\begin{itemize}
\item
  \begin{itemize}
  \tightlist
  \item
    School officials in Des Moines are refusing to hold in-person
    classes,
    \href{https://www.nytimes3xbfgragh.onion/2020/09/10/us/des-moines-school-opening-coronavirus.html?action=click\&pgtype=Article\&state=default\&region=MAIN_CONTENT_3\&context=storylines_keepup}{despite
    an order from Iowa's governor and a judge's ruling}, risking school
    funding and their jobs because they think it's unsafe.
  \item
    The University of Illinois at Urbana-Champaign had one of the most
    comprehensive plans by a major college to keep the virus under
    control. But it
    \href{https://www.nytimes3xbfgragh.onion/2020/09/10/health/university-illinois-covid.html?action=click\&pgtype=Article\&state=default\&region=MAIN_CONTENT_3\&context=storylines_keepup}{failed
    to account for students partying}.
  \item
    College students are
    \href{https://www.nytimes3xbfgragh.onion/2020/09/10/technology/coronavirus-quarantines-college.html?action=click\&pgtype=Article\&state=default\&region=MAIN_CONTENT_3\&context=storylines_keepup}{using
    apps to shame their schools}~into better coronavirus plans.
  \item
    For some families, the pandemic
    \href{https://www.nytimes3xbfgragh.onion/2020/09/10/parenting/family-second-language-coronavirus.html?action=click\&pgtype=Article\&state=default\&region=MAIN_CONTENT_3\&context=storylines_keepup}{has
    meant a return to their native languages}.
  \end{itemize}
\end{itemize}

James said that most of the time, we
\href{https://www.washingtonpost.com/technology/2018/11/14/your-smartphone-photos-are-totally-fake-you-love-it/}{want
cameras to tinker with our snapshots}. People like me who aren't capable
photographers don't want to think about exposure times, shutter speeds
or color balance. And I want my phone to make my photos less blurry or
brighten images from a dark restaurant. Reality is overrated.

But for people who are frustrated that their smartphones aren't
accurately capturing what they see, there are apps like Snapseed and
Halide that let people adjust the color on their smartphone-shot photos.
(Check out these
\href{https://twitter.com/sarahfrier/status/1303706996873461760}{before-and-after
app-adjusted shots} from a Bloomberg News journalist in San Francisco.)

James said apps like Photos included on iPhones have edit options, and
choosing ``warmer'' colors will restore those photos of the orange skies
to something closer to what people see with their own eyes.

``They are extraordinary cameras in general,'' James said about our
smartphones. Some of his iPhone photos have been published in The Times,
too.

\emph{You deserve more interesting and fun things for your ear holes.
Let me point you to ``Sway,'' a new podcast about power and influence
from my colleagues at Times Opinion and the tech journalist Kara
Swisher.}
\href{https://www.nytimes3xbfgragh.onion/2020/09/10/opinion/sway-kara-swisher-trailer.html}{\emph{Check
out the trailer}}\emph{.}

\begin{center}\rule{0.5\linewidth}{\linethickness}\end{center}

\hypertarget{before-we-go-}{%
\subsection{Before we go \ldots{}}\label{before-we-go-}}

\begin{itemize}
\item
  \textbf{Who is responsible for workers who aren't employees?} Uber and
  other ``gig'' companies classify their workers as contractors and not
  employees, leaving a legal gray area about who is responsible for
  injuries or mistreatment on the job.

  My colleagues Kellen Browning and Kate Conger write that a civil
  rights nonprofit is asking California regulators to step up
  protections for house cleaners who booked work through a gig app
  called Handy and said they were
  \href{https://www.nytimes3xbfgragh.onion/2020/09/10/business/handy-service-cleaners-harassment.html}{sexually
  harassed by clients} and couldn't get Handy to address it.
\item
  \textbf{He helps make sure ``the babies'' can do remote school:}
  Online school stinks, but the education news website The 74 has a
  lovely article about the
  \href{https://www.the74million.org/article/from-i-t-guy-to-mvp-the-pandemic-thrusts-san-antonio-isds-ken-thompson-into-the-center-of-the-action/}{head
  of information technology for San Antonio's schools}. He helped prep
  teachers for remote instruction and set up a tech support help desk
  that fielded up to 1,400 calls from families on the first day of
  virtual school. He and other staff members refer to students,
  affectionately, as ``the babies.''

  (I first read about this in The Times's
  \href{https://www.nytimes3xbfgragh.onion/2020/09/09/us/schools-reopening-coronavirus.html}{Coronavirus
  Schools Briefing}, which you should
  \href{https://www.nytimes3xbfgragh.onion/newsletters/coronavirus-schools-briefing}{sign
  up} for!)
\item
  \textbf{``We need ways to politely disconnect.''} YES, PLEASE, to this
  OneZero columnist's plea for
  \href{https://onezero.medium.com/now-is-the-time-to-bring-back-away-messages-d53b3fcf0af3}{universal
  digital ``away messages.''} These pop-up notices, popularized by
  2000s-era AOL, automatically notify people who are emailing, messaging
  and texting us that we are trying not to be distracted and will read
  all that stuff later. Or never.
\end{itemize}

\hypertarget{hugs-to-this}{%
\subsubsection{Hugs to this}\label{hugs-to-this}}

I envy
\href{https://www.tiktok.com/@caitlindorann/video/6866129235405245702?_d=secCgsIARCbDRgBIAIoARI\%2BCjwkb1OiMDqnufxPGDyf3tc4Sx0KYUTRMxXC2zak\%2BQx1xVyp4PXqIZn7N7NvJcV\%2FgsFY2olKQMBxp9XhwhYaAA\%3D\%3D\&language=en\&preview_pb=0\&sec_user_id=MS4wLjABAAAAMRA6cray9TND3Cj4-GeI-c6RkJGLCHeGC0tcYt2Htx6QIRj5ul4usQ41y9ncry57\&share_app_name=musically\&share_item_id=6866129235405245702\&share_link_id=befd23d1-d72e-48a0-9419-13638bb45a8d\&timestamp=1599097671\&u_code=dbilh20db37fag\&user_id=6811584884517651461\&utm_campaign=client_share\&utm_medium=android\&utm_source=copy\&source=h5_m}{the
life of Tiptoe the 175-pound tortoise}, whose big outing was a stroll
across the street --- motivated by his ``walking snackies.''

\begin{center}\rule{0.5\linewidth}{\linethickness}\end{center}

\emph{We want to hear from you. Tell us what you think of this
newsletter and what else you'd like us to explore. You can reach us at}
\href{mailto:ontech@NYTimes.com?subject=On\%20Tech\%20Feedback}{\emph{ontech@NYTimes.com.}}
**

\emph{If you don't already get this newsletter in your inbox,}
\href{https://www.nytimes3xbfgragh.onion/newsletters/signup/OT}{\emph{please
sign up here}}\emph{.}

Advertisement

\protect\hyperlink{after-bottom}{Continue reading the main story}

\hypertarget{site-index}{%
\subsection{Site Index}\label{site-index}}

\hypertarget{site-information-navigation}{%
\subsection{Site Information
Navigation}\label{site-information-navigation}}

\begin{itemize}
\tightlist
\item
  \href{https://help.nytimes3xbfgragh.onion/hc/en-us/articles/115014792127-Copyright-notice}{©~2020~The
  New York Times Company}
\end{itemize}

\begin{itemize}
\tightlist
\item
  \href{https://www.nytco.com/}{NYTCo}
\item
  \href{https://help.nytimes3xbfgragh.onion/hc/en-us/articles/115015385887-Contact-Us}{Contact
  Us}
\item
  \href{https://www.nytco.com/careers/}{Work with us}
\item
  \href{https://nytmediakit.com/}{Advertise}
\item
  \href{http://www.tbrandstudio.com/}{T Brand Studio}
\item
  \href{https://www.nytimes3xbfgragh.onion/privacy/cookie-policy\#how-do-i-manage-trackers}{Your
  Ad Choices}
\item
  \href{https://www.nytimes3xbfgragh.onion/privacy}{Privacy}
\item
  \href{https://help.nytimes3xbfgragh.onion/hc/en-us/articles/115014893428-Terms-of-service}{Terms
  of Service}
\item
  \href{https://help.nytimes3xbfgragh.onion/hc/en-us/articles/115014893968-Terms-of-sale}{Terms
  of Sale}
\item
  \href{https://spiderbites.nytimes3xbfgragh.onion}{Site Map}
\item
  \href{https://help.nytimes3xbfgragh.onion/hc/en-us}{Help}
\item
  \href{https://www.nytimes3xbfgragh.onion/subscription?campaignId=37WXW}{Subscriptions}
\end{itemize}
