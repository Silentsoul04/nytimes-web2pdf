Sections

SEARCH

\protect\hyperlink{site-content}{Skip to
content}\protect\hyperlink{site-index}{Skip to site index}

\href{https://www.nytimes3xbfgragh.onion/section/well/family}{Family}

\href{https://myaccount.nytimes3xbfgragh.onion/auth/login?response_type=cookie\&client_id=vi}{}

\href{https://www.nytimes3xbfgragh.onion/section/todayspaper}{Today's
Paper}

\href{/section/well/family}{Family}\textbar{}The Causes of Estrangement,
and How Families Heal

\url{https://nyti.ms/2ZpCXUG}

\begin{itemize}
\item
\item
\item
\item
\item
\item
\end{itemize}

\href{https://www.nytimes3xbfgragh.onion/spotlight/at-home?action=click\&pgtype=Article\&state=default\&region=TOP_BANNER\&context=at_home_menu}{At
Home}

\begin{itemize}
\tightlist
\item
  \href{https://www.nytimes3xbfgragh.onion/2020/09/07/travel/route-66.html?action=click\&pgtype=Article\&state=default\&region=TOP_BANNER\&context=at_home_menu}{Cruise
  Along: Route 66}
\item
  \href{https://www.nytimes3xbfgragh.onion/2020/09/04/dining/sheet-pan-chicken.html?action=click\&pgtype=Article\&state=default\&region=TOP_BANNER\&context=at_home_menu}{Roast:
  Chicken With Plums}
\item
  \href{https://www.nytimes3xbfgragh.onion/2020/09/04/arts/television/dark-shadows-stream.html?action=click\&pgtype=Article\&state=default\&region=TOP_BANNER\&context=at_home_menu}{Watch:
  Dark Shadows}
\item
  \href{https://www.nytimes3xbfgragh.onion/interactive/2020/at-home/even-more-reporters-editors-diaries-lists-recommendations.html?action=click\&pgtype=Article\&state=default\&region=TOP_BANNER\&context=at_home_menu}{Explore:
  Reporters' Google Docs}
\end{itemize}

Advertisement

\protect\hyperlink{after-top}{Continue reading the main story}

Supported by

\protect\hyperlink{after-sponsor}{Continue reading the main story}

\hypertarget{the-causes-of-estrangement-and-how-families-heal}{%
\section{The Causes of Estrangement, and How Families
Heal}\label{the-causes-of-estrangement-and-how-families-heal}}

For those who reconcile with estranged relatives, the key is ``letting
go of the attempt to have the other person see the past as they saw
it,'' the author of a new book says.

\includegraphics{https://static01.graylady3jvrrxbe.onion/images/2020/09/09/well/00well-pillemer2/00well-pillemer2-articleLarge.jpg?quality=75\&auto=webp\&disable=upscale}

By \href{https://www.nytimes3xbfgragh.onion/by/paula-span}{Paula Span}

\begin{itemize}
\item
  Sept. 10, 2020
\item
  \begin{itemize}
  \item
  \item
  \item
  \item
  \item
  \item
  \end{itemize}
\end{itemize}

A 21-year-old college student who hasn't spoken to her mother since high
school.

A woman who cannot get along with her daughter-in-law, and who therefore
has no contact with her son.

Three siblings who stopped speaking because of a disputed inheritance 30
years ago.

Family estrangement --- a topic once so distressing and shameful that
people hesitated to discuss it --- is drawing more attention as some
tell their stories and researchers delve into its causes and
consequences.

Karl Pillemer, a family sociologist at Cornell University, has just
published
\href{https://www.penguinrandomhouse.com/books/595000/fault-lines-by-karl-pillemer-phd/}{``Fault
Lines: Fractured Families and How to Mend Them,''} a book that provides
something rare in this realm --- actual data.

He asked participants in a representative national survey, ``Is there a
relative with whom you have no contact?'' Among the 1,340 people who
answered an online questionnaire, a substantial 27 percent reported
being estranged from a family member. And half had been estranged for
four years or more.

During a five-year period, Dr. Pillemer and his colleagues conducted
hundreds of interviews with people estranged from their parents, adult
children, siblings or other relatives. They also interviewed many who
had reconciled, and Dr. Pillemer has passed along their advice in his
book.

(Another book on the topic,
\href{https://www.penguinrandomhouse.com/books/622584/rules-of-estrangement-by-joshua-coleman-phd/}{``Rules
of Estrangement''} by Joshua Coleman, a psychologist in the Bay Area, is
coming in March.)

I spoke by phone with Dr. Pillemer about his findings. Our conversation
has been edited and condensed.

Image

Karl Pillemer

\textbf{Paula Span:} \textbf{We appear to be hearing more about
estrangement, when for so long it seemed to be something people just
didn't talk about.}

\textbf{Karl Pillemer:} It was astonishing to me to find so little
scientific literature on it.

But high-profile celebrities have brought it to the forefront. Prince
Harry and Meghan Markle. Angelina Jolie, famously estranged from her
father, Jon Voight. Tara Westover's book, ``Educated.''

Estrangement may have been less common when families lived closer to one
another and there was more routine interaction, a social norm that you
maintain contact at all costs. When I interview older people, they often
describe hanging in with their families no matter what.

With the baby boomers and younger, there's more of a sense that if the
relationship's not working out, they can move on.

This phenomenon of cutting off or being cut off from a family member is
strikingly common in America.

\textbf{Plus, you didn't find differences when considering gender or
race or education level --- this can happen to anyone. Can you explain
what you call pathways, the most common reasons or explanations for
estrangement?}

One is difficult childhood histories: Abusive parenting, harsh
parenting, memories of parental favoritism --- people don't always get
over those. They carry them into adulthood.

Second, divorce, no matter when it appears in the life cycle. Children
are more likely to lose contact with fathers, the research shows, but
the disruption can weaken the ties to both parents.

Also, the problematic in-law. In a striking number of cases, someone in
the family of origin thinks you've married the wrong person, and the
classic conflict between the demands of your own family and your partner
can't be resolved.

Then there's money. There's lots of resentment around how inheritances
are distributed. You can divide your money among your kids, but you
can't divide tangible property like heirlooms or a summer house. But
also business deals gone wrong can contribute, or loans unrepaid.

And unmet expectations. An archetypal example involves caregiving for
aging parents: Sibling A is left with all the care and Sibling B doesn't
do anything, so Sibling A says, ``I'm done.''

Finally, lifestyle and value discrepancies, especially in parent and
child relationships. A kid coming out as gay or lesbian. A religious
conversion. Different politics.

\textbf{You point out that when people look back at what went wrong,
they have divergent views of the past. They can't even agree on what
actually occurred or who said what.}

Right. It's not a realistic expectation to believe that a sibling, a
parent, an adult child is going to come over to your view of these past
events. But it's an almost indelible wish. People are often in long-term
estrangements because the other person supposedly can't see the reality
of the past.

We know from psychology that we love our own narratives and we don't
give them up. You're not going to align the perspectives of the sister
who felt she was emotionally abused and the brother who thought he was
just doing normal teasing.

\textbf{You describe estrangement as a wound that won't heal.}

People experience estrangement as isolating and shameful. They often
experience guilt. And there's stigma attached. Other people think
there's something wrong with your family.

Analyzing the survey data, there were correlations between being
estranged and feeling anxious or depressed or isolated.

\textbf{Your ``reconcilers'' --- about 100 of them in your interview
sample --- weren't obviously different from the others, were they?}

They were remarkably similar in what caused the estrangement, how
upsetting it was and how long it had gone on. If I showed you accounts
of how the estrangement occurred and how difficult it was, you wouldn't
be able to distinguish between those who eventually reconciled and those
that haven't.

\textbf{What shifted for these reconcilers? After years of estrangement,
what made contact possible?}

The situation had changed or the person had changed. If the issue was a
problematic in-law and there was a divorce, the barrier wasn't there
anymore.

Or people began to feel the pressure of a limited time horizon.
Observing their own or others' health problems made them think they
could no longer put it off.

And just the passage of time. It let some of the angry feelings
dissipate. One of my interviewees said, ``Boy, the argument that started
it seems so trivial now.''

\textbf{Your reconcilers offered some helpful strategies, one of which
was letting go of the past. They don't mean that you forgive and forget,
but that you accept that you and the other person won't ever have the
same view of what happened.}

People who reconcile describe the experience as letting go of the
attempt to have the other person see the past as they saw it.

\textbf{They also talk about changing their expectations.}

Reconciliation is usually imperfect, even if it's good. So determining
the least you can accept in the relationship was a very useful exercise.

It did involve settling for less, in most cases. It was still worth it
to be back in the relationship.

\textbf{Here's a popular word: boundaries. How do they work in resolving
estrangement?}

The reconcilers developed very clear terms, specific conditions under
which the relationship could exist. ``If you're in my house, you can't
say anything negative about my husband. That's the rule.''

\textbf{I know some readers will respond that they feel fully justified
in cutting off contact. And that anyone urging them to reconcile --- or
simply telling them how to reconcile --- doesn't accept their view that
they did the right thing.}

I'm not recommending that individuals reconcile. But for the vast
majority who do, it turned out to be a positive, sometimes even
life-changing experience. They found it to be a major life
accomplishment.

The number of people who were completely estranged from a close relative
and identified that as a positive event, one they were glad had
occurred, were certainly a minority.

I would say to the people who feel that it was the best thing they ever
did and they feel liberated as a result: More power to you. But for most
people in estrangement, that's not their experience. They feel there's
something missing from their lives.

Advertisement

\protect\hyperlink{after-bottom}{Continue reading the main story}

\hypertarget{site-index}{%
\subsection{Site Index}\label{site-index}}

\hypertarget{site-information-navigation}{%
\subsection{Site Information
Navigation}\label{site-information-navigation}}

\begin{itemize}
\tightlist
\item
  \href{https://help.nytimes3xbfgragh.onion/hc/en-us/articles/115014792127-Copyright-notice}{©~2020~The
  New York Times Company}
\end{itemize}

\begin{itemize}
\tightlist
\item
  \href{https://www.nytco.com/}{NYTCo}
\item
  \href{https://help.nytimes3xbfgragh.onion/hc/en-us/articles/115015385887-Contact-Us}{Contact
  Us}
\item
  \href{https://www.nytco.com/careers/}{Work with us}
\item
  \href{https://nytmediakit.com/}{Advertise}
\item
  \href{http://www.tbrandstudio.com/}{T Brand Studio}
\item
  \href{https://www.nytimes3xbfgragh.onion/privacy/cookie-policy\#how-do-i-manage-trackers}{Your
  Ad Choices}
\item
  \href{https://www.nytimes3xbfgragh.onion/privacy}{Privacy}
\item
  \href{https://help.nytimes3xbfgragh.onion/hc/en-us/articles/115014893428-Terms-of-service}{Terms
  of Service}
\item
  \href{https://help.nytimes3xbfgragh.onion/hc/en-us/articles/115014893968-Terms-of-sale}{Terms
  of Sale}
\item
  \href{https://spiderbites.nytimes3xbfgragh.onion}{Site Map}
\item
  \href{https://help.nytimes3xbfgragh.onion/hc/en-us}{Help}
\item
  \href{https://www.nytimes3xbfgragh.onion/subscription?campaignId=37WXW}{Subscriptions}
\end{itemize}
