Sections

SEARCH

\protect\hyperlink{site-content}{Skip to
content}\protect\hyperlink{site-index}{Skip to site index}

\href{https://www.nytimes3xbfgragh.onion/section/world/europe}{Europe}

\href{https://myaccount.nytimes3xbfgragh.onion/auth/login?response_type=cookie\&client_id=vi}{}

\href{https://www.nytimes3xbfgragh.onion/section/todayspaper}{Today's
Paper}

\href{/section/world/europe}{Europe}\textbar{}Brexit Talks Hit Crisis as
Boris Johnson Rejects Ultimatum

\url{https://nyti.ms/3hj85LN}

\begin{itemize}
\item
\item
\item
\item
\item
\end{itemize}

Advertisement

\protect\hyperlink{after-top}{Continue reading the main story}

Supported by

\protect\hyperlink{after-sponsor}{Continue reading the main story}

\hypertarget{brexit-talks-hit-crisis-as-boris-johnson-rejects-ultimatum}{%
\section{Brexit Talks Hit Crisis as Boris Johnson Rejects
Ultimatum}\label{brexit-talks-hit-crisis-as-boris-johnson-rejects-ultimatum}}

Europe has demanded the quick withdrawal of draft Brexit legislation
that breaks international law. Britain has refused.

\includegraphics{https://static01.graylady3jvrrxbe.onion/images/2020/09/10/world/10brexit/10brexit-articleLarge.jpg?quality=75\&auto=webp\&disable=upscale}

\href{https://www.nytimes3xbfgragh.onion/by/stephen-castle}{\includegraphics{https://static01.graylady3jvrrxbe.onion/images/2018/10/08/multimedia/author-stephen-castle/author-stephen-castle-thumbLarge.png}}\href{https://www.nytimes3xbfgragh.onion/by/mark-landler}{\includegraphics{https://static01.graylady3jvrrxbe.onion/images/2019/10/22/reader-center/author-mark-landler/author-mark-landler-thumbLarge-v3.png}}

By \href{https://www.nytimes3xbfgragh.onion/by/stephen-castle}{Stephen
Castle} and
\href{https://www.nytimes3xbfgragh.onion/by/mark-landler}{Mark Landler}

\begin{itemize}
\item
  Sept. 10, 2020
\item
  \begin{itemize}
  \item
  \item
  \item
  \item
  \item
  \end{itemize}
\end{itemize}

LONDON --- Britain and the European Union were on a collision course
Thursday, after Brussels demanded the speedy withdrawal of proposed
Brexit legislation that the government of Prime Minister Boris Johnson
\href{https://www.nytimes3xbfgragh.onion/2020/09/08/world/europe/boris-johnson-brexit-northern-ireland.html?searchResultPosition=1}{has
admitted would breach international law}.

The ultimatum, which Mr. Johnson's government swiftly rejected, is the
most serious crisis yet to hit negotiations on a trade agreement for
after Britain leaves the European Union's trade zone. The talks have
failed to make any significant progress yet have somehow remained alive.

The dispute suggests that the moment of truth is fast approaching.

The proposed legislation would override aspects of
\href{https://www.nytimes3xbfgragh.onion/2020/09/08/world/europe/boris-johnson-brexit-northern-ireland.html}{a
landmark Brexit withdrawal agreement} involving the treatment of the
border between Northern Ireland, which is part of the United Kingdom,
and Ireland, which will remain in the European Union.

In a toughly worded statement that underscored the growing tension, the
European Commission --- the bloc's executive arm --- suggested it was
ready to take legal action against the British government, accusing it
of threatening Northern Ireland's fragile peace process.

Mr. Johnson himself struck the withdrawal agreement with the European
Union and championed it during last year's election on his way to a
landslide victory. It was crafted in part to prevent the creation of a
hard border between Northern Ireland and Ireland.

But the vice president of the European Commission, Maros Sefcovic, said
Mr. Johnson's draft bill would constitute ``an extremely serious
violation'' of international law, and demanded its withdrawal by the end
of the month. As worded, the legislation would give Britain the right to
decide how to implement sensitive aspects of the treaty unilaterally
rather than through negotiation with the European Union as stipulated
under law.

That demand was rejected by Michael Gove, a senior British cabinet
minister, who said the British government had made it ``perfectly
clear'' it would not withdraw the bill even though it has admitted that
it breaks international law in a limited way.

\includegraphics{https://static01.graylady3jvrrxbe.onion/images/2020/09/10/world/10brexit2/10brexit2-articleLarge.jpg?quality=75\&auto=webp\&disable=upscale}

In issuing its ultimatum, the European Union stopped short of taking an
irreversible step or walking away from the trade talks. And the two
sides have agreed to keep on talking next week after what David Frost,
Britain's main negotiator, described as ``useful exchanges'' that took
place separately on Thursday.

He and his counterpart, Michel Barnier, need to reach a deal on trade
next month if it is to be ratified by the end of the year when Britain
stops trading under European Union's rules.

``With the E.U. saying `you have until the end of the month,' one side
has to back down,'' said David Henig, director of the UK Trade Policy
Project at the European Center for International Political Economy, a
research institute, ``and that is a difficult situation if we are to get
a deal by mid-October. The E.U. is not going to back down.''

There is no sign Mr. Johnson plans to back down to either, claiming
earlier this week that failing to reach a trade deal with the European
Union would still be
\href{https://www.nytimes3xbfgragh.onion/2020/09/08/world/europe/boris-johnson-brexit-northern-ireland.html}{a
``good outcome.''} Though he has shown flexibility in the past, and his
government has made a series of policy reversals, the government has
taken an uncompromising line on Brexit, giving few hints that it intends
to compromise, let alone give way to Brussels.

While another crisis in Brexit negotiation seemed almost inevitable, few
expected the confrontation to be about putting in place an agreement
that had already been signed and sealed. Britain argues that its new
draft legislation is to provide a fallback option in case it cannot
strike a trade deal with Brussels, but it not even clear that it will
pass through Parliament unscathed.

Critics said the risk of miscalculation for Mr. Johnson was high. Far
from being a peripheral issue, the provisions on the Irish border are
the centerpiece of the withdrawal agreement. They united the bloc's 27
members during protracted, divisive negotiations with Britain over
Brexit, because they are seen as so closely intertwined with maintaining
Irish peace.

``The E.U. is not going to allow peace in Ireland to be leveraged in a
negotiation,'' said Mujtaba Rahman, an expert on Brexit at the political
risk consultancy, Eurasia Group. ``They see very clearly what the
government is doing, and they're not going to be bullied. Britain is
playing a very dangerous game of chicken with Europe.''

Image

Movement between Northern Ireland and Ireland remains a key holdup in
Brexit negotiations.Credit...Paulo Nunes dos Santos for The New York
Times

There are other international consequences to the gamesmanship.

Democratic leaders in Washington warned Mr. Johnson that he was putting
his hopes for a trans-Atlantic trade deal at risk. Undermining the
Northern Ireland provisions in the withdrawal agreement, they said,
would jeopardize the Good Friday Accord, a peace deal brokered under
President Bill Clinton that ended decades of sectarian violence in
Ireland.

``If the U.K. violates that international treaty and Brexit undermines
the Good Friday Accord, there will be absolutely no chance of a
U.S.-U.K. trade agreement passing the Congress,'' Speaker Nancy Pelosi
of California said in a statement on Wednesday.

Representative Richard E. Neal, a Massachusetts Democrat who is chairman
of the powerful House Ways and Means Committee, urged Britain to uphold
the withdrawal agreement.

``I sincerely hope the British government upholds the rule of law and
delivers on the commitments it made during Brexit negotiations,
particularly in regard to the Irish border,'' Mr. Neal said in a
statement.

``Every political party on the island opposes a return of a hard
border,'' he said.

The Democratic nominee, former Vice President Joseph R. Biden Jr., is a
staunch defender of Ireland who said he would have voted against Brexit
if he were a British citizen. One of his top foreign policy advisers,
Antony Blinken, suggested in a tweet this week that Mr. Biden was
watching the situation carefully.

Mr. Biden, he wrote, ``is committed to preserving the hard-earned peace
\& stability in Northern Ireland.'' He added, ``As the UK and EU work
out their relationship, any arrangements must protect the Good Friday
Agreement and prevent the return of a hard border.''

Advertisement

\protect\hyperlink{after-bottom}{Continue reading the main story}

\hypertarget{site-index}{%
\subsection{Site Index}\label{site-index}}

\hypertarget{site-information-navigation}{%
\subsection{Site Information
Navigation}\label{site-information-navigation}}

\begin{itemize}
\tightlist
\item
  \href{https://help.nytimes3xbfgragh.onion/hc/en-us/articles/115014792127-Copyright-notice}{©~2020~The
  New York Times Company}
\end{itemize}

\begin{itemize}
\tightlist
\item
  \href{https://www.nytco.com/}{NYTCo}
\item
  \href{https://help.nytimes3xbfgragh.onion/hc/en-us/articles/115015385887-Contact-Us}{Contact
  Us}
\item
  \href{https://www.nytco.com/careers/}{Work with us}
\item
  \href{https://nytmediakit.com/}{Advertise}
\item
  \href{http://www.tbrandstudio.com/}{T Brand Studio}
\item
  \href{https://www.nytimes3xbfgragh.onion/privacy/cookie-policy\#how-do-i-manage-trackers}{Your
  Ad Choices}
\item
  \href{https://www.nytimes3xbfgragh.onion/privacy}{Privacy}
\item
  \href{https://help.nytimes3xbfgragh.onion/hc/en-us/articles/115014893428-Terms-of-service}{Terms
  of Service}
\item
  \href{https://help.nytimes3xbfgragh.onion/hc/en-us/articles/115014893968-Terms-of-sale}{Terms
  of Sale}
\item
  \href{https://spiderbites.nytimes3xbfgragh.onion}{Site Map}
\item
  \href{https://help.nytimes3xbfgragh.onion/hc/en-us}{Help}
\item
  \href{https://www.nytimes3xbfgragh.onion/subscription?campaignId=37WXW}{Subscriptions}
\end{itemize}
