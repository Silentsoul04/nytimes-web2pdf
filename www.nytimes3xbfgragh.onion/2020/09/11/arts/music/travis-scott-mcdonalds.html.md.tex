Sections

SEARCH

\protect\hyperlink{site-content}{Skip to
content}\protect\hyperlink{site-index}{Skip to site index}

\href{https://www.nytimes3xbfgragh.onion/section/arts/music}{Music}

\href{https://myaccount.nytimes3xbfgragh.onion/auth/login?response_type=cookie\&client_id=vi}{}

\href{https://www.nytimes3xbfgragh.onion/section/todayspaper}{Today's
Paper}

\href{/section/arts/music}{Music}\textbar{}Travis Scott Meets McDonald's
(It's Lit!)

\url{https://nyti.ms/2RirtOo}

\begin{itemize}
\item
\item
\item
\item
\item
\item
\end{itemize}

Advertisement

\protect\hyperlink{after-top}{Continue reading the main story}

Supported by

\protect\hyperlink{after-sponsor}{Continue reading the main story}

Critic's Notebook

\hypertarget{travis-scott-meets-mcdonalds-its-lit}{%
\section{Travis Scott Meets McDonald's (It's
Lit!)}\label{travis-scott-meets-mcdonalds-its-lit}}

A partnership between the rapper and the fast-food chain is a melding of
the merchandising minds.

\includegraphics{https://static01.graylady3jvrrxbe.onion/images/2020/09/12/arts/10travis-mcd1/merlin_176780883_ebcfcf8c-fe8d-43d5-b41a-65c25cb5d218-articleLarge.jpg?quality=75\&auto=webp\&disable=upscale}

By \href{https://www.nytimes3xbfgragh.onion/by/jon-caramanica}{Jon
Caramanica}

\begin{itemize}
\item
  Sept. 11, 2020
\item
  \begin{itemize}
  \item
  \item
  \item
  \item
  \item
  \item
  \end{itemize}
\end{itemize}

During my childhood years, my family's kitchen cabinets used to be lousy
with the sorts of commemorative glassware you'd get at Burger King or
McDonald's --- a Luke Skywalker here, a Snoopy there. These were part of
huge global marketing pushes for creative projects being milked for
every last ounce of intellectual property, but also savvy positioning by
the restaurants. Fast-food companies have long attempted to stave off
disposability by piggybacking on broader cultural moments, hoping to
extend their reach beyond the comestible into the permanently tangible.

In 2020, a fast-food chain looking for equivalent big-tent cultural
relevance has few more compelling places to turn than hip-hop, the
cultural arena with the most natural and ambitious gift for
merchandising. And in hip-hop, there are fewer more ambitious personal
branders than Travis Scott, who has his own festival, several Nike
collaborations, a cereal, a Hot Wheels and much more
\href{https://www.complex.com/style/travis-scott-brand-collaboration-timeline}{to
his name}.

That said, the collaboration between McDonald's and Scott, which began
this week and includes a range of merchandise and a limited-edition
\href{https://www.mcdonalds.com/us/en-us/travis-scott-meal.html}{meal},
initially seems preposterous --- what does McDonald's know about the
right singing-to-rapping ratio? What does Scott know about the right
salt-to-fry ratio?

Juggernauts gonna juggernaut, though. And each gets something from the
other. For Scott, it's the scale of the flex --- a partnership with a
brand the magnitude of McDonald's is essentially unheard-of. (What's
next: Walmart? Berkshire Hathaway?) It's a way to slip his aesthetics
into the global mainstream through ads and products, and also something
that doesn't exist in music anymore: physical distribution locations.
(There are over 13,000 McDonald's restaurants in the United States.)

In exchange, McDonald's gets some refracted cool and the satisfaction of
knowing that thousands of young people might find their way --- through
the co-branded merchandise --- into becoming walking billboards,
especially crucial given that while McDonald's remains among the most
valuable fast-food restaurant brands on the planet, with total global
revenue of around \$21 billion each of the last two years, it's still a
business in
\href{https://www.macrotrends.net/stocks/charts/MCD/mcdonalds/revenue}{overall
decline}, from a high of \$28 billion in 2013. Partnering with Scott is
a way to advertise to young people without all the burdens and potential
misfires of actually advertising to young people.

It would all be so sinister, so savagely instrumental, if it weren't so
effective. The range of products in the
\href{https://shop.travisscott.com/}{merchandise drop} is frankly
staggering. There are umpteen T-shirts --- some insert Scott's imprint
name, Cactus Jack, into the Golden Arches; some are inspired by early
1990s sports aesthetics. There are rugs, a lunchbox, socks, a tie, a
\$90 McNugget body pillow. As with most of Scott's merch, it's
well-designed, colorful, playful. The brown work jacket with ``Billions
and Billions Served'' embroidered on the back (\$128) could have been
right out of a dawn-of-the-'90s Beastie Boys video.

These garments are likely to look better in the rearview a couple of
decades from now. Though they're well-designed, wearing clothing
advertising the leading fast-food brand is, in a Sweetgreen era, an
unprogressive choice --- nostalgia tends to soften capitalist excesses,
though.

There's a television commercial, too, in which an action figure version
of Scott --- speaking in his real(?) voice --- showcases his meal:
``same order since back in Houston.'' Here, too, a mutual compromise:
McDonald's, potentially still skittish about aligning with a rapper,
swaps in an animated version in the ad. (Some franchisees apparently
\href{https://www.restaurantbusinessonline.com/marketing/mcdonalds-reveals-its-multi-level-deal-travis-scott\#}{opposed}
the partnership, citing Scott's risqué lyrics.)

And Scott retains a bit of personal mystery. From this action figure
commercial to his recent
\href{https://www.theverge.com/2020/4/23/21233637/travis-scott-fortnite-concert-astronomical-live-report}{concert
on the video game Fortnite}, he has been moving toward full time avatar
territory. He is already among the most reluctant of hip-hop stars,
almost never photographed with his eyes engaging the camera. And his
voice is generally digitally processed practically beyond recognition,
merely shrugging off the texture of reality. He is becoming an A.I.
musician long before the algorithms take over.

His aesthetics, though, he's willing to share. A collaboration at this
scale is maybe a final stop before a full-fledged brand of one's own ---
a Yeezy or a Fenty. Despite hip-hop's complete dominance of pop culture,
there is still a bit of a lag when it comes to the willingness of large
mainstream brands to work with hip-hop stars. It's still a light shock
to see DJ Khaled hawking for Geico, or Snoop Dogg for Corona (or Dunkin'
or the General or Tostitos).

McDonald's partnership with Scott may well be the savviest music/food
pairing since the Starbucks music program, which
\href{https://www.nytimes3xbfgragh.onion/2004/11/03/arts/music/would-you-like-an-extra-shot-of-music-with-that-macchiato.html}{placed
CDs from its Hear Records label} next to its registers. Which brings us
back to food. There is of course also a Travis Scott Meal, which costs
\$6 --- a specialty burger something like an amped-up Quarter Pounder
With Cheese, fries with barbecue sauce and a Sprite --- that sadly does
not come with a toy. Part of why the Scott/McDonald's alliance feels
different is because of the intimacy of food --- it's one thing to
attach a celebrity to a luxury item, but to attach one to a commodity
product is a far bolder statement.

A couple of days ago, Scott had a not very socially distanced
\href{https://losangeles.cbslocal.com/2020/09/08/rapper-travis-scott-mobbed-by-fans-at-downey-mcdonalds/}{launch
event at a McDonald's} in Downey, Calif. Scott's buddies wore special
shirts made for employees and cheesed for pictures
\href{https://www.instagram.com/p/CE5af3pJCR3/?utm_source=ig_web_copy_link}{over
the griddle}.

\includegraphics{https://static01.graylady3jvrrxbe.onion/images/2020/09/11/arts/11travis-mcd2/11travis-mcd2-articleLarge.jpg?quality=75\&auto=webp\&disable=upscale}

At the McDonald's closest to my house on Wednesday, though, there was
little hubbub --- just another day in the fry-guy trenches. A sign in
the outer window featured a glam shot of the meal and referred to it as
a ``limited time collab.'' On the video menu screen inside, a picture of
the sandwich appeared next to a scrawled Travis Scott catchphrase:
``It's Lit!'' I bought one and can confirm the sandwich tasted \ldots{}
exactly like McDonald's. I lasted one bite --- the Sprite was a deeply
necessary palate cleanse.

As merch goes, the Travis Scott Meal is imperfect in that it disappears
--- you've got nothing to show for it apart from oily skin and a mild
gastric hangover. As a collector, I was much more interested in the
grill slip, the small, grease-mottled piece of paper stuck to the top of
the box that indicates a special order, and which was marked ``The
Travis Scott.'' It is peak ephemera, utilitarian debris of a peculiar
cultural moment. I threw out the sandwich, and pocketed the slip.

Advertisement

\protect\hyperlink{after-bottom}{Continue reading the main story}

\hypertarget{site-index}{%
\subsection{Site Index}\label{site-index}}

\hypertarget{site-information-navigation}{%
\subsection{Site Information
Navigation}\label{site-information-navigation}}

\begin{itemize}
\tightlist
\item
  \href{https://help.nytimes3xbfgragh.onion/hc/en-us/articles/115014792127-Copyright-notice}{©~2020~The
  New York Times Company}
\end{itemize}

\begin{itemize}
\tightlist
\item
  \href{https://www.nytco.com/}{NYTCo}
\item
  \href{https://help.nytimes3xbfgragh.onion/hc/en-us/articles/115015385887-Contact-Us}{Contact
  Us}
\item
  \href{https://www.nytco.com/careers/}{Work with us}
\item
  \href{https://nytmediakit.com/}{Advertise}
\item
  \href{http://www.tbrandstudio.com/}{T Brand Studio}
\item
  \href{https://www.nytimes3xbfgragh.onion/privacy/cookie-policy\#how-do-i-manage-trackers}{Your
  Ad Choices}
\item
  \href{https://www.nytimes3xbfgragh.onion/privacy}{Privacy}
\item
  \href{https://help.nytimes3xbfgragh.onion/hc/en-us/articles/115014893428-Terms-of-service}{Terms
  of Service}
\item
  \href{https://help.nytimes3xbfgragh.onion/hc/en-us/articles/115014893968-Terms-of-sale}{Terms
  of Sale}
\item
  \href{https://spiderbites.nytimes3xbfgragh.onion}{Site Map}
\item
  \href{https://help.nytimes3xbfgragh.onion/hc/en-us}{Help}
\item
  \href{https://www.nytimes3xbfgragh.onion/subscription?campaignId=37WXW}{Subscriptions}
\end{itemize}
