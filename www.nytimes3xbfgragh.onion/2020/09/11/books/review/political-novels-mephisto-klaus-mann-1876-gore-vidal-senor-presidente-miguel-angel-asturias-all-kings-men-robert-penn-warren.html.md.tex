Sections

SEARCH

\protect\hyperlink{site-content}{Skip to
content}\protect\hyperlink{site-index}{Skip to site index}

\href{https://www.nytimes3xbfgragh.onion/section/books/review}{Book
Review}

\href{https://myaccount.nytimes3xbfgragh.onion/auth/login?response_type=cookie\&client_id=vi}{}

\href{https://www.nytimes3xbfgragh.onion/section/todayspaper}{Today's
Paper}

\href{/section/books/review}{Book Review}\textbar{}Stories of Then That
Still Hold Up Now

\url{https://nyti.ms/2Ri17w1}

\begin{itemize}
\item
\item
\item
\item
\item
\end{itemize}

Advertisement

\protect\hyperlink{after-top}{Continue reading the main story}

Supported by

\protect\hyperlink{after-sponsor}{Continue reading the main story}

\hypertarget{stories-of-then-that-still-hold-up-now}{%
\section{Stories of Then That Still Hold Up
Now}\label{stories-of-then-that-still-hold-up-now}}

Margaret Atwood, Héctor Tobar, Thomas Mallon and Brenda Wineapple on
older political novels they admire that have a lot to say about the
world today.

\begin{itemize}
\item
  Sept. 11, 2020
\item
  \begin{itemize}
  \item
  \item
  \item
  \item
  \item
  \end{itemize}
\end{itemize}

\emph{Inevitably, 2020 has been a year filled to the brim with books
about politics --- and not just in nonfiction. Novelists are as focused
on the state of the world as any journalist or Washington insider. We
decided to ask four accomplished writers to revisit a favorite political
novel from the past --- telling us why they admire it, and why it
remains relevant and timely (or timeless, if you prefer). --- John
Williams}

\hypertarget{margaret-atwood-on-mephisto-by-klaus-mann}{%
\subsection{Margaret Atwood on `Mephisto,' by Klaus
Mann}\label{margaret-atwood-on-mephisto-by-klaus-mann}}

\includegraphics{https://static01.graylady3jvrrxbe.onion/images/2020/09/09/books/review/09PoliticsFiction-04/09PoliticsFiction-04-articleLarge.jpg?quality=75\&auto=webp\&disable=upscale}

It was 1984 --- in real life, not the book. My family and I were living
in West Berlin, where I was beginning to write ``The Handmaid's Tale.''
Berlin was iconic for me: Having been born two months after the start of
World War II, I'd lived all my life in the long shadows it cast. The
Soviet Union and its satellites were still in place, and showed no signs
of vanishing: Every Sunday, the East German Air Force made sonic booms,
just to let us know it was right next door. The Berlin Wall was still
firmly standing, and people were still being shot while trying to
escape. No one suspected that in a mere five years we'd be buying
fragments of it for souvenirs.

We were in Berlin at the invitation of the D.A.A.D, an academic exchange
group that brought foreign artists into West Berlin so that local
artists would not feel so cut off. West Berlin at that time was partly
empty --- young men could avoid the draft there, but young families
hesitated to expose their children to the risks --- so the D.A.A.D had a
range of rental apartments available for their visiting artists. Ours
had a large iron safe in the living room. Who had lived here? I
wondered. What had they kept in that safe? What had become of them? I
didn't have a good feeling about that. The echoes of jackboots on the
stairs were not audible, but they were there.

The D.A.A.D provided German lessons, so, as I had some elementary German
left over from high school and college, I took them.

My teacher was a stickler who was worried about the decline of the
dative case, and who discouraged me from using expressions I picked up
on the street. But I wanted to use expressions I picked up on the
street. I copied slang from ads, and read popular magazines.

In aid of my German, I sought out a novel with short sentences. This is
how I came to read ``Mephisto.'' It could not have been a more
appropriate choice for the book I was writing, and it chimes eerily with
the times we are living through now.

``Mephisto'' was written by Klaus Mann, the son of the famous writer
Thomas Mann, and was first published in 1936, when Hitler's Third Reich
had been in power for three years and Klaus Mann was already in exile.

It tells the story of an actor named Hendrik Höfgen, who, having started
out as a Brechtian radical socialist activist, changes course and rises
to great heights in the theater world of National Socialist Germany. But
he rises at a cost: As he scrambles up the ladder, Höfgen betrays his
former associates and renounces his Black lover, while slipping on the
required Nazi ideology like a costume.

Höfgen's most acclaimed role --- and yes, he's talented --- is as the
demon Mephistopheles in Goethe's ``Faust*,''* who persuades the hero to
sell his soul in return for worldly wealth, status and pleasure. In
life, however, Höfgen plays Faust, the weak, tempted one, while the part
of Mephistopheles is taken by the Nazi state and its functionaries. Of
course, Höfgen could have left --- gone into exile, as Klaus Mann did.
But he was an actor, and an actor without an audience is nothing.

In no political system do artists have real power. They may have
influence of a kind, but they don't control the purse strings or give
the marching orders, and they're always at the mercy of prevailing
winds. Patrons and gatekeepers decide who's hot and who's not, who gets
the grants, and, in locked-tight regimes, even who gets the theatrical
roles. Is Höfgen only doing what he has to in order to fully achieve his
own greatness? Does art justify everything? How much complicity in a
criminal regime, how much collaboration, how much failure to speak up,
before your soul is damned?

These are the questions ``Mephisto'' raises. They were both pertinent
and prescient in 1936, and they're still with us today. Imagine an
America in which an increasingly ruthless authoritarian regime has laid
its hands not only on the judiciary and the environment and the Postal
Service, but on all media and all educational and artistic institutions.
Then imagine trying to function as an artist. That's the sort of world
Höfgen is navigating. It's difficult to picture such a state of affairs
coming to exist in America; but, after the last four years, it's not
impossible.

\emph{Margaret Atwood is the author of more than 20 works of fiction.
Her latest novel, ``The Testaments,'' was recently published in
paperback. Her new collection of poems, ``Dearly,'' will be published in
November.}

\begin{center}\rule{0.5\linewidth}{\linethickness}\end{center}

\hypertarget{huxe9ctor-tobar-on-el-seuxf1or-presidente-by-miguel-uxe1ngel-asturias}{%
\subsection{Héctor Tobar on `El Señor Presidente,' by Miguel Ángel
Asturias}\label{huxe9ctor-tobar-on-el-seuxf1or-presidente-by-miguel-uxe1ngel-asturias}}

Image

Miguel Ángel Asturias in 1967, the year he won the Nobel Prize in
Literature.Credit...Author photograph from Associated Press

The title character of Miguel Ángel Asturias's novel ``El Señor
Presidente'' (1946) is a shameless egomaniac. He's vain, insecure and
unpredictable, and he's the commander in chief. When a priest takes down
a poster announcing the birthday of the president's mother, he has the
priest arrested. It's an open secret that he keeps assorted prostitutes
as his mistresses.

Asturias transformed Latin American literature when he published this
book. It helped spawn a new genre: the ``dictator novel.'' The author's
use of extended dream sequences and rich, figurative language inspired
what would become known many years later as magical realism.

But what I find most compelling about ``El Señor Presidente'' is how
much it speaks to the here and now. We live in an age of demagogues.
We've seen how the whims and fears of a leader, transformed into deeds
by an army of sycophants, can spread chaos through a nation's
institutions. Asturias saw this madness, too, and created art from it.

The country where the novel is set isn't named, but most of the book
unfolds in a place recognizable as Guatemala City, where Asturias was
born and raised. The dictator isn't named either, but he is based on a
real person: Manuel Estrada Cabrera, Guatemala's president for the first
two decades of the 20th century.

Most of ``El Señor Presidente'' unfolds from the point of view of one of
the president's minions, the handsome fixer known as Miguel Cara de
Ángel, or Angel Face, in the archaic English translation from 1963.
Angel Face is ``as beautiful and as wicked as Satan.'' The president
asks him to neutralize a general who is an incipient political rival.
But then Angel Face falls in love with the general's daughter.

To feel worthy of love, Angel Face performs a good deed: He rescues the
life of another army officer the president wants dead. Then he gives the
officer some friendly advice about how to stay alive in a dictatorship.
``Try and find a way of getting on the right side of the president,'' he
counsels. The best way to gain the president's good will is to break the
law on his behalf. ``Commit a public outrage on defenseless people,''
Angel Face says. Show the public ``the superiority of force. \ldots{}
Get rich at the expense of the nation.''

In real life, Estrada Cabrera modernized Guatemala by opening it to U.S.
capital. The president lined his pockets in the process, and helped
create the culture of venality that has plagued Guatemala ever since. As
a young man, Asturias saw how Estrada Cabrera ruled Guatemala with
ever-increasing doses of cynicism and sadism. In ``El Señor
Presidente,'' the regime's prisons are hell on earth; those awaiting
execution are kept in a lightless cell where they are forced to stand in
their own excrement.

For generations before and after the novel was published, Guatemala's
idealists went into exile; or they stayed home and were murdered. As
Asturias writes: ``The men of this town who desired their country's good
are far away now: some of them begging outside houses in a foreign land,
others rotting in a common grave.''

In today's Guatemala, criminal gangs have privatized violence and
corruption: Just like the dictators, they've left a trail of mutilated
corpses and a terrified populace in their wake. Guatemalans migrate away
from their country, in part, to escape the collapse of the rule of law.
They suffer existential torments not in the dungeons depicted in ``El
Señor Presidente,'' but in desert holding cells on the U.S.-Mexico
border.

My parents left Guatemala in the early 1960s. I was born in Los Angeles.
When Asturias won the Nobel Prize in Literature in 1967, I was about to
start elementary school in East Hollywood. The Nobel was a source of
great pride for my Guatemalan expatriate parents, who purchased several
of his books; these were the first novels I ever laid eyes on.

In a family that was a generation removed from illiteracy, Asturias's
books and his Nobel stood for our right as \emph{guatemaltecos} to claim
to be people of letters. Owning copies of ``El Papa Verde,'' ``Hombres
de Maíz'' and other books from Asturias's oeuvre was an act of cultural
preservation.

Today, I read Asturias with the eyes of a novelist. I see a writer using
every tool at his disposal to make us feel how one man can inflict a
daily assault on the collective psyche of a people.

In ``El Señor Presidente,'' Asturias shows us how a writer can vanquish
the darkest and most omnipotent leader. He exposes the lies of a
strongman and shrinks him into the artist's own pliable creation. The
novelist condemns the ``great leader'' to a terrible fate: spending
eternity as a character trapped between the covers of a book.

\emph{Héctor Tobar's latest novel is ``The Last Great Road Bum.'' He is
a contributing writer for The Times's opinion pages.}

\begin{center}\rule{0.5\linewidth}{\linethickness}\end{center}

\hypertarget{thomas-mallon-on-1876-by-gore-vidal}{%
\subsection{Thomas Mallon on `1876,' by Gore
Vidal}\label{thomas-mallon-on-1876-by-gore-vidal}}

Image

Gore Vidal in 1973.Credit...Author photograph from Associated Press

In the summer of 1975, Gore Vidal was completing ``1876,'' what would be
the third novel in his seven-volume ``Narratives of Empire.'' He
asserted, in the book's afterword, that ``1876 was probably the low
point in our republic's history'' --- quite a claim from a writer who
regarded most of the republic's points as being close to rock bottom,
and whose readers had just lived through Watergate. His novel allowed
Americans to view their bicentennial through the commemorative year of a
century before; present-day readers, six years away from the
semiquincentennial of the republic (if we can keep it), can discern some
of their own grotesque times through the author's vision of 1876.

Vidal's narrator is the fictional Charles Schermerhorn Schuyler, a
widower and ``very old,'' he tells us, at the age of 62. An illegitimate
son of Aaron Burr (the hero of Vidal's previous volume in the series),
Schuyler is a diplomat long since turned writer who has spent the last
40 years in Europe. He is now returning to the States with his daughter
--- a titled, 35-year-old widow named Emma --- because they've gone
broke in the speculative ``Panic of '73.''

Beset with heart, lung and mobility problems, Schuyler must hustle like
a man decades younger. Just being ``the New York press's perennial
authority on European matters'' will no longer be enough to keep him
afloat. He now needs to chase after the big stories of his native land,
from the Philadelphia Centennial Exhibition to the scandals of the Grant
administration to the presidential hopes of New York's surprisingly
honest Democratic governor, Samuel J. Tilden. If Schuyler succeeds ---
pleasing editors and Tilden's own circle with his commentaries --- he
may wind up not only financially revived but as ambassador to France. If
a year in the States also helps to find a proper new husband for Emma,
his happiness will be complete.

The real purpose of Schuyler's fictional existence is to serve as
Vidal's eyes and ears, to \emph{notice} the cultural changes that would
strike a man who can remember shaking Andrew Jackson's hand in his
youth. He's aghast over the girth and beardedness and nasality of voice
that has befallen the Yankee male. Even that subspecies' potency has
been sapped --- ``something tragical,'' an Irish prostitute tells
Schuyler --- by the economic panic. Urchins swarm the sidewalks of New
York, and dogfighting is a ``new, dreadful, illegal sport.'' The
citizenry guzzles ``razzle-dazzle'' cocktails, and munches a new snack
called popcorn. The protocol affectations of Mrs. Astor's dinner parties
bore Schuyler, but recent polyglot waves of immigration display to him a
``new world, more like a city from the `Arabian Nights' than that small
staid English-Dutch town or village of my youth.''

Washington's rapid modernizations include the finally completed Capitol,
``floating like a dream carved in whitest soap.'' Inside it, Schuyler
finds ``the old red hangings and tobacco-stained rugs have been replaced
by a delicate gray décor with hints here and there of imperial gilt''
--- ornamental foreshadowings of Vidal's preoccupation with empire. But
in ``1876,'' the theme is corruption, the kind facilitated by the Senate
cloakroom's informality: ``the practical tribune of the people prefers
making himself easily accessible to those who want to give him money.''
The political class, more awash in cologne than soap, literally smells
bad, and it howls whenever anyone is honest enough to notice its hands
in the till. ``God save us!'' cries Mrs. Puss Belknap, the thieving wife
of the thieving secretary of war.

Vidal speedily animates a whole gallery of political figures --- the
``plumed Knight,'' James G. Blaine; the charmingly venal Chester Arthur;
the nobly dyspeptic Tilden --- as they prepare for what will be the
wildly disputed election to choose Grant's successor. The contest's
defining elements --- the implacable partisan divide; the electorate's
inability to become aroused against plunder; racial division and
anti-immigrant sentiment; the ineffectuality of ``the better sort of
Republicans'' --- will hurl readers, allegorically, smack into the
present. What citizens of our gerontocracy won't recognize is the
general youthfulness of the novel's key political figures.

Schuyler shares Vidal's taste for aphorism and paradox (``like most
people who hate everyone, he desperately needs company''), and seems
vulnerable to the idea that there is no history except for ``fictions of
varying degrees of plausibility.'' And yet, Schuyler's self-induced pep,
his need to be back on the make in his 60s, gives his voice a verve that
the later Vidal, compulsively world-weary and mandarin, would sometimes
lack when writing in the third person.

Vidal's dark wit almost single-handedly awakened the American historical
novel from its costumed midcentury slumbers, but his Schuyler is also
capable of a Dickensian warmth. When he greets the impending birth of a
child with a kind of shudder --- ``Poor boy! What a world to come
into!'' --- he is expressing the dread that every era somehow believes
is unique to itself, but which historical fiction consolingly shows was
ever thus.

\emph{Thomas Mallon has written 10 novels, including ``Landfall,''
``Finale'' and ``Watergate,'' as well as several works of nonfiction and
essays.}

\begin{center}\rule{0.5\linewidth}{\linethickness}\end{center}

\hypertarget{brenda-wineapple-on-all-the-kings-men-by-robert-penn-warren}{%
\subsection{Brenda Wineapple on `All the King's Men,' by Robert Penn
Warren}\label{brenda-wineapple-on-all-the-kings-men-by-robert-penn-warren}}

Image

Robert Penn Warren in 1946.Credit...Author photograph from Associated
Press

Last March, during the first weeks of the pandemic, I began pulling old
novels down from the shelves, hoping to find the comfort or momentary
escape they once delivered. When I opened Robert Penn Warren's Pulitzer
Prize-winning ``All the King's Men'' around Super Tuesday, I assumed it
would offer a consoling picture of a demagogue's demise but not much
more. Then the world changed, and so did the novel.

Context is all, or at least a lot. Sure, the book's central character,
Willie Stark, has come to signify chicanery, political bossdom and
populism run amok. Sure, it's the story of a charismatic politician in a
Southern state (resembling Louisiana) who rises to power after learning
he'd been taken for a fool. Lackeys of the former governor, Joe
Harrison, persuaded Willie to run for office, assuming he will split the
``hick'' vote with Harrison's rival and allow Harrison to waltz to
re-election.

Known as ``Cousin Willie from the country,'' Stark is so shaken when he
learns of the scheme that he gets drunk for the first time. The
self-taught and somewhat naïve, even idealistic, county treasurer who
had studied his secondhand law books --- and Emerson and Macaulay and
Shakespeare --- in front of a rusty old stove at the family farm then
reaches deep and finds his calling. Appealing to the resentments of his
poor white constituency, he rallies crowd after crowd almost to madness.
Pretty soon, he's sitting in the governor's mansion.

Willie Stark has gotten really good at beating corrupt politicians at
their own game. In his state, machine politics has replaced the illusory
rectitude of old boy aristocrats, but Willie is neither an aristocrat
nor a machine pol; he's a solo act. He's also an authoritarian
grandstander who uses all the means at his disposal, whether
court-packing or blackmail. ``There's always something,'' he tells his
expert dirt-digger, Jack Burden, who narrates the novel.

Yet Willie sincerely wants to bring roads and schools to his state, to
tax the rich and to create a more equitable social structure.
Unfortunately, though, he thinks only he can deliver the goods, and that
how he chooses to do it doesn't matter. He believes he embodies the
people's will. ``Your need is my justice,'' he shouts to them. He's
their ruthless, energetic ``Willie.''

But what surprised me most on rereading the novel was that a hospital
--- a \emph{hospital} --- lay at its center. When the basically
unsympathetic but complex Governor Stark escapes impeachment
(\emph{impeachment}), he promises a roaring crowd that he's going to
build a big, beautiful, free hospital to ease pain and sickness. ``You
shall not be deprived of hope,'' he tells them. Willie's dream is not a
dream of meretricious beauty, like Jay Gatsby's. It's a dream of health
care as a basic human right.

Before that hospital goes up, Willie is fatally shot by the priggish and
somewhat self-deluded romantic who happens to be the famous physician
picked to run the place. ``It might have been all different, Jack,'' a
dying Willie tells the narrator. ``You got to believe that.''

Maybe so; maybe he'd have built his beautiful hospital just the way he
said he wanted, without graft or sin. I like to think so. But maybe he'd
build it however he could, simply to get it done. As it is, we're left
with the slightly portentous narrator, who bears the novel's ``burden,''
having finally discovered that we're all connected, for better and
worse: ``The world is like an enormous spider web and if you touch it,
however lightly, at any point, the vibration ripples to the remotest
perimeter,'' he says. ``It does not matter whether or not you meant to
brush the web of things.'' We're all connected, yes, and in connection
lies responsibility. Willie Stark may finally grasp that. But frankly, I
don't care who builds that hospital: Just build it.

\emph{Brenda Wineapple is the author of several works of history and
literary biography, most recently ``The Impeachers: The Trial of Andrew
Johnson and the Dream of a Just Nation.''}

Advertisement

\protect\hyperlink{after-bottom}{Continue reading the main story}

\hypertarget{site-index}{%
\subsection{Site Index}\label{site-index}}

\hypertarget{site-information-navigation}{%
\subsection{Site Information
Navigation}\label{site-information-navigation}}

\begin{itemize}
\tightlist
\item
  \href{https://help.nytimes3xbfgragh.onion/hc/en-us/articles/115014792127-Copyright-notice}{©~2020~The
  New York Times Company}
\end{itemize}

\begin{itemize}
\tightlist
\item
  \href{https://www.nytco.com/}{NYTCo}
\item
  \href{https://help.nytimes3xbfgragh.onion/hc/en-us/articles/115015385887-Contact-Us}{Contact
  Us}
\item
  \href{https://www.nytco.com/careers/}{Work with us}
\item
  \href{https://nytmediakit.com/}{Advertise}
\item
  \href{http://www.tbrandstudio.com/}{T Brand Studio}
\item
  \href{https://www.nytimes3xbfgragh.onion/privacy/cookie-policy\#how-do-i-manage-trackers}{Your
  Ad Choices}
\item
  \href{https://www.nytimes3xbfgragh.onion/privacy}{Privacy}
\item
  \href{https://help.nytimes3xbfgragh.onion/hc/en-us/articles/115014893428-Terms-of-service}{Terms
  of Service}
\item
  \href{https://help.nytimes3xbfgragh.onion/hc/en-us/articles/115014893968-Terms-of-sale}{Terms
  of Sale}
\item
  \href{https://spiderbites.nytimes3xbfgragh.onion}{Site Map}
\item
  \href{https://help.nytimes3xbfgragh.onion/hc/en-us}{Help}
\item
  \href{https://www.nytimes3xbfgragh.onion/subscription?campaignId=37WXW}{Subscriptions}
\end{itemize}
