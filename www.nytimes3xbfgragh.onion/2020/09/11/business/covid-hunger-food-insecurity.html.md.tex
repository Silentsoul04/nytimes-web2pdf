\href{/section/business}{Business}\textbar{}The Other Way Covid Will
Kill: Hunger

\url{https://nyti.ms/35rJ12U}

\begin{itemize}
\item
\item
\item
\item
\item
\item
\end{itemize}

\hypertarget{the-coronavirus-outbreak}{%
\subsubsection{\texorpdfstring{\href{https://www.nytimes3xbfgragh.onion/news-event/coronavirus?name=styln-coronavirus-markets\&region=TOP_BANNER\&block=storyline_menu_recirc\&action=click\&pgtype=Article\&impression_id=75e51e10-f52b-11ea-9a16-5d2ad197bf40\&variant=undefined}{The
Coronavirus
Outbreak}}{The Coronavirus Outbreak}}\label{the-coronavirus-outbreak}}

\begin{itemize}
\tightlist
\item
  live\href{https://www.nytimes3xbfgragh.onion/2020/09/12/world/covid-19-coronavirus.html?name=styln-coronavirus-markets\&region=TOP_BANNER\&block=storyline_menu_recirc\&action=click\&pgtype=Article\&impression_id=75e51e11-f52b-11ea-9a16-5d2ad197bf40\&variant=undefined}{Latest
  Updates}
\item
  \href{https://www.nytimes3xbfgragh.onion/interactive/2020/us/coronavirus-us-cases.html?name=styln-coronavirus-markets\&region=TOP_BANNER\&block=storyline_menu_recirc\&action=click\&pgtype=Article\&impression_id=75e54520-f52b-11ea-9a16-5d2ad197bf40\&variant=undefined}{Maps
  and Cases}
\item
  \href{https://www.nytimes3xbfgragh.onion/interactive/2020/science/coronavirus-vaccine-tracker.html?name=styln-coronavirus-markets\&region=TOP_BANNER\&block=storyline_menu_recirc\&action=click\&pgtype=Article\&impression_id=75e54521-f52b-11ea-9a16-5d2ad197bf40\&variant=undefined}{Vaccine
  Tracker}
\item
  \href{https://www.nytimes3xbfgragh.onion/2020/09/10/us/politics/fda-coronavirus-vaccine.html?name=styln-coronavirus-markets\&region=TOP_BANNER\&block=storyline_menu_recirc\&action=click\&pgtype=Article\&impression_id=75e54522-f52b-11ea-9a16-5d2ad197bf40\&variant=undefined}{F.D.A.
  Regulators' Self-Defense}
\item
  \href{https://www.nytimes3xbfgragh.onion/2020/09/09/upshot/coronavirus-surprise-test-fees.html?name=styln-coronavirus-markets\&region=TOP_BANNER\&block=storyline_menu_recirc\&action=click\&pgtype=Article\&impression_id=75e54523-f52b-11ea-9a16-5d2ad197bf40\&variant=undefined}{Surprise
  Test Fees}
\end{itemize}

\includegraphics{https://static01.graylady3jvrrxbe.onion/images/2020/09/13/business/11hunger-1/merlin_176527119_f5a154c2-d39f-440e-969d-fb8ab138948d-articleLarge.jpg?quality=75\&auto=webp\&disable=upscale}

Sections

\protect\hyperlink{site-content}{Skip to
content}\protect\hyperlink{site-index}{Skip to site index}

\hypertarget{the-other-way-covid-will-kill-hunger}{%
\section{The Other Way Covid Will Kill:
Hunger}\label{the-other-way-covid-will-kill-hunger}}

Worldwide, the population facing life-threatening levels of food
insecurity is expected to double, to more than a quarter of a billion
people.

The malnutrition ward of Indira Gandhi Children's Hospital in Kabul,
Afghanistan.Credit...Jim Huylebroek for The New York Times

Supported by

\protect\hyperlink{after-sponsor}{Continue reading the main story}

\href{https://www.nytimes3xbfgragh.onion/by/peter-s-goodman}{\includegraphics{https://static01.graylady3jvrrxbe.onion/images/2018/02/16/multimedia/author-peter-s-goodman/author-peter-s-goodman-thumbLarge-v2.png}}\href{https://www.nytimes3xbfgragh.onion/by/abdi-latif-dahir}{\includegraphics{https://static01.graylady3jvrrxbe.onion/images/2020/08/14/reader-center/author-abdi-latif-dahir/author-abdi-latif-dahir-thumbLarge.png}}\href{https://www.nytimes3xbfgragh.onion/by/karan-deep-singh}{\includegraphics{https://static01.graylady3jvrrxbe.onion/images/2019/12/02/reader-center/author-karan-deep-singh/author-karan-deep-singh-thumbLarge.png}}

By \href{https://www.nytimes3xbfgragh.onion/by/peter-s-goodman}{Peter S.
Goodman},
\href{https://www.nytimes3xbfgragh.onion/by/abdi-latif-dahir}{Abdi Latif
Dahir} and
\href{https://www.nytimes3xbfgragh.onion/by/karan-deep-singh}{Karan Deep
Singh}

\begin{itemize}
\item
  Sept. 11, 2020
\item
  \begin{itemize}
  \item
  \item
  \item
  \item
  \item
  \item
  \end{itemize}
\end{itemize}

Long before the pandemic swept into her village in the rugged southeast
of Afghanistan, Halima Bibi knew the gnawing fear of hunger. It was an
omnipresent force, an unrelenting source of anxiety as she struggled to
nourish her four children.

Her husband earned about \$5 a day, hauling produce by wheelbarrow from
a local market to surrounding homes. Most days, he brought home a loaf
of bread, potatoes and beans for an evening meal.

But when the coronavirus arrived in March, taking the lives of her
neighbors and shutting down the market, her husband's earnings plunged
to about \$1 a day. Most evenings, he brought home only bread. Some
nights, he returned with nothing.

``We hear our children screaming in hunger, but there is nothing that we
can do,'' said Ms. Bibi, speaking in Pashto by telephone from a hospital
in the capital city of Kabul, where her 6-year-old daughter was being
treated for severe malnutrition. ``That is not just our situation, but
the reality for most of the families where we live.''

\includegraphics{https://static01.graylady3jvrrxbe.onion/images/2020/09/13/business/11hunger-2/merlin_176527140_ef63cf5a-14e0-4e5e-b5b5-17c54e3bd1b9-articleLarge.jpg?quality=75\&auto=webp\&disable=upscale}

It is increasingly the reality for hundreds of millions of people around
the planet. As the global economy absorbs the most punishing reversal of
fortunes since the Great Depression, hunger is on the rise. Those
confronting potentially life-threatening levels of so-called food
insecurity in the developing world are expected to nearly double this
year to 265 million, according to the
\href{https://www.wfp.org/news/covid-19-will-double-number-people-facing-food-crises-unless-swift-action-taken}{United
Nations World Food Program}.

Worldwide, the number of children younger than 5 caught in a state of
so-called wasting --- their weight so far below normal that they face an
elevated risk of death, along with long-term health and developmental
problems --- is likely to grow by nearly seven million this year, or 14
percent, according to a recent paper published in
\href{https://www.thelancet.com/journals/lancet/article/PIIS0140-6736(20)31647-0/fulltext}{The
Lancet}, a medical journal.

The largest numbers of vulnerable communities are concentrated in South
Asia and Africa, especially in countries that are already confronting
trouble, from military conflict and extreme poverty to climate-related
afflictions like drought, flooding and soil erosion.

At least for now, the unfolding tragedy falls short of a famine, which
is typically set off by a combination of war and environmental disaster.
Food remains widely available in most of the world, though prices have
climbed in many countries, as fear of the virus disrupts transportation
links, and as currencies fall in value, increasing the costs of imported
items.

Rather, with the
\href{https://www.imf.org/en/Publications/WEO/Issues/2020/06/24/WEOUpdateJune2020}{world
economy expected to contract} nearly 5 percent this year, households are
cutting back sharply on spending. Among those who went into the pandemic
in extreme poverty, hundreds of millions of people are suffering an
intensifying crisis over how to secure their basic dietary needs.

The pandemic has reinforced basic economic inequalities, none more
defining than access to food.

Image

The main wholesale market in Kabul. Communities most vulnerable to food
insecurity are concentrated in South Asia and Africa --- especially
those that are already confronting trouble, like military conflict,
extreme poverty and climate-related afflictions like drought or
flooding.Credit...Jim Huylebroek for The New York Times

\hypertarget{shock-upon-shock-upon-shock}{%
\subsection{`Shock upon shock upon
shock'}\label{shock-upon-shock-upon-shock}}

In South Africa, more than a quarter-century has passed since the
official ending of apartheid, yet the Black majority remains
\href{https://www.nytimes3xbfgragh.onion/2017/10/24/business/south-africa-economy-apartheid.html}{overwhelmingly
confined to poor townships} that are far removed from jobs and services
in the cities. When the pandemic emerged in March, the government
ordered the shutdown of informal food vendors and township shops,
unleashing the military to detain merchants who violated orders. That
forced residents to rely on supermarkets --- suddenly farther away than
ever, given the lockdown of already woeful bus service.

At the same time, South Africa closed its schools, eliminating school
lunches --- the only reliable meal for millions of students --- just as
breadwinners lost their means of getting to jobs. By the end of April,
nearly half of all South African households had exhausted their funds to
buy food, according to an
\href{https://cramsurvey.org/wp-content/uploads/2020/07/Wills-household-resource-flows-and-food-poverty-during-South-Africa\%E2\%80\%99s-lockdown-2.pdf}{academic
study.} Social unrest eventually prompted a loosening of the country's
restrictions.

Far from a danger confined to the world's poorest countries, hunger is a
growing scourge even in the wealthiest countries. Previously working
people who have never felt compelled to seek help are now lining up at
food banks in the
\href{https://www.nytimes3xbfgragh.onion/interactive/2020/09/02/magazine/food-insecurity-hunger-us.html}{United
States},
\href{https://www.independent.co.uk/news/world/europe/food-banks-spain-hunger-coronavirus-poverty-covid-19-a9536341.html}{Spain}
and
\href{https://www.nytimes3xbfgragh.onion/2020/03/19/world/europe/coronavirus-uk-food-banks.html}{Britain}.
Even people of relative means are cutting their purchases of fresh
fruits and vegetables, while relying more on the cheap calories of fast
food.

In wealthier countries, the economic strains are cushioned by government
programs like unemployment benefits, subsidized wage plans and cash
grants for food. In the poorest countries, the coronavirus is
intensifying a litany of already potent afflictions.

``Covid has been yet another shock in what has been a terrible year in
this region,'' said Michael Dunford, regional director for East Africa
at the \href{https://www.wfp.org/}{World Food Program}. ``In addition to
already having 21 million people acutely food insecure at the beginning
of the year, we've then had flooding, locusts, and now we've got Covid.
So it's shock upon shock upon shock, which is just increasing
vulnerability throughout the region.''

Just as the need for help intensifies, the threat of the virus is
forcing relief agencies to scrap public health campaigns and limit their
outreach. Lockdowns imposed to halt the pandemic will this year deprive
250 million children in poor countries of scheduled supplements of
Vitamin A, elevating the threat of premature death, according to
\href{https://www.unicefusa.org/mission/starts-with-u/protection-for-children?}{UNICEF}.
The supplements strengthen the immune system, limiting vulnerability to
diseases that opportunistically exploit malnutrition.

The virus has also forced the delay of other immunization programs,
which are typically combined with doses of deworming medicine ---
another bulwark against malnutrition.

``I'm increasingly concerned about the socioeconomic impacts of the
pandemic on the nutrition situation of children,'' said Victor Aguayo,
chief of nutrition programs at UNICEF in New York. ``It's a perfect
storm to see an increase in malnutrition rates if appropriate measures
and programs are not put in place.''

Image

The Mangateen IDP camp in Juba, South Sudan, in April. In the early days
of the pandemic, South Sudan was already one of the world's poorest
countries, with 80 percent of its roughly 11 million people living in a
state of absolute poverty, surviving on less than \$2 a
day.Credit...Alex Mcbride/Agence France-Presse --- Getty Images

\hypertarget{only-the-latest-plague}{%
\subsection{Only the latest plague}\label{only-the-latest-plague}}

In Juba, the capital of South Sudan, the pandemic was merely the most
recent form of grave danger.

A sense of crisis has prevailed since
\href{https://www.nytimes3xbfgragh.onion/2016/03/12/world/africa/south-sudan-civil-war.html}{a
paroxysm of violence four years ago} in a long-running civil war fueled
by ethnic division. Amid the fighting, people fled the surrounding
countryside for refuge in camps inside the city. Without access to their
fields, many became dependent on food distributed by relief agencies
along with anything they could buy at the market.

\hypertarget{latest-updates-the-coronavirus-outbreak-and-the-economy}{%
\section{\texorpdfstring{\href{https://www.nytimes3xbfgragh.onion/live/2020/09/11/business/stock-market-today-coronavirus?action=click\&pgtype=Article\&state=default\&region=MAIN_CONTENT_1\&context=storylines_live_updates}{Latest
Updates: The Coronavirus Outbreak and the
Economy}}{Latest Updates: The Coronavirus Outbreak and the Economy}}\label{latest-updates-the-coronavirus-outbreak-and-the-economy}}

\href{https://www.nytimes3xbfgragh.onion/live/2020/09/11/business/stock-market-today-coronavirus?action=click\&pgtype=Article\&state=default\&region=MAIN_CONTENT_1\&context=storylines_live_updates\#the-nyse-may-move-its-data-center-out-of-new-jersey-in-response-to-a-proposed-tax}{23h
ago}

\href{https://www.nytimes3xbfgragh.onion/live/2020/09/11/business/stock-market-today-coronavirus?action=click\&pgtype=Article\&state=default\&region=MAIN_CONTENT_1\&context=storylines_live_updates\#the-nyse-may-move-its-data-center-out-of-new-jersey-in-response-to-a-proposed-tax}{The
N.Y.S.E. may move its data center out of New Jersey in response to a
proposed tax.}

\href{https://www.nytimes3xbfgragh.onion/live/2020/09/11/business/stock-market-today-coronavirus?action=click\&pgtype=Article\&state=default\&region=MAIN_CONTENT_1\&context=storylines_live_updates\#the-federal-budget-deficit-hit-3-trillion-as-of-august}{25h
ago}

\href{https://www.nytimes3xbfgragh.onion/live/2020/09/11/business/stock-market-today-coronavirus?action=click\&pgtype=Article\&state=default\&region=MAIN_CONTENT_1\&context=storylines_live_updates\#the-federal-budget-deficit-hit-3-trillion-as-of-august}{The
federal budget deficit hit \$3 trillion as of August.}

\href{https://www.nytimes3xbfgragh.onion/live/2020/09/11/business/stock-market-today-coronavirus?action=click\&pgtype=Article\&state=default\&region=MAIN_CONTENT_1\&context=storylines_live_updates\#warner-bros-pushes-the-release-of-wonder-woman-1984-to-christmas}{25h
ago}

\href{https://www.nytimes3xbfgragh.onion/live/2020/09/11/business/stock-market-today-coronavirus?action=click\&pgtype=Article\&state=default\&region=MAIN_CONTENT_1\&context=storylines_live_updates\#warner-bros-pushes-the-release-of-wonder-woman-1984-to-christmas}{Warner
Bros. pushes the release of `Wonder Woman 1984' to Christmas.}

\href{https://www.nytimes3xbfgragh.onion/live/2020/09/11/business/stock-market-today-coronavirus?action=click\&pgtype=Article\&state=default\&region=MAIN_CONTENT_1\&context=storylines_live_updates}{See
more updates}

More live coverage:
\href{https://www.nytimes3xbfgragh.onion/2020/09/11/world/covid-19-coronavirus.html?action=click\&pgtype=Article\&state=default\&region=MAIN_CONTENT_1\&context=storylines_live_updates}{Global}

South Sudan was already one of the world's poorest countries, with 80
percent of its roughly 11 million people living in a state of absolute
poverty, surviving on less than \$2 a day, according to the
\href{https://www.worldbank.org/en/country/southsudan/publication/south-sudan-economic-update-peace-agreement-spurs-economic-recovery-but-poverty-remains}{World
Bank}. The reinvigorated conflict posed an economic shock. As the
government printed currency to pay its bills, runaway inflation
resulted, dropping teachers' salaries from the equivalent of \$100 a
month to \$1.

Food prices soared. Most items were trucked in from neighboring Kenya
and Uganda and priced in dollars, making them more expensive as the
nation's currency plunged. A 50 kilogram (110 pound) bag of corn flour
that fetched \$20 four years ago was more than \$30 by late last year.

Poverty and hunger proved mutually reinforcing. As mosquito nets
increased in price, that enhanced the risks of a lethal strain of
malaria, which itself reduced appetites and worsened malnutrition among
children.

Last year, heavy rains that fell in too short a time created torrential
flooding that decimated crops and killed cattle.

By the beginning of 2020, roughly six million people in South Sudan were
technically food insecure, meaning they could not reliably count on
satisfying their dietary requirements.

``Nutrition is a lot more than food,'' said Mads Oyen, chief of field
operations for UNICEF in South Sudan, speaking by videoconference from
Juba. ``You've got malaria and measles and a lack of nutrients and other
health issues. It's about lack of clean water, which means cholera.''

This was all before the arrival of the worst pandemic in a century.

As the virus sowed chaos in transportation networks across East Africa,
the price of staple foods sold in Juba leapt another 25 percent. At the
same time, a lockdown imposed by the government derailed local
businesses like food stalls, decimating incomes.

These were the forces that brought Mary Pica to a primary health care
center in Juba in early May. It was run by the international relief
organization World Vision. She carried her then-10-month-old son. He
weighed only 5.4 kilograms (11.9 pounds), well below healthy.

Ms. Pica lived with her husband's family in a household of nine people.
Her husband had worked loading baggage onto buses. That job was a
casualty of the fighting, as bus service largely shut down.

Her mother-in-law grew greens on a small plot of land outside Juba,
using the proceeds to buy other items that balanced their diet ---
yogurt, fruit, fish and eggs. With the market closed, she could not earn
cash. The family was subsisting almost entirely on greens. Ms. Pica, who
had become pregnant again, was no longer breastfeeding her baby. He was
wasting away.

The clinic provided her with a peanut-based paste donated by UNICEF.
Every two weeks, she goes back to pick up another supply. The baby has
been gaining weight.

But Ms. Pica sees dangers everywhere. Her sister-in-law's child, a
2-year-old boy, has malaria. The pandemic is unrelenting.

``I'm worried,'' she said, speaking in Arabic by phone from Juba. ``I
have no hope that the situation will change tomorrow. I can only pray to
God that it changes.''

\hypertarget{money-is-the-law}{%
\subsection{`Money is the law'}\label{money-is-the-law}}

Food prices have been rising in much of Africa for the same reason that
Samuel Omondi has endured nearly six months without seeing his wife and
five children in western Kenya --- because of the chaos gripping the
roads.

A father of five, Mr. Omondi, 42, makes his living driving a truck,
typically hauling wheat. It used to take him four days to complete his
usual round-trip from the Kenyan port of Mombasa to the Ugandan capital
of Kampala, a distance of 1,400 miles. Now, the same journey requires
eight to 10 days.

Drivers cannot enter either country without certificates showing they
are free of Covid. Uganda has required that every driver submit to a
test at the border, waiting as long as four days for results.

Throughout the region, immigration and customs checks have become so
onerous that lines form 40 miles before borders. Trucks progress slowly,
in low gear, consuming extra fuel. Drivers submit to the maddening wait
while fretting over increased costs.

``You know you are going to spend three days in the truck without taking
a bath,'' Mr. Omondi said. ``You can't even park on the side of the road
and relax. People will pass you.''

Along their journeys, drivers get hostility from communities that view
them as disease carriers. They bring their own groceries, fearful of
stopping in major towns and drawing attention.

``People are saying we are bringing Covid,'' Mr. Omondi said. ``There
was a child in Uganda who looked at us truck drivers and said, `Mama, do
you see these people with corona?'''

Yet he cannot go home, knowing that the chief in his area will force him
into quarantine. ``We are suffering a lot,'' he said.

Given the delays and the bothers, he and other truck drivers have been
making fewer trips a month, diminishing their income and diminishing the
supply of food in many cities.

As convoys roll slowly toward border crossings in the heat, containers
full of fish, chicken, bananas and other perishable goods are rotting.

The movement of food has also been hampered by corruption. In many
countries, police stop truck drivers to inspect their Covid
certificates, engendering a flourishing trade in fake documents. Border
officials exploit the pandemic as a fresh opportunity to extract bribes.

``There's no law at the borders,'' said Joel Ombaso, a wholesale fruit
dealer in Nairobi. ``Money is the law.''

He buys oranges from Tanzania and pineapples and bananas from Uganda. He
must usually dispense hundreds of dollars in bribes to get his cargo
into Kenya, he said. There, he sells the fruit to local grocery stores.
A curfew in Nairobi has prevented delivery at night, imposing further
delays that have damaged shipments. Since the pandemic began, Mr.
Ombaso's profits have plunged by nearly three-fourths, he said.

Image

Tsige Alelign, 24, earns a living making coffee in Addis Ababa,
Ethiopia. A limited supply of food and other factors have pushed prices
higher, just as vast numbers of people have seen their incomes
depleted.Credit...Hilina Abebe for The New York Times

Image

A market in Addis Ababa. In a recent survey conducted by the
International Committee of the Red Cross in 11 African countries, 94
percent of respondents said that food prices had
increased.Credit...Hilina Abebe for The New York Times

An outbreak of pandemic-related nationalism --- with countries blaming
one another for the spread of the disease --- has produced an escalating
wave of trade barriers that has amplified the trouble on the roads.
Rwanda has refused to allow Tanzanian truck drivers to haul goods into
the country, forcing a time-consuming change of driver at the border.

All of these factors have combined to limit the supply of food, pushing
prices higher, just as vast numbers of people have seen their incomes
depleted.

In a recent survey conducted by the International Committee of the Red
Cross in 11 African countries --- among them Kenya, Ethiopia, Nigeria
and the Democratic Republic of Congo --- 85 percent of the respondents
said food was available in their local markets. But 94 percent reported
that prices had increased, and 82 percent said incomes were down.

Ethiopians are voracious consumers of onions, folding them into
seemingly every dish. Much of this staple is imported from neighboring
Sudan. But with the border now shut, the price of onions has skyrocketed
in Addis Ababa, Ethiopia's capital, home to six million.

This has tightened the pressure on Mulunesh Moges, 38, a mother of two
who sells clothes at an open air market.

``My customers are almost down to zero,'' Ms. Moges said. ``I sit at my
shop the whole day without doing anything.'' Her daily earnings used to
run about 200 Ethiopian birr (about \$5) --- enough to feed her family.
Lately, she has earned next to nothing.

``We used to eat three times a day,'' she said. ``Now it's once or
twice. I'm always calculating what to feed my children.''

Birchat Abdala runs a street-side tea and coffee kiosk. Her daily
earnings have dropped by more than two-thirds to 30 birr (about 83
cents).

``In the morning, I used to feed my children eggs and bread,'' she said.
``Now, I feed them only bread, or whatever is left over from my
business. We eat whatever we can get our hands on.''

Image

Indian traders wait for customers at a vegetable market in New
Delhi.Credit...Rebecca Conway for The New York Times

\hypertarget{a-counterintuitive-problem-falling-demand}{%
\subsection{A counterintuitive problem: Falling
demand}\label{a-counterintuitive-problem-falling-demand}}

Across the Arabian Sea, in the Indian capital of New Delhi, Champa Devi
and her family have responded to a loss of income by downgrading their
diet.

She earns her living cleaning homes. Her husband lost his job as a
driver early in the year. Then the pandemic emerged, prompting Prime
Minister Narendra Modi to impose a lockdown, and making it virtually
impossible for her husband to find another job. Their favorite fruits,
bananas and apples, have become luxuries they can no longer afford.

``We have to squeeze our wallets," said Ms. Devi, 29, the mother of a
9-month-old daughter. ``Now, we're surviving on dal and roti*'' ---* the
Indian staple of watery lentils and flatbread.

The shutdown eliminated paychecks for office workers in major cities.
Migrant workers lost their construction jobs. The poorest of the poor
were deprived of meager livings gained by gathering scraps of metal and
plastic from streets. This translated into a monumental reduction of
spending power in a nation of 1.3 billion.

And that produced what seems like a counterintuitive problem in the
midst of rising hunger: Falling demand for crops.

In the northern Indian state of Haryana, Satbir Singh Jatain last month
relinquished his bottle gourds to the elements, allowing them to rot on
the vine rather than wasting the effort to harvest them. The price they
would have fetched would not have covered the cost of labor or
transportation.

Image

Farm workers plant onions in the village of Sahori in Rajasthan,
India.Credit...Rebecca Conway for The New York Times

Image

An Indian farm worker and his son herd sheep in the village of Tulera. A
crashing economy in big cities has produced a strange problem in the
midst of rising hunger: falling demand for crops.Credit...Rebecca Conway
for The New York Times

``There's no point in even picking them and taking them to the market,''
he said.

Since the lockdown, Mr. Jatain, a third-generation farmer, has lost over
700,000 rupees (\$10,000), he said.

Initially, he could not get his tomatoes to market. What little he
gained by selling the crop near his village covered less than a third of
his costs. As the tomatoes began rotting, he became so enraged that he
ran them over with a tractor.

``The lockdowns have destroyed farmers,'' he said. ``Now, we have no
money to buy seeds or pay for fuel.''

Across India, farm laborers complain they are
\href{https://www.nytimes3xbfgragh.onion/2020/09/08/world/asia/india-coronavirus-farmer-suicides-lockdown.html}{not
being paid}, forcing their families to cut their spending on food.

Mr. Jatain is on the hook for bank loans reaching nearly \$18,000. He
owes money lenders in his village. ``I can never pay it back, and soon
they will come for my land,'' he said. ``There is nothing left for us.''

\hypertarget{the-perils-of-seeking-help}{%
\subsection{The perils of seeking
help}\label{the-perils-of-seeking-help}}

In Afghanistan, Ms. Bibi felt a mixture of dread and terror as her
6-year-old daughter, Zinab, sank further into a state of malnutrition.
Her skin was going pale as her body diminished. She was losing energy.

``I could see with my own eyes that the child was withering away,'' Ms.
Bibi said.

She had taken her daughter to several supposed doctors around her
village. They administered folk remedies, advised prayer and urged Zinab
to eat. But her appetite was minimal. And the family had little food.

The prices of staples like flour, rice, cooking oil and sugar were all
rising. Many of these products were trucked in from Pakistan, Iran and
Kazakhstan. So long as the market remained closed, Ms. Bibi's husband
was without work.

By the middle of July, Zinab required serious medical attention,
necessitating a trip to the capital city of Khost Province. Ms. Bibi was
deeply reluctant to make the journey. Getting to the city entailed a
90-minute drive through a forbidding landscape rife with tribal
conflicts, the territory controlled neither by the Afghan government nor
the insurgent Taliban. The roads were too frequently lined with deadly
explosive devices.

And now a new horror was layered atop the usual sources of fear. The
coronavirus had killed more than 15 people in her village of perhaps
500. Beyond its confines lay a seemingly infinite number of potential
carriers.

This was the calculation that was preventing people from seeking
critical care throughout Afghanistan. Between January and May, the
number of Afghan children under 5 who were suffering from severe acute
malnutrition --- a condition requiring hospitalization --- increased to
780,000 from 690,000, according to Zakia Maroof, a nutrition expert with
UNICEF in Kabul. Since March, the number of children admitted to
hospitals has declined 40 percent.

But if Ms. Bibi was frightened to venture out, she was even more
disturbed by the alternative.

Image

The children's hospital in Kabul.~Worldwide, the number of children
younger than 5 caught in a state of so-called wasting is expected to
grow by nearly seven million this year, or 14 percent.Credit...Jim
Huylebroek for The New York Times

Image

An Afghan girl, 10, with her 1½-year-old sister.Credit...Jim Huylebroek
for The New York Times

``It was either be afraid of the coronavirus and watch my child die,''
she said, ``or at least tell my heart that I did what I had to do.''

Her husband borrowed from relatives to cover their medical bills, and
they climbed aboard a minibus.

At a rudimentary hospital in the city of Khost, doctors administered a
diet of powdered milk. After three weeks there, with bills mounting,
Zinab was still losing weight. The doctors pronounced their capabilities
exhausted. The family would have to go to Kabul, another seven-hour ride
away.

Her husband went out to the streets and begged, amassing the funds for a
ride in a beat-up station wagon headed for Afghanistan's capital.

They rode through the blazing August heat, arriving in a bustling city
they had never visited, and where they knew no one. They beseeched
strangers to direct them to a children's hospital. A kindly soul led
them to the Indira Gandhi hospital, which was run by the Indian
government and supported by UNICEF.

Zinab was admitted and administered regular feeding by a tube inserted
through her nose. She weighed only 8.5 kilograms (less than 19 pounds).
Two weeks later, she was still shedding weight, her system struggling to
hold food down.

Ms. Bibi sat by her side, keeping vigil, fretting about the bills and
wondering how they might find their way home.

Donate now to the 109th annual campaign of The New York Times Neediest
Cases Fund. All proceeds go to 10 organizations providing assistance to
those facing economic hardship. Make a tax-deductible donation through
GoFundMe.

\href{https://charity.gofundme.com/o/en/campaign/neediest-cases-fund-2020}{Donate
Now}

Peter S. Goodman reported from London, Abdi Latif Dahir from Nairobi,
Kenya, and Karan Deep Singh from Delhi, India. Simon Marks contributed
reporting from Addis Ababa, Ethiopia.

Advertisement

\protect\hyperlink{after-bottom}{Continue reading the main story}

\hypertarget{site-index}{%
\subsection{Site Index}\label{site-index}}

\hypertarget{site-information-navigation}{%
\subsection{Site Information
Navigation}\label{site-information-navigation}}

\begin{itemize}
\tightlist
\item
  \href{https://help.nytimes3xbfgragh.onion/hc/en-us/articles/115014792127-Copyright-notice}{©~2020~The
  New York Times Company}
\end{itemize}

\begin{itemize}
\tightlist
\item
  \href{https://www.nytco.com/}{NYTCo}
\item
  \href{https://help.nytimes3xbfgragh.onion/hc/en-us/articles/115015385887-Contact-Us}{Contact
  Us}
\item
  \href{https://www.nytco.com/careers/}{Work with us}
\item
  \href{https://nytmediakit.com/}{Advertise}
\item
  \href{http://www.tbrandstudio.com/}{T Brand Studio}
\item
  \href{https://www.nytimes3xbfgragh.onion/privacy/cookie-policy\#how-do-i-manage-trackers}{Your
  Ad Choices}
\item
  \href{https://www.nytimes3xbfgragh.onion/privacy}{Privacy}
\item
  \href{https://help.nytimes3xbfgragh.onion/hc/en-us/articles/115014893428-Terms-of-service}{Terms
  of Service}
\item
  \href{https://help.nytimes3xbfgragh.onion/hc/en-us/articles/115014893968-Terms-of-sale}{Terms
  of Sale}
\item
  \href{https://spiderbites.nytimes3xbfgragh.onion}{Site Map}
\item
  \href{https://help.nytimes3xbfgragh.onion/hc/en-us}{Help}
\item
  \href{https://www.nytimes3xbfgragh.onion/subscription?campaignId=37WXW}{Subscriptions}
\end{itemize}
