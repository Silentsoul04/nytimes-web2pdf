Sections

SEARCH

\protect\hyperlink{site-content}{Skip to
content}\protect\hyperlink{site-index}{Skip to site index}

\href{https://www.nytimes3xbfgragh.onion/section/business/economy}{Economy}

\href{https://myaccount.nytimes3xbfgragh.onion/auth/login?response_type=cookie\&client_id=vi}{}

\href{https://www.nytimes3xbfgragh.onion/section/todayspaper}{Today's
Paper}

\href{/section/business/economy}{Economy}\textbar{}A Lifeline to the
Jobless Has Problems With Fraud, and With Math

\url{https://nyti.ms/2ZuTNlr}

\begin{itemize}
\item
\item
\item
\item
\item
\end{itemize}

\hypertarget{the-coronavirus-outbreak}{%
\subsubsection{\texorpdfstring{\href{https://www.nytimes3xbfgragh.onion/news-event/coronavirus?name=styln-coronavirus-markets\&region=TOP_BANNER\&block=storyline_menu_recirc\&action=click\&pgtype=Article\&impression_id=15b96c00-f4b6-11ea-89a3-d9fd6cbe23e4\&variant=undefined}{The
Coronavirus
Outbreak}}{The Coronavirus Outbreak}}\label{the-coronavirus-outbreak}}

\begin{itemize}
\tightlist
\item
  live\href{https://www.nytimes3xbfgragh.onion/2020/09/11/world/covid-19-coronavirus.html?name=styln-coronavirus-markets\&region=TOP_BANNER\&block=storyline_menu_recirc\&action=click\&pgtype=Article\&impression_id=15b99310-f4b6-11ea-89a3-d9fd6cbe23e4\&variant=undefined}{Latest
  Updates}
\item
  \href{https://www.nytimes3xbfgragh.onion/interactive/2020/us/coronavirus-us-cases.html?name=styln-coronavirus-markets\&region=TOP_BANNER\&block=storyline_menu_recirc\&action=click\&pgtype=Article\&impression_id=15b99311-f4b6-11ea-89a3-d9fd6cbe23e4\&variant=undefined}{Maps
  and Cases}
\item
  \href{https://www.nytimes3xbfgragh.onion/interactive/2020/science/coronavirus-vaccine-tracker.html?name=styln-coronavirus-markets\&region=TOP_BANNER\&block=storyline_menu_recirc\&action=click\&pgtype=Article\&impression_id=15b99312-f4b6-11ea-89a3-d9fd6cbe23e4\&variant=undefined}{Vaccine
  Tracker}
\item
  \href{https://www.nytimes3xbfgragh.onion/2020/09/10/us/politics/fda-coronavirus-vaccine.html?name=styln-coronavirus-markets\&region=TOP_BANNER\&block=storyline_menu_recirc\&action=click\&pgtype=Article\&impression_id=15b99313-f4b6-11ea-89a3-d9fd6cbe23e4\&variant=undefined}{F.D.A.
  Regulators' Self-Defense}
\item
  \href{https://www.nytimes3xbfgragh.onion/2020/09/09/upshot/coronavirus-surprise-test-fees.html?name=styln-coronavirus-markets\&region=TOP_BANNER\&block=storyline_menu_recirc\&action=click\&pgtype=Article\&impression_id=15b99314-f4b6-11ea-89a3-d9fd6cbe23e4\&variant=undefined}{Surprise
  Test Fees}
\end{itemize}

Advertisement

\protect\hyperlink{after-top}{Continue reading the main story}

Supported by

\protect\hyperlink{after-sponsor}{Continue reading the main story}

\hypertarget{a-lifeline-to-the-jobless-has-problems-with-fraud-and-with-math}{%
\section{A Lifeline to the Jobless Has Problems With Fraud, and With
Math}\label{a-lifeline-to-the-jobless-has-problems-with-fraud-and-with-math}}

An emergency federal program faces growing issues with spurious claims,
and the flood of applicants may have led to overcounting the unemployed.

\includegraphics{https://static01.graylady3jvrrxbe.onion/images/2020/09/11/business/11virus-benefits2/11virus-benefits2-articleLarge.jpg?quality=75\&auto=webp\&disable=upscale}

By \href{https://www.nytimes3xbfgragh.onion/by/ben-casselman}{Ben
Casselman},
\href{https://www.nytimes3xbfgragh.onion/by/patricia-cohen}{Patricia
Cohen},
\href{https://www.nytimes3xbfgragh.onion/by/conor-dougherty}{Conor
Dougherty} and
\href{https://www.nytimes3xbfgragh.onion/by/nelson-d-schwartz}{Nelson D.
Schwartz}

\begin{itemize}
\item
  Sept. 11, 2020
\item
  \begin{itemize}
  \item
  \item
  \item
  \item
  \item
  \end{itemize}
\end{itemize}

Two weeks ago, shortly after she advertised an apartment for rent in the
Bay Area, Barbara Lamb found five envelopes from the state's
unemployment office in the building's communal mail slot. They kept
coming, day after day, until a stack of more than 30 piled up, bulging
with notices of benefit approvals, questionnaires about job status ---
and debit cards with money.

``They could barely get them through the mail slot, they were so
thick,'' she said.

But Ms. Lamb had not applied for benefits, and had never heard of the
people to whom the envelopes were sent. Fearing the address of the
vacant unit was being used as part of a fraud scheme to collect the
money, she contacted the F.B.I.

California is at the center of increasing concerns about extensive fraud
in a federal program to push unemployment benefits to freelancers,
part-timers and others lacking a safety net in the coronavirus pandemic.

\includegraphics{https://static01.graylady3jvrrxbe.onion/images/2020/09/11/business/11virus-benefits1/merlin_176862939_16c66641-e057-4594-ac55-21b05ce28aec-articleLarge.jpg?quality=75\&auto=webp\&disable=upscale}

At the same time, there is growing evidence of problems keeping track of
how many people are being paid through the program. The Labor Department
reports about 15 million claims for benefits nationwide. A comparison of
state and federal records by The New York Times suggests that total may
overstate the number of recipients by five million or more.

If the number of people getting unemployment benefits is lower than
officially reported, it could affect thinking about the scale of the
pandemic's economic impact. In addition, the taint of fraud could
undermine support for the program, and efforts to combat abuses may make
it harder for legitimate applicants to collect benefits, which are
distributed by the states.

The program, Pandemic Unemployment Assistance, is part of a \$2.2
trillion relief package hurriedly enacted in March. In the latest Labor
Department tally, the program accounted for nearly half the total
recipients collecting jobless benefits of any kind.

Those figures imply that nearly seven million people are collecting
Pandemic Unemployment Assistance benefits in California alone, far more
than its population would suggest. The state's own data suggests the
number may be less than two million. Experts on the unemployment system
say such discrepancies seem to reflect multiple counting as states
rushed out payments.

But a surge in new claims in California --- where they have risen to
more than 400,000 a week, twice the level in August --- is attributed
not to accounting, but to fraud.

``We do suspect that a big part of the unusual recent rise in P.U.A.
claims is linked to fraud,'' said Loree Levy, a spokeswoman for the
California Employment Development Department.
\href{https://www.nytimes3xbfgragh.onion/live/2020/09/10/business/stock-market-today-coronavirus/california-is-looking-into-fraud-in-a-pandemic-unemployment-program}{She
said the state was investigating} ``unscrupulous attacks'' exploiting
identity theft and vulnerabilities in the system.

\hypertarget{latest-updates-the-coronavirus-outbreak-and-the-economy}{%
\section{\texorpdfstring{\href{https://www.nytimes3xbfgragh.onion/live/2020/09/11/business/stock-market-today-coronavirus?action=click\&pgtype=Article\&state=default\&region=MAIN_CONTENT_1\&context=storylines_live_updates}{Latest
Updates: The Coronavirus Outbreak and the
Economy}}{Latest Updates: The Coronavirus Outbreak and the Economy}}\label{latest-updates-the-coronavirus-outbreak-and-the-economy}}

\href{https://www.nytimes3xbfgragh.onion/live/2020/09/11/business/stock-market-today-coronavirus?action=click\&pgtype=Article\&state=default\&region=MAIN_CONTENT_1\&context=storylines_live_updates\#the-nyse-may-move-its-data-center-out-of-new-jersey-in-response-to-a-proposed-tax}{9h
ago}

\href{https://www.nytimes3xbfgragh.onion/live/2020/09/11/business/stock-market-today-coronavirus?action=click\&pgtype=Article\&state=default\&region=MAIN_CONTENT_1\&context=storylines_live_updates\#the-nyse-may-move-its-data-center-out-of-new-jersey-in-response-to-a-proposed-tax}{The
N.Y.S.E. may move its data center out of New Jersey in response to a
proposed tax.}

\href{https://www.nytimes3xbfgragh.onion/live/2020/09/11/business/stock-market-today-coronavirus?action=click\&pgtype=Article\&state=default\&region=MAIN_CONTENT_1\&context=storylines_live_updates\#the-federal-budget-deficit-hit-3-trillion-as-of-august}{11h
ago}

\href{https://www.nytimes3xbfgragh.onion/live/2020/09/11/business/stock-market-today-coronavirus?action=click\&pgtype=Article\&state=default\&region=MAIN_CONTENT_1\&context=storylines_live_updates\#the-federal-budget-deficit-hit-3-trillion-as-of-august}{The
federal budget deficit hit \$3 trillion as of August.}

\href{https://www.nytimes3xbfgragh.onion/live/2020/09/11/business/stock-market-today-coronavirus?action=click\&pgtype=Article\&state=default\&region=MAIN_CONTENT_1\&context=storylines_live_updates\#warner-bros-pushes-the-release-of-wonder-woman-1984-to-christmas}{11h
ago}

\href{https://www.nytimes3xbfgragh.onion/live/2020/09/11/business/stock-market-today-coronavirus?action=click\&pgtype=Article\&state=default\&region=MAIN_CONTENT_1\&context=storylines_live_updates\#warner-bros-pushes-the-release-of-wonder-woman-1984-to-christmas}{Warner
Bros. pushes the release of `Wonder Woman 1984' to Christmas.}

\href{https://www.nytimes3xbfgragh.onion/live/2020/09/11/business/stock-market-today-coronavirus?action=click\&pgtype=Article\&state=default\&region=MAIN_CONTENT_1\&context=storylines_live_updates}{See
more updates}

More live coverage:
\href{https://www.nytimes3xbfgragh.onion/2020/09/11/world/covid-19-coronavirus.html?action=click\&pgtype=Article\&state=default\&region=MAIN_CONTENT_1\&context=storylines_live_updates}{Global}

Pandemic Unemployment Assistance is meant to provide benefits to the
self-employed, independent contractors, gig workers, part-timers and
others ordinarily ineligible for state unemployment insurance. Set up to
last through the end of the year, it was a major element of the CARES
Act, which economists widely agree has kept the country from a far
greater economic calamity. According to the Labor Department,
\href{https://oui.doleta.gov/unemploy/docs/cares_act_funding_state.html}{\$47
billion in pandemic unemployment benefits} have been paid so far.

Fraud is not uncommon in hastily assembled disaster programs,
\href{https://www.nytimes3xbfgragh.onion/2020/09/10/us/politics/ppp-fraud-coronavirus.html}{including
the Paycheck Protection Program}, the component of the CARES Act that
provided forgivable loans to small businesses to help weather the
pandemic without layoffs.

But signs of trouble with the Pandemic Unemployment Assistance program
have surfaced for months as people who did not file claims --- including
the \href{https://apnews.com/9b9a84686d61e1822486e9b04423640d}{governor
of Arkansas} --- found benefits issued in their names. A growing number
of states have signaled that the problems with the program go beyond the
routine.

California has warned that it is cutting off recipients when it detects
irregularities, like mailings stacking up at a given address. ``These
situations are believed to be fraud, and scammers will often try to
intercept, redirect, or gather mail associated with these claims,'' the
state's employment agency wrote.

Colorado said Thursday that in a six-week stretch this summer, 77
percent of new claims under the program were not legitimate.

``Nationally, it's just presented an opportunity for criminals to take
advantage of a program that doesn't have a lot of safety measures in
place,'' said Cher Haavind, deputy executive director of the Colorado
Department of Labor.

Citing a significant increase in fraud, the Labor Department
\href{https://wdr.doleta.gov/directives/attach/UIPL/UIPL_28-20.pdf}{set
aside \$100 million} recently to help states prevent, detect and
investigate misuse of Pandemic Unemployment Assistance and a smaller
federal jobless benefits program. But fraud is not the only issue
raising questions about the surge in recipients reflected in official
data.

\hypertarget{questions-about-counting}{%
\subsection{Questions About Counting}\label{questions-about-counting}}

Image

Waiting for a Kentucky unemployment office to open in June. The Pandemic
Unemployment Assistance program is funded by the federal government, but
administered by the states.Credit...Bryan Woolston/Reuters

Experts on the unemployment system
\href{https://www.bloomberg.com/news/articles/2020-06-29/u-s-jobless-claims-figures-inflated-by-states-backlog-clearing?srnd=premium\&sref=vuYGislZ}{figured
out months ago} that the tallies being reported to the Labor Department
were overstated in many states, most likely because of processing
backlogs that led to multiple counting of individual recipients. They
expected the issue to fade as backlogs cleared and job losses slowed.
Instead, the overcounting issue may even have become more serious in
some states.

``It's a perfect storm,'' said Stephen A. Wandner, a former top Labor
Department official who is now a senior fellow at the National Academy
of Social Insurance. ``You've got insane numbers of applications
compared to what the states are used to and inadequate numbers of staff
to process and adjudicate claims.''

Determining the scale of the problem on a national level has proved
difficult, however. Overwhelmed state employment offices have struggled
to provide timely data to the federal government, and there have been
several examples of outright errors making their way into the official
data.

At least some of the overcounting appears to reflect the way the Labor
Department collects statistics on unemployment benefits. The government
does not track the number of individual people receiving benefits, but
rather the total number of weeks of benefits claimed. During normal
times, when claims are processed on a weekly basis, the number of
recipients and the number of weeks are essentially the same --- each
person files for one week of benefits each week. (Further complicating
matters, the department tracks claims for benefits, not all of which are
approved.)

During the pandemic, however, the flood of claims overwhelmed state
employment offices. Because benefits are paid retroactively, processing
delays meant that by the time many people were approved for benefits,
they were owed several weeks at once --- so they counted as multiple
``continuing claims'' in a single week.

In the absence of a reliable count from the Labor Department, economists
have tried to estimate the number of recipients using data from surveys,
federal spending data from the Treasury Department and other sources.
Those approaches yield a wide range of estimates, but most suggest that
the official total overstates the true number of recipients by millions.

``It's almost certainly lower than is being reported,'' said Daniel
Zhao, senior economist for the career site Glassdoor. He said it was
hard to come up with a precise estimate, but that the true number was
most likely below 10 million, not the nearly 15 million counted by the
Labor Department.

The Labor Department did not immediately respond Friday to a query about
the reporting discrepancies.

Mr. Zhao said that the counting issues did not fundamentally alter the
bigger picture: Millions of Americans are still relying on unemployment
benefits to pay rent and buy food, and that number has fallen only
slowly over time.

\hypertarget{the-downside-of-streamlining}{%
\subsection{The Downside of
Streamlining}\label{the-downside-of-streamlining}}

Image

A demonstrator urged Congress in July to extend pandemic relief. Some
CARES Act provisions have lapsed, but Pandemic Unemployment Assistance
is authorized through December.Credit...Jonathan Ernst/Reuters

Pandemic Unemployment Assistance aims to capture those lacking a path
into traditional state benefits and accounts for the pandemic's
particular disruptions. A college student could qualify if she
interviewed for a job in February and was set to start working in March
but never did. So could people with limited earnings histories, and some
of those unable to work because of child-care needs arising from school
shutdowns.

The minimum payment is usually half the average weekly benefit paid
under a state's regular unemployment program. The maximum for an
individual ranges from \$235 a week in Mississippi to \$823 in
Massachusetts,
\href{https://www.ziprecruiter.com/blog/unemployment-benefits-by-state/}{according
to the job site ZipRecruiter}.

And the claims process is streamlined compared with conventional
unemployment insurance, making it more vulnerable to fraud, said Michele
Evermore, senior researcher and policy analyst at the National
Employment Law Project.

Before collecting state unemployment insurance, applicants usually must
provide proof of past work or have state agencies contact employers.
With Pandemic Unemployment Assistance, many people can start collecting
the minimum with far less documentation. Then they generally have 21
days to provide evidence of lost work, like a pay stub or a 1099 form
from the Internal Revenue Service.

In an emergency program like Pandemic Unemployment Assistance, Ms.
Evermore said, there is a natural tension between the need to get
payments flowing and the risk that some people will take advantage and
fraudulently apply for benefits.

``There is a choice between denying benefits or accidentally overpaying
people,'' she said. ``With Pandemic Unemployment Assistance, scammers
may be getting money that is meant for the unemployed.''

Erica Quealy, communications director of the Michigan Department of
Labor and Economic Opportunity, said the program had become the prey of
``large fraud rings.'' Michigan's attorney general has conducted
hundreds of investigations, and the state has appointed a special fraud
adviser and brought in the consulting firm Deloitte to help.

Some schemes involve using false Social Security cards and fake driver's
licenses to apply. One man was charged with filing applications in
Pennsylvania under false names, and then having benefits worth \$150,000
in debit cards mailed to addresses in Michigan, according to the
\href{https://www.oig.dol.gov/public/Press\%20Releases/Baker\%20complaint\%20pr.pdf}{state
attorney general.} Prosecutors said he used the money to buy a \$45,000
Rolex watch.

The rate of fraudulent claims in Colorado has been striking. After
adding more screening measures to catch fraud, Colorado found that more
than three out of four claims filed over a six-week period for jobless
benefits under the federal Pandemic Unemployment Assistance program were
bogus.

On Thursday, the state said it had reduced its count of new claims filed
from July 12 to Aug. 22 by 48,000 because of new fraud-detection
efforts. Before being discovered, though, those responsible for the
fraud were able to collect \$40 million during that period, said Jeff
Fitzgerald, head of the state's unemployment insurance program.

Officials estimated that the state's screening tools had saved the
federal government \$750 million to \$1 billion over eight weeks by
halting wrongful payments or by flagging them before they were made.

``What we're looking at is quite sophisticated,'' Mr. Fitzgerald said.
``It is something that a common individual would not be able to do, and
really it points to orchestrated, very sophisticated, large fraud
schemes. These aren't onesies and twosies.''

The fraud detection efforts are putting an enormous burden on the
states. Mr. Fitzgerald said that Colorado had assigned 60 people to
investigate unemployment fraud, compared with five in normal times.

In the meantime, the mail keeps coming. Ms. Lamb, whose East Bay rental
unit had been inundated with envelopes, rubber-banded them into neat
stacks Thursday to send back to the state unemployment office. She had
given the five addressees' names to the F.B.I.

On Friday, two more envelopes arrived from the state, bearing a new
name.

Tara Siegel Bernard contributed reporting, and Sheelagh McNeill
contributed research.

Advertisement

\protect\hyperlink{after-bottom}{Continue reading the main story}

\hypertarget{site-index}{%
\subsection{Site Index}\label{site-index}}

\hypertarget{site-information-navigation}{%
\subsection{Site Information
Navigation}\label{site-information-navigation}}

\begin{itemize}
\tightlist
\item
  \href{https://help.nytimes3xbfgragh.onion/hc/en-us/articles/115014792127-Copyright-notice}{©~2020~The
  New York Times Company}
\end{itemize}

\begin{itemize}
\tightlist
\item
  \href{https://www.nytco.com/}{NYTCo}
\item
  \href{https://help.nytimes3xbfgragh.onion/hc/en-us/articles/115015385887-Contact-Us}{Contact
  Us}
\item
  \href{https://www.nytco.com/careers/}{Work with us}
\item
  \href{https://nytmediakit.com/}{Advertise}
\item
  \href{http://www.tbrandstudio.com/}{T Brand Studio}
\item
  \href{https://www.nytimes3xbfgragh.onion/privacy/cookie-policy\#how-do-i-manage-trackers}{Your
  Ad Choices}
\item
  \href{https://www.nytimes3xbfgragh.onion/privacy}{Privacy}
\item
  \href{https://help.nytimes3xbfgragh.onion/hc/en-us/articles/115014893428-Terms-of-service}{Terms
  of Service}
\item
  \href{https://help.nytimes3xbfgragh.onion/hc/en-us/articles/115014893968-Terms-of-sale}{Terms
  of Sale}
\item
  \href{https://spiderbites.nytimes3xbfgragh.onion}{Site Map}
\item
  \href{https://help.nytimes3xbfgragh.onion/hc/en-us}{Help}
\item
  \href{https://www.nytimes3xbfgragh.onion/subscription?campaignId=37WXW}{Subscriptions}
\end{itemize}
