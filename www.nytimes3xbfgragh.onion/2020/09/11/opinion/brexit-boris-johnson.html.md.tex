Sections

SEARCH

\protect\hyperlink{site-content}{Skip to
content}\protect\hyperlink{site-index}{Skip to site index}

\href{https://myaccount.nytimes3xbfgragh.onion/auth/login?response_type=cookie\&client_id=vi}{}

\href{https://www.nytimes3xbfgragh.onion/section/todayspaper}{Today's
Paper}

\href{/section/opinion}{Opinion}\textbar{}Remember Brexit?

\url{https://nyti.ms/3moPFgC}

\begin{itemize}
\item
\item
\item
\item
\item
\end{itemize}

Advertisement

\protect\hyperlink{after-top}{Continue reading the main story}

\href{/section/opinion}{Opinion}

Supported by

\protect\hyperlink{after-sponsor}{Continue reading the main story}

\hypertarget{remember-brexit}{%
\section{Remember Brexit?}\label{remember-brexit}}

The drama continues to unfold.

By
\href{https://www.nytimes3xbfgragh.onion/interactive/opinion/editorialboard.html}{The
Editorial Board}

The editorial board is a group of opinion journalists whose views are
informed by expertise, research, debate and certain longstanding ****
\href{https://www.nytimes3xbfgragh.onion/interactive/2018/opinion/editorialboard.html}{values}.
It is separate from the newsroom.

\begin{itemize}
\item
  Sept. 11, 2020
\item
  \begin{itemize}
  \item
  \item
  \item
  \item
  \item
  \end{itemize}
\end{itemize}

\includegraphics{https://static01.graylady3jvrrxbe.onion/images/2020/09/12/opinion/11brexit/11brexit-articleLarge.jpg?quality=75\&auto=webp\&disable=upscale}

Before the coronavirus pandemic, Britain's tortuous debate on how to
exit the European Union --- while maintaining all the perks of
membership --- was a perennial of global news. The long-running
political drama went from one ``final'' Brexit deadline to the next
until, at 11 p.m. Coordinated Universal Time on Jan. 31, 2020, Britain
left the union.

But not entirely. That moment started an
\href{https://www.bbc.com/news/uk-politics-50838994}{11-month
transition} period during which British and E.U. leaders were supposed
to negotiate their future relationship. Until that period is concluded,
Britain remains in the E.U. customs union and single market, and is
bound by E.U. law, though it is without a seat on any E.U. institution.

The talks have made no progress, and this week they descended into a
nasty dispute over a proposed British law that, in the words of an E.U.
executive, amounts to ``an extremely serious violation'' of
international law. E.U. leaders have demanded that the law be withdrawn;
the British say no way.

Britain's prime minister, Boris Johnson, who got elected 10 months ago
on the claim that he had an ``oven-ready'' Brexit deal, is now
\href{https://www.nytimes3xbfgragh.onion/2020/09/10/world/europe/brexit-boris-johnson-ireland.html}{threatening
to leave the European Union}, with or without a deal, on Oct. 15. Mr.
Johnson says that leaving without a deal would be a
``\href{https://www.nytimes3xbfgragh.onion/2020/09/08/world/europe/boris-johnson-brexit-northern-ireland.html}{good
outcome}'' for Britain, a claim as ill-founded today as it was when the
British were on the brink of a deal-less exit last year.

The latest twist in the saga revolves around a bill recently introduced
by Mr. Johnson's Conservative government that proposes to override a key
element in the withdrawal agreement that Mr. Johnson signed in January.
Not surprisingly, the problem is over Northern Ireland, the Gordian knot
that has bedeviled all negotiations past.

The question has been how to keep the border across Ireland open once
Britain, of which Northern Ireland is part, leaves the European Union.
As a fallback, the withdrawal treaty created some complex customs
arrangements between Northern Ireland and the rest of Britain that would
go into effect if other arrangements were not agreed to. With that
possibility looming large, Mr. Johnson's new law would effectively allow
Britain to decide on its own how to manage the movement of goods across
the Irish Sea. That, a Cabinet minister conceded, would break
international law, though only
``\href{https://www.bbc.com/news/uk-politics-54073836}{in a very
specific and limited way}.''

That qualifier has failed to prevent a brouhaha. Theresa May, who
wrestled bravely and futilely with Brexit as Mr. Johnson's predecessor,
questioned why any future partner would trust Britain if it openly
violated its legal obligations.

Across the Atlantic, American politicians declared they would not permit
anything that might undo the 1998 Good Friday agreement, which ended
decades of sectarian strife in Ireland. Nancy Pelosi, the Democratic
speaker of the House, warned on Wednesday that if Britain violated the
treaty there would be ``absolutely no chance'' of getting a U.S.-British
trade agreement through Congress. Mr. Johnson, it will be recalled, had
argued that leaving the European Union would allow Britain to strike a
better deal with the United States.

So much of the Brexit serial is back, yet again: the ``final''
deadlines, the brinkmanship, the talk of a ``hard'' (read no-deal)
Brexit, the grumbling in Brussels about British intransigence, the spat
over the Irish border.

If the soundtrack is familiar, however, the background is far different.
An economic crisis spawned by the coronavirus pandemic has diminished
the popularity of Mr. Johnson and his Conservatives. The European Union
has shown no sign of yielding on key issues --- state subsidies for
British industry and fishing rights. President Trump, an ardent
supporter of Brexit, is facing a tough re-election battle. The
opposition Labour Party is now led by
\href{https://www.nytimes3xbfgragh.onion/2020/04/04/world/europe/labour-party-keir-starmer.html}{Keir
Starmer}, who is regarded as a greater challenge to Conservatives than
his predecessor, Jeremy Corbyn. And with a no-deal exit looking more
likely, polls in Scotland have shown a sharp uptick in support for
independence.

The hard fact is that without a deal there can be no good outcome. A
no-deal Brexit would mean that as of Jan. 1, 2021, there will be tariffs
and border checks, higher prices for many goods, long lines of trucks at
the English Channel. The British service industry would lose guaranteed
access to the European Union, affecting numerous people ranging from
bankers to musicians.

All these consequences have been endlessly hashed over since the British
voted for Brexit in a referendum in June 2016. But four years of
anguished debates and wrenching negotiations appear to have done little
to close the gap between those who see membership in the European Union
as a leash on British sovereignty --- as demonstrated recently when
Britain's chief negotiator, David Frost, snapped that Britain would
never become a
``\href{https://www.bbc.com/news/uk-politics-54045653}{client state}''
of the union --- and those who view the union as a community of shared
economic rules and common values that benefit both sides.

Mr. Johnson was thrust into office less on the strength of his often
theatrical speaking skills than on fatigue with the Brexit process.
Breaking the withdrawal treaty is not the way forward, and that dodge
should be abandoned. There is still time, and it would be best for
Britain and the Continent if Mr. Johnson were to use that time to reach
a deal that would adhere to international law and minimize the
disruption of a painful separation.

\emph{The Times is committed to publishing}
\href{https://www.nytimes3xbfgragh.onion/2019/01/31/opinion/letters/letters-to-editor-new-york-times-women.html}{\emph{a
diversity of letters}} \emph{to the editor. We'd like to hear what you
think about this or any of our articles. Here are some}
\href{https://help.nytimes3xbfgragh.onion/hc/en-us/articles/115014925288-How-to-submit-a-letter-to-the-editor}{\emph{tips}}\emph{.
And here's our email:}
\href{mailto:letters@NYTimes.com}{\emph{letters@NYTimes.com}}\emph{.}

\emph{Follow The New York Times Opinion section on}
\href{https://www.facebookcorewwwi.onion/nytopinion}{\emph{Facebook}}\emph{,}
\href{http://twitter.com/NYTOpinion}{\emph{Twitter (@NYTopinion)}}
\emph{and}
\href{https://www.instagram.com/nytopinion/}{\emph{Instagram}}\emph{.}

Advertisement

\protect\hyperlink{after-bottom}{Continue reading the main story}

\hypertarget{site-index}{%
\subsection{Site Index}\label{site-index}}

\hypertarget{site-information-navigation}{%
\subsection{Site Information
Navigation}\label{site-information-navigation}}

\begin{itemize}
\tightlist
\item
  \href{https://help.nytimes3xbfgragh.onion/hc/en-us/articles/115014792127-Copyright-notice}{©~2020~The
  New York Times Company}
\end{itemize}

\begin{itemize}
\tightlist
\item
  \href{https://www.nytco.com/}{NYTCo}
\item
  \href{https://help.nytimes3xbfgragh.onion/hc/en-us/articles/115015385887-Contact-Us}{Contact
  Us}
\item
  \href{https://www.nytco.com/careers/}{Work with us}
\item
  \href{https://nytmediakit.com/}{Advertise}
\item
  \href{http://www.tbrandstudio.com/}{T Brand Studio}
\item
  \href{https://www.nytimes3xbfgragh.onion/privacy/cookie-policy\#how-do-i-manage-trackers}{Your
  Ad Choices}
\item
  \href{https://www.nytimes3xbfgragh.onion/privacy}{Privacy}
\item
  \href{https://help.nytimes3xbfgragh.onion/hc/en-us/articles/115014893428-Terms-of-service}{Terms
  of Service}
\item
  \href{https://help.nytimes3xbfgragh.onion/hc/en-us/articles/115014893968-Terms-of-sale}{Terms
  of Sale}
\item
  \href{https://spiderbites.nytimes3xbfgragh.onion}{Site Map}
\item
  \href{https://help.nytimes3xbfgragh.onion/hc/en-us}{Help}
\item
  \href{https://www.nytimes3xbfgragh.onion/subscription?campaignId=37WXW}{Subscriptions}
\end{itemize}
