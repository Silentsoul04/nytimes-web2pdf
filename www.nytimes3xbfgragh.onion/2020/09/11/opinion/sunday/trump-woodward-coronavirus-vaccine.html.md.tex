Sections

SEARCH

\protect\hyperlink{site-content}{Skip to
content}\protect\hyperlink{site-index}{Skip to site index}

\href{https://www.nytimes3xbfgragh.onion/section/opinion/sunday}{Sunday
Review}

\href{https://myaccount.nytimes3xbfgragh.onion/auth/login?response_type=cookie\&client_id=vi}{}

\href{https://www.nytimes3xbfgragh.onion/section/todayspaper}{Today's
Paper}

\href{/section/opinion/sunday}{Sunday Review}\textbar{}The Towering Lies
of President Trump

\url{https://nyti.ms/2GQInBX}

\begin{itemize}
\item
\item
\item
\item
\item
\end{itemize}

Advertisement

\protect\hyperlink{after-top}{Continue reading the main story}

\href{/section/opinion}{Opinion}

Supported by

\protect\hyperlink{after-sponsor}{Continue reading the main story}

\hypertarget{the-towering-lies-of-president-trump}{%
\section{The Towering Lies of President
Trump}\label{the-towering-lies-of-president-trump}}

It's simple. Everything that benefits Mr. Trump is true and everything
that inconveniences him is false.

By Greg Weiner

Mr. Weiner is a political scientist and was a senior Senate aide to Bob
Kerrey, Democrat of Nebraska.

\begin{itemize}
\item
  Sept. 11, 2020
\item
  \begin{itemize}
  \item
  \item
  \item
  \item
  \item
  \end{itemize}
\end{itemize}

\includegraphics{https://static01.graylady3jvrrxbe.onion/images/2020/09/13/opinion/sunday/11weiner/11weiner-articleLarge-v3.jpg?quality=75\&auto=webp\&disable=upscale}

President Trump's
\href{https://www.nytimes3xbfgragh.onion/2020/09/09/us/politics/woodward-trump-book-virus.html}{taped
admission} to Bob Woodward that he deliberately misled Americans about
the danger of the coronavirus makes him morally culpable in the ensuing
tragedy. But Mr. Trump actually forfeited the battle against the
pandemic long before it erupted in Wuhan, China, in the first place. His
tragic failures are a result of a systematic skepticism --- an
opportunistic cynicism --- whose orienting principle is that everything
that benefits Mr. Trump is true and everything that inconveniences him
is false.

Five centuries ago, Niccolò Machiavelli called this the ``effectual
truth'': Claims that are true,
\href{https://www.google.com/books/edition/The_Prince/ehzOd8DVlNkC?hl=en\&gbpv=1\&dq=mansfield+the+prince\&printsec=frontcover}{he
wrote} in ``The Prince,'' are so not because they correspond to
objective reality but because they are politically ``useful.'' Mr.
Trump's use of skepticism as a standard for interpreting reality now
risks undermining public faith in an eventual vaccine.

In his capstone address to the Republican National Convention, Mr. Trump
\href{https://www.nytimes3xbfgragh.onion/live/2020/08/27/us/rnc-fact-check}{promised}
a vaccine ``before the end of the year or maybe even sooner.'' That
might be written off as a mere prediction of events, except that Mr.
Trump tipped his hand in a pair of tweets directed against the Food and
Drug Administration. Together, they displayed the systematic skepticism
that has derailed the administration's pandemic response. But that
skepticism is not isolated to the pandemic. Its degrading effects on
politics may, instead, be one of Mr. Trump's most regrettable and
enduring legacies.

In one tweet, Mr. Trump
\href{https://twitter.com/realdonaldtrump/status/1297148038385991680?s=11}{rebuked}
the F.D.A. for revoking its emergency authorization of
hydroxychloroquine, the antimalarial drug the president has repeatedly
touted as a coronavirus treatment: ``Many doctors and studies disagree
with this!''

This is almost certainly accurate in the strict sense that ``many''
refers to an unspecified number exceeding one. Especially in the early
and uncertain period of a new disease, scientific opinions are like
parenting books and diet fads: One can go looking for something that
confirms a prior conviction and be certain to find it.

That is different from trying to make the best judgment one can on the
basis of competing views, some of which are more credible and
authoritative than others. The fact that ``many'' people are willing to
say anything, especially if the president cues them to do so, proves
nothing except adherence to an intellectually and morally corrosive
belief that nothing can be proved.

Early in this administration, it was reasonable to say that
\href{https://www.nationalaffairs.com/publications/detail/trump-and-truth}{plausibility
was the new truth}: The mere existence of a believable opposing claim
was sufficient to rebut a balanced assessment of the totality of the
evidence. But when skepticism is a whole system of thought, assertion is
the new truth. A claim no longer needs to be believable. It simply needs
to be made.

Mr. Trump started down that road before he was inaugurated,
preposterously claiming that millions of fraudulent ballots cheated him
out of the 2016 popular vote. Machiavelli would recognize the terms in
which Mr. Trump recently recast this claim: that he won in
``\href{https://www.msnbc.com/rachel-maddow-show/despite-reality-trump-said-he-won-popular-vote-true-sense-n1239070}{a
true sense}.'' Mr. Trump continued within hours of taking the
presidential oath, forcing his staff to insist against photographic
evidence that the crowd at his inauguration had been the largest in
history.

That skepticism now appears to be guiding the search for a vaccine. The
same morning Mr. Trump insisted on the effectiveness of
hydroxychloroquine, another presidential tweet
\href{https://twitter.com/realdonaldtrump/status/1297138862108663808?s=11}{accused}
``the deep state, or whoever, over at the FDA'' of ``making it very
difficult for drug companies to get people in order to test the vaccines
and therapeutics. Obviously, they are hoping to delay the answer until
after November 3rd.''

He proceeded to demand a ``a focus on speed, and saving lives.'' But the
reference to Election Day was the giveaway: The question for Mr. Trump
was not simply whether the F.D.A. was dragging its feet or elevating
medical caution over the urgency of a vaccine. That is a legitimate
debate that must weigh the unknown long-term effects of an
insufficiently tested vaccine with the known and immediate effects of
its unavailability. By contrast, Mr. Trump was concerned not with
prudential or objective assessments of proposed vaccines but rather with
their ``effectual truth.'' By definition, the F.D.A. was moving too
slowly because moving more quickly would help him politically.

A degree of skepticism can be a healthy disposition. Politically, it
encourages the wariness of concentrated power appropriate to a republic.
Intellectually, it values a prudent humility about what we can know over
the certitude to which expertise is often prone.

But skepticism as a system for interpreting the world --- which
conservatives might once have called ``nihilism'' --- is different from
skepticism as a disposition. Systematic skepticism axiomatically
questions the truth or relevance of anything that does not serve Mr.
Trump's personal ambition.

What has happened since the pandemic arrived in the United States has
simply been the application of that system of thought to unfolding
events. What is good for Mr. Trump is good for the nation, and a
pandemic is not good for Mr. Trump. That is a more plausible explanation
for his early denials than the president's claim that he downplayed the
coronavirus to prevent panic. Honesty about dangers, and a clear
strategy for overcoming them, prevent panic. Lies fuel it, unless the
panic to which Mr. Trump referred was personal fear about his own
interests.

As a result of all this, the still evolving science surrounding Covid-19
is routinely interpreted based on its perceived effects on Mr. Trump's
re-election. The president's personal refusal to wear a mask except when
it seems politically convenient to do so --- and his outright mocking of
those who have worn them --- has transformed the single most effective
device for preventing contagion into a statement of political loyalties.
That has vastly more to do with the pandemic's tragic toll than Mr.
Trump's policy choices do.

A Republican sheriff in Florida recently went as far as
\href{https://www.ocala.com/story/news/politics/county/2020/08/11/marion-county-deputies-ordered-not-to-wear-masks/113049372/}{banning
masks} for deputies on duty and even for visitors to his office. The
sheriff's reasoning about wearing masks, which made
\href{https://www.washingtonpost.com/nation/2020/08/12/masks-florida-ban-billy-woods/}{national
news}, was rooted in systematic skepticism. He claimed that the
effectiveness of masks was disproved by the fact that someone,
somewhere, disputed it: ``The fact is, the amount of professionals that
give the reasons we should, I can find the exact same amount of
professionals that say why we shouldn't.'' On the other side of the
landscape are Trump supporters who claim that his lies are obvious and
transparent whoppers that no one believes, which is evidently untrue.

At this point, there is every reason to worry that a vaccine will be
approved on an emergency basis before its safety has been fully
established. The absurdity of the situation distills to this:
Pharmaceutical executives, who have
\href{https://www.nytimes3xbfgragh.onion/2020/09/08/health/9-drug-companies-pledge-coronavirus-vaccine.html}{pledged}
not to seek authorization for a vaccine until adequate scientific
evidence is gathered, are now acting as a check on federal regulators
rather than the other way around.

Absent systematic skepticism, the balance of considerations might
justify rapid authorization. But Mr. Trump has undermined any reason for
confidence in such a judgment by staking his claim to the effectual
rather than the actual truth. When a vaccine is actually ready, public
trust in it will be a vital tool for ending the pandemic. On the other
hand, should a vaccine suddenly appear on the eve of the election under
obvious political pressure, the systematic skeptics who interpret truth
through Mr. Trump's eyes could prove their fidelity to him by being the
first in line to receive it.

Greg Weiner (\href{https://twitter.com/GregWeiner1}{@GregWeiner1}) is a
political scientist at Assumption University, a visiting scholar at the
American Enterprise Institute and the author of ``The Political
Constitution: The Case Against Judicial Supremacy.''

\emph{The Times is committed to publishing}
\href{https://www.nytimes3xbfgragh.onion/2019/01/31/opinion/letters/letters-to-editor-new-york-times-women.html}{\emph{a
diversity of letters}} \emph{to the editor. We'd like to hear what you
think about this or any of our articles. Here are some}
\href{https://help.nytimes3xbfgragh.onion/hc/en-us/articles/115014925288-How-to-submit-a-letter-to-the-editor}{\emph{tips}}\emph{.
And here's our email:}
\href{mailto:letters@NYTimes.com}{\emph{letters@NYTimes.com}}\emph{.}

\emph{Follow The New York Times Opinion section on}
\href{https://www.facebookcorewwwi.onion/nytopinion}{\emph{Facebook}}\emph{,}
\href{http://twitter.com/NYTOpinion}{\emph{Twitter (@NYTopinion)}}
\emph{and}
\href{https://www.instagram.com/nytopinion/}{\emph{Instagram}}\emph{.}

Advertisement

\protect\hyperlink{after-bottom}{Continue reading the main story}

\hypertarget{site-index}{%
\subsection{Site Index}\label{site-index}}

\hypertarget{site-information-navigation}{%
\subsection{Site Information
Navigation}\label{site-information-navigation}}

\begin{itemize}
\tightlist
\item
  \href{https://help.nytimes3xbfgragh.onion/hc/en-us/articles/115014792127-Copyright-notice}{©~2020~The
  New York Times Company}
\end{itemize}

\begin{itemize}
\tightlist
\item
  \href{https://www.nytco.com/}{NYTCo}
\item
  \href{https://help.nytimes3xbfgragh.onion/hc/en-us/articles/115015385887-Contact-Us}{Contact
  Us}
\item
  \href{https://www.nytco.com/careers/}{Work with us}
\item
  \href{https://nytmediakit.com/}{Advertise}
\item
  \href{http://www.tbrandstudio.com/}{T Brand Studio}
\item
  \href{https://www.nytimes3xbfgragh.onion/privacy/cookie-policy\#how-do-i-manage-trackers}{Your
  Ad Choices}
\item
  \href{https://www.nytimes3xbfgragh.onion/privacy}{Privacy}
\item
  \href{https://help.nytimes3xbfgragh.onion/hc/en-us/articles/115014893428-Terms-of-service}{Terms
  of Service}
\item
  \href{https://help.nytimes3xbfgragh.onion/hc/en-us/articles/115014893968-Terms-of-sale}{Terms
  of Sale}
\item
  \href{https://spiderbites.nytimes3xbfgragh.onion}{Site Map}
\item
  \href{https://help.nytimes3xbfgragh.onion/hc/en-us}{Help}
\item
  \href{https://www.nytimes3xbfgragh.onion/subscription?campaignId=37WXW}{Subscriptions}
\end{itemize}
