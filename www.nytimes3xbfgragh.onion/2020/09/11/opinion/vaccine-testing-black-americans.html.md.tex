Sections

SEARCH

\protect\hyperlink{site-content}{Skip to
content}\protect\hyperlink{site-index}{Skip to site index}

\href{https://myaccount.nytimes3xbfgragh.onion/auth/login?response_type=cookie\&client_id=vi}{}

\href{https://www.nytimes3xbfgragh.onion/section/todayspaper}{Today's
Paper}

\href{/section/opinion}{Opinion}\textbar{}We Need to Recruit More Black
Americans in Vaccine Trials

\url{https://nyti.ms/33itm3d}

\begin{itemize}
\item
\item
\item
\item
\item
\end{itemize}

Advertisement

\protect\hyperlink{after-top}{Continue reading the main story}

\href{/section/opinion}{Opinion}

Supported by

\protect\hyperlink{after-sponsor}{Continue reading the main story}

\hypertarget{we-need-to-recruit-more-black-americans-in-vaccine-trials}{%
\section{We Need to Recruit More Black Americans in Vaccine
Trials}\label{we-need-to-recruit-more-black-americans-in-vaccine-trials}}

Pharmaceutical companies and the government must step up their efforts
to enroll a diverse group of volunteers.

By Wayne A. I. Frederick, Valerie Montgomery Rice, David M. Carlisle and
James E. K. Hildreth

Wayne A. I. Frederick is the president of Howard University; Valerie
Montgomery Rice is the president of Morehouse School of Medicine; David
M. Carlisle is president of Charles R. Drew University of Medicine and
Science; James E. K. Hildreth is president of Meharry Medical College.

\begin{itemize}
\item
  Sept. 11, 2020
\item
  \begin{itemize}
  \item
  \item
  \item
  \item
  \item
  \end{itemize}
\end{itemize}

\includegraphics{https://static01.graylady3jvrrxbe.onion/images/2020/09/14/opinion/09Frederick1/merlin_174762276_06a88f82-9650-4559-bf3d-d4b7c0f7e054-articleLarge.jpg?quality=75\&auto=webp\&disable=upscale}

The global race is on for a vaccine to combat the coronavirus, but the
question is: Who will be included?

To date, several companies have reached Phase 3 trials for an
experimental vaccine --- including Moderna, Pfizer, AstraZeneca and
CanSino. AstraZeneca
\href{https://www.statnews.com/2020/09/09/astrazeneca-covid19-vaccine-trial-hold-patient-report/}{recently
announced} a pause in its process to check a complication with one
participant. Despite this setback, the early results are encouraging.

Yet these trials have not met an important challenge: recruiting an
appropriately diverse group of participants --- even though Covid-19 has
taken a disproportionate toll on communities of color and on Black
Americans in particular.

Drugmakers approved for Phase 3 trials have been slow to report the
breakdown of participants. But Dr. Francis Collins, director of the
National Institutes of Health,
\href{https://www.cnn.com/2020/08/16/health/covid-19-vaccine-trial-black-minority-recruitment/index.html}{told
CNN} that Moderna deserved a C for recruiting minorities. As of~Sept. 4,
Moderna \href{https://www.modernatx.com/cove-study}{reported} 26 percent
of study participants from communities of color, including Black or
African-American, Latinx, American Indian and Alaskan Native.

Granted, this is an improvement from most studies. In fact, in clinical
trials overall, African-American participation hovers around an abysmal
\href{https://clinicalresearchpathways.org/diversity/diversity-statistics-infographic/}{5
percent}, despite being 13 percent of the U.S. population. Research
participants should look like the population --- that would be 32
percent for those four groups. And Anthony Fauci, director of the
National Institute of Allergy and Infectious Diseases, has called for
twice that number because of how hard Covid-19 has hit them.

The paucity of diversity in these clinical trials creates problems on
two fronts: treatment and trust.

On the treatment front, a vaccine with limited testing could have
unanticipated effects on Black bodies. As with all drug trials, the
impact of medication can differ significantly, depending on the genetic
makeup of the population. This is even more so with vaccines that depend
on altering the immune system. It is therefore vital that the trials,
which usually hold about 30,000 participants, include as diverse a set
of participants as possible.

As it is, African-Americans cope with higher rates of cancer, diabetes,
heart disease and hypertension. Because these conditions can put people
infected with the virus at risk, it's extra important that
African-Americans play a critical role in testing.

Trust is also an issue. Unsurprisingly, Black Americans are suspicious
of leading philanthropists, the pharmaceutical industry and the American
health care system. The litany of abuses committed by health
professionals in the name of ``research'' that inflicted harm on
thousands of Black Americans will forever be a stain on the soul of our
nation. The African-American community knows well the infamous racism of
the Tuskegee syphilis experiment and the exploitation of Henrietta
Lacks.

Black doctors are the best way to build trust in our communities. But
they need help. Without significant participation in clinical trials,
there will be no proof that our patients should trust the vaccine.

Morehouse School of Medicine and Meharry Medical College have been
identified as clinical trial sites, and are in the early stages of
volunteer recruitment. But an expansion is necessary. Researchers and
the medical industry should engage the remaining two Black and minority
serving medical schools --- Charles R. Drew College of Medicine and
Howard University College of Medicine --- in the vaccine trials now. In
addition, the 104 Historically Black Colleges and Universities can serve
as credible messengers to distribute information and foster trust in
communities throughout the country.

Their involvement should include the recruitment of patients,
participation in the science, and development of the plan to distribute
the vaccine to the most vulnerable communities. Unlike what happened
with the development of antiviral treatment for AIDS, the
African-American population should not be last to get access to the
lifesaving medication.

Economic barriers must also be lifted. Institutions must work with
African-Americans who can't take time away from work, by engaging with
employers to provide time for employee participation as a health
incentive. And because our communities suffer from a lack of reliable
transportation, institutions must also conduct trials where we live.

The African-American community must also be willing to engage: ask hard
questions and consult trusted sources in order to assuage legitimate
concerns.

The Black Lives Matter movement reminds us that we do not have to be
confined by the ugliest parts of our nation's history or our fallen
human nature. We have an opportunity to do something better in this
moment. Simply put, the largest population being killed by Covid-19
should have a significant role in development of a treatment.

The human rights activist
\href{https://blogs.cdc.gov/healthequity/2015/04/30/mhmonth/}{Fannie Lou
Hamer} grew up under the brutality of Jim Crow in the Mississippi Delta
that included forced sterilization --- an abhorrent practice so common
it became known as a ``Mississippi appendectomy.'' Mrs. Hamer reached a
point where the status quo would simply not do, famously remarking ``I
am sick and tired of being sick and tired.'' She demanded full inclusion
in American democracy for all Black Americans.

We find ourselves at another inflection point where the status quo
cannot stand. True change requires that government and industry make
every effort to achieve true diversity in clinical trials. Black lives
depend on it.

Wayne A. I. Frederick is the president of Howard University; Valerie
Montgomery Rice is the president of Morehouse School of Medicine; David
M. Carlisle is president of Charles R. Drew University of Medicine and
Science; James E. K. Hildreth is president of Meharry Medical College.

\emph{The Times is committed to publishing}
\href{https://www.nytimes3xbfgragh.onion/2019/01/31/opinion/letters/letters-to-editor-new-york-times-women.html}{\emph{a
diversity of letters}} \emph{to the editor. We'd like to hear what you
think about this or any of our articles. Here are some}
\href{https://help.nytimes3xbfgragh.onion/hc/en-us/articles/115014925288-How-to-submit-a-letter-to-the-editor}{\emph{tips}}\emph{.
And here's our email:}
\href{mailto:letters@NYTimes.com}{\emph{letters@NYTimes.com}}\emph{.}

\emph{Follow The New York Times Opinion section on}
\href{https://www.facebookcorewwwi.onion/nytopinion}{\emph{Facebook}}\emph{,}
\href{http://twitter.com/NYTOpinion}{\emph{Twitter (@NYTopinion)}}
\emph{and}
\href{https://www.instagram.com/nytopinion/}{\emph{Instagram}}\emph{.}

Advertisement

\protect\hyperlink{after-bottom}{Continue reading the main story}

\hypertarget{site-index}{%
\subsection{Site Index}\label{site-index}}

\hypertarget{site-information-navigation}{%
\subsection{Site Information
Navigation}\label{site-information-navigation}}

\begin{itemize}
\tightlist
\item
  \href{https://help.nytimes3xbfgragh.onion/hc/en-us/articles/115014792127-Copyright-notice}{©~2020~The
  New York Times Company}
\end{itemize}

\begin{itemize}
\tightlist
\item
  \href{https://www.nytco.com/}{NYTCo}
\item
  \href{https://help.nytimes3xbfgragh.onion/hc/en-us/articles/115015385887-Contact-Us}{Contact
  Us}
\item
  \href{https://www.nytco.com/careers/}{Work with us}
\item
  \href{https://nytmediakit.com/}{Advertise}
\item
  \href{http://www.tbrandstudio.com/}{T Brand Studio}
\item
  \href{https://www.nytimes3xbfgragh.onion/privacy/cookie-policy\#how-do-i-manage-trackers}{Your
  Ad Choices}
\item
  \href{https://www.nytimes3xbfgragh.onion/privacy}{Privacy}
\item
  \href{https://help.nytimes3xbfgragh.onion/hc/en-us/articles/115014893428-Terms-of-service}{Terms
  of Service}
\item
  \href{https://help.nytimes3xbfgragh.onion/hc/en-us/articles/115014893968-Terms-of-sale}{Terms
  of Sale}
\item
  \href{https://spiderbites.nytimes3xbfgragh.onion}{Site Map}
\item
  \href{https://help.nytimes3xbfgragh.onion/hc/en-us}{Help}
\item
  \href{https://www.nytimes3xbfgragh.onion/subscription?campaignId=37WXW}{Subscriptions}
\end{itemize}
