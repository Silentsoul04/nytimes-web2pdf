Sections

SEARCH

\protect\hyperlink{site-content}{Skip to
content}\protect\hyperlink{site-index}{Skip to site index}

\href{https://www.nytimes3xbfgragh.onion/section/realestate}{Real
Estate}

\href{https://myaccount.nytimes3xbfgragh.onion/auth/login?response_type=cookie\&client_id=vi}{}

\href{https://www.nytimes3xbfgragh.onion/section/todayspaper}{Today's
Paper}

\href{/section/realestate}{Real Estate}\textbar{}Dream Homes for Golfers
and Wine Lovers

\url{https://nyti.ms/35pRgMY}

\begin{itemize}
\item
\item
\item
\item
\item
\end{itemize}

Advertisement

\protect\hyperlink{after-top}{Continue reading the main story}

Supported by

\protect\hyperlink{after-sponsor}{Continue reading the main story}

\hypertarget{dream-homes-for-golfers-and-wine-lovers}{%
\section{Dream Homes for Golfers and Wine
Lovers}\label{dream-homes-for-golfers-and-wine-lovers}}

Communities that are tailored to bons vivants looking for world-class
chipping and sipping in beautiful settings.

\includegraphics{https://static01.graylady3jvrrxbe.onion/images/2020/09/12/multimedia/12sp-golfhomes-wineries-inyt1/merlin_176288838_ef8b2e45-012b-486c-b658-645d9fb93806-articleLarge.jpg?quality=75\&auto=webp\&disable=upscale}

By Nora Walsh

\begin{itemize}
\item
  Sept. 11, 2020
\item
  \begin{itemize}
  \item
  \item
  \item
  \item
  \item
  \end{itemize}
\end{itemize}

If your idea of a dream home includes a combination of golf, nature,
quality architecture and fine wine, there are master-planned communities
around the world offering the best of all four.

``Some of the top golf clubs in the world also have some of the finest
wine cellars,'' said Greg Nathan, chief business officer of the
\href{https://www.ngf.org/}{National Golf Foundation}.

Perhaps the combination shouldn't be a surprise: The British Open's
coveted trophy is in the shape of a claret jug after all ---
traditionally used to serve Bordeaux wine in the 19th century. And top
golfers like Jack Nicklaus, Greg Norman, Ernie Els, David Frost and Luke
Donald all have their own wine labels, said Paul T. Stringer, president
of \href{https://www.nicklausdesign.com/}{Nicklaus Design}.

Here is a selection of lifestyle-driven communities tailored to bons
vivants looking for world-class chipping and sipping in breathtaking
settings.

\hypertarget{italy}{%
\subsection{Italy}\label{italy}}

Set in the Tuscan hills where the Medici family used to holiday,
\href{https://www.castelfalfi.com/}{Toscana Resort Castelfalfi} offers a
taste of la dolce vita. The fertile 2,700-acre retreat is a patchwork of
vineyards, olive groves, lakes and woodlands centered on a medieval
village of stone structures outfitted with modern amenities.

``Unlike other golf-driven communities, this doesn't have a suburban
feel,'' said Marco Boni, a homeowner who lives in Dubai, the United Arab
Emirates.

\includegraphics{https://static01.graylady3jvrrxbe.onion/images/2020/09/12/multimedia/12sp-golfhomes-wineries-inyt2/12sp-golfhomes-wineries-inyt2-articleLarge.jpg?quality=75\&auto=webp\&disable=upscale}

Six sites for villas overlooking the golf course start at 1.7 million
euros (about \$2 million) for a 2,580-square-foot property. New
constructions (a rarity in Tuscany) must adhere to strict green-building
codes. For those looking for a fixer-upper, 14 additional farmhouse
ruins, some dating to the 15th century, are scattered throughout the
property and are available for restoration. Refurbishment (conducted
through the resort) costs about €740 per square foot. (Homeowners'
resort fees start at €4,000 per year.)

Two hillside golf courses span about 32,000 square feet across varied
terrain. Designed by Wilfried Moroder and Rainer Preissmann, the 18-hole
Mountain Course tests veteran golfers while the nine-hole Lake Course
caters to beginners.

``If I didn't play golf, I'd walk the course every day because the views
are so visually arresting,'' said Tim Wade, a homeowner who lives in
London.

To enjoy the literal fruits of Tuscany, homeowners can adopt a row of
grape vines or an olive grove to receive personalized bottles of wine or
extra-virgin olive oil refined at the on-site organic winery. Homeowners
also have exclusive access to wine tours, tastings and discounts on the
resort's various wine labels.

\hypertarget{south-africa}{%
\subsection{South Africa}\label{south-africa}}

The 2,265-acre \href{https://valdevie.co.za/}{Val de Vie} estate is in
the heart of the vineyard-lined Paarl-Franschhoek Valley in South
Africa's Western Cape.

Image

The Val de Vie estate is a residential development in the
Paarl-Franschhoek Valley in the country's Western Cape wine region. The
estate has an 18-hole golf course designed by Jack Nicklaus.Credit...Val
de Vie

\href{https://www.pearlvalley.co.za/}{Pearl Valley}, the residential
development's 18-hole golf course, was designed by Jack Nicklaus and has
been consistently ranked a top course by Golf Digest magazine.

``It's such a memorable course because the layout is challenging and the
mountain backdrops are magnificent,'' said Hein Koegelenberg, a resident
of Val de Vie and owner of the on-site winery,
\href{https://valdevie.co.za/wine/cellar/}{L'Huguenot Cellar}, which
produces the estate's five signature wines. Residents can also create
their own wine blends at the winery; prices start at \$700 per barrel.

Image

The houses in Val de Vie feature French Provençal or Cape Vernacular
styles with prices ranging from \$250,000 for an entry-level
1,990-square-foot home to \$5 million for a 15,000-square-foot
house.Credit...Val de Vie

While a barrel of wine will go a long way toward pleasing adults, Val de
Vie also places a huge focus on family entertainment, said Mr.
Koegelenberg, citing a list of kid-friendly facilities, including junior
golf, tennis and equestrian academies, cricket, soccer, batting cages,
polo fields, a wildlife camp and 26 miles of trails.

About 80 of the 1,700 lots are currently available. Homes feature French
Provençal or Cape Vernacular styles with prices ranging from \$250,000
for an entry-level 1,990-square-foot house to \$5 million for a
15,000-square-foot home. Construction rates are about \$100 per square
foot, and monthly fees start at \$225.

\hypertarget{new-zealand}{%
\subsection{New Zealand}\label{new-zealand}}

Near Queenstown on New Zealand's South Island is Jack's Point, a
3,138-acre lakeside development at the base of the Remarkables mountain
range.

Image

Jack's Point is a 3,138-acre lakeside development at the base of the
Remarkables mountain range. The development is near the wine-growing
region of Gibbston.Credit...Touch of Spice

Thirteen hundred lots ranging from 3,230 square feet to more than 12
acres have prices from 350,000 to 3 million New Zealand dollars (about
\$231,000 to \$2 million), with construction costs starting at about 325
New Zealand dollars per square foot. (Homeowners' association fees start
at 3,500 New Zealand dollars per year.)

Thirty-six home sites with alpine and lake views are in The Preserve, a
neighborhood fringing the 18-hole golf course designed by the project's
developer, John Darby.

Image

Jack's Point has 1,300 lots ranging in price from about \$231,000 to \$2
million.Credit...Touch of Spice

``Our house is low-slung and built with local schist stone, dark-stained
timber and a flat roof covered with pebbles to blend into the natural
landscape,'' said Jude Roberts, a full-time resident whose sun-drenched
four-bedroom home overlooks the vista.

The developer said residents were attracted to Jack's Point for its four
distinct seasons and recreational facilities, which include tennis,
water sports, 15.5 miles of trails, local golf courses and ski resorts.

Oenophiles can taste the terroir at Gibbston, a nearby wine-growing
region famous for its pinot noir and home to dozens of wineries and wine
cellars.

\hypertarget{argentina}{%
\subsection{Argentina}\label{argentina}}

\href{https://www.algodonwineestates.com/}{Algodon Wine Estates} in San
Rafael, Mendoza, has luxury vineyard living on a 4,138-acre estate
planted with heritage vines, olive groves and fruit orchards at the foot
of the Sierra Pintada mountains.

Image

Algodon Wine Estates in San Rafael, Mendoza, is a 4,138-acre estate
planted with heritage vines, olive groves and fruit orchards at the foot
of the Sierra Pintada mountains.Credit...Fernando Arcuri

The region's sunny climate is ideal for producing the estate's signature
bonarda and malbec wines, as well as playing its nine-hole golf course
designed by Ricardo Jurado Jr., the grandson of the Argentine golf
legend José Jurado.

More than 100 vineyard lots, some showcasing Spanish Revival homes,
overlook the golf course. (Another nine holes with adjacent lots are in
the works.) Home sites range from 21,527 square feet (\$105,000) to
almost seven acres (\$785,000), with construction costs around \$80 per
square foot. Homeowners' association fees run about \$350 per month.

``I quickly realized I could get much more for my money in Mendoza
compared to Napa, Tuscany or Provence,'' said John Raffaeli, a homeowner
and wine entrepreneur.

Image

Algodon Wine Estates is in a region of Argentina with a sunny climate
that is ideal for producing the estate's signature bonarda and malbec
wines.

At the on-site winery, homeowners can create their own private-label in
collaboration with the winemaker Mauro Nosenzo. Rates start at \$2,900
and include the cost of labor, a new French-oak barrel and about 290
bottles of wine.

Further north in the province of Salta is
\href{http://www.lec.com.ar/}{La Estancia de Cafayate}, a 1,360-acre
residential vineyard estate in the Calchaqui Valley, a premier wine
region known for its Torrontes.

The 400-lot development, featuring an 18-hole Bob Cupp-designed golf
course, will debut 17 vineyard homes with rustic touches like
terra-cotta roofs, wrought iron and verandas with typical Argentine
grills. Prices start at \$290,000 plus homeowner association fees.

``Cafayate is a quaint boutique-winery town --- think Napa 50 years
ago,'' said David Galland, a homeowner and minority partner in La
Estancia de Cafayate.

\hypertarget{canada}{%
\subsection{Canada}\label{canada}}

\href{https://www.predatorridge.com/}{Predator Ridge}, a 1,200-acre
residential community featuring over 700 homes and fitness and wellness
amenities in the lush Okanagan countryside of British Columbia, has
attracted Canadians from across the country --- in part because of its
amenities, Rob Davidson, Predator Ridge's vice president of product and
planning, said.

Image

Residents of the Predator Ridge community in British Columbia don't have
to drive more than 15 minutes to reach the wineries surrounding the
Okanagan Valley wine region.Credit...Predator Ridge

``People buy our community before they buy a home,'' Mr. Davidson said.
``We have over a thousand community events every year that residents can
participate in, from fitness classes to wine-pairing dinners, cooking
classes and trail walks.''

Predator Ridge has 36 holes of championship golf, including the par-72
Ridge course designed by Doug Carrick that stretches 7,000 yards across
rolling hills. The similarly sized Predator course features a par-71 Les
Furber layout.

Image

Predator Ridge has community events that include fitness classes,
wine-pairing dinners, cooking classes and trail walks.Credit...Predator
Ridge

Homes in the Commonage neighborhood overlooking the Predator course have
modern-ranch architecture, outdoor living areas and low-impact
landscaping. Lots start at 270,000 Canadian dollars (about \$202,550)
for 7,405 square feet, with construction costs around 265 Canadian
dollars per square foot. Homeowners' association fees are about 200
Canadian dollars per month.

Wine lovers don't have to drive more than 15 minutes along scenic back
roads to sip varietals like pinot gris and pinot noir at a handful of
more than 180 wineries peppering the surrounding Okanagan Valley wine
region.

Advertisement

\protect\hyperlink{after-bottom}{Continue reading the main story}

\hypertarget{site-index}{%
\subsection{Site Index}\label{site-index}}

\hypertarget{site-information-navigation}{%
\subsection{Site Information
Navigation}\label{site-information-navigation}}

\begin{itemize}
\tightlist
\item
  \href{https://help.nytimes3xbfgragh.onion/hc/en-us/articles/115014792127-Copyright-notice}{©~2020~The
  New York Times Company}
\end{itemize}

\begin{itemize}
\tightlist
\item
  \href{https://www.nytco.com/}{NYTCo}
\item
  \href{https://help.nytimes3xbfgragh.onion/hc/en-us/articles/115015385887-Contact-Us}{Contact
  Us}
\item
  \href{https://www.nytco.com/careers/}{Work with us}
\item
  \href{https://nytmediakit.com/}{Advertise}
\item
  \href{http://www.tbrandstudio.com/}{T Brand Studio}
\item
  \href{https://www.nytimes3xbfgragh.onion/privacy/cookie-policy\#how-do-i-manage-trackers}{Your
  Ad Choices}
\item
  \href{https://www.nytimes3xbfgragh.onion/privacy}{Privacy}
\item
  \href{https://help.nytimes3xbfgragh.onion/hc/en-us/articles/115014893428-Terms-of-service}{Terms
  of Service}
\item
  \href{https://help.nytimes3xbfgragh.onion/hc/en-us/articles/115014893968-Terms-of-sale}{Terms
  of Sale}
\item
  \href{https://spiderbites.nytimes3xbfgragh.onion}{Site Map}
\item
  \href{https://help.nytimes3xbfgragh.onion/hc/en-us}{Help}
\item
  \href{https://www.nytimes3xbfgragh.onion/subscription?campaignId=37WXW}{Subscriptions}
\end{itemize}
