Sections

SEARCH

\protect\hyperlink{site-content}{Skip to
content}\protect\hyperlink{site-index}{Skip to site index}

\href{https://www.nytimes3xbfgragh.onion/section/science}{Science}

\href{https://myaccount.nytimes3xbfgragh.onion/auth/login?response_type=cookie\&client_id=vi}{}

\href{https://www.nytimes3xbfgragh.onion/section/todayspaper}{Today's
Paper}

\href{/section/science}{Science}\textbar{}Cancer Projects to Diversify
Genetic Research Receive New Grants

\url{https://nyti.ms/3hnXRd0}

\begin{itemize}
\item
\item
\item
\item
\item
\end{itemize}

Advertisement

\protect\hyperlink{after-top}{Continue reading the main story}

Supported by

\protect\hyperlink{after-sponsor}{Continue reading the main story}

\hypertarget{cancer-projects-to-diversify-genetic-research-receive-new-grants}{%
\section{Cancer Projects to Diversify Genetic Research Receive New
Grants}\label{cancer-projects-to-diversify-genetic-research-receive-new-grants}}

Because much cancer research and clinical trials have been based on
white populations, efforts to explore the ways race and ethnicity
influence disease are underway.

\includegraphics{https://static01.graylady3jvrrxbe.onion/images/2020/09/11/science/11GENOME1/11GENOME1-articleLarge-v2.jpg?quality=75\&auto=webp\&disable=upscale}

By \href{https://www.nytimes3xbfgragh.onion/by/emma-goldberg}{Emma
Goldberg}

\begin{itemize}
\item
  Sept. 11, 2020
\item
  \begin{itemize}
  \item
  \item
  \item
  \item
  \item
  \end{itemize}
\end{itemize}

The New York Genome Center awarded six cancer research grants this week
as part of an initiative examining the role of race and ethnicity in
major types of cancer.

The projects will investigate a variety of cancers including pancreatic,
colorectal and endometrial cancer in African-Americans; lung cancer in
Asian-American patients; breast and prostate cancer in patients of
African ancestry; and the role of ethnicity in bladder cancer.

The Genome Center's two-year-old initiative, called
\href{https://www.nygenome.org/research-areas/cancer/polyethnic-1000/}{Polyethnic-1000},
is aimed at closing the knowledge gap that exists largely because
decades of genetic studies focused mainly on white patient populations.
Dr. Harold Varmus, a professor of medicine at Weill Cornell overseeing
the initiative, said he hoped the projects would advance the
understanding of racial disparities in the prevalence of different
cancer types, as well as patient responses to different cancer
therapies.

``The disparities are there but the explanations are not,'' said Dr.
Varmus, who previously served as head of the National Institutes of
Health and of Memorial Sloan Kettering Cancer Center. Expanding genetic
research to become more representative of the broader American
population will also further researchers' understanding of cancer.
``Leaving people out is an equity issue and a knowledge issue,'' he
added.

Black Americans have the
\href{https://pubmed.ncbi.nlm.nih.gov/30762872/}{highest death rate} of
any racial group for most cancers; some 73,000 African-Americans die of
cancer each year. But cancer research has focused disproportionately on
white patients because it tends to be conducted at research centers with
rich resources that have more affluent and white patient populations.
That sometimes leaves researchers unsure of their work's relevance to
Black, Latino and Asian patients, and unable to fully anticipate how
these populations will respond to drugs and therapies.

``When we generate results, we don't know if they apply to
underrepresented minority communities,'' said Dr. Deborah Schrag, an
oncologist at the Dana-Farber Cancer Institute who reviewed grant
proposals for the initiative. ``If we're not profiling people of all
races and ethnicities, we're missing opportunities to treat people
strategically.''

The coronavirus pandemic has thrown a glaring light on the
disproportionately devastating effects the disease has had on nonwhite
people, especially in the United States. Death and hospitalization rates
are higher for Black, Latino and Indigenous people.

Many social and socioeconomic factors affect racial disparities in
cancer. Black men are more likely than white men to forgo colonoscopy
screenings, and the rate of new colon cancer cases is about
\href{https://www.nytimes3xbfgragh.onion/2020/08/29/health/colon-cancer-chadwick-boseman.html}{20
percent higher} in African-Americans than non-Hispanic white people.
Black women are less likely to undergo preventive screenings for breast
cancer, and they are also more likely to die of the disease.

But some forms of cancer affect racial groups differently regardless of
socioeconomic status, leading researchers to consider that genetics
could also play a critical role. This week, New England Journal of
Medicine published a
\href{https://www.nejm.org/doi/full/10.1056/NEJMc2000069}{letter}
emphasizing the importance of taking race into account in genomic
studies of cancer.

``We think there's more to it than just social factors,'' said Dr. Laura
Martello-Rooney, one of the grant recipients, who studies pancreatic and
colon cancer in African-Americans. ``We think there are underlying
molecular and cellular differences that impact the incidence as well as
its treatment.''

Dr. Bishoy Faltas, an oncologist at Weill Cornell who is leading the
study on bladder cancer, said conducting genetic research on patients of
different ethnic backgrounds was important because the way that
patients' immune systems respond to cancer and cancer therapies is
determined by their genetic makeup.

The recipients of the new grants include the team at Weill Cornell, as
well as at Cold Spring Harbor Laboratory, Northwell Health, SUNY
Downstate Medical Center, SUNY Downstate Health Sciences University,
Kings County Medical Center, Mount Sinai Hospital and
NewYork-Presbyterian. While the scope of the six projects is small,
leaders of the Polyethnic initiative hope to see it replicated in other
U.S. cities to continue broadening the diversity of patients represented
in genetic databases.

One difficulty in recruiting African-American patients to clinical
research has been the Black
\href{https://www.nytimes3xbfgragh.onion/2020/01/13/upshot/bad-medicine-the-harm-that-comes-from-racism.html}{community's
mistrust} in the medical system. Physicians involved with the Genome
Center initiative said they were trying to build trust by partnering
with oncologists in community hospitals in diverse neighborhoods across
New York City.

``The way to build trust isn't to have a bunch of people from Sloan
Kettering go into these areas and say `Hey, sign this consent form and
give me your tumor,''' said Dr. Charles Sawyers, a physician at Memorial
Sloan Kettering Cancer Center and one of the initiative's leaders.
``It's to work with local doctors and oncologists.''

\textbf{\emph{{[}}\href{http://on.fb.me/1paTQ1h}{\emph{Like the Science
Times page on Facebook.}}} ****** \emph{\textbar{} Sign up for the}
\textbf{\href{http://nyti.ms/1MbHaRU}{\emph{Science Times
newsletter.}}\emph{{]}}}

Advertisement

\protect\hyperlink{after-bottom}{Continue reading the main story}

\hypertarget{site-index}{%
\subsection{Site Index}\label{site-index}}

\hypertarget{site-information-navigation}{%
\subsection{Site Information
Navigation}\label{site-information-navigation}}

\begin{itemize}
\tightlist
\item
  \href{https://help.nytimes3xbfgragh.onion/hc/en-us/articles/115014792127-Copyright-notice}{©~2020~The
  New York Times Company}
\end{itemize}

\begin{itemize}
\tightlist
\item
  \href{https://www.nytco.com/}{NYTCo}
\item
  \href{https://help.nytimes3xbfgragh.onion/hc/en-us/articles/115015385887-Contact-Us}{Contact
  Us}
\item
  \href{https://www.nytco.com/careers/}{Work with us}
\item
  \href{https://nytmediakit.com/}{Advertise}
\item
  \href{http://www.tbrandstudio.com/}{T Brand Studio}
\item
  \href{https://www.nytimes3xbfgragh.onion/privacy/cookie-policy\#how-do-i-manage-trackers}{Your
  Ad Choices}
\item
  \href{https://www.nytimes3xbfgragh.onion/privacy}{Privacy}
\item
  \href{https://help.nytimes3xbfgragh.onion/hc/en-us/articles/115014893428-Terms-of-service}{Terms
  of Service}
\item
  \href{https://help.nytimes3xbfgragh.onion/hc/en-us/articles/115014893968-Terms-of-sale}{Terms
  of Sale}
\item
  \href{https://spiderbites.nytimes3xbfgragh.onion}{Site Map}
\item
  \href{https://help.nytimes3xbfgragh.onion/hc/en-us}{Help}
\item
  \href{https://www.nytimes3xbfgragh.onion/subscription?campaignId=37WXW}{Subscriptions}
\end{itemize}
