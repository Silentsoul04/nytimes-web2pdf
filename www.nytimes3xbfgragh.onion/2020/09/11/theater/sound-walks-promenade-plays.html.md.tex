\href{/section/theater}{Theater}\textbar{}For These Shows, Take a Hike

\url{https://nyti.ms/32l2sIG}

\begin{itemize}
\item
\item
\item
\item
\item
\end{itemize}

\includegraphics{https://static01.graylady3jvrrxbe.onion/images/2020/09/12/arts/10sound-theater-1/merlin_176827971_9c49be4e-d432-49df-b11a-452d438f6d0b-articleLarge.jpg?quality=75\&auto=webp\&disable=upscale}

Sections

\protect\hyperlink{site-content}{Skip to
content}\protect\hyperlink{site-index}{Skip to site index}

critic's notebook

\hypertarget{for-these-shows-take-a-hike}{%
\section{For These Shows, Take a
Hike}\label{for-these-shows-take-a-hike}}

If you participate in a sound walk and no one is there to applaud, does
it count as theater? Our critic argues that it does. Or at least that it
can.

A listener takes in ``Cairns,'' a sound walk created for Green-Wood
Cemetery, in Brooklyn.Credit...Sasha Arutyunova for The New York Times

Supported by

\protect\hyperlink{after-sponsor}{Continue reading the main story}

By \href{https://www.nytimes3xbfgragh.onion/by/alexis-soloski}{Alexis
Soloski}

\begin{itemize}
\item
  Sept. 11, 2020
\item
  \begin{itemize}
  \item
  \item
  \item
  \item
  \item
  \end{itemize}
\end{itemize}

There's a moment late in
\href{https://here.org/shows/cairns/}{``Cairns,''} a lovely, peaceable
sound walk created by the singer and scholar
\href{https://gelseybell.com/}{Gelsey Bell} and presented by Here, in
which Bell will ask you to do something drastic: Take out your earbuds.
Maybe that doesn't seem so extreme, but when was the last time you put
away your phone, shut your eyes, stilled the mental whirl of worries,
statistics and undone errands, and just listened?

People who have tired of Zoom plays (don't raise your hands all at once,
please!), will welcome the opportunity to listen --- outdoors and
screen-free. After all, if a sound walk doesn't get you into the
theater, at least it gets you out of the house.

\href{https://www.nytimes3xbfgragh.onion/2020/08/12/theater/promenade-theater-walking-pandemic.html}{Promenade
plays}, in which audience members walk from physically distanced scene
to physically distanced scene, have become a mainstay of pandemic
theater. In ``Cairns'' and
\href{https://www.inversiontheatre.com/park}{``Intralia, the Weird
Park,''} another recent audio play, you still walk --- for miles --- but
the scenes are staged in your mind's eye and mind's ear only. These are
participatory shows, but in a solitary and covert way that seems like
some kind of theatrical koan. If you participate and no one is there to
applaud, does it even count? I'd argue that it does. Or at least that it
can.

\includegraphics{https://static01.graylady3jvrrxbe.onion/images/2020/09/10/arts/10sound-theater-4/merlin_176827992_0e7cd26d-c38c-4f23-a778-efa48853c602-articleLarge.jpg?quality=75\&auto=webp\&disable=upscale}

Since both sound walks take place in Brooklyn --- ``Cairns'' in
\href{https://www.green-wood.com/}{Green-Wood Cemetery} and ``Intralia''
in \href{https://www.prospectpark.org/}{Prospect Park}, neither too far
from my apartment --- I hiked them one after the other on a sunny
Thursday. Actually, I began the night before, buying Bell's album on
\href{https://bandcamp.com/}{Bandcamp} and downloading the tracks, plus
a map, onto my faltering Samsung Galaxy. The next morning, before the
heat kicked in, I slid on some sneakers, reached for a mask, tramped the
two and a half miles to Green-Wood's Sunset Park entrance and clicked
play.

I'd seen Bell onstage, severe and sylphlike in the musicals
\href{https://www.nytimes3xbfgragh.onion/2016/11/15/theater/natasha-pierre-and-the-great-comet-of-1812-review.html}{``Natasha,
Pierre and the Great Comet of 1812''} and
\href{https://www.nytimes3xbfgragh.onion/2014/10/11/theater/dave-malloys-ghost-quartet-at-the-bushwick-starr.html}{``Ghost
Quartet,''} and knew enough of her experimental music to feel curious.
Still, with the anxieties of the last several months, a morning spent
contemplating mortality, however vanguard the accompaniment, didn't
hugely appeal. Because at the end of the day, Green-Wood, a nature
preserve and sculpture garden --- and in the 1850s, an insanely popular
tourist attraction --- is still a cemetery.

Image

Gelsey Bell, the show's creator and narrator, wonders what these trees
might think, ``watching us short-lived meat bags the way we watch
humming birds.''Credit...Sasha Arutyunova for The New York Times

Image

Credit...Sasha Arutyunova for The New York Times

I shouldn't have worried. Bell couches her work in deeply humane terms,
even as she looks beyond the human and toward the natural world. Even
the grimmer observations are somehow delightful. Passing beneath some
purple-leafed beeches, she wonders what these trees might think,
``watching us short-lived meat bags the way we watch humming birds.''

During the walk, which lasts a little over an hour, Bell stays virtually
by your side. An informed, supportive friend, she casts you as her
companion, lending an aural hand to pull you onto each new gravel path.
She gives precise and particular directions --- take a soft left, make a
sharp right --- and even someone like me, with the directional acumen of
a demagnetized compass, never felt lost.

Generously, Bell wants you to notice what she has noticed, and in that
spirit, she takes you past a few graves like that of Do-Hum-Me, an
Indigenous woman exhibited by the showman P.T. Barnum; or Eunice Newton
Foote, a 19th-century climate scientist; or Susan McKinney Smith
Steward, the first Black woman to become a doctor in New York state.
Bell also directs your eyes toward Lady Liberty, far away in the harbor,
and to the bottom of a headstone that reads, ``Have an egg cream.''

Image

The ``Cairns'' tour guides you past the grave of Do-Hum-Me, an
Indigenous woman who was exhibited by the showman P.T.
Barnum.Credit...Sasha Arutyunova for The New York Times

But she also leaves space for your explorations, encouraging you to
pause the audio whenever needed. Often, in music composed with Joseph
White, she lets her own talk give way to whispering, humming, chanting
as her voice loops atop itself. Moving from tree to tree and plot to
plot, she encourages you to make your own sense, your own story, your
own theater. And even though I'm a meditation dud, the five minutes she
asked me to spend just listening --- to birds, leaves, an airplane,
insects that chittered like a bicycle with a playing card in its spokes
--- left me feeling quieter.

Another two miles took me to the Ocean Avenue entrance to Prospect Park
and the opening track for ``Intralia,'' from
\href{https://www.inversiontheatre.com/}{InVersion Theater}, with music
and sound design by Jordan Hall. At least, I think that was where I was
meant to begin. ``Intralia'' exists as an app for iPhone users, but for
the rest of us, it's a SoundCloud link and a sketchy map. The catch:
There are nine tracks and only six locations marked on the map.

The piece begins with an instrumental --- ominous strings --- then
offers some language apparently borrowed from E.P.A. Superfund site
reports regarding the nearby Gowanus Canal, describing dangerous
contaminants. What this has to do with Prospect Park, fed by the city's
aqueducts and not the canal, is anyone's guess. (Though the
cyanobacteria that has turned some of the park's lakes and ponds a
retina-jolting green seems a likely source of inspiration.)

Image

Bell encourages listeners to pause ``Cairns'' whenever needed and
explore.Credit...Sasha Arutyunova for The New York Times

Image

Credit...Sasha Arutyunova for The New York Times

A press released had described ``Intralia'' as a story of two municipal
workers, Eve and Ash, who confront strange doings. Except for a
throwaway line about the ``good luck to be assigned here, Prospect
Park,'' I would never have known it as the piece never bothers to
establish character or place. Other tracks reveal the discovery of
entrails and hanging goat heads. Which failed to feel creepy. Because
while ``Cairns'' entrenches itself in its surroundings, ``Intralia''
disregards them.

Those strings and scares elide what is authentically strange and really
beautiful about this particular urban park. Walking on what was maybe
the path, I saw anglers, bicyclists, CrossFit enthusiasts, children on
scooters, teenagers smoking weed, a man air-drying his laundry on a
piece of park equipment, and a lady acrobat balancing atop her partner's
head. I saw a memorial for the
\href{https://www.nycgovparks.org/parks/B073/highlights/19641}{Battle of
Brooklyn} and an
\href{https://www.prospectpark.org/news-events/news/alliance-transforms-historic-wellhouse/}{extremely
pretty composting toilet}. ``Intralia'' ignores them all.

The best examples of environmental theater (I'm thinking of pieces like
\href{https://www.nytimes3xbfgragh.onion/2003/07/03/theater/theater-that-uses-the-city-as-a-stage.html}{``The
Angel Project''} and
\href{https://www.nytimes3xbfgragh.onion/2014/10/31/theater/the-dreary-coast-on-the-banks-of-the-gowanus-canal.html}{``The
Dreary Coast''}) take a familiar place and make you see it through new
eyes. But ``Intralia'' didn't seem to see the park at all, with no
consideration given to who you are and how and why you might be
listening.

Image

Bell gives precise and particular directions during ``Cairns,''~which
lasts a little over an hour.Credit...Sasha Arutyunova for The New York
Times

I had hiked to the top of Lookout Hill --- for a final track that never
mentioned it --- and when ``Intralia'' ended, I hiked down then walked
the mile or so back home, alone again with my own internal soundtrack.
(My Samsung battery held out until I reached the road that rings the
park, then died.) I thought about how generous traditional theater is
and how the actors, designers and directors conspire to deliver a total
work of art. Sound walks don't do that, but even within these
constraints, a good one, like ``Cairns,'' can conjure a world and a
worldview, too.

Maybe this form seems stingy --- no costumes, no lights, no tap numbers,
just a few words murmured in your ear --- but advance the track and
think of it as generous instead, a reminder not only of how much theater
can give us, but how much it trusts us to imagine, too.

Advertisement

\protect\hyperlink{after-bottom}{Continue reading the main story}

\hypertarget{site-index}{%
\subsection{Site Index}\label{site-index}}

\hypertarget{site-information-navigation}{%
\subsection{Site Information
Navigation}\label{site-information-navigation}}

\begin{itemize}
\tightlist
\item
  \href{https://help.nytimes3xbfgragh.onion/hc/en-us/articles/115014792127-Copyright-notice}{©~2020~The
  New York Times Company}
\end{itemize}

\begin{itemize}
\tightlist
\item
  \href{https://www.nytco.com/}{NYTCo}
\item
  \href{https://help.nytimes3xbfgragh.onion/hc/en-us/articles/115015385887-Contact-Us}{Contact
  Us}
\item
  \href{https://www.nytco.com/careers/}{Work with us}
\item
  \href{https://nytmediakit.com/}{Advertise}
\item
  \href{http://www.tbrandstudio.com/}{T Brand Studio}
\item
  \href{https://www.nytimes3xbfgragh.onion/privacy/cookie-policy\#how-do-i-manage-trackers}{Your
  Ad Choices}
\item
  \href{https://www.nytimes3xbfgragh.onion/privacy}{Privacy}
\item
  \href{https://help.nytimes3xbfgragh.onion/hc/en-us/articles/115014893428-Terms-of-service}{Terms
  of Service}
\item
  \href{https://help.nytimes3xbfgragh.onion/hc/en-us/articles/115014893968-Terms-of-sale}{Terms
  of Sale}
\item
  \href{https://spiderbites.nytimes3xbfgragh.onion}{Site Map}
\item
  \href{https://help.nytimes3xbfgragh.onion/hc/en-us}{Help}
\item
  \href{https://www.nytimes3xbfgragh.onion/subscription?campaignId=37WXW}{Subscriptions}
\end{itemize}
