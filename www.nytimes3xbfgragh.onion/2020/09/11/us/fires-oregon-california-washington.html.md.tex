Sections

SEARCH

\protect\hyperlink{site-content}{Skip to
content}\protect\hyperlink{site-index}{Skip to site index}

\href{/section/us}{U.S.}\textbar{}A Line of Fire South of Portland and a
Yearslong Recovery Ahead

\url{https://nyti.ms/3mest4x}

\begin{itemize}
\item
\item
\item
\item
\item
\end{itemize}

\hypertarget{wildfires-in-the-west}{%
\subsubsection{\texorpdfstring{\href{https://www.nytimes3xbfgragh.onion/spotlight/california-wildfires?name=styln-california-wildfires\&region=TOP_BANNER\&block=storyline_menu_recirc\&action=click\&pgtype=Article\&impression_id=964aa140-f4b6-11ea-bc0c-2dd0f27298e1\&variant=undefined}{Wildfires
in the West}}{Wildfires in the West}}\label{wildfires-in-the-west}}

\begin{itemize}
\tightlist
\item
  live\href{https://www.nytimes3xbfgragh.onion/2020/09/11/us/wildfires-live-updates.html?name=styln-california-wildfires\&region=TOP_BANNER\&block=storyline_menu_recirc\&action=click\&pgtype=Article\&impression_id=964aa141-f4b6-11ea-bc0c-2dd0f27298e1\&variant=undefined}{Fires
  Updates}
\item
  \href{https://www.nytimes3xbfgragh.onion/interactive/2020/us/fires-map-tracker.html?name=styln-california-wildfires\&region=TOP_BANNER\&block=storyline_menu_recirc\&action=click\&pgtype=Article\&impression_id=964aa142-f4b6-11ea-bc0c-2dd0f27298e1\&variant=undefined}{Maps
  of the Fires}
\item
  \href{https://www.nytimes3xbfgragh.onion/2020/09/10/us/climate-change-california-wildfires.html?name=styln-california-wildfires\&region=TOP_BANNER\&block=storyline_menu_recirc\&action=click\&pgtype=Article\&impression_id=964ac850-f4b6-11ea-bc0c-2dd0f27298e1\&variant=undefined}{A
  Climate Reckoning}
\item
  \href{https://www.nytimes3xbfgragh.onion/article/wildfires-california-oregon-washington.html?name=styln-california-wildfires\&region=TOP_BANNER\&block=storyline_menu_recirc\&action=click\&pgtype=Article\&impression_id=964ac851-f4b6-11ea-bc0c-2dd0f27298e1\&variant=undefined}{Answers
  to Your Questions}
\item
  \href{https://www.nytimes3xbfgragh.onion/article/wildfires-photos-california-oregon-washington-state.html?name=styln-california-wildfires\&region=TOP_BANNER\&block=storyline_menu_recirc\&action=click\&pgtype=Article\&impression_id=964ac852-f4b6-11ea-bc0c-2dd0f27298e1\&variant=undefined}{Photos}
\item
  \href{https://www.nytimes3xbfgragh.onion/2020/09/09/us/california-wildfires.html?name=styln-california-wildfires\&region=TOP_BANNER\&block=storyline_menu_recirc\&action=click\&pgtype=Article\&impression_id=964ac853-f4b6-11ea-bc0c-2dd0f27298e1\&variant=undefined}{Newsletter}
\end{itemize}

\includegraphics{https://static01.graylady3jvrrxbe.onion/images/2020/09/11/us/11wildfires-oregon01/merlin_176874867_619c9016-5c29-4d85-9f2a-2222ac19fc24-articleLarge.jpg?quality=75\&auto=webp\&disable=upscale}

\hypertarget{a-line-of-fire-south-of-portland-and-a-yearslong-recovery-ahead}{%
\section{A Line of Fire South of Portland and a Yearslong Recovery
Ahead}\label{a-line-of-fire-south-of-portland-and-a-yearslong-recovery-ahead}}

Firefighters continued to battle blazes along the West Coast that have
now charred nearly five million acres. At least 17 people are dead, with
dozens still missing.

The parking lot of a wildfire evacuation center in Happy Valley, Ore.,
was filled with R.V.s on Friday.Credit...Kristina Barker for The New
York Times

Supported by

\protect\hyperlink{after-sponsor}{Continue reading the main story}

By \href{https://www.nytimes3xbfgragh.onion/by/jack-healy}{Jack Healy},
\href{https://www.nytimes3xbfgragh.onion/by/jack-nicas}{Jack Nicas} and
\href{https://www.nytimes3xbfgragh.onion/by/mike-baker}{Mike Baker}

\begin{itemize}
\item
  Published Sept. 11, 2020Updated Sept. 12, 2020, 12:33 a.m. ET
\item
  \begin{itemize}
  \item
  \item
  \item
  \item
  \item
  \end{itemize}
\end{itemize}

SALEM, Ore. --- A 36-mile-wide line of flames edged into the towns
around Portland, Ore., and cities along the West Coast were smothered in
acrid smoke and ash on Friday as history-making wildfires remained
unchecked, killing at least 17 and leaving dozens of people missing.

``We are preparing for a mass fatality incident based on what we know
and the numbers of structures that have been lost,'' Andrew Phelps,
director of the Oregon Office of Emergency Management, said as
firefighters struggled to contain blazes that have spread across
millions of acres of the Pacific Northwest.

Fires in California, Oregon and Washington have torn through idyllic
mountain towns, reduced neighborhoods to ash and spewed so much smoke
that pilots were unable to pursue aerial attacks that can be critical in
preventing such mass wildfires from encroaching on communities.
Portland's mayor, fearing the possibility that fires could start and
spread in the city, has declared a state of emergency.

Combined, the states have seen nearly five million acres consumed by
fire --- a land mass approaching the size of New Jersey --- in a
record-setting made worse, scientists say, by the climate change caused
by the burning of fossil fuels like coal and oil. Such disasters will
only become worse as the planet continues to warm.

The flames also left a humanitarian disaster in their wake, including
three more deaths in Oregon that were confirmed on Friday. Hundreds, if
not thousands, of homes have been lost, most of them in Oregon, where an
estimated 40,000 people have been evacuated and as many as 500,000 live
in evacuation alert zones, poised to flee with a change in the winds.

Tens of thousands of people have sought refuge in shelters, with friends
and in parking lots up and down Interstate 5 --- with emergency
responders struggling to create safe shelter for all of them in the
middle of the coronavirus pandemic.

On the outskirts of Portland, a site set up to shelter evacuees had to
be evacuated itself as the fire line continued expanding toward suburban
towns south of the city.

State fire officials said winds had pressed a 36-mile-wide wildfire
front toward those outlying Portland suburbs on Thursday, with fire
jumping over the community of Estacada and threatening others around the
foothills of the Cascade Mountains.

But Friday brought measures of relief with winds calming or shifting,
weather cooling and potential rain forecast for the days to come. That
appeared to ease the threat that fires could move through the Portland
suburbs or into the city.

\includegraphics{https://static01.graylady3jvrrxbe.onion/images/2020/09/11/us/11wildfires-oregon02/merlin_176874765_39c00e2b-bf6e-411c-bb64-448b4607d9bf-articleLarge.jpg?quality=75\&auto=webp\&disable=upscale}

``We were very fortunate that the winds did not sustain another day like
we had experienced'' in the previous four days, said Doug Grafe, the
chief of fire protection at the Oregon Department of Forestry.

As residents flee fire-ravaged communities, officials have struggled to
manage a series of migrations reminiscent of a war zone, with distraught
families showing up with little in hand beyond an overwhelming fear that
their homes have been lost for good. Emergency responders have only
begun to get a sense of how many victims they have and the grueling
effort to rebuild that will lie ahead.

``The long-term recovery is going to last years,'' said Mr. Phelps, the
emergency management director.

Parking lots up and down the state were transformed into improvised
campgrounds. A dozen campers and motor homes posted up outside a
supermarket in Milwaukie, down the hill from the fires ravaging much of
Clackamas County. Adriana Amaro said 25 members of her family had
decided to leave before they were ordered out, concerned about how a
frantic flight in the smoke would affect the children. They have been
sleeping on air mattresses inside the back of a relative's cargo truck
and going to the bathroom at a gas station on the corner.

Hotel and motel rooms have become so scarce that officials who had been
running a shelter at the Oregon State Fairgrounds in Salem said they had
to allow people to sleep on spaced-out cots indoors --- a step they had
been avoiding because of coronavirus fears.

``At this point it's all we can do,'' said Bethany Jones, a volunteer.

On Thursday night, about 2,300 people slept in emergency accommodations
provided by the American Red Cross and its partners --- 520 of them in
traditional mass shelters. Tens of thousands more were crashing with
friends or family members. Others were pitching tents in high school
football fields or sleeping in shopping mall parking lots --- many of
them unsure whether they would be displaced for days, or weeks, or more.

For Carla Heath and Cindy Essman, two sisters who fled their home in
Lyons, Ore., with their dogs and birds, Friday morning was Day 5 of
sleeping in their gray Buick hatchback. They spent two nights in a
shopping center parking lot before arriving at the shelter at the
fairgrounds.

``You do what you have to do at this point,'' Ms. Heath, 64, said. ``The
coronavirus is a concern, but that's so far back right now. I'm not even
thinking about it.''

Trish Reagan and her husband, Lee, have been staying at a Days Inn since
they had to evacuate their manufactured home in Mill City. Pictures of
their neighborhood show total devastation, and they think their home
burned to the ground. They have been making daily trips to emergency-aid
centers to bring food and clothing back to other evacuees at the motel
who are stuck without cars, basically sheltering in place.

Image

Volunteers distributed food to evacuees of the Bear Fire in Oroville,
Calif., on Friday.Credit...Christian Monterrosa for The New York Times

``I know God's going to provide,'' Ms. Reagan, 49, said. ``I don't know
how yet.''

On Friday morning, as carloads of evacuees from the fire pulled into the
fairgrounds looking for a place to sleep, a table of volunteers called
hotels. The nearby town of Woodburn was full. There was little left in
Salem. They were putting people up in Corvallis, about 35 miles away.

``We have to search far and wide,'' John Allen, a volunteer, said. ``You
can't go too far because you're going to find another disaster.''

Neighbors and friends raced through various communities helping families
evacuate and find temporary shelter.

Just south of Portland, Shashana Packus said she and her father spent
Thursday helping some friends closer to the fires get out of the
evacuation zone with their belongings and animals.

By Thursday night, five more people were spending the night at their
house in Oregon City, along with 15 chickens, three goats and a dog.
Their home was also at risk of falling under evacuation orders, so they
began gathering old home videos and taking seats out of the minivan so
the animals could fit, waiting for the sound of an emergency alert.

``Your phone goes off every few minutes, and your heart kind of sinks,''
Ms. Packus said.

In southern Oregon, the Jackson County Expo had been closed since March
as a result of the pandemic, unable to offer its usual rodeos, concerts
and festivals.

But on Friday it was teeming with local officials, Red Cross volunteers
and hundreds of fire refugees sleeping on cots surrounded by curtains,
or in tents or R.V.s outside. The center is the main evacuation shelter
in Jackson County, where an estimated 500 homes were destroyed by the
Almeda Fire on Tuesday, displacing thousands of people.

The initial plan had been to issue vouchers for hotels so people could
stay safely isolated, but the fires advanced so quickly that county
officials had to ditch those plans and set up the emergency shelter,
said Helen Funk, the Expo's director.

Image

Wildfire smoke blanketed downtown Portland, Ore., on
Friday.Credit...Amanda Lucier for The New York Times

Fortunately, one building had already been set up as a quarantine center
in case of a coronavirus outbreak, with the floor space divided into
10-square-foot sections, each with a cot and curtains. Officials
immediately began directing people there, and then began creating
identical setups in other buildings. If it were not for the pandemic,
they would have packed people closer together, Ms. Funk said.

``We have to make sure that we keep them socially distanced,'' she said.
On Friday, signs at each door said masks were required to enter, and
hand sanitizer was at each table in the cafeteria, where families ate
Subway sandwiches and pork tamales.

Miny Williams, 90, and Lois Gould, 83, were cleaning their hands with
alcohol wipes in a quiet corner of one bunkhouse. They had been there
for three days.

Ms. Williams, balancing on a walker and wearing a name tag and a mask,
said a sheriff's deputy had helped her escape from her mobile home on
Tuesday as the fire rushed toward her small town of Phoenix, much of
which was destroyed.

``After being evacuated, I'm very pleased to be here actually,'' she
said.

On Friday, authorities charged a 41-year-old man with starting part of
that inferno, the Almeda Fire, one of this year's most destructive
blazes. It burned through much of two towns along Interstate 5 in
southern Oregon but is now 50 percent contained.

The Jackson County Sheriff's Office said the fire started on Tuesday in
two areas, and that police officers arrested the man, Michael Jarrod
Bakkela, near one of those points of origin. Authorities charged him
with arson, criminal mischief and reckless endangering. Mr. Bakkela, who
could not be reached on Friday, had denied starting the fire when he was
arrested, the authorities said.

All told, the fires in Oregon have burned about one million acres, Gov.
Kate Brown said. California's footprint was even larger: Fires there
have now consumed about 3.1 million acres --- a modern record.

Gov. Gavin Newsom of California acknowledged that poor forest management
over decades had contributed to the severity of the state's wildfires in
recent years. But he said that serious droughts and record-breaking heat
waves were undeniable evidence that many of the most dire predictions
about climate change
\href{https://www.nytimes3xbfgragh.onion/2020/09/10/us/climate-change-california-wildfires.html}{had
already arrived}.

``California is America in fast forward,'' he said. ``What we're
experiencing right now is coming to communities all across the
country.''

Advertisement

\protect\hyperlink{after-bottom}{Continue reading the main story}

\hypertarget{site-index}{%
\subsection{Site Index}\label{site-index}}

\hypertarget{site-information-navigation}{%
\subsection{Site Information
Navigation}\label{site-information-navigation}}

\begin{itemize}
\tightlist
\item
  \href{https://help.nytimes3xbfgragh.onion/hc/en-us/articles/115014792127-Copyright-notice}{©~2020~The
  New York Times Company}
\end{itemize}

\begin{itemize}
\tightlist
\item
  \href{https://www.nytco.com/}{NYTCo}
\item
  \href{https://help.nytimes3xbfgragh.onion/hc/en-us/articles/115015385887-Contact-Us}{Contact
  Us}
\item
  \href{https://www.nytco.com/careers/}{Work with us}
\item
  \href{https://nytmediakit.com/}{Advertise}
\item
  \href{http://www.tbrandstudio.com/}{T Brand Studio}
\item
  \href{https://www.nytimes3xbfgragh.onion/privacy/cookie-policy\#how-do-i-manage-trackers}{Your
  Ad Choices}
\item
  \href{https://www.nytimes3xbfgragh.onion/privacy}{Privacy}
\item
  \href{https://help.nytimes3xbfgragh.onion/hc/en-us/articles/115014893428-Terms-of-service}{Terms
  of Service}
\item
  \href{https://help.nytimes3xbfgragh.onion/hc/en-us/articles/115014893968-Terms-of-sale}{Terms
  of Sale}
\item
  \href{https://spiderbites.nytimes3xbfgragh.onion}{Site Map}
\item
  \href{https://help.nytimes3xbfgragh.onion/hc/en-us}{Help}
\item
  \href{https://www.nytimes3xbfgragh.onion/subscription?campaignId=37WXW}{Subscriptions}
\end{itemize}
