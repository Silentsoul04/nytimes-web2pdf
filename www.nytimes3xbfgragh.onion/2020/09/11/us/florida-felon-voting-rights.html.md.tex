Sections

SEARCH

\protect\hyperlink{site-content}{Skip to
content}\protect\hyperlink{site-index}{Skip to site index}

\href{https://www.nytimes3xbfgragh.onion/section/us}{U.S.}

\href{https://myaccount.nytimes3xbfgragh.onion/auth/login?response_type=cookie\&client_id=vi}{}

\href{https://www.nytimes3xbfgragh.onion/section/todayspaper}{Today's
Paper}

\href{/section/us}{U.S.}\textbar{}Ex-Felons in Florida Must Pay Fines
Before Voting, Appeals Court Rules

\url{https://nyti.ms/3kayqO5}

\begin{itemize}
\item
\item
\item
\item
\item
\end{itemize}

Advertisement

\protect\hyperlink{after-top}{Continue reading the main story}

Supported by

\protect\hyperlink{after-sponsor}{Continue reading the main story}

\hypertarget{ex-felons-in-florida-must-pay-fines-before-voting-appeals-court-rules}{%
\section{Ex-Felons in Florida Must Pay Fines Before Voting, Appeals
Court
Rules}\label{ex-felons-in-florida-must-pay-fines-before-voting-appeals-court-rules}}

In a reversal, a court said Floridians who had completed sentences for
felonies must pay fines and fees before voting. The State Constitution
was amended in 2018 to restore their rights.

\includegraphics{https://static01.graylady3jvrrxbe.onion/images/2020/09/11/us/11FLORIDA/merlin_175868121_76151c14-3957-4bdf-8651-d2cec4b05839-articleLarge.jpg?quality=75\&auto=webp\&disable=upscale}

\href{https://www.nytimes3xbfgragh.onion/by/patricia-mazzei}{\includegraphics{https://static01.graylady3jvrrxbe.onion/images/2018/11/28/multimedia/author-patricia-mazzei/author-patricia-mazzei-thumbLarge.png}}

By \href{https://www.nytimes3xbfgragh.onion/by/patricia-mazzei}{Patricia
Mazzei}

\begin{itemize}
\item
  Sept. 11, 2020
\item
  \begin{itemize}
  \item
  \item
  \item
  \item
  \item
  \end{itemize}
\end{itemize}

MIAMI --- Four months after a federal judge ruled that it was akin to an
unconstitutional poll tax for Florida to require that people with
serious criminal convictions pay court fines and fees before they can
register to vote, an appeals court narrowly overturned that decision on
Friday.

The court's 6-4 ruling dealt a significant blow to civil rights groups
that have fought to expand the voter rolls with hundreds of thousands of
people who had completed prison time and parole for felony convictions.
It also undermined what had seemed like a major referendum victory in
2018 and served as another reminder of the decisive role that a slew of
legal cases could play before the presidential election.

The U.S. Court of Appeals for the 11th Circuit in Atlanta ruled that a
Florida law passed in 2019 was constitutional, reversing
\href{https://www.nytimes3xbfgragh.onion/2020/05/24/us/florida-felon-voting-court-judge-ruling.html}{the
lower court ruling in May} that said it discriminated against people who
had been convicted of felonies, many of whom are indigent, by imposing
an unlawful ``pay-to-vote system.''

The legal battle followed an amendment to the State Constitution in
2018, when Florida's voters decided to
\href{https://www.nytimes3xbfgragh.onion/2018/11/07/us/florida-felon-voting-rights.html}{end
the disenfranchisement of those convicted of felonies}, except for
murder and sexual offenses. Florida is a perennially close state in
presidential elections, and any effort to limit ballot access could play
a role in November, particularly if it affects a mostly low-income and
disadvantaged population likely to lean more toward Democrats. The
deadline to register is Oct. 5.

This week, a federal appeals court ruled that Texas could keep
restricting mail voting for people under 65, and the Wisconsin Supreme
Court ruled that the mailing of absentee ballots should be paused until
it decides whether the Green Party's presidential nominee should be on
the ballot. Both were seen as potential impediments to voting that were
likely to benefit Republicans.

In the Florida case, the appeals court sided with the administration of
Gov. Ron DeSantis, a Republican, and found that the felons who sued had
failed to prove a violation to the equal protection clause of the 14th
Amendment to the Constitution.

``If a State may decide that those who commit serious crimes are
presumptively unfit for the franchise,'' the 11th Circuit ruled, ``it
may also conclude that those who have completed their sentences are the
best candidates for re-enfranchisement.''

Five of the six judges who supported the 60-page decision were appointed
to the court by President Trump. Two of those judges were also former
Florida Supreme Court justices named to that bench by Mr. DeSantis. (One
of the former justices, Judge Barbara Lagoa, was named by Mr. Trump this
week as among those he would consider
\href{https://www.nytimes3xbfgragh.onion/2020/09/09/us/trump-supreme-court-list.html}{nominating
to a potential future seat on the Supreme Court}.)

Restoring felons' voting rights could vastly grow the electorate in the
nation's biggest presidential battleground state. An expert for the
American Civil Liberties Union and other civil rights groups testified
at trial that more than 774,000 felons in Florida owe legal financial
obligations.

``This ruling runs counter to the foundational principle that Americans
do not have to pay to vote,'' Julie Ebenstein, a senior staff attorney
with the A.C.L.U.'s Voting Rights Project, said in a statement. ``The
gravity of this decision cannot be overstated. It is an affront to the
spirit of democracy.''

The civil rights groups representing the felons pledged to keep fighting
and could appeal to the Supreme Court. But the court has already sided
once in the case with the state of Florida,
\href{https://www.nytimes3xbfgragh.onion/2020/07/16/us/supreme-court-felons-voting-florida.html}{rejecting
an emergency application} to lift the appeals court's stay while the
outcome was pending.

In a statement, Fred Piccolo, a spokesman for Mr. DeSantis, said
Friday's decision underscored that Amendment 4, as the referendum was
known, would restore the rights of felons only if they had completed the
entirety of sentences, including paying court fines and fees. (At the
time Amendment 4 passed, Florida was one of three states that prevented
people with felony records from voting.)

``All terms of a sentence means all terms,'' Mr. Piccolo said. ``There
are multiple avenues to restore rights, pay off debts and seek financial
forgiveness from one's victims. Second chances and the rule of law are
not mutually exclusive.''

Four judges dissented in a pair of lengthy and scathing opinions. ``I
doubt that today's decision --- which blesses Florida's neutering of
Amendment 4 --- will be viewed as kindly by history,'' Judge Adalberto
Jordan, who was appointed to the appeals court by President Barack
Obama, wrote in one of them.

The DeSantis administration has argued that voters knew that felons
would have to pay their outstanding debts before becoming eligible to
vote. The state has no centralized system to let felons know how much
they might owe, and the appeals court said states were not required to
provide a process for felons to learn whether they are eligible.

The Florida Rights Restoration Coalition, which organized the Amendment
4 campaign, has raised about \$4 million to help more than 4,000
``returning citizens'' pay their outstanding court fines and fees,
according to Neil G. Volz, the coalition's political director.

Florida's division of elections had received 85,000 voter registrations
as of May from former felons who believed they had been re-enfranchised
by Amendment 4. The division must screen those registrations to see
whether the would-be voters had paid their financial obligations. Only
then could any of them be removed from the voter rolls, the appeals
court said.

``Florida has yet to complete its screening of any of the
registrations,'' the appeals ruling noted. ``Until it does, it will not
have credible and reliable information supporting anyone's removal from
the voter rolls, and all 85,000 felons will be entitled to vote.''

Rebecca R. Ruiz contributed reporting from New York.

Advertisement

\protect\hyperlink{after-bottom}{Continue reading the main story}

\hypertarget{site-index}{%
\subsection{Site Index}\label{site-index}}

\hypertarget{site-information-navigation}{%
\subsection{Site Information
Navigation}\label{site-information-navigation}}

\begin{itemize}
\tightlist
\item
  \href{https://help.nytimes3xbfgragh.onion/hc/en-us/articles/115014792127-Copyright-notice}{©~2020~The
  New York Times Company}
\end{itemize}

\begin{itemize}
\tightlist
\item
  \href{https://www.nytco.com/}{NYTCo}
\item
  \href{https://help.nytimes3xbfgragh.onion/hc/en-us/articles/115015385887-Contact-Us}{Contact
  Us}
\item
  \href{https://www.nytco.com/careers/}{Work with us}
\item
  \href{https://nytmediakit.com/}{Advertise}
\item
  \href{http://www.tbrandstudio.com/}{T Brand Studio}
\item
  \href{https://www.nytimes3xbfgragh.onion/privacy/cookie-policy\#how-do-i-manage-trackers}{Your
  Ad Choices}
\item
  \href{https://www.nytimes3xbfgragh.onion/privacy}{Privacy}
\item
  \href{https://help.nytimes3xbfgragh.onion/hc/en-us/articles/115014893428-Terms-of-service}{Terms
  of Service}
\item
  \href{https://help.nytimes3xbfgragh.onion/hc/en-us/articles/115014893968-Terms-of-sale}{Terms
  of Sale}
\item
  \href{https://spiderbites.nytimes3xbfgragh.onion}{Site Map}
\item
  \href{https://help.nytimes3xbfgragh.onion/hc/en-us}{Help}
\item
  \href{https://www.nytimes3xbfgragh.onion/subscription?campaignId=37WXW}{Subscriptions}
\end{itemize}
