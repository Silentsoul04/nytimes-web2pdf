Sections

SEARCH

\protect\hyperlink{site-content}{Skip to
content}\protect\hyperlink{site-index}{Skip to site index}

\href{https://www.nytimes3xbfgragh.onion/section/us}{U.S.}

\href{https://myaccount.nytimes3xbfgragh.onion/auth/login?response_type=cookie\&client_id=vi}{}

\href{https://www.nytimes3xbfgragh.onion/section/todayspaper}{Today's
Paper}

\href{/section/us}{U.S.}\textbar{}Suspect Is Charged With Arson in
Oregon Wildfire

\url{https://nyti.ms/3hm5vEI}

\begin{itemize}
\item
\item
\item
\item
\item
\item
\end{itemize}

\hypertarget{wildfires-in-the-west}{%
\subsubsection{\texorpdfstring{\href{https://www.nytimes3xbfgragh.onion/spotlight/california-wildfires?name=styln-california-wildfires\&region=TOP_BANNER\&block=storyline_menu_recirc\&action=click\&pgtype=Article\&impression_id=d7e50210-f52b-11ea-b1f6-7dcd7185e0b5\&variant=undefined}{Wildfires
in the West}}{Wildfires in the West}}\label{wildfires-in-the-west}}

\begin{itemize}
\tightlist
\item
  live\href{https://www.nytimes3xbfgragh.onion/2020/09/12/us/wildfires-live-updates.html?name=styln-california-wildfires\&region=TOP_BANNER\&block=storyline_menu_recirc\&action=click\&pgtype=Article\&impression_id=d7e52920-f52b-11ea-b1f6-7dcd7185e0b5\&variant=undefined}{Fires
  Updates}
\item
  \href{https://www.nytimes3xbfgragh.onion/interactive/2020/us/fires-map-tracker.html?name=styln-california-wildfires\&region=TOP_BANNER\&block=storyline_menu_recirc\&action=click\&pgtype=Article\&impression_id=d7e52921-f52b-11ea-b1f6-7dcd7185e0b5\&variant=undefined}{Maps
  of the Fires}
\item
  \href{https://www.nytimes3xbfgragh.onion/article/wildfires-photos-california-oregon-washington-state.html?name=styln-california-wildfires\&region=TOP_BANNER\&block=storyline_menu_recirc\&action=click\&pgtype=Article\&impression_id=d7e52922-f52b-11ea-b1f6-7dcd7185e0b5\&variant=undefined}{Photos}
\item
  \href{https://www.nytimes3xbfgragh.onion/2020/09/10/us/climate-change-california-wildfires.html?name=styln-california-wildfires\&region=TOP_BANNER\&block=storyline_menu_recirc\&action=click\&pgtype=Article\&impression_id=d7e52923-f52b-11ea-b1f6-7dcd7185e0b5\&variant=undefined}{A
  Climate Reckoning}
\item
  \href{https://www.nytimes3xbfgragh.onion/article/wildfires-california-oregon-washington.html?name=styln-california-wildfires\&region=TOP_BANNER\&block=storyline_menu_recirc\&action=click\&pgtype=Article\&impression_id=d7e52924-f52b-11ea-b1f6-7dcd7185e0b5\&variant=undefined}{Answers
  to Your Questions}
\item
  \href{https://www.nytimes3xbfgragh.onion/2020/09/09/us/california-wildfires.html?name=styln-california-wildfires\&region=TOP_BANNER\&block=storyline_menu_recirc\&action=click\&pgtype=Article\&impression_id=d7e52925-f52b-11ea-b1f6-7dcd7185e0b5\&variant=undefined}{Newsletter}
\end{itemize}

Advertisement

\protect\hyperlink{after-top}{Continue reading the main story}

Supported by

\protect\hyperlink{after-sponsor}{Continue reading the main story}

\hypertarget{suspect-is-charged-with-arson-in-oregon-wildfire}{%
\section{Suspect Is Charged With Arson in Oregon
Wildfire}\label{suspect-is-charged-with-arson-in-oregon-wildfire}}

Oregon's governor said there were concerns for dozens of people reported
missing in a state where more than a million acres have burned.
California and Washington State are also battling fires.

Published Sept. 11, 2020Updated Sept. 12, 2020, 12:29 a.m. ET

\begin{itemize}
\item
\item
\item
\item
\item
\item
\end{itemize}

\hypertarget{heres-what-you-need-to-know}{%
\subsubsection{Here's what you need to
know:}\label{heres-what-you-need-to-know}}

\begin{itemize}
\tightlist
\item
  \protect\hyperlink{link-1e628466}{Oregon and California brace for a
  rising death toll.}
\item
  \protect\hyperlink{link-7a0c7fbe}{False rumors are complicating the
  fight against the fires around Portland.}
\item
  \protect\hyperlink{link-18416023}{A man is charged with arson in a
  southern Oregon blaze.}
\item
  \protect\hyperlink{link-25b5d74}{Climate change is a real and urgent
  threat in California.}
\item
  \protect\hyperlink{link-5b86b2c4}{`This is a fathomless loss': Some
  searches for the missing end in tragedy.}
\item
  \protect\hyperlink{link-38ec109a}{Getting prisoners out of harm's way
  raises the risk of spreading the coronavirus.}
\item
  \protect\hyperlink{link-3e1cece7}{States are in a desperate search for
  help battling the fires.}
\end{itemize}

\includegraphics{https://static01.graylady3jvrrxbe.onion/images/2020/09/12/us/11fire-vid/11fire-vid-videoSixteenByNine3000.jpg}

\hypertarget{oregon-and-california-brace-for-a-rising-death-toll}{%
\subsection{Oregon and California brace for a rising death
toll.}\label{oregon-and-california-brace-for-a-rising-death-toll}}

The governors of California and Oregon delivered blunt and alarming
details on Friday about the massive wildfires that have consumed
millions of acres across their two states and Washington, killing at
least 17 people. State leaders also braced for that death toll to
increase, with an Oregon official saying the state was preparing for a
``mass fatality incident.''

Oregon, Washington and California are enduring a wildfire season of
historic proportions, with the firefighting effort compounded by the
coronavirus pandemic and
\href{https://www.nytimes3xbfgragh.onion/2020/09/10/us/wildfires-misinformation-arson-activists.html}{misinformation
online}.

But as residents readied themselves for more pain, they also looked to
the skies and hoped that changing weather might help them this weekend
in their fight. Doug Grafe, chief of Fire Protection for the Oregon
Department of Forestry, said that the strong winds that had spread the
fires had dissipated, and that cooler temperatures and higher humidity
would help fire crews move ``from just life safety to the offense'' in
fighting the blazes.

\href{https://www.nytimes3xbfgragh.onion/interactive/2020/us/fires-map-tracker.html}{}

\includegraphics{https://static01.graylady3jvrrxbe.onion/images/2020/09/11/us/fires-map-tracker-1599839565497/fires-map-tracker-1599839565497-articleLarge.png}

\hypertarget{california-oregon-and-washington-fire-tracking-maps}{%
\subsection{California, Oregon and Washington Fire Tracking
Maps}\label{california-oregon-and-washington-fire-tracking-maps}}

Maps showing air quality and where major fires are burning in the
Western states.

Gov. Gavin Newsom of California also noted the dying winds and said that
a ``modest amount'' of precipitation could be on the way in his state.

In her news conference, Gov. Kate Brown of Oregon noted that well over 1
million acres of land --- over 1,500 square miles --- has been burned in
the state and that the state's air quality ranks the worst in the world.
``Almost anywhere in the state you can feel this right now,'' she said.

More than 40,000 Oregonians have already been evacuated, and about
500,000 are in zones that may be evacuated as the fires continue to
grow. Mayor Ted Wheeler of Portland
\href{https://www.portland.gov/sites/default/files/2020-09/cop-fire-emergency-declaration-september-10-24-2020.pdf}{declared
a state of emergency} on Thursday night, and residents of Molalla, about
30 miles to the south, packed highways as they fled from the approaching
fires.

But tragedy has already befallen some, with towns like Talent and
Phoenix all but obliterated. Andrew Phelps, director of the Oregon
Office of Emergency Management, said mass casualties were a possibility
``based on what we know and the numbers of structures that have been
lost.''

Three new deaths in Oregon were reported on Friday: Officials in Marion
County located two additional victims from the Beachie Creek Fire east
of Salem, and crews found a victim in a residence within the perimeter
of the Holiday Farm Fire east of Eugene.

Sheriff Kory Honea of Butte County, Calif., lowered the death toll of
the Bear Fire north of Sacramento to nine, saying that deputies had
mistakenly identified an anatomical model of a skeleton as human
remains.

In California, where more than three million acres have burned, Mr.
Newsom held a surreal livestreamed news conference, speaking among
charred trees in the midst of a yellowish, smoky haze left by the raging
North Complex Fires.

To his west, the August Complex Fire --- which this week became the
largest in the state's history --- had now burned across 747,000 acres,
Mr. Newsom said.

\includegraphics{https://static01.graylady3jvrrxbe.onion/images/2020/09/11/us/politics/Screen-Shot-2020-09-11-at-4/Screen-Shot-2020-09-11-at-4-videoSixteenByNineJumbo1600.png}

Mr. Newsom, who said he feared that more bodies would be found,
emphasized the unprecedented scale of the challenges facing
firefighters, who have been strained by enormous blazes up and down the
coast.

``It's just something we've never seen in our lifetime,'' Mr. Newsom
said.

Mr. Newsom said he spoke with President Trump for about a half an hour
on Thursday about the fires and said the president ``enforced his
commitment'' to sending aid for both businesses and individuals.

While the governor acknowledged that poor forest management over decades
has contributed to the severity of wildfires, he said that mega-droughts
and record heat waves are evidence that the most dire predictions about
climate change have already arrived.

While California, he said, was investing in green technology and
regulating vehicle emissions, the fires ravaging the entire West Coast
were a grim preview of what the rest of the country may soon face if
policies and priorities did not change nationally.

``California is America in fast forward,'' he said. ``What we're
experiencing right now is coming to communities all across the
country.''

\emph{{[}Sign up}
\href{https://www.nytimes3xbfgragh.onion/newsletters/california-today}{\emph{for
California Today}}\emph{, our daily newsletter from the Golden
State.{]}}

\hypertarget{false-rumors-are-complicating-the-fight-against-the-fires-around-portland}{%
\subsection{False rumors are complicating the fight against the fires
around
Portland.}\label{false-rumors-are-complicating-the-fight-against-the-fires-around-portland}}

\includegraphics{https://static01.graylady3jvrrxbe.onion/images/2020/09/11/us/11FIRES-BRIEFING-rumors2/merlin_176834859_604e8de9-36fe-4f66-969d-3ae87f7954f5-articleLarge.jpg?quality=75\&auto=webp\&disable=upscale}

Every natural disaster has its holdouts. But the political fear-stoking
that accompanied a tumultuous summer of racial-justice protests in
Oregon has become a volatile new complication in the catastrophic
wildfires that pushed closer to Portland on Friday, as authorities try
to evacuate thousands of people.

Law-enforcement officials across the state said they had been swamped
with calls about social-media misinformation and begged people to
``\href{https://www.facebookcorewwwi.onion/DouglasCoSO/posts/3294100377341103}{STOP.
SPREADING. RUMORS!}'' In the line of fire, the swirl of rumors actually
helped goad some people into defying evacuation orders so they could
stay and guard their homes.

As a Level 3 evacuation on Thursday urged people to ``leave now,'' an
eerie stillness fell over Molalla, an old timber town of 9,000 an hour's
drive south of Portland, and the holdout residents girded themselves for
two threats. One was the very real
\href{https://inciweb.nwcg.gov/}{125,000-acre Riverside Fire} burning
just east of town. The other was the imagined invasion of left-wing mobs
and arsonists that multiple law-enforcement agencies have sought to
\href{https://www.nytimes3xbfgragh.onion/2020/09/10/us/wildfires-misinformation-arson-activists.html?referringSource=articleShare}{refute}.

Residents who remained hosed down their roofs and soaked their lawns.
They organized go-bags of baby supplies and clothes, just in case. They
scouted for unfamiliar cars on the roads.

``I'm protecting my city,'' Troy McNeeley said as he stood in front of
the 900-square foot home he shares with his son, his son's partner and
several cats. ``If I see people doing crap, I'm going to hurt them.''

The rumors played into some conservative residents'
\href{https://www.nytimes3xbfgragh.onion/2020/09/05/us/portland-political-chasm-protests-unrest.html}{fears
and anger} over months of protests in Portland, where left-wing and
right-wing groups have occasionally
\href{https://www.nytimes3xbfgragh.onion/2020/08/22/us/portland-protests.html}{clashed}.

On Wednesday, the police in Portland warned protesters about lighting
fires --- a seemingly innocuous public-safety message that was followed
by waves of rumor about arsonists and mayhem. Sheriff's offices and fire
departments already coping with wildfires that have consumed 900,000
acres were flooded with phone calls.

``We are inundated with questions about things that are FAKE stories,''
the Jackson County Sheriff's Office in Medford posted on Facebook. ``One
example is a story circulating that varies about what group is involved
as to setting fires and arrests being made. THIS IS NOT TRUE!''

Law-enforcement agencies in southern Oregon announced on Thursday that
they had begun to investigate whether the Almeda fire had been
deliberately set. The fire has burned hundreds of homes around Medford
and is tied to two deaths. But the police chief of Ashland
\href{https://www.oregonlive.com/crime/2020/09/arson-investigation-underway-where-human-remains-found-at-almeda-fire-in-ashland-chief-says.html}{told
The Oregonian} that there was no evidence pointing to anti-fascist
activists.

\hypertarget{a-man-is-charged-with-arson-in-a-southern-oregon-blaze}{%
\subsection{A man is charged with arson in a southern Oregon
blaze.}\label{a-man-is-charged-with-arson-in-a-southern-oregon-blaze}}

Image

A man was charged with arson on Friday after residents said he had
started a blaze during the Almeda Fire, which destroyed much of Phoenix,
Ore.Credit...Gillian Flaccus/Associated Press

Authorities in southern Oregon charged a 41-year-old man with starting
part of one of this year's most destructive fires, saying he lit the
fire in a small Oregon town as a larger blaze moved toward the area.

The Jackson County Sheriff's Office said the Almeda Fire started around
11 a.m. on Tuesday in Ashland, Ore., and then began spreading north
along Interstate 5. Around 5 p.m., residents reported that a man had
started a fire in Phoenix, a town of about 4,500 people just north of
Ashland that was under orders to evacuate, authorities said.

The Sheriff's Office said police officers discovered Michael Jarrod
Bakkela at the scene, denying that he started the large fire nearby.
Police arrested him on a parole violation.

On Friday, the Jackson County district attorney charged him with arson,
criminal mischief and reckless endangering.

Mr. Bakkela, who could not be reached, had not yet been appointed a
lawyer, said Beth Heckert, the county's district attorney. He was
scheduled to be arraigned in court on Monday, she said.

Mike Moran, a public information officer for the Jackson County sheriff,
said Mr. Bakkela had a criminal record and was well known to local law
enforcement. A news release from the Sheriff's Office described him as
``a local transient.''

While many wildfires on the West Coast this year have burned through
remote areas and parts of rural communities, the Almeda Fire hit a
series of towns along the freeway in southern Oregon, destroying an
estimated 500 homes and 100 businesses. Mr. Moran said authorities were
still investigating the fire's initial point of origin in Ashland. He
said that they suspected arson there, too, and that they found the
remains of one man near the fire's start.

\hypertarget{climate-change-is-a-real-and-urgent-threat-in-california}{%
\subsection{Climate change is a real and urgent threat in
California.}\label{climate-change-is-a-real-and-urgent-threat-in-california}}

Image

Scorched land left behind by the Bear Fire in Feather Falls, Calif., on
Thursday.Credit...Max Whittaker for The New York Times

Multiple mega fires burning more than three million acres. Millions of
residents smothered in toxic air. Rolling blackouts and triple-digit
heat waves. Climate change, in the words of one scientist, is smacking
California in the face.

The crisis in the nation's most populous state is more than just an
accumulation of individual catastrophes. It is also an example of
something climate experts have long worried about, but which few
expected to see so soon: a cascade effect, in which a series of
disasters overlap, triggering or amplifying each other.

``You're toppling dominoes in ways that Americans haven't imagined,''
said Roy Wright, who directed resilience programs for the Federal
Emergency Management Agency until 2018 and grew up in Vacaville, Calif.,
near one of this year's largest fires. ``It's apocalyptic.''

The same could be said for the entire West Coast this week, to
Washington and Oregon, where towns were decimated by infernos as
firefighters were stretched to their limits.

\href{https://www.nytimes3xbfgragh.onion/spotlight/california-wildfires}{Wildfires
in the West ›}

\hypertarget{live-updates}{%
\subsection{\texorpdfstring{\href{https://www.nytimes3xbfgragh.onion/2020/09/12/us/wildfires-live-updates.html}{Live
Updates}}{Live Updates}}\label{live-updates}}

Updated~

Sept. 12, 2020, 2:53 p.m. ET

\begin{itemize}
\tightlist
\item
  \href{https://www.nytimes3xbfgragh.onion/2020/09/12/us/wildfires-live-updates.html\#link-f3961ff}{President
  Trump will visit California on Monday after destructive fires.}
\item
  \href{https://www.nytimes3xbfgragh.onion/2020/09/12/us/wildfires-live-updates.html\#link-7e503ae9}{Shifting
  weather may improve firefighting conditions on the West Coast.}
\item
  \href{https://www.nytimes3xbfgragh.onion/2020/09/12/us/wildfires-live-updates.html\#link-5e4c548d}{Oregon's
  fire marshal is temporarily replaced as firefighters battle blazes.}
\end{itemize}

California's simultaneous crises illustrate how the ripple effect works.
A scorching summer led to dry conditions never before experienced. That
aridity helped make the season's wildfires the biggest ever recorded.
Six of the 20 largest wildfires in modern California history have
occurred this year.

If climate change was a somewhat abstract notion a decade ago, today it
is all too real for Californians. The intensely hot wildfires are not
only chasing thousands of people from their homes but causing dangerous
chemicals to leach into drinking water. Excessive heat warnings and
suffocating smoky air have threatened the health of people already
struggling during the pandemic. And the threat of more wildfires has led
insurance companies to cancel homeowner policies and the state's main
utility to shut off power to tens of thousands of people pre-emptively.

\hypertarget{this-is-a-fathomless-loss-some-searches-for-the-missing-end-in-tragedy}{%
\subsection{`This is a fathomless loss': Some searches for the missing
end in
tragedy.}\label{this-is-a-fathomless-loss-some-searches-for-the-missing-end-in-tragedy}}

As the blazes rage across California, Oregon and Washington, family and
friends are desperately searched for missing loved ones who remained
unaccounted for.

Zygy Roe-Zurz, whose family lives in Berry Creek, Calif., said he had
been waiting for days for news from his mother, his aunt and his uncle.
On Thursday, he learned that his aunt was killed as the Bear Fire ripped
through the community, and that his mother remained missing. Authorities
told the family that Mr. Roe-Zurz's uncle was likely dead as well, he
said.

``I feel barren --- this is a fathomless loss and I will never be the
same,'' said Mr. Roe-Zurz, 37, who is in Arkansas and last spoke to his
mother on Tuesday night, before the flames intensified. ``This cruel
fire took everything.''

He said that his family members staying at the property in Berry Creek
had been under the impression that the fire was getting under control,
but that the situation changed dramatically as the Bear Fire jumped an
astonishing 230,000 acres overnight Tuesday into Wednesday.

``It's pretty much a nightmare scenario,'' Mr. Roe-Zurz said. ``I'm
devastated.''

There was better news for other families who found out that loved ones
they believed to be missing were found safe on Thursday.

Katy Carmel said her daughter, Natalie Anderson, had been on a camping
trip with her boyfriend near the McKenzie Bridge east of Eugene, Ore.
But when the Holiday Farm Fire broke out on Monday evening, Ms. Carmel
could no longer reach Ms. Anderson.

Ms. Carmel could not sleep, fearing the worst. Days passed and the
anxiety built. On Thursday, authorities notified the families that both
Ms. Anderson and her boyfriend, Enmanuel Rodriguez, were safe and
evacuated.

Ms. Carmel said she was relieved to hear the news, but added, ``I'll be
better once she's actually home.''

\hypertarget{getting-prisoners-out-of-harms-way-raises-the-risk-of-spreading-the-coronavirus}{%
\subsection{Getting prisoners out of harm's way raises the risk of
spreading the
coronavirus.}\label{getting-prisoners-out-of-harms-way-raises-the-risk-of-spreading-the-coronavirus}}

Image

The Oregon State Penitentiary in Salem was surrounded by smoke on
Thursday after inmates were relocated from other prisons.Credit...Carlos
Barria/Reuters

Officials in Oregon's state corrections system this week began moving
hundreds of inmates out of the path of the wildfires creeping toward
some of their prisons. But the introduction of large groups of prisoners
into different facilities may be exposing them to another risk ---
contracting the coronavirus.

Juan Chavez, a lawyer with the Oregon Justice Resource Center, a
nonprofit legal advocacy group, said that relocated inmates were
sleeping on mattresses crammed close together, but it's ``picking your
poison.''

``You have two crises that are stacked on top of each other --- Covid-19
and these fires --- and they're out of good options,'' Mr. Chavez said.
He added that he fears the relocated inmates could contribute to a
superspreader event for the virus in the prisons.

But few other options exist for the Oregon Department of Corrections,
which has evacuated four prisons so far.

As the Beachie Creek and Lionshead wildfires raged in an area southeast
of Portland, officials hastily relocated 1,450 inmates from three
prisons in Marion County --- Oregon State Correctional Institution,
Santiam Correctional Institution and Mill Creek Correctional Facility.
Inmates were moved west, to emergency beds in the Oregon State
Penitentiary in Salem on Tuesday, according to the agency.

On Thursday officials sent 1,303 inmates from Coffee Creek Correctional
Facility, a prison north of Salem in Wilsonville, to the Deer Ridge
Correctional Institute more than 100 miles to the southeast, said
Jennifer Black, a spokeswoman for the prison system.

Those inmates were moved to avoid a third blaze, the Riverside wildfire,
which is north of the Beachie Creek and Lionshead wildfires. Each of the
three blazes is more than 100,000 acres in size.

Inmates will be ``housed with others from their home institution
whenever possible,'' and officials are aware of the potential
coronavirus spread, Ms. Black said.

``We are taking all available steps to mitigate that impact,'' Ms. Black
said. ``As we have said from the beginning, prisons were not constructed
to allow for optimal social distancing.''

The coronavirus has already ravaged the state prison population. In
June, Gov. Kate Brown of
Oregon\href{https://www.oregon.gov/newsroom/Pages/NewsDetail.aspx?newsid=36856}{commuted
the sentences of 57 inmates} who were vulnerable to the coronavirus.
There have been 829 confirmed coronavirus cases in prison system
facilities, including staff members and inmates, according to the
department's\href{https://www.oregon.gov/doc/covid19/Pages/covid19-tracking.aspx}{records}.
Six people have died.

At the Oregon State Penitentiary, 36 staff members and 143 inmates have
tested positive for the virus.

\hypertarget{states-are-in-a-desperate-search-for-help-battling-the-fires}{%
\subsection{States are in a desperate search for help battling the
fires.}\label{states-are-in-a-desperate-search-for-help-battling-the-fires}}

Image

Inmate firefighters worked on the Bear Fire near Oroville, Calif., on
Thursday. Emergency responders are struggling to keep pace with fires
that have destroyed entire towns.Credit...Max Whittaker for The New York
Times

As
\href{https://www.nytimes3xbfgragh.onion/article/wildfires-photos-california-oregon-washington-state.html}{wildfires}
began consuming communities across Oregon this week, leaders at the
state emergency management office fired off an email to counterparts
around the country, pleading for 10 firefighting strike teams that could
bring 50 extra engines to the region.

The state got one commitment: Utah would send a team with five engines.

Facing a historic year of wildfire destruction across the West Coast,
including more than three million acres consumed in California, the
national emergency systems that rely on state-to-state assistance have
been buckling under the strain. That has left emergency responders
struggling to keep pace with fires that have destroyed entire towns and
led to at least 17 deaths, with seven more people found dead on Thursday
from a fire north of Sacramento.

``I don't know that we have any fires where we can say we have got
enough resources to do what we need to do,'' Andrew Phelps, the director
of the Oregon Office of Emergency Management, said.

Fires continued to rage in southern Oregon, where hundreds of homes have
been razed, as well as east of Salem, where two bodies have been found,
and along the state's coast. More than 900,000 acres have burned, nearly
double a typical season. Hundreds of thousands of people have been
ordered to evacuate, including parts of the Portland suburbs, where
fires were still on the move.

In California, firefighters continued to battle the blazes of a
remarkable wildfire season, including the August Complex burning in the
Mendocino National Forest that is now the largest fire in the state's
recorded history.

In Washington, hundreds of homes and other structures were at risk of
wildfires that continued to burn, even as a deadly stretch of dry winds
from the East began to ease. Hilary Franz, the state's commissioner of
public lands, said the state was searching for help from elsewhere in
the country.

So many state aid requests have gone to the National Multi-Agency
Coordinating Group, which helps direct wildfire resources, that the
group has been left to decide which ones get priority. Dan Smith, a
member of the group who is also fire director for the National
Association of State Foresters, said that as of Thursday morning there
were over 300 requests for support that could not be fulfilled.

On Friday, Gov. Greg Abbott of Texas said around 190 more firefighters,
50 more fire trucks and 10 command vehicles from 56 fire departments
across Texas were set to be deployed to California. The state had
already sent 44 firefighters, 10 fire trucks and two command vehicles to
California late last month.

\hypertarget{wildfire-smoke-is-dangerous-to-your-health-heres-how-to-protect-yourself}{%
\subsection{Wildfire smoke is dangerous to your health. Here's how to
protect
yourself.}\label{wildfire-smoke-is-dangerous-to-your-health-heres-how-to-protect-yourself}}

Image

Heavy smoke conditions in Clackamas County, Ore., on
Thursday.Credit...Kristina Barker for The New York Times

Smoke from wildfires, which can include toxic substances from burned
buildings, has been linked to serious health problems.

Studies have shown that when waves of smoke hit
\href{https://insights.ovid.com/epidemiology/epide/2017/01/000/wildfire-specific-fine-particulate-matter-risk/13/00001648}{the
rate of hospital visits rises} and many of the additional patients
experience
\href{https://www.ncbi.nlm.nih.gov/pmc/articles/PMC6015400/}{respiratory
problems, heart attacks and strokes}.

The health effects of wildfire smoke don't go away when skies clear. A
\href{https://www.sciencedirect.com/science/article/pii/S0160412019326935}{recent
study on Montana residents} suggested a long tail for wildfire smoke
exposure.

Erin Landguth, an associate professor in the school of public and
community health science at the University of Montana and the lead
author on the study, said research had shown that ``after bad fire
seasons, one would expect to see three to five times worse flu seasons''
months later.

If you can't leave an area that has high levels of smoke, the C.D.C.
recommends
\href{https://www3.epa.gov/airnow/smoke_fires/prepare-for-fire-season-508.pdf}{limiting
exposure} by staying indoors with windows and doors closed and running
air-conditioners in recirculation mode so that outside air isn't drawn
into your home.

Portable air purifiers are also recommended, though, like
air-conditioners, they require electricity. If utilities cut off power,
\href{https://www.nytimes3xbfgragh.onion/2020/08/18/us/california-blackouts.html}{as
has happened in California}, those options are limited.

If you do have power, avoid frying food, which can increase indoor
smoke.

Experts say it is especially important to avoid cigarettes. They also
recommend avoiding strenuous outdoor activities when the air is bad.
When outside, well-fitted N95 masks are also recommended, though they
are in short supply because of the coronavirus pandemic.

Some other masks, particularly tightly woven ones made of different
layers of fabric, can provide ``pretty good filtration,'' if they are
fitted closely to the face, said Sarah Henderson, senior scientist in
environmental health services at the British Columbia Center for Disease
Control.

Reporting was contributed by Davey Alba, Tim Arango, Mike Baker, Kate
Conger, Jill Cowan, Richard Fausset, Marie Fazio, Christopher Flavelle,
Thomas Fuller, Jack Healy, Giulia McDonnell Nieto del Rio, Jack Nicas,
Bryan Pietsch, John Schwartz, Will Wright and Alan Yuhas.

Advertisement

\protect\hyperlink{after-bottom}{Continue reading the main story}

\hypertarget{site-index}{%
\subsection{Site Index}\label{site-index}}

\hypertarget{site-information-navigation}{%
\subsection{Site Information
Navigation}\label{site-information-navigation}}

\begin{itemize}
\tightlist
\item
  \href{https://help.nytimes3xbfgragh.onion/hc/en-us/articles/115014792127-Copyright-notice}{©~2020~The
  New York Times Company}
\end{itemize}

\begin{itemize}
\tightlist
\item
  \href{https://www.nytco.com/}{NYTCo}
\item
  \href{https://help.nytimes3xbfgragh.onion/hc/en-us/articles/115015385887-Contact-Us}{Contact
  Us}
\item
  \href{https://www.nytco.com/careers/}{Work with us}
\item
  \href{https://nytmediakit.com/}{Advertise}
\item
  \href{http://www.tbrandstudio.com/}{T Brand Studio}
\item
  \href{https://www.nytimes3xbfgragh.onion/privacy/cookie-policy\#how-do-i-manage-trackers}{Your
  Ad Choices}
\item
  \href{https://www.nytimes3xbfgragh.onion/privacy}{Privacy}
\item
  \href{https://help.nytimes3xbfgragh.onion/hc/en-us/articles/115014893428-Terms-of-service}{Terms
  of Service}
\item
  \href{https://help.nytimes3xbfgragh.onion/hc/en-us/articles/115014893968-Terms-of-sale}{Terms
  of Sale}
\item
  \href{https://spiderbites.nytimes3xbfgragh.onion}{Site Map}
\item
  \href{https://help.nytimes3xbfgragh.onion/hc/en-us}{Help}
\item
  \href{https://www.nytimes3xbfgragh.onion/subscription?campaignId=37WXW}{Subscriptions}
\end{itemize}
