Sections

SEARCH

\protect\hyperlink{site-content}{Skip to
content}\protect\hyperlink{site-index}{Skip to site index}

\href{https://www.nytimes3xbfgragh.onion/section/world/middleeast}{Middle
East}

\href{https://myaccount.nytimes3xbfgragh.onion/auth/login?response_type=cookie\&client_id=vi}{}

\href{https://www.nytimes3xbfgragh.onion/section/todayspaper}{Today's
Paper}

\href{/section/world/middleeast}{Middle East}\textbar{}Bahrain Will
Normalize Relations With Israel, in Deal Brokered by Trump

\url{https://nyti.ms/32ja5zy}

\begin{itemize}
\item
\item
\item
\item
\item
\item
\end{itemize}

Advertisement

\protect\hyperlink{after-top}{Continue reading the main story}

Supported by

\protect\hyperlink{after-sponsor}{Continue reading the main story}

\hypertarget{bahrain-will-normalize-relations-with-israel-in-deal-brokered-by-trump}{%
\section{Bahrain Will Normalize Relations With Israel, in Deal Brokered
by
Trump}\label{bahrain-will-normalize-relations-with-israel-in-deal-brokered-by-trump}}

The island kingdom in the Persian Gulf becomes the second Arab nation in
a month to more openly embrace Israel, dismissing Palestinian
objections.

\includegraphics{https://static01.graylady3jvrrxbe.onion/images/2020/09/11/us/politics/11dc-prexy-sub/11dc-prexy-sub-videoSixteenByNineJumbo1600-v2.jpg}

\href{https://www.nytimes3xbfgragh.onion/by/michael-crowley}{\includegraphics{https://static01.graylady3jvrrxbe.onion/images/2019/10/25/reader-center/author-michael-crowley/author-michael-crowley-thumbLarge-v2.png}}\href{https://www.nytimes3xbfgragh.onion/by/david-m-halbfinger}{\includegraphics{https://static01.graylady3jvrrxbe.onion/images/2018/10/10/multimedia/author-david-m-halbfinger/author-david-m-halbfinger-thumbLarge.png}}

By \href{https://www.nytimes3xbfgragh.onion/by/michael-crowley}{Michael
Crowley} and
\href{https://www.nytimes3xbfgragh.onion/by/david-m-halbfinger}{David M.
Halbfinger}

\begin{itemize}
\item
  Sept. 11, 2020
\item
  \begin{itemize}
  \item
  \item
  \item
  \item
  \item
  \item
  \end{itemize}
\end{itemize}

WASHINGTON --- President Trump announced on Friday that Bahrain would
establish full diplomatic relations with Israel, following the United
Arab Emirates, in another sign of shifting Middle East dynamics that are
bringing Arab nations closer to Israel.

Mr. Trump announced the news
\href{https://twitter.com/realDonaldTrump/status/1304464848831631361}{on
Twitter}, releasing a joint statement with Bahrain and Israel and
calling the move ``a historic breakthrough to further peace in the
Middle East.'' Speaking to reporters, the president said the anniversary
of the Sept. 11, 2001, terrorist attacks was a fitting day for the
announcement.

``There's no more powerful response to the hatred that spawned 9/11,''
he said.

The announcement came after a
\href{https://www.nytimes3xbfgragh.onion/2020/08/13/us/politics/trump-israel-united-arab-emirates-uae.html}{similar
one} last month by Israel and the United Arab Emirates that they would
normalize relations, on the condition that Prime Minister Benjamin
Netanyahu of Israel not follow through with plans to annex portions of
the West Bank. Trump administration officials said they hoped that
agreement would encourage other Arab countries with historically hostile
--- though recently thawing --- relations with Israel to take similar
steps.

The deal, which isolates the Palestinians, comes as Mr. Trump tries to
position himself as a peacemaker before the elections in November.

Bahrain's move was not unexpected: The tiny Persian Gulf kingdom was
widely seen as the low-hanging fruit to be picked if all went well in
the aftermath of the Emiratis' announcement, analysts said. Bahrain,
strategically important as the home port for the U.S. Navy's Fifth
Fleet, had already opened its airspace to new commercial passenger
\href{https://www.nytimes3xbfgragh.onion/2020/08/31/world/middleeast/israel-uae-flight.html}{flights
between Tel Aviv and Abu Dhabi}; Secretary of State Mike Pompeo
\href{https://www.nytimes3xbfgragh.onion/2020/08/24/world/middleeast/pompeo-jerusalem-rnc.html}{visited}
the region last month in an effort to close the deal.

It was unclear whether the United States or Israel had made any
concessions to Bahrain in exchange for the agreement. When asked during
a briefing for reporters, Mr. Trump's son-in-law and adviser, Jared
Kushner, who helped to broker the deal, did not respond directly.

Israel would always welcome the addition of another Arab country to the
short list of those with diplomatic ties, but in Jerusalem the
announcement landed with neither the surprise nor the weight of the
Emirati decision.

``Any Arab country is very important, for sure,'' said Amos Gilead, a
retired Israeli major general who leads the Institute for Policy and
Strategy at the Interdisciplinary Center Herzliya. ``It's another
precedent. But with all due respect, when you are small, you are
small.''

But Bahrain has outsize significance, said Kirsten Fontenrose, a former
National Security Council senior director for Gulf affairs in the Trump
White House who is now a director at the Atlantic Council. She noted
that Bahrain was a close ally of Saudi Arabia, the true diplomatic prize
for Israel.

``Its importance is mostly because it's an indication that the new
leadership in Saudi Arabia supports normalization,'' Ms. Fontenrose
said. ``Bahrain doesn't make a foreign policy move without Saudi
Arabia's express permission.''

Toby Matthiesen, a research fellow at St. Antony's College, Oxford
University, said that the Saudis ``could be testing the waters'' for
their own normalization with Israel. But, he noted, Saudi Arabia
participated in the 1948 Arab-Israeli war and has a long history of
support to the Palestinian cause, and its autocratic leaders are
sensitive to public opinion on such issues.

After Mohammed bin Salman, the powerful Saudi crown prince, appeared to
make an overture toward Israel in 2018 by saying in an interview that
Israelis ``have the right'' to a homeland, his father, King Salman,
quickly countered with a rare public rebuke.

Israeli experts hold little hope of normalization with Saudi Arabia as
long as King Salman retains power, saying that the country's custody of
Islamic holy sites complicates its ability to strike a deal with Israel
if the Palestinians see it as a betrayal.

``I hope I'm wrong, but I cannot imagine them signing a normalization of
relations with us,'' General Gilead said.

For now, Mr. Trump will happily settle for Bahrain, which will send
officials to a White House signing ceremony planned for Tuesday for the
Israel-Emirates agreement.

On Friday, Mr. Kushner called the Bahrain agreement ``a historic
breakthrough for the president and also for the world.'' Mr. Trump
boasted that ``things are happening in the Middle East that nobody
thought was even possible to think about.''

But Democrats and many Middle East analysts in Washington called such a
self-congratulatory tone hyperbolic, particularly given that Israel's
relations with the Gulf's Sunni Arab governments had been warming for
years, driven by a common animus toward Shiite Iran. (Bahrain is a
Shiite-majority state ruled by Sunni monarchs.)

``This latest agreement by itself is an encouraging sign of progress in
a region that has been racked with conflict and civil wars,'' said Brian
Katulis, a senior fellow and Middle East expert at the liberal Center
for American Progress. ``But it's hard to credit the Trump
administration with this deal.''

And Mr. Trump played a far less direct role than Mr. Kushner, who has
led the administration's effort to strike a peace deal between Israel
and the Palestinians. That project has largely been on pause since the
administration's release in January of a
\href{https://www.nytimes3xbfgragh.onion/2020/01/28/world/middleeast/peace-plan.html}{peace
plan} heavily slanted in Israel's favor that the Palestinians rejected
out of hand.

Since then, Mr. Kushner and other Trump officials have turned their
energies toward Israel's relations with other Arab countries, partly as
a means of showing the Palestinians that their demands would no longer
dictate the region's wider dynamics.

Speaking to reporters on Friday, Mr. Kushner said Bahrain's move would
``separate the Palestinian issue from their own national interests, from
their foreign policy, which should be focused on their domestic
priorities.''

Bahrain's decision indicated that the Arab world was abandoning the Arab
Peace Initiative of 2002, a proposal endorsed by the Arab League that
called on Israel to withdraw from occupied territories it had captured
in the 1967 Arab-Israeli conflict in return for normal relations with
Arab and Islamic countries, Palestinian analysts said.

``The Arab position that demands the establishment of an independent
Palestinian state before normalizing with Israel is collapsing,'' said
Jehad Harb, an analyst of Palestinian politics who is based in Ramallah,
in the West Bank. ``The Bahraini move is an affirmation of this new
reality.''

The Palestinians tried this week to persuade the Arab League to condemn
the United Arab Emirates, only to
\href{https://www.timesofisrael.com/in-blow-to-palestinians-arab-league-refuses-to-condemn-israel-uae-deal/\#gs.fvfqe9}{receive
a scolding from the group} --- traditionally at least a reliable
rhetorical backer of the Palestinian cause --- for meddling in the
``sovereign foreign policy decisions'' of its member states.

One Arab government official said the diplomatic steps by Bahrain and
the Emirates reflected no loss of sympathy among Gulf leaders for the
Palestinian cause itself, but they did signal a deepening impatience
with what they saw as a dysfunctional and intransigent Palestinian
leadership.

In a statement, the Palestinian leadership declared its ``strong
rejection and condemnation'' of the announcement, which it called ``a
betrayal of Jerusalem, the Aqsa Mosque and the Palestinian cause, as
well as support for legitimizing the Israeli occupation's crimes against
the Palestinian people.'' A Palestinian Authority official also
\href{https://www.facebookcorewwwi.onion/mofa.pna/posts/3502604999782610}{announced}
that the Palestinian ambassador to Bahrain was being recalled.

Israel and Bahrain have had unofficial ties on and off since the 1990s
and enjoyed warm relations for several years. In 2019, Bahrain played
host to a Trump administration conference promoting the economic aspects
of its proposal to resolve the Israeli-Palestinian conflict, during
which Sheikh Khalid, a member of the Bahraini royal family who is now a
diplomatic adviser to the king, gave friendly interviews to visiting
Israeli journalists. ``Israel is part of this heritage of this whole
region, historically,'' he said, adding that ``the Jewish people have a
place amongst us.''

Mr. Netanyahu announced the deal to his people as an event on par with
Israel's 1994 peace treaty with Jordan.

``It took us 26 years to reach the second peace agreement with an Arab
country for the third peace agreement, 26 years,'' he said. ``But 29
days to reach a peace agreement between the third Arab state and the
fourth Arab state, and there will be more.''

Acknowledging that the agreements did not come from out of the blue, Mr.
Netanyahu said they ``were made through hard work behind the scenes for
years'' but credited Mr. Trump for providing ``important help.''

Michael Crowley reported from Washington, and David M. Halbfinger from
Jerusalem. Adam Rasgon contributed reporting from Tel Aviv, Edward Wong
from Washington, and Declan Walsh from Cairo.

Advertisement

\protect\hyperlink{after-bottom}{Continue reading the main story}

\hypertarget{site-index}{%
\subsection{Site Index}\label{site-index}}

\hypertarget{site-information-navigation}{%
\subsection{Site Information
Navigation}\label{site-information-navigation}}

\begin{itemize}
\tightlist
\item
  \href{https://help.nytimes3xbfgragh.onion/hc/en-us/articles/115014792127-Copyright-notice}{©~2020~The
  New York Times Company}
\end{itemize}

\begin{itemize}
\tightlist
\item
  \href{https://www.nytco.com/}{NYTCo}
\item
  \href{https://help.nytimes3xbfgragh.onion/hc/en-us/articles/115015385887-Contact-Us}{Contact
  Us}
\item
  \href{https://www.nytco.com/careers/}{Work with us}
\item
  \href{https://nytmediakit.com/}{Advertise}
\item
  \href{http://www.tbrandstudio.com/}{T Brand Studio}
\item
  \href{https://www.nytimes3xbfgragh.onion/privacy/cookie-policy\#how-do-i-manage-trackers}{Your
  Ad Choices}
\item
  \href{https://www.nytimes3xbfgragh.onion/privacy}{Privacy}
\item
  \href{https://help.nytimes3xbfgragh.onion/hc/en-us/articles/115014893428-Terms-of-service}{Terms
  of Service}
\item
  \href{https://help.nytimes3xbfgragh.onion/hc/en-us/articles/115014893968-Terms-of-sale}{Terms
  of Sale}
\item
  \href{https://spiderbites.nytimes3xbfgragh.onion}{Site Map}
\item
  \href{https://help.nytimes3xbfgragh.onion/hc/en-us}{Help}
\item
  \href{https://www.nytimes3xbfgragh.onion/subscription?campaignId=37WXW}{Subscriptions}
\end{itemize}
