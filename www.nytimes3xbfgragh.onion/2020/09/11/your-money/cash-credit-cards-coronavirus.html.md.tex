Sections

SEARCH

\protect\hyperlink{site-content}{Skip to
content}\protect\hyperlink{site-index}{Skip to site index}

\href{https://www.nytimes3xbfgragh.onion/section/your-money}{Your Money}

\href{https://myaccount.nytimes3xbfgragh.onion/auth/login?response_type=cookie\&client_id=vi}{}

\href{https://www.nytimes3xbfgragh.onion/section/todayspaper}{Today's
Paper}

\href{/section/your-money}{Your Money}\textbar{}Who Gets Hurt When the
World Stops Using Cash

\url{https://nyti.ms/2RiI98I}

\begin{itemize}
\item
\item
\item
\item
\item
\end{itemize}

\hypertarget{the-coronavirus-outbreak}{%
\subsubsection{\texorpdfstring{\href{https://www.nytimes3xbfgragh.onion/news-event/coronavirus?name=styln-coronavirus-national\&region=TOP_BANNER\&block=storyline_menu_recirc\&action=click\&pgtype=Article\&impression_id=8c17d640-f52c-11ea-bf06-e35e969503ac\&variant=undefined}{The
Coronavirus
Outbreak}}{The Coronavirus Outbreak}}\label{the-coronavirus-outbreak}}

\begin{itemize}
\tightlist
\item
  live\href{https://www.nytimes3xbfgragh.onion/2020/09/12/world/covid-19-coronavirus.html?name=styln-coronavirus-national\&region=TOP_BANNER\&block=storyline_menu_recirc\&action=click\&pgtype=Article\&impression_id=8c17d641-f52c-11ea-bf06-e35e969503ac\&variant=undefined}{Latest
  Updates}
\item
  \href{https://www.nytimes3xbfgragh.onion/interactive/2020/us/coronavirus-us-cases.html?name=styln-coronavirus-national\&region=TOP_BANNER\&block=storyline_menu_recirc\&action=click\&pgtype=Article\&impression_id=8c17d642-f52c-11ea-bf06-e35e969503ac\&variant=undefined}{Maps
  and Cases}
\item
  \href{https://www.nytimes3xbfgragh.onion/interactive/2020/science/coronavirus-vaccine-tracker.html?name=styln-coronavirus-national\&region=TOP_BANNER\&block=storyline_menu_recirc\&action=click\&pgtype=Article\&impression_id=8c17fd50-f52c-11ea-bf06-e35e969503ac\&variant=undefined}{Vaccine
  Tracker}
\item
  \href{https://www.nytimes3xbfgragh.onion/2020/09/10/us/politics/fda-coronavirus-vaccine.html?name=styln-coronavirus-national\&region=TOP_BANNER\&block=storyline_menu_recirc\&action=click\&pgtype=Article\&impression_id=8c17fd51-f52c-11ea-bf06-e35e969503ac\&variant=undefined}{F.D.A.
  Regulators' Self-Defense}
\item
  \href{https://www.nytimes3xbfgragh.onion/2020/09/09/upshot/coronavirus-surprise-test-fees.html?name=styln-coronavirus-national\&region=TOP_BANNER\&block=storyline_menu_recirc\&action=click\&pgtype=Article\&impression_id=8c17fd52-f52c-11ea-bf06-e35e969503ac\&variant=undefined}{Surprise
  Test Fees}
\end{itemize}

Advertisement

\protect\hyperlink{after-top}{Continue reading the main story}

Supported by

\protect\hyperlink{after-sponsor}{Continue reading the main story}

your money adviser

\hypertarget{who-gets-hurt-when-the-world-stops-using-cash}{%
\section{Who Gets Hurt When the World Stops Using
Cash}\label{who-gets-hurt-when-the-world-stops-using-cash}}

Some people don't have credit or debit cards, so a growing number of
state and local governments are requiring businesses to accept cash.

\includegraphics{https://static01.graylady3jvrrxbe.onion/images/2020/09/11/business/11Adviser-illo/11Adviser-illo-articleLarge.jpg?quality=75\&auto=webp\&disable=upscale}

By \href{https://www.nytimes3xbfgragh.onion/by/ann-carrns}{Ann Carrns}

\begin{itemize}
\item
  Sept. 11, 2020
\item
  \begin{itemize}
  \item
  \item
  \item
  \item
  \item
  \end{itemize}
\end{itemize}

Cash doesn't have the status it used to.

In fact, some state and local governments are forcing businesses like
restaurants and retail shops to continue accepting cash --- concerned
that cashless businesses effectively discriminate against consumers who
do not have bank accounts or credit cards.

New York City will require most stores and restaurants to accept cash as
of Nov. 19, joining cities including San Francisco; Berkeley, Calif.;
and Philadelphia, all of which mandated acceptance of cash last year.
New Jersey required acceptance of cash statewide in 2019, and it has
been illegal for businesses to refuse cash in Massachusetts for decades.
Many other cities and states are
\href{https://www.capradio.org/articles/2020/05/12/as-more-california-businesses-go-cashless-during-pandemic-lawmaker-continues-push-to-ban-the-practice/}{considering}similar
steps.

Concerns about a decline in the acceptance of cash surfaced well before
the coronavirus arrived, as consumers grew more comfortable shopping
online with credit or debit cards and paying quickly with mobile apps.
Many businesses like electronic payments because they speed up purchases
and reduce concern about theft.

Then, during the pandemic, restaurants and stores emphasized online
ordering and digital payment to reduce interactions, and the risk of
infection, among customers and employees. And as consumers stayed home,
\href{https://www.nytimes3xbfgragh.onion/2020/06/25/business/economy/coin-shortage-coronavirus.html}{coin
shortages}occurred, making it difficult for some stores to give change.
That added to a preference for electronic payments.

``The concern has been heightened as a result of the pandemic,'' said
Susan Grant, director of consumer protection and privacy at the Consumer
Federation of America, a nonprofit advocacy group.

\hypertarget{latest-updates-the-coronavirus-outbreak}{%
\section{\texorpdfstring{\href{https://www.nytimes3xbfgragh.onion/2020/09/11/world/covid-19-coronavirus.html?action=click\&pgtype=Article\&state=default\&region=MAIN_CONTENT_1\&context=storylines_live_updates}{Latest
Updates: The Coronavirus
Outbreak}}{Latest Updates: The Coronavirus Outbreak}}\label{latest-updates-the-coronavirus-outbreak}}

Updated 2020-09-12T12:04:20.515Z

\begin{itemize}
\tightlist
\item
  \href{https://www.nytimes3xbfgragh.onion/2020/09/11/world/covid-19-coronavirus.html?action=click\&pgtype=Article\&state=default\&region=MAIN_CONTENT_1\&context=storylines_live_updates\#link-dfb8a16}{Fauci
  cautions the virus could disrupt life in the U.S. until `maybe even
  towards the end of 2021.'}
\item
  \href{https://www.nytimes3xbfgragh.onion/2020/09/11/world/covid-19-coronavirus.html?action=click\&pgtype=Article\&state=default\&region=MAIN_CONTENT_1\&context=storylines_live_updates\#link-7104d154}{From
  Asia to Africa, China promotes its vaccine candidates to win friends.}
\item
  \href{https://www.nytimes3xbfgragh.onion/2020/09/11/world/covid-19-coronavirus.html?action=click\&pgtype=Article\&state=default\&region=MAIN_CONTENT_1\&context=storylines_live_updates\#link-393ad215}{The
  other way the virus will kill: hunger.}
\end{itemize}

\href{https://www.nytimes3xbfgragh.onion/2020/09/11/world/covid-19-coronavirus.html?action=click\&pgtype=Article\&state=default\&region=MAIN_CONTENT_1\&context=storylines_live_updates}{See
more updates}

More live coverage:
\href{https://www.nytimes3xbfgragh.onion/live/2020/09/11/business/stock-market-today-coronavirus?action=click\&pgtype=Article\&state=default\&region=MAIN_CONTENT_1\&context=storylines_live_updates}{Markets}

But as digital payments become more widespread, ``we're concerned that
people aren't going to be able to pay for necessities,'' said Linda
Sherry, director of national priorities at Consumer Action, an advocacy
group.

Businesses that refuse cash put at a disadvantage people who lack
traditional bank accounts or can't qualify for credit cards, consumer
advocates say. About one-fourth of American adults were unbanked or
underbanked in 2019 --- meaning they lacked a bank account or had one
but also used alternatives like check-cashing services, the
\href{https://www.federalreserve.gov/publications/files/2019-report-economic-well-being-us-households-202005.pdf}{Federal
Reserve}found. Those consumers are more likely to be in a racial or
ethnic minority group, have lower incomes and be less educated.

Some may like cash because it helps them budget their money or teach
their children about spending. Others may be wary of a loss of privacy
and vulnerability to hacking with electronic payments, or simply prefer
cash, Ms. Grant said. ``The decision should be the consumer's.''

The federation and dozens of other advocacy and privacy rights groups
are backing
\href{https://consumerfed.org/wp-content/uploads/2020/09/Support_Payment_Choice_Act.pdf}{federal
legislation} that would prohibit brick-and-mortar retailers from
refusing to accept cash. (It's unclear if the bill will be considered
this year, given the menu of pandemic-related issues before Congress.)

Consumers still use cash for more than one-quarter of all payments,
according to \href{https://www.frbsf.org/cash/publications/}{Federal
Reserve} data from October, its latest comprehensive study of payment
behavior. Cash was used for almost half of the payments under \$10.

In a narrower Fed survey in April and May, aimed at spotting payment
changes during the pandemic, 70 percent of participants said they were
not avoiding cash because of concern about the virus.

Cash remains important to consumers despite a menu of competing payment
options. ``Many consumers value and prefer to use cash for everyday
purchases, while others use cash as a backup, or for the convenience of
small value payments,'' Mark Gould, chief operating officer of the
Federal Reserve Bank of San Francisco, said in a
\href{https://www.frbsf.org/our-district/press/news-releases/2020/fed-report-shows-consumers-are-keeping-more-cash-on-hand-during-the-covid-19-pandemic/}{statement}
last month that accompanied the narrowed Fed survey.

Shelle Santana, a visiting scholar at Harvard Business School who has
studied payment trends, said it was unclear how aggressive the
enforcement of the cash requirements had been during the pandemic. She
said she foresaw a ``less cash'' society, rather than a truly cashless
one, in the near term, since many people continue to rely on hard
currency.

Some businesses that stopped accepting cash have reversed their policies
voluntarily, Ms. Santana noted, after realizing they were excluding some
customers.

``No one,'' she said, ``wants to turn away business.''

Here are some questions and answers about paying with cash:

\textbf{Is it legal to refuse to accept cash?}

There is no federal requirement that businesses accept cash or coins as
payment, according to the
\href{https://www.federalreserve.gov/faqs/currency_12772.htm}{Federal
Reserve Board}. ``Private businesses are free to develop their own
policies on whether or not to accept cash'' unless state law says
otherwise, the board explains on its website. Businesses like movie
theaters, convenience stores and gas stations may refuse to accept bills
over \$20, and bus lines may ban payment of fares in pennies, the
\href{https://www.treasury.gov/resource-center/faqs/currency/pages/legal-tender.aspx}{Treasury
Department} says.

\textbf{How will New York City enforce its cash requirement?}

The city's Department of Consumer Affairs is responsible for enforcing
the new rule, which was enacted
\href{https://legistar.council.nyc.gov/LegislationDetail.aspx?ID=3763665\&GUID=7800AFC9-D8B1-41FD-9C31-172565712686\&Options=ID\%7CText\%7C\&Search=cashless}{this
year}. The department said that enforcement would be based on complaints
and that it would issue instructions for filing a complaint before the
rule took effect. Businesses that fail to comply may face fines of up to
\$1,000 for the first violation and \$1,500 for subsequent violations.

The rule has some exceptions. For instance, a business can decline cash
if it offers a cash conversion kiosk, which transfers the cash value to
a debit card and is sometimes called a ``reverse A.T.M.,'' if the
machine meets certain criteria.

\textbf{Is it safe to pay with cash during the pandemic?}

The virus that causes Covid-19 is mainly transmitted through close
person-to-person contact, the Centers for Disease Control and Prevention
says. It's possible that someone could become infected by touching a
surface or an object with the virus on it, but ``this is not thought to
be the main way the virus spreads,'' according to
\href{https://www.cdc.gov/coronavirus/2019-ncov/prevent-getting-sick/how-covid-spreads.html}{the
agency}.

While there has been concern that handling cash can
\href{https://apnews.com/7f96f57e3c327e7dbe2be79f2bbdd543}{spread
germs}, touching a payment terminal or handling a plastic card to a
clerk may also pose a risk. The
\href{https://www.cdc.gov/coronavirus/2019-ncov/daily-life-coping/essential-goods-services.html}{C.D.C.}suggests
using touchless payment if possible. ``If you must handle money, a card
or use a keypad, use hand sanitizer right after paying,'' it says.

The World Health Organization
\href{https://www.nytimes3xbfgragh.onion/2020/03/27/learning/what-questions-do-you-have-about-the-coronavirus.html}{has
said} that ``it is good hygiene practice to wash your hands after
handling money, especially if eating or handling food.''

Advertisement

\protect\hyperlink{after-bottom}{Continue reading the main story}

\hypertarget{site-index}{%
\subsection{Site Index}\label{site-index}}

\hypertarget{site-information-navigation}{%
\subsection{Site Information
Navigation}\label{site-information-navigation}}

\begin{itemize}
\tightlist
\item
  \href{https://help.nytimes3xbfgragh.onion/hc/en-us/articles/115014792127-Copyright-notice}{©~2020~The
  New York Times Company}
\end{itemize}

\begin{itemize}
\tightlist
\item
  \href{https://www.nytco.com/}{NYTCo}
\item
  \href{https://help.nytimes3xbfgragh.onion/hc/en-us/articles/115015385887-Contact-Us}{Contact
  Us}
\item
  \href{https://www.nytco.com/careers/}{Work with us}
\item
  \href{https://nytmediakit.com/}{Advertise}
\item
  \href{http://www.tbrandstudio.com/}{T Brand Studio}
\item
  \href{https://www.nytimes3xbfgragh.onion/privacy/cookie-policy\#how-do-i-manage-trackers}{Your
  Ad Choices}
\item
  \href{https://www.nytimes3xbfgragh.onion/privacy}{Privacy}
\item
  \href{https://help.nytimes3xbfgragh.onion/hc/en-us/articles/115014893428-Terms-of-service}{Terms
  of Service}
\item
  \href{https://help.nytimes3xbfgragh.onion/hc/en-us/articles/115014893968-Terms-of-sale}{Terms
  of Sale}
\item
  \href{https://spiderbites.nytimes3xbfgragh.onion}{Site Map}
\item
  \href{https://help.nytimes3xbfgragh.onion/hc/en-us}{Help}
\item
  \href{https://www.nytimes3xbfgragh.onion/subscription?campaignId=37WXW}{Subscriptions}
\end{itemize}
