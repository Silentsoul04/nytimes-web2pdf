Sections

SEARCH

\protect\hyperlink{site-content}{Skip to
content}\protect\hyperlink{site-index}{Skip to site index}

\href{https://www.nytimes3xbfgragh.onion/section/politics}{Politics}

\href{https://myaccount.nytimes3xbfgragh.onion/auth/login?response_type=cookie\&client_id=vi}{}

\href{https://www.nytimes3xbfgragh.onion/section/todayspaper}{Today's
Paper}

\href{/section/politics}{Politics}\textbar{}Targeting Conservative
Terrain, Democrats Look to Independent Challengers in Alaska

\url{https://nyti.ms/3hrsOx7}

\begin{itemize}
\item
\item
\item
\item
\item
\end{itemize}

Advertisement

\protect\hyperlink{after-top}{Continue reading the main story}

Supported by

\protect\hyperlink{after-sponsor}{Continue reading the main story}

\hypertarget{targeting-conservative-terrain-democrats-look-to-independent-challengers-in-alaska}{%
\section{Targeting Conservative Terrain, Democrats Look to Independent
Challengers in
Alaska}\label{targeting-conservative-terrain-democrats-look-to-independent-challengers-in-alaska}}

The bids of two congressional candidates are shaping up as crucial tests
of whether a centrist label can overcome resistance to Democrats in a
conservative-leaning state.

\includegraphics{https://static01.graylady3jvrrxbe.onion/images/2020/09/13/us/politics/13dc-alaska1/merlin_176868747_253a3353-557a-4e03-afd6-e1f905d42ca2-articleLarge.jpg?quality=75\&auto=webp\&disable=upscale}

\href{https://www.nytimes3xbfgragh.onion/by/carl-hulse}{\includegraphics{https://static01.graylady3jvrrxbe.onion/images/2018/06/14/multimedia/author-carl-hulse/author-carl-hulse-thumbLarge.png}}

By \href{https://www.nytimes3xbfgragh.onion/by/carl-hulse}{Carl Hulse}

\begin{itemize}
\item
  Sept. 12, 2020, 11:27 a.m. ET
\item
  \begin{itemize}
  \item
  \item
  \item
  \item
  \item
  \end{itemize}
\end{itemize}

PALMER, Alaska --- Reaching into remote territory where they usually
have little chance of victory, Democrats are mounting serious efforts to
pick up Senate and House conservative-leaning seats here, reflecting
rising hopes about their chances of winning control of the Senate and
tightening their grip on the House.

And the party doesn't even have candidates on the November ballot.

The candidates challenging incumbent Republicans, Senator Dan Sullivan
and Representative Don Young, who is currently the longest-serving
member of the House, are both running as independents. Their bids, once
viewed as long shots, have become increasingly competitive in recent
weeks, and are shaping up as crucial tests of whether a centrist label
can overcome resistance to Democrats in a conservative-leaning state
that has been rocked by the economic consequences of the coronavirus
pandemic.

That steep downturn --- and a grim national political environment for
Republicans that has tracked with President Trump's sagging approval
ratings for much of the year --- have helped put the challengers, Al
Gross in the Senate contest and Alyse Galvin in the House fight, in
striking distance of incumbents who were once considered safe. Those
candidates say that, if elected, their lack of partisan allegiance will
allow them to focus more on what is best for the state in a Congress in
which there is rapidly dwindling tolerance for daylight between
lawmakers and their parties on policy matters.

``I would say I am not a very good Democrat and I'm not a very good
Republican,'' said Dr. Gross, a former orthopedic surgeon and commercial
fisherman who has not previously sought office but whose family has a
political pedigree in Alaska. ``Some of my values are in alignment with
the Democratic Party and some are in alignment with the Republican
Party. As a senator, I will always do what I think is best for the
state, irrespective of partisanship.''

Their opponents are toiling to tie them as closely as possible to
Democrats, noting that both Dr. Gross and Ms. Galvin have been endorsed
by the party's leaders in Washington and have said they would align with
their caucuses in Congress if elected.

Republicans call the centrist label a charade intended to distract
voters from the fact that if Democrats swept into power, the two
independents would be members of Democratic majorities led by Speaker
Nancy Pelosi and Senator Chuck Schumer of New York.

``I think they are absolutely pretending,'' Mr. Sullivan said last
Sunday as a largely mask-eschewing crowd enthusiastically greeted him at
an agriculture fair in the conservative Mat-Su Valley about 40 miles
outside Anchorage. ``If people identify him with the policies and being
part of a Chuck Schumer Senate majority, I think they will reject it.''

``It is way to try to get votes,'' said Mr. Young, who called the
independent status ``nonsense.'' ``I think it is a little bit
dishonest.''

Image

``I would say I am not a very good Democrat and I'm not a very good
Republican,'' said Al Gross, who is running for the Senate.Credit...Ash
Adams for The New York Times

Image

``I think they are absolutely pretending,'' said Senator Dan Sullivan,
the Republican incumbent.Credit...Ash Adams for The New York Times

Ms. Galvin, an education activist who lost to Mr. Young two years ago,
is the latest to try to dethrone the irascible dean of the House, who
has repelled many challengers over his 24 terms. But polls show a close
race.

Ms. Galvin said that the terrible economy, crushing expense of health
care in Alaska and the erosion of Mr. Young's power because of committee
term limits and minority-party status have provided an opening to topple
the only person the vast majority of Alaskans have ever known as their
congressman. He was first elected in a special election in 1973.

``Unfortunately, he can no longer deliver,'' she said, adding that with
only a single House seat, Alaska cannot afford to surrender influence.
``There is only one voice there for us. It has to be somebody who is
energetic, passionate, knows Alaska.''

With most voters in Alaska unaffiliated with either party and state
residents famously embracing rugged individualism, the independent label
can be powerful and has bolstered candidates for governor in the past.
In 2016, Democrats began allowing independents to run in the party's
primaries, opening the door to candidacies like those of Ms. Galvin and
Dr. Gross, who both easily won the Democratic contests in August.

The congressional battle in Alaska is being waged against a backdrop of
a suffering state. Alaska's economy was in recession and hurting before
the onset of the pandemic, at times recording the highest unemployment
rate in the nation. Now, the virus and its consequences have devastated
three pillars of the state's economy: the oil industry, commercial
fisheries and tourism.

It was a lost summer for the state, which typically draws hundreds of
thousands of visitors to experience its sweeping vistas and abundant
wildlife --- many via cruise ship. The shutdown of the tourism industry
was painfully visible over Labor Day weekend as downtown Anchorage was
virtually empty, devoid of the usual throngs of people browsing at
souvenir shops and enjoying fresh seafood at craft breweries and
restaurants.

``It is like a ghost town,'' Ms. Galvin said as she sat outside her
aging motor home plastered with campaign signs.

Image

The congressional battle in Alaska is being waged against a backdrop of
a suffering state.Credit...Ash Adams for The New York Times

Image

The virus and its consequences have devastated the state's oil industry,
commercial fisheries and tourism.Credit...Ash Adams for The New York
Times

Dr. Gross, who said he had been drawn into the race by Mr. Sullivan's
vote to repeal the Affordable Care Act, said his opponent had failed to
come to grips with the depth of the state's economic plight.

``There is not a lot of hope for jobs and new opportunities here in
Alaska and Dan is not recognizing what is holding back the economy and
doing anything about it,'' said Dr. Gross, who wrote an op-ed asking,
``Where are the cranes?,'' referring to the sight of construction cranes
as a sign of economic growth. ``There is fricking construction all over
the place. There is none here.''

Mr. Sullivan conceded that Alaska had ``taken it on the chin'' and
agreed that the struggling economy could encourage some voters to
consider a change. But he said a Democratically controlled Washington
would be disastrous for his state, pointing to the Green New Deal, which
would restrict energy development.

Working to leverage his relationship with President Trump, Mr. Sullivan,
who remains a colonel in the Marine reserves, said he had won gains for
Alaska in Pentagon funding, federal law enforcement presence and energy
projects that would be endangered by a Democratic takeover and what he
called the party's ``anti-Alaska agenda.''

``The question for Alaskans in this election in my view is, `Do you have
a senator who is going to continue to fight it, or empower it?''' he
said.

In a state that prizes Alaskan heritage, the Senate candidates have both
sought to establish their bona fides. Dr. Gross, whose father was
attorney general in the 1970s, has emphasized a personal history that
includes killing a grizzly bear he says sneaked up on him as he was duck
hunting. Mr. Sullivan, a native of Ohio, opened his campaign with an ad
featuring his wife, Julie, a member of a native Athabascan family that
she says ``goes back generations with deep ties to our land and
culture.''

In the House race, Ms. Galvin, who lost to Mr. Young by six points in
2018 while running as an independent, has assembled a grass-roots
campaign, using her activist background to organize more than 150
virtual meetings with groups of Alaskans gathered in their homes.

Image

Ms. Galvin said the chance to have a representative with the standing to
influence Democrats was all the more reason to elect her to promote
Alaskan views on energy and other issues.Credit...Ash Adams for The New
York Times

Image

Mr. Young dismissed Ms. Galvin's claim that he had lost his clout in
Washington and said she would accomplish little if elected because she
lacked seniority.Credit...Ash Adams for The New York Times

Her campaign material is careful to note that she is a third-generation
Alaskan. She is promoting a ``cradle all-the-way-to career'' education
system and, noting that her grandmother had to ration prescription
drugs, a push to cut health care costs that are consistently among the
highest in the nation. She has raised more money than Mr. Young and
received more total votes in the primary than he did.

Mr. Young, 87, acknowledged that he was in a fight and had begun
behaving accordingly. He flew back to Washington last month during the
House's summer recess to vote in favor of a Democratic bill
\href{https://www.nytimes3xbfgragh.onion/2020/08/22/us/politics/usps-bill-congress-vote.html}{to
bolster the Postal Service}, one of only 26 Republicans to support it
over Mr. Trump's veto threat. (Mr. Young has been outspoken in defense
of Alaska's bypass mail program, which subsidizes shipments to the
state.)

But he dismissed his opponent's claim that he had lost his clout in
Washington, and said that Ms. Galvin would accomplish little if elected
because she lacked seniority.

``I have the expertise and the know-how,'' he said in an interview.
``She has never held elective office. Freshmen cannot deliver,'' he
added. ``You don't hang around that body that long and not make friends
on both sides of the aisle.''

As for how long he intends to remain in office, Mr. Young replied, ``God
will decide that, or the voters.''

In his new television ad titled ``Allegiance,'' Mr. Young argues that
Ms. Galvin would be beholden to Democrats, noting that she has been
endorsed by Mr. Pelosi, who he says would ``shut down Alaska.''

But Ms. Galvin said the influence she held with Democrats would give her
stronger standing to promote Alaskan views on energy and other issues.

``I will be in the majority as an independent, helping them to
understand Alaska,'' she said. ``Their ears will be open more.''

``Fresh voices right now are needed,'' she said.

Advertisement

\protect\hyperlink{after-bottom}{Continue reading the main story}

\hypertarget{site-index}{%
\subsection{Site Index}\label{site-index}}

\hypertarget{site-information-navigation}{%
\subsection{Site Information
Navigation}\label{site-information-navigation}}

\begin{itemize}
\tightlist
\item
  \href{https://help.nytimes3xbfgragh.onion/hc/en-us/articles/115014792127-Copyright-notice}{©~2020~The
  New York Times Company}
\end{itemize}

\begin{itemize}
\tightlist
\item
  \href{https://www.nytco.com/}{NYTCo}
\item
  \href{https://help.nytimes3xbfgragh.onion/hc/en-us/articles/115015385887-Contact-Us}{Contact
  Us}
\item
  \href{https://www.nytco.com/careers/}{Work with us}
\item
  \href{https://nytmediakit.com/}{Advertise}
\item
  \href{http://www.tbrandstudio.com/}{T Brand Studio}
\item
  \href{https://www.nytimes3xbfgragh.onion/privacy/cookie-policy\#how-do-i-manage-trackers}{Your
  Ad Choices}
\item
  \href{https://www.nytimes3xbfgragh.onion/privacy}{Privacy}
\item
  \href{https://help.nytimes3xbfgragh.onion/hc/en-us/articles/115014893428-Terms-of-service}{Terms
  of Service}
\item
  \href{https://help.nytimes3xbfgragh.onion/hc/en-us/articles/115014893968-Terms-of-sale}{Terms
  of Sale}
\item
  \href{https://spiderbites.nytimes3xbfgragh.onion}{Site Map}
\item
  \href{https://help.nytimes3xbfgragh.onion/hc/en-us}{Help}
\item
  \href{https://www.nytimes3xbfgragh.onion/subscription?campaignId=37WXW}{Subscriptions}
\end{itemize}
