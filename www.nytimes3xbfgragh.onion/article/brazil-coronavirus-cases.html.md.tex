Sections

SEARCH

\protect\hyperlink{site-content}{Skip to
content}\protect\hyperlink{site-index}{Skip to site index}

\href{https://www.nytimes3xbfgragh.onion/section/world/americas}{Americas}

\href{https://myaccount.nytimes3xbfgragh.onion/auth/login?response_type=cookie\&client_id=vi}{}

\href{https://www.nytimes3xbfgragh.onion/section/todayspaper}{Today's
Paper}

\href{/section/world/americas}{Americas}\textbar{}Coronavirus in Brazil:
What You Need to Know

\url{https://nyti.ms/3eeU0hD}

\begin{itemize}
\item
\item
\item
\item
\item
\end{itemize}

\hypertarget{the-coronavirus-outbreak}{%
\subsubsection{\texorpdfstring{\href{https://www.nytimes3xbfgragh.onion/news-event/coronavirus?name=styln-coronavirus-national\&region=TOP_BANNER\&block=storyline_menu_recirc\&action=click\&pgtype=Article\&impression_id=4ef19c10-f1d0-11ea-bd53-232e4c613195\&variant=undefined}{The
Coronavirus
Outbreak}}{The Coronavirus Outbreak}}\label{the-coronavirus-outbreak}}

\begin{itemize}
\tightlist
\item
  live\href{https://www.nytimes3xbfgragh.onion/2020/09/08/world/covid-19-coronavirus.html?name=styln-coronavirus-national\&region=TOP_BANNER\&block=storyline_menu_recirc\&action=click\&pgtype=Article\&impression_id=4ef1c320-f1d0-11ea-bd53-232e4c613195\&variant=undefined}{Latest
  Updates}
\item
  \href{https://www.nytimes3xbfgragh.onion/interactive/2020/us/coronavirus-us-cases.html?name=styln-coronavirus-national\&region=TOP_BANNER\&block=storyline_menu_recirc\&action=click\&pgtype=Article\&impression_id=4ef1c321-f1d0-11ea-bd53-232e4c613195\&variant=undefined}{Maps
  and Cases}
\item
  \href{https://www.nytimes3xbfgragh.onion/interactive/2020/science/coronavirus-vaccine-tracker.html?name=styln-coronavirus-national\&region=TOP_BANNER\&block=storyline_menu_recirc\&action=click\&pgtype=Article\&impression_id=4ef1c322-f1d0-11ea-bd53-232e4c613195\&variant=undefined}{Vaccine
  Tracker}
\item
  \href{https://www.nytimes3xbfgragh.onion/2020/09/02/your-money/eviction-moratorium-covid.html?name=styln-coronavirus-national\&region=TOP_BANNER\&block=storyline_menu_recirc\&action=click\&pgtype=Article\&impression_id=4ef1c323-f1d0-11ea-bd53-232e4c613195\&variant=undefined}{Eviction
  Moratorium}
\item
  \href{https://www.nytimes3xbfgragh.onion/interactive/2020/09/02/magazine/food-insecurity-hunger-us.html?name=styln-coronavirus-national\&region=TOP_BANNER\&block=storyline_menu_recirc\&action=click\&pgtype=Article\&impression_id=4ef1c324-f1d0-11ea-bd53-232e4c613195\&variant=undefined}{American
  Hunger}
\end{itemize}

Advertisement

\protect\hyperlink{after-top}{Continue reading the main story}

Supported by

\protect\hyperlink{after-sponsor}{Continue reading the main story}

\hypertarget{coronavirus-in-brazil-what-you-need-to-know}{%
\section{Coronavirus in Brazil: What You Need to
Know}\label{coronavirus-in-brazil-what-you-need-to-know}}

How did Brazil become a global epicenter of the outbreak? After seeming
to ease, is the virus making a comeback?

By Manuela Andreoni

\begin{itemize}
\item
  Sept. 1, 2020
\item
  \begin{itemize}
  \item
  \item
  \item
  \item
  \item
  \end{itemize}
\end{itemize}

\hypertarget{heres-what-you-need-to-know}{%
\subsubsection{Here's what you need to
know:}\label{heres-what-you-need-to-know}}

\begin{itemize}
\tightlist
\item
  \protect\hyperlink{link-6cf7a36a}{A world leader in infections and
  deaths.}
\item
  \protect\hyperlink{link-7aa8053f}{The president scorned the virus,
  then got it himself.}
\item
  \protect\hyperlink{link-1c5e3f23}{How did Brazil get here?}
\item
  \protect\hyperlink{link-6a723006}{What is the country doing to fight
  the outbreak?}
\item
  \protect\hyperlink{link-41ddb021}{What have been the consequences of
  Brazil's response?}
\end{itemize}

\includegraphics{https://static01.graylady3jvrrxbe.onion/images/2020/08/18/world/00brazil-explainer-new/merlin_174332442_11f6fd1f-a6a6-4e92-ac6f-8e741930dbb0-articleLarge.jpg?quality=75\&auto=webp\&disable=upscale}

\hypertarget{a-world-leader-in-infections-and-deaths}{%
\subsection{A world leader in infections and
deaths.}\label{a-world-leader-in-infections-and-deaths}}

Latin America became
\href{https://www.nytimes3xbfgragh.onion/2020/05/12/world/americas/latin-america-virus-death.html}{an
epicenter of the coronavirus pandemic} in May, driven by
\href{https://www.nytimes3xbfgragh.onion/interactive/2020/world/americas/brazil-coronavirus-cases.html}{Brazil's
ballooning caseload}, as the number of known infections in Europe fell.
Six months after its first known case, Brazil has had at least 3.9
million cases --- more than all of Europe --- and over 121,000 deaths.

The day after the country passed 100,000 deaths, President Jair
Bolsonaro posted a message on
\href{https://www.facebookcorewwwi.onion/jairmessias.bolsonaro/posts/2045837842231858}{Facebook}
defending his government's response to the virus. ``There was no
shortage of resources, equipment or medication,'' he wrote.

In early June, Brazil began averaging about 1,000 deaths per day from
Covid-19, joining the United States --- and later India --- as the
countries with the world's worst death toll.

Then came signs the spread of the virus was easing in Brazil.

By late August, the Ministry of Health said the weekly number of cases
had finally started to fall. Thirteen of the country's 26 states have
seen cases decrease; another seven, and the capital, Brasília, have been
stable.

``We need to see how the disease will behave in the next two or three
weeks to confirm there is a significant drop,'' Secretary of Health
Surveillance Arnaldo Medeiros told reporters.

The trend was confirmed by an Imperial College London study that put
Brazil's contagion rate below 1.0, local news media reported. That means
every 100 people infected with the virus are passing it on to fewer than
100 additional people, slowing the epidemic.

Now, numbers from Brazil's surveillance system on the hospitalization of
people with respiratory syndromes suggest that some cities might be
facing a new wave of the virus. Because the surveillance system does not
depend on testing for the virus, it can be quicker to spot trends.

The data suggests that, in some state capitals, like Rio de Janeiro, the
number of cases has plateaued at a high level. Other, like Maceió and
Palmas, are now reporting a similar or even a higher number of cases
then they had in their first wave.

Marcelo Gomes, who analyses data at Fiocruz, a government institute that
studies health care trends, says the numbers are a warning that the
recent fall in the number of Covid-19 cases should not be taken as
``carte blanche'' to resume normal life.

``The numbers only fell because we acted,'' he said. ``If we change our
attitude again the number of cases can go up again.''

\hypertarget{latest-updates-the-coronavirus-outbreak}{%
\section{\texorpdfstring{\href{https://www.nytimes3xbfgragh.onion/2020/09/08/world/covid-19-coronavirus.html?action=click\&pgtype=Article\&state=default\&region=MAIN_CONTENT_1\&context=storylines_live_updates}{Latest
Updates: The Coronavirus
Outbreak}}{Latest Updates: The Coronavirus Outbreak}}\label{latest-updates-the-coronavirus-outbreak}}

Updated 2020-09-08T12:22:35.182Z

\begin{itemize}
\tightlist
\item
  \href{https://www.nytimes3xbfgragh.onion/2020/09/08/world/covid-19-coronavirus.html?action=click\&pgtype=Article\&state=default\&region=MAIN_CONTENT_1\&context=storylines_live_updates\#link-46162376}{Trillions
  of dollars separate lawmakers' proposals for virus relief.}
\item
  \href{https://www.nytimes3xbfgragh.onion/2020/09/08/world/covid-19-coronavirus.html?action=click\&pgtype=Article\&state=default\&region=MAIN_CONTENT_1\&context=storylines_live_updates\#link-679303d7}{Nine
  drugmakers pledge to thoroughly vet any coronavirus vaccine.}
\item
  \href{https://www.nytimes3xbfgragh.onion/2020/09/08/world/covid-19-coronavirus.html?action=click\&pgtype=Article\&state=default\&region=MAIN_CONTENT_1\&context=storylines_live_updates\#link-1c973131}{`The
  lockdown killed my father': Farmer suicides add to India's virus
  misery.}
\end{itemize}

\href{https://www.nytimes3xbfgragh.onion/2020/09/08/world/covid-19-coronavirus.html?action=click\&pgtype=Article\&state=default\&region=MAIN_CONTENT_1\&context=storylines_live_updates}{See
more updates}

More live coverage:
\href{https://www.nytimes3xbfgragh.onion/live/2020/09/08/business/stock-market-today-coronavirus?action=click\&pgtype=Article\&state=default\&region=MAIN_CONTENT_1\&context=storylines_live_updates}{Markets}

Shopping malls and restaurants reopened and beaches started to draw
crowds again,

And so far, four states have allowed schools to reopen --- but mostly
private ones, which cater to less than 20 percent of children in a
state. Twelve states have announced plans to reopen at least some
schools by October.

In several cities, teachers have gone to court in an effort not to be
forced to return to classrooms while the virus is still spreading. In
Manaus, public-school teachers have protested the reopening of schools
after several people who went back to work tested positive for the
virus.

The virus has also spread to the vast countryside and in smaller towns.
Now, almost all of Brazil's more than 5,600 cities have reported
Covid-19 cases.

More than 90 percent of Brazil's cities lack intensive care units, and
more than half didn't have ventilators until February, according to a
study by Fiocruz. Some
\href{https://www1.folha.uol.com.br/cotidiano/2020/06/noves-estados-estao-com-mais-de-80-das-utis-lotadas.shtml}{hospital
systems have come close to running out}of intensive care beds.

\includegraphics{https://static01.graylady3jvrrxbe.onion/images/2017/01/29/podcasts/the-daily-album-art/the-daily-album-art-articleInline-v2.jpg?quality=75\&auto=webp\&disable=upscale}

\hypertarget{listen-to-the-daily-what-went-wrong-in-brazil}{%
\subsubsection{Listen to `The Daily': What Went Wrong in
Brazil}\label{listen-to-the-daily-what-went-wrong-in-brazil}}

Despite a history of success in public health crises, the country has
one of the world's worst coronavirus outbreaks.

transcript

Back to The Daily

bars

0:00/29:24

-29:24

transcript

\hypertarget{listen-to-the-daily-what-went-wrong-in-brazil-1}{%
\subsection{Listen to `The Daily': What Went Wrong in
Brazil}\label{listen-to-the-daily-what-went-wrong-in-brazil-1}}

\hypertarget{hosted-by-michael-barbaro-produced-by-rachel-quester-and-adizah-eghan-with-help-from-robert-jimison-and-sydney-harper-and-edited-by-liz-o-baylen-and-lisa-tobin}{%
\subsubsection{Hosted by Michael Barbaro; produced by Rachel Quester and
Adizah Eghan, with help from Robert Jimison and Sydney Harper; and
edited by Liz O. Baylen and Lisa
Tobin}\label{hosted-by-michael-barbaro-produced-by-rachel-quester-and-adizah-eghan-with-help-from-robert-jimison-and-sydney-harper-and-edited-by-liz-o-baylen-and-lisa-tobin}}

\hypertarget{despite-a-history-of-success-in-public-health-crises-the-country-has-one-of-the-worlds-worst-coronavirus-outbreaks}{%
\paragraph{Despite a history of success in public health crises, the
country has one of the world's worst coronavirus
outbreaks.}\label{despite-a-history-of-success-in-public-health-crises-the-country-has-one-of-the-worlds-worst-coronavirus-outbreaks}}

\begin{itemize}
\item
  michael barbaro\\
  From The New York Times, I'm Michael Barbaro. This is ``The Daily.''

  Today: Brazil has a long and distinguished history of navigating
  public health crises, until now. Ernesto Londoño on what went wrong
  with the coronavirus.

  It's Thursday, July 2.

  OK, Ernesto, as we're talking, Brazil is second only to the U.S. in
  cases of Covid-19. Where do we need to start to understand how we got
  here?
\item
  ernesto londoño\\
  Well, Michael, while we're seeing kind of the first concerns about
  coronavirus rippling beyond China, it was the farthest thing from
  Brazilians' minds. You know, it's the peak of summer in the southern
  hemisphere and especially in February ---
\item
  archived recording\\
  {[}CARNIVAL MUSIC{]}
\end{itemize}

ernesto londoño

People are in party mode.

\begin{itemize}
\tightlist
\item
  archived recording\\
  {[}CARNIVAL MUSIC{]}
\end{itemize}

ernesto londoño

You know, all across Brazil in the big cities, people are celebrating
Carnival, which turns into this weeks-long celebration.

You know, the beaches are thronged. There's block parties on the street
where everybody is dancing, and there's a lot of people kissing
strangers. Everybody is sort of a sweaty mess. There's a lot of heavy
drinking. And just about the time when people are starting to sort of
nurse their post-Carnival hangover in late February, the first case is
diagnosed in São Paolo, in Brazil's largest city. And the first case
involves a man who had traveled home from Italy. And I don't think there
were panic alarms that went off initially, but that quickly changed in
early March. And it changed as a result of an extraordinary trip.

The Brazilian president, Jair Bolsonaro, traveled with a large entourage
to Florida, where he dined in Mar-a-Lago with President Trump.

\begin{itemize}
\tightlist
\item
  archived recording (donald trump)\\
  He's doing a fantastic job, great job. Brazil loves him, and the
  U.S.A. loves him.
\end{itemize}

michael barbaro

And as they come back from that trip, a number of people who have been
part of that delegation start testing positive.

\begin{itemize}
\item
  archived recording 1\\
  The latest sign of the spread of the virus is it is reaching and
  threatening the most powerful offices in the world.
\item
  archived recording 2\\
  There are new virus concerns stemming from the meeting President Trump
  had with a delegation from Brazil in Florida.
\item
  archived recording 3\\
  This video shows Mr. Trump last weekend at Mar-a-Lago next to a man
  who has since tested positive for the virus. He is the press secretary
  for Brazil's president.
\item
  archived recording 4\\
  We now know that a dozen Brazilians who were there have tested
  positive for the virus.
\end{itemize}

ernesto londoño

So Brazil really becomes sort of consumed because their leadership, the
people running the country all of a sudden started falling sick. But
even as the virus is clearly within the halls of power, the president is
not expressing any alarm. Quite the contrary.

\begin{itemize}
\item
  archived recording (jair bolsonaro)\\
  {[}IN PORTUGUESE{]}
\item
  archived recording (translator)\\
  My obligation as head of state is to anticipate problems, to bring the
  truth to the Brazilian people. But this truth should not incite panic.
\end{itemize}

ernesto londoño

President Bolsonaro starts calling this virus something that the media
was obsessed with in order to bring down his popularity numbers.

\begin{itemize}
\tightlist
\item
  archived recording (jair bolsonaro)\\
  {[}IN PORTUGUESE{]}
\end{itemize}

ernesto londoño

He called it a fantasy. He said, this is a measly cold. He said, even if
I were to get this virus, because I have an athlete's background, I
would shake it off within days.

michael barbaro

Huh.

\begin{itemize}
\tightlist
\item
  archived recording (jair bolsonaro)\\
  {[}IN PORTUGUESE{]}
\end{itemize}

ernesto londoño

(PARAPHRASING PRESIDENT JAIR BOLSONARO) This is nothing to worry about,
and this is not something that justifies shutting the country down for.

And on the other hand, you had mayors and governors saying, actually, we
have to listen to the scientists. We have to shut down businesses. We
have to keep people home. We have to act now if we want to save lives.
But it was really hard to enforce when you had the president trying to
persuade Brazilians that this was all a mistake and that they shouldn't
be listening to these local leaders.

So what this did was it really polarized Brazilian society. You had
people who are very loyal to the president, essentially taking his side
and sort of feeling empowered not to be quarantined, not to have to stay
at home, not to have to give up their jobs. But on the other hand, you
had a lot of people who despise the president and who became very
worried.

So by late March, something really striking started happening across the
country in big cities.

\begin{itemize}
\tightlist
\item
  archived recording\\
  {[}PROTESTORS BANGING ON POTS{]}
\end{itemize}

ernesto londoño

People protesting by banging pots from their windows. It was the only
way that people who were actually taking this virus seriously and were
worried about it could make their voices viewed. So every night like
clockwork at about 8:30 p.m., I would hear from my window people banging
pots and screaming, out with Bolsonaro, out with Bolsonaro!

And it was a really striking sounded. It almost sounded primal. It was
like these voices piercing through the night, and voices that kind of
conveyed a degree of despair and anguish.

\begin{itemize}
\tightlist
\item
  archived recording\\
  {[}PROTESTORS BANGING ON POTS{]}
\end{itemize}

ernesto londoño

So that's when we start seeing that much of the country was not feeling
safe in Bolsonaro's hands at a moment of crisis.

michael barbaro

And did this approach from the president, from Bolsonaro, did it
surprise you?

ernesto londoño

Well, I think we've learned not to be terribly surprised by anything
that Bolsonaro does. As you might remember, he's a far-right populist
leader who has been very divisive ever since he was elected in 2018. But
it was very surprising that Brazil would be caught flat-footed in a
health care crisis of this magnitude. And the reason for that is that
the country has, in the past, risen to the challenge of very serious
health care challenges, and deployed its pretty robust and very
sophisticated public health care service to go after really complicated
problems with very innovative solutions.

I'll give you a couple of examples. Back in the `90s when the first
H.I.V. medicines were on the market and were allowing people to live
healthy and productive lives, these drugs were very expensive for people
in poor countries. And Brazil took a pretty maverick approach to this.
Brazil essentially challenged to the drug companies and said, we think
that this is a matter of human rights, and people should have access to
lifesaving medicine without having to fill your pockets for years on
end. So that argument was so powerful. It led to drug manufacturers
making concessions, and that led to these drugs becoming cheaper and
more widely available. And Brazil scored a pretty significant victory on
the world stage by taking what was a pretty bold stance at the time.

More recently, Brazil had to contend with the Zika crisis, which led to
babies being born with deformities that were very, very difficult to
manage. And once again, it threw everything it had in the way of
scientific expertise. And one of the most interesting solutions that
Brazil came up with was genetically modified mosquitoes. And the plan
was that by creating a genetically altered breed of mosquitoes, they
would be preventing the dangerous types of mosquitoes from reproducing,
and in doing so, sort of gradually stamp out Zika from areas where
people were catching it.

So in recent decades, Brazil has been regarded as a really top-tier
player when it came to standing up to really complicated health care
challenges and to rising to the challenge, even for a country that has
enormous problems. You know, lots of people live in poverty. Lots of
people don't have access to clean water. But when it came to saving
lives, Brazil has a proven track record of acting in a bold and decisive
way.

But this year, we've seen something very different.

michael barbaro

We'll be right back.

So Ernesto, what explains why Brazil's leader would take this
essentially denialist approach to the pandemic, especially in light of
Brazil's very long history of so aggressively confronting public health
crises?

ernesto londoño

Bolsonaro was elected as a typical populist, who took the reins of a
country that had been reeling from a really brutal economic recession
and was only starting to sort of sputter back to life on his watch. So I
think for him and his followers, the idea of an economic unraveling on
his watch, considering how polarizing a figure he is, was just ruinous.
I think he felt that if businesses shut down and jobs disappeared in
very significant numbers, his base of support would crumble. And I think
another element that might inform Bolsonaro's behavior is he's somebody
who has always looked up to President Trump for cues on how he should
respond to things.

michael barbaro

Interesting.

ernesto londoño

And President Trump was also taking the approach that this virus was not
that big a deal, that this was going to disappear by itself. So you
know, there's a striking similarity in how these two embattled leaders
are selling this crisis to their bases and to the broader audience that
listens to them.

michael barbaro

Right. And in both cases, and it sounds like especially for Bolsonaro, a
strong economy is the basis for his staying in office, for being a
leader. And a strong economy and a very strong reaction to this pandemic
are almost, by definition, incompatible. So in his mind, the greatest
threat to his power is an economy that starts to sputter and stop, not a
virus that may infect and start to kill the people of Brazil?

ernesto londoño

That's right, but there's another element at play here. President
Bolsonaro has been consumed by political scandals, pretty much from the
beginning of his administration. And in recent months, he's begun to
face some legislative and criminal investigations that have called into
question his ability to serve out his term. One of these involves an
investigation into a money laundering scheme that one of his sons is a
target for. And the president is also now being investigated by the
Supreme Court over his efforts to switch a police chief, in what his
former justice minister thought was an abuse of power and an effort to
shield his family members and allies from corruption investigations.

So as the virus really starts taking hold of the country, you're left
with a president who is also in a really politically precarious
situation, and who is clinging onto his hopes for a strong economy ---
an economy that won't go off the rails --- because he sees that as the
key to his political survival.

michael barbaro

So given all that, Ernesto, how does Bolsonaro's approach here play out
inside Brazil's public health system? What does it look like?

ernesto londoño

Well, you have this extraordinary split-screen reality. On the one hand,
you have the health minister going on television every night during
press conferences ---

\begin{itemize}
\tightlist
\item
  archived recording (luiz mandetta)\\
  (IN PORTUGUESE) There is nothing that will influence this response
  more than how the Brazilian society will behave in the next month or
  in a few days.
\end{itemize}

ernesto londoño

--- preaching the merits of social distancing, saying quarantines are
the only tried-and-tested tool we can throw at this virus right now,
people who can stay at home should stay at home, business shutdowns make
perfect sense.

\begin{itemize}
\tightlist
\item
  archived recording (luiz mandetta)\\
  (IN PORTUGUESE) We need to have focus, discipline and science.
\end{itemize}

ernesto londoño

So you essentially had a health minister who was adhering to the
conventional wisdom and the scientific consensus on what countries
should be doing.

\begin{itemize}
\tightlist
\item
  archived recording (luiz mandetta)\\
  (IN PORTUGUESE) So that we can get out of this situation together.
\end{itemize}

ernesto londoño

On the other hand, you had the president leaving the palace and joining
pro-government demonstrations.

\begin{itemize}
\tightlist
\item
  archived recording\\
  {[}CHEERING{]}
\end{itemize}

ernesto londoño

You had him shaking hands. He certainly wasn't wearing a mask at that
point.

And the only thing he's really expressing interest in as sort of a cure
for the virus is the anti-malaria drug hydroxychloroquine, which he goes
as far as ordering the armed forces to mass produce, even though there's
really no scientific consensus that this is a good idea. And there's
some signals that it could actually be dangerous for coronavirus
patients. So it became this unsustainable rift where people were asking
the health minister, how on earth should we expect Brazilians to listen
to what you're asking them to do when they're seeing their president
take the exact opposite approach?

And it reaches a breaking point in mid-April. The health minister, in
the midst of a rising epidemic that is starting to spiral out of
control, gets fired.

michael barbaro

Huh.

\begin{itemize}
\tightlist
\item
  archived recording (luiz mandetta)\\
  {[}IN PORTUGUESE{]}
\end{itemize}

ernesto londoño

You know, and on his way out, he delivered a pretty blistering
indictment of the president's handling of this. And he essentially said,
I stayed in as long as I could to try to keep Brazil on a responsible
path, to try to work within the margins of my authorities, but I can no
longer serve under this president because we are too far apart when it
comes to our vision and our values on this.

\begin{itemize}
\tightlist
\item
  archived recording (luiz mandetta)\\
  Thank you very much, and thank you very much to the ministry of
  health.
\end{itemize}

ernesto londoño

So with the first health minister getting tossed out, the president
appoints a new minister who's a physician, who had very little name
recognition and had never sort of run a large bureaucracy. He often
looks like he has a deer-in-the-headlights look. Brazilians started
making fun of him in memes online. It just never felt like he was
getting any traction or laying out a vision. And just short of
completing a month on the job, he convenes a press conference and says
---

\begin{itemize}
\tightlist
\item
  archived recording (nelson teich)\\
  (IN PORTUGUESE) Life is made of choices, and today I chose to leave.
  So you tell me if I didn't do my best during this phase, during this
  period.
\end{itemize}

ernesto londoño

(PARAPHRASING NELSON TEICH) This is as far as I can go.

\begin{itemize}
\tightlist
\item
  archived recording (nelson teich)\\
  (IN PORTUGUESE) Its not easy to be heading a secretariat like this in
  such a difficult period.
\end{itemize}

ernesto londoño

Doesn't really give a clear explanation for why he's leaving, but it's
pretty clear that he, too, just couldn't live with being the face of
this response that was being led by the president.

\begin{itemize}
\tightlist
\item
  archived recording (nelson teich)\\
  It was an honor for me to have been part of this. Thank you.
\end{itemize}

michael barbaro

So this is definitively not going well. You're churning through two
health ministers in the middle of a deadly pandemic.

ernesto londoño

Yeah, and the numbers are just spiraling.

\begin{itemize}
\item
  archived recording 1\\
  Brazil now has the most cases of coronavirus in South America with
  more than 5,800 confirmed Covid-19 cases and growing.
\item
  archived recording 2\\
  A new study out over the last couple of days showing that Brazil might
  have eight times more cases that have so far been recorded.
\item
  archived recording 3\\
  Brazil has officially reported about 4,500 deaths. The true number are
  believed to be much higher, due to the lack of testing.
\item
  archived recording 4\\
  Topping 90,000 confirmed cases and with more than 6,000 deaths.
\item
  archived recording 5\\
  100,000 with more than 7,000 deaths.
\item
  archived recording 6\\
  Brazil now has more confirmed cases of coronavirus deaths than China.
\end{itemize}

ernesto londoño

You know, at this point ---

\begin{itemize}
\tightlist
\item
  archived recording\\
  Well, as the number of coronavirus cases goes up in Brazil, so does
  the threat to communities in the Amazon region.
\end{itemize}

ernesto londoño

--- you start seeing real strain in some states.

\begin{itemize}
\tightlist
\item
  archived recording\\
  Already, the biggest city, Manaus, has seen its health system
  collapse.
\end{itemize}

ernesto londoño

Up in the Amazon, for example, grave diggers started digging mass graves
because people were dying so quickly that officials were completely
overwhelmed at hospitals and funeral homes.

\begin{itemize}
\tightlist
\item
  archived recording\\
  In Rio de Janeiro, hundreds of men, women, and children stood on a
  line to get food and water.
\end{itemize}

ernesto londoño

So ---

\begin{itemize}
\tightlist
\item
  archived recording\\
  Health systems across the country are struggling.
\end{itemize}

ernesto londoño

--- you know, across the country in some cities, panic really starts
setting in.

\begin{itemize}
\tightlist
\item
  archived recording\\
  But experts don't expect Brazil to reach the peak for a few weeks yet.
\end{itemize}

ernesto londoño

And at the national level, the health ministry is without a minister.
And instead of appointing another expert in the field, the president
leaves the ministry in the hands of an active duty army general who was
an expert in logistics but had no real track record when it came to
medicine. And one of the first things the ministry does when it is
essentially run by this army general is endorse this anti-malaria pill,
hydroxychloroquine, and say that health care professionals in the
country should give it to all coronavirus patients who want it at any
stage of contagion.

michael barbaro

So on his third try, he finally found a health minister willing to take
that position.

ernesto londoño

Absolutely. And he had a general on the job, and the general had to take
orders from the president.

michael barbaro

And has Bolsonaro backtracked at all as these infection numbers have
surged, as the death toll has risen? Or has he stayed with this same
approach as he started with?

ernesto londoño

There have been times when he has acknowledged that this is a very
significant problem and this is a crisis, but he has been very
consistent in saying that it is crucial to save the economy and to put
economic recovery ahead of fighting the virus.

At one point, when there had been sort of a milestone in the death toll,
somebody asked him outside the palace what he thought, and he said, so
what? What do you want me to do? My name might be Messiah, which was a
reference to his middle name, but I'm not here to perform miracles.
Earlier this month, Bolsonaro said, I regret the loss of life. But at
the end of the day, this is everybody's destiny.

michael barbaro

Everybody eventually dies is what he's saying.

ernesto londoño

Yes. So he hasn't really backtracked. And I think in the long run, what
some political experts think is that he is betting on the possibility
that when the real economic pain is felt --- six months, a year down the
road, when it becomes clear just how big of a beating Brazilians took
--- that Brazilians may take out their wrath on the governors and the
mayors who imposed these quarantines, and that the president may be able
to sort of carve out a role for himself as the person who consistently
wanted to save jobs, wanted to keep the economy on track. And I think
it's too early to tell. As a political strategy, that may pay dividends
down the road.

michael barbaro

Ernesto, I'm curious if you think that given Brazil's history of beating
back public health crises, that under different leadership --- not a
president like Bolsonaro who is skeptical of the science and who
explicitly puts economics over public health --- that Brazil would be in
a very different place right now?

ernesto londoño

Well, I think it bears watching what other countries in the region have
done. And there's cases where governments had very decisive and
well-thought-out responses that are wrestling with very high numbers. So
there's something kind of mystifying about where this virus strikes with
wrath, even in the face of a decisive and sophisticated response.

I think one common theme we're seeing that certainly applies to Brazil
is countries with very stark inequality have had a harder time reigning
the virus in. You have to think about kind of the way Brazilian cities
are laid out. You have many Brazilians living in impoverished, teeming
communities where people are sort of stacked up together, where they
live numerous families to a small dwelling. And this virus has
underscored the privilege some Brazilians have in adhering to
conventional social distancing norms and the extent to which, for many
Brazilians, for millions of them, it's just not a possibility.

But there's no question that Brazil had the expertise, had a track
record of responding to health care challenges in a really decisive way.
And it never really mounted a response that was coherent or
sophisticated. It's been mired in this political fight that has
prevented it from having a plan that makes sense to people, a plan it
can explain to its population.

At the end of the day, without a clear national policy, without a
political consensus, and without effective enforcement mechanisms for
some of these quarantines and lockdowns, there was no reining in the
virus.

But pretty much everybody I spoke to who has spent years working in
health care policy in Brazil said, we were equipped and ready and
trained to rise to the challenge. There was so much we could have done
in the precious early days of the epidemic to strike back, to prepare
and to save lives.

{[}music{]}

michael barbaro

Ernesto, thank you very much. We appreciate it.

ernesto londoño

My pleasure, Michael.

michael barbaro

As of Wednesday, the number of infections in Brazil has risen to 1.4
million, and the number of deaths has surpassed 60,000, confirming
Brazil's outbreak as the second worst in the world, after the United
States. We'll be right back.

Here's what else you need to know today.

\begin{itemize}
\tightlist
\item
  archived recording (bill de blasio)\\
  So I want to make very clear, we cannot go ahead at this point in time
  with indoor dining in New York City. Look, even a week ago, honestly,
  I was hopeful we could. But the news we have gotten from around the
  country gets worse and worse all the time.
\end{itemize}

michael barbaro

As U.S. infections continue to break records, New York City delayed a
plan to resume indoor dining, Miami Beach reinstated a curfew to keep
residents from congregating at night, and California shut down bars and
indoor dining in 19 counties. During a news conference on Wednesday,
California's governor Gavin Newsom said that he knew the decision would
be disappointing.

\begin{itemize}
\tightlist
\item
  archived recording (gavin newsom)\\
  And I deeply respect people's liberty, their desire to go back to the
  way things once were. But I cannot impress upon you more, our actions
  have an impact on other people.
\end{itemize}

michael barbaro

As of Wednesday night, the U.S. death toll from the virus neared
128,000.

\begin{itemize}
\tightlist
\item
  archived recording\\
  {[}PROTESTORS IN HONG KONG{]}
\end{itemize}

michael barbaro

A new national security law, imposed on Hong Kong by China, was put to
the test on Wednesday as thousands of protesters took to the streets
there, demanding greater freedom and independence from China.

The law, which went into effect on Tuesday, forbids a wide variety of
activities, including chanting slogans and carrying banners that China
considers seditious. At times, citing the new law, police arrested about
370 people, including a 15-year-old girl waving a flag, calling for Hong
Kong's independence.

``The Daily'' is made by Theo Balcomb, Andy Mills, Lisa Tobin, Rachel
Quester, Lynsea Garrison, Annie Brown, Clare Toeniskoetter, Paige
Cowett, Michael Simon Johnson, Brad Fisher, Larissa Anderson, Wendy
Dorr, Chris Wood, Jessica Cheung, Stella Tan, Alexandra Leigh Young,
Jonathan Wolfe, Lisa Chow, Eric Krupke, Mark George, Luke Vander Ploeg,
Adizah Eghan, Kelly Prime, Julia Longoria, Sindhu Gnanasambandan, M.K.
Davis Lin, Austin Mitchell, Sayre Quevedo, Neena Pathak, Dan Powell,
Dave Shaw, Sydney Harper, Daniel Guillemette, Hans Buetow, Robert
Jimison, Mike Benoist, Bianca Giaever, Asthaa Chaturvedi and Rachelle
Bonja. Our theme music is by Jim Brunberg and Ben Landsverk of Wonderly.
Special thanks to Sam Dolnick, Mikayla Bouchard, Lauren Jackson, Julia
Simon, Mahima Chablani, Nora Keller and Lis Moriconi.

That's it for ``The Daily.'' I'm Michael Barbaro. See you on Monday
after the holiday.

The country's response to the crisis
\href{https://www.nytimes3xbfgragh.onion/2020/07/02/podcasts/the-daily/brazil-coronavirus.html}{has
been widely criticized at home and abroad}. Mr. Bolsonaro has
\href{https://www.nytimes3xbfgragh.onion/2020/04/01/world/americas/brazil-bolsonaro-coronavirus.html}{dismissed
the danger} posed by the virus,
\href{https://www.nytimes3xbfgragh.onion/2020/05/16/world/americas/virus-brazil-deaths.html}{sabotaged
quarantine measures} adopted at the state level and called on Brazilians
to continue working to keep the economy from collapsing.

\hypertarget{the-president-scorned-the-virus-then-got-it-himself}{%
\subsection{The president scorned the virus, then got it
himself.}\label{the-president-scorned-the-virus-then-got-it-himself}}

Image

In a Facebook video, President Jair Bolsonaro showed he was taking
hydroxychloroquine after testing positive for the
coronavirus.Credit...via Facebook

The president disclosed on July 7 that he had the coronavirus,
turbocharging the debate over his
\href{https://www.nytimes3xbfgragh.onion/2020/04/01/world/americas/brazil-bolsonaro-coronavirus.html?searchResultPosition=1}{cavalier
handling} of the pandemic.

Mr. Bolsonaro, 65, said he had taken a test after experiencing fatigue,
muscle pain and a fever. He said the demands of his job had put him at
risk.

Three days before he tested positive, Mr. Bolsonaro had attended a
luncheon hosted by the American ambassador in Brazil to celebrate the
Fourth of July. In photos, participants are seen standing
shoulder-to-shoulder or embracing, without wearing masks.

During the time he was sick, Mr. Bolsonaro was seen outside without a
mask and
\href{https://www1.folha.uol.com.br/poder/2020/07/com-covid-19-bolsonaro-passeia-de-moto-e-conversa-sem-mascara-com-garis-no-alvorada.shtml}{talking
to other people}on more than one occasion. Three weeks after revealing
he had been infected, Mr. Bolsonaro said he had tested negative for the
virus and
\href{https://twitter.com/jairbolsonaro/status/1286994557440348160}{posted
a picture of himself} holding a box of hydroxychloroquine, an unproven
treatment against the virus.

\hypertarget{how-did-brazil-get-here}{%
\subsection{How did Brazil get here?}\label{how-did-brazil-get-here}}

Image

Passengers at the Luz Station in São Paulo in mid-June.~Credit...Victor
Moriyama for The New York Times

Brazil declared a
\href{https://www1.folha.uol.com.br/equilibrioesaude/2020/02/governo-decreta-estado-de-emergencia-por-causa-de-surto-do-coronavirus.shtml}{public
health emergency} in early March, and the Ministry of Health urged state
officials to cancel public events and put social-distancing measures in
place.

Some experts thought Brazil would be well equipped to rise to the
challenge, based on its
\href{https://www.nytimes3xbfgragh.onion/2020/05/16/world/americas/virus-brazil-deaths.html}{track
record in past public health emergencies}. It has a public health care
system that, while underfunded, provides robust coverage across the
country. And it had time to study the responses of the first countries
hit by the virus.

\href{https://www.nytimes3xbfgragh.onion/interactive/2020/world/americas/brazil-coronavirus-cases.html}{}

\includegraphics{https://static01.graylady3jvrrxbe.onion/images/2020/04/20/us/brazil-coronavirus-cases-promo-1587423141828/brazil-coronavirus-cases-promo-1587423141828-articleLarge-v116.png}

\hypertarget{brazil-coronavirus-map-and-case-count}{%
\subsection{Brazil Coronavirus Map and Case
Count}\label{brazil-coronavirus-map-and-case-count}}

A detailed map shows the extent of the coronavirus outbreak, with charts
and tables of the number of cases and deaths.

But Brazil's response quickly went awry, with efforts by state
governments to combat the virus often at odds with the positions adopted
by the president.

Mr. Bolsonaro pressured public health officials to do away with social
distancing recommendations, calling Covid-19 a ``measly cold.''

His response to the pandemic pitted him against governors and officials
at the Ministry of Health, who were urging people to stay home to the
extent possible.
\href{https://www.nytimes3xbfgragh.onion/2020/05/15/world/americas/brazil-health-minister-bolsonaro.html}{Two
health ministers} left after clashes with Mr. Bolsonaro, one having been
fired and one quitting, leaving a military general with no public health
training in charge of the response.

The conflicting messages from the government, experts say, left
Brazilians uncertain about the merits of isolation measures, which led
to low compliance.

\hypertarget{what-is-the-country-doing-to-fight-the-outbreak}{%
\subsection{What is the country doing to fight the
outbreak?}\label{what-is-the-country-doing-to-fight-the-outbreak}}

Image

A field hospital being built at Ibirapuera Stadium in São Paulo in
April.Credit...Victor Moriyama for The New York Times

The Ministry of Health has yet to present a comprehensive plan to fight
the virus. One of its main initiatives has been boosting the production
of
\href{https://www.nytimes3xbfgragh.onion/2020/05/20/world/coronavirus-world-tracker.html}{hydroxychloroquine}
and encouraging doctors in the public health care system to prescribe
it.

\href{https://www.nytimes3xbfgragh.onion/news-event/coronavirus?action=click\&pgtype=Article\&state=default\&region=MAIN_CONTENT_3\&context=storylines_faq}{}

\hypertarget{the-coronavirus-outbreak-}{%
\subsubsection{The Coronavirus Outbreak
›}\label{the-coronavirus-outbreak-}}

\hypertarget{frequently-asked-questions}{%
\paragraph{Frequently Asked
Questions}\label{frequently-asked-questions}}

Updated September 4, 2020

\begin{itemize}
\item ~
  \hypertarget{what-are-the-symptoms-of-coronavirus}{%
  \paragraph{What are the symptoms of
  coronavirus?}\label{what-are-the-symptoms-of-coronavirus}}

  \begin{itemize}
  \tightlist
  \item
    In the beginning, the coronavirus
    \href{https://www.nytimes3xbfgragh.onion/article/coronavirus-facts-history.html?action=click\&pgtype=Article\&state=default\&region=MAIN_CONTENT_3\&context=storylines_faq\#link-6817bab5}{seemed
    like it was primarily a respiratory illness}~--- many patients had
    fever and chills, were weak and tired, and coughed a lot, though
    some people don't show many symptoms at all. Those who seemed
    sickest had pneumonia or acute respiratory distress syndrome and
    received supplemental oxygen. By now, doctors have identified many
    more symptoms and syndromes. In April,
    \href{https://www.nytimes3xbfgragh.onion/2020/04/27/health/coronavirus-symptoms-cdc.html?action=click\&pgtype=Article\&state=default\&region=MAIN_CONTENT_3\&context=storylines_faq}{the
    C.D.C. added to the list of early signs}~sore throat, fever, chills
    and muscle aches. Gastrointestinal upset, such as diarrhea and
    nausea, has also been observed. Another telltale sign of infection
    may be a sudden, profound diminution of one's
    \href{https://www.nytimes3xbfgragh.onion/2020/03/22/health/coronavirus-symptoms-smell-taste.html?action=click\&pgtype=Article\&state=default\&region=MAIN_CONTENT_3\&context=storylines_faq}{sense
    of smell and taste.}~Teenagers and young adults in some cases have
    developed painful red and purple lesions on their fingers and toes
    --- nicknamed ``Covid toe'' --- but few other serious symptoms.
  \end{itemize}
\item ~
  \hypertarget{why-is-it-safer-to-spend-time-together-outside}{%
  \paragraph{Why is it safer to spend time together
  outside?}\label{why-is-it-safer-to-spend-time-together-outside}}

  \begin{itemize}
  \tightlist
  \item
    \href{https://www.nytimes3xbfgragh.onion/2020/05/15/us/coronavirus-what-to-do-outside.html?action=click\&pgtype=Article\&state=default\&region=MAIN_CONTENT_3\&context=storylines_faq}{Outdoor
    gatherings}~lower risk because wind disperses viral droplets, and
    sunlight can kill some of the virus. Open spaces prevent the virus
    from building up in concentrated amounts and being inhaled, which
    can happen when infected people exhale in a confined space for long
    stretches of time, said Dr. Julian W. Tang, a virologist at the
    University of Leicester.
  \end{itemize}
\item ~
  \hypertarget{why-does-standing-six-feet-away-from-others-help}{%
  \paragraph{Why does standing six feet away from others
  help?}\label{why-does-standing-six-feet-away-from-others-help}}

  \begin{itemize}
  \tightlist
  \item
    The coronavirus spreads primarily through droplets from your mouth
    and nose, especially when you cough or sneeze. The C.D.C., one of
    the organizations using that measure,
    \href{https://www.nytimes3xbfgragh.onion/2020/04/14/health/coronavirus-six-feet.html?action=click\&pgtype=Article\&state=default\&region=MAIN_CONTENT_3\&context=storylines_faq}{bases
    its recommendation of six feet}~on the idea that most large droplets
    that people expel when they cough or sneeze will fall to the ground
    within six feet. But six feet has never been a magic number that
    guarantees complete protection. Sneezes, for instance, can launch
    droplets a lot farther than six feet,
    \href{https://jamanetwork.com/journals/jama/fullarticle/2763852}{according
    to a recent study}. It's a rule of thumb: You should be safest
    standing six feet apart outside, especially when it's windy. But
    keep a mask on at all times, even when you think you're far enough
    apart.
  \end{itemize}
\item ~
  \hypertarget{i-have-antibodies-am-i-now-immune}{%
  \paragraph{I have antibodies. Am I now
  immune?}\label{i-have-antibodies-am-i-now-immune}}

  \begin{itemize}
  \tightlist
  \item
    As of right
    now,\href{https://www.nytimes3xbfgragh.onion/2020/07/22/health/covid-antibodies-herd-immunity.html?action=click\&pgtype=Article\&state=default\&region=MAIN_CONTENT_3\&context=storylines_faq}{~that
    seems likely, for at least several months.}~There have been
    frightening accounts of people suffering what seems to be a second
    bout of Covid-19. But experts say these patients may have a
    drawn-out course of infection, with the virus taking a slow toll
    weeks to months after initial exposure.~People infected with the
    coronavirus typically
    \href{https://www.nature.com/articles/s41586-020-2456-9}{produce}~immune
    molecules called antibodies, which are
    \href{https://www.nytimes3xbfgragh.onion/2020/05/07/health/coronavirus-antibody-prevalence.html?action=click\&pgtype=Article\&state=default\&region=MAIN_CONTENT_3\&context=storylines_faq}{protective
    proteins made in response to an
    infection}\href{https://www.nytimes3xbfgragh.onion/2020/05/07/health/coronavirus-antibody-prevalence.html?action=click\&pgtype=Article\&state=default\&region=MAIN_CONTENT_3\&context=storylines_faq}{.
    These antibodies may}~last in the body
    \href{https://www.nature.com/articles/s41591-020-0965-6}{only two to
    three months}, which may seem worrisome, but that's~perfectly normal
    after an acute infection subsides, said Dr. Michael Mina, an
    immunologist at Harvard University. It may be possible to get the
    coronavirus again, but it's highly unlikely that it would be
    possible in a short window of time from initial infection or make
    people sicker the second time.
  \end{itemize}
\item ~
  \hypertarget{what-are-my-rights-if-i-am-worried-about-going-back-to-work}{%
  \paragraph{What are my rights if I am worried about going back to
  work?}\label{what-are-my-rights-if-i-am-worried-about-going-back-to-work}}

  \begin{itemize}
  \tightlist
  \item
    Employers have to provide
    \href{https://www.osha.gov/SLTC/covid-19/standards.html}{a safe
    workplace}~with policies that protect everyone equally.
    \href{https://www.nytimes3xbfgragh.onion/article/coronavirus-money-unemployment.html?action=click\&pgtype=Article\&state=default\&region=MAIN_CONTENT_3\&context=storylines_faq}{And
    if one of your co-workers tests positive for the coronavirus, the
    C.D.C.}~has said that
    \href{https://www.cdc.gov/coronavirus/2019-ncov/community/guidance-business-response.html}{employers
    should tell their employees}~-\/- without giving you the sick
    employee's name -\/- that they may have been exposed to the virus.
  \end{itemize}
\end{itemize}

But while the hydroxychloroquine supply has increased, with Brazil's
army mass-producing it, essential medical supplies have been depleted
for months. The country has
\href{https://www.nytimes3xbfgragh.onion/2020/04/09/world/coronavirus-equipment-rich-poor.html}{struggled}
to import essential and lifesaving equipment, like coronavirus tests and
ventilators.

Experts believe Brazil's official coronavirus figures significantly
understate the scope of the problem --- an occurrence in many countries.

Though Brazil struggled to amass resources to respond to the Covid-19
crisis, a
\href{https://www1.folha.uol.com.br/cotidiano/2020/07/ministerio-da-saude-gastou-menos-de-13-da-verba-para-covid-diz-tcu.shtml}{recent
audit} by Congress found that the Ministry of Health used less than a
third of its emergency funds even as the country lacked tests and
hospital beds. Prosecutors are investigating whether officials were
negligent.

To ease the economic pain, Brazil offered monthly installments of \$120
to \$240 to people who had lost their income as a result of the
pandemic. That program has been marred by widespread allegations of
fraud and by the difficulty many eligible people experienced in gaining
access to the funds.

While it is unclear whether there will be additional payments,
\href{https://g1.globo.com/rn/rio-grande-do-norte/noticia/2020/08/21/bolsonaro-diz-que-auxilio-emergencial-vai-ser-ate-dezembro-so-nao-sei-o-valor.ghtml}{Mr.
Bolsonaro} has promised to continue making them until December, though
most likely for smaller amounts.

\hypertarget{what-have-been-the-consequences-of-brazils-response}{%
\subsection{What have been the consequences of Brazil's
response?}\label{what-have-been-the-consequences-of-brazils-response}}

Image

Residents of the Paraisópolis favela in São Paulo protested for better
health care in May.Credit...Victor Moriyama for The New York Times

Brazil's chaotic response has deepened political polarization among
supporters and critics of the president. Hospital systems have largely
coped with the crush of patients, but the virus has taken a devastating
toll on health care workers. Dozens of
\href{https://www.nytimes3xbfgragh.onion/2020/05/16/world/americas/virus-brazil-deaths.html}{nurses
and hospital technicians died} after contracting the virus at work.

In the deeply unequal country, some groups have been hit worse than
others. According to
a\href{https://covid19.ibge.gov.br/pnad-covid/}{recent survey} from the
Brazilian Institute of Geography and Statistics, Black people were twice
as likely as white people to have had Covid-19 symptoms. The study also
found that Black Brazilians were likelier to lose their jobs or face pay
cuts than white people during the pandemic.

The death rate in poorer cities has been substantially higher than in
rich ones.

In the Amazon region, which has long suffered a lack of resources and
government attention, The Times found that many health care workers
assigned to work with Indigenous groups
\href{https://www.nytimes3xbfgragh.onion/2020/07/19/world/americas/coronavirus-brazil-indigenous.html}{were
themselves carriers} of the virus. Without access to enough testing and
protective gear, they most likely exposed the populations they intended
to help.

The pandemic has exacerbated several of Brazil's chronic problems.
Deforestation in the Amazon rainforest has
\href{https://www.nytimes3xbfgragh.onion/2020/06/06/world/americas/amazon-deforestation-brazil.html}{surged},
and
\href{https://g1.globo.com/monitor-da-violencia/noticia/2020/06/17/em-plena-quarentena-brasil-tem-alta-de-8percent-no-numero-de-assassinatos-em-abril.ghtml}{homicide
rates} have gone up.

Brazil's economy is expected to contract by 6 percent this year, and
capital flight is reaching levels not seen since the 1990s, when the
country was grappling with hyperinflation.

And as Brazil reels from its worst crisis in decades, Mr. Bolsonaro and
his allies are using
\href{https://www.nytimes3xbfgragh.onion/2020/06/10/world/americas/bolsonaro-coup-coronavirus-brazil.html?smid=nytcore-ios-share}{the
prospect of military intervention} to protect his grip on power.

\hypertarget{brazil-and-the-race-for-a-vaccine}{%
\subsection{Brazil and the race for a
vaccine.}\label{brazil-and-the-race-for-a-vaccine}}

Image

Testing customers' temperatures in São Paulo in June.Credit...Victor
Moriyama for The New York Times

Brazil has emerged as
\href{https://www.nytimes3xbfgragh.onion/2020/08/15/world/americas/brazil-coronavirus-vaccine.html?smid=tw-share}{a
potentially vital player} in the global effort to find a vaccine.

Three of the most promising and advanced vaccine studies in the world
are relying on scientists and volunteers in Brazil, according to the
World Health Organization's report on the progress of vaccine research.
A \href{http://portal.anvisa.gov.br/estudos-clinicos-covid-19}{fourth
vaccine} trial is to start this month.

The country's role in research helped it cut two deals to get
preferential access to a vaccine if one is developed. One such agreement
is with China's Sinovac, for 120 million doses, and another is with
AstraZeneca, for 100 million doses year. Both vaccines would be produced
by Brazilian manufacturers.

Advertisement

\protect\hyperlink{after-bottom}{Continue reading the main story}

\hypertarget{site-index}{%
\subsection{Site Index}\label{site-index}}

\hypertarget{site-information-navigation}{%
\subsection{Site Information
Navigation}\label{site-information-navigation}}

\begin{itemize}
\tightlist
\item
  \href{https://help.nytimes3xbfgragh.onion/hc/en-us/articles/115014792127-Copyright-notice}{©~2020~The
  New York Times Company}
\end{itemize}

\begin{itemize}
\tightlist
\item
  \href{https://www.nytco.com/}{NYTCo}
\item
  \href{https://help.nytimes3xbfgragh.onion/hc/en-us/articles/115015385887-Contact-Us}{Contact
  Us}
\item
  \href{https://www.nytco.com/careers/}{Work with us}
\item
  \href{https://nytmediakit.com/}{Advertise}
\item
  \href{http://www.tbrandstudio.com/}{T Brand Studio}
\item
  \href{https://www.nytimes3xbfgragh.onion/privacy/cookie-policy\#how-do-i-manage-trackers}{Your
  Ad Choices}
\item
  \href{https://www.nytimes3xbfgragh.onion/privacy}{Privacy}
\item
  \href{https://help.nytimes3xbfgragh.onion/hc/en-us/articles/115014893428-Terms-of-service}{Terms
  of Service}
\item
  \href{https://help.nytimes3xbfgragh.onion/hc/en-us/articles/115014893968-Terms-of-sale}{Terms
  of Sale}
\item
  \href{https://spiderbites.nytimes3xbfgragh.onion}{Site Map}
\item
  \href{https://help.nytimes3xbfgragh.onion/hc/en-us}{Help}
\item
  \href{https://www.nytimes3xbfgragh.onion/subscription?campaignId=37WXW}{Subscriptions}
\end{itemize}
