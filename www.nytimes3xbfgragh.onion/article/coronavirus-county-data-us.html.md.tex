Sections

SEARCH

\protect\hyperlink{site-content}{Skip to
content}\protect\hyperlink{site-index}{Skip to site index}

\href{https://www.nytimes3xbfgragh.onion/section/us}{U.S.}

\href{https://myaccount.nytimes3xbfgragh.onion/auth/login?response_type=cookie\&client_id=vi}{}

\href{https://www.nytimes3xbfgragh.onion/section/todayspaper}{Today's
Paper}

\href{/section/us}{U.S.}\textbar{}We're Sharing Coronavirus Case Data
for Every U.S. County

\url{https://nyti.ms/2yeqKHZ}

\begin{itemize}
\item
\item
\item
\item
\item
\end{itemize}

\hypertarget{the-coronavirus-outbreak}{%
\subsubsection{\texorpdfstring{\href{https://www.nytimes3xbfgragh.onion/news-event/coronavirus?name=styln-coronavirus-national\&region=TOP_BANNER\&block=storyline_menu_recirc\&action=click\&pgtype=Article\&impression_id=606bee20-f4b7-11ea-ab3e-6b17423f7877\&variant=undefined}{The
Coronavirus
Outbreak}}{The Coronavirus Outbreak}}\label{the-coronavirus-outbreak}}

\begin{itemize}
\tightlist
\item
  live\href{https://www.nytimes3xbfgragh.onion/2020/09/11/world/covid-19-coronavirus.html?name=styln-coronavirus-national\&region=TOP_BANNER\&block=storyline_menu_recirc\&action=click\&pgtype=Article\&impression_id=606bee21-f4b7-11ea-ab3e-6b17423f7877\&variant=undefined}{Latest
  Updates}
\item
  \href{https://www.nytimes3xbfgragh.onion/interactive/2020/us/coronavirus-us-cases.html?name=styln-coronavirus-national\&region=TOP_BANNER\&block=storyline_menu_recirc\&action=click\&pgtype=Article\&impression_id=606c1530-f4b7-11ea-ab3e-6b17423f7877\&variant=undefined}{Maps
  and Cases}
\item
  \href{https://www.nytimes3xbfgragh.onion/interactive/2020/science/coronavirus-vaccine-tracker.html?name=styln-coronavirus-national\&region=TOP_BANNER\&block=storyline_menu_recirc\&action=click\&pgtype=Article\&impression_id=606c1531-f4b7-11ea-ab3e-6b17423f7877\&variant=undefined}{Vaccine
  Tracker}
\item
  \href{https://www.nytimes3xbfgragh.onion/2020/09/10/us/politics/fda-coronavirus-vaccine.html?name=styln-coronavirus-national\&region=TOP_BANNER\&block=storyline_menu_recirc\&action=click\&pgtype=Article\&impression_id=606c1532-f4b7-11ea-ab3e-6b17423f7877\&variant=undefined}{F.D.A.
  Regulators' Self-Defense}
\item
  \href{https://www.nytimes3xbfgragh.onion/2020/09/09/upshot/coronavirus-surprise-test-fees.html?name=styln-coronavirus-national\&region=TOP_BANNER\&block=storyline_menu_recirc\&action=click\&pgtype=Article\&impression_id=606c1533-f4b7-11ea-ab3e-6b17423f7877\&variant=undefined}{Surprise
  Test Fees}
\end{itemize}

Advertisement

\protect\hyperlink{after-top}{Continue reading the main story}

Supported by

\protect\hyperlink{after-sponsor}{Continue reading the main story}

\hypertarget{were-sharing-coronavirus-case-data-for-every-us-county}{%
\section{We're Sharing Coronavirus Case Data for Every U.S.
County}\label{were-sharing-coronavirus-case-data-for-every-us-county}}

With no detailed government database on where the thousands of
coronavirus cases have been reported, a team of New York Times
journalists is attempting to track every case.

By The New York Times

\begin{itemize}
\item
  March 28, 2020
\item
  \begin{itemize}
  \item
  \item
  \item
  \item
  \item
  \end{itemize}
\end{itemize}

Download county-level data for coronavirus cases in the United States
from The New York Times
\href{https://github.com/nytimes/covid-19-data}{on GitHub}.

As the coronavirus has spread across the United States, killing hundreds
of people and sickening tens of thousands more, comprehensive data on
the extent of the outbreak has been difficult to come by.

No single agency has provided the public with an accurate, up-to-date
record of coronavirus cases, tracked to the county level. To fill the
gap, The New York Times has launched a round-the-clock effort to tally
every known coronavirus case in the United States. The data, which The
Times will continue to track, is being made available to the public on
Friday.

Individual states and counties have tracked their own cases and
presented them to the public with varying degrees of speed and accuracy,
but those tallies provide only limited snapshots of the nation's
outbreak. A
\href{https://www.cdc.gov/coronavirus/2019-ncov/cases-updates/cases-in-us.html}{publicly
available tracker from the federal Centers for Disease Control and
Prevention}, updated five times a week, includes only state-level data.
Other entities have made efforts, including
\href{https://www.arcgis.com/apps/opsdashboard/index.html\#/bda7594740fd40299423467b48e9ecf6}{a
notable one by Johns Hopkins University}, to track cases worldwide or
\href{https://coronavirus.1point3acres.com/en}{within the United
States}.

In late January, not long after the first known case was reported in
Washington State, The Times began tracking each known U.S. case as
counties and states began reporting results of testing. Such testing,
which had been delayed, gradually became more widely available. For the
last eight weeks, a team of Times journalists has recorded an array of
details --- locations, dates, ages and conditions, when possible ---
about newly confirmed cases reported by state and local officials.

By Friday morning, The Times had tracked more than 85,000 cases in all
50 states, the District of Columbia and three U.S. territories. There
have been
\href{https://www.nytimes3xbfgragh.onion/2020/03/26/health/usa-coronavirus-cases.html}{more
known cases in the United States than in China, Italy or any other
country}, and more than 1,200 people have died in the United States.
Researchers, scientists, government officials and business executives
have requested access to the information. The Times is releasing its
data publicly in an effort to broaden understanding of the virus's toll.

``We hope the data set can help inform the ongoing public health
response to the pandemic and ultimately, save lives,'' said Dean Baquet,
the executive editor of The Times. ``We believe the data may help reveal
how Covid-19 has spread through communities and clusters; which
geographic areas may be hit the hardest; and how its spread in hard-hit
areas may offer clues for regions that could face wider outbreaks in the
future.''

The tracking has shown how quickly a single known case can mushroom into
an uncontrolled outbreak, as has happened in
\href{https://www.nytimes3xbfgragh.onion/2020/03/26/us/coronavirus-louisiana-new-orleans.html}{Louisiana}.

The database has shown how the detection of a small cluster in one area,
like New Rochelle, N.Y., can precede the discovery of thousands more
cases in nearby cities and states.

And it has shown, with tragic frequency, how vulnerable older adults are
to the worst of the virus. Public health officials have linked at least
37 deaths to the
\href{https://www.nytimes3xbfgragh.onion/2020/03/21/us/coronavirus-nursing-home-kirkland-life-care.html}{Life
Care nursing facility in Kirkland, Wash}. The Times has also tracked
outbreaks at other nursing and senior living facilities in Washington
State, Colorado, Florida and Louisiana.

The goal of the tracking was to compile a historical record,
\href{https://www.nytimes3xbfgragh.onion/interactive/2020/03/21/us/coronavirus-us-cases-spread.html}{with
as much detail about individual patients as can be obtained,}of the
largest public health crisis in modern American history. By collecting
the data continuously, and from multiple levels of government, The Times
has been able to
\href{https://www.nytimes3xbfgragh.onion/interactive/2020/us/coronavirus-us-cases.html}{map
the spread of the virus}, with updated information published several
times a day.

The Times's database has already informed academic research that showed
\href{https://www.nytimes3xbfgragh.onion/interactive/2020/03/20/us/coronavirus-model-us-outbreak.html}{how
the virus might be slowed}, and
\href{https://www.gainesville.com/opinion/20200325/stephen-j-hagen-and-peter-j-hirschfeld-florida-must-be-locked-down-now}{what
might happen} if it is not. In the weeks and months ahead, it is our
hope that the data can continue to inform researchers, policymakers and
journalists as they seek to understand how this pandemic evaded
containment, how it still might be mitigated and how similar disasters
might be avoided in the future.

The tracking effort grew from a handful of Times correspondents to a
large team of journalists that includes experts in data and graphics,
staff news assistants and freelance reporters, as well as journalism
students from Northwestern University, the University of Missouri and
the University of Nebraska-Lincoln. The reporting continues nearly all
day and night, seven days a week, across U.S. time zones, to record as
many details as possible about every case in real time. The Times is
committed to collecting as much data as possible in connection with the
outbreak and is collaborating with the University of California,
Berkeley, on an effort in that state.

In tracking the cases, the reporting process is labor-intensive but
straightforward much of the time. But with dozens of states and hundreds
of local health departments using their own reporting methods --- and
sometimes moving patients from county to county or state to state with
no explanation --- judgment calls have sometimes been required.

When the federal government arranged flights to the United States for
Americans exposed to the coronavirus in China and Japan, The Times
recorded their cases in the states where they subsequently were treated,
even though local health departments generally did not.

When a
\href{https://www.nytimes3xbfgragh.onion/2020/03/22/us/coronavirus-deaths-united-states.html}{resident
of Florida died in Los Angeles}, we recorded her death as having
occurred in California, though officials in Florida counted her case in
their own records. When officials in some states reported new cases
without immediately identifying where they were being treated, our team
attempted to add county information later, once it became available.

This accounting is an attempt to record how and where the coronavirus
has spread across the United States. But it is a product of a fragmented
American public health system in which overwhelmed officials at the
state or local level have sometimes struggled to report accurately and
consistently.

On several occasions, officials have corrected information hours or days
after they first reported it. Or cases have disappeared from a local
database. Or states have created confusing, sometimes duplicative ways
of tracking the cases of people who became ill while traveling. In those
instances, which have become more common as caseloads have increased,
The Times has sought to reflect the most current, accurate information
while ensuring that every known case,
\href{https://www.nytimes3xbfgragh.onion/2020/03/22/us/coronavirus-deaths-united-states.html}{each
of them representing someone's loved one}, is counted.

Advertisement

\protect\hyperlink{after-bottom}{Continue reading the main story}

\hypertarget{site-index}{%
\subsection{Site Index}\label{site-index}}

\hypertarget{site-information-navigation}{%
\subsection{Site Information
Navigation}\label{site-information-navigation}}

\begin{itemize}
\tightlist
\item
  \href{https://help.nytimes3xbfgragh.onion/hc/en-us/articles/115014792127-Copyright-notice}{©~2020~The
  New York Times Company}
\end{itemize}

\begin{itemize}
\tightlist
\item
  \href{https://www.nytco.com/}{NYTCo}
\item
  \href{https://help.nytimes3xbfgragh.onion/hc/en-us/articles/115015385887-Contact-Us}{Contact
  Us}
\item
  \href{https://www.nytco.com/careers/}{Work with us}
\item
  \href{https://nytmediakit.com/}{Advertise}
\item
  \href{http://www.tbrandstudio.com/}{T Brand Studio}
\item
  \href{https://www.nytimes3xbfgragh.onion/privacy/cookie-policy\#how-do-i-manage-trackers}{Your
  Ad Choices}
\item
  \href{https://www.nytimes3xbfgragh.onion/privacy}{Privacy}
\item
  \href{https://help.nytimes3xbfgragh.onion/hc/en-us/articles/115014893428-Terms-of-service}{Terms
  of Service}
\item
  \href{https://help.nytimes3xbfgragh.onion/hc/en-us/articles/115014893968-Terms-of-sale}{Terms
  of Sale}
\item
  \href{https://spiderbites.nytimes3xbfgragh.onion}{Site Map}
\item
  \href{https://help.nytimes3xbfgragh.onion/hc/en-us}{Help}
\item
  \href{https://www.nytimes3xbfgragh.onion/subscription?campaignId=37WXW}{Subscriptions}
\end{itemize}
