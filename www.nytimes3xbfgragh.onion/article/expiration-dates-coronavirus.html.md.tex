Sections

SEARCH

\protect\hyperlink{site-content}{Skip to
content}\protect\hyperlink{site-index}{Skip to site index}

\href{https://www.nytimes3xbfgragh.onion/section/food}{Food}

\href{https://myaccount.nytimes3xbfgragh.onion/auth/login?response_type=cookie\&client_id=vi}{}

\href{https://www.nytimes3xbfgragh.onion/section/todayspaper}{Today's
Paper}

\href{/section/food}{Food}\textbar{}The Food Expiration Dates You Should
Actually Follow

\url{https://nyti.ms/2Voi2ii}

\begin{itemize}
\item
\item
\item
\item
\item
\item
\end{itemize}

\href{https://www.nytimes3xbfgragh.onion/spotlight/at-home?action=click\&pgtype=Article\&state=default\&region=TOP_BANNER\&context=at_home_menu}{At
Home}

\begin{itemize}
\tightlist
\item
  \href{https://www.nytimes3xbfgragh.onion/2020/09/07/travel/route-66.html?action=click\&pgtype=Article\&state=default\&region=TOP_BANNER\&context=at_home_menu}{Cruise
  Along: Route 66}
\item
  \href{https://www.nytimes3xbfgragh.onion/2020/09/04/dining/sheet-pan-chicken.html?action=click\&pgtype=Article\&state=default\&region=TOP_BANNER\&context=at_home_menu}{Roast:
  Chicken With Plums}
\item
  \href{https://www.nytimes3xbfgragh.onion/2020/09/04/arts/television/dark-shadows-stream.html?action=click\&pgtype=Article\&state=default\&region=TOP_BANNER\&context=at_home_menu}{Watch:
  Dark Shadows}
\item
  \href{https://www.nytimes3xbfgragh.onion/interactive/2020/at-home/even-more-reporters-editors-diaries-lists-recommendations.html?action=click\&pgtype=Article\&state=default\&region=TOP_BANNER\&context=at_home_menu}{Explore:
  Reporters' Google Docs}
\end{itemize}

Advertisement

\protect\hyperlink{after-top}{Continue reading the main story}

Supported by

\protect\hyperlink{after-sponsor}{Continue reading the main story}

\hypertarget{the-food-expiration-dates-you-should-actually-follow}{%
\section{The Food Expiration Dates You Should Actually
Follow}\label{the-food-expiration-dates-you-should-actually-follow}}

The first thing you should know? The dates, as we know them, have
nothing to do with safety. J. Kenji López-Alt explains.

\includegraphics{https://static01.graylady3jvrrxbe.onion/images/2020/04/15/dining/15Kenji-Cover-Illustration/15Kenji-Cover-Illustration-articleLarge.jpg?quality=75\&auto=webp\&disable=upscale}

By \href{https://www.nytimes3xbfgragh.onion/by/j-kenji-lopez-alt}{J.
Kenji López-Alt}

\begin{itemize}
\item
  April 15, 2020
\item
  \begin{itemize}
  \item
  \item
  \item
  \item
  \item
  \item
  \end{itemize}
\end{itemize}

With most of us quarantined in our homes, chances are you've been
reacquainting yourself with the forgotten spices and fusty beans from
the depths of your pantry. But how fusty is \emph{too} fusty? When is
the right time to throw something out? And what about fresh ingredients?
If I'm trying to keep supermarket trips to a minimum, how long can my
eggs, dairy and produce keep?

Here's the first thing you should know: Expiration dates are not
expiration dates.

Food product dating, as the U.S. Department of Agriculture calls it, is
completely voluntary for all products (with the exception of baby food,
more on that later). Not only that, but it has nothing to do with
safety. It acts solely as the manufacturer's best guess as to when its
product will no longer be at peak quality, whatever that means. Food
manufacturers also tend to be rather conservative with those dates,
knowing that not all of us keep our pantries dark and open our
refrigerators as minimally as necessary. (I, for one, would never leave
the fridge door open for minutes at a time as I contemplate what to
snack on.)

Let's start with the things you definitely don't have to worry about.
Vinegars, honey, vanilla or other extracts, sugar, salt, corn syrup and
molasses will last virtually forever with little change in quality.
Regular steel-cut or rolled oats will last for a year or so before they
start to go rancid, but parcooked oats (or instant oats) can last nearly
forever. (Same with grits versus instant grits.)

\emph{{[}}\href{https://www.nytimes3xbfgragh.onion/article/recipes-cooking-tips-coronavirus.html}{\emph{More
recipes and tips for quarantine cooking}}\emph{.{]}}

White flour is almost certainly fine to use, no matter its age.
Whole-wheat and other whole-grain flours can acquire a metallic or soapy
odor within a few months. This whiter-equals-longer rule of thumb is
true for nonground grains as well. Refined white rice, for example, will
last for years, while brown rice will last only for months.

This is because unrefined grains contain fats, and fats are the first
thing to go off when it comes to dry pantry staples. Tree nuts,
typically high in fat, will go rancid within a few months in the pantry.
(Store them in the freezer to extend that to a few years.)

For things that go stale, it's the opposite: Shelf-stable supermarket
breads made with oils (and preservatives) can stay soft for weeks in the
fridge, but the lean, crusty sourdough from the corner bakery will be
stale by the next day and probably start to mold before the week is up.
(I
\href{https://www.nytimes3xbfgragh.onion/2016/02/24/dining/how-to-make-toast.html}{slice
and freeze} my fancy bread, taking it out a slice at a time to toast.)

Dried beans and lentils will remain safe to eat for years after
purchase, but they'll become tougher and take longer to cook as time
goes on. If you aren't sure how old your dried beans are, avoid using
them in recipes that include acidic ingredients like molasses or
tomatoes. Acid can drastically increase the length of time it takes
beans to soften.

We all make fun of our parents for using spices that expired in the
1980s, but, other than losing potency, there's nothing criminal in using
them (unless you consider flavorless
\href{https://cooking.nytimes3xbfgragh.onion/recipes/1018068-chicken-paprikash}{chicken
paprikash} a crime).

What about canned and jarred goods? As a rule, metal lasts longer than
glass, which lasts longer than plastic.

So long as there is no outward sign of spoilage (such as bulging or
rust), or visible spoilage when you open it (such as cloudiness,
moldiness or rotten smells), your canned fruits, vegetables and meats
will remain as delicious and palatable as the day you bought them for
years (or in the case of, say, Vienna sausages **** at least as good as
they were to begin with). The little button on the top of jarred goods,
which will bulge if there has been significant bacterial action inside
the jar, **** is still the best way to tell if the contents are going to
be all right to eat. Depending on storage, that could be a year or a
decade. Similarly, cans of soda will keep their fizz for years, glass
bottles for up to a year and plastic bottles for a few months. (Most
plastics are gas-permeable.)

Oils, even rancidity-prone unrefined oils, stored in sealed cans are
nearly indestructible as well (as evidenced by the two-gallon tin of
roasted sesame oil that I've been working through since 2006). Oils in
sealed glass bottles, less so. Oil in open containers can vary greatly
in shelf life, but all will last longer if you don't keep them near or
above your stovetop, where heat can get to them.

How do you tell if your oil is good? The same way you would with most
foods: Follow your nose. Old oil will start to develop metallic, soapy
or in some cases --- such as with canola oil --- fishy smells. Don't
trust your nose? Put a drop on your fingertip and squeeze it. Rancid oil
will feel tacky as opposed to slick.

Also from the oil-and-vinegar aisle: Salad dressings will last for
months or over a year in the fridge, especially if they come in bottles
with narrow squeeze openings (as opposed to open-mouthed jars).

Mustard lasts forever. Ketchup will start to turn color before the year
is out, but will still remain palatable. Contrary to popular belief,
mayonnaise --- especially when it doesn't contain ingredients like fresh
lemon juice or garlic --- has an exceptionally long shelf life. (High
concentrations of fat, salt and acid are all enemies of bacteria and
mold.)

The international aisle is a den of long-lasting sauces, pickles and
condiments. I've yet to find the quality inflection point for oyster
sauce, pickled chiles, chile sauces (like sambal oelek or Sriracha),
fermented bean sauces (like hoisin or Sichuan broad-bean chile paste) or
fish sauce. Soy sauce has a reputation for longevity, but I keep mine in
the refrigerator to fend off the fishy aromas that can start to develop
after a few months in the pantry.

We all know what a rotten egg smells like, right? Why else would it be a
benchmark for describing so many other bad smells? But how many times
have you actually smelled one: Once? Twice? Never? Probably never, at
least according to
\href{https://twitter.com/kenjilopezalt/status/1244897822668816385?s=20}{the
impromptu poll I conducted on Twitter}. That's because it takes a
\emph{long} time for eggs to go bad.

How long? The Julian date printed on each carton (that's the three-digit
number ranging from 001 for Jan. 1 to 365 for Dec. 31) represents the
date the eggs were packed, which, in most parts of the country, can be
up to 30 days after the egg was actually laid. The sell-by stamp can be
another 30 days after the pack date.

That's 60 full days! But odds are good that they'll still be palatable
for several weeks longer than that. You'll run out of hoarded toilet
paper before those eggs go bad.

We've all accidentally poured some clumpy spoiled milk into our cereal
bowls. It seems as if our milk is perfectly fine, until it's suddenly
not. How does it go bad overnight? The truth is, it doesn't. From the
moment you open a carton of milk, bacteria start to digest lactose (milk
sugars), and produce acidic byproducts. Once its pH hits 4.6, that's
when casein (milk protein) clumps.

Want longer-lasting milk? Look for ``ultrahigh temperature,'' or
``UHT,'' on the label. Milk in these cartons has been pasteurized at
high temperatures (275 degrees Fahrenheit: hot enough to destroy not
only viruses and bacteria, but bacterial spores as well), then
aseptically pumped and sealed into cartons. Most organic milk brands
undergo UHT. (Bonus: In the blind taste tests I've conducted, most
people \emph{preferred} the sweeter flavor of UHT milk.)

And as for
\href{https://cooking.nytimes3xbfgragh.onion/guides/57-how-to-make-baby-food}{baby
food} --- the only food with federally mandated use-by dating --- that
expiration date represents the latest date that the manufacturer can
guarantee that the food contains not less of each nutrient than what is
printed on the label, or, in the case of formula, that it can still pass
through an ordinary rubber nipple.

If it comes down to it, rest assured that you'll still be able to eat
the baby food and gain some nutritional benefit long after the zombie
apocalypse.

\emph{Follow} \href{https://twitter.com/nytfood}{\emph{NYT Food on
Twitter}} \emph{and}
\href{https://www.instagram.com/nytcooking/}{\emph{NYT Cooking on
Instagram}}\emph{,}
\href{https://www.facebookcorewwwi.onion/nytcooking/}{\emph{Facebook}}\emph{,}
\href{https://www.youtube.com/nytcooking}{\emph{YouTube}} \emph{and}
\href{https://www.pinterest.com/nytcooking/}{\emph{Pinterest}}\emph{.}
\href{https://www.nytimes3xbfgragh.onion/newsletters/cooking}{\emph{Get
regular updates from NYT Cooking, with recipe suggestions, cooking tips
and shopping advice}}\emph{.}

Advertisement

\protect\hyperlink{after-bottom}{Continue reading the main story}

\hypertarget{site-index}{%
\subsection{Site Index}\label{site-index}}

\hypertarget{site-information-navigation}{%
\subsection{Site Information
Navigation}\label{site-information-navigation}}

\begin{itemize}
\tightlist
\item
  \href{https://help.nytimes3xbfgragh.onion/hc/en-us/articles/115014792127-Copyright-notice}{©~2020~The
  New York Times Company}
\end{itemize}

\begin{itemize}
\tightlist
\item
  \href{https://www.nytco.com/}{NYTCo}
\item
  \href{https://help.nytimes3xbfgragh.onion/hc/en-us/articles/115015385887-Contact-Us}{Contact
  Us}
\item
  \href{https://www.nytco.com/careers/}{Work with us}
\item
  \href{https://nytmediakit.com/}{Advertise}
\item
  \href{http://www.tbrandstudio.com/}{T Brand Studio}
\item
  \href{https://www.nytimes3xbfgragh.onion/privacy/cookie-policy\#how-do-i-manage-trackers}{Your
  Ad Choices}
\item
  \href{https://www.nytimes3xbfgragh.onion/privacy}{Privacy}
\item
  \href{https://help.nytimes3xbfgragh.onion/hc/en-us/articles/115014893428-Terms-of-service}{Terms
  of Service}
\item
  \href{https://help.nytimes3xbfgragh.onion/hc/en-us/articles/115014893968-Terms-of-sale}{Terms
  of Sale}
\item
  \href{https://spiderbites.nytimes3xbfgragh.onion}{Site Map}
\item
  \href{https://help.nytimes3xbfgragh.onion/hc/en-us}{Help}
\item
  \href{https://www.nytimes3xbfgragh.onion/subscription?campaignId=37WXW}{Subscriptions}
\end{itemize}
