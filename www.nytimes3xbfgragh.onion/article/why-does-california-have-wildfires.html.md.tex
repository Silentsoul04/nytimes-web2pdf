Sections

SEARCH

\protect\hyperlink{site-content}{Skip to
content}\protect\hyperlink{site-index}{Skip to site index}

\href{https://www.nytimes3xbfgragh.onion/section/climate}{Climate}

\href{https://myaccount.nytimes3xbfgragh.onion/auth/login?response_type=cookie\&client_id=vi}{}

\href{https://www.nytimes3xbfgragh.onion/section/todayspaper}{Today's
Paper}

\href{/section/climate}{Climate}\textbar{}Why Does California Have So
Many Wildfires?

\url{https://nyti.ms/3aIdnyW}

\begin{itemize}
\item
\item
\item
\item
\item
\end{itemize}

\hypertarget{climate-and-environment}{%
\subsubsection{\texorpdfstring{\href{https://www.nytimes3xbfgragh.onion/section/climate?name=styln-climate\&region=TOP_BANNER\&block=storyline_menu_recirc\&action=click\&pgtype=Article\&impression_id=054f5af0-f29f-11ea-9e00-ad25b2d773f7\&variant=undefined}{Climate
and
Environment}}{Climate and Environment}}\label{climate-and-environment}}

\begin{itemize}
\tightlist
\item
  \href{https://www.nytimes3xbfgragh.onion/article/why-does-california-have-wildfires.html?name=styln-climate\&region=TOP_BANNER\&block=storyline_menu_recirc\&action=click\&pgtype=Article\&impression_id=054f5af1-f29f-11ea-9e00-ad25b2d773f7\&variant=undefined}{California
  Fires}
\item
  \href{https://www.nytimes3xbfgragh.onion/interactive/2020/climate/trump-environment-rollbacks.html?name=styln-climate\&region=TOP_BANNER\&block=storyline_menu_recirc\&action=click\&pgtype=Article\&impression_id=054f5af2-f29f-11ea-9e00-ad25b2d773f7\&variant=undefined}{Trump's
  Changes}
\item
  \href{https://www.nytimes3xbfgragh.onion/interactive/2020/04/19/climate/climate-crash-course-1.html?name=styln-climate\&region=TOP_BANNER\&block=storyline_menu_recirc\&action=click\&pgtype=Article\&impression_id=054f5af3-f29f-11ea-9e00-ad25b2d773f7\&variant=undefined}{Climate
  101}
\item
  \href{https://www.nytimes3xbfgragh.onion/interactive/2018/08/30/climate/how-much-hotter-is-your-hometown.html?name=styln-climate\&region=TOP_BANNER\&block=storyline_menu_recirc\&action=click\&pgtype=Article\&impression_id=054f8200-f29f-11ea-9e00-ad25b2d773f7\&variant=undefined}{Is
  Your Hometown Hotter?}
\end{itemize}

Advertisement

\protect\hyperlink{after-top}{Continue reading the main story}

Supported by

\protect\hyperlink{after-sponsor}{Continue reading the main story}

\hypertarget{why-does-california-have-so-many-wildfires}{%
\section{Why Does California Have So Many
Wildfires?}\label{why-does-california-have-so-many-wildfires}}

There are four key ingredients to the disastrous wildfire seasons in the
West, and climate change is a key culprit.

\includegraphics{https://static01.graylady3jvrrxbe.onion/images/2020/09/08/science/08CLI-CALFIRES-update/merlin_176700381_709524cd-fbfe-499c-af77-596f80821067-articleLarge.jpg?quality=75\&auto=webp\&disable=upscale}

\href{https://www.nytimes3xbfgragh.onion/by/kendra-pierre-louis}{\includegraphics{https://static01.graylady3jvrrxbe.onion/images/2018/07/16/multimedia/author-kendra-pierre-louis/author-kendra-pierre-louis-thumbLarge.png}}\href{https://www.nytimes3xbfgragh.onion/by/john-schwartz}{\includegraphics{https://static01.graylady3jvrrxbe.onion/images/2018/02/16/multimedia/author-john-schwartz/author-john-schwartz-thumbLarge.jpg}}

By
\href{https://www.nytimes3xbfgragh.onion/by/kendra-pierre-louis}{Kendra
Pierre-Louis} and
\href{https://www.nytimes3xbfgragh.onion/by/john-schwartz}{John
Schwartz}

\begin{itemize}
\item
  Sept. 8, 2020
\item
  \begin{itemize}
  \item
  \item
  \item
  \item
  \item
  \end{itemize}
\end{itemize}

Again, California is aflame --- and it isn't close to being over yet.

As of Tuesday, more than
\href{https://www.fire.ca.gov/daily-wildfire-report/}{two million acres
have burned across the state} so far in 2020, which makes this a record
year, surpassing 2018, according to the California Department of
Forestry and Fire Protection. About 200 Labor Day visitors to the Sierra
National Forest
\href{https://www.nytimes3xbfgragh.onion/2020/09/06/us/mammoth-pools-fires-california.html}{had
to be evacuated by helicopter} after being trapped by the Creek Fire,
and Governor Gavin Newsom
\href{https://www.gov.ca.gov/wp-content/uploads/2020/09/9.6.20-September-Fires-Emergency.pdf}{declared
a state of emergency} in five counties.

High temperatures and strong winds have made the situation even worse.
A\href{https://www.nytimes3xbfgragh.onion/article/california-weather.html?smid=tw-nytimes\&smtyp=cur}{heat
wave baked Southern California over the weekend}(Los Angeles County hit
a record 121 degrees)
\href{https://www.nytimes3xbfgragh.onion/2020/08/17/climate/death-valley-hottest-temperature-on-earth.html}{and
Death Valley recently reached 130 degrees}, which, if confirmed, would
be the highest temperature ever reliably recorded on the planet.

Residents being evacuated must weigh the risks of seeking refuge
\href{https://www.nytimes3xbfgragh.onion/2020/08/20/us/ca-wildfires-covid.html}{in
evacuation shelters in the midst of the coronavirus pandemic} and people
living far beyond the burn zone are struggling with the
\href{https://twitter.com/NWSBayArea/status/1296355443158036481?ref_src=twsrc\%5Etfw\%7Ctwcamp\%5Etweetembed\%7Ctwterm\%5E1296355443158036481\%7Ctwgr\%5E\&ref_url=https\%3A\%2F\%2Fwww.nytimes3xbfgragh.onion\%2F2020\%2F08\%2F20\%2Fus\%2Fca-fires.html}{smoke}.

What is it about California that makes
\href{https://www.nytimes3xbfgragh.onion/2020/09/08/us/california-wildfires-helicopter-rescue.html}{wildfires}
so catastrophic? There are four key ingredients.

\hypertarget{the-changing-climate}{%
\subsection{The (changing) climate}\label{the-changing-climate}}

The first is California's climate.

``Fire, in some ways, is a very simple thing,'' said Park Williams, a
bioclimatologist at Columbia University's Lamont-Doherty Earth
Observatory. ``As long as stuff is dry enough and there's a spark, then
that stuff will burn.''

\href{https://www.nytimes3xbfgragh.onion/2019/10/30/us/getty-fire-california-kincade.html}{California},
like much of the West, gets most of its moisture in the fall and winter.
Its vegetation then spends much of the summer slowly drying out because
of a lack of rainfall and warmer temperatures. That vegetation then
serves as kindling for fires.

\emph{{[}Follow our live}
\href{https://www.nytimes3xbfgragh.onion/interactive/2020/08/20/us/california-wildfire-maps.html}{\emph{California
wildfires map tracker}}\emph{.{]}}

But while California's climate has always been fire prone,
\href{https://agupubs.onlinelibrary.wiley.com/doi/full/10.1029/2019EF001210}{the
link between climate change and bigger fires is inextricable}. ``This
climate-change connection is straightforward: warmer temperatures dry
out fuels. In areas with abundant and very dry fuels, all you need is a
spark,'' he said.

\includegraphics{https://static01.graylady3jvrrxbe.onion/images/2020/08/21/climate/21CLI-CALFIRES1/merlin_175921338_d037db18-c773-44e1-9761-d11a44e63c94-articleLarge.jpg?quality=75\&auto=webp\&disable=upscale}

\href{https://www.nytimes3xbfgragh.onion/interactive/2019/03/18/business/pge-california-wildfires.html}{California's
fire} record dates back to 1932;
\href{https://www.fire.ca.gov/media/5510/top20_acres.pdf}{the 10 largest
fires since then} have occurred since 2000, including the 2018 Mendocino
Complex Fire,
\href{https://www.nytimes3xbfgragh.onion/2018/08/07/us/mendocino-complex-fire-california.html}{the
largest in state history}, and this year's L.N.U. Lightning Complex,
\href{https://www.fire.ca.gov/incidents/2020/8/17/lnu-lightning-complex-includes-hennessey-gamble-15-10-spanish-markley-13-4-11-16-walbridge/}{which
is now 91 percent contained}.

``In pretty much every single way, a perfect recipe for fire is just
kind of written in California,'' Dr. Williams said. ``Nature creates the
perfect conditions for fire, as long as people are there to start the
fires. But then climate change, in a few different ways, seems to also
load the dice toward more fire in the future.''

\href{\%3Ca\%20href=\%22https://www.nytimes3xbfgragh.onion/section/climate?action=click\&pgtype=Article\&state=default\&region=MAIN_CONTENT_1\&context=storylines_keepup\%22\%3Ehttps://www.nytimes3xbfgragh.onion/section/climate?action=click\&pgtype=Article\&state=default\&region=MAIN_CONTENT_1\&context=storylines_keepup\%3C/a\%3E}{}

\hypertarget{climate-and-environment-}{%
\subsubsection{Climate and Environment
›}\label{climate-and-environment-}}

\hypertarget{keep-up-on-the-latest-climate-news}{%
\paragraph{Keep Up on the Latest Climate
News}\label{keep-up-on-the-latest-climate-news}}

Updated Sept. 6, 2020

Here's what you need to know this week:

\begin{itemize}
\item
  \begin{itemize}
  \tightlist
  \item
    Americans back
    \href{https://www.nytimes3xbfgragh.onion/2020/09/04/climate/flood-fire-building-restrictions.html?action=click\&pgtype=Article\&state=default\&region=MAIN_CONTENT_1\&context=storylines_keepup}{tough
    limits on building in fire and flood zones}, new research shows.
  \item
    California's wildfires are driving another crisis: More and more
    \href{https://www.nytimes3xbfgragh.onion/2020/09/02/climate/wildfires-insurance.html?action=click\&pgtype=Article\&state=default\&region=MAIN_CONTENT_1\&context=storylines_keepup}{homeowners
    can't get insurance}.
  \item
    The Trump administration has relaxed Obama-era rules limiting the
    release of
    \href{https://www.nytimes3xbfgragh.onion/2020/08/31/climate/trump-coal-plants.html?action=click\&pgtype=Article\&state=default\&region=MAIN_CONTENT_1\&context=storylines_keepup}{toxic
    waste from coal plants}.
  \end{itemize}
\end{itemize}

\hypertarget{people}{%
\subsection{People}\label{people}}

Even if the conditions are right for a wildfire, you still need
something or someone to ignite it. Sometimes the trigger is nature, like
the unusual lightning strikes that set off the LNU Lightning Complex
fires in August, but more often than not humans are responsible, said
Nina S. Oakley, a research scientist at the Center for Western Weather
and Water Extremes at the Scripps Institution of Oceanography,
University of California, San Diego.

Many deadly fires have been
\href{https://www.nytimes3xbfgragh.onion/interactive/2019/03/18/business/pge-california-wildfires.html}{started
by downed power lines}. The 2018 Carr Fire, the state's sixth-largest on
record, started when a truck blew out its tire and its rim scraped the
pavement, sending out sparks. And some are started through bad
decisions, like the fire that was ignited over the weekend by
smoke-generating fireworks
\href{https://www.nytimes3xbfgragh.onion/2020/09/07/us/gender-reveal-party-wildfire.html}{as
part of a gender-reveal party} and has consumed thousands of acres east
of Los Angeles;
\href{https://www.fire.ca.gov/incidents/2020/9/5/el-dorado-fire/}{Cal
Fire says the fire is currently 16 percent contained}.

Dr. Oakley, was driving away from her home in Santa Rosa, Calif., on
Tuesday with her husband, another climate scientist, Benjamin Hatchett
of the Desert Research Institute in Reno, Nev., to the ocean to get a
respite from the smoke over her home and power outages that made it
impossible to work. Events like gender reveal parties can trigger a
blaze, she noted, but ``you also have the human contribution to
wildfire,'' which includes the warming that has been caused by
greenhouse gas emissions and the accompanying increased drying. Both
contribute ``to creating a situation favorable to wildfire,'' she said.
Dr. Hatchett called it a ``perfectly primed scenario'' for this
disaster.

``California has a lot of people and a really long dry season,'' Dr.
Williams said. ``People are always creating possible sparks, and as the
dry season wears on and stuff is drying out more and more, the chance
that a spark comes off a person at the wrong time just goes up. And
that's putting aside arson.''

There's another way people have contributed to wildfires: in their
choices of where to live.
\href{https://www.nytimes3xbfgragh.onion/2018/11/15/climate/california-fires-wildland-urban-interface.html}{People
are increasingly moving into areas near forests}, known as the
urban-wildland interface, that are inclined to burn.

Image

The L.N.U. Lightning Complex fire tore through the property of Bill
Nichols, 84, who worked to tamp out flames in Vacaville, Calif., last
month.Credit...Noah Berger/Associated Press

\hypertarget{fire-suppression}{%
\subsection{Fire suppression}\label{fire-suppression}}

It's counterintuitive, but the United States' history of suppressing
\href{https://www.nytimes3xbfgragh.onion/interactive/2019/03/18/business/pge-california-wildfires.html}{wildfires}
has actually made present-day wildfires worse.

``For the last century we fought fire, and we did pretty well at it
across all of the Western United States,'' Dr. Williams said. ``And
every time we fought a fire successfully, that means that a bunch of
stuff that would have burned didn't burn. And so over the last hundred
years we've had an accumulation of plants in a lot of areas.

``And so in a lot of
\href{https://www.nytimes3xbfgragh.onion/interactive/2019/03/18/business/pge-california-wildfires.html}{California
now when fires start}, those fires are burning through places that have
a lot more plants to burn than they would have if we had been allowing
fires to burn for the last hundred years.''

In recent years, the United States Forest Service has been trying to
rectify the previous practice through the use of prescribed or
``controlled'' burns.

\hypertarget{the-santa-ana-winds}{%
\subsection{The Santa Ana winds}\label{the-santa-ana-winds}}

The second and stage of this year's fire season is yet to come.

Each fall, strong gusts known as the Santa Ana winds bring dry air from
the Great Basin area of the West into Southern California, said Fengpeng
Sun, an assistant professor in the department of geosciences at the
University of Missouri-Kansas City.

Dr. Sun is a co-author of a
\href{http://iopscience.iop.org/article/10.1088/1748-9326/10/9/094005}{2015
study} that suggests that California has two distinct fire seasons. One,
which runs from June through September and is driven by a combination of
warmer and drier weather, is the Western fire season that most people
think of. Those wildfires tend to be more inland, in higher-elevation
forests.

But Dr. Sun and his co-authors also identified a second fire season that
runs from October through April and is driven by the Santa Ana winds.
Those fires tend to spread three times faster and burn closer to urban
areas, and they were responsible for 80 percent of the economic losses
over two decades beginning in 1990.

It's not just that the Santa Ana winds dry out vegetation; they also
move embers around, spreading fires.

Which brings us back to climate change.

Ultimately, determining the links between any individual fire and
climate change takes time, and analysis from the evolving discipline of
\href{https://www.nytimes3xbfgragh.onion/2016/08/02/science/looking-quickly-for-the-fingerprints-of-climate-change.html}{attribution
science}. But the effects of the greenhouse gases humans produce
underlie everything that occurs in the atmosphere, and the tendency of
climate change to make dry places more dry over time is a warning to the
West of a fiery future.

This article originally appeared in 2018. It was updated in September
2020.

Advertisement

\protect\hyperlink{after-bottom}{Continue reading the main story}

\hypertarget{site-index}{%
\subsection{Site Index}\label{site-index}}

\hypertarget{site-information-navigation}{%
\subsection{Site Information
Navigation}\label{site-information-navigation}}

\begin{itemize}
\tightlist
\item
  \href{https://help.nytimes3xbfgragh.onion/hc/en-us/articles/115014792127-Copyright-notice}{©~2020~The
  New York Times Company}
\end{itemize}

\begin{itemize}
\tightlist
\item
  \href{https://www.nytco.com/}{NYTCo}
\item
  \href{https://help.nytimes3xbfgragh.onion/hc/en-us/articles/115015385887-Contact-Us}{Contact
  Us}
\item
  \href{https://www.nytco.com/careers/}{Work with us}
\item
  \href{https://nytmediakit.com/}{Advertise}
\item
  \href{http://www.tbrandstudio.com/}{T Brand Studio}
\item
  \href{https://www.nytimes3xbfgragh.onion/privacy/cookie-policy\#how-do-i-manage-trackers}{Your
  Ad Choices}
\item
  \href{https://www.nytimes3xbfgragh.onion/privacy}{Privacy}
\item
  \href{https://help.nytimes3xbfgragh.onion/hc/en-us/articles/115014893428-Terms-of-service}{Terms
  of Service}
\item
  \href{https://help.nytimes3xbfgragh.onion/hc/en-us/articles/115014893968-Terms-of-sale}{Terms
  of Sale}
\item
  \href{https://spiderbites.nytimes3xbfgragh.onion}{Site Map}
\item
  \href{https://help.nytimes3xbfgragh.onion/hc/en-us}{Help}
\item
  \href{https://www.nytimes3xbfgragh.onion/subscription?campaignId=37WXW}{Subscriptions}
\end{itemize}
