\href{/es/section/ciencia-y-tecnologia}{Ciencia y
Tecnología}\textbar{}Es la ley de la selva en el mercado de las
mascarillas

\url{https://nyti.ms/2JKWuac}

\begin{itemize}
\item
\item
\item
\item
\item
\end{itemize}

\hypertarget{el-brote-de-coronavirus}{%
\subsubsection{\texorpdfstring{\href{https://www.nytimes3xbfgragh.onion/es/spotlight/coronavirus?name=styln-coronavirus-es\&region=TOP_BANNER\&block=storyline_menu_recirc\&action=click\&pgtype=Article\&impression_id=1d6307c0-f4c7-11ea-bca3-a182bcc486e2\&variant=undefined}{El
brote de
coronavirus}}{El brote de coronavirus}}\label{el-brote-de-coronavirus}}

\begin{itemize}
\tightlist
\item
  \href{https://www.nytimes3xbfgragh.onion/es/interactive/2020/espanol/mundo/coronavirus-en-estados-unidos.html?name=styln-coronavirus-es\&region=TOP_BANNER\&block=storyline_menu_recirc\&action=click\&pgtype=Article\&impression_id=1d6307c1-f4c7-11ea-bca3-a182bcc486e2\&variant=undefined}{Casos
  en Estados Unidos}
\item
  \href{https://www.nytimes3xbfgragh.onion/es/interactive/2020/08/06/espanol/ciencia-y-tecnologia/tengo-covid-19-sintomas.html?name=styln-coronavirus-es\&region=TOP_BANNER\&block=storyline_menu_recirc\&action=click\&pgtype=Article\&impression_id=1d632ed0-f4c7-11ea-bca3-a182bcc486e2\&variant=undefined}{Síntomas}
\item
  \href{https://www.nytimes3xbfgragh.onion/es/2020/09/11/espanol/ciencia-y-tecnologia/cerebro-coronavirus.html?name=styln-coronavirus-es\&region=TOP_BANNER\&block=storyline_menu_recirc\&action=click\&pgtype=Article\&impression_id=1d632ed1-f4c7-11ea-bca3-a182bcc486e2\&variant=undefined}{El
  efecto en el cerebro}
\item
  \href{https://www.nytimes3xbfgragh.onion/es/2020/09/09/espanol/ciencia-y-tecnologia/salud-mental-coronavirus.html?name=styln-coronavirus-es\&region=TOP_BANNER\&block=storyline_menu_recirc\&action=click\&pgtype=Article\&impression_id=1d632ed2-f4c7-11ea-bca3-a182bcc486e2\&variant=undefined}{Convalecencia
  prolongada}
\item
  \href{https://www.nytimes3xbfgragh.onion/es/2020/09/08/espanol/ciencia-y-tecnologia/dentistas-covid-dientes.html?name=styln-coronavirus-es\&region=TOP_BANNER\&block=storyline_menu_recirc\&action=click\&pgtype=Article\&impression_id=1d632ed3-f4c7-11ea-bca3-a182bcc486e2\&variant=undefined}{La
  salud dental y el virus}
\end{itemize}

\includegraphics{https://static01.graylady3jvrrxbe.onion/images/2020/04/05/business/06Virus-mascarillas-ES-1/00virus-mask4-articleLarge.jpg?quality=75\&auto=webp\&disable=upscale}

Sections

\protect\hyperlink{site-content}{Skip to
content}\protect\hyperlink{site-index}{Skip to site index}

\hypertarget{es-la-ley-de-la-selva-en-el-mercado-de-las-mascarillas}{%
\section{Es la ley de la selva en el mercado de las
mascarillas}\label{es-la-ley-de-la-selva-en-el-mercado-de-las-mascarillas}}

Hospitales, gobiernos, benefactores y vendedores improvisados están
compitiendo. Las estafas y los precios están en aumento.

Susan HoughtellingCredit...Shane Lavalette para The New York Times

Supported by

\protect\hyperlink{after-sponsor}{Continue reading the main story}

\href{https://www.nytimes3xbfgragh.onion/by/jack-nicas}{\includegraphics{https://static01.graylady3jvrrxbe.onion/images/2018/11/26/multimedia/author-jack-nicas/author-jack-nicas-thumbLarge.png}}

Por \href{https://www.nytimes3xbfgragh.onion/by/jack-nicas}{Jack Nicas}

\begin{itemize}
\item
  6 de abril de 2020
\item
  \begin{itemize}
  \item
  \item
  \item
  \item
  \item
  \end{itemize}
\end{itemize}

\href{https://www.nytimes3xbfgragh.onion/2020/04/03/technology/coronavirus-masks-shortage.html}{Read
in English}

\href{https://www.nytimes3xbfgragh.onion/newsletters/el-times}{Regístrate
para recibir nuestro boletín} por correo con lo mejor de The New York
Times.

\begin{center}\rule{0.5\linewidth}{\linethickness}\end{center}

El mes pasado, Susan Houghtelling, gerente de suministros hospitalarios
en el norte de Nueva York, enfrentaba una escasez de artículos médicos
cuando, de pronto, su buzón de entrada se llenó de ofertas.

Había anuncios de galones de gel antibacterial, cajas de trajes de
aislamiento y, lo más crucial, palés de cubrebocas N95, quizás el
producto más buscado en el planeta. Todas las ofertas tenían precios que
eran varias veces más altos de lo que normalmente pagaba.

``Todas estas personas están saliendo de la nada y, misteriosamente,
ahora todas tienen acceso a un suministro abundante'', dijo
Houghtelling, quien trabaja para tres hospitales propiedad de Arnot
Health, con sede en Elmira, Nueva York. Le reenvió decenas de los
mensajes de los nuevos vendedores a The New York Times. Uno le ofrecía
cajas de 50 mascarillas quirúrgicas por 70 dólares cada una; solía pagar
2,28 dólares.

Un negociante en específico llamó su atención: Blank Industries, una
compañía que ofrecía tapabocas N95 por casi cinco dólares cada una, pero
solo si Houghtelling ordenaba un millón. Se imaginó que era un fraude.

Blank Industries es una compañía real, pero se trata de un fabricante de
productos para derretir hielo en Hudson, Massachusetts. En una
entrevista, Andrew Blank, el fundador, dijo que había transformado su
negocio para vender cubrebocas después de tener noticias de un proveedor
chino al que contrató una vez para fabricar un nuevo tipo de cepillo de
dientes (Blank lo había inventado). Después de la aparición del
coronavirus, el proveedor convirtió su planta de productos dentales en
una fábrica de tapabocas. Blank les dijo a sus doce empleados que
dejaran de vender halita y comenzaran a vender mascarillas.

¿Por qué está cobrando 4,92 dólares por cada tapabocas N95? ``Para ser
honesto, ni siquiera sé en cuánto se vende normalmente un cubrebocas
N95'', comentó.

Le dije cuánto costaban. ``¿Cincuenta centavos?'', repitió. Su proveedor
le estaba cobrando 4,75 dólares (su margen cubriría los costos de envío;
planeaba no obtener ganancias).

La explosión en la demanda debido a las cantidades cada vez más escasas
de tapabocas ha provocado un alboroto en la cadena de suministro. En
Estados Unidos, el gobierno federal ha decidido no ordenar a las
fábricas estadounidenses que se pongan a hacer mascarillas para aumentar
el suministro. En vez de esto, los funcionarios federales están
compitiendo con estados, hospitales y proveedores médicos por la misma
reserva de cubrebocas, que vienen en su mayor parte de China.

Sin embargo, estados y hospitales, cuyos proveedores de siempre están
abrumados y saturados, tienen poca experiencia negociando directamente
con la cadena de suministro china. Miles de intermediarios
---emprendedores, buenas personas y oportunistas--- se han apresurado a
llenar el vacío.

Ese frenesí ha provocado un lío confuso, de acuerdo con entrevistas a
hospitales, fábricas y compradores de tapabocas. La producción de
mascarillas está aumentando, pero también los fraudes, los obstáculos
logísticos y, desde luego, los precios.

\hypertarget{nos-estuxe1n-bombardeando}{%
\subsection{`Nos están bombardeando'}\label{nos-estuxe1n-bombardeando}}

\includegraphics{https://static01.graylady3jvrrxbe.onion/images/2020/04/02/business/06Virus-mascarillas-ES-2/merlin_171201603_b7580d91-22b9-496c-a18c-7bbaedafa2d6-articleLarge.jpg?quality=75\&auto=webp\&disable=upscale}

Después de que comenzó el brote de coronavirus, China importó
\href{https://www.nytimes3xbfgragh.onion/2020/04/01/business/coronavirus-china-masks.html}{2000
millones de tapabocas}.
\href{https://www.nytimes3xbfgragh.onion/reuters/2020/03/31/world/europe/31reuters-health-coronavirus-france-masks.html}{Francia
ordenó mil millones} y prometió volverse autosuficiente para finales del
año. El gobierno estadounidense ha hecho comparativamente poco para
coordinar las compras y asegurar que los gobiernos y los hospitales
estadounidenses no compitan entre sí.

El mes pasado, los funcionarios federales de Estados Unidos aceptaron
comprar aproximadamente 600 millones de mascarillas N95 a lo largo de
los siguientes dieciocho meses. Sin embargo, muchos estados y hospitales
están desesperados por obtener suministros en este momento, y al
gobierno ya casi se le ha acabado el suministro de equipo de protección
de la reserva nacional. El jueves, la Casa Blanca dijo que había
\href{https://www.nytimes3xbfgragh.onion/2020/04/02/world/coronavirus-live-news-updates.html\#link-711303f6}{recurrido
a la Ley de Producción para la Defensa}, una ley de la década de 1950,
para asegurar que el gigante de la manufactura 3M envíe cierta
proporción de sus cubrebocas a Estados Unidos.

Algunos de los emprendedores que han ocupado el lugar del gobierno han
tenido éxito. Operation Masks, una organización sin fines de lucro que
fue creada hace dos semanas por ejecutivos del sector tecnológico, dijo
que acababa de llegar a acuerdos para obtener un millón de tapabocas N95
para el estado de Nueva York y 200.000 para Hawái, con un costo de poco
más de tres dólares por mascarilla, sin incluir el envío ni otros
costos. El jueves, Massachusetts recibió 1,2 millones de cubrebocas N95
con el
\href{https://www.wsj.com/articles/a-million-n95-masks-are-coming-from-chinaon-board-the-new-england-patriots-plane-11585821600}{avión
del equipo de fútbol americano de los Patriotas de Nueva Inglaterra}.

Aun así, varios ejecutivos de hospitales dijeron que, aunque apreciaban
la ayuda de las personas bienintencionadas, estaban abrumados por los
nombres nuevos que aparecen en sus buzones, todos ofreciendo productos
que necesitan a precios mucho mayores de los que generalmente pagan.

``Nos están bombardeando'', dijo Ed Bonetti, director de la cadena de
suministro de la red hospitalaria UMass Memorial en Worcester,
Massachusetts.

El hospital está preparado para pagar más por los tapabocas, pero no
quiere adquirir equipo defectuoso. ``Es un territorio virgen en el que
se hace difícil incluso verificar la calidad de las compras'', dijo
Bonetti. ``Lo último que queremos es darle un producto a un médico que
no lo proteja''.

\hypertarget{el-arbitraje-de-los-suministros-muxe9dicos}{%
\subsection{El arbitraje de los suministros
médicos}\label{el-arbitraje-de-los-suministros-muxe9dicos}}

Image

Yaear Weintroub, un estudiante de 22 años de Brooklyn, ha cambiado la
venta de dispositivos electrónicos por la de suministros
médicos.Credit...Chang W. Lee/The New York Times

No todos los nuevos participantes del mercado son buenos samaritanos.
Grupos en Facebook, WhatsApp y Telegram están llenos de publicaciones
que anuncian miles de tapabocas a precios inflados.

Algunos son vendedores al mayoreo que compraron palés de cubrebocas
provenientes de China o en ventas de liquidación y después aumentaron su
precio de venta. Muchos más simplemente son intermediarios que se hacen
llamar agentes. Exploran los grupos en busca de mascarillas anunciadas a
un precio relativamente bajo, y después vuelven a publicar la oferta por
algunos miles de dólares más. No se encargan de los tapabocas ni ponen
su propio dinero.

Yaear Weintroub es uno de esos agentes. Es un estudiante de 22 años de
la universidad comunitaria de Brooklyn que vendía productos electrónicos
a mayoreo a vendedores en Amazon. Pero en marzo, los foros en línea en
los que Weintroub busca ofertas se llenaron con listados de cubrebocas,
por lo que ahora pasa el tiempo tratando de conectar a compradores y
vendedores para hacer un tipo de arbitraje de suministros médicos.

En una entrevista reciente, dijo que estaba trabajando con un socio para
cerrar un acuerdo para vender 280.000 mascarillas médicas con un aumento
de precio de un 20 por ciento que le daría una ganancia neta de
aproximadamente 40.000 dólares. Dijo que muchos de los agentes vendieron
los productos a otros agentes, cada uno imponiendo su precio, hasta que
los tapabocas llegan, supuestamente, a algún hogar de ancianos o un
hospital. Él aseguró que preferiría vender directamente a los
hospitales.

``Son más serios'', dijo. ``Si tengo los productos, quiero un comprador
serio para ellos. Y además, es una obra moralmente buena''.

Para estos vendedores, los suministros médicos son simplemente otro
producto de moda con el cual pueden obtener ganancias. Avraham
Eisenberg, un mayorista de Nueva York que intenta transportar
mascarillas desde China, comparó la prisa por estas con la moda hace
algunos años por los
\href{https://www.nytimes3xbfgragh.onion/2017/08/15/magazine/the-rise-of-the-fidget-spinner-and-the-fall-of-the-well-managed-fad.html}{\emph{fidget
spinners}}.

El Departamento de Justicia de Estados Unidos dijo el mes pasado que
investigaría a las personas que manipulan el mercado de los suministros
médicos. Cinco días después, las autoridades federales
\href{https://www.bloomberg.com/news/articles/2020-03-30/suspected-mask-profiteer-arrested-for-coughing-on-fbi-agents}{acusaron
a un hombre de Brooklyn} de mentir sobre la manipulación de precios tras
haber intentado vender mil tapabocas y otros suministros a un médico por
12.000 dólares. (También lo acusaron de agresión después de que afirmó
que tenía coronavirus y les tosió a unos agentes del FBI). Los
funcionarios federales ahora están distribuyendo los más de
\href{https://www.hhs.gov/about/news/2020/04/02/doj-and-hhs-partner-to-distribute-more-than-half-a-million-medical-supplies-confiscated-from-price-gougers.html}{medio
millón de suministros que le confiscaron}.

\hypertarget{de-los-flotantes-para-piscina-a-los-cubrebocas}{%
\subsection{De los flotantes para piscina a los
cubrebocas}\label{de-los-flotantes-para-piscina-a-los-cubrebocas}}

Image

Dan Schonfeld, quien usualmente vende flotantes para piscina, intentó
conseguir 100.000 mascarillas de China para abastecer hospitales
estadounidenses.Credit...Joe Carrotta para The New York Times

La gente que incursiona en el mercado de las mascarillas proviene de
todas partes. Dan Schonfeld, por ejemplo, vende flotantes para piscina.
Es bastante bueno en eso. Encontró un proveedor de confianza en China,
les pegó logos de equipos deportivos y creó un negocio estable a través
de PoolPartsToGo.com.

Cuando el coronavirus se propagó el mes pasado en el estado donde vive,
Nueva York, Schonfeld pensó que podría usar sus conexiones en China para
obtener cubrebocas para los médicos estadounidenses. Dejó de lado su
negocio de artículos para piscinas y comenzó a ir tras los tapabocas,
prometiendo no ganar ni un centavo.

``Fue como entrar a un periodo de cámara rápida, y no se ha detenido
desde entonces'', dijo Schonfeld, de 40 años. ``No creo haber dormido
cuatro noches seguidas''.

Usó su iPhone todo el día y llamó a hospitales estadounidenses de día y
a contactos chinos de noche. Todos los hospitales estaban interesados,
pero no había muchas mascarillas de buena calidad.

Después, justo antes de la medianoche del 19 de marzo, su proveedor de
flotantes para piscina en Ningbo, China, Jensen Jiang, le envió un
correo electrónico con noticias. Había cerrado un acuerdo con una
fábrica cercana para que le hicieran 100.000 tapabocas N95 a 2,70
dólares cada una. Pero habían llegado pedidos simultáneos, dijo, así que
Schonfeld debía decidir rápidamente.

``Mañana es demasiado tarde'', escribió Jiang. Schonfeld le dijo que
hiciera el depósito de 35.000 dólares.

Al día siguiente, Schonfeld llamó emocionado a los hospitales. Sin
embargo, los ejecutivos que habían expresado tanta desesperación por
obtener cubrebocas de pronto estaban preocupados de entregarle 270.000
dólares a un hombre que vendía artículos de piscina unos días antes. Uno
respondió: ``Simplemente no lo conocemos'', dijo Schonfeld. ``Ahora era
yo quien necesitaba ayuda''.

Al final, sus abogados encontraron un nuevo comprador: una red de
organizaciones sin fines de lucro que cuida de 35.000 neoyorquinos con
discapacidades intelectuales. Transfirieron el dinero, y Schonfeld
reservó un vuelo de carga.

Después, se despertó para recibir más malas noticias. ``Me temo que le
he provocado grandes problemas'', dijo Jiang en un correo electrónico el
26 de marzo. ``Todas las mascarillas fueron incautadas por el
gobierno''. El correo electrónico incluía la fotografía de un aviso de
clausura en las puertas de la fábrica, con fecha de once días antes.
Schonfeld no supo qué pensar.

Image

Schonfeld recibió esta foto de su proveedor en China. No sabe si es
auténtico. Dice: ``Estimados visitantes: por órdenes del departamento
gubernamental que supervisa nuestro trabajo, suspenderemos la entrega de
mascarillas. Le informaremos el momento en el que reanudaremos labores.
Disculpe las molestias.''

Mientras Jiang negociaba un reembolso, que aún no ha llegado, decidieron
encontrar y enviar un tapabocas distinto: el llamado KN95, la versión
eficaz de China del N95. Schonfeld ordenó 150.000 de una nueva fábrica y
reservó un avión de carga para el 3 de abril.

Sin embargo, hubo otro detalle: los lineamientos de la Administración de
Alimentos y Medicamentos de Estados Unidos (FDA, por su sigla en inglés)
sobre el uso médico de los tapabocas KN95 en Estados Unidos son difusos,
y los abogados de Schonfeld le advirtieron que los funcionarios podrían
incautarlos. (El 2 de abril, la agencia
\href{https://www.buzzfeednews.com/article/kenbensinger/the-fda-now-says-it-will-allow-imports-of-kn95-masks-an}{dijo}
que no bloquearía las importaciones de cubrebocas).

``Todos los días me despierto para encontrar un nuevo obstáculo'',
comentó. ``Jamás creí que sería así de difícil ayudar''.

Sin importar qué ocurra con el envío, dará por terminada su incursión en
el mundo de los suministros médicos, comentó.

Un día después, mencionó que estaba investigando sobre proveedores de
respiradores. ``Le dije a mi esposa: `Muy bien. Ya terminé con esto''',
comentó. ``Pero si veo en las noticias que están suplicando que les den
respiradores y nadie hace nada, no podré quedarme de brazos cruzados''.

Cao Li y Sheera Frenkel colaboraron con reporteo.

Jack Nicas cubre tecnología en el área de San Francisco. Antes de unirse
al Times, pasó siete años en The Wall Street Journal reporteando sobre
tecnología, aviación y noticias de Estados Unidos.
\href{https://twitter.com/jacknicas?lang=es}{@jacknicas}

\begin{center}\rule{0.5\linewidth}{\linethickness}\end{center}

Advertisement

\protect\hyperlink{after-bottom}{Continue reading the main story}

\hypertarget{site-index}{%
\subsection{Site Index}\label{site-index}}

\hypertarget{site-information-navigation}{%
\subsection{Site Information
Navigation}\label{site-information-navigation}}

\begin{itemize}
\tightlist
\item
  \href{https://help.nytimes3xbfgragh.onion/hc/en-us/articles/115014792127-Copyright-notice}{©~2020~The
  New York Times Company}
\end{itemize}

\begin{itemize}
\tightlist
\item
  \href{https://www.nytco.com/}{NYTCo}
\item
  \href{https://help.nytimes3xbfgragh.onion/hc/en-us/articles/115015385887-Contact-Us}{Contact
  Us}
\item
  \href{https://www.nytco.com/careers/}{Work with us}
\item
  \href{https://nytmediakit.com/}{Advertise}
\item
  \href{http://www.tbrandstudio.com/}{T Brand Studio}
\item
  \href{https://www.nytimes3xbfgragh.onion/privacy/cookie-policy\#how-do-i-manage-trackers}{Your
  Ad Choices}
\item
  \href{https://www.nytimes3xbfgragh.onion/privacy}{Privacy}
\item
  \href{https://help.nytimes3xbfgragh.onion/hc/en-us/articles/115014893428-Terms-of-service}{Terms
  of Service}
\item
  \href{https://help.nytimes3xbfgragh.onion/hc/en-us/articles/115014893968-Terms-of-sale}{Terms
  of Sale}
\item
  \href{https://spiderbites.nytimes3xbfgragh.onion}{Site Map}
\item
  \href{https://help.nytimes3xbfgragh.onion/hc/en-us}{Help}
\item
  \href{https://www.nytimes3xbfgragh.onion/subscription?campaignId=37WXW}{Subscriptions}
\end{itemize}
