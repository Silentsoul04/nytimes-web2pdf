Sections

SEARCH

\protect\hyperlink{site-content}{Skip to
content}\protect\hyperlink{site-index}{Skip to site index}

\href{https://www.nytimes3xbfgragh.onion/es/section/america-latina}{América
Latina}

\href{https://myaccount.nytimes3xbfgragh.onion/auth/login?response_type=cookie\&client_id=vi}{}

\href{https://www.nytimes3xbfgragh.onion/section/todayspaper}{Today's
Paper}

\href{/es/section/america-latina}{América Latina}\textbar{}Brasil
retiene las estadísticas del coronavirus y se desata la furia

\url{https://nyti.ms/3dL5rgN}

\begin{itemize}
\item
\item
\item
\item
\item
\end{itemize}

\hypertarget{el-brote-de-coronavirus}{%
\subsubsection{\texorpdfstring{\href{https://www.nytimes3xbfgragh.onion/es/spotlight/coronavirus?name=styln-coronavirus-es\&region=TOP_BANNER\&block=storyline_menu_recirc\&action=click\&pgtype=Article\&impression_id=6fc61760-f4c1-11ea-9caa-f145e0172db9\&variant=undefined}{El
brote de
coronavirus}}{El brote de coronavirus}}\label{el-brote-de-coronavirus}}

\begin{itemize}
\tightlist
\item
  \href{https://www.nytimes3xbfgragh.onion/es/interactive/2020/espanol/mundo/coronavirus-en-estados-unidos.html?name=styln-coronavirus-es\&region=TOP_BANNER\&block=storyline_menu_recirc\&action=click\&pgtype=Article\&impression_id=6fc61761-f4c1-11ea-9caa-f145e0172db9\&variant=undefined}{Casos
  en Estados Unidos}
\item
  \href{https://www.nytimes3xbfgragh.onion/es/interactive/2020/08/06/espanol/ciencia-y-tecnologia/tengo-covid-19-sintomas.html?name=styln-coronavirus-es\&region=TOP_BANNER\&block=storyline_menu_recirc\&action=click\&pgtype=Article\&impression_id=6fc61762-f4c1-11ea-9caa-f145e0172db9\&variant=undefined}{Síntomas}
\item
  \href{https://www.nytimes3xbfgragh.onion/es/2020/09/11/espanol/ciencia-y-tecnologia/cerebro-coronavirus.html?name=styln-coronavirus-es\&region=TOP_BANNER\&block=storyline_menu_recirc\&action=click\&pgtype=Article\&impression_id=6fc61763-f4c1-11ea-9caa-f145e0172db9\&variant=undefined}{El
  efecto en el cerebro}
\item
  \href{https://www.nytimes3xbfgragh.onion/es/2020/09/09/espanol/ciencia-y-tecnologia/salud-mental-coronavirus.html?name=styln-coronavirus-es\&region=TOP_BANNER\&block=storyline_menu_recirc\&action=click\&pgtype=Article\&impression_id=6fc61764-f4c1-11ea-9caa-f145e0172db9\&variant=undefined}{Convalecencia
  prolongada}
\item
  \href{https://www.nytimes3xbfgragh.onion/es/2020/09/08/espanol/ciencia-y-tecnologia/dentistas-covid-dientes.html?name=styln-coronavirus-es\&region=TOP_BANNER\&block=storyline_menu_recirc\&action=click\&pgtype=Article\&impression_id=6fc61765-f4c1-11ea-9caa-f145e0172db9\&variant=undefined}{La
  salud dental y el virus}
\end{itemize}

Advertisement

\protect\hyperlink{after-top}{Continue reading the main story}

Supported by

\protect\hyperlink{after-sponsor}{Continue reading the main story}

Sudamérica

\hypertarget{brasil-retiene-las-estaduxedsticas-del-coronavirus-y-se-desata-la-furia}{%
\section{Brasil retiene las estadísticas del coronavirus y se desata la
furia}\label{brasil-retiene-las-estaduxedsticas-del-coronavirus-y-se-desata-la-furia}}

A medida que las cifras aumentan, el gobierno del presidente Jair
Bolsonaro dejó de revelar datos completos sobre los casos y muertes por
coronavirus.

\includegraphics{https://static01.graylady3jvrrxbe.onion/images/2020/06/08/world/09virus-Brazil-ES-01/08virus-brazil-articleLarge-v2.jpg?quality=75\&auto=webp\&disable=upscale}

Por \href{https://www.nytimes3xbfgragh.onion/by/ernesto-londono}{Ernesto
Londoño}

\begin{itemize}
\item
  8 de junio de 2020
\item
  \begin{itemize}
  \item
  \item
  \item
  \item
  \item
  \end{itemize}
\end{itemize}

\href{https://www.nytimes3xbfgragh.onion/2020/06/08/world/americas/brazil-coronavirus-statistics.html}{Read
in English}

\href{https://www.nytimes3xbfgragh.onion/newsletters/el-times}{Regístrate
para recibir nuestro boletín} con lo mejor de The New York Times.

\begin{center}\rule{0.5\linewidth}{\linethickness}\end{center}

RÍO DE JANEIRO --- Cuando el coronavirus arrasaba Brasil, el presidente
Jair Bolsonaro fue criticado por sabotear las medidas de aislamiento
impuestas por los estados, al alentar marchas masivas entre sus
simpatizantes y arremeter contra la creciente cifra de muertos,
diciendo: ``¿Qué quieren que haga?''.

Ahora que el brote en Brasil ha empeorado aún más ---con más infecciones
que cualquier otro país excepto Estados Unidos--- el gobierno de
Bolsonaro apareció con una singular respuesta a la creciente alarma:
decidió dejar de informar por completo las cifras acumulativas del
virus.

El viernes 5 de junio el Ministerio de Salud de Brasil retiró
\href{https://covid.saude.gov.br/}{el sitio web} donde había estado
reportando las estadísticas de coronavirus. Y luego, cuando la página
volvió a estar en línea el sábado, omitía los datos históricos, dejando
fuera la cantidad de personas que ya habían sido infectadas o que
murieron debido al virus.

Legisladores y expertos en salud atacaron rápidamente a Bolsonaro de
forma inusualmente virulenta. No solo condenaron la decisión del
gobierno de ocultar estadísticas exhaustivas al tiempo que las muertes y
los contagios continúan aumentando, sino que criticaron rotundamente la
práctica reiterada del gobierno de Bolsonaro de minimizar el peligro del
virus, más allá de lo que puedan decir los científicos y sus propios
ministros de salud.

Gilmar Mendes, uno de los jueces de la Corte Suprema,
calificó\href{https://twitter.com/gilmarmendes/status/1269399102242119686?s=20}{la
``manipulación} de estadísticas del gobierno una táctica de regímenes
totalitarios'', y agregó que el ``truco no absolverá al gobierno de un
eventual genocidio''.

\includegraphics{https://static01.graylady3jvrrxbe.onion/images/2020/06/08/world/09virus-Brazil-ES-02/merlin_172778076_514b20e8-0114-4083-9aa3-4519b0cbcccf-articleLarge.jpg?quality=75\&auto=webp\&disable=upscale}

La pandemia ---y, específicamente, las respuestas de los gobiernos a
ella--- ha sido muy polémica en todo el mundo. Pero en pocos lugares los
problemas han polarizado tanto como en Brasil, un país ya separado por
un abismo político entre los furiosos detractores de Bolsonaro y sus
igualmente fervientes seguidores.

El presidente Bolsonaro, quien inicialmente describió al virus como
``una gripecita'', dijo que el desafío del virus se ve empequeñecido
frente a las consecuencias económicas de las medidas para quedarse en
casa, y que el verdadero peligro es el aumento del desempleo que dejará
a las personas en la pobreza.

Pero también ha sido criticado por unirse a las grandes manifestaciones
favor de su gobierno que corren el riesgo de propagar el virus, por
ordenar a las fuerzas armadas producir masivamente un medicamente no
probado para el virus ---la
\href{https://www.nytimes3xbfgragh.onion/2020/06/03/health/hydroxychloroquine-coronavirus-trump.html}{hidroxicloroquina}---
y por luchar contra sus propios funcionarios de salud a medida que la
crisis se intensificaba.

Ahora Brasil sufre la mayor cantidad diaria de muertes en el mundo ---a
menudo más de 1000 al día--- y el gobierno ha dejado de informar el
número acumulado de víctimas del brote.

``Al alterar los números, el Ministerio de Salud intenta tapar el sol
con un dedo'', dijo Rodrigo Maia, presidente de la Cámara Baja del
Congreso
\href{https://twitter.com/RodrigoMaia/status/1269841732780294144?s=20}{en
un mensaje en Twitter}publicado poco después de la medianoche del
domingo. ``Es urgente restaurar la credibilidad de las estadísticas. Un
ministerio que distorsiona los números crea un universo paralelo para
evitar enfrentar la realidad de los hechos''.

Carlos Wizard Martins, un empresario que recientemente fue designado
para ayudar a liderar la respuesta del gobierno, le dijo la semana
pasada al diario O Globo que las estadísticas de coronavirus del país
estaban siendo auditadas porque funcionarios federales creían que los
estados reportaban cifras infladas en un esfuerzo por captar más fondos.

La explicación, que no fue respaldada con evidencias, fue vista
ampliamente como el último paso en falso del gobierno en su respuesta al
brote.

Image

Durante la pandemia, vecinos de la favela Paraisópolis en São Paulo
protestaron para exigir mejores condiciones sanitarias.Credit...Victor
Moriyama para The New York Times

El Ministerio de Salud se ha visto sacudido por la rotación de personal
en las últimas semanas, cuando el virus se apoderaba de Brasil. A
mediados de abril, Bolsonaro despidió a un ministro de Salud, Luiz
Henrique Mandetta, después de que ambos se enfrentaron a causa del
desdén del presidente hacia las medidas de distanciamiento social que el
ministerio y los gobernadores de los estados estaban promoviendo.

Luego Nelson Teich, el sucesor en la cartera de salud,
\href{https://www.nytimes3xbfgragh.onion/2020/05/15/world/americas/brazil-health-minister-bolsonaro.html}{renunció
después de menos de un mes en el cargo,} y dejó el ministerio al mando
de un general en activo sin experiencia en atención médica.

El domingo 8 de junio, el gobierno emitió dos cifras diferentes sobre el
último número de muertes diarias, reportando inicialmente 1382 muertes,
solo para actualizar el número a 525. El ministerio dijo que la cifra
inicial incluía muertes reportadas erróneamente.

El ministerio de Salud dijo en un comunicado, también el domingo, que su
nuevo método de registro ofrecería ``una visión más realista de lo que
está sucediendo a nivel nacional''.

El gobierno no explicó la nueva metodología para el seguimiento de
casos.

Durante el fin de semana, el Consejo Nacional de Secretarios de Salud,
que representa a los funcionarios de salud locales,
\href{http://www.conass.org.br/painelconasscovid19/}{lanzó un sitio web}
que recopila datos completos. Según dicho recuento, hasta el domingo
Brasil tenía más de 680.400 casos confirmados de coronavirus y, por lo
menos, 36.151 muertes.

El consejo respondió con indignación a la acusación de que los
funcionarios estatales estaban proporcionando números ficticios para
obtener ganancias monetarias, refiriéndose a la acusación formulada por
Wizard.

Durante el fin de semana, brasileños indignados llamaron a boicotear los
negocios de Wizard. El domingo por la noche,
\href{https://noticias.r7.com/saude/carlos-wizard-diz-que-nao-vai-mais-contribuir-com-ministerio-da-saude-07062020}{Wizard
anunció} que dejaría su cargo en el gobierno.

``Pido disculpas por cualquier acto o declaración de mi autoría que haya
sido interpretada como irrespetuosa hacia los familiares de las víctimas
de la COVID-19 o los profesionales de salud que asumieron la noble
misión de salvar vidas'', dijo en un comunicado.

Image

Carlos Wizard, designado para ayudar a dirigir la respuesta del
gobierno, dijo que las estadísticas de coronavirus se auditarían debido
a la sospecha de que los estados inflaban las cifras. Después
renunció.Credit...Miguel Schincariol/Agence France-Presse --- Getty
Images

Brasil, que tiene un robusto sistema de atención de salud pública, se ha
destacado
\href{https://www.nytimes3xbfgragh.onion/es/2020/05/18/espanol/america-latina/covid-brasil.html}{históricamente
en vigilancia epidemiológica}.

En todo caso, los expertos dicen que una auditoría rigurosa de los casos
de COVID-19 revelaría que la enfermedad ha matado a más personas que lo
que los datos oficiales han registrado, porque las pruebas han estado
severamente limitadas. Un
\href{https://www.nytimes3xbfgragh.onion/es/2020/05/12/espanol/america-latina/virus-latinoamerica.html}{análisis
del Times} descubrió que en Manaos, una metrópoli en las profundidades
de la Amazonía, el número de muertes en abril fue tres veces su promedio
histórico para ese mes.

``La manipulación de los datos de la pandemia por parte del Ministerio
de Salud es, por decir lo mínimo, angustiante'', dijo
\href{https://www.sabin.org/leadership\#DG}{Denise Garrett}, una
epidemióloga brasileña estadounidense que trabajó en los Centros para el
Control y la Prevención de Enfermedades durante más de dos décadas.
``Los datos deben comunicarse de manera transparente, precisa y
oportuna. Esto es crucial para la toma de decisiones y también de suma
importancia para evitar la confusión pública''.

Manuela Andreoni y Letícia Casado colaboraron con reportería.

Ernesto Londoño es el jefe de la corresponsalía de Brasil, con sede en
Río de Janeiro. Antes fue escritor parte del Comité Editorial y, antes
de unirse a The New York Times, era reportero en The Washington Post.
\href{https://twitter.com/londonoe}{@londonoe}
•\href{https://www.facebookcorewwwi.onion/londono}{Facebook}

\begin{center}\rule{0.5\linewidth}{\linethickness}\end{center}

Advertisement

\protect\hyperlink{after-bottom}{Continue reading the main story}

\hypertarget{site-index}{%
\subsection{Site Index}\label{site-index}}

\hypertarget{site-information-navigation}{%
\subsection{Site Information
Navigation}\label{site-information-navigation}}

\begin{itemize}
\tightlist
\item
  \href{https://help.nytimes3xbfgragh.onion/hc/en-us/articles/115014792127-Copyright-notice}{©~2020~The
  New York Times Company}
\end{itemize}

\begin{itemize}
\tightlist
\item
  \href{https://www.nytco.com/}{NYTCo}
\item
  \href{https://help.nytimes3xbfgragh.onion/hc/en-us/articles/115015385887-Contact-Us}{Contact
  Us}
\item
  \href{https://www.nytco.com/careers/}{Work with us}
\item
  \href{https://nytmediakit.com/}{Advertise}
\item
  \href{http://www.tbrandstudio.com/}{T Brand Studio}
\item
  \href{https://www.nytimes3xbfgragh.onion/privacy/cookie-policy\#how-do-i-manage-trackers}{Your
  Ad Choices}
\item
  \href{https://www.nytimes3xbfgragh.onion/privacy}{Privacy}
\item
  \href{https://help.nytimes3xbfgragh.onion/hc/en-us/articles/115014893428-Terms-of-service}{Terms
  of Service}
\item
  \href{https://help.nytimes3xbfgragh.onion/hc/en-us/articles/115014893968-Terms-of-sale}{Terms
  of Sale}
\item
  \href{https://spiderbites.nytimes3xbfgragh.onion}{Site Map}
\item
  \href{https://help.nytimes3xbfgragh.onion/hc/en-us}{Help}
\item
  \href{https://www.nytimes3xbfgragh.onion/subscription?campaignId=37WXW}{Subscriptions}
\end{itemize}
