Sections

SEARCH

\protect\hyperlink{site-content}{Skip to
content}\protect\hyperlink{site-index}{Skip to site index}

\href{/es/section/mundo}{Mundo}\textbar{}Así fue que al mundo se le
escapó la propagación silenciosa de la COVID-19

\url{https://nyti.ms/31uSRPM}

\begin{itemize}
\item
\item
\item
\item
\item
\item
\end{itemize}

\hypertarget{the-coronavirus-outbreak}{%
\subsubsection{\texorpdfstring{\href{https://www.nytimes3xbfgragh.onion/news-event/coronavirus?name=styln-coronavirus-national\&region=TOP_BANNER\&block=storyline_menu_recirc\&action=click\&pgtype=Article\&impression_id=0a23ee80-f1d0-11ea-a506-3174ba8d379e\&variant=undefined}{The
Coronavirus
Outbreak}}{The Coronavirus Outbreak}}\label{the-coronavirus-outbreak}}

\begin{itemize}
\tightlist
\item
  live\href{https://www.nytimes3xbfgragh.onion/2020/09/08/world/covid-19-coronavirus.html?name=styln-coronavirus-national\&region=TOP_BANNER\&block=storyline_menu_recirc\&action=click\&pgtype=Article\&impression_id=0a241590-f1d0-11ea-a506-3174ba8d379e\&variant=undefined}{Latest
  Updates}
\item
  \href{https://www.nytimes3xbfgragh.onion/interactive/2020/us/coronavirus-us-cases.html?name=styln-coronavirus-national\&region=TOP_BANNER\&block=storyline_menu_recirc\&action=click\&pgtype=Article\&impression_id=0a241591-f1d0-11ea-a506-3174ba8d379e\&variant=undefined}{Maps
  and Cases}
\item
  \href{https://www.nytimes3xbfgragh.onion/interactive/2020/science/coronavirus-vaccine-tracker.html?name=styln-coronavirus-national\&region=TOP_BANNER\&block=storyline_menu_recirc\&action=click\&pgtype=Article\&impression_id=0a241592-f1d0-11ea-a506-3174ba8d379e\&variant=undefined}{Vaccine
  Tracker}
\item
  \href{https://www.nytimes3xbfgragh.onion/2020/09/02/your-money/eviction-moratorium-covid.html?name=styln-coronavirus-national\&region=TOP_BANNER\&block=storyline_menu_recirc\&action=click\&pgtype=Article\&impression_id=0a241593-f1d0-11ea-a506-3174ba8d379e\&variant=undefined}{Eviction
  Moratorium}
\item
  \href{https://www.nytimes3xbfgragh.onion/interactive/2020/09/02/magazine/food-insecurity-hunger-us.html?name=styln-coronavirus-national\&region=TOP_BANNER\&block=storyline_menu_recirc\&action=click\&pgtype=Article\&impression_id=0a241594-f1d0-11ea-a506-3174ba8d379e\&variant=undefined}{American
  Hunger}
\end{itemize}

\includegraphics{https://static01.graylady3jvrrxbe.onion/images/2020/06/22/world/29Asymptomatic-ES/xxasymptomatic-articleLarge-v3.jpg?quality=75\&auto=webp\&disable=upscale}

Europa

\hypertarget{asuxed-fue-que-al-mundo-se-le-escapuxf3-la-propagaciuxf3n-silenciosa-de-la-covid-19}{%
\section{Así fue que al mundo se le escapó la propagación silenciosa de
la
COVID-19}\label{asuxed-fue-que-al-mundo-se-le-escapuxf3-la-propagaciuxf3n-silenciosa-de-la-covid-19}}

El equipo de Camilla Rothe estuvo entre los primeros en advertir sobre
la transmisión asintomática.Credit...Laetitia Vancon para The New York
Times

Supported by

\protect\hyperlink{after-sponsor}{Continue reading the main story}

\href{https://www.nytimes3xbfgragh.onion/by/matt-apuzzo}{\includegraphics{https://static01.graylady3jvrrxbe.onion/images/2018/06/14/multimedia/author-matt-apuzzo/author-matt-apuzzo-thumbLarge.png}}\href{https://www.nytimes3xbfgragh.onion/by/selam-gebrekidan}{\includegraphics{https://static01.graylady3jvrrxbe.onion/images/2018/10/10/multimedia/author-selam-gebrekidan/author-selam-gebrekidan-thumbLarge.png}}\href{https://www.nytimes3xbfgragh.onion/by/david-d-kirkpatrick}{\includegraphics{https://static01.graylady3jvrrxbe.onion/images/2018/10/15/multimedia/author-david-d-kirkpatrick/author-david-d-kirkpatrick-thumbLarge-v2.png}}

Por \href{https://www.nytimes3xbfgragh.onion/by/matt-apuzzo}{Matt
Apuzzo},
\href{https://www.nytimes3xbfgragh.onion/by/selam-gebrekidan}{Selam
Gebrekidan} y
\href{https://www.nytimes3xbfgragh.onion/by/david-d-kirkpatrick}{David
D. Kirkpatrick}

\begin{itemize}
\item
  29 de junio de 2020
\item
  \begin{itemize}
  \item
  \item
  \item
  \item
  \item
  \item
  \end{itemize}
\end{itemize}

\href{https://www.nytimes3xbfgragh.onion/2020/06/27/world/europe/coronavirus-spread-asymptomatic.html}{Read
in
English}\href{https://www.nytimes3xbfgragh.onion/2020/06/27/world/europe/coronavirus-spread-asymptomatic.html}{Read
in English}

La transmisión asintomática hace que el coronavirus sea mucho más
difícil de combatir. Pero los funcionarios de salud desestimaron el
riesgo durante meses, impulsando afirmaciones engañosas y
contradictorias frente a la evidencia en aumento.

\href{https://www.nytimes3xbfgragh.onion/newsletters/el-times}{Regístrate
para recibir nuestro boletín} con lo mejor de The New York Times.

\begin{center}\rule{0.5\linewidth}{\linethickness}\end{center}

MÚNICH --- Camilla Rothe estaba a punto de salir a cenar cuando el
laboratorio del gobierno la llamó para darle la sorprendente noticia de
que la prueba que había solicitado había sido positiva. Era el 27 de
enero y acababa de descubrir el primer caso del nuevo coronavirus en
Alemania.

Sin embargo, el diagnóstico no tenía sentido. Su paciente, un empresario
de una compañía de autopartes cercana, solo pudo haberse infectado a
través de una persona: una colega que estaba de visita desde China.
Nadie pensó que esa colega tuviera el virus.

La visitante había parecido perfectamente sana durante su estancia en
Alemania. No tosió ni estornudó ni mostró señales de fatiga ni fiebre
durante dos días de largas reuniones. Les dijo a sus colegas que había
comenzado a sentirse enferma después de su vuelo de regreso a China.
Días más tarde, dio positivo a la prueba de coronavirus.

En ese momento, los científicos creían que solo las personas con
síntomas podían propagar el coronavirus. Asumieron que actuaba como su
primo genético, el SARS.

``Las personas que saben mucho más acerca del coronavirus que yo estaban
absolutamente seguras'', recordó Rothe, especialista en enfermedades
infecciosas del Hospital de la Universidad de Múnich.

Pero si los expertos se equivocaban ---si el virus era capaz de
propagarse mediante enfermos aparentemente sanos que aún no habían
desarrollado síntomas--- las ramificaciones eran posiblemente
catastróficas. Las campañas públicas de concientización, los filtros en
los aeropuertos y las políticas de confinamiento en caso de enfermedad
quizá no iban a poder detenerlo. Tal vez se requerirían medidas más
agresivas: ordenarle a las personas sanas que lleven cubrebocas, por
ejemplo, o restringir los viajes internacionales.

Rothe y sus colegas fueron de los primeros en advertir al mundo. Sin
embargo, aunque se acumulaban las pruebas de otros científicos, los
principales funcionarios de salud expresaron con una seguridad
inquebrantable que la propagación asintomática no era de importancia.

En los días y semanas siguientes, los políticos, funcionarios de salud
pública y académicos rivales menospreciaron o ignoraron al equipo de
Múnich. Algunos se esforzaron de manera activa para socavar sus
advertencias en un momento crucial, mientras la enfermedad se propagaba
imperceptiblemente en
\href{https://www.reuters.com/article/us-health-coronavirus-france-church-spec/special-report-five-days-of-worship-that-set-a-virus-time-bomb-in-france-idUSKBN21H0Q2}{iglesias
francesas}, estadios de fútbol italianos y bares en estaciones de esquí
austríacos. Un crucero, el Diamond Princess, se convertiría en un
presagio mortal de la propagación sin síntomas.

\includegraphics{https://static01.graylady3jvrrxbe.onion/images/2020/06/28/world/29Asymptomatic-ES-02/merlin_170540889_773e07a6-e978-4dc4-ab9a-610c9b70b23f-articleLarge.jpg?quality=75\&auto=webp\&disable=upscale}

Image

Funcionarios con equipo de protección abordaron el Diamond Princess en
febrero para trasladar a una persona con coronavirus.Credit...Kim
Kyung-Hoon/Reuters

Entrevistas con médicos y funcionarios de salud pública en más de una
decena de países muestran que durante dos meses cruciales ---y ante las
pruebas genéticas cada vez más numerosas--- los funcionarios de salud y
los líderes políticos occidentales restaron importancia o negaron el
riesgo de la propagación asintomática. Las principales agencias en
materia de salud, incluyendo la Organización Mundial de la Salud y el
Centro Europeo para la Prevención y el Control de las Enfermedades,
emitieron consejos contradictorios y a menudo confusos. Un diálogo
crucial de salud pública se convirtió en un debate semántico acerca de
cómo llamar a las personas que estaban infectadas y no tenían síntomas
evidentes.

El retraso de dos meses fue el resultado de suposiciones científicas
erradas, rivalidades académicas y, quizá lo más importante, una
reticencia a aceptar que contener el virus requeriría medidas drásticas.
La resistencia a las evidencias emergentes fue parte de la respuesta
lenta del mundo al virus.

Es imposible calcular el número de víctimas que causó ese retraso, pero
los modelos sugieren que las acciones tempranas y agresivas quizá
habrían salvado decenas de miles de vidas. A países como Singapur y
Australia, que utilizaron pruebas y rastreo de contactos y pasaron
rápidamente a poner en cuarentena a viajeros aparentemente sanos, les
fue mejor que a aquellos que no lo hicieron.

Image

Las personas disfrutaban de un día soleado en el Louvre en París a
mediados de marzo.Credit...Dmitry Kostyukov para The New York Times

Image

Pacientes que esperaban los resultados de sus exámenes en marzo en un
hospital en Brescia, Italia, uno de los primeros lugares en Europa en
ser golpeado fuertemente por el coronavirus.Credit...Alessandro Grassani
para The New York Times

Ahora se acepta de manera generalizada que las personas al parecer sanas
pueden propagar el virus, aunque sigue siendo incierto cuánto han
contribuido a la pandemia. Aunque varían los cálculos, los modelos que
usan datos de
\href{https://www.nature.com/articles/s41591-020-0869-5}{Hong Kong},
\href{https://www.eurosurveillance.org/content/10.2807/1560-7917.ES.2020.25.17.2000257}{Singapur
y China} sugieren que del 30 al 60 por ciento de la propagación ocurre
cuando las personas no muestran síntomas.

``Esto fue, creo, una verdad muy simple'', dijo Rothe. ``Me sorprendió
que causara tamaña tormenta. No puedo explicarlo''.

Incluso ahora, con más de nueve millones de casos en todo el mundo y
\href{https://www.nytimes3xbfgragh.onion/interactive/2020/world/coronavirus-maps.html}{un
número de muertes que superó las 500.000}, la COVID-19 sigue siendo un
acertijo sin respuesta. Es demasiado pronto para saber si lo peor ya ha
pasado, o si una segunda ola mundial de infecciones está a punto de
estallar. No obstante, está claro que varios países, de regímenes
herméticos a democracias demasiado seguras de sí mismas, han errado en
su respuesta, subestimado el virus y desestimado sus propios planes de
emergencia.

También es dolorosamente claro que el tiempo era un elemento esencial
para frenar el virus, y que se malgastó demasiado.

\hypertarget{ella-no-estaba-enferma}{%
\subsection{`Ella no estaba enferma'}\label{ella-no-estaba-enferma}}

La noche de la primera prueba positiva de Alemania, el virus parecía muy
lejano. Se habían reportado menos de 100 muertes en todo el mundo.
Italia, que se convertiría en la zona cero de Europa, no registraría sus
primeros casos por otros tres días.

Algunos informes de China ya habían sugerido la posibilidad de una
propagación sin síntomas. Pero nadie había demostrado que pudiera
ocurrir.

Aquella noche, Rothe redactó un correo electrónico dirigido a una decena
de médicos y funcionarios de salud pública.

``Las infecciones de hecho pueden transmitirse durante el periodo de
incubación'', escribió.

Tres empleados más de Webasto, la compañía de autopartes, dieron
positivo el día siguiente. Sus síntomas eran tan leves que, normalmente,
es probable que ninguno hubiera sido señalado para hacerse la prueba ni
hubieran pensado en quedarse en casa.

Rothe decidió que debía sonar la alarma. Su jefe, Michael Hoelscher,
envió un correo electrónico a The New England Journal of Medicine.
``Creemos que esta observación es de suma importancia'', escribió.

Los editores respondieron de inmediato. ¿Cuán pronto podrían ver el
artículo?

Image

El doctor Michael Hoelscher en su despacho durante una entrevista con
una estación local de televisiónCredit...Laetitia Vancon para The New
York Times

Image

La doctora Rothe le hizo la prueba a un voluntario durante un estudio
sobre la COVID-19 en un asilo de ancianos en Múnich en
junio.Credit...Laetitia Vancon para The New York Times

La mañana siguiente, el 30 de enero, funcionarios de salud entrevistaron
a la empresaria china por teléfono. Hospitalizada en Shanghái, explicó
que había empezado a sentirse enferma en el vuelo de regreso a casa. En
retrospectiva, quizá había tenido dolores o fatiga leves, pero los había
atribuido a un largo día de viajes.

\hypertarget{latest-updates-the-coronavirus-outbreak}{%
\section{\texorpdfstring{\href{https://www.nytimes3xbfgragh.onion/2020/09/08/world/covid-19-coronavirus.html?action=click\&pgtype=Article\&state=default\&region=MAIN_CONTENT_1\&context=storylines_live_updates}{Latest
Updates: The Coronavirus
Outbreak}}{Latest Updates: The Coronavirus Outbreak}}\label{latest-updates-the-coronavirus-outbreak}}

Updated 2020-09-08T12:22:35.182Z

\begin{itemize}
\tightlist
\item
  \href{https://www.nytimes3xbfgragh.onion/2020/09/08/world/covid-19-coronavirus.html?action=click\&pgtype=Article\&state=default\&region=MAIN_CONTENT_1\&context=storylines_live_updates\#link-46162376}{Trillions
  of dollars separate lawmakers' proposals for virus relief.}
\item
  \href{https://www.nytimes3xbfgragh.onion/2020/09/08/world/covid-19-coronavirus.html?action=click\&pgtype=Article\&state=default\&region=MAIN_CONTENT_1\&context=storylines_live_updates\#link-679303d7}{Nine
  drugmakers pledge to thoroughly vet any coronavirus vaccine.}
\item
  \href{https://www.nytimes3xbfgragh.onion/2020/09/08/world/covid-19-coronavirus.html?action=click\&pgtype=Article\&state=default\&region=MAIN_CONTENT_1\&context=storylines_live_updates\#link-1c973131}{`The
  lockdown killed my father': Farmer suicides add to India's virus
  misery.}
\end{itemize}

\href{https://www.nytimes3xbfgragh.onion/2020/09/08/world/covid-19-coronavirus.html?action=click\&pgtype=Article\&state=default\&region=MAIN_CONTENT_1\&context=storylines_live_updates}{See
more updates}

More live coverage:
\href{https://www.nytimes3xbfgragh.onion/live/2020/09/08/business/stock-market-today-coronavirus?action=click\&pgtype=Article\&state=default\&region=MAIN_CONTENT_1\&context=storylines_live_updates}{Markets}

``Desde su perspectiva, no estaba enferma'', dijo Nadie Schian, una
portavoz de Webasto que estaba en la llamada. ``Ella dijo, `OK, me
sentía cansada. Pero he estado en Alemania muchas veces antes y siempre
tengo desfase horario'''.

Cuando los funcionarios de salud describieron la llamada, Rothe y
Hoelscher rápidamente terminaron y enviaron su artículo. Rothe no llamó
a la paciente, pero dijo que recurrió al resumen de la autoridad
sanitaria.

En cuestión de horas,
\href{https://www.nejm.org/doi/full/10.1056/NEJMc2001468}{se publicó en
línea}. Fue una modesta observación clínica en un momento clave. Solo
unos días antes, la Organización Mundial de la Salud había dicho que
necesitaba más información sobre este mismo tema.

Sin embargo, lo que no sabían los autores era que, en un suburbio a 20
minutos de distancia, otro grupo de médicos también se había apresurado
a publicar un informe. Ninguno sabía en qué estaba trabajando el otro,
una diferencia aparentemente pequeña que tendría implicaciones globales.

\hypertarget{divisiones-acaduxe9micas}{%
\subsection{Divisiones académicas}\label{divisiones-acaduxe9micas}}

El segundo grupo estaba conformado por funcionarios de la autoridad
sanitaria bávara y la agencia nacional de salud de Alemania, conocida
como Instituto Robert Koch. En una oficina suburbana, los médicos
desplegaron papel mural y rastrearon rutas de infección usando
bolígrafos de colores.

Ese equipo, dirigido por la epidemióloga bávara Merle Böhmer, envió un
artículo a The Lancet, otra revista médica de primera. Sin embargo, el
grupo hospitalario de Múnich los había superado por tres horas. Böhmer
dijo que el artículo de su equipo, que por eso no fue publicado, había
llegado a conclusiones similares, pero las redactaron de manera
ligeramente distinta.

Rothe había escrito que los pacientes parecían ser contagiosos antes del
inicio de \emph{cualquier} síntoma. El equipo del gobierno había escrito
que los pacientes parecían ser contagiosos antes del inicio de
\emph{todos} los síntomas, en un momento en que los síntomas eran tan
leves que la gente quizá ni siquiera los reconocía.

La mujer china, por ejemplo, se había despertado a mitad de la noche
sintiendo desfase horario. Como quería estar lista para sus reuniones,
se tomó una medicina china llamada 999 ---que contiene el equivalente a
una tableta de Tylenol--- y volvió a la cama.

¿Quizás eso había ocultado una fiebre leve? ¿Quizás el desfase horario
era en realidad fatiga? Se había puesto un chal durante una reunión,
¿tal vez eso fue una señal de escalofríos?

Image

La doctora Merle Böhmer y su equipo escribieron que los pacientes
parecían ser contagiosos antes de mostrar todos los síntomas, no antes
de mostrar síntomas.Credit...Laetitia Vancon para The New York Times

Image

El doctor Hoelscher dijo que se negó a cambiar la redacción del informe
de Rothe y reemplazar su nombre con el de los miembros del grupo de
trabajo del gobierno.Credit...Laetitia Vancon para The New York Times

Después de dos largas llamadas telefónicas con la mujer, los médicos del
Instituto Robert Koch estaban convencidos de que simplemente ella no
había podido reconocer sus síntomas. Escribieron al editor de The New
England Journal of Medicine para poner en duda los hallazgos de Rothe.

Los editores decidieron que la disputa equivalía a detalles nimios. Si
se necesitó una larga entrevista para identificar los síntomas, ¿cómo
podría esperarse que alguien lo haga en el mundo real?

``La pregunta era si ella tenía algo consistente con la COVID-19 o si
alguien habría reconocido en ese momento que era COVID-19'', dijo Eric
Rubin, médico y editor de la revista.

``La respuesta parecía ser no''.

La revista no publicó la carta. Sin embargo, ese no sería el fin de la
historia.

Ese fin de semana, Andreas Zapf, dirigente de la autoridad sanitaria
bávara, llamó a Hoelscher, de la clínica de Múnich. ``Mira, la gente de
Berlín está muy enojada por tu publicación'', dijo Zapf, según
Hoelscher.

Sugirió cambiar la redacción del informe de Rothe y reemplazar su nombre
con el de los miembros del grupo de trabajo del gobierno, según
Hoelscher. No obstante, él se rehusó.

La agencia sanitaria no quiso hacer comentarios sobre la llamada
telefónica.

Hasta entonces, dijo Hoelscher, su informe había parecido directo. Ahora
era claro: ``Políticamente, este era un gran, gran problema''.

\hypertarget{un-completo-tsunami}{%
\subsection{`Un completo tsunami'}\label{un-completo-tsunami}}

El lunes 3 de febrero, la revista Science publicó
\href{https://www.sciencemag.org/news/2020/02/paper-non-symptomatic-patient-transmitting-coronavirus-wrong}{un
artículo en el que describió el informe de Rothe como ``defectuoso''}.
Science informó que el Instituto Robert Koch le había escrito al New
England Journal para rebatir sus hallazgos y corregir un error.

El Instituto Robert Koch rechazó varias solicitudes de entrevista a lo
largo de varias semanas y no respondió a las preguntas que se le
enviaron por escrito.

El informe de Rothe rápidamente se convirtió en símbolo de las
investigaciones apresuradas. Los científicos dijeron que ella debió
haber hablado con la paciente china antes de publicar el artículo y que
la omisión había socavado el trabajo de su equipo. En Twitter, ella y
sus colegas fueron desacreditados tanto por científicos como por
comentaristas todólogos. ``Cayó sobre nosotros como un completo
tsunami'', dijo Hoelscher.

La controversia también eclipsó otro desarrollo crucial fuera de Múnich.

La mañana siguiente, Clemens-Martin Wendtner
\href{https://instmikrobiobw.de/aktuelles/ansicht/pressemitteilung}{hizo
un anuncio sorprendente}. Wendtner supervisaba el tratamiento de los
pacientes con COVID-19 en Múnich ---ahora eran ocho--- y había tomado
muestras de cada uno.

Descubrió el virus en la nariz y la garganta en niveles mucho más altos
y mucho antes de lo que se había observado en pacientes de SARS. Eso
significaba que, probablemente, podría propagarse antes de que las
personas supieran que estaban enfermas.

Image

El trabajo de Clemens-Martin Wendtner también sugirió el riesgo de que
los pacientes pudieran transmitir el virus antes de darse cuenta de que
lo tenían.Credit...Laetitia Vancon para The New York Times

Image

La doctora Rothe ayuda a una participante a completar un formulario en
un asilo de ancianos.Credit...Laetitia Vancon para The New York Times

Pero la historia de Science acalló esa noticia. Si el artículo de Rothe
implicaba que los gobiernos quizá debían tomar más medidas en contra de
la COVID-19, la reacción negativa del Instituto Robert Koch era una
defensa implícita del pensamiento convencional.

La agencia de salud pública de Suecia declaró que el informe de Rothe
contenía errores importantes. El sitio web de la agencia dijo,
inequívocamente, que ``no hay evidencia de que las personas sean
contagiosas durante el período de incubación'', una afirmación que
permanecería en línea de algún modo durante meses.

Los funcionarios de salud franceses tampoco dieron lugar para el debate:
``Una persona es contagiosa solo cuando aparecen los síntomas'',
\href{https://www.lemonde.fr/les-decodeurs/article/2020/02/06/coronavirus-une-affiche-du-ministere-ecarte-trop-vite-le-risque-de-contagion-lors-de-l-incubation_6028658_4355770.html}{se
leía en una hoja volante distribuida por el gobierno}. ``Sin síntomas =
sin riesgo de ser contagioso''.

Mientras Rothe y Hoelscher seguían aturdidos por la crítica, unos
médicos japoneses se preparaban para abordar el crucero Diamond
Princess. Un ex pasajero había dado positivo por coronavirus.

Sin embargo, en el crucero las fiestas continuaban. Después de todo, el
pasajero infectado había estado fuera del barco durante días. Y no había
reportado síntomas mientras estuvo a bordo.

\hypertarget{un-debate-semuxe1ntico}{%
\subsection{Un debate semántico}\label{un-debate-semuxe1ntico}}

Inmediatamente después del informe de Rothe, la OMS señaló que los
pacientes podrían transmitir el virus antes de mostrar síntomas. Sin
embargo, la organización también enfatizó un punto que sigue
defendiendo: los pacientes con síntomas son los principales motores de
la pandemia.

No obstante, en cuanto se publicó el artículo de Science, la
organización entró directamente al debate en torno al trabajo de Rothe.
El martes 4 de febrero, Sylvie Briand, directora de preparación ante las
enfermedades infecciosas de la agencia, publicó un enlace en Twitter que
llevaba al artículo de Science, y dijo que el informe de Rothe era
defectuoso.

Con ese tuit, la OMS se enfocó en una distinción semántica que iba a
nublar el diálogo durante meses: ¿la paciente era asintomática, lo que
significa que jamás mostraría síntomas? ¿O presintomática, lo cual
implicaría que se iba a enfermar más tarde? O algo aún más confuso: ¿era
oligosintomática, es decir, que tenía síntomas tan leves que no los
reconoció?

Para algunos médicos, centrarse en estas distinciones arcanas era como
ponerle al mal tiempo buena cara. Una persona que se siente sana no
tiene forma de saber que es portadora de un virus o que está a punto de
enfermarse. Los controles de temperatura del aeropuerto no son capaces
de identificar a esas personas. Tampoco resulta efectivo preguntarles
sobre sus síntomas o decirles que se queden en casa cuando se sientan
enfermas.

La OMS dijo más tarde que el tuit no intentaba ser una crítica.

Un grupo prestó poca atención a este debate en gestación: los médicos
del área de Múnich que trabajaban para contener el foco de infección de
la compañía de autopartes. Ellos hablaban a diario con personas
potencialmente enfermas, monitoreaban sus síntomas y rastreaban a sus
contactos.

Image

Rothe y su equipo se preparan para el día de trabajo.Credit...Laetitia
Vancon para The New York Times

Image

Hoelscher dijo que el artículo del The New England Journal of Medicine
se ha convertido en un tema político ``muy, muy importante'' para
él.Credit...Laetitia Vancon para The New York Times

``Para nosotros, pronto fue bastante claro que esta enfermedad puede
transmitirse antes de los síntomas'', dijo Monika Wirth, una médica que
rastreó los contactos en el cercano condado de Fürstenfeldbruck.

Rothe, sin embargo, estaba conmocionada. No podía entender por qué gran
parte de la comunidad científica parecía ansiosa por minimizar el
riesgo.

``Todo lo que necesitas es un par de ojos'', dijo. ``No necesitas
virología avanzada''.

Pero se mantuvo confiada.

``Se demostrará que tenemos razón'', le dijo a Hoelscher.

Esa noche, Rothe recibió un correo electrónico de Michael Libman, un
especialista en enfermedades infecciosas en Montreal. Él pensaba que las
críticas al artículo fueron de semántica. Su artículo lo había
convencido de algo: ``La enfermedad probablemente se propagará por todo
el mundo''.

\hypertarget{paruxe1lisis-poluxedtica}{%
\subsection{Parálisis política}\label{paruxe1lisis-poluxedtica}}

El 4 de febrero, el comité científico de Gran Bretaña se reunió y,
aunque sus expertos no descartaron la posibilidad de una transmisión
asintomática, nadie puso mucha atención al artículo de Rothe.

``Fue en gran medida un estudio de oídas'', dijo Wendy Barclay, viróloga
y miembro del comité, conocido como el Grupo Científico Asesor para
Emergencias. ``En ausencia de una epidemiología y un rastreo realmente
robustos, no es obvio hasta que se ven los datos''.

Los datos llegarían pronto, y de una fuente inesperada. En la segunda
semana de febrero, Böhmer, del equipo bávaro de salud, recibió una
sorprendente llamada telefónica.

Los virólogos habían descubierto una mutación genética sutil en las
infecciones de dos pacientes del foco infeccioso de Múnich. Se habían
cruzado durante un momento muy breve: uno le pasó el salero al otro en
la cafetería de la compañía, cuando ninguno tenía síntomas. Su mutación
compartida dejó claro que uno había infectado al otro.

Böhmer se había mostrado escéptica acerca de la propagación
asintomática. Pero ahora no había duda: ``Solo se puede explicar con la
transmisión pre-sintomática'', dijo Böhmer.

Ahora fue Böhmer quien sonó la alarma. Dijo que de inmediato compartió
el hallazgo, y su importancia, con la OMS y el Centro Europeo para la
Prevención y el Control de Enfermedades.

\href{https://www.nytimes3xbfgragh.onion/news-event/coronavirus?action=click\&pgtype=Article\&state=default\&region=MAIN_CONTENT_3\&context=storylines_faq}{}

\hypertarget{the-coronavirus-outbreak-}{%
\subsubsection{The Coronavirus Outbreak
›}\label{the-coronavirus-outbreak-}}

\hypertarget{frequently-asked-questions}{%
\paragraph{Frequently Asked
Questions}\label{frequently-asked-questions}}

Updated September 4, 2020

\begin{itemize}
\item ~
  \hypertarget{what-are-the-symptoms-of-coronavirus}{%
  \paragraph{What are the symptoms of
  coronavirus?}\label{what-are-the-symptoms-of-coronavirus}}

  \begin{itemize}
  \tightlist
  \item
    In the beginning, the coronavirus
    \href{https://www.nytimes3xbfgragh.onion/article/coronavirus-facts-history.html?action=click\&pgtype=Article\&state=default\&region=MAIN_CONTENT_3\&context=storylines_faq\#link-6817bab5}{seemed
    like it was primarily a respiratory illness}~--- many patients had
    fever and chills, were weak and tired, and coughed a lot, though
    some people don't show many symptoms at all. Those who seemed
    sickest had pneumonia or acute respiratory distress syndrome and
    received supplemental oxygen. By now, doctors have identified many
    more symptoms and syndromes. In April,
    \href{https://www.nytimes3xbfgragh.onion/2020/04/27/health/coronavirus-symptoms-cdc.html?action=click\&pgtype=Article\&state=default\&region=MAIN_CONTENT_3\&context=storylines_faq}{the
    C.D.C. added to the list of early signs}~sore throat, fever, chills
    and muscle aches. Gastrointestinal upset, such as diarrhea and
    nausea, has also been observed. Another telltale sign of infection
    may be a sudden, profound diminution of one's
    \href{https://www.nytimes3xbfgragh.onion/2020/03/22/health/coronavirus-symptoms-smell-taste.html?action=click\&pgtype=Article\&state=default\&region=MAIN_CONTENT_3\&context=storylines_faq}{sense
    of smell and taste.}~Teenagers and young adults in some cases have
    developed painful red and purple lesions on their fingers and toes
    --- nicknamed ``Covid toe'' --- but few other serious symptoms.
  \end{itemize}
\item ~
  \hypertarget{why-is-it-safer-to-spend-time-together-outside}{%
  \paragraph{Why is it safer to spend time together
  outside?}\label{why-is-it-safer-to-spend-time-together-outside}}

  \begin{itemize}
  \tightlist
  \item
    \href{https://www.nytimes3xbfgragh.onion/2020/05/15/us/coronavirus-what-to-do-outside.html?action=click\&pgtype=Article\&state=default\&region=MAIN_CONTENT_3\&context=storylines_faq}{Outdoor
    gatherings}~lower risk because wind disperses viral droplets, and
    sunlight can kill some of the virus. Open spaces prevent the virus
    from building up in concentrated amounts and being inhaled, which
    can happen when infected people exhale in a confined space for long
    stretches of time, said Dr. Julian W. Tang, a virologist at the
    University of Leicester.
  \end{itemize}
\item ~
  \hypertarget{why-does-standing-six-feet-away-from-others-help}{%
  \paragraph{Why does standing six feet away from others
  help?}\label{why-does-standing-six-feet-away-from-others-help}}

  \begin{itemize}
  \tightlist
  \item
    The coronavirus spreads primarily through droplets from your mouth
    and nose, especially when you cough or sneeze. The C.D.C., one of
    the organizations using that measure,
    \href{https://www.nytimes3xbfgragh.onion/2020/04/14/health/coronavirus-six-feet.html?action=click\&pgtype=Article\&state=default\&region=MAIN_CONTENT_3\&context=storylines_faq}{bases
    its recommendation of six feet}~on the idea that most large droplets
    that people expel when they cough or sneeze will fall to the ground
    within six feet. But six feet has never been a magic number that
    guarantees complete protection. Sneezes, for instance, can launch
    droplets a lot farther than six feet,
    \href{https://jamanetwork.com/journals/jama/fullarticle/2763852}{according
    to a recent study}. It's a rule of thumb: You should be safest
    standing six feet apart outside, especially when it's windy. But
    keep a mask on at all times, even when you think you're far enough
    apart.
  \end{itemize}
\item ~
  \hypertarget{i-have-antibodies-am-i-now-immune}{%
  \paragraph{I have antibodies. Am I now
  immune?}\label{i-have-antibodies-am-i-now-immune}}

  \begin{itemize}
  \tightlist
  \item
    As of right
    now,\href{https://www.nytimes3xbfgragh.onion/2020/07/22/health/covid-antibodies-herd-immunity.html?action=click\&pgtype=Article\&state=default\&region=MAIN_CONTENT_3\&context=storylines_faq}{~that
    seems likely, for at least several months.}~There have been
    frightening accounts of people suffering what seems to be a second
    bout of Covid-19. But experts say these patients may have a
    drawn-out course of infection, with the virus taking a slow toll
    weeks to months after initial exposure.~People infected with the
    coronavirus typically
    \href{https://www.nature.com/articles/s41586-020-2456-9}{produce}~immune
    molecules called antibodies, which are
    \href{https://www.nytimes3xbfgragh.onion/2020/05/07/health/coronavirus-antibody-prevalence.html?action=click\&pgtype=Article\&state=default\&region=MAIN_CONTENT_3\&context=storylines_faq}{protective
    proteins made in response to an
    infection}\href{https://www.nytimes3xbfgragh.onion/2020/05/07/health/coronavirus-antibody-prevalence.html?action=click\&pgtype=Article\&state=default\&region=MAIN_CONTENT_3\&context=storylines_faq}{.
    These antibodies may}~last in the body
    \href{https://www.nature.com/articles/s41591-020-0965-6}{only two to
    three months}, which may seem worrisome, but that's~perfectly normal
    after an acute infection subsides, said Dr. Michael Mina, an
    immunologist at Harvard University. It may be possible to get the
    coronavirus again, but it's highly unlikely that it would be
    possible in a short window of time from initial infection or make
    people sicker the second time.
  \end{itemize}
\item ~
  \hypertarget{what-are-my-rights-if-i-am-worried-about-going-back-to-work}{%
  \paragraph{What are my rights if I am worried about going back to
  work?}\label{what-are-my-rights-if-i-am-worried-about-going-back-to-work}}

  \begin{itemize}
  \tightlist
  \item
    Employers have to provide
    \href{https://www.osha.gov/SLTC/covid-19/standards.html}{a safe
    workplace}~with policies that protect everyone equally.
    \href{https://www.nytimes3xbfgragh.onion/article/coronavirus-money-unemployment.html?action=click\&pgtype=Article\&state=default\&region=MAIN_CONTENT_3\&context=storylines_faq}{And
    if one of your co-workers tests positive for the coronavirus, the
    C.D.C.}~has said that
    \href{https://www.cdc.gov/coronavirus/2019-ncov/community/guidance-business-response.html}{employers
    should tell their employees}~-\/- without giving you the sick
    employee's name -\/- that they may have been exposed to the virus.
  \end{itemize}
\end{itemize}

Ninguna organización incluyó el descubrimiento en sus informes
regulares.

Una semana después de recibir la información de Böhmer, los funcionarios
de salud europeos aún declaraban: ``Todavía no estamos seguros de si los
casos leves o asintomáticos pueden transmitir el virus''. No se mencionó
la evidencia genética.

Image

La doctora Böhmer había sido escéptica de la propagación asintomática,
pero su investigación finalmente arrojó pruebas genéticas de lo que
estaba ocurriendo.Credit...Laetitita Vancon para The New York Times

Image

``Esta fue una declaración engañosa de la OMS'', dijo Wendtner sobre los
comentarios de febrero del líder técnico de la organización sobre la
propagación asintomática.Credit...Laetitia Vancon para The New York
Times

Los funcionarios de la OMS dicen que el descubrimiento genético influyó
en su razonamiento, pero no lo anunciaron. Los funcionarios europeos de
salud dicen que la información alemana fue una de las primeras piezas de
una imagen emergente que todavía estaban armando.

Los médicos en Múnich estaban cada vez más frustrados y confundidos con
la Organización Mundial de la Salud. Primero, el grupo acreditó
erróneamente al gobierno chino de alertar a las autoridades alemanas
sobre la primera infección. Los funcionarios de gobierno y los médicos
dicen que la propia compañía de autopartes fue la que hizo sonar la
alarma.

Luego, el director de emergencias de la Organización Mundial de la
Salud, Michael Ryan, dijo el 27 de febrero que la importancia de la
propagación asintomática se estaba convirtiendo en un mito. Y Maria Van
Kerkhove, la líder técnica de la respuesta al coronavirus de la
organización, sugirió que no había nada de qué preocuparse.

``Es raro pero posible'',
\href{https://www.youtube.com/watch?v=SCgCzYAHusA\&t=22m55s}{dijo}. ``Es
muy raro''.

La OMS aún afirma que las personas que tosen o estornudan son más
contagiosas que las personas que no. Sin embargo, no hay consenso
científico acerca de la importancia de esta diferencia o cómo afecta la
propagación del virus.

Y así, con la evidencia acumulándose, el equipo de Múnich no podía
entender cómo la OMS podía estar tan segura de que la propagación
asintomática era insignificante.

``En este punto, para nosotros era claro'', dijo Wendtner, el médico
principal que supervisa el tratamiento de los pacientes de la COVID-19.
``Esta fue una declaración engañosa de la OMS''.

\hypertarget{si-esto-es-verdad-estamos-en-problemas}{%
\subsection{`Si esto es verdad, estamos en
problemas'}\label{si-esto-es-verdad-estamos-en-problemas}}

El foco de infección de Múnich no fue la única advertencia.

Las autoridades de salud chinas habían advertido explícitamente que los
pacientes eran contagiosos antes de mostrar síntomas. Un chofer de bus
japonés se infectó al transportar a turistas aparentemente sanos desde
Wuhan.

Y hacia mediados de febrero, 355 personas a bordo del crucero Diamond
Princess habían dado positivo. Alrededor de un tercio de los pasajeros y
el personal infectados no tenían síntomas.

Pero los funcionarios de salud pública consideraron que era peligroso
promover el riesgo de los propagadores silenciosos. Si poner en
cuarentena a las personas enfermas y rastrear a sus contactos no podía
contener de manera confiable la enfermedad, los gobiernos quizá
abandonarían esos esfuerzos por completo.

En Suecia y Gran Bretaña, por ejemplo, se discutió sobre cómo soportar
la epidemia hasta que la población obtuviera ``inmunidad de rebaño''.
Los funcionarios de salud pública temían que eso pudiera llevar a
hospitales abrumados y muertes innecesarias.

Image

Comensales disfrutaron de una noche en Estocolmo, en
abril.Credit...Andres Kudacki para The New York Times

Image

Un tren lleno de gente en São Paulo, Brasil, el mes
pasadoCredit...Victor Moriyama para The New York Times

Además, evitar la propagación silenciosa requería una iniciativa
enérgica de despistaje generalizado, pero en ese entonces eso era
imposible en la mayoría de los países.

``No es que tuviéramos una alternativa fácil'', dijo Libman, el médico
canadiense. ``El mensaje era básicamente: `Si esto es verdad, estamos en
problemas'''.

Los funcionarios europeos de salud dijeron que se mostraban renuentes a
reconocer la propagación silenciosa porque las pruebas estaban llegando
poco a poco y las consecuencias de una falsa alarma habrían sido graves.
``Estos informes se ven en todas partes, en todo el mundo'', dijo Josep
Jansa, un alto funcionario de salud de la Unión Europea. ``Lo que sea
que saquemos, no hay vuelta atrás''.

En retrospectiva, los funcionarios de salud debieron haber dicho que sí:
la propagación asintomática estaba ocurriendo y no entendían cuál era su
prevalencia, dijo Agoritsa Baka, médica de alto nivel de la Unión
Europea.

Sin embargo, hacerlo, dijo, habría sido igual a una advertencia
implícita a los países: lo que están haciendo quizá no sea suficiente.

\hypertarget{dejen-de-comprar-cubrebocas}{%
\subsection{`¡Dejen de comprar
cubrebocas!'}\label{dejen-de-comprar-cubrebocas}}

Si bien los funcionarios de salud pública dudaron, algunos médicos
actuaron. En una conferencia en Seattle a mediados de febrero, Jeffrey
Shaman, profesor de la Universidad de Columbia, dijo que
\href{https://science.sciencemag.org/content/368/6490/489}{su
investigación sugirió} que la rápida propagación de la COVID-19 solo
podía ser explicada si hubiera pacientes infectados con síntomas poco
notables o sin síntomas.

En la audiencia ese día estaba Steven Chu, físico ganador del Nobel y ex
secretario de energía de Estados Unidos. ``Si se deja desatendida, esta
enfermedad se diseminará a toda la población'', recuerda la advertencia
del profesor Shaman.

Luego, Chu comenzó a insistir en que los colegas sanos de su laboratorio
de la Universidad de Stanford usaran cubrebocas. Médicos en Cambridge,
Inglaterra, concluyeron que la transmisión asintomática era una gran
fuente de infecciones y aconsejaron a los empleados de salud locales y a
los pacientes que usaran cubrebocas mucho antes de que el gobierno
británico reconociera el riesgo de los esparcidores silenciosos.

Las autoridades estadounidenses, ante la escasez, desalentaron
activamente al público a comprar cubrebocas. ``En serio, gente: ¡DEJEN
DE COMPRAR CUBREBOCAS!'', tuiteó el 29 de febrero el cirujano general
Jerome M. Adams, el más alto funcionario de salud del gobierno federal
de Estados Unidos.

Para principios de marzo, cuando la OMS seguía afirmando que la
transmisión asintomática era poco común, la ciencia estaba tomando la
dirección opuesta.

Image

Compradores con cubrebocas en fila afuera de un Costco en Livermore,
CaliforniaCredit...Max Whittaker para The New York Times

Image

Producción de cubrebocas de tela en BangkokCredit...Adam Dean para The
New York Times

Los investigadores de Hong Kong
\href{https://www.nature.com/articles/s41591-020-0869-5.pdf}{calcularon
que} el 44 por ciento de las transmisiones de COVID-19 ocurrían antes de
que comenzaran los síntomas, un cálculo que coincidía con
\href{https://science.sciencemag.org/content/368/6491/eabb6936}{un
estudio británico} que señaló que ese número alcanzaba el 50 por ciento.

El estudio de Hong Kong
\href{https://www.nature.com/articles/s41591-020-0869-5.pdf}{concluyó}
que las personas se volvían infecciosas alrededor de dos días antes de
que surgiera su enfermedad, con un punto álgido en su primer día de
síntomas. Para cuando los pacientes sentían el primer dolor de cabeza o
la primera molestia en la garganta, quizá ya habían estado propagando la
enfermedad durante días.

En Bélgica, los médicos vieron esa matemática en acción, cuando la
COVID-10 atravesó los asilos de ancianos y mató a casi 5000 personas.

``Pensamos que al monitorear los síntomas y pedirle a las personas
enfermas que se quedaran en casa, seríamos capaces de controlar la
propagación'', dijo Steven Van Gucht, jefe del comité científico para la
COVID-19 en Bélgica. ``Llegó a través de personas con casi ningún
síntoma''.

Más de 700 personas a bordo del Diamond Princess estaban enfermas.
Catorce murieron. Los investigadores
\href{https://www.eurosurveillance.org/content/10.2807/1560-7917.ES.2020.25.10.2000180\#html_fulltext}{estiman}
que la mayor parte de la infección ocurrió mucho antes, cuando los
pasajeros aparentemente sanos socializaban y festejaban.

Científicos gubernamentales en Gran Bretaña
\href{https://assets.publishing.service.gov.uk/government/uploads/system/uploads/attachment_data/file/888804/S0399_Thirtieth_SAGE_meeting_on_Covid-19_.pdf}{concluyeron}
a fines de abril que del cinco al seis por ciento de los trabajadores de
la salud estaban infectados y asintomáticos y podrían haber propagado el
virus.

En Múnich, Hoelscher se había preguntado varias veces si las cosas
habrían sido distintas si los líderes del mundo se hubieran tomado en
serio el asunto antes. Comparó su respuesta a la de un conejo tropezando
con una serpiente venenosa.

``Estábamos viendo esa serpiente y de alguna manera estábamos
paralizados''.

\hypertarget{aceptaciuxf3n-o-no}{%
\subsection{Aceptación. O no}\label{aceptaciuxf3n-o-no}}

A medida que en marzo las investigaciones confluyeron, los oficiales de
salud europeos estaban convencidos.

``Está bien, este es realmente un gran problema'', recuerda Baka haber
pensado. ``Juega un papel importante en la transmisión''.

A finales del mes, los Centros para el Control de Enfermedades
anunciaron que estaban replanteando su política sobre los cubrebocas. Se
llegó a la conclusión de que hasta el
\href{https://www.nytimes3xbfgragh.onion/2020/03/31/health/coronavirus-asymptomatic-transmission.html}{25
por ciento} de los pacientes podrían no tener síntomas.

Desde entonces, los CDC, los gobiernos de todo el mundo y, finalmente,
la Organización Mundial de la Salud han recomendado que las personas
usen cubrebocas en público.

Aun así, la OMS está enviando señales confusas. A principios de este
mes, Maria Van Kerkhove, la dirigente técnica de la respuesta al
coronavirus de la organización, repitió que la transmisión de los
pacientes asintomáticos era ``muy poco común''. Después de que los
médicos se quejaron, la agencia dijo que hubo un malentendido.

``Para ser totalmente honesta, aún no tenemos un panorama claro al
respecto'', comentó Van Kerkhove. Ella dijo que se había estado
refiriendo a algunos estudios que mostraban una transmisión limitada de
pacientes asintomáticos.

Image

La doctora Rothe en casaCredit...Laetitia Vancon para The New York Times

Image

La doctora Böhmer publicó un estudio en The Lancet el mes pasado que
encontró una transmisión ``sustancial'' de personas sin síntomas o
síntomas excepcionalmente leves y no específicos.~Credit...Laetitia
Vancon para The New York Times

Los recientes anuncios en internet confundieron el asunto aún más. Una
búsqueda en Google a mediados de junio de estudios sobre transmisión
asintomática tuvo como respuesta un anuncio de la OMS titulado:
``Personas asintomáticas: rara vez propagan el coronavirus''.

Sin embargo, al hacer clic en el enlace, aparecía una imagen mucho más
matizada: ``Algunos informes han indicado que las personas asintomáticas
pueden transmitir el virus. Todavía no se sabe con qué frecuencia
sucede''.

Después de que el Times preguntó sobre esas discrepancias, la
organización eliminó los anuncios.

De regreso en Múnich, quedan pocas dudas. Böhmer, la médico del gobierno
bávaro, publicó
\href{https://www.thelancet.com/journals/laninf/article/PIIS1473-3099(20)30314-5/fulltex}{un
estudio en The Lancet} el mes pasado basado en entrevistas e información
genética exhaustivas para rastrear de manera metódica todos los casos
del foco de infección.

En los meses posteriores a las pruebas que Rothe le hizo a su primer
paciente, 16 personas infectadas fueron identificadas y detectadas de
manera temprana. Todos sobrevivieron. Las iniciativas enérgicas de
pruebas y el rastreo impecable de contactos contuvieron la propagación.

El estudio de Böhmer halló transmisiones ``importantes'' de personas
asintomáticas o con síntomas excepcionalmente leves y no específicos.

Rothe y sus colegas recibieron una nota al pie.

Advertisement

\protect\hyperlink{after-bottom}{Continue reading the main story}

\hypertarget{site-index}{%
\subsection{Site Index}\label{site-index}}

\hypertarget{site-information-navigation}{%
\subsection{Site Information
Navigation}\label{site-information-navigation}}

\begin{itemize}
\tightlist
\item
  \href{https://help.nytimes3xbfgragh.onion/hc/en-us/articles/115014792127-Copyright-notice}{©~2020~The
  New York Times Company}
\end{itemize}

\begin{itemize}
\tightlist
\item
  \href{https://www.nytco.com/}{NYTCo}
\item
  \href{https://help.nytimes3xbfgragh.onion/hc/en-us/articles/115015385887-Contact-Us}{Contact
  Us}
\item
  \href{https://www.nytco.com/careers/}{Work with us}
\item
  \href{https://nytmediakit.com/}{Advertise}
\item
  \href{http://www.tbrandstudio.com/}{T Brand Studio}
\item
  \href{https://www.nytimes3xbfgragh.onion/privacy/cookie-policy\#how-do-i-manage-trackers}{Your
  Ad Choices}
\item
  \href{https://www.nytimes3xbfgragh.onion/privacy}{Privacy}
\item
  \href{https://help.nytimes3xbfgragh.onion/hc/en-us/articles/115014893428-Terms-of-service}{Terms
  of Service}
\item
  \href{https://help.nytimes3xbfgragh.onion/hc/en-us/articles/115014893968-Terms-of-sale}{Terms
  of Sale}
\item
  \href{https://spiderbites.nytimes3xbfgragh.onion}{Site Map}
\item
  \href{https://help.nytimes3xbfgragh.onion/hc/en-us}{Help}
\item
  \href{https://www.nytimes3xbfgragh.onion/subscription?campaignId=37WXW}{Subscriptions}
\end{itemize}
