Sections

SEARCH

\protect\hyperlink{site-content}{Skip to
content}\protect\hyperlink{site-index}{Skip to site index}

\href{https://www.nytimes3xbfgragh.onion/es/section/mundo}{Mundo}

\href{https://myaccount.nytimes3xbfgragh.onion/auth/login?response_type=cookie\&client_id=vi}{}

\href{https://www.nytimes3xbfgragh.onion/section/todayspaper}{Today's
Paper}

\href{/es/section/mundo}{Mundo}\textbar{}Un hombre de 30 años muere
después de asistir a una `fiesta COVID', dice un hospital de Texas

\url{https://nyti.ms/30cIFJw}

\begin{itemize}
\item
\item
\item
\item
\item
\end{itemize}

\hypertarget{el-brote-de-coronavirus}{%
\subsubsection{\texorpdfstring{\href{https://www.nytimes3xbfgragh.onion/es/spotlight/coronavirus?name=styln-coronavirus-es\&region=TOP_BANNER\&block=storyline_menu_recirc\&action=click\&pgtype=Article\&impression_id=64d8ec20-f52e-11ea-ac8a-f5eef0546dac\&variant=undefined}{El
brote de
coronavirus}}{El brote de coronavirus}}\label{el-brote-de-coronavirus}}

\begin{itemize}
\tightlist
\item
  \href{https://www.nytimes3xbfgragh.onion/es/interactive/2020/espanol/mundo/coronavirus-en-estados-unidos.html?name=styln-coronavirus-es\&region=TOP_BANNER\&block=storyline_menu_recirc\&action=click\&pgtype=Article\&impression_id=64d91330-f52e-11ea-ac8a-f5eef0546dac\&variant=undefined}{Casos
  en Estados Unidos}
\item
  \href{https://www.nytimes3xbfgragh.onion/es/interactive/2020/08/06/espanol/ciencia-y-tecnologia/tengo-covid-19-sintomas.html?name=styln-coronavirus-es\&region=TOP_BANNER\&block=storyline_menu_recirc\&action=click\&pgtype=Article\&impression_id=64d91331-f52e-11ea-ac8a-f5eef0546dac\&variant=undefined}{Síntomas}
\item
  \href{https://www.nytimes3xbfgragh.onion/es/2020/09/11/espanol/ciencia-y-tecnologia/cerebro-coronavirus.html?name=styln-coronavirus-es\&region=TOP_BANNER\&block=storyline_menu_recirc\&action=click\&pgtype=Article\&impression_id=64d91332-f52e-11ea-ac8a-f5eef0546dac\&variant=undefined}{El
  efecto en el cerebro}
\item
  \href{https://www.nytimes3xbfgragh.onion/es/2020/09/09/espanol/ciencia-y-tecnologia/salud-mental-coronavirus.html?name=styln-coronavirus-es\&region=TOP_BANNER\&block=storyline_menu_recirc\&action=click\&pgtype=Article\&impression_id=64d91333-f52e-11ea-ac8a-f5eef0546dac\&variant=undefined}{Convalecencia
  prolongada}
\item
  \href{https://www.nytimes3xbfgragh.onion/es/2020/09/08/espanol/ciencia-y-tecnologia/dentistas-covid-dientes.html?name=styln-coronavirus-es\&region=TOP_BANNER\&block=storyline_menu_recirc\&action=click\&pgtype=Article\&impression_id=64d91334-f52e-11ea-ac8a-f5eef0546dac\&variant=undefined}{La
  salud dental y el virus}
\end{itemize}

Advertisement

\protect\hyperlink{after-top}{Continue reading the main story}

Supported by

\protect\hyperlink{after-sponsor}{Continue reading the main story}

\hypertarget{un-hombre-de-30-auxf1os-muere-despuuxe9s-de-asistir-a-una-fiesta-covid-dice-un-hospital-de-texas}{%
\section{Un hombre de 30 años muere después de asistir a una `fiesta
COVID', dice un hospital de
Texas}\label{un-hombre-de-30-auxf1os-muere-despuuxe9s-de-asistir-a-una-fiesta-covid-dice-un-hospital-de-texas}}

``Pensé que era un engaño'', dijo el hombre a su enfermero, informó una
funcionaria del hospital.

\includegraphics{https://static01.graylady3jvrrxbe.onion/images/2020/07/12/multimedia/13Coronafiesta-ES-1/merlin_174003033_d9beba7a-f0eb-4516-a79d-c7d1b944a6f1-articleLarge.jpg?quality=75\&auto=webp\&disable=upscale}

Por \href{https://www.nytimes3xbfgragh.onion/by/bryan-pietsch}{Bryan
Pietsch}

\begin{itemize}
\item
  Publicado 13 de julio de 2020Actualizado 14 de julio de 2020
\item
  \begin{itemize}
  \item
  \item
  \item
  \item
  \item
  \end{itemize}
\end{itemize}

\href{https://www.nytimes3xbfgragh.onion/2020/07/12/us/30-year-old-covid-party-death.html}{Read
in English}

\href{https://www.nytimes3xbfgragh.onion/newsletters/el-times}{Regístrate
para recibir nuestro boletín} con lo mejor de The New York Times.

\begin{center}\rule{0.5\linewidth}{\linethickness}\end{center}

Un hombre de 30 años que creía que el coronavirus era un engaño y
asistió a una ``Fiesta COVID'' murió tras ser infectado con el virus,
según un hospital de Texas.

El hombre había asistido a una reunión con una persona infectada para
probar si el coronavirus era real, dijo Jane Appleby, directora médica
del Hospital Metodista en San Antonio, donde murió el hombre.

Ella no dijo cuándo tuvo lugar la fiesta, cuántas personas asistieron o
cuánto tiempo después del evento el hombre fue hospitalizado con la
COVID-19, la enfermedad causada por el coronavirus. El hombre no fue
públicamente identificado.

La premisa de tales fiestas es probar si el virus realmente existe o
exponer intencionalmente a las personas al coronavirus en un intento de
obtener inmunidad. Algunos expertos y funcionarios públicos de salud han
tenido dudas sobre si hay ``fiestas COVID'' o en qué medida se celebran.

{[}\emph{Consulta el}
\textbf{\href{https://www.nytimes3xbfgragh.onion/es/interactive/2020/espanol/mundo/coronavirus-en-estados-unidos.html}{\emph{mapa
de coronavirus en Estados
Unidos}}}\href{https://www.nytimes3xbfgragh.onion/es/interactive/2020/espanol/mundo/coronavirus-en-estados-unidos.html}{**}\emph{con
los datos detallados de casos y muertes{]}}

Appleby declaró que el hombre le había dicho a su enfermero que había
asistido a una fiesta COVID. Antes de morir, contó, el paciente le dijo
a su enfermero: ``Creo que cometí un error. Pensé que esto era un
engaño, pero no lo es''.

Appleby declaró que estaba compartiendo la historia para advertir a
otros, especialmente en Texas, donde los casos de coronavirus están
aumentando.

El sábado, hubo 8332 nuevos casos de coronavirus confirmados en el
estado, según
\href{https://www.nytimes3xbfgragh.onion/interactive/2020/us/texas-coronavirus-cases.html}{una
base de datos de The New York Times}. Hasta el momento, se han
registrado más de 260.000 casos y más de 3200 muertes en Texas.

Las fiestas COVID son ``peligrosas, irresponsables y potencialmente
mortales'', dijo Robert Glatter, médico de emergencias en el Hospital
Lenox Hill en Manhattan.

``Asistir a una fiesta de ese tipo puede ser la vía a un deceso
temprano, si no fatiga crónica e implacable, dolor en el pecho,
dificultad para respirar y fiebres diarias, si sobrevives'', dijo
Glatter.

Los funcionarios de salud del condado en el sureste de Washington
\href{https://www.nytimes3xbfgragh.onion/2020/05/06/us/coronavirus-covid-parties.html}{informaron
en mayo} que tenían evidencia de que al menos dos casos de coronavirus
estaban relacionados con una o más de las llamadas fiestas COVID-19,
pero rápidamente se retractaron, y dijeron que las fiestas podrían haber
sido reuniones más inocentes.

Antes de que existiera una vacuna contra la varicela, las personas
organizaban fiestas de varicela para infectar a sus hijos con la
enfermedad, ya que se pensaba que era más peligroso contraerla de
adulto.

Ahora que está disponible, la vacuna es la forma más segura de
protegerse contra la varicela, aunque algunos,
\href{https://www.nytimes3xbfgragh.onion/2019/03/21/us/kentucky-governor-chickenpox.html}{incluido
el exgobernador de Kentucky, Matt Bevin}, aún permiten que sus hijos
participen en tales reuniones para contraer la enfermedad.

El coronavirus no se comporta como la varicela, dijo Glatter, y las
fiestas para ambos virus no se deben celebrar.

En Alabama,
\href{https://www.wbrc.com/2020/07/03/covid-parties-tuscaloosa-whats-really-going/}{informes
de que estudiantes se reunían} para apostar quién se infectaría primero
con el virus ---el enfermo ganador se llevaba a casa una cantidad de
dinero---
\href{https://abc3340.com/news/local/university-of-alabama-responds-to-covid-parties-being-thrown-in-tuscaloosa}{provocó
advertencias a los estudiantes de la Universidad de Alabama} sobre los
riesgos de tales fiestas, aunque los funcionarios estatales de salud no
pudieron confirmar que existieran los eventos.

Estados Unidos alcanzó recientemente
\href{https://www.nytimes3xbfgragh.onion/es/interactive/2020/espanol/mundo/coronavirus-en-estados-unidos.html}{un
número récord} de nuevos casos por día, con más de 68.000 casos
confirmados el viernes 10.

Todavía no se ha comprobado que infectarse con el coronavirus
proporcione inmunidad, así que una reinfección aún es posible.

En un
\href{https://www.nytimes3xbfgragh.onion/2020/04/08/opinion/coronavirus-parties-herd-immunity.html}{artículo
de opinión} para el Times, Greta Bauer, profesora de epidemiología y
bioestadística, advirtió contra las llamadas fiestas de coronavirus y
señaló que incluso los jóvenes pueden ser hospitalizados y sufrir daños
a largo plazo por el virus.

``Es importante que no tomemos riesgos innecesarios con consecuencias
desconocidas'', escribió Bauer. ``Si podemos evitar la infección,
tenemos que hacer exactamente eso''.

Los Centros para el Control y la Prevención de Enfermedades de Estados
Unidos advierten que las personas infectadas con el coronavirus
\href{https://espanol.cdc.gov/coronavirus/2019-ncov/community/large-events/considerations-for-events-gatherings.html}{no
deben asistir a reuniones}, y que hay un alto riesgo inherente en
cualquier evento en el que las personas se mezclan sin cubrebocas o
distanciamiento social.

Un hombre de California
\href{https://www.washingtonpost.com/nation/2020/07/02/man-who-went-party-warned-people-not-be-an-idiot-like-me-day-before-dying-covid-19/}{murió
de la COVID-19 después de asistir a una fiesta} ---que no fue celebrada
con el propósito de infectar a los asistentes--- en la que las personas
no usaron mascarillas y a la que una persona infectada había asistido.

Bryan Pietsch es un reportero de temas generales en la sección Express
en The New York Times.
\href{https://twitter.com/bryan_pietsch}{@bryan\_pietsch}

Advertisement

\protect\hyperlink{after-bottom}{Continue reading the main story}

\hypertarget{site-index}{%
\subsection{Site Index}\label{site-index}}

\hypertarget{site-information-navigation}{%
\subsection{Site Information
Navigation}\label{site-information-navigation}}

\begin{itemize}
\tightlist
\item
  \href{https://help.nytimes3xbfgragh.onion/hc/en-us/articles/115014792127-Copyright-notice}{©~2020~The
  New York Times Company}
\end{itemize}

\begin{itemize}
\tightlist
\item
  \href{https://www.nytco.com/}{NYTCo}
\item
  \href{https://help.nytimes3xbfgragh.onion/hc/en-us/articles/115015385887-Contact-Us}{Contact
  Us}
\item
  \href{https://www.nytco.com/careers/}{Work with us}
\item
  \href{https://nytmediakit.com/}{Advertise}
\item
  \href{http://www.tbrandstudio.com/}{T Brand Studio}
\item
  \href{https://www.nytimes3xbfgragh.onion/privacy/cookie-policy\#how-do-i-manage-trackers}{Your
  Ad Choices}
\item
  \href{https://www.nytimes3xbfgragh.onion/privacy}{Privacy}
\item
  \href{https://help.nytimes3xbfgragh.onion/hc/en-us/articles/115014893428-Terms-of-service}{Terms
  of Service}
\item
  \href{https://help.nytimes3xbfgragh.onion/hc/en-us/articles/115014893968-Terms-of-sale}{Terms
  of Sale}
\item
  \href{https://spiderbites.nytimes3xbfgragh.onion}{Site Map}
\item
  \href{https://help.nytimes3xbfgragh.onion/hc/en-us}{Help}
\item
  \href{https://www.nytimes3xbfgragh.onion/subscription?campaignId=37WXW}{Subscriptions}
\end{itemize}
