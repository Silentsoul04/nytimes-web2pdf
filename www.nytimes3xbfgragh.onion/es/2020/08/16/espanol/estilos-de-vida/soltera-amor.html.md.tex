Sections

SEARCH

\protect\hyperlink{site-content}{Skip to
content}\protect\hyperlink{site-index}{Skip to site index}

\href{https://www.nytimes3xbfgragh.onion/es/section/estilos-de-vida}{Estilos
de Vida}

\href{https://myaccount.nytimes3xbfgragh.onion/auth/login?response_type=cookie\&client_id=vi}{}

\href{https://www.nytimes3xbfgragh.onion/section/todayspaper}{Today's
Paper}

\href{/es/section/estilos-de-vida}{Estilos de Vida}\textbar{}Cuando se
sobrestima el matrimonio

\url{https://nyti.ms/3kR4UOj}

\begin{itemize}
\item
\item
\item
\item
\item
\end{itemize}

Advertisement

\protect\hyperlink{after-top}{Continue reading the main story}

Supported by

\protect\hyperlink{after-sponsor}{Continue reading the main story}

Modern Love

\hypertarget{cuando-se-sobrestima-el-matrimonio}{%
\section{Cuando se sobrestima el
matrimonio}\label{cuando-se-sobrestima-el-matrimonio}}

Estar soltera a tus 30 años puede ser lo mismo que esperar en fila para
entrar a un club de moda: cuando logras entrar puedes pensar, ¿esto era
todo?

\includegraphics{https://static01.graylady3jvrrxbe.onion/images/2020/08/09/fashion/09MODERN-MARRIAGECLUB/09MODERN-MARRIAGECLUB-articleLarge.jpg?quality=75\&auto=webp\&disable=upscale}

Por Katerina Tsasis

\begin{itemize}
\item
  16 de agosto de 2020
\item
  \begin{itemize}
  \item
  \item
  \item
  \item
  \item
  \end{itemize}
\end{itemize}

\href{https://www.nytimes3xbfgragh.onion/2020/08/07/style/modern-love-when-marriage-is-just-another-overhyped-nightclub.html}{Read
in English}

\href{https://www.nytimes3xbfgragh.onion/newsletters/el-times}{Regístrate
para recibir nuestro boletín} con lo mejor de The New York Times.

\begin{center}\rule{0.5\linewidth}{\linethickness}\end{center}

La gente te trata de manera distinta cuando eres soltera. No todos ni
todo el tiempo, no de manera explícita ni necesariamente grosera. Te
preguntan por qué nadie se ha fijado en ti, ofrecen organizarte citas a
ciegas y llevarte a mesas de solteros en eventos formales. Te invitan a
cenas a último momento cuando alguien más les cancela.

Te hacen sentir como si no fueras parte de la norma, a pesar de que los
datos del censo estadounidense nos dicen que, de hecho, la soltería es
cada vez más la norma.

De niña, pertenecía a una comunidad migrante que consideraba el
matrimonio y la maternidad como el principal objetivo de las mujeres en
la vida. Las historias que me rodeaban estaban llenas de bodas y finales
felices: \emph{Friends}, \emph{Sex and the City}, \emph{Full House}.
Todas las comedias románticas, todas las comedias. \emph{Pride and
Prejudice}, \emph{Little Women}, todos los cuentos de hadas. Brangelina,
Kim y Kanye, el interés desbordado que los estadounidenses muestran por
las bodas de la realeza británica.

Yo hice lo típico: ir a la universidad, trabajar, hacer amigos, salir,
conocer hombres en bares, en la escuela, en la oficina. Conocer personas
era fácil: forjar relaciones era difícil. Era a principios de la década
de 2000 en Los Ángeles, un lugar donde parecía que todos querían
mantener abiertas sus opciones. Con frecuencia me encontraba en el
purgatorio de las relaciones: veía a alguien pero, en realidad, no era
una cita; tenía citas pero no tenía una relación; o estaba en una
relación pero no tenía un futuro en ella.

Fue más o menos en esa época que mi hermana menor terminó la universidad
y anunció su compromiso. Yo estaba a punto de mudarme al extranjero para
estudiar una maestría de administración empresarial. Los comentarios de
las señoras de mi familia se volvieron más incisivos. ``¡No esperes
demasiado!'', decían, en broma pero a la vez en serio. Desde su
perspectiva, estaba invirtiendo demasiado tiempo en las prioridades
equivocadas. A los 26 años, necesitaba ponerme manos a la obra.

``¿Aún planeas irte?'', me preguntó mi madre.

Esa es otra cosa que ocurre cuando estás soltera: tu tiempo y tus planes
se perciben como algo menos fijo y menos válido, a diferencia de los
casados.

Se espera que te esfuerces por ver a tus seres queridos en las
festividades o que te quedes más tarde en el trabajo cuando tus colegas
deben recoger a sus hijos. Con la boda de mi hermana en el horizonte,
había una expectativa implícita de que no me perdería ninguna de las
etapas que llevarían al feliz suceso.

De cualquier manera me fui a Europa.

Cuando viajé a casa para ir a la boda de mi hermana, el agente de
aduanas quedó confundido por mi maleta de gimnasio arrugada con dos
mudas de ropa.

``¿Eso es todo lo que traes?'', me preguntó.

Jamás he sentido menos estorbos antes ni desde entonces, pues había
empacado de manera tan ligera que sentía como si estuviera flotando,
dispuesta a regresar a mis aventuras.

A lo largo del año siguiente aprendí nuevas cosas, viajé a una decena de
países, practiqué otros idiomas, vi una ópera montada en los escalones
de un castillo, subí el monte Kilimanjaro, manejé por la aterradora
glorieta del Arco de Triunfo.

También fue un año en el que experimenté agresivos acercamientos por
parte de compañeros varones, ``frases de chicos'' con las que aderezaban
las conversaciones casuales, así como un flujo constante de sexismo
sutil y explicaciones condescendientes. La idea de salir a citas jamás
me pareció más desalentadora ni menos atractiva.

Cuando regresé a California, me encontré con que muchas de mis amigas
habían comenzado relaciones serias que se dirigían al matrimonio. En ese
momento, había dejado de creer que necesitamos una pareja para tener una
vida plena, pero aún pensaba que debía carecer de algo fundamental
---quizá no era lo suficientemente buena, no tenía el atractivo
necesario, ni la amabilidad suficiente o algo no era suficiente--- en
comparación.

Los amigos, familiares, conocidos e incluso extraños con cortesía dirán
que tú, como soltera, pareces estar rezagada. Una amiga mía fue a ver a
un doctor respecto a una pregunta de salud mental, y su consejo fue que
necesitaba un novio. Los familiares bienintencionados la animaron a ir a
la iglesia para encontrar a un hombre, aunque es agnóstica.

Me han dicho que soy demasiado selectiva, que no me volveré más joven,
que debería salir más, que debo luchar por el amor y que debo buscar a
un hombre que sea más atractivo y menos atractivo, más cerebrito y menos
cerebrito, más asertivo y menos asertivo.

Hombres que apenas conozco o que no conozco del todo me han dicho que
debo maquillarme más, cambiar mi actitud, hacer más sentadillas,
vestirme distinto, sonreír más. Lo he escuchado en una primera cita,
mientras caminaba por la calle haciendo mis cosas y en medio de una
conversación sobre un tema totalmente distinto.

Es extraño seguir buscando a la persona ``adecuada'' mientras combato la
expectativa de hacer precisamente eso. Seguía conociendo a personas: en
horas felices, grupos de reunión, citas en línea. Probé cosas nuevas:
clases de salsa, viajes en \emph{scooter}, exploración de cuevas. Hice
amistades, practiqué pasatiempos y viví aventuras.

Mezclados con la diversión hubo momentos tristes y solitarios, malas
relaciones y rompimientos dolorosos, pero ya no creía que me faltara
algo, a pesar de las pistas que me seguían enviando amigos, familiares y
la sociedad. La vida me parecía buena, satisfactoria y plena. No tenía
que esperar a que alguien más hiciera mi final feliz.

Pero a mediados de mis treinta, me había mudado a Austin, Texas, y mis
padres comenzaron a mostrarse preocupados a larga distancia. Sus vidas
no habían sido fáciles, y solo habían podido depender el uno del otro. A
mi padre le preocupaba que no tuviera a alguien que me cuidara. ¿Y si me
enfermaba? ¿Y si necesitaba ayuda?

Mi madre, consternada por mi incapacidad de encontrar a alguien, dijo:
``¡No es como si ella tuviera tres cabezas!''.

Tuve más citas. Tuve citas que se esfumaban más rápido que la espuma de
un capuchino. Fui a una hora feliz en la que bebí demasiado con el
estómago vacío y les pagué una ronda a todos en el bar. Tuve una cena
con alguien que seguía disculpándose por responder su celular. Estuve en
una relación con alguien que no estaba listo para comprometerse. Mantuve
una relación con alguien que anhelaba estar con su ex.

Y después, tuve una relación que funcionó.

No había nada de magia al respecto, ningún despertar del alma, ningún
reconocimiento personal, ninguna razón organizada ni clara de por qué
funcionaba y las demás no. Conocí a un hombre que es un ser humano
encantador. Encontramos intereses compartidos y química. Nos tratamos
con amabilidad y respeto. Estoy segura de que, si lo hubiera conocido
años antes o años después, el resultado habría sido el mismo: nos
casamos.

Soy la misma persona que vive en el mismo lugar y hace el mismo trabajo
con los mismos amigos y los mismos pasatiempos. No había nada peor en mí
antes. No hay nada mejor en mí ahora. Sin embargo, la gente que veía mi
soltería con curiosidad, lástima o menosprecio ahora son más cálidos y
cordiales. Es como si me hubiera unido al club.

Me hacen menos preguntas sobre mi vida personal. Nos invitan a mi marido
y a mí a salidas con otras parejas. Cuando decido quedarme en casa en
las festividades en lugar de viajar para visitar a la familia nadie lo
cuestiona y nadie se queja. Los acercamientos no deseados se acaban con
las palabras ``estoy casada'' cuando antes un ``no, gracias'' resultaba
insuficiente.

¿En realidad qué dice la declaración legal de alguien más acerca de ti?
¿Te valida? ¿Te hace parecer más normal? ¿Establece nuevos límites a tu
alrededor? ¿Te hace parecer más segura?

Amo a mi pareja y disfruto compartir nuestro día a día, pero el
matrimonio ---esa cosa que les enseñan a venerar a las jóvenes--- no ha
transformado mi vida. Es más como tejer nuevos hilos en un tapiz
existente, que tirar a la basura un patrón para dar lugar a otro más
colorido.

Cuando vivía en Los Ángeles, solía salir con amigos y esperar por horas
en la cola para entrar a un club nuevo y exclusivo, solo para por fin
entrar y ver que no era la gran cosa. La presión social sobre el
matrimonio se siente así, un énfasis en entrar por la puerta sin
preocuparse de qué hay más allá.

Cada experiencia es distinta. Solo puedo describir la mía. Castigamos y
recompensamos a las personas por la manera en que se ajustan a nuestros
ideales sin siquiera darnos cuenta. Nos castigamos cuando las cosas que
nos dicen evitan que apreciemos y disfrutemos las cosas que tenemos.

Quizá alguien lea esto y le parezca que mis ideas son evidentes,
trilladas o anticuadas. Quizá alguien lea esto y crea que me he perdido
de cosas en la vida. Pero igual lo estoy escribiendo por las ocasiones
en que he pensado: ``Quizá estoy imaginando cosas'', ``quizá tienen
razón'' o ``quizá hay algo que está mal en mi vida''.

Lo que quiero decirles a mis amigos que sienten la presión de la familia
o la sociedad mientras exploran las citas, las relaciones o la soltería,
y a quienes les han dicho que de alguna manera están incompletos porque
están solos es esto: no es cierto. Todos tenemos derecho a una vida
plena y llena de propósitos, sin necesidad de tener que celebrar una
boda.

\href{https://www.kt.marketing/}{Katerina Tsasis} es estratega de
mercadotecnia y escritora en Austin, Texas.

Advertisement

\protect\hyperlink{after-bottom}{Continue reading the main story}

\hypertarget{site-index}{%
\subsection{Site Index}\label{site-index}}

\hypertarget{site-information-navigation}{%
\subsection{Site Information
Navigation}\label{site-information-navigation}}

\begin{itemize}
\tightlist
\item
  \href{https://help.nytimes3xbfgragh.onion/hc/en-us/articles/115014792127-Copyright-notice}{©~2020~The
  New York Times Company}
\end{itemize}

\begin{itemize}
\tightlist
\item
  \href{https://www.nytco.com/}{NYTCo}
\item
  \href{https://help.nytimes3xbfgragh.onion/hc/en-us/articles/115015385887-Contact-Us}{Contact
  Us}
\item
  \href{https://www.nytco.com/careers/}{Work with us}
\item
  \href{https://nytmediakit.com/}{Advertise}
\item
  \href{http://www.tbrandstudio.com/}{T Brand Studio}
\item
  \href{https://www.nytimes3xbfgragh.onion/privacy/cookie-policy\#how-do-i-manage-trackers}{Your
  Ad Choices}
\item
  \href{https://www.nytimes3xbfgragh.onion/privacy}{Privacy}
\item
  \href{https://help.nytimes3xbfgragh.onion/hc/en-us/articles/115014893428-Terms-of-service}{Terms
  of Service}
\item
  \href{https://help.nytimes3xbfgragh.onion/hc/en-us/articles/115014893968-Terms-of-sale}{Terms
  of Sale}
\item
  \href{https://spiderbites.nytimes3xbfgragh.onion}{Site Map}
\item
  \href{https://help.nytimes3xbfgragh.onion/hc/en-us}{Help}
\item
  \href{https://www.nytimes3xbfgragh.onion/subscription?campaignId=37WXW}{Subscriptions}
\end{itemize}
