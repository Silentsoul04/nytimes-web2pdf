Sections

SEARCH

\protect\hyperlink{site-content}{Skip to
content}\protect\hyperlink{site-index}{Skip to site index}

\href{https://www.nytimes3xbfgragh.onion/es/section/america-latina}{América
Latina}

\href{https://myaccount.nytimes3xbfgragh.onion/auth/login?response_type=cookie\&client_id=vi}{}

\href{https://www.nytimes3xbfgragh.onion/section/todayspaper}{Today's
Paper}

\href{/es/section/america-latina}{América Latina}\textbar{}Brasil es el
laboratorio ideal para buscar la vacuna contra el coronavirus

\url{https://nyti.ms/3iNdZWL}

\begin{itemize}
\item
\item
\item
\item
\item
\end{itemize}

\hypertarget{el-brote-de-coronavirus}{%
\subsubsection{\texorpdfstring{\href{https://www.nytimes3xbfgragh.onion/es/spotlight/coronavirus?name=styln-coronavirus-es\&region=TOP_BANNER\&block=storyline_menu_recirc\&action=click\&pgtype=Article\&impression_id=e3193ab0-f4b9-11ea-b32a-256dc0696bd9\&variant=undefined}{El
brote de
coronavirus}}{El brote de coronavirus}}\label{el-brote-de-coronavirus}}

\begin{itemize}
\tightlist
\item
  \href{https://www.nytimes3xbfgragh.onion/es/interactive/2020/espanol/mundo/coronavirus-en-estados-unidos.html?name=styln-coronavirus-es\&region=TOP_BANNER\&block=storyline_menu_recirc\&action=click\&pgtype=Article\&impression_id=e3193ab1-f4b9-11ea-b32a-256dc0696bd9\&variant=undefined}{Casos
  en Estados Unidos}
\item
  \href{https://www.nytimes3xbfgragh.onion/es/interactive/2020/08/06/espanol/ciencia-y-tecnologia/tengo-covid-19-sintomas.html?name=styln-coronavirus-es\&region=TOP_BANNER\&block=storyline_menu_recirc\&action=click\&pgtype=Article\&impression_id=e3193ab2-f4b9-11ea-b32a-256dc0696bd9\&variant=undefined}{Síntomas}
\item
  \href{https://www.nytimes3xbfgragh.onion/es/2020/09/11/espanol/ciencia-y-tecnologia/cerebro-coronavirus.html?name=styln-coronavirus-es\&region=TOP_BANNER\&block=storyline_menu_recirc\&action=click\&pgtype=Article\&impression_id=e3193ab3-f4b9-11ea-b32a-256dc0696bd9\&variant=undefined}{El
  efecto en el cerebro}
\item
  \href{https://www.nytimes3xbfgragh.onion/es/2020/09/09/espanol/ciencia-y-tecnologia/salud-mental-coronavirus.html?name=styln-coronavirus-es\&region=TOP_BANNER\&block=storyline_menu_recirc\&action=click\&pgtype=Article\&impression_id=e31961c0-f4b9-11ea-b32a-256dc0696bd9\&variant=undefined}{Convalecencia
  prolongada}
\item
  \href{https://www.nytimes3xbfgragh.onion/es/2020/09/08/espanol/ciencia-y-tecnologia/dentistas-covid-dientes.html?name=styln-coronavirus-es\&region=TOP_BANNER\&block=storyline_menu_recirc\&action=click\&pgtype=Article\&impression_id=e31961c1-f4b9-11ea-b32a-256dc0696bd9\&variant=undefined}{La
  salud dental y el virus}
\end{itemize}

Advertisement

\protect\hyperlink{after-top}{Continue reading the main story}

Supported by

\protect\hyperlink{after-sponsor}{Continue reading the main story}

América del Sur

\hypertarget{brasil-es-el-laboratorio-ideal-para-buscar-la-vacuna-contra-el-coronavirus}{%
\section{Brasil es el laboratorio ideal para buscar la vacuna contra el
coronavirus}\label{brasil-es-el-laboratorio-ideal-para-buscar-la-vacuna-contra-el-coronavirus}}

El contagio generalizado, una amplia reserva nacional de científicos y
una sólida infraestructura de fabricación de inmunizaciones han
convertido al país en un actor importante en la búsqueda de una vacuna.

\includegraphics{https://static01.graylady3jvrrxbe.onion/images/2020/08/16/world/17Brazil-vac-ES-00/merlin_175106148_7983adc4-cfe5-443f-9c34-7dab22b57205-articleLarge.jpg?quality=75\&auto=webp\&disable=upscale}

Por Manuela Andreoni y
\href{https://www.nytimes3xbfgragh.onion/by/ernesto-londono}{Ernesto
Londoño}

\begin{itemize}
\item
  17 de agosto de 2020
\item
  \begin{itemize}
  \item
  \item
  \item
  \item
  \item
  \end{itemize}
\end{itemize}

\href{https://www.nytimes3xbfgragh.onion/2020/08/15/world/americas/brazil-coronavirus-vaccine.html}{Read
in English}

\href{https://www.nytimes3xbfgragh.onion/newsletters/el-times}{Regístrate
para recibir nuestro boletín} con lo mejor de The New York Times.

\begin{center}\rule{0.5\linewidth}{\linethickness}\end{center}

RÍO DE JANEIRO --- La respuesta caótica al coronavirus en Brasil, donde
el patógeno ha ocasionado el fallecimiento de más de
\href{https://covid.saude.gov.br/}{105.000 personas}, hizo que la
experiencia del país fuera un ejemplo admonitorio que muchos han
contemplado con alarma en todo el mundo.

Pero conforme el número de casos aumentaba, los investigadores de
vacunas vieron una oportunidad única.

Debido a un contagio generalizado y sostenido, una gran reserva de
especialistas en inmunización, una infraestructura robusta de
manufactura médica y miles de voluntarios para realizar pruebas
clínicas, Brasil ha surgido como un jugador potencialmente vital en la
carrera global para acabar con la pandemia.

Tres de las investigaciones de vacunas más prometedoras y avanzadas en
el mundo cuentan con científicos y voluntarios en Brasil, según el
informe de la Organización Mundial de la Salud sobre los avances en la
investigación para encontrar una vacuna.

El gobierno, que tantos conflictos enfrenta, espera que sus ciudadanos
estén entre los primeros en ser inoculados. Asimismo, los expertos
médicos están pensando que Brasil incluso podría fabricar la vacuna y
exportarla a los países vecinos ---una posibilidad que los llena de
orgullo, algo que este año ha escaseado para los brasileños---.

``Me siento muy optimista'', dijo Dimas Covas, el director del Instituto
Butantan, un productor farmacéutico con reconocimiento internacional que
se asoció con Sinovac de China en uno de los estudios que ha llegado a
la tercera etapa de la investigación, durante la cual se probarán
vacunas potenciales en 9000 personas.

``Brasil será uno de los primeros países en tener la vacuna'', afirmó
Covas.

\includegraphics{https://static01.graylady3jvrrxbe.onion/images/2020/08/15/world/17Brazil-vac-ES-01/merlin_175470378_00df1661-ed97-47ca-98e3-193817ba871c-articleLarge.jpg?quality=75\&auto=webp\&disable=upscale}

También se ha reclutado a unos 5000 brasileños para apoyar unas pruebas
de vacunas realizadas por AstraZeneca, una empresa farmacéutica
británica y sueca en una colaboración con la Universidad de Oxford.
Otros mil voluntarios fueron reclutados en Brasil para probar una vacuna
desarrollada por Pfizer, empresa con sede en Nueva York.

Para poder evaluar si una vacuna servirá, los investigadores necesitan
países con brotes considerablemente grandes. Algunos voluntarios reciben
la vacuna potencial y a otros se les da un placebo, pero deben estar en
un lugar donde haya suficiente virus circulando para que se pueda poner
a prueba la eficacia de la vacuna.

Brasil, donde el virus ha infectado a más de tres millones de personas,
cumple con las condiciones para estas pruebas. Además, será el único
otro país además de Estados Unidos que jugará un papel importante en
tres de los principales estudios pues la búsqueda ---sin precedentes---
de una vacuna logró que se logren las aprobaciones regulatorias y ha
forjado alianzas que fueron negociadas con gran premura.

Sin embargo, según los expertos no hay certeza de que las pruebas que
actualmente se realizan en Brasil son las que ganarán la carrera.

Muchos países en todo el mundo compiten por ser los primeros en tener
acceso a la vacuna que querrán miles de millones de personas. En India,
una de las familias más acaudaladas del país
\href{https://www.nytimes3xbfgragh.onion/2020/08/01/world/asia/coronavirus-vaccine-india.html}{está
apostando en grande} al producir en masa la vacuna de Oxford con la
esperanza de que será la primera en sortear los obstáculos regulatorios
y de seguridad.

La semana pasada, Rusia aprobó
\href{https://www.nytimes3xbfgragh.onion/es/2020/08/12/espanol/ciencia-y-tecnologia/vacuna-rusia-coronavirus.html}{una
vacuna de fabricación nacional}que aún no pasa las pruebas finales de
seguridad y eficacia. Si funciona, ese país podría posicionarse para
afirmar que produjo la primera vacuna efectiva contra el coronavirus.

La explosión de casos en Brasil ha hecho que este sea el país más
afectado después de Estados Unidos. Si bien otros países en la región
tienen tasas per cápita más altas, los expertos han arremetido contra el
presidente Jair
Bolsonaro\href{https://www.nytimes3xbfgragh.onion/es/2020/04/02/espanol/america-latina/bolsonaro-coronavirus-brasil.html?action=click\&module=RelatedLinks\&pgtype=Article}{por
manejar la crisis con ligereza}.

El mandatario,
que\href{https://www.nytimes3xbfgragh.onion/es/2020/07/07/espanol/america-latina/bolsonaro-coronavirus.html}{en
julio se contagió del virus}, calificó a la enfermedad como una
``gripecita'' y saboteó las peticiones de cuarentenas y cierres.
Asimismo, designó a un general del ejército sin ninguna experiencia
médica como encargado del ministerio de Salud después de que dos
ministros tuvieron desacuerdos con el presidente debido a su
displicencia por los enfoques basados en la ciencia.

Debido a la respuesta desorganizada de combate al virus, los brasileños
han tenido que soportar prohibiciones de viajes, sus vecinos han
militarizado los cruces fronterizos y los sindicatos que representan a
los trabajadores médicos hace poco le pidieron a la Corte Penal
Internacional que juzgara a Bolsonaro por crímenes de lesa humanidad,
pues sostienen que le ha dado rienda suelta al virus.

Image

~La vacuna SinovacCredit...Diego Vara/Reuters

El sistema sanitario público de Brasil tiene
\href{https://www.nytimes3xbfgragh.onion/es/2020/05/18/espanol/america-latina/covid-brasil.html}{uno
de los mejores programas de inmunización}de los países en vías de
desarrollo, por lo cual ha logrado contener brotes de fiebre amarilla,
sarampión y otras enfermedades.

Pero en los últimos años, a medida que la economía se ha contraído, el
programa se ha visto afectado por recortes presupuestales. También ha
tenido que luchar contra campañas de desinformación que han tenido mucha
repercusión en las redes sociales.

En 2019, por primera vez en 25 años, Brasil no cumplió su meta de
inoculación de ninguna de las vacunas que suele proporcionar.

Un logro en el coronavirus podría impulsar el sector de las vacunas.
También podría darle más fuerza a las instituciones científicas, que
contratan a científicos de clase mundial pero se han visto afectadas
tras años de
\href{https://www.nytimes3xbfgragh.onion/es/2020/04/09/espanol/coronavirus-paises-desarrollo.html}{recortes
presupuestales que debilitaron el sistema de salud público} y han dañado
la reputación del país como una potencia en investigaciones.

Katherine O'Brien, la directora de inmunización de la OMS, recibió con
gusto la noticia de la colaboración de Brasil en la producción de
vacunas para la COVID-19, la enfermedad causada por el virus. Sin
embargo, afirmó que los acuerdos bilaterales como aquellos en los que
está involucrado Brasil de todos modos eran una apuesta.

``Algunos países tendrán suerte, los que firmaron contratos con un
candidato que resulte eficaz'', sostuvo O'Brien. ``Otros países tendrán
acuerdos con candidatos que fracasarán y no tendrán nada''.

Image

El laboratorio de Bio-Manguinhos producirá la vacuna de Oxford en Río de
Janeiro.Credit...Antonio Lacerda/EPA, vía Shutterstock

Brasil, con una población de cerca de 210 millones de personas, tiene la
capacidad de producir unos 500 millones de vacunas al año. En el marco
de los actuales acuerdos sobre vacunas contra el coronavirus en los que
participa el país, inicialmente las plantas brasileñas se encargarían de
las etapas finales de la producción de vacunas después de importar las
materias primas, y luego las producirían en su totalidad.

Brasil ha firmado dos acuerdos para obtener acceso preferencial a una
vacuna. Uno, entre el Instituto Butantan del estado de Sao Paulo y
Sinovac, que proveería a los brasileños con 120 millones de dosis de la
vacuna para principios de 2021. El segundo, entre Bio-Manguinhos del
gobierno federal y AstraZeneca, que garantiza el acceso a 100 millones
de dosis de la vacuna para principios del próximo año.

Ambos contratos incluyen un acuerdo de transferencia de tecnología que
permitiría a Brasil fabricar posteriormente vacunas por su cuenta. Los
funcionarios del gobierno esperan comenzar a vacunar a algunos
brasileños en el primer semestre de 2021, aunque la fecha exacta
dependerá de los resultados de los estudios en curso y de un futuro
proceso de aprobación con el organismo regulador local.

Carla Domingues, la epidemióloga que dirigió el programa de inmunización
del país hasta el año pasado, dijo que las campañas de desinformación
sobre la inmunización han obstaculizado los esfuerzos para proteger a
las personas del VPH, una infección de transmisión sexual.

``Desafortunadamente, esta tendencia que venimos viendo en otros países
durante muchos años ahora está aquí en Brasil'', dijo. ``Y no hemos
logrado revertirla''.

Image

Vacunas contra la neumonía en el laboratorio de Bio-Manguinhos, donde se
producirá la vacuna de Oxford en BrasilCredit...Antonio Lacerda/EPA, vía
Shutterstock

Sin embargo, reclutar voluntarios para los estudios en curso en Brasil
no ha sido un desafío, dijo Soraya Smaili, presidenta de la Universidad
Federal de São Paulo, que participa en el estudio de AstraZeneca y
Oxford.

``No ha sido difícil encontrar voluntarios'', dijo. ``La gente dio un
paso al frente y todos quieren ser parte de la solución. Ha sido un
movimiento social muy bonito''.

Denise Abranches, una cirujana dental que ha pasado meses tratando a
pacientes con coronavirus con llagas en la boca en unidades de cuidados
intensivos, fue una de las primeras en ofrecerse como voluntaria para
recibir la vacuna. Dijo que su único miedo era no hacer fila lo
suficientemente pronto para recibir la inyección.

``Veo esto como una forma de recuperar un papel de liderazgo'' en la
comunidad científica internacional, dijo. ``El mundo nos mira en busca
de respuestas y esta es una vacuna que podría ayudar a todos en el
planeta''.

Maurício Zuma, director en Bio-Manguinhos, uno de los fabricantes que
espera producir vacunas contra la COVID-19 en Brasil, dijo que después
de que el país satisfaga su demanda interna, espera exportar viales a
países vecinos que también han tenido problemas con grandes cantidades
de casos.

``Nuestra intención es formar parte de un movimiento de solidaridad'',
dijo. ``Si logramos producir la vacuna aquí y terminamos con un
excedente, obviamente nos aseguraremos de que se use en otros países de
América Latina''.

Pero mientras los investigadores aplauden el papel de Brasil en la
carrera mundial por la vacuna, también se han sentido obligados a
recordar a los ciudadanos que estas buenas noticias no acabarán por sí
solas con el sufrimiento que el virus ha desatado en el país.

``No deberían suponer que ya se acabó y eso fue todo'', dijo Maria Elena
Bottazzi, una desarrolladora de vacunas en la Facultad de Medicina de
Baylor. ``Aún hay mucho trabajo que Brasil necesita hacer para
fortalecer su infraestructura de salud pública y reducir la transmisión
del virus''.

Ernesto Londoño es el jefe de la corresponsalía de Brasil, con sede en
Río de Janeiro. Antes fue escritor parte del Comité Editorial y, antes
de unirse a The New York Times, era reportero en The Washington Post.
\href{https://twitter.com/londonoe}{@londonoe}•\href{https://www.facebookcorewwwi.onion/londono}{Facebook}

Advertisement

\protect\hyperlink{after-bottom}{Continue reading the main story}

\hypertarget{site-index}{%
\subsection{Site Index}\label{site-index}}

\hypertarget{site-information-navigation}{%
\subsection{Site Information
Navigation}\label{site-information-navigation}}

\begin{itemize}
\tightlist
\item
  \href{https://help.nytimes3xbfgragh.onion/hc/en-us/articles/115014792127-Copyright-notice}{©~2020~The
  New York Times Company}
\end{itemize}

\begin{itemize}
\tightlist
\item
  \href{https://www.nytco.com/}{NYTCo}
\item
  \href{https://help.nytimes3xbfgragh.onion/hc/en-us/articles/115015385887-Contact-Us}{Contact
  Us}
\item
  \href{https://www.nytco.com/careers/}{Work with us}
\item
  \href{https://nytmediakit.com/}{Advertise}
\item
  \href{http://www.tbrandstudio.com/}{T Brand Studio}
\item
  \href{https://www.nytimes3xbfgragh.onion/privacy/cookie-policy\#how-do-i-manage-trackers}{Your
  Ad Choices}
\item
  \href{https://www.nytimes3xbfgragh.onion/privacy}{Privacy}
\item
  \href{https://help.nytimes3xbfgragh.onion/hc/en-us/articles/115014893428-Terms-of-service}{Terms
  of Service}
\item
  \href{https://help.nytimes3xbfgragh.onion/hc/en-us/articles/115014893968-Terms-of-sale}{Terms
  of Sale}
\item
  \href{https://spiderbites.nytimes3xbfgragh.onion}{Site Map}
\item
  \href{https://help.nytimes3xbfgragh.onion/hc/en-us}{Help}
\item
  \href{https://www.nytimes3xbfgragh.onion/subscription?campaignId=37WXW}{Subscriptions}
\end{itemize}
