Sections

SEARCH

\protect\hyperlink{site-content}{Skip to
content}\protect\hyperlink{site-index}{Skip to site index}

\href{https://www.nytimes3xbfgragh.onion/es/section/opinion}{Opinión}

\href{https://myaccount.nytimes3xbfgragh.onion/auth/login?response_type=cookie\&client_id=vi}{}

\href{https://www.nytimes3xbfgragh.onion/section/todayspaper}{Today's
Paper}

\href{/es/section/opinion}{Opinión}\textbar{}Colombia y la justicia
portátil

\url{https://nyti.ms/31k9RHJ}

\begin{itemize}
\item
\item
\item
\item
\item
\end{itemize}

Advertisement

\protect\hyperlink{after-top}{Continue reading the main story}

\href{/es/section/opinion}{Opinión}

Supported by

\protect\hyperlink{after-sponsor}{Continue reading the main story}

Comentario

\hypertarget{colombia-y-la-justicia-portuxe1til}{%
\section{Colombia y la justicia
portátil}\label{colombia-y-la-justicia-portuxe1til}}

La noticia de la detención del expresidente Álvaro Uribe provocó una
reacción peligrosa que puede erosionar la institucionalidad de nuestras
democracias: antes de dejar que la justicia avance con el caso, algunos
líderes internacionales se solidarizaron automáticamente con el político
colombiano.

\includegraphics{https://static01.graylady3jvrrxbe.onion/images/2020/08/23/multimedia/23Barrera-ES-1/merlin_176001447_c8c8f3a4-33c9-4cb9-82d2-1e588afd45d6-articleLarge.jpg?quality=75\&auto=webp\&disable=upscale}

Por Alberto Barrera Tyszka

Es colaborador regular de The New York Times.

\begin{itemize}
\item
  23 de agosto de 2020
\item
  \begin{itemize}
  \item
  \item
  \item
  \item
  \item
  \end{itemize}
\end{itemize}

\href{https://www.nytimes3xbfgragh.onion/newsletters/el-times}{Regístrate
para recibir nuestro boletín} con lo mejor de The New York Times.

\begin{center}\rule{0.5\linewidth}{\linethickness}\end{center}

CIUDAD DE MÉXICO --- Álvaro Uribe Vélez es un personaje polémico.
Siempre ha estado en medio de grandes controversias y su manera de
pronunciarse y de debatir puede ser muy irritante. No en balde, en más
de una ocasión, se le ha comparado con Hugo Chávez: tan distintos en el
terreno ideológico y tan parecidos en sus formas de ejercer el
liderazgo. Producían experiencias carismáticas parecidas. Tanto el
uribismo como el chavismo son movimientos devocionales. Su definición es
la ceguera gozosa, la militancia política trabucada en fervor religioso.

La
\href{https://www.nytimes3xbfgragh.onion/es/2020/08/04/espanol/america-latina/alvaro-uribe-detencion-colombia.html}{noticia
del arresto domiciliario} al expresidente colombiano, ordenado por la
Corte Suprema de Justicia de su país, desnuda nuevamente los peligros de
la polarización: la solidaridad automática que cuestiona y descalifica a
la institucionalidad. Otro paso más en el paradójico proceso que están
viviendo nuestras sociedades: los políticos asesinan a la política.

La polarización es una dinámica suicida para los políticos. Se exhiben
en permanente ejercicio de destrucción, denunciándose unos a otros,
alertando sobre planes malévolos y constantes confabulaciones. Lo único
que le queda a los ciudadanos y a las sociedades son las instituciones.
Es necesario un acuerdo mínimo, un pacto de respeto para defenderlas. El
sistema de justicia no puede ser un poder portátil que se evalúa o se
define a conveniencia, según las circunstancias y según el acusado.

El 4 de agosto, el propio Álvaro Uribe
\href{https://twitter.com/AlvaroUribeVel/status/1290712262504779784}{anunció}
a través de su cuenta de Twitter que por una orden judicial había sido
privado de libertad. No es casual que se haya adelantado a las
autoridades y haya decidido mediatizar de inmediato el hecho. De esta
manera, desde su estreno, trasladó el caso del sistema de justicia a la
dimensión del espectáculo, en clara clave de melodrama: ``La privación
de mi libertad me causa profunda tristeza por mi señora, por mi familia
y por los colombianos que todavía creen que algo bueno he hecho por la
patria''. Solo le faltó la música de violines.

El arresto domiciliario de Uribe responde a una investigación por
``\href{https://www.bbc.com/mundo/noticias-america-latina-53604677}{posibles
riesgos de obstrucción a la justicia}'', con supuestos sobornos y
manipulación de testigos en un caso de 2014. Era previsible que esta
detención ---inédita en la historia colombiana--- sacudiera internamente
la ya agitada y dividida situación del país. Pero no deja de llamar la
atención cómo, de manera casi inmediata, en los días siguientes, también
comenzaron a aparecer reacciones internacionales casi instintivas,
maquinales, que ---mostrando su adhesión a Uribe--- despojaban de
cualquier legitimidad a la institucionalidad de Colombia.

En una
\href{https://www.eltiempo.com/politica/congreso/alvaro-uribe-expresidentes-de-espana-y-america-latina-se-solidarizan-con-el-senador-527784}{carta
pública} , 21 expresidentes de Latinoamérica y España cuestionaron la
decisión judicial y alertaron sobre la ``ideologización'' y la
``manipulación'' de las garantías en Colombia. Tal fue el desatino que
la organización Human Rights Watch debió responder ---también
públicamente--- con
\href{https://www.eltiempo.com/mundo/eeuu-y-canada/la-carta-de-hrw-a-expresidentes-que-criticaron-la-detencion-de-uribe-531646}{otra
misiva} donde prácticamente regañaba a los exmandatarios.

También el gobierno de Estados Unidos, a través de su vicepresidente,
Mike Pence,
\href{https://elpais.com/internacional/2020-08-14/el-vicepresidente-de-ee-uu-mike-pence-pide-el-fin-del-arresto-domiciliario-de-alvaro-uribe.html}{reaccionó
rápidamente} y pidió el fin del arresto y ---humildemente--- condecoró a
Uribe con la palabra ``héroe''.

Pero nada, sin duda, como María Corina Machado, líder de la derecha
venezolana, quien
\href{https://www.alertatolima.com/noticias/politica/maria-corina-machado-advirtio-que-detencion-de-uribe-hace-parte-de-una-operacion}{envió
un mensaje de alerta} al pueblo colombiano, definiendo el hecho como una
``operación'' del ``conglomerado criminal'' que se instaló en el país,
como parte de una conspiración internacional destinada a convertir a
Colombia en una nueva Venezuela.

Todas estas reacciones son un espejo perfecto donde puede verse la
mediocridad absoluta que produce la polarización. El liderazgo que
---apelando al modelo y a los valores liberales--- ha defendido siempre
la independencia de poderes, repite ahora las misma acciones y las
mismas declaraciones que tanto le ha criticado a sus adversarios. De
pronto desacreditan al poder público y desautorizan a la
institucionalidad colombiana por una sola razón, por un nombre. Por
haberse metido con un aliado.

La relación de Álvaro Uribe Vélez con la justicia es larga y tiene un
extenso expediente. Hay investigaciones, incluso
\href{https://www.nytimes3xbfgragh.onion/es/2018/05/25/espanol/cables-uribe-narcotrafico-colombia.html}{documentos
desclasificados} en Estados Unidos, que lo vinculan al narcotráfico en
los años noventa. En su propio país, el expresidente acumula
\href{https://www.semana.com/nacion/articulo/alvaro-uribe-polemicas-e-investigaciones/691756}{29
procesos en la Corte Suprema y más de 200} en otras instancias
judiciales. Y Colombia no es Venezuela, no es una dictadura donde un
partido controla y maneja a su antojo el sistema de justicia. Visto
desde esta perspectiva, parece muy temerario e irresponsable establecer
respaldos instantáneos que, además, sin aportar ninguna prueba,
erosionan la credibilidad de las instituciones de un país extranjero.

En Latinoamérica, donde seguimos batallando tanto por la autonomía de
los poderes y en contra de la impunidad, la defensa de las instituciones
debería ser un acuerdo indiscutible para la supervivencia de la
democracia. Es muy saludable que en cualquier sociedad se pueda
investigar y juzgar ---con transparencia y apego a las leyes--- a
cualquier exfuncionario público. Aunque se llame Álvaro Uribe Vélez.

Alberto Barrera Tyszka es escritor. Su libro más reciente es la novela
\emph{Mujeres que matan}.

Advertisement

\protect\hyperlink{after-bottom}{Continue reading the main story}

\hypertarget{site-index}{%
\subsection{Site Index}\label{site-index}}

\hypertarget{site-information-navigation}{%
\subsection{Site Information
Navigation}\label{site-information-navigation}}

\begin{itemize}
\tightlist
\item
  \href{https://help.nytimes3xbfgragh.onion/hc/en-us/articles/115014792127-Copyright-notice}{©~2020~The
  New York Times Company}
\end{itemize}

\begin{itemize}
\tightlist
\item
  \href{https://www.nytco.com/}{NYTCo}
\item
  \href{https://help.nytimes3xbfgragh.onion/hc/en-us/articles/115015385887-Contact-Us}{Contact
  Us}
\item
  \href{https://www.nytco.com/careers/}{Work with us}
\item
  \href{https://nytmediakit.com/}{Advertise}
\item
  \href{http://www.tbrandstudio.com/}{T Brand Studio}
\item
  \href{https://www.nytimes3xbfgragh.onion/privacy/cookie-policy\#how-do-i-manage-trackers}{Your
  Ad Choices}
\item
  \href{https://www.nytimes3xbfgragh.onion/privacy}{Privacy}
\item
  \href{https://help.nytimes3xbfgragh.onion/hc/en-us/articles/115014893428-Terms-of-service}{Terms
  of Service}
\item
  \href{https://help.nytimes3xbfgragh.onion/hc/en-us/articles/115014893968-Terms-of-sale}{Terms
  of Sale}
\item
  \href{https://spiderbites.nytimes3xbfgragh.onion}{Site Map}
\item
  \href{https://help.nytimes3xbfgragh.onion/hc/en-us}{Help}
\item
  \href{https://www.nytimes3xbfgragh.onion/subscription?campaignId=37WXW}{Subscriptions}
\end{itemize}
