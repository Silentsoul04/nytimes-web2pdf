Sections

SEARCH

\protect\hyperlink{site-content}{Skip to
content}\protect\hyperlink{site-index}{Skip to site index}

\href{https://www.nytimes3xbfgragh.onion/es/}{en Español}

\href{https://myaccount.nytimes3xbfgragh.onion/auth/login?response_type=cookie\&client_id=vi}{}

\href{https://www.nytimes3xbfgragh.onion/section/todayspaper}{Today's
Paper}

\href{/es/}{en Español}\textbar{}No se puede escapar de los piojos, ni
siquiera a casi 2000 metros bajo la superficie del océano

\url{https://nyti.ms/31FgxjO}

\begin{itemize}
\item
\item
\item
\item
\item
\end{itemize}

Advertisement

\protect\hyperlink{after-top}{Continue reading the main story}

Supported by

\protect\hyperlink{after-sponsor}{Continue reading the main story}

Mundo animal

\hypertarget{no-se-puede-escapar-de-los-piojos-ni-siquiera-a-casi-2000-metros-bajo-la-superficie-del-ocuxe9ano}{%
\section{No se puede escapar de los piojos, ni siquiera a casi 2000
metros bajo la superficie del
océano}\label{no-se-puede-escapar-de-los-piojos-ni-siquiera-a-casi-2000-metros-bajo-la-superficie-del-ocuxe9ano}}

Una especie de insectos vive en el cuerpo de los elefantes marinos
mientras pasan meses en el mar, soportando los cambios de presión de las
profundas y largas inmersiones de esos mamíferos.

\includegraphics{https://static01.graylady3jvrrxbe.onion/images/2020/09/01/science/26tb-lice-ES-1/merlin_176170911_4d2970b6-edc4-45ea-a810-14b32c4a56d3-articleLarge.jpg?quality=75\&auto=webp\&disable=upscale}

Por Priyanka Runwal

\begin{itemize}
\item
  29 de agosto de 2020
\item
  \begin{itemize}
  \item
  \item
  \item
  \item
  \item
  \end{itemize}
\end{itemize}

\href{https://www.nytimes3xbfgragh.onion/2020/08/26/science/lice-elephant-seals.html}{Read
in English}

\href{https://www.nytimes3xbfgragh.onion/newsletters/el-times}{Regístrate
para recibir nuestro boletín} con lo mejor de The New York Times.

\begin{center}\rule{0.5\linewidth}{\linethickness}\end{center}

Bajo el mar, la vida es más sabrosa, a menos que seas un insecto. Quizás
has visto algunos bichos flotando en la superficie de un estanque o
incluso
\href{https://www.nytimes3xbfgragh.onion/2017/11/21/science/diving-flies-mono-lake.html}{creando
su propia burbuja de buceo} para sumergirse en los lagos interiores.
Pero, en mar abierto, prácticamente no hay insectos.

Sin embargo, si miras de cerca las aletas posteriores de los elefantes
marinos del sur, encontrarás unos insectos que han logrado tener una
vida parcialmente acuática. Los piojos de la especie Lepidophthirus
macrorhini habitan en las extremidades traseras de estos mamíferos
acuáticos de gran tamaño, los cuales pasan casi 10 meses del año en
aguas antárticas y se sumergen hasta 1981 metros por debajo de la
superficie oceánica en busca de comida e incluso pueden quedarse ahí
casi dos horas seguidas.

Estos piojos tal vez sean los insectos que pueden sobrevivir a mayores
profundidades en los ecosistemas marinos, según un
\href{https://jeb.biologists.org/content/early/2020/07/16/jeb.226811}{estudio}
publicado en julio en la revista Journal of Experimental Biology. Al
tolerar ambientes tan extremos, los piojos de los elefantes marinos
pueden ayudar a los científicos a resolver el misterio de por qué hay
tan pocos insectos en la vastedad del océano.

Los especímenes de L. macrorhini son piojos parasitarios que chupan
sangre y anidan en la capa superior de la piel del elefante marino para
alimentarse. En 2015, María Soledad Leonardi, bióloga marina del
Instituto de Biología de Organismos Marinos en Argentina, encontró
piojos vivos en los elefantes marinos machos que salían a la superficie
para aparearse en la isla Rey Jorge, cerca de la costa de la Antártida.

``Se pueden ver a simple vista'', afirmó. ``Parecen cangrejos
miniatura''.

En su opinión, la presencia de piojos en los elefantes marinos adultos
que salen a la superficie tras extensas excursiones submarinas sugiere
que los insectos pueden sobrevivir a las inmersiones profundas y las
subidas pronunciadas en los trayectos acuáticos de los elefantes
marinos. Eso querría decir que los piojos tal vez son capaces de
soportar la presión aplastante de las profundidades del océano.

\includegraphics{https://static01.graylady3jvrrxbe.onion/images/2020/08/26/science/26tb-lice-ES-2/26TB-LICE2-articleLarge.jpg?quality=75\&auto=webp\&disable=upscale}

Soledad Leonardi comentó que atrapar elefantes marinos de 3628
kilogramos en alta mar para ver si los piojos sobreviven a estas
condiciones extremas sería complicado. Por lo tanto, su equipo decidió
llevar los piojos al laboratorio.

Con ayuda de pinzas pequeñas, extrajeron los insectos de las aletas
traseras de quince crías de elefante marino que nacieron en las playas
de la península Valdés en Argentina. Los cachorros
\href{https://www.publish.csiro.au/zo/ZO9650437}{tienen piojos adultos}
transferidos de los cuerpos de sus madres a los pocos días de su
nacimiento. Los piojos se reproducen rápidamente, aprovechando las
primeras semanas en las que las crías deben quedarse en tierra, puesto
que \href{https://www.publish.csiro.au/zo/ZO9650437}{sus huevos no
pueden eclosionar bajo el agua}.

En el laboratorio, el equipo sumergió a los piojos en cámaras
individuales del tamaño de una memoria USB llenas de agua de mar y
conectadas a un tanque de buceo. Luego, expusieron a cada piojo a una
gama de presiones de agua, hasta 200 veces mayor a la de la superficie
marina y equivalente a profundidades de entre 298 y 1981 metros. Tras
experimentar 10 minutos en este ambiente en aguas profundas, 69 de los
75 piojos salieron vivos.

``Para mí fue fascinante ver que sobrevivieron a la alta presión'', dijo
Claudio Lazzari, fisiólogo de insectos en la Universidad de Tours en
Francia y uno de los autores del estudio. ``Esto demuestra que estos
piojos son resistentes. Podemos excluir la posibilidad de que
simplemente mueren''.

Después, los investigadores expusieron a los piojos que habían
sobrevivido a una presión de agua más alta o más baja que la que habían
experimentado antes.

``La idea era reproducir la situación a la que estarían sujetos los
piojos cuando sus portadores se sumergen a distintos niveles de
presión'', comentó Lazzari. Todos los piojos lograron tolerar el cambio
rápido de presión; los adultos se recuperaron más pronto y presentaron
movilidad tras el experimento, en comparación con las ninfas.

Stuart Humphries, biofísico evolutivo de la Universidad de Lincoln en
Inglaterra, se refirió al estudio como ``satisfactorio''. Y agregó: ``Me
interesaría saber cómo hacen esto los piojos''.

Hasta el momento, los investigadores no saben si los piojos marinos
tienen adaptaciones especiales.

``Me imagino que estos pequeños solo suprimen y bloquean su sistema
traqueal'', especuló Humphries, lo cual significa que los piojos podrían
contener la respiración en aguas profundas.

Los investigadores quieren realizar experimentos para ver si estos
insectos frenan sus actividades y su gasto energético en lo profundo del
océano o si siguen respirando.

``Comprender cómo este grupo de insectos se las ingenia para sobrevivir
bajo el agua será la clave para entender por qué otros grupos no lo
logran'', explicó Lazzari.

No obstante, algunos científicos piensan que los piojos podrían ser un
caso único.

``Estos piojos son un caso especializado; solo viven adheridos a su
portador en los ambientes marinos y se reproducen cuando los elefantes
marinos están en tierra'', mencionó Lanna Cheng, bióloga marina emérita
de la Institución Scripps de Oceanografía en San Diego. ``No tenemos
idea de si tienen o no la habilidad de sobrevivir como insectos de vida
libre a esas profundidades''.

Advertisement

\protect\hyperlink{after-bottom}{Continue reading the main story}

\hypertarget{site-index}{%
\subsection{Site Index}\label{site-index}}

\hypertarget{site-information-navigation}{%
\subsection{Site Information
Navigation}\label{site-information-navigation}}

\begin{itemize}
\tightlist
\item
  \href{https://help.nytimes3xbfgragh.onion/hc/en-us/articles/115014792127-Copyright-notice}{©~2020~The
  New York Times Company}
\end{itemize}

\begin{itemize}
\tightlist
\item
  \href{https://www.nytco.com/}{NYTCo}
\item
  \href{https://help.nytimes3xbfgragh.onion/hc/en-us/articles/115015385887-Contact-Us}{Contact
  Us}
\item
  \href{https://www.nytco.com/careers/}{Work with us}
\item
  \href{https://nytmediakit.com/}{Advertise}
\item
  \href{http://www.tbrandstudio.com/}{T Brand Studio}
\item
  \href{https://www.nytimes3xbfgragh.onion/privacy/cookie-policy\#how-do-i-manage-trackers}{Your
  Ad Choices}
\item
  \href{https://www.nytimes3xbfgragh.onion/privacy}{Privacy}
\item
  \href{https://help.nytimes3xbfgragh.onion/hc/en-us/articles/115014893428-Terms-of-service}{Terms
  of Service}
\item
  \href{https://help.nytimes3xbfgragh.onion/hc/en-us/articles/115014893968-Terms-of-sale}{Terms
  of Sale}
\item
  \href{https://spiderbites.nytimes3xbfgragh.onion}{Site Map}
\item
  \href{https://help.nytimes3xbfgragh.onion/hc/en-us}{Help}
\item
  \href{https://www.nytimes3xbfgragh.onion/subscription?campaignId=37WXW}{Subscriptions}
\end{itemize}
