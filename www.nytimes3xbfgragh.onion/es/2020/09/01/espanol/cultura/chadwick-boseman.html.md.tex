Sections

SEARCH

\protect\hyperlink{site-content}{Skip to
content}\protect\hyperlink{site-index}{Skip to site index}

\href{https://www.nytimes3xbfgragh.onion/es/section/cultura}{Cultura}

\href{https://myaccount.nytimes3xbfgragh.onion/auth/login?response_type=cookie\&client_id=vi}{}

\href{https://www.nytimes3xbfgragh.onion/section/todayspaper}{Today's
Paper}

\href{/es/section/cultura}{Cultura}\textbar{}Chadwick Boseman: el
difícil arte de hacer que la dignidad sea interesante

\url{https://nyti.ms/3beH4b9}

\begin{itemize}
\item
\item
\item
\item
\item
\item
\end{itemize}

Advertisement

\protect\hyperlink{after-top}{Continue reading the main story}

Supported by

\protect\hyperlink{after-sponsor}{Continue reading the main story}

\hypertarget{chadwick-boseman-el-difuxedcil-arte-de-hacer-que-la-dignidad-sea-interesante}{%
\section{Chadwick Boseman: el difícil arte de hacer que la dignidad sea
interesante}\label{chadwick-boseman-el-difuxedcil-arte-de-hacer-que-la-dignidad-sea-interesante}}

El actor, que murió a los 43 años, llevó al límite los parámetros del
cine biográfico: más que un imitador o alguien ``parecido a'', era un
historiador del magnetismo y la voluntad de otras personas.

\includegraphics{https://static01.graylady3jvrrxbe.onion/images/2014/07/30/multimedia/31BOSEMAN-ES-00/getonup-anatomy-articleLarge-v3.jpg?quality=75\&auto=webp\&disable=upscale}

\href{https://www.nytimes3xbfgragh.onion/by/wesley-morris}{\includegraphics{https://static01.graylady3jvrrxbe.onion/images/2018/06/13/multimedia/author-wesley-morris/author-wesley-morris-thumbLarge.jpg}}

Por \href{https://www.nytimes3xbfgragh.onion/by/wesley-morris}{Wesley
Morris}

\begin{itemize}
\item
  1 de septiembre de 2020
\item
  \begin{itemize}
  \item
  \item
  \item
  \item
  \item
  \item
  \end{itemize}
\end{itemize}

\href{https://www.nytimes3xbfgragh.onion/2020/08/29/movies/chadwick-boseman-appraisal.html}{Read
in English}

\href{https://www.nytimes3xbfgragh.onion/newsletters/el-times}{Regístrate
para recibir nuestro boletín} con lo mejor de The New York Times.

\begin{center}\rule{0.5\linewidth}{\linethickness}\end{center}

El problema de la dignidad es que un actor no puede hacer mucho con
ella. No cuando interpreta a Jackie Robinson, a Thurgood Marshall, o
cuando encarna al líder de un reino africano ficticio como Wakanda.

Para un actor, la dignidad puede parecer un ancla o un vacío. ¿Qué puede
mostrarnos de una leyenda del béisbol o de un titán de la jurisprudencia
que no haya sido revelado previamente?

Al interpretar la dignidad, Chadwick Boseman,
\href{https://www.nytimes3xbfgragh.onion/2020/08/28/movies/chadwick-boseman-dead.html}{quien
murió el viernes a los 43 años de cáncer de colon}, a menudo parecía
tener la tarea de representar el peso que significa. Pero en esos
papeles, él siempre transmitía más que solo fuerza. Más bien se centró
en proyectar lo opuesto: ligereza. En \emph{Marshall}, en vez de
insistir en la genialidad y la sabiduría del hombre, Boseman convirtió
el concepto de lo que era procesable de manera jurídica en acciones
físicas. Era ligero, rápido, suave y elegante. Solía rociar la verdad
con hierbas y especias.

Sorprendentemente, entre sus actuaciones como Robinson y Marshall,
Boseman también interpretó a la gran superestrella estadounidense James
Brown en \emph{Get On Up}. ¿Algún otro actor ha pasado tanto tiempo
poniéndose los zapatos de personajes tan enormes en un periodo tan
breve?
(\href{https://www.nytimes3xbfgragh.onion/2013/04/12/movies/42-with-chadwick-boseman-as-jackie-robinson.html}{La
película de Jackie Robinson}, \emph{42}, salió en 2013; \emph{Marshall}
se lanzó cuatro años después). En el negocio del cine no me viene nadie
a la mente. Quizás Sidney Poitier. Pero fue el primero y le tocó hacerse
sus propios zapatos.

Confieso que me parece extraño que Boseman haya interpretado estos tres
papeles tan rápidamente. Al principio, parecía una broma acerca de la
obsesión constante de las películas sobre las historias de
estadounidenses negros excepcionales o que en Hollywood eran demasiado
vagos como para imaginar a alguien más encarnando estos papeles
excepcionales. La verdad es que Boseman arrinconó un mercado con su
elasticidad interna y, al menos para mí, explotó los parámetros de lo
que debería ser la realización de películas biográficas. En su caso, el
hecho de ``parecerse a'' importaba más que ``luce igual a''. Era
atrevido y ni siquiera parecía consciente de los riesgos.

¿Qué puede mostrarnos un actor cuando ni siquiera se parece a los
personajes que interpreta? Siempre me pareció peculiar que no se
pareciera a ninguno de los tres hombres. Pero Chadwick Boseman tenía
aquellos ojos. No eran los de Robinson, ni los de Marshall o los de
Brown. En cada caso, los ojos de Boseman eran demasiado grandes (y su
figura, ahora que estamos en materia, era demasiado pequeña). Pero,
vaya, su sinceridad y ternura te llegaban a lo más profundo. Eso es lo
que sus ojos podían lograr con personalidades enteras: llegar a su punto
e ir más allá.

Durante esa racha de ``grandes hombres'', la idea de Boseman sobre las
leyendas que encarnaba se impuso sobre la verosimilitud. Las películas
en sí mismas no eran lo suficientemente audaces como para dejarlo
sumergirse de manera profunda o tocar aspectos muy oscuros: \emph{42} se
centra más en cómo el propietario de los Brooklyn Dodgers, Branch Rickey
(Harrison Ford), manejó el equipo del que formaba parte Robinson. No
obstante, Boseman hizo que cada hombre fuera sexy, contemplativo y
certero.

``Parecerse a'' lo llevó a lugares fascinantes en \emph{Get On Up},
\href{https://grantland.com/features/guardians-of-the-galaxy-get-on-up-review/}{la
película de James Brown} de 2014. Obtuvo la cinética de Brown y sus
ademanes de percusionista al conversar, así como su encanto y la
mercurial mecha corta de su personalidad. Una audiencia podría haber
tenido problemas para armonizar las contradicciones de Brown: los
impulsos libertinos y conservadores, su tiranía, paranoia y generosidad,
era un hombre que amaba a las mujeres y las golpeaba.

Boseman convirtió la fricción de la personalidad de Brown en fuego. La
rebeldía de la película, su dura historia de vida, su divergencia con la
realidad, son aspectos que probablemente habrían abrumado a un actor
normal. Pero Boseman estaba lejos de ser un actor normal.

La película pasó desapercibida ese verano. Lo que todos se perdieron no
fue solo una de las mejores actuaciones del año, sino un hito para un
género desgastado. A diferencia de
\href{https://www.nytimes3xbfgragh.onion/2005/11/18/movies/the-man-in-black-on-stage-and-off.html}{Joaquin
Phoenix} (que interpretó a Johnny Cash) y, finalmente, Rami Malek
(Freddie Mercury) y Renée Zellweger (Judy Garland), Boseman no intentó
cantar. Lo que escuchas es la voz de James Brown. Pero Boseman evita
cualquier truco de edición. La cámara se acerca mientras, digamos, él
permanece inmóvil ---inmóvil al estilo de Brown--- y graba \emph{Try
Me}, \href{https://www.youtube.com/watch?v=_OIcuozqEjo}{a capella}.
Boseman era tan fluido al hacer la curvatura de la lengua de Brown y en
la apertura de su boca mientras esculpía y escupía ``I need you'' y ``I
want you to stop my heart from crying'' y ``heh!'' que la voz del
cantante bien podría haber sido la del actor.

El impacto de la sincronización labial de Boseman difiere de la de
Marion Cotillard en \emph{La Vie en Rose} o la de Jamie Foxx en
\emph{Ray}, porque Boseman realmente se ve muy mal para el papel, la
ropa, por ejemplo, que apretaba al Brown en sus últimos años de carrera,
cuelga del atlético cuerpo de Boseman. La simulación oral forjó su
camino a la credibilidad; no fueron el cabello o el maquillaje. Lo que a
su ``Padrino del Soul'' le faltaba en semejanza, lo compensó con
entusiasmo espiritual.

La carrera de Boseman no despegó hasta que cumplió los 30. Así que un
gran ``y qué habría pasado si'' se cierne sobre su carrera, la mayor
parte de la cual la pasó, por supuesto, en el universo Marvel, donde
prosperó como T'Challa, rey de Wakanda, el país que defiende como
Pantera Negra. Cuando T'Challa interviene en la primera secuela de
\emph{Capitán América}, hay un ardor en Boseman que lo convierte en la
persona más convincente de la película cuando aparece en escena, que no
es mucho, pero es más de lo que esperaba. Pero Marvel siempre tiene un
plan, y el plan para Boseman era una película independiente de
\emph{Pantera Negra}. Era su cóctel característico entre pensativo y
tranquilo. La corona no le pesaba. Así interpretó el papel que lo
convertiría en la estrella de cine.

\includegraphics{https://static01.graylady3jvrrxbe.onion/images/2019/02/07/arts/31BOSEMAN-ES-01/merlin_133340099_9a718148-7246-4bce-bb7c-878a2b4dda8d-articleLarge.jpg?quality=75\&auto=webp\&disable=upscale}

Un aspecto maravilloso de la fama de Boseman era lo poco que parecía
importarle estar ligado a esa franquicia. Lo que sea que signifique
\emph{Pantera Negra} para millones de personas también significó algo
para él. Caminaba por las alfombras rojas con abrigos de diseñador hasta
el suelo, trajes bordados, capas de caballero y muchos patrones
brillantes, al punto que la ropa se convirtió en su propio sello.
Aparente y sorprendentemente, lo hizo mientras también luchaba contra el
cáncer. En público, cruzaba los brazos sobre el pecho como lo hacen en
Wakanda, un saludo que es una promesa perdurable. En 2018, presentó
\emph{Saturday Night Live} y, como T'Challa, compitió graciosamente
contra Shanice y Rashad en uno de los segmentos del programa llamado
``Black Jeopardy!''. Sus categorías incluían \emph{Hecho y derecho};
\emph{¿Estás loca?/ Chau, chica}; y \emph{Gente blanca}.

\href{https://www.youtube.com/watch?v=hzMzFGgmQOc}{En un momento},
Shanice escoge la primera categoría por 600 dólares y recibe la pista:
``Envías a tu niña sabelotodo aquí porque cree que ha crecido''.
T'Challa interviene, hablando con el cadencioso pragmatismo wakandiano
de Boseman: ``Qué es `a una de nuestras universidades gratuitas donde
puede aplicar su inteligencia y quizás algún día convertirse en una gran
científica'''. Su dignidad es más de lo que el juego requiere. Es
pedirle al programa que quiera más para sí. La comedia surge de la
tensión entre las bajas y las altas expectativas, entre la exasperación
de Kenan Thompson, como presentador, y la alegre rectitud de Boseman,
entre la gente común y la realeza.

El misterio más emocionante era pensar hacia dónde se enfilaría la
sofisticación de Boseman, además de Wakanda. Había terminado una
\href{https://deadline.com/2020/08/netflix-delays-ma-raineys-black-bottom-virtual-preview-event-following-chadwick-boseman-death-1203026802/}{versión
cinematográfica} de la obra de August Wilson \emph{Ma Rainey's Black
Bottom,} para George C. Wolfe, con Viola Davis. Y aunque quizás habría
dudado sobre si volver a encarnar a otro estadounidense extraordinario,
era bueno en eso. ¿Por qué detenerse en Thurgood Marshall? La solemnidad
y los ojos redondos, serios y escrutadores de Boseman coincidían mejor
con los de James Baldwin. Esa mezcla podría haber sido algo interesante:
la mediana edad de Baldwin coincidiría con la de Boseman, el método
digno, y diestro, del actor al abordar las demandas infinitas del
pensador que luchó para que este país respetara la dignidad de los
afroestadounidenses.

Su escaso parecido con Baldwin resulta secundario ante lo que Boseman
podría haber hecho con la erudición y elocución de Baldwin. Porque
Boseman no era un imitador. A su manera, era un historiador del
magnetismo y la voluntad de otras personas. La excelencia y el liderazgo
le hablaban y lo encendían. Por fuerza. Nadie se acerca a tanta
grandeza, sin tener una considerable reserva de grandeza en su interior.

Wesley Morris es crítico. Fue galardonado con el premio Pulitzer en 2012
por su crítica mientras estuvo en el Boston Globe. También ha trabajado
en Grantland, The San Francisco Chronicle y The San Francisco Examiner.
\href{https://twitter.com/wesley_morris}{@wesley\_morris}

\begin{center}\rule{0.5\linewidth}{\linethickness}\end{center}

Advertisement

\protect\hyperlink{after-bottom}{Continue reading the main story}

\hypertarget{site-index}{%
\subsection{Site Index}\label{site-index}}

\hypertarget{site-information-navigation}{%
\subsection{Site Information
Navigation}\label{site-information-navigation}}

\begin{itemize}
\tightlist
\item
  \href{https://help.nytimes3xbfgragh.onion/hc/en-us/articles/115014792127-Copyright-notice}{©~2020~The
  New York Times Company}
\end{itemize}

\begin{itemize}
\tightlist
\item
  \href{https://www.nytco.com/}{NYTCo}
\item
  \href{https://help.nytimes3xbfgragh.onion/hc/en-us/articles/115015385887-Contact-Us}{Contact
  Us}
\item
  \href{https://www.nytco.com/careers/}{Work with us}
\item
  \href{https://nytmediakit.com/}{Advertise}
\item
  \href{http://www.tbrandstudio.com/}{T Brand Studio}
\item
  \href{https://www.nytimes3xbfgragh.onion/privacy/cookie-policy\#how-do-i-manage-trackers}{Your
  Ad Choices}
\item
  \href{https://www.nytimes3xbfgragh.onion/privacy}{Privacy}
\item
  \href{https://help.nytimes3xbfgragh.onion/hc/en-us/articles/115014893428-Terms-of-service}{Terms
  of Service}
\item
  \href{https://help.nytimes3xbfgragh.onion/hc/en-us/articles/115014893968-Terms-of-sale}{Terms
  of Sale}
\item
  \href{https://spiderbites.nytimes3xbfgragh.onion}{Site Map}
\item
  \href{https://help.nytimes3xbfgragh.onion/hc/en-us}{Help}
\item
  \href{https://www.nytimes3xbfgragh.onion/subscription?campaignId=37WXW}{Subscriptions}
\end{itemize}
