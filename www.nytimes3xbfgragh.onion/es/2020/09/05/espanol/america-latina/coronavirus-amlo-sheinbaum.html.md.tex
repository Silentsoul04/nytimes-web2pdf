Sections

SEARCH

\protect\hyperlink{site-content}{Skip to
content}\protect\hyperlink{site-index}{Skip to site index}

\href{https://www.nytimes3xbfgragh.onion/es/section/america-latina}{América
Latina}

\href{https://myaccount.nytimes3xbfgragh.onion/auth/login?response_type=cookie\&client_id=vi}{}

\href{https://www.nytimes3xbfgragh.onion/section/todayspaper}{Today's
Paper}

\href{/es/section/america-latina}{América Latina}\textbar{}Entre la
pandemia y el presidente: el delicado equilibrio de la jefa de gobierno
de Ciudad de México

\url{https://nyti.ms/2Z9CndA}

\begin{itemize}
\item
\item
\item
\item
\item
\end{itemize}

\hypertarget{el-brote-de-coronavirus}{%
\subsubsection{\texorpdfstring{\href{https://www.nytimes3xbfgragh.onion/es/spotlight/coronavirus?name=styln-coronavirus-es\&region=TOP_BANNER\&block=storyline_menu_recirc\&action=click\&pgtype=Article\&impression_id=aa591640-f2bc-11ea-a620-0defd8f835ac\&variant=undefined}{El
brote de
coronavirus}}{El brote de coronavirus}}\label{el-brote-de-coronavirus}}

\begin{itemize}
\tightlist
\item
  \href{https://www.nytimes3xbfgragh.onion/es/interactive/2020/08/06/espanol/ciencia-y-tecnologia/tengo-covid-19-sintomas.html?name=styln-coronavirus-es\&region=TOP_BANNER\&block=storyline_menu_recirc\&action=click\&pgtype=Article\&impression_id=aa593d50-f2bc-11ea-a620-0defd8f835ac\&variant=undefined}{Síntomas}
\item
  \href{https://www.nytimes3xbfgragh.onion/es/2020/09/02/espanol/ciencia-y-tecnologia/vacunas-experimentales-coronavirus.html?name=styln-coronavirus-es\&region=TOP_BANNER\&block=storyline_menu_recirc\&action=click\&pgtype=Article\&impression_id=aa593d51-f2bc-11ea-a620-0defd8f835ac\&variant=undefined}{Vacunas
  experimentales}
\item
  \href{https://www.nytimes3xbfgragh.onion/es/2020/08/31/espanol/mundo/rebrote-espana.html?name=styln-coronavirus-es\&region=TOP_BANNER\&block=storyline_menu_recirc\&action=click\&pgtype=Article\&impression_id=aa593d52-f2bc-11ea-a620-0defd8f835ac\&variant=undefined}{Rebrote
  en España}
\item
  \href{https://www.nytimes3xbfgragh.onion/es/2020/09/02/espanol/negocios/desalojos-trump.html?name=styln-coronavirus-es\&region=TOP_BANNER\&block=storyline_menu_recirc\&action=click\&pgtype=Article\&impression_id=aa593d53-f2bc-11ea-a620-0defd8f835ac\&variant=undefined}{Moratoria
  a los desalojos}
\item
  \href{https://www.nytimes3xbfgragh.onion/es/2020/08/26/espanol/ciencia-y-tecnologia/coronavirus-afecta-hombres.html?name=styln-coronavirus-es\&region=TOP_BANNER\&block=storyline_menu_recirc\&action=click\&pgtype=Article\&impression_id=aa593d54-f2bc-11ea-a620-0defd8f835ac\&variant=undefined}{El
  impacto en los hombres}
\end{itemize}

Advertisement

\protect\hyperlink{after-top}{Continue reading the main story}

Supported by

\protect\hyperlink{after-sponsor}{Continue reading the main story}

\hypertarget{entre-la-pandemia-y-el-presidente-el-delicado-equilibrio-de-la-jefa-de-gobierno-de-ciudad-de-muxe9xico}{%
\section{Entre la pandemia y el presidente: el delicado equilibrio de la
jefa de gobierno de Ciudad de
México}\label{entre-la-pandemia-y-el-presidente-el-delicado-equilibrio-de-la-jefa-de-gobierno-de-ciudad-de-muxe9xico}}

Claudia Sheinbaum es la primera mujer, y la primera persona judía,
elegida para gobernar la capital y necesita el apoyo del presidente.
Pero, ¿cuán cerca puede permanecer de un hombre que ha minimizado la
pandemia?

\includegraphics{https://static01.graylady3jvrrxbe.onion/images/2020/09/03/world/05mexico-mayor-ES-00/merlin_176416098_ddde0fca-2696-4855-a1ed-2a3ef2e47e64-articleLarge.jpg?quality=75\&auto=webp\&disable=upscale}

\href{https://www.nytimes3xbfgragh.onion/by/natalie-kitroeff}{\includegraphics{https://static01.graylady3jvrrxbe.onion/images/2019/03/01/multimedia/author-natalie-kitroeff/author-natalie-kitroeff-thumbLarge.png}}

Por
\href{https://www.nytimes3xbfgragh.onion/by/natalie-kitroeff}{Natalie
Kitroeff}

\begin{itemize}
\item
  5 de septiembre de 2020
\item
  \begin{itemize}
  \item
  \item
  \item
  \item
  \item
  \end{itemize}
\end{itemize}

\href{https://www.nytimes3xbfgragh.onion/2020/09/05/world/americas/mexico-mayor-amlo-sheinbaum.html}{Read
in English}

\href{https://www.nytimes3xbfgragh.onion/newsletters/el-times}{Regístrate
para recibir nuestro boletín en español}

\begin{center}\rule{0.5\linewidth}{\linethickness}\end{center}

CIUDAD DE MÉXICO --- Fue el retrato perfecto de la delicada relación
entre el presidente mexicano y su protegida.

En medio del auge de la pandemia, el presidente Andrés Manuel López
Obrador convocó a sus aliados para una fotografía. Sin mascarillas y
ansiosos por complacerlo, todos se apretujaron junto a él, excepto por
una persona: Claudia Sheinbaum, una de sus colaboradoras de mayor
confianza.

Sheinbaum, jefa de gobierno de la Ciudad de México, estaba recelosa de
acercarse demasiado. Así que ese día de abril se ubicó en un extremo.
Fue literalmente la excepción, la única persona que mantuvo el
distanciamiento social en la sala.

La distancia ---física y política--- que ponga entre ella y el hombre
más poderoso de México definirá el legado de Sheinbaum, su futuro
político y el destino de millones de personas que viven en
la\href{https://worldpopulationreview.com/world-cities}{quinta ciudad
más grande del mundo}.

Desde el principio,
\href{https://www.nytimes3xbfgragh.onion/es/2020/05/08/espanol/america-latina/mexico-coronavirus.html}{López
Obrador minimizó la pandemia} y cuestionó de manera repetida los
argumentos científicos sobre el uso de las mascarillas, además de hacer
pocas pruebas. Buscando evadir los problemas económicos, apenas ha
restringido los viajes. Durante su mandato, México se ha convertido
\href{https://www.nytimes3xbfgragh.onion/es/interactive/2020/espanol/america-latina/coronavirus-en-mexico.html}{en
el cuarto país con el mayor número de muertes} por coronavirus en todo
el mundo.

Para Sheinbaum, una científica con un doctorado en ingeniería
energética, mantenerse demasiado cerca del presidente implicaría ignorar
las prácticas que sabe que beneficiarán a la salud pública. Si se aleja
demasiado, corre el riesgo de perder el apoyo de un hombre que forja
líderes políticos y que se dice la está considerando ---a la primera
mujer y la primera persona judía en ser elegida para gobernar a la
capital de la nación--- como la próxima candidata presidencial de su
partido.

Hasta ahora, su estrategia ha sido seguir los preceptos de la ciencia,
mientras se niega a criticar al presidente.

``Yo no voy a permitir que esto sea un conflicto político'', dijo
Sheinbaum, de 58 años. Estaba sentada rígidamente en su escritorio, y su
voz sonaba apagada por una mascarilla de tela. ``Pero también creo que
me corresponde una parte aquí en la ciudad y voy a cumplir con lo que yo
creo''.

\includegraphics{https://static01.graylady3jvrrxbe.onion/images/2020/09/03/world/05mexico-mayor-ES-01/merlin_176416074_5b191239-056e-49c8-877c-b3f97581abe2-articleLarge.jpg?quality=75\&auto=webp\&disable=upscale}

Mientras López Obrador todavía besaba a bebés en los mítines y comparaba
el virus con la gripe, Sheinbaum estaba haciendo planes para una
pandemia prolongada. Impulsó una agresiva campaña de pruebas y rastreo
de contactos. También instaló quioscos donde se hacen pruebas de manera
gratuita.

Exigió que todos en Ciudad de México se cubrieran la cara en el
transporte público y usa mascarilla cada vez que se dirige a la prensa.

Sheinbaum discute en privado con Hugo López-Gatell, el funcionario de
salud designado por el presidente para dirigir la respuesta del país al
coronavirus. Pero su personal ha recibido instrucciones de enfatizar, en
público, cuán alineados están los gobiernos de la ciudad y el federal y
cuánto tienen en común.

``Esa es la manera en que hemos actuado siempre respetando, siempre
informando'', dijo. ``Tratándonos de coordinar en lo más posible''.

La jefa de gobierno envía su primer mensaje de texto del día poco
después de las 05:00 a. m., a menudo suele estar dirigido a uno de los
expertos de su equipo que mide el progreso de la contención del brote en
la Ciudad de México, que es el peor a nivel nacional.

Todas las mañanas, pregunta cuántas personas acudieron a los hospitales
el día anterior, cuántas se fueron a casa, cuántas fueron intubadas y
cuántas murieron. Supervisa el rastreo de los vecindarios que organizan
fiestas, cuántas personas usaban mascarillas en el metro y, si en
realidad, la llevaban más bien como una bufanda para la barbilla.

El virus ha prosperado en la congestionada capital, hogar de nueve
millones de personas, donde la mitad de los habitantes son pobres. Y
aunque el número de víctimas ha sido espantoso (más de 11.000 han
muerto), los analistas dicen que podría haber sido peor sin las
estrategias de la jefa de gobierno.

Image

Familiares de Victor Bailón, quien murió de COVID-19 en julio, alrededor
de sus tumba. México tiene una de las tasas más altas de mortalidad en
el mundo.Credit...Daniel Berehulak para The New York Times

Al principio, Sheinbaum creó una línea telefónica donde las personas
podían reportar los síntomas del coronavirus y recibir un paquete
gratuito de mascarillas, un termómetro, gel antibacterial y analgésicos.

Los médicos le dijeron que las máscaras N95 que el gobierno federal
había importado de China eran demasiado estrechas para adaptarse a los
rostros de los mexicanos, por lo que convirtió una fábrica local en una
operación de fabricación de mascarillas.

Solo alrededor de 600 camas de unidades de cuidados intensivos estaban
equipadas para tratar a pacientes con coronavirus en la ciudad, por lo
que compró cientos de ventiladores de Estados Unidos, Alemania y China,
lo que ayudó a incrementar el número de camas de las unidades de
cuidados intensivos a más de 2000.

Para evaluar cómo están las cosas, Sheinbaum se centra en la cantidad de
personas ingresadas en los hospitales y, por estos días, le gusta lo que
ve. Cuando la capital reabrió gran parte de su economía el 1 de julio,
seis de cada diez camas de hospital estaban ocupadas, en comparación con
las cuatro de cada diez que se registran ahora.

``Lo que nos importa es que los hospitales no se saturen'', dice.

Según los epidemiólogos, el problema con su estrategia es que transmite
una percepción baja de la prevalencia del virus entre los jóvenes, que
tienen menos probabilidades de acudir al hospital. Cuando las personas
enfermas llegan a las salas de emergencia,
\href{https://www.nytimes3xbfgragh.onion/es/2020/08/10/espanol/america-latina/mexico-covid-hospitales.html}{suele
ser demasiado tarde para romper la cadena de transmisión}.

``Durante las dos semanas que estuvieron infectadas antes de llegar al
hospital, estuvieron expuestas a decenas o quizá a cientos de
personas'', dijo Thomas Tsai, del Instituto de Salud Global de Harvard.

La alternativa son las pruebas masivas, que la ciudad no está haciendo,
incluso después de invertir dinero en ese problema y triplicar las tasas
de pruebas. Ahora, Ciudad de México realiza 40 pruebas por cada 100.000
habitantes, en comparación con la cifra de solo nueve por cada 100.000
habitantes en todo el país. Pero sigue siendo un número bajo en
comparación con las 322 por cada 100.000 personas que se realizan en
Nueva York, o la tasa de 130 que se registra en Los Ángeles.

La proporción de personas que dan positivo en la ciudad de México ha
disminuido, pero se mantiene en alrededor del 30 por ciento, seis veces
la tasa que la Organización Mundial de la Salud considera segura para
reabrir la economía.

``Esto no es Estocolmo. Esto no es Singapur. Tenemos recursos
limitados'', dijo José Merino, quien dirige la agencia que coordina el
grupo de trabajo sobre el coronavirus en la capital. ``Y no podemos
evitar que la gente salga a la calle y trate de alimentar a sus
familias''.

Image

Sheinbaum es una científica reconocida: ``Voy a cumplir con lo que yo
creo'', dijo.Credit...Meghan Dhaliwal para The New York Times

La ciudad tendría que gastar aproximadamente una décima parte de su
presupuesto anual en pruebas si quisiera alcanzar los niveles de Nueva
York. Y el gobierno federal no está ayudando mucho. López-Gatell ha
dicho que cree que las pruebas masivas son una ``pérdida de tiempo'', lo
cual explica, en parte, por qué las tasas de pruebas nacionales de
México se encuentran entre las más bajas.

López-Gatell ha sido criticado por prometer desde un inicio el final
inminente de la pandemia y proyectar solo 6000 fallecimientos. Ahora hay
más de 65.000.

Y, sin embargo, el presidente de México confía completamente en él.
Sheinbaum acude a las reuniones con el mandatario para ``presentarle los
escenarios para la ciudad'' y transmitirle su creencia en la efectividad
de las mascarillas. ``Y él me dijo, `siempre ponte de acuerdo con
Hugo''', explica la funcionaria.

La tarea no ha sido fácil.

La jefa de gobierno se sintió profundamente incómoda cuando, a mediados
de marzo,
\href{https://www.nytimes3xbfgragh.onion/es/2020/03/16/espanol/deportes/mexico-coronavirus-amlo-futbol.html}{se
llevó a cabo un concierto en la capital} porque López-Gatell lo
permitió. Generalmente él se dirige a la prensa sin mascarillas y dijo
que exigir el uso del cubrebocas podría ``violar los derechos humanos''.

En julio, López-Gatell anunció en una conferencia de prensa que
Sheinbaum le había dado un regalo: un paquete de mascarillas. Pero no
usó ninguna.

La funcionaria dijo que había tenido ``diferencias públicas y notorias''
con López-Gatell, pero se niega a cuestionarlo.

``Yo no voy a entrar en contradicción con el gobierno de México'',
afirma.

Hija de dos judíos de izquierda, Sheinbaum fue criada atea en un país
católico que fue gobernado por el mismo partido durante siete décadas.

Conoció a López Obrador cuando él visitó su casa para reunirse con quien
ahora es su exesposo, Carlos Imaz, un líder político de izquierda, y
otros activistas. ``Yo preparé el café y las galletas'', recuerda.

Con el tiempo, se convirtió en una de las principales investigadoras
climáticas del país, y cuando López Obrador ganó la alcaldía de la
Ciudad de México en el año 2000, la nombró su secretaria de medio
ambiente.

En 2018, cuando López Obrador ---en su tercer intento--- fue elegido
para el cargo más alto en la lista del partido que fundó, Sheinbaum se
postuló con la coalición de él y resultó electa jefa de gobierno de la
Ciudad de México.

Image

Sheinbaum y el presidente López Obrador, han trabajado juntos desde hace
décadas y sus asesores dicen que entre ellos hay ``afecto y
admiración''.Credit...Fernando Llano/Associated Press

``Hay una parte muy importante de cariño y admiración que aún es común
con que te peleas, con quien te peleaste a lo largo de dos décadas,
hombro a hombro, desde la oposición, desde no tener poder, desde no
tener dinero, desde ser saboteado, desde ser perseguido'', dijo Ana
Laura Magaloni, una profesora de Derecho que asesoró la campaña de
Sheinbaum. ``De repente este grupo llega al poder y esto es un poco
también: `esa historia nos hace ser equipo'''.

Sentada en su oficina, frente a una foto de ella y del presidente,
Sheinbaum se enrolló el cordón de un oxímetro en el dedo. Después de que
un integrante de su equipo dio positivo por el virus, comenzó a medir
sus niveles de oxígeno varias veces al día.

``La pandemia en el momento que haya vacunas se va acabar'', dijo. Y
agregó: ``Entonces si hay una diferencia particular del uso del
cubrebocas o no, de si hacer más pruebas en determinado momento o no,
eso es menor frente al fondo de la transformación de nuestro país''.

Varias personas dijeron que su relación con López Obrador era como de
padre e hija. El presidente ``la ama y la protege'', dijo Marta Lamas,
una académica feminista que asesoró a la campaña de Sheinbaum. ``Y ella
es totalmente leal a él, y a su proyecto''.

Pero quienes han trabajado con López Obrador dicen que puede volverse
desconfiado, incluso con sus aliados más cercanos.

``Una relación paternal es como, te voy a proteger, pase lo que pase, no
es el caso de Andrés Manuel'', dijo Paola Ojeda, quien trabajó con López
Obrador cuando era jefe de gobierno de la ciudad y en tres de sus
campañas presidenciales.

Él no va a escoger a su sucesor hasta el último momento, asegura.

``Claudia se ha ganado, cada día, ese respeto y ese respaldo'', dice
Ojeda. ``Y ella sabe, como todos los que están cerca, que se puede
perder en el momento en el que haga algo indebido''.

Advertisement

\protect\hyperlink{after-bottom}{Continue reading the main story}

\hypertarget{site-index}{%
\subsection{Site Index}\label{site-index}}

\hypertarget{site-information-navigation}{%
\subsection{Site Information
Navigation}\label{site-information-navigation}}

\begin{itemize}
\tightlist
\item
  \href{https://help.nytimes3xbfgragh.onion/hc/en-us/articles/115014792127-Copyright-notice}{©~2020~The
  New York Times Company}
\end{itemize}

\begin{itemize}
\tightlist
\item
  \href{https://www.nytco.com/}{NYTCo}
\item
  \href{https://help.nytimes3xbfgragh.onion/hc/en-us/articles/115015385887-Contact-Us}{Contact
  Us}
\item
  \href{https://www.nytco.com/careers/}{Work with us}
\item
  \href{https://nytmediakit.com/}{Advertise}
\item
  \href{http://www.tbrandstudio.com/}{T Brand Studio}
\item
  \href{https://www.nytimes3xbfgragh.onion/privacy/cookie-policy\#how-do-i-manage-trackers}{Your
  Ad Choices}
\item
  \href{https://www.nytimes3xbfgragh.onion/privacy}{Privacy}
\item
  \href{https://help.nytimes3xbfgragh.onion/hc/en-us/articles/115014893428-Terms-of-service}{Terms
  of Service}
\item
  \href{https://help.nytimes3xbfgragh.onion/hc/en-us/articles/115014893968-Terms-of-sale}{Terms
  of Sale}
\item
  \href{https://spiderbites.nytimes3xbfgragh.onion}{Site Map}
\item
  \href{https://help.nytimes3xbfgragh.onion/hc/en-us}{Help}
\item
  \href{https://www.nytimes3xbfgragh.onion/subscription?campaignId=37WXW}{Subscriptions}
\end{itemize}
