Sections

SEARCH

\protect\hyperlink{site-content}{Skip to
content}\protect\hyperlink{site-index}{Skip to site index}

\href{https://www.nytimes3xbfgragh.onion/es/section/opinion}{Opinión}

\href{https://myaccount.nytimes3xbfgragh.onion/auth/login?response_type=cookie\&client_id=vi}{}

\href{https://www.nytimes3xbfgragh.onion/section/todayspaper}{Today's
Paper}

\href{/es/section/opinion}{Opinión}\textbar{}Ahora todos somos
socialmente torpes

\url{https://nyti.ms/3m1fbrO}

\begin{itemize}
\item
\item
\item
\item
\item
\item
\end{itemize}

\hypertarget{el-brote-de-coronavirus}{%
\subsubsection{\texorpdfstring{\href{https://www.nytimes3xbfgragh.onion/es/spotlight/coronavirus?name=styln-coronavirus-es\&region=TOP_BANNER\&block=storyline_menu_recirc\&action=click\&pgtype=Article\&impression_id=fba36710-f2af-11ea-a5f1-d358cc56c051\&variant=undefined}{El
brote de
coronavirus}}{El brote de coronavirus}}\label{el-brote-de-coronavirus}}

\begin{itemize}
\tightlist
\item
  \href{https://www.nytimes3xbfgragh.onion/es/interactive/2020/08/06/espanol/ciencia-y-tecnologia/tengo-covid-19-sintomas.html?name=styln-coronavirus-es\&region=TOP_BANNER\&block=storyline_menu_recirc\&action=click\&pgtype=Article\&impression_id=fba36711-f2af-11ea-a5f1-d358cc56c051\&variant=undefined}{Síntomas}
\item
  \href{https://www.nytimes3xbfgragh.onion/es/2020/09/02/espanol/ciencia-y-tecnologia/vacunas-experimentales-coronavirus.html?name=styln-coronavirus-es\&region=TOP_BANNER\&block=storyline_menu_recirc\&action=click\&pgtype=Article\&impression_id=fba38e20-f2af-11ea-a5f1-d358cc56c051\&variant=undefined}{Vacunas
  experimentales}
\item
  \href{https://www.nytimes3xbfgragh.onion/es/2020/08/31/espanol/mundo/rebrote-espana.html?name=styln-coronavirus-es\&region=TOP_BANNER\&block=storyline_menu_recirc\&action=click\&pgtype=Article\&impression_id=fba38e21-f2af-11ea-a5f1-d358cc56c051\&variant=undefined}{Rebrote
  en España}
\item
  \href{https://www.nytimes3xbfgragh.onion/es/2020/09/02/espanol/negocios/desalojos-trump.html?name=styln-coronavirus-es\&region=TOP_BANNER\&block=storyline_menu_recirc\&action=click\&pgtype=Article\&impression_id=fba38e22-f2af-11ea-a5f1-d358cc56c051\&variant=undefined}{Moratoria
  a los desalojos}
\item
  \href{https://www.nytimes3xbfgragh.onion/es/2020/08/26/espanol/ciencia-y-tecnologia/coronavirus-afecta-hombres.html?name=styln-coronavirus-es\&region=TOP_BANNER\&block=storyline_menu_recirc\&action=click\&pgtype=Article\&impression_id=fba38e23-f2af-11ea-a5f1-d358cc56c051\&variant=undefined}{El
  impacto en los hombres}
\end{itemize}

Advertisement

\protect\hyperlink{after-top}{Continue reading the main story}

\href{/es/section/opinion}{Opinión}

Supported by

\protect\hyperlink{after-sponsor}{Continue reading the main story}

Comentario

\hypertarget{ahora-todos-somos-socialmente-torpes}{%
\section{Ahora todos somos socialmente
torpes}\label{ahora-todos-somos-socialmente-torpes}}

Si se priva a la gente de las interacciones con las personas, sus
habilidades sociales se atrofiarán. Este es otro efecto secundario de la
pandemia.

\includegraphics{https://static01.graylady3jvrrxbe.onion/images/2020/09/01/opinion/04murphy-ES-00/01Murphy1-articleLarge.jpg?quality=75\&auto=webp\&disable=upscale}

Por \href{https://www.nytimes3xbfgragh.onion/by/kate-murphy}{Kate
Murphy}

Es periodista.

\begin{itemize}
\item
  7 de septiembre de 2020
\item
  \begin{itemize}
  \item
  \item
  \item
  \item
  \item
  \item
  \end{itemize}
\end{itemize}

\href{https://www.nytimes3xbfgragh.onion/2020/09/01/sunday-review/coronavirus-socially-awkward.html}{Read
in English}

\href{https://www.nytimes3xbfgragh.onion/newsletters/el-times}{Regístrate
para recibir nuestro boletín} con lo mejor de The New York Times.

\begin{center}\rule{0.5\linewidth}{\linethickness}\end{center}

Mientras el año escolar comienza en medio de una
\href{https://www.nytimes3xbfgragh.onion/2020/09/02/world/covid-19-coronavirus.html}{pandemia}
mundial,
\href{https://www.bbc.com/future/article/20200603-how-covid-19-is-changing-the-worlds-children}{mucha
gente está preocupada} por el
\href{https://www.theatlantic.com/family/archive/2020/06/how-quarantine-will-affect-kids-social-development/613381/}{impacto
negativo} que el aprendizaje virtual o con distanciamiento social podría
tener en el desarrollo de las habilidades sociales de los niños.

Sin embargo, ¿qué pasa con los adultos? Parece que los adultos privados
de un contacto variado y constante con sus pares pueden volverse tan
torpes en las interacciones sociales como los niños sin experiencia.

Investigaciones realizadas con
\href{https://www.ncjrs.gov/App/abstractdb/AbstractDBDetails.aspx?id=142494}{presos},
\href{https://scalar.usc.edu/works/bodies/Hikikomori:\%20The\%20Postmodern\%20Hermits\%20of\%20Japan\%20\%7C\%20Warscapes_thumb.pdf}{ermitaños},
\href{https://www.tandfonline.com/doi/abs/10.1080/15332985.2018.1522607}{soldados},
\href{https://www.nasa.gov/hrp/bodyinspace}{astronautas}, exploradores
polares y otras personas que han pasado largos periodos en aislamiento
indican que las habilidades sociales son como un músculo que se atrofia
por la falta de uso. La gente separada de la sociedad ---por las
circunstancias o por decisión propia--- reporta haberse sentido más
ansiosa, impulsiva, torpe e intolerante en términos sociales cuando
regresó a la vida normal.

Los psicólogos y los neurólogos aseguran que ahora nos está pasando algo
similar a todos nosotros, debido a la pandemia. Estamos perdiendo, de
una manera sutil pero inexorable, nuestra facilidad y agilidad en
situaciones sociales\ldots{} nos percatemos o no. Las señales están por
todas partes: la gente que comparte de más en Zoom, la exageración o la
malinterpretación de los comportamientos del otro, el anhelo de tener
contacto con los demás para luego no disfrutarlo de verdad.

Es un malestar social extraño que se puede arraigar con facilidad si no
reconocemos por qué está ocurriendo y no tomamos las medidas para
minimizar sus efectos.

``Lo primero que se debe entender es que hay razones biológicas detrás
de esto'', comentó Stephanie Cacioppo, la directora del
\href{https://braindynamics.uchicago.edu/}{Laboratorio de Dinámica
Cerebral} de la Universidad de Chicago. ``No es una patología ni un
trastorno mental''.

Según Cacioppo, hasta los más introvertidos de nosotros estamos
programados para querer compañía. Es un imperativo evolutivo porque, en
términos históricos, hay seguridad en los grupos grandes de personas. La
gente solitaria tenía dificultades para matar
\href{https://www.thevintagenews.com/2019/01/19/mammoth-bone/}{mamuts
lanudos} y para defenderse de los ataques enemigos.

Por lo tanto, cuando estamos desconectados de los demás, nuestros
cerebros lo interpretan como una amenaza mortal. Sentirse solo o aislado
es una señal biológica como el hambre y la sed. Además, al igual que no
comer cuando tienes hambre o no beber algo cuando estás deshidratado, no
interactuar con más personas cuando te sientes solo produce
\href{https://www.ncbi.nlm.nih.gov/pmc/articles/PMC5783394/}{efectos
negativos cognitivos, emocionales y psicológicos}, los cuales
probablemente muchos de nosotros estemos sintiendo ahora, según explica
Cacioppo.

Aunque te sientas muy cómodo en una burbuja pandémica con una pareja
romántica o familiares, de todas maneras te puedes sentir solo ---una
sensación que a menudo se disfraza de tristeza, irritabilidad, enojo y
letargo--- porque no estás obteniendo la gama completa de interacciones
humanas que necesitas, casi como no comer una dieta balanceada.
Subestimamos el beneficio de la camaradería casual en la oficina, el
gimnasio, las prácticas del coro o las clases de arte, sin mencionar los
intercambios espontáneos con extraños.

\includegraphics{https://static01.graylady3jvrrxbe.onion/images/2020/09/01/opinion/04murphy-ES-01/01Murphy2-articleLarge.jpg?quality=75\&auto=webp\&disable=upscale}

Muchos de nosotros no hemos conocido a nadie en meses.

``La interacción diaria con individuos en el mundo te da una sensación
de pertenencia y seguridad que proviene de sentirte parte de una
comunidad y una red más amplias, o de tener acceso a ellas'', comentó
Stefan Hofmann, profesor de Psicología de la Universidad de Boston. ``El
aislamiento social destruye esa red''.

La privación pone a nuestros cerebros en modo de supervivencia, lo cual
disminuye nuestra capacidad para reconocer y responder de manera
apropiada a las sutilezas y complejidades inherentes en las situaciones
sociales. En cambio, nos volvemos
\href{https://www.ncbi.nlm.nih.gov/pmc/articles/PMC3874845/}{hipervigilantes}
e hipersensibles. Si encima de eso agregas un virus aparentemente
caprichoso, todos estamos programados para pelear o escapar.

Si te miran de reojo, de inmediato crees que le caes mal a la otra
persona. Un comentario confuso es interpretado como un insulto. Al mismo
tiempo, te sientes más cohibido y temes que cualquier paso en falso te
pondrá en riesgo. Como resultado, las situaciones sociales, incluso una
llamada amigable, se convierten en problemas que es mejor evitar. La
gente comienza a retraerse y encuentra como justificación que está
demasiado cansada, que no le caía muy bien esa persona desde el
principio o que hay algo que preferiría ver en Netflix.

Es un fenómeno que la médica británica Beth Healey conoce muy bien.
Healey pasó un año en un
\href{https://www.esa.int/Science_Exploration/Human_and_Robotic_Exploration/Concordia/Spaceship_Concordia}{puesto
remoto} en la Antártida como parte de un equipo que realizaba
investigaciones para la Agencia Espacial Europea.

``Antes de empezar el proyecto tuvimos mucha capacitación acerca de lo
complicado que puede ser regresar a casa'', comentó. ``Te lo tomas un
poco en broma y piensas que no te sucederá a ti''.

Sin embargo, como era de esperarse, cuando Healey regresó a la
civilización a inicios de 2016, dijo que se sintió intranquila. ``Me vi
con una buena amiga en Nueva Zelanda y me di cuenta cómo me escondía un
poco detrás de ella al momento de registrarnos en el hotel'', mencionó.
``Normalmente habría tomado la iniciativa con gusto, pero prefería que
hablaran con ella''.

Durante meses, le daba ansiedad subirse a un autobús y se sentía
abrumada de ir al supermercado. ``Fue muy extraño y se siente parecido a
lo que estamos viendo ahora después del aislamiento'' por el
coronavirus, comentó. ``De cierta manera fue más fácil salir de la
Antártida al mundo porque nadie más se sentía así, pero ahora todo el
mundo está un poco raro''.

A algunos de sus compañeros de equipo les costó tanto readaptarse que de
inmediato se apuntaron para regresar a la Antártida. A menudo les sucede
lo mismo a los soldados que regresan después de estar mucho tiempo
apostados y también a los presos que han pasado años en confinamiento
solitario. Aunque regresan a casa con familias que los apoyan, después
de días o semanas, quieren volver.

``No quiero hacer una equivalencia entre los presos en confinamiento
solitario y la situación que todos estamos pasando, pero definitivamente
hay similitudes'', comentó Craig Haney, profesor de Psicología de la
Universidad de California, campus Santa Cruz, que estudia los
\href{https://scholarlycommons.law.northwestern.edu/nulr/vol115/iss1/5/}{efectos
del aislamiento en presos}. ``El hecho de que la gente se sienta
incómoda con otras personas es parte de lo que ocurre cuando se nos
niega el contacto social normal del que tanto dependemos''.

En todas las interacciones debes hacer incontables juicios intuitivos:
interpretar palabras, gestos y expresiones, y reaccionar acorde a ello.
También debes comprender el tiempo y el ritmo correctos, así como
valorar qué tanto compartir y a quién. La interacción social es una de
las cosas más complicadas que les pedimos a nuestros cerebros. En
circunstancias normales, podemos practicar mucho, por eso casi no se
siente. No piensas en ello. Sin embargo, cuando tienes menos
oportunidades para practicar, pierdes el toque. La calidad surreal y
tosca de las interacciones virtuales o con mascarilla
\href{https://www.nytimes3xbfgragh.onion/2020/04/29/sunday-review/zoom-video-conference.html}{solo
empeoran la situación}.

Image

Credit...Ashley Gilbertson/VII Photo

Los expertos en aislamiento aseguran que este es un terreno complicado y
aconsejan tomar medidas para mantener tus habilidades sociales lo más
activas que puedas durante este tiempo antisocial. Haney comentó que los
presos que regresan después de pasar un tiempo en confinamiento
solitario son los que se percataron de que el aislamiento era una
amenaza grave para su sentido de identidad y seguridad, y aprovecharon
todas las oportunidades para tener contacto con otras personas.

``La gente que sobrelleva la situación de mejor manera es la que escribe
cartas y tiene visitas, y quienes mantienen comunicación con otras
personas, aunque sea tan solo a través de los muros de un pabellón'',
señaló. ``A los que les va peor son los que se retraen profundamente y
le rehúyen al contacto con otras personas''.

Por eso es importante apartar tiempo todos los días para conectar con
los demás, sin importar que sea por medio de un chat, una llamada
telefónica o, como mínimo, un texto amable, manteniendo el
distanciamiento social.

Además, conforme todos volvamos a salir de nuestro confinamiento y
ampliemos nuestros círculos sociales, no esperes que todo ni todos se
hayan mantenido iguales. Healey comentó que los miembros de su equipo de
expedición polar que tuvieron más dificultades para reintegrarse fueron
los que esperaban regresar a sus trabajos y relaciones exactamente como
las habían dejado. Es inevitable que la gente cambie con el tiempo y,
sin duda, después de un evento significativo, como una pandemia, que
cambia su vida de forma drástica y la hace desconfiar de lo que creía
conocer. Los valores cambian. Las personalidades se alteran. No somos
los mismos.

Por lo tanto, no seas tan duro contigo mismo ni con los demás. Ten
paciencia con tu rareza y la de otras personas.

Kate Murphy, colaboradora frecuente de The New York Times, es la autora
de \href{https://us.macmillan.com/books/9781250297198}{``You're Not
Listening: What You're Missing and Why It Matters''}.

Advertisement

\protect\hyperlink{after-bottom}{Continue reading the main story}

\hypertarget{site-index}{%
\subsection{Site Index}\label{site-index}}

\hypertarget{site-information-navigation}{%
\subsection{Site Information
Navigation}\label{site-information-navigation}}

\begin{itemize}
\tightlist
\item
  \href{https://help.nytimes3xbfgragh.onion/hc/en-us/articles/115014792127-Copyright-notice}{©~2020~The
  New York Times Company}
\end{itemize}

\begin{itemize}
\tightlist
\item
  \href{https://www.nytco.com/}{NYTCo}
\item
  \href{https://help.nytimes3xbfgragh.onion/hc/en-us/articles/115015385887-Contact-Us}{Contact
  Us}
\item
  \href{https://www.nytco.com/careers/}{Work with us}
\item
  \href{https://nytmediakit.com/}{Advertise}
\item
  \href{http://www.tbrandstudio.com/}{T Brand Studio}
\item
  \href{https://www.nytimes3xbfgragh.onion/privacy/cookie-policy\#how-do-i-manage-trackers}{Your
  Ad Choices}
\item
  \href{https://www.nytimes3xbfgragh.onion/privacy}{Privacy}
\item
  \href{https://help.nytimes3xbfgragh.onion/hc/en-us/articles/115014893428-Terms-of-service}{Terms
  of Service}
\item
  \href{https://help.nytimes3xbfgragh.onion/hc/en-us/articles/115014893968-Terms-of-sale}{Terms
  of Sale}
\item
  \href{https://spiderbites.nytimes3xbfgragh.onion}{Site Map}
\item
  \href{https://help.nytimes3xbfgragh.onion/hc/en-us}{Help}
\item
  \href{https://www.nytimes3xbfgragh.onion/subscription?campaignId=37WXW}{Subscriptions}
\end{itemize}
