Sections

SEARCH

\protect\hyperlink{site-content}{Skip to
content}\protect\hyperlink{site-index}{Skip to site index}

\href{https://www.nytimes3xbfgragh.onion/es/section/ciencia-y-tecnologia}{Ciencia
y Tecnología}

\href{https://myaccount.nytimes3xbfgragh.onion/auth/login?response_type=cookie\&client_id=vi}{}

\href{https://www.nytimes3xbfgragh.onion/section/todayspaper}{Today's
Paper}

\href{/es/section/ciencia-y-tecnologia}{Ciencia y
Tecnología}\textbar{}Los dentistas enfrentan una epidemia de dientes
rotos. ¿Qué está pasando?

\url{https://nyti.ms/2ZlW1TV}

\begin{itemize}
\item
\item
\item
\item
\item
\item
\end{itemize}

\hypertarget{el-brote-de-coronavirus}{%
\subsubsection{\texorpdfstring{\href{https://www.nytimes3xbfgragh.onion/es/spotlight/coronavirus?name=styln-coronavirus-es\&region=TOP_BANNER\&block=storyline_menu_recirc\&action=click\&pgtype=Article\&impression_id=36a4d0b0-f283-11ea-8f04-735a2495ff97\&variant=undefined}{El
brote de
coronavirus}}{El brote de coronavirus}}\label{el-brote-de-coronavirus}}

\begin{itemize}
\tightlist
\item
  \href{https://www.nytimes3xbfgragh.onion/es/interactive/2020/08/06/espanol/ciencia-y-tecnologia/tengo-covid-19-sintomas.html?name=styln-coronavirus-es\&region=TOP_BANNER\&block=storyline_menu_recirc\&action=click\&pgtype=Article\&impression_id=36a4d0b1-f283-11ea-8f04-735a2495ff97\&variant=undefined}{Síntomas}
\item
  \href{https://www.nytimes3xbfgragh.onion/es/2020/09/02/espanol/ciencia-y-tecnologia/vacunas-experimentales-coronavirus.html?name=styln-coronavirus-es\&region=TOP_BANNER\&block=storyline_menu_recirc\&action=click\&pgtype=Article\&impression_id=36a4d0b2-f283-11ea-8f04-735a2495ff97\&variant=undefined}{Vacunas
  experimentales}
\item
  \href{https://www.nytimes3xbfgragh.onion/es/2020/08/31/espanol/mundo/rebrote-espana.html?name=styln-coronavirus-es\&region=TOP_BANNER\&block=storyline_menu_recirc\&action=click\&pgtype=Article\&impression_id=36a4d0b3-f283-11ea-8f04-735a2495ff97\&variant=undefined}{Rebrote
  en España}
\item
  \href{https://www.nytimes3xbfgragh.onion/es/2020/09/02/espanol/negocios/desalojos-trump.html?name=styln-coronavirus-es\&region=TOP_BANNER\&block=storyline_menu_recirc\&action=click\&pgtype=Article\&impression_id=36a4d0b4-f283-11ea-8f04-735a2495ff97\&variant=undefined}{Moratoria
  a los desalojos}
\item
  \href{https://www.nytimes3xbfgragh.onion/es/2020/08/26/espanol/ciencia-y-tecnologia/coronavirus-afecta-hombres.html?name=styln-coronavirus-es\&region=TOP_BANNER\&block=storyline_menu_recirc\&action=click\&pgtype=Article\&impression_id=36a4d0b5-f283-11ea-8f04-735a2495ff97\&variant=undefined}{El
  impacto en los hombres}
\end{itemize}

Advertisement

\protect\hyperlink{after-top}{Continue reading the main story}

Supported by

\protect\hyperlink{after-sponsor}{Continue reading the main story}

\hypertarget{los-dentistas-enfrentan-una-epidemia-de-dientes-rotos-quuxe9-estuxe1-pasando}{%
\section{Los dentistas enfrentan una epidemia de dientes rotos. ¿Qué
está
pasando?}\label{los-dentistas-enfrentan-una-epidemia-de-dientes-rotos-quuxe9-estuxe1-pasando}}

A principios de junio retorné a mi consultorio dental y las fracturas de
dientes empezaron a aparecer: al menos una al día, todos los días.

\includegraphics{https://static01.graylady3jvrrxbe.onion/images/2020/09/09/well/well-teeth/well-teeth-articleLarge.jpg?quality=75\&auto=webp\&disable=upscale}

Por Tammy Chen, D.D.S.

\begin{itemize}
\item
  8 de septiembre de 2020
\item
  \begin{itemize}
  \item
  \item
  \item
  \item
  \item
  \item
  \end{itemize}
\end{itemize}

\href{https://www.nytimes3xbfgragh.onion/2020/09/08/well/live/dentists-tooth-teeth-cracks-fractures-coronavirus-stress-grinding.html}{Read
in English}

\href{https://www.nytimes3xbfgragh.onion/newsletters/el-times}{Regístrate
para recibir nuestro boletín} con lo mejor de The New York Times.

\begin{center}\rule{0.5\linewidth}{\linethickness}\end{center}

``¿Cómo está tu clínica dental?'', me preguntó una amiga, con el ceño
fruncido y un gesto de preocupación evidente en su rostro.

Últimamente, he visto mucho esa mirada. Desde el comienzo de la
pandemia, con el confinamiento de la ciudad y las medidas de
distanciamiento social firmemente arraigadas, varios amigos y familiares
han asumido que debo estar al borde del cierre. Pero les hice saber que
estoy más ocupada que nunca.

``¿En serio?'', preguntó. ``¿Cómo es posible?''.

``He visto más fracturas de dientes en las últimas seis semanas que en
los últimos seis años'', le expliqué.

Desafortunadamente, no es una exageración.

A mediados de marzo cerré mi consultorio del centro de Manhattan y no
atendí nada que no fuera una emergencia dental, de acuerdo con las
directrices de la Asociación Dental Americana y el mandato del gobierno
estatal. Casi de inmediato, noté un aumento en las llamadas telefónicas:
dolor de mandíbula, sensibilidad dental, dolor en las mejillas y
migrañas. A la mayoría de esos pacientes los traté eficazmente a través
de la telemedicina.

Pero cuando reabrí mi consulta, a inicios de junio, las fracturas
comenzaron a aparecer: al menos una al día, todos los días que he estado
en el consultorio. En promedio, veo de tres a cuatro; los días malos son
más de seis fracturas.

¿Qué ocurre?

Una respuesta obvia es el estrés. Desde las
\href{https://www.nytimes3xbfgragh.onion/2020/04/13/style/why-weird-dreams-coronavirus.html}{pesadillas
inducidas por la COVID} hasta el
``\href{https://www.nytimes3xbfgragh.onion/es/2020/07/22/espanol/negocios/doomscrolling-que-es.html}{doomsurfing}''
o la
``\href{https://www.wellandgood.com/coronavirus-anxiety-scale-coronaphobia/}{coronafobia}'',
no es un secreto que la ansiedad relacionada con la pandemia ha afectado
nuestra salud mental colectiva. Ese estrés, a su vez, lleva a apretar y
rechinar los dientes, lo que puede dañarlos.

Pero más específicamente, el aumento que estoy viendo en el traumatismo
dental puede ser el resultado de dos factores adicionales.

Primero, un número sin precedentes de estadounidenses comenzaron a
trabajar desde casa, a menudo en cualquier lugar donde puedan improvisar
una estación de trabajo: en el sofá, encaramados en un taburete,
escondidos en una esquina de la barra de la cocina. Las incómodas
posiciones corporales que se producen pueden hacer que
\href{https://www.nytimes3xbfgragh.onion/2020/09/04/well/live/ergonomics-work-from-home-injuries.html}{encorven
los hombros hacia adelante}, curvando la columna vertebral en algo
parecido a una C.

Si te preguntas por qué una dentista se preocupa por la ergonomía, la
simple verdad es que los nervios del cuello y los músculos de los
hombros conducen a la articulación temporomandibular, o ATM, que conecta
la mandíbula con el cráneo. Una mala postura durante el día puede
traducirse en un problema de rechinamiento por la noche.

En segundo lugar, la mayoría no estamos teniendo el sueño reparador que
necesitamos. Desde el comienzo de la pandemia, he escuchado a los
pacientes describir su repentina inquietud e insomnio. Estos son signos
de un sistema nervioso simpático hiperactivo o dominante, que impulsa la
respuesta de ``lucha o huída'' del cuerpo. Piensa en un gladiador
preparándose para la batalla: cierra los puños, aprieta la mandíbula.
Debido al estrés del coronavirus, el cuerpo se mantiene en un estado de
excitación, listo para la batalla, en vez de descansar y recargarse.
Toda esa tensión va directamente a los dientes.

Entonces, ¿qué podemos hacer?

Te sorprendería saber cuánta gente no se da cuenta de que está apretando
y rechinando. Incluso los pacientes que vienen a la oficina quejándose
de dolor y sensibilidad suelen ser incrédulos cuando se lo señalo. ``Oh,
no. No rechino los dientes'', es una frase que escucho una y otra vez, a
pesar de que a menudo los \emph{veo} hacerlo.

La conciencia es clave. ¿Tus dientes se están tocando en este momento?
¿Incluso mientras lees este artículo? Si es así, es una señal segura de
que estás haciendo algún daño, tus dientes no deberían tocarse durante
el día a menos que estés comiendo y masticando la comida activamente. En
cambio, tu mandíbula debería estar activamente relajada, con un poco de
espacio entre los dientes cuando los labios están cerrados. Quédate
atento e intenta evitar rechinar los dientes cuando te sorprendas
haciéndolo.

Si tienes un protector bucal nocturno, guarda oclusal o un retenedor,
dispositivos que mantienen los dientes alineados y evitan el
rechinamiento, intenta colocártelos durante el día. Estos aparatos
proporcionan una barrera física, al absorber y dispersar la presión.
Como a menudo les digo a mis pacientes, prefiero que se rompa un
protector bucal nocturno que un diente. Tu dentista puede hacer un
protector bucal a la medida para asegurar un ajuste adecuado.

Y como muchos seguiremos trabajando desde casa durante meses, es
imperativo establecer un puesto de trabajo adecuado. Lo ideal es que, al
sentarse, los hombros estén alineados con las caderas y las orejas con
los hombros. Las pantallas de las computadoras deben estar a la altura
de los ojos; apoya tu monitor o portátil en una caja o una pila de
libros si no tienes una silla o escritorio ajustable.

Considera, también, que en nuestras nuevas oficinas en casa no es raro
salir de la cama, encontrar un sofá y luego sentarse durante nueve horas
al día. Intenta ponerte de pie, cuando sea posible, e incorporar más
movimiento. Utiliza cada descanso para ir al baño, o cada llamada
telefónica, como una oportunidad de dar más pasos, sin importar lo
pequeña que sea tu casa o apartamento.

Al final del día de trabajo, aconsejo a mis pacientes que ---disculpen
el término médico tan técnico--- ``se meneen como un pez''. Acuéstate
boca arriba, con los brazos extendidos por encima de la cabeza, y mueve
suavemente los brazos, hombros, caderas y pies de un lado a otro. El
objetivo es descomprimir y alargar la columna vertebral, así como
liberar y aliviar parte de esa tensión y presión.

Si tienes una bañera, considera tomar un baño de sales de Epson de 20
minutos por la noche. Concéntrate en respirar por la nariz y relajarte,
en vez de pensar en el trabajo, revisar los correos electrónicos o
contemplar el horario de regreso a clases de tus hijos (sé que esto es
más fácil decirlo que hacerlo).

Luego, justo antes de acostarte, tómate cinco minutos para tranquilizar
tu mente. Cierra los ojos, aspira la lengua hasta el paladar e inhala y
exhala por la nariz, inhala y exhala. Es una solución nada tecnológica,
pero la respiración profunda es una de las formas más efectivas de
estimular el nervio vago, que controla el sistema nervioso parasimpático
del cuerpo. Como contrapartida a la respuesta de lucha o huida, el
sistema nervioso parasimpático activa el mecanismo de ``descanso y
digestión'' del cuerpo: disminuye el ritmo cardíaco, baja la presión
sanguínea, permite un sueño más tranquilo y reparador. Cuando más
relajado esté el cuerpo, más probable es que te despiertes con menos
tensión en la mandíbula. Eso significa rechinar menos por la noche.

Los dientes son naturalmente frágiles, y todos tienen pequeñas fisuras
en sus dientes por masticar, rechinar y el uso diario. Pueden soportar
solo un trauma antes de romperse. Piensa en una pared que tiene una
pequeña grieta en forma de araña que, con el tiempo, puede hacerse cada
vez más grande hasta que se convierte en un agujero enorme. Queremos
evitar cualquier estrés adicional por rechinamiento que podría causar
que estas grietas microscópicas se conviertan en grietas más grandes y,
en última instancia, en un fallo catastrófico que requiera un
tratamiento de conducto, una corona u otro tratamiento dental
importante.

Si aún no lo has hecho, agenda una cita con tu dentista. Mantente al día
con tu programa de evaluación y limpieza de cada semestre.

Y si no haces nada más, consigue un protector bucal nocturno.

Tammy Chen es prostodoncista y propietaria de
\href{https://www.cpdanyc.com/}{Central Park Dental Aesthetics} en el
centro de Manhattan.

Advertisement

\protect\hyperlink{after-bottom}{Continue reading the main story}

\hypertarget{site-index}{%
\subsection{Site Index}\label{site-index}}

\hypertarget{site-information-navigation}{%
\subsection{Site Information
Navigation}\label{site-information-navigation}}

\begin{itemize}
\tightlist
\item
  \href{https://help.nytimes3xbfgragh.onion/hc/en-us/articles/115014792127-Copyright-notice}{©~2020~The
  New York Times Company}
\end{itemize}

\begin{itemize}
\tightlist
\item
  \href{https://www.nytco.com/}{NYTCo}
\item
  \href{https://help.nytimes3xbfgragh.onion/hc/en-us/articles/115015385887-Contact-Us}{Contact
  Us}
\item
  \href{https://www.nytco.com/careers/}{Work with us}
\item
  \href{https://nytmediakit.com/}{Advertise}
\item
  \href{http://www.tbrandstudio.com/}{T Brand Studio}
\item
  \href{https://www.nytimes3xbfgragh.onion/privacy/cookie-policy\#how-do-i-manage-trackers}{Your
  Ad Choices}
\item
  \href{https://www.nytimes3xbfgragh.onion/privacy}{Privacy}
\item
  \href{https://help.nytimes3xbfgragh.onion/hc/en-us/articles/115014893428-Terms-of-service}{Terms
  of Service}
\item
  \href{https://help.nytimes3xbfgragh.onion/hc/en-us/articles/115014893968-Terms-of-sale}{Terms
  of Sale}
\item
  \href{https://spiderbites.nytimes3xbfgragh.onion}{Site Map}
\item
  \href{https://help.nytimes3xbfgragh.onion/hc/en-us}{Help}
\item
  \href{https://www.nytimes3xbfgragh.onion/subscription?campaignId=37WXW}{Subscriptions}
\end{itemize}
