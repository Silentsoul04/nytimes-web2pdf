Sections

SEARCH

\protect\hyperlink{site-content}{Skip to
content}\protect\hyperlink{site-index}{Skip to site index}

\href{https://www.nytimes3xbfgragh.onion/es/section/cultura}{Cultura}

\href{https://myaccount.nytimes3xbfgragh.onion/auth/login?response_type=cookie\&client_id=vi}{}

\href{https://www.nytimes3xbfgragh.onion/section/todayspaper}{Today's
Paper}

\href{/es/section/cultura}{Cultura}\textbar{}Una Mulán para cada época:
la heroína más adaptable

\url{https://nyti.ms/3haNoSb}

\begin{itemize}
\item
\item
\item
\item
\item
\item
\end{itemize}

Advertisement

\protect\hyperlink{after-top}{Continue reading the main story}

Supported by

\protect\hyperlink{after-sponsor}{Continue reading the main story}

\hypertarget{una-muluxe1n-para-cada-uxe9poca-la-herouxedna-muxe1s-adaptable}{%
\section{Una Mulán para cada época: la heroína más
adaptable}\label{una-muluxe1n-para-cada-uxe9poca-la-herouxedna-muxe1s-adaptable}}

¿Es una novata o una luchadora hábil? ¿Acaba siendo una novia radiante o
se suicida? La leyenda de la guerrera se ha transformado mucho a lo
largo de 1500 años.

\includegraphics{https://static01.graylady3jvrrxbe.onion/images/2020/09/06/arts/08Mulan-ES-00/merlin_170347383_c6dc432f-3704-4951-9941-07dc3dcd4945-articleLarge.jpg?quality=75\&auto=webp\&disable=upscale}

Por \href{https://www.nytimes3xbfgragh.onion/by/robert-ito}{Robert Ito}

\begin{itemize}
\item
  8 de septiembre de 2020
\item
  \begin{itemize}
  \item
  \item
  \item
  \item
  \item
  \item
  \end{itemize}
\end{itemize}

\href{https://www.nytimes3xbfgragh.onion/2020/09/03/movies/mulan-history.html}{Read
in English}

\href{https://www.nytimes3xbfgragh.onion/newsletters/el-times}{Regístrate
para recibir nuestro boletín} con lo mejor de The New York Times.

\begin{center}\rule{0.5\linewidth}{\linethickness}\end{center}

Cuando los rumores de una versión con personajes de carne y hueso y no
musical de empezaron a correr hace unos años, muchos fanáticos de la
película de Disney de 1998
\href{https://www.slashfilm.com/mulan-remake-is-not-a-musical/}{se
quejaron}. ¿Sin grandes números musicales ni baladas vertiginosas? ¿Sin
Mushu, el dragón bromista, o Li Shang, el conflictivo interés amoroso de
la película? ¿Sin
\href{https://www.youtube.com/watch?v=sL9bUSfaO6M}{\emph{Mi reflejo}}?
Muchos sintieron que los cineastas le eran infieles a la leyenda de
Mulán, o al menos a la versión que había creado Disney.

Pero Mulán siempre ha sido la más adaptable de las heroínas. Mucho antes
de que los fans criticasen a Disney por tomarse libertades con la
querida heroína animada, poetas, escritores, dramaturgos y cineastas
crearon decenas de versiones muy diferentes de la legendaria guerrera.
En algunas es una generala de ejército endurecida; en otras tiene
poderes mágicos; en algunas más es una arquera impecable. En una versión
animada, es un bicho.

A lo largo de los siglos, se la ha celebrado en obras de teatro y
óperas, en musicales y series de televisión, en libros ilustrados y
novelas y ficción para jóvenes adultos. En la pantalla grande ha
protagonizado películas mudas (\emph{Hua Mulán se une al ejército} de
1927); un magnífico musical a todo color de los legendarios hermanos
Shaw (\emph{Generala Hua Mu-Lan}, 1964); una película épica, bélica y
llena de acción (\emph{Hua Mulán}, de 2009, con Zhao Wei), así como
cierta película animada de Disney que presentaba a un pequeño dragón
rojo.

En la última Mulán, que se estrenó el 4 de septiembre en Disney+, la
actriz chino-estadounidense Yifei Liu protagoniza un relato que combina
impresionantes secuencias de batalla (el presupuesto de 200 millones de
dólares de la película incluía una parte para 80 jinetes acrobáticos de
Kazajstán y Mongolia) con un relato que hace mucho con el trasfondo del
desafío a los roles de género de la historia.

Y aunque no hay ningún Mushu (``necesitábamos que Mulán se enfrentase a
sus propios desafíos y tomase sus propias decisiones'', señaló la
directora Niki Caro), hay varias referencias a la película animada de
1998. También hay varios guiños a varias versiones antiguas de la
historia, entre las que destaca la
\href{https://es.wikipedia.org/wiki/Hua_Mulan}{\emph{Balada
de}}\emph{Mulán}, el poema del siglo V o VI que inició todo.

La
\href{https://en.wikisource.org/wiki/Translation:Ballad_of_Mulan}{\emph{Balada
de}}\emph{Mulán} es un cuento relativamente simple, de solo seis
estrofas: Mulán deja su pueblo para tomar el lugar de su padre enfermo
en el ejército del emperador. Durante una docena de años, sirve
noblemente, siempre disfrazada de hombre; al final, rechaza las
recompensas y los honores para volver a casa, donde sus antiguos
camaradas por fin se enteran que, ¡sorpresa!, Mulán es mujer.

El poema termina con la imagen de dos conejos (``¿cómo puedes distinguir
a la hembra del macho?'') que corren uno al lado del otro, una escena
que se repite en la nueva película.

``Cada vez que había una imagen de la balada, quería traerla a la
película'', dijo Caro. ``Obviamente, mucha de la audiencia internacional
puede no conocer la balada, pero para los que sí la conocen, es lindo''.

Después del poema original, las versiones posteriores de la historia de
Mulán añadieron tramas y detalles para dar cuerpo a la historia. En la
obra de teatro del siglo XVI \emph{La heroína Mulán va a la guerra en
lugar de su padre}, ella tiene los pies vendados. ``En esa época, las
mujeres de las clases altas se vendaban los pies, y el dramaturgo quería
asegurarse de que Mulán fuera vista como el icono ideal de la
feminidad'', dijo Lan Dong, autor de
Mulán\href{https://muse.jhu.edu/book/9786}{\emph{'s Legend and Legacy in
China and the United States}} (La leyenda y el legado de Mulán en China
y Estados Unidos) y profesor de inglés en la Universidad de Illinois
Springfield. ``Tenía que ser perfecta''.

En la novela de 1695 \emph{El romance de las dinastías Sui y Tang},
Mulán conoce a una compañera guerrera que por un juramento se convierte
en su hermana; al final, Mulán se suicida cuando el Khan la convoca para
ser su concubina. ``Muchas versiones enfatizan su virtud'', dijo Dong.
``Incluso después de todos esos años y de todo lo que ha pasado, se
mantuvo intacta''.

Las adaptaciones al cine ampliaron aún más la leyenda. En la película
china de 1939
Mulán\href{https://www.youtube.com/watch?v=B99xRkrwdTs\&t=2288s\&ab_channel=ModernChineseCulturalStudies}{\emph{se
une al ejército}}, la heroína es una hábil cazadora, guerrera y,
finalmente, generala; la película termina con Mulán como una novia
radiante.

\includegraphics{https://static01.graylady3jvrrxbe.onion/images/2020/09/06/arts/08Mulan-ES-01/29mulan-depiction4-articleLarge.jpg?quality=75\&auto=webp\&disable=upscale}

El filme de ópera de Huangmei
\href{https://www.youtube.com/watch?v=IxNRvoIdLHQ}{\emph{Generala Hua
Mu-Lan}} es quizás la más exuberante de las anteriores a Disney. Además
de las llamativas secuencias de combate, los vibrantes trajes y los
juegos de beber al estilo papa caliente (durante los cuales Mulán se
emborracha), la película presenta un montón de canciones. Todos cantan,
sobre todo lo imaginable: el asma de papá; la importancia de la devoción
filial; los roles de género y la división desigual del trabajo en el
hogar; esos bárbaros ``imprudentes y agresivos'' que invaden nuestra
patria, etc.

Cuando los cineastas de Disney comenzaron a trabajar en la última
historia de Mulán, recurrieron a una serie de versiones para inspirarse.
Estaba la balada original, por supuesto, así como variaciones
regionales, que examinaron con la ayuda de asesores de China. Miraron
obras de teatro y películas, incluido el drama con Zhao Wei. ``Excavamos
muy profundamente para ver el arco de la historia'', dijo Jason Reed,
uno de los productores, ``para ver qué elementos habían permanecido
consistentes a lo largo del tiempo y qué elementos habían sido adaptados
para encajar en el tiempo y el medio en que la historia estaba siendo
contada''.

En muchos relatos, Mulán es una luchadora experimentada desde antes de
unirse al ejército. La versión animada retrataba a Mulán como una novata
(antes de que esa humillante secuencia del campamento de entrenamiento
la convirtiera en un ``hombre''), pero en su última versión, nos
enteramos de que Mulán ha sido entrenada por su padre desde que era una
niña.

Otro tema central de la leyenda es la devoción filial, con Mulán
recibiendo la bendición de sus padres antes de ir a la guerra. La
devoción filial también dicta que ella regrese a casa con sus padres
después de terminar su periodo de servicio. Su travestismo es perdonado
(después de todo, hubo una guerra), siempre y cuando vuelva a su lugar
como hija y esposa, después del conflicto. ``Por eso, a pesar de sus
transgresiones, fue puesta en un pedestal incluso en la China
premoderna'', dijo Dong. ``Ella rompe las reglas sin amenazar el
sistema''.

Image

~La película animada de 1998 de Disney fue una de las inspiraciones de
esta versión con actores de carne y hueso.Credit...Disney Enterprises,
Inc.

En ambas películas de Disney, Mulán se escapa al amparo de la oscuridad,
difícilmente la hija obediente. La nueva, sin embargo, modifica aún más
la leyenda de Mulán, incluso cuando juega con la virtud de la devoción
filial de maneras inexploradas en la animación original. ``En todas las
versiones anteriores que pudimos encontrar, siempre termina por volver y
regresar a su antigua vida, y pensamos que no era un final satisfactorio
para su jornada'', dijo Amanda Silver, quien coescribió el guión con
Rick Jaffa (ellos comparten crédito con Lauren Hynek y Elizabeth
Martin).

Silver y Jaffa se inspiraron especialmente en la emoción y el alcance de
la balada. (``Habla muy sucintamente de lo que ella pasa en la
batalla'', dijo Silver). Pero el original animado siempre fue una de sus
principales inspiraciones, y se pueden ver guiños y más guiños en toda
la película.

Todos los involucrados en la nueva película tenían escenas y elementos
favoritos de la original de Disney, cosas que tenían que tener en esta
última versión. A Jaffa le encantaba la secuencia en la que los soldados
hablan de su mujer ideal, aunque en esta versión, dijo, ``pensamos que
era super importante decirlo más claramente desde el punto de vista de
Mulán''.

Para Caro, era la escena de la casamentera, en la que Mulán cómica y
espectacularmente falla en su prueba de ``buena esposa'', y la
avalancha, una escena de batalla clave en la animación original. ``Con
toda la tecnología a nuestra disposición, por supuesto que íbamos a
hacerla'', dijo.

Image

Zhao Wei interpretó el papel principal en Hua Mulan, de
2009.Credit...Polybona Films

Y siendo esta una épica de acción, hay mucha más lucha que en la
original, particularmente de Mulán. La película tiene el aspecto y la
sensación de las épicas \emph{wuxias} de Zhang Yimou (piensa en
\href{https://www.nytimes3xbfgragh.onion/2004/08/27/movies/film-review-crouching-tiger-hidden-truths-court-king-who-would-be-emperor.html}{\emph{Hero}}
y
\href{https://www.nytimes3xbfgragh.onion/2004/10/09/movies/silk-brocade-soaked-in-blood-and-passion.html}{\emph{La
casa de las dagas voladoras}}), con túnicas fluidas que parecen tener
vida propia y espadas centelleantes, soldados que saltan sobre los
tejados y corren por las paredes. Incluso los famosos protagonistas de
sus películas (Gong Li, Donnie Yen, Jet Li) tienen papeles principales.
``Me inspiré enormemente en su trabajo'', dijo Caro. (``La marca de
Disney es que no se puede mostrar violencia'', señaló, por lo que no hay
destripamientos al estilo \emph{wuxia} o arcos de sangre que salgan a
chorros).

No solo teníamos que ver a Mulán pelear, dijo Caro, sino que teníamos
que verla pelear como mujer, de ahí todas esas tomas de Yifei Liu sin la
armadura que oculta el cuerpo, su pelo largo y suelto sin sombrero ni
casco. ``En esta versión, lo que aprende es que no será verdaderamente
poderosa hasta que sea ella misma, hasta que acepte el poder que tiene
como una mujer joven'', añadió Caro.

La película también añadió personajes como la hechicera que cambia de
forma de Gong Li, un llamativo contrapunto al estricto soldado que
encarna Mulán. También hay suficientes miradas de anhelo y escenas de
amor no correspondido para satisfacer al más ferviente fanático de las
comedias románticas. ``Me encanta la fluidez de género inherente a la
historia'', dijo Caro. ``Y hay una escena entre Mulán y el personaje de
Gong Li que está literalmente dirigida como una escena de amor. Es todo
consciente, pero la película también puede existir felizmente para el
público en general''.

¿Cómo les resultará esta versión a los fanáticos del original, ya sea la
balada o la animación de Disney? ``Sé que no vamos a complacer a todo el
mundo'', dijo Caro. ``Pero creo que hay una razón por la que la historia
ha sido tan relevante y tenido tanto eco durante, ¿cuánto? ¿más de 1300
años? Y al contarla como una película de acción con personajes de carne
y hueso mi esperanza es que hiciera posible que todos, incluso aquellos
que fueron tan protectores con la animación, la disfrutaran de nuevo de
una manera distinta''.

\begin{center}\rule{0.5\linewidth}{\linethickness}\end{center}

Advertisement

\protect\hyperlink{after-bottom}{Continue reading the main story}

\hypertarget{site-index}{%
\subsection{Site Index}\label{site-index}}

\hypertarget{site-information-navigation}{%
\subsection{Site Information
Navigation}\label{site-information-navigation}}

\begin{itemize}
\tightlist
\item
  \href{https://help.nytimes3xbfgragh.onion/hc/en-us/articles/115014792127-Copyright-notice}{©~2020~The
  New York Times Company}
\end{itemize}

\begin{itemize}
\tightlist
\item
  \href{https://www.nytco.com/}{NYTCo}
\item
  \href{https://help.nytimes3xbfgragh.onion/hc/en-us/articles/115015385887-Contact-Us}{Contact
  Us}
\item
  \href{https://www.nytco.com/careers/}{Work with us}
\item
  \href{https://nytmediakit.com/}{Advertise}
\item
  \href{http://www.tbrandstudio.com/}{T Brand Studio}
\item
  \href{https://www.nytimes3xbfgragh.onion/privacy/cookie-policy\#how-do-i-manage-trackers}{Your
  Ad Choices}
\item
  \href{https://www.nytimes3xbfgragh.onion/privacy}{Privacy}
\item
  \href{https://help.nytimes3xbfgragh.onion/hc/en-us/articles/115014893428-Terms-of-service}{Terms
  of Service}
\item
  \href{https://help.nytimes3xbfgragh.onion/hc/en-us/articles/115014893968-Terms-of-sale}{Terms
  of Sale}
\item
  \href{https://spiderbites.nytimes3xbfgragh.onion}{Site Map}
\item
  \href{https://help.nytimes3xbfgragh.onion/hc/en-us}{Help}
\item
  \href{https://www.nytimes3xbfgragh.onion/subscription?campaignId=37WXW}{Subscriptions}
\end{itemize}
